\chapter{Frequencies of conjunctions}\label{ch:7}\label{bkm:Ref1727395}\label{bkm:Ref41324909}\label{bkm:Ref50992543}\label{sec:7}

This chapter follows up on \citegen[462]{Hilpert2013b} suggestion that frequencies of occurrence can be relevant for studying constructions and \isi{constructional variation} and \is{constructional change}change. However, the purely frequency-based approach, common in much of traditional corpus linguistics, needs to be viewed critically: Frequencies of forms result from the need to express certain semantic or grammatical relations, and questions concerning those underlying factors should therefore be primary. For example, higher or lower frequencies of particular concessive conjunctions in certain varieties may be the result of
(i)~differences in the proportion of concessives that are expressed by means of subordination,
(ii)~general differences in the frequency of concessives (that is, subordinating and other),
(iii)~a \isi{topic bias} in the sampled corpus material that favours or disfavours the use of concessives, or
(iv)~a combination of several of these factors. This chapter nevertheless pursues the traditional approach of counting surface frequencies of conjunctions. In combination with \chapref{sec:8}, it serves as a point of departure for later analyses.

  Concerning the \is{text frequency}text frequencies of conjunctions, no precise hypotheses were formulated in \sectref{fig:5.3}, mainly for the reason that more profound insights are expected from a variationist perspective, i.e. when investigating the relative frequencies (proportions or percentages) of \SchuetzlerIndexExpression{although}, \SchuetzlerIndexExpression{though} and \SchuetzlerIndexExpression{even though} as variant forms. However, the general expectations would be that
(i)~the frequency ranking found in the literature will be confirmed, namely \SchuetzlerIndexExpression{although} > \SchuetzlerIndexExpression{though} > \SchuetzlerIndexExpression{even though}, perhaps with some uncertainty as to the position of the latter two, and that
(ii)~all three conjunctions should be more frequent in writing, perhaps less so with regard to \SchuetzlerIndexExpression{even though}. No general difference between \is{varieties of English!L1}L1 and \is{varieties of English!L2}L2 varieties is anticipated. Based on some of my earlier research, \sectref{sec:7.1} briefly inspects the frequencies of a range of markers (conjunc\-tions, \isi{prepositions}/adpositions and \isi{conjuncts}) beyond the ones targeted in this monograph. Section~\ref{sec:7.2} introduces the statistical model used for the main frequency analysis presented in \sectref{sec:7.3}, which then focuses on the three conjunctions \textit{although}, \textit{though} and \textit{even though}. Results are summarised in \sectref{sec:7.4}.

\section{\label{bkm:Ref481681511}\label{bkm:Ref488072060}Overview: Frequencies of different concessive markers}\label{sec:7.1}

This section provides some minimal context for the three conjunctions under investigation in this study by showing their rates of use relative to other, functionally related markers. A general overview of the frequencies of concessive markers that belong to different grammatical categories is given in \figref{fig:7.1}, which is based on data from twelve \is{International Corpus of English}ICE corpora and thus goes beyond the selection of varieties included in the present volume (see \citealt{Schützler2018c}). Panel (a) shows global mean frequencies while panel (b) contrasts frequencies in speech and writing.\footnote{Additional varieties included for \figref{fig:7.1} were
(i)~US-\ili{American English}, represented by a combination of ICE-USA and the \textit{\is{corpora!Santa Barbara Corpus}Santa Barbara Corpus of Spoken American English} (\citealt{DuBoisEtAl2000}); 
(ii)~\ili{Philippine English}; and
(iii)~\ili{New Zealand English}. The procedure for both parts of the figure was to determine for each marker in each variety frequencies in speech and writing, as well as their \isi{geometric mean} (cf. Footnote~\ref{fn81} on p.~\pageref{fn81}). Each data point shown in the figure was then arrived at by calculating the \isi{geometric mean} of the twelve respective variety-based values (cf. the online appendix at \url{https://osf.io/m4tfc/}). Frequencies were thus not estimated using a regression model (cf. \sectref{sec:7.3}) but simply counted and normalised within the two broad categories of speech and writing.}

  Three broad groups can be identified. The \is{conjuncts}conjunct \SchuetzlerIndexExpression{however} and the subordinator \SchuetzlerIndexExpression{although} are the most frequent connectives; \SchuetzlerIndexExpression{though} (both as a conjunction and a \is{though@\textit{though}!as conjunct}\is{conjuncts}conjunct), \SchuetzlerIndexExpression{despite} and \SchuetzlerIndexExpression{even though} constitute a cluster of markers that are of intermediate frequency; a third group of relatively rare markers comprises \SchuetzlerIndexExpression{nevertheless}, \SchuetzlerIndexExpression{in spite of}, the conjunction \SchuetzlerIndexExpression{however} and the adposition \SchuetzlerIndexExpression{notwithstanding}. For patterns in individual varieties that diverge slightly from this general picture, see \citet[158]{Schützler2018c}.

\begin{figure}
\includegraphics{figures/CCs.Fig.7.1.pdf}
\caption{\label{bkm:Ref477783402}\label{fig:7.1}Frequencies of different concessive markers (ICE)}
\end{figure}

\figref{fig:7.1}b indicates that in the vast majority of cases concessive markers are considerably more frequent in written language. The only items that deviate from this pattern are the \is{conjuncts}conjunct \is{though@\textit{though}!as conjunct}\SchuetzlerIndexExpression{though}, which is preferred in speech (cf. \citealt{Schützler2020a}), and the conjunction \SchuetzlerIndexExpression{even though}, which seems to be equally frequent in both \is{mode of production}modes of production. The result for the latter anticipates tendencies also found in the more detailed, regression-based analyses in \sectref{sec:7.3} below, with more insights provided in subsequent chapters. Counts of the type presented in \figref{fig:7.1} can go some way towards an estimate of the total number of concessives used in varieties of English. However, one must bear in mind that it is also possible for concessive meaning to be expressed without an explicit grammatical marker, and that the coordinator \SchuetzlerIndexExpression{but} poses a certain problem since it is very frequent and quite variable in function (e.g. concessive vs purely \is{adverbials!contrast}adversative).

\section{\label{bkm:Ref52283842}Statistical model}\label{sec:7.2}
\begin{sloppypar}
Frequencies of all three conjunctions are estimated with a \is{Bayesian statistics}Bayesian \is{regression!negative binomial}negative binomial \is{regression!mixed-effects}mixed-effects regression model, which is given the denomination “Model~A” and breaks down into nine variety-specific submodels of exactly the same form, each based on the respective subset of the data. The output from these models is used in all analyses in \sectref{sec:7.3}. The only \is{cluster variables}cluster variable in the model is \textsc{genre}: The smallest unit of observation in the (\is{regression!negative binomial}negative binomial) count model is the individual text, which is why \textsc{text} does not enter the model as a \is{cluster variables}cluster variable. As discussed earlier, variety, too, does not feature as a \is{cluster variables}cluster variable, due to the one-model-per-variety approach that was taken~– a characteristic shared by all models in this study. There are only two predictor variables in the fixed part of Model~A,
(i)~\isi{mode of production} and
(ii)~the marker itself (\textit{although}, \textit{though}, \textit{even though}), corresponding to the independent variables \textsc{spoken.ct} and \textsc{marker}. The model syntax is shown in (\ref{eq:7.1}); for a definition of variables, see \tabref{tab:6.3} in \sectref{sec:6.3.6}. The predictors \textsc{spoken.ct} and \textsc{marker} interact in the model. Additionally, slopes for \textsc{marker} are specified as varying randomly across \textsc{genre}. The component labelled “offset” refers the model to the variable \textsc{log\_words}, which quantifies the logged number of words per text. Note that this variable is not specified in \tabref{tab:6.3}.
\end{sloppypar}

\ea
\label{bkm:Ref41307899}\label{eq:7.1}Model A: Syntax
\begin{lstlisting}
count ~ spoken.ct * marker
        + (marker | genre)
        + offset(log_words)
\end{lstlisting}
\z

Appendix~\ref{appendix:B.1} provides more information concerning token numbers, the number of levels of the random factor \textsc{genre}, the implemented \isi{priors} and the number of posterior samples. Data, scripts and comprehensive regression tables are published online (cf. \sectref{sec:1.4}).

\section{\label{bkm:Ref3905874}Results}\label{sec:7.3}

There will first be a discussion of global frequency patterns, i.e. a comparison of the estimated average frequencies of the three markers in all varieties, which is then unfolded into a comparison of frequencies in speech and writing. Finally, the focus shifts from a variety-based approach to a marker-based approach by ranking for each connective the conditions that favour (or disfavour) its occurrence. This latter part does not provide new information but a new perspective.

A first general assessment is shown in \figref{fig:7.2}. Here, median frequency estimates of the three conjunctions are summed for each variety, and varieties are ranked by this total number of CCs. The four \is{varieties of English!L1}L1 varieties are shown in black, while the five \is{varieties of English!L2}L2 varieties are shown in grey.

\begin{figure}
\includegraphics{figures/CCs.Fig.7.2.pdf}
\caption{\label{bkm:Ref58235298}\label{fig:7.2}Summed frequencies of concessive subordinators in nine varieties; black~= L1 varieties, grey~= L2 varieties}
\end{figure}

This basic display highlights that there is a core frequency range for subordinating CCs, which extends from 334 pmw (in \il{Jamaican English}JamE) to 388 pmw (in \il{Singapore English}SingE), suggesting a fairly stable rate of use of this kind of construction in most varieties. However, the total number of CCs is substantially higher in \il{British English}BrE (537 pmw) and substantially lower in \il{Nigerian English}NigE (284 pmw). \figref{fig:7.2} draws attention to this phenomenon, which tends to be obscured in more detailed visualisations (e.g. in \figref{fig:7.3} below), and raises the question of its potential implications. It seems difficult to motivate the dramatic difference between \il{Nigerian English}NigE and \il{British English}BrE. Does it arise
(i)~because speakers and writers of these varieties stand out in using (or not using) CCs as a semantico-pragmatic device,
(ii)~because these particular \is{International Corpus of English!components}components of \is{International Corpus of English}ICE contain texts (i.e. speakers/writers) that are atypical, from a cross-varietal perspective, or
(iii)~because the subject matters in the respective corpora happen to favour (or disfavour) the use of CCs? More speculative reasons of this kind could be adduced, but it should be clear that purely frequency-based results of this kind are difficult to interpret. Similar concerns apply when inspecting the log-scaled visualisations further below: Since intra-constructional semantics and the arrangement of subordinate and matrix clauses in a CC are hypothesised (and indeed shown) to play a role in the choice of conjunction (see \chapref{sec:10}), a frequency analysis that blinds itself to these factors must be of limited explanatory power. Thus, it cannot be emphasised enough that the present chapter should be understood as a general background against which the results in later chapters emerge all the more clearly in their interpretability. At the same time, it represents a traditional corpus-linguistic approach based on surface frequencies, which makes it comparable to earlier studies.

Frequencies of the three conjunctions \textit{although}, \textit{though} and \textit{even though} in all nine varieties are plotted in \figref{fig:7.3}, based on the respective component models of Model~A (for a concise summary of model parameters, see the online appendix). Effects of the \isi{mode of production} are controlled for by estimating average values based on the written and spoken conditions.

\begin{figure}
\includegraphics{figures/CCs.Fig.7.3.pdf}
\caption{\label{bkm:Ref482364726}\label{fig:7.3}Average frequencies of concessive subordinators; A~=~\SchuetzlerIndexExpression{although}, T~=~\SchuetzlerIndexExpression{though}, E~=~\SchuetzlerIndexExpression{even though}}
\end{figure}

As expected, \SchuetzlerIndexExpression{although} is the most frequent one of the three conjunctions in all varieties except \il{Indian English}IndE, with an average text frequency of 184 pmw across varieties (not shown).\footnote{For the calculation of this and the following two values, the \isi{geometric mean} was used, i.e. the average frequencies (pmw) in all nine varieties were logged, averaged, and then re-transformed by exponentiation. Thus, the result matches the visual impression conveyed by the plot. The following R function was used, where \texttt{x} is a vector of values to be averaged using the \isi{geometric mean}: \texttt{MGeom = function(x) \{exp(mean(log(x)))\}}.\label{fn81}} In six out of nine varieties, \SchuetzlerIndexExpression{though} is the second most frequent conjunction, with an average text frequency of 103 pmw. Finally, \SchuetzlerIndexExpression{even though} is generally least frequent, with only two exceptions (CanE and HKE); its average text frequency is 62 pmw. These findings are in reasonable agreement with what is shown in \figref{fig:7.1} above.\largerpage

\begin{sloppypar}
Against the background of general differences between varieties outlined above, the following paragraphs will focus on frequency differences between spoken and written language in the nine varieties. Based on the same statistical model (Model~A), the approach here is not to control for \isi{mode of production} but to show two estimates for each of the three conjunctions in each variety. The visualisation in \figref{fig:7.4} is subdivided into nine parts, corresponding to varieties, each of which takes two perspectives. In the lower panel, estimated \is{text frequency}text frequencies are plotted, showing expected values in speech and writing. Values for the three conjunctions are connected with dotted lines in each \isi{mode of production} to facilitate the comparison of frequency patterns (cf. \citealt{Schützler2023}). In the upper panel of each subplot, differences between frequencies in the two modes are highlighted by focusing on the frequency ratio of $f_{\text{W}}$ divided by $f_{\text{S}}$. Like the frequency values themselves, this measure of difference is log-scaled to ensure that the relative differences come across more clearly in the visual display (cf. \citealt{Schützler2023}).
\end{sloppypar}

In the vast majority of cases, the three markers are more frequent in writing, which is readily seen when inspecting the upper panels for the nine varieties in \figref{fig:7.4}: Virtually all median ratios are greater than (or equal to) 1, with a single exception in \il{Nigerian English}NigE (see below). The most extreme frequency difference between writing and speech is found for the conjunction \SchuetzlerIndexExpression{though} in \il{Hong Kong English}HKE, with \textit{Ro}\textsubscript{~W/S}~=~6.7 [2.9; 16.2]. For \SchuetzlerIndexExpression{even though}, the difference between modes is not substantially different from \textit{Ro}~=~1 in seven out of the nine varieties, namely \il{Singapore English}SingE (1.7 [0.8; 3.6]), \il{Indian English}IndE (1.6 [0.9; 2.9]), \il{Irish English}IrE (1.4 [0.8; 2.8]), \il{Canadian English}CanE (1.3 [0.8; 2.1]), \il{Jamaican English}JamE (1.2 [0.5; 2.7]), \il{Australian English}AusE (1.0 [0.6; 1.8]), and \il{Nigerian English}NigE (0.9 [0.5; 1.5]). In \il{Nigerian English}NigE, the ratio is even slightly in favour of spoken language. In \il{British English}BrE and \il{Hong Kong English}HKE, the written-to-spoken ratio for \SchuetzlerIndexExpression{even though} is most substantially different from 1, with \textit{Ro}\textsubscript{~W/S}~=~2.1 [1.0; 4.4] and \textit{Ro}\textsubscript{~W/S}~=~1.9 [1.1; 3.4], respectively. The general difference between \SchuetzlerIndexExpression{although} and \SchuetzlerIndexExpression{though} (treated as a pair) on the one hand and \SchuetzlerIndexExpression{even though} on the other is highlighted by several “hockey-stick” patterns in the upper panels of \figref{fig:7.4}.

\begin{figure}
\includegraphics{figures/CCs.Fig.7.4.pdf}
\caption{\label{bkm:Ref1849948}\label{fig:7.4}Frequencies of concessive subordinators in speech and writing; A~=~\SchuetzlerIndexExpression{although}, T~=~\SchuetzlerIndexExpression{though}, E~=~\SchuetzlerIndexExpression{even though}, W~=~written, S~=~spoken}
\end{figure}

Using frequency differences between speech and writing as a very rough indicator of \is{style}stylistic function, it appears even from the purely visual inspection of \figref{fig:7.4} that both \SchuetzlerIndexExpression{although} and \SchuetzlerIndexExpression{though} are more sensitive (or specialised) in this regard, being much more common in writing than in speech; \SchuetzlerIndexExpression{even though}, on the other hand, is much more evenly distributed between \is{mode of production}modes of production. If we average across the written-to-spoken ratios of all varieties for the three conjunctions, using the \isi{geometric mean} (cf. Footnote~\ref{fn81} on p.~\pageref{fn81}), this impression is fully confirmed: The average written-to-spoken ratio is 3.0 for \SchuetzlerIndexExpression{although}, 3.6 for \SchuetzlerIndexExpression{though}, but only 1.4 for \SchuetzlerIndexExpression{even though}.

The effect of mode has the same direction in \is{varieties of English!L1}L1 and \is{varieties of English!L2}L2 varieties, but in the latter group it seems to be smaller for the two more frequent conjunctions, with a (geometric) mean ratio of 2.8 for \SchuetzlerIndexExpression{although}~(as compared to 3.1 in \is{varieties of English!L1}L1 varieties) and 2.9 for \SchuetzlerIndexExpression{though} (as compared to 3.8 in \is{varieties of English!L1}L1 varieties). For \SchuetzlerIndexExpression{even though}, the two values are about the same (1.3 in \is{varieties of English!L2}L2; 1.4 in \is{varieties of English!L1}L1). If we again accept speech and writing as very rough \is{style}stylistic categories, the more level pattern in \is{varieties of English!L2}L2 varieties at least for \SchuetzlerIndexExpression{although} and \SchuetzlerIndexExpression{though} can very tentatively be interpreted as a lack of differentiation according to \citegen{Schneider2003} Dynamic Model (cf. \sectref{sec:4.3.2}). As pointed out above, however, conclusions of this kind must be tentative if drawn on the basis of a model that is not truly \is{multifactorial approaches}multifactorial as it ignores other underlying functional and formal factors of potential importance. It will therefore be necessary to revisit the results presented here when discussing findings in \chapref{sec:10}.

Further interesting nuances are revealed as we shift our perspective by focusing on speech and writing in isolation, taking a step back from the direct comparison of the two modes of \isi{production}. In writing, \il{Indian English}IndE constitutes the single exception to the otherwise perfectly regular ranking $f_A > f_T > f_E$, with \SchuetzlerIndexExpression{though} being the most frequent conjunction of the three. The regularity of the pattern in all other written varieties suggests that there is considerable agreement as to which conjunctions are most generally usable in this mode, and this is consistent with \citegen{McArthur1987} postulation of a written \ili{World Standard English} (cf. \figref{fig:4.4}b on p. \pageref{bkm:Ref497947644}). In speech, the frequency ranking of the three conjunctions is much more variable. Four different patterns exist, the most common one being $f_A > f_E > f_T$, which is found in \il{Irish English}IrE, \il{Canadian English}CanE, \il{Australian English}AusE, \il{Jamaican English}JamE and \il{Hong Kong English}HKE, followed by the pattern $f_A > f_T > f_E$ in \il{British English}BrE and \il{Singapore English}SingE. \il{Indian English}IndE has the unique pattern $f_T > f_A > f_E$, and \il{Nigerian English}NigE also stands out in having very similar (low) frequencies of both \SchuetzlerIndexExpression{though} and \SchuetzlerIndexExpression{even though}, which can strictly be ranked as $f_E > f_T > f_A$. Thus, four out of the six possible frequency rankings do in fact occur in spoken varieties, while patterns in writing are essentially uniform. From a general perspective, the conjunction \SchuetzlerIndexExpression{even though} seems to be characterised by a considerably higher (relative) rate of occurrence in spoken English, the result at the level of the individual variety very often being a pronounced difference in pattern between speech and writing~– only \il{British English}BrE, \il{Singapore English}SingE and \il{Indian English}IndE are characterised by the same general ranking of conjunctions in both modes.

\figref{fig:7.5} focuses entirely on the conjunction \SchuetzlerIndexExpression{although} and ranks all spoken and written varieties (\textit{n~}=~18) by their absolute (normalised) \is{text frequency}text frequencies of this marker. This display and its minimal discussion does not go beyond \figref{fig:7.4} above, but it arranges the information in a different way and thus provides another perspective on the data. On the one hand, the actual frequency range of \SchuetzlerIndexExpression{although} [44; 461] is more clearly visible here. On the other hand, the black and white boxes on the right highlight the ranking of conditions according to variety type (\is{varieties of English!L1}L1 vs \is{varieties of English!L2}L2) and \isi{mode of production}. This part of the figure is further supported by triangular indicators that show the mean ranks for the two groups that are compared in each column, using the respective colours. We see that \SchuetzlerIndexExpression{although} is more frequent in \is{varieties of English!L1}L1 varieties (mean rank: 8.0) than in \is{varieties of English!L2}L2 varieties (mean rank: 10.7), but the more striking contrast is between written and spoken varieties, with mean ranks of 5.2 and 13.8, respectively.

The same perspective is taken for the conjunction \SchuetzlerIndexExpression{though} in \figref{fig:7.6}. Compared to \SchuetzlerIndexExpression{although}, the range of median values is shifted towards lower values [28; 319].{\interfootnotelinepenalty=4000\footnote{The range for \SchuetzlerIndexExpression{though} also appears to be narrower in absolute terms, but on a logarithmic scale the difference between maximum and minimum is very similar for both conjunctions.}} In terms of absolute \is{text frequency}text frequencies, this marker is somewhat more common in the \is{varieties of English!L2}L2 varieties under investigation (mean rank: 8.6) compared to the \is{varieties of English!L1}L1 varieties (mean rank: 10.6). Once again, frequencies in written varieties are much higher than in spoken varieties, with mean ranks of 5.4 and 13.6, respectively.

\begin{figure}
\includegraphics{figures/CCs.Fig.7.5.pdf}
\caption{\label{bkm:Ref52266654}\label{fig:7.5}Text frequencies: Ranking of specific conditions for \SchuetzlerIndexExpression{although}; W~= written, S~=spoken}
\end{figure}

\begin{figure}
\includegraphics{figures/CCs.Fig.7.6.pdf}
\caption{\label{bkm:Ref52266684}\label{fig:7.6}Text frequencies: Ranking of specific conditions for \SchuetzlerIndexExpression{though}; W~= written, S~=spoken}
\end{figure}

Finally, \figref{fig:7.7} shows the frequency rankings of all \textit{n~}=~18 conditions for \SchuetzlerIndexExpression{even though}. Like \SchuetzlerIndexExpression{though}, this conjunction occurs slightly more frequently in \is{varieties of English!L2}L2 varieties (mean rank: 8.8) than in \is{varieties of English!L1}L1 varieties (mean rank: 10.4), and it is also much more frequent in written varieties, with a mean rank of 6.1 as compared to a mean rank of 12.9 in spoken varieties. The lower bound of the range of median values for \SchuetzlerIndexExpression{even though} is the same as for \SchuetzlerIndexExpression{though}, but it is more restricted at higher values [28; 112].

\begin{figure}
\includegraphics{figures/CCs.Fig.7.7.pdf}
\caption{\label{bkm:Ref52266690}\label{fig:7.7}Text frequencies: Ranking of specific conditions for \SchuetzlerIndexExpression{even though}; W~= written, S~=spoken}
\end{figure}

Regarding \is{text frequency}text frequencies, then, the main division in the data runs between spoken and written varieties, while variety status (\is{varieties of English!L1}L1 vs \is{varieties of English!L2}L2) plays no more than a subsidiary role. Differences between \is{varieties of English!L1}L1 and \is{varieties of English!L2}L2 varieties will feature in \chapref{sec:10} as well, but the patterns that emerge there are somewhat difficult to reconcile with what is shown here. The implications will be discussed in \sectref{sec:10.3}.

\section{\label{bkm:Ref51831577}Summary and discussion}\label{sec:7.4}

The discussion of \isi{text frequency} patterns in this concluding section will be kept to a minimum and needs to be preceded by a few notes of caution. The purely form-driven investigation of linguistic phenomena based on \isi{text frequency} has a long tradition in corpus linguistics, possibly because \isi{text frequency} as such is the most immediately observable and objective aspect in the study of linguistic constructions. The general approach in this study, however, rests on the belief that the construction of a linguistic expression is motivated from the desire to express certain content or certain relations (i.e. the functional side of language), and that it is therefore necessary to treat those functions as predictor variables when analysing or counting forms. This will be the general approach in Chapters \ref{sec:9}–\ref{sec:11}, as explained and shown in schematic form in \sectref{sec:4.1.3} above. Against this background assumption, the aim of this chapter has been twofold.

Firstly, it followed up on a notion formulated by \citeauthor{Hilpert2013b} (\citeyear[462]{Hilpert2013b}; see the very beginning of this chapter): A general investigation of frequencies can be a useful point of departure for the more detailed (or \is{multifactorial approaches}multifactorial) investigation of constructions, since \isi{text frequency} patterns can be symptoms of underlying cognitive or functional mechanisms and processes. In particular, the frequent exposure to a particular construction type or a particular connective may have consequences for the \isi{entrenchment} of such linguistic forms. Secondly, findings in this chapter can be juxtaposed with findings in \chapref{sec:10} below, mainly to gauge whether or not the purely form-driven approach can be meaningfully related to analyses that are motivated functionally and consider multiple factors. In many respects, the investigation of text frequency in this chapter will be qualified in the light of the later analyses.

Turning to the results proper, the aggregated frequencies of the three conjunctions for individual varieties fell within a reasonably narrow range, with two notable exceptions: \il{British English}BrE (with a very much higher total rate of use) and \il{Nigerian English}NigE (with a very much lower rate). Individual outlier varieties of this kind may arise from the nature and quality of the data (e.g. a lack of corpus \is{corpora!comparability of}comparability due to topic-related \isi{sampling error}), or there may indeed be a fundamental difference between varieties, in the sense that culture-specific aspects play a role in the use of CCs as discourse-structuring, \is{rhetorics}rhetorical devices. However, explanatory approaches of this kind must at present remain speculative and need to be addressed by independent studies of a different methodological orientation (see comments in \chapref{sec:1}).

While the typical frequency ranking in written varieties is $f_A > f_T > f_E$, \SchuetzlerIndexExpression{even though} is occasionally the second most frequent marker in speech. This is because \isi{mode of production} has a strong impact on the \is{text frequency}text frequencies of \SchuetzlerIndexExpression{although} and \SchuetzlerIndexExpression{though} (with considerably lower frequencies in speech), while \SchuetzlerIndexExpression{even though} regularly seems to be immune to this effect. As a result, the role of \SchuetzlerIndexExpression{even though} in speech is strengthened, relative to the other markers. This has certain implications if we treat \isi{mode of production} as a basic, binary \is{style}stylistic variable; the more fine-grained analyses (particularly in \chapref{sec:10}) will provide a more detailed picture of this phenomenon.

Finally, let me briefly consider how results relate to the previous research summarised in \sectref{sec:5.1.1}. The general frequency pattern, with \SchuetzlerIndexExpression{although} and \SchuetzlerIndexExpression{though} much more frequent than \SchuetzlerIndexExpression{even though}, as described, for instance, by \citet{QuirkEtAl1985} and confirmed by \citet{Altenberg1986} and \citet{Aarts1988}, is also found in the present study. There is, however, not much evidence to support Quirk et al.’s (\citeyear{QuirkEtAl1985}: 1097–1099) treatment of \SchuetzlerIndexExpression{though} as “more \is{informality}informal” than \SchuetzlerIndexExpression{although} (see also \citealt{BiberEtAl1999} and \citealt{HuddlestonPullum2002}), except that in many cases \SchuetzlerIndexExpression{though} responds somewhat less strongly to the spoken/written dimension of variation and thus seems to be less sensitive (or specialised) in this regard. The tendency for \SchuetzlerIndexExpression{although} to respond most vigorously to a difference in the \isi{mode of production} confirms a pattern evident in \citegen{Altenberg1986} study based on data from \is{corpora!LLC}LLC and \is{corpora!LOB}LOB. Results in the present study also agree with \citet{Aarts1988}, who finds that, stylistically, \SchuetzlerIndexExpression{although} is most sensitive and \SchuetzlerIndexExpression{even though} is least sensitive in the comparison of the three markers. The \is{emphasis}emphatic character \citet{QuirkEtAl1985} ascribe to \SchuetzlerIndexExpression{even though} may play a role in this conjunction’s higher rates of use in spoken language, if we take a somewhat more generous view on the concept of \textit{emphasis} and extend it to more \is{involvement}involved or personal speech styles as found in some kinds of spoken discourse. It must be borne in mind, however, that only \textit{some} spoken \is{International Corpus of English!genres}\is{genre}genres in \is{International Corpus of English}ICE can be characterised as \is{involvement}involved.

This chapter has shown no more than general patterns, and~– with the necessary caveats concerning text frequency analyses in mind~– more detailed studies of \is{style}stylistic variation are called for. There are a few surprising findings in specific varieties of English~– see, for instance, patterns in \il{Indian English}IndE (both spoken and written) and spoken \il{Nigerian English}NigE in \figref{fig:7.4}. It remains to be seen whether these exceptions are confirmed when more complex, functionally motivated analyses are undertaken in \chapref{sec:10}.

