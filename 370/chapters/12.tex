\chapter{Conclusion and outlook}\label{sec:12}\label{ch:12}\label{bkm:Ref4146401}\label{bkm:Ref60664348}

This study set out to generate insights concerning the \is{function and form}functional and formal variation of a certain set of \isi{concessive constructions} (CCs), namely complex sentences with the subordinating conjunctions \SchuetzlerIndexExpression{although}, \SchuetzlerIndexExpression{though} or \SchuetzlerIndexExpression{even though}. The original point of departure (as in \citealt{Schützler2018c}) was the question as to why these three markers coexist in English, and what the division of labour between them is. This book goes some way beyond this original question: It is no longer only the connectives that are under scrutiny, but the general correlations that exist between functional and formal properties of the constructions in which they occur. Highlighting the ties between these different facets of a CC, the study fills some of the gaps that are left by grammars of English.

The book also proposes one particular approach to \isi{constructional variation}, essentially dealing with the questions of how to build theories for complex, multi-faceted constructions and their variability, and how to capture those constructions in statistical terms.\footnote{Apart from their introduction in \sectref{sec:6.3}, statistical techniques were not foregrounded in the analytic chapters of this volume. The online appendix (\url{https://osf.io/m4tfc/})~– perhaps together with the published data (\citealt{Schützler2021}; \url{https://doi.org/10.18710/1JMFVR})~– provides much more detail and can serve as a point of departure for further analyses (see also \sectref{sec:1.4}).} The resulting kind of quantitative \isi{Construction Grammar} treats the different components of a CC (and, by extension, other constructions) as hierarchically ordered and embedded within each other. This may seem to be in conflict with some of the basic tenets of \is{Construction Grammar}CxG~– for instance, the fusion (or inextricability) of form and function. On the other hand, it can be argued that the model agrees well with the notion that more general constructions break down into \is{constructions!subconstructions}subconstructions at different levels of granularity. This paradoxical situation~– with the scientific model partly supporting, partly contradicting CxG-based thinking~– will be discussed in some more detail in \sectref{sec:12.3}.

Apart from its contributions to the description of CCs and to CxG-based theories, the present study is also relevant in the context of \isi{varieties of English} world-wide (see \sectref{sec:4.3}). However, general results in this dimension of variation suggest that the phenomenon at hand is not a \is{salience}salient marker of variety affiliation, as most of the inter-varietal differences that do exist are relatively slight or unsystematic, particularly when inspecting the general contrast between \is{varieties of English!L1}L1 and \is{varieties of English!L2}L2 varieties.\footnote{Many of the patterns are perhaps best regarded as reflections of \isi{sampling error}, the diachronic dimension of \is{International Corpus of English}ICE, or differences between the individual compilation processes.} On the whole, CCs and their structured variation seem to be a relatively stable and homogeneous part of English grammar, at least from the synchronic perspective.

Beyond all of the above, the present study has provided, categorised and discussed in detail a wealth of corpus examples, highlighting the semantic and pragmatic versatility of CCs. The tension between the propositions juxtaposed in a construction can be based on generally understood pieces of \isi{world knowledge} (so-called \textit{topoi}) concerning facts that are typically incompatible, yielding what was called \textit{anticausal} concessives or their inverse, \textit{epistemic} concessives. In \textit{dialogic} concessives, on the other hand, the contrast may be based not on the expected incompatibility of facts but merely on the \is{modification (or qualification)}qualification of one proposition by another. \citet[166]{Hilpert2013a} calls concessives of this kind “mixed-messages”, because they allow for different overall interpretations or evaluations and may therefore trigger different, perhaps even diametrically opposed, courses of action. For all types of CCs~– \is{concessives (types of)!anticausal}anticausal, \is{concessives (types of)!epistemic}epistemic and \is{concessives (types of)!dialogic}dialogic~– the number of possible \is{topos}topoi and semantic patterns is vast, and the possible propositional content of CCs is virtually limitless. In a way, what is produced by a collection of CCs and the precise relations holding between their component propositions is essentially a mirror image of human reasoning and argumentation.

The paragraphs above can stand as a broad summary of the main contributions of this book. The remainder of this chapter serves three purposes:
(i)~It summarises the main results of the quantitative analyses (\sectref{sec:12.1});
(ii)~it points to wider contexts of investigation in which we can place CCs, and more comprehensive ways of looking at these constructions (\sectref{sec:12.2}), venturing recommendations as well as warnings; and
(iii)~it reflects in more detail upon the advantages and disadvantages~– as well as the overall plausibility~– of the \is{constructional choice model}choice model of constructional variation that was proposed (\sectref{sec:12.3}), including the discussion of alternative views. Finally, \sectref{sec:12.4} concludes the book with a few final remarks.

\section{\label{bkm:Ref502620948}Summary of results}\label{sec:12.1}

Throughout the book, a distinction was made between Chapters \ref{sec:7} \& \ref{sec:8} on the one hand and Chapters \ref{sec:9}, \ref{sec:10} \& \ref{sec:11} on the other. The two earlier chapters work with the \is{text frequency}text frequencies of conjunctions and semantic types (using count models), while the three later chapters inspect choices in \isi{variable contexts} (using binary and \is{regression!multinomial}multinomial regression models). As will be discussed in \sectref{sec:12.1.1} below, the former type of analysis is somewhat limited compared to the latter: We may be in a position to explain much of what determines choices made in the relevant contexts, but it is more difficult to explain the frequency of a phenomenon as a whole. For instance, the \isi{text frequency} of a particular semantic type may well depend on the discourse topic and other factors not normally of interest in a variationist approach. On the other hand, \isi{text frequency} has often been central in corpus-linguistic studies, and its discussion in this book~– particularly vis-à-vis the contributions made by Chapters \ref{sec:9}, \ref{sec:10} \& \ref{sec:11}~– can highlight certain methodological issues. Results from the latter three chapters are discussed in \sectref{sec:12.1.2}, drawing on the \is{constructional choice model}choice model of constructional variation that was proposed, and thus forming a more integrated whole. Finally, \sectref{sec:12.1.3} returns to a question that originally inspired the investigation as a whole (cf. \citealt{Schützler2018c}). This concerns the functional differences between the three conjunctions \textit{although}, \textit{though} and \textit{even though}, i.e. the question as to how exactly they divide between them the task of introducing concessive subordinate clauses.

\subsection{\label{bkm:Ref80357338}Frequency-based accounts: Uses and limitations}\label{sec:12.1.1}

As discussed in \sectref{sec:7.4}, investigations of the \is{text frequency}text frequencies of phenomena have traditionally taken centre stage in quantitative corpus linguistics. However, they may come with the risk of presenting an oversimplified picture. Absolute (normalised) frequencies sometimes do, but often enough do \textit{not} give us the answers we are looking for. The summary of results from Chapters~\ref{sec:7} \& \ref{sec:8} will therefore be brief, and it will to an extent serve the purpose of throwing the discussion of \isi{variable contexts} in \sectref{sec:12.1.2} into sharper relief.

\begin{sloppypar}
In an inspection of the cumulated frequencies of all three conjunctions it turned out that the total number of subordinating CCs was reasonably similar in most varieties, but also that rather extreme outliers do exist, e.g. \il{British English}BrE with its much higher overall frequency, and \il{Nigerian English}NigE with a very much lower overall rate. While, in a \is{multifactorial approaches}multifactorial design, the researcher can with some success discuss the reasons why one of several possible realisations was selected, general frequency differences of this kind may result from data quality issues, or from hard-to-gauge characteristics of varieties and their underlying cultures. They are therefore difficult to interpret. Concerning the individual conjunctions, \SchuetzlerIndexExpression{although} is usually most frequent in written English, while \SchuetzlerIndexExpression{even though} is most of the time least frequent. In speech, \SchuetzlerIndexExpression{even though} is more frequent relative to the others, mostly because it is much less susceptible to the tendency of spoken language to generate fewer complex sentences. The general pattern in the present study agrees with much of the literature (e.g. \citealt{QuirkEtAl1985,Altenberg1986,Aarts1988}), although it does not seem plausible to describe \SchuetzlerIndexExpression{though} as less \is{formality}formal than \SchuetzlerIndexExpression{although} (cf. \citealt{QuirkEtAl1985}: 1097–1099, \citealt{BiberEtAl1999,HuddlestonPullum2002}), at least not on the basis of a simplistic operationalisation of \isi{style} as “spoken vs written”. There is, however, a tendency for \SchuetzlerIndexExpression{although} to respond most strongly and for \SchuetzlerIndexExpression{even though} to respond least strongly to differences in \isi{mode of production}, all of which aligns relatively well with patterns found by \citet{Altenberg1986} and \citet{Aarts1988}, for instance.  \citegen{QuirkEtAl1985} characterisation of \SchuetzlerIndexExpression{even though} as \is{emphasis}emphatic is difficult to confirm, unless we stretch our definition of \textit{emphasis} to simply include notions like “\isi{involvement}” or “directness”, which would then partly account for this marker’s popularity in speech. The meaning of the results described here will be brought out more clearly by considering not only what we know about the currency of semantic types (see following paragraph) but also by the \is{multifactorial approaches}multifactorial investigation summarised in \sectref{sec:12.1.2}.
\end{sloppypar}

As regards the \is{text frequency}text frequencies of semantic types, the present study somewhat surprisingly found that \is{concessives (types of)!dialogic}dialogic CCs are by far the most frequent type in all varieties~– “surprisingly”, because grammars primarily tend to cite \is{concessives (types of)!anticausal}anticausal examples and the literature seems to treat these as prototypical. However, the \is{concessives (types of)!anticausal}anticausal type only comes second in frequency, followed at a considerable distance by \is{concessives (types of)!epistemic}epistemic (and \is{concessives (types of)!narrow-scope dialogic}narrow-scope dialogic) CCs. Contrary to expectation, \is{concessives (types of)!dialogic}dialogic CCs do not associate with speech. This correlation was hypothesised because the two component propositions in this semantic type are pragmatically on a par (i.e. not captured by an \textsc{if~→~then} relation), the entire construction is therefore (cognitively) more coordinated in character, and paratactic structures are generally more common in speech. The finding that \is{concessives (types of)!narrow-scope dialogic}narrow-scope CCs are considerably more frequent in writing casts further doubt on the usefulness of \is{text frequency}text frequencies as outcomes. It can be shown that, at the syntactic level, this particular semantic type is most commonly constructed with a \is{nonfinite clauses}nonfinite complement of the conjunction. Since nonfiniteness is generally characteristic of writing, the large number of \is{concessives (types of)!narrow-scope dialogic}narrow-scope CCs found in that mode may in fact be an artefact of this particular syntactic property, which considerably complicates the interpretation of results.

Much of the literature remains silent on the issue of semantic types of CCs. In direct contrast with findings in the present study, results reported by \citet{Hilpert2013a} suggest that the \is{concessives (types of)!anticausal}anticausal type is most frequent. As argued in \sectref{sec:5.1.2}, Hilpert’s study does not include the conjunction \textit{even though} and moreover focuses on specific constructions with \is{co-referentiality}co-referential subjects in matrix clause and subordinate clause. Particularly the second point can probably account for much of the discrepancy between results. Concerning the present study, the fact that the predominance of \is{concessives (types of)!dialogic}dialogic CCs holds true in all varieties under investigation inspires a certain degree of confidence in this finding.\label{bkm:Ref80357452}\label{bkm:Ref80604785}

\subsection{Multifactorial analyses at different levels}\label{sec:12.1.2}\label{bkm:Ref82434670}

The notion of the different “levels” of a construction and its theoretical implications will feature more prominently in \sectref{sec:12.3} below. Here, suffice it to remind the reader that the quantitative analyses of Chapters \ref{sec:9}, \ref{sec:10} \& \ref{sec:11} assume a nestedness of lower-level (or more local) constructional properties within higher-level (or more general) properties. The highest formal level involves the placement of the basic building blocks in a CC, \isi{matrix clause} and subordinate clause. The intermediate level involves the selection of a concessive conjunction, which serves as the node between the two clauses. At the lowest level, the subordinate structure is syntactically unfolded into a \is{finite clauses}finite or \is{nonfinite clauses}nonfinite clause. It is in this order that results will be summarised in the following paragraphs.

\emph{Clause position} was treated as a binary variable, taking the values “\is{final position}final” and “\is{nonfinal position}nonfinal”, with the latter comprising \is{initial position}initial and \is{medial position}medial positions (see \sectref{sec:2.3.1} and \sectref{sec:6.3.6}). The main results from the analysis of variable clause positions in CCs are summarised in the following three points. A more detailed summary and discussion follows below.

\begin{enumerate}
  \item Sentence-\isi{final position} of subordinate clauses is more common in \is{varieties of English!L1}L1 varieties than in \is{varieties of English!L2}L2 varieties;
  \item sentence-\isi{final position} is more likely in speech; and
  \item there is no systematic general link between the intra-constructional semantic relation (here: \is{concessives (types of)!anticausal}anticausal vs \is{concessives (types of)!dialogic}dialogic) and the arrangement of clauses.
\end{enumerate}

\begin{sloppypar}
Concerning the first result, it was initially hypothesised that \is{varieties of English!L2}L2 varieties would favour subordinate clauses in \isi{final position}. This was based on the view that the \is{final position}final placement of subordinate structures is cognitively optimal, both in terms of \isi{production} and \isi{parsing} (cf. \sectref{sec:2.3.1}). Due to the somewhat less central and secure status of English in \is{varieties of English!L2}L2 varieties, it was argued, the cognitively less complex (perhaps: more natural) patterns would tend to prevail. I suggested that a potential reason for the unexpected inverse pattern may lie in the generally more scholastic \is{language acquisition}acquisition of English in \is{varieties of English!L2}L2 contexts and the predominance of prototypical cases of \is{concessives (types of)!anticausal}anticausal CCs with preposed subordinate clauses in such settings (as foregrounded in grammar books, for instance). However, post-hoc speculations of this type can only be substantiated with an independent research effort and will not be pursued any further here.
\end{sloppypar}

The finding that subordinate clauses are more likely to be placed in \isi{final position} in the spoken mode agrees with the respective hypothesis, even if the effect is generally not large and \is{nonfinal position}nonfinal placement remains the majority variant in many spoken varieties. From the perspectives of both \isi{production} and \isi{processing}, \is{final position}final placement was considered cognitively less demanding than \is{nonfinal position}nonfinal placement (again, see \sectref{sec:2.3.1}), and mechanisms of this kind should of course be all the more effective in speech, due to its transient nature.

The absence of a systematic relationship between intra-constructional semantics and clause position undermines the hypothesis that clausal arrangements in \is{concessives (types of)!anticausal}anticausal CCs should be \is{iconicity}iconic of the semantic relation between propositions (once more, see \sectref{sec:2.3.1} and ‎\sectref{sec:5.3}). This hypothesis seemed particularly appealing, as its confirmation would have provided a plausible link between \is{function and form}functional and formal parameters internal to the construction. However, we see an unsystematic array of patterns across varieties, some supporting, some undermining the hypothesis. In combination with the relatively weak effects for \isi{mode of production}, they leave us with an uneasy feeling regarding clause position as an outcome variable. It was argued that important~– and perhaps central~– factors were not taken into consideration in this study. These could include the discourse-structuring intentions of \is{speaker/writer}SP/W, who may have a certain \isi{theme-rheme} (or \isi{topic-comment}) structure in mind. Thus, structures and the ways in which they present and foreground information have their motivation in the wider discourse context and in \is{speaker/writer}SP/W’s construal of it. Since the present study treated CCs as hermetic (i.e. restricted to exactly two component clauses and their relation), other, possibly central factors must necessarily slip the net of the analysis. These issues and their implications for future research will be discussed further in \sectref{sec:12.2.2}.

Results for the \emph{choice of conjunction} can be summarised in five points, one of them addressing the general picture, the other four commenting on specific factors and their impact on the probability of occurrence of each of the three markers. Note that in this discussion we are still moving through the hierarchy imposed by the \is{constructional choice model}choice model. An alternative, more holistic perspective on markers is provided in \sectref{sec:12.1.3}.

\begin{enumerate}
  \item Controlling for individual factors, \SchuetzlerIndexExpression{although} is generally the most frequent marker; frequencies of \SchuetzlerIndexExpression{though} and \SchuetzlerIndexExpression{even though} are similar to each other, but much lower.
  \item \SchuetzlerIndexExpression[although]{Although} is selected particularly in the written mode, to express \is{concessives (types of)!dialogic}dialogic meaning, and if the subordinate clause is in \isi{nonfinal position}.
  \item \SchuetzlerIndexExpression[though]{Though} is also more frequent in writing (with a weaker effect compared to \SchuetzlerIndexExpression{although}), when \is{concessives (types of)!dialogic}dialogic CCs are expressed, and if the subordinate clause is in \isi{final position}.
  \item \SchuetzlerIndexExpression[even though]{Even though} tends to be selected more often in the spoken mode, to express \is{concessives (types of)!anticausal}anticausal meaning, and when the subordinate clause is in sentence-\isi{final position}.
  \item Compared to \is{varieties of English!L2}L2 varieties, \is{varieties of English!L1}L1 varieties tend to use higher rates of \SchuetzlerIndexExpression{although} (at the expense of \SchuetzlerIndexExpression{though}), but it was argued that this pattern is mostly due to the idiosyncratic pattern found in \il{Indian English}IndE.
\end{enumerate}

The fact that \SchuetzlerIndexExpression{although} responds somewhat more strongly to differences in \isi{mode of production} lends some support to  \citegen{QuirkEtAl1985} claim that it is more \is{formality}formal than \SchuetzlerIndexExpression{though} (see also \citealt{BiberEtAl1999,HuddlestonPullum2002,Aarts1988}). More fine-grained \is{style}stylistic analyses would of course be required to substantiate this further. Associations between semantic types and particular conjunctions have thus far only been explored by \citet{Hilpert2013a} and the author himself (\citealt{Schützler2017,Schützler2018b}). Conflicts between results in the present study and Hilpert’s findings (e.g. concerning the connection between \textit{although} and \is{concessives (types of)!dialogic}dialogic meaning) have been commented on before (e.g. in \sectref{sec:10.3}). On a methodological note, it is intuitively plausible that the strong link between the most frequent semantic type (\is{concessives (types of)!dialogic}dialogic) and the conjunction \SchuetzlerIndexExpression{although} can explain the high \isi{text frequency} of this connective, as seen in \chapref{sec:7}. The case of \SchuetzlerIndexExpression{though} also supports this argument: If we ignore semantics (by controlling for this predictor), this conjunction appears to be of similar frequency as \SchuetzlerIndexExpression{even though}. If, however, we consider that \SchuetzlerIndexExpression{though} also associates strongly with the most frequent type (\is{concessives (types of)!dialogic}dialogic), we have the explanation for its rather high \isi{text frequency} (again, see \chapref{sec:7}).

The conjunction \SchuetzlerIndexExpression{even though} stands out quite strongly from the other two, as it associates with the spoken mode and with the less frequent \is{concessives (types of)!anticausal}anticausal semantics. The semantic dimension explains why this marker has a much lower \isi{text frequency} than \SchuetzlerIndexExpression{although} and \SchuetzlerIndexExpression{though}~– again, this is not apparent from the analyses in \chapref{sec:7}. The association of \SchuetzlerIndexExpression{even though} with speech may be read as weakly supporting the claim that this conjunction has an \is{emphasis}emphatic character (suggested by the adverb \textit{even}), which is sometimes made in the literature. In a vague sense, \isi{emphasis} and more immediate modes of communication are characteristics of speech, rather than writing, but beyond this we can say little about the socio-stylistics of \SchuetzlerIndexExpression{even though}.

With regard to \emph{clause structure}, i.e. the selection of a \is{finite clauses}finite or \is{nonfinite clauses}nonfinite subordinate clause, neither the difference between \is{varieties of English!L1}L1 and \is{varieties of English!L2}L2 varieties nor the intra-constructional semantics of a CC seem to have systematic effects. The most important patterns are associated with \isi{mode of production}, the positions of clauses and the subordinating conjunction, as follows:

\begin{enumerate}
  \item \is{nonfinite clauses}Nonfinite subordinate clauses are more common in writing;
  \item \is{nonfinite clauses}nonfinite clauses are more likely in \isi{nonfinal position}; and
  \item the conjunction \SchuetzlerIndexExpression{though} is most likely (and \SchuetzlerIndexExpression{even though} is least likely) to be followed by a \is{nonfinite clauses}nonfinite subordinate clause.
\end{enumerate}

\begin{sloppypar}
\is{nonfinite clauses}Nonfinite constructions are generally considered to be cognitively more complex since they imply, rather than overtly express, some of the necessary grammatical information. Their somewhat higher frequency in writing therefore comes as no surprise, since time constraints are considerably lower when producing and decoding written language.
\end{sloppypar}

The relationship between (non)finiteness and clause position is less straightforward. On the one hand, an association of \is{nonfinite clauses}nonfinite clauses with \is{nonfinal position}nonfinal positions does not seem ideal because it not only suspends the central (matrix-clause) proposition, but it additionally withholds grammatical information. Thus, \is{addressee/reader}AD/R has to store incomplete material at several levels until the gaps are filled in by the matrix clause. On the other hand, \is{nonfinite clauses}nonfinite structures are typically shorter than (and thus not as heavy as) \is{finite clauses}finite structures. \citegen[1036]{QuirkEtAl1985} notion of \textit{resolution} would in this case predict that heavier (\is{finite clauses}finite) structures should follow shorter (\is{nonfinite clauses}nonfinite) ones. In the present study, no hypotheses were attached to the possible correlation of clause positions and (non)finiteness. However, results suggest that a typical \is{nonfinite clauses}nonfinite CC presents the subordinate clause early and thus follows the principle of (sentence-level) \isi{end-weight}, placing the matrix clause in focus position.

The strong link between the conjunction \SchuetzlerIndexExpression{though} and \is{nonfinite clauses}nonfinite subordinate clauses corresponds to findings by \citet{Hilpert2013a}, while findings concerning \SchuetzlerIndexExpression{even though} (which strongly favours \is{finite clauses}finite clauses) and \SchuetzlerIndexExpression{although} (which is intermediate between the other two) are novel. The detected correlations result in a \is{constructional focusing}focusing of possible formal variants into more precisely defined \is{constructions!subconstructions}subconstructions (see \sectref{sec:12.3.3}). It is the shortest marker (\SchuetzlerIndexExpression{though}) that is most likely to combine with \is{nonfinite clauses}nonfinite (and therefore, on average, also shorter) clauses, and the longest marker (\SchuetzlerIndexExpression{even though}) that is most likely to introduce \is{finite clauses}finite (and thus longer) clauses. As a result, there will be a tendency to get short/\is{nonfinite clauses}nonfinite constructions with \SchuetzlerIndexExpression{though}, long/\is{finite clauses}finite constructions with \SchuetzlerIndexExpression{even though}, and intermediate constructions with \SchuetzlerIndexExpression{although}. While the effects of those correlations will in reality be quite subtle, it is noteworthy that, rather than balancing out the differences, the grammatical system seems to favour types of \is{constructions!subconstructions}subconstructions that are more clearly differentiated in terms of \is{length (of clauses)}length, or \isi{weight}.

\subsection{\label{bkm:Ref82434669}A marker-based summary}\label{sec:12.1.3}

This brief section looks directly at each of the three conjunctions and highlights what contextual, functional and formal parameters they typically associate with. This is partly a \is{semasiological approaches}semasiological view: Instead of asking what forms are typically selected, given a certain set of conditioning factors, it focuses on the typical functions, contexts of use, or concomitant formal characteristics of a given item~– in this case a certain conjunction. This approach does not provide entirely novel insights but inverts the perspective on the results. In effect, it constitutes a brief return to the original point of view in \citet{Schützler2018c}, where the objective was to describe the differences between connectives, taking directly observable surface forms as the starting point for the analysis, without tying them into a more complex system of \is{constructional choice model}constructional choices, as in this volume.

The presentation in \figref{fig:12.1} is based on information given in Figures \ref{fig:10.7}–\ref{fig:10.9} and \ref{fig:11.9}–\ref{fig:11.11}, rescaling it in a standardised way.\footnote{Take, for instance, the affinity of \SchuetzlerIndexExpression{even though} and \is{concessives (types of)!anticausal}anticausal CCs: Inspecting \figref{fig:10.9}, I determined the average rank of the top 36 slots and the average rank of the bottom 36 slots. The resulting values (18.5 and 54.5) were equated with 0 and 1, respectively, the ranks between them were rescaled accordingly, and the actual ranks of scenarios involving \is{concessives (types of)!anticausal}anticausal meanings were then placed on this standardised scale and plotted horizontally in \figref{fig:12.1} for each conjunction. In concrete terms: If all 36 of the scenarios most favourable to the use of \SchuetzlerIndexExpression{even though} were \is{concessives (types of)!anticausal}anticausal in meaning, the symbol “E” would be placed on the very left of the plot. Conversely, if the 36 most favourable scenarios were all \is{concessives (types of)!dialogic}dialogic in meaning, the symbol would be placed on the very right. See the online appendix for details.\label{fn95}} Each level of the plot describes the associations of the three conjunctions with the two levels of a dichotomous variable across all nine varieties that were considered. If a conjunction is placed near the grey vertical line in the centre, it is relatively unresponsive to the respective variable. The further it is placed to the left or to the right, the greater its affinity to the condition indicated in the respective margin.

\begin{figure}
\includegraphics{figures/CCs.Fig.12.1.pdf}
\caption{\label{bkm:Ref80960298}\label{fig:12.1}Association of markers with basic conditions; A~= \SchuetzlerIndexExpression{although}, T~= \SchuetzlerIndexExpression{though}, E~= \SchuetzlerIndexExpression{even though}}
\end{figure}

Concerning the association with \is{varieties of English!L1}L1 and \is{varieties of English!L2}L2 \isi{varieties of English}, \SchuetzlerIndexExpression{although} tends towards the former, \SchuetzlerIndexExpression{though} tends towards the latter, and \SchuetzlerIndexExpression{even though} is very much indifferent to this dimension of variation. However, we saw (in all the relevant plots in \sectref{sec:10.2}) that \il{Indian English}IndE was exceptional in very strongly preferring the conjunction \SchuetzlerIndexExpression{though}. Since this highly erratic pattern did not correspond to any general tendency among \is{varieties of English!L2}L2 varieties, we must treat it (and its effect on the overall picture) with a certain suspicion. As discussed in the respective parts of the analysis, the other results seem more reliable: \SchuetzlerIndexExpression{even though} typically occurs in speech while the other two conjunctions associate with writing; \SchuetzlerIndexExpression{even though} is much more likely if a CC has \is{concessives (types of)!anticausal}anticausal meaning, while both \SchuetzlerIndexExpression{although} and \SchuetzlerIndexExpression{though} are much more common with \is{concessives (types of)!dialogic}dialogic CCs; \SchuetzlerIndexExpression{although} typically introduces subordinate clauses in \isi{nonfinal position}~– the “\isi{grounding} function” discussed by \citet[22]{Altenberg1986}~– while clauses with the other two markers are more likely to follow the matrix clause; and finally, \SchuetzlerIndexExpression{even though} is least likely and \SchuetzlerIndexExpression{though} most likely to combine with \is{nonfinite clauses}nonfinite subordinate clauses, while \SchuetzlerIndexExpression{although} is intermediate in this regard. Evidently, the three markers pattern rather differently for different factors~– affinities between any two of them can be found for individual variables but cannot be generalised. Differences between the three conjunctions are complex and not easy to detect, since they require involved semantic and syntactic analyses, and it is therefore unsurprising that the literature has thus far lacked precise descriptions.

Based on these findings, the three conjunctions can nevertheless be shown in their overall (dis)similarity. To this end, multi-dimensional scaling was applied to the values plotted in \figref{fig:12.1} above. Euclidean distances between the three markers were calculated across the standardised values indicating their affinities to different factors (see Footnote~\ref{fn95} on p.~\pageref{fn95}), and these values were then reduced to coordinates in a two-dimensional space, as shown in \figref{fig:12.2}~– see \citet{Schützler2022} for the technical details involved in this procedure. The plot shows three separate scenarios: One in which all five factors are included (variety status, mode, semantics, clause position and subordinate clause structure; see \figref{fig:12.1} above), and two alternative scenarios in which one or two factors are excluded, as indicated.

\begin{figure}
\includegraphics{figures/CCs.Fig.12.2.pdf}
\caption{\label{bkm:Ref80963762}\label{fig:12.2}Similarities of conjunctions based on associations with predictors; A~=~\SchuetzlerIndexExpression{although},~T~=~\SchuetzlerIndexExpression{though}, E~=~\SchuetzlerIndexExpression{even though}}
\end{figure}

The general arrangement of conjunctions relative to each other is relatively similar, irrespective of whether we base the analysis on the full set of variables or on a subset: \SchuetzlerIndexExpression{although} and \SchuetzlerIndexExpression{though} are somewhat closer to each other, while \SchuetzlerIndexExpression{even though} stands apart. The respective distances for the three-factor scenario are 0.52 between \SchuetzlerIndexExpression{although} and \SchuetzlerIndexExpression{though}, 0.82 between \SchuetzlerIndexExpression{although} and \SchuetzlerIndexExpression{even though}, and 0.84 between \SchuetzlerIndexExpression{though} and \SchuetzlerIndexExpression{even though}, respectively.\footnote{Note that these distances are based on the two-dimensional (potentially reductive) representation, not on the original underlying distances; again, see \citet{Schützler2022}.} The tentative claim made in \chapref{sec:10} to the effect that \SchuetzlerIndexExpression{though} is more “multi-role” in character cannot be upheld from the perspective shown here: It may be true that \SchuetzlerIndexExpression{though} responds less sensitively to certain predictors than the other two conjunctions, but it is by no means positioned between them as regards general correlations with certain factor levels. The functional differentiation of the three conjunctions is of a much more complex and overlapping nature.

\section{Wider contexts of investigation}\label{sec:12.2}\label{bkm:Ref82933818}\label{bkm:Ref80005891}\label{bkm:Ref80607069}

This section outlines a few suggestions concerning potential directions for future work on CCs. Some of these are stock commentaries found at the end of any major book or research article, laying out what could have been done in an ideal world, with no restrictions on resources. Some of them are less promising and will accordingly be discussed rather briefly. Others, however, arise directly from the experience of this particular study and can be understood as serious suggestions for future work. Issues that concern the plausibility of the \isi{constructional choice model} are reserved for \sectref{sec:12.3} below.

\begin{sloppypar}
Two possible expansions are briefly mentioned here, but not discussed at length, because, at least to the author, they seem ambitious beyond the manageable and branch out into domains far more general than English linguistics. The first concerns the discussion of concessives from a cross-linguistic perspective. Comparing the patterns that were found in the present study with patterns in other (Germanic) languages would be very much in the spirit of work by König (e.g. \citeyear{König1988,König1994,König2006}), \citet{Kortmann1996} and \citet{Rudolph1996}, for instance. This kind of undertaking would require historical and cross-linguistic expertise and would thus best be tackled collaboratively. Another aspect that branches out into far more general areas of knowledge concerns a more systematic investigation and categorisation of the \is{topos}topoi at work in \is{concessives (types of)!anticausal}anticausal and \is{concessives (types of)!epistemic}epistemic CCs (cf. \sectref{sec:2.2.1} and \sectref{sec:2.2.2}). This could theoretically be undertaken not only for concessives but also for \is{adverbials!condition}conditional and \is{adverbials!reason}causal relations, since most of the implicational structures will be shared. Such an investigation would shed light on human cognition and the construction of a functional human world. However, knowledge of effects based on causes, results based on actions, or behaviours based on predispositions is psychologically so pervasive and basic, as well as culturally diverse, that it may well prove too vast an object of investigation. Similarly, the precise types of qualification and modification that operate between propositions in \is{concessives (types of)!dialogic}dialogic CCs (cf. \sectref{sec:2.2.3})~– which I called \textit{themes} in this study~– could be investigated more systematically. Like \is{topos}topoi, however, relations of this kind form an open class and establishing an inventory may well turn out to be a Sisyphean task.
\end{sloppypar}

\subsection{\label{bkm:Ref81995769}From constructional subset to complete inventory}\label{sec:12.2.1}

The present study was \is{onomasiological approaches}onomasiological in orientation, as semantic (and extra-linguistic) \is{function and form}functions were treated as primary and formal choices as secondary. However, since the analysis was restricted to CCs involving the three conjunctions \textit{although}, \textit{though} and \textit{even though}, it remains unknown to what extent other means of encoding CCs are employed, for instance prepositional or coordinated constructions. Even more problematically, there are also constructions that cannot even be automatically retrieved from a corpus, since they generate concessive meaning purely from the content of propositions or from the discourse context. The problems involved in more comprehensive approaches to concessives are highlighted by \citeauthor{Hoffmann2005} (\citeyear{Hoffmann2005}: 111; see also \sectref{sec:1.3} above). However, casting the net wider in this way might be successful if the analysis was restricted to a corpus of suitable size~– that is, a corpus large enough to contain a sufficiently wide range of constructions, but small enough for the researcher to essentially read it without recourse to automatic retrieval. Due to the enormous amount of manual work involved, this comprehensive approach would probably need to focus on a limited range of varieties, and it would not generate enough material for meaningful \isi{register} analyses. All of this, however, would depend on the resources that are invested.

Another problem that is faced when looking at all possible \isi{concessive constructions} is that, unlike the complex sentences in the present study, they will in many cases not be syntactically equivalent. For instance, the clausal complements of conjunctions were classified as \is{finite clauses}finite and \is{nonfinite clauses}nonfinite in this book, but this kind of classification does of course not apply to the complements of \isi{prepositions} such as \SchuetzlerIndexExpression{despite} or \SchuetzlerIndexExpression{in spite of}. Other markers introduce their own specific complications, as, for instance, in the use of \SchuetzlerIndexExpression{notwithstanding} as a post- or preposition \citep{Schützler2018a}; \isi{conjuncts} like \SchuetzlerIndexExpression{however} or \SchuetzlerIndexExpression{nevertheless} would need to be treated differently; and “\is{concessives (types of)!universal con\-di\-tion\-al-con\-ces\-sive}universal conditional-concessives” (e.g. \textit{whatever you do}; \textit{however hard we tried}; see \sectref{sec:2.1.1}) also evade the straightforward classifications applied in this study.

Finally, there is the question of syntactic structures that are perhaps not very frequent but quite \is{salience}salient, like the use of certain \isi{correlative markers}, particularly in varieties beyond the \is{varieties of English!Inner Circle}Inner Circle (e.g. \SchuetzlerIndexExpression{although}…\SchuetzlerIndexExpression{but}; cf. \sectref{sec:3.5}). These appear as syntactic hybrids, since they combine coordinating and subordinating markers in a single construction. How exactly to classify those complex connectives~– and whether to treat them as markers in their own right or as variants of existing \isi{subordinators}~– is very much an open question and has potential implications for syntactic theory.

Thus, there is certainly scope for expanding the focus of the present study and aiming at a fuller treatment of \isi{concessive constructions}. The methodological challenges, however, are quite considerable, and the structure of a unified analytical framework would need to be developed along strongly modified or altogether different lines, compared to the present study.

\subsection{\label{bkm:Ref82931998}Expanding the functional dimension}\label{sec:12.2.2}

In a narrower sense, the function of a CC was defined at the interface of the two involved propositions, and it has been variously called a “semantic” and/or “pragmatic” function. The discussion of more complex views of CCs in this section looks beyond the construction but does not touch upon \is{language-external factors}language-external (e.g. socio-\is{style}stylistic) functions (but see \sectref{sec:12.2.4} below).

The somewhat unsatisfactory results concerning clause position (see \chapref{sec:9}) raised the question of whether it is enough to look at the relation between propositions in a CC to predict formal realisations, or whether we should include the wider discourse context to this end. For instance, there was no systematic, cross-varietal link between semantic types and the decision to place a subordinate clause in final or nonfinal position in the sentence. Particularly the hypothesis that the arrangement of clauses should be \is{iconicity}iconic of the semantic relation between propositions was not supported by the data. The general arrangement of component clauses might in fact be more systematically conditioned by content preceding or following the actual CC in question. For instance, does the proposition in one of the component clauses relate to propositions or arguments found earlier or later in the discourse, and is the respective clause therefore placed in proximity to those points of reference to increase textual \isi{cohesion} and facilitate \isi{planning} and \isi{processing}? In order to address this issue we would in many cases have to look quite some distance to the left and right of a sentence to find clues that link the wider discourse to the respective CC. We would also need to categorise different kinds of anticipation in what precedes, different kinds of elaboration in what follows, their interactions, as well as instances in which no obvious discourse connection can be found. In other words: In addition to the intra-constructional relations identified in this study, we would need similar relations that apply to the wider discourse. Apart from being challenging at the coding stage, any such expansion would considerably increase the complexities of statistical models~– that is, if a quantitative approach is still considered feasible under these circumstances in the first place. It was precisely for reasons like these that a discourse-analytic component was not included in the present study (see \sectref{sec:1.1}).

Even if we stay at the level of intra-constructional semantic/pragmatic functions as operationalised in this study, semantically \is{ambiguity}ambiguous constructions (see \sectref{sec:3.4}) could be explicitly addressed in the analysis, for example via the addition of levels to the predictor variable \textsc{type} (cf. \sectref{sec:6.3.6}). However, before this is considered, the \is{concessives (types of)!epistemic}epistemic type should be re-included, in spite of its relatively low overall frequency.

Thus, there is much that could be done concerning the expansion of the functional side of CCs. Like the expansion of candidate constructions discussed in \sectref{sec:12.2.1}, however, putting these ideas into practice would in many cases involve a considerable reworking of the analytic framework, particularly concerning quantitative methods.

\subsection{The diachronic dimension}\label{sec:12.2.3}\largerpage

Section \ref{sec:2.1} provided the general historical background for the concessive class of adverbials in general and the conjunctions \textit{although}, \textit{though} and \textit{even though} in particular, but the present study did not actively engage with the \is{diachronic approaches}diachronic dimension of variation. The issues in filling this gap are once again mainly to do with the availability of data and the amount of manual coding and \is{disambiguation}disambiguation involved in the analysis. Diachronic work could make valuable contributions on several counts, as sketched in the following paragraphs.

One \is{diachronic approaches}diachronic research question could concern the relatedness and development of semantic types of concessives. In particular, the more or less implicit treatment of \is{concessives (types of)!anticausal}anticausal CCs as somehow primary or prototypical and the associated notion that \is{concessives (types of)!epistemic}epistemic and \is{concessives (types of)!dialogic}dialogic CCs are derived from them (cf. \citealt{Hilpert2013a,Sweetser1990}) merit closer inspection. For instance, are \is{concessives (types of)!dialogic}dialogic CCs only notionally “later” than \is{concessives (types of)!anticausal}anticausal ones, in the sense that a “pragmaticalised” function is conceptualised as derived from a notionally more “logical” one? Or can we actually \textit{show} that they appear later in the history of English? Further, if such \is{diachronic approaches}diachronic processes can be traced: Do \is{concessives (types of)!epistemic}epistemic CCs take an intermediate position, or do they play some other role? \citegen{Schützler2018b} study of \il{American English}AmE finds some evidence that all three conjunctions were more likely to carry \is{concessives (types of)!dialogic}dialogic meaning in the late 20\textsuperscript{th} century, compared to the late 19\textsuperscript{th} century. However, the statistical approach that was used is unlikely to stand the test of more rigorous methods, and only relatively weak tendencies were found. If confirmed, the \is{diachronic approaches}diachronic derivation of \is{concessives (types of)!epistemic}epistemic meanings from \is{concessives (types of)!anticausal}anticausal meanings could be interpreted as a case of \isi{subjectification}, because it results in a greater visibility of the active reasoning and inferencing of \is{speaker/writer}SP/W. This is remotely related to the well-known development of \is{modality}modal constructions from deontic to epistemic meanings, as discussed by \citet[91]{Krug2000} in the context of \isi{grammaticalisation}, for instance. The development of (putatively \is{intersubjectivity}intersubjective) \is{concessives (types of)!dialogic}dialogic CCs would then be yet another step away from purely content-oriented readings. These possible \is{diachronic approaches}diachronic trajectories would need to be investigated in new research efforts, however. Contra the notion of a diachronically increasing number of \is{concessives (types of)!dialogic}dialogic CCs, \citeauthor{Burnham1911} (\citeyear{Burnham1911}: 33; cf. Footnote~\ref{bkm:Ref82082041} on p. \pageref{bkm:Ref82082041}) suggests that it was quite common for \ili{Old English} to use \SchuetzlerIndexExpression[þéah/þéh]{þéah}~– the predecessor of \SchuetzlerIndexExpression{though}~– in a \is{concessives (types of)!dialogic}dialogic, quasi-\is{adverbials!contrast}adversative function (although Burnham does of course not use these terms). The secondary/derived status of the \is{concessives (types of)!dialogic}dialogic type can therefore not be taken for granted~– on the contrary, it is not only possible that \is{concessives (types of)!dialogic}dialogic CCs have for a long time coexisted with \is{concessives (types of)!anticausal}anticausal (and \is{concessives (types of)!epistemic}epistemic) CCs, but they may actually have been the dominant (because more general) type to start with.\largerpage

Another \is{diachronic approaches}diachronic question concerns the changing fates of different markers concerning their availability. Frequency changes can shed light both on the \isi{grammaticalisation} status of conjunctions and their \is{style}stylistic values. Like the present study, such efforts would ideally use a complex framework that takes \is{function and form}functional and formal parameters into account and thus goes beyond the mere measuring of \is{text frequency}text frequencies, as in \citet[165]{Schützler2018c}, for example.

\subsection{\label{bkm:Ref81829707}More varieties and predictors?}\label{sec:12.2.4}

The present study includes data from nine different \isi{varieties of English}. On the whole, differences between those varieties were relatively slight or unsystematic. Including more varieties in follow-up studies would probably not shed more light on general differences between, say, inner-circle and outer-circle varieties, as those differences are apparently not particularly pronounced with regard to the phenomenon under investigation. Further, expanding the range of varieties would likely reveal more instances of idiosyncratic patterns that are hard to account for. Against this background, the inclusion of further (\is{varieties of English!L2}L2) varieties seems warranted only if there are specific, theoretically motivated expectations attached to those particular varieties. Such an expansion would then require a more careful consideration of the sociolinguistic realities in the respective territories. On the whole, however, CCs are probably not particularly \is{salience}salient and therefore play a relatively minor role as variety-based \is{social identity}identity markers, as argued in \sectref{sec:12.1} based on general patterns mostly characterised by inter-varietal similarity. However, the absence of US-\ili{American English} from the investigated set of varieties constitutes a regrettable gap~– as explained in \sectref{sec:6.1}, this is due to the incomplete status of ICE-USA. Rather than including more \is{varieties of English!L2}L2 varieties, a systematic comparison of \il{British English}BrE and \il{American English}AmE (i.e. English in the USA) might therefore be a valuable contribution. This would then need to be based on corpora beyond those from the ICE-family.

This study treated speech and writing as macro-\is{style}stylistic categories and also considered what is involved in their \isi{production} and \isi{processing}. Getting a better idea of the \is{style}stylistic value of different concessive markers would require a more fine-grained inspection of \is{register}registers (or genres). ICE-corpora, however, are too small to investigate \is{style}stylistic variation in detail, particularly if the construction is of medium frequency and is given a complex definition with multiple formal parameters, as in the present study. Distinguishing production-and-\isi{processing} effects from truly \is{style}stylistic effects will not always be easy and probably requires a careful operationalisation of \isi{genre}. Truly social factors might also be of interest: If individual markers respond to differences in \isi{register}~– as claimed in some of the literature~– they may also vary systematically between groups of speakers and writers. Once more, investigating this dimension of variation either requires corpora that include the appropriate \isi{metadata}, or the adoption of more controlled (e.g. \is{experimental approaches}experimental) methodologies. On the whole, however, there seem to be several aspects of CCs more deserving of closer inspection than their precise sociolinguistic behaviour. Some have been outlined in this section as well as in \sectref{sec:12.2.1} and \sectref{sec:12.2.2} above, others will be discussed in the next section.

\section{\label{bkm:Ref82691473}Concessives and Construction Grammar}\label{sec:12.3}

In this study, CCs were conceptualised as a hierarchically organised system of choices. This section evaluates the success of the approach, suggests alternative approaches, and discusses theoretical implications.

\subsection{\label{bkm:Ref81558829}Constructions as hierarchical choices}\label{sec:12.3.1}

The hierarchical view of CCs in the \is{constructional choice model}choice model introduced in \sectref{sec:4.1.3} means that we take a top-down perspective on constructions, with more general, broadly defined constructions at the top and more fully specified constructions at the bottom. This can be incorporated into an even more general hierarchy with three levels, the second and third of which comprise the choice model as implemented in this study:
(i)~\is{language-external factors}language-external function,
(ii)~non-situational (or perhaps, language-internal) function, and
(iii)~form.

In the present study, \is{language-external factors}language-external, socio-\is{style}stylistic (or contextual) factors include the varieties themselves~– perhaps grouped into \is{varieties of English!L1}L1 and \is{varieties of English!L2}L2~– and the two \is{mode of production}modes of production, writing and speech. Even if they are non-linguistic, factors like these can broadly be classified as functional: For instance, \is{speaker/writer}SP/W may consciously or unconsciously wish to flag up their association with a certain variety, and the selected formal realisations will then serve that function. \is{mode of production}Mode of production cannot be captured in exactly the same terms, as we can hardly claim that \is{speaker/writer}SP/W feels the need to express the fact that they are speaking (or writing). However, speaking and writing are clearly functions (or uses) that language can be put to, and they place certain constraints on how things are expressed. The consequent formal choices will thus again serve a higher function, namely making writing or speech work for both \is{speaker/writer}SP/W and \is{addressee/reader}AD/R. The relevant mechanisms can be viewed as production- or processing-related or as macro-\is{style}stylistic (cf. \sectref{sec:4.2}). It is crucial to bear in mind that, as language-external factors, both variety and mode can potentially inform all lower-ranking properties of a construction.

One level below the two \is{language-external factors}language-external parameters there are functions that cannot easily be linked to the situation or context in which language is produced. In the present study, only the semantic or pragmatic relationship between propositions within a CC was included at this level. Other functions of this kind, applicable to other constructions, could be found in the domain of \isi{modality} (e.g. obligation meanings of different strengths), in other \is{adverbials}adverbial domains (e.g. \is{adverbials!time}temporal relations of different kinds) or in slight differences between semantic roles (e.g. different kinds of possessor-possessum relationships). Such functions have in common that, while they do of course refer to some real-world situation or some relation external to the linguistic form, there is no immediate link with the situation (or the conditions) under which language is produced. An indicator for the identification of such intermediate functions may be the question of whether they are rooted in the socio-\is{style}stylistic or cognitive characteristics of the situation or directly linked to what \is{speaker/writer}SP/W wishes to express. In the present study, for example, constructing an \is{concessives (types of)!anticausal}anticausal CC is not motivated socio-stylistically but from a specific message that needs to be conveyed.

\chapref{sec:8} explored the notion that the number of CCs of different semantic types may vary across \is{mode of production}modes of production and different \isi{varieties of English}. However, the general view taken of CCs in Chapters \ref{sec:9}, \ref{sec:10} \& \ref{sec:11} was that the construction proper only begins at the intermediate, message-related level of the semantic \is{function and form}function, which then finds expression in the various possible formal realisations. The construction is thus implicitly treated as context-free: If we look at a CC in isolation, we can identify its internal semantic make-up, the arrangement of clauses, the conjunction that is involved, as well as the structure of the subordinate clause. Delimiting a CC in this way results in what was called a \textit{hermetic} view (see \sectref{sec:4.1.3}): All parameters relevant for the analysis of the construction can be recovered from its propositional content and its form.

The formal level is then comprised of
(i)~clause position,
(ii)~a marker, and
(iii)~the syntactic structure of the subordinate clauses. These properties not only rank lowest in the general hierarchy (\textsc{extralinguistic function}~→ \textsc{semantic function}~→ \textsc{form}), but they can also be ranked internally. In this study, I took the view that the primary, highest-level decision concerns where to place the component clauses, which are, after all, the largest elements involved. Next, the link between the two clauses (i.e. the conjunction) was given precedence over the formal realisation of the subordinate clause, which was motivated from traditional grammatical thinking whereby the clause depends upon (or complements) its conjunction.

If we accept the notion that \isi{form follows function}, we can still question the assumption that certain formal properties are conditioned by others in a unidirectional (or hierarchical) way. In other words: Can we even identify formal properties of different ranks, or should we treat form as a single, multi-dimensional component of a construction? In the context of the present study, for instance, is it reasonable to frame the dependency as \textsc{position}~→ \textsc{marker}~→ \textsc{clausal~structure}? Or would a partly different arrangement, or the rejection of any hierarchy, be more plausible? These issues will be discussed in the following section.

\subsection{\label{bkm:Ref80788003}Alternative models of constructional variation}\label{sec:12.3.2}

A critique of the \is{constructional choice model}choice model of \isi{constructional variation} (cf. \sectref{sec:4.1.3}) can be based on  a particular view of \is{usage-based approach}usage-based \is{Construction Grammar}CxG, as outlined in \sectref{sec:4.1.2} and illustrated in \figref{fig:4.1}. The \isi{schema} introduced there involves the three formal properties of CCs along with the semantic/pragmatic dimension. These parameters are all put on an equal footing, as shown by the lines that establish all possible cross-connections. In this model, it does not seem contradictory to assume that it is the semantic \is{function and form}function that drives formal variation and the establishment of typical formal patterns: Meaning is primary and needs to be formally expressed, and we therefore have an unavoidable function-to-form hierarchy (see \sectref{sec:12.3.1} above). However, a \is{usage-based approach}usage-based assumption would be that certain formal correlations (e.g. more instances of \SchuetzlerIndexExpression{although} in \isi{nonfinal position}; cf. \chapref{sec:10}) become cognitively strengthened simply through their co-occurrence. That is, instead of being guided by some mechanism that works its way through the different formal layers in a top-down fashion, \is{speaker/writer}SP/W intuitively accesses all relevant formal levels at the same time, producing a formally complex construction whose internal dependencies (apart from \textsc{function} → \textsc{form}) are not even theoretically relevant. The contrast between the two views is shown in \figref{fig:12.3}. In the hierarchical model in panel~(a), fully \isi{language-external factors} like \textsc{variety} and \textsc{mode} as well as semantic or pragmatic factors have an impact on all formal aspects of a CC. Concerning formal parameters, however, the model implies that \is{speaker/writer}SP/W has stored inventories of likely realisations at different levels of granularity, from a general syntactic grid of sequentially ordered component clauses via the selection of a \is{subordinators}subordinator to the eventual realisation of the subordinate clause as \is{finite clauses}finite or \is{nonfinite clauses}nonfinite. These could then be called \textit{subconstructions}, with more specific ones nested in more general (or \is{schematicity}schematic) ones. In panel~(b), on the other hand, external and semantic/pragmatic factors have a direct impact on a single (if still multi-faceted) formal choice.

\begin{figure}
\includegraphics[width=9cm]{figures/CCs.Fig.12.3.pdf}
\caption{\label{bkm:Ref81509266}\label{fig:12.3}Hierarchical (a) vs holistic (b) views of formal dependencies in CCs}
 \end{figure}

As a model of real-time language \isi{production}, the model in panel~(a) of \figref{fig:12.3} seems less efficient, as it suggests that \is{speaker/writer}SP/W accesses the different layers of a construction in a sequential way, going through a chain of decisions. Perspective~(b), on the other hand, is more economical and therefore plausible, since all formal properties are directly accessed in bulk. However, even in this holistic view the internal relation between formal properties still needs to be established~– we still want to take a look into the black box that contains \textsc{position}, \textsc{marker} and \textsc{clause} in panel~(b) in order to find out how its content is patterned.

To resolve the conflict between the two panels in \figref{fig:12.3}, I will argue that they simply make two different contributions to answering the same question, namely: What are the typically expected formal properties of a CC, given a particular semantic function and context of production? The two components of \figref{fig:12.3} approach this issue from two perspectives. Panel (a) represents a particular view of how \is{constructions!subconstructions}subconstructions are organised at the formal level, proceeding from higher-level, general syntactic grids to the more local properties of sentence-internal linkage and the structure of embedded clauses. This schematic and idealised view is directly aligned with the quantitative analyses in this book, based on regression models that become increasingly complex as we move from the most general and \is{schematicity}schematic \is{constructions!subconstructions}subconstructions to more fully specified ones. Panel (b), on the other hand, establishes a more direct link between functions (both extra- and intra-linguistic) and forms and treats the latter as more unitary. This reflects that \is{speaker/writer}SP/W does of course make a single choice when encoding a CC in a particular situation.

The three equations shown in (\ref{eq:12.1}) return to the syntax of statistical models that were used in Chapters \ref{sec:9}, \ref{sec:10} \& \ref{sec:11}, in order to further illustrate the hierarchical thinking that was applied. Note that several variables are given more general names here (e.g. \textsc{semantics}, \textsc{position} and \textsc{clause}), and that random parts are not restated, because they are irrelevant for the discussion at hand. The logic of these related models is that the higher-level outcomes \textsc{position} and \textsc{marker} become predictor variables at lower levels. In a sense, \isi{constructional variation} (involving three formal parameters) is operationalised as three separate alternations, with a single formal parameter as the outcome in each case.

\ea
\label{bkm:Ref81816324}\label{eq:12.1}Statistical models in the hierarchical perspective
\begin{lstlisting}
position ~ mode * semantics
marker   ~ mode * (semantics + position)
clause   ~ mode * (semantics + position + marker)
\end{lstlisting}
\z

Above, it was argued that the two components of \figref{fig:12.3} merely take two perspectives on essentially the same question, and that the hierarchical view breaks the holistic view up into more manageable units (i.e. binary or ternary alternations) but otherwise serves the same purpose. The three formulations in (\ref{eq:12.1}) show that this is not strictly true. For instance, \textsc{position} can impact upon \textsc{marker}, and \textsc{marker} can impact upon \textsc{clause}, but not vice versa. Based on this design and on results shown in \chapref{sec:11}, we can argue that using the conjunction \SchuetzlerIndexExpression{though} makes a \is{nonfinite clauses}nonfinite subordinate clause more likely, but it does not follow that using a \is{nonfinite clauses}nonfinite subordinate clause makes using the conjunction \SchuetzlerIndexExpression{though} more likely. An ideal model, however, should perhaps accommodate both views: Neither does \is{speaker/writer}SP/W select a certain type of clause on the basis of a certain marker, nor is the marker selected on the basis of a clause type, but the two of them are selected together. Not only can we question the exact hierarchy of formal levels, but we can question the very idea of hierarchies. What, then, would be the methodological consequences of a truly holistic perspective concerning the formal side of CCs? The three equations in (\ref{eq:12.2}) show options that will be discussed in more detail below.

\ea
\label{bkm:Ref81816371}\label{eq:12.2}Possible models for the holistic perspective
\begin{lstlisting}
form                       ~ mode * semantics
form_func                  ~ mode
position | marker | clause ~ mode * semantics
\end{lstlisting}
\z\largerpage

The outcome variable \textsc{form} in the first model has an exceptional structure: Its levels correspond to all twelve possible discrete combinations of clause~positions~(×2),~conjunctions (×3) and clause structures (×2). Using only extra-lin\-guis\-tic and semantic predictors, probabilities of these outcome variants could theoretically be predicted using a \is{regression!multinomial}multinomial model, but this is indeed no more than a theoretical possibility: Ternary outcomes are already difficult enough to handle (see supplementary materials for \chapref{sec:10}; see also \citealt{FahyEtAl2022}), and analyses with more than three outcome categories (as in \citealt{SchützlerHerzky2021}) are very much the exception. However, based on such models we could easily focus on \is{constructions!subconstructions}subconstructions at a more general level, for instance by comparing the cumulated proportion of all formal variants that involve the marker \textit{though} to the cumulated proportion of formal variants involving the other two conjunctions, or by comparing the cumulated proportion of all variants with subordinate clauses in \isi{final position} to the respective value for clauses in \isi{nonfinal position}.{\interfootnotelinepenalty=10000\footnote{Note that formal parameters subsumed into the outcome variable~– here, as well as in the count models discussed below~– have to be categorical; continuous characteristics (like clause \is{length (of clauses)}length) must either be excluded or converted into categories.}}

To a similar effect, a count model (cf. Chapters \ref{sec:7} \& \ref{sec:8}) could be employed to measure the rates of occurrence of variants, to be then converted into proportions or percentages~– see, for instance, the approach in \citet{Schützler2022}. The respective (theoretical) outcome variable is labelled \textsc{form\_func} in the second line of (\ref{eq:12.2}): We need to include the functional (i.e. semantic) dimension in the outcome that is counted, because only properties of text units can be used as predictors. This approach is computationally easier to handle than the complex \is{regression!multinomial}multinomial model sketched above, but it should in principle yield the same results. This, too, comes with its own complications, however. For instance, counts need to be established for each outcome category in each text. Given the number of \textit{n}~=~4,902 texts in the present study, we would need \textit{n}~=~24 observations per text (12~forms~× 2~semantics), blowing the dataset up to \textit{n}~=~117,648 observations.\footnote{Compare the count models in Chapters \ref{sec:7} \& \ref{sec:8}, which had “only” \textit{n}~=~14,706 and \textit{n~}=~19,608 observations, respectively, as described in Appendices~\ref{appendix:B.1} \&~\ref{appendix:B.2} and documented in the published data \citep{Schützler2021}.} In analogy to the approach described for the \is{regression!multinomial}multinomial model, typical \is{constructions!subconstructions}subconstructions can be captured by summing up the estimated rates for specific outcomes across contrasting broader categories (e.g. final vs nonfinal; \textit{although} vs \textit{though} vs \textit{even though}).

A third option is the application of regression modelling with a \is{multivariate outcomes}multivariate outcome (e.g. \citealt{AfifiEtAl2020,JohnsonWichern2007}), as indicated in the third equation in (\ref{eq:12.2}): Separate regressions are formulated for each formal parameter (in this case: \textsc{position}, \textsc{marker} and \textsc{clause}), probably using identical predictor structures for theoretical reasons. However, the model not only contains information about the relations between outcomes and predictors, but also knows about the correlations between levels of the three outcomes.

Finally, an approach that does not strictly fall into the domain of regression is \isi{Structural Equation Modelling} (e.g. \citealt{Hoyle2012,Kaplan2009}). Among other things, the flexible and powerful procedures that it provides can account for correlated dependents and complex interrelationships between all variables involved. However, a fuller discussion of these techniques cannot be provided here.

The alternative modelling strategies sketched in this section have in common that they target \is{constructions!subconstructions}subconstructions not in a hierarchical way~– as shown schematically in (\ref{eq:12.1})~– but holistically. The relative benefits of such techniques remain to be tested. Their discussion, however, highlights that the present study has proposed merely one particular perspective on \isi{constructional variation}, which must not be taken as the final word but as a point of departure for future approaches to \isi{constructional variation} (and, perhaps, change). In particular, such approaches would reconcile the view of constructions as formally holistic with the fact that they can also break down into syntactically more \is{schematicity}schematic (or, less fully specified) \is{constructions!subconstructions}subconstructions. Notionally, the holistic view of constructions was already a driving force behind the analyses in the present volume. Quantitatively, however, it could be implemented more rigorously in future research.

\subsection{\label{bkm:Ref82082423}Processes in constructional variation and change}\label{sec:12.3.3}

Beyond fundamental quantitative issues involved in the analysis of formally and functionally complex constructions, future research should take a closer look at the relation between function and form. What, for instance, defines a \is{constructions!subconstructions}subconstruction? What specific processes can be involved in what we popularly refer to as \textit{constructionalisation}? And, again, how can we support our analysis of such processes using our quantitative toolkit? This section makes some suggestions concerning these points. The focus is on the constructions that were the topic of this volume (CCs), but the notions that are developed have wider applicability.

A basic, common-sense assumption is that the relationship between \is{function and form}functional and formal variation is not random but structured. There is at the very least a tendency for certain (e.g. semantic) functions to be expressed using certain (e.g. syntactic) forms. This kind of correlation can be extended to the relations between different formal parameters as well, as explained earlier. While these notions are in fact relatively theory-neutral and do not in themselves advance \isi{Construction Grammar}, their theoretical relevance is strengthened if we link them to \isi{production} and \isi{processing}. Both are assumedly facilitated if a certain function correlates with distinct formal properties; conversely, they are made more difficult if the relation between function and form is fuzzy and unsystematic. In other words: If there is a lot of overlap between the formal means used to express different functions, decoding a message will be cognitively more challenging. In the study at hand, the simplified functional view focused on \is{concessives (types of)!anticausal}anticausal and \is{concessives (types of)!dialogic}dialogic meanings. In this context, \isi{constructionalisation} would consist of a tidier mapping of particular formal properties onto each of the two semantics. Measuring the discreteness of form-function patterns would then be one concern of quantitative \isi{Construction Grammar}.

The concept that is introduced below will be called \is{constructional focusing}\textit{constructional focusing}. It was also used in \citet[124–128]{Schützler2018c}, alongside two other concepts, “standardised constructional difference” and “constructional specialisation”. As compared to the present study, the original framing of \isi{constructional focusing} is somewhat problematic: On the one hand, the terms “focusing” and “specialisation” are relatively similar; on the other hand, the earlier study was \is{semasiological approaches}semasiological (form-driven), not \is{onomasiological approaches}onomasiological (function-driven), in outlook. I would suggest that it is more efficient to use a single concept developed strictly from the perspective of functions.

\figref{fig:12.4} draws a schematic comparison between relatively \is{constructional focusing}\is{constructions!unfocused}unfocused (or “\is{constructions!diffuse}diffuse”) and relatively \is{constructional focusing}\is{constructions!focused}focused constructions. It suggests that function and form are continuous dimensions. While this is theoretically true, functional and formal \textit{categories} are more likely to be used in practice, as in the present volume.

\begin{figure}
\includegraphics[width=10cm]{figures/CCs.Fig.12.4.pdf}
\caption{\label{bkm:Ref502692541}\label{fig:12.4}Diffuse and focused macro constructions}
 \end{figure}

In \figref{fig:12.4}a, instances of a previously defined \is{constructions!macro constructions}macro construction (e.g. CCs) are scattered rather randomly across the available \is{function and form}functional and formal space. As semantic properties vary (horizontally), formal properties also vary (vertically), but not in a particularly systematic way: The same formal means are available when encoding different functions, and the function-to-form mapping is therefore relatively \is{constructions!diffuse}\textit{diffuse}. In \figref{fig:12.4}b, on the other hand, variation is much more structured: \is{function and form}Functions on the left are expressed by formal means not available to functions on the right to the same extent. A large number of form-function pairings are still theoretically possible (and will therefore appear in data), but the probabilities are high for certain combinations and low for others, and formal overlap between different functions is much more limited. We would therefore speak of a macro construction that has \is{constructions!focused}focused into relatively distinct subconstructions.

This concept of \is{constructional focusing}focusing nicely dovetails with the notion that constructions (and thus, \is{constructions!subconstructions}subconstructions) are certain combinations of form and function that are processed, stored and produced by language users to generate meaning. If, in an exemplar-based representation, certain parameters at different levels combine more often than others, they are strengthened and will be cognitively more readily accessed, as discussed in \sectref{sec:4.1.2}. This would apply both to constructions with fairly general, productive grammatical properties (e.g. CCs) and more idiosyncratic and unpredictable constructions like the covariational-conditional construction discussed in \sectref{sec:4.1.1}, for instance.

In quantitative terms, measuring \isi{constructional focusing} would require the kind of holistic approach outlined in \sectref{sec:12.3.2}. It would perhaps be most interesting to trace \is{diachronic approaches}diachronic changes in \is{constructional focusing}focusing and thus the emergence of subcon\-struc\-tions, but variation may well exist between national \isi{varieties of English} or \is{genre}genres, too. In cognitive terms, changing (or systematically varying) degrees of \isi{constructional focusing} would shed light on how language users classify and store mental (e.g. semantic/pragmatic) categories, and how they relate them to linguistic forms.

\section{\label{bkm:Ref82929985}In conclusion}\label{sec:12.4}

The constructions under investigation in this book turned out to be relatively stable across varieties as regards the general constraints that regulate them; speech and writing, on the other hand, have a greater impact on formal variation. On the whole, however, CCs seem to be most interesting if we look at them with a focus on the coherence and interplay of intra-constructional \is{function and form}functional and formal parameters. Semantics, clause positions, the specific conjunctions themselves as well as the syntactic realisations of subordinate clauses~– all of these interact in a systematic way and tend to form a structured set of \is{constructions!subconstructions}subconstructions within the broader class of CCs. The precise relationships between \is{function and form}functional and formal facets in these and other (possibly rather different) constructions deserve even closer attention in the future, and research efforts of this kind can contribute to the development of new theories and quantitative methods in \isi{Construction Grammar}. The present volume has provided a few pointers in this direction.

