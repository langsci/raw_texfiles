\chapter{\label{bkm:Ref466898909}\label{bkm:Ref3457867}Previous findings and research questions}\label{sec:5}

This chapter sets out by reporting previous research on the concessive constructions that are the focus of the present study (\sectref{sec:5.1}). The content is arranged in sections that roughly correspond to the sequence of the analytical chapters. Each section proceeds from short summaries of what is reported in the major standard reference grammars of English (\citealt{QuirkEtAl1985,BiberEtAl1999,HuddlestonPullum2002}) to discussions of empirical research.\footnote{Two studies by \citet{Burnham1911} and \citet{Quirk1954} are not included in this chapter because they deal exclusively with concessives in \ili{Old English}.} Section \ref{sec:5.2} summarises findings in the literature and highlights research gaps (see, in particular, \tabref{tab:5.1}), while \sectref{sec:5.3} formulates the concrete research questions and expectations that will be explored quantitatively. These build upon the broader questions formulated in \sectref{sec:1.3}; their late appearance is due to the fact that they depend on the content of Chapters \ref{sec:2} \& \ref{sec:4}, as well as the research background summarised in this chapter.

\section{\label{bkm:Ref74132044}Previous research}\label{sec:5.1}

The sequence of content in \sectref{sec:5.1.1}–\ref{sec:5.1.4} roughly parallels the sequence of Chapters \ref{sec:7}–\ref{sec:11}, but direct comparability is in many respects limited: While the main analyses in this book apply \is{regression!negative binomial}negative binomial, binary \is{regression!logistic}logistic or \is{regression!multinomial}multinomial \is{regression!mixed-effects}mixed-effects regression models (informed by the \isi{constructional choice model} introduced in \sectref{sec:4.1.3}), research in the literature is in most cases based on more basic statistical approaches. That is, \is{text frequency}text frequencies or percentages/proportions are established, but neither \isi{nested data} structures nor the interrelatedness of different factors are considered. Moreover, the literature provides hardly any information concerning syntactic types of subordinate clauses, beyond a comment in \citet{HuddlestonPullum2002}. Some of the reported earlier findings will therefore need to be treated with some reservation.

\subsection{\label{bkm:Ref35511425}\label{bkm:Ref35777602}Frequencies of conjunctions}\label{sec:5.1.1}

The three current major standard grammars of English have rather similar views on the three conjunctions under investigation in this study. \citet[1097–1099]{QuirkEtAl1985} regard \SchuetzlerIndexExpression{although} and \SchuetzlerIndexExpression{though} as the central markers of concession, the latter being “more \is{informality}informal”. \SchuetzlerIndexExpression[even though]{Even though} is treated as an \is{emphasis}emphatic variant, with the modifier \textit{even} “expressing unexpectedness” (\citealt{QuirkEtAl1985}: 1099). Concerning the basic equivalence of and subtle \is{style}stylistic difference between \SchuetzlerIndexExpression{although} and \SchuetzlerIndexExpression{though}, both \citet[845]{BiberEtAl1999} and  \citet[736]{HuddlestonPullum2002} largely agree with \citet{QuirkEtAl1985}. Biber et al.’s \is{style}stylistic evaluation of the two conjunctions is based on frequency differences indicating that \SchuetzlerIndexExpression{though} (including \SchuetzlerIndexExpression{even though}) is more frequent in conversation and fiction, while \SchuetzlerIndexExpression{although} is more frequent in \is{academic language}academic writing (\citealt{BiberEtAl1999}: 842). They further argue that in preferring \SchuetzlerIndexExpression{although} to \SchuetzlerIndexExpression{though} in \is{formality}formal \is{genre}genres, language users may be influenced by the homonymous \is{conjuncts}conjunct \is{though@\textit{though}!as conjunct}\textit{though} (as in \textit{He’s quite old, though.}), which is perceived as \is{informality}informal or colloquial (\citeyear{BiberEtAl1999}: 846; see \citealt{Schützler2020a}). Interestingly, both \citet{BiberEtAl1999} and \citet{HuddlestonPullum2002} treat \textit{even though} as an intensified variant of \textit{though}, rather than a distinct conjunction in its own right. \citet[821]{BiberEtAl1999} further comment that concessive clauses in general~– i.e. irrespective of the connective that is used~– occur predominantly in written language, which is unsurprising in view of the correlation between writing and the use of complex sentences (see \sectref{sec:4.2}). On the basis of the information provided by the three major standard grammars of English, one would thus expect \SchuetzlerIndexExpression{although} and \SchuetzlerIndexExpression{though} to be the most frequent of the three conjunctions, and one would further expect them to be characterised by different \is{style}stylistic distributions, even between the two basic \is{mode of production}modes of production, speech and writing.

\citet{Altenberg1986} presents a study that provides useful descriptive detail on several markers of concession, both in terms of their frequencies in spoken and written \il{British English}BrE and their syntactic positions (concerning the latter, see \sectref{sec:5.1.3} below). His work is based on 100,000 words each from the \textit{London-Lund Corpus of Spoken English} (\is{corpora!LLC}LLC, \citealt{SvartvikQuirk1980,GreenbaumSvartvik1990}) and the written \is{corpora!LOB}\textit{Lancaster-Oslo/Bergen Corpus} (LOB; \citealt{JohanssonEtAl1978}). Reproducing only results for the three conjunctions relevant in this study, \figref{fig:5.1} shows frequencies in \is{corpora!LLC}LLC and \is{corpora!LOB}LOB as well as differences in frequencies between writing and speech, represented as ratios. Raw frequencies are shown in the table on the right of the figure; particularly concerning \textit{even though}, data are sparse and results need to be interpreted with due caution.

\begin{figure}
\includegraphics{figures/CCs.Fig.5.1.pdf}
\caption{\label{bkm:Ref471492512}\label{fig:5.1}Frequencies of concessive conjunctions in spoken and written BrE \citep{Altenberg1986} }
\end{figure}

In both the spoken and the written material, there is a substantial frequency gap between \SchuetzlerIndexExpression{though} and \SchuetzlerIndexExpression{although} on the one hand and \SchuetzlerIndexExpression{even though} on the other. All three conjunctions are more frequent in writing; \SchuetzlerIndexExpression{although} is most sensitive and \SchuetzlerIndexExpression{even though} is least sensitive in this regard.\footnote{Note that \citet{Altenberg1986} quantifies the written-spoken difference not as a ratio but using a different index. Further, he does not make any claims concerning \SchuetzlerIndexExpression{even though} due to its low frequency in his data.}

Only slightly later than Altenberg, \citet{Aarts1988} investigates the frequencies of \SchuetzlerIndexExpression{although}, \SchuetzlerIndexExpression{though} and \SchuetzlerIndexExpression{even though} (as well as other concessive markers) based on a sample of 305,000 words from 12 genres in the corpus of the \textit{Survey of English Usage} (\is{Survey of English Usage}SEU; written \il{British English}BrE). As shown in \figref{fig:5.2}, \SchuetzlerIndexExpression{though} is most frequent ($n=133$; 436 pmw), followed closely by \SchuetzlerIndexExpression{although} ($n=121$; 397 pmw); again, \SchuetzlerIndexExpression{even though} is considerably less frequent ($n=16$; 52 pmw). The general frequency patterns of the three conjunctions are thus remarkably similar between \citet{Aarts1988} and \citeauthor{Altenberg1986} (\citeyear{Altenberg1986}; see above).

\begin{figure}
\includegraphics{figures/CCs.Fig.5.2.pdf}
\caption{\label{bkm:Ref471516304}\label{fig:5.2}Frequencies of concessive conjunctions in written BrE \citep{Aarts1988} }
 \end{figure}

Regarding \is{style}stylistic preferences, \citet[47–48]{Aarts1988} finds that \SchuetzlerIndexExpression{although} is used most frequently in exam essays, medical correspondences, scientific writing, administrative/official language and letters, while it occurs least frequently in journals and non-fiction. By contrast, \SchuetzlerIndexExpression{though} is stylistically more evenly distributed than \SchuetzlerIndexExpression{although}, i.e. “there are no high peaks in relative frequency for this subordinator”, which is interpreted to the effect “that \SchuetzlerIndexExpression{though} is stylistically less marked than \SchuetzlerIndexExpression{although}” (\citealt{Aarts1988}: 50). These findings dovetail nicely with descriptions found in the major grammars. Concerning the third conjunction, \SchuetzlerIndexExpression{even though}, \citet[50]{Aarts1988} states that its \is{style}stylistic distribution is even more level.

\citet{Schützler2017} inspects \SchuetzlerIndexExpression{although}, \SchuetzlerIndexExpression{though} and \SchuetzlerIndexExpression{even though} in British, Canadian and \ili{New Zealand English} (\il{British English}BrE, \il{Canadian English}CanE and \il{New Zealand English}NZE), the first two of which are also investigated in this book. Data are taken from the respective components of the \textit{International Corpus of English} (\is{International Corpus of English}ICE; cf. \sectref{sec:6.1}). While the study thus foreshadows and is related to the present research, analyses are not regression-based and take a more traditional approach. The main results are shown in \figref{fig:5.3}.

\begin{figure}
\includegraphics{figures/CCs.Fig.5.3.pdf}
\caption{\label{bkm:Ref35585270}\label{fig:5.3}Frequencies of concessive conjunctions in three varieties of English \citep{Schützler2017}}
 \end{figure}

In all three varieties, \SchuetzlerIndexExpression{although} is most frequent, followed by \SchuetzlerIndexExpression{though}. There is a clear difference between speech and writing, with higher frequencies in the latter. With regard to \SchuetzlerIndexExpression{even though}, however, this difference is virtually non-existent in \il{Canadian English}CanE and \il{New Zealand English}NZE (\citealt{Schützler2017}: 177–178). Ratios of writing over speech in \il{British English}BrE (not plotted) are somewhat more regular than in \citegen{Altenberg1986} data (see \figref{fig:5.1}), but they are also close to the value of $R=2$. That is, frequencies in writing are roughly twice those in speech, with $R=2.1$ for \SchuetzlerIndexExpression{although}, $R=2.2$ for \SchuetzlerIndexExpression{though}, and $R=2.0$ for \SchuetzlerIndexExpression{even though}.

Based on the \textit{Corpus of Historical American English} (\is{corpora!COHA}COHA, \citealt{Davies2010}), another study by \citet{Schützler2018b} is not so much preliminary but complementary to the present study, using a different corpus and tracing \is{diachronic approaches}\is{diachronic approaches}diachronic developments of \SchuetzlerIndexExpression{although}, \SchuetzlerIndexExpression{though} and \SchuetzlerIndexExpression{even though} in \il{American English}AmE from the 1860s to the present day. \figref{fig:5.4} summarises frequencies of occurrence; semantic patterns found in this study are discussed in the next section (\sectref{sec:5.1.2}).

\begin{figure}
\includegraphics{figures/CCs.Fig.5.4.pdf}
\caption{\label{bkm:Ref35585292}\label{fig:5.4}Frequencies of concessive conjunctions in COHA \citep[205]{Schützler2018b}}
 \end{figure}

Frequencies of \SchuetzlerIndexExpression{although} and \SchuetzlerIndexExpression{even though} increase over time, while those of \SchuetzlerIndexExpression{though} decline. Semantic properties and the double function of \SchuetzlerIndexExpression{though}~– i.e. the competition between the use of this form as a conjunction and a \is{though@\textit{though}!as conjunct}\is{conjuncts}conjunct, respectively~– are proposed as explanations (see also \citealt{Schützler2020a}).\footnote{The general frequency changes shown in \figref{fig:5.4} progress in a similar fashion in the four broad \is{genre}genres of \is{corpora!COHA}COHA \citep[207]{Schützler2018b}.} Based on \is{corpora!COHA}COHA, the \il{American English}AmE situation at around the turn of the third millennium is broadly comparable to patterns found in the other studies summarised above. Additionally, the \is{diachronic approaches}diachronic trends can inform hypotheses concerning different patterns in \isi{varieties of English}, if we assume that the three conjunctions are affected by ongoing processes of \isi{grammaticalisation} in such varieties.

  From a surface perspective, then, the literature suggests that \SchuetzlerIndexExpression{although} and \SchuetzlerIndexExpression{though} are used more frequently than \SchuetzlerIndexExpression{even though}, potentially with \SchuetzlerIndexExpression{though} taking an intermediate position between the other two conjunctions. Rates of use of \SchuetzlerIndexExpression{although} and \SchuetzlerIndexExpression{though} are considerably lower in speech than in writing. Most sources suggest that this is also the case for \SchuetzlerIndexExpression{even though}, but there is some evidence that this conjunction is used at similar rates in both \is{mode of production}modes of production, at least in certain varieties. It will be one of the main tasks of the present study to take a closer look at possible underlying functional reasons that may to some extent account for frequency differences.

\subsection{\label{bkm:Ref80603607}Semantics}\label{sec:5.1.2}

The \is{corpora!LLC}LLC (cf. \citealt{Altenberg1986} above) also forms the basis of \citegen{Mondorf2004} study on gender-conditioned syntactic variation in \il{British English}BrE. One set of outcome variables includes finite adverbial clauses, among them concessives. \citet[85–86]{Mondorf2004} finds that, within the set of adverbials she investigates, only concessives are used more frequently by men than by women. In particular, men use more preposed (i.e. sentence-\is{initial position}initial) concessive clauses than women, while there is no such difference for sentence-\is{final position}final clauses (\citeyear{Mondorf2004}: 99). Furthermore, male speakers use a higher number of concessives that are “propositional” in meaning, which corresponds to what is called the \is{concessives (types of)!anticausal}\textit{anticausal} type in the present study (\citeyear{Mondorf2004}: 135–136; cf. \sectref{sec:2.2.1}). Mondorf concludes that this is because constructions of this type highlight a particularly strong commitment to the truth of a proposition, which correlates with the traditional male domains of \isi{authority} and power in sociolinguistics (\citeyear{Mondorf2004}: 185–186). As pointed out by \citet{Azar1997}, \is{concessives (types of)!anticausal}anticausal CCs can strengthen the main clause proposition by anticipating (and defusing) potential counter-arguments, and their use by male speakers is here interpreted as evidence of men’s tendency to resort to “linguistic strategies that are least likely to be challenged” \citep[186]{Mondorf2004}. Mondorf’s findings contribute sociolinguistically relevant aspects to a discussion of CCs, but in the present study this dimension of variation is not considered.\footnote{\citegen{Mondorf2004} data contain information concerning both the semantics of CCs and the position of clauses, but the relationship (or correlation) between these two parameters is not explored.}

\begin{sloppypar}
\citet{Hilpert2013a} studies \isi{constructional change} in a number of different linguistic structures, including “concessive \isi{parentheticals}” (155–203). This part of his study is exceptional in presenting a quantitative approach to the in\-tra-con\-struc\-tional semantics of concessives in English (as in \citealt{Schützler2017,Schützler2018b}). As will be explained below, this semantic aspect is not the main focus of Hilpert's study, which nevertheless provides very important background information for the present investigation. Hilpert concentrates on \SchuetzlerIndexExpression{although} and \SchuetzlerIndexExpression{though} (together with other connectives) in written twentieth-century \ili{American English}, based on data from \is{corpora!COHA}COHA (\citealt{Davies2010}). Based on the syntactic behaviour of concessive \isi{parentheticals}, Hilpert tests two complementary hypotheses concerning their source constructions, namely unreduced concessives (the “reduction hypothesis”) and \is{adverbials!time}temporal/\is{adverbials!condition}conditional \isi{parentheticals} (the “analogy hypothesis”). Formally, \citet[179]{Hilpert2013a} defines concessive \isi{parentheticals} as reduced clauses that lack a copula and a pronoun as in the following example, in which square brackets indicate the possible reduction:
\end{sloppypar}

\ea\label{ex:75} 
\textbf{Although} [\textit{it was}] \textit{rare}, family violence did occur. \citep[179]{Hilpert2013a}
\z

As discussed in \sectref{sec:2.3.2}, the subjects in both clauses of such constructions (matrix clause and reduced concessive clause) are \is{co-referentiality}co-referential, and as a category of comparison Hilpert therefore uses only unreduced (full-clause) constructions in which the subjects of both clauses are also \is{co-referentiality}co-referential. His results can therefore not necessarily be assumed to hold for concessives more generally, but rather highlight an interesting correlation of semantics and a particular subject configuration. Moreover, the approach is \is{semasiological approaches}semasiological, as it quantifies the proportion of semantic types by marker, not vice versa. As shown in \figref{fig:5.5}, in Hilpert’s data \SchuetzlerIndexExpression{although} is more often a marker of \is{concessives (types of)!anticausal}anticausal concession than \SchuetzlerIndexExpression{though}, both in full-clause constructions and \isi{parentheticals}. In both syntactic types, \is{concessives (types of)!epistemic}epistemic and \is{concessives (types of)!dialogic}dialogic concessives are more frequent in combination with \SchuetzlerIndexExpression{though}.\footnote{In Hilpert’s study, the attributes “content”, “\is{concessives (types of)!epistemic}epistemic” and “\is{concessives (types of)!speech-act}speech-act” are used, based on \citet{Sweetser1990}. I have taken the liberty of translating them according to the conventions followed in the present study (cf. \sectref{sec:2.2}).}

\begin{figure}
\includegraphics{figures/CCs.Fig.5.5.pdf}
\caption{\label{bkm:Ref474698939}\label{fig:5.5}Semantic types in full and parenthetical clauses with \textit{although} and \textit{though} \citep[189]{Hilpert2013a} }
 \end{figure}

The main semantic difference between unreduced concessives and \isi{parentheticals} appears to be that the latter encode a higher proportion of \is{concessives (types of)!dialogic}dialogic concessives while the former associate more with \is{concessives (types of)!anticausal}anticausal concessives. Hilpert’s study thus suggests that particular semantic types of concessives correlate with particular \isi{connectives} and with particular syntactic realisations, even though his results are valid only for a subset of constructions, as described above. What is notable in comparison to the results of the present study are the rather high percentages of \is{concessives (types of)!anticausal}anticausal and \is{concessives (types of)!epistemic}epistemic concessives.

  Like \citet{Hilpert2013a}, \citet{Schützler2017} takes a \is{semasiological approaches}semasiological approach to concessive marking; in this case, however, the construction type is not limited to combinations of clauses with \is{co-referentiality}co-referential subjects. As shown in \figref{fig:5.6}, there is a clear tendency for \SchuetzlerIndexExpression{although} and \SchuetzlerIndexExpression{though} to encode \is{concessives (types of)!dialogic}dialogic concessives, while \SchuetzlerIndexExpression{even though} mostly surfaces in constructions of the \is{concessives (types of)!anticausal}anticausal type (\citealt{Schützler2017}: 179–180).\footnote{In contrast to the present study, but in this case in parallel to \citet{Hilpert2013a}, \citet{Schützler2017} uses the terms originally introduced by Sweetser, i.e. “content”, “\is{concessives (types of)!epistemic}epistemic” and “\is{concessives (types of)!speech-act}speech-act” (cf. \sectref{sec:2.2}).} The semantic characteristics of concessives employing the three conjunctions are rather robust, not only across the three varieties but also across speech and writing (not plotted). A very similar pattern is also found in \citeauthor{Schützler2018b}’s (\citeyear{Schützler2018b}; see \sectref{sec:5.1.1}) \is{diachronic approaches}diachronic study of \il{American English}AmE:  \SchuetzlerIndexExpression{although} and \SchuetzlerIndexExpression{though} predominantly occur in \is{concessives (types of)!dialogic}dialogic concessives, while \SchuetzlerIndexExpression{even though} associates more strongly with \is{concessives (types of)!anticausal}anticausal concessives.

\begin{figure}
\includegraphics{figures/CCs.Fig.5.6.pdf}
\caption{\label{bkm:Ref35585846}\label{fig:5.6}The semantics of concessive conjunctions in three varieties of English \citep{Schützler2017}}
 \end{figure}

Apart from the contribution by \citet{Mondorf2004}, only two  authors \citealt{Hilpert2013a,Schützler2017,Schützler2018b}) have undertaken quantitative analyses that involve the intra-constructional semantics of CCs. The evidence that they present is conflicting: With the exception of \isi{parentheticals} with \SchuetzlerIndexExpression{though}, \citet{Hilpert2013a} finds a ranking of semantic types that seems to reflect the historical trajectory of change; that is, the (allegedly original or prototypical) \is{concessives (types of)!anticausal}anticausal type is more frequent than the \is{concessives (types of)!epistemic}epistemic type, and the pragmaticalised \is{concessives (types of)!dialogic}dialogic type is least frequent. In contrast, \citet{Schützler2017, Schützler2018b} finds that the two conjunctions \SchuetzlerIndexExpression{although} and \SchuetzlerIndexExpression{though} are most frequently of the \is{concessives (types of)!dialogic}dialogic type, followed by the \is{concessives (types of)!anticausal}anticausal type, while the opposite is the case for \SchuetzlerIndexExpression{even though}. Comparability of these two sources is limited because Hilpert focuses on constructions with \is{co-referentiality}co-referential subjects in both clauses~– which apparently correlate with \is{concessives (types of)!anticausal}anticausal meaning~– and moreover did not include \SchuetzlerIndexExpression{even though} in his analysis.

\subsection{\label{bkm:Ref35532447}Clause position}\label{sec:5.1.3}

\citet[1088]{QuirkEtAl1985} claim that~– as in conditionals and adversatives~– subordinate clauses in CCs tend to be placed before the matrix clause. They give no reason for this pattern, however, and it is quite obviously contra \citeauthor{Diessel2005}’s (\citeyear{Diessel2005}; cf. \sectref{sec:2.3.1}) general theory of clause placement. In contrast to what is claimed by \citet[1088]{QuirkEtAl1985}, \citet[834]{BiberEtAl1999} find that, across \is{register}registers, subordinate clauses in concessives are placed \textit{after} the matrix clause in 60\% of the cases. Their analysis is limited to \is{finite clauses}finite clauses, however.

\figref{fig:5.7} shows \citegen[22]{Altenberg1986} findings in \is{corpora!LLC}LLC and \is{corpora!LOB}LOB (see \sectref{sec:5.1.1} above). Subordinate clauses with \SchuetzlerIndexExpression{though} and \SchuetzlerIndexExpression{even though} occur more frequently in \isi{final position}~– a tendency that is further strengthened in speech. \SchuetzlerIndexExpression[although]{Although}, on the other hand, is more likely found in sentence-\isi{initial position}, but in speech this tendency is less pronounced. \is{medial position}Medially placed clauses introduced by the three conjunctions are rare overall. Concerning the general pattern, \citet[22–23]{Altenberg1986} concludes that \SchuetzlerIndexExpression{although} has a more important “\isi{grounding}” function for the following discourse than \SchuetzlerIndexExpression{though} (and \SchuetzlerIndexExpression{even though}, which he does not discuss due to low numbers). He further argues that different \isi{planning} strategies in speech and writing are responsible for differences in clause positions between the two \is{mode of production}modes \citep[20-22]{Altenberg1986}: In speech, there is less advance \isi{planning}, and a main clause is therefore more often qualified by a postposed subordinate clause (cf. \citealt{Diessel2005}, as discussed in \sectref{sec:2.3.1}).

\begin{figure}
\includegraphics{figures/CCs.Fig.5.7.pdf}
\caption{\label{bkm:Ref471494167}\label{fig:5.7}Positions of subordinate clauses in spoken and written BrE \citep[22]{Altenberg1986}}
\end{figure}

Concerning the position of clauses, \citet[43–44]{Aarts1988} finds that in written \il{British English}BrE \SchuetzlerIndexExpression{although} is nonfinal in 54\% of all cases, while \SchuetzlerIndexExpression{though} is nonfinal 36\% of the time~– his findings are shown in \figref{fig:5.8} in greater detail, including the three categories “\is{initial position}initial’”, “\is{medial position}medial” and “\is{final position}final”. The ordering of clauses is thus rather similar between the two studies by \citet{Altenberg1986} and \citet{Aarts1988}. In both cases, clauses headed by \SchuetzlerIndexExpression{although} are considerably more likely to precede the matrix clause than clauses headed by \SchuetzlerIndexExpression{though}. For \SchuetzlerIndexExpression{even though}, no data on clause ordering are presented by Aarts.

\begin{figure}
\includegraphics{figures/CCs.FIg.5.8.pdf}
\caption{\label{bkm:Ref35783219}\label{fig:5.8}Clause positions of concessive conjunctions in written BrE \citep{Aarts1988} }
 \end{figure}

A study by \citet{WiechmannKerz2013} investigates the position of concessive subordinate clauses in the written part of the \is{corpora!BNC}\textit{British National Corpus} (BNC). Written data are used, “as concessive clauses occur predominantly in written \is{register}registers” (\citeyear{WiechmannKerz2013}: 7; cf. \citealt{BiberEtAl1999}: 821). The study compares $n=1{,}000$ clauses with the conjunction \SchuetzlerIndexExpression{although} to $n=1{,}000$ clauses with \SchuetzlerIndexExpression{whereas}. In contrast to \SchuetzlerIndexExpression{although}, \SchuetzlerIndexExpression{whereas} regularly marks constructions that are purely \is{adverbials!contrast}adversative, rather than concessive, in meaning (cf. \citealt{Altenberg1986}: 22). For instance, it is difficult to use \SchuetzlerIndexExpression{whereas} in most of the examples in \sectref{sec:2.2} of this monograph. The following summary will therefore focus mostly on \textit{although}.

  As independent variables,  \citet[3–7]{WiechmannKerz2013} take the following properties of the subordinate clause into account: (i)~relative clause \is{length (of clauses)}length,
(ii)~finiteness/nonfiniteness,
(iii)~clause \is{complexity (of clauses)}complexity (i.e. constructions with or without an embedded clause within the subordinate clause),
(iv)~the presence of a “\isi{bridging} context” that refers explicitly to the preceding discourse with an \is{anaphora}anaphoric \textsc{pro}-form, and
(v)~the choice of the subordinator itself. With regard to \SchuetzlerIndexExpression{although}, \citet[11–20]{WiechmannKerz2013} find that subordinate clauses are more likely to occur in sentence-\isi{initial position} if they contain an \is{anaphora}anaphoric reference to an earlier part of the discourse. Long, complex and \is{finite clauses}finite clauses tend to follow the matrix clause rather than precede it. However, those factors play only “subsidiary roles”. Of greater importance is the choice of subordinator: \SchuetzlerIndexExpression{whereas} tends much more strongly to be placed in sentence-\isi{final position}. \citet{WiechmannKerz2013} argue that the semantic difference between (concessive) \SchuetzlerIndexExpression{although} and (\is{adverbials!contrast}adversative) \SchuetzlerIndexExpression{whereas} motivates differences in syntactic behaviour. The contrast between prototypically concessive and \is{adverbials!contrast}adversative meaning can to some extent be applied to the semantic types of CCs investigated in the present study, too: The \is{concessives (types of)!dialogic}dialogic type seems to be closer in meaning to \is{adverbials!contrast}adversativity, lacking the semantic integration (via a \isi{topos}) characteristic of the other types. While \citet{WiechmannKerz2013} thus identify semantics as an underlying factor, the term (as they use it) refers to rather general categories (e.g. “\is{adverbials!contrast}contrastive/\is{adverbials!contrast}adversative” vs “concessive”), not to the more specific categories established by \citet{Sweetser1990} and used in the present study.

Drawing on written \il{American English}AmE data from \is{corpora!COHA}COHA (\citealt{Davies2010}), \citet{Schützler2019} predicts the positions of concessive subordinate clauses (\is{final position}final vs \is{nonfinal position}nonfinal) based on several independent variables, among them the subordinating conjunction (\SchuetzlerIndexExpression{although}, \SchuetzlerIndexExpression{though} and \SchuetzlerIndexExpression{even though}) and the semantic type of the construction.\footnote{Like \citet{Hilpert2013a} and \citet{Schützler2017, Schützler2018b}, \citet{Schützler2019} uses different labels for semantic types (cf. \citealt{Sweetser1990}; see \sectref{sec:2.2} above).} Results are shown in \figref{fig:5.9}.

\begin{figure}
\includegraphics{figures/CCs.Fig.5.9.pdf}
\caption{\label{bkm:Ref35862401}\label{fig:5.9}Clause positions of concessive conjunctions in COHA \citep[261]{Schützler2019}; a~= anticausal, e~= epistemic, d~= dialogic}
 \end{figure}

The conjunction itself has the greatest impact, with \SchuetzlerIndexExpression{even though} strongly associating with subordinate clauses in \isi{final position}. Regarding \SchuetzlerIndexExpression{although} and \SchuetzlerIndexExpression{though}, there is a regular effect of intra-constructional semantics: Dialogic CCs tend to be more often realised with subordinate clauses in \isi{final position}, while preposed subordinate clauses correlate with \is{concessives (types of)!anticausal}anticausal semantics.

Finally, \citet{Schützler2020b} also inspects the positions of subordinate clauses, focusing on the conjunction \SchuetzlerIndexExpression{although} in six \is{varieties of English!L1}L1 and \is{varieties of English!L2}L2 \isi{varieties of English}.\footnote{The varieties that are inspected are \il{British English}BrE, \il{Canadian English}CanE, \il{New Zealand English}NZE, \il{Nigerian English}NigE, \il{Indian English}IndE, as well as \ili{Philippine English} (PhilE).} The syntactic behaviour is predicted based on variety status (\is{varieties of English!L1}L1 vs \is{varieties of English!L2}L2), \isi{mode of production} (written vs spoken) and semantic type (anticausal vs dialogic). Results are summarised in \figref{fig:5.10}. There is no systematic difference between \is{varieties of English!L1}L1 and \is{varieties of English!L2}L2 varieties. Both spoken language and \is{concessives (types of)!anticausal}anticausal semantics increase the likelihood of subordinate clauses to appear in \isi{final position}. However, the effect is more moderate in written language. While \citet{Schützler2020b} partly draws on the same data as the present study, the perspective on clause arrangements is rather different: In \citet{Schützler2020b}, the choice of marker (\textit{although}) is treated as primary, while in this book the selection of a clausal sequence is regarded as primary in the \isi{constructional choice model}.

\begin{figure}
\includegraphics{figures/CCs.Fig.5.10.pdf}
\caption{\label{bkm:Ref35890523}\label{fig:5.10}Positions of subordinate clauses with \textit{although} in six varieties in ICE \citep{Schützler2020b}}
 \end{figure}

  To sum up the research presented in this section: Speaking for concessives in general, \citet{QuirkEtAl1985} and \citet{BiberEtAl1999} do not agree on the typical position of subordinate clauses. The empirical studies that were reported indicate that finally placed subordinate clauses are more likely in speech. \SchuetzlerIndexExpression[though]{Though} and \SchuetzlerIndexExpression{even though} tend to introduce clauses in \isi{final position} in some studies, in others it is only \SchuetzlerIndexExpression{even though} that shows this association. Across the board, however, \SchuetzlerIndexExpression{although} correlates with clauses in \isi{nonfinal position}. Further, there is evidence that CCs of the \is{concessives (types of)!anticausal}anticausal type are more likely to be found in \isi{nonfinal position} than \is{concessives (types of)!dialogic}dialogic ones.\footnote{However, only one of the relevant studies is based on an independent dataset, while the other uses a subset of the data underlying the present study.} This tendency complicates the assessment of earlier studies like \citet{Altenberg1986} and \citet{Aarts1988}, which do not consider the semantic structure of CCs as a potential factor.

\subsection{\label{bkm:Ref35854988}Clause types}\label{sec:5.1.4}

Information on the relative frequencies of \is{finite clauses}finite and \is{nonfinite clauses}nonfinite concessive subordinate clauses is virtually non-existent in the literature, most likely because there appeared to be no reason to make a special case for concessives in this regard. The only result that can inform the present research is found in Hilpert’s study (\citeyear{Hilpert2013a}: 183), where \SchuetzlerIndexExpression{though} is followed by \is{parentheticals}parenthetical (reduced) structures more often (45\% of all cases) than \SchuetzlerIndexExpression{although} (31\%). However, Hilpert’s analysis is based exclusively on combinations of matrix and subordinate clauses that share a single, \is{co-referentiality}co-referential subject~– the prerequisite of clause reduction. That is, the percentage of reduced clauses in the present study will probably be lower, since subjects in the component clauses of a CC will in many cases be hetero-referential, resulting in irreducible subordinate clauses. Nevertheless, Hilpert’s results can be taken to indicate that the shorter connective (\SchuetzlerIndexExpression{though}) associates with \is{length (of clauses)}shorter (i.e. reduced) clausal complements.

\section{\label{bkm:Ref35511277}Summary and identification of research gaps}\label{sec:5.2}

The main findings of quantitative studies on the three conjunctions are summarised in \tabref{tab:5.1}, ordered by the three relevant parameters of variation: frequency, semantics and syntax. Asterisks indicate (partial) agreement and mutual support of different studies within the respective category. Superscript daggers indicate disagreement between studies.

\begin{sidewaystable}
\small
\caption{\label{bkm:Ref490845890}\label{tab:5.1}Concessive conjunctions: Summary of previous research.}
\tabcolsep=0.6\tabcolsep
\begin{tabularx}{\textwidth}{ll Q}
\lsptoprule
\multicolumn{2}{l}{Parameter/Source} & Findings and comments\\\midrule
\multicolumn{2}{l}{\textsc{frequency}}\\
& *\citet{Altenberg1986} & Frequency ranking: \textit{though} > \textit{although} > \textit{even though}; higher frequencies in writing ({BrE})\\
& *\citet{Aarts1988} & Frequency ranking: \textit{though} > \textit{although} > \textit{even though}; 
\textit{although}: more frequent in \is{formality}formal styles; \textit{though} and \textit{even though} stylistically less restricted ({BrE})\\
& \textsuperscript{\phantom{§}}\citet{Mondorf2004} & Higher rates of use of subordinate concessive clauses by men, most strongly if in \isi{initial position} and/or with proposition-oriented semantics ({BrE})\\
& *\citet{Schützler2017} &  Frequency ranking: \textit{although}  > \textit{though} >  \textit{even though}; more frequent in writing; \textit{even though} less affected ({BrE}, {CanE}, {NZE})\\
& \textsuperscript{\phantom{§}}\citet{Schützler2018b}  & \textit{although}\slash\textit{even though}: increasing frequency; \textit{though}: decreasing frequency ({AmE})\\\midrule

\multicolumn{2}{l}{\textsc{semantics}}\\
&\textsuperscript{†}\citet{Hilpert2013a}  & \textit{although}: \is{concessives (types of)!anticausal}anticausal; \textit{though}: mostly \is{concessives (types of)!anticausal}anticausal; in \isi{parentheticals}: weaker preference of \is{concessives (types of)!anticausal}anticausal type ({AmE})\\
& \textsuperscript{†,}*\citet{Schützler2017} & \textit{although}/\textit{though}: \is{concessives (types of)!dialogic}dialogic; \textit{even though}: \is{concessives (types of)!anticausal}anticausal ({BrE}, {CanE}, {NZE})\\
& *\citet{Schützler2018b} & \textit{although}\slash\textit{though}: \is{concessives (types of)!dialogic}dialogic; \textit{even though}: \is{concessives (types of)!anticausal}anticausal ({AmE})\\\midrule


\multicolumn{2}{l}{\textsc{syntax}}\\
& \textsuperscript{†}\citet{QuirkEtAl1985} & Subordinate clauses precede matrix clauses\\
& \textsuperscript{†}\citet{BiberEtAl1999} &Subordinate clauses follow matrix clauses\\
& *\citet{Altenberg1986} & \textit{although}: \isi{initial position}; \textit{though}\slash\textit{even though}: \isi{final position}; speech: more \isi{final position} ({BrE})\\
& *\citet{Aarts1988} & \textit{although}: \isi{nonfinal position}; \textit{though}: mostly \isi{final position} ({BrE})\\
& \textsuperscript{\phantom{†}}\citet{WiechmannKerz2013} & \textit{although}: long clauses tend to be in \isi{final position}; \is{anaphora}anaphoric (“\isi{bridging}”) elements make nonfinal placement more likely ({BrE})\\
& \textsuperscript{\phantom{†}}\citet{Hilpert2013a} & \textit{though} more likely to take reduced clauses than \textit{although} ({AmE})\\
& *\citet{Schützler2019} & \textit{even though}: \isi{final position}; \is{concessives (types of)!anticausal}anticausal: \isi{nonfinal position}; \is{concessives (types of)!dialogic}dialogic: \isi{final position} ({AmE})\\
& *\citet{Schützler2020b} & \textit{although}: higher probability of \is{final position}final placement in speech and in combination with \is{concessives (types of)!dialogic}dialogic concessives ({BrE}, {CanE}, {NZE}, {NigE}, {IndE}, PhilE)\\
\lspbottomrule
\end{tabularx}
\end{sidewaystable}

Concerning the properties that characterise concessive constructions, clause ordering and frequencies have been investigated in several studies, the latter also with a view to genre-related variation. By contrast, semantic aspects~– more precisely: the intra-constructional semantic relations between propositions~– have received little attention. Several authors have discussed them in theoretical terms (cf. \sectref{sec:2.2}), but~– with the exceptions of \citet{Mondorf2004} and a few previous investigations by the author of the present volume~– the only quanti\-ta\-tive study of those aspects is \citet{Hilpert2013a}. It is also striking that, except for \citet{Schützler2017, Schützler2020b}, concessive markers have not been studied in \isi{varieties of English} other than \il{British English}BrE and \il{American English}AmE.

Most importantly, the existing research lacks in \is{multifactorial approaches}multifactorial approaches to CCs. As sketched in \tabref{tab:5.1}, disconnected results exist for several dimensions of variation (\is{text frequency}text frequencies, semantics and syntax), but their interrelatedness (or interaction) remains largely unexplored. Much of the conflicting or inconclusive evidence is therefore likely due to the large number of unknowns in each individual study~– i.e. underlying factors that are not operationalised. It is this aspect in particular that the present study will address.

\section{\label{bkm:Ref3388422}\label{bkm:Ref3390733}\label{bkm:Ref61176821}Research questions and hypotheses}\label{sec:5.3}

The general research questions for the present study were formulated in \sectref{sec:1.3} and will not be restated here. A more precise definition of expectations relies on two components:
(i)~the \isi{constructional choice model} introduced in \sectref{sec:4.1.3}, and
(ii)~the two extralinguistic factors introduced in \sectref{sec:4.2} and \sectref{sec:4.3}, mode of \isi{production} and variety status. The diagram in \figref{fig:5.11} shows the expected relationships between external predictors and outcomes at different levels of the construction, as well as intra-constructional relationships between parameters. This will be the main point of reference for the analyses implemented in Chapters \ref{sec:7}–\ref{sec:11} of the book. Note that the model is developed only for \is{concessives (types of)!anticausal}anticausal and \is{concessives (types of)!wide-scope dialogic}wide-scope dialogic CCs. That is, \is{concessives (types of)!epistemic}epistemic and \is{concessives (types of)!narrow-scope dialogic}narrow-scope dialogic CCs are excluded: Both of them are relatively rare, and, in addition, narrow-scope dialogic CCs do not participate in the full range of syntactic variation (see \sectref{sec:2.2.4}). External factors are represented by triangles (black “W”: written language; grey “S”: spoken language) and circles (black “\is{varieties of English!L1}L1”: first-language/inner-circle varieties; grey “\is{varieties of English!L2}L2”: second-language/outer-circle varieties), respectively. Connecting lines between these symbols and components of the construction indicate expected positive correlations (e.g. between spoken language and the selection of \SchuetzlerIndexExpression{though}, or between \is{varieties of English!L1}L1 varieties and \is{nonfinite clauses}nonfinite clause structures). In addition, the general intra-constructional relations sketched in \sectref{sec:4.1.3} (\figref{fig:4.2}) are unfolded into more precisely defined expectations, as shown by the connecting arrows. This means that we not only expect higher-order properties to influence lower-order ones, but that we have concrete expectations concerning the way this happens. In the following paragraphs, the reasoning underlying \figref{fig:5.11} will be explained.

\begin{figure}
\includegraphics[width=10cm]{figures/CCs.Fig.5.11.pdf}
\caption{\label{bkm:Ref35882786}\label{fig:5.11}A choice model for CCs, including expected correlations}
 \end{figure}

At the intra-constructional level, subordinate clauses in \is{concessives (types of)!anticausal}anticausal CCs are expected to be more likely in \isi{nonfinal position}, since such an arrangement is \is{iconicity}iconic of the conditional (or \is{cause and effect}cause-and-effect) relation that exists between propositions. In configurations of this kind, the sequence (or dependency) of real-world phenomena finds its correlate in syntactic structure, which is regarded as cognitively more ideal, both in \isi{planning}/\isi{production} and in \isi{processing}.\footnote{There is thus a correspondence between mental \isi{presupposition} and formal preposition.} \is{concessives (types of)!dialogic}Dialogic CCs, on the other hand, are not based on such underlying relations and are more often characterised by some sort of (seemingly post-hoc) \is{modification (or qualification)}qualification of the primary statement. They are therefore expected to be more often characterised by subordinate clauses in \isi{final position}, which has also been argued to be the ideal default configuration, irrespective of adverbial meaning (cf. \citealt{Diessel2005}; see \sectref{sec:2.3.1}).

Concerning the relationship between clause positions and the selection of a conjunction, several studies have found that there is a positive correlation between \SchuetzlerIndexExpression{although} and subordinate clauses in \isi{nonfinal position} and between \SchuetzlerIndexExpression{even though} and subordinate clauses in \isi{final position}. The present study also expects to find these patterns, despite the absence of a firm theoretical basis~– except perhaps that a functional specialisation of a marker (or markers) in this regard is generally plausible. Concerning the preferred clause configuration (final vs nonfinal) of \SchuetzlerIndexExpression{though}, previous research is divided; accordingly, this link is not specified in \figref{fig:5.11}. Finally, \citet{Hilpert2013a} provides some evidence of a tendency for \SchuetzlerIndexExpression{though} to combine with \is{nonfinite clauses}nonfinite clauses more often than \SchuetzlerIndexExpression{although}, even if his results are only valid for a specific construction type. In this case, it is the conjunction \SchuetzlerIndexExpression{even though} for which we lack prior information; once again the respective link in the model is left unspecified. Clause positions may also have an effect on the realisation of subordinate clauses as either \is{finite clauses}finite or \is{nonfinite clauses}nonfinite. However, two conflicting hypotheses can be generated. On the one hand, \is{nonfinite clauses}nonfinite (and therefore subjectless) clauses withhold grammatical information, which makes their placement before the grammatically more \is{explicitness}explicit matrix clause cognitively demanding and therefore less likely. On the other hand, \is{nonfinite clauses}nonfinite clauses will tend to be shorter than \is{finite clauses}finite ones, and according to the principle of “\isi{resolution}” (\citealt{QuirkEtAl1985}: 1036; cf. \sectref{sec:2.3.1}) they would be more likely in \isi{initial position}, leaving the \is{final position}final slot to the \is{weight}heavier matrix clause. No hypothesis is formulated concerning this particular relationship, since the clash of plausible explanations cannot be resolved at this stage.

\begin{sloppypar}
\is{mode of production}Mode of production is expected to correlate with the general frequencies as well as the functional and formal parameters of CCs in four ways. Firstly, propositions in \is{concessives (types of)!dialogic}dialogic concessives are more loosely connected in that they lack a \isi{topos} (or underlying causal\slash conditional \isi{presupposition}). Often enough, their component parts constitute mutual qualifications, and the construction as a whole comes across as quasi-coordinated in meaning, if not in syntax. Dialogic CCs are therefore considered more compatible with spoken discourse, which puts greater temporal constraints on \isi{planning}, \isi{production} and \isi{processing}. Anticausal CCs, on the other hand, are based on complex inventories of \is{topos}topoi, which need to be accessed by both SP/W and AD/R. In terms of economy and complexity, CCs of this type would therefore be expected to be employed more frequently in writing. Secondly, a positive correlation of spoken language with the \is{final position}final placement of subordinate clauses is expected, while clauses in \isi{nonfinal position} should associate more with writing. The arguments that underpin these expectations were discussed in \sectref{sec:4.2}. They are based on the assumption that, from the perspectives of \isi{production} and \isi{processing}, subordinate clauses in \isi{final position} are more straightforward, while subordinate clauses in \isi{initial position} are cognitively more challenging (see e.g. \citealt{Hawkins1994, Hawkins2000}). Thirdly, based on vague \is{style}stylistic patterns discussed in the literature, a higher proportion of \SchuetzlerIndexExpression{although} is expected in writing, while \SchuetzlerIndexExpression{though} is expected to be relatively more common in speech. The third conjunction, \SchuetzlerIndexExpression{even though}, is not explicitly discussed concerning its \is{style}stylistic value in most grammars. \citet{Aarts1988} even makes a point of this marker’s equal distribution across different text types, which is why I am reluctant to predict its behaviour across speech and writing. Finally, it is expected that \is{nonfinite clauses}nonfinite clauses should be more frequent in writing than in speech, since they are characterised by a less \is{explicitness}explicit mapping of surface form onto propositional content and are therefore more easily processed in contexts characterised by lower time pressure. Moreover, using reduced clauses is simply one way of producing more \is{compression}compressed and grammatically less redundant language, which is also more typical of writing (cf. \sectref{sec:4.2}).
\end{sloppypar}

Regarding the correlation between variety status (\is{varieties of English!L1}L1 vs \is{varieties of English!L2}L2) and semantic types, it is expected that \is{concessives (types of)!anticausal}anticausal CCs should be relatively more frequent in \is{varieties of English!L2}L2 varieties, while \is{concessives (types of)!dialogic}dialogic CCs are more common in \is{varieties of English!L1}L1 varieties. This assumption is based on the general tendency for English to be acquired scholastically in \is{varieties of English!L2}L2 contexts~– that is, language users’ inventories of constructions and \is{constructions!subconstructions}subconstructions are to a larger extent based on tendencies and instructions explicitly codified in grammars and the teaching materials based on them. Even a cursory inspection of grammars like \citet{QuirkEtAl1985} reveals that it is predominantly \is{concessives (types of)!anticausal}anticausal CCs that are used to exemplify concessive adverbials, and it is expected that this will have at least some effect on language use in varieties that depend on explicit learning to a greater extent. Further, subordinate clauses in \isi{final position} are expected to be more frequent in \is{varieties of English!L2}L2 varieties: (i)~If we accept that finally placed subordinate clauses are the default based on principles of \isi{production} and \isi{processing}, then this tendency should be adhered to more strongly in varieties in which contexts of use for English are somewhat more restricted; and (ii)~if exposure to English pervades all every-day contexts (as in \is{varieties of English!L1}L1 varieties), there will be more low-level variation and a more flexible handling of syntactic patterns. However, the factor of scholastic \is{language acquisition}acquisition in \is{varieties of English!L2}L2 varieties may have a contrary effect concerning clause placement. This is because the general tendency for standard grammars (and derived materials) is not only to showcase \is{concessives (types of)!anticausal}anticausal CCs, but also to present them with preposed subordinate clauses. There are thus several conflicting hypotheses, and expectations concerning the correlation of \is{varieties of English!L2}L2 varieties with subordinate clauses in \isi{final position} cannot be formulated with confidence. Next, \is{varieties of English}variety status is not expected to have an effect on the selection of markers. In \is{varieties of English!L2}L2 varieties, \SchuetzlerIndexExpression{although} and \SchuetzlerIndexExpression{even though} may be less grammaticalised, and such varieties may therefore be similar to earlier stages of English, as shown in \figref{fig:5.4} for \il{American English}AmE. It seems difficult, however, to position \is{varieties of English!L2}L2 varieties on this kind of historical trajectory, since contexts of \is{language acquisition}acquisition and use are likely to override purely historical factors. Explicitly codified patterns of use may be disproportionally influential, and we would thus expect an even stronger predominance of \SchuetzlerIndexExpression{although} in such varieties, since this marker is regularly treated as the primary concessive conjunction. These are no more than informed speculations, however, and no clear hypothesis or expectation is formulated regarding the effect of variety status on the choice of conjunction. Finally, I assume that \is{nonfinite clauses}nonfinite clauses will be somewhat less routinely used in \is{varieties of English!L2}L2 varieties: The greater degree of transparency and \isi{explicitness} that comes with \is{finite clauses}finite clauses may be beneficial in societies characterised by a less pervasive role of English.

  The discussion of expected patterns of variation in this section is admittedly multi-faceted and complex. The interpretation of results in Chapters \ref{sec:7}–\ref{sec:11} will strongly rely on the information presented here, and the reader is invited to refer back to \figref{fig:5.11} when reading the chapters below.

