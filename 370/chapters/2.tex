\chapter{\label{bkm:Ref34993759}\label{bkm:Ref35421461}\label{bkm:Ref35521616}Concessive clauses: Development, function and form}\label{sec:2}

In this chapter, three aspects are addressed. First, \sectref{sec:2.1} discusses the close relationship between concessives and other kinds of adverbial relation and shows some of the paths along which concessives (and concessive markers) have developed. Next, \sectref{sec:2.2} is an introduction to the semantic categories relevant in the present study. Finally, \sectref{sec:2.3} discusses the possible syntactic realisations and the grammatical characteristics of concessive constructions in Present-day English, both in terms of sentence structure and complement-internal syntax.

Following König \& Eisenberg (\citeyear{KönigEisenberg1984}: 322; cf. \citealt{König1991b}: 632, \citealt{KönigSiemund2000}: 341), I use the terms \is{connectives}\textit{connective} and \is{markers}\textit{marker} interchangeably. While the focus of this study is on \is{subordinators}subordinating conjunctions, two other broad classes of \isi{connectives} can be identified: \isi{prepositions} and \isi{conjuncts} (\citealt{KönigEisenberg1984}: 322, \citealt{König1991b}: 632, \citealt{Hoffmann2005}: 110, \citealt{König2006}: 821).\footnote{Conjuncts are sometimes referred to simply as “adverbs” \citep[41]{Aarts1988}, “connective adjuncts” (\citealt{HuddlestonPullum2002}: 736), “linking adverbials” (\citealt{BiberEtAl1999}: 850–851) or “conjunctional adverbs” (\cites[821]{König2006}[632]{König1991b}[110]{Hoffmann2005}). For a study of the conjunct \textit{though}, see \citealt{Schützler2020a}.} This diversity of grammatically different concessive \isi{markers} reflects the fact that conces\-sive relations do not exclusively hold between clauses within a sentence, but may involve other structures, for instance nominalisations, entire sentences or larger discourse chunks.

The majority of examples stem from the literature, the \textit{\isi{International Corpus of English}} (\is{International Corpus of English}ICE; see \sectref{sec:6.1}), Brown-family corpora (see beginning of \chapref{sec:3}), the \textit{Corpus of Historical American English} (\is{corpora!COHA}COHA; \citealt{Davies2010}) or \is{corpora!ARCHER}ARCHER \citep{Yáñez-Bouza2011}, and their provenance is cited accordingly. If no further information is given, examples were constructed by the author.

\section{\label{bkm:Ref82082809}\label{bkm:Ref427224630}Historical background}\label{sec:2.1}

Although diachronic developments of concessive markers play no central role in this study, this section provides some historical background for the contextualisation of the analyses presented in Chapters \ref{sec:7}–\ref{sec:11}. The first part (\sectref{sec:2.1.1}) of the discussion focuses on the development of English concessives more generally and is followed by short histories of the relevant individual markers (\sectref{sec:2.1.2}).

\subsection{\label{bkm:Ref467055763}General aspects}\label{sec:2.1.1}

According to \citet[190]{König1991a}, there are (at least) two classes of \is{adverbials}adverbial relations:
(i)~“elementary” or “primary” relations (\is{adverbials!place}place, \is{adverbials!time}time, \is{adverbials!manner}manner), which can often be “expressed by monomorphemic, non-anaphoric adverbs” (e.g. \textit{there}, \textit{then, fast}) and corre\-sponding simple interrogative pronouns (e.g. \textit{where}, \textit{when}, \textit{how}), and
(ii) “logical” rela\-tions (e.g. \is{adverbials!reason}causal, concessive, \is{adverbials!instrumental}instrumental, and \is{adverbials!purpose}purposive). Historically, \isi{logical relations} can emerge from primary ones through “\isi{secondary grammaticalisation}” (\citealt{Hilpert2013a}: 167–168; cf. examples in \citealt{KönigTraugott1988}: 113–114), but not vice versa. This corre\-sponds to the typical order in which these expressions are acquired by learners (\citealt{König1991a}: 190–191). In the history of a language, concessives usually develop relatively late, if they develop at all \parencites[1--2]{König1985}[151]{König1988}[632]{König1991b}[319]{Kortmann1996}[821]{König2006}[167--168]{Hilpert2013a}.

König (\citeyear{König2006}: 821–822; also \citealt{König1991a}: 192–195) identifies five types of concessive \isi{connectives} on historical grounds (cf. also \citealt{König1985}: 10–11 and \citealt{KönigEisenberg1984}: 323–325, both of whom do not list type 4):

\begin{enumerate}
\item Connectives that grammati\-calised out of nouns that carry meanings such as ‘obstinacy’, ‘contempt’ and ‘spite’, e.g. the English \isi{prepositions} \SchuetzlerIndexExpression{in spite of} and \SchuetzlerIndexExpression{despite}, which originally depended on some kind of human agentivity (cf. German \SchuetzlerIndexExpression{trotz}; French \SchuetzlerIndexExpression{en dépit de})

\item Connectives one of whose components origi\-nated as a free-choice (or universal) quanti\-fier (“\isi{allquantor}”), e.g. \SchuetzlerIndexExpression{albeit}, \SchuetzlerIndexExpression{however}, \SchuetzlerIndexExpression{whatever}, \SchuetzlerIndexExpression{anyway}, or \SchuetzlerIndexExpression{for all that} (cf. German \is{wie auch immer@\textit{wie auch immer}} \textit{wie}/\SchuetzlerIndexExpression{was auch immer})

\item\sloppy Connectives that developed via \isi{secondary grammaticalisation}, particularly building on \is{adverbials!time}temporal, \is{adverbials!reason}causal and \is{adverbials!condition}conditional meanings and often combining with an additive focus particle, e.g. \SchuetzlerIndexExpression{if}, \SchuetzlerIndexExpression{even if}, \SchuetzlerIndexExpression{even though}, \SchuetzlerIndexExpression{even so} (cf. German \SchuetzlerIndexExpression{obgleich}, \SchuetzlerIndexExpression{obwohl}, \SchuetzlerIndexExpression{wenngleich}, \SchuetzlerIndexExpression{obschon})

\item Connectives that derive from constructions originally expressing factuality or \is{emphasis}emphatic affirmation (König cites the construction \textit{True} p\textit{, but} q; cf. German \textit{Zwar…, doch…})

\item Connectives developed from expressions that highlight a state of remarkable co-occurrence or coexistence, e.g. \SchuetzlerIndexExpression{nevertheless}, \SchuetzlerIndexExpression{still}, \SchuetzlerIndexExpression{notwithstanding} (cf. German \SchuetzlerIndexExpression{nichtsdestoweniger}, \SchuetzlerIndexExpression{gleichwohl}).\footnote{Cf. \citet[293–295]{DiMeola2004}, who undertakes a similar classification of concessive connectors in \ili{German}, using properties of their components as criteria.}
\end{enumerate}

Connectives like \SchuetzlerIndexExpression{nichts\-desto\-trotz} (German) or \SchuetzlerIndexExpression{in spite of all} show that mixed etymologies exist, in these cases between categories 1 \& 5 and categories 1 \& 2, respectively. In addition, it can be argued that there is another class of markers that superficially look like nonfinite verb phrases (VPs) but have (partly) grammaticalised into a connective (e.g. \SchuetzlerIndexExpression{seeing that}, \SchuetzlerIndexExpression{considering}, \SchuetzlerIndexExpression{having said that}). The members of this group belong to the category of “\is{subordinators!marginal}marginal subordinators” (e.g. \SchuetzlerIndexExpression{supposing}, \SchuetzlerIndexExpression{provided}, \citealt{QuirkEtAl1985}: 1002–1003, cf. \citealt{Schützler2018a}).\footnote{Very generously (if, of course, also unorthodoxly) one could argue that certain deadjectival lexical items also convey a degree of concessive meaning. Thus, adverbs like \SchuetzlerIndexExpression{surprisingly} and \SchuetzlerIndexExpression{unexpectedly} imply the existence of some underlying \isi{presupposition} which disagrees with the proposition modified by those adverbs.}

The developmental path of concessive linkers from markers of \is{primary adverbial relations}primary to markers of \isi{logical relations} (see above) can still be felt in Present-day English (PDE). For example, \citet[40]{Aarts1988} describes concession as “a fuzzy semantic notion”, which shades into the neighbouring semantic domains of \is{adverbials!condition}condition, \is{adverbials!time}time and \is{adverbials!contrast}contrast (cf. \citealt{QuirkEtAl1985}: 1088). The overlap between concession and other kinds of \is{adverbials}adverbial relations is also discussed in \citet[66]{Burnham1911}, with a focus on \ili{Old English}, and in Couper-Kuhlen \& Kortmann (\citeyear{Couper-KuhlenKortmann2000Introduction}: 2; cf. \citealt{KönigSiemund2000,Harris1988}). As pointed out by \citet[168]{Hilpert2013a}, two sources of \isi{secondary grammaticalisation} are \is{adverbials!time}temporal and \is{adverbials!condition}conditional markers (also cf. \citealt{Kortmann1996}: 321, \citealt{HeineKuteva2002}: 93, 292), shown in \REF{ex:1} and \REF{ex:2}. As in the other examples throughout the present study, connectives will be highlighted in bold print, with their clausal complements in italics.

\ea\label{ex:1}
    \ea\label{bkm:Ref118500031}\textbf{If} \textit{you try a little harder}, you will succeed. (\is{adverbials!condition}conditional)\is{if@\textit{if}}\\
    \ex\label{ex:1b}The film was nice, \textbf{if} \textit{perhaps a bit cheesy}. (concessive)\is{if@\textit{if}}\\
  \z
\ex\label{ex:2}
    \ea\label{bkm:Ref118500042}  \textbf{While} \textit{living in London}, she decided to become a human rights activist. (\is{adverbials!time}temporal)\is{while@\textit{while}}\\
    \ex\label{ex:2b}\textbf{While} \textit{clearly a right-leaning person}, he voted socialist on this occasion. (concessive)\is{while@\textit{while}}\\
  \z
\z

If one accepts the primacy of \isi{adverbials} of \is{adverbials!place}place, \is{adverbials!time}time and \is{adverbials!manner}manner, examples like these are symptoms of a process of \isi{grammaticalisation} in which certain adverbial connectives have developed additional grammatical (namely concessive) functions. The older functions continue to exist, which can then be interpreted as a kind of \textit{\isi{divergence}} (cf. \citealt{Hopper1991}: 22, \citealt{HopperTraugott2003}: 124–126), whereby primary and secondary grammatical functions can occur alongside each other. At the same time, the principle of \textit{\isi{persistence}} (see also \citealt{Hopper1991}) applies, which means that \is{adverbials!condition}conditional or \is{adverbials!time}temporal meanings and associations remain part of a concessive marker’s function even after \isi{secondary grammaticalisation} has taken place.

Two particular types of concessives illustrate the kinship of \is{adverbials!condition}conditional and concessive adverbial relations. They are what \citet[1099–1102]{QuirkEtAl1985} call \is{concessives (types of)!alternative con\-di\-tion\-al-con\-ces\-sive}“alternative conditional-concessive” and “\is{concessives (types of)!universal con\-di\-tion\-al-con\-ces\-sive}universal conditional-concessive”, respectively. Both are subsumed under the category of \is{concessives (types of)!irrelevance conditionals}“irrelevance conditionals” by König (\citeyear{König1991b}: 635; cf. \citealt{KönigEisenberg1984}: 315).\footnote{\citet[262–263]{ThompsonEtAl2007} use the term \is{concessives (types of)!indefinite}“indefinite concessive” instead of  \citegen{QuirkEtAl1985} “\is{concessives (types of)!universal con\-di\-tion\-al-con\-ces\-sive}universal conditional-concessive”, because such constructions “contain some unspecified element”. Other concessives they call “\is{concessives (types of)!definite}definite concessive”.} Example (\ref{ex:3}a) shows an \is{concessives (types of)!alternative con\-di\-tion\-al-con\-ces\-sive}alternative conditional-concessive. If one of two logically opposed conditions is met (\textit{It is my turn} vs \textit{It is not my turn}), the proposition in the matrix clause will hold true. The focus in this case is of course on the negative condition, which is why the example can be rephrased as shown in (\ref{ex:3}b) and (\ref{ex:3}c).

\ea\label{ex:3}
    \ea\label{ex:3a}\label{bkm:Ref467066205}In the mornings I scoured the breakfast pans \textbf{whether} \textbf{or} \textbf{not} \textit{it was my turn}. (COHA, 1992, fiction)\is{whether or not@\textit{whether or not}}\\
    \ex\label{ex:3b}In the mornings I scoured the breakfast pans \textbf{even} \textbf{if} \textit{it was not my turn}.\is{even if@\textit{even if}}\\
    \ex\label{ex:3c}This morning I scoured the breakfast pans \textbf{although} \textit{it was not my turn}.\\
  \z
\z

Examples \REF{ex:4} and \REF{ex:5} illustrate \is{concessives (types of)!universal con\-di\-tion\-al-con\-ces\-sive}universal conditional-concessives.\footnote{\citet[64–65]{Hermodsson1994} calls concessives like German \textit{Was auch geschieht}… “generell-inkonditional”, a category he developed in an earlier study \citep[80]{Hermodsson1978}.} This type of concessive occurs in combination with the marker \SchuetzlerIndexExpression{however}, which thus has two possible functions, either as a special kind of “fused” conjunction (as in these cases) or as a \is{conjuncts}conjunct.\footnote{I call \textit{however} and \textit{whatever} “fused” in such contexts, because they appear to be conjunctions and components of the following clause at the same time.}  In the two examples, there are not two alternatives as in \REF{ex:3}, but “any number of choices” (\citealt{QuirkEtAl1985}: 1101), including those that would under normal circumstances prevent what is stated in the matrix clause proposition.

\ea\label{ex:4}   \label{bkm:Ref467067268}[\textbf{H}]\textbf{owever} \textit{hard she strove}, she could not suppress a slight quivering of her lips. (COHA, 1877, fiction)\is{however@\textit{however}}
    \ex\label{ex:5}\label{bkm:Ref467067270}\textbf{Whatever} \textit{I say to them}, I can’t keep them quiet. (\citealt{QuirkEtAl1985}: 1101)\is{whatever@\textit{whatever}}
\z

According to \citet[638]{König1991b}, constructions of this kind, as well as conditionals more generally, are one important source construction for present-day concessives.

\is{adverbials!reason}Causal adverbials are less often mentioned in the context of \isi{secondary grammaticalisation} \citep[168]{Hilpert2013a}, although the connection between \is{adverbials!reason}causality and concession is often pointed out (e.g. \citealt{Verhagen2000}: 373–375). However, example \REF{ex:6} demonstrates that, like \is{adverbials!condition}conditional or \is{adverbials!time}temporal meaning, \is{adverbials!reason}causal meaning can also develop into concessive meaning.

\ea\label{ex:6}
    \ea\label{bkm:Ref431802276}He couldn’t move \textbf{for} \textit{fear}. (\is{adverbials!reason}causal)\is{for@\textit{for}}
    \ex\label{ex:6b}She loved him \textbf{for} \textbf{all} \textit{his faults}. (concessive)\is{for all@\textit{for all}}
\z
\z

Although it can be argued that \SchuetzlerIndexExpression{for all} is a complex connective different from simple \SchuetzlerIndexExpression{for}, the connection between concession and cause is nevertheless evident. Example ‎\REF{ex:7} illustrates another interesting construction which blends \is{adverbials!reason}causality and concession.\footnote{Very similar constructions exist in \ili{German}.}

\ea\label{ex:7}\label{bkm:Ref470803084}\textbf{Just} \textbf{because} \textit{the lights are on} doesn’t mean that John is in his office. (from \citealt{Hilpert2007}: 31; cf. \citealt{Hilpert2013a}: 168)\is{just because@\textit{just because}}
\z

The development of concessive connectives out of other markers is also reflected in typology: Connectives with a truly and uniquely concessive meaning do not seem to exist in all languages, while \is{adverbials!contrast}adversative coordinating conjunctions (German \SchuetzlerIndexExpression{aber}; English \SchuetzlerIndexExpression{but}) seem to be very common \citep[632]{König1991b}. It is also possible for a language to rely on the context of an utterance to disambiguate a multifunctional connective. In English, expressions with a clear concessive meaning exist alongside markers in which primary grammatical functions persist. \citet[40]{Aarts1988} calls the former “centrally concessive” (e.g. \SchuetzlerIndexExpression{although}) and the latter “peripherally concessive” (e.g. \SchuetzlerIndexExpression{whereas}, \SchuetzlerIndexExpression{if}).\footnote{Note that \citet[343–348]{DiMeola1998} uses the label “peripheral” to refer to certain pragmatic types of concessives (cf. \sectref{sec:2.2.3}).}

\subsection{\label{bkm:Ref485723430}The histories of \textit{although}, \textit{though} and \textit{even though}}\label{sec:2.1.2}

According to the \textit{Oxford English Dictionary}\nocite{OED} (OED; \textit{s.v.} “though”), \ili{Old English} (OE) \SchuetzlerIndexExpression[þéah/þéh]{þéah}~– or one of its variants~– seems to have been the original form out of which the etymologically related other items of the set have developed. With the exception of certain dialects (e.g. East Anglian), this was replaced in \ili{Middle English} by the form \SchuetzlerIndexExpression[þóh/þou]{þóh} (or one of its variants), which derived from \ili{Old Norse} and had a back vowel. These forms were the basis for developments in the standard language up to PDE. As early as \il{Old English}OE, it was possible to add a preceding \is{intensification}intensifying form, \textit{eall} (cf. \citealt{Burnham1911}: 12–14, \citealt{Eitle1914}: 114, \citealt{Chen2000}: 104–105). In \ili{Middle English}, this became \is{al-@\textit{al-}}\SchuetzlerIndexExpression[al-]{alle}/\SchuetzlerIndexExpression[al-]{all}/\SchuetzlerIndexExpression[al-]{al}, either free-standing or hyphenated to a variant form of \SchuetzlerIndexExpression{though}. Original and expanded forms have coexisted up to the present day. The development of \SchuetzlerIndexExpression{although} constitutes a case of \isi{grammaticalisation}, with the particle \SchuetzlerIndexExpression[al-]{alle}/\SchuetzlerIndexExpression[al-]{all}/\SchuetzlerIndexExpression[al-]{al} not only becoming firmly attached to \SchuetzlerIndexExpression{though}~but also losing its \is{intensification}intensifying character. It is interesting to note that the first uses of \il{Old English}OE \SchuetzlerIndexExpression[þéah/þéh]{þéah}/\SchuetzlerIndexExpression[þéah/þéh]{þéh} as \isi{conjuncts} (or \isi{sentence adverbs}) date from roughly the same time as its use as a conjunction (cf. \citealt{Eitle1914}: 112), and apparently \il{Old English}OE did not make a clear syntactic distinction between the two uses. The OED states that, like \SchuetzlerIndexExpression{although}, \SchuetzlerIndexExpression{even though} came to be used as an \is{emphasis}emphatic, \is{intensification}intensifying variant.

The OED establishes a relatively clear chronology. The earliest attestations of the predecessors of \SchuetzlerIndexExpression{though} as a conjunction are given for the 9th and 10th centuries. A variant approximating the modern, Norse-based form (\SchuetzlerIndexExpression[þóh/þou]{þou}) is cited from the 14th century. It is around the same time that \SchuetzlerIndexExpression{although} as a conjunction seems to have arisen, if of course in variant spellings. Finally, \SchuetzlerIndexExpression{even though} is attested considerably later, in 1697.

In PDE, \SchuetzlerIndexExpression{although}~– a marked (\is{emphasis}emphatic) form at the time of its emergence (cf. \citealt{Burnham1911}: 19–20, \citealt{Bryant1962}: 216)~– is the most frequent of the three conjunctions (see, for instance, results in \citealt{Schützler2017}). While it no longer stands out as an \is{emphasis}emphatic variant, \SchuetzlerIndexExpression{although} may have developed a different kind of markedness, if we accept \citegen{QuirkEtAl1985} claim that it is more \is{formality}formal than \SchuetzlerIndexExpression{though} (see \chapref{sec:5}). That is, if \SchuetzlerIndexExpression{even though} is an \is{emphasis}emphatic variant (\citealt{QuirkEtAl1985}: 1099), \SchuetzlerIndexExpression{although} may be a \is{style}stylistically (slightly) elevated variant.

\section{\label{bkm:Ref431556473}\label{bkm:Ref425763480}Semantic types of concessives}\label{sec:2.2}

The analyses in this study reckon with three semantic types of concessives, as discussed by \citet[76–78]{Sweetser1990}.\footnote{There are some contributions that overlap with (and partly antedate) \citegen{Sweetser1990} influential three-way categorisation. See, for example, \citegen[51]{Borkin1980} distinction between a “dissonance of an empirical nature” and a “dissonance of a rhetorical nature” in concessives, or \citegen[250–253]{HallidayHasan1976} discussion of “external” and “internal” adversatives.} Sweetser's “\is{concessives (types of)!content}content” and “\is{concessives (types of)!speech-act}speech-act” types will be referred to as \is{concessives (types of)!anticausal}\textit{anticausal} and \is{concessives (types of)!dialogic}\textit{dialogic}, respectively, while the label “\is{concessives (types of)!epistemic}epistemic” is left unchanged. Furthermore, the \is{concessives (types of)!dialogic}dialogic type is subdivided into two variants, as follows:

\begin{enumerate}
\item \is{concessives (types of)!anticausal}anticausal concessives
\item \is{concessives (types of)!epistemic}epistemic concessives
\item \is{concessives (types of)!dialogic}dialogic concessives:
  \begin{enumerate}
  \item \is{concessives (types of)!wide-scope dialogic}wide scope
  \item \is{concessives (types of)!narrow-scope dialogic}narrow scope\footnote{These labels and the subdivision of \is{concessives (types of)!dialogic}dialogic concessives were introduced in \citet{Schützler2020b}. For further semantic categories of adverbial linking, see \citet[315–317]{Crevels2000}, \citet{Lang2000} and \citet{Tsunoda2012}.}
  \end{enumerate}
\end{enumerate}

Using these types goes beyond more general definitions as provided by \citet[1098]{QuirkEtAl1985}, for example, according to whom “[c]oncessive clauses indicate that the situation in the matrix clause is contrary to expectation in the light of what is said in the concessive clause”. While all concessives share an element of surprise or unexpectedness, the more fine-grained semantic categories are needed to describe them precisely.

The following four examples from the \textit{\isi{International Corpus of English}} (\is{International Corpus of English}ICE; see \sectref{sec:6.1}) illustrate the four semantic types. The main points of difference will be highlighted with short, non-technical paraphrases, while more theoretical and detailed discussions will be provided in the respective individual sections below.

\ea\label{ex:8} \is{concessives (types of)!anticausal}Anticausal concessive:\label{bkm:Ref500312809}\\
  \textbf{Although} \textit{no one will publish his work}, he refuses to compromise his artistic integrity by catering to the marketplace. (ICE-CAN:W2F-019)
\ex\label{ex:9} \is{concessives (types of)!epistemic}Epistemic concessive:\label{bkm:Ref496793868}\\\relax
  [\textbf{A}]\textbf{lthough} \textit{he’s always brandishing his bolo}, Bonifacio never fought with the bolo. (ICE-PHI:S2A-034)
\ex\label{ex:10} \is{concessives (types of)!dialogic}Dialogic concessive (\is{concessives (types of)!wide-scope dialogic}wide scope):\label{bkm:Ref496794740}\\
  On the basis of this material historians have been able to establish some unassailable facts, \textbf{although} \textit{they are not agreed on how this evidence should be interpreted}. (ICE-IRE:S2B-036)
\ex\label{ex:11} \is{concessives (types of)!dialogic}Dialogic concessive (\is{concessives (types of)!narrow-scope dialogic}narrow scope):\label{bkm:Ref496794746}\\
  I have tried to phone you a couple of times this week, \textbf{although} \textit{not persistently}. (ICE-GB:W1B-001)\\
\z

Example \REF{ex:8} is an \is{concessives (types of)!anticausal}anticausal concessive, whose semantic structure can be paraphrased as follows: ‘He is unsuccessful as a writer, but, \textit{unexpectedly, this does not result in a change of writing style or subject matter}.’ The italic part of the sentence points to what will be called a \is{topos}\textit{topos} in \sectref{sec:2.2.1} below, i.e. the assumption that, under normal circumstances, a certain set of circumstances will have certain consequences or results. The term \is{concessives (types of)!anticausal}\textit{anticausal} refers to the fact that the causal trajectory that is triggered is not consonant with the presented facts. That is, preconceptions held by the \is{addressee/reader}addressee or reader concerning causes and effects are activated for the decoding of the concessive, but the normally assumed \is{cause and effect}cause-and-effect relation remains unrealised.

The \is{concessives (types of)!epistemic}epistemic concessive in \REF{ex:9} can be paraphrased as follows: ‘Pictures always show Bonifacio [a Filipino revolutionary leader; OS] with a bolo, but, \textit{although this portrayal will naturally make us think so, he never actually fought with this kind of weapon}.’ The dependent clause in \REF{ex:9} expresses observed facts or phenomena that suggest or encourage (or, indeed, cause) certain conclusions. However, the observed facts cannot be construed as real-world causes (the brandishing of bolos~– presumably in portraits~– cannot possibly cause Bonifacio to have used them in the past).

\begin{sloppypar}
Inferences of the types discussed above (i.e. \is{inference}inferences concerning either causes or effects) are not central in constructing and decoding \is{concessives (types of)!dialogic}dialogic concessives like \REF{ex:10}, which can be paraphrased as ‘On the one hand, there is agreement concerning the facts of the matter~– this is helpful; on the other hand, historians are not agreed concerning what those facts mean~– this complicates things.’ In this case, two propositions make differently-angled contributions to the overall evaluation of a situation. In the example, the positive tone of the first one is dampened by the second one. Referring to such constructions as \is{concessives (types of)!dialogic}\textit{dialogic concessives} highlights the fact that the two propositions enter into some kind of dialogue, in the sense that both are subject to reciprocal pragmatic qualification and modification. The relationship between propositions might be argued to be \is{adverbials!contrast}adversative, rather than concessive in the strict sense of the word.
\end{sloppypar}

Finally, \is{concessives (types of)!narrow-scope dialogic}narrow-scope \is{concessives (types of)!dialogic}dialogic concessives are regarded not as an entirely independent category but as a subtype of \is{concessives (types of)!dialogic}dialogic concessives. Like in \REF{ex:10}, the subclause in \REF{ex:11} modifies the proposition in the matrix clause without triggering causal inferences. In this case, however, the (adverb) phrase introduced by \textit{although} does not have \isi{scope} over the entire main clause but only over its verb phrase. While \is{concessives (types of)!dialogic}dialogic in nature, \is{concessives (types of)!narrow-scope dialogic}narrow-scope \is{concessives (types of)!dialogic}dialogic concessives are treated as a separate category, since~– like in \REF{ex:11}~– the dependent clause does not present a new proposition, but essentially functions like a negatively-phrased adverbial of \is{adverbials!manner}manner. \tabref{tab:2.1} summarises the main traits of the different semantic types. More detailed discussions will follow in \sectref{sec:2.2.1}--\ref{sec:2.2.4}.\footnote{\label{bkm:Ref504053828}For a discussion of the different semantic types of concessives regarding their degrees of \isi{subjectivity}, see \citet{Crevels2000,Hilpert2013a} and \citet{Schützler2018b}; for general discussions of \isi{subjectivity} that can contribute to this kind of approach, see \textcites{Benveniste1971}[26–27]{HallidayHasan1976}{Traugott1989}{Traugott2010}{Traugott2014} and \citet{Langacker1985,Langacker1990}.}

\begin{table}
\caption{\label{tab:2.1}Semantic types of concessives}
\begin{tabularx}{\textwidth}{lX}
\lsptoprule
Type & Short description\\\midrule
\is{concessives (types of)!anticausal}anticausal & Proposition in dependent structure presents real-world obstacle to matrix clause proposition; understood and decoded on the basis of a \isi{topos}, i.e. an assumed “normal” relation of \isi{cause and effect}\\\tablevspace
\is{concessives (types of)!epistemic}epistemic & Proposition in dependent structure suggests certain conclusions; typically also based on a \isi{topos}; often functions like an inverted \is{concessives (types of)!anticausal}anticausal concessive, i.e. the observed effect suggests an assumed cause\\\tablevspace
\is{concessives (types of)!dialogic}dialogic (\is{concessives (types of)!wide-scope dialogic}wide scope) & Propositions in dependent structure and matrix clause present different perspectives on a single situation; proposition in dependent structure qualifies, modifies or corrects the matrix-clause proposition, but stands in no relation of \isi{cause and effect}\\\tablevspace
\is{concessives (types of)!dialogic}dialogic (\is{concessives (types of)!narrow-scope dialogic}narrow scope) & Semantic/pragmatic function like \is{concessives (types of)!dialogic}dialogic concessives; only part of the matrix-clause proposition (e.g. the VP and its complements) is qualified, modified or corrected \\
\lspbottomrule
\end{tabularx}
\end{table}

Like the syntactic categories discussed in \sectref{sec:2.3}, semantic categories will be simplified for the central quantitative analyses by including only \is{concessives (types of)!anticausal}anticausal and \is{concessives (types of)!wide-scope dialogic}wide-scope dialogic CCs. \is{concessives (types of)!epistemic}Epistemic concessives, while theoretically interesting, are rare in the data, and their inclusion in regression models would generate more problems than real insights. \is{concessives (types of)!narrow-scope dialogic}Narrow-scope \is{concessives (types of)!dialogic}dialogic CCs~– apart from also being relatively rare~– are syntactically highly restricted, i.e. they hardly partake in formal variation as defined in this study.

\subsection{Anticausal concessives}\label{sec:2.2.1}\label{bkm:Ref470953076}\label{bkm:Ref487191503}\label{bkm:Ref490761556}\label{bkm:Ref496799325}\label{bkm:Ref81906278}\label{bkm:Ref118500084}

As already outlined above, an \is{concessives (types of)!anticausal}anticausal concessive is constructed and decoded based on a \is{topos}\textit{topos} (\citealt{Azar1997}: 306, \citealt{Anscombre1989}), which is a \isi{presupposition} in the form of an \textsc{if}\,→\,\textsc{then} relation shared (i.e. understood) by the \is{speaker/writer}speaker or writer (for short: \is{speaker/writer}SP/W) and the \is{addressee/reader}addressee or reader (\is{addressee/reader}AD/R; cf. \citealt{König2006,Givón1990}: 835).\footnote{There is a wealth of alternative terms, e.g. “hypothesis” (\citealt{Burnham1911}: 1–2), “\isi{presupposition}” \citep{König1991a}, and “assumption” \citep{König2006}.} Topoi can be very general or nearly universal, in which case they require very little (or no) context; on the other hand, they can also be highly context-specific, in which case the \isi{topos} is valid only for a particular communicative situation, a certain time period, or a certain speech community or culture. An example of a universal \isi{topos} is perhaps \textsc{little sleep}\,→\,\textsc{tiredness}, i.e. ‘if you sleep little, you will be tired’.\footnote{I will use small caps to highlight generalised relationships between propositions in concessive constructions.} It is reasonable to assume that this mechanism will be operative irrespective of time, place and social factors, because it is part of the physical human condition. On the other hand, a \isi{topos} may also be more restricted, e.g. regionally or historically. Take, for example, the construction \textit{Although only 22 years old, he was allowed to vote}. This would make little sense in present-day western societies. In late nineteenth-century Prussia, however, men were allowed to vote only if they were at least 24 years old, so different assumptions hold for this particular historical political system, making the above sentence perfectly functional and easy to decode in that context.

The view of concessives as based on \is{adverbials!reason}causal or \is{adverbials!condition}conditional relationships is also reflected in \citeauthor{QuirkEtAl1985}’s (\citeyear[484]{QuirkEtAl1985}) treatment of concession as “an ‘inverted’ condition” or “a ‘blocked’ or inoperative cause”, as well as in \citegen[272]{HallidayMatthiessen2004} description of concessives as construing “frustrated cause”. The term \is{concessives (types of)!anticausal}\textit{anticausal} in the present study is intended to be a more transparent reflection of this underlying mechanism than the term “\is{concessives (types of)!content}content concessive” (\citealt{Sweetser1990,Crevels2000,Hilpert2013a}: 78).

Example \REF{ex:12} hinges upon the (general) \isi{topos} \textsc{making haste} → \textsc{punctuality}.\footnote{Other typical \is{topos}topoi would be \textsc{hard work} → \textsc{success}, \textsc{little to drink} → \textsc{thirst}, and \textsc{active social life} → \textsc{not feeling lonely}.} It appears sensible to describe \is{topos}topoi by using maximally general formulations like this, so that they can capture a large number of actual realisations (cf. \citealt{König1991b}: 633, \citealt{Hermodsson1994}: 73).

\ea\label{ex:12}   \label{bkm:Ref470956499}\textbf{Although} \textit{I ran fast}, I missed the bus.
\z

Example \REF{ex:13} is reproduced from \citet[79]{Sweetser1990}. Someone who is not aware of an emergency (because they have not heard the call for help) will normally not come to the rescue. A more general \isi{topos} on which the construction is based could be formulated as \textsc{unawareness of problem} → \textsc{inactivity}.

\ea\label{ex:13}   \label{bkm:Ref489004948}\textbf{Although} \textit{he didn’t hear me calling}, he came and saved my life.
\z

In the terminological framework proposed by \citet[308]{Azar1997}, \is{concessives (types of)!anticausal}anticausal concessives are “\isi{persuaders}”. They do not provide additional evidence in favour of the (unmarked) primary statement but make it more convincing: Firstly, they anticipate, make explicit and thus disarm facts that might otherwise be used to undermine or discredit the main proposition; secondly, they increase the credibility of \is{speaker/writer}SP/W, who presents a more multifaceted, balanced and complete picture of the situation and thus comes across as circumspect and thorough. \Citet[341]{DiMeola1998} argues that by mentioning an obstacle in the antecedent, the consequent proposition is highlighted, appearing less natural. In addition, \is{addressee/reader}AD/R’s curiosity may be piqued, and their thoughts may be directed towards a yet unknown cause for the non-realisation of the default causality, thus potentially contributing to the coherence of a text by foreshadowing further evidence provided in subsequent parts of the discourse.

 The cause in a \isi{topos} may be construed indirectly, as in the following example. Here, tipsiness does not result directly in incomprehensible speech. Rather, it may result in slurred or indistinct speech, which in turn is likely to make spoken messages incomprehensible.

\ea\label{ex:14}   But we gathered, \textbf{although} \textit{they were tipsy}, that their first names were Lily and May. (ICE-IRE:S2B-021)
\z

The logical chain of causal relations in the example could run as follows:
(i)~‘Because they were tipsy, their speech was indistinct’;
(ii)~‘although their speech was indistinct, we understood that their names were Lily and May’. Thus, two topoi are effectively fused into one: \textsc{drunkenness} → \textsc{obscure} \textsc{speech} → \textsc{communication} \textsc{problems}. The intermediate ‘obscure speech’ element is left entirely implicit, but \is{speaker/writer}SP/W can safely rely on \is{addressee/reader}AD/R’s ability to fill in the gap, based on their \isi{world knowledge}.

\subsection{\label{bkm:Ref479420416}Epistemic concessives}\label{sec:2.2.2}

Like anticausal concessives, \is{concessives (types of)!epistemic}epistemic concessives are often based on \is{topos}topoi. In this case, however, the two propositions are not held together by an \textsc{if}\,→\,\textsc{then} relation~– at least not in the same way as in an anticausal CC. \citet[165–167]{Hilpert2013a} describes the difference as follows: Epistemic concessives, like anticausal concessives, invoke a “\isi{causality frame}” (which can, for practical purposes, be equated with a \isi{topos}; see \sectref{sec:2.2.1} above). In \is{concessives (types of)!epistemic}epistemic concessives, however, the \isi{causality frame} is not based on “real-world causation” but on “\isi{inference}” (\citealt{Hilpert2013a}: 165; cf. \citealt{Crevels2000}: 318).

In \REF{ex:15}, observing that someone has failed her final exams may lead to certain conclusions, among them perhaps that the person in question is not a particularly gifted student. This conclusion, however, turns out not to be in harmony with reality in this case:

\ea\label{ex:15} \label{bkm:Ref488739088}  She is a very clever student, \textbf{although} \textit{she failed her finals}.\\
\z

Two important notes need to be made. First, it is crucial that the conclusion based on the proposition in the subordinate clause is what could be called \textit{regressive}, i.e. it concerns states of affairs (or processes and actions) that are prior to (or underlying) the italicised proposition. In the example, the relevant relationship is not between failing the exams and its possible real-world consequences, but between failing the exams and possible causes, facts or personality traits that can account for it. Secondly~– and this characteristic is shared between \is{concessives (types of)!anticausal}anticausal and \is{concessives (types of)!epistemic}epistemic concessives~– we cannot assume that the content of the subclause triggers a highly specific conclusion. Rather, the semantic structure of the construction as a whole is such that the main-clause proposition contrasts with one of numerous possible \is{inference}inferences triggered by the proposition in the subordinate. Thus, in the example, failing one’s exams could be due to a lack of talent, preparation, interest in the subject, or physical/mental fitness.

Example \REF{ex:16} is taken from \citet[79]{Sweetser1990}, where it is presented along with the sentence reproduced as \REF{ex:13} above. The propositional content is the same, but in \REF{ex:16} the past perfect is used to make explicit that the subordinate clause is shown in relation to a prior event (see comments above).

\ea\label{ex:16}   \label{bkm:Ref489005139}\textbf{Although} \textit{he came and saved me}, he hadn’t heard me calling for help.\\
\z

As the examples in this and the previous section show, \is{concessives (types of)!anticausal}an anticausal concessive can in many cases be turned into an \is{concessives (types of)!epistemic}epistemic concessive (and vice versa) by re-attaching the concessive marker and thus changing the status of main and subclause, possibly supported by an additional adjustment of tenses. On the basis of examples like these, \is{concessives (types of)!epistemic}epistemic concessives can be regarded as \is{concessives (types of)!anticausal}anticausal concessives with inverted semantic polarity.\footnote{Both \citet[1098]{QuirkEtAl1985} and    \citet[735]{HuddlestonPullum2002} point out that the concessive marker can be attached to either of two clauses. However, only Huddleston \& Pullum discuss the fact that moving the connective from the head of one clause to the head of another changes “[t]he implicature”~– they effectively describe the difference between anticausal and epistemic types, without using those terms.} In the same vein, \citet[345–346]{DiMeola1998} calls \is{concessives (types of)!epistemic}epistemic concessives “reconstructive” and points out that they are characterised by a reversal of the “argumentative direction”.\footnote{German: “rekonstruktiv”; “Argumentationsrichtung”.}

\subsection{\label{bkm:Ref426019070}Dialogic concessives}\label{sec:2.2.3}

In this study, the term \is{concessives (types of)!dialogic}\textit{dialogic} will be used for a relatively broad category of concessives. What they have in common is the absence of \is{inference}inferences concerning likely effects and outcomes (as in \is{concessives (types of)!anticausal}anticausal CCs) or likely underlying causes or states of affairs (as in \is{concessives (types of)!epistemic}epistemic CCs). Instead, in the most general definition of \is{concessives (types of)!dialogic}dialogic concessives, one of the two propositions qualifies the other, qualitatively or in degree, or both propositions present conflicting evidence and thus suggest different courses of action or different evaluations of the situation as a whole. \citet{Sweetser1990}, \citet{Crevels2000} and \citet{Hilpert2013a} use the term “\is{concessives (types of)!speech-act}speech-act concessive” for this type of construction.\footnote{\label{bkm:Ref82082041}Publishing her book in the year of J.~L.~Austin’s birth, \citet[33]{Burnham1911} naturally does not use the term “\is{speech acts}speech act”, although she describes the \is{concessives (types of)!speech-act}speech-act type (which I call \textit{dialogic}) when she writes that \il{Old English}OE concessives with \textit{þeah} are “sometimes used loosely to relate or contrast two ideas between which there is no logical opposition”, in which case the concessive clause “is added simply as a qualifier”.} I would argue that, although this term is a good label for certain types of concessives, a broader designation is needed, particularly since different notions exist as to what precisely constitutes a \is{speech acts}speech act. Still, the term \is{concessives (types of)!dialogic}\textit{dialogic concessive} as I use it is co-extensive in meaning with the term “\is{concessives (types of)!speech-act}speech-act concessive”, and therefore \citegen[167]{Hilpert2013a} following definition of \is{concessives (types of)!speech-act}speech-act concessives clearly applies here:

\begin{quote}
With the first element, the speaker makes a pragmatic commitment that would, in a default scenario, cause her or him to make subsequent statements consistent with that commitment. Yet, the commitment is withdrawn, and this is signalled with the use of a concessive conjunction.
\end{quote}

The withdrawal of \isi{pragmatic commitment} described by Hilpert is what I call \is{modification (or qualification)}\textit{qualification} or \is{modification (or qualification)}\textit{modification}: One proposition sets the discourse off on a certain pragmatic trajectory, suggesting certain evaluations or courses of action, while the second proposition qualifies, weakens, or indeed cancels that pragmatic trajectory. \citet[318]{Crevels2000} argues more purely in terms of \isi{speech acts}:

\begin{quote}
In the speech-act domain the content of the concessive clause does not form an obstacle for the realization of the event or the state of affairs described in the main clause, but raises obstacles for the realization of the \is{speech acts}speech act expressed by the speaker in the main clause.
\end{quote}

Any qualification, weakening or withdrawal of “\isi{pragmatic  commitment}” \citep[167]{Hilpert2013a} can be regarded as an obstacle to the realisation of a \is{speech acts}speech act, if \isi{speech acts} are defined broadly enough. As a motivation of the term \is{concessives (types of)!dialogic}\textit{dialogic}, it could be argued that the two propositions in constructions of this type enter into a dialogue with each other, one of them promoting certain evaluations or courses of action, the other providing additional and potentially conflicting information with an impact on how the concessive as a whole is to be interpreted. At the level of discourse, it could further be argued that \is{speaker/writer}SP/W presents an unresolved situation, and that meaning-making ultimately depends on how \is{addressee/reader}AD/R engages with this. Thus, the process involves an inter-propositional dialogue as well as an \is{intersubjectivity}intersubjective one.

A prototypical case of what is called “speech-act concessive” in the literature is shown in \REF{ex:17}: The unmarked clause contains a declaration (\textit{I’m innocent}), possibly meant to encourage a supportive course of action in \is{addressee/reader}AD/R; at the same time, however, \is{speaker/writer}SP/W presents a second proposition (\textit{I know you won’t believe me}) which reduces the probability of the matrix-clause \is{speech acts}speech act being felicitous. There is no \isi{topos}-based, factual incompatibility between being innocent and expecting to be disbelieved. The relatedness of different types of CCs becomes once again evident if we expand the matrix clause in the example (\textit{I am saying that I’m innocent, although...}), for instance.

\ea\label{ex:17}\label{bkm:Ref427151328}I’m innocent, \textbf{although} \textit{I know you won’t believe me}. (from \citealt{Sweetser1990}: 81)\\
\z

Example \REF{ex:18} from ICE-GB is another instance that is not only \is{concessives (types of)!dialogic}dialogic but truly speech-act in nature. Here, the unmarked clause contains the writer’s birthday congratulations, while the concessive clause expresses the certainty that the letter expressing them will arrive late, effectively making the congratulating \is{speech acts}speech act less felicitous.

\ea\label{ex:18}\label{bkm:Ref427151342}Happy 25th Birthday for Monday, \textbf{although} \textit{this letter will arrive days and days after your birthday}. (ICE-GB:W1B-005; comma added)\\
\z

In (\ref{ex:19}a), taken from \citet[165]{Hilpert2013a}, the \is{speech acts}speech act is less clearly identifiable as such. It is the frequent occurrence of CCs of this kind that led me to explore alternative, more general labels for this semantic category, resulting in the term \is{concessives (types of)!dialogic}\textit{dialogic} concessive.

\ea\label{ex:19}
    \ea\label{ex:19a}\label{bkm:Ref427151356}\textbf{Although} \textit{surgery is best}, it is not always possible. \citep[165]{Hilpert2013a}\\
    \ex\label{ex:19b}Surgery is best, \textbf{although} \textit{it is not always possible}.\\
\z
\z

The proposition in the subordinate clause of (\ref{ex:19}a) focuses on the fact that surgery is the best option available in a certain situation, while the matrix clause states that it is not always feasible. There is no \isi{topos} whereby the best solution can generally be expected to be viable; what the construction as a whole does is qualify the pragmatic stance of one proposition (implicitly recommending/promoting surgery) by introducing another, which indicates certain complications or restrictions. In constructions of this type, it is possible to simply re-attach the concessive marker and thus change the status of clauses with only minor effects on the function of the construction as a whole, as shown in (\ref{ex:19}b). I would argue that this is because, unlike \is{concessives (types of)!anticausal}anticausal and \is{concessives (types of)!epistemic}epistemic concessives, \is{concessives (types of)!dialogic}dialogic concessives do not involve \is{inference}inferences in the stricter sense, and thus the link between propositions has no particular directionality.

In the next (constructed) example, two characteristics of a person~– good looks and low intelligence~– are contrasted, using \textit{although}. There is no conflict between the propositions themselves (beauty vs lack of intelligence), i.e. there is no real-world reason to assume that the two should not co-occur. However, the positive stance of the main clause (presenting the subject as physically attractive) is downgraded by the proposition contained in the subordinate clause (presenting the subject as intellectually \textit{un}attractive). In this case, the concessive relation holds between two evaluative stances, and the final position needs to be negotiated dialogically, based on the evidence. This sentence could be paraphrased as follows: ‘His good looks make him an exciting companion; but then again his lack of intelligence might make him boring or even embarrassing company.’\footnote{Concessives like \REF{ex:20} are called “evaluative” (German “evaluativ”) by \citet[345]{DiMeola1998}, who provides a similar German example. He (\citeyear[347–348]{DiMeola1998}) also discusses “limiting” and “corrective” CCs (German “limitativ”, “korrektiv”), which form a continuum, the latter expressing a stronger \is{modification (or qualification)}qualification or correction than the former. \citet[823–824]{König2006} speaks of “rectifying” concessives, in which “the content of the main clause is weakened”. He further claims that the (marked) rectifying clause always follows the matrix clause.}

\ea\label{ex:20}\label{bkm:Ref427176372}He is really good-looking, \textbf{although} \textit{he’s not very bright}.\\
\z

Finally, in \REF{ex:21} the fact that undernourishment in Argentina is at a relatively low rate (presented as an achievement) is qualified by adding that this is only possible due to state support (i.e. the achievement comes at a cost); the subordinate addition clearly makes the matrix-clause message appear less impressive.

\eanoraggedright\label{ex:21}\label{bkm:Ref427227390}\sloppy In Argentina, […] only some 8 per cent of the population is under\-nourished, \textbf{though} \textit{the National Food Programme is now needed to ensure that food is available}. (ICE-GB:W2A-019)
\z

Dialogic concessives, then, can serve a number of purposes that are not mutually exclusive:
(i)~They can (ostensibly) express a complex situation in a more objective way by providing contrasting perspectives, which may enhance \is{speaker/writer}SP/W’s standing in the eyes of \is{addressee/reader}AD/R, as they will appear more circumspect and considerate;
(ii)~they enable \is{speaker/writer}SP/W to avoid taking a clear stance and thus responsibility for consequent actions and decisions; and
(iii)~they can give \is{addressee/reader}AD/R more interpretative leeway. Because \is{speaker/writer}SP/W avoids taking an entirely clear stance and several interpretations are possible, \is{concessives (types of)!dialogic}dialogic concessives are pragmatically “mixed messages” \citep[166]{Hilpert2013a}.

\subsection{\label{bkm:Ref496776303}\label{bkm:Ref496778892}Narrow-scope dialogic concessives}\label{sec:2.2.4}

As the label suggests, \is{concessives (types of)!narrow-scope dialogic}narrow-scope dialogic concessives are treated as a subtype of \is{concessives (types of)!dialogic}dialogic concessives. They are more limited in semantic scope and the dependent structure lacks syntactic mobility.

The narrow semantic scope of this type of CC can be seen in \REF{ex:22}. The dependent negative adverb phrase introduced by the connective does not comment on the entire matrix clause proposition but constitutes a qualifying addition to one aspect only, namely the degree of improvement.

\ea\label{ex:22}   \label{bkm:Ref496860619}It improved on a standard Philips design, \textbf{though} \textit{not a great deal}. (ICE-GB:W2B-040)\\
\z

Example \REF{ex:23} is perhaps an even clearer illustration of this semantic type: \textit{reluctantly} is a modification of the VP (\textit{agreed}). In order to give the dependent part of the CC \isi{scope} over the entire matrix clause, one could use an adjective phrase (AdjP) instead of an adverb phrase (AdvP): \textit{Though reluctant, the child agreed}. Alternatively, one could restate the subject along with a \is{resumptive predicate}resumptive (dummy) predicate: \textit{…though she did so reluctantly}.

\ea\label{ex:23}   \label{bkm:Ref496860621}The child agreed, \textbf{though} \textit{reluctantly}. (ICE-IND:W2B-018)\\
\z

\is{concessives (types of)!narrow-scope dialogic}Narrow-scope concessives will be treated as strictly of the dialogic type and strictly \is{nonfinite clauses}nonfinite.\footnote{As discussed in the text, it is possible to re-construct \REF{ex:23} using a finite clause (\textit{…though she did so reluctantly}), but this kind of construction simply did not occur in the data.} Concerning syntactic ordering in \is{concessives (types of)!narrow-scope dialogic}narrow-scope CCs, it is almost categorically the case that the connective and its complement~– usually an AdvP or a preposition phrase (PP)~– follow the matrix clause. Rearrangements will result in ungrammatical constructions (*\textit{Though not very quickly he answered the phone}; *\textit{He though not very quickly answered the phone}). In ICE-Philippines~– not included in the quantitative part of this study~– there was a single example in which the typical sequence of elements was inverted, as shown in \REF{ex:24}. The canonical form would be either \textit{significantly, though not fully}, or perhaps \textit{not fully, though significantly}. The fact that this occurs in an \is{varieties of English!L2}L2 variety is in accordance with the finding that (some of) those varieties may treat connectives somewhat differently, sometimes using a second, \is{correlative markers}correlative marker (in this case: \textit{though… but…}).\footnote{See, for instance, \REF{ex:69} and \REF{ex:70} from \il{Indian English}IndE and \il{Hong Kong English}HKE (\sectref{sec:3.5},  p. \pageref{bkm:Ref488839457}).}

\ea\label{ex:24}   \label{bkm:Ref496862678}This phenomenon was, \textbf{though} \textit{not fully} \textbf{but} significantly, explained by the Sapir-Whorf theory, […]. (ICE-PHI:W1A-007)\is{if@\textit{but}}\\
\z

\begin{sloppypar}
Examples like \REF{ex:25} and \REF{ex:26} were also categorised as \is{concessives (types of)!narrow-scope dialogic}narrow-scope, even though they function somewhat differently. In both cases, the attribute of a following noun is postmodified by an AdjP marked for concession.
\end{sloppypar}

\ea\label{ex:25}   \label{bkm:Ref496860631}[A]nd the study of Latin occupied a subsidiary, \textbf{though} \textit{nonetheless important}, place in the curriculum of the Scottish universities. (ICE-CAN:W2A-008)\\
    \ex\label{ex:26} \label{bkm:Ref496860632}They have a surprisingly loud, \textbf{though} \textit{squeaky}, voice for so tiny a bird. (ICE-JAM:W2B-021)\\
\z

Crucially, all examples in this section are characterised by subclausal postmodification, be it within AdvPs, AdjPs or relative to entire VPs. Secondly, rigid constraints are in place concerning the syntactic placement of \is{concessives (types of)!narrow-scope dialogic}narrow-scope CCs. And, finally, it is grammatically not possible for \is{concessives (types of)!narrow-scope dialogic}narrow-scope dialogic CCs to be constructed with \is{finite clauses}finite clauses, which is a direct result of subclausal status. Rather, CCs of this type employ an AdjP (if postmodifying an AdjP) or an AdvP/PP (if postmodifying an AdvP or a VP).

\section{\label{bkm:Ref35085173}The syntax of concessive subordination}\label{sec:2.3}

This section deals with the syntactic properties of CCs with \textit{although}, \textit{though} and \textit{even though}. Section \ref{sec:2.3.1} discusses the general aspect of syntactic ordering, i.e. the positioning of subclauses relative to matrix clauses, while \sectref{sec:2.3.2} focuses on basic types of clauses (or clause-equivalent structures) that combine with the three connectives.

Subordinating conjunctions introduce clauses that depend on another clause, the super\-ordinate clause. In English, this dependent status is made syntactically explicit by the subordinator, as in \REF{ex:27} and \REF{ex:28}, in which only the conjunctions \textit{although} and \textit{though} indicate that their complement clauses are subordinate to the matrix clause.\footnote{\ili{German}, in contrast, employs a subordinator and verb-final syntax in the subordinate clause.}

\ea\label{ex:27} \label{bkm:Ref470886714}  \label{bkm:Ref489777815}\textbf{Although} \textit{he collected only 603 votes}, he remains undaunted. (ICE-CAN:W2C-020)\\
    \ex\label{ex:28} \label{bkm:Ref470886719}  \label{bkm:Ref489777822}Representatives of many different nations camp in the town, \textbf{though} \textit{most are French and Spanish}. (ICE-GB:S2B-027)\\
\z

The subordinate clause is treated as a constituent of the sentence by \citet[987]{QuirkEtAl1985}, who argue that it is “downgraded to a subclausal unit”. This is essentially why a clausal construction can be substituted with a \is{preposition phrases}prepositional one (e.g. \textit{although he failed} → \textit{despite his failure}).

\begin{sloppypar}
It has been proposed that concessives are characterised by syntactic constraints that set them apart from other adverbials. Since they are of little relevance to the central analyses of this study, those aspects will only be discussed in the following summary. \citeauthor{König2006} (\citeyear{König2006}: 821; cf. \cites[679]{König1994}[192]{König1991a}[149–151]{König1988}) highlights four syntactic properties of concessive constructions, focusing on subordinate clauses:
\end{sloppypar}

\begin{enumerate}
\item There are no interrogative adverbs for concessives, corresponding to \SchuetzlerIndexExpression{when} for \is{adverbials!time}temporal clauses and \SchuetzlerIndexExpression{why} for \is{adverbials!reason}causal clauses;
\item it is not possible to combine concessives with \isi{focusing particles} like \textit{only}, \textit{just}, \textit{especially} (*\textit{He answered her only although he didn’t want to}.);
\item concessives cannot be focused in a \is{cleft sentences}cleft sentence (*\textit{It was although} X \textit{that} Y.); and
\item it is not possible to focus a concessive by negation or using a polar interrogative.
\end{enumerate}\largerpage

In illustration of the fourth point, the question \textit{Did he fail the exam because he was unprepared?} can be answered with \textit{No, he failed because the questions were not fair}, while it is not possible to answer the question \textit{Did he fail the exam although he was well prepared?} by saying \textit{No, he failed the exam although he cheated}. The question using \textit{although} is not ungrammatical, but~– in contrast to the question using \textit{because}~– it can only have wide scope and will receive the respective answer.

As argued by König (\citeyear{König2006}: 821; cf. \citealt{König1991a}: 191, \citealt{Crevels2000}: 314), the specific constraints listed above point to a more general constraint whereby concessive clauses “cannot be focused against the background of the rest of the sentence”, which is interpreted as a symptom of their lack of syntactic integration into the matrix structure and as an indication that, in certain respects, concessive subordinates behave more like paratactic elements \citep[821]{König2006}.

Concerning the subordinators themselves, \citet[998–999]{QuirkEtAl1985} subdivide them into \is{subordinators!simple}“simple”, \is{subordinators!complex}“complex” and \is{correlative markers}“correlative”. \is{subordinators!simple}Simple \isi{subordinators} like \SchuetzlerIndexExpression{although} and \SchuetzlerIndexExpression{though} consist of a single word, while \is{subordinators!complex}complex ones consist of several words (e.g. \SchuetzlerIndexExpression{even though}, which Quirk et al. do not mention in this context, however). \is{correlative markers}Correlative subordinators consist of two markers, one attached to the subclause and one to the matrix clause (cf. \citealt{Rudolph1996}: 227). Some CCs with a correlative use of markers will be discussed in \sectref{sec:3.5} in the next chapter.

\subsection{\label{bkm:Ref487447653}Syntactic ordering}\label{sec:2.3.1}\largerpage

This section focuses on the positions of subordinate clauses relative to the corresponding matrix clauses. Subordinate adverbial clauses can occur in \is{initial position}initial, \is{medial position}medial and \isi{final position}, and this is the terminology that will be used in this study. \citet[1037]{QuirkEtAl1985} also refer to \isi{initial position} as “left-branching”, \isi{medial position} as “nested”, and \isi{final position} as “right-branching”, while   \citet[779]{HuddlestonPullum2002} speak of “front”, “central” and “end” position.

According to \citet[780]{HuddlestonPullum2002}, elements in \is{initial position}initial position are placed before the subject; elements in \is{medial position}medial position are placed before the verb (and after the subject); and elements in \is{final position}final position are placed after the verb. \citet[1039–1040]{QuirkEtAl1985} argue that subclauses in \is{final position}final position are easiest to process, while \is{initial position}initial and \is{medial position}medial clauses are more difficult to process, particularly if the subclause is long or complex (cf. \citealt{HuddlestonPullum2002}: 780; see also discussion below). Some authors focus entirely on the difference between \is{initial position}initial and \is{final position}final placement (e.g. \citealt{Chafe1984}: 437, \citealt{WiechmannKerz2013}: 1, 7, \citealt{Diessel2005}: 452), which also makes quantitative analyses more straightforward.\footnote{Thus, \citet[footnote 2]{WiechmannKerz2013} disregard concessive clauses in sentence-medial position, partly perhaps because they fit a binary \is{regression!logistic}logistic regression model to their data. Both initial and medial position are coded as “nonfinal”.} In this study, the approach followed in the statistical analyses is to treat clause position as a binary variable, with the two categories “final” and “nonfinal” (cf.~\sectref{sec:6.3.6}).

\citet[21]{Altenberg1986} argues that a preposed (i.e. sentence-\is{initial position}initial) subordinate clause has a “\isi{grounding}” function~– that is, it provides background information against which the (more important) information in the main clause is presented. This arrangement of clauses entails more rigorous advance planning on the part of SP/W, and for Altenberg this is the reason why sentence-\is{initial position}initial subordinate clauses are more likely in writing, which is charac\-te\-rised by lower time pressure and allows for post-hoc editing (cf. \citealt{Diessel2005}: 452). By contrast, conversation is characterised by planning in “real time”, with “locally managed” units (including main and subordinate clauses): “[W]hen planning is not far ahead of production, it is easier to qualify a superordinate idea retrospectively (by postposition) than to anticipate it by means of \isi{grounding} (pre-position)” \citep[21]{Altenberg1986}.\footnote{\citet[20–24]{Altenberg1986} uses the term “\isi{contrastive sequencing}” to refer to the ordering of clauses in contrastive (including concessive) constructions. In the present study, more neutral terms like \textit{syntactic ordering} or \textit{clause position} are used, since the phenomenon is far from being unique to adversative/concessive constructions.}

\citet[1036]{QuirkEtAl1985} apply the principle of \is{resolution}\textit{resolution} to account for the ordering of clauses in complex sentences, saying that “the \is{final position}final clause should be the point of maximum \isi{emphasis}” (cf. “\isi{communicative dynamism}” in \citealt{QuirkEtAl1985}: 1556–1557). They use this concept as a sentence-level equivalent of \is{end-focus}\textit{end-focus}, which applies at the level of the clause, i.e. as a mechanism that can account for the variable arrangement of subordinate and matrix structures in terms of \isi{information structure} and focus (cf. \citealt{Chafe1984}: 440; see also discussion in \citealt{Schützler2018a}).

A fine discussion of competing motivations in the placement of an adverbial clause relative to a matrix clause is provided by \citet{Diessel2005}, who does not, however, deal specifically with concessives. The three factors Diessel identifies are related to \isi{processing}, discourse-pragmatics and semantics. In support of the first principle, and largely based on Hawkins’s (\citeyear{Hawkins1990, Hawkins1992, Hawkins1994, Hawkins2000}) “performance theory of order and constituency”, \citet[458–459]{Diessel2005} argues that adverbial clauses in sentence-\isi{final position} are preferable, from the perspectives of both \isi{production} and \isi{parsing}:
(i)~Since the matrix clause is constructed first, SP/W does not need to make an early commitment to a complex sentence structure and is thus relieved of advance \isi{planning};
(ii)~since the subordinator follows the matrix clause, it marks the entire sentence as complex and indicates the boundary between matrix and subordinate clause at the same time; finally,
(iii)~no (or at least much weaker) constraints are placed upon a subordinate clause in \isi{final position} concerning its \is{length (of clauses)}length (or \isi{weight}). On this basis, the \is{initial position}initial placement of subordinate clauses appears as the marked solution which needs to be motivated.

One such motivating factor competing with processing-based constraints is what \citet[459–461]{Diessel2005} calls “\is{discourse pragmatics}discourse-pragmatic forces”. Although the two concepts are not exactly coextensive, I will discuss \isi{discourse pragmatics} in information-structural terms (cf. \citealt{Chafe1976,Chafe1984}: 440, \citealt{Lambrecht1994,Krifka2008,BrintonBrinton2010}: 324–329; also cf. \citealt{WiechmannKerz2013}: 3, 6). For example, in \REF{ex:29} the sentence-\is{initial position}initial subordinate clause headed by \textit{although} is placed at the junction of two somewhat differently angled sections of the discourse, establishing an elegant transition between them and anchoring the following passage on the antecedent. With regard to \isi{information structure}, it is also the case that certain specifics concerning the “initiative” referred to in the italicised subordinate clause have been established in the foregoing discourse. We would therefore expect a strong tendency for subordinate structures of this kind to precede the respective matrix clause.

\ea\label{ex:29}   \label{bkm:Ref469857390}The economic and social cost of the robberies prevented in the first two years of the initiative is estimated to have been between 107 and 130m, which exceeds the average annual cost (24.1m per year) of the initiative. \textbf{Although} \textit{the initiative itself has ended}, funding has been made available to the ten Street Crime Initiative forces in 2005/2006. (BE06, miscellaneous prose)\\
\z

Reference to earlier parts of the discourse is particularly obvious if there is what I call an \is{anaphora}\textit{anaphoric} element, i.e. a demonstrative \textsc{pro}-form, e.g. \textit{this}, as shown in \REF{ex:30}.\footnote{See the discussion of \citet{WiechmannKerz2013} in \sectref{sec:5.1.3}.}    \citet[6]{WiechmannKerz2013} call this a “\isi{bridging}” context, because the concessive construction (consisting of main and subordinate clause) is explicitly tied into the earlier discourse.

\eanoraggedright\label{ex:30}   \label{bkm:Ref490744512}A notable feature of this study was the number of patients who died before their third dialysis session, often during or immediately after the first dialysis. \textbf{Although}~\textit{this group is biased in favour of patients with the most severe disease} it may indicate the stress of acute haemodialysis on a compromised cardiovascular system has an adverse effect. (ARCHER, medical writing, 1985)
\z

Finally, Diessel (\citeyear{Diessel2005}: 461–465; cf. \citealt{Diessel2008}) discusses semantic factors that influence clause placement. He argues, for example, that prototypical \SchuetzlerIndexExpression{if}-clauses are predominantly placed in sentence-\isi{initial position}: They establish a specific semantic frame for the interpretation of following clauses, namely “if A then B (otherwise C)”, and it is implied that the early position of the \SchuetzlerIndexExpression{if}-clause is needed to enable a smooth \isi{processing} of the clauses following the \SchuetzlerIndexExpression{if}-condition. Further, in a rearranged sentence of the form “B if A (otherwise C)”, B will initially be interpreted as factual, but then needs to be reinterpreted as hypothetical, which “disturbs the information flow” \citep[462]{Diessel2005} and is thus not ideal. I would argue that \isi{iconicity} (which Diessel mainly discusses for temporal and causal clauses) can also account for the typical sequence: The \SchuetzlerIndexExpression{if}-condition in A needs to be met before B can be realised, so that the natural chronology of events (\textsc{condition~}→~\textsc{consequence}) finds its correlate in the syntactic arrangement of clauses. The argument for \isi{iconicity} as a motivating factor can also be made with regard to concessives: As shown in \sectref{sec:2.2}, many concessive constructions are based upon an underlying \textit{if}-\textit{then} relation. Even though the expected outcome or consequence is suspended (or unrealised), one can hypothesise that the natural sequence (\textsc{if}\,→\,\textsc{then}) will be iconically represented by the respective arrangement of clauses. This is perhaps why, as pointed out by \citet[232]{Rudolph1996}, examples in theoretical discussions often (and in disagreement with actual usage, as shown in \chapref{sec:9}) seem to suggest that the subordinate clause typically precedes the matrix clause.

\subsection{\label{bkm:Ref35178961}\label{bkm:Ref35269006}\label{bkm:Ref35363629}\label{bkm:Ref35418089}\label{bkm:Ref35769817}Clause types}\label{sec:2.3.2}

In the present study, two complement types of \textit{although}, \textit{though} and \textit{even though} are accepted as distinct syntactic categories and will be explained and exemplified in detail below:
(i)~finite clauses, including \is{subjunctive}subjunctives and \is{though@\textit{though}!inversion@\textit{though}-inversion}\textit{though}-in\-ver\-sion, and
(ii)~\is{nonfinite clauses}nonfinite clauses, including present and past participle clauses, as well as \isi{verbless clauses}.\footnote{Grammatical descriptions of subordinate clauses usually distinguish between \is{finite clauses}finite, \is{nonfinite clauses}nonfinite and \isi{verbless clauses} (e.g. \citealt{QuirkEtAl1985}: 992, \citealt{Diessel2005}: 451; cf. \citealt{Givón1990}: 839). The latter two are treated as a single category in this study.}

Finite indicative clauses in combination with concessive conjunctions~– as shown in \REF{ex:27} and \REF{ex:28} above~– are the most frequent type in the present study. Subjunctives are extremely rare, with only a single example found in the nine components of ICE.\footnote{More \is{subjunctive}subjunctives are found in the extended \is{corpora!xBrown}Brown family of corpora, particularly in \is{corpora!Brown}Brown, \is{corpora!BBrown}BBrown and \is{corpora!BLOB}BLOB, i.e. in data that are older and/or from \il{American English}AmE (cf. \citealt{Crawford2009,Kjellmer2009,Schlüter2009}).} This is shown in \REF{ex:31}, while \REF{ex:32} is another example from the BLOB corpus (\citealt{LeechSmith2005}):\largerpage

\ea\label{ex:31}   \label{bkm:Ref489781174}Mr Dodds says he is quite sorry, and even shook him by the hand when he said goodbye, which is going a bit far to my way of thinking, \textbf{though} \textit{he be a fine upstanding young fellow}. (ICE-GB:W2F-005)
\ex\label{ex:32} \label{bkm:Ref489781592}It is sometimes necessary to remove the second molar, \textbf{even} \textbf{though} \textit{it be sound}, in order to give the wisdom tooth enough elbow room to come through. (BLOB; popular lore)\\
\z

In both examples, the \isi{subjunctive} mood presents the content of the subordinate clause as less factual: In \REF{ex:31}, the positive personal evaluation appears as somewhat less committed, while in \REF{ex:32} a hypothetical situation is discussed. With verbs in the \isi{subjunctive}, the meaning of concessives shades more strongly into that of conditionals; as discussed in \sectref{sec:2.1.1}, the latter are an adverbial category to which concessives are related, or out of which they have developed via \isi{secondary grammaticalisation}.

In the category of \is{finite clauses}finite clauses, there is a word order phenomenon restricted to the conjunction \textit{though}. In (\ref{ex:33}a), the AdjP complement \textit{difficult} in the original corpus finding is not in its default post-verbal position~– shown in the alternative (constructed) subordinate clause of \REF{ex:33b}~– but in a slot not only before the subject, but before the subordinator.

\ea\label{ex:33}
    \ea
{\label{bkm:Ref489783741}  I do not see the whole system of local government finance grinding to a complete halt during the next two years, \textit{difficult} \textbf{though} \textit{those years will be}. (ICE-GB:S1B-034; comma added)}\\
    \ex\label{ex:33b}
{ …\textbf{though} \textit{those years will be difficult}.}\\
\z
\z

Within a \isi{Generative Grammar} framework, \citet[166–167]{Culicover1976} and \citet[213]{Radford1981} call this “\textit{though}-attraction” and “\textit{though}-movement”, respectively (cf. also \citealt{Aarts1988}: 44–45), \citet[908]{BiberEtAl1999} refer to the phenomenon as “fronting in dependent clauses”, while \citet[634]{HuddlestonPullum2002} call it “preposing in PP structure” (on the basis of their use of the term \textit{preposition}; \citeyear{HuddlestonPullum2002}: 599–600). In the present study, the phenomenon will be referred to as \is{though@\textit{though}!inversion@\textit{though}-inversion}\textit{though}-inversion. \citet[909]{BiberEtAl1999} argue that the main purpose in such inverted constructions is to emphasise the preposed element. While both \citet{Culicover1976} and \citet{Radford1981} refer to AdjPs only, the following three examples show that it is also possible for an NP, AdvP, or an entire (nonfinite) VP to precede the conjunction \textit{though} in a similar way.\footnote{The NP in the subordinate clause in \REF{ex:34} would require a determiner, if, for example, the clause was re-constructed into the unmarked variant (\textit{though he was a brilliant artist}). Intriguingly, the behaviour of such “fronted” NPs resembles that of NPs preceding postpositional \textit{notwithstanding} (e.g. \textit{bad cough notwithstanding} vs \textit{notwithstanding} \textbf{\textit{his}} \textit{bad cough}), which could be argued to be an equally marked construction (cf. \citealt{Schützler2018a}).}

\eanoraggedright\label{ex:34}   \label{bkm:Ref489283807}But Goldie, \textit{brilliant artist} \textbf{though} \textit{he was}, isn’t in the international league and a lot of overseas collectors aren’t all that interested in slightly-known New Zealanders who are a big deal only in their own country. \\(ICE-NZ:W2E)
\ex\label{ex:35} The outlaw, \textit{fast} \textbf{though} \textit{he was going} […] would have noticed it. (BBrown; adventure and western)
\ex\label{ex:36} \textit{Shout at Eichmann} \textbf{though} \textit{he might}, the Prosecutor could not establish that the defendant was falsifying the way he felt about Jews.\\(Brown; popular lore)
\z

\is{nonfinite clauses}Nonfinite and \isi{verbless clauses} are analysable into the same components (or “functional elements”) as \is{finite clauses}finite clauses (\citealt{QuirkEtAl1985}: 992). However, the subject is always missing in \is{nonfinite clauses}nonfinite clauses introduced by a subordinator. This can be seen in the following three examples, in which it is impossible to add a subject to the subordinate clause without adding a \is{finite clauses}finite verb as well:

\ea\label{ex:37}   \label{bkm:Ref489778084}The job, \textbf{though} \textit{lacking a certain prestige}, allowed me to write much of the day […]. (AmE06; humour)\\
    \ex\label{ex:38}\label{bkm:Ref489778311}Browning’s poem ‘The Grammarian’s Funeral’ is a psychoanalysis of a Renaissance scholar who, \textbf{although} \textit{born a poet}, devoted his maturity to the examination of abstruse problems in Greek etymology. (BBrown; belles lettres)\\
    \ex\label{ex:39}\label{bkm:Ref488843907}\textbf{Although} \textit{a rural and predominantly agricultural area}, no part of the Vale is more than 12 miles from major industrial and urban centres. (LOB; learned and scientific)\\
\z

In \REF{ex:37}, subject (\textit{it}) and finite verb (\textit{was}) are not overtly expressed, and the main verb in the subordinate clause appears as a present participle.\footnote{Simple-present and perfective uses of the \textit{ing}-participle (as in \textit{seeing him} vs \textit{having seen him}) are not differentiated in the present study.} In \REF{ex:38}, subject (\textit{he}) and finite verb (\textit{was}) are also omitted from the subordinate clause, which hinges upon the past participle \textit{born}.  Finally, the subordinate clause in \REF{ex:39} neither contains an overt subject (which would have to be \textit{the Vale}) nor a form of the verb \textsc{be}; it consists only of the subject complement, which is the complex NP in italics.\footnote{The distinction between \is{nonfinite clauses}nonfinite subordinate clauses with or without verbs is a rather fine (and, for some purposes, unnecessary) one, as is the distinction between nonfinite components of finite VPs and subject complements. Compare the surface equivalence of \textit{He was tall}\slash \textit{He was a solicitor}\slash \textit{He was waiting}.} Examples \REF{ex:37} and \REF{ex:38} further illustrate another property of \is{nonfinite clauses}nonfinite subordinate clauses, namely that their subject is typically \is{co-referentiality}co-referential with the one in the matrix clause (\citealt{QuirkEtAl1985}: 1005; cf. \citealt{Givón1990}: 836)~– a very strong tendency stated as the “\isi{normal attachment rule}” (cf. \citealt{QuirkEtAl1985}: 1121, \citealt{Schützler2018a}). Example \REF{ex:39} is an interesting, if not particularly jarring, departure from that rule: The implied subject of the subordinate clause can only be assumed to be \textit{the Vale}, while, strictly speaking, the overt matrix-clause subject is \textit{no part of the Vale}.

  The above descriptions are clearly not an exhaustive account of all aspects relevant in the syntactic description of CCs with subordinating conjunctions. An additional point made by \citet[41–43]{Aarts1988} is that subordinate clauses of concession come in different degrees of complexity. He distinguishes three: a “simple” type with no embedded clauses; a “complex” type that contains additional embedded clauses (e.g. \textit{although he stopped when he saw the obstacle}); and a “coordinated” type, in which the concessive marker relates to several independent clauses linked by coordinating conjunc\-tions (e.g. \textit{although the food was bad and the staff were unfriendly}).\footnote{Two thirds of all clauses in Aarts’s data were simple; “considerably fewer” were complex; and “only a handful” \citep[43]{Aarts1988} were coordinated.} In the present study, subordinate clauses were not coded for degree of complexity, in order not to inflate the quantitative apparatus necessary for analysis, and also because there appeared to be no theoretical reason for doing so. Three syntactic phenomena will be discussed in some more detail in \sectref{sec:3.5}:
(i)~the use of \is{correlative markers}correlative conjuncts,
(ii)~“overlapping” (or “double”) concessives, and
(iii) the marker \textit{even although}.\footnote{These phenomena are based on what was found in the data. Of course, many other marginal construction types are likely to exist, and may be found in other corpora.}

\section{Summary}\label{sec:2.4}

This chapter set out by providing the historical context for concessives in general and for the particular conjunctions under investigation in this study. Further, different semantic types of concessives~– anticausal, epistemic and dialogic~– were discussed. Finally, the syntax of present-day CCs involving the three conjunctions was examined, focusing on the position of dependent structures relative to matrix clauses and the structure of complements within subordinate clauses. While the historical background was provided mainly for the general contextualisation of results in this study, the discussion of semantic and syntactic aspects outlines the range of functional and formal variants on which the subsequent quantitative analyses (particularly in Chapters \ref{sec:9}–\ref{sec:11}) will be based.

As pointed out above, only \is{concessives (types of)!anticausal}anticausal and \is{concessives (types of)!dialogic}dialogic CCs will be included in the quantitative analyses that are to follow. This is due to the rarity of \is{concessives (types of)!epistemic}epistemic and \is{concessives (types of)!narrow-scope dialogic}narrow-scope concessives, as well as to the syntactic inflexibility of the latter, which would considerably complicate statistical analyses and generate results so lacking in robustness that they would likely distract from a meaningful interpretation. Similarly, the syntactic options that are considered in the quantitative approach are an idealised abstraction: Apart from the simplified distinction between \is{finite clauses}finite and \is{nonfinite clauses}nonfinite clauses that complement the three conjunctions, clause positions are analysed using a binary scheme, with only a contrast between \is{final position}final and \is{nonfinal position}nonfinal positions.

The next chapter will be qualitative in nature, discussing a number of typical (and a few less typical) corpus examples in illustration of the semantic and syntactic structures that were introduced above.
