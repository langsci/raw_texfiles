\author{Ole Schützler}
\title{Concessive constructions\newlineCover{} in varieties of English}
% \subtitle{Add subtitle here if it exists}
\BookDOI{10.5281/zenodo.8375010}
% \typesetter{}
\proofreader{Amir Ghorbanpour,
Annika Schiefner,
Barthe Bloom,
Brett Reynolds,
Caroline Pajančič,
Elen Le Foll,
Elliott Pearl,
Janina Rado,
Katja Politt,
Lachlan Mackenzie,
Lea Schäfer,
Tom Bossuyt,
Yvonne Treis}
% \lsCoverTitleSizes{51.5pt}{17pt}
\renewcommand{\lsSeries}{lv}
\renewcommand{\lsSeriesNumber}{9} 
\dedication{To Steffi \& Arved}
\renewcommand{\lsID}{370}
\BackBody{This volume presents a synchronic investigation of concessive constructions in nine varieties of English, based on data from the \textit{International Corpus of English}. The structures of interest are complex sentences with a subordinate clause introduced by \textit{although}, \textit{though} or \textit{even though}. Various functional and formal features are taken into account: (i) the semantic/pragmatic relation that holds between the propositions involved, (ii) the position of the subordinate clause, (iii) the conjunction that is used, and (iv) the syntax of the subordinate clause. By exploring patterns of variation from a Construction Grammar perspective, the study works towards an explanatory model whose point of departure is at the functional (semantic/pragmatic) level and which makes hierarchically organised predictions for different formal levels (clause position, choice of connective and realisation of the subordinate clause). It treats concessives as complex form-function pairings and develops arguments and routines that may inform quantitative approaches to constructional variation more generally.}

\renewcommand{\lsISBNdigital}{978-3-96110-422-2}
\renewcommand{\lsISBNhardcover}{978-3-98554-080-8} 
