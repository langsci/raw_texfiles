\author{Christian Döhler}
\title{Speaking the map}
\subtitle{Komnzo texts}
\renewcommand{\lsISBNdigital}{978-3-96110-494-9}
\renewcommand{\lsISBNhardcover}{978-3-98554-123-2}
\BookDOI{10.5281/zenodo.14214653}
\typesetter{Christian Döhler}
\proofreader{Sebastian Nordhoff}
\lsCoverTitleSizes{45pt}{15mm}
\renewcommand{\lsSeries}{otc}
\renewcommand{\lsSeriesNumber}{1}
\renewcommand{\lsID}{480}
\dedication{I dedicate this book to my \textit{ŋafe} Marua† and my \textit{ngom} Kurai, who grew up speaking the neighbouring languages Wära and Anta and wishes that his children will speak Komnzo.}
\BackBody{This collection of fourteen texts in Komnzo offers an insight into the language and culture of the Farem people, their storytelling tradition, oral history, mythology and everyday life. It contains stories from nine narrators, which were transcribed, translated and analysed by the researcher with the help of the Komnzo language committee. All texts are presented in a parallel text version arranged in columns (Komnzo/English) and in an interlinearised and glossed version. The book focuses thematically on landscape, place names and locality. It includes a description and analysis of the way Komnzo speakers conceptualise this semantic domain.}

\renewcommand{\lsDedicationFont}{\fontsize{15pt}{8mm}\selectfont}
