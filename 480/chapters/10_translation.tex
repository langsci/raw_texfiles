\begin{Parallel}{0.47\textwidth}{0.47\textwidth}
    \ParallelLText{\noindent \textit{Bänema kwa ŋatrikwé nzefé. Fenz ane mane ŋonathrth kwosifr kabeaneme bthan kabeyé. Trikasi zrethkäfé.}}
    \ParallelRText{\noindent I will talk about the body fluid from a dead man's corpse and how sorcerers are drinking this. I will start the story.}
\end{Parallel}

    \vspace{.4cm}

\begin{Parallel}{0.47\textwidth}{0.47\textwidth}
    \ParallelLText{\noindent \textit{Bthan kabe fthé fenz yonasi. Bänemr zrethkäfth mätraksir. Kzi kwa yafiyokwrth. Srafiyokwrth karesama. Kzi srä-rzirth. Watik kwa eyak nima kwosifr fthé kabe fthé ynänzüthzrth baden. Fthé one week srakor. Fthé fof krefar ane bthan kabe bobo fokam znfo. Fokam mnzfo. Sikwankwanme zbär thd kabef keke kwa sremar. Süsübäthen kwa yak tosinmäre kwayanmäre. Kwa yak. Yfrsé gwonyamekarä kwa yé. Keke kwa kwayanthé gwonyamekarä bäne-ma kabeyé sremarth. Fi fthé srayak e fokam. Watik yfö katanr kwa yarenzr. Katan yfö fthé zremar ebarfa. Fä fof kwa bäne ythorthr mrrab. Mrrab zbo zanfr byé. Ane fof sräsryöfth bobo yfön. Watik fobo fof srayak. Kzi zräzin nabi tonze mrrab tonze. Fenzane bäne mrrab bäne kwa wämneme yrthakunzr. Watik fenz ane kwa ŋankarkwr. Naf fobo fof krayagunzr kzifo. Nafa watikthmenzo. Keke kwa krärtf ane kzi. Fthé zremar nima watikthmenzo zfrä. Mrrab ane sräfum. Watik kwot zrarmänwr yfö. Watik krefar fof. Bäne zrazänzr fenz kzikaf mä keke kwa kabef sremar ane yam fiyoksin. Kwa wrifthzr.}}
    \ParallelRText{\noindent When these sorcerers want to drink bodily fluids, they first take out the bark tray they have made. They make it from the paperbark tree by bending the bark into shape. Next they go to the corpse. About a week after the funeral, the sorcerer makes his way to the grave, to the grave house. He creeps there in the middle of the night. Nobody will catch sight of him. He makes his way in the dark without a flashlight, without any light. He will wear black clothes. Not in white clothes, because someone might see him. Then he walks up to the grave and looks around for a small hole. When he finds a small hole at the head end, he inserts a small bamboo pipe. The pipe is about this long. He pushes it right into the hole. Next he puts the bark dish close to the bamboo pipe. He sucks up the fluid and pours it on the bark dish until he's got enough. He won't fill it right up. When he sees that there is enough there, he pulls out that bamboo pipe. Then he closes up the hole properly. Then he sets off. He will carry these bodily fluids with him in the bark dish, where nobody would see him. He will hide them.}
\end{Parallel}

    \vspace{.4cm}

\begin{Parallel}{0.47\textwidth}{0.47\textwidth}
    \ParallelLText{\noindent \textit{Watik fi zöbthé zane bäne kramonzikn-wr zzarfa. Bäne ferä ymd thäbu nzabu. Watik ane thrma ane fof krefar fokamfo. Ane fthé zrarinakwr. Kzin sräbth. Watik yonasir fof zrärifthm. Zöbthé bäneme kwa wrthakunzr z-zarfame. Bänema gatha miyosé rä. Nafane miyo keke namä wärä. Zra-rthakunzr zräbth. Wati bäne ane kwa yfethakwr ymd nzabu. Srafethakwr. Keke kwa zane touch srarär ane fenzme. Kwan krakurwr. Zrarär kwanen bäne-ma thafma gatha miyoma. Zrarär zra-fethakwr. We zbo sranakwr krafigthkwr. We nimanzo kwot e zräbth ane fenz. Fthé zräbth kzi ane kwa yfönzr mnime. Fewama mnime sräföf. Watik kräbrim nafane mnzfo.}}
    \ParallelRText{\noindent Before he starts anything, he will go and get whatchamacallit, ginger roots. Those things will be prepared and also the feather of a bird. Only then does he go to the grave, fills up the bark dish and hides to drink. First, he sprinkles the liquid with the ginger because it has a bad taste. Its taste is not good. He sprinkles it all over. Then he dips in the feather. He carefully dips it in because fluid should not touch his mouth. It would scratch his mouth. He puts it carefully into the mouth because it's bitter, because of its terrible taste. He does that and he dips it in. And he puts it down his throat and licks it. He continues doing this until he finishes up the fluid. When he's done, we will burn the bark tray in the fire because of the bad smell. He burns it in the fire and he returns to his house.}
\end{Parallel}

    \vspace{.4cm}

\begin{Parallel}{0.47\textwidth}{0.47\textwidth}
    \ParallelLText{\noindent \textit{Kwa yrugr. E baf fthé sräbth nima kabe zan miyof. Okay fthé fof krefar. Keke kwa mnzen ane tmatm zrafiyokwr ane yam. Zagr kwa yak ksi karen. Bä sramnzr. Foba fof krefar kabe zanr. Si kwa zöbthé ŋazübrakwr warfo kabedbo. Warfo kabe kwa ykonzr ``Befé mitafo sabrim! Nzun fefe kwagathif.'' Watik ane kabe kwa yfänzr. Kabe yf kwa ybräknwr. Nima ``Bäi! Bäiane mitafo be sabrim! Nzun fefe kwagathif!'' Watik fthé krefar. Kabef keke kwa sremar. Bänema mnzen fthé srarugr. Nagayé disturb or ŋare disturb srarär. Watik anema fof krämätr outside. Nä karfo ksi karen fä sramnzr. Fä ane tmatm kwa kabe yafiyokwr. Btha zan yfnzr. Foba fof krethfär. Mobo fthzé. Nima zba fthé roukuma nge srarä. Zbär kwa yam zä wäfiyokwr. Zba krethfär safsfo. Bä btha zan srafnzr. Bthanme srafnzr. Kränbrim we ane we zbär keke kwa bä srarugr o srawäkwr. Zbär we kwa ŋanbrigwr keke kwa mothen fi srayak. Fi krathfänzr.}}
    \ParallelRText{\noindent He will sleep peacefully until the blood lust overcomes him. Okay, then he gets going. He doesn't do his magic in the house. He will go far somewhere in the bush. He has to go away in order to kill people. First, he will pray to god. He will say to god: ``You take back the spirit but leave the flesh for me.'' He will point out a particular person and mention their name, for example ``Bäi! Take Bäi's spirit back and leave his flesh behind for me!'' Then he makes his way. He won't visit that person because his wife or his children will distract him. Therefore, he will go out to some place in the bush. He will perform this action there, but his magic power will strike that person. He can fly from there where ever he wishes, for example if a \textit{Rouku} boy is the sorcerer, he will do this in the night and he will fly from here to \textit{Safés}. He will strike the victim with a spell. Then he returns in the night. He doesn't stay there overnight. He will not walk on the road, but he will fly.}
\end{Parallel}

    \vspace{.4cm}

\begin{Parallel}{0.47\textwidth}{0.47\textwidth}
    \ParallelLText{\noindent \textit{Nima ane wäfiyokwr. Fthé sräbth kabe bthan zan srethkäf watik fä mane kwik erä fof. Keke. Taurifo tmatm zrafiyokwr o ŋathafo o faso rrokar berä. Ane rrokar-fo kwa tmatm yafiyokwr keke kwa nima nä kabedben. Fi ane kabeane mitafo kwa wthorthr. Nä faso rokarfo o fthzé ŋatha zräthb. Ra fthzé srarä ymd. Watik ane fof kwa tmatm yafiyokwr. Ŋatha yafiyokwr nafane yfkaf. Nezä kabe kwa kwosi yé. Keke ŋatha kwa kwosi srarä yakme. Mon tariäsi fthé kratariwr ŋatha. We kabe nimäwä kwa ŋatariwr. Kwot e ŋatha fthé zä kwosi srarä. Kabe bä kwa kwosi yé. Bänema ŋatha ane nafane yfkaf sfrä. Nimame ane fof bthan erä.}}
    \ParallelRText{\noindent He will do this: When he has finished drinking the fluids, he starts casting his spell. Next, someone will fall ill. He can also put the spell on a wallaby or on a dog, or on any other animal. He might do this to animal, and not to a human being. But the sorcerer's spirit will go inside some animal or whatever. It could be a bird, or anything. For example, he performs these rituals on a dog and gives it the name of a man. This man will die, but he will not die quickly. While the dog dies slowly, the man will suffer in the same way. One day the dog will die. The man will also die because the dog has been marked with his name. That's how they do the spell.}
\end{Parallel}

    \vspace{.4cm}

\begin{Parallel}{0.47\textwidth}{0.47\textwidth}
    \ParallelLText{\noindent \textit{Okay ane fenz mane ŋonathrth. Tmä naf fof ärithr. Kwosifr kabeane tmäf fof ezänzr nä karfo. Nä kayé kam kwa emätrakwrth kabe kam kwosifr kam. Watik ane fof thfäsir fof. Ane kamf kwa yzänzr bobo nä karfo. Fi fenz ane bänemrnzo rä tmä yarisir. Kamf fi ane kwa yzänzr bobo nima safs o wämnefr. Nima zagr kwa ŋathfänzr weam. Fthzé bobomrwä arufe krathfänzr zagr karfo. Ane tmäf kwa yzänzr kam a fenz.}}
    \ParallelRText{\noindent These sorcerers only become strong, when they drink these body fluids. The strength of the corpse carries them to another village. Sometimes they also take bones, human bones, from the corpse. That's really something to fly with. These bones allow them to go to other places. And these bodily fluids give them magic power. The bones will carry them, for example to \textit{Safés} or to \textit{Wämnefr}, or to far away to places such as \textit{Weam}, or all the way to \textit{Arufi}. This power can take them there. They use the bones and the body fluids.}
\end{Parallel}

    \vspace{.4cm}

\begin{Parallel}{0.47\textwidth}{0.47\textwidth}
    \ParallelLText{\noindent \textit{Eso kafar. Anenzo katan trikasi zfrä. Trikasi nimanzo worä kabeyé mane watrikwrth. Fi srakwä fthzé kwot kratrikwrth. Gadmöwä.}}
    \ParallelRText{\noindent Thank you. That was the small story. I have told you this as I have heard it from others. The boys talk about this topic all the time. Thank you.}
\end{Parallel}