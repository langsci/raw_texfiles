The story was prompted by a comment about Lucy's mother Naimér. Her mother was know as \textit{ausi kamkam}, which is a nickname that means something like `bony old woman'. Her mother had passed away during my absence a few months earlier. She must have been well over 90 years old, as she was the sole survivor of the last headhunting raid in the area.

Lucy talks about her own youth growing up in small hamlet close to Yokwa village, and later at \textit{Kanathér}. Lucy's father died when she was still very young. It was her mother and another woman who took care of her and her elder sister. She describes her mother as a fierce woman who went hunting by herself killing pigs and cassowaries.

In the mid-70s, Lucy married Abia Bai and they moved to \textit{Kiunga} for a few years. Abia was working there for the mining company. Later they returned to the Morehead district and lived at \textit{Kanathér}, \textit{Masu} and \textit{Rouku}, and Lucy looked after the old woman.

The story is both a praise of her mother and an account of her own life. 