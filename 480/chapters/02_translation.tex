% \begin{Parallel}{0.47\textwidth}{0.47\textwidth}
%     \ParallelLText{\noindent \textit{XXX}}
%     \ParallelRText{\noindent XXX}
% \end{Parallel}

% \vspace{.4cm}

\begin{Parallel}{0.47\textwidth}{0.47\textwidth}
    \ParallelLText{\noindent \textit{Trikasi kwa wthkärwé masuma. Kar yf fä mane znthkäfe yabär, tratrabäk, mdri. Masu zf rä. Masu zf rä kar.}}
    \ParallelRText{\noindent I will start the story about \textit{Masu}. We started walking and came from \textit{Yambär}, to \textit{Tratrambäk}, to \textit{Méndri}, and now to \textit{Masu} right here.}
\end{Parallel}

    \vspace{.2cm}
    
\begin{Parallel}{0.47\textwidth}{0.47\textwidth}
    \ParallelLText{\noindent \textit{Nzenme menz zöbthé fefe mobo sa-thora. Menz frükakmenzo fthé kwan-kwirwrmth mrn frükakmenzo. Menz berä mayawa bagu sagara. Nafanme mothme zumarwrmth. Nimame kwakwirwrmth. Kar fthé thunthorakwrmth.}}
    \ParallelRText{\noindent This is the village of \textit{Masu}, where our ancestor first arrived. The ancestors came one by one, each clan. There were different ancestors of the Mayawa, the Bangu, and the Sangara. They were looking for their way when they came. They were looking for a place where they could settle down.}
\end{Parallel}
    
\begin{Parallel}{0.47\textwidth}{0.47\textwidth}
    \ParallelLText{\noindent \textit{Nzenmenz zä zf ŋankwira. Komo foba fthé kwankwirwrm. Swänze bä zänrita. Mare komnzo bithn bä mä zänrita. Zä zf ŋakwira zf zyé. Sathora zyé. Ymd foba zankarisa afa kfokfo ``Kar töna bä rä brä.'' Santhora. Zamara. Yf zwärako kar yf masu. Wati nimame komnzo zäritako ziyaro. Nimame ``Kwa n komnzo wiyako.''}}
    \ParallelRText{\noindent Our ancestor came to this place here. He came from \textit{Komo} and he crossed over the creek at \textit{Swänze}. That log bridge is still there, where he went across. Then he came right here. He heard the butcherbird calling out and thought ``This is a place on the high ground.'' He looked at the place, and he gave the name \textit{Masu} to it. Then he continued further saying: ``I will try and walk a little further.''}
\end{Parallel}

    \vspace{.4cm}
    
\begin{Parallel}{0.47\textwidth}{0.47\textwidth}
    \ParallelLText{\noindent \textit{Fi yakako farem kar. Afa kfokfo mane zakarisa ``Kar bä rä brä.'' Yakako. Nima n sräzigrthm ``Kabe z zämsa.'' Kabe zöbthwé z zämsa ane fof farem. Farem thdma bres orot. Nafane menz zöbthé z zämsa. Thartharfa näbun mane thwa-niyakm nima wazu berä muthrata. Kar yf fä z erä. Kar yf rä wazu faremkar thartharfa. Neba tharthar foba bräro muthrata. Neba wazi zba zünrä fisor bthan. Ane fof nzenme menz moba zänbrima. Kma zanmara ``Keke. Kabe z zäms.'' Zäbrimo. Watik yfnzo zwänra fisor bthan. Zänbrima zbo zf masu. Zf rä ane menz kar. Zä zf zämsa zf nzenme menz masu mayawanme menz. Edawä mrn mrzar mayawa a banibani mayawa.}}
    \ParallelRText{\noindent He got to \textit{Faremkar}, when he heard a butcherbird again: ``There is a settlement there.'' He walked further and he looked around: ``Someone has settled here already.'' The ancestors of the Farem clan had settled down there first in the middle of \textit{Faremkar}, from Bres and Orot's lineage. Their ancestor had settled there first. Others like the Wazu clan and Muthérata clan settled on the side of \textit{Faremkar}. They have named places of their own there, for example \textit{Wazu} right beside \textit{Faremkar}. And there is \textit{Muthérata} on the other side. In this direction is \textit{Fisor Béthan}, the place from where our ancestor returned here. He must have looked thinking ``Someone has settled here already.'' So he only named that place \textit{Fisor Béthan}, and then he returned to \textit{Masu}. So this is the story place right here. Our ancestor settled down right here, the ancestor of the Masu Mayawas. There are two clans: Mérzar Mayawa and Banibani Mayawa.}
\end{Parallel}

\vspace{.4cm}

\begin{Parallel}{0.47\textwidth}{0.47\textwidth}
    \ParallelLText{\noindent \textit{Zä fof zämsa. Nafane msaksi zn zane zf rä. Zane zf kar töna rä. No trkr kafar fthé rera. Zane faf zn keke nof zafafa. Nä karnzo nof thefafa. Zane keke. Masu. Manema nzräkorth masu kar. Masu kar ni zf nrä. Kar yf zane zf rä masu.}}
    \ParallelRText{\noindent  He settled down right here. His house was right here. This is high ground. When the water level rises during the rainy season, this place won't be affected by the water. Only other places are submerged, not \textit{Masu} here. This is why they call us \textit{Masu} people. We are the \textit{Masu} people. It is based on this place \textit{Masu}.}
\end{Parallel}

    \vspace{.4cm}
    
\begin{Parallel}{0.47\textwidth}{0.47\textwidth}
    \ParallelLText{\noindent \textit{Fz rera zane kafar fz. Daor zufarwrmth zane brä fz. Zena keke nima rä fz brä. Zena yusi frnzo rä. Kafar fz rera zane. Ŋafyé ethawä erera. Foba we enrera. Bada ethäwä erera. Ŋafyé ethäwä enrera. We foba nanfiyokwath. Ni zena znrä tüfr. Niwä komnzo ŋarake bäfiyokwre we zane faf znen zf zerä. Zagathinzake kwa we bä nzfräro ŋazäthe. Nä bad bäne bä mane eräro bad tnztnz nzenme bä mane eräro. Bä fof nzfrä e fthmäsü zabthe bä we kwanbrigwre we znrä zena. Zane ysakwren zf zathkäfe zrä. Ŋarake thunbrigwre zena. Wawa mnz bthuwore fof. we zathkäfe we fof ŋarakeme fof ŋarake thikysime.}}
    \ParallelRText{\noindent There used to be a dense forest here. They cut down the forest to create gardens. Today there is no more forest here, only savannah grass. It used to be a thick forest. In our fathers generation, there were only few people. All the way back. There were only a few people in the generation of our ancestors and fathers. They made us what we are today. And we are now many people. We are still gardening here, right here, in this place. At the moment, we are letting this place recover and have our gardens in \textit{Ŋazäthe} where we also have some land. We have a small piece of land there. When we have used up the land there, we will return here. We have already started to bring some of the gardens back here. We have started building some storehouses. Next season we will start the actual layout of the gardens and put up fences.}
\end{Parallel}

    \vspace{.4cm}

\begin{Parallel}{0.47\textwidth}{0.47\textwidth}
    \ParallelLText{\noindent \textit{Watik trikasi masuanema nima zf rä brä. Ni zf kar zf rä. Zf masu manema nzräkorth masu kar. Ni masu kar zf nrä zf. Trikasi fof zwaythk katan brä nimame.}}
    \ParallelRText{\noindent This is the story about \textit{Masu}. This place here is \textit{Masu}, and that is why they called us \textit{Masu} people. We are the \textit{Masu} people. The story ends like this.}
\end{Parallel}

    \vspace{.4cm}
    
\begin{Parallel}{0.47\textwidth}{0.47\textwidth}
    \ParallelLText{\noindent CD: \textit{Rma kabe zefarath zane karfa?}}
    \ParallelRText{\noindent CD: Why did the people move away?}
\end{Parallel}

    \vspace{.4cm}
    
\begin{Parallel}{0.47\textwidth}{0.47\textwidth}
    \ParallelLText{\noindent \textit{Zba kabe bänema zefarath. Zöbthé mane thfrärm fthé kafmd kwot kafmd fthé keke zfrärm. Wati frükakmenzo kabe thwamnzrmo katan karen. Nafan-me menz mä zämsa. Fi fobo thwam-nzrm nima. ni masun z nzwamnzrm. Boba wazi bämnzr nima. mifinin akrimogon nä thwamnzrm. Kar katakatan beräro nima. Safsen berä. Nä nima thwamnzrm kwaikr. Nä fä thwamnzrm nimame mefath. Wati, markai fthé yaniyaka. kwayan kabe fthé yaniyaka. Wati, näbi karfonzo nmosinza. Wati, näbi kafar kar thwafiyokwrme. Masuma eyakako näbi roukufo. Ŋazäthe-ma äniyaka  faremkar dmädr. Näbi karfonzo ŋamosinzrake. Kafar kar wäfiyokwrake rouku.}}
    \ParallelRText{\noindent When they used to live here, there was no government. People used to live in their clans, in small settlements, where their ancestors had settled initially. They were living like this. We were here at \textit{Masu}, and those other people lived on the other side of the river, for example in \textit{Mibini} or \textit{Akrimogo}. There used to be small hamlets like \textit{Safés} or \textit{Kwaikér}. Others lived in \textit{Mefath}. When the white man came, we shifted to one place. We built larger villages. The \textit{Masu} people went to \textit{Rouku} for good, likewise those ones from \textit{Ŋazäthe}, \textit{Faremkar}, and \textit{Démändér}. We came together in one village. We built one big village, and that's \textit{Rouku}.}
\end{Parallel}

\vspace{.4cm}

\begin{Parallel}{0.47\textwidth}{0.47\textwidth}
    \ParallelLText{\noindent \textit{Zöbthé mane nzfrärm frükakmenzo nzwamnzrm. Ane mrn fämnzr. Ane mrn fämnzr. Ane mrn fämnzr. Nima-me mrnmenzo nzwamnzrm. Zagr sime kwamarwrme. Markai fthé yaniyaka fof. Baibel buk trikasi fthé änyaka fof. Wati, bäne naf wnzänza nima. Yuniti wnzänza. Wati, näbi kwot ŋawagrwake fof. Tar ŋakonzrake. Zena znrä nimame. Watik, anema nimame fof nrä fof. Näbi karenzo bä bnamnzr. Anema zane zf. Kar mä nzenme bada zämsa o menz zämsa kar anema thugathikwrame. Näbi karenzo bobo fof nzfiyakmo.}}
    \ParallelRText{\noindent Before we were like this, we lived individually. This clan lived over there, that clan lived over there and the third clan lived over there. We each lived in our clans. We only saw each other from a distance. When the white man came, when the gospel came, it also brought whatchamacallit, it brought unity. From then on, we met regularly. We became friends. That's how we still live today. So that is the reason why we live like this, why we live in one village. As for the places where our ancestors had first settled, our story men, we have left these places. We've moved into a village together.}
\end{Parallel}

\vspace{.4cm}

\begin{Parallel}{0.47\textwidth}{0.47\textwidth}
    \ParallelLText{\noindent \textit{Trikasi mane rä nima komnzo fof rä nima. Nzenme kar fefe zane zfrä masu. Nzenme menz mä zämsa zöbthé fefe. Nä mrn we beräro nima. Trikasikaräsü fi we nimäwä eräro. Nimame fof brä. Eso.}}
    \ParallelRText{\noindent This is how the story goes. This is really our place here, \textit{Masu}, where our ancestor settled down first.  Other clans also have their places with their stories. It is just like this! Thank you!}
\end{Parallel}