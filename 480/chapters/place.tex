\section{Introduction}\label{sec:introtolandscape}
Mary Ayres writes that ``Morehead people never cease to amaze me with their ability to recall, to the square meter, the precise location where an event occurred'' (\citeyear[174]{Ayres:1983dw}). To keep such memories alive, the Farem leave tangible traces in the physical world. They often make cuts into tree trunks to mark a path. They tie leaves around trees to signal that a particular place should not be entered. They leave objects as a sign of remembrance of some event. \figref{fig:arrow} shows the broken shaft of an arrow that a man had set up on the side of a path after he had killed a large pig at this spot the previous year.

\begin{figure}
    \includegraphics[width=.53\textwidth]{figures/huntermark1s.jpg}
    \caption{A broken arrow is a reminder of the hunter's luck and skill}
    \label{fig:arrow}
\end{figure}

The interest in places has also left its mark on the language of the Farem people. I will argue here that communication about route-finding and about shared locations is made possible by a sophisticated system of landscape terms and a dense network of place names. The words making up these subsystems are used to talk about everyday events, but they also bring the mythical past into the here and now. They help to regulate the rightful use of land, and in some cases may even restrict access to it. Lastly, they provide a spatial anchor for a large part of social life, for example by linking people to their respective places of origin. As Thornton puts it in his overview of Native American place naming, they ``tell us something not only about the structure and content of the physical environment itself but also how people perceive, conceptualize, classify, and utilize their environment'' (\cite[209]{Thornton:1997uv}).

While the study of place names has a long tradition, research from the perspective of semantic typology has gained momentum in the last ten years. Good examples for a set of systematic studies of the topic can be found in the contributions to (\cite{Burenhult:2008jv}). A particularly useful distinction introduced by \citet{Burenhult:2008fp} looks at ``feature names'' versus ``area names''. While the former pick out particularly salient features of concrete geography, e.g.\ rivers, mountains, or valleys, the latter are not based on the physical environment, but more abstract concepts are involved, e.g.\ ownership, history, myth, or ethnicity. Feature names are good for individuating many different types of landscape, but they cannot reach a complete coverage of a given environment. After all, in this system you can only name what is there. Area names, on the other hand, can potentially reach complete coverage.

Feature names and area names show varying degrees of internal structure in Komnzo. An example for the hierarchical organisation of feature names comes from the expression \textit{kafar fz} `big forest' which is used for `jungle; thick, dense, dark, cool forest'. There are several subtypes of this vegetation type which vary in size and shape, for example \textit{fokufoku} `small patch of forest’ and \textit{fz minz} `thin strip of forest’. For area names, we may take \textit{kar} which describes an `inhabited place', typically these are villages, but there is also the word \textit{menz kar} `story/creator place' for a place that was once or is still inhabited by a mythical creator spirit, and that has a creation story to it. Entering such places should be avoided at all costs, as it can lead to illness and death. There are many \textit{menz kar} around \textit{Rouku}, and each one has a place name.

The following description shows that Komnzo speakers employ a mix of both systems. Feature names capture a variety of landforms (e.g. hill, slope, creek, river mouth, island), man-made places (e.g. village, garden, path), and biological entities (e.g. forest, swamp, savannah, grove). Area names, especially in the form of place names, are abundant and this subsystem reaches almost complete coverage. Recall FE Williams' comment from the introduction to this book: ``if you ask your guide where you stand at any moment, he will be able to give a name to the land'' (\citeyear[207]{Williams:1936hb}).

Area names and place names often develop from the names of features. Take the three major cities in the German federal state of Saxony, which all come from Slavic feature names: Dresden from \textit{Draždany} `swampy forest', Leipzig from \textit{Lipsko} `linden place', and Chemnitz from \textit{Kamenica} `stony river'. This pattern is also found in Komnzo place names such as \textit{swäri zfth} ‘\textit{swäri} base’. This must have started as a descriptor of a place with an especially large or beautiful \textit{swäri} tree (\textit{Alstonia actinifila}), but over time it lost its descriptive function. Today it is used even though the \textit{swäri} tree was cut down decades ago. \textcite{Merlan:2001it} described place name systems of this kind as ``non-arbitrary'', because they establish a direct relationship to the designated places.

Although landscape terms and place names in Komnzo are well-developed semantic fields, it is place names that carry the greater part of the functional load. In other words, speakers almost always coordinate through shared knowledge of place names rather than physical features of the environment. The latter is used as a fallback option if there is no name for a particular place or if the speaker assumes that the other person does not know the name.

\section{The physical environment}\label{sec:physicalenvironment}
The landscape of the Morehead District has been characterised by outsiders as uneventful, boring, and featureless. An extreme comment to this end was made by Wilfried Norman Beaver, the Resident-Magistrate of the colonial administration, who had briefly visited the region in September 1908. He writes: ``The timber is poor, consisting of stunted gum, ti-tree and paper bark. The grass is poor, and the soil is but third rate. I should say this is one vast swamp in the rainy, and parched up in the dry season''; concluding that ``there is nothing to induce settlement, nor would I ever advise anyone to go there.'' (\cite[12]{Murray:1909ly}). Having spent a total of two years in the region, I can only disagree with him. However, we do not need to move much further in time from Beaver, to find a more nuanced description of the landscape. Francis Edgar Williams, the Government Anthropologist for the Territory of Papua, who had visited the region repeatedly throughout the 1920s, introduces the area by stating that ``its scenery often has a mild, almost dainty, attractiveness in detail, but represents on the whole the extreme of monotony'' (\citeyear[1]{Williams:1936hb}). As we shall see, it is the small details on the ground that are highly salient to the people.

The wider area of Southern New Guinea, that part of the land between the Fly River in the north and the coastline in the south, has seen dramatic geomorphological changes over the last millennia. Recall that the initial settlement of the paleo-continent Sahul, consisting of New Guinea, mainland Australia, Tasmania, and the Aru Islands, is dated back to 40,000 BP (\cite{Allen:2008wi}). Most of Southern New Guinea was inundated at the maximum height of the sea level at 6,000 BP (\cite{Chappell:2005coastal}). It was slowly rebuilt with sediments carried by the Fly River and Digul River. As a consequence, large parts of the land consist of alluvial soil of poor quality. A notable exception is the area under investigation here, the so-called ``Morehead ridge'' (\cite[15]{Paijmans:1971morehead}), which is a slightly elevated stretch of land that runs in West-East direction. Here the soil is well drained and lies above the water level during the rainy season. For these reasons, most villages in the region are situated along the ridge. The village of \textit{Rouku} is located on the highest part of the ridge.

The broader Southern New Guinea region lies in a zone of tropical savannah or sub-humid tropical climate. The annual rainfall is around 2000mm, and 70\% of this falls during the wet season (\cite[12]{Paijmans:1971morehead}).\footnote{As a comparison, the Berlin-Brandenburg area received 579 mm annually on average between 1991-2010 (\cite{Senstadt:2022ew}).} The annual monsoon cycle brings a long dry season (June--November) and an intense wet season (December--May).  Small differences in elevation are barely noticeable during the dry part of the year, but they become clear landmarks in the wet season. Rivers overflow their banks, ditches turn into creeks, recesses become stagnant ponds. All these changes affect the vegetation and other types of biota, e.g. the types of animals that occur in the different ecological niches at different times of the year. \figref{fig:watermark1} shows the high-water mark near the village of \textit{Rouku}. During the previous rainy season, the paperbark trees on the right side of the picture were submerged to about 1 metre, while the bamboo groves on the left side remained dry.

\begin{figure}
    \includegraphics[width=.8\textwidth]{figures/watermark1s.jpg}
    \caption{The high water mark is visible in the centre of the picture}
    \label{fig:watermark1}
\end{figure}

Humans have interacted and changed the physical environment of the Morehead district, and they continue to do so. The most obvious result of these interventions are villages, roads and gardens that disrupt the natural environment. But even if a village was abandoned a long time ago or a garden has been left fallow for many years, there are unmistakable traces that remain. Old villages, for example, can be recognised not only by old house posts, but also by coconut palms that were once planted. Abandoned gardens can also be recognised by the absence of larger trees or dead trees whose naked trunks tower above the regrowing plants, or by bamboo groves that were once planted as building material for fences.

Human meddling with the physical environment has yet another, much more subtle dimension. \textit{Rouku} men often go hunting carrying the smouldering inflorescence of a local species of banskia. This \textit{dagu zthé} `firestick' is used to burn small patches of land along their routes, resulting in many low-intensity fires during the dry season. I was told that this practice serves to create a favourable habitat for wallabies and bandicoots or to increase the local supply of certain plants. In other parts of the world, especially Australia, this technique of ``cultural burning'' has lead to substantial changes in the geographic range and demographic structure of many vegetation types (\cite[217ff.]{Flannery:2002bs} and \cite{Bowman:1998cl}).

\section{Landscape in the lexicon}\label{sec:landscapelex}
The Komnzo lexicon marks off at least four major ecological zones: (i) \textit{kafar fz} `big forest’, which is a type of monsoon rain forest (\figref{fig:kafarfz}), (ii) \textit{fz} `forest’, which is a much thinner forest type covered by a grass floor and dotted with red anthills (\figref{fig:fz}), (iii) \textit{ksi kar} `bushy place’, which is a type of savannah that lacks trees, but is covered with high grass (\figref{fig:ksikar}), and (iv) \textit{zra} `swamp’, which is a place entirely inundated during the wet season, often timbered by paperbark trees and a ground cover of dead leaves (\figref{fig:zra}). While there is a general distinction between \textit{fz} `forest’, \textit{ksi kar} `open grassland’ and \textit{fath} `clear place, savannah’, we also find fine-grained distinctions like \textit{fokufoku} `small patch of forest’, \textit{fz minz} `thin strip of forest’, and \textit{morthr} `edge of forest with a smaller patch forest close by’. For a further description of the landscape types, see \citet[10ff.]{Dohler:2018qt}.

\begin{figure}
    \includegraphics[width=.8\textwidth]{figures/rainforest1s.jpg}
    \caption{A road through the rain forest (\textit{kafar fz})}
    \label{fig:kafarfz}
\end{figure}
\begin{figure}
    \includegraphics[width=.8\textwidth]{figures/lightforest2s.jpg}
    \caption{Thinner type of forest (\textit{fz})}
    \label{fig:fz}
\end{figure}
\begin{figure}
    \includegraphics[width=.8\textwidth]{figures/savannah3s.jpg}
    \caption{Edge of the forest transitioning to savannah (\textit{ksi kar})}
    \label{fig:ksikar}
\end{figure}
\begin{figure}
    \includegraphics[width=.8\textwidth]{figures/swamp2s.jpg}
    \caption{Villagers catching fish in the swamp (\textit{zra})}
    \label{fig:zra}
\end{figure}

Large parts of the Morehead district are inundated by rising water during the wet season. Consequently, small changes in the landscape may have drastic effects during this time. This has found its way into the lexicon of Komnzo. Moving from the highest point, usually the villages and settlements, to the lowest point, the Morehead River, we find a number of specific terms. I translate the term \emph{töna} as `high ground'. It is that part of the land, regardless of vegetation type, which is virtually never covered by water. Villages, hamlets, storage houses and yam gardens are located on \emph{töna}. Small hills are referred to by \emph{märmär} or by the Motu loan \emph{ororo}. These areas may become islands (\emph{bod}) during high floods. Wide, gentle slopes (\emph{rsrs}) lacerated by many small creeks (\emph{ttfö}) lead to lower areas. It is often along creeks where people plant sago palms. Taro gardens are made here if the ground stays wet throughout the year. Closer to the river, the ground can be very uneven and bumpy due to running water. This is called \emph{kore}. A little lower lies that part of the land which is always covered by water during the rainy season. Often backwater stays in stagnant pools, which dry up only during the height of the dry season. These places are called \emph{zra}, which I translate with `swamp', but maybe the term `billabong', commonly used in Australia, is more fitting. In this area, we find smaller pools of water which dry up (\emph{nawan}) and larger pools which are permanent (\emph{dmgu}). The ankle-deep, muddy water covered with leaves is called \emph{nzäwi}. Walking towards the river, the land rises again in many places. This difference in elevation is almost unnoticeable, but it is enough so that this area dries up first at the end of the wet season. These places between the swamp and the river are called \emph{for} and people plant cassava, sweet potato and taro here. The steep riverbanks along the Morehead river are called \emph{rokuroku}, a word from which the village name \textit{Rouku} originates. The sides of the river are covered with patches of \emph{süfi} `floating grass', and in some places this layer is called \emph{tüf} when it is thick enough to support the cultivation of sweet potatoes. Finally, there is the Morehead River (\figref{fig:moreheadriver}), and much of daily life takes place between the village and the river as people go and fetch water, wash, catch fish, or simply enjoy sitting by the river during the hot part of the year. The word for river is \emph{ŋars}, and there is the verb \textit{frezsi} meaning ‘to come up from the river’ (cf. \sectref{talkingplaces}). Most of the words mentioned here can be found in the texts included in this book, either as descriptors or as parts of place names (cf. \sectref{sec:placenames}).

\begin{figure}
        \includegraphics[width=.8\textwidth]{figures/river1s.jpg}
    \caption{The Morehead River at \textit{Kanathr}}
    \label{fig:moreheadriver}
\end{figure}

Other parts of the lexicon feature human-made structures. For example, \textit{kar} `village, place' describes a place that is suitable for settlement, or that is in fact settled permanently by people (\figref{fig:rouku}). Garden structures are described as \textit{dao} `garden' and by metonymical extension as \textit{ŋarake} `fence, garden' (\figref{fig:fence}). There is more specialised terminology, for example in \textit{thaba} which describes a clearing in the forest that is used for gardening (\figref{fig:thaba}), or \textit{dao kr} for an abandoned garden, \textit{for ŋarake} for a garden by the river. Another example for a human-made structure is \textit{swäyé} and somewhat archaic \textit{ethf} for `berth', or `canoe place'.

\begin{figure}
        \includegraphics[width=.8\textwidth]{figures/village1s.jpg}
    \caption{Houses in Rouku village}
    \label{fig:rouku}
\end{figure}
\begin{figure}
    \includegraphics[width=.8\textwidth]{figures/garden1s.jpg}
    \caption{A newly constructed garden fence (\textit{ŋarake})}
    \label{fig:fence}
\end{figure}
\begin{figure}
        \includegraphics[width=.8\textwidth]{figures/forestgarden5s.jpg}
    \caption{Two images of \textit{thaba} `forest garden' marked by dead trees}
    \label{fig:thaba}
\end{figure}

Area names of mythological significance are called \textit{menz kar} `story place', in which the word \textit{menz} stands for a mythical creator. These place are always connected to a creation story, and therefore I adopt Ayres' term `story place' for them. More generally, sacred places are called \textit{thak} and the reduplication of this word \textit{thakthak} means `law, taboo, rule'. This example illustrates something that \textcite{Ayres:1983dw} argues for in her dissertation, namely that the abstract is often anchored in the concrete landscape in the Morehead district.

\section{Landscape in metaphors}\label{landmetaphor}
Other ways of conceptualizing landscape in language comes in the form of meta-phorical expressions. Before discussing one example of a metaphor at length, I want to point to a gap. The domain of body parts, which is commonly used in many languages around the world (\cite{Tjuka:2019if}) for landscape features such as river mouths or mountain faces, is largely absent in Komnzo. So far, the only example is the word \textit{thm} `nose', which can be used for river bends or curves in a road. Speakers translate \textit{thm} with English `point', as in \textit{ŋars thm} `point in the river' (lit. `river nose'). 

Plants, especially trees, are a fertile source domain for landscape metaphors. The mouth of a creek or a river is referred to by the word \textit{zfth}. This word is a colexification (\cite{Francois:2008wc}) of a number of concepts. In its most basic sense, it means `base of a tree'. It can also mean `origin' and `reason’, much like `root' in the English expression `the root of all evil'. \textit{Zfth} can also refer to one's clan and lineage.\footnote{In \textref{text:fawbrigsi}, the speaker uses the nominal compound \textit{kabe zfth} [people tree\_base] to refer to the elders of his clan.} While the tree metaphor is commonly found in languages around the world, its extension to landscape and bodies of water, as in `river mouth', seems to be specific to the Morehead district. This is also found in other riverine terminology such as \textit{ttfö tuti} `creek branches, creek twigs’ or \textit{ttfö minz} `creek vines’, which can both be used to refer to smaller streams. The place where a creek starts can be called either \textit{ttfö ker} `creek tail’, but also \textit{ttfö zrminz} `creek root'. The same terminology is applied to different parts of the much larger Morehead river.

The fact that bodies of water are conceptualised through a tree metaphor can be traced back to the myth of \textit{Kwafar}, in which the place from which all humans descend is a tree. \textit{Kwafar} is located somewhere in the Arafura Sea between New Guinea and Australia. In the myth, the tree burns down and the people are left homeless. The burning of the tree is connected to two brothers who kill a mythical creature. This event triggers a flood that drives the people to the north and south. As the water rises, the channels in the ground left by the charred roots of the tree become creeks and rivers. In other versions of the myth, the tree falls to the north and the impressions left in the ground are filled by the water of the approaching flood. Text \ref{text:kwafar} is a version of the \textit{Kwafar} myth. Other versions have been recorded by \citet[306]{Williams:1936hb} and \citet[50]{Ayres:1983dw}.

\section{Place names}\label{sec:placenames}
The two quotations from Williams and Ayres cited in the introduction already indicate the abundance and importance of place names, or toponyms (cf. \sectref{sec:introduction}). In \figref{fig:ayresfieldnotes}, we can get an impression of what Ayres means when she writes that the system is ``staggering in its elaboration'' (\citeyear[38]{Ayres:1983dw}). The image shows a page from her notebook containing a map of \textit{Rouku} and the surrounding area -- no more than eight kilometers in diameter -- with well over 100 named places. When I compared her sketch with my own data, I realised that the map is by no means exhaustive, and that there are many named places missing, especially closer to settlements.

\begin{figure}
        \includegraphics[width=\textwidth]{figures/ayres-fieldnotes.png}
    \caption{Place names around Rouku from Ayres' fieldnotes}
    \label{fig:ayresfieldnotes}
\end{figure}

While the place names are common knowledge and known to most Farem people, the details of each little path and the stories that go with them are only known to the rightful owners of that piece of land. In this sense, toponymic knowledge can be likened to a proof of ownership. For this reason, I have deliberately refrained from providing a complete list of the place names I have collected as well as a detailed map.

All place names in Komnzo are proper names, but they differ in their composition and meaning. Some place names have no other meaning than the places they refer to, for example \textit{Fthi}, \textit{Kanathr} or \textit{Ŋazäthe}. At some point in the past, they may have been segmentable into meaningful parts or formed a meaningful word in themselves, but this knowledge has faded. Place names usually preserve features that have become unproductive or lexemes that have become archaic. This can also be observed in Komnzo. For example, the place name \textit{Thmefi} can be analysed as \textit{thm} `nose' and \textit{efi} `hair'. However, the word \textit{efi} is archaic and \textit{thäbu} is used instead. In fact, some speakers are not aware of the possible segmentation.

Komnzo place names usually consist of two nouns that form a compound. Semantically, they range from generic descriptions such as \textit{Gani zfth} `base of the \textit{gani} tree' (\textit{Endiandra brassii}) to specific illustrations such as \textit{Nzga warsi} `vulva chewing' or \textit{Kwanz fath} `baldness clearing'. Many of these compounds consist of a plant name and a landscape term or a term used for the part of a plant. The most common landscape terms in these compounds are: \textit{zra} `swamp, waterhole’ and \textit{ttfö} `creek’. The most common plant part terms are: \textit{zfth} `base’ and \textit{fr} `stem, grove’. A few examples are: \textit{Karesa zfth} `\textit{karesa} base' (\textit{Melaleuca sp}), \textit{Atätö fr} `\textit{atätö} stem' (\textit{Pouteria sp}), \textit{Wsws zra} `\textit{wsws} swamp' (\textit{Combretum sp}). These are not descriptions of places or plant parts, as the translation might indicate, but they are proper names. An expression like \textit{karesa zfth} can refer to the base of any \textit{karesa} tree, but the proper name \textit{Karesa Zfth} refers only to a single place.

Only few place names are inflected verbs. The two above mentioned place names \textit{Kanathr} and \textit{Ŋazäthe} look like verbs, but they are no longer analyzable as such. A clearer example is the place name \textit{Karifthe} which relates to one of the stories in this collection (\textref{text:safakfaikore}). \textit{Karifthe} is the place where the protagonists set off in different directions after their argument. The place name is an inflected verb that can be translated as: `You two should send each other away!'\footnote{Second person dual imperative of the verb \textit{rifthaksi} `send, send off'.}

\section{Mixed-language place names}\label{sec:mixed}

The study of toponymy is a fruitful source of data about prior linguistic occupations of an area. In many places, as in the case of Celtic place names through much of Central and Eastern Europe (\cite{Sims-Williams:2006it}), there is an inference that language choice in toponyms is evidence for land-language associations different from those that currently obtain. However, we should beware of assuming that reigning conditions of monolingualism are the most potent force shaping place naming. Traditional egalitarian multilingualism is still practised throughout the Morehead District, which is reflected in mixed-language toponyms. Two such examples are: the place name \textit{Sandir Mit} close to Bimadbn village, which mixes Nen \textit{sandir} `banksia species' and Idi \textit{mit} `stem'\footnote{I thank Daniel Gbae (Guvae) from the village of Bimadbn who gave me a handwritten document with a dozen bilingual place names around his village after I spoke to him about this issue. The place name \textit{Sandir Mit} and the word \textit{sandir} are listed in Evans' dictionary of Nen (\citeyear{Evans:2019aa}). The Idi word \textit{mit} is listed in the lexical database Yamfinder (\cite{Carroll:2021ta}).}, and the place name \textit{Ormogo} close to \textit{Rouku} village, which mixes Komnzo \textit{or} `Emerald dove' and Nama \textit{mogo} `house'.

\begin{table}
\caption{Mixed-language place names}
\label{tab:mixedtoponyms}
	\begin{tabularx}{\textwidth}{llll}
	\lsptoprule
		\textsc{name}&\textsc{language 1}&\textsc{language 2}&\textsc{translation}\\
		\midrule 
        \textit{Düdüsam}&Na: \textit{\textbf{düdü} wkwr}&Ko: \textit{dödö \textbf{sam}}&‘broom liquid’\\
        \textit{Fakwr}&Na: \textit{\textbf{fa} fak}&Ko: \textit{thrma \textbf{kwr}}&‘after ashes’\\
        \textit{Fotnz}&Wt: \textit{\textbf{fo} tg}&Ko: \textit{ŋazi \textbf{tnz}}&‘coconut short’\\
        \textit{Füsari}&Na: \textit{\textbf{fü} bilé}&Ko: \textit{ŋanz \textbf{sari}}&‘garden plot axe’\\
        \textit{Makozanzan}&Ar: \textit{\textbf{maxo} kamakama}&Ko: \textit{nzga \textbf{zanzan}}&‘vagina beating’\\        
        \textit{Mefath}&We: \textit{\textbf{me} faf}&Ko: \textit{mni \textbf{fath}}&‘fire place’\\
        \textit{Mnzärfr}&Na: \textit{\textbf{mnzär} sèrásèr}&Ko: \textit{msar \textbf{fr}}&‘ant post’\\
		\textit{Säzäri}&Wt: \textit{\textbf{sä} ytho}&Ko: \textit{karesa \textbf{zäri}}&‘paperbark bending’\\
        \textit{Tratratbäk}&Ka: \textit{\textbf{tratra} bak}&Ko: \textit{drädrä \textbf{bäk}}&‘lapwing's back’\\        
        \textit{Wästhak}&Na: \textit{\textbf{wäs} näk}&Ko: \textit{wäsü \textbf{thak}}&‘tree (sp.) place’\\
        \midrule
        \textit{Märofak}&Ko: \textit{\textbf{märo} kwr}&Na: \textit{mane \textbf{fak}}&‘tree (sp.) ashes’\\
        \textit{Ormogo}&Ko: \textit{\textbf{or} mnz}&Na: \textit{bänz \textbf{mogo}}&‘dove house’\\        
        \textit{Snzäzwär}&Ko: \textit{\textbf{snzä} zfth}&Wt: \textit{dawi \textbf{zwär}}&‘crayfish base’\\
        \textit{Wamamogo}&Ko: \textit{\textbf{wama} mnz}&Na: \textit{féfé \textbf{mogo}}&‘yam house’\\
        \textit{Yemgifaf}&Ko: \textit{\textbf{yem} zan \textbf{faf}}&Na: \textit{awyé \textbf{gi} \textbf{faf}}&‘cassow. hunt place’\\
        \textit{Zthékabir}&Ko: \textit{\textbf{zthé} etfth}&Wä: \textit{zthk \textbf{kabir}}&‘penis sleeping’\\
        \midrule
        \textit{Dimsathak}&An: \textit{\textbf{dimsa} \textbf{thak}}&Ko: \textit{nzrmsé \textbf{thak}}&‘sour place’\\
        \textit{Gawe}&Wt: \textit{\textbf{ga we}}&Ko: \textit{nzä \textbf{we}}&‘I also’\\
        \textit{Sizwär}&Ko: \textit{\textbf{si} zfth}&Wt: \textit{\textbf{si} \textbf{zwär}}&‘eye base’\\
        \textit{Zöfäthak}&Wä: \textit{\textbf{zöfä} \textbf{thak}}&Ko: \textit{ymnd \textbf{thak}}&‘bird place’\\
	\lspbottomrule
        \multicolumn{4}{l}{\footnotesize Abbreviations: Anta (An), Arammba (Ar), Kánchá (Ka), Komnzo (Ko), Nama (Na), Wära (Wä), }\\
        \multicolumn{4}{l}{\footnotesize Wèré (We), Wartha Thuntai (Wt)}
	\end{tabularx}
\end{table}%mixedtoponyms

Mixed-language place names account for a significant proportion of toponyms recorded for Komnzo. \tabref{tab:mixedtoponyms} gives a list of 20 of them. They involve words from almost all surrounding language varieties (cf. map in \figref{fig:sng-map}).\footnote{Kémä and Namat are absent from the list. Both of these languages are so close to other varieties that a clear identification of the source language is difficult. In the case of Kémä these are Wära and Wèré, and in the case of Namat it is Nama.} They are sorted by the order of elements: in the first set the second element comes from Komnzo, in the second set it is the first element. The last four place names involve cases in which the second variety is very close to Komnzo. Thus, there is one word that belongs to both languages, and one word that clearly belongs to the other language. For each place name, I provide the English translation in the rightmost column. In the two columns labeled ``language 1'' and ``language 2'', I provide the translation of the mixed-language place name into each of the two languages; i.e. if it were a monolingual place name.

I have argued in \citet{Dohler:2021tb} that mixed-language place names must have been coined in a deliberate act; in a conscious decision that a particular place would henceforth be labeled with a bilingual expression. The names themselves are proof of this, as they consist of semantically coherent expressions such as \textit{Fotnz} `short coconut' with Wartha Thuntai \textit{fo} `coconut' and Komnzo \textit{tnz} `short' or \textit{Mnzärfr} `ant post' with Nama \textit{mnzär} `ant' and Komnzo \textit{fr} `stem, post'. The point about semantic coherence is that it links words from two languages directly. One piece of evidence against intentional naming would be if mixed-language place names were formed from a semantically coherent expression in one language, e.g. \textit{fo tg} `short coconut' in Wartha Thuntai, which is then combined with a generic place name in another language, e.g. \textit{thak} `place' in Komnzo, resulting in \textit{fo tg thak}. However, such structures have not been documented for mixed-language place names. Further supporting evidence comes from the observation that these place names pattern roughly with geography, in that places containing a word from Wartha Thuntai are found west of \textit{Rouku}, while places containing a word from Nama are found east of \textit{Rouku}. Note that code-switching can be ruled out as a path of development, since code-switching is heavily sanctioned by the local language ideology. Instead, speakers converse in a dual-lingual mode (\cite{Lincoln:1976sl}), sometimes also called receptive multilingualism (\cite{Singer:2023zj}). The pattern of language use is such that in conversations between speakers of two different languages each consistently speaks one language in response to utterances in the other language.

\section{Talking about places}\label{talkingplaces}
The first observation that can be made from the texts in this collection is the abundance of place names. Almost all stories are anchored at some named place, and each scene takes place in a specific location. This becomes clear in the first text, the \textit{Kwafar} myth (\textref{text:kwafar}), which belongs to a genre of ``traveling creator myths'' known from Australia (cf. \cite[5ff.]{Evans:2010vh}). They are also known from \textcite{Wagner:1996qz}, who shows that traveling myths are found along the New Guinea south coasts of the Western and Gulf Provinces, in adjoining areas of West Papua, in southern areas of the Simbu Province, and on the Torres Strait Islands.

In the \textit{Kwafar} myth in this collection, the ancestor travels and creates the landscape, for example by dropping crumbs from his yamcake that turn into small pebbles, or carving a bow and arrows for himself, the remains of which turn into bamboo groves. He brings with him certain sacred items such as rainmaking stones, and he plays a role in the dispersal of people from this original place called \textit{Kwafar}. His name and route may change depending on where the story is recorded, but what gives the story its ultimate credibility is the account of the places he visited along the way. The version in this text collection tells the story of \textit{Mathkwi}, the ancestor of the Mayawa clan in \textit{Rouku}. In some episodes of the plot, the narrator Abia Bai lists the places like pearls on a string, as in the excerpt in (\ref{ex:stringofpearls}).

\ea\label{ex:stringofpearls}
    \ea
        watik foba yaniyaka misa zfth\\
        \glt `Then he continued and came to \textit{Misa Zéféth}.'
    \ex
        mäbri misa zfth yrn\\
        \glt `\textit{Mämbri} was first, then \textit{Misa Zéféth} and then \textit{Yérén}.'
    \ex
        fä zänrsöfätha fof yaniyaka benzü zfth\\
        \glt `He went down there (to the river) and walked to \textit{Benzü Zéféth}.'\exsource{\textref{text:kwafar}, line \ref{ex:1:a3712}--\ref{ex:1:a3717}}
\z
\z

While such spatial anchoring is especially important for myths of origin (cf. Texts \ref{text:kwafar} and \ref{text:masu}), there is another genre that shows this connection clearly. For the genre of \textit{nzürna trikasi} ‘spirit stories’, one must assume that each clan should have its own \textit{nzürna trikasi} (\cite{Dohler:2025hy}). There are three \textit{nzürna trikasi} in this book (cf. Texts \ref{text:nzurna-marua}, \ref{text:nzurna-maraga}, and \ref{text:nzurna-kurai}). In the introduction to the first story, I outline the common elements found in all \textit{nzürna trikasi}. What is more important now are the differences, because within a broader genre that contains many stories, the uniqueness of one's own story must be emphasised. This is often achieved in the afterword by linking the story to one's clan land, and thereby to one's lineage. In each of the three stories we can recognise the narrator's attempt to claim the respective story as his own by mentioning the place. As an example, I cite Maraga Kwozi's afterword to the \textit{nzürna} story, which is set in his native village of \textit{Firra} in (\ref{ex:afterwordnzurna}).

\ea\label{ex:afterwordnzurna}
    \ea
        nä karen nima näbuné bänema\\
        \glt `There are other (stories) at other places.'
    \ex 
        nä nzürna ŋare zokwasi trikasi bä räro fi ane kar woga mane erä fi ane miyatha erä\\
        \glt `There are other Nzürna stories, but the villagers there know them.'
    \ex 
        nzefé nzüwäbragwé nima ni miyatha nrä\\
        \glt `I followed this one because we know it.'
    \ex 
        nzekaren ane yam kwafiyokwrm\\
        \glt `This happened in our village.'\exsource{\textref{text:nzurna-maraga}, line \ref{ex:7:a4886}--\ref{ex:7:a4890}}
\z
\z

On first mention of a particular place, narrators often provide additional explanations about certain features of the place or about its position in relation to other named places. Two examples are given below in (\ref{ex:zäzrmnz}) and (\ref{ex:waisam}). Such elaboration occurs somewhat more frequently in the stories than in natural language usage, since an unknowing outside linguist is involved in the recording situation. In (\ref{ex:zäzrmnz}), Lucy Abia points out a place that most people from \textit{Rouku} would know very well. A case like (\ref{ex:waisam}) is different in that the place Maraga Kwozi mentions is further away and few people know the details of this location. In such circumstances, further explanations would also be used in more natural speech.

\ea\label{ex:zäzrmnz}
    \ea
        wati zenfarake zäzr mnz bä rä\\
        \glt `So we departed and moved to \textit{Zäzér Ménz}.'
    \ex 
        safs ane mothen rä nima ŋars zawe\\
        \glt `That's on the road to \textit{Safés}, on the side towards the river.'\exsource{\textref{text:ausi}, line \ref{ex:14:a3076}--\ref{ex:14:a3078}}
\z
\z

\ea\label{ex:waisam}
    \ea
        kafar wäsü sukogrm mrab fren waisamen\\
        \glt `The big \textit{Wäsü} tree stood in a bamboo grove at \textit{Waisam}.'
    \ex
        waisam ane kar yf rä\\
        \glt `That's the name of the place, \textit{Waisam},'
    \ex        
        mobo zwamnzrm\\
        \glt `where she lived.'
    \ex
        mrab fr thden\\
        \glt `It is located in the middle of a bamboo grove.'\exsource{\textref{text:nzurna-maraga}, line \ref{ex:7:a4814}--\ref{ex:7:a4817}}
\z
\z

Not only scenes within stories, but also people are explicitly associated with certain places. Narrators pay attention to this spatial anchoring even if (i) the character does not play a major role in the further course of the plot, (ii) the character is not mentioned by name or (iii) is identified only with a kinship term. Example (\ref{ex:bres}) comes from the introduction of a story in which the narrator Kurai Tawéth mentions in passing that the protagonist has two wives.

\ea\label{ex:bres}
    \ea
        okay bres bä swamnzrm yirkon\\
        \glt `Okay, Bres was living in \textit{Yirko}.'
    \ex
        edäwä ŋare swarärm\\
        \glt `He had two wives.'
    \ex
        nä kanfokma nä masuma\\        
        \glt `One was from \textit{Kanfok} and the other one was from \textit{Masu}.'\exsource{\textref{text:nzurna-kurai}, line \ref{ex:8:a1655}--\ref{ex:8:a1657}}
\z
\z

Place names are often used metonymically referring to the inhabitants of that place, as in (\ref{ex:akrimogo}), where the notion of ``people from place X'' is inferred from the use of the dative case on the place name: \textit{akrimogo kar=nm} [\gl{pln} village=\gl{dat}].

\ea\label{ex:akrimogo}
    akrimogo karnm naf thätrifa\\
    \glt `She told the \textit{Akrimongo} people.'\exsource{\textref{text:ebarzanfirran}, line \ref{ex:4:a1738}}
\z

Especially in dry season much of people's daily life involves coming and going from the high ground to the river. This movement has left some impact in the verb lexicon. For example, the verb \emph{frezsi} usually means `take something out of the water'. The verb can be used in a quasi-reflexive construction with the meaning `come up from the river'. The latter construction is used when disembarking a canoe, or when walking back from a river camp to the village. We can find \textit{frezsi} with this meaning in a number of texts, for example in \textit{Kwafar} (\textref{text:kwafar}), \textit{Faw brigsi} (\textref{text:fawbrigsi}), and \textit{Kukufia} (\textref{text:kukufia}). The verbs \textit{rsörsi} ‘climb down’ and \textit{sogsi} ‘climb up’, as in ‘climb down or up a tree’, can also be used to mean `go downhill or uphill'. For example, we find two short episodes that make use of the two verbs in the hunting story \textit{Fiyaf trikasi} (\textref{text:fiyaf-maembu}, lines \ref{ex:13:a1341}--\ref{ex:13:a1343}, \ref{ex:13:a1410}--\ref{ex:13:a1412}). The first of these is given below in (\ref{ex:mndüfi}) with the verbs for `climb up and down' printed in bold font. Note that these route descriptions are located in places where the difference in elevation is barely noticeable in the dry season.

\ea\label{ex:mndüfi}
    \ea
        wiyak e nä töna \textbf{kresöbätho}\\
        \glt `I walked up to the high ground.'
    \ex 
        \textbf{krärsöfäthé} \textbf{kresöbätho} mdüfi tönafo fof\\
        \glt `I went down and went up again to the high ground at \textit{Méndüfi}.'\exsource{\textref{text:fiyaf-maembu}, line \ref{ex:13:a1341}--\ref{ex:13:a1343}}
\z
\z

\section{Conclusion}
Komnzo speakers do not use vectors on a grid, such as maps, compasses or other cartographic aids for coordination. In lieu of a grid system, or a landscape with highly salient features such as mountains and valleys, people use established tracks between named places in order to navigate, or rather to talk about such navigation. Thus, the local system of orientation is one of ``tracks-between-named-places.'' People do not just wander off and stroll around, except when hunting for an animal. When visiting unfamiliar territory, they are usually accompanied by someone who knows the area. Most tracks are visible even to the untrained eye, and should a track become overgrown because it has not been used for some time, people have a habit of cutting marks in the trees along the path.

Based on the abundance of place names in narratives and in Komnzo speech in general, I argue that the Farem conceptualise their environment through a network of named locations. The exact locations, as would be indicated on Western maps, are less important. What matters is a certain degree of coordination with other speakers. Such a mnemonic system is kept active in the memory and passed on to the next generation by constantly referring to places, which explains the abundance of place names in the texts. In that sense, the Farem people quite literally ``speak the map''.
