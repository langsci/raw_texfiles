A group of headhunters from \textit{Mere}, \textit{Wartha}, \textit{Safés}, and \textit{Nzrari} planned an attack on the village of \textit{Firra}. One of the elders of the Mayawa clan named Yasi knew about the attack and left the village. He tried to convince his brother Srank to leave also, but Srank stayed behind. Srank survived the attack and ran to \textit{Masu} where he informed his Mayawa brothers.

Only few people survived the attack. Among them was Srank, a young woman named Zafo from \textit{Akrimongo}, and a man named Géfi from \textit{Safés}. Zafo from \textit{Akrimongo} had overheard the headhunters and could identify where they had come from. Most of the other families residing at \textit{Firra} at the time were killed, and their patrilines were cut off because of the attack. The speaker emphasises that his patriline is the only one from \textit{Firra} that has survived until today.

The Mayawa elders from \textit{Masu}, together with Yasi and Srank, took revenge on the headhunters, but the story does not go into detail about this payback raid.

The story ends in statement that these events were the last headhunting raids in the region. Not long after the raids, the colonial administration and the local missionaries ``pacified'' the region.