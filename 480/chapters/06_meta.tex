\textit{Nzürna trikasi watayan} is a recording lasting roughly 6min. It was recorded by Christian Döhler on November 19\textsuperscript{th} 2011, in both audio and video.\footnote{The original recording session for this text is labelled tci20111119-06. It is archived at: \href{https://doi.org/10.5281/zenodo.11189830}{https://doi.org/10.5281/zenodo.11189830}} The story teller is Marua Bai. The recording took place in a camp by the Morehead river the vicinity of \textit{Rouku} village. This recording belongs to the wider genre of \textit{nzürna trikasi} `Nzürna stories' which are told publicly and can be considered common knowledge. During my fieldwork in \textit{Rouku}, I was able to record three such stories, but I was told many more. In the following, I will give a kind of Gestalt description of the genre and introduce some of the recurring motifs.

The Nzürna character is a female forest being, often an older woman, with extraordinarily long fingernails and eyelashes, sometimes also earlobes. In many versions, she lives with a husband and with children, but these play only a minor role in the plot. Nzürna can change her appearance into a normal human being of any age and sex. She puts this skill to use in order to trick clueless villagers into deadly traps. The fate of her victims is often quite gruesome: She kills and eats them raw, she plucks out their eyeballs, she decapitates them and carries their heads home, with her long fingernails she rips out their intestines through the anus, and places them somewhere visible to mark the event. These chilling details add to the colourful retellings of Nzürna stories. In fact most stories contain some kind of a warning message: either the poor protagonist was warned not to go alone to these places in the forest, or the audience is directly addressed not to do such foolish things.

Nzürna stories are often imaginative and comical, and both the audience and narrator laugh a lot at certain passages that recur in the stories. For example, the doomed protagonist often tries to escape the Nzürna by hiding under a pile of yams in the storage house. But the Nzürna has the ability to command centipedes and she sends them to give him painful bites. In another scene towards the end, the Nzürna realises that the villagers have sealed her fate by setting her house on fire. Trapped inside, she tries in vain to put out the fire by urinating and defecating on the flames getting all burned in her attempts. These parts of the story are often acted out verbally, sometimes even physically, and they usually elicit hysteric laughter from the audience.

In all versions of Nzürna stories that I have heard, she and her family live in the hollow inside of a Wäsü tree (\textit{Ficus elastica}). These ``strangler figs'' or ``curtain figs'' have a remarkable appearance when fully matured. They are parasitic plants whose seeds are usually carried by birds to a host tree, where they germinate in crevices and develop long aerial roots downwards. Strangler figs grow until they fully envelop the host tree, which eventually dies. This creates the characteristic hollow central core, which is described as the entrance to the Nzürna's house. Although most of the stories locate the Wäsü tree in a specific place, narrators can never point to the actual tree, because at the end of each story the Wäsü tree is burned to the ground by revengeful villagers, thereby killing the Nzürna.