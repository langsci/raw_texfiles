\textit{Ebar zan firran} is a narrative lasting roughly 10min. It was recorded by Christian Döhler on November 7\textsuperscript{th} 2011, in both audio and video formats.\footnote{The original recording session for this text is labelled tci20111107. It is archived at: \href{https://doi.org/10.5281/zenodo.11189781}{https://doi.org/10.5281/zenodo.11189781}} The story teller is Maraga Kwozi, and the recording took place in the garden of his house at \textit{Morehead Station}. 
Maraga was around 55 years old at the time of recording. He belongs to the Banibani Mayawa clan from \textit{Firra}, and he lives with his family at \textit{Morehead Station}. In fact all of Firra had become a semi-permanent settlement, which is mostly used as a garden place.
The story focusses on a headhunting raid that must have taken place in the 1930s. The exact date can only be guessed from the age of one of the protagonists, Yasi, who was Maraga's grandfather. The story explains why there are only few people from \textit{Firra}, and all of them are from the Mayawa section. The transcription of the recording prompted Abia Bai to tell me the story of the revenge in \textref{text:fawbrigsi}.