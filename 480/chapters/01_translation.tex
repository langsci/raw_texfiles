\begin{Parallel}{0.47\textwidth}{0.47\textwidth}
    \ParallelLText{\noindent \textit{Moba zrathkäfe?}}
    \ParallelRText{\noindent Where will we start?}
\end{Parallel}

    \vspace{.2cm}
    
\begin{Parallel}{0.47\textwidth}{0.47\textwidth}
    \ParallelLText{\noindent CD: \textit{Wawa moba enrera?}}
    \ParallelRText{\noindent CD: Where did the yams come from?}
\end{Parallel}

    \vspace{.2cm}

\begin{Parallel}{0.47\textwidth}{0.47\textwidth}
    \ParallelLText{\noindent \textit{``Wawa moba enrera?'' okay! kwa zöb-thé zrathkäfe nimame trikasi fof kwafar. Nimame fof nzranyan e zbo zrabthe. Ra nzigfu enfathwath. Ra fofosa nzigfu enfathwath. Watik zbo zf zrabthe ane-me fof. Trikasi näbi kwa wäniyak. Zane trikasi mane rä ŋafyf bäyf mane ŋatrikwa. Nzenm natrikwa. Watik ane trikasi fof zena ŋaritakwr.}}
    \ParallelRText{\noindent ``Where did the yams come from?'' Okay! We will start the story in this way, with \textit{Kwafar}. We will come like this and then finish up with the rain making stones. What rain stones they used to have. We will finish with this part. So it will be one story. As for this story, it was my father Bäi who told it. He told it to us. Well, this story will be passed on today.}
\end{Parallel}

    \vspace{.2cm} 
 
\begin{Parallel}{0.47\textwidth}{0.47\textwidth}
    \ParallelLText{\noindent \textit{Trikasi mane rä kwafarma rä. ``Kwafar'' Ŋafyf nima fof kwatrikwrm ``Kwafar mane rera thden rera.'' Zane zena mane bad mane wythk. Mazo mä ŋakonzr a australiane bad mä wythk. Fä mä fi zfrärm ane. Kwafar fof kabe mä kwamosinzrmth. Wäsi warfo thfrugrm. Wäsi bäne ykonzrth. Nä bä bikogro zärkarä. Kabe fä fof thwamnzrm fof zokwasi ffrümenzo. Nä zfthen thwam-nzrm. Nä thden thwamnzrm. Nä kerker thwamnzrm. Watik zokwasi ane ffrümenzo kwanafrmth.}}
    \ParallelRText{\noindent This story is about \textit{Kwafar}. ``\textit{Kwafar}", that's how father began the story, ``\textit{Kwafar} was in the centre, where this continent ends today and where the ocean begins, to where the Australian land ends.'' That is where \textit{Kwafar} was located. The people used to live together in \textit{Kwafar}. They stayed on top of a Wäsi tree. They call this one Wäsi. Over there is one that casts a shadow. The people were living there, and they spoke different languages. Some lived at the base of the tree, others in the middle and still others at the top. They all spoke different languages.}
\end{Parallel}

\begin{Parallel}{0.47\textwidth}{0.47\textwidth}
    \ParallelLText{\noindent \textit{Nä kayé wäsi ane zäföfa fof. Zästha fof. Nä kabe nima kwakwirwrmth. Nä kabe nima mnin kwarsirwrmth. Watik wäsi ane kwot ŋarsira zäbtha. Kabe bä mane thwägrm warfo nä mrmr. Fi we nimäwä kwarsirwrmth. Watik ezi kabe ane frümenzo tnägsi zethkäfath. Bä frümenzo thwamnzrm. Watik mni fthé ŋarsira. Kar ane bramöwä ŋarsira fof. Thgathg zfrärm fath thefath. Watik menzmenz ane fof yabun kafar thgathg dagonr. Ane fof zenfara. Kabe mane thfthnm kwosi.}}
    \ParallelRText{\noindent One day, the Wäsi tree started to burn. It really lit up. Some people ran away, while others burned in the fire. So the Wäsi tree burned to the ground. Those people who stayed at the top, and those who stayed inside, they all died. The next morning, the people began to disperse and they have been living apart ever since. When the fire was burning, it burned everything down. It became a charred place, a clear place. There was a huge spirit being, a very big creature. It roamed around and ate the people, the dead that were lying there.}
\end{Parallel}

    \vspace{.2cm}

\begin{Parallel}{0.47\textwidth}{0.47\textwidth}
    \ParallelLText{\noindent \textit{Watik gwamf ŋatha thäsa ezi ane thefath thgathgen fof yaser. Watik ŋatha anenzo fof sathkäfa. Ŋatha ane swaru-thrm gwam mon nima yarera. Eda erna kabe. Kafar yf mane thfrnm. Nafangthrwä gwam muri. Gwam yara nafanane. Muri nafangth. Wati gwamf ane fof ezi ŋatha thäsa thgathgen. E anenzo fof ŋatha yayamgwa. Ane menznzo fof kabe maf änatha fof. Fewakaf kwosi thwanathrm. Gwamf zagr ymarwa fof ``Ra bäne yé?'' Nima né samara o ``Ra menzmenz yé?'' Kabe nrma fi fobo fof ŋagathikwa fof ane menzmenz. Kabe ane zenthkäfath yak. Ŋatha mane kwaruthrm yf ŋatha yara ane tifr. Wati sathkäfath. Kabeyé ane dunzi kma sfruthrmth. Keke.}}
    \ParallelRText{\noindent In the morning, Gwam called his dogs to go hunting at that burned place. He started with one of the dogs. The dog was barking, and Gwam was thinking like this. In fact, there were two men. They were well known: Gwam with his younger brother Muri. Gwam was the older brother and Muri was the younger brother. So it was Gwam who called the dogs in the morning to go hunting and that one dog stopped because of this creature. It was startled. It was that creature eating the human bodies. He ate the dead, the decaying corpses. Gwam saw it from a distance. ``What's that?'' he said and tried to take a look at it. ``What creature is this?'' The creature stopped with a full stomach. As for the barking dog, it's name was Tifér. Some of the people there tried to shoot the creature, but without success.}
\end{Parallel}
\newpage
\begin{Parallel}{0.47\textwidth}{0.47\textwidth}
    \ParallelLText{\noindent \textit{Gwamf nafangth sräkor ``Muri zba känrit nzuzawe. Nzefé biruthro!'' Ŋa naf nima ``Samg! Bänema näbuné fof yruthrth byé. Keke kwosi yathizr.'' Naf nima ``Keke. Fi miyamr erä fofosa mä rä. Nze komnzo zimarwé fof.'' Zirkn thfrnm. Näbun kwanafrm. Näbun kwa-nafrm. ``Watik ngth biruthé. Famkaräsü gnräré!'' Nafananaf. Ane fof trikasi nima rä. Nafangth kma markai nabi-karä sfrärm. Watik nafangth mane yara naf keke samga. Nafananafnzo nabi ŋathunza zf sfthnm. Yo kwan! Fof sargosira fofosa fefen.}}
    \ParallelRText{\noindent Gwam told his small brother: ``Muri, come over here to my side! I'll shoot him from here'' and Muri said, ``Shoot now! The others are shooting at it, but it's not dying.'' Gwam said: ``No, they don't know where the heart is. Only I can see it from here.'' They kept on doing this. One was talking and then the other one was talking. Gwam said ``Okay brother, I will shoot now! Watch out!'' The story goes that it was the older brother who shot. The younger brother is said to have owned a shotgun. The younger brother did not hit the animal. The older brother drew the bow and hit it. The creature dropped down. He speared it. The arrow pierced right through to the heart.}
\end{Parallel}

    \vspace{.4cm}

\begin{Parallel}{0.47\textwidth}{0.47\textwidth}
    \ParallelLText{\noindent \textit{No fof zärftha. No ane zamatha. Wati no mane kwakwirm fof. Wäsi zrminz mä ŋanrsira fof, mni mä ŋanrsira, no fä kwanthorthrm fof. Ane zrminz fof. Nof nä nima thäkothmako. Nä nima thänkothma nzezawe. Gwamane nima zenmathath muriane nima. Mane ŋan-kwirwath zentnäthath. Nä enrera bawi. Wartha nima bämnzr wartha a kondomarin smärki. Nafanme foba fof ŋankwira fof. Fi foba fof ŋankwirwath bawi. Watik gwamf fä fof mni bäne zafrafa fof. No dödöme zakwra. Watik no fä fof zäkora. Keke kwa nof zanmäyofa. Fobo fof no ŋagathikwa fof.}}
    \ParallelRText{\noindent Water began to break out. Water was flowing. Water that was running everywhere. Where the roots of the Wäsi tree burned, where the fire burned them, the water flowed along those roots. The water chased some of the people in this direction. Others were chased away towards our side here. The people from Gwam ran here and those from Muri in the other direction. Those who ran away scattered in various directions. Some came to \textit{Bawi}. For example, the \textit{Wartha} people who live there, and the \textit{Kondomarin} and the \textit{Sémärki}. The ancestors of those people dispersed there. They escaped first to \textit{Bawi}. Then Gwam extinguished the fire. No! I mean, he hit the running water with the broom and stopped the flood. The water did not flow any further; it stopped there.}
\end{Parallel}

    \vspace{.4cm}

\begin{Parallel}{0.47\textwidth}{0.47\textwidth}
    \ParallelLText{\noindent \textit{Watik fi mane enrera e zwari wartha fof. Watik fä fof zwarin. Zämsath. Zokwasi fthé emarwath ffrümenzo. Watik kondomarin nima feräro. Zena boba wazi fi berä merauken. Nä mane erera zwarifa ŋafrezath thoro. Watik thoron fä fthé zemarath. We nimäwä fof zokwasi ffrümenzo. Watik foba zethkäfath nimame kwasogwrmth. Okay, nä mane enrera zwarifa e bäne thamgakar. Nima bä ämnzr safs. Wati fi fä fof thfyakm. Nzenme mane yanra mä ŋankwirwath komo. Fä ŋanfrezath komo. Nzenme mayawama kabe nä fä thägathizath. We foba thden nä kwot we mayawama kabe fof. Foba baguma kabe foba zena mifinin zämnzr. Sagara fä thägathizath. Okay fi nima erera mogarkam. Nä mane erera nima erera bäne a drdr nä sagara fof. Bagu mane enrera bäne mäta. Sagara mane enrera garaita. Mayawa ni zbo zf nnrera.}}
    \ParallelRText{\noindent So the people were coming up to \textit{Zwari}, the \textit{Wartha} people really. They settled there in \textit{Zwari}, where they realised that they spoke different languages. So the \textit{Kondomarin} people continued further this way. Today, they live in \textit{Merauke}. Others came up from \textit{Zwari} towards \textit{Thoro}. At \textit{Thoro} it was the same again. They realised that they spoke different languages. So some of them moved further inland. Others came from \textit{Zwari} straight to \textit{Tamgakar}. For example, the people who live at \textit{Safés}. They were walking this way. As for our ancestor and his people, they walked to \textit{Komo}. They came up at \textit{Komo}. They left some of our Mayawa people there. Some Mayawa people and some Bangu people live there in the centre. Nowadays, these people live at \textit{Mibini}. The ancestor also left a few Sangara people there, and they spread out further to \textit{Mogarkam}. Other Sangara people continued until \textit{Déridéri}. Some of Bangu people continued further to \textit{Mäta} and some of the Sangara people to \textit{Garaita}. We Mayawas came right here.}
\end{Parallel}

    \vspace{.3cm}
    
\begin{Parallel}{0.47\textwidth}{0.47\textwidth}
    \ParallelLText{\noindent \textit{Nzenme bada mrzarane bada mane yanra yf ane yanra mathkwi. Math-kwif ane enfathwa wawa fofosa. Naf ane ynfathwa fof. Wati näbi ane komnzo fofosa yara. Wawama nasi duga biskar dagon nä berä fof. Watik fi anekarä fof yanra fof. Mane yanyaka e wm bä ythn zabrta. Fä fof ŋanritakwath fof. Kwanritakwrmth trkren. Watik nima né fam zära ``Garaita zawe? Keke. nä kabe foba z sfyak.'' Watik nima zethkäfa fi safs. Nimame ane zethkäfa mothr. Mane yanra e akrimogo. Yam fä fof thremar fof ``Oh nä nima z eräro.'' Watik keräfi foba fof zäzira fof. E kar yf rä ymnz. Watik fobo fof ``Oh kabe bä yé. Watik nimame wiyak.'' Watik foba fof akrimogofa zenfara fof. Foba näbi yaniyaka. Karane yf erä füsari. Füsärifa rarafü kar. Rarafü karfa kafrir. Fä ttfön zänrita. E bäne zofok. Zofok fä yamthiza. Nabi komnzo bekogr. Nabi ŋatr fä fof zurärm. Zwa-frmnzrm. Zurzirakwa fof. Nasi nömä yanatha. Rfarrfar futhfuth mane erera wmr ane fof ŋakwthenzath fof. Zäkwtherath. Watik komnzo berästhgr. Wm mane yé ynfathwa fof no nzigfu. Watik ane fof yräza fof zofok kar. Watik foba yaniyaka misa zfth. Mäbri misa zfth yrn. Fä zänrsöfätha fof. Yani-yaka benzü zfth. Foba fof ymd thren-karis fof afa kfokfo ythama. Fam zära ``Kar bä rä. ŋa kar töna fobo fof wiyak fof.'' Yaniyaka. Fä fof zänrita. Fof rä. Kukwrb fr zra. Mnzärfr neba. Wati fä fof yaniyaka fof mä zänfrefa. Nömä futhfuth fä fof ŋantnägwath.}}
    \ParallelRText{\noindent As for our ancestor, the Mérzar's ancestor, his name was Mathkwi. Mathkwi was bringing along those things: the yam stone. He had this one. It was for the cultivation of yams, long yams, taro and cassava, and for other kinds of crops. He came with that stone. When he approached the place where the rocks are near \textit{Zambérta}, he crossed the river. He crossed the river during the flood season. He thought to himself ``Should I go to \textit{Garaita}? No, someone's already gone there.'' So he started walking towards \textit{Safés}. He got going and walked until \textit{Akrimongo}. He saw footprints there. ``Oh, others are already here.'' Then he shot an arrow that flew all the way to \textit{Yéménz}. He heard people shouting from there and thought: ``Oh, there's already someone there. Then I will go this way.'' He left \textit{Akrimongo} and started walking. He came straight to a place called \textit{Füsari}. From \textit{Füsari} to \textit{Rarafü Kar}. From \textit{Rarafü Kar} to \textit{Kafrir}, where he crossed the creek over to what's-that, to \textit{Zofok}. He rested in \textit{Zofok}. Those bamboos are still there, where he prepared his bow and arrow, where he put on the bowstring. He ate some yam cake. Those crumbs from the yam cake turned into rocks. They are still there. As for the stone that he had brought, the magic stone, he named it ``\textit{Zofok}'' at that place. Then he continued further to \textit{Misa Zéféth}. \textit{Mämbri} was first, then \textit{Misa Zéféth} and then \textit{Yérén}. He went down to the river and walked to \textit{Benzü Zéféth}. From there he heard  birds calling, the butcherbird and the bird of paradise. ``This must be high ground. I am going to walk up there.'' He continued a little further. He crossed the river at \textit{Kukwérémb Fér}, opposite \textit{Ménzär Fér}. That's where he walked up from the river. He dropped also some crumbs of the yam cake there.}
\end{Parallel}

    \vspace{.4cm}
    
\begin{Parallel}{0.47\textwidth}{0.47\textwidth}
    \ParallelLText{\noindent \textit{Mane yanra e zrä zöfäthak bä brä. Zafe ŋazi fr nä fä fof ethn berä. Watik nä fä fof ŋantnägwath fof. Fä fof sakuka ``Oh zane zf zunthorakwé watik!'' Fz zamara. Afa kfokfo zakarisa. Bäne zakarisa ythama. Watik krenafth ``Nima wiyak. Zbo kar rä farem kar.'' Watik fthé yaka bobo foba krekaris ``Oh füthan nä zbo kabe yamnzr.'' We foba krekaris ``Oh farem karen kabe yé.'' Watik yako. Faremaneme kabe z sathora. Bafane bada fof fatamaane. Farem thden watik foba fof sräkor. Foba fof ``Bä fä fof gnamnzé! Ey, fisor bthanen käms!'' Wati we nä sräthoro wazu. Fä fof sräkor watik ``Foba fof käms wazufa!''}}
    \ParallelRText{\noindent He came right here to \textit{Zöfäthak}. There are some more crumbs over there by those old coconuts. He dropped some more over there. Then he stopped and said ``That's the place I was looking for.'' He looked at the forest and again he heard the butcherbird and the other one, the bird of paradise. Then he said ``I will go this way. There is a place, \textit{Faremkar}.'' When he came there, he heard something and said ``Someone is staying in \textit{Fütha}.'' He listened again and said ``Oh, there is already someone at \textit{Faremkar}.'' He continued and saw that the ancestor of the \textit{Farem} was already there. Those one's ancestor, Fatama's ancestor. He was staying in the centre of \textit{Faremkar}. He said ``You stay over there! You can settle in \textit{Fisor Béthan}!'' Then another ancestor arrived, from the \textit{Wazu} clan. He said: ``You can settle there in \textit{Wazu}!''}
\end{Parallel}

    \vspace{.3cm}
    
\begin{Parallel}{0.47\textwidth}{0.47\textwidth}
    \ParallelLText{\noindent \textit{Watik fthé zamara. Katan fäth ane zfrärm ``Kwa nzä zä zf kwramnzr? Nima ŋabrigwé.'' Moba fthmäsü zänbrima. Watik nasi nömä ane fof tnägsi thenthkäfa rrfar futhfuth. Watik mane yaniyaka zä zf. E zane zf zethno zerä. Fä fof ane futhfuth thuntnägwrm. Nä bä enthn. E nima ziyaro ŋe masu. Foba fof sathora ``Nzukar zä zf ämnzr.'' Watik menz kar ane fof zräkorth yari.}}
    \ParallelRText{\noindent Mathkwi looked around and saw that it was a very small piece of land. He replied ``Oh, this is where you want me to live? I will go back this way.'' On the way, he was dropping crumbs of the yam cake, those scraps. They're right here. They are lying here. He kept dropping these crumbs. There are some lying right over there. He continued along this way to \textit{Masu}. When he arrived, he said: ``This is my place. My people will live here.'' They call this story place \textit{Yari}.}
\end{Parallel}

    \vspace{.3cm}

\begin{Parallel}{0.47\textwidth}{0.47\textwidth}
    \ParallelLText{\noindent \textit{Yari sathora fof. Watik fä fof no nzigfukarä fi fof sathora fof. Fi mane yanra nzigfu nä fofosa yfathwa fof. Nasi wawa duga biskar ranzo fä dagon eräro. Anekaräsü swamnzrm fof. Fthé wawa thuworthrmth. Watik sfrärm e wawa taga kwot thkarthé kwafiyokwrmth. Watik fthé fof wawa taga nä thurtnwrm. Nasi taga kemar taga taga bäne berä biskar duga. Watik nzigfu mrmr foba sfrärm. Ane tagame su-myuknwrm. surdiknwrm. Watik wawa zfthen swäzin. Sfthnm e wawa fthé thwemar nima thkarthé zäkorth. Watik ausiausi thukonzrm: ``Käthfe kabe!'' Ausiausi thfyakm ŋanz ffrümenzoma. Wawa ane ebar fr wawa ebar fr kafar. Watik nä ŋanzma wawa näbi nä ŋanzma nä ŋanzma nä ŋanzma nimanzo watik. Mnime thufränzrmth. Watik thufthakwrmth. Foba karome thurzathrmth. Zizi ane fof thfzänzrmth bobo far mä suräzrmth. Mathkwi o karawa o kukuma o ote. Watik ane far fof sfrästhgrm. Wawa fobo fof thufnakwrm. Fobo fefe né wawa kwanäbü-nzrmth. Kwosi kwakonzrmth bänemr e tayo tfotfo.}}
    \ParallelRText{\noindent So he arrived in \textit{Yari}. He brought the rain making stone and another stone, the yam stone, for long yam, yam, taro, and cassava, for all these other types of crops. He was holding on to those items. The planting season came. Then the leaves of the yams became dry. He pulled off some of the yam leaves or long yam leaves, or sweet potato leaves, all those ones, cassava leaves also. He put the stone inside the leaves and rolled it up. He tied it up in a bundle and put it down at the base of the plants. It stayed there until he saw that the yam leaves became really dried out and wrinkled. Then he called the women of the village: ``Go people!'' The women went and took one yam from each garden plot. They took one especially big yam. One yam from this plot, and another one from that plot, and so on. They burned off the hair on the yam tubers. Then they took them out of the fire and put them in the ground oven. In the afternoon they took them out and brought them to the place where they had set up a post. All those grandfathers were doing that: Mathkwi, Karawa, Kukuma and Ote. They had set up a post. They laid the yams at the base and left them there until they were about to spoil and start to rot. They waited until the other yams were almost ripe.}
\end{Parallel}

\newpage
\begin{Parallel}{0.47\textwidth}{0.47\textwidth}
    \ParallelLText{\noindent \textit{Tayo wawa fthé kwakonzrmth. Rfnak-sir bobo zarfa thfrärm mäta garaita. Nafa fthé kwänrfnth ``Ayo!'' Wrwr fof zefaro swänrifthth. Watik we masu karé kwekaristh ``Oh nafa z zärfnth.'' We kwot we näbikakme we nä wawa thf-rärmth. Katan o kafar thuwok-nzrmth. Watik kwarzathrmth. Tawar ane thfrä-rmth. ŋazi thurwrmth. Kwot thufathwrmth kobakob. Watik kwarfnakwrmth. Watik nima kwanrzrmth. Fatr nima thwafiyokwrmth ``Ayo farem benm fräro!'' Watik ane fthé kwärit ane tmatm rfnaksi tmatm. Thumarwrmth e rrr kwan fthé bäne kwäkorth. Tayo kwot thuwäkwrm. Watik fthé fof yaka swefafth.}}
    \ParallelRText{\noindent When the yams were ready, they heard from \textit{Mäta} or \textit{Garaita} that they had already started tasting the yams. Every time they tasted the new yams, they shouted, ``Listen up!'' and the wind carried the message here. The \textit{Masu} people heard it and said: ``Oh, they have already tasted their yams.'' So they also took one yam from each plot. They chose big or small ones. They put it in the oven. They scraped the fibres out of the inside, mixed them with coconut and formed round balls. They tasted them and then threw some of the balls in this direction. They threw their arms in the air and shouted: ``Listen, you Farem people, this is for you!'' That's what they did. This tasting ritual then moved from place to place further west. Now, the leaves had really dried out and the yams were ready. They picked up the digging stick and started harvesting.}
\end{Parallel}

    \vspace{.3cm}
\begin{Parallel}{0.47\textwidth}{0.47\textwidth}
    \ParallelLText{\noindent \textit{Anenzo fof. Ane tmatm kwaritakwrm e zbo bäydbo. Bäi kafar zäkora. Nafa-ŋafyf nafane ŋafyf ane fof sara fof foba fof otef. Watik naf we ane fof thwamonegwrm no bäne no nzigfu a fofosa frä dagon fofosa fof. Foba e ni kafar ŋankonzake. Nzesinenwä ane fof komnzo thfrnm ane eda rokar fof. E nama masun ane yam tmatm z zwabragwre fof. E watik foba zänbrimake zena mänwä zä namnzr. Zf znrä. Watik fi fthmäsü kwik kwosi yara. Greg täwdben ane thfrärm ane bäne nzigfu thfrnm edawä. Watik nzenme ŋafe fthmäsü kwosi yara. Watik foba ni miyamr nrä mafadben zena ethn. Z thrifthmath fof.}}
    \ParallelRText{\noindent This ritual was passed down all the way to Bäi. When Bäi became a man, his father gave him the yam stone. His father, Ote had also looked after the rain stone and the yam stone. This continued until we became men and those two stones were under our care. Until recently, we followed this tradition in \textit{Masu}. But then we moved here, to where we live now. That was the time when father became ill and died. Those things were with Greg's father then, these two whatchamacallit, these two stones. Since our father passed away, we don't know who they are with now. Someone must have hidden them.}
\end{Parallel}

    \vspace{.3cm}

\begin{Parallel}{0.47\textwidth}{0.47\textwidth}
    \ParallelLText{\noindent \textit{Watik trikasi mane nŋatrikwé fof, ŋaf-ynm badafa ane fof ŋanritakwa fof. Bada aki kwark benrera fof. Zath kwark enrera e ŋafydbo we nzedbo fof né zänrita nima. Watik zena maf keke wäbragwr ane.}}
    \ParallelRText{\noindent Okay, the story I just told was passed down from our ancestors to the grandfathers, until it came to our father. It was about to be passed down to us. But today, no one follows this tradition anymore.}
\end{Parallel}