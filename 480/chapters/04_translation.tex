\begin{Parallel}{0.47\textwidth}{0.47\textwidth}
    \ParallelLText{\noindent \textit{Zöbthé kabe mane thwamnzrm kabe yf ane thfrärm. Nzone miyatha monme rä manenzo z ŋafyé thebräknath. Anenzo né threbräkné. Tüfr kabe thwamnzrm mayawama. Mayawa mrmrma o maya-wama kafar mane thwamnzrm. Yasi nzone aki a srank. Fi mayawama fof thfrnm. Bagu fi tüfrwä thwamnzrm. Fi baguanme kafar mane swamnzrm a mawoi. Firra kar mrmren sagarama bthi. Okay, ane etha mrn mrmren ane fof thwamnzrm. Fi kabe tüfr thwam-nzrm firra kar mrmren.}}
    \ParallelRText{\noindent The people who lived in \textit{Firra} before, they all had names. I only know the ones my fathers have mentioned. So I will only mention them. Many people from the Mayawa used to live there. They belonged to the Mayawa clan or were elders of the Mayawa. Yasi, my grandfather, and Srank. They were from the Mayawa clan. The Bangu were numerous. One of the Bangu elders who lived there was Mawoi. And there was Béthi of the Sangara. Okay, there were a lot of people within these three clans in \textit{Firra} village.}
\end{Parallel}

\vspace{.4cm}


\begin{Parallel}{0.47\textwidth}{0.47\textwidth}
    \ParallelLText{\noindent \textit{Ane ebar yam fthé zewära firran. Nafa wäfiyokwath bobafa mere. Mere mane enrera warthaŋä zänthafrath. A näbun mane erera nzrarima. Safs kabe zena mane ekonzrth. Nzrari mane thfam-nzrm. Ane kabe fof enrera. Firran mafa thäkwrath kabe.}}
    \ParallelRText{\noindent At this time, a headhunt was taking place in \textit{Firra}. Those who did this came from \textit{Mere}. The ones from \textit{Mere} have joined forces with the \textit{Wartha} people. Others cames from \textit{Nzrari}. Nowadays, they call them \textit{Safés} people, including the ones living at \textit{Nzrari}. These were the men who killed the people at \textit{Firra}.}
\end{Parallel}

\vspace{.5cm}

\begin{Parallel}{0.47\textwidth}{0.47\textwidth}
    \ParallelLText{\noindent \textit{Okay, ane ebar fthé ŋafiyokwa. Nzone aki yasi, nzone ŋafe fthé fof katan fäth sfrärm, fi zizi zenfara. Nima zena zf thräkwrth ebar zan nafaŋarerwä. Nafaŋare yf noko. Matama emoth. Nge nzone ŋafe ane fof kwozi. Nafä fof zenfarath zizi fefe. Fthé efoth kwabthakwrm. Fi z miyatha zäkora nafanemäwä bad yamme. Nima ``Zena kwa zan zbo ŋathorthr. Kabe kwa zena efnzrth firran.'' Watik, srank kma sakora ``Srank ni krafare!'' Srankf zenaftha ``Keke. Efoth zizi fefe rä. Nzä kayé woräro.'' Watik, fi yasi mane yara fi fefe zenfara masu. Masufo fam thänra. Karbu ote mane fobo thwamnzrm ane badabada. Koyä ttfr mayawama kkafar. Watik, fi ane fefe zenfara ane miyathan ane zizi.}}
    \ParallelRText{\noindent When that headhunting raid took place, my grandfather Yasi, my father was a young baby boy, he left in the afternoon. He knew that ``They will kill them today.'' He left with his wife. His wife's name was Noko, a woman from \textit{Mata}. The child, my father, was Kwozi. They set off together very late in the afternoon when the sun was going down. He had already worked it out with his own magic. He thought something like this: ``Today the killing will come here'' or ``They will kill the people today in \textit{Firra}.'' He tried to tell Srank, ``Srank, let's go!'' Srank said: ``No, it's already late in the day. I'll come tomorrow.'' Then Yasi really went to \textit{Masu}. He thought of the people in \textit{Masu}, Karémbu and Ote, who lived over there, these ancestors. Koyä and Tétéfér, the elders of the Mayawa. With this goal in mind, he set off in the afternoon.}
\end{Parallel}

\vspace{.6cm}

\begin{Parallel}{0.47\textwidth}{0.47\textwidth}
    \ParallelLText{\noindent \textit{Zan woga thäthora zbär. Z ekrkwath nwir. Thden kar zakorath. Zbär zäthbath. Zan thefafath. Srank a nafangth mane yara. Srank sakwrath. Zätra. Fi keke kwosirwä sakwrath. Watik, thogr thwarenzrm mr rtmaksir. Z zäbrima-koth thogr. Thwarenzrm. Srank fthé mane yara yakä. Nafangth fi z kwosir-wä sakwrath. Watik, srank näbi yasi fam sanra. Yasi ynbragwa masu. Naf fof zätrifa fobo nima ``Firran z thä-kwrth!''}}
    \ParallelRText{\noindent The headhunters arrived at night. They were already blocking the space for the attack. They surrounded the village and came in at night and killed them. As for Srank and his little brother, they struck Srank down. But they did not kill him. They looked around for the small bamboo with which they wanted to cut his neck. When they returned to look for the bamboo knife, Srank ran off. They had already killed his little brother. Srank thought only of Yasi. He followed Yasi to \textit{Masu}. On the way he told everyone: ``They have already killed them in \textit{Firra}.''}
\end{Parallel}

\newpage

\begin{Parallel}{0.47\textwidth}{0.47\textwidth}
    \ParallelLText{\noindent \textit{Fthé kwan swanorm ŋarsen masun neba wazifa. Yasif karbu sakora ``Srank byanor. Garda sarofäth! Kayé kma fof fi sanmisé. Kma né sräkwrmth zan wogané.'' Watik srank trtha byara. Srank ngemäre. Ane efoth yföyfön fthé kwankwirmth. Foba kwanfarkwrmth. Fobo nä karma thwamnzrm nima safs-ma. Safsma woga nä fobo swamnzrm gfi yf. Mafnemäwä bamu yé thbithé zena mafnemäwä erä. Fi z zenfara ane kayé nima ``kayé zan zrarä.'' Fi z zenfara. Fi anema trtha zena zf erä. Fi fthé nyamnzrm nafäsü kwa thräkwrmth. Kabe fobo nä firran mane erera. Kabe karma mane ämnza fi z thäkwrath. Fobo nafä thäthfrath. Fi anenzo trtha woga ferna srank yasi. Yasifa foba fof ni zane zewärake zena znrä. Nzenme ŋafyé berna. Nzone ŋafe kwozi maiti. Fi trtha berna. Srank ngemäre zäbtha. Ŋare ffé zefafa. Fi ngemär bana yara.}}
    \ParallelRText{\noindent Then he shouted at the river in \textit{Masu} from the other side. Yasi said to Karé-mbu ``Srank is shouting there. Take the canoe for him! I tried to take him with me yesterday. These headhunters nearly killed him.'' So Srank survived, but he didn't leave any children behind. Back then, when they ran away and left, other people had settled there, for example people from \textit{Safés}. There was a man from \textit{Safés} by the name of Géfi. His father was Bamu, but they actually come from \textit{Thémbithé}. Géfi had left already. He knew there would be an attack. So he left. That's why they his children are still alive. If he had stayed, he would have been killed along with others. The other people in \textit{Firra}, people from other villages who had lived there, were also killed and mixed with them. Only those two survived, Srank and Yasi. We go back to Yasi; from that time until today. Our fathers were Kwozi and Maiti. This family line survived, but Srank died later without leaving children. He did get married, but the poor chap had no children. }
\end{Parallel}

\vspace{.5cm}

\begin{Parallel}{0.47\textwidth}{0.47\textwidth}
    \ParallelLText{\noindent \textit{Ausi fäth mä rera ane zan mrmren. Nafane yf zafo. Akrimogma emoth zafo. Watik fi mane rera. Mathmath gamo zäzira. Nasi yfön zäthba. Watik zan wogané komnzo zfnagwrmth. Watik zan woga fthé zäzinbrath. Fi fthé fof zänmätra nasi yföfa akrimogo. Akrimogo karnm naf thätrifa ``Mawoi z sakwrth!'' Kafar mane yara firrama kafar mawoi. ``Mawoi z safafth.'' Watik fobo fof miyatha zäkorath ``Oh firran z thäkwrth!'' Nezä faw brigsir mane rera thrmawä eyaka. Ŋarsen kma thethräfath. Fi z zaföwä zäritakoth. Naf fof ŋariza yf fof. Wära zokwasime mane kwanafrmth fobo. Wartha zokwasime kwanafrmth. Kanza zokwasime. Wära zokwasi woga nä foba yariza. Woga mane ŋanafa emoth wäthräkwa ``Nzone ymoth fob rä mafo kmam zethfro!'' Nafadba ane ausi fäthdba miyatha zo-kwasi fof zathorath nima ``Okay, keke zagr woga zanr thwaniyak. Fi zba zrä. Safsma woga thwaniyak nzrarima.''}}
    \ParallelRText{\noindent There was another woman in that headhunting raid. Her name was Zafo. Zafo was from \textit{Akrimongo}. She used magic and hid in the hole of a long yam. The headhunters simply missed her. When the headhunters had passed,  she came out of the long yam hole and ran to \textit{Akrimongo}. She told the \textit{Akrimongo} people ``They've already killed Mawoi.'' Mawoi was an elder from \textit{Firra}. ``They've already got Mawoi.'' That's how they found out: ``Oh, they killed them in \textit{Firra}.'' As for their revenge, that came much later. They should have stopped the headhunters at the river, but they had already crossed the river. The woman had heard their names. Some were talking in Wära language. Some talked in Wartha and Kánchá. She heard one guy speaking Wära, a man who prevented the others from killing his sister saying ``That's my sister over there. Don't mix her up with the others!'' From this young woman the knowledge spread and people said ``Okay, they did not come from far away to kill them. There are from close by, from \textit{Safés}, from \textit{Nzrari}.''}
\end{Parallel}

\vspace{.6cm}

\begin{Parallel}{0.47\textwidth}{0.47\textwidth}
    \ParallelLText{\noindent \textit{Watik nezä faw mane rera. Ane firra zanane faw. Nezä faw z wbrigrnath. Nzenme aki a ote mafanemäwä zena kaunsl yé nane marua ane family bramöwä kadikafu. Okay nafa nezä z faw wbrigrnath. Bänema nafanme mayawa kakafar z bramöwä thäkwr-ath firran. Ane ebar mane rera fawkarä-sü rera. Nezä mane erera zagr keke eyaka. Mane né merefo themiyarath. nezä faw merema woga kma thrä-kwrmth. Komnzo zäwthefath. Safsma woga thäkwrath nzrarin. Krsi znen fobo fof thethräfath. Fobo fof thäkwr-ath. Watik faw z ŋabrigwa. Ane ebar nimame firran fof rera ane.}}
    \ParallelRText{\noindent Well, the revenge that took place was the payback for \textit{Firra}. Our grandfathers paid them back and also Ote whose grandson is now the councillor, brother Marua, Kadikafu and the whole family. Okay, they took revenge because all their Mayawa elders were killed at \textit{Firra}. Those cut-off heads would be avenged. They did not go far for their revenge. They were about to go to \textit{Mere}. They almost took revenge on the people from \textit{Mere}. But they decided to turn around and kill the \textit{Safés} people at \textit{Nzrari}. They surrounded them there and killed them. That was the revenge. This is what happened during the \textit{Firra} headhunt.}
\end{Parallel}

\newpage

\begin{Parallel}{0.47\textwidth}{0.47\textwidth}
    \ParallelLText{\noindent \textit{Fobafa mane rä fobo fof kerker kwan-bthakrm ane fof. Monme zane ebarane zan zfrärm fof. Woga thufnzrmth. Ebar thwärtmth. Thfzänzrmth karfo. Ane mane rera edaŋane ane fof rera. Ane fof rera. Last fefe mä zermäna fethkakaren. Zbo mane thäkwrath. Fethkakar karen fthé thäkwrath. Näbi fthé fof zemathakoth. Fobo fof zäkora. Ebar fobo fof zäbtha. Zanemr zena znrä. Wati zenafa ni tüfr nagayé kwakonzre. Fi ane ebar-ane zokwasi fthé z maf kratrikwro. Fi zena mrmren fthé niné wbrigwre näbuné. Fthé zmarwre zane nzenme kar mrmren thd moreheaden. Nimame krenafth-mé thd moreheaden kar mane erä. Ni woga tüfrmäre nrä. Bänema nzenme thden ane fof kwakwirm. Woga finzo finzo kwafnzrmth. Nä kar wogané nä kar thwemiyarth. Nä kar wogané nä kar woga thwemiyarth. Okay ni nimame fof ŋafiyokwake.}}
    \ParallelRText{\noindent From then on, the hunting raids came to an end. What they did in the headhunt, they killed the people, cut off their heads and carried them to the village. This incident was the second to last headhunt. The very last one, where they killed people, took place at \textit{Fethka}. When they killed the people of \textit{Fethka}, the survivors ran away for good. From this point on, headhunting was over. That is why we are alive today. We are many children now. But this headhunting story, whoever may tell it, when we or others retell it, you just need to look inside our village, in the middle of \textit{Morehead}. I can tell you, in the middle of \textit{Morehead}, those are the villages where people came from. We are not many people, because this happened in our midst. People killed each other. One group of villagers attacked other villages, and again another group of villagers attacked other villagers. Okay, that is what we have done.}
\end{Parallel}

\vspace{.4cm}

 \begin{Parallel}{0.47\textwidth}{0.47\textwidth}
    \ParallelLText{\noindent \textit{Nzenme mrmren nima fof ŋakwira yam. Anema kabekaräsü ni nrera. Ni mrnmäwä nrera. Firran mane thä-kwrath woga fi fefe zäbthath. Nafane story nimame firraŋane fof rä brä. Nä keke mane fä sathfära. ninzo fefe nznethfära. Yasiane family zena znrä. Ni yasiane mrnma. Nge nafane. Nima ni nafane ane badabadama fof nrä. Nzenme mrn mrmren z monme miyatha ŋakwir. Gadmöwä fi zbo komnzo ämrna karbu ote. Firranzo bä bthrarärm. Mane fthé z thräkwrmth fobo fof. Keke kwa ni tüfr nzrarärm. Fi mon zbo ämrna. Yasif ane fof fam thnra. Nafane-mäwä mrn fof. Watik zänbrima fi zanmatha. Nzone ŋafe näbinzo finzo.}}
    \ParallelRText{\noindent This custom was common with us. We were part of that. We come from that clan. The \textit{Firra} headhunt really finished off these other clans. This is the story about the \textit{Firra} headhunt. Nobody could escape there, only we could escape. From Yasi's family to us today.  We are from Yasi's family. We are his children. We go back to his ancestors. We know about this in our family. Fortunately, Ote and Karémbu had stayed here. If they had been there at \textit{Firra}, they would have been killed too, and we would not be many people today. Because those two were staying here, Yasi thought of them, really his own clan, he came back, he ran really. My father was the only son.}
\end{Parallel}

\vspace{.4cm}

\begin{Parallel}{0.47\textwidth}{0.47\textwidth}
    \ParallelLText{\noindent \textit{Fobo fof wythik ane zokwasi fof firra ebarma. Keke zanfrme rä. Fobo komnzo zänbtha fof. Nima kerkeren efothen zfwiyakm. Monwä z nznanafthmé. ker-ker efothen zfwiyakm. Fthé fefe markai zänthba. We mizi zänthbath. Fobo fof wythka. Fthé dibura fathasi thethkäf-ath mafa ane woga thufnzrmth nima-me. Fobo fof zwaythk ane zokwasi.}}
    \ParallelRText{\noindent The story of the \textit{Firra's} headhunt ends here. It is not a long story. At that time it ended in the last days of headhunting. Like I just told you. This happened in the last days of headhunting. This happened when the white man came and also when the missionaries arrived. After that, the headhunting ended. They started throwing people in prison who had killed someone. This story is now over.}
\end{Parallel}