The story is about two ancestors who each represent a totemic animal and are connected to a specific clan. There is the Bushfowl man of the Masu-Mayawa clan and the Saratoga man of the Garaita-Sangara clan. They used to live in good neighbourhood and shared the land on the same side of the Morehead River. At the same time, each held back a certain resource from the other. The Saratoga man hid Saratoga fish from his neighbour, and likewise the Bushfowl man hid the bushfowls.

One day they learn of each other's deeds and after an argument, the Saratoga man decides to move further east to the other side of the river. The Bushfowl man destroys the crossing over the river and moves further west.

The story explains some physical and cultural aspects of contemporary life, such as the location of the settlement of the two clans. Another example is the totemic relationship of the two groups to the respective animal species. The Garaita Sangara do not fish or eat saratogas, and the Masu Mayawas do not hunt bush fowl. Furthermore, the story is intended to explain why the two clans do not intermarry, but give each other wives if one of the groups lacks a sister to feed into the cycle of exchange marriages.