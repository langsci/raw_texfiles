\addchap{\lsAbbreviationsTitle}
% \addchap{Abbreviations and symbols}
The category labels for abbreviations follow the Leipzig Glossing Rules.\footnote{\url{http://www.eva.mpg.de/lingua/resources/glossing-rules.php}}
\vspace{.5cm}

\begin{tabularx}{.45\textwidth}{lQ}
    ∅ & zero form \\
    \textbackslash.../ & verb stem\\
    . & multi-item gloss (3\textsc{sg.m})\\
    \_ & multi-item lexemes (like\_that)\\
    - & for affixes (\textit{-thé})\\
    = & for clitics (\textit{=en})\\
    | & syncretism (2|3 person)\\
    > & argument structure (1>3 `first person acting on third person')\\
    1 & first person\\
    2 & second person\\
    3 & third person\\
    \textsc{abs} & absolutive\\    
    \textsc{absc} & absconditive (attention getter, `look here')\\
	\textsc{adjz} & adjectivizer\\
    \textsc{all} & allative\\
    \textsc{and} & andative (`away')\\
    \textsc{anim} & animate\\
    \textsc{appr} & apprehensive\\
    \textsc{assoc} & associative case\\
    \textsc{char} & characteristic case\\
    \textsc{dat} & dative case\\
    \textsc{dem} & anaphoric demonstrative\\
\end{tabularx}
\begin{tabularx}{.45\textwidth}{lQ}
    \textsc{dia} & diathetic prefix\\
    \textsc{dim} & diminutive\\
    \textsc{dist} & distal deictic\\
    \textsc{distr} & ditributive\\
    \textsc{du} & dual number\\
    \textsc{dur} & durative\\
    \textsc{emph} & emphatic\\
    \textsc{erg} & ergative case\\
    \textsc{etc} & et cetera (`and all')\\
    \textsc{f} & feminine\\
    \textsc{fut} & future\\
    \textsc{futimp} & future imperative\\
    \textsc{iam} & iamitive (`already')\\
    \textsc{ic} & inclusory case\\
    \textsc{imn} & imminent (`about to')\\
    \textsc{imp} & imperative\\
    \textsc{indf} & indefinite\\
    \textsc{ins} & instrumental case\\
    \textsc{io} & indirect object\\
    \textsc{ipfv} & imperfective\\
    \textsc{ipst} & immediate past\\
    \textsc{irr} & irrealis\\
    \textsc{iter} & iterative\\
    \textsc{loc} & locative case\\
    \textsc{lpl} & large plural\\
    \textsc{m} & masculine\\
    \textsc{med} & medial deictic\\
\end{tabularx}
\newpage
\begin{tabularx}{.45\textwidth}{lQ}
    \textsc{nd} & non-dual\\
    \textsc{neg} & negator\\
    \textsc{nmlz} & nominalizer\\
    \textsc{npl} & non-plural\\
	\textsc{npst} & non-past\\
	\textsc{nsg} & non-singular\\
    \textsc{only} & exclusive marker (`only', `just')\\
    \textsc{pfv} & perfective\\
    \textsc{ph} & placeholder (`thingamajig')\\
    \textsc{pl} & plural\\
    \textsc{pn} & proper name\\
    \textsc{pln} & place name\\
    \textsc{poss} & possessive\\
\end{tabularx}
\begin{tabularx}{.45\textwidth}{lQ}
    \textsc{pot} & potential\\
    \textsc{priv} & privative case\\
	\textsc{prop} & proprietive case\\
    \textsc{prox} & proximal deictic\\
    \textsc{pst} & past\\
    \textsc{purp} & purposive case\\
    \textsc{q} & question\\    
    \textsc{quot} & quotative\\
    \textsc{redup} & reduplication\\
	\textsc{rpst} & recent past\\
    \textsc{sg} & singular\\
    \textsc{simil} & similative case\\
    \textsc{stat} & stative\\
    \textsc{temp} & temporal case\\
    \textsc{vent} & venitive (`towards')\\
\end{tabularx}