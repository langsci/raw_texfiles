\section{Introduction}\label{sec:introduction}

This collection of Komnzo texts has the thematic focus of landscape, place names, and locality. One does not need to dig very deep into Komnzo oral literature in order to investigate this topic. Judging just by the occurrence of place names in speech of any genre, one has to conclude that the anchoring of events in space is of great importance to Komnzo speakers. An early visitor to the region was the Government Anthropologist for the Territory of Papua, Francis Edgar Williams, who observed that ``if you ask your guide where you stand at any moment, he will be able to give a name to the land'' (\citeyear[207]{Williams:1936hb}). In the late 1970s, the anthropologist Mary Ayres was based in the Komnzo-speaking village \textit{Rouku}.\footnote{As a convention, all Komnzo place names are printed in italic font in this book.} Ayres chose locality as the topic of her doctoral thesis — entitled \textit{This side, that side: locality and exogamous group definition in Morehead area, Southwestern Papua} (\cite{Ayres:1983dw}). She points out that the system of place names ``is staggering in its elaboration'' (\citeyear[38]{Ayres:1983dw}). In the following sections of this introduction, I will include some analysis of the way Komnzo speakers use their language to talk about places. 

This book is structured as follows. As most of the cultural, demographic, anthropological and historical background information is given in the grammar of Komnzo (\cite[8ff.]{Dohler:2018qt}), I include here only a summary thereof (\sectref{sec:setting}). After a brief typological overview of the linguistic structure (\sectref{sec:typoloverview}) and a description of the glossing and orthography conventions (\sectref{sec:conventions}), I describe the provenance of the texts and the editing process (\sectref{sec:data}). Next, I address topics and text genres (\sectref{sec:genres}), introduce the speakers (\sectref{sec:speakers}) and the texts (\sectref{sec:texts}). The conceptualisation of landscape, place and locality is topic of Chapter \ref{chapter:space}. The main body of the book, consists of the 14 texts (Texts~\ref{text:kwafar}--\ref{text:ausi}). For each text, I provide the metadata in the introduction, a short summary of the content and some background information, a parallel text arranged in columns (Komnzo/English), and finally the interlinearised and fully glossed text.

\section{The setting}\label{sec:setting}

Komnzo speakers refer to themselves as the ``Farem tribe''. \textit{Faremkar} is the name for the place of origin; the place where the apical ancestors of all Komnzo speaking clans met before they settled their respective places. Thus, in the oral history of Farem people, the concept of a shared place of origin overlaps with speech variety. Henceforth, I use the terms ``Komnzo speakers'' and ``Farem people'' interchangeably.

Komnzo is spoken in the village of \textit{Rouku}, which is located about seven kilometers west of \textit{Morehead} and about one kilometer north of the Morehead River. See \figref{fig:sng-map} for an overview of the area. It is situated on the road that connects \textit{Morehead} with \textit{Weam} in the west. Traditional lands expand about 20km east-west and 25km north-south. Other settlements are \textit{Morehead Station}, 7 km east of \textit{Rouku}, and \textit{Gunana}, 2 km west of \textit{Rouku}. While the people in one of these settlements have permanent houses, their lives are quite mobile, as they spend a lot of time in their gardens or in camps along the river, where they have storage houses and temporary shelters for sleeping.

\begin{figure}
    \centering
    \includegraphics[width=\textwidth]{figures/Yam-family-2.png}
    \caption{Languages in Southern New Guinea with Komnzo in the centre of the map}
    \label{fig:sng-map}
\end{figure}

As is the case for most speech communities in the Morehead district, Farem people are very multilingual. There are demographic as well as cultural reasons for this. Like many other languages in New Guinea, Komnzo is a small language. I estimate the total number of speakers at around 250, a figure that is blurred by the widespread multilingualism. The local marriage pattern dictates that the women move their place of residence to the place of their respective husbands. They almost always come from a different village with a different language. The system can be described as symmetrical sister-exchange. Even though exogamous groups are based on clan and place, Mary Ayres states that it was explained to her sometimes as ``a rule of dialect exogamy: \emph{``We should not intermarry because we talk the same language''} is a phrase sometimes stated by informants'' (\citeyear[186]{Ayres:1983dw}). As a consequence of this quasi-linguistic exogamy, children acquire not just the languages of both parents, but often the languages of their aunts and grandmothers, who might come from yet another language community. A repertoire of four to five local languages is quite common before entering the school system which is taught in English. The high levels of multilingualism — individual as well as societal — are contrasted by a conversational practice in which interactants converse in their respective languages: A Komnzo speaker meeting a Nama speaker on the road converses in Komnzo, while his interlocutor replies in Nama. This pattern is sometimes called ``dual-lingualism'' (\cite{Lincoln:1979vs}) or ``receptive multilingualism'' (\cite{Rehbein:2011ft}, \cite{Singer:2023zj}). Essentially, Komnzo speakers remain loyal to the language of their fathers — their patrilect — mainly for reasons of language ideology, emblematicity and linguistic purism. Over the recent years, a body of descriptive literature has emerged that focuses on this pattern in the Morehead District (\cite{Evans:2012cq}, \cite{Evans:2017lf}, \cite[34-36]{Dohler:2018qt}, \cite{Kashima:2020va}, \cite{Dohler:2021tb}, \cite{Schokkin:2021ct}, \cite{Schokkin:2023dz}, \cite{Dohler:2024kd}).

Superimposed on the indigenous pattern of multilingualism is English as a lingua franca. English is the language of religion, education, media, and administration. The church services are held in English, with important phrases being translated for rhetorical purposes in Komnzo. The language of instruction in the primary school in Rouku and in the high school in Morehead is English. The national and provincial political sphere is conducted in English, as are the few available media channels such as the newspaper Post Courier and Radio Fly. With a few exceptions, everyone is proficient in English. Speakers over 65–70 also know Motu, as this was the language of the church and the school system until the 1970s. Younger men who have travelled and sometimes worked outside the region speak Tok Pisin and Bahasa Indonesia.

The languages of the Morehead District are not under immediate threat from English or from Tok Pisin, whose influence is growing only slowly. Nevertheless, they should be considered endangered languages due to their small size and limited distribution. Most are spoken in a single village. The traditional multilingualism is certainly at risk, as English is becoming more important in communication between different communities. This process can already be observed in the town of Morehead, where people with different linguistic backgrounds meet regularly. Changes of this kind have been the harbingers of language loss in other parts of New Guinea (\cite{Aikhenvald:2002ul}).

The social segmentation of people in the Morehead district is divided into a multi-layered structure (cf. \cite{Williams:1936hb}, \cite{Ayres:1983dw}). The most basic level of organisation is the clan. Clan membership is passed through the father's line, and clans are localised in the sense that they exist in and are associated with a particular tract of land. At the second level, all clans belong to one out of three groups: \textit{Bangu}, \textit{Sangara}, \textit{Mayawa}. I use the term ``section'' for this level. If there were not three (but two) groups at this level, the more established term ``moiety'' would be appropriate here.\footnote{Moiety from French `one-half' is used in anthropological literature to classify kinship systems that are characterised by a system of dual-oppositions.} Sections are non-local in that one finds them in almost all villages in the Morehead district. For example, there are two Mayawa clans in the Komnzo speaking territory (\textit{Mrzar Mayawa} and \textit{Banibani Mayawa}), their lands are adjacent, they belong to the same exogamous group, they share the same totemic animals and most mythological stories. At the same time, different Mayawa clans are found at other villages in the Morehead district. At the third level, all clans of all sections in a specific area claim prior unity at a particular place. This often involves the travelling myths of the respective apical ancestor. One such myth is given in \textref{text:kwafar}. For \textit{Rouku}, this place is called \textit{Faremkar}. The third level aligns with language variety, thus, members of all clans (of all three sections; and all clans within these sections) who claim their mythical origin at \textit{Faremkar} are also speakers of Komnzo. An overview of the clans and sections of the Farem people is given in \tabref{tab:clans}.

\begin{table}
\caption{Clans of the Farem people}
\label{tab:clans}
\centering
	\begin{tabularx}{.75\textwidth}{llr}
	\lsptoprule
		\textsc{section}&\textsc{clan name} (`translation')&\textsc{\# of patrilines}\\
		\midrule 
		Bangu&\emph{Nümgar} (`crocodile')&1\\
		Sangara&\emph{Farem} (place name)&2\\
		Sangara&\emph{Wazu} (place name)&2\\
		Sangara&\emph{Mutherata} (place name)&1\\
		Mayawa&\emph{Banibani} (`Brahminy Kite')&2\\
		Mayawa&\emph{Mrzar} (proper name)&1\\
	\lspbottomrule
	\end{tabularx}
\end{table}%clans

In addition to self-attribution, there is a web of more or less visible markers that distinguish a member of one group from that of another group regardless of the level. Such markers include certain designs printed on grass skirts, particular patterns carved on arrows or drums, special songs and dance styles. Furthermore, there are totemic animals which one must not hunt or eat. Some of these markers overlap, for example the Swamp Eel (\textit{dobakwr}) is a totem for all Mayawa and all Sangara clans. The most important fact about sections and clans lies in land ownership and exogamy. Moreover, clan affiliation is not only a theme in many stories, but some stories are also the property of a particular clan (cf. \sectref{talkingplaces}). For more information on the topic of social segmentation see Döhler (\citeyear[26ff.]{Dohler:2018qt}).

\section{Grammar sketch}\label{sec:typoloverview}

\subsection{Introduction}

Komnzo is a Papuan language of the Yam language family. The term ``Papuan'' was introduced by \textcite[16]{Ray:1895jw}, who defined these languages ex-negativo as the languages of the region near New Guinea that are neither Austronesian nor Australian. The number of distinct Papuan language families that have been proposed ranges from ten (\cite{Wurm:1975dk}) to 23 (\cite{Ross:2005cu}) up to 60 (\cite{Foley:1986jl}). Although authors acknowledge the incredible diversity within New Guinea, there have been some attempts at defining grammatical properties which are characteristic for Papuan languages, for example the presence of tone, the absence of dependent marking (flagging), and switch-reference systems (\cite{Foley:1986jl}, \citeyear{Foley:2000uh}).\footnote{Other authors have shown that the languages of New Guinea do not share a set of typological features of Papuan type that set them apart from the languages of the world (\cite{Comrie:2012yo}).} Komnzo, the languages of the Yam family, and possibly the whole Southern New Guinea area deviate from this Papuan type in a number of ways. In the following sections, I briefly introduce the typologically most striking features of the language. For an in-depth description, I refer the reader to the grammar (\cite{Dohler:2018qt}).

\subsection{Phonology}

The Komnzo phoneme inventory consists of eight vowels and 18 consonants. The vowels are the five cardinal vowels /i/, /e/, /a/, /ɔ/, /u/ plus a low front unrounded vowel /æ/, and two front rounded vowels /y/ and /œ/, which are unusual for Papuan languages.\footnote{Outside of the Yam family front rounded vowels are also found in Awyu-Dumut languages (\cite[60]{vanEnk:1997tl}).} The most frequent vowel is the epenthetic vowel, schwa [ə], which not analysed as a phoneme. \figref{fig-01-vowels} shows the vowel inventory.
\vspace{-.3cm}
\begin{figure}
	{
		\begin{vowel}[plain]
			\putvowel{i}{0,5\vowelhunit}{0,4\vowelvunit}
			\putvowel{y}{1,0\vowelhunit}{0,4\vowelvunit}
			\putvowel{e}{1,3\vowelhunit}{1,5\vowelvunit}
			\putvowel{œ}{1,8\vowelhunit}{1,5\vowelvunit}
			\putvowel{æ}{2,1\vowelhunit}{2,5\vowelvunit}
			\putvowel{u}{3,7\vowelhunit}{0,4\vowelvunit}
			\putvowel{a}{2,9\vowelhunit}{2,7\vowelvunit}
			\putvowel{(ə)}{2,7\vowelhunit}{1,3\vowelvunit}
			\putvowel{ɔ}{3,7\vowelhunit}{1,5\vowelvunit}
		\end{vowel}
	}%
	\caption{Vowels}
	\label{fig-01-vowels}
\end{figure}%vowels
\vspace{-.3cm}
\begin{table}
\caption{Consonants}
\label{tab-01-cons}\small
	\begin{tabularx}{\textwidth}{p{2cm}XXXXXXX}
		\lsptoprule
		& {bilabial}& {dental} & {alveolar} & {palato-alveolar}	& {palatal} & {velar} & {labio-velar} \\ \midrule
		{plosive} \&&&&t&&&k&kʷ\\
		{affricate}&&&&ts&&&\\
		&&&&&&&\\
		{prenasalised} &ᵐb&&ⁿd&&&ᵑg&ᵑgʷ\\
		plosive \& &&&&&&&\\
		affricate&&&&ⁿdz&&&\\
		&&&&&&&\\
		{fricative} & ɸ&ð&s &&&&\\
		&&&&&&&\\
		{nasal} & m && n &&& ŋ & \\
		&&&&&&&\\
		{trill/tap} &&& r &&&&\\
		&&&&&&&\\
		{semivowel} &&&&&j && w\\
		\lspbottomrule
	\end{tabularx}
\end{table}

There are 18 consonantal phonemes, as shown in \tabref{tab-01-cons}. These follow a set of pairs of voiceless and prenasalised plosives at the alveolar and velar points of articulation: /t/, /ⁿd/, /k/, /ᵑg/. There are labialised velars: /kʷ/, /ᵑgʷ/. At the bilabial point of articulation there is only a prenasalised plosive /ᵐb/, while its oral counterpart /b/ only occurs in loanwords. There are three {nasals} /m/, /n/, /ŋ/, one trill/tap /r/, two semivowels /j/, /w/ and, again unusual for Papuan languages, three {fricatives} /ɸ/, /ð/, /s/ and two affricates /ts/, /ⁿdz/. It follows that we can identify three main points of articulation: bilabial, alveolar and velar. Further points of articulation include dental /ð/, palato-alveolar /ts/ and /ⁿdz/ as well as palatal /j/.

As in other Papuan languages such as {Kalam} (\cite{Blevins:2010ur}) many syllables lack phonemically specified vowels. In this case, an epenthetic vowel may be inserted, usually a short central vowel [ə]. Many words lack phonemically specified vowels altogether, for example the inflected verb \emph{ymgthkwrmth} [jə̆mə̆ᵑgə̆θkʷə̆rə̆mə̆θ] `they were feeding him'.

The {syllable} structure allows for complex onsets of the type CRV, as in \emph{gru} `shooting star' or \emph{srak} `boy'. Otherwise onsets are simply CV. Even though vowel-initial words exist, they are always produced with a {glottal stop}, as in \emph{ane} [ʔane] `that' or \emph{ebar} [ʔeᵐbar] `head'. Syllable codas do exist, but they consist of maximally one consonant.
\vspace{-.1cm}
\subsection{Morphology}
\vspace{-.1cm}
Komnzo morphology can be used to easily distinguish nominals from verbs. As in other Yam languages such as Nama (\citealt{Siegel:2017ku, Siegel:2023ay}) and Nen (\cite{Evans:2015aa}), Komnzo verb morphology exhibits a high degree of complexity. Verbal morphology is highly synthetic, while nominal morphology is agglutinative and almost entirely suffixing.

Komnzo nouns are inflected for number if their referent is animate. Otherwise number marking only takes place in the verb. Furthermore, nouns are marked for case by enclitics, which attach to the last element of the noun phrase. \tabref{tab-01-case} shows the case markers for the inanimate noun \emph{efoth} `sun, day' and the animate noun \emph{kabe} `man, people'. For more information on the range of meanings of case marked noun phrases, I refer the reader to the grammar (\cite[135ff.]{Dohler:2018qt}).

\begin{table}
\caption{Case markers on \textit{efoth} `sun, day' and \textit{kabe} `man, human'}
\label{tab-01-case}
\centering
	\begin{tabularx}{\textwidth}{Xlll}
	\lsptoprule
		&\textsc{inanimate}&\textsc{animate} (\gl{sg})& \textsc{animate} (\gl{nsg})\\
		\midrule 
		Absolutive&\emph{efoth}&\emph{kabe}&\emph{kabe=é}\\
		Ergative&\emph{efoth=f}&\emph{kabe=f}&\emph{kabe=é}\\
		Dative&\emph{efoth=n}&\emph{kabe=n}&\emph{kabe=nm}\\
		Possessive&\emph{efoth=ane}&\emph{kabe=ane}&\emph{kabe=aneme}\\
		Locative&\emph{efoth=en}&\emph{kabe=dben}&\emph{kabe=medben}\\
		Allative&\emph{efoth=fo}&\emph{kabe=dbo}&\emph{kabe=medbo}\\
		Ablative&\emph{efoth=fa}&\emph{kabe=dba}&\emph{kabe=medba}\\
		Temp. locative&\emph{efoth=thamen}&n/a&n/a\\
		Temp. purposive&\emph{efoth=thamar}&n/a&n/a\\
		Temp. possessive&\emph{efoth=thamane}&n/a&n/a\\
		Instrumental&\emph{efoth=me}&n/a&n/a\\
		Purposive&\emph{efoth=r}&n/a&n/a\\
		Characteristic&\emph{efoth=ma}&\emph{kabe=anema}&\emph{kabe=anemema}\\
		Proprietive&\emph{efoth=karä}&\emph{kabe=karä}&n/a\\
		Privative&\emph{efoth=mär}&\emph{kabe=mär}&n/a\\
		Associative&\emph{efoth=ä}&n/a&n/a\\
		Inclusory&n/a&\emph{kabe=r}&\emph{kabe=ä}\\
		Similative&\emph{efoth=thatha}&\emph{kabe=thatha}&n/a\\
	\lspbottomrule
	\end{tabularx}
\end{table}%case

Nominal morphology in Komnzo is comparatively simple. Case marking is shown by enclitics that attach to the rightmost element of a noun phrase, which is usually a head noun as in (\ref{exintro2}), but may sometimes be a modifier as in (\ref{exintro3}).

\ea \label{exintro}
	\ea \label{exintro2}
        kafar kabefnzo\\
        \gll kafar kabe=f=nzo\\
		big man=\gl{erg.sg}=\gl{only}\\
		\glt `only the big man (did sth.)'
	\ex \label{exintro3}
        kabe kafarfnzo\\
        \gll kabe kafar=f=nzo\\
		man big=\gl{erg.sg}=\gl{only}\\
		\glt `only the big man (did sth.)'
    \z
\z

In contrast to nominals, verb morphology is highly synthetic. Verbs may index up to two arguments showing agreement in person, number and gender. Verbs encode 18 TAM categories, valency, directionality and deictic status. Complexity lies not only in the number of categories that verbs express, but also in the way how these categories are encoded.

\subsection{Distributed exponence}\label{sec:distributedexp}

Komnzo verbs exhibit what may be called ``distributed exponence'' (\cite{Carroll:2016bf}); an exponence type that is characterised by morphemic underspecification. For many grammatical categories, different slots have to be taken into account in order to arrive at the value of a grammatical category. This phenomenon is different from multiple exponence (e.g. circumfixes) in that each morphological slot can be manipulated independently. The basic principle is shown in \tabref{tab-01-thoraksi} in the expression of a few selected tense-aspect-mood (TAM) values for the verb \emph{thoraksi} `arrive, appear'; all in \gl{3sg.m}.

\begin{table}
\caption{Distributed exponence: TAM}
\label{tab-01-thoraksi}
\centering
	\begin{tabularx}{.66\textwidth}{Xl}
	\lsptoprule
		Non-past imperfective & \emph{y-thorak-wr}\\
		Recent-past imperfective & \emph{su-thorak-wr}\\
		Recent-past durative & \emph{y-thorak-wr-m}\\
		Recent-past perfective & \emph{sa-thor}\\
		Past imperfective & \emph{y-thorak-wr-a}\\
		Past durative & \emph{su-thorak-wr-m}\\
		Past perfective & \emph{sa-thor-a}\\
		Iterative & \emph{su-thor}\\
	\lspbottomrule
	\end{tabularx}
\end{table}

Distributed exponence means that we cannot gloss the prefix \emph{y-} for a tense value, because it is used for the inflections of non-past, recent past and past. Furthermore, glossing the suffix \emph{-m} as a durative is only half of its function as it backshifts tense as well from non-past to recent past and again from recent past to past tense. In fact, the only TAM morpheme in the table that serves only one function is the past suffix \emph{-a}. Moreover, the table shows that the verb stem itself is also an exponent of TAM \emph{thorak} versus \emph{thor}. Indeed, most Komnzo verbs possess two stems which are sensitive to aspect, and for many verbs these are suppletive pairs (\cite[180ff.]{Dohler:2018qt}). Again, the stem alone is not sufficient to express the aspectual values (imperfective, perfective, iterative, durative), but it is the combination of stem type, prefix and suffix.

Distributed exponence is best explained with the way Komnzo marks number on verbs. The four possible values are singular, dual, plural, and large plural.\footnote{Large plurals are available only for a small subset of verbs (\cite[219ff.]{Dohler:2018qt}).} The exponents of number are distributed over two morphological slots. There is a binary distinction in the prefix (\emph{y-} vs. \emph{e-}) and the suffix (\emph{-thgr} vs. \emph{-thgn}). The four possible combinations of these exponents encode the four number values. This is shown with the intransitive verb \emph{migsi} `hang' in a third person frame in \tabref{tab-01-migsi}.

\begin{table}
\caption{Distributed exponence: number}
\label{tab-01-migsi}
    \centering
	\begin{tabularx}{.3\textwidth}{ll}
    \lsptoprule
		\gl{3sg} & \emph{y-mi-thgr}\\
		\gl{3du} & \emph{e-mi-thgn}\\
		\gl{3pl} & \emph{e-mi-thgr}\\
		\gl{3lpl} & \emph{y-mi-thgn}\\
    \lspbottomrule
	\end{tabularx}
\end{table}%migsi

\subsection{Syntax}\label{sec:overview-syntax}

Komnzo is a double-marking language. The case marking is organised in an ergative-absolutive system. In addition to three core cases (absolutive, ergative and dative), there are 14 semantic cases (cf. \tabref{tab-01-case}). Verbs index up to two arguments. The undergoer argument is indexed by a prefix and the actor argument is indexed by a suffix, as in (\ref{exintro4c}) and (\ref{exintro4d}) below. One-place predicates split along the lines of stative versus dynamic event types. The latter employ the suffix for indexing, as in (\ref{exintro4b}), while the former make use of the prefix, as in (\ref{exintro4a}). However, in many cases the assignment of a verb lexeme to those inflectional patterns is rather idiosyncratic. Valency changing morphology, glossed as \gl{dia} (for diathetic) in (\ref{exintro4d}), enables the indexing of a goal, beneficiary or possessor in the prefix. This is shown below with the verbs `sleep' (\ref{exintro4a}), `return' (\ref{exintro4b}), `see' (\ref{exintro4c}) and `give' (\ref{exintro4d}). I use the term ``template'' to describe the different inflectional patterns in which verb stems are found. Templates are fixed for some lexemes, but on the whole the system is remarkably fluid (\citealt{Dohler:2022ab, Dohler:2023rc, Dohler:2023oz}).

\ea \label{exintro4a}
    fi yrugr.\\
    \gll fi y-rugr\\
	\gl{3.abs} \gl{3sg.m}-sleep\\
	\glt `He sleeps.'
\z
\ea \label{exintro4b}
    fi ŋabrigwrth.\\
    \gll fi ŋabrigwr-th\\
	\gl{3.abs} return-\gl{3pl}\\
	\glt `They return.'
\z
\ea \label{exintro4c}
    nafa fi ymarth.\\
	\gll nafa fi y-mar-th\\
	\gl{3pl.erg} \gl{3.abs} \gl{3sg.m}-see-\gl{3pl}\\
	\glt `They see him.'
\z
\ea \label{exintro4d}
    nafa yare kaben yarithrth.\\
    \gll nafa yare kabe=n y-a-rithr-th\\
	\gl{3pl.erg} bag(\gl{abs}) man=\gl{dat.sg} \gl{3sg.m}-\gl{dia}-give-\gl{3pl}\\
	\glt `They give the man the bag.'
\z
\newpage
The most frequent word order in Komnzo is SOV, more accurately AUV\footnote{AUV: actor undergoer verb.}, since there is only weak evidence for a subject category. At the same time, the flagging of noun phrases with case allows for considerable freedom in the word order patterns. Nominal compounds and noun phrases are typically head-final, although modifying elements in the noun phrase, for example adjectives or quantifiers, may occur after the head. Relative clauses follow their head.

Subordinate clauses in Komnzo are usually non-finite employing nominalised verbs with appropriate case markers. Verb chaining and the distinction between medial and final verb forms, which are typical for Papuan languages, are not found in Komnzo. The following examples show a phase complement (\ref{exintro5}) and a complement of desire (\ref{exintro6}).

\ea
	\label{exintro5}
	nafa with rkusi thethkäfath.\\
    \gll nafa with rku-si the-thkäfa-th\\
	\gl{3nsg.erg} banana(\gl{abs}) knock\_down-\gl{nmlz} \gl{2|3pl}-start-\gl{2|3pl}\\
	\glt `They started knocking down the bananas.'
\z
\ea
	\label{exintro6}
    fi miyo yé nge fathasir.\\
	\gll fi miyo yé nge fatha-si=r\\
	\gl{3.abs} desire \gl{3sg.m}.be child hold-\gl{nmlz}=\gl{purp}\\
	\glt `He wants to hold the child.'
\z

In addition to nominalised verbs, clauses can be connected with conjunctions (\ref{exintro7}), relative pronouns (\ref{exintro8}) or a placeholder pronominal inflected for case (\ref{exintro9}).

\ea \label{exintro7}
    fi z zebnaf o komnzo yrugr?\\
    \gll fi z zebnaf-ø o komnzo y-rugr\\
	\gl{3abs} already wake\_up-\gl{3sg} or still \gl{3sg.m}-sleep\\
	\glt `Did he wake up already or is he still sleeping?'
\z
\ea \label{exintro8}
    kabe sathor kayé mane sfmarwrme.\\
    \gll kabe sa-thor kayé mane sf-marwrm-e\\
	man(\gl{abs}) \gl{3sg.m}-arrive yesterday which \gl{3sg.m}-see-\gl{1pl}\\
	\glt `The man who we saw yesterday arrived.'
\z
\newpage
\ea \label{exintro9}
    ŋare z zefar bänema nafane kkauna zwarithrth.\\
	\gll ŋare z zefar-ø bäne=ma nafane kkauna zu-a-rithr-th\\
	woman(\gl{abs}) already set\_off-\gl{3sg} \gl{ph}=\gl{char} \gl{3sg.poss} things \gl{3sg.f}-\gl{dia}-give-\gl{3pl}\\
	\glt `The woman has left already, because they gave back her belongings to her.'
\z

\section{Conventions}\label{sec:conventions}
\subsection{Glossing conventions}

The interlinearisation follows the Leipzig Glossing Rules.\footnote{\url{http://www.eva.mpg.de/lingua/resources/glossing-rules.php}} Additional abbreviations that appear in the gloss line are listed in the preamble to this book.

With the exception of verbs, I apply the standard item-and-arrangement model for interlinearisation, i.e., words are segmented on one line, and then glossed on the line below. The phenomenon of distributed exponence (cf. \sectref{sec:distributedexp}) prompts us to take a different approach for verbs, namely the word-in-paradigm model (\cite{Matthews:1974yw}), which takes the word, rather than the morpheme, as the level of analysis. Hence, I do not provide a morpheme segmentation, but instead the verb stem is separated from all affixal material by {\textbackslash}slanted lines/ on the morpheme tier. On the gloss tier, the inflected verb form is placed in its paradigm by listing information in the following order: argument structure, TAM, directionality, and (following a backslash) lexeme translation, as in (\ref{exintro9a}). The values of grammatical categories are separated with a colon (:), e.g. \gl{npst}:\gl{ipfv}:\gl{and} for non-past, imperfective, andative. Grammatical values that belong to the same domain are not separated or separated with a fullstop (.), e.g. \gl{3sg.f} for third singular feminine. The glossing of argument structure does not employ labels such as \gl{sbj} and \gl{obj}, instead an arrow symbol indicates an actor > undergoer relationship, e.g. \gl{1sg}>\gl{3sg.m} for `I (did sth. to) him'. The only exception are ditransitivised verbs, which include the label \gl{io} for indirect object, usually a goal, possessor, or recipient, as the verb `give' in (\ref{exintro9a}).

Note that for the overview section in this publication, especially in \sectref{sec:overview-syntax}, I have used a simplified gloss for explanatory purposes. A fully glossed version of example (\ref{exintro9}) from this section would look like (\ref{exintro9a}).
\newpage
\ea \label{exintro9a}
    ŋare z zefar bänema nafane kkauna zwarithrth.\\
	\gll ŋare z ze\stem{far} bäne=ma nafane kkauna zwa\stem{ri}thrth.\\
	woman(\gl{abs}) already \gl{2|3sg}:\gl{rpst}:\gl{pfv}{\textbackslash}set\_off \gl{ph}=\gl{char} \gl{3sg.poss} things \gl{2|3pl}>\gl{3sg.f}.\gl{io}:\gl{rpst}:\gl{ipfv}{\textbackslash}give\\
	\glt `The woman has left already, because they gave back her belongings to her.'\\
\z

\subsection{Orthographic conventions}\label{sec:orthography}

There is no writing tradition in Komnzo, but most people can read and write in one of the official languages, namely English and Motu. The mission school, which was based at \textit{Rouku} during the 1960s, operated in Motu, but today English is the teaching language at the school in \textit{Morehead}. Thus, reading and writing in Komnzo has not been promoted in the past. As a consequence, literacy in one's mother tongue is an alien concept for most Komnzo speakers. Writing Komnzo words is often limited to proper nouns.

The first attempt to develop an orthography for Komnzo took place at an alphabet workshop organised by Marco and Alma Bouvé of the Summer Institute of Linguistics (SIL) at \textit{Morehead Station} in 2000. Representatives from several villages took part in this workshop, and the result was a number of different orthographies. The orthography developed for Komnzo was not in use at the time I began my work in 2010. Together with the Komnzo Language Committee, the spelling was revised several times. The texts in this book make use of the graphemes shown in \tabref{orthogcons} and \figref{orthogvow}. 

\begin{table}[h]
\caption{Consonant graphemes}
\label{orthogcons}\small
	\begin{tabularx}{\textwidth}{p{2cm}XXXXXXX}
		\lsptoprule
		& {bilabial}& {dental} & {alveolar} & {palato-alveolar}	& {palatal} & {velar} & {labio-velar} \\ \midrule
		{plosive} \&&&&t&&&k&kw\\
		{affricate}&&&&z&&&\\
		&&&&&&&\\
		{prenasalised} &&&&&&&\\
		{plosive \&}&b&&d&&&g&gw\\
        {affricate} &&&&nz&&&\\
        &&&&&&&\\
		{fricative} & f	& th & s &&&&\\
		&&&&&&&\\
		{nasal} & m && n &&& ŋ & \\
		&&&&&&&\\
		{trill/tap} &&& r &&&&\\
		&&&&&&&\\
		{semivowel} &&&&&y && w\\
		\lspbottomrule
	\end{tabularx}
\end{table}%

\begin{figure}
\centering
{
\begin{vowel}[plain]
	\putvowel{i}{0,3\vowelhunit}{0,4\vowelvunit}
	\putvowel{ü}{1,3\vowelhunit}{0,4\vowelvunit}
	\putvowel{e}{0,85\vowelhunit}{1,5\vowelvunit}
	\putvowel{ö}{1,7\vowelhunit}{1,5\vowelvunit}
	\putvowel{ä}{1,6\vowelhunit}{2,55\vowelvunit}
	\putvowel{u}{4\vowelhunit}{0,4\vowelvunit}
	\putvowel{a}{2,9\vowelhunit}{2,7\vowelvunit}
	\putvowel{é}{2,7\vowelhunit}{1,3\vowelvunit}
	\putvowel{o}{4\vowelhunit}{1,5\vowelvunit}
\end{vowel}
}%
\caption{Vowel graphemes} \label{orthogvow}
\end{figure}

Note that the epenthetic vowel [ə̆] is not written in the orthography because it is entirely predictable from the syllable structure (cf. \cite[67ff.]{Dohler:2018qt}). There are in fact many roots which lack specified vowels altogether, for example \emph{mnz} [mə̆\textsuperscript{n}ts] `house', \emph{zfth} [tsə̆ɸə̆θ] `tree base', and \emph{ggrb} [{ᵑ}gə̆{ᵑ}gə̆ɾə̆\textsuperscript{m}p] `small, unripe coconut'. Only in word final position, the epenthetic vowel will be written as <é>. The quality of the epenthetic vowel shows only little variation. In almost all environments, it is realised as a mid central vowel of very short duration [ə̆].

The spelling of proper nouns is adapted to a more English-like orthography in the translations. I follow here some of the spelling conventions that Komnzo speakers have adopted, e.g. spelling pre-nasalised consonants with a digraph, and writing the epenthetic vowel. Examples are the personal name \textit{karbu} [karə̆ᵐbu], which is spelled ``Karémbu'' in the English translation, and the place name \textit{mnzärfr} [mə̆ntsær ɸə̆r], which is spelled ``Ménzärfér''.

Punctuation in the interlinearised text is kept to a minimum. Only exclamation marks, question marks and quotation marks are used. In the parallel text version, where there is continuous text in Komnzo, I use full stops and sometimes commas. These punctuation marks signalise syntactic elements such as sentences, clauses and postposed elements. In addition, each sentence is introduced with a capital letter. For the definition of these elements, I refer the reader to the grammar (\cite{Dohler:2018qt}).

\newpage
\section{Provenance}\label{sec:data}

The provenance and metadata of the individual stories is given in the introductory text for each text. The documentation of Komnzo was part of my dissertation project at the Australian National University, Canberra. The project was funded by the \textsc{dobes} initiative of the Volkswagen Foundation.\footnote{Grant number: 85606, Title: ``Nen and Tonda: Two languages of Southern Papua New Guinea'' (Komnzo belongs to the ``Tonda'' subgroup of the Yam family.)}

\subsection{Ethics}

This project began with a pilot fieldtrip to the Morehead district in September of 2010. At the time, my goal was to establish contact to a community that speaks one of the Tonda languages. I did not know which village or variety I was going to work on when I first arrived in the provincial capital \textit{Daru}. It was Abia Bai from \textit{Rouku} who invited me to accompany him to his natal village, where I received a warm and friendly welcome from the community. I explained my intentions and people agreed that I may return regularly. I stayed for eight weeks, and was eventually adopted in the Mrzar Mayawa clan. My village name is \textit{Bäi} [ᵐbæi].

The community set up a language committee with representatives from all clans in \textit{Rouku} to oversee the documentation project. Some members remained passive representatives, while others took on more active roles, such as suggesting topics and stories to be recorded, checking my dictionary, helping me with elicitation, showing me plants and animals, inviting me to various feasts, and visiting places with me. The committee consisted of the following individuals: Abraham Maembu, Albert Maembu, Anau Weni, Caspar Mokai, Daure Kaumb, Kalés Tawéth, Kaumb Bai, Maembu Kwozi, Mai Karémbu, Marua Bai†, Mokai Orot, Railey Abia, Sémoi Weni†, Steven Karémbu, Turaki Damaya, and Wermang Maembu. The two people I worked with most of the time were my adoptive father Abia Bai and his daughter Nakre. In particular, Nakre helped me with the transcription and translation of all the texts in this collection.

On my first fieldtrip, I took with me a written consent form as was required by the ethics committee at ANU. This did not work well for people who do not regularly come into contact with bureaucratic processes and who do not have to deal with forms that they have to sign. Moreover, official paperwork is hardly a substitute for a personal conversation. In my case, it worked much better to explain myself in casual conversation, and to ask later if people agree to being recorded. I have only made and archived recordings in which the narrators were clear about the further terms of use.

\subsection{Archived material}

The original recordings of this text collection are archived at different places. All audio-visual footage collected during the Komnzo documentation project can be accessed at The Language Archive (TLA) under the following link: \url{https://hdl.handle.net/1839/a4d3a01c-0705-4583-8fb7-2fb479fe4e11}.

A second storage location for Komnzo recordings is the Zenodo data repository. Only a part of the original recordings is stored here, namely the Komnzo text corpus, i.e. the transcribed, translated and glossed recordings. This is available in the so-called ``Komnzo community''\footnote{Communities at Zenodo are topical collections. The Komnzo community also includes: the grammar, the dictionary, this text collection, scientific articles, specific datasets.} under the following link: \url{https://zenodo.org/communities/komnzo}. While the footage files are stored in separate records at Zenodo, the corresponding annotation files (\textsuperscript{$\ast$}.eaf) are available as a zip-file in a dedicated record under the following link: \url{https://doi.org/10.5281/zenodo.1306246}, which is regularly updated.\footnote{The titles of the respective records for footage files and the file name of annotation files are labelled with the same source code, e.g.  tci20100905.pdf (scan of the notebook), tci20100905.eaf (transcription), tci20100905.wav (audio), tci20100905.mpg (video).} 

In the next section, I describe the editing process. The edited annotation files and the comma-separated files (\textsuperscript{$\ast$}.csv) generated from them can be found under this link: \url{https://zenodo.org/records/14267763}.

\subsection{Editing decisions}\label{subsec:editing}

The texts in this collection are based on recordings that were made during my visits to \textit{Rouku} between 2010 and 2017. With the help of various Komnzo teachers, especially Nakre Abia and her father Abia Bai, I have transcribed and translated these recordings. For this publication, I could not return to \textit{Rouku}, which means that I was not able to consult with any of the original narrators or with my language teachers for the editing process. The editing choices are based on (1) my own understanding of the language, (2) comments made by my collaborators during the transcription which I noted down, (3) general comments made by my language teachers. As a consequence, the changes that I have made to the transcriptions are kept to a minimum. They involve standard procedures like removing some disfluencies such as false starts (\ref{ex:abia2-orig}), self-corrections (\ref{ex:maraga-orig}), and pauses (\ref{ex:abia-orig}) in order to make the texts better to read. I decided not to remove fillers such as the placeholder pro-form \textit{bäne/baf} or the manner demonstrative \textit{nima} since these are conventionalised lexical items (\cite{Dohler:0wb}).\footnote{These two items are glossed as \textit{nima} [like\_this] and \textit{bäne/baf} [\gl{ph}] in the texts.}

\ea 
    \label{ex:abia2-orig}
    tfrisam \textbf{-/ŋawa/-} ŋawathknwa.\\
    \glt `He packed up at Téférisam.' \exsource{tci20110802 ABB 106}
\z

\ea 
    \label{ex:maraga-orig}
    nafane yf \textbf{zafgo} ... \textbf{zafo} Akrimogoma emoth.\\
    \glt `Her name was Zafgo ... Zafo, a sister from Akrimongo.' \exsource{tci20111107-01 MAK 92}
\z

In many cases, the deletion of pauses has led to a rearrangement of units that correspond more to a sentence structure. For example, in the \textit{Kukufia} story (\textref{text:kukufia}), the narrator made several pauses which were removed. Consequently, the three annotation units in the transcription file in (\ref{ex:abia-orig-1}-c) were merged into one line of text in (\ref{ex:abia-orig-4}′).

\ea 
    \label{ex:abia-orig}
    \ea
        \label{ex:abia-orig-1}
        nä kayé ...\\
        \glt `One day ...' \exsource{tci20100905 ABB 25}
    \ex
        \label{ex:abia-orig-2}
        kukufia zenfara ...\\
        \glt `Kukufia set off ...' \exsource{tci20100905 ABB 26}
    \ex
        \label{ex:abia-orig-3}
        kofär.\\
        \glt `to go fishing.' \exsource{tci20100905 ABB 27}        
    \z
\exp{ex:abia-orig}\label{ex:abia-orig-4}
    nä kayé kukufia zenfara kofär.\\
    \glt `One day, Kukufia set off to go fishing.' \exsource{\textref{text:kukufia}, line \ref{ex:9:a2238}}
\z 
    
The opposite scenario is also possible, i.e., narrators often produced postposed elements (e.g. a clause, noun phrase, inflected verb) that is intonationally part of the following unit, but belongs with the former unit in terms of syntax and semantics. In these cases, I have rearranged the annotation units to better reflect a kind of sentence notion. Below is an example of such rearrangement. The verb \textit{kwakwirwrmth} `they ran away' belongs to the second intonation unit (in \ref{ex:kaumb-orig}), but it is part of the topic construction in the first unit (\ref{ex:kaumb-orig}').

\ea \label{ex:kaumb-orig}
    \ea \label{ex:kaumb-orig-1}
        yaw kabe wathr mane enrära\\
        \glt `As for the dancers'
    \ex \label{ex:kaumb-orig-2}
        kwakwirwrmth ruga sagathifath dagon ra bramöwä egathikwath.\\
        \glt `they ran away. They forgot about the pig, the food and everything else.' \exsource{tci20120909-06 KAB 72-73}
    \z
    \exp{ex:kaumb-orig}\label{ex:kaumb-orig-4}
        \ea \label{ex:kaumb-1}
            yaw kabe wathr mane enrära kwakwirwrmth\\
            \glt `As for the dancers, they ran away.'
        \ex \label{ex:kaumb-2}
            ruga sagathifath dagon ra bramöwä egathikwath.\\
            \glt `They forgot about the pig, the food and everything else.' \exsource{\textref{text:masis}, lines \ref{ex:11:a2548}--\ref{ex:11:a2549}}
    \z
\z

Some texts are excerpts from a longer recording. For example, the text \textit{Ausi} (\textref{text:ausi}) comes from a recording session that includes a number of different stories. Such information about excerpts or possible thematic links between the texts are given in the introduction to each text.

Meta-comments have not been removed from the texts. These include personal introductions, afterword comments, asides that respond to something that was going on during the recording session, or comments directed to someone in the audience, or to myself. A typical example is given in (\ref{ex:maraga2-orig}), where the narrator addresses the linguist directly (\ref{ex:maraga2-orig1}). After a comment by his wife, he continues the narration (\ref{ex:maraga2-orig2}). There are also cases in which the linguist prompts the narrator by asking a clarification question or making a comment. Such situational information is given in the footnotes.

\ea \label{ex:maraga2-orig}
    \ea\label{ex:maraga2-orig1}
        nä kayé fthé boba gnyako nima kwa ymarwr ane kafar wämne.\\
        \glt `If you walk there one day, you will see that big tree.'
    \ex\setcounter{xnumi}{0}\label{ex:maraga2-orig2}
        z niyakako. zba mothfa mane ykogr füni.\\
        \glt `Ah, you’ve already been there? The \textit{füni} tree is right by the road.' \exsource{\textref{text:nzurna-maraga}, lines \ref{ex:7:a4767}--\ref{ex:7:a4769}}
    \z
\z

Insertions from other languages are kept in the texts. These include code-switches, usually into English, or ad-hoc borrowings. Some examples are words like \textit{okay}, numbers (\textit{15}, \textit{fifteen}), place names (\textit{Daru}, \textit{Australia}). In such cases, the orthography is not adapted. Loanwords on the other hand are phonologically integrated and, thus, do not count as insertions. Loanwords are adapted to the Komnzo orthography (eg. \textit{bicycle} > \textit{basikol}, \textit{hospital} > \textit{ospitor}).

There a few code-switches for narrative purposes, for example in direct speech by someone from another place who speaks another language. This usually involves related languages or dialects, for example the code-switch to Wära in the two headhunting stories (in Texts \ref{text:ebarzanfirran} and \ref{text:fawbrigsi}). Inserted material is not segmented or glossed in the interlinearised text.

In the English translations of the texts, I make some orthographic changes to proper nouns and place names to make them more readable. As mentioned earlier, there is no tradition of writing Komnzo words. Words that are commonly written are personal names and place names. Examples of such adaptations are the personal name \textit{thbithé} [ðəᵐbiðə] which is spelled ``Thémbithé'' and the place name \textit{karesa zfth} [karesa tsəɸəð] which is spelled ``Karesa Zéféth'' in the English translation. In the text line, I use the Komnzo orthography.

\section{Genres and narrative style}\label{sec:genres}

Komnzo speakers have terms for only some of the text genres included in this text collection. The names of genres reflect content rather than narrative structure, performance style or other conventions. Narratives are described as \textit{ebar trikasi} `headhunting stories' (Texts \ref{text:ebarzanfirran} and \ref{text:fawbrigsi}), \textit{menz trikasi} `ancestor stories' (Texts \ref{text:kwafar} and \ref{text:masu}),\footnote{The word \textit{menz} describes a mythical being from the creation time. Such beings reside at specific places (\textit{menz kar}), and they always have a story that goes with the place. Therefore, I translate the word as `ancestor', `story man', `mythical being'.} \textit{nzürna trikasi} `spirit stories' (Texts \ref{text:nzurna-marua}, \ref{text:nzurna-maraga}, and \ref{text:nzurna-kurai}).\footnote{The word \textit{nzürna} describes a shapeshifter, usually an old woman, who lives inside a particular tree. Nzürnas roam the forest and try to trick and kill unsuspecting villagers.} The only exception is the genre \textit{se zokwasi} `bark torch speech', a type of public speech given during an all night feast that is unspecified for its content (\cite{Dohler:2019do}).

The titles of the stories reflect no local naming strategy, nor were the titles given by the narrators. Titles were created during the transcription and translation by myself together with my language teachers, often they are descriptive titles such as \textit{fiyaf trikasi} `hunting story' (Texts \ref{text:fiyaf-lucy} and \ref{text:fiyaf-maembu}).

Komnzo narratives are rarely the kind of staged performances that take place in a typical recording session. In most cases, narratives are told as ``conversational narratives'' that involve a lot of interaction between the narrator and their audience. If it is a conversation between a small number of participants, the main narrator is often interrupted by clarification questions or by side comments. If the audience is larger, there will also be loud emotional reactions and applause. An example of the latter is the story about the introduction of matches (\textref{text:masis}), especially the scene in which people run away in fear of the small matchbox. This interactive dimension of the Komnzo narratives cannot be rendered here in the printed text. However, the interested reader can listen to the original audio recordings.

\section{Narrators}\label{sec:speakers}

In \tabref{tab:narrators}, I present basic information about the narrators who appear in this text collection. As becomes clear from the table, most narrators are from one particular clan, namely the Mrzar Mayawa clan from \textit{Masu}, a small hamlet close to \textit{Rouku} village. There is also a male bias in the selection of narrators. Biases like these are found already in the archived collection, i.e., in the recordings made during the documentation project. They were caused by the circumstances under which I was introduced to and later lived in \textit{Rouku}. Because I was a man, I had more contact with men than with women. Because I was a Mayawa man, I spent more time with other men of the Mayawa clan than with men of other clans.

\begin{table}
\caption{Narrators sorted by age}
\label{tab:narrators}
    \centering
	\begin{tabularx}{\textwidth}{XlrXll}
    \lsptoprule
        \textsc{name} &\textsc{sex} &\textsc{age}\super{a} &\textsc{clan} &\textsc{home} &\textsc{text}\\
	\midrule
		Ruth Abia &\female& 28 & Mrzar Mayawa & Rouku & \ref{text:fenz-nakre}\\
        Daure Kaumb &\male& 35 & Mrzar Mayawa & Rouku & \ref{text:masu}\\
        Maembu Kwozi &\male& 35 & Banibani Mayawa & Rouku & \ref{text:fiyaf-maembu}\\
        Kurai Tawéth &\male& 40 & Nümgar Bangu & Yirko & \ref{text:nzurna-kurai}\\
        Lucy Abia &\female& 55 & Sangara & Yokwa & \ref{text:fiyaf-lucy}, \ref{text:ausi}\\
		Maraga Kwozi &\male& 55 & BaniBani Mayawa & Firra & \ref{text:ebarzanfirran}, \ref{text:nzurna-maraga}\\
        Abia Bai &\male& 60 & Mrzar Mayawa & Masu & \ref{text:kwafar}, \ref{text:safakfaikore}, \ref{text:fawbrigsi}, \ref{text:kukufia}\\
        Kaumb Bai &\male& 65 & Mrzar Mayawa & Masu & \ref{text:masis}\\
        Marua Bai† &\male& 70 & Mrzar Mayawa & Masu & \ref{text:nzurna-marua}\\
    \lspbottomrule
	\multicolumn{5}{l}{\footnotesize \super{a}Estimated age at the time of recording.}
\end{tabularx}
\end{table}
\vspace{-.5cm}
\section{The texts}\label{sec:texts}

The texts in this collection are not grouped according to supposed genres, nor by narrator or time of recording. Instead, they are arranged thematically, as shown in \tabref{tab:texts}. The first three texts are about places of origin and the creation of people. The following two texts are historical reports of two headhunting raids. The next three stories belong to the genre of \textit{nzürna trikasi} `spirit stories'. This is followed by three stories about other supernatural phenomena. The text collection ends with three personal stories: two hunting stories and one life history.

Originally, the texts in this collection were selected to cover a wide range of narrative genres. It was only during the curation phase that I decided to give this book a thematic twist by exploring concepts of landscape, place names, and locality. Some of the texts lend themselves to this thematic analysis. For example, the first text \textit{Kwafar} is a myth of wandering creators (\textref{text:kwafar}) and the second text \textit{Masu} was recorded in an attempt to explore Komnzo place names (\textref{text:masu}). Other texts were selected for different reasons. Some texts are linked such as the three texts belonging to the genre of \textit{Nzürna trikasi} (Texts \ref{text:nzurna-marua}--\ref{text:nzurna-kurai}), or the two stories about headhunting in which one (\textref{text:fawbrigsi}) is the continuation of the other (\textref{text:ebarzanfirran}). As the next chapter shows, all the texts thematise landscape, place names and locality.

\begin{table}
\caption{Texts}
\label{tab:texts}
    \centering
	\begin{tabularx}{\textwidth}{llQr}
    \lsptoprule
        \textsc{text} & \textsc{title} & \textsc{topic} & \textsc{words}\\
	\midrule
		\ref{text:kwafar}&\textit{Kwafar} & Origin myth, wandering creator myth, yam cultivation & 1351\\
        \ref{text:masu}&\textit{Masu} & Myth and history about the home of the Mayawa clan & 690\\
        \ref{text:safakfaikore}&\textit{Safak menz a faikore menz}& Myth about two ancestors&531\\
        \ref{text:ebarzanfirran}&\textit{Ebar zan firran}&Story about a headhunting attack on the village of \textit{Firra}&909\\
        \ref{text:fawbrigsi}&\textit{Faw brigsi}&Story about the revenge attack against the village of \textit{Téndöfi}&780\\
        \ref{text:nzurna-marua}&\textit{Nzürna trikasi watayan}&Spirit story at \textit{Wataya} forest&543\\
        \ref{text:nzurna-maraga}&\textit{Nzürna trikasi firran}&Spirit story at \textit{Firra} village&900\\
        \ref{text:nzurna-kurai}&\textit{Nzürna trikasi bresanema}&Spirit story at \textit{Yirko} hamlet&773\\
        \ref{text:kukufia}&\textit{Kukufia}&Story about two children and the short man Kukufia&441\\
        \ref{text:fenz-nakre}&\textit{Fenz yonasi}&Beliefs about sorcerers&612\\
        \ref{text:masis}&\textit{Masis}&Story about the introduction of matchboxes&619\\
        \ref{text:fiyaf-lucy}&\textit{Fiyaf trikasi}&Short hunting story&320\\
        \ref{text:fiyaf-maembu}&\textit{Fiyaf trikasi}&Story about an all day hunting trip&668\\
        \ref{text:ausi}&\textit{Ausi}&Autobiographical story&1102\\
    \midrule
        \textbf{total}&&&\textbf{10239}\\
    \lspbottomrule
\end{tabularx}
\end{table} 
