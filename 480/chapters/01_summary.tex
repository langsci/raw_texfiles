The text was prompted by my question ``Where did the yams come from?'' and it should be seen as a compendium rather than a single story line. The text can be cut into three story lines, which I have been told independently by others. The first part is the \textit{Kwafar} flood myth, which also appears in \citet{Williams:1936hb} and \citet{Ayres:1983dw}. \textit{Kwafar} is a place off the coast between the island of New Guinea and the Australian continent. 

According to the story, there was a large Wäsi tree at \textit{Kwafar} and the people of different tribes and languages lived together in this tree. Eventually, the tree burned down and the people started spreading out from there. Many clans of the Morehead district have an apical ancestor who came from \textit{Kwafar}. One of the many stories located at \textit{Kwafar} involves two brothers. While hunting in the area, the brothers come across a mysterious creature that devours the bodies of those who died in the fire. The two brothers try to shoot the creature, but only the older brother is successful. As his arrow pierces the creature, a flood of water bursts out of the wound. In recent versions of the myth, the younger brother is said to be white like Europeans. He owns a shotgun instead of a bow. He runs south towards what is now Australia. The older brother runs north. He stops the flood by beating the water with branches of \textit{dödö} (\textit{Melaleuca sp}). 

At this point, Abia transitions into the second part. This is the story of Mathkwi, the apical ancestor of his clan. This story involves many small episodes about the route that Mathkwi took and all the things he carried and brought along.

The third part is about customs and traditions around yam cultivation, which involved a ritual about tasting the first yams of the season. In this part of the story, Abia also talks about a two magic stones that were passed down through the family, but lost in his father's generation.