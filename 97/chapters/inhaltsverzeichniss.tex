\chead{Inhalt.}
\chapter*{Inhalts-Verzeichniss.}
\fed{{\textbar}VIII{\textbar}}\sed{{\textbar}{\textbar}IX{\textbar}{\textbar}}
\begin{inhaltsverzeichniss}
\pdfbookmark[0]{Inhalt.}{inhaltsverzeichniss}
\subsection*{Erstes Buch.}
\subsection*{Allgemeiner Theil.}

\tabcolsep=1mm
\begin{longtable}{b{0.04\linewidth} b{0.05\linewidth} b{0.75\linewidth} p{0.05\linewidth}}
 & & & \textbf{Seite} \\
\contentchap{I.}{Begriff der Sprachwissenschaft.}
\contentnumsec{§. 1.}{Nothwendigkeit der Definition}{I.I.1}
\contentnumsec{§. 2.\newline\newline}{Begriff der menschlichen Sprache: Deutbarkeit, Eindeutigkeit, Absichtlichkeit. Geberdensprache: Hörbarkeit. Sprachen der stimmbegabten Thiere: Gliederung}{I.I.2}
\contentnumsec{§. 3.}{Lautsprache, Articulation (\textsc{Techmer})}{I.I.3}
\contentnumsec{§. 4.}{Der Gedanke, Begriff des Denkens (\textsc{Steinthal}, \textsc{Lotze})}{I.I.4}  

\contentchap{II.}{Aufgaben der Sprachwissenschaft.}
\contentnumsec{§. 1.}{Spracherlernung. – Sprachwissenschaft}{I.II.1}
\contentnumsec{§. 2.}{A. Die Einzelsprachen}{I.II.2}  
\contentnumsec{§. 3.}{B. Sprachgeschichte, Sprachstämme}{I.II.3}
\contentnumsec{§. 4.}{C. Das Sprachvermögen; die allgemeine Sprachwissenschaft}{I.II.4}
\contentsec{\sed{Rückblick}}{I.II.rueckblick}

\contentchap{III.}{Stellung der Sprachwissenschaft.} 
\contentsec{Anthropologie. Ethnographie. Geschichte. Naturwissenschaft. Psychologie; Logik und Metaphysik. – Gegen die Einreihung der Sprachwissenschaft in die Naturwissenschaften}{I.III}

\contentchap{IV.}{Anregungen zur Sprachwissenschaft.}
\contentsec{Verwunderung: Frage nach den Gründen. Zweck des Lernens und Forschens}{I.IV}
\contentsec{Die Ägypter: Zerlegung der Sprache in Laute, Buchstaben}{I.IV.aegypter}

\contentsec{Die Assyrer: Assyrisch-sumerische Grammatiken und Wörterverzeichnisse}{I.IV.assyrer}
\contentsec{Die Chinesen: Bücherverbrennung und philologische Restauration. Sprachphilosophie}{I.IV.chinesen}
\contentsec{Griechen und Römer: Geringschätzung der Barbaren. Sprachphilosophie}{I.IV.grci}
\contentsec{Ihre Epigonen: Alexandria, Byzanz}{I.IV.byzanz}
\contentsec{Das Christenthum}{I.IV.chistenthum}
\contentsec{Der \inlineupdate{Islâm.}{Islâm} Antheil der Perser}{I.IV.islam}
\contentsec{Die Juden: Massoreten und Punctatoren}{I.IV.juden}
\contentsec{Die Pârsî}{I.IV.parsi}
\contentsec{Die Inder: Begabung und Anregungen. Pânini’s Grammatik}{I.IV.inder}
\contentsec{Die Japaner: Anregungen von Aussen; schnelle Wandelungen in der eigenen Sprache. Grammatik}{I.IV.japaner}
\contentsec{Rückblick: Sprachphilosophie, einzelsprachliche Forschung, vorzugsweise in der Muttersprache. Verfall der Sprache und Classicität\hspace*{-0.1mm}}{I.IV.rueckblick}
\multicolumn{2}{l}{\fed{{\textbar}IX{\textbar}} \sed{{\textbar}{\textbar}X{\textbar}{\textbar}}} \\
\contentsec{Neuere Zeit: Humanismus. Missionare und Reisende. Wissenschaftliche Ahnungen, \textsc{Sassetti}, \textsc{Varo}, \edins{Marshman, {\textbar}{\textbar}\textsuperscript{1891 und 1901:}\textsc{Marsmhan,}} \textsc{Premare}, \textsc{Wesdin}, \textsc{Relandus}, \textsc{Sajnovics}, \textsc{Gyarmathy}}{I.IV.neuerezeit}
\contentsec{Die Sanskritstudien: \edins{Engländer, {\textbar}{\textbar}\textsuperscript{1891 und 1901:}\textsc{Engländer,}} \textsc{Schlegel}, \textsc{Bopp}, \textsc{Rask}, \textsc{J. Grimm}, \textsc{Pott}}{I.IV.sanskritstudien}
\contentsec{Die Polyglotten: \textsc{Duret}, \textsc{Leibniz}, das Vocabularium Catharinae, \textsc{Hervas}, \textsc{Adelung} und \textsc{Vater} (Mithridates). \textsc{Fr. Müller}}{I.IV.polyglotten}
\contentsec{\sed{Entzifferung der Keilschriften und Hieroglyphen}}{I.IV.keilschriften}
\contentsec{\textsc{Wilh. von Humboldt}}{I.IV.humboldt}
\contentsec{Die Indogermanistik}{I.IV.indogermanistik}
\contentsec{Verzweigungen und Annäherungen}{I.IV.verzweigung}

\contentchap{V.}{Schulung des Sprachforschers.}
\contentnumsec{§. 1.}{Verschiedene Ausgangspunkte und Richtungen.}{I.V.1}
\contentnumsec{§. 2.\newline\newline}{a) Phonetische Schulung. Werth der Phonetik, Stellung derselben zur Sprachwissenschaft. Lautunterscheidung, Articulation. Übung des Sprach- und Gehörsorgans. Phonetische Schriftsysteme}{I.V.2}
\contentnumsec{ }{\fed{~Zusatz: Schreibung fremder Sprachen.}}{I.V.zusatz}
\contentnumsec{§. 3.\newline\newline\newline\newline\newline}{b) Psychologische Schulung. Denken und Sprechen. Die Sprachen als Weltanschauungen. Wortschöpfungen und Übertragungen. Arten derselben. Beobachtungen an der Muttersprache. Sprachfehler. Einfluss des Volksthums, der Berufsart. Jede Neuerung ursprünglich ein Fehler; was bedingt ihre Annahme oder Ablehnung? Sprache des gemeinen Mannes. Erhaltende Kräfte. Selbstbeobachtung\hspace*{-0.1mm}}{I.V.3}
\contentnumsec{§. 4.\newline}{c) Logische Schulung, praktische und theoretische. Wichtigkeit für den Sprachforscher}{I.V.4}
\contentnumsec{§. 5.\newline\newline\newline\newline}{d) Allgemein sprachwissenschaftliche Schulung. Sprachtalent. Erlernung fremder Sprachen. Anregungen vom Entlegensten her. Wahl der Sprachen. Nothwendigkeit der Praxis. Wichtigkeit der Indogermanistik. Lectüre allgemein sprachwissenschaftlicher Werke\hspace*{-0.1mm}}{I.V.5}
\contentnumsec{ }{~Zusatz: Phantasie und Menschenkenntniss. Geschichte, Völkerkunde: Philosophie, Naturwissenschaften als Nebenstudien}{I.V.zusatz2}
\fed{{\textbar}XII{\textbar}} \\
\end{longtable}

\subsection*{Zweites Buch.}
\subsection*{Die einzelsprachliche Forschung.}

\begin{longtable}{b{0.04\linewidth} b{0.06\linewidth} b{0.74\linewidth} p{0.05\linewidth}}

\contentchap{I.\newline}{Umfang der Einzelsprache. Sprache, Dialekte, Unterdialekte.}
\contentsec{Wieviele Sprachen giebt es auf der Erde? Schwankende Terminologie, fliessende Grenzen. Doppelte Eintheilungsgründe: a) politisch-social, b) nach dem gegenseitigen Verständnisse. Individual- und Volkssprachen. Sprachgemeinschaft: a) mit den Mitlebenden, – b) mit den Vorfahren. Merkmale der Verschiedenheit von Sprachen, Haupt- und Unterdialekten}{II.I}

\contentchap{II.\newline}{Die besondere Aufgabe der einzelsprachlichen Forschung.}
\contentsec{Veränderungen der Sprachen, dialektische Spaltungen. Die Vorgeschichte und das jeweilige Sprachgefühl: Verschiebungen, Lösung alter, Anknüpfung neuer Verbindungen. Die s.~g. isolirten Sprachen}{II.II}

\multicolumn{4}{l}{\fed{{\textbar}X{\textbar}}}\\

\contentchap{III.}{Sprachkenntniss.}
\contentnumsec{1.}{Jede Sprache will erlernt sein}{II.III.erlernt}
\contentnumsec{2.}{Fehlerlose Handhabung der Muttersprache}{II.III.handhabung}
\contentnumsec{3.}{Diese geschieht unbedacht}{II.III.unbedacht}
\multicolumn{4}{l}{\sed{{\textbar}{\textbar}XI{\textbar}{\textbar}}} \\
\contentnumsec{4.\newline}{Aber nach Gesetzen, die unter sich ein organisches System bilden. Stoff und Form, innere Sprachform}{II.III.gesetze}
\contentnumsec{5.}{Gedächtnisserwerb und unbewusste Abstraction}{II.III.gedaechtnisserwerb}

\contentchap{IV.}{Spracherlernung.}

\contentnumsec{§. 1.\newline\newline\newline\newline\newline\newline}{A. Durch mündlichen Umgang. Erlernung der Muttersprache. Frei gebildete Kindersprachen. Einwirkung der Erwachsenen, Aneignung der Regeln. – Verschiedensprachige Menschen: stumme Verständigung; Ablauschen der fremden Sprache; rasche Auffassung und Verständigung bei Ungebildeten. – Missionare; Reisende. Gefahr der Missverständnisse. Methode. – Zwei- und mehrsprachige Erziehung}{II.IV.1}
\contentnumsec{§. 2.\newline\newline}{B. Durch methodischen Unterricht. Lehrer und Lehrbücher. Welche Methode ist die beste? Gefahr des Übersetzungswesens. Neuere Verbesserungen}{II.IV.2}
\contentnumsec{§. 3.\newline}{C. Aus Texten. Bekannte und unbekannte Grössen, Grade der Schwierigkeit. Methode: Naives Verhalten – Collectaneen}{II.IV.3}

\contentchap{V.}{Erforschung der Einzelsprache.}

\contentnumsec{§. 1.}{Die Erkenntniss als Ziel. Was soll erkannt werden?}{II.V.1}
\contentnumsec{§. 2.\newline}{A. Anlegung und Führung der Collectaneen. Abwechselndes cursorisches Lesen. Form der Collectaneen; verschiedene Methoden\hspace*{-0.1mm}}{II.V.2}
\contentnumsec{§. 3.\newline}{B. Prüfung und Ordnung der Collectaneen. Neue Sichtung, Umordnung}{II.V.3}

\contentchap{VI.}{Die Darstellung der Einzelsprache.}
\contentchap{ }{A. Die Grammatik.}
\contentnumsec{§. 1.\newline\newline}{Innere und äussere Sprachform. Sprachbau, Satzbau, – Vollständigkeit und Richtigkeit der Grammatik. Das System; Fehler dagegen. Selbstschilderung}{II.VI.1}  
\contentnumsec{§. 2.\newline}{a) Zeitpunkt zur Selbstprüfung. Selbstbeobachtung bei der Erlernung einer fremden Sprache. Naives Verhalten, Congenialität}{II.VI.2}
\contentnumsec{§. 3.\newline\newline\newline}{b) Bestandtheile des grammatischen Wissens; die beiden Systeme. Die Sprache als zu deutende Erscheinung und als anzuwendendes Mittel. Beides im Geiste durchwoben, in der Darstellung auseinanderzuhalten}{II.VI.3}
\contentnumsec{§. 4.\newline\newline}{c) Die Prolegomena.~ a) Laut- und Accentlehre.~ b) Grundgesetze des Sprachbaues. Die Sprache des Grammatikers und der Leser gleichgültig}{II.VI.4}
\contentnumsec{§. 5.\newline\newline\newline\newline\newline\newline}{d) Das analytische System. – Die Entdeckung der Regeln; die Grammatik in entdeckender Methode. Der analytische Weg: vom Weiteren zum Engeren, und zwar in Rücksicht sowohl auf den Stoff wie auf die Gesetze und Regeln. Kein gemeingülti\-ges Schema möglich. Gleichartiges gehört zusammen. Was ist gleichartig? Beispiel eines analytischen Systems. Gemischte Systeme. Nothwendigkeit, Alles zu beweisen}{II.VI.5}
\contentnumsec{ }{~Zusatz: Beispiel am Arabischen}{II.VI.zusatz}
\fed{{\textbar}XI{\textbar}}\\
\contentnumsec{§. 6.\newline\newline\newline\newline\newline\newline\newline\newline}{e) Das synthetische System. Seine Aufgabe. Unterschied vom analytischen Systeme. Weg von den Theilen zum Ganzen. Andere Gesichtspunkte. Grammatische Synonymik. Beispiele: Was stellt der Sprache ihre Aufgaben? Die psychologische Modalität. Werth des synthetischen Systems: praktischer und theoretischer. Wissenschaftliche Berechtigung dieses Systems. Individuelles Verhalten der Synonymik gegenüber, verwaschene Grenzen und hieraus entstehende Schwierigkeiten. Grundsätze für die Forschung. – Eintheilung des synthetischen Systems}{II.VI.6} 
\contentnumsec{§. 7.\newline}{Zusatz I. Stilistik und Grammatik. Individueller und nationaler Stil}{II.VI.7}
\contentnumsec{§. 8.}{Zusatz II. Die Appendices}{II.VI.8}
\sed{{\textbar}{\textbar}XII{\textbar}{\textbar}} \\
\contentnumsec{§. 9.\newline}{Allgemeines über die Schreibweise und äussere Ausstattung. Berechtigte Klagen. Erleichterungen}{II.VI.9}
\contentnumsec{§. 10.\newline\newline}{Arten von Grammatiken:~ a) Systematische – methodische.~ b) Vollständige Grammatiken – Elementarbücher; grammatische Vorschulen.~ c) Kritische und didaktische Grammatiken}{II.VI.10}
\contentnumsec{§. 11.}{Die grammatische Terminologie}{II.VI.11}
\contentnumsec{§. 12.}{Die Beispiele}{II.VI.12}
\contentnumsec{§. 13.}{Paradigmen und Formeln. Die Anubandhas der Inder}{II.VI.13}
\contentnumsec{§. 14.}{Übungsstücke}{II.VI.14}
\contentnumsec{§. 15.}{Die Sprache des Grammatikers und die darzustellende Sprache}{II.VI.15} 
\contentnumsec{§. 16.\newline\newline\newline\newline\newline\newline\newline\newline}{B. \so{Das Wörterbuch}\newline Als Nachschlagebuch dem Bequemlichkeitszwecke dienend. Möglichkeit eines wisssenschaftlichen Wörterbuchs. Volksthum und Wortschatz. Grenze zwischen Grammatik und Wörterbuch praktisch geboten und wissenschaftlich gerechtfertigt. Idee eines wissenschaftlichen einzelsprachlichen Wörterbuchs:~ I. Wortschatz als Erscheinung:~ a) etymologisch;~ b) morphologisch.~ II. Wortschatz als Ausdrucksmittel: encyklopädische Synonymik. Wieviel ist davon erreichbar und zweckmässig?}{II.VI.16}
\contentnumsec{§. 17.\newline\newline\newline\newline}{C. \so{Berücksichtigung zeitlicher und örtlicher Besonderheiten in Grammatik und Wörterbuch.}\newline Wahrung des einzelsprachlichen Standpunktes. Grenzen des Zulässigen. Die Mode in der Sprache. Akademien zur Regelung der Sprachen. Bühnenmässige Sprache}{II.VI.17}  
\contentnumsec{§. 18.\newline\newline\newline\newline\newline\newline\newline\newline\newline\newline}{D. \so{Sprache und Schrift.}\newline Vorläufer und Ursprung der Schrift. Trieb zu bildnerischem Schaffen, zur Selbstverewigung. Gedächtnisshülfen. Conventionelle Zeichen, Bilder und Symbole. Grenze zwischen der Schrift und ihren Vorläufern: Lesbarkeit, – die Schrift stellt Sprache dar. Stilisirung. Wechselwirkung zwischen Sprache und Schrift. Kunstschriften. Eintheilung der Schriften in Wort-, Sylben- und Buchstabenschriften. – Zwischenstufen. Die Orthographie: historische und phonetische. Neuerungsversuche und ihre Schwierigkeit. Werth der historischen Orthographien für die Sprachwissenschaft. Transscriptionen}{II.VI.18}
\multicolumn{4}{l}{\fed{{\textbar}XII{\textbar}}}\\
\end{longtable}

\subsection*{Drittes Buch.}
\subsection*{Die genealogisch-historische Sprachforschung.}
\begin{longtable}{b{0.04\linewidth} b{0.07\linewidth} b{0.73\linewidth} p{0.05\linewidth}}

\contentteil{\so{Einleitung.}}
\contentsec{Nächste Aufgaben. Aus der Geschichte der vergleichenden Indogermanistik. Nähere Fassung der Aufgabe: im Grunde überall Geschichte einer einzigen Sprache. Unterschied von der einzelsprachlichen Auffassung: nicht räumlich, auch nicht zeitlich, sondern artlich. \sed{Einseitigkeit des historischen Standpunktes. Ursprachen und Urdialekte. Zufälligkeiten in den Standpunkten und Masstäben.} \inlineupdate{Äussere}{Aeussere} und innere Geschichte}{III.I}

\contentteil{\so{Erster Theil. Die äussere Sprachgeschichte. Der Verwandtschaftsnachweis.}}
\contentnumsec{§. 1.\newline\newline\newline\newline}{Aufgaben der Sprachengenealogie. Jetziger Stand unseres Wissens von den Sprachfamilien. Deren Menge. Ob noch grössere Einheiten, ob Ur\sed{{\textbar}{\textbar}XIII{\textbar}{\textbar}}\-einheit aller Sprachen anzunehmen? Möglichkeiten und Übereilungen. Methode, Zufall und Einfall. Bestimmung der Verwandtschaftsgrade}{III.I.I.1}
\contentnumsec{§. 2.}{Entdeckung und Erweiterung der Sprachstämme}{III.I.I.2}
\contentnumsec{ }{~~A. Das Aufsuchen von Anzeichen}{III.I.I.2A}
\contentnumsec{ }{~~~~a) Geographische Momente}{III.I.I.2Aa}
\contentnumsec{ }{~~~~b) Anthropologische Momente}{III.I.I.2Ab} 
\contentnumsec{ }{~~~~c) Ethnographische und culturgeschichtliche Momente}{III.I.I.2Ac}
\contentnumsec{ }{~~~~d) Sprachliche Momente}{III.I.I.2Ad} 
    \contentnumsec{ }{~~~~~~\inlineupdate{aa)}{α)} Ähnlichkeiten im Lautwesen}{III.I.I.2Adalpha}
    \contentnumsec{ }{~~~~~~\inlineupdate{bb)}{β)} Im Sprachbaue}{III.I.I.2Adbeta}
    \contentnumsec{ }{~~~~~~\inlineupdate{cc)}{γ)} In der inneren Sprachform}{III.I.I.2Adgamma}
    \contentnumsec{ }{~~~~~~\inlineupdate{dd)}{δ)} In Wörtern und Lautformen}{III.I.I.2Addelta}

\contentnumsec{ }{\hangindent=0.7cm~~B. Zur Methodik der Sprachenvergleichung. Voreiligkeiten. Die „turanischen Sprachen“. Beweis der Verwandtschaft}{III.I.I.2B}
\contentnumsec{ }{~~~~1. Aufsuchen der ältesten Lautformen}{III.I.I.2B1}
\contentnumsec{ }{~~~~~~a) Die vollere Lautgestalt}{III.I.I.2B1a}
\contentnumsec{ }{~~~~~~b) Spätere Zuwüchse}{III.I.I.2B1b}
\contentnumsec{ }{~~~~~~c) Ursprüngliche Bedeutungen}{III.I.I.2B1c}
\contentnumsec{ }{~~~~2. Prärogativinstanzen}{III.I.I.2B2}
\contentnumsec{ }{~~~~3. Inductive Probe}{III.I.I.3}
\contentnumsec{§. 3.\newline\newline\newline}{Arten und Grade der Verwandtschaft. Voll- und halbbürtige Verwandtschaft, Mischsprachen. Nähe und Ferne: Ähnlichkeit, zumal lexikalische Übereinstimmungen und gemeinsame Neubildungen}{III.I.I.3}
\contentnumsec{§. 4.}{Zusatz I. Zur Anwendung der obigen Lehren}{III.I.I.4}
\contentnumsec{ }{~~I. Die hamito-semitische Sprachfamilie}{III.I.I.4.I}
\contentnumsec{ }{~II. Verwandtschaft des Nahuatl mit den Algonkin-Sprachen}{III.I.I.4.II}
\contentnumsec{§. 5.\newline}{Zusatz II. Stammbaum- und Wellentheorie. \textsc{Schleicher}, \textsc{Johannes Schmidt}}{III.I.I.5}
\contentnumsec{§. 6.\newline}{Zusatz III. Die Sprachen von Kabakada und Neulauenburg, ein Ausnahmefall}{III.I.I.6}
\contentnumsec{§. 7.\newline\newline}{Zur Technik. Collectaneen zum Verwandtschaftsnachweise. Encyklopädisches Wörterbuch. Regelmässige Lautvertretungen. Grammatische Vergleichung}{III.I.I.7}

\multicolumn{4}{l}{\fed{{\textbar}XIII{\textbar}}} \\

\contentteil{\so{Zweiter Theil. Die innere Sprachgeschichte.}}
\contentteil{~~~~Erstes Hauptstück. \so{Allgemeines}.}
\contentnumsec{§. 1.\newline\newline\newline}{Ihre Aufgaben. – Ideale Ziele und bisherige Bestrebungen. Die Indogermanistik und die drei vergleichenden Grammatiken von \textsc{Bopp}, \textsc{Schleicher} und \textsc{Brugmann}. Die indogermanische Ursprache}{III.II.1}
\contentnumsec{ }{~Zusatz: \textsc{Delbrück} über \textsc{Bopp} und \textsc{Schleicher}}{III.II.zusatz}
\contentnumsec{§. 2.\newline\newline\newline}{Alte und neuere Sprachen. – Aus der Geschichte der Indogermanistik. Die „Ursprache“ des Stammes. Ihr Werth vorläufig der einer Formel. Die Gesetze des sprachlichen Werdens als Problem: Werth der lebenden Sprachen neben den todten}{III.II.2}
\contentnumsec{§. 3.\newline\newline}{Die vereinzelte Sprache. – Ausschluss fremder Einflüsse. Isolirte Völker. Die einheimischen Mächte in der Regel stärker als die fremden}{III.II.3}
\contentnumsec{§. 4.\newline\newline\newline}{Die Etymologie. – Analytische Methode. Etymologische Wörterbücher. Begriff der Etymologie. Herkunft der Wörter und grammatischen Formen; Wurzeln. Weitgehende Bestrebungen und Skepticismus. Wichtigkeit der Etymologie}{III.II.4}
\contentteil{~~~~Zweites Hauptstück. \so{Die sprachgeschichtlichen Mächte.}}

\contentnumsec{§. 1.\newline\newline\newline\newline\newline}{Deutlichkeit und Bequemlichkeit. – Voraussetzung der Deutlichkeit. Be\sed{{\textbar}{\textbar}XIV{\textbar}{\textbar}}\-quemlichkeit des Gewohnten; Streben nach weiterer Kraftersparniss, körperlicher und geistiger. Erhaltung, Zerstörung, Neuschöpfung. Anstrengung zum Zwecke der Deutlichkeit, Einfluss auf das Lautwesen. Anschaulichkeit und Eindringlichkeit}{III.II.II.1}
\contentnumsec{§. 2.\newline}{Der Lautwandel. – Aus der Geschichte der Indogermanistik. Gesetzlichkeit und Gründe der Unregelmässigkeiten}{III.II.II.2}
\contentnumsec{ }{~~a) Irrige Vergleichungen}{III.II.II.2a}
\contentnumsec{ }{~~b) Verschiedene Laute in der Ursprache}{III.II.II.2b}
\contentnumsec{ }{~~c) Verschiedene Voraussetzungen der Lautentwicklung}{III.II.II.2c}
\contentnumsec{ }{~~d) Falsche Analogie}{III.II.II.2d}
\contentnumsec{ }{~~e) Entlehnungen}{III.II.II.2e}
\contentsec{Das Axiom von der Unverbrüchlichkeit der Lautgesetze. Seine Voraussetzungen. Wieweit berechtigt? Hodegetischer Werth. Unerklärliches}{III.II.II.2axiom}
\contentnumsec{ }{~Zusatz: Beispiele zur Lehre von der Articulation und der Lautverschiebung. 1. Samoanisch. 2.~Batta, Dajak, Malaisch. 3. Australische Sprachen. 4. Amerikanische Sprachen: Pima, Hidatsa, Chilenisch}{III.II.II.2zusatz}
\contentnumsec{§. 3\sed{a.}\newline\newline\newline\newline\newline\newline}{Die Euphonik (Sandhi). – Zweck oder Ursache? Lautliche Neigungen der Einzelsprachen und Sprachfamilien. Bequemlichkeit.~ 1. Richtungen der Beeinflussung.~ 2. Was wirkt, und was wird beeinflusst?~ 3. Innerer und äusserer Sandhi.~ 4. Ergebniss. – Physiologisches und psychologisches Moment. Verschiedenes Verhalten der Sprachen. Zetacismus. Unorganische Dentale. Erhaltende Mächte}{III.II.II.3a}
\contentnumsec{\sed{§. 3b.}\newline\newline\newline}{\sed{Bevorzugung und Verwahrlosung in der Articulation. Nachdruck und Flüchtigkeit. Was verflüchtigt sich? Entähnlichungen lautlicher Nebenformen. Geschäftliche Kürzungen, Zahlwörter, Rufnamen. Sprachen mit raschem Lautverschliff}}{III.II.II.3b}
\contentnumsec{§. 4.\newline}{Naturlaute als Ausnahmen von den Lautgesetzen. – Onomatopöien, Kinderlaute, Interjektionen}{III.II.II.3b}
\contentnumsec{§. 5.\newline\newline\newline\newline}{Die Analogie. – Gemeingültigkeit, Vieldeutigkeit und Gefährlichkeit der Sache. Die Analogie bei der Spracherlernung. Verhalten der Indogermanistik. Was verleiht der Analogie Wirkung und Anklang? \fed{{\textbar}XIV{\textbar}} Warum nicht überall Analogie? – Die einzelnen Fälle der Analogiewirkungen}{III.II.II.5}
\contentnumsec{§. 6.\newline}{Die falsche Congruenz. – Umladung der \inlineupdate{Formativa}{Formative} von einem Redetheile auf den anderen}{III.II.II.6}

\contentnumsec{§. 7.\newline\newline}{Das etymologische Bedürfniss. \fed{–} Aufbauen und Zerlegen. Falsche Zerlegungen; Volksetymologien; Übertritt der Wörter aus einer etymologischen Familie in die andere}{III.II.II.7}
\contentnumsec{§. 8.\newline\newline\newline\newline\newline\newline}{Das lautsymbolische Gefühl. – Naives Verhalten zur Muttersprache. Wo ähnliche Klänge und ähnliche Vorstellungen zusammentreffen, da verbinden sie sich im Sprachgefühle. Erklärung des Herganges aus der Spracherlernung der Kinder. Wir\-kungen des lautsymbolischen \inlineupdate{Gefühles:}{Gefühles;} Zusammensetzungen und Redensarten; Bedeutungswandel; Neubildungen; unorganischer Lautwandel; Einfluss auf Formenbildung und Syntax?}{III.II.II.8}

\contentnumsec{§. 9.\newline\newline}{Gebundene Rede. – Mechanik bei längerem Sprechen, Pause und neuer Anlauf. Rhythmik. Erhaltung des Veraltenden, Zerstörungen im Laut- und Formenwesen. Betonung der Antithesen}{III.II.II.9}

\multicolumn{4}{l}{\sed{{\textbar}{\textbar}XV{\textbar}{\textbar}}} \\

\contentteil{\so{Bedeutungswandel, Verluste und Neuschöpfungen.}}
\contentnumsec{§. 10.}{Einleitung. – Schwierigkeiten}{III.II.II.10}

\contentnumsec{§. 11.\newline\newline\newline\newline}{Classification der einschlägigen Thatsachen. Die Begriffe und ihre Grenzen. Grenzverschiebungen. Zusammenfliessen und Spaltung. Abschaffung, Verengung, Erhöhung oder Erniedrigung. Neuschöpfungen, Übertragungen, Entlehnungen. Wirkungen des Culturerwerbes und Verkehrs}{III.II.II.11}
\contentnumsec{§. 12.}{Die bewegenden Mächte}{III.II.II.12}
\contentnumsec{ }{~~1. Ähnlichkeit der Vorstellungen, – auch der Laute}{III.II.II.12.1} 
\contentnumsec{ }{~~2. Composition und Construction}{III.II.II.12.2}
\contentnumsec{ }{~~~~a) Das Gleichniss}{III.II.II.12.2a}
\contentnumsec{ }{\hangindent=0.8cm~~~~b) Phraseologische Verbindungen; achtungsvolle und geringschätzige Ausdrücke}{III.II.II.12.2b}
\contentnumsec{ }{~~~~c) Eigentliche Composita}{III.II.II.12.2c}
\contentnumsec{ }{~~~~d) Kürzungen derselben}{III.II.II.12.2d}
\contentnumsec{ }{~~3. Entähnlichung der Bedeutung bei Doubletten}{III.II.II.12.3}
\contentnumsec{ }{\hangindent=0.6cm~~4. Verdeutlichungen und Verstärkungen. – Periphrastische Formen, Diminutiva, Übertreibungen. Composita}{III.II.II.12.4}
\contentnumsec{ }{~~5. Ironie und rhetorische Frage}{III.II.II.12.5}
\contentnumsec{ }{\hangindent=0.6cm~~6. Sitte und Satzung. Ge- und verbotene Ausdrücke; Tabuwesen. Einfluss der \inlineupdate{socialen}{sozialen} Stellung. Keuschheit und Zote. Männer- und Weibersprachen. Die Karaiben, die Kolarier. Aristokratie der Sprache; Entwerthung des Vornehmen}{III.II.II.12.6}  

\contentnumsec{\sed{§. 13a.}\newline}{\sed{Nach- und Neuschöpfungen von Wurzeln \sed{nnd}\edins{{\textbar}{\textbar}und} Wortstämmen}}{III.II.II.13a}
\contentnumsec{§. 13\sed{b.}\newline\newline\newline}{Schwund alter und Entstehen neuer grammatischer Kategorien. – Neue Tempora. Verlust gewisser Tempora und Casus, des Duals. Doppelte Pluralformen. Das Neutrum bei den Neuromanen. Die Kategorie des Belebten im Slavischen}{III.II.II.13b}
\contentteil{\so{Rückblick.}}

\contentnumsec{§. 14.\newline\newline\newline\newline\newline}{Der Spirallauf der Sprachgeschichte, die Agglutinationstheorie. – Älteste Wörter nicht nothwendig einsylbig, nicht nothwendig unveränderlich. Die Afformativa ursprünglich selbständige Wörter. \fed{{\textbar}XVI{\textbar}} Abnutzung der Laute, Deutlichkeitstrieb, daher neuer Ersatz für das Schwindende. Die indogermanischen Sprachen; die indochinesischen: tertiäre Isolation. Der Polysynthetismus}{III.II.II.14}

\contentnumsec{§. 15.\newline}{Hemmende und beschleunigende Kräfte. – Verkehr der verschiedenen \inlineupdate{Alterstufen}{Altersstufen} untereinander}{III.II.II.15}

\contentteil{\so{Einfluss des Verkehres, Sprachmischung.}}
\contentnumsec{§. 16.\newline\newline}{Einleitung. – Verständlichkeit; was sie erfordern und erlauben \inlineupdate{kann.}{kann} Verständigung mit verschiedensprachigen Menschen. Gemischte Bevölkerungen}{III.II.II.16}
\contentnumsec{§. 17.\newline}{Aussterben der Sprachen. – Recht des \inlineupdate{Stärkeren;}{Stärkeren:} Lebenskraft der Völker; Vergewaltigungen. Auswanderer}{III.II.II.17}

\contentnumsec{§. 18.\newline\newline\newline\newline\newline}{Entlehnungen. – Internationaler Verkehr mit Waaren und Begriffen. Fremdwörter und Nachbildungen. Wanderung von Wörtern. Was wird entlehnt? Lehnwörter als Überbleibsel verklungener Sprachen. Merkmale der Fremdlinge. Einbürgerung und Angleichung. Finnisch und Germanisch. Dialektische Doubletten. Namen der Culturpflanzen}{III.II.II.18}
\contentnumsec{§. 19.\newline}{Beeinflussung des Lautwesens durch Nachbar-Sprachen und \mbox{-Dialekte}}{III.II.II.19}

\multicolumn{4}{l}{\sed{{\textbar}{\textbar}XVI{\textbar}{\textbar}}} \\

\contentnumsec{§. 20.\newline\newline\newline\newline}{Entlehnte Redensarten. \sed{Einführung fremder grammatischer und stilistischer Formen.} Annahme fremdsprachlicher Gewohnheiten. Einfluss fremder \inlineupdate{Literaturen:}{Litteraturen:} der chinesischen, der indischen, der arabischen. Die griechisch-römische Prosa, – die französische. Kopten, Äthiopier, Syrjänen}{III.II.II.20}

\contentnumsec{§. 21.\newline\newline\newline\newline\newline}{Sprachmischung innerhalb der Muttersprache. – Die kleinsten Kräfte und Wirkungen. Neuerwerb, Auffrischen, Vergessen. Selbststeigerung der eigenen Gewohnheiten, Annahme fremder. Wirkung mächtiger Individualitäten. Nachwirkung sprachlicher Eindrücke im Traumleben. Abstumpfung des sprachlichen Gewissens}{III.II.II.21}

\contentnumsec{§. 22.\newline\newline}{Einfluss der Kindersprache. – Deren Eigenthümlichkeiten. Nachahmung \inlineupdate{seiten}{seitens} Erwachsener; Nachwirkungen: Diminutiva, Koseformen der Rufnamen}{III.II.II.22}

\contentnumsec{§. 23.\newline\newline}{Eigentliche Mischsprachen. – Unzählige Menge der Möglichkeiten. Creolensprachen. Die Melanesier, Australier und Kolarier. \textsc{Lepsius’} Theorie von den Sprachen Afrikas}{III.II.II.23}

\contentnumsec{§. 24.\newline\newline\newline\newline}{Dialektforschung. – Mikroskopische Arbeit der historischen Sprachforschung. Kleinste Wandelungen. Schärfe des Unterscheidungsvermögens bei engem Gesichtskreis. Doppelformen in Dialekten. Rückschluss auf die Vorzeit. Wissenschaftlicher Werth der Dialektforschung}{III.II.II.24}

\contentnumsec{§. 25.\newline\newline\newline}{Ständesprachen. – Spaltung der Volksclassen, Zuzug von Aussen. Beruf, Denk- und Sprachgewohnheiten. \textit{Slang}, \textit{argot} u.~s.~w., Herkunft der Ausdrücke, Aufnahme derselben in den nationalen Sprachschatz}{III.II.II.25}

\contentnumsec{§. 26.\newline\newline\newline}{Zusatz I. Anregungen zu sprachgeschichtlichen Untersuchungen. Irrlichter. – 1. Japanisch und Mandschu.~ 2. Chinesisch, Koreanisch, Mandschu.~ 3. u. 4. Scheinbare Lautvertretungen; Bedenken.~ 5. Irrlichter auf indogermanischem Gebiete}{III.II.II.26}

\contentnumsec{§. 27.\newline\newline}{Zusatz II. Sprachvergleichung und Urgeschichte. – Probleme. Sprache und Volkstypus. Der gemeinsame Wortschatz als Zeuge vom wirthschaftlichen und geistigen Inventare der Vorfahren\hspace*{-0.1mm}}{III.II.II.27}

\multicolumn{4}{l}{\fed{{\textbar}XVI{\textbar}}} \\

\contentnumsec{§. 28.\newline\newline\newline}{Zusatz III. Die Wurzeln. – Begriff der Wurzel. – Wurzel im einzelsprachlichen Sinne, – im Sinne der Stammes-Ursprache. Apriorische und aposteriorische Wurzeln. Das Problem der Ursprache. Übergang zur allgemeinen Sprachwissenschaft}{III.II.II.28}

\contentnumsec{\sed{§. 29.}}{\sed{Zusatz IV. Laut- und Sachvorstellung}}{III.II.II.29}

\end{longtable}

\subsection*{Viertes Buch.}
\subsection*{Die allgemeine Sprachwissenschaft.}
\begin{longtable}{b{0.04\linewidth} b{0.05\linewidth} b{0.75\linewidth} p{0.05\linewidth}}


\contentteil{\inlineupdate{Capitel I.}{I. Capitel.} \so{Ihre Aufgaben.}}

\contentsec{Grundlagen des Sprachvermögens; Verschiedenheit seiner Entfaltungen; Werthschätzung der Sprache. Urzustand der menschlichen Rede}{IV.I}

\contentteil{\inlineupdate{Capitel II.}{II. Capitel.} \so{Die Grundlagen des menschlichen Sprachvermögens.}}

\contentnumsec{§. 1.}{Allgemeines}{IV.II.1}

\contentnumsec{§. 2.\newline\newline}{Physische Grundlagen. Sprechende Thiere. Die Hand. Die Nahrung. Keine periodisch wiederkehrenden Paarungszeiten. Hülfsbedürftigkeit der Kinder, Familienleben}{IV.II.2}

\contentnumsec{§. 3.\newline\newline\newline\newline\newline\newline}{Psychische Grundlagen. Familienleben und Liebe. Horden; Gemeinsinn und Neid. Spieltrieb. Nachahmungstrieb. Sanguinisches Temperament. \sed{{\textbar}{\textbar}XVII{\textbar}{\textbar}} Eitelkeit. Neugier und Geschwätzigkeit. Gemeinsame Arbeit. Die Laute als ständige Symbole. Neugier und Frage: Analyse. Zank und Lüge. Sprachspielerei. Verschiedenheit der Stimme nach Alter und Geschlecht: conventionelle Laute. Vorzüge der akustischen Mittel vor den optischen}{IV.II.3}

\contentnumsec{§. 4.\newline}{Laute und Töne in der Ursprache. – \inlineupdate{Mannichfaltigkeit.}{Mannigfaltigkeit.} Auch Mehrsylbler. Stilisirung}{IV.II.4}

\contentnumsec{§. 5.\newline\newline}{Die Personificirung, Beseelung und Belebung. – Übertragung des Willens auf Willenloses. Das Widerstrebende. Übertragungen im Ausdrucke. Woher die Vergleiche? Die Etymologie}{IV.II.5}

\contentteil{\inlineupdate{Capitel III.}{III. Capitel.} \so{Inhalt und Form der Rede.}}

 & \multicolumn{3}{l}{~~I. \so{Die Rede.}}\\
 \contentsec{Logische Verknüpfungen. Das Ich und das Du}{IV.III.I}
\contentnumsec{ }{~~1. Mittheilende Rede im engeren Sinne}{IV.III.I.1}
\contentnumsec{ }{~~2. Fragende Rede}{IV.III.I.2}
\contentnumsec{ }{~~3. Gebietende u. s. w. Rede}{IV.III.I.3}
\contentnumsec{ }{~~4. Ausrufende Rede}{IV.III.I.4}
\contentsec{Prüfung dieser Eintheilung. Mittelstufen. Schema. – Arten der ausrufenden Rede}{IV.III.I.5}
\contentnumsec{ }{~~A. Voller Satz}{IV.III.I.A}
\contentnumsec{ }{~~B. Ellipse}{IV.III.I.B}
\contentnumsec{ }{~~C. Vocative u. dgl.}{IV.III.I.C}
\contentnumsec{ }{~~D. Reine Interjectionen}{IV.III.I.D}
\contentnumsec{ }{~~~~a) Nachahmende}{IV.III.I.a}
\contentnumsec{ }{~~~~b) Subjective}{IV.III.I.b}
\contentsec{Eintheilung der Rede in Rücksicht auf die Formung: Schema. – Schema nach Art und Grund der Erregung. – Zwitterformen}{IV.III.I.6}

 & \multicolumn{3}{l}{~~II. \so{Eintheilung der Rede in Stoff und Form.}} \\

\contentnumsec{§. 1.\newline\newline\newline}{A. Der Stoff. – Worin besteht er? Seine Gliederung und Zerlegung. \fed{{\textbar}XVII{\textbar}} Geistiger Standpunkt der Völker: Perspective \inlineupdate{und}{nnd} Horizont; geistiges Auge. Die Beziehungen (Bindemittel) als Stoff, \fed{–} als Form}{IV.III.II.1}

\contentnumsec{§. 2.}{B. Die Form}{IV.III.II.2}
\contentnumsec{§. 3.\newline\newline}{1. Die innere Form. \textsc{Humboldt}, \textsc{Pott}, \textsc{Steinthal}, \textsc{Misteli}, \textsc{Fr. Müller}. – Beurtheilung. Werth der Etymologie. Genetische Erklärung}{IV.III.II.3}
\contentnumsec{§. 4.}{Die äussere Sprachform. Die morphologische Classification}{IV.III.II.4}
\contentnumsec{ }{~~1. Ungeformte Satzwörter}{IV.III.II.4.1}
\contentnumsec{ }{~~2. Häufung derselben}{IV.III.II.4.2}
\contentnumsec{ }{~~3. Isolirung}{IV.III.II.4.3}
\contentnumsec{ }{~~4. Composition}{IV.III.II.4.4}
\contentnumsec{ }{~~5. Hülfswörter}{IV.III.II.4.5}
\contentnumsec{ }{~~6. Agglutination, Prä-, Sub- und Infixe}{IV.III.II.4.6}
\contentnumsec{ }{~~7. Fliessende Grenzen: was befördert die Agglutination?}{IV.III.II.4.7}
\contentnumsec{ }{~~8. Unterabtheilungen der Agglutination}{IV.III.II.4.8}
\contentnumsec{ }{~~~~a) Sub- und Präfixe}{IV.III.II.4.8a}
\contentnumsec{ }{~~~~b) Umfang der Agglutination}{IV.III.II.4.8b}
\contentnumsec{ }{\hangindent=0.8cm~~~~c) Innigkeit der \inlineupdate{Verbindung}{Verbindungen} von Form und Stoffelementen}{IV.III.II.4.8c}
\contentnumsec{ }{~~~~d) Grammatische Functionen}{IV.III.II.4.8d}
\contentnumsec{ }{~~9. Anbildung und Agglutination. Defektivsystem}{IV.III.II.4.9}  
\multicolumn{4}{l}{\sed{{\textbar}{\textbar}XVIII{\textbar}{\textbar}}}\\

\contentnumsec{ }{~~10. Symbolisation}{IV.III.II.4.10}
\contentnumsec{ }{~~11. Fliessende Grenzen}{IV.III.II.4.11}  
\contentnumsec{ }{~~12. Die angeblich flectirende Classe}{IV.III.II.4.12} 
\contentnumsec{ }{\hangindent=0.8cm~~13. Einverleibung, Polysynthetismus. \textsc{Humboldt}. Arten d. Einverleibung}{IV.III.II.4.13}
\contentnumsec{ }{~~14. Die syntaktischen Composita}{IV.III.II.4.14}
\contentnumsec{ }{~~15. Die Erscheinungen der Wortstellung}{IV.III.II.4.15}  

\contentnumsec{§. 5.\newline\newline\newline}{C. Der Formungstrieb. – Scheinbarer und wirklicher Überfluss im sprachlichen Ausdrucke; dessen Ursprung und Abschaffung. Der Zweck der Sprache nicht nur geschäftlich. Synonymformen. Das zu Grunde liegende Bedürfniss. Formungstrieb überall}{IV.III.II.5}
\contentnumsec{ }{~~III. \so{Die Wortstellung.}\newline Psychologisches Subject und Prädicat. – Die Agglutinationen als Zeugen vorgeschichtlicher Stellungsgesetze. Die ungegliederte Rede. Rede und Antwort. Die Ellipse. Häufung einwortiger Äusserungen; logisches Band zwischen solchen. Anfänge zusammenhängender Rede. Unbestimmtheit des Zusammenhanges zwischen den Redegliedern. Die Ordnung frei, aber bedeutsam, zunächst vom Standpunkte des Hörenden, dann aber auch von dem des Redenden aus. Inductiver Beweis; Grundsätze für die Wahl der Beispiele. Isolirung des psychologischen Subjectes. Das Verbum vor dem Subjecte. – Das psychologische Subject als grammatische Kategorie}{IV.III.III}

\contentnumsec{ }{~~IV. \so{Die }\inlineupdate{\so{Betonung.}}{\so{Betouung.}}\newline Herkömmliche Erklärung der Stellungserscheinungen. Wann und was betont man? Lautes Reden. Betonung bestimmter Theile der Rede, immer polemisch, gegensätzlich. Warum so oft das erste Satzglied betont? Bedeutsamkeit der Betonung}{IV.III.IV}

\contentnumsec{ }{~~V. \so{Ausspracheweise oder Stimmungsmimik.}\newline Einwirkung der Stimmung des Redenden auf Laut- und Tonbildung. Verwendung der Modulationen zu Form- und Wortbildung\hspace*{-0.1mm}}{IV.III.V}

\multicolumn{4}{l}{\fed{{\textbar}XVIII{\textbar}}} \\

\contentnumsec{ }{~~VI. \so{Zusammenwirken des Stellungsgesetzes und der Stimmungsmimik.}\newline Beide schon der Ursprache eigen, dieser verhältnissmässige Frische und Mannichfaltigkeit verleihend. Möglichkeit zu fester Gestaltung. Gegensinn (\textsc{C. Abel}), ironische Redeweise. Rohe Sprachen}{IV.III.VI}

\contentnumsec{ }{~~VII. \so{Classification der Wörter nach Begriffskategorien, grammatische Redetheile.}\newline Das Prädicat als Name des Subjectes. Verschiedene Subjecte mit gleichen Prädicaten. Gleiche oder ähnliche Subjecte mit entgegengesetzten Prädicaten. Unterschied zwischen Ding, Eigenschaft und Thätigkeit durch die Betrachtung der Welt gegeben. Denkgewohnheiten. Das Gewohnte wird zur Regel. Verschiedenes Verhalten der Sprachen, wie möglich? Hybride Fälle. Verschiedene Auffassungsweisen. Das verbale Prädicat im Gegensatz zum nominalen. Die ursprünglichen Kategorien}{IV.III.VII}

\contentnumsec{ }{~~VIII. \so{Möglichkeit} – \so{Regel} – \so{Gesetz.}\newline Rückblick: Ursprünglich schrankenlose \inlineupdate{Mannichfaltigkeit.}{Mannigfaltigkeit.} Ob schon grammatisches System? Antriebe zu weiterem grammatischen Ausbaue der Sprache: gesteigertes Geistesleben, – das Gewöhnliche gewinnt die Alleinherrschaft. Die s. g. Naturvölker. Wechselwirkung zwischen Sprache und Volksgeist}{IV.III.VIII}

\multicolumn{4}{l}{\sed{{\textbar}{\textbar}XIX{\textbar}{\textbar}}} \\

\contentteil{\inlineupdate{Capitel IV.}{IV. Capitel.} \so{Sprachwürderung. Gesichtspunkte für die} \inlineupdate{\so{Werthsbestimmung}}{\so{Werthbestimmung}} \so{der Sprachen.}}

\contentnumsec{1.}{Einleitung. – Wechselwirkung zwischen Sprache und Volksgeist}{IV.IV.1}
\contentnumsec{2.\newline\newline}{Grundlagen der Induction. – Culturwerth der Völker. Dualismus zwischen Formsprachen und formlosen Sprachen oder Gradunterschiede? Die Grundlagen der Induction verschoben}{IV.IV.2}
\contentnumsec{3.\newline\newline\newline\newline\newline}{Massstab auf Seiten der Sprachen. – Verschiedenheit der Organe und Functionen. Angebliche Vorzüge auch bei Sprachen niedrig stehender Völker: Congruenz, grammati\-sches Geschlecht, innerer Lautwandel, prädicative Conjugation, die dritte Person in der Conjugation, der Nominativ, Durchdringung von Stoff und Form. Ergebniss}{IV.IV.3}
\contentnumsec{4.}{Geschichtliche Einflüsse}{IV.IV.4}
\contentnumsec{5.\newline}{Werth der \inlineupdate{Etymologie.}{Etymologie –} Gefahr ungerechter Beurtheilung}{IV.IV.5}
\contentnumsec{6.\newline}{Wesen der indogermanischen Flexion. – Das Defectivsystem und die geistige Anlage der Rasse}{IV.IV.6}
\contentnumsec{7.\newline}{Lautgesetze, Sandhi u. s. w. – Anticipationen und Nachwirkungen. Die indogermanischen und die uralaltaischen Sprachen}{IV.IV.7}
\contentnumsec{8.\newline}{Agglutination. – Bedenklichkeit der Etymologien. Stoffliches kann formal werden}{IV.IV.8}
\contentnumsec{9.}{Das etymologische Bewusstsein. – Logische Klarheit}{IV.IV.9}
\contentnumsec{10.\newline\newline\newline}{Missgriffe bei der Analyse und Beurtheilung; störende Factoren. – Die zwischenzeiligen Übersetzungen. Die Terminologie. Fehler der Grammatiken. Gefahr voreiliger Geringschätzung. Mischsprachen\hspace*{-0.1mm}}{IV.IV.10}
\contentnumsec{11.\newline}{Die semitischen Sprachen. – Vocalwandel. Ähnliches in anderen Sprachen}{IV.IV.11}
\contentnumsec{12.\newline\newline}{Malaien und Semiten. – Ähnlichkeiten in der Syntax. Verglei\-\fed{{\textbar}XIX{\textbar}}chung der Völkerstämme. Schlussfolgerungen. Die semitische Wortstammbildung}{IV.IV.12}
\contentnumsec{13.\newline}{Malaien und Uralaltaier. – Gegensätze im Sprachbaue. Vergleichung \inlineupdate{der}{dor} Rassenanlagen}{IV.IV.13}
\contentnumsec{14.}{Die Bantuvölker. Sprachbau und Volkscharakter}{IV.IV.14}
\contentnumsec{15.\newline\newline}{Indianersprachen Amerikas. – Der incorporirend polysynthetische Bau der Sprachen und das Geistesleben der Rasse. Verbale und nominale Auffassungsweise}{IV.IV.15}
\contentnumsec{16.}{Andere Völker und Sprachen. – Die Australier. Die Indochinesen. Die Kaukasusvölker}{IV.IV.16}
\contentnumsec{17.\newline}{\textsc{Byrne}’s Principles of the Structure of Language. – \inlineupdate{\textsc{Humboldt},}{\textsc{Humboldtt},} \textsc{Steinthal} und \textsc{Byrne}}{IV.IV.17}

\contentnumsec{18.\newline\newline\newline\newline\newline}{Einzelerscheinungen und Einzelsprachen. – Weiterbildungen und Abänderungen. Verhalten den Neuerungen gegenüber. Zähigkeit und Trägheit, Empfänglichkeit und Geschmeidigkeit. Gesichtspunkte zur Beurtheilung. Sprach- und Volksgeschichte. Erziehung und Verwahrlosung der Sprachen; Neuschöpfungen und Einbussen}{IV.IV.18}

 & & \multicolumn{2}{l}{\so{Einzelne Factoren:}}\\
\contentnumsec{ }{A. Das Lautwesen}{IV.IV.18A}
\contentnumsec{ }{B. Innere Articulation. – Begriff, Flüchtigkeiten und Kürzungen; besonders scharfe Articulation. Anwendung auf die Beurtheilung der sprachlichen Formen. Einfluss auf die Sprachgeschichte. Rückschluss auf den Volksgeist}{IV.IV.18B}

\multicolumn{4}{l}{\sed{{\textbar}{\textbar}XX{\textbar}{\textbar}}}\\

\contentnumsec{ }{C. Der Sprachbau oder der grammatische Gesichtspunkt. – Die Grammatik und die nationalen Denkgewohnheiten. Objective und subjective Momente}{IV.IV.18C} 

\contentteil{I. \inlineupdate{Die Objectivität}{\so{Die Objectivität.}}}
\contentnumsec{ }{a) In der formellen Eintheilung des Wortschatzes. – Werth der classificirenden Merkmale. Nominale und verbale Prädicate. Schwund der Unterscheidungsmerkmale. Was und wie wird classificirt? (Mafoor, Süd-Andamanisch)}{IV.IV.Ia}
\contentnumsec{ }{b) Wortformen von absoluter Bedeutung. – Im Mafoor. Vergrössernde oder verklei\-nernde, \edins{lobende{\textbar}{\textbar}\textsuperscript{1891 und 1901}lebende} und tadelnde Formen (Italienisch, Spanisch). Zahl, Zeit, Ort. Besonderes und Allgemeines}{IV.IV.Ib}
\contentnumsec{ }{c) Ausdrücke für die Beziehungen der Satzglieder und Sätze untereinander und des Sprechenden zur Rede}{IV.IV.Ic}
\contentnumsec{ }{\hangindent=0.7cm~~α) Im Allgemeinen. – Grammatik und Logik. \inlineupdate{Mannichfaltigkeit}{Mannigfaltigkeit} des Ausdruckes. Analyse eines Beispieles. Schwierigkeiten und Voraussetzungen der Beurtheilung\hspace*{-0.1mm}}{IV.IV.Icalpha}
\contentnumsec{ }{\hangindent=0.7cm~~β) Prädicat und Attribut, Satz und Satztheil. – Die Wortfolge. \mbox{Einseitig} prädicativer und einseitig attributiver Sprachbau. Zwischenstufen und Entwegungen}{IV.IV.Icbeta}
\contentnumsec{ }{\hangindent=0.7cm~~γ) Prädicativattribute, Zwischenprädicate. – Relativsätze. Stellung des Attributes hinter seinem Träger}{IV.IV.Icgamma}
\contentnumsec{ }{\hangindent=0.7cm~~δ) Attributivprädicate, Prädicatsprädicate und secundäre Prä\-dicate. – Fälle im Chinesischen}{IV.IV.Icdelta}
\multicolumn{4}{l}{\fed{{\textbar}XX{\textbar}}}\\
\contentnumsec{ }{\hangindent=0.7cm~~ε) Active und passive Redeweise, Incorporation.~~–~~Einverlei\-bende Form des Satzbaues}{IV.IV.Icepsilon}

\contentnumsec{ }{\hangindent=0.7cm~~ζ) Nominales und verbales Prädicat, prädicative und possessive Conjugation}{IV.IV.Iczeta}

\contentnumsec{ }{\hangindent=0.7cm~~η)~~Casus, Prä- und Postpositionen.~~–~~Die möglichen Beziehungen substantivischer Satztheile zu anderen. Der Objectivcasus als Unterart der adverbialen. Der Anschaulichkeitszweck. Einzelnes:~ I. \inlineupdate{Objectscasus}{Objectcasus} im Chinesischen.~ II. Hülfswörter für die Attributivverhältnisse ebenda.~ III. Adnominale und adverbiale Attribute.~ IV. Bevorzugung der letzteren.~ V. Welcher Art adverbiale Beziehungen in den einzelnen Sprachen?~ VI. Die beiden \inlineupdate{Genitive}{Genetive} bei den Polynesiern}{IV.IV.Iceta}

\contentnumsec{ }{\hangindent=0.7cm~~θ) Verwandlung der Sätze in Satztheile. – Die Aufgabe, ihre Wichtigkeit und Schwierigkeit.~ I. Unvermitteltes Aneinanderreihen kurzer Sätze.~ II. Eintönige Conjunctionen.~ III. Participial- und Gerundialconstructionen.~ IV. Planmässigeres Verfahren:~ a) Zeiten und Modi einander bedingend.~ b) Satzwörter und Quasiwörter:~ α) Zusammengesetzte Verbalnomina.~ β) Quasi-Substantiva, syntaktisch erzeugt.~ γ) Kennzeichnung durch veränderte Wortstellung}{IV.IV.Ictheta}

\contentnumsec{ }{\hangindent=0.7cm~~ι)~~Logische Modalität.~~–~~Schwierigkeit der Beurtheilung.~~Beispiel aus dem Chinesischen}{IV.IV.Iciota}

\multicolumn{4}{l}{\sed{{\textbar}{\textbar}XXI{\textbar}{\textbar}}}\\ 

\contentteil{~~~~II. \so{Die Subjectivität.}}
\contentnumsec{ }{a) Psychologische Modalität. – Mittheilsamkeit, ihre Voraussetzungen und Richtungen}{IV.IV.IIa}

\contentnumsec{ }{b) Die \inlineupdate{sociale}{soziale} Modalität. – Rückschlüsse auf das gesellschaftliche und staatliche Leben}{IV.IV.IIb}

\multicolumn{3}{l}{~~~~\sed{III. \so{Der Stil}}\dotfill} & ~~\pageref{IV.IV.III} \\

\contentteil{\inlineupdate{Capitel V.}{V. Capitel.} \so{Die Sprachschilderung.}}
\contentsec{Bedürfniss derselben. Voraussetzungen und Aufgaben}{IV.V}

\contentteil{\inlineupdate{Capitel VI.}{VI. Capitel.} \so{Die allgemeine Grammatik.}}
\contentsec{Allgemeine Gesichtspunkte. Das Lautwesen. Der Sprachbau. Die grammatischen Erscheinungen der Sprachen, ihre Bedeutung, ihre Regelmässigkeit, ihre Geschichte. Das synthetische System und die Monographien}{IV.VI}

\contentteil{\inlineupdate{Capitel VII.}{VII. Capitel.} \so{Die allgemeine Wortschatzkunde.}}
\contentsec{Ihre Voraussetzungen und Aufgaben. Welt der Vorstellungen. Etymologie. Phraseologie}{IV.VII}

\contentteil{\inlineupdate{Capitel VIII.}{VIII. Capitel.} \so{Schluss.}}
\contentsec{Rückblick auf die Aufgaben der einzelsprachlichen und sprachgeschichtlichen Forschung. Erhaltende und gestaltende Kräfte. Sprach- und Volksleben. Letzte Ziele der allgemeinen Sprachwissenschaft}{IV.VIII}

%\contentsec{\sed{\so{Register}}}

\end{longtable}
\end{inhaltsverzeichniss}