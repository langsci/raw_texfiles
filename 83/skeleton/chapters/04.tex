% CHAPTER 4 VERBS
\chapter{Verbs}\label{ch:verbs}
This chapter covers the verbal system of Southern Yauyos Quechua. Its four sections treat verb stems, verb types, verbal inflection and verbal derivation, in that order.

\section{Verb stems}\label{sec:verbstem}
In Southern Yauyos Quechua, as in other Quechuan languages, verb\index[sub]{verbs} stems always end in a vowel (\phono{yanapa-} ‘help’). Verb stems are bound forms: with the single exception of \phono{haku} ‘let’s go!’ they never appear in isolation. They are subject to both inflectional and derivational processes, both suffixing (\phono{wañu-n}, die-3, ‘they die’; \phono{wañu-chi-n}, die-\lsc{caus}-3, ‘they kill’). The order of inflectional suffixes is fixed; the order of derivational suffixes is highly regular but admits exception. Inflection for person is obligatory (\phono{*qawa-katra-ya} see-\lsc{freq}-\lsc{prog}); derivational processes are optional (\phono{qawa-n} see-3). The different person suffixes are mutually exclusive; different derivational suffixes may attach in series (\phono{qipi-ra-chi-ku-sa} carry-\lsc{urgt}-\lsc{caus}-\lsc{refl}-\lsc{npst} ‘she got herself carried’).

\section{Types of verbs}
Quechua verb stems are usually classed as (di-)transitive (\phono{qu-} ‘give’, \phono{riku-} ‘see’), intransitive (\phono{puñu-} ‘sleep’), or copulative (\phono{ka-} ‘be’). A fourth class can be set apart: onomatopoetic verbs (\phono{chuqchuqya-} ‘nurse, make the sound of a calf nursing’). Special cases include the deictic verb \phono{hina-}, the dummy verb \phono{na-}, and the combining verbs \phono{\pb{-naya-}} ‘give desire’ (§~\ref{ssec:senspsy}) and \phono{-na-} ‘do what, matter, and happen’ (§~\ref{ssec:todona}). §~\ref{sec:transitiveverbs}--\ref{sec:onomatopoeicverbs} cover transitive, intransitive, equational, and onomatopoetic verbs, in turn.

\subsection{Transitive verbs}\label{sec:transitiveverbs}
Transitive verbs\index[sub]{verbs!transitive} are standardly defined for Quechuan languages as those that can take regular-noun direct objects case-marked accusative (\phono{llama-\pb{ta}} \phono{maqa-rqa} ‘They hit the llama’)~(\ref{Glo4:WakKasha}--\ref{Glo4:Vakatall}).\\

% 1
\gloexe{Glo4:WakKasha}{}{lt}%
{Wak Kashapatapiñam maqarura \pb{César Mullidata}.}%lt que first line
{\morglo{wak}{\lsc{dem.d}}\morglo{Kashapata-pi-ña-m}{Kashapata-\lsc{loc}-\lsc{disc}-\lsc{evd}}\morglo{maqa-ru-ra}{beat-\lsc{urgt}-\lsc{pst}}\morglo{César}{César}\morglo{Mullida-ta}{Mullida-\lsc{acc}}}%morpheme+gloss
\glotran{They beat \pb{César Mullida} there in Kashapata.}{}%eng+spa trans
{}{}%rec - time

% 2
\gloexe{Glo4:Asnuqa}{}{sp}%
{Asñuqa nin, “Ñuqa tarisisayki \pb{sugaykita}qa”.}%sp que first line
{\morglo{asnu-qa}{donkey-\lsc{top}}\morglo{ni-n,}{say-\lsc{3}}\morglo{ñuqa}{I}\morglo{tari-si-sayki}{find-\lsc{acmp}-\lsc{1>2.fut}}\morglo{suga-yki-ta-qa}{rope-\lsc{2}-\lsc{acc}-\lsc{top}}}%morpheme+gloss
\glotran{The mule said, “I’m going to help you find your \pb{rope}.”}{}%eng+spa trans
{}{}%rec - time

% 3
\gloexe{Glo4:Maqtakunata}{}{amv}%
{¿\pb{Maqtakunata} pushanki icha \pb{pashñata}?}%amv que first line
{\morglo{maqta-kuna-ta}{young.man-\lsc{pl}-\lsc{acc}}\morglo{pusha-nki}{bring.along-\lsc{2}}\morglo{icha}{or}\morglo{pashña-ta}{girl-\lsc{acc}}}%morpheme+gloss
\glotran{Are you going to take \pb{the boys} or \pb{the girl}?}{}%eng+spa trans
{}{}%rec - time

% 4
\gloexe{Glo4:Vakatall}{}{amv}%
{¡\pb{Vakata} \pb{lliwta} qaquruy! Rikurushaq hanaypim.}%amv que first line
{\morglo{vaka-ta}{cow-\lsc{acc}}\morglo{lliw-ta}{all-\lsc{acc}}\morglo{qaqu-ru-y}{toss.out-\lsc{urgt}-\lsc{imp}}\morglo{ri-ku-ru-shaq}{go-\lsc{refl}-\lsc{urgt}-\lsc{1.fut}}\morglo{hanay-pi-m}{up.hill-\lsc{loc}-\lsc{evd}}}%morpheme+gloss
\glotran{Toss out \pb{the cows}, \pb{all of them}! I’m going to go up hill.}{}%eng+spa trans
{}{}%rec - time

\noindent
In addition to regular transitives, verbs of motion (\phono{lluqsi-} ‘leave’)~(\ref{Glo4:Yakupis}) and impersonal (“weather”) verbs (\phono{riti-} ‘snow’)~(\ref{Glo4:Llaqtayki}),~(\ref{Glo4:Tukuy}) may appear in clauses with regular nouns case-marked \phono{-ta}. In these instances, however, \phono{-ta} does not indicate accusative case.\footnote{An anonymous reviewer points out that the verbs in~(\ref{Glo4:Llaqtayki}) and~(\ref{Glo4:Tukuy}) could be interpreted as transitive (telic) verbs with accusative arguments. \phono{para-}, for example, is interpretable as ‘rain on’ and \phono{pukuta-} as ‘cloud over’, in which case \phono{-ta} in \phono{llaqta-yki-ta} and\phono{-kta} in \phono{llaqta-kta} would have to be interpreted as genuine accusatives.}\\

% 5
\gloexe{Glo4:Yakupis}{}{amv}%
{Yakupis tukuy pampa\pb{ta} rikullaq.}%amv que first line
{\morglo{yaku-pis}{water-\lsc{add}}\morglo{tukuy}{all}\morglo{pampa-ta}{ground-\lsc{acc}}\morglo{ri-ku-lla-q}{go-\lsc{refl}-\lsc{rstr}-\lsc{ag}}}%morpheme+gloss
\glotran{The water used to run all \pb{over} the ground.}{}%eng+spa trans
{}{}%rec - time

% 6
\gloexe{Glo4:Llaqtayki}{}{amv}%
{¿Llaqtayki\pb{ta} paranchu?}%amv que first line
{\morglo{llaqta-yki-ta}{town-\lsc{2}-\lsc{acc}}\morglo{para-n-chu}{rain-\lsc{3}-\lsc{q}}}%morpheme+gloss
\glotran{Does it rain \pb{on} your town?}{}%eng+spa trans
{}{}%rec - time

% 7
\gloexe{Glo4:Tukuy}{}{ch}%
{Tukuy puntraw pukutalunqa llaqta\pb{kta}.}%ch que first line
{\morglo{tukuy}{all}\morglo{puntraw}{day}\morglo{pukuta-lu-nqa}{cloud-\lsc{urgt}-\lsc{3.fut}}\morglo{llaqta-kta}{town-\lsc{acc}}}%morpheme+gloss
\glotran{It’s going to cloud over \pb{on} the town all day.}{}%eng+spa trans
{}{}%rec - time

\subsection{Intransitive verbs}\label{subsec:intransitiveverbs}
Intransitive verbs\index[sub]{verbs!intransitive} are those, like \phono{puñu-} ‘sleep’~(\ref{Glo4:Kamapamnuqa}) and \phono{wiña-} ‘grow’~(\ref{Glo4:Chaypaqa}), that cannot occur in clauses including a regular noun case-marked accusative (\phono{*puñu-ni} \phono{kama-ta} target meaning: ‘I sleep the bed’). Also included among the intransitives are the impersonal weather verbs, like \phono{qasa-} ‘freeze’, which do not take subjects (\phono{qasa-ya-n} ‘it’s freezing’).\footnote{The weather verbs admit only their corresponding weather nouns for subjects. \phono{Para para-ya-n}. ‘The rain is raining.’}\\

% 1
\gloexe{Glo4:Kamapamnuqa}{}{ach}%
{Kamapam ñuqa \pb{puñu}kuya: ishkayni:.}%ach que first line
{\morglo{kama-pa-m}{bed-\lsc{loc}-\lsc{evd}}\morglo{ñuqa}{I}\morglo{puñu-ku-ya-:}{sleep\lsc{refl}-\lsc{prog}-\lsc{1}}\morglo{ishkay-ni-:}{two-\lsc{euph}-\lsc{1}}}%morpheme+gloss
\glotran{We were both \pb{sleep}ing in bed.}{}%eng+spa trans
{}{}%rec - time

% 2
\gloexe{Glo4:Chaypaqa}{}{amv}%
{Chaypaqa \pb{wiña}raptinqa, ¿ayka puntrawnintataq riganchik?}%amv que first line
{\morglo{chay-pa-qa}{\lsc{dem.d}-\lsc{loc}-\lsc{top}}\morglo{wiña-ra-pti-n-qa}{grow-\lsc{unint}-\lsc{subds}-\lsc{3}-\lsc{top}}\morglo{ayka}{how.many}\morglo{puntraw-ni-n-ta-taq}{day-\lsc{euph}-\lsc{3}-\lsc{acc}-\lsc{seq}}\morglo{riga-nchik}{irrigate-\lsc{1pl}}}%morpheme+gloss
\glotran{When it \pb{grow}s, at how many days do we water it?}{}%eng+spa trans
{}{}%rec - time

\noindent
Verbs of motion (\phono{hamu-} ‘come’, \phono{lluqsi-} ‘exit’) form a subclass of intransitive verbs. These often have adverbial complements marked with the directional suffixes \phono{-ta} (accusative), \phono{-man} (allative, dative), \phono{-paq} (ablative) and \phono{-kama} (limitative)~(\ref{Glo4:madrugaw}),~(\ref{Glo4:Hinashpachay}), and they may occur in clauses that include a nominalization with the agentive suffix \phono{-q} indicating the purpose of movement~(\ref{Glo4:Llaman}),~(\ref{Glo4:Kabraykiwan}).\\

% 3
\gloexe{Glo4:madrugaw}{}{amv}%
{Chay huk madrugaw \pb{trinta i unu di abrilta} \pb{lluqsi}run waway.}%amv que first line
{\morglo{chay}{\lsc{dem.d}}\morglo{huk}{one}\morglo{madrugaw}{morning}\morglo{trinta}{thirty}\morglo{i}{and}\morglo{unu}{one}\morglo{di}{of}\morglo{abril-ta}{April-\lsc{acc}}\morglo{lluqsi-ru-n}{go.out-\lsc{urgt}-\lsc{3}}\morglo{wawa-y}{baby-\lsc{1}}}%morpheme+gloss
\glotran{On that morning, \pb{the thirty-first of April}, my son \pb{left} the house [and was kidnapped].}{}%eng+spa trans
{}{}%rec - time

% 4
\gloexe{Glo4:Hinashpachay}{}{ach}%
{Hinashpa chay\pb{paq} wichay\pb{man} \pb{pasa}chisa chay Amador kaq\pb{man}ñataq.}%ach que first line
{\morglo{hinashpa}{then}\morglo{chay-paq}{\lsc{dem.d}-\lsc{abl}}\morglo{wichay-man}{up.hill-\lsc{all}}\morglo{pasa-chi-sa}{pass-\lsc{caus}-\lsc{npst}}\morglo{chay}{\lsc{dem.d}}\morglo{Amador}{Amador}\morglo{ka-q-man-ña-taq}{be-\lsc{ag}-\lsc{all}-\lsc{disc}-\lsc{seq}}}%morpheme+gloss
\glotran{Then, \pb{from} there they made them \pb{march} [\pb{to}] up high \pb{to} Don Amador’s place.}{}%eng+spa trans
{}{}%rec - time

% 5
\gloexe{Glo4:Llaman}{}{amv}%
{Llaman \pb{qutuq} \pb{ri}sa, mayuta pawayashpash saqakarusa.}%amv que first line
{\morglo{llama-n}{llama-\lsc{3}}\morglo{qutu-q}{gather-\lsc{ag}}\morglo{ri-sa}{go-\lsc{npst}}\morglo{mayu-ta}{river-\lsc{acc}}\morglo{pawa-ya-shpa-sh}{jump-\lsc{prog}-\lsc{subis}-\lsc{evr}}\morglo{saqa-ka-ru-sa}{go.down-\lsc{passacc}-\lsc{urgt}-\lsc{npst}}}%morpheme+gloss
\glotran{She \pb{went to gather} her llamas and when she jumped the river, she fell.}{}%eng+spa trans
{}{}%rec - time

% 6
\gloexe{Glo4:Kabraykiwan}{}{amv}%
{Kabraykiwan \pb{qatishiq} hamusa ninkimiki.}%amv que first line
{\morglo{kabra-yki-wan}{goat-\lsc{2}-\lsc{instr}}\morglo{qati-shi-q}{follow-\lsc{acmp}-\lsc{ag}}\morglo{hamu-sa}{come-\lsc{npst}}\morglo{ni-nki-mi-ki}{say-\lsc{2}-\lsc{evd}-\lsc{iki}}}%morpheme+gloss
\glotran{He came \pb{to help bring} your goats, you said.}{}%eng+spa trans
{}{}%rec - time

\subsection{Copulative/equational verbs}\label{ssec:copu}
\SYQ{} counts a single copulative verb\index[sub]{verbs!copulative}, \phono{ka-}. Like the English verb \emph{be}, \phono{ka-} has both copulative (‘I am a llama’)~(\ref{Glo4:fwirti}),~(\ref{Glo4:salvasyunniy}) and existential (‘There are llamas’)~(\ref{Glo4:turu}),~(\ref{Glo4:Rantiqpis}) interpretations.\\

% 1
\gloexe{Glo4:fwirti}{}{amv}%
{Ñuqa-nchik fwirti \pb{kanchik}, patachita, matrkata, trakranchik lluqsiqta mikushpam.}%amv que first line
{\morglo{ñuqa-nchik}{I-\lsc{1pl}}\morglo{fwirti}{strong}\morglo{ka-nchik}{be-\lsc{1pl}}\morglo{patachi-ta}{wheat.soup-\lsc{acc}}\morglo{matrka-ta}{ground.cereal.meal-\lsc{acc}}\morglo{trakra-nchik}{field-\lsc{1pl}}\morglo{lluqsi-q-ta}{come.out-\lsc{ag}-\lsc{acc}}\morglo{miku-shpa-m}{eat-\lsc{subis}-\lsc{evd}}}%morpheme+gloss
\glotran{We \pb{are} strong because we eat what comes out of our fields -- wheat soup and machka.}{}%eng+spa trans
{}{}%rec - time

% 2
\gloexe{Glo4:salvasyunniy}{}{amv}%
{Qammi salvasyunniy \pb{kanki}.}%amv que first line
{\morglo{qam-mi}{you-\lsc{evd}}\morglo{salvasyun-ni-y}{salvation-\lsc{euph}-\lsc{1}}\morglo{ka-nki}{be-\lsc{2}}}%morpheme+gloss
\glotran{You \pb{are} my salvation.}{}%eng+spa trans
{}{}%rec - time

% 3
\gloexe{Glo4:turu}{}{amv}%
{\pb{Kan}ña piña turu.}%amv que first line
{\morglo{ka-n-ña}{be-\lsc{3}-\lsc{disc}}\morglo{piña}{mad}\morglo{turu}{bull}}%morpheme+gloss
\glotran{\pb{There are} mean bulls.}{}%eng+spa trans
{}{}%rec - time

% 4
\gloexe{Glo4:Rantiqpis}{}{amv}%
{Rantiqpis \pb{kan}taqmi.}%amv que first line
{\morglo{ranti-q-pis}{buy-\lsc{ag}-\lsc{add}}\morglo{ka-n-taq-mi}{be-\lsc{3}-\lsc{seq}-\lsc{evd}}}%morpheme+gloss
\glotran{\pb{There are} also buyers.}{}%eng+spa trans
{}{}%rec - time

\noindent
Combined with the progressive, \phono{ya-}, it may but need not have a stative interpretation as well (equivalent to the Spanish \spanish{estar})~(\ref{Glo4:maypahinañatr}),~(\ref{Glo4:sumaqsumaq}).\\

% 5
\gloexe{Glo4:maypahinañatr}{}{amv}%
{¿Cañete, maypahinañatr ka\pb{ya}nchik? Karru, mutu, ¡Asu machu!}%amv que first line
{\morglo{Cañete,}{Cañete}\morglo{may-pa-hina-ña-tr}{where-\lsc{loc}-\lsc{comp}-\lsc{disc}-\lsc{evc}}\morglo{ka-ya-nchik}{be-\lsc{prog}-\lsc{1pl}}\morglo{karru}{bus}\morglo{mutu}{motorcycle}}%morpheme+gloss
\glotran{Cañete, like we \pb{are} where already? Cars, motorcycles -- My Lord!}{}%eng+spa trans
{}{}%rec - time

% 6
\gloexe{Glo4:sumaqsumaq}{}{ach}%
{Qam sumaq sumaq warmim ka\pb{ya}nki.}%ach que first line
{\morglo{qam}{you}\morglo{sumaq}{pretty}\morglo{sumaq}{pretty}\morglo{warmi-m}{woman-\lsc{evd}}\morglo{ka-ya-nki}{be-\lsc{prog}-\lsc{2}}}%morpheme+gloss
\glotran{You \pb{are} a very pretty woman.}{}%eng+spa trans
{}{}%rec - time

\noindent
\phono{ka-} is irregular: the third person singular present tense form, \phono{ka-n}, never appears in equational statements, but only in existential statements~(\ref{Glo4:WiraWira}),~(\ref{Glo4:Llutan}).\footnote{The verbal system includes just two irregularities, the second being that \phono{haku} ‘let’s go’ is never conjugated.}\\

% 7
\gloexe{Glo4:WiraWira}{}{amv}%
{Wira wira\pb{m} matraypi puñushpa, allin pastuta mikushpam.}%amv que first line
{\morglo{wira}{fat}\morglo{wira-m}{fat-\lsc{evd}}\morglo{matray-pi}{cave-\lsc{loc}}\morglo{puñu-shpa}{sleep-\lsc{subis}}\morglo{allin}{good}\morglo{pastu-ta}{pasture.grass-\lsc{acc}}\morglo{miku-shpa-m}{eat-\lsc{subis}-\lsc{evd}}}%morpheme+gloss
\glotran{Sleeping in a cave and eating good pasture, my cow \pb{is} really fat.}{}%eng+spa trans
{}{}%rec - time

% 8
\gloexe{Glo4:Llutan}{}{lt}%
{Llutan\pb{shi}ki.}%lt que first line
{\morglo{llutan-shi-ki}{deformed-\lsc{evr}-\lsc{iki}}}%morpheme+gloss
\glotran{They \pb{are} deformed, they say.}{}%eng+spa trans
{}{}%rec - time

\noindent
In these cases, \phono{ka-\pb{ya}-n} may be employed instead~(\ref{Glo4:Watunqa}),~(\ref{Glo4:Alpakachu}).\\

% 9
\gloexe{Glo4:Watunqa}{}{amv}%
{Watunqa fiyu fiyu wiqam ka\pb{ya}n.}%amv que first line
{\morglo{watu-n-qa}{rope-\lsc{3}-\lsc{top}}\morglo{fiyu}{ugly}\morglo{fiyu}{ugly}\morglo{wiqa-m}{twisted-\lsc{evd}}\morglo{ka-ya-n}{be-\lsc{prog}-\lsc{3}}}%morpheme+gloss
\glotran{Her rope \pb{is} really horrid twisted wool.}{}%eng+spa trans
{}{}%rec - time

% 10
\gloexe{Glo4:Alpakachu}{}{amv}%
{¿Alpakachu wak \pb{kaya}n?}%amv que first line
{\morglo{alpaka-chu}{alpaca-\lsc{q}}\morglo{wak}{\lsc{dem.d}}\morglo{ka-ya-n}{be-\lsc{prog}-\lsc{3}}}%morpheme+gloss
\glotran{\pb{Is that} alpaca [wool]?}{}%eng+spa trans
{}{}%rec - time

\subsection{Onomatopoetic verbs}\label{sec:onomatopoeicverbs}
Onomatopoetic verbs\index[sub]{verbs!onomatopoetic} can be distinguished from other verbs by the shape of their stem. The majority involve the repetition --~two to four times~-- of a syllable or syllable group, most often with the suffixation of \phono{-ya}. Four patterns dominate:\\[1em]

\noindent
Pattern~1: ([C\tss{1}V\tss{1}(C\tss{2})]\tss{S1})[C\tss{1}V\tss{1}(C\tss{2})]\tss{S1}[C\tss{1}V\tss{1}(C\tss{2})]\tss{S1} (\phono{-ya})(\phono{-ku})\\[1em]
Pattern 1 involves the repetition of a single syllable twice or three times, generally with \phono{-ya} or, more rarely, \phono{-ku} or \phono{-ya-ku}, i.e., (S\tss{1})S\tss{1}S\tss{1} (\phono{-ya})(\phono{-ku}).\\[1em]
Example: \phono{qurqurya-} ‘snore’, \phono{luqluqluqya-} ‘boil’.

% Table Onoi
\newcounter{TabOnoi}\setcounter{TabOnoi}{0}%
\newcommand{\TabEntOnoi}[2]{\refstepcounter{TabOnoi}(\theTabOnoi)\label{TabOnoi:#1} &\Qyell{\phono{#1}} & #2\\}%
\begin{table}[!ht]
\small\centering
\caption{Onomatopoetic verbs Pattern~1 examples}
\begin{tabularx}{\textwidth}{clL}
\lsptoprule
\TabEntOnoi{taqtaq-ya-}{knock, make the sound of knocking on wood}
\TabEntOnoi{qurqur-ya-}{snore, make the sound of snoring}
\TabEntOnoi{kurrkurr-ya-}{ribbit (make the sound of a frog)}
\TabEntOnoi{punpun-ya-}{flub-dub, beat (make the sound of the heart)}
\TabEntOnoi{qasqas-ya-}{make the sound of dry leaves}
\TabEntOnoi{katkat-ya-}{tremble, shake (intrans.)}
\TabEntOnoi{chuqchuq-ya-}{nurse, make the sound of an animal nursing}
\TabEntOnoi{pakpak-ya-ku-}{make the sound of a guinea pig}
\TabEntOnoi{qullqullqull-ya-}{gurgle, make the sound of a stomach}
\TabEntOnoi{luqluqluq-ya-}{boil, make the sound of water boiling}
\TabEntOnoi{quququ-ya-ku-}{croak (make the sound of a frog)}
\lspbottomrule
\end{tabularx}
\end{table}

\noindent
Pattern~2: [C\tss{1}V\tss{1}(C\tss{2})]\tss{S1}[C\tss{3}V\tss{1}]\tss{S2}[C\tss{3}V\tss{1}]\tss{S2}[C\tss{3}V\tss{1}]\tss{S2}(\phono{-ya})(\phono{-ku})\\[1em]
Pattern~2, like Pattern~1, involves the repetition of a single syllable generally with \phono{-ya} or, more rarely, \phono{-ku} or
\phono{-ya-ku}. Pattern~2 differs from Pattern~1, however, in that the repeated syllable is (1)~always repeated three times; (2)~never includes a coda; and (3)~is preceded by a non-cognate syllable which generally if not always includes the same vowel as does the repeated syllable, i.e., S\tss{1}S\tss{2}S\tss{2}S\tss{2}(\phono{-ya})(\phono{-ku}).\\[1em]
Example: \phono{bunrururu-} ‘thunder’.

% Table Onoii
\newcounter{TabOnoii}\setcounter{TabOnoii}{0}%
\newcommand{\TabEntOnoii}[2]{\refstepcounter{TabOnoii}(\theTabOnoii)\label{TabOnoii:#1} &\Qyell{\phono{#1}} & #2\\}%
\begin{table}[!ht]
\small\centering
\caption{Onomatopoetic verbs Pattern~2 examples}
\begin{tabularx}{\textwidth}{clL}
\lsptoprule
\TabEntOnoii{taqlalala-}{clang, make the sound of a can knocking against something}
\TabEntOnoii{bunrururu-}{thunder, make the sound of thunder}
\TabEntOnoii{challallalla-}{drip, make the sound of water dripping}
\TabEntOnoii{lapapapa-ya-}{make the sound of a billy goat chasing a female goat}
\lspbottomrule
\end{tabularx}
\end{table}

\noindent
Pattern~3:\\
\resizebox{\textwidth}{1.5ex}{([[C\tss{1}V\tss{1}(C\tss{2})]\tss{S1}[C\tss{1}V\tss{1}(C\tss{2})]\tss{S2}]\tss{U1})[[C\tss{1}V\tss{1}(C\tss{2})]\tss{S1}[C\tss{1}V\tss{1}(C\tss{2})]\tss{S2}]\tss{U1}[[C\tss{1}V\tss{1}(C\tss{2})]\tss{S1}[C\tss{1}V\tss{1}(C\tss{2})]\tss{S2}]\tss{U1}(\phono{-ya})(\phono{-ku})}\\[1em]
Pattern~3 replaces the single syllable of Pattern~1 with a two-syllable unit, \ie,\\
([S\tss{1}S\tss{2}]\tss{U1})[S\tss{1}S\tss{2}]\tss{U1}[S\tss{1}S\tss{2}]\tss{U1}(-ya)(-ku).\\[1em] 
Example: \phono{chiplichipli-} ‘sparkle’.

% Table Onoiii
\newcounter{TabOnoiii}\setcounter{TabOnoiii}{0}%
\newcommand{\TabEntOnoiii}[2]{\refstepcounter{TabOnoiii}(\theTabOnoiii)\label{TabOnoiii:#1} &\Qyell{\phono{#1}} & #2\\}%
\begin{table}[!ht]
\small\centering
\caption{Onomatopoetic verbs Pattern~3 examples}
\begin{tabularx}{\textwidth}{clL}
\lsptoprule
\TabEntOnoiii{chiplichipli-}{shine, sparkle}
\TabEntOnoiii{piiiiichiwpiiiichiw-}{make the sound of a pichusa}
\TabEntOnoiii{iraniraniran-ya-ku-}{moo (make the sound of a cow)}
\TabEntOnoiii{wilwichwilwich-ya-ku-}{make the sound of a pheasant}
\lspbottomrule
\end{tabularx}
\end{table}

\noindent
Pattern~4: Pattern~4, like Patterns~1 and~3, involves the repetition of a single syllable or two-syllable unit two or three times, generally with \phono{-ya} or \phono{-ku}. Pattern~4 differs from Patterns~1 and~3, however, in that the final consonant in the final iteration is eliminated or changed.\\[1em]
Examples: \phono{waqwaqwaya-} ‘guffaw’, \phono{chalaqchalanya-} ‘clang’.

% Table Onoiv
\newcounter{TabOnoiv}\setcounter{TabOnoiv}{0}%
\newcommand{\TabEntOnoiv}[2]{\refstepcounter{TabOnoiv}(\theTabOnoiv)\label{TabOnoiv:#1} &\Qyell{\phono{#1}} & #2\\}%
\begin{table}[!ht]
\small\centering
\caption{Onomatopoetic verbs Pattern~4 examples}
\begin{tabularx}{\textwidth}{clL}
\lsptoprule
\TabEntOnoiv{chalaqchalan/ya-}{clang, make the sound of metal things coming into contact with each other}
\TabEntOnoiv{waqwaqwa-ya-}{laugh heartily, guffaw}
\TabEntOnoiv{chiwachiwa-ya-ku-}{make the sound of a chivillo bird}
\lspbottomrule
\end{tabularx}
\end{table}

There are further, less-common variations. For example, \phono{kurutukutu-} ‘make the sound of a male guinea pig chasing a female guinea pig’ involves the repetition of a three-syllable unit with the elimination of the second syllable in the final iteration.\\

% 1
\gloexe{Glo4:Fwirapapis}{}{ach}%
{Fwirapapis \pb{katkatyakuyanchik}.}%ach que first line
{\morglo{fwira-pa-pis}{outside-\lsc{loc}-\lsc{add}}\morglo{katkatyaku-ya-nchik}{tremble-\lsc{prog}-\lsc{1pl}}}%morpheme+gloss
\glotran{Outside, too, we’re \pb{trembling}.}{}%eng+spa trans
{}{}%rec - time

% 2
\gloexe{Glo4:kaballiriya}{}{amv}%
{Tutaña killapa sumaq sumaq kaballiriya hamukuyasa pampata \pb{chiplichiplishpa}.}%amv que first line
{\morglo{tuta-ña}{night-\lsc{disc}}\morglo{killa-pa}{moon-\lsc{loc}}\morglo{sumaq}{pretty}\morglo{sumaq}{pretty}\morglo{kaballiriya}{horse}\morglo{hamu-ku-ya-sa}{come-\lsc{refl}-\lsc{prog}-\lsc{npst}}\morglo{pampa-ta}{ground-\lsc{acc}}\morglo{chiplichipli-shpa}{sparkle-\lsc{subis}}}%morpheme+gloss
\glotran{At night, under the moon, a beautiful horse was coming across the ground, \pb{sparkling}.}{}%eng+spa trans
{}{}%rec - time

% 3
\gloexe{Glo4:rayu}{}{amv}%
{Unayqa wamaq wamaq rayu kakullaq. “¡Qangran! ¡Qangran!” \pb{taqlaqyaku}q.}%amv que first line
{\morglo{unay-qa}{before-\lsc{top}}\morglo{wamaq}{a.lot}\morglo{wamaq}{a.lot}\morglo{rayu}{thunder}\morglo{ka-ku-lla-q}{be-\lsc{rel}-\lsc{rstr}-\lsc{ag}}\morglo{qangra-n}{growl-\lsc{3}}\morglo{qangra-n}{growl-\lsc{3}}\morglo{taqlaqyaku-q}{rumble-\lsc{ag}}}%morpheme+gloss
\glotran{Before, there was a whole lot of thunder. “Bbrra-boom! Bbrra-boom!” it \pb{rumbled}.}{}%eng+spa trans
{}{}%rec - time

% 4
\gloexe{Glo4:Chitchityakushpa}{}{lt}%
{\pb{Chitchityaku}shpa rikullan kabrakunaqa.}%lt que first line
{\morglo{chitchityaku-shpa}{say.chit.chit-\lsc{subis}}\morglo{ri-ku-lla-n}{go-\lsc{refl}-\lsc{rstr}-\lsc{3}}\morglo{kabra-kuna-qa}{goat-\lsc{pl}-\lsc{top}}}%morpheme+gloss
\glotran{\emph{\pb{Chit-chitting}}, the goats left.}{}%eng+spa trans
{}{}%rec - time

\section{Verb inflection}
\subsection{Summary}\label{sec:verbs summary}
Verbs in \SYQ, as in other Quechuan languages, inflect\index[sub]{verbs!inflection} for person, number, tense, conditionality, imperativity, aspect, and subordination. 

In practice, \SYQ{} counts three persons: first, second, and third (\phono{ñuqa}, \phono{qam}, and \phono{pay}). \SYQ{} verbs inflect for plurality in the first person (\phono{-nchik}); singular and plural suffixes are identical in the second and third persons (\phono{-nki}, \phono{-n}). Although \SYQ{} makes available a three-way distinction between dual, inclusive and exclusive in the first person plural (\phono{ñuqanchik}, \phono{ñuqanchikkuna}, \phono{nuqakuna}), in practice, in all but the \CH{} dialect, the dual form is employed in all three cases; inclusive and exclusive interpretations are supplied by context, both linguistic and extra-linguistic. 

Transitive verbs with non-reflexive first or second person objects inflect for actor-object reference (\phono{-wan}, \phono{-yki}, \etc) Verbal inflection in \SYQ{} marks three tenses, present, past (\phono{-RQa}), and future (portmanteau); the perfect (\phono{-sHa}); the progressive (\phono{-ya}); the present and past conditional (\phono{-man} (\phono{karqa})); and the second person and first person plural imperative (\phono{-y}, \phono{-shun}) and third person injunctive (\phono{-chun}). In practice, \SYQ{} counts two adverbial subordinating suffixes, one employed when the subjects of the main and subordinated clauses are different (\phono{-pti}); the other when they are the identical (\phono{-shpa}). A third subordinating suffix (\phono{-shtin}), also employed when the subjects of the two clauses are identical, is recognized, if not frequently used. Inflectional suffixes (\lsc{ia}) follow derivational suffixes (\lsc{da}), if any are present; derivational suffixes attach to the verb stem (\lsc{vs}). Thus, a \SYQ{} verb is built: \lsc{vs} --~(\lsc{da})~-- \lsc{ia} (see~§~\ref{sec:conord} and~\ref{sec:comple} on constituent order and sentences).

The dialects of \SYQ{} differ in the suffixes they employ in the first person. One set --~\AMV{} and \LT~-- follow the pattern of the \QII{} languages, employing \phono{-ni} to mark the first-person singular nominative and \phono{-wa} to mark the accusative/dative; another set --~\ACH, \SP, \CH~-- follow the \QI{} pattern, employing \phono{-:} (vowel length) for the first-person singular nominative and \phono{-ma} for the accusative dative. The person-number suffixes are: \phono{-ni} or -: (1\lsc{p}), \phono{-nki} (2\lsc{p}), \phono{-n} (3\lsc{p}), and \phono{-nchik} or \phono{-ni}/ \phono{-:} (1\lsc{pl}). \SYQ{} verbs also inflect for actor-object reference. The subject-object suffixes are: \phono{-yki} (1>2), \phono{-wanki} or \phono{-manki} (2>1), \phono{-wan} or \phono{-man} (3>1, \phono{-shunki} (3>2), \phono{-wanchik} or \phono{-manchik} (3>1\lsc{pl}), and \phono{-sHQayki} (1>2.\lsc{\lsc{fut}}). Examples: \phono{ni-nki} ‘you say’; \phono{qawa-yki} ‘I see you’ (see~§~\ref{sec:personandnumber}).

The simple present tense is unspecified for time. It generally indicates temporally unrestricted or habitual action. The simple present tense is indicated by the suffixation of person-number suffixes alone, unaccompanied by any other inflectional markers. Example: \phono{yanu-ni} (\phono{sapa puntraw}) ‘I cook (every day)’ (see~§~\ref{ssec:simplepresent}).

Future suffixes simultaneously indicate person, number and tense. The future suffixes are: \phono{-shaq} (1\lsc{p}), \phono{-nki} (2\lsc{p}), \phono{-nqa} (3\lsc{p}), and \phono{-shun} (1\lsc{pl}). Note that the second person future suffix is identical to the second person simple present suffix. Examples: \phono{chawa-shaq} ‘I will milk’; \phono{pawa-nki} ‘you will jump’; \phono{picha-nqa} ‘they will sweep’ (see~§~\ref{ssec:future}).

The simple past tense alone generally does not receive a completive interpretation; indeed, speakers generally translate it into Spanish with the present perfect. The simple past tense is indicated by the suffix \phono{-RQa}, realized as \phono{-rqa} in \AMV, \phono{-ra} in \ACH, \LT, \SP, and \phono{-la} in \CH. These are immediately followed by person-number suffixes which are identical to the present tense person-number suffixes with the single exception that the third person is realized not as \phono{-n} but as \phono{-\uo}. Examples: \phono{qawa-rqa-ni} ‘I saw’ or ‘I have seen’; \phono{patrya-la-\uo} ‘it/they exploded or ‘it/they has/have exploded’; \phono{hamu-ra-nki} ‘you came’ or ‘you have come’ (see~§~\ref{par:simplepast}).

The quotative simple past tense can be used in story-telling. The quotative simple past is indicated by the suffix \phono{-sHQa}, realized as \phono{-sa} in \ACH, \AMV{} and \SP{} and \phono{-sha} in \CH{} and \LT. It is sometimes realized in all dialects as \phono{-shqa} in the first and sometimes last line of a story. Examples: \phono{nasi-sa-:} ‘I was born’; \phono{ri-shqa} ‘he went’; \phono{hamu-sa-\uo} ‘they came’ (see~§~\ref{par:QSPT}).

Within the morphological paradigm, \phono{-sHa} --~realized as \phono{-sa} in \ACH, \AMV{} and \SP{} and \phono{-sha} in \CH{} and \LT~-- occupies a slot that seems to be reserved for the perfect. Its interpretation, however, is more subtle and it is most often employed as a completive past. \phono{-sHa} is immediately followed by the same person-number suffixes as is simple past (i.e., the third person is realized as \phono{-\uo}). Example: \phono{ri-sa-nki} ‘you have gone’ (see~§~\ref{par:perfect}).

The iterative past is indicated by the combination --~as independent words~-- of the agentive verb form (V\phono{-q}) and --~in the first and second persons~-- the corresponding present tense form of the verb \phono{-ka} ‘to be’. Examples: \phono{ri-q} ‘she used to go’; \phono{ri-q ka-nchik} ‘we used to go’ (see~§~\ref{par:iterative}).

The conditional (also called “potential” or “irrealis”) covers more territory than does the conditional in English. It corresponds to the existential and universal ability, circumstantial, deontic, epistemic, and teleological modals of English. The regular conditional is indicated by the suffix \phono{-man}. \phono{-man} is immediately preceded by person-number suffixes. In the case of the first person singular, the suffixes of the nominal (possessive) paradigm are employed: \phono{-y} in the \AMV{} and \LT{} dialects and \phono{-:} in the \ACH, \CH, and \SP{} dialects. Alternative conditional forms are attested in the second person both singular and plural in the \AMV{} dialect and first person plural in all dialects. \phono{-waq} indicates the second person conditional; \phono{-chuwan}, the first person plural conditional. Both these morphemes simultaneously indicate person and conditionality and are in complementary distribution both with tense and inflectional morphemes. The past conditional is formed by the addition of \phono{ka-RQa} --~the third person simple past tense form of \phono{ka-} ‘be’ to either the regular or alternative present-tense conditional form. Examples: \phono{ri-nki-man} ‘you can go’; \phono{ri-chuwan} ‘we can go’ (see~§~\ref{ssec:conditional}).

Imperative suffixes simultaneously indicate person, number and imperativity. The imperative suffixes are: \phono{-y} (2\lsc{p}) and \phono{-shun} (1\lsc{pl}); the injunctive suffix is \phono{-chun} (1\lsc{pl}). Examples: \phono{¡Ri-y!} ‘Go!’, \phono{¡Ruwa-shun!} ‘Let’s do it!’, and \phono{¡Lluqsi-chun!} ‘Let him leave!’ (see~§~\ref{ssec:impinj}).

Progressive aspect is indicated by the derivational suffix \phono{-ya}. \phono{-ya} precedes\footnote{The derivational affixes \phono{-mu}, \phono{-chi}, and \phono{-ru} may intervene between \phono{-ya} and the inflectional affixes.} person-number suffixes and time suffixes, if any are present are present. Example: \phono{ri-ya-n} ‘she/he/they is/are going’; \phono{ri-ya-ra-\uo} ‘she/he/they was/were going’ (see~§~\ref{ssec:aspect}).

Subordination is not entirely at home with verbal inflection. Subordinating suffixes are different from inflectional suffixes in that, first, they cannot combine with tense, imperativity, or conditionality suffixes, and, second, they are inflected with the person-number suffixes of the nominal paradigm and not those of the verbal paradigm. \SYQ{} makes use of three subordinating suffixes: \phono{-pti}, \phono{-shpa} and \phono{-shtin}: \phono{-pti} is used when the subjects of the main and subordinate clauses are different; \phono{-shpa} and \phono{-shtin}, when the subjects are identical. Cacra, following the pattern of the \QI{} languages, uses \phono{-r} (realized \textipa{[l]}) in place of \phono{-shpa}. \phono{-pti} is generally translated ‘when’, but also occasionally receives the translations ‘if’, ‘because’, or ‘although’. \phono{-shpa} may receive any of these translations, but is most often translated with a gerund. \phono{-shtin} is translated with a gerund exclusively. All three inherit tense, conditionality, and aspect specification from the main-clause verb. \phono{-pti} always inflects for person-number; \phono{-shpa} and \phono{-shtin} never do. Person-number suffixes are those of the nominal paradigm: \phono{-y} or \phono{-:} (1\lsc{p}), \phono{-Yki} (2\lsc{p}), \phono{-n} (3\lsc{p}), and \phono{-nchik} (1\lsc{pl}). Examples: \phono{Hamu-\pb{pti}-ki lluqsi-rqa-\uo} ‘when/because you came, she left’; \phono{Kustumbra-ku-\pb{shpa} hawka-m yatra-ku-nchik} ‘When/if we adjust, we live peacefully’ (see~§~\ref{ssec:subordination}).

Table~\ref{Tab12} summarizes this information. In this and the tables that follow, for reasons of space, unless otherwise specified, all dialects employ the same forms.

\noindent
The following abbreviations and conventions are employed.

\begin{center}
\small
\begin{tabular}{l@{~→~}l}
\lsptoprule
 ‘you’ 				&you.\lsc{s}\lsc{/}you.\lsc{pl}\\
 ‘he’ 				&he/she/it/they\\
 ‘can~\dots{}’ 		&can/could/will/would/shall/should/may/might\\
 ‘could~\dots{}’ 	&could/would/should/might\\
 ‘when~\dots{}’ 	&when/if/because/although/not until or V-ing\\
\lspbottomrule
\end{tabular}
\end{center}

A verb appearing inside angled brackets <like this> indicates a root without tense, conditionality or aspect specified. 

Dialects differ from each other in four sets of cases. They diverge in terms of (1)~their treatment of the first person singular and the first person plural exclusive; (2)~their realization of the simple past tense morpheme \phono{-RQa}; (3)~their realization of the perfect morpheme \phono{-sHa} and (4)~their realization of \textipa{*/r/}. 

Table~\ref{Tab12} displays the differences among the dialects that are relevant to verbal inflection.\\

% TABLE 12
\begin{table}[!ht]
\small\centering
\caption{Verbal inflectional suffixes with different realizations in \SYQ{} dialects}\label{Tab12}
\begin{tabularx}{\textwidth}{p{7ex}LLLL}
\lsptoprule
		&First person singular & Past tense suffix \phono{-RQa} &Perfect \phono{-sHa} &Second-person alternative conditional	\\
\midrule
\AMV{} 	&\phono{-ni} & \phono{-rqa} & \phono{-sa} & yes	\\
\ACH{} 	&\phono{-:} & \phono{-ra} & \phono{-sa} & no		\\
\CH{} 	&\phono{-:} & \phono{-la} & \phono{-sha} & no		\\
\SP{} 	&\phono{-:} & \phono{-ra} & \phono{-sa} & no		\\
\LT{} 	&\phono{-ni} & \phono{-ra} & \phono{-sha} & no		\\
\lspbottomrule
\end{tabularx}
\end{table}

Tables~\ref{Tab13a} and~\ref{Tab13b} give the verbal inflection paradigm of \SYQ. All processes are suffixing, i.e., a verb root precedes all inflectional morphemes. Translations are given as if for the verb \phono{ni-} ‘say.’ Details of form and use as well as extensive examples follow in~§~\ref{sec:personandnumber}--\ref{ssec:subordination}.

% TABLE 13a
\begin{landscape}
\small
\begin{longtable}{@{\hspace{1ex}}p{15ex}@{\hspace{2ex}}l@{\hspace{2ex}}l@{\hspace{2ex}}l@{\hspace{2ex}}l@{\hspace{1ex}}}
\caption{Verbal inflection paradigm}\label{Tab13a}

\\[2ex]
\lsptoprule
Tense 	& 1P & 2P & 3P 	& 1\lsc{pl}	\\
\midrule
\endfirsthead

\multicolumn{5}{c}{\tablename\ \thetable. Continued from previous page} \\
\lsptoprule
Tense 	& 1P & 2P & 3P 	& 1\lsc{pl}	\\
\midrule
\endhead

\lspbottomrule \multicolumn{5}{r}{{\footnotesize Continued on next page~\dots}} \\
\endfoot

\lspbottomrule
\endlastfoot

\multirow{3}{15ex}{Present} 	
& -ni\tss{\AMV,\LT}		& -nki	& -n	& -nchik	\\
\nopagebreak& -:\tss{\ACH,\CH,\SP}& & & 	\\
\nopagebreak& ‘I say’	&‘you say’&‘he says’&‘we say’\\

\cmidrule{2-5}
\multirow{2}{15ex}{Future}
& -shaq		& -nki		& -nqa 		& -shun	\\
\nopagebreak&‘I will say’ 	&‘you will say’ 	& ‘he will say’ 	&‘we will say’\\

\cmidrule{2-5}
\multirow{5}{15ex}{Past}
& -rqa-ni\tss{\AMV}	& -rqa-nki\tss{\AMV}	& -rqa-\uo\tss{\AMV}	& -rqa-nchik\tss{\AMV}\\
\nopagebreak&-ra-ni\tss{\LT}	&-ra-nki\tss{\ACH,\LT,\SP}	&-ra-\uo\tss{\ACH,\LT,\SP}	&-ra-nchik\tss{\ACH,\LT,\SP}\\
\nopagebreak&-ra-:\tss{\ACH,\SP}	&-la-nki\tss{\CH}	&-la-\uo\tss{\CH}	&-la-nchik\tss{\CH}\\
\nopagebreak&-la-:\tss{\CH}			&&&\\
\nopagebreak& ‘I (have) said’ 	&‘you (have) said’ 	&‘he (has) said’ 	&‘we (have) said’\\

\cmidrule{2-5}
\multirow{5}{15ex}{Narrative past}
& -sa-ni\tss{\AMV}	 & -sa-nki\tss{\ACH,\AMV,\SP}	 & -sa-\uo\tss{\ACH,\AMV, \SP}	 & -sa-nchik\tss{\ACH,\AMV,\SP}	\\
\nopagebreak&-sha-ni\tss{\LT}	 & -sha-nki\tss{\CH,\LT}	 & -sha-\uo\tss{\CH,\LT}	&-sha-nchik\tss{\CH,\LT}	\\
\nopagebreak&-sa-:\tss{\ACH,\SP}	 & 	 & 	 & 	\\
\nopagebreak&-sha-:\tss{\CH}	 & 	 & 	 & 	\\
\nopagebreak&‘I have said’ 	 & ‘you have said’ 	 & ‘he has said’ 	 & ‘we have said’\\

\cmidrule{2-5}
\multirow{3}{15ex}{Habitual past}
& -q ka-ni\tss{\AMV,\LT}	 & -q ka-nki	 & -q	 & -q ka-nchik	\\
\nopagebreak& -q ka-:\tss{\ACH,\CH,\SP}	 & 	 & 	 & 	\\
\nopagebreak& ‘I used to say’ 	 & ‘you used to say’ 	 & ‘he used to say’ 	 & ‘we used to say’	\\

\cmidrule{2-5}
\multirow{3}{15ex}{Continuative}
& -ya-ni\tss{\AMV,\LT}	 & -ya-nki	 & -ya-n	 & -ya-nchik	\\
\nopagebreak& -ya-:\tss{\ACH,\CH,\SP}	 & 	 & 	 & 	\\
\nopagebreak& ‘I am saying’ 	 & ‘you are saying’ 	 & ‘he is saying’ 	 & ‘we are saying’\\

\cmidrule{2-5}
\multirow{3}{15ex}{Conditional (potential)}
& -y-man\tss{\AMV,\LT}	 & -nki-man	 & -n-man	 & -nchik-man	\\
\nopagebreak& -:-man\tss{\ACH,\CH,\SP}	 & 	 & 	 & 	\\
\nopagebreak& ‘I can~\dots{} say’ 	 & ‘you can~\dots{} say’ 	 & ‘he can~\dots{} say’ 	 & ‘we can~\dots{} say’\\

\cmidrule{2-5}
\multirow{2}{15ex}{Alternative conditional}
& \ding{53} 	 & -waq\tss{\AMV}	 & \ding{53} 	 & -chuwan	\\
\nopagebreak& 	 & ‘you could~\dots{} say’	 & 	 & ‘we could~\dots{} say’\\

\cmidrule{2-5}
\multirow{5}{15ex}{Past conditional}
& -y-man karqa\tss{\AMV}	 & -nki-man ka-rqa\tss{\AMV}	 & -n-man ka-rqa\tss{\AMV}	 & -nchik-man ka-rqa\tss{\AMV}	\\
\nopagebreak& -y-man ka-ra\tss{\LT}	 & -nki-man ka-ra\tss{\ACH,\LT,\SP}	 & -n-man ka-ra	 & -nchik-man ka-ra\tss{\ACH,\LT,\SP}	\\
\nopagebreak& -:-man ka-ra\tss{\ACH,\SP}	 & -nki-man ka-la\tss{\CH}	 & \ACH, \LT, \SP	 & - nchik-man ka-la\tss{\CH}	\\
\nopagebreak& -:-man ka-la\tss{\CH}	 & 	 & -n-man ka-la\tss{\CH}	 & 	\\
\nopagebreak& ‘I could~\dots{} have said’ 	 & ‘you could~\dots{} have said’ 	 & ‘he could~\dots{} have said’ 	 & ‘we could~\dots{} have said’\\

\cmidrule{2-5}										
\multirow{4}{15ex}{Alternative past conditional}
& 	 & -waq ka-rqa\tss{\AMV}	 & 	 & -chuwan ka-rqa\tss{\AMV}	\\
\nopagebreak& 	 & -waq ka-ra\tss{\LT}	 & 	 & -chuwan ka-ra\tss{\ACH,\SP,\LT}	\\
\nopagebreak& 	 & 	 & 	 & -chuwan ka-la\tss{\CH}	\\
\nopagebreak& \ding{53} 	 & ‘you could~\dots{} have said’ & \ding{53} 	 & ‘we could~\dots{} have said’\\

\cmidrule{2-5}
\multirow{2}{15ex}{Imperative}
& 	 	& -y	 & -chun 	 & -shun	\\
\nopagebreak& \ding{53} 	& ‘Say!’	 & ‘Let him say!’ 	 & ‘Let’s say!’\\

\cmidrule{2-5}
\multirow{3}{15ex}{Subordinator different subjects}
& -pti-y\tss{\AMV,\LT}& -pti-ki & -pti-n & -pti-nchik	\\
\nopagebreak& -pti-:\tss{\ACH,\CH,\SP}&&&		\\
\nopagebreak& when~\dots{} I <say> 	&when~\dots{} you <say> 	& when~\dots{} he <say> & when~\dots{} we <say>	\\

\cmidrule{2-5}
\multirow{2}{15ex}{Subordinator identical subj. 1}
& -shpa	& -shpa	& -shpa	& -shpa	\\
\nopagebreak& ‘when~\dots{} I <say>’ 	& ‘when~\dots{} you <say>’ 	& ‘when~\dots{} he <say>’ 	& ‘when~\dots{} we <say>’	\\

\cmidrule{2-5}
\multirow{2}{15ex}{Subordinator identical subj. 2}
& -shtin	& -shtin	& -shtin		& -shtin	\\
\nopagebreak& ‘saying’ 	& ‘saying’	& ‘saying’	& ‘saying’	\\
\end{longtable}

\bigskip

\begin{longtable}{@{\hspace{1ex}}p{13ex}*{5}{@{\hspace{2ex}}>{\raggedright\arraybackslash}p{19ex}}@{\hspace{1ex}}}
\caption{Verbal inflection paradigm, actor-object suffixes}\label{Tab13b}

\\[2ex]
\lsptoprule
Tense 	& \lsc{2>1} & \lsc{3>1} & \lsc{3>1pl} 	& \lsc{1>2} & \lsc{3>2}	\\
\midrule
\endfirsthead

\multicolumn{6}{c}{\tablename\ \thetable. Continued from previous page} \\
\lsptoprule
Tense 	& \lsc{2>1} & \lsc{3>1} & \lsc{3>1pl} 	& \lsc{1>2} & \lsc{3>2}	\\
\midrule
\endhead

\lspbottomrule \multicolumn{6}{r}{{\footnotesize Continued on next page~\dots}} \\
\endfoot

\lspbottomrule
\endlastfoot

\multirow{3}{13ex}{Present}	
&	-wa-nki\tss{\AMV,\LT}	&	-wa-n\tss{\AMV,\LT}	&	-wa-nchik\tss{\AMV,\LT}	&	-yki	&	-shu-nki	\\
\nopagebreak& -ma-nki\tss{\ACH,\CH,\SP}	&	-ma-n\tss{\ACH,\CH,\SP}	&	-man-chik\tss{\ACH,\CH,\SP}	&	 	&	 	\\
\nopagebreak& ‘you say to me’	&	‘he says to me’	&	‘he says to us’	&	‘I say to you’	&	‘he says to you’	\\

\cmidrule{2-6}
\multirow{3}{13ex}{Future}	
&	-wa-nki\tss{\AMV,\LT}	&	-wa-nga\tss{\AMV,\LT}	&	-wa-shun\tss{\AMV,\LT}	&	-sHQayki	&	-shu-nki	\\
\nopagebreak&	-ma-nki\tss{\ACH,\CH,\SP}	&	-ma-nga\tss{\ACH,\CH,\SP} & -ma-shun\tss{\ACH,\CH,\SP}	&	 	&	 	\\ 
\nopagebreak&	‘you will say to me’	&	‘he will say to me’	&	‘he will say to us’	&	‘I will say to you’	&	‘he will say to you’	\\

\cmidrule{2-6}
\multirow{6}{13ex}{Past}	
&	-wa-rqa-nki\tss{\AMV}	&	-wa-rqa-\uo\tss{\AMV}	&	-wa-rqa-nchik\tss{\AMV}	& -rqa-yki\tss{\AMV}	&	-shu-rqa-nki\tss{\AMV}	\\
\nopagebreak&	-wa-ra-nki\tss{\LT}	&	-wa-ra-\uo\tss{\LT}	&	-wa-ra-nchik\tss{\LT}	&	-ra-yki\tss{\LT, \ACH, \SP}	&	-shu-ra-nki\tss{\LT, \ACH, \SP}	\\
\nopagebreak&	-ma-ra-nki\tss{\ACH,\SP}	&	-ma-ra-\uo\tss{\ACH,\SP}	&	-ma-ra-nchik\tss{\ACH,\SP}	&	 	&	 	\\
\nopagebreak&	-ma-la-nki\tss{\CH}	&	-ma-la-\uo\tss{\CH}	&	-ma-la-nchik\tss{\CH}	&	-la-yki\tss{\CH}	&	-shu-la-nki\tss{\CH}	\\
\nopagebreak& ‘you (have) said to me’	&	‘he (has) said to me’	&	‘he (has) said to us’	&	‘I (have) said to you’	&	‘he (has) said to you’\\

\cmidrule{2-6}
\multirow{6}{13ex}{Narrative past}	
&	-wa-sa-nki\tss{\AMV}	&	-wa-sa-\uo\tss{\AMV}	&	-wa-sa-nchik\tss{\AMV}	&	-sa-yki\tss{\AMV, \ACH, \SP}	&	N/A	\\
\nopagebreak&	-wa-sha-nki\tss{\LT}	&	-wa-sha-\uo\tss{\LT}	&	-wa-sha-nchik\tss{\LT}	&	-sha-yki\tss{\LT, \CH}	&	N/A	\\
\nopagebreak&	-ma-sa-nki\tss{\ACH,\SP}	&	-ma-sa-\uo\tss{\ACH,\SP}	&	-ma-sa-nchik\tss{\ACH,\SP}	&	 	&	 	\\
\nopagebreak&	-ma-sha-nki\tss{\CH}	&	-ma-sha-\uo\tss{\CH}	&	-ma-sha-nchik\tss{\CH}	&	 	&	 	\\
\nopagebreak&	‘you (have) said to me’	&	‘he (has) said to me’	&	‘he (has) said to us’	&	‘I (have) said to you’	&	‘he (has) said to you’\\

\cmidrule{2-6}
\multirow{3}{13ex}{Habitual past}	
&	-wa-q ka-nki\tss{\AMV,\LT}	&	-wa-q\tss{\AMV,\LT}	&	N/A	&	N/A	&	N/A \\
\nopagebreak&	-ma-q ka-nki\tss{\ACH,\CH,\SP}	&	-ma-q\tss{\ACH,\CH,\SP}	&	N/A	&	N/A	&	N/A\\

\cmidrule{2-6}
\multirow{5}{13ex}{Continuous}	
&	-ya-wa-nki\tss{\AMV,\LT}	&	-ya-wa-n\tss{\AMV,\LT}	&	-ya-wa-nchik\tss{\AMV,\LT}	&	-ya-yki\tss	&	-ya-shu-nki	\\
\nopagebreak&	-ya-ma-nki\tss{\ACH,\CH,\SP}	&	-ya-ma-n\tss{\ACH,\CH,\SP}	&	-ya-ma-nchik\tss{\ACH,\CH,\SP}	&	 	&	 	\\
\nopagebreak&	‘you are saying to me’	&	‘he is saying to me’	&	‘he is saying to us’	&	‘I am saying to you’	&	‘he is saying to you’	\\

\cmidrule{2-6}
\multirow{6}{13ex}{Conditional}	
&	-wa-nki-man\tss{\AMV,\LT}	&	-wa-n-man\tss{\AMV,\LT}	&	-wa-nchik-man\tss{\AMV,\LT}	&	-yki-man	&	-shu-nki-man \\
\nopagebreak&	-ma-nki-man\tss{\ACH,\CH,\SP}	&	-ma-n-man\tss{\ACH,\CH,\SP}	&	-ma-nchik-man\tss{\ACH,\CH,\SP}	&	 	&	 	\\
\nopagebreak&	‘you can~\dots{} say to me’	&	‘he can~\dots{} say to me’	&	‘he can~\dots{} say to us’	&	‘I can~\dots{} say to you’	&	‘he can~\dots{} say to you’	\\

\cmidrule{2-6}
\multirow{4}{13ex}{Alternative conditional}	
&	\ding{53}	&	\ding{53}	&	-wa-chuwan\tss{\AMV,\LT}	&	\ding{53}	&	\ding{53}	\\
\nopagebreak&	\ding{53}	&	\ding{53}	&	-ma-chuwan\tss{\ACH,\CH,\SP}	&	\ding{53}	&	\ding{53}	\\
\nopagebreak&	 	&	 	&	‘he ca~\dots{} say to us’	&	&	 	\\

\cmidrule{2-6}
\multirow{10}{13ex}{Past conditional}	
&	-wa-nki-man ka-rqa\tss{\AMV}	& -wa-n-man ka-rqa\tss{\AMV}	&	-wa-nchik-man ka-rqa\tss{\AMV}	& -yki-man ka-rqa \tss{\AMV}	& -shu-nki-man ka-rqa\tss{\AMV} \\
\nopagebreak& -wa-nki-man ka-ra\tss{\LT}	& -wa-n-man ka-ra\tss{\LT}	&	-wa-nchik-man ka-ra\tss{\LT}	&	-yki-man ka-ra \tss{\LT}	&	-shu-nki-man ka-ra\tss{\LT} \\
\nopagebreak& -ma-nki-man ka-ra\tss{\ACH,\SP}	&	-ma-n-man ka-ra\tss{\ACH,\SP}	&	-ma-nchik-man ka-ra\tss{\ACH,\SP}	&	 	&	\\
\nopagebreak&	-ma-nki-man ka-la\tss{\CH}	&	-ma-n-man ka-la\tss{\CH}	&	-ma-nchik-man ka-la\tss{\CH}	&	 	&	 	\\
\nopagebreak&	‘you could~\dots{} have said to me’	&	‘he could~\dots{} have said to me’	&	‘he could~\dots{} have said to us’	&	‘I could~\dots{} have said to you’	&	‘he could~\dots{} have said to you’	\\

\cmidrule{2-6}
\multirow{8}{13ex}{Alternative past conditional} 
& \ding{53} 	 &	\ding{53}	 & -wa-chuwan ka-rqa\tss{\AMV} 	 &	\ding{53} &	\ding{53}	\\
\nopagebreak&	\ding{53} 	 &	\ding{53}	 & -ma-chuwan ka-ra\tss{\LT,\ACH,\SP} 	 &	\ding{53} &	\ding{53}	\\
\nopagebreak&	\ding{53} 	 &	\ding{53}	 & -ma-chuwan ka-la\tss{\CH} 	 &	\ding{53} &	\ding{53}	\\
\nopagebreak&	 	&	‘he could~\dots{} say to us’	&	 	&	 	\\

\cmidrule{2-6}
\multirow{5}{13ex}{Subordinator different subjects}	
&	-wa-pti-ki\tss{\AMV,\LT}	&	-wa-pti-n\tss{\AMV,\LT}	& -wa-pti-nchik\tss{\AMV,\LT}	& -pti-ki	&	-shu-pti-ki \\
\nopagebreak& -ma-pti-ki\tss{\ACH,\CH,\SP}	&	-ma-pti-n\tss{\ACH,\CH,\SP}	&	-ma-pti-nchik\tss{\ACH,\CH,\SP}	&	 	&	 	\\
\nopagebreak& ‘when~\dots{} you say to me’	&	‘when~\dots{} he says to me’	&	‘when~\dots{} he says to us’	&	‘when~\dots{} I say to you’	&	‘when~\dots{} he says to you’	\\
\end{longtable}
\end{landscape}

\subsection{Person and number}\label{sec:personandnumber}
\SYQ{} non-subordinate verbs inflect for actor and object reference; substantives inflect for allocation. 

\subsubsection{Subject}\label{ssec:subjallo}
The first person is indicated in both the verbal and substantive paradigms in \ACH, \CH, and \SP{} by \phono{-:}\tss{\ACH,\CH,\SP}; in \AMV, \LT; these are indicated by \phono{-ni}\tss{\AMV,\LT}, and \phono{-y}\tss{\AMV,\LT}, respectively. \phono{-:} and \phono{-ni} attach to verb stems (plus derivational or inflectional suffixes, if any are present, with the single exception that \phono{-ni} cannot precede the conditional suffix \phono{-man}) (\phono{puri-\pb{ni}}, \phono{puri-\pb{:}} ‘I walk’). \phono{-:} and \phono{-y} attach to the subordinating suffix \phono{-pti} (\phono{qawa-pti-\pb{y}}, \phono{qawa-pti-\pb{:}} ‘when~\dots{} I see’) and to the verb stem in the conditional (\phono{lluqsi-\pb{y}} \phono{-man}, \phono{lluqsi-\pb{:}-man} ‘I could leave’).

In all dialects the second person is indicated in the verbal paradigm by \phono{-nki} and in the substantive paradigm by \phono{-yki}. \phono{-nki} attaches to verb stems (plus derivational or inflectional suffixes, if any are present, except \phono{-man}) (\phono{puri-\pb{nki}} ‘you walk’); the \phono{-yki} allomorph \phono{-ki} attaches to the subordinator \phono{-pti} (\phono{qawa-pti-\pb{ki}} ‘when~\dots{} you see’. In Cacra, \phono{-k} indicates that the second person is the object of an action by the first person in the present tense (\phono{qu-k} ‘I give you’).

\phono{-n} indicates the third person and \phono{-nchik} refers to a group that includes the speaker and the addressee and, potentially, others in both the verbal and substantive paradigms. \phono{-n} and \phono{-nchik} attach to verb roots (plus derivational and inflectional suffixes, if any are present) (\phono{puri-n} ‘he/they walk/s’; \phono{puri-\pb{nchik}} ‘we walk’) and the the subordinating suffix \phono{-pti} as well (\phono{qawa-\pb{pti}-\pb{n}} ‘when~\dots{} you see’ \phono{qawa-\pb{pti}-\pb{nchik}} ‘when~\dots{} you see’). This information is summarized in Table~\ref{Tab14}.

% TABLE 14
\begin{table}[!ht]
\footnotesize\centering
\caption{Person suffixes by environment}\label{Tab14}
\begin{tabularx}{\textwidth}{@{}lLLLLLLL@{}}
\lsptoprule
Person	&	verb stem + suffixes	&	subordina-tor \phono{-shpa}	&	subordina-tor \phono{-pti}	&	substantive (short) \textipa{i} final	&	substantive (short) \textipa{a}, \textipa{u} final	&	substantive \Cons{} (or long \Vowe) final	&	condi\-tional \Vowe{} stem + suffixes\\
\midrule														
1	&	-ni\tss{\AMV,\LT} & -y\tss{\AMV,\LT}	& -y\tss{\AMV,\LT} &	-y\tss{\AMV,\LT} &	-y\tss{\AMV,\LT} &	-ni-y\tss{\AMV,\LT} &	-y\tss{\AMV,\LT} \\
					&-:\tss{\ACH,\CH,\SP} 	& -:\tss{\ACH,\CH,\SP} 	& -:\tss{\ACH,\CH,\SP} 	& -:\tss{\ACH,\CH,\SP} 	& -:\tss{\ACH,\CH,\SP} 	& -ni-:\tss{\ACH,\CH,\SP} 	& -:\tss{\ACH,\CH,\SP} \\[2ex]
%\midrule
2	&	-nki	&	-yki	&	-ki	&	-ki	&	-yki	&	-ni-ki	&	-nki	\\[2ex]
%\midrule
3	&	-n	&	-n	&	-n	&	-n	&	-n	&	-ni-n	&	-n	\\
1\lsc{pl}	&	-nchik	&	-nchik	&	-nchik	&	-nchik	&	-nchik	&	-ni-nchik	&	-nchik	\\
\lspbottomrule														
\end{tabularx}
\end{table}

\subsubsection{Actor and object reference}\label{ssec:actorobjref}\index[sub]{actor and object reference}
\phono{-wa}\tss{\AMV,\LT} and \phono{-ma}\tss{\ACH,\CH,\SP} indicate a first person object. Followed by the second person verbal suffix (\phono{-nki}) \phono{-wa} and \phono{-ma} indicate that the speaker is the object of action by the addressee (\phono{qu-\pb{wa}-\pb{nki}}, \phono{qu-\pb{ma}-\pb{nki}} ‘you give me’)~(\ref{Glo4:Dios}),~(\ref{Glo4:Kuyurayanchu}); followed by third person verbal suffix (\phono{-n}), they indicate that the speaker is the object of action by a third person (\phono{qu-\pb{wa}\pb{-n}}, \phono{qu-\pb{ma}-\pb{n}} ‘he/she/they give/s me’)~(\ref{Glo4:pampachi}),~(\ref{Glo4:Hapira}).\\

% 1
\gloexe{Glo4:Dios}{}{amv}%
{¡Dios Tayta! ¿Imata willakuya\pb{wanki}?}%amv que first line
{\morglo{Dios}{God}\morglo{tayta}{father}\morglo{ima-ta	}{what-\lsc{acc}}\morglo{willa-ku-ya-wa-nki}{tell-\lsc{refl}-\lsc{prog}-\lsc{1.obj}-\lsc{2}}}%morpheme+gloss
\glotran{My God! What are \pb{you} telling \pb{me}?}{}%eng+spa trans
{}{}%rec - time

% 2
\gloexe{Glo4:Kuyurayanchu}{}{sp}%
{Qam ni\pb{ma}ra\pb{nki}, “¿Kuyurayanchu?”}%sp que first line
{\morglo{qam}{you}\morglo{ni-ma-ra-nki,}{say-\lsc{1.obj}-\lsc{pst}-\lsc{2}}\morglo{kuyu-ra-ya-n-chu}{move-\lsc{passacc}-\lsc{prog}-\lsc{3}-\lsc{q}}}%morpheme+gloss
\glotran{\pb{You} asked \pb{me}, “Was it moving?”}{}%eng+spa trans
{}{}%rec - time

% 3
\gloexe{Glo4:pampachi}{}{amv}%
{Kaywan pampachi\pb{wan}.}%amv que first line
{\morglo{kay-wan}{\lsc{dem.p}-\lsc{instr}}\morglo{pampa-chi-wa-n}{bury-\lsc{caus}-\lsc{1.obj}-\lsc{3}}}%morpheme+gloss
\glotran{\pb{He}’ll bury \pb{me} with this.}{}%eng+spa trans
{}{}%rec - time

% 4
\gloexe{Glo4:Hapira}{}{ach}%
{Hapira\pb{man}.}%ach que first line
{\morglo{hapi-ra-ma-n}{grab-\lsc{urgt}-\lsc{1.obj}-\lsc{3}}}%morpheme+gloss
\glotran{\pb{It} took hold of \pb{me}.}{}%eng+spa trans
{}{}%rec - time

\noindent
\phono{-nchik} pluralizes a first-person object (\phono{qu-\pb{wa}-\pb{nchik}}, \phono{qu-\pb{ma}-\pb{-nchik}} ‘he/she/they give/s us’)~(\ref{Glo4:kambyachi}--\ref{Glo4:Mitamik}).\\

% 5
\gloexe{Glo4:kambyachi}{}{amv}%
{Lliw lliw mushuq kambyachi\pb{wanchik} rupanchiktam hinashpam kahunman wina\pb{wanchik}.}%amv que first line
{\morglo{lliw}{all}\morglo{lliw}{all}\morglo{mushuq}{new}\morglo{kambya-chi-wa-nchik}{change-\lsc{caus}-\lsc{1.obj}-\lsc{1pl}}\morglo{rupa-nchik-ta-m}{clothes-\lsc{1pl}-\lsc{acc}-\lsc{evd}}\morglo{hinashpa-m}{then}\morglo{kahun-man}{coffin-\lsc{all}}\morglo{wina-wa-nchik}{toss.in-\lsc{1.obj}-\lsc{1pl}}}%morpheme+gloss
\glotran{\pb{They} change \pb{us} into brand new clothes. Then \pb{they} toss \pb{us} into a coffin.}{}%eng+spa trans
{}{}%rec - time

% 6
\gloexe{Glo4:Mancharichi}{}{ach}%
{Mancharichi\pb{manchik} tuta.}%ach que first line
{\morglo{mancha-ri-chi-man-chik}{scare-\lsc{incep}-\lsc{caus}-\lsc{1.obj}-\lsc{1pl}}\morglo{tuta}{night}}%morpheme+gloss
\glotran{\pb{It} scares \pb{us} at night.}{}%eng+spa trans
{}{}%rec - time

% 7
\gloexe{Glo4:Mitamik}{}{ch}%
{Mitamik. Trura\pb{manchik} kwadirnuman sutinchikta.}%ch que first line
{\morglo{mita-mi-k}{quota-\lsc{evd}-\lsc{ik}}\morglo{trura-ma-nchik}{put-\lsc{1.obj}-\lsc{1pl}}\morglo{kwadirnu-man}{notebook-\lsc{all}}\morglo{suti-nchik-ta}{name-\lsc{1pl}-\lsc{acc}}}%morpheme+gloss
\glotran{A water quota. \pb{They} put \pb{us}, our names, in a notebook.}{}%eng+spa trans
{}{}%rec - time

\noindent
Followed by second person imperative suffix (\phono{-y}), \phono{-wa/-ma} indicates that the speaker is the object of action by the addressee (\phono{¡Qu-\pb{wa}-\pb{y}!}, \phono{¡Qu-\pb{ma}-\pb{y}!} ‘Give me!’)~(\ref{Glo4:Qawaykachi}),~(\ref{Glo4:dihara}).\\

% 8
\gloexe{Glo4:Qawaykachi}{}{amv}%
{¡Qawaykachi\pb{way} chay kundinawpa wasinta!}%amv que first line
{\morglo{qawa-yka-chi-wa-y}{see-\lsc{excep}-\lsc{caus}-\lsc{1.obj}-\lsc{imp}}\morglo{chay}{\lsc{dem.d}}\morglo{kundinaw-pa}{zombie-\lsc{gen}}\morglo{wasi-n-ta}{house-\lsc{3}-\lsc{acc}}}%morpheme+gloss
\glotran{Show \pb{me} the zombie’s house!}{}%eng+spa trans
{}{}%rec - time

% 9
\gloexe{Glo4:dihara}{}{ach}%
{“¡Amayá dihara\pb{may}chu!” nishpa lukuyakuyan.}%ach que first line
{\morglo{ama-yá}{\lsc{proh}-\lsc{emph}}\morglo{diha-ra-\pb{ma-y}-chu}{leave-\lsc{urgt}-\lsc{1.obj}-\lsc{imp}-\lsc{neg}}\morglo{ni-shpa}{say-\lsc{subis}}\morglo{luku-ya-ku-ya-n}{crazy-\lsc{inch}-\lsc{refl}-\lsc{prog}-\lsc{3}}}%morpheme+gloss
\glotran{Saying, “Don’t leave \pb{me}!” he is going crazy.}{}%eng+spa trans
{}{}%rec - time

\noindent
\phono{-shu}, followed by a second person verbal suffix (\phono{-nki}), indicates that the addressee is the object of action by a third person (\phono{qu-shu-nki} ‘he/she/they give/s you’)~(\ref{Glo4:Makinchikqa}).\\

% 10
\gloexe{Glo4:Makinchikqa}{}{amv}%
{Makinchikqa tusku kaptinqa vakapa nanachinqa chichinta saytarushpa diharu\pb{shunki}.}%amv que first line
{\morglo{maki-nchik-qa}{hand-\lsc{1pl}-\lsc{top}}\morglo{tusku}{rough}\morglo{ka-pti-n-qa}{be-\lsc{subds}-\lsc{3}-\lsc{top}}\morglo{vaka-pa}{cow-\lsc{gen}}\morglo{nana-chi-nqa}{hurt-\lsc{caus}-\lsc{3.fut}}\morglo{chichi-n-ta}{teat-\lsc{3}-\lsc{acc}}\morglo{sayta-ru-shpa}{kick-\lsc{urgt}-\lsc{subis}}\morglo{diha-ru-shunki}{leave-\lsc{urgt}-\lsc{3>2}}}%morpheme+gloss
\glotran{When our hands are rough, they make the cow’s teats hurt and \pb{she} kicks and leaves \pb{you}.}{}%eng+spa trans
{}{}%rec - time

\noindent
\phono{-sHQayki} indicates that the addressee is the object of future action by the speaker (\phono{qu\pb{-sa-yki}} ‘I give you’)~(\ref{Glo4:Wirayachi}--\ref{Glo4:qipiru}).\\

% 11
\gloexe{Glo4:Wirayachi}{}{ach}%
{Wirayachi\pb{sayki}.}%ach que first line
{\morglo{wira-ya-chi-sayki}{fat-\lsc{inch}-\lsc{caus}-\lsc{1>2.fut}}}%morpheme+gloss
\glotran{\pb{I’m going to} fatten \pb{you} up.}{}%eng+spa trans
{}{}%rec - time

% 12
\gloexe{Glo4:Kanallan}{}{sp}%
{Kanallan shuyakaramu\pb{sayki}.}%sp que first line
{\morglo{kanallan}{just.now}\morglo{shuya-ka-ra-mu-sayki}{wait-\lsc{passacc}-\lsc{urgt}-\lsc{cisl}-\lsc{1>2.fut}}}%morpheme+gloss
\glotran{Right now, \pb{I’m going to} wait for \pb{you}.}{}%eng+spa trans
{}{}%rec - time

% 13
\gloexe{Glo4:qullqita}{}{amv}%
{Kay qullqita qu\pb{sqayki}.}%amv que first line
{\morglo{kay}{\lsc{dem.p}}\morglo{qullqi-ta}{money-\lsc{acc}}\morglo{qu-sqayki}{give-\lsc{1>2.fut}}}%morpheme+gloss
\glotran{\pb{I’m going to} give \pb{you} this money.}{}%eng+spa trans
{}{}%rec - time

% 14
\gloexe{Glo4:qipiru}{}{amv}%
{Ñuqa qipiru\pb{shqayki} llaqtayta.}%amv que first line
{\morglo{ñuqa}{I}\morglo{qipi-ru-shqayki}{carry-\lsc{urgt}-\lsc{1>2.fut}}\morglo{llaqtayta}{town-\lsc{1}-\lsc{acc}}}%morpheme+gloss
\glotran{\pb{I’m going to} carry \pb{you} to my town.}{}%eng+spa trans
{}{}%rec - time

\noindent
The object suffixes --~\phono{-wa/-ma}, \phono{-shu} and \phono{-sHQa}~-- succeed aspect suffixes~(\ref{Glo4:Munashantanam}--\ref{Glo4:Huktriki}) and preceed tense~(\ref{Glo4:Chaynam}--\ref{Glo4:Nirayki}) and subordinating suffixes~(\ref{Glo4:Hamullarqani}--\ref{Glo4:Wasari}), as well as the nominalizing suffix \phono{-na}~(\ref{Glo4:fakultayku}),~(\ref{Glo4:shunaykipaq}) (\phono{qu\pb{-ya}-\pb{-wa}-nki} ‘you are giving me’; \phono{qu\pb{-wa}-\pb{rqa}-\uo} ‘you gave me’; \phono{qu\pb{-su-pti}-ki} ‘when he/she/they gave you’; \phono{qu\pb{-wa-na}-n-paq} ‘so he/she/they give/s me’).\\

% 15
\gloexe{Glo4:Munashantanam}{}{lt}%
{Munashantañam ruwan runaqa tantya\pb{yawan}triki.}%lt que first line
{\morglo{muna-sha-n-ta-ña-m}{want-\lsc{prf}-\lsc{3}-\lsc{acc}-\lsc{disc}-\lsc{evd}}\morglo{ruwa-n}{make-\lsc{3}}\morglo{runa-qa}{person-\lsc{top}}\morglo{tantya-ya-wa-n-tri-ki}{size.up-\lsc{prog}-\lsc{1.obj}-\lsc{3}-\lsc{evc}-\lsc{iki}}}%morpheme+gloss
\glotran{People do what they want already. \pb{They} must \pb{be} siz\pb{ing} \pb{me} up, for sure.}{}%eng+spa trans
{}{}%rec - time

% 16
\gloexe{Glo4:Kwirpum}{}{ach}%
{Kwirpum nanayan. Kaymi kay runam aysa\pb{yaman}ña.}%ach que first line
{\morglo{kwirpu-m}{body-\lsc{evd}}\morglo{nana-ya-n}{hurt-\lsc{prog}-\lsc{3}}\morglo{kay-mi}{\lsc{dem.p}-\lsc{evd}}\morglo{kay}{\lsc{dem.p}}\morglo{runa-m}{person-\lsc{evd}}\morglo{aysa-ya-ma-n-ña}{pull-\lsc{prog}-\lsc{1.obj}-\lsc{3}-\lsc{disc}}}%morpheme+gloss
\glotran{[My] body is hurting. These \pb{people are} pull\pb{ing} \pb{me} over here like this.}{}%eng+spa trans
{}{}%rec - time

% 17
\gloexe{Glo4:Huktriki}{}{amv}%
{Huktriki apa\pb{yashunki}. ¿Kikillaykichu puriyanki mutuwan?}%amv que first line
{\morglo{huk-tri-ki}{one-\lsc{evc}-\lsc{iki}}\morglo{apa-ya-shunki}{bring-\lsc{prog}-\lsc{3>2}}\morglo{kiki-lla-yki-chu}{self-\lsc{rstr}-\lsc{2}-\lsc{q}}\morglo{puri-ya-nki}{walk-\lsc{prog}-\lsc{2}}\morglo{mutu-wan}{motorcycle-\lsc{instr}}}%morpheme+gloss
\glotran{\pb{Someone} else must \pb{be} bring\pb{ing} \pb{you}. Or are you yourself wandering around with a motorbike?}{}%eng+spa trans
{}{}%rec - time

% 18
\gloexe{Glo4:Chaynam}{}{amv}%
{Chaynam kundur qipi\pb{warqa} matrayta.}%amv que first line
{\morglo{chayna-m}{thus-\lsc{evd}}\morglo{kundur}{condor}\morglo{qipi-wa-rqa}{carry-\lsc{1.obj}-\lsc{pst}}\morglo{matray-ta}{cave-\lsc{acc}}}%morpheme+gloss
\glotran{Like that, \pb{the condor} carri\pb{ed} \pb{me} to his cave.}{}%eng+spa trans
{}{}%rec - time

% 19
\gloexe{Glo4:Imapaq}{}{sp}%
{“¿Imapaq aysapa\pb{maranki} ñuqa hawka puñukupti:?” nishpash.}%sp que first line
{\morglo{imapaq}{why}\morglo{aysa-pa-ma-ra-nki}{pull-\lsc{ben}-\lsc{1.obj}-\lsc{pst}-\lsc{2}}\morglo{ñuqa}{I}\morglo{hawka}{tranquil}\morglo{puñu-ku-pti-:}{sleep-\lsc{refl}-\lsc{subds}-\lsc{1}}\morglo{ni-shpa-sh}{say-\lsc{subis}-\lsc{evr}}}%morpheme+gloss
\glotran{“Why \pb{did} \pb{you} tug at \pb{me} when I was sleeping peacefully?” said [the zombie].}{}%eng+spa trans
{}{}%rec - time

% 20
\gloexe{Glo4:Nirayki}{}{sp}%
{Ni\pb{rayki}.}%sp que first line
{\morglo{ni-ra-yki}{say-\lsc{pst}-\lsc{1>2}}}%morpheme+gloss
\glotran{\pb{I} sa\pb{i}d to \pb{you}.}{}%eng+spa trans
{}{}%rec - time

% 21
\gloexe{Glo4:Hamullarqani}{}{amv}%
{Hamullarqani chikchik paralla \pb{tapallawaptin} yana puyulla \pb{ñitillawaptin}.}%amv que first line
{\morglo{hamu-lla-rqa-ni}{come-\lsc{rstr}-\lsc{pst}-\lsc{1}}\morglo{chikchik}{hail}\morglo{para-lla}{rain-\lsc{rstr}}\morglo{tapa-lla-wa-pti-n}{cover-\lsc{rstr}-\lsc{1.obj}-\lsc{subds}-\lsc{3}}\morglo{yana}{black}\morglo{puyu-lla}{cloud-\lsc{rstr}}\morglo{ñiti-lla-wa-pti-n}{crush-\lsc{rstr}-\lsc{1.obj}-\lsc{subds}-\lsc{3}}}%morpheme+gloss
\glotran{I came \pb{when} the freezing rain was \pb{covering me}, \pb{when} the black fog was \pb{crushing me}.}{}%eng+spa trans
{}{}%rec - time

% 22
\gloexe{Glo4:pampamanqa}{}{amv}%
{¡Kay pampaman qatimuchun! Wakpa \pb{ñitiruwaptin}qa.}%amv que first line
{\morglo{kay}{\lsc{dem.p}}\morglo{pampa-man}{plain-\lsc{all}}\morglo{qati-mu-chun}{follow-\lsc{cisl}-\lsc{injunc}}\morglo{wak-pa}{\lsc{dem.d}-\lsc{loc}}\morglo{ñiti-ru-wa-pti-n-qa}{crush-\lsc{urgt}-\lsc{1.obj}-\lsc{subds}-\lsc{3}-\lsc{top}}}%morpheme+gloss
\glotran{Let him bring it toward that plain -- over there \pb{he would crush me}.}{}%eng+spa trans
{}{}%rec - time

% 23
\gloexe{Glo4:yakukta}{}{ch}%
{Mana yakukta qu\pb{maptin}, ¿Imaynataq alfa:pis planta:pis kanqa?}%ch que first line
{\morglo{mana}{no}\morglo{yaku-kta}{water-\lsc{acc}}\morglo{qu-ma-pti-n,}{give-\lsc{1.obj}-\lsc{subds}-\lsc{3}}\morglo{imayna-taq}{how-\lsc{seq}}\morglo{alfa-:-pis}{alfalfa-\lsc{add}}\morglo{planta-:-pis}{plant-\lsc{1}-\lsc{add}}\morglo{ka-nqa}{be-\lsc{3.fut}}}%morpheme+gloss
\glotran{\pb{If they} don’t \pb{give me} water, how will I have alfalfa and plants?}{}%eng+spa trans
{}{}%rec - time

% 24
\gloexe{Glo4:Wamra}{}{lt}%
{Wamra willa\pb{suptiki}.}%lt que first line
{\morglo{wamra}{child}\morglo{willa-su-pti-ki}{tell-\lsc{2.obj}-\lsc{subds}-\lsc{2}}}%morpheme+gloss
\glotran{\pb{When the} children told \pb{you}.}{}%eng+spa trans
{}{}%rec - time

% 25
\gloexe{Glo4:Sudarachi}{}{amv}%
{Sudarachi\pb{shuptiki} kapasmi surqurunman.}%amv que first line
{\morglo{suda-ra-chi-shu-pti-ki}{sweat-\lsc{urgt}-\lsc{caus}-\lsc{2.obj}-\lsc{subds}-\lsc{2}}\morglo{kapas-mi}{perhaps-\lsc{evd}}\morglo{surqu-ru-n-man}{take.out-\lsc{urgt}-\lsc{3}-\lsc{cond}}}%morpheme+gloss
\glotran{\pb{When it} makes \pb{you} sweat, it’s possible he could remove it.}{}%eng+spa trans
{}{}%rec - time

% 26
\gloexe{Glo4:Tantya}{}{lt}%
{Tantya\pb{washpa} chayta ruwan.}%lt que first line
{\morglo{tantya-wa-shpa}{size.up-\lsc{1.obj}-\lsc{subis}}\morglo{chay-ta}{\lsc{dem.d}-\lsc{acc}}\morglo{ruwa-n}{make-\lsc{3}}}%morpheme+gloss
\glotran{Sizing \pb{me} up, \pb{they} do that.}{}%eng+spa trans
{}{}%rec - time

% 27
\gloexe{Glo4:Wasari}{}{sp}%
{Wasari\pb{mashpa}m nuchipis kwintakuq.}%sp que first line
{\morglo{wasa-ri-ma-shpa-m}{wake-\lsc{incep}-\lsc{1.obj}-\lsc{subis}-\lsc{evd}}\morglo{nuchi-pis}{night-\lsc{add}}\morglo{kwinta-ku-q}{tell.story-\lsc{refl}-\lsc{ag}}}%morpheme+gloss
\glotran{At night, \pb{they} would wake \pb{me} up and tell stories.}{}%eng+spa trans
{}{}%rec - time

% 28
\gloexe{Glo4:fakultayku}{}{lt}%
{Pipis fakultayku\pb{wananpaq}.}%lt que first line
{\morglo{pi-pis}{who-\lsc{add}}\morglo{fakulta-yku-wa-na-n-paq}{assist-\lsc{excep}-\lsc{1.obj}-\lsc{nmlz}-\lsc{3}-\lsc{purp}}}%morpheme+gloss
\glotran{\pb{So someone} can help \pb{me} out.}{}%eng+spa trans
{}{}%rec - time

% 29
\gloexe{Glo4:shunaykipaq}{}{amv}%
{Raki\pb{shunaykipaq}.}%amv que first line
{\morglo{raki-shu-na-yki-paq}{separate-\lsc{2.obj}-\lsc{nmlz}-\lsc{2}-\lsc{purp}}}%morpheme+gloss
\glotran{\pb{So he} sets some aside \pb{for you}.}{}%eng+spa trans
{}{}%rec - time

\noindent
Both object and subject suffixes --~\phono{-wa/-ma}, \phono{-shu} and \phono{-sHQa}, as well as \phono{-nki}, \phono{-YkI}, and \phono{-n}~-- preceed the conditional suffix \phono{-man} (\phononb{qu\pb{-wa}\-\pb{-nki}\pb{-man}} ‘you could give me’)~(\ref{Glo4:Sarurulla}--\ref{Glo4:Kwidaducha}).\\

% 30
\gloexe{Glo4:Sarurulla}{}{amv}%
{Sarurulla\pb{wankiman}. Manam saru\pb{wanan}taq munaniñachu.}%amv que first line
{\morglo{saru-ru-lla-wa-nki-man}{trample-\lsc{urgt}-\lsc{rstr}-\lsc{1.obj}-\lsc{cond}-\lsc{2}}\morglo{mana-m}{no-evd}\morglo{saru-wa-na-n-taq}{trample-\lsc{1.obj}-\lsc{nmlz}-\lsc{3}-\lsc{seq}}\morglo{muna-ni-ña-chu}{want-\lsc{1}-\lsc{disc}-\lsc{neg}}}%morpheme+gloss
\glotran{\pb{You could} trample \pb{me}. I don’t want \pb{him} to trample \pb{me} any more.}{}%eng+spa trans
{}{}%rec - time

% 31
\gloexe{Glo4:chichiyuqchi}{}{amv}%
{Mana chichiyuq kaptikiqa chayna lluqari\pb{shunkiman}tri.}%amv que first line
{\morglo{mana}{no}\morglo{chichi-yuq}{breast-\lsc{poss}}\morglo{ka-pti-ki-qa}{be-\lsc{subds}-\lsc{2}-\lsc{top}}\morglo{chayna}{thus}\morglo{lluqa-ri-shu-nki-man-tri}{top-\lsc{incep}-\lsc{2.obj}-\lsc{2}-\lsc{cond}-\lsc{evc}}}%morpheme+gloss
\glotran{When you don’t have breasts \pb{they can} top \pb{you}.}{}%eng+spa trans
{}{}%rec - time

% 32
\gloexe{Glo4:Kwidaducha}{}{ch}%
{¡Kwidadu! Chaypitaq qalqali mikulu\pb{shunkiman}.}%ch que first line
{\morglo{kwidadu}{careful}\morglo{chay-pi-taq}{\lsc{dem.d}-\lsc{loc}-\lsc{seq}}\morglo{qalqali}{zombie}\morglo{miku-lu-shunki-man}{eat-\lsc{urgt}-\lsc{2.obj}-\lsc{2}-\lsc{cond}}}%morpheme+gloss
\glotran{Be careful! \pb{A demon} \pb{could} eat \pb{you} there.}{}%eng+spa trans
{}{}%rec - time

\noindent
Exceptions to these rules arise when object is 1\lsc{pl}. First, the first-person object pluralizer, \phono{-nchik}, does not preceed aspect, tense, subordinating, nominalizing and conditional suffixes, but, rather, succeeds them (\phono{ñiti-ru-wa-n\pb{-man-chik}} ‘it could crush us’)~(\ref{Glo4:tumaytam}--\ref{Glo4:wanmanchik}).\\

% 33
\gloexe{Glo4:tumaytam}{}{amv}%
{Mana kanan tumaytam munanchu qaninpaq \pb{shinkarachiwarqanchik}.}%amv que first line
{\morglo{mana}{no}\morglo{kanan}{now}\morglo{tuma-y-ta-m}{drink-\lsc{inf}-\lsc{acc}-\lsc{evd}}\morglo{muna-n-chu}{want-\lsc{3}-\lsc{neg}}\morglo{qanin-paq}{previous-\lsc{abl}}\morglo{shinka-ra-chi-wa-rqa-nchik}{get.drunk-\lsc{urgt}-\lsc{caus}-\lsc{1.obj}-\lsc{pst-\lsc{1pl}}}}%morpheme+gloss
\glotran{She doesn’t want to drink now. Earlier, \pb{they had got us drunk}.}{}%eng+spa trans
{}{}%rec - time

% 34
\gloexe{Glo4:pasawaptinchikpis}{}{amv}%
{Chiri \pb{pasawaptinchikpis}, wiksa nanaykunapaq.}%amv que first line
{\morglo{chiri}{cold}\morglo{pasa-wa-pti-nchik-pis}{pass-\lsc{31.obj}-\lsc{subds}-\lsc{1pl}-\lsc{add}}\morglo{wiksa}{stomach}\morglo{nana-y-kuna-paq}{hurt-\lsc{inf}-\lsc{pl}-\lsc{abl}}}%morpheme+gloss
\glotran{\pb{When we get chills} or for stomach pain [this plant is good].}{}%eng+spa trans
{}{}%rec - time

% 35
\gloexe{Glo4:wanmanchik}{}{amv}%
{Ñitiru\pb{wanmanchik}.}%amv que first line
{\morglo{ñiti-ru-wan-ma-nchik}{crush-\lsc{urgt}-\lsc{1.obj}-\lsc{1pl}-\lsc{cond}-\lsc{3>1pl}}}%morpheme+gloss
\glotran{\pb{It} could crush \pb{us}.}{}%eng+spa trans
{}{}%rec - time

\noindent
Second, 3>1\lsc{pl} future is not indicated by \phono{*-wa/ma\pb{-nqa-nchik}}, but rather by \phononb{-wa/ma\pb{shun}} (\ref{Glo4:Mundum}), (\ref{Glo4:Watyarunshi}).\\

% 36 (44)
\gloexe{Glo4:Mundum}{}{sp}%
{Mundum \pb{ñitiramashun}. Kaytam sustininkiqa.}%sp que first line
{\morglo{mundu-m}{world-\lsc{evd}}\morglo{ñiti-ra-ma-shun}{crush-\lsc{urgt}-\lsc{1.obj}-\lsc{1pl.fut}}\morglo{kay-ta-m}{\lsc{dem.p}-\lsc{acc}-\lsc{evd}}\morglo{sustini-nki-qa}{sustain-\lsc{2}-\lsc{top}}}%morpheme+gloss
\glotran{\pb{The world is going to crush us}. Hold this one up.}{}%eng+spa trans
{}{}%rec - time

% 37 (45)
\gloexe{Glo4:Watyarunshi}{}{ach}%
{Watyarunshi. Chaynatr \pb{watyaramashun} ñuqanchiktapis.}%ach que first line
{\morglo{watya-ru-n-shi}{bake-\lsc{urgt}-\lsc{3}-\lsc{evr}}\morglo{chayna-tr}{thus-\lsc{evc}}\morglo{watya-ra-ma-shun}{bake-\lsc{urgt}-\lsc{1.obj}-\lsc{1pl.fut}}\morglo{ñuqa-nchik-ta-pis}{I-\lsc{1pl}-\lsc{acc}-\lsc{add}}}%morpheme+gloss
\glotran{They got baked, they say. Like that, \pb{we’re going to get baked}, us, too.}{}%eng+spa trans
{}{}%rec - time

\noindent
Finally, third, just as the 1\lsc{pl} conditional may be indicated by either of two forms, one regular (\phono{-nchik-man}) one alternative/portmaneau (\phono{-chuwan}), the 3>1\lsc{pl} conditional, too, may be indicated by either a regular (\phono{-wa/ma-n-man-chik}) or a portmanteau form (\phono{-wa/ma-chuwan}) (\phono{chuka-ru\pb{-wa-chuwan}} ‘it can make us sick’)~(\ref{Glo4:qullqiyuqpaq}), (\ref{Glo4:machuwan}).\\

% 38 (37)
\gloexe{Glo4:qullqiyuqpaq}{}{amv}%
{Kayanmi uniku qullqiyuqpaq. ¿Maypam rigala\pb{wachuwan} runaqa?}%amv que first line
{\morglo{ka-ya-n-mi}{be-\lsc{prog}-\lsc{3}-\lsc{evd}}\morglo{uniku}{only}\morglo{qullqi-yuq-paq}{money-\lsc{poss}-\lsc{ben}}\morglo{may-pa-m}{where-\lsc{loc}-\lsc{evd}}\morglo{rigala-wa-chuwan}{gift-\lsc{1.obj}-\lsc{1pl.cond}}\morglo{runa-qa}{person-\lsc{top}}}%morpheme+gloss
\glotran{There are only for rich people. Where \pb{can people give us} things for free?}{}%eng+spa trans
{}{}%rec - time

% 39 (38)
\gloexe{Glo4:machuwan}{}{ach}%
{Miku\pb{machuwan}tri.}%ach que first line
{\morglo{miku-ma-chuwan-tri}{eat-\lsc{1.obj}-\lsc{1pl.cond}-\lsc{evc}}}%morpheme+gloss
\glotran{\pb{He} could eat \pb{us}.}{}%eng+spa trans
{}{}%rec - time

\noindent
In all other cases, subject-object suffixes combine with standard morphology~(\ref{Glo4:kuntistamu}--\ref{Glo4:Tirruristam}).\\

% 40 (41)
\gloexe{Glo4:kuntistamu}{}{amv}%
{Qampis kuntistamu\pb{wanki}má.}%amv que first line
{\morglo{qam-pis}{you-\lsc{add}}\morglo{kuntista-mu-wa-nki-m-á}{answer-\lsc{cisl}-\lsc{1.obj}-\lsc{2}-\lsc{evd}-\lsc{emph}}}%morpheme+gloss
\glotran{\pb{You}, too, are \pb{going} to answer \pb{me}.}{}%eng+spa trans
{}{}%rec - time

% 41 (42)
\gloexe{Glo4:Allichawanqa}{}{amv}%
{¿\pb{Allichawanqa}chu manachu? Yatrarunqaña kukantaqa qawaykushpa.}%amv que first line
{\morglo{alli-cha-wa-nqa-chu}{good-\lsc{fact}-\lsc{1.obj}-\lsc{3.fut}-\lsc{q}}\morglo{mana-chu}{no-\lsc{q}}\morglo{yatra-ru-nqa-ña}{know-\lsc{urgt}-\lsc{3.fut}-\lsc{disc}}\morglo{kuka-n-ta-qa}{coca-\lsc{3}-\lsc{acc}-\lsc{top}}\morglo{qawa-yku-shpa}{see-\lsc{excep}-\lsc{subis}}}%morpheme+gloss
\glotran{Is \pb{he going to heal me} or not? He’ll find out by looking at his coca.}{}%eng+spa trans
{}{}%rec - time

% 42 (43)
\gloexe{Glo4:Tirruristam}{}{ach}%
{Tirruristam hamuyan. Wak turutatr pagaykushaqqa manam \pb{wañuchimanqa}chu.}%ach que first line
{\morglo{tirrurista-m}{terrorist-\lsc{evd}}\morglo{hamu-ya-n}{come-\lsc{prog}-\lsc{3}}\morglo{wak}{\lsc{dem.d}}\morglo{turu-ta-tr}{bull-\lsc{acc}-\lsc{evc}}\morglo{paga-yku-shaq-qa}{pay-\lsc{excep}-\lsc{1.fut}-\lsc{top}}\morglo{mana-m}{no-\lsc{evd}}\morglo{wañu-chi-ma-nqa-chu}{die-\lsc{caus}-\lsc{1.obj}-\lsc{3.fut}-\lsc{neg}}}%morpheme+gloss
\glotran{The terrorists are coming. I’ll pay them a bull and \pb{they won’t kill me}.}{}%eng+spa trans
{}{}%rec - time

\noindent
A typological note: number is expressed in spontaneously-occurring examples only in those cases in which there is a first-person plural object~(\ref{Glo4:Yatra}). In these cases all \SYQ{} dialects follow the Southern \QII{} pattern ordering suffixes: \lsc{obj}-\lsc{tns}-\lsc{sbj}-\lsc{num}. Note, though, that while in the Southern \QII{} languages \phono{-chik} pluralizes the subject, in \SYQ{} \phono{-chik} pluralizes the object. There are no spontaneous examples following the Central \QII{} pattern \lsc{num}-\lsc{obj}-\lsc{tns}-\lsc{sbj}.\\

% 43 (36)
\gloexe{Glo4:Yatra}{}{amv}%
{Mana riq\pb{kuna}, ¿Imatam rima\pb{sayki}? Yatra\pb{nchik}chu.}%amv que first line
{\morglo{mana}{no}\morglo{ri-q-kuna}{go-\lsc{ag}-\lsc{pl}}\morglo{ima-ta-m}{what-\lsc{acc}-\lsc{evd}}\morglo{rima-sayki}{talk-\lsc{1>2}}\morglo{yatra-nchik-chu}{know-\lsc{1pl}-\lsc{neg}}}%morpheme+gloss
\glotran{\pb{People} who haven’t gone, what am \pb{I going to say to you}? \pb{We} don’t know.}{}%eng+spa trans
{}{}%rec - time

\noindent
There are no special forms for third-person objects. A third-person object is indicated by the case-marking of the third-person pronoun \phono{pay} with either accusative \phono{-ta} or allative/dative \phono{-man} (\phono{\pb{pay-ta}} \phono{qawa-nchik} ‘we see him/her,’ \phononb{\pb{pay}\-\pb{-kuna}\pb{-man}} \phono{qu-nki} ‘you give them’)~(\ref{Glo4:swirupis}).\\

% 44 (39)
\gloexe{Glo4:swirupis}{}{lt}%
{Kay swirupis allquypaqpis~\dots{} nikurunshi \pb{subrinuntaqa}.}%lt que first line
{\morglo{kay}{\lsc{dedm.p}}\morglo{swiru-pis}{whey-\lsc{add}}\morglo{allqu-y-paq-pis}{dog-\lsc{1}-\lsc{ben}-\lsc{add}}\morglo{ni-ku-ru-n-shi}{say-\lsc{refl}-\lsc{urgt}-\lsc{3}-\lsc{evr}}\morglo{subrinu-n-ta-qa}{nephew-\lsc{3}-\lsc{acc}-\lsc{top}}}%morpheme+gloss
\glotran{This whey also for my dog also~\dots{} he said, they say, \pb{to his nephew}.}{}%eng+spa trans
{}{}%rec - time

\noindent
First-and second-person object suffixes may be reinforced with similarly case-marked pronouns~(\ref{Glo4:uywamara}).\\

% 45 (40)
\gloexe{Glo4:uywamara}{}{sp}%
{\pb{Ñuqata} \pb{uywamara} mamacha: tiyu: tiya:.}%sp que first line
{\morglo{ñuqa-ta}{I-\lsc{acc}}\morglo{uywa-ma-ra}{raise-\lsc{1.obj}-\lsc{pst}}\morglo{mama-cha-:}{mother-\lsc{dim}-\lsc{1}}\morglo{tiyu-:}{uncle	-\lsc{1}}\morglo{tiya-:}{aunt-\lsc{1}}}%morpheme+gloss
\glotran{My grandmother and my uncle and aunt \pb{raised} \emph{\pb{me}}.}{}%eng+spa trans
{}{}%rec - time

There are no special forms for actors acting on themselves or any group that includes them: reflexive action is indicated with the derivational suffix \phono{-ku}. ‘I see myself ‘ is \phono{ñuqa} \phono{qawa-\pb{ku}-ni/-:} and ‘I see us’ is ‘\phono{ñuqa} \phono{ñuqanchik-\pb{ta}} \phono{qawa-ni/-:}.

Actor-object suffixes are employed both with transitive and ditransitive verbs (\phono{Miku-ru-\pb{shunki}} ‘He’s going to eat you’; \phono{Kay qullqi-ta qu-\pb{sqayki}} ‘I’m going to give you this money’). Actor-object suffixes may be reinforced --~but not replaced~-- by accusative- and dative-marked personal pronouns (\phono{\pb{Ñuqa}-\pb{-ta}-s} \phononb{harqu-ru-\pb{wa}-ra-\uo} ‘He tossed \pb{me} out, too’).

Except in the two cases 2>1\lsc{pl} and 3>1\lsc{pl}, where \phono{-chik} indicates a plural object, when either the actor or the object is plural, the verb optionally takes the joint action suffix \phono{-pakU}\index[sub]{joint action} (3\lsc{pl}>2 \phono{Pay-kuna} \phono{qu-\pb{paku}-shunki} \phono{tanta-ta} \phono{qam-man}. ‘They give you.\lsc{s} bread’; 1>2\lsc{pl} \phono{Ñuqa} \phono{qu-\pb{paku}-yki} \phono{tanta-ta qam-kuna-man} ‘I give you.\lsc{pl} bread’). In practice, the plural forms, although recognized, are not spontaneously invoked.

This information is summarized in Table~\ref{Tab15}. Naturally-occurring examples of the five principal subject-object reference processes (1>2, 2>1, 3>1, 3>2, 3>1\lsc{pl}) are presented in~(\ref{Glo4:Dios}--\ref{Glo4:uywamara}).

% TABLE 15
\begin{table}[!ht]
\small\centering
\caption{Actor-object inflectional suffixes}\label{Tab15}
\begin{tabular}{llll}
\lsptoprule
				& 1\lsc{obj}				& 2\lsc{obj} 							& 1\lsc{pl} \lsc{obj}		\\
\midrule
{1 \lsc{sbj}}	& \ding{53}					& Present: -YkI\tss{\ACH,\AMV,\LT,\SP}	& \ding{53}		\\
				& 							& Future: -sHQa-yki						& 	\\[2ex]
%\midrule
{2 \lsc{sbj}}	& -wa-nki\tss{\AMV,\LT}		& \ding{53}								& 		\\
				& -ma-nki\tss{\ACH,\CH,\SP}	& 										& 		\\[2ex]
%\midrule
{3 \lsc{sbj}}	& -wa-N\tss{\AMV,\LT}		& 	-shu-nki							& -wa-nchik\tss{\AMV,\LT}	\\
				& -ma-N\tss{\ACH,\CH,\SP}	& 										& -ma-nchik\tss{\ACH,\CH,\SP}		\\
\lspbottomrule
\end{tabular}
\end{table}

\subsection{Tense}
\SYQ{} counts three tenses: present, past, and future (\phono{maska-nchik} ‘we look for’, \phono{maska}\phono{-rqa-nchik} ‘we looked for’, \phono{maska-shun} ‘we will look for’). With the exception of the first person plural, person suffixes in \SYQ{} are unmarked for number. \phono{-nki} corresponds to the second person singular and plural (\phono{yanapa-nki} ‘you.\lsc{s/pl} help; \phono{maylla-nki} ‘you.\lsc{s/pl} wash’). \phono{-N} corresponds to the third person singular and plural (\phono{taki-n} ‘she/he/it/they sing(s)’). §~\ref{ssec:simplepresent}--\ref{ssec:past} cover the simple present, future and past tenses, in turn. 

\subsubsection{Simple present}\label{ssec:simplepresent}\index[sub]{simple present}
The present tense subject suffixes in \SYQ{} are \phono{-ni} and \phono{-:} (1\lsc{P}), \phono{-nki} (2\lsc{P}), \phono{-n} (3\lsc{P}), and \phono{-nchik} (1\lsc{pl}) (\phono{atrqay-tuku-\pb{ni}/\pb{-:}} ‘\pb{I} pretend to be an eagle’, \phono{kundur-tuku\pb{-nki}} ‘\pb{you} pretend to be a condor’, \phono{rutu-tuku-\pb{n}} ‘\pb{he} pretends to be a \emph{rutu}’ (small mountain bird), \phono{qari-tuku\pb{-nchik}} ‘\pb{we} pretend to be men’)~(\ref{Glo4:Wasiytanuqa}--\ref{Glo4:Suqta}).\\

% 1
\gloexe{Glo4:Wasiytanuqa}{}{amv}%
{Wasiyta \pb{ñuqa}qa pichaku\pb{ni} tallawanmi.}%amv que first line
{\morglo{wasi-y-ta}{house-\lsc{1}-\lsc{acc}}\morglo{ñuqa-qa}{I-\lsc{top}}\morglo{picha-ku-ni}{sweep-\lsc{refl}-\lsc{1}}\morglo{talla-wan-mi}{straw-\lsc{instr}-\lsc{evd}}}%morpheme+gloss
\glotran{\pb{I} sweep my house with straw.}{}%eng+spa trans
{}{}%rec - time

% 2
\gloexe{Glo4:yatra}{}{ach}%
{Manam \pb{ñuqa} yatra\pb{:}chu.}%ach que first line
{\morglo{mana-m}{no-\lsc{evd}}\morglo{ñuqa}{I}\morglo{yatra-:-chu}{know-\lsc{1}-\lsc{neg}}}%morpheme+gloss
\glotran{\pb{I} don’t know (how).}{}%eng+spa trans
{}{}%rec - time

% 3
\gloexe{Glo4:ritamu}{}{lt}%
{\pb{Qam}qa ritamu\pb{nki} urquta.}%lt que first line
{\morglo{qam-qa}{you-\lsc{top}}\morglo{ri-tamu-nki}{go-\lsc{irrev}-\lsc{2}}\morglo{urqu-ta}{hill-\lsc{acc}}}%morpheme+gloss
\glotran{\pb{You} left for the hill for good.}{}%eng+spa trans
{}{}%rec - time

% 4
\gloexe{Glo4:mikuku}{}{amv}%
{\pb{Allqu} mikuku\pb{n} wakchuchataqa.}%amv que first line
{\morglo{allqu}{dog}\morglo{miku-ku-n}{eat-\lsc{refl}-\lsc{3}}\morglo{wakchu-cha-ta-qa}{lamb-\lsc{dim}-\lsc{acc}-\lsc{top}}}%morpheme+gloss
\glotran{\pb{The dog} ate up the lamb.}{}%eng+spa trans
{}{}%rec - time

% 5
\gloexe{Glo4:Viyhunchikta}{}{amv}%
{Viyhunchikta ruwa\pb{nchik} hinashpaqa kaña\pb{nchik}mi.}%amv que first line
{\morglo{viyhu-nchik-ta}{effigy-\lsc{1pl}-\lsc{acc}}\morglo{ruwa-nchik}{make-\lsc{1pl}}\morglo{hinashpa-qa}{then-\lsc{top}}\morglo{kaña-nchik-mi}{burn-\lsc{1pl}-\lsc{evd}}}%morpheme+gloss
\glotran{\pb{We} make our effigy then burn it.}{}%eng+spa trans
{}{}%rec - time

% 6
\gloexe{Glo4:Familyallan}{}{ch}%
{Familyallan \pb{ñuqakuna} suya:.}%ch que first line
{\morglo{familya-lla-n}{family-\lsc{rstr}-\lsc{3}}\morglo{ñuqa-kuna}{I-\lsc{pl}}\morglo{suya-:}{wait-\lsc{1}}}%morpheme+gloss
\glotran{Just their relatives -- \pb{we} waited.}{}%eng+spa trans
{}{}%rec - time

% 7
\gloexe{Glo4:qamkuna}{}{sp}%
{Kanan \pb{qamkuna}tr hamuyanki.}%sp que first line
{\morglo{kanan}{now}\morglo{qam-kuna-tr}{you-\lsc{pl}-\lsc{evc}}\morglo{hamu-ya-nki}{come-\lsc{prog}-\lsc{2}}}%morpheme+gloss
\glotran{Now \pb{you.\lsc{pl}} are coming.}{}%eng+spa trans
{}{}%rec - time

% 8
\gloexe{Glo4:Suqta}{}{amv}%
{\pb{Suqta wanka} vakata tumban.}%amv que first line
{\morglo{suqta}{six}\morglo{wanka}{hired.hand}\morglo{vaka-ta}{cow-\lsc{acc}}\morglo{tumba-n}{tackle-\lsc{3}}}%morpheme+gloss
\glotran{\pb{Six hired hands} tackle the cow.}{}%eng+spa trans
{}{}%rec - time

\noindent
Although it generally indicates temporally unrestricted or habitual action, the simple present is unmarked for time. Present tense forms may receive past and future tense interpretations in different contexts (\phono{qawa-chi-n} ‘he showed/shows/will show’)~(\ref{Glo4:qawaykushpa}).\\

% 9
\gloexe{Glo4:qawaykushpa}{}{amv}%
{Chaytaqa qawaykushpa valurta hapi\pb{ni}.}%amv que first line
{\morglo{chay-ta-qa}{\lsc{dem.d}-\lsc{acc}-\lsc{top}}\morglo{qawa-yku-shpa}{see-\lsc{excep}-\lsc{subis}}\morglo{valur-ta}{courage-\lsc{acc}}\morglo{hapi-ni}{grab-\lsc{1}}}%morpheme+gloss
\glotran{Looking at that, I gather\pb{ed} courage.}{}%eng+spa trans
{}{}%rec - time

\noindent
\SYQ{} makes available a three-way distinction in the first person plural, between \phono{ñuqanchik} (dual), \phono{ñuqanchikkuna} (inclusive), and \phono{ñuqakuna} (exclusive). In practice, \phono{ñuqanchik} is employed with dual, inclusive and exclusive interpretations to the virtual complete exclusion of the other two forms, except in the \CH{} dialect. Verbs and substantives appearing with the inclusive \phono{ñuqanchikkuna} inflect following the same rules as do verbs and substantives appearing with the dual/default \phono{ñuqanchik}~(\ref{Glo4:kustumbrawmi}); verbs and substantives appearing with the exclusive \phono{ñuqakuna} inflect following the same rules as do verbs and substantives appearing with the singular \phono{ñuqa}~(\ref{Glo4:taytachaymi}).\\

% 10
\gloexe{Glo4:kustumbrawmi}{}{amv}%
{Kaypi \pb{ñuqanchikkuna}qa kustumbrawmi kaya\pb{nchik}.}%amv que first line
{\morglo{kay-pi}{\lsc{dem.p}-\lsc{loc}}\morglo{~nuqa-nchik-kuna-qa}{I-\lsc{1pl}-\lsc{pl}-\lsc{top}}\morglo{kustumbraw-mi}{accustomed-\lsc{evd}}\morglo{ka-ya-nchik}{be-\lsc{prog}-\lsc{1pl}}}%morpheme+gloss
\glotran{Here, \pb{we}’re accustomed to it.}{}%eng+spa trans
{}{}%rec - time

% 11
\gloexe{Glo4:taytachaymi}{}{amv}%
{Wañuq taytachaymi chaytaqa \pb{ñuqakuna}man willawarqa.}%amv que first line
{\morglo{wañu-q}{die-\lsc{ag}}\morglo{tayta-cha-y-mi}{father-\lsc{dim}-\lsc{1}-\lsc{evd}}\morglo{chay-ta-qa}{\lsc{dem.d}-\lsc{acc}-\lsc{top}}\morglo{ñuqa-kuna-man}{I-\lsc{pl}-\lsc{all}}\morglo{willa-wa-rqa}{tell-\lsc{1.obj}-\lsc{pst}}}%morpheme+gloss
\glotran{Our late grandfather told that to \pb{us}.}{}%eng+spa trans
{}{}%rec - time

\noindent
Although \phono{ñuqa} is generally interpreted as singular --~likely an implicature attributable to the availability of plural forms in the first person~-- it is, in fact, unspecified for number and may receive plural interpretations~(\ref{Glo4:Kamapam}).\\

% 12
\gloexe{Glo4:Kamapam}{}{ach}%
{Kamapam \pb{ñuqa} puñukuya\pb{:} ishkayni:.}%ach que first line
{\morglo{kama-pa-m}{bed-\lsc{loc}-\lsc{evd}}\morglo{ñuqa}{I}\morglo{puñu-ku-ya-:}{sleep-\lsc{refl}-\lsc{prog}-\lsc{1}}\morglo{ishkay-ni-:}{two-\lsc{euph}-\lsc{1}}}%morpheme+gloss
\glotran{\pb{We} were \pb{both} sleeping in bed.}{}%eng+spa trans
{}{}%rec - time

% 13
\gloexe{Glo4:Dispidichin}{}{amv}%
{Dispidichin churinkunata hinashpaqa kañan.}%
{\morglo{dispidi-chi-n	}{bid.farewell-\lsc{caus}-\lsc{3}}\morglo{churi-n-kuna-ta}{child-\lsc{3}-\lsc{pl}-\lsc{acc}}\morglo{hinashpa-qa}{then-\lsc{top}}\morglo{kaña-n}{burn-\lsc{3}}}%morpheme+gloss
\glotran{One has their children say good bye and then burns it [the effigy].}%eng
{‘Se hace depedir a sus hijos y después se lo quema’.}%spa
{}{}%{Vinac\_HQ\_Lamb\_NewYear}{00:50--00:55}%

% TABLE 16a
\begin{table}[!ht]
\small\centering
\caption{Present tense inflection}\label{Tab16a}
\begin{tabular}{llll}
\lsptoprule
Person	& Singular				& \multicolumn{2}{l}{Plural}	\\
\midrule
1		& 						&-nchik				&(dual, incl.)	\\
		&-ni\tss{\AMV,\LT}		&-ni\tss{\AMV,\LT}				&(excl.)		\\
					&-:\tss{\ACH,\CH,\SP}	&-:\tss{\ACH,\CH,\SP}				&(excl.)		\\[2ex]
%\midrule
2		&-nki					&-nki			&\\[2ex]
%\midrule
3		&-n						&-n				&\\
\lspbottomrule
\end{tabular}
\end{table}

% TABLE 16b
\begin{table}[!ht]
\small\centering
\caption{Present tense inflection -- actor-object suffixes}\label{Tab16b}
\begin{tabular}{lllll}
\lsptoprule
2>1	&	3>1	&	3>1pl	&	1>2	&	3>2	\\
\midrule
-wa-nki\tss{\AMV,\LT}	
&	-wa-n\tss{\AMV,\LT}	
&	-wa-nchik\tss{\AMV,\LT}	
&	-yki	
&	-shunki	\\
-ma-nki\tss{\ACH,\CH,\SP}	
&	-ma-n\tss{\ACH,\CH,\SP}	
&	-ma-nchik\tss{\ACH,\CH,\SP}	
&	 	&	 	\\
\lspbottomrule
\end{tabular}
\end{table}

\subsubsection{Future}\label{ssec:future}\index[sub]{future}
The future tense suffixes in \SYQ{} are \phono{-shaq} (1\lsc{pl}), \phono{-nki}~(\ref{Glo4:michimu}), \phono{-nqa}~(\ref{Glo4:Vakatash}), and \phono{-shun} (1\lsc{s})~(\ref{Glo4:iskapa}--\ref{Glo4:Kaytatrpaqa}).\\

% 1
\gloexe{Glo4:iskapa}{}{amv}%
{Manam iskapa\pb{nqa}chu. Wañurachi\pb{shaq}mi.}%amv que first line
{\morglo{mana-m}{no-\lsc{evd}}\morglo{iskapa-nqa-chu}{escape-\lsc{3.fut}-\lsc{neg}}\morglo{wañu-ra-chi-shaq-mi}{die-\lsc{urgt}-\lsc{caus}-\lsc{1.fut}-\lsc{evd}}}%morpheme+gloss
\glotran{\pb{She’s} not \pb{going to} escape. \pb{I’ll} kill her.}{}%eng+spa trans
{}{}%rec - time

% 2
\gloexe{Glo4:michimu}{}{sp}%
{Ubiha:ta michimu\pb{shaq} vaka:ta chawaru\pb{shaq} kisuta ruwaru\pb{shaq}.}%sp que first line
{\morglo{ubiha-:-ta}{sheep-\lsc{1}-\lsc{acc}}\morglo{michi-mu-shaq}{pasture-\lsc{cisl}-\lsc{1.fut}}\morglo{vaka-:-ta}{cow-\lsc{1}-\lsc{acc}}\morglo{chawa-ru-shaq}{milk-\lsc{urgt}-\lsc{1.fut}}\morglo{kisu-ta}{cheese-\lsc{acc}}\morglo{ruwa-ru-shaq}{make-\lsc{urgt}-\lsc{1.fut}}}%morpheme+gloss
\glotran{\pb{I’m going to} herd my sheep; \pb{I’m going to} milk my cows; \pb{I’m going to} make cheese.}{}%eng+spa trans
{}{}%rec - time

% 3
\gloexe{Glo4:Vakatash}{}{amv}%
{Vakatash harka\pb{nki} vakata chawa\pb{nki}.}%amv que first line
{\morglo{vaka-ta-sh}{cow-\lsc{acc}-\lsc{evr}}\morglo{harka-nki}{herd-\lsc{2}}\morglo{vaka-ta}{cow-\lsc{acc}}\morglo{chawa-nki}{milk-\lsc{2}}}%morpheme+gloss
\glotran{\pb{You’ll} herd the cows; \pb{you’ll} milk the cows.}{}%eng+spa trans
{}{}%rec - time

% 4
\gloexe{Glo4:Rupari}{}{amv}%
{Rupari\pb{nqa}tr.}%amv que first line
{\morglo{rupa-ri-nqa-tr}{burn-\lsc{incep}-\lsc{3.fut}-\lsc{evc}}}%morpheme+gloss
\glotran{\pb{It will} be warm [tomorrow].}{}%eng+spa trans
{}{}%rec - time

% 5
\gloexe{Glo4:Shimikita}{}{sp}%
{Shimikita siraru\pb{shun}.}%sp que first line
{\morglo{shimi-ki-ta}{mouth-\lsc{2}-\lsc{acc}}\morglo{sira-ru-shun}{sew-\lsc{urgt}-\lsc{1pl.fut}}}%morpheme+gloss
\glotran{\pb{We’re going to} sew your mouth shut.}{}%eng+spa trans
{}{}%rec - time

% 6
\gloexe{Glo4:Kaytatrpaqa}{}{amv}%
{Kaytatr paqariku\pb{shun}.}%amv que first line
{\morglo{kay-ta-tr}{\lsc{dem.p}-\lsc{acc}-\lsc{evc}}\morglo{paqa-ri-ku-shun}{wash-\lsc{incep}-\lsc{refl}-\lsc{1pl.fut}}}%morpheme+gloss
\glotran{\pb{We’ll} wash this.}{}%eng+spa trans
{}{}%rec - time

\noindent
The second person suffix is ambiguous between present and future tense. Second person and third person plural suffixes are the same as those for the second and third persons singular~(\ref{Glo4:Qamkunallam}--\ref{Glo4:Manalaq}).\\

% 7
\gloexe{Glo4:Qamkunallam}{}{ch}%
{Qamkunallam parla\pb{nki}.}%ch que first line
{\morglo{qam-kuna-lla-m}{you-\lsc{pl}-\lsc{rstr}-\lsc{evd}}\morglo{parla-nki}{talk-\lsc{2}}}%morpheme+gloss
\glotran{Just \pb{you.\lsc{pl}} \pb{are going to} talk.}{}%eng+spa trans
{}{}%rec - time

% 8
\gloexe{Glo4:mamaykis}{}{ach}%
{Qampa mamaykis taytaykis wañuku\pb{nqa} turikipis ñañaykipis.}%ach que first line
{\morglo{qam-pa}{you-\lsc{gen}}\morglo{mama-yki-s}{mother-\lsc{2}-\lsc{add}}\morglo{tayta-yki-s}{father-\lsc{2}-\lsc{add}}\morglo{wañu-ku-nqa}{die-\lsc{refl}-\lsc{3.fut}}\morglo{turi-ki-pis}{brother-\lsc{2}-\lsc{add}}\morglo{ñaña-yki-pis}{sister-\lsc{2}-\lsc{add}}}%morpheme+gloss
\glotran{\pb{Your mother and father will} die, your brother and your sister, too.}{}%eng+spa trans
{}{}%rec - time

% 9
\gloexe{Glo4:Manalaq}{}{ch}%
{Manalaq yakukta quma\pb{nqa}chu.}%ch que first line
{\morglo{mana-laq}{no-\lsc{cont}}\morglo{yaku-kta}{water-\lsc{acc}}\morglo{qu-ma-nqa-chu}{give-\lsc{1.obj}-\lsc{3.fut}-\lsc{neg}}}%morpheme+gloss
\glotran{\pb{They} still \pb{aren’t going to} give me water.}{}%eng+spa trans
{}{}%rec - time

% TABLE 17a
\begin{table}[!ht]
\small\centering
\caption{Future tense inflection}\label{Tab17a}
\begin{tabular}{lll}
\lsptoprule
Person		& Singular		& Plural	\\
\midrule
1			&-shaq			&-shun		\\[2ex]
%\midrule
2			&-nki			&-nki		\\[2ex]
%\midrule
3			&-nqa			&-nqa		\\
\lspbottomrule
\end{tabular}
\end{table}

% TABLE 17b
\begin{table}[!ht]
\small\centering
\caption{Future tense inflection -- actor-object suffixes}\label{Tab17b}
\begin{tabular}{lllll}
\lsptoprule
2>1	&	3>1	&	3>1pl	&	1>2	&	3>2	\\
\midrule
-wa-nki\tss{\AMV,\LT}	
&	-wa-nqa-\uo\tss{\AMV,\LT}	
&	-wa-shun\tss{\AMV,\LT}	
&	-sHQayki	
& -shunki\\
-ma-nki\tss{\ACH,\CH,\SP}	
&	-ma-nqa-\uo\tss{\ACH,\CH,\SP}	
&	-ma-shun\tss{\ACH,\CH,\SP}	
&	 	
& 	\\
\lspbottomrule
\end{tabular}
\end{table}

\subsubsection{Past}\label{ssec:past}\index[sub]{past}
\SYQ{} distinguishes between the simple past, the perfect, and the iterative past. The simple past is indicated by the past tense morpheme \phono{-RQa} (\phononb{rima-\pb{rqa/ra}-nchik} ‘we spoke’). In practice \phono{-RQa} is assigned both simple past and present perfect (non-completive) interpretations. The quotative simple past (\phono{-sHQa}) is used in story-telling (\phono{apa-mu\pb{-sa}-\uo} ‘she brought it’). The past tense (completive) is indicated by the suffix \phono{-sHa} (\phono{uyari\pb{-sa}-ni} ‘I heard’). The habitual past is indicated by the agentive noun --~formed by the suffixation of \phono{-q} to the verb stem~-- in combination with the relevant present-tense form of \phono{ka-} ‘be’ (\phono{taki\pb{-q}} \phono{\pb{ka-nki}} ‘you used to sing’). §~\ref{par:simplepast}--\ref{par:iterative} cover the simple past, the narrative past, the perfect, and the iterative past, in turn. The past conditional is covered in~§~\ref{ssec:altcond}.

\paragraph{Simple past \phono{-RQa}}\label{par:simplepast}\index[sub]{simple past}
\phono{-RQa} indicates the past tense.\footnote{\phono{-RQa} signals the preterite in all Quechuan languages; \phono{-RU}, according to \citet{CerroP87}\index[aut]{Cerrón-Palomino, Rodolfo M.}, is a later evolution in some Quechuan languages from the modal suffix \phono{-RQu} (outward direction). In Tarma Q and Pacaraos Q \phono{-rQu} is now a perfective aspect marker \citet[18--29]{Adelaar88}.\index[aut]{Adelaar, Willem F. H.} An anonymous reviewer points out that in Southern Conchucos Quechua, \phono{-ru} in Southern Conchucos Q originally indicated outward direction. It became a derivational perfective then an inflectional past \citep[see][192--197]{Hintz}.} The morpheme is realized \phono{-rqa} in \AMV{}~(\ref{Glo4:Iskwilanta}),~(\ref{Glo4:Imapaqtaqni}); \phono{-ra} in \ACH{}~(\ref{Glo4:ganawnintin}), \LT{}~(\ref{Glo4:Primitivoqa}),~(\ref{Glo4:pasaypaqtriki}), and \SP{}~(\ref{Glo4:Antaylumatata}); and \phono{-la} in \CH{}~(\ref{Glo4:Suwanakushpatr}),~(\ref{Glo4:Manachurimi}).\\

% 1
\gloexe{Glo4:Iskwilanta}{}{amv}%
{Iskwilanta lliwta ya wamrayta puchukachi\pb{rqani}.}%amv que first line
{\morglo{iskwila-n-ta}{school-\lsc{3}-\lsc{acc}}\morglo{lliw-ta}{all-\lsc{acc}}\morglo{ya}{\lsc{emph}}\morglo{wamra-y-ta}{child-\lsc{1}-\lsc{acc}}\morglo{puchuka-chi-rqa-ni}{finish-\lsc{caus}-\lsc{pst}-\lsc{1}}}%morpheme+gloss
\glotran{\pb{I made} all my children finish their schooling.}{}%eng+spa trans
{}{}%rec - time

% 2
\gloexe{Glo4:Imapaqtaqni}{}{amv}%
{¿Imapaqtaq niwa\pb{rqa}\pb{nki}? ¡Pagarullawanmantri karqa!}%amv que first line
{\morglo{ima-paq-taq}{what-\lsc{purp}-\lsc{seq}}\morglo{ni-wa-rqa-nki}{say-\lsc{1.obj}-\lsc{pst}-\lsc{2}}\morglo{paga-ru-lla-wa-n-man-tri}{pay-\lsc{urgt}-\lsc{rstr}-\lsc{1.obj}-\lsc{3}-\lsc{cond}-\lsc{evc}}\morglo{ka-rqa}{be-\lsc{pst}}}%morpheme+gloss
\glotran{Why \pb{did you say} that to me? He would have sacrificed me!}{}%eng+spa trans
{}{}%rec - time

% 3
\gloexe{Glo4:ganawnintin}{}{ach}%
{Kutikamu\pb{ra:} lliw ganawnintin wamra: lliw listu hishpiruptinña.}%ach que first line
{\morglo{kuti-ka-mu-ra-:}{return-\lsc{refl}-\lsc{cisl}-\lsc{pst}-\lsc{1}}\morglo{lliw}{all}\morglo{ganaw-ni-ntin}{cattle-\lsc{euph}-\lsc{incl}}\morglo{wamra-:}{child-\lsc{1}}\morglo{lliw}{all}\morglo{listu}{ready}\morglo{hishpi-ru-pti-n-ña}{educate-\lsc{urgt}-\lsc{subds}-\lsc{3}-\lsc{disc}}}%morpheme+gloss
\glotran{\pb{I came} back with all my cattle when my children had been educated.}{}%eng+spa trans
{}{}%rec - time

% 4
\gloexe{Glo4:Primitivoqa}{}{lt}%
{Kanan Primitivoqa ñuqa istankamu\pb{rani}.}%lt que first line
{\morglo{kanan}{now}\morglo{Primitivo-qa}{Primitovo-\lsc{top}}\morglo{ñuqa}{I}\morglo{istanka-mu-ra-ni}{fill.reservoir-\lsc{cisl}-\lsc{pst}-\lsc{1}}}%morpheme+gloss
\glotran{Now Primitivo [says] \pb{I filled} the reservoir.}{}%eng+spa trans
{}{}%rec - time

% 5
\gloexe{Glo4:pasaypaqtriki}{}{lt}%
{Qam pasaypaqtriki ri\pb{ranki} Diosninchikta tariq.}%lt que first line
{\morglo{qam}{you}\morglo{pasaypaq-tri-ki}{completely-\lsc{evc}-\lsc{iki}}\morglo{ri-ra-nki}{go-\lsc{pst}-\lsc{2}}\morglo{Dios-ni-nchik-ta}{God-\lsc{euph}-\lsc{1pl}-\lsc{acc}}\morglo{tari-q}{find-\lsc{ag}}}%morpheme+gloss
\glotran{\pb{You} surely \pb{went} to look for our God.}{}%eng+spa trans
{}{}%rec - time

% 6
\gloexe{Glo4:Antaylumatata}{}{sp}%
{Antaylumata tarirushpaqa pallakulla\pb{ra} hinaptinshi.}%sp que first line
{\morglo{antayluma-ta}{antayluma.berry-\lsc{acc}}\morglo{tari-ru-shpa-qa}{find-\lsc{urgt}-\lsc{subis}-\lsc{top}}\morglo{palla-ku-lla-ra}{pick-\lsc{refl}-\lsc{rstr}-\lsc{pst}}\morglo{hinaptin-shi}{then-\lsc{evr}}}%morpheme+gloss
\glotran{When she found the antayluma berries, \pb{she picked} them then, they say.}{}%eng+spa trans
{}{}%rec - time

% 7
\gloexe{Glo4:Suwanakushpatr}{}{ch}%
{Suwanakushpatr lluqsi\pb{la}.}%ch que first line
{\morglo{suwa-naku-shpa-tr}{steal-\lsc{recip}-\lsc{subis}-\lsc{evc}}\morglo{lluqsi-la}{go.out-\lsc{pst}}}%morpheme+gloss
\glotran{\pb{They left} eloping.}{}%eng+spa trans
{}{}%rec - time

% 8
\gloexe{Glo4:Manachurimi}{}{ch}%
{¿Manachu rimidyukta apakamu\pb{la}nki?}%ch que first line
{\morglo{mana-chu}{no-\lsc{q}}\morglo{rimidyu-kta}{remedy-\lsc{acc}}\morglo{apa-ka-mu-la-nki}{bring-\lsc{passacc}-\lsc{cisl}-\lsc{pst}-\lsc{2}}}%morpheme+gloss
\glotran{\pb{You didn’t bring} any medicine?}{}%eng+spa trans
{}{}%rec - time

\noindent
In all five dialects, person-number inflection in the past tense is as in the present tense, with the exception that in the third person, \phono{-n} is replaced by \phono{-\uo}~(\ref{Glo4:Llaqtaykipa}),~(\ref{Glo4:tirruku}).\\

% 9
\gloexe{Glo4:Llaqtaykipa}{}{amv}%
{¿Llaqtaykipa pasa\pb{rqa}chu?}%amv que first line
{\morglo{llaqta-yki-pa}{town-\lsc{2}-\lsc{loc}}\morglo{pasa-rqa-chu}{pass-\lsc{pst}-\lsc{q}}}%morpheme+gloss
\glotran{\pb{Did} [the earthquake] \pb{go} through your town?}{}%eng+spa trans
{}{}%rec - time

% 10
\gloexe{Glo4:tirruku}{}{ch}%
{Unaymi chayna puli\pb{la\uo} chay tirruku. Awturidadkunakta ashushpa wañuchiyta muna\pb{la}.}%ch que first line
{\morglo{unay-mi}{before-\lsc{evd}}\morglo{chayna}{thus}\morglo{puli-la}{walk-\lsc{pst}}\morglo{chay}{\lsc{dem.d}}\morglo{tirruku}{Shining.Path}\morglo{awturidad-kuna-kta}{authority-\lsc{pl}-\lsc{acc}}\morglo{ashu-shpa}{approach-\lsc{subis}}\morglo{wañu-chi-y-ta}{die-\lsc{caus}-\lsc{inf}-\lsc{acc}}\morglo{muna-la}{want-\lsc{pst}}}%morpheme+gloss
\glotran{The Shining Path \pb{walked} about like that. They \pb{approached} the officials. They \pb{wanted} to kill them.}{}%eng+spa trans
{}{}%rec - time

\noindent
In all five dialects, \phono{-RQa} indicates tense but not aspect and is thus consistent with both perfective~(\ref{Glo4:Alliallitayari}) and imperfective aspect~(\ref{Glo4:talpu}--\ref{Glo4:Ripukuytam}).\\

% 11
\gloexe{Glo4:Alliallitayari}{}{lt}%
{Alliallitayari lucha\pb{ra}nchik wak hurquruptinqa.}%lt que first line
{\morglo{alli-alli-ta-ya-ri}{good-good-\lsc{acc}-\lsc{emph}-\lsc{ari}}\morglo{lucha-ra-nchik}{fight-\lsc{pst}-\lsc{1pl}}\morglo{wak}{\lsc{dem.d}}\morglo{hurqu-ru-pti-n-qa}{remove-\lsc{urgt}-\lsc{subds}-\lsc{3}-\lsc{top}}}%morpheme+gloss
\glotran{We \pb{fought} really well when they took that out.}{}%eng+spa trans
{}{}%rec - time

% 12
\gloexe{Glo4:talpu}{}{ch}%
{Manam ñuqakunaqa talpu\pb{la}:chu.}%ch que first line
{\morglo{mana-m}{no-\lsc{evd}}\morglo{ñuqa-kuna-qa}{I-\lsc{pl}-\lsc{top}}\morglo{talpu-la-:-chu}{plant-\lsc{pst}-\lsc{1}-\lsc{neg}}}%morpheme+gloss
\glotran{We \pb{haven’t} planted.}{}%eng+spa trans
{}{}%rec - time

% 13
\gloexe{Glo4:Chayllatam}{}{amv}%
{Chayllatam tumachi\pb{rqa}ni. Manam iksisti\pb{rqa}chu chay rantiypaq kay Viñacpaqa wak Gloria.}%amv que first line
{\morglo{chay-lla-ta-m}{\lsc{dem.d}-\lsc{rstr}-\lsc{acc}-\lsc{evd}}\morglo{tuma-chi-rqa-ni}{drink-\lsc{caus}-\lsc{pst}-\lsc{1}}\morglo{mana-m}{no-\lsc{evd}}\morglo{iksisti-rqa-chu}{exist-\lsc{pst}-\lsc{neg}}\morglo{chay}{\lsc{dem.d}}\morglo{ranti-y-paq}{sell-\lsc{inf}-\lsc{abl}}\morglo{kay}{\lsc{dem.p}}\morglo{Viñac-pa-qa}{Viñac-\lsc{loc}-\lsc{top}}\morglo{wak}{\lsc{dem.d}}\morglo{Gloria}{Gloria}}%morpheme+gloss
\glotran{I \pb{fed} them only goat milk and cheese. Gloria, milk for sale, \pb{didn’t exist} here in Viñac.}{}%eng+spa trans
{}{}%rec - time

% 14
\gloexe{Glo4:chunyaku}{}{ch}%
{Chay limpu limpu chunyaku\pb{la}nchik ayvis.}%ch que first line
{\morglo{chay}{\lsc{dem.d}}\morglo{limpu}{all}\morglo{limpu}{all}\morglo{chunya-ku-la-nchik}{silent-\lsc{refl}-\lsc{pst}-\lsc{1pl}}\morglo{ayvis}{sometimes}}%morpheme+gloss
\glotran{But we \pb{were} completely \pb{silent} here sometimes.}{}%eng+spa trans
{}{}%rec - time

% 15
\gloexe{Glo4:Ripukuytam}{}{amv}%
{Ripukuytam muna\pb{rqa}nchik.}%amv que first line
{\morglo{ripu-ku-y-ta-m}{go-\lsc{refl}-\lsc{inf}-\lsc{acc}-\lsc{evd}}\morglo{muna-rqa-nchik}{want-\lsc{pst}-\lsc{1pl}}}%morpheme+gloss
\glotran{We \pb{wanted} to run away.}{}%eng+spa trans
{}{}%rec - time

\noindent
Perfective aspect is, rather, indicated by the derivational suffix \phono{-RU}~(\ref{Glo4:kasarashpa}--\ref{Glo4:talilushpaqa}).\\

% 16
\gloexe{Glo4:kasarashpa}{}{amv}%
{Uyqa, chayta kasarashpa puchuka\pb{ru}nchik.}%amv que first line
{\morglo{uyqa}{sheep}\morglo{chay-ta}{\lsc{dem.d}-\lsc{acc}}\morglo{kasara-shpa}{marry-\lsc{subis}}\morglo{puchuka-ru-nchik}{finish-\lsc{urgt}-\lsc{1pl}}}%morpheme+gloss
\glotran{When we got married, we \pb{finished} with those, the sheep.}{}%eng+spa trans
{}{}%rec - time

% 17
\gloexe{Glo4:wawanta}{}{amv}%
{Wak runaqa wawanta pampa\pb{ru}n qipichaykushpam.}%amv que first line
{\morglo{wak}{\lsc{dem.d}}\morglo{runa-qa}{person-\lsc{top}}\morglo{wawa-n-ta}{baby-\lsc{3}-\lsc{acc}}\morglo{pampa-ru-n}{bury-\lsc{urgt}-\lsc{3}}\morglo{qipi-cha-yku-shpa-m}{carry-\lsc{dim}-\lsc{refl}-\lsc{subis}-\lsc{evd}}}%morpheme+gloss
\glotran{The people \pb{buried} their son, carrying him.}{}%eng+spa trans
{}{}%rec - time

% 18
\gloexe{Glo4:Yaqam}{}{ach}%
{Yaqam wañu\pb{ru}n.}%ach que first line
{\morglo{yaqa-m}{almost-\lsc{evd}}\morglo{wañu-ru-n}{die-\lsc{urgt}-\lsc{3}}}%morpheme+gloss
\glotran{He almost \pb{died}.}{}%eng+spa trans
{}{}%rec - time

% 19
\gloexe{Glo4:Pusuman}{}{lt}%
{Pusuman hiqayku\pb{ru}ni. kaypaq urayman.}%lt que first line
{\morglo{pusu-man}{reservoir-\lsc{all}}\morglo{hiqa-yku-ru-ni}{go.down-\lsc{excep}-\lsc{urgt}-\lsc{1}}\morglo{kay-paq}{\lsc{dem.p}-\lsc{abl}}\morglo{uray-man}{down.hill-\lsc{all}}}%morpheme+gloss
\glotran{I \pb{fell} towards the reservoir. From here down hill.}{}%eng+spa trans
{}{}%rec - time

% 20
\gloexe{Glo4:uywaqkunaman}{}{sp}%
{Mana ganaw uywaqkunaman chayman partiku\pb{ru}n.}%sp que first line
{\morglo{mana}{no}\morglo{ganaw}{cattle}\morglo{uywa-q-kuna-man}{raise-\lsc{ag}-\lsc{pl}-\lsc{all}}\morglo{chay-man}{\lsc{dem.d}-\lsc{all}}\morglo{parti-ku-ru-n}{divide-\lsc{refl}-\lsc{urgt}-\lsc{3}}}%morpheme+gloss
\glotran{They \pb{distributed} it to those who don’t raise cattle.}{}%eng+spa trans
{}{}%rec - time

% 21
\gloexe{Glo4:Disparisi}{}{sp}%
{Disparisi\pb{ru}nñam. Manam uyari:chu.}%sp que first line
{\morglo{disparisi-ru-n-ña-m}{disappear-\lsc{urgt}-\lsc{3}-\lsc{disc}-\lsc{evd}}\morglo{mana-m}{no-\lsc{evd}}\morglo{uyari-:-chu}{hear-\lsc{1}-\lsc{neg}}}%morpheme+gloss
\glotran{They \pb{disappeared} already. I don’t hear them [anymore].}{}%eng+spa trans
{}{}%rec - time

% 22
\gloexe{Glo4:talilushpaqa}{}{ch}%
{Chay walmita talilushpaqa apa\pb{lu}nñam uspitalman.}%ch que first line
{\morglo{chay}{\lsc{dem.d}}\morglo{walmi-ta}{woman-\lsc{acc}}\morglo{tali-lu-shpa-qa}{find-\lsc{urgt}-\lsc{subis}-\lsc{top}}\morglo{apa-lu-n-ña-m}{bring-\lsc{urgt}-\lsc{3}-\lsc{disc}-\lsc{evd}}\morglo{uspital-man}{hospital-\lsc{all}}}%morpheme+gloss
\glotran{\pb{When they found} the woman they took her to the hospital.}{}%eng+spa trans
{}{}%rec - time

\noindent
\phono{-rQa} and \phono{-Ru} are thus not in paradigmatic opposition and differ in their distribution. \phono{-RQa,} but not \phono{-Ru}, is used in the construction of the habitual past~(\ref{Glo4:Dumingunpa}), (\ref{Glo4:manchachiku}) and the past conditional~(\ref{Glo4:Imapaqtaqni}),~(\ref{Glo4:Kundinakurun}); while \phono{-Ru}, but not \phono{-RQa}, may be used in combination with \phono{-sHa}~(\ref{Glo4:ayarikura}),~(\ref{Glo4:Chayllapaq}) as well as with \phono{-shpa}~(\ref{Glo4:Antaylumatata}), (\ref{Glo4:talilushpaqa}) and \phono{-pti}~(\ref{Glo4:ganawnintin}), (\ref{Glo4:trayarun}), (\ref{Glo4:shinkaqqa}), in which case it indicates the precedence of the subordinated event to the main-clause event.\\

% 23
\gloexe{Glo4:Dumingunpa}{}{ach}%
{Dumingunpa kisuta \pb{apaq kara:} (*karu:) ishkay.}%ach que first line
{\morglo{dumingu-n-pa}{Sunday-\lsc{3}-\lsc{loc}}\morglo{kisu-ta}{cheese-\lsc{acc}}\morglo{apa-q}{bring-\lsc{ag}}\morglo{ka-ra-:}{be-\lsc{pst}-\lsc{1}}\morglo{ishkay}{two}}%morpheme+gloss
\glotran{On Sundays, \pb{I would bring} two cheeses.}{}%eng+spa trans
{}{}%rec - time

% 24
\gloexe{Glo4:manchachiku}{}{ch}%
{Trayamushpa manchachiku\pb{q} ka\pb{la}.}%ch que first line
{\morglo{traya-mu-shpa}{arrive-\lsc{cisl}-\lsc{subis}}\morglo{mancha-chi-ku-q}{scare-\lsc{caus}-\lsc{refl}-\lsc{ag}}\morglo{ka-la}{be-\lsc{pst}}}%morpheme+gloss
\glotran{When she came, \pb{she would scare} them.}{}%eng+spa trans
{}{}%rec - time

% 25
\gloexe{Glo4:Kundinakurun}{}{sp}%
{Kundinakurun\pb{man}tri ka\pb{ra} (*karun) qullqi chay kasa.}%sp que first line
{\morglo{kundina-ku-ru-n-man-tri}{condemn-\lsc{refl}-\lsc{urgt}-\lsc{3}-\lsc{cond}-\lsc{evc}}\morglo{ka-ra}{be-\lsc{pst}}\morglo{qullqi}{money}\morglo{chay}{\lsc{dem.d}}\morglo{ka-sa}{be-\lsc{npst}}}%morpheme+gloss
\glotran{She \pb{would have condemned} herself -- that was money.}{}%eng+spa trans
{}{}%rec - time

% 26
\gloexe{Glo4:ayarikura}{}{ach}%
{Cañeteta ayarikura:. Ispusu:ta listaman trura\pb{rusa} (*trurarqasa, *trurasarqa).}%ach que first line
{\morglo{Cañete-ta}{Cañete-\lsc{acc}}\morglo{ayari-ku-ra-:}{escape-\lsc{refl}-\lsc{pst}-\lsc{1}}\morglo{ispusu-:-ta}{husband-\lsc{1}-\lsc{acc}}\morglo{lista-man}{list-\lsc{all}}\morglo{trura-ru-sa}{put-\lsc{urgt}-\lsc{npst}}}%morpheme+gloss
\glotran{I escaped to Cañete. They \pb{had put} my husband on the list.}{}%eng+spa trans
{}{}%rec - time

% 27
\gloexe{Glo4:Chayllapaq}{}{ach}%
{Chayllapaq willaka\pb{ru}sa. (*willakarqasa).}%ach que first line
{\morglo{chay-lla-paq}{\lsc{dem.d}-\lsc{rstr}-\lsc{abl}}\morglo{willa-ka-ru-sa}{tell-\lsc{passacc}-\lsc{urgt}-\lsc{npst}}}%morpheme+gloss
\glotran{That’s why \pb{they had told} on him.}{}%eng+spa trans
{}{}%rec - time

% 28
\gloexe{Glo4:trayarun}{}{sp}%
{Chay hawla\pb{rupti}nshi, atuq trayarun (*hawlaraptin).}%sp que first line
{\morglo{chay}{\lsc{dem.d}}\morglo{hawla-ru-pti-n-shi}{cage-\lsc{urgt}-\lsc{subds}-\lsc{3}-\lsc{evr}}\morglo{atuq}{fox}\morglo{traya-ru-n}{arrive-\lsc{urgt}-\lsc{3}}}%morpheme+gloss
\glotran{\pb{When he had caged} [the rabbit], the fox arrived.}{}%eng+spa trans
{}{}%rec - time

% 29
\gloexe{Glo4:shinkaqqa}{}{ach}%
{Chay mulapaq siqayku\pb{rupti}n puñukuratrik shinkaqqa.}%ach que first line
{\morglo{chay}{\lsc{dem.d}}\morglo{mula-paq}{mule-\lsc{abl}}\morglo{siqa-yku-ru-pti-n}{go.down-\lsc{excep}-\lsc{urgt}-\lsc{subds}-\lsc{3}}\morglo{puñu-ku-ra-tri-k}{sleep-\lsc{refl}-\lsc{pst}-\lsc{evc}-\lsc{ik}}\morglo{shinka-q-qa}{get.drunk-\lsc{ag}-\lsc{top}}}%morpheme+gloss
\glotran{\pb{When he fell} off that mule, the drunk must have been asleep.}{}%eng+spa trans
{}{}%rec - time

% TABLE 18
\begin{table}[!ht]
\small\centering
\caption{Past tense inflection}\label{Tab18}
\begin{tabular}{lll}
\lsptoprule
Person		& Singular		& Plural			\\
\midrule
1	& -rqa-ni\tss{\AMV}		& -rqa-nchik\tss{\AMV}		\\
	& -ra-ni\tss{\LT}		& -ra-nchik\tss{\ACH,\SP,\LT}	\\
	& -ra-:\tss{\ACH,\SP}		& -la-nchik\tss{\CH}		\\
	& -la-:\tss{\CH}		&	\\[2ex]
%\midrule
2	& -rqa-nki\tss{\AMV}		& -rqa-nki\tss{\AMV}		\\
	& -ra-nki\tss{\ACH,\SP,\LT}	& -ra-nki\tss{\ACH,\SP,\LT}	\\
	& -la-nki\tss{\CH}		& -la-nki\tss{\CH}		\\[2ex]
%\midrule
3	& -rqa-\uo\tss{\AMV}		& -rqa-\uo\tss{\AMV}		\\
	& -ra-\uo\tss{\ACH,\SP,\LT}	& -ra-\uo\tss{\ACH,\SP,\LT}	\\
	& -la-\uo\tss{\CH}		& -la-\uo\tss{\CH}		\\
\lspbottomrule
\end{tabular}
\end{table}

% TABLE 18b

\begin{table}[!ht]
\small\centering
\caption{Past tense inflection -- actor-object suffixes}\label{Tab18b}
\resizebox{\textwidth}{!}{%
\begin{tabular}{l@{\hspace{2ex}}l@{\hspace{2ex}}l@{\hspace{2ex}}l@{\hspace{2ex}}l}
\lsptoprule
2>1	&	3>1	&	3>1pl	&	1>2	&	3>2	\\
\midrule
-wa-rqa-nki\tss{\AMV}	&	-wa-rqa-\uo\tss{\AMV}	&	-wa-rqa-nchik\tss{\AMV}	&	-rqa-yki\tss{\AMV}	&	-shu-rqa-nki\tss{\AMV}\\
-wa-ra-nki\tss{\LT}	&	-wa-ra-\uo\tss{\LT}	&	-wa-ra-nchik\tss{\LT}	&	-ra-yki\tss{\LT,\ACH,\SP}	&	-shu-ra-nki\tss{\LT,\ACH,\SP}	\\
-ma-ra-nki\tss{\ACH,\SP} &	-ma-ra-\uo\tss{\ACH,\SP}	& -ma-ra-nchik\tss{\ACH,\SP}	& 	& \\
-ma-la-nki\tss{\CH}	&	-ma-la-\uo\tss{\CH}	&	-ma-la-nchik\tss{\CH}	&	-la-yki\tss{\CH}	&	-shu-la-nki\tss{\CH}	\\
\lspbottomrule
\end{tabular}}
\end{table}

\paragraph{Quotative simple past tense \phono{-sHQa}}\label{par:QSPT}\index[sub]{simple past!quotative tense}
In \SYQ, as in other Quechuan languages, when speakers have only second-hand knowledge of the events they report, they may recur to a another past tense form, \phono{-sHQa}, often referred to as the “narrative past” because it is used systematically in story-telling. In \SYQ, \phono{-sHQa} --~realized as \phono{-sa} in \ACH, \AMV{} and \SP{} and as \phono{-sha} in \CH{} and \LT~-- is used predominantly in story-telling~(\ref{Glo4:Huklla}), (\ref{Glo4:maqtatukushpa}), historical narrative~(\ref{Glo4:miniruwanshi}--\ref{Glo4:Tariramu}), and, generally, in relating information one has received from others~(\ref{Glo4:Matalo}--\ref{Glo4:aychata}).\\

% 1
\gloexe{Glo4:Huklla}{}{sp}%
{Huklla atuqshi ka\pb{sa}.}%sp que first line
{\morglo{huk-lla}{one-\lsc{rstr}}\morglo{atuq-shi}{fox-\lsc{evr}}\morglo{ka-sa}{be-\lsc{npst}}}%morpheme+gloss
\glotran{[Once upon a time] there \pb{was} a fox, they say.}{}%eng+spa trans
{}{}%rec - time

% 2
\gloexe{Glo4:maqtatukushpa}{}{amv}%
{Chay ukucha ka\pb{sa} maqtatukushpa.}%amv que first line
{\morglo{chay}{\lsc{dem.d}}\morglo{ukucha}{mouse}\morglo{ka-sa}{be-\lsc{npst}}\morglo{maqta-tuku-shpa}{young.man-\lsc{simul}-\lsc{subis}}}%morpheme+gloss
\glotran{\pb{It was} a rat pretending to be a man.}{}%eng+spa trans
{}{}%rec - time

% 3
\gloexe{Glo4:miniruwanshi}{}{ach}%
{Hinashpa qalay qalay Chavin miniruwanshi parti\pb{sa}.}%ach que first line
{\morglo{hinashpa}{then}\morglo{qalay}{all}\morglo{qalay}{all}\morglo{Chavin}{Chavin}\morglo{miniru-wan-shi}{miner-\lsc{instr}-\lsc{evr}}\morglo{parti-sa}{divide-\lsc{npst}}}%morpheme+gloss
\glotran{Then they divid\pb{ed} everything up with the Chavin miners.}{}%eng+spa trans
{}{}%rec - time

% 4
\gloexe{Glo4:intanadanqa}{}{ach}%
{Chay intanadanqa ayqiku\pb{sa}.}%ach que first line
{\morglo{chay}{\lsc{dem.d}}\morglo{intanada-n-qa}{step.daughter-\lsc{3}-\lsc{top}}\morglo{ayqi-ku-sa}{escape-\lsc{refl}-\lsc{npst}}}%morpheme+gloss
\glotran{His step-daughter escap\pb{ed}.}{}%eng+spa trans
{}{}%rec - time

% 5
\gloexe{Glo4:Tariramu}{}{lt}%
{Tariramu\pb{sha} armata.}%lt que first line
{\morglo{tari-ra-mu-sha}{find-\lsc{urgt}-\lsc{cisl}-\lsc{npst}}\morglo{arma-ta}{weapon-\lsc{acc}}}%morpheme+gloss
\glotran{They \pb{found} firearms.}{}%eng+spa trans
{}{}%rec - time

% 6
\gloexe{Glo4:Matalo}{}{ch}%
{“¡Mátalo!” ni\pb{sha}shiki.}%ch que first line
{\morglo{mátalo}{{}[Spanish]}\morglo{ni-sha-shi-ki}{say-\lsc{npst}-\lsc{evr}-\lsc{iki}}}%morpheme+gloss
\glotran{“Kill him!” she \pb{said}, they say.}{}%eng+spa trans
{}{}%rec - time

% 7
\gloexe{Glo4:disiyananpaq}{}{amv}%
{Wañukachishpash qipiru\pb{sa} karuta mana disiyananpaq.}%amv que first line
{\morglo{wañu-ka-chi-shpa-sh}{die-\lsc{passacc}-\lsc{caus}-\lsc{subis}-\lsc{evr}}\morglo{qipi-ru-sa}{carry-\lsc{urgt}-\lsc{npst}}\morglo{karu-ta}{far-\lsc{acc}}\morglo{mana}{no}\morglo{disya-na-n-paq}{suspect-\lsc{nmlz}-\lsc{3}-\lsc{purp}}}%morpheme+gloss
\glotran{When she kill\pb{ed} him, they say, she carri\pb{ed} him far, so they wouldn’t suspect.}{}%eng+spa trans
{}{}%rec - time

% 8
\gloexe{Glo4:warmiqall}{}{amv}%
{Wak warmiqa llaman qutuq ri\pb{sa}. Mayuta pawayashpash siqaykuru\pb{sa}; karu karutash aparu\pb{sa}.}%amv que first line
{\morglo{wak}{\lsc{dem.d}}\morglo{warmi-qa}{woman-\lsc{top}}\morglo{llama-n}{llama-\lsc{3}}\morglo{qutu-q}{gather-\lsc{ag}}\morglo{ri-sa}{go-\lsc{pst}}\morglo{mayu-ta}{river-\lsc{acc}}\morglo{pawa-ya-shpa-sh}{jump-\lsc{prog}-\lsc{subis}-\lsc{evr}}\morglo{siqa-yku-ru-sa}{go.down-\lsc{excep}-\lsc{urgt}-\lsc{npst}}\morglo{karu}{far}\morglo{karu-ta-sh}{far-\lsc{acc}-\lsc{evr}}\morglo{apa-ru-sa}{bring-\lsc{urgt}-\lsc{npst}}}%morpheme+gloss
\glotran{That woman \pb{went} to gather up her llamas. Jumping the river, she \pb{fell} and [the river] \pb{took} her far, they say.}{}%eng+spa trans
{}{}%rec - time

% 9
\gloexe{Glo4:Fiystaman}{}{ach}%
{Fiystaman hamushpa siqaykuru\pb{sha}.}%ach que first line
{\morglo{fiysta-man}{festival-\lsc{all}}\morglo{hamu-shpa}{come-\lsc{subis}}\morglo{siqa-yku-ru-sha}{go.down-\lsc{excep}-\lsc{urgt}-\lsc{npst}}}%morpheme+gloss
\glotran{When they were coming to the festival they \pb{fell} [into the canyon].}{}%eng+spa trans
{}{}%rec - time

% 10
\gloexe{Glo4:aychata}{}{amv}%
{Wak runaqa achka aychata aparamu\pb{sa} llama aycha\pb{sh} sibadawan kambyakunanpaq.}%amv que first line
{\morglo{wak}{\lsc{dem.d}}\morglo{runa-qa}{person-\lsc{top}}\morglo{achka}{a.lot}\morglo{aycha-ta}{meat-\lsc{acc}}\morglo{apa-ra-mu-sa}{bring-\lsc{urgt}-\lsc{cisl}-\lsc{npst}}\morglo{llama}{llama}\morglo{aycha-sh}{meat-\lsc{evr}}\morglo{sibada-wan}{barley-\lsc{instr}}\morglo{kambya-ku-na-n-paq}{exchange-\lsc{refl}-\lsc{nmlz}-\lsc{3}-\lsc{purp}}}%morpheme+gloss
\glotran{Those people \pb{brought} a lot of meat -- llama meat, they say, to exchange for barley.}{}%eng+spa trans
{}{}%rec - time

\noindent
It may also be used in dream reports~(\ref{Glo4:kuchihinam}).\\

% 11
\gloexe{Glo4:kuchihinam}{}{sp}%
{Lliw lliw kuchihinam mituman yaykuru\pb{sa}.}%sp que first line
{\morglo{lliw}{all}\morglo{lliw}{all}\morglo{kuchi-hina-m}{pig-\lsc{comp}-\lsc{evd}}\morglo{mitu-man}{mud-\lsc{all}}\morglo{yayku-ru-sa}{enter-\lsc{urgt}-\lsc{npst}}}%morpheme+gloss
\glotran{All, like pigs, \pb{entered} the mud.}{}%eng+spa trans
{}{}%rec - time

\noindent
The morpheme is realized as \phono{-shqa}, it seems, only in the first or culminating line of a story, and rarely even there~(\ref{Glo4:IshkayWa}).\\

% 12
\gloexe{Glo4:IshkayWa}{}{amv}%
{Ishkay Wanka samaku\pb{shqa} huk matraypi, tarukapa ka\pb{sa}npi. Wama wamaq karka kasa.}%amv que first line
{\morglo{ishkay}{two}\morglo{Wanka}{Wanka}\morglo{sama-ku-shqa}{rest-\lsc{refl}-\lsc{npst}}\morglo{huk}{one}\morglo{matray-pi,}{cave-\lsc{loc}}\morglo{taruka-pa}{taruka-\lsc{gen}}\morglo{ka-sa-n-pi}{be-\lsc{prf}-\lsc{3}-\lsc{loc}}\morglo{wama}{a.lot}\morglo{wamaq}{a.lot}\morglo{karka}{manure}\morglo{ka-sa}{be-\lsc{npst}}}%morpheme+gloss
\glotran{Two Huancayoans were resting in a cave, in some tarucas’ place. There was a whole lot of manure.}{}%eng+spa trans
{}{}%rec - time

\noindent
\phono{-RQa} and \phono{-Ru}, may also be employed in the same contexts as is \phono{-sHQa}, even in combination with the reportative evidential, \phono{-shI}~(\ref{Glo4:Rutupis}), (\ref{Glo4:Millisunqa}).\\

% 13
\gloexe{Glo4:Rutupis}{}{amv}%
{Rutupis ingaña\pb{rqash} maqtatukushpa pashñata.}%amv que first line
{\morglo{rutu-pis}{rutu.bird-\lsc{add}}\morglo{ingaña-rqa-sh}{trick-\lsc{pst}-\lsc{evr}}\morglo{maqta-tuku-shpa}{young.man-\lsc{simul}-\lsc{subis}}\morglo{pashña-ta}{girl-\lsc{acc}}}%morpheme+gloss
\glotran{A rutu-bird, too, \pb{deceived} a girl by making himself out to be a young man, \pb{they say}.}{}%eng+spa trans
{}{}%rec - time

% 14
\gloexe{Glo4:Millisunqa}{}{amv}%
{Millisunqa wañuru\pb{rqash} huknin.}%amv que first line
{\morglo{millisu-n-qa}{twin-\lsc{3}-\lsc{top}}\morglo{wañu-ru-rqa-sh}{die-\lsc{urgt}-\lsc{pst}-\lsc{evr}}\morglo{huk-ni-n}{one-\lsc{euph}-\lsc{3}}}%morpheme+gloss
\glotran{His twin, the other one, \pb{died}, \pb{they say}.}{}%eng+spa trans
{}{}%rec - time

\noindent
Inside quotations in story-telling, \phono{RQa} and \phono{-Ru} are generally employed~(\ref{Glo4:nshari}), (\ref{Glo4:kundur}).\\

% 15
\gloexe{Glo4:nshari}{}{amv}%
{Traya\pb{ru}nshari, ‘¿Maymi chay warmiy?’}%amv que first line
{\morglo{traya-\pb{ru}-n-sh-ari,}{arrive-\lsc{urgt}-\lsc{evr}-\lsc{ari}}\morglo{may-mi}{where-\lsc{evd}}\morglo{chay}{\lsc{dem.d}}\morglo{warmi-y}{woman-\lsc{1}}}%morpheme+gloss
\glotran{The condor \pb{arrived}, they say, [and said], “Where is my wife?”}{}%eng+spa trans
{}{}%rec - time

% 16
\gloexe{Glo4:kundur}{}{amv}%
{Chaynam kundur qipiwa\pb{rqa} matrayta chaypi wawaku\pb{ru}ni.}%amv que first line
{\morglo{chayna-m}{thus-\lsc{evd}}\morglo{kundur}{condor}\morglo{qipi-wa-rqa}{carry-\lsc{1.obj}-\lsc{pst}}\morglo{matray-ta}{cave-\lsc{acc}}\morglo{chaypi}{\lsc{dem.d}-\lsc{loc}}\morglo{wawa-ku-ru-ni}{give.birth-\lsc{refl}-\lsc{urgt}-\lsc{1}}}%morpheme+gloss
\glotran{That condor \pb{carried} me like that to a cave and I \pb{gave birth} there.}{}%eng+spa trans
{}{}%rec - time

\paragraph{\phono{Perfect}}\label{par:perfect}\index[sub]{perfect}
\phono{-sHa} --~realized as \phono{-sa} in \ACH, \AMV{} and \SP{} and as \phono{-sha} in \CH{} and \LT~-- may be argued sometimes to admit interpretations cognate with the English perfect, indicating events beginning in the past and either continuing into the present or with effects continuing into the present~(\ref{Glo4:alkulta}--\ref{Glo4:MikuMiku}).\\

% 1
\gloexe{Glo4:alkulta}{}{amv}%
{Chay alkulta mana tapa\pb{sani}chu.}%amv que first line
{\morglo{chay}{\lsc{dem.d}}\morglo{alkul-ta}{alcohol-\lsc{acc}}\morglo{mana}{no}\morglo{tapa-sa-ni-chu}{cover-\lsc{sa}-\lsc{1}-\lsc{neg}}}%morpheme+gloss
\glotran{\pb{I haven’t} capped that alcohol.}{}%eng+spa trans
{}{}%rec - time

% 2
\gloexe{Glo4:Grasyusu}{}{amv}%
{Grasyusu ka\pb{sanki}.}%amv que first line
{\morglo{grasyusu}{funny}\morglo{ka-sa-nki}{be-\lsc{sa}-\lsc{2}}}%morpheme+gloss
\glotran{\pb{You’ve been} funny.}{}%eng+spa trans
{}{}%rec - time

% 3
\gloexe{Glo4:MikuMiku}{}{lt}%
{Miku\pb{sha}yari. Miku\pb{sha}yari.}%lt que first line
{\morglo{miku-sha-y-ari}{miku-\lsc{sha}-\lsc{emph}-\lsc{ari}}\morglo{miku-\pb{sha}-y-ari}{eat-\lsc{sha}-\lsc{emph}-\lsc{ari}}}%morpheme+gloss
\glotran{\pb{They’ve eaten} them, all right. \pb{They’ve eaten} them.}{}%eng+spa trans
{}{}%rec - time

\noindent
That said, the non-nominalizing instances of \phono{-sHa} in the corpus, almost without exception, have more readily-available interpretations as narrative pasts (see §~\ref{par:QSPT}) (\ref{Glo4:Mulankunawan}).\footnote{The corpus counts~1157 instances of \phono{-sHa}; a sample of~50 turned up no translation to the Spanish perfect.}\\

% 4
\gloexe{Glo4:Mulankunawan}{}{ach}%
{Mulankunawan kargarikushpa pasan wañurichishpa wak Chavin lawpash. Hinashpa qalay qalay Chavin miniruwanshi parti\pb{sa}.}%ach que first line
{\morglo{mula-n-kuna-wan}{mule-\lsc{3}-\lsc{pl}-\lsc{instr}}\morglo{karga-ri-ku-shpa}{carry-\lsc{incep}-\lsc{refl}-\lsc{subis}}\morglo{pasa-n}{pass-\lsc{3}}\morglo{wañu-ri-chi-shpa}{die-\lsc{incep}-\lsc{caus}-\lsc{subis}}\morglo{wak}{\lsc{dem.d}}\morglo{Chavin}{Chavin}\morglo{law-pa-sh}{side-\lsc{loc}-\lsc{evr}}\morglo{hinashpa}{then}\morglo{qalay}{all}\morglo{qalay}{all}\morglo{Chavin}{Chavin}\morglo{miniru-wan-shi}{miner-\lsc{instr}-\lsc{evr}}\morglo{parti-sa}{divide-\lsc{sa}}}%morpheme+gloss
\glotran{Carrying everything with their mules, they left, killing people over by Chavin, they say. Then they \pb{divided} up absolutely everything with the miners.}{}%eng+spa trans
{}{}%rec - time

\noindent
Indeed, speakers offer only simple past translations for verbs suffixed with \phono{-sHa}; perfect translations may be offered, rather, for \phono{-Rqa}, \phono{-RU} (very rarely), or the present\footnote{In elicitation sessions, speakers of \SYQ{} do interpret \phono{-ri} as indicating the present perfect; in a sample of 50 of the 353 instances of \phono{-Ri} in the corpus, however, only once did the speakers assign it a perfect interpretation (\spkr~1: \phono{Yapa-mi-k} \phono{kuti-nqa}, \phono{¿aw?} \spkr~2: \phono{Puchuka\pb{-ri}} \phono{-n-chu}. ‘She’s going to go back again, no?’ ‘She hasn’t finished yet.’)} (\ref{Glo4:Maypaqtaq}--\ref{Glo4:tarin}) (see~§~\ref{par:simplepast}).\footnote{The the translations in~(\ref{Glo4:alkulta}--\ref{Glo4:MikuMiku}) were proposed only to suggest possible perfect interpretations of sentences that, I argued, are better interpreted as narrative pasts.}\\

% 5
\gloexe{Glo4:Maypaqtaq}{}{lt}%
{‘¿Maypaqtaq \pb{suwamura}nki?’ nishpa.}%lt que first line
{\morglo{may-paq-taq}{where-\lsc{abl}-\lsc{seq}}\morglo{suwa-mu-ra-nki}{steal-\lsc{cisl}-\lsc{pst}-\lsc{2}}\morglo{ni-shpa}{say-\lsc{subis}}}%morpheme+gloss
\glotran{“Where \pb{have} \pb{you stolen} these from?” he said.}{}%eng+spa trans
{}{}%rec - time

% 6
\gloexe{Glo4:Kananqa}{}{sp}%
{Kananqa shimi:lla \pb{qacharu}n hat-hatun.}%sp que first line
{\morglo{kanan-qa}{now-\lsc{top}}\morglo{shimi-:-lla}{mouth-\lsc{1}-\lsc{rstr}}\morglo{qacha-ru-n}{rip-\lsc{urgt}-\lsc{3}}\morglo{hat-hatun}{big-big}}%morpheme+gloss
\glotran{Now my mouth \pb{has ripped} open wide.}{}%eng+spa trans
{}{}%rec - time

% 7
\gloexe{Glo4:tarin}{}{ach}%
{Ni pi \pb{qawan}chu ni pi \pb{tarin}chu.}%ach que first line
{\morglo{ni}{nor}\morglo{pi}{who}\morglo{qawa-n-chu}{see-\lsc{3}-\lsc{neg}}\morglo{ni}{nor}\morglo{pi}{who}\morglo{tari-n-chu}{find-\lsc{3}-\lsc{neg}}}%morpheme+gloss
\glotran{No one \pb{has seen} her and no one \pb{has found} her.}{}%eng+spa trans
{}{}%rec - time

\noindent
Speakers do consistently translate the combination of \phono{-RU} and \phono{-sHa} with the Spanish past perfect~(\ref{Glo4:liyunqa}--\ref{Glo4:Payllatam}); in Andean Spanish, however, this construction does not share the semantics of the Standard Spanish.\footnote{This construction generally can only awkwardly be translated as a past perfect in English, however.}\\

% 8
\gloexe{Glo4:liyunqa}{}{amv}%
{¡Wak suwa liyunqa ubihayta tumba\pb{rusa}!}%amv que first line
{\morglo{wak}{\lsc{dem.d}}\morglo{suwa}{thief}\morglo{liyun-qa}{lion-\lsc{top}}\morglo{ubiha-y-ta}{sheep-\lsc{1}-\lsc{acc}}\morglo{tumba-ru-sa}{knock.down-\lsc{urgt}-\lsc{sa}}}%morpheme+gloss
\glotran{That thieving puma \pb{had knocked off} my sheep!}{}%eng+spa trans
{}{}%rec - time

% 9
\gloexe{Glo4:Trakraymi}{}{amv}%
{Trakraymi tuñirun. Yakutam \pb{katraykurusa}.}%amv que first line
{\morglo{trakra-y-mi}{field-\lsc{1}-\lsc{evd}}\morglo{tuñi-ru-n}{crumble-\lsc{urgt}-\lsc{3}}\morglo{yaku-ta-m}{water-\lsc{acc}-\lsc{evd}}\morglo{katra-yku-ru-sa}{release	-\lsc{excep}-\lsc{urgt}-\lsc{sa}}}%morpheme+gloss
\glotran{My field washed away. They \pb{had released} water.}{}%eng+spa trans
{}{}%rec - time

% 10
\gloexe{Glo4:Payllatam}{}{ach}%
{Payllatam wañurachira runa~\dots{} \pb{hapirusa} karrupi.}%ach que first line
{\morglo{pay-lla-ta-m}{he-\lsc{rstr}-\lsc{acc}-\lsc{evd}}\morglo{wañu-ra-chi-ra}{die-\lsc{urgt}-\lsc{caus}-\lsc{pst}}\morglo{runa}{person}\morglo{hapi-ru-sa}{grab-\lsc{urgt}-\lsc{sa}}\morglo{karrupi}{car-\lsc{loc}}}%morpheme+gloss
\glotran{The people killed just him~\dots{} They \pb{had grabbed} him on the bus.}{}%eng+spa trans
{}{}%rec - time

\noindent
Given, however, the restrictions on the distribution of \phono{-RU-sHa} --~it inflects only for third person\footnote{The corpus counts 330 instances of \phono{-RU} (\phono{-\uo/-chi/-mu}) \phono{-sHa}; in only two cases is it not inflected for third person.} and it is not contentful either with stative verbs or with the copulative, \phono{ka-}~-- it is improbable that it that would constitute the language’s principal strategy for rendering the past perfect. Rather, to indicate the sequence of two completed events, speakers of \SYQ{} generally employ ether the subordinator \phono{-pti}~(\ref{Glo4:nqa}), (\ref{Glo4:Hinaptinshiis}) or a connective like \phono{hinashpa} or \phono{hinaptin}~(\ref{Glo4:Suyarusa}).\footnote{It has been suggested to me that an additional function of \phono{-sHa} might be to indicate ‘sudden discovery’ \citep{Adelaar77}\index[aut]{Adelaar, Willem F. H.} or surprise. That is, \phono{-sHa} might indicate the mirative, as do the perfect marker \phono{-shka} in Ecuadorian Q \citep{muysken1977syntactic} and ‘non-experienced’ past tense marker \phono{-sqa} in Cuzco Q \citep{Faller03}\index[aut]{Faller, Martina} \citep[as cited in][223--33]{Peterson14}.\index[aut]{Peterson, Tyler} This is a hypothesis I am currently investigating.}\\

% 11
\gloexe{Glo4:nqa}{}{ch}%
{Lilu\pb{pti}nqa, li:.}%ch que first line
{\morglo{li-lu-pti-n-qa}{go-\lsc{urgt}-\lsc{subds}-\lsc{3}-\lsc{top}}\morglo{li-:}{go-\lsc{1}}}%morpheme+gloss
\glotran{When (\pb{after}) he went, I went.}{}%eng+spa trans
{}{}%rec - time

% 12
\gloexe{Glo4:Hinaptinshiis}{}{ach}%
{Hinaptinshi iskinapa kaya\pb{pti}n baliyarun.}%ach que first line
{\morglo{hinaptin-shi}{then-\lsc{evr}}\morglo{iskina-pa}{corner-\lsc{loc}}\morglo{ka-ya-\pb{pti}-n}{be-\lsc{prog}-\lsc{subds}-\lsc{3}}\morglo{baliya-ru-n}{shoot-\lsc{urgt}-\lsc{3}}}%morpheme+gloss
\glotran{Then, they say, \pb{when} he was in the corner, they shot him.}{}%eng+spa trans
{}{}%rec - time

% 13
\gloexe{Glo4:Suyarusa}{}{amv}%
{Suyarusa \pb{hinashpa} maqarusa. Chayshi nirqamik tumarun.}%amv que first line
{\morglo{suya-ru-sa}{wait-\lsc{urgt}-\lsc{sa}}\morglo{\pb{hinashpa}}{then}\morglo{maqa-ru-sa}{beat-\lsc{urgt}-\lsc{sa}}\morglo{chay-shi}{\lsc{dem.d}-\lsc{evr}}\morglo{ni-rqa-mi-k}{say-\lsc{pst}-\lsc{evd}-\lsc{ik}}\morglo{tuma-ru-n}{take-\lsc{urgt}-\lsc{3}}}%morpheme+gloss
\glotran{She had waited for her \pb{then} she had hit her. That’s why he took [the poison], they say.}{}%eng+spa trans
{}{}%rec - time

% TABLE 19a
\begin{table}[!ht]
\small\centering
\caption{Inflection of \phono{-sHa}}\label{Tab19a}
\begin{tabular}{lll}
\lsptoprule
Person		& Singular		& Plural	\\
\midrule
1	& -sa-ni\tss{\AMV}			& -sa-nchik\tss{\AMV,\ACH,\SP}		\\
					& -sha-ni\tss{\LT}			& -sha-nchik\tss{\CH,\LT}	\\
					& -sha-:\tss{\CH}		& 			\\
					& -sa-:\tss{\AMV,\SP}			&	\\[2ex]
%\midrule
2	& -sa-nki\tss{\AMV,\ACH,\SP}		& -sa-nki\tss{\AMV,\ACH,\SP}			\\
					& -sha-nki\tss{\CH,\LT}	& -sha-nki\tss{\CH,\LT}	\\
		\\[2ex]
%\midrule
3	& -sa-\uo\tss{\AMV,\ACH,\SP}			& -sa-\uo\tss{\AMV,\ACH,\SP}			\\
					& -sha-\uo\tss{\CH,\LT}	& -sha-\uo\tss{\CH,\LT}	\\
			\\
\lspbottomrule
\end{tabular}
\end{table}

% TABLE 19b
\begin{table}[!ht]
\small\centering
\caption{Inflection of \phono{sHa} -- actor-object suffixes}\label{Tab19b}
\begin{tabular}{lllll}
\lsptoprule
2>1	&	3>1	&	3>1pl	&	1>2	&	3>2	\\
\midrule
-wa-sa-nki\tss{\AMV}	&	-wa-sa-\uo\tss{\AMV}	&	-wa-sa-nchik\tss{\AMV}	&	-sa-yki\tss{\AMV,\ACH,\SP}	&	N/A	\\
-wa-sha-nki\tss{\LT}	&	-wa-sha-\uo\tss{\LT}	&	-wa-sha-nchik\tss{\LT}	&	-sha-yki\tss{\LT,\CH}	&	N/A	\\
-ma-sa-nki\tss{\ACH,\SP}	&	-ma-sa-\uo\tss{\ACH,\SP}	&	-ma-sa-nchik\tss{\ACH,\SP}	&	 	&	 	\\
-ma-sha-nki\tss{\CH}	&	-ma-sha-\uo\tss{\CH}	&	-ma-sha-nchik\tss{\CH}	&	 	&	 	\\
\lspbottomrule
\end{tabular}
\end{table}

\paragraph{Habitual past \phono{-q ka-}}\label{par:iterative}\index[sub]{iterative past}
The habitual past is indicated by the combination of the agentive noun --~formed by the addition of \phono{-q} to the verb stem~-- and the relevant present-tense form of \phono{ka-} ‘be’ (zero in the third person)~(\ref{Glo4:Marcopukyopa}--\ref{Glo4:tirruristawan}).\\

% 1
\gloexe{Glo4:Marcopukyopa}{}{amv}%
{Wak Marcopukyopa, triguta hurqupakamu\pb{q} \pb{kani}.}%amv que first line
{\morglo{wak}{\lsc{dem.d}}\morglo{Marcopukyo-pa,}{Marcopukyo-\lsc{loc}}\morglo{trigu-ta}{wheat-\lsc{acc}}\morglo{hurqu-paka-mu-q}{remove-\lsc{mutben}-\lsc{cisl}-\lsc{ag}}\morglo{ka-ni}{be-\lsc{1}}}%morpheme+gloss
\glotran{There in Marcopukyo, I \pb{used to} harvest wheat.}{}%eng+spa trans
{}{}%rec - time

% 2
\gloexe{Glo4:Chayhina}{}{amv}%
{Chayhina puri\pb{q} \pb{kanchik} ayvis fusfuru puchukaru\pb{q}.}%amv que first line
{\morglo{chay-hina}{\lsc{dem.d}-\lsc{comp}}\morglo{puri-q}{walk-\lsc{ag}}\morglo{ka-nchik}{be-\lsc{1pl}}\morglo{ayvis}{sometimes}\morglo{fusfuru}{match}\morglo{puchuka-ru-q}{finish-\lsc{urgt}-\lsc{ag}}}%morpheme+gloss
\glotran{We \pb{would} walk around like that; sometimes the matches \pb{would} run out.}{}%eng+spa trans
{}{}%rec - time

% 3
\gloexe{Glo4:Awturidadkunaqa}{}{ch}%
{Awturidadkunaqa pakaku\pb{q} huk law liku\pb{q}.}%ch que first line
{\morglo{awturidad-kuna-qa}{authority-\lsc{pl}-\lsc{top}}\morglo{paka-ku-q}{hide-\lsc{refl}-\lsc{ag}}\morglo{huk}{one}\morglo{law}{side}\morglo{li-ku-q}{go-\lsc{refl}-\lsc{ag}}}%morpheme+gloss
\glotran{The officials \pb{would} hide, they \pb{would} go other places.}{}%eng+spa trans
{}{}%rec - time

% 4
\gloexe{Glo4:tirruristawan}{}{ach}%
{Chay tirruristawan kay Azángaropaq rikuya\pb{q}. Wama wamaq piliyakuya\pb{q}.}%ach que first line
{\morglo{chay}{\lsc{dem.d}}\morglo{tirrurista-wan}{terrorist-\lsc{instr}}\morglo{kay}{\lsc{dem.p}}\morglo{Azángaro-paq}{Azángaro-\lsc{abl}}\morglo{riku-ya-q}{go-\lsc{prog}-\lsc{ag}}\morglo{wama}{a.lot}\morglo{wamaq}{a.lot}\morglo{piliya-ku-ya-q}{fight-\lsc{refl}-\lsc{prog}-\lsc{ag}}}%morpheme+gloss
\glotran{They \pb{would} be going from Azángaro with the terrorists. They \pb{would} be fighting a lot.}{}%eng+spa trans
{}{}%rec - time

\noindent
Generally translated in Spanish with the imperfect, the structure can be translated in English as ‘used to V’ or ‘would V’. Object suffixes precede \phono{-q}~(\ref{Glo4:Wasiytaham}), (\ref{Glo4:Taytacha}).\\

% 5
\gloexe{Glo4:Wasiytaham}{}{amv}%
{Wasiyta hamuruptiy uquchi\pb{waq}. Huk vidatam wakwanqa pukllarirqani.}%amv que first line
{\morglo{wasi-y-ta}{house-\lsc{1}-\lsc{acc}}\morglo{hamu-ru-pti-y}{come-\lsc{urgt}-\lsc{subds}-\lsc{1}}\morglo{uqu-chi-wa-q}{wet-\lsc{caus}-\lsc{1.obj}-\lsc{ag}}\morglo{huk}{one}\morglo{vida-ta-m}{life-\lsc{acc}-\lsc{evd}}\morglo{wak-wan-qa}{\lsc{dem.d}\lsc{}-\lsc{instr}-\lsc{top}}\morglo{puklla-ri-rqa-ni}{play-\lsc{incep}-\lsc{pst}-\lsc{1}}}%morpheme+gloss
\glotran{When I \pb{would} come home, they \pb{would} get me wet. I played around with them a lot.}{}%eng+spa trans
{}{}%rec - time

% 6
\gloexe{Glo4:Taytacha}{}{sp}%
{Taytacha: willa\pb{maq} chayhinam antigwu viyhukuna purira nishpa.}%sp que first line
{\morglo{tayta-cha-:}{father-\lsc{dim}-\lsc{1}}\morglo{willa-ma-q}{tell-\lsc{1.obj}-\lsc{ag}}\morglo{chay-hina-m}{\lsc{dem.d}-\lsc{comp}-\lsc{evd}}\morglo{antigwu}{ancient}\morglo{viyhu-kuna}{old-\lsc{pl}}\morglo{puri-ra}{walk-\lsc{pst}}\morglo{ni-shpa}{say-	\lsc{subis}}}%morpheme+gloss
\glotran{My grandfather \pb{used to} tell \pb{me} [stories]. The ancients walked about like that, he said.}{}%eng+spa trans
{}{}%rec - time

% TABLE 20a
\begin{table}[!ht]
\small\centering
\caption{Habitual past inflection}\label{Tab20a}
\begin{tabular}{lll}
\lsptoprule
Person		& Singular		& Plural	\\
\midrule
1	& -q ka-ni\tss{\AMV,\LT}	& -q ka-nchik	\\
	& -q ka-:\tss{\ACH,\CH,\SP}	&				\\[2ex]
%\midrule
2	&	-q ka-nki				&-q ka-nki		\\[2ex]
%\midrule
3	&	-q						&-q				\\
\lspbottomrule
\end{tabular}
\end{table}

% TABLE 20b
\begin{table}[!ht]
\small\centering
\caption{Habitual past inflection -- actor-object suffixes}\label{Tab20b}
\begin{tabular}{lllll}
\lsptoprule
2>1		& 3>1		& 3>1pl	& 1>2	& 3>2	\\
\midrule
-wa-q ka-nki\tss{\AMV,\LT}	&	-wa-q\tss{\AMV,\LT}	&	N/A	&	N/A	&	N/A	\\
-ma-q ka-nki\tss{\ACH,\CH,\SP}	&	-ma-q\tss{\ACH,\CH,\SP}	&	 	&	 	&	 	\\
\lspbottomrule
\end{tabular}
\end{table}

\subsection{Conditional}\label{ssec:conditional}
\SYQ{} verbs inflect for conditionality, present and past. Two different forms indicate the conditional in \SYQ. The first, the regular conditional, is attested in all persons, singular and plural, in all dialects. Alternative conditional forms are attested in the first person plural in all dialects and in the second person both singular and plural in the \AMV{} dialect. Both the regular and alternative conditional may be interpreted as ability, circumstantial, deontological, epistemological, and teleological modals, both existential and universal, at least. For more extensive discussion of the interpretation of the conditional under the scope of the various evidential enclitics and their modifiers, see~§~\ref{ssec:evidence}.

\subsubsection{Regular conditional (potential) \phono{-man}}\label{ssec:regcond}\index[sub]{conditional}
All \SYQ{} dialects indicate the conditional with the suffix \phono{-man}. In the first person, it is the person-number suffixes of the nominal (possessive) paradigm that are used in combination with \phono{-man} (i.e., \phono{-y} and not \phono{-ni} is used for the first-person singular in the \QII{}-alligned dialects)~(\ref{Glo4:lliwlliw}). \phono{-man} follows all other inflectional suffixes (\phono{ri\pb{-nki-man}} \phono{*ri\pb{-man-ni-nki}})~(\ref{Glo4:chichiyuq}); \phono{-man} is in complementary distribution with tense morphemes (\phono{*ri\pb{-rqa}-nki\pb{-man}}) (the examples cited are given in~§~\ref{ssec:modality}).

% TABLE 21a
\begin{table}[!ht]
\small\centering
\caption{Regular conditional inflection}\label{Tab21a}
\begin{tabular}{lll}
\lsptoprule
Person		& Singular		& Plural	\\
\midrule
1	& -y-man\tss{\AMV,\LT}		& -nchik-man		\\
	& -:-man\tss{\ACH,\CH,\SP}	& 		\\[2ex]
%\midrule
2	& -nki-man		& -nki-man		\\[2ex]
%\midrule
3	& -n-man		& -n-man		\\
\lspbottomrule
\end{tabular}
\end{table}

% TABLE 21b
\begin{table}[!ht]
\small\centering
\caption{Regular conditional inflection -- actor-object suffixes}\label{Tab21b}
\resizebox{\textwidth}{!}{
\begin{tabular}{@{\hspace{1ex}}l@{\hspace{1ex}}l@{\hspace{1ex}}l@{\hspace{1ex}}l@{\hspace{1ex}}l@{\hspace{1ex}}}
\lsptoprule
2>1		& 3>1		& 3>1pl	& 1>2	& 3>2	\\
\midrule
-wa-nki-man\tss{\AMV,\LT}	&	-wa-n-man\tss{\AMV,\LT}	&	-wa-nchik-man\tss{\AMV,\LT}	&	-yki-man	&	-shu-nki-man \\
-ma-nki-man\tss{\ACH,\CH,\SP}	&	-ma-n-man\tss{\ACH,\CH,\SP}	&	-ma-nchik-man\tss{\ACH,\CH,\SP}	&	 	&	 	\\
\lspbottomrule
\end{tabular}}
\end{table}

\subsubsection{Excursis: modality}\label{ssec:modality}\index[sub]{modals}
The \SYQ{} conditional covers far more territory than does the conditional in Spanish or English, receiving ability~(\ref{Glo4:Kanancha}--\ref{Glo4:kawsa}), circumstantial~(\ref{Glo4:lawpa}), (\ref{Glo4:Sarurullawan}), (\ref{Glo4:Suwapis}), deontic~(\ref{Glo4:Wawakunki}), (\ref{Glo4:Yatarunki}), (\ref{Glo4:Chayshi}), (\ref{Glo4:Ishchallata}), teleological~(\ref{Glo4:surqunaykipaq}), (\ref{Glo4:Agua}), and epistemological~(\ref{Glo4:Wasikunapis}), (\ref{Glo4:waqayan}), (\ref{Glo4:mantriki}) modal readings, both existential and universal.\\

% 1 (2)
\gloexe{Glo4:Kanancha}{}{lt}%
{Kanan chayta rin\pb{man}.}%lt que first line
{\morglo{kanan}{now}\morglo{chay-ta}{\lsc{dem.d}-\lsc{acc}}\morglo{ri-n-man}{go-\lsc{3}-\lsc{cond}}}%morpheme+gloss
\glotran{Now, he \pb{could} go there.}{}%eng+spa trans
{}{}%rec - time

% 2 (3)
\gloexe{Glo4:Manachukuska}{}{ch}%
{¿Manachu kuska lin\pb{man}?}%ch que first line
{\morglo{mana-chu}{no-\lsc{q}}\morglo{kuska}{together}\morglo{li-n-man}{go-\lsc{3}-\lsc{cond}}}%morpheme+gloss
\glotran{\pb{Can}’t they go together?}{}%eng+spa trans
{}{}%rec - time

% 3 (4)
\gloexe{Glo4:Ulvidaru}{}{sp}%
{Ulvidaru:, manayá yuyari:\pb{man}chu.}%sp que first line
{\morglo{ulvida-ru-:}{forget-\lsc{urgt}-\lsc{1}}\morglo{mana-yá}{no-\lsc{emph}}\morglo{yuyari-:-man-chu}{remember-\lsc{1}-\lsc{cond}-\lsc{neg}}}%morpheme+gloss
\glotran{I’ve forgotten. I \pb{can}’t remember.}{}%eng+spa trans
{}{}%rec - time

% 4 (5)
\gloexe{Glo4:Imatataq}{}{ach}%
{¿Imatataq ruwanki\pb{man}? ¿Imatataq ruwanman?}%ach que first line
{\morglo{ima-ta-taq}{what-\lsc{acc}-\lsc{seq}}\morglo{ruwa-nki-man}{make-\lsc{2}-\lsc{cond}}\morglo{ima-ta-taq}{what-\lsc{acc}-\lsc{seq}}\morglo{ruwa-n-man}{make-\lsc{3}-\lsc{cond}}}%morpheme+gloss
\glotran{What \pb{can} you do? What \pb{can} they do?}{}%eng+spa trans
{}{}%rec - time

% 5 (6)
\gloexe{Glo4:kawsa}{}{ch}%
{Manaña\pb{m} kawsa:\pb{man}chu.}%ch que first line
{\morglo{mana-ña-m}{no-\lsc{disc}-\lsc{evd}}\morglo{kawsa-:-man-chu}{live-\lsc{1}-\lsc{cond}-\lsc{neg}}}%morpheme+gloss
\glotran{I \pb{can’t} live any more.}{}%eng+spa trans
{}{}%rec - time

% 6 (7)
\gloexe{Glo4:lawpa}{}{amv}%
{Manatr wak lawpa pastu kan\pb{man}chu.}%amv que first line
{\morglo{mana-tr}{no-\lsc{evc}}\morglo{wak}{\lsc{dem.d}}\morglo{law-pa}{side-\lsc{loc}}\morglo{pastu}{pasture.grass}\morglo{ka-n-man-chu}{be-\lsc{3}-\lsc{cond}-\lsc{neg}}}%morpheme+gloss
\glotran{There \pb{can’t} be any pasture on that side.}{}%eng+spa trans
{}{}%rec - time

% 7 (8)
\gloexe{Glo4:Sarurullawan}{}{amv}%
{Sarurullawan\pb{man}.}%amv que first line
{\morglo{saru-ru-lla-wa-n-man}{trample-\lsc{urgt}-\lsc{rstr}-\lsc{1.obj}-\lsc{3}-\lsc{cond}}}%morpheme+gloss
\glotran{She \pb{could} trample me.}{}%eng+spa trans
{}{}%rec - time

% 8 (29)
\gloexe{Glo4:Suwapis}{}{ach}%
{Suwapis rikaru\pb{nman} chaypa.}%ach que first line
{\morglo{suwa-pis}{thief-\lsc{add}}\morglo{rika-ru-n-man}{see-\lsc{urgt}-\lsc{3}-\lsc{cond}}\morglo{chay-pa}{\lsc{dem.d}-\lsc{loc}}}%morpheme+gloss
\glotran{Thieves also \pb{can} pop up around there.}{}%eng+spa trans
{}{}%rec - time

% 9 (10)
\gloexe{Glo4:Wawakunki}{}{ach}%
{Wawakunki\pb{man}mi hukllatas.}%ach que first line
{\morglo{wawa-ku-nki-man-mi}{give.birth-\lsc{refl}-\lsc{2}-\lsc{cond}-\lsc{evd}}\morglo{huk-lla-ta-s}{one-\lsc{rstr}-\lsc{acc}-\lsc{add}}}%morpheme+gloss
\glotran{You \pb{should} give birth to at least one [child].}{}%eng+spa trans
{}{}%rec - time

% 10 (11)
\gloexe{Glo4:Yatarunki}{}{amv}%
{Yatarunki\pb{man}taq.}%amv que first line
{\morglo{yata-ru-nki-man-taq}{catch-\lsc{urgt}-\lsc{2}-\lsc{cond}-\lsc{seq}}}%morpheme+gloss
\glotran{\pb{Be careful} not to catch it.}{}%eng+spa trans
{}{}%rec - time

% 11 (26)
\gloexe{Glo4:Chayshi}{}{lt}%
{Chayshi manash invidyusu kaytaq \pb{atipanchikman}chu.}%lt que first line
{\morglo{chay-shi}{\lsc{dem.d}-\lsc{evr}}\morglo{mana-sh}{no-\lsc{evr}}\morglo{invidyusu}{jealous}\morglo{kay-taq}{\lsc{dem.p}-\lsc{seq}}\morglo{atipa-nchik-man-chu}{be.able-\lsc{1pl}-\lsc{cond}-\lsc{neg}}}%morpheme+gloss
\glotran{That’s why we \pb{shouldn’t} be jealous.}{}%eng+spa trans
{}{}%rec - time

% 12 (33)
\gloexe{Glo4:Ishchallata}{}{amv}%
{Ishchallataña shutuykachi\pb{yman}, ¿aw?}% que first line
{\morglo{ishcha-lla-ta-ña}{little-\lsc{rstr}-\lsc{acc}-\lsc{disc}}\morglo{shutu-yka-chi-y-man}{drip-\lsc{excep}-\lsc{caus}-\lsc{1}-\lsc{cond}}\morglo{aw}{yes}}%morpheme+gloss
\glotran{\pb{I should} make it drip just a little, right?}{}%eng+spa trans
{}{}%rec - time

% 13 (12)
\gloexe{Glo4:surqunaykipaq}{}{amv}%
{Allin nutata surqunaykipaq istudyanki\pb{man}miki.~\updag}%amv que first line
{\morglo{allin}{good}\morglo{nuta-ta}{grade-\lsc{acc}}\morglo{surqu-na-yki-paq}{take.out-\lsc{nmlz}-\lsc{2}-\lsc{purp}}\morglo{istudya-nki-man-mi-ki}{study-\lsc{2}-\lsc{cond}-\lsc{evd}-\lsc{iki}}}%morpheme+gloss
\glotran{\pb{If you want} to get good grades, \pb{you have to} study.}{}%eng+spa trans
{}{}%rec - time

% 14 (28)
\gloexe{Glo4:Agua}{}{ach}%
{Agua floridata u krisutapis apamu\pb{nkiman}mi.}%ach que first line
{\morglo{agua}{water}\morglo{florida-ta}{florida-\lsc{acc}}\morglo{u}{or}\morglo{krisu-ta-pis}{Croesus-\lsc{acc}-\lsc{add}}\morglo{apa-mu-nki-man-mi}{bring-\lsc{cisl}-\lsc{2}-\lsc{cond}-\lsc{evd}}}%morpheme+gloss
\glotran{You \pb{can} bring florida water or croesus [so as not to get sick].}{}%eng+spa trans
{}{}%rec - time

% 15 (9)
\gloexe{Glo4:Wasikunapis}{}{amv}%
{Wasikunapis saqaykun\pb{man}tri fwirti kaptinqa.}%amv que first line
{\morglo{wasi-kuna-pis}{house-\lsc{pl}-\lsc{add}}\morglo{saqa-yku-n-man-tri}{go.down-\lsc{excep}-\lsc{3}-\lsc{cond}-\lsc{evc}}\morglo{fwirti}{strong}\morglo{ka-pti-n-qa}{be-\lsc{subds}-\lsc{3}-\lsc{top}}}%morpheme+gloss
\glotran{The houses, also, \pb{could} fall if there were a strong one [earthquake].}{}%eng+spa trans
{}{}%rec - time

% 16 (30)
\gloexe{Glo4:waqayan}{}{sp}%
{Chayqa waqayan. ¿Imataq ka\pb{nman}?}%sp que first line
{\morglo{chay-qa}{\lsc{dem.d}-\lsc{top}}\morglo{waqa-ya-n}{cry-\lsc{prog}-\lsc{3}}\morglo{ima-taq}{what-\lsc{seq}}\morglo{ka-n-man}{be-\lsc{3}-\lsc{cond}}}%morpheme+gloss
\glotran{It’s crying. What \pb{could} that be?}{}%eng+spa trans
{}{}%rec - time

% 17 (31)
\gloexe{Glo4:mantriki}{}{ach}%
{Wañukun\pb{mantriki}.¿Imayna mana kutikamunmanchu?}%ach que first line
{\morglo{wañu-ku-n-man-tri-ki}{die-\lsc{refl}-\lsc{3}-\lsc{cond}-\lsc{evc}-\lsc{iki}}\morglo{	imayna}{how}\morglo{mana}{no}\morglo{kuti-ka-mu-n-man-chu}{return-\lsc{refl}-\lsc{cisl}-\lsc{3}-\lsc{cond}-\lsc{neg}}}%morpheme+gloss
\glotran{He \pb{might} have died. Why can’t he come back?}{}%eng+spa trans
{}{}%rec - time

\noindent
As detailed in~§~\ref{ssec:evidence}, \SYQ{} modals are themselves unspecified for force: modal force is determined by context and is generally specified by the evidential modifiers. Weak modal readings result when the modal is under the scope either of no evidential or of an evidential modified by the evidential modifier \uo; strong universal readings result when the evidential is modified by the evidential modifier \phono{-iki} (\phono{siqa-yku-n-man-tri-\pb{\uo}} ‘it \pb{might} fall’, \phono{siqa-yku-n-man-tri-\pb{ki}} ‘it \pb{will most likely} fall’; \phono{istudya-nki-man-mi-\pb{\uo}} ‘you \pb{should} study’, \phono{istudya-nki-man-mi-\pb{ki}} ‘you \pb{must} study’); moderately strong modal readings result when the modifier \phono{-ik} takes scope over the modal. Ability modals also result from the combination of the infinitive and the verb \phono{atipa-} ‘be able’~(\ref{Glo4:pishipakuyan}--\ref{Glo4:kaptinqa}).\\

% 18 (15)
\gloexe{Glo4:pishipakuyan}{}{amv}%
{Manaña riyta \pb{atipa}nchu pishipakuyan.}%amv que first line
{\morglo{mana-ña}{no-\lsc{disc}}\morglo{ri-y-ta}{go-\lsc{inf}-\lsc{acc}}\morglo{atipa-n-chu}{be.able-\lsc{3}-\lsc{neg}}\morglo{pishipa-ku-ya-n}{tire-\lsc{refl}-\lsc{prog}-\lsc{3}}}%morpheme+gloss
\glotran{They \pb{can’t} go -- they’re getting tired.}{}%eng+spa trans
{}{}%rec - time

% 19 (16)
\gloexe{Glo4:kaptinqa}{}{ach}%
{Wawan kaptinqa, manaña uywayta \pb{atipa}nchu.}%ach que first line
{\morglo{wawa-n}{baby-\lsc{3}}\morglo{ka-pti-n-qa,}{be-\lsc{subds}-\lsc{3}-\lsc{top}}\morglo{mana-ña}{no-\lsc{disc}}\morglo{uywa-y-ta}{raise-\lsc{inf}-\lsc{acc}}\morglo{atipa-n-chu}{be.able-\lsc{3}-\lsc{neg}}}%morpheme+gloss
\glotran{When they have babies, \pb{they can’t} raise [cattle] any more.}{}%eng+spa trans
{}{}%rec - time

\noindent
The verbs \phono{usHachi-} and \phono{puydi-}, both translated ‘be able,’ as well as \phono{yatra-} ‘know’ may also be employed in this construction~(\ref{Glo4:pawayta}--\ref{Glo4:Puriyta}).\\

% 20 (18)
\gloexe{Glo4:pawayta}{}{amv}%
{Chay ninaman pawayta hawanta munayan mana \pb{usachi}nchu.}%amv que first line
{\morglo{chay}{\lsc{dem.d}}\morglo{nina-man}{fire-\lsc{all}}\morglo{pawa-y-ta}{jump-\lsc{inf}-\lsc{acc}}\morglo{hawa-n-ta}{above-\lsc{3}-\lsc{acc}}\morglo{muna-ya-n}{want-\lsc{prog}-\lsc{3}}\morglo{mana}{no}\morglo{usachi-n-chu}{be.able-\lsc{3}-\lsc{neg}}}%morpheme+gloss
\glotran{They want to jump over the fire, but they \pb{can’t}.}{}%eng+spa trans
{}{}%rec - time

% 21 (19)
\gloexe{Glo4:Piluntaqa}{}{amv}%
{Piluntaqa yupayanshari chay chapupaqta. \pb{Ushachi}nchu yupayta.}%amv que first line
{\morglo{pilu-n-ta-qa}{hair-\lsc{3}-\lsc{acc}-\lsc{top}}\morglo{yupa-ya-n-sh-ari}{count-\lsc{prog}-\lsc{3}-\lsc{evr}-\lsc{ari}}\morglo{chay}{\lsc{dem.d}}\morglo{chapu-paq-ta}{little.dog-\lsc{gen}-\lsc{acc}}\morglo{ushachi-n-chu}{be.able-\lsc{3}-\lsc{neg}}\morglo{yupa-y-ta}{count-\lsc{inf}-\lsc{acc}}}%morpheme+gloss
\glotran{[The zombie] is counting the hairless dog’s hairs. He \pb{can’t} count them.}{}%eng+spa trans
{}{}%rec - time

% 22 (20)
\gloexe{Glo4:Puriyta}{}{amv}%
{Puriyta \pb{yatra}nñam.}%amv que first line
{\morglo{puri-y-ta}{walk-\lsc{inf}-\lsc{acc}}\morglo{yatra-n-ña-m}{know-\lsc{3}-\lsc{disc}-\lsc{evd}}}%morpheme+gloss
\glotran{She \pb{can} already walk.}{}%eng+spa trans
{}{}%rec - time

\noindent
\phono{atipa-}, \phono{usHachi-}, and \phono{puydi-} appear in verbal constructions only when negated; they appear non-negated only in nominalizations~(\ref{Glo4:Hinashpatra}), (\ref{Glo4:Burrunchikwan}).\\

% 23 (21)
\gloexe{Glo4:Hinashpatra}{}{ch}%
{Hinashpa trayarushpaqa~\dots{} waqtakuyanchikña \pb{atipasa}nchikkama.}%ch que first line
{\morglo{hinashpa}{then}\morglo{traya-ru-shpa-qa}{arrive-\lsc{urgt}-\lsc{subis}-\lsc{top}}\morglo{waqta-ku-ya-nchik-ña}{hit-\lsc{refl}-\lsc{prog}-\lsc{1pl}-\lsc{disc}}\morglo{atipa-sa-nchik-kama}{be.able-\lsc{prf}-\lsc{1pl}-\lsc{lim}}}%morpheme+gloss
\glotran{Then, when you get there, when there is any, you’re already hitting it as much as you \pb{can}.}{}%eng+spa trans
{}{}%rec - time

% 24 (22)
\gloexe{Glo4:Burrunchikwan}{}{amv}%
{Burrunchikwan rinchik Cañetekama maykamapis \pb{atipa}sanchikkama.}%amv que first line
{\morglo{burru-nchik-wan}{donkey-\lsc{1pl}-\lsc{instr}}\morglo{ri-nchik}{go-\lsc{1pl}}\morglo{Cañete-kama}{Cañete-\lsc{lim}}\morglo{may-kama-pis}{where-\lsc{lim}-\lsc{add}}\morglo{atipa-sa-nchik-kama}{be.able-\lsc{prf}-\lsc{1pl}-\lsc{lim}}}%morpheme+gloss
\glotran{With our donkeys we went to Cañete, to wherever, wherever we \pb{could}.}{}%eng+spa trans
{}{}%rec - time

\noindent
Universal deontic readings additionally follow from the combination of the nominalizer, \phono{-na} with nominal (possessive) person inflection~(\ref{Glo4:Chaymi}); they are available, too, with the simple present tense.\\

% 25 (23)
\gloexe{Glo4:Chaymi}{}{amv}%
{Chaymi vaka harkaq riku\pb{nayki}miki.}%amv que first line
{\morglo{chay-mi}{\lsc{dem.d}-\lsc{evd}}\morglo{vaka}{cow}\morglo{harka-q}{herd-\lsc{ag}}\morglo{riku-na-yki-mi-ki}{go-\lsc{nmlz}-\lsc{2}-\lsc{evd}-\lsc{iki}}}%morpheme+gloss
\glotran{That’s why \pb{you have to} go pasture the cows.}{}%eng+spa trans
{}{}%rec - time

\noindent
In~(\ref{Glo4:tushu}), the adverb \phono{hawka} ‘tranquil’ modifying a future tense verb receives an existential deontic modal reading. As detailed in~§~\ref{ssec:conjectural}, under the scope of the conjectural evidential, \phono{-trI}, conditionals are generally restricted to epistemic interpretations; under the scope of the direct evidential \phono{-mI}, they receive all but conjectural interpretations.\\

% 26 (24)
\gloexe{Glo4:tushu}{}{amv}%
{\pb{Hawka}ñam tushu\pb{nqa}.}%amv que first line
{\morglo{hawka-ña-m}{tranquil-\lsc{disc}-\lsc{evd}}\morglo{tushu-nqa}{dance-\lsc{3.fut}}}%morpheme+gloss
\glotran{She \pb{can} go dancing.}{}%eng+spa trans
{}{}%rec - time

\noindent
Attaching to verbs inflected with second-person \phono{-iki}, \phono{-man}, may be interpreted as a caution~(\ref{Glo4:rishpa}).\\

% 27 (34)
\gloexe{Glo4:rishpa}{}{amv}%
{Viñacta rishpa kichkata \pb{manam} saruramunkiman.}%amv que first line
{\morglo{Viñac-ta}{Viñac-\lsc{acc}}\morglo{ri-shpa}{go-\lsc{subis}}\morglo{kichka-ta}{thorn-\lsc{acc}}\morglo{mana-m}{no-\lsc{evd}}\morglo{saru-ra-mu-nki-man}{trample-\lsc{urgt}-\lsc{cisl}-\lsc{2}-\lsc{cond}}}%morpheme+gloss
\glotran{Be careful not to step on thorns when you go to Viñac.}{}%eng+spa trans
{}{}%rec - time

\noindent
And finally, it appears that \phono{-man} never attaches to either of the alternative-conditional morphemes, \phono{-waq} or \phono{-chuman.}\footnote{I have not yet tested these for grammaticality in elicitation sessions. I can only say that in a corpus with~85 instances of \phono{-iki-man} and~24 instances of \phono{-nchick-man}, \phono{*-waq-man} and \phono{*-chuwan-man} remain unattested.} This information is summarized in Table~\ref{Tab22} (examples are given for the third person with the verb \phono{qawa-} ‘see’).\\

% TABLE 22
\begin{table}[!ht]
\small\centering
\caption{Modal system}\label{Tab22}\index[sub]{modal system}
\begin{tabular}{lll}
\lsptoprule
				&	Existential								&	Universal*	\\
\midrule
Ability			&	V-\lsc{cond}-\lsc{evd}								&	x	\\
				&	{\phono{qawa-n-man-mi}}					&		\\
				& manam V-\lsc{inf}-\lsc{acc} 	&	\\
				& atipa-\lsc{infl}-chu *\lsc{ev}			&			\\
				&	{\phono{manam qawa-y-ta atipa-n-chu}}&			\\[2ex]
%\midrule
Circumstantial	&	V-\lsc{cond}-\lsc{evd}								&	x	\\
				&	{\phono{wiña-n-man-mi}}				&			\\[2ex]
%\midrule
Deontic			&	V-\lsc{cond}-\lsc{evd}								&	V-\lsc{cond}-\lsc{evd}	\\
				&	{\phono{qawa-n-man-mi}}				&	{\phono{qawa-n-man-mi}}	\\
				&	Hawka V-\lsc{fut}-\lsc{evd}							&	V-\lsc{nmlz}-\lsc{poss}-\lsc{evd} (be-\lsc{pst})	\\
				&	{\phono{hawka qawa-nqa-m}}			&	{\phono{qawa-na-n-mi}}	\\[2ex]
%\midrule
Epistemic		&	V-\lsc{cond}-\lsc{evc}								&	V-\lsc{cond}-\lsc{evc} (be-\lsc{pst})	\\
				&	{\phono{qawa-n-man-tri}}			&	{\phono{qawa-n-man-tri}}	\\[2ex]
%\midrule
Teleological	&	V-\lsc{cond}-\lsc{evd}								&	V-\lsc{cond}-\lsc{evd}	\\
				&	{\phono{qawa-n-man-mi}}				&	{\phono{qawa-n-man-mi}}		\\
				&	V-\lsc{pres}-\lsc{evd}								&	V-\lsc{pres}-\lsc{evd}	\\
				&	{\phono{qawa-n-mi}}					&	{\phono{qawa-n-m}}	\\
\lspbottomrule							
\multicolumn{3}{l}{\footnotesize *The verbs \phono{usHachi-} ‘be able’, \phono{puydi-} ‘be able’, and \phono{yatra-} ‘know’ can replace \phono{atipa-}.}		\\
\end{tabular}
\end{table}

% 28+ (1)
\gloexe{Glo4:lliwlliw}{}{amv}%
{Ruwa\pb{yman} lliw lliw.}%amv que first line
{\morglo{ruwa-y-man}{make-\lsc{1}-\lsc{cond}}\morglo{lliw}{all}\morglo{lliw}{all}}%morpheme+gloss
\glotran{\pb{I can} do everything.}{}%eng+spa trans
{}{}%rec - time

% 29+ (13)
\gloexe{Glo4:Suwakun}{}{lt}%
{Suwakun\pb{mantriki}.}%lt que first line
{\morglo{suwa-ku-n-man-tri-ki}{rob-\lsc{refl}-\lsc{3}-\lsc{cond}-\lsc{evc}-\lsc{iki}}}%morpheme+gloss
\glotran{[Where it’s abandoned] \pb{it’s very likely} they will rob [you].}{}%eng+spa trans
{}{}%rec - time

% 30+ (14)
\gloexe{Glo4:siqaykurusa}{}{amv}%
{Turantin siqaykurusa. Chay ukupaqa puchukarun\pb{mantriki}.}%amv que first line
{\morglo{tura-ntin}{bull-\lsc{incl}}\morglo{siqa-yku-ru-sa}{go.down-\lsc{excep}-\lsc{urgt}-\lsc{npst}}\morglo{chay}{\lsc{dem.d}}\morglo{uku-pa-qa}{inside-\lsc{loc}-\lsc{top}}\morglo{puchuka-ru-n-man-tri-ki}{finish-\lsc{urgt}-\lsc{3}-\lsc{cond}-\lsc{evc-\lsc{iki}}}}%morpheme+gloss
\glotran{He fell [from the roof] with the bull. He \pb{really might} [have] been finished off inside.}{}%eng+spa trans
{}{}%rec - time

% 31+ (17)
\gloexe{Glo4:Qutrash}{}{amv}%
{Qutrash. Manash pawayta \pb{atipa}nchu chaypaq.}%amv que first line
{\morglo{qutra-sh}{reservoir-\lsc{evr}}\morglo{mana-sh}{no-\lsc{evr}}\morglo{pawa-y-ta}{jump-\lsc{inf}-\lsc{acc}}\morglo{atipa-n-chu}{be.able-\lsc{3}-\lsc{neg}}\morglo{chaypaq}{\lsc{dem.d}-\lsc{abl}}}%morpheme+gloss
\glotran{It’s a lake, they say. They \pb{can’t} jump out of there, they say.}{}%eng+spa trans
{}{}%rec - time

% 32+ (25)
\gloexe{Glo4:Kwidaduqal}{}{ch}%
{¡Kwidadu! Chaypitaq qalqali mikulu\pb{shunkiman}.}%ch que first line
{\morglo{kwidadu}{be.careful}\morglo{chay-pi-taq}{\lsc{dem.d}-\lsc{loc}-\lsc{seq}}\morglo{qalqali}{zombie}\morglo{miku-lu-shunki-man}{eat-\lsc{urgt}-\lsc{3>2}-\lsc{cond}}}%morpheme+gloss
\glotran{Be careful! A zombie \pb{could} eat you there.}{}%eng+spa trans
{}{}%rec - time

% 33+ (27)
\gloexe{Glo4:Manamwanu}{}{sp}%
{Manam wañu:\pb{man}chu.}%sp que first line
{\morglo{mana-m}{no-\lsc{evd}}\morglo{wañu-:-man-chu}{die-\lsc{1}-\lsc{cond}-\lsc{neg}}}%morpheme+gloss
\glotran{I \pb{can’t} die.}{}%eng+spa trans
{}{}%rec - time

% 34+ (32)
\gloexe{Glo4:chichiyuq}{}{amv}%
{Mana chichiyuq kaptikiqa chayna lluqari\pb{shunkiman}tri.}%amv que first line
{\morglo{mana}{no}\morglo{chichi-yuq}{breast-\lsc{poss}}\morglo{ka-pti-ki-qa}{be-\lsc{subds}-\lsc{2}-\lsc{top}}\morglo{chayna}{thus}\morglo{lluqa-ri-shu-nki-man-tri}{walk.grabbing-\lsc{incep}-\lsc{2.obj}-\lsc{2}-\lsc{cond}-\lsc{evc}}}%morpheme+gloss
\glotran{If you don’t have breasts \pb{they might lean on you}.}{}%eng+spa trans
{}{}%rec - time

% 35+ (35)
\gloexe{Glo4:Sarurullawanman}{}{amv}%
{Sarurullawanman manam saruwanantaq munanichu.}%
{\morglo{saru-ru-lla-wa-n-man}{trample{}-\lsc{urgt}-\lsc{rstr}-1.\lsc{obj}-3-\lsc{cond}}\morglo{mana-m}{no-\lsc{evd}}\morglo{saru-wa-na-n-taq}{trample{}-1.\lsc{obj}-\lsc{nmlz}-3-\lsc{seq}}\morglo{muna-ni-chu}{want-1-\lsc{neg}}}%morpheme+gloss
\glotran{She might trample me. I don’t want her to trample me.}%eng
{‘Me podría pisotear. No quiero que me pisotee’.}%spa
{}{}%{Llanka\_BC\_LostCow\_Milking}{02:59-03:03}%

\subsubsection{Alternative conditional \phono{-waq} and \phono{-chuwan}}\label{ssec:altcond}\index[sub]{alternative conditional}
Alternative conditional forms are attested in the second person both singular and plural in the \AMV{} dialect and first person plural in all dialects. \phono{-waq} indicates the second person conditional~(\ref{Glo4:Imallatapis}--\ref{Glo4:lawman}); \phono{-chuwan} indicates the first person plural conditional~(\ref{Glo4:ratum}--\ref{Glo4:Tutayaqpaq}); \phono{-waq} may be explicitly pluralized with \phono{-pa(:)ku}~(\ref{Glo4:Yanapapa}).\\

% 1
\gloexe{Glo4:Imallatapis}{}{amv}%
{¿Imallatapis mikuchayku\pb{waq}chu mamay?}%amv que first line
{\morglo{ima-lla-ta-pis}{what-\lsc{rstr}-\lsc{acc}-\lsc{add}}\morglo{miku-cha-yku-waq-chu}{eat-\lsc{dim}-\lsc{excep}-\lsc{2.cond}-\lsc{q}}\morglo{mama-y?}{mother-\lsc{1}}}%morpheme+gloss
\glotran{\pb{Can you} eat any little thing, Miss?}{}%eng+spa trans
{}{}%rec - time

% 2
\gloexe{Glo4:tinapa}{}{amv}%
{Wak tinapa alcha\pb{waq}.}%amv que first line
{\morglo{wak}{\lsc{dem.d}}\morglo{tina-pa}{tub-\lsc{loc}}\morglo{alcha-waq}{fix-\lsc{2.cond}}}%morpheme+gloss
\glotran{\pb{You can} fix it in that tub.}{}%eng+spa trans
{}{}%rec - time

% 3
\gloexe{Glo4:lawman}{}{amv}%
{¡Ama! Huk lawman hitraykurulla\pb{waq}.}%amv que first line
{\morglo{ama}{\lsc{proh}}\morglo{huk}{one}\morglo{law-man}{side-\lsc{all}}\morglo{hitra-yku-ru-lla-waq}{spill-\lsc{excep}-\lsc{urgt}-\lsc{rstr}-\lsc{2.cond}}}%morpheme+gloss
\glotran{Don’t! \pb{Be careful you don’t} spill it on the other side.}{}%eng+spa trans
{}{}%rec - time

% 4
\gloexe{Glo4:ratum}{}{ach}%
{Ratu ratum chaywanqa shinkaru\pb{chuwan}.}%ach que first line
{\morglo{ratu}{moment}\morglo{ratu-m}{moment-\lsc{evd}}\morglo{chay-wan-qa}{\lsc{dem.d}-\lsc{instr}-\lsc{top}}\morglo{shinka-ru-chuwan}{get.drunk-\lsc{urgt}-\lsc{1pl.cond}}}%morpheme+gloss
\glotran{\pb{We can} get drunk really quickly with that.}{}%eng+spa trans
{}{}%rec - time

% 5
\gloexe{Glo4:quptinqa}{}{ach}%
{Huk quptinqa mikuru\pb{chuwan}mi.}%ach que first line
{\morglo{huk}{one}\morglo{qu-pti-n-qa}{give-\lsc{subds}-\lsc{3}-\lsc{top}}\morglo{miku-ru-chuwan-mi}{eat-\lsc{urgt}-\lsc{1pl.cond}-\lsc{evd}}}%morpheme+gloss
\glotran{When another gives, \pb{we can} eat.}{}%eng+spa trans
{}{}%rec - time

% 6
\gloexe{Glo4:kwintaku}{}{lt}%
{Manañam kwintaku\pb{chuwan}ñachu.}%lt que first line
{\morglo{mana-ña-m}{no-\lsc{disc}-\lsc{evd}}\morglo{kwinta-ku-chuwan-ña-chu}{account-\lsc{refl}-\lsc{1pl.cond}-\lsc{disc}-\lsc{neg}}}%morpheme+gloss
\glotran{\pb{We can} no longer become aware of it.}{}%eng+spa trans
{}{}%rec - time

% 7
\gloexe{Glo4:Tutayaqpaq}{}{amv}%
{Tutayaqpaq, manam imatapis ruwa\pb{chuwan}.}%amv que first line
{\morglo{tuta-ya-q-paq}{night-\lsc{inch}-\lsc{ag}-\lsc{loc}}\morglo{mana-m}{no-\lsc{evd}}\morglo{ima-ta-pis}{what-\lsc{acc}-\lsc{add}}\morglo{ruwa-chuwan}{make-\lsc{1pl.cond}}}%morpheme+gloss
\glotran{In the darkness, \pb{we could}n’t do anything.}{}%eng+spa trans
{}{}%rec - time

% 8
\gloexe{Glo4:Yanapapa}{}{amv}%
{Yanapa\pb{pakuwaq}.}%amv que first line
{\morglo{yanapa-paku-waq}{help-\lsc{jtacc}-\lsc{2.cond}}}%morpheme+gloss
\glotran{\pb{You.\lsc{pl} should} help.}{}%eng+spa trans
{}{}%rec - time

\noindent
Both morphemes simultaneously indicate person and conditionality and are in complementary distribution both with tense and inflectional morphemes. \phono{-w/ma-chuwan} is used with a first-person plural object~(\ref{Glo4:Vinina}--\ref{Glo4:Midiku}).\\

% 9
\gloexe{Glo4:Vinina}{}{ach}%
{Vinina\pb{machuwan}tri.}%ach que first line
{\morglo{vinina-ma-chuwan-tri}{poison-\lsc{1.obj}-\lsc{1pl.cond}-\lsc{evc}}}%morpheme+gloss
\glotran{It \pb{can} poison \pb{us}.}{}%eng+spa trans
{}{}%rec - time

% 10
\gloexe{Glo4:Sapallanchi}{}{ach}%
{Sapallanchiktaqa mikuru\pb{machuwan}tri.}%ach que first line
{\morglo{sapa-lla-nchik-ta-qa}{alone-\lsc{rest}-\lsc{1pl}-\lsc{acc}-\lsc{top}}\morglo{miku-ru-ma-chuwan-tri}{eat-\lsc{urgt}-\lsc{1.obj}-\lsc{1pl.cond}}}%morpheme+gloss
\glotran{[When we’re] alone, [the Devil] \pb{can} eat \pb{us}.}{}%eng+spa trans
{}{}%rec - time

% 11
\gloexe{Glo4:Dibil}{}{amv}%
{Dibil kaptinchik chukaru\pb{wachuwan}yá.}%amv que first line
{\morglo{dibil}{weak}\morglo{ka-pti-nchik}{be-\lsc{subds}-\lsc{1pl}}\morglo{chuka-ru-wa-chuwan-yá}{crash-\lsc{urgt}-\lsc{1.obj}-\lsc{1pl.cond}-\lsc{emph}}}%morpheme+gloss
\glotran{When we’re weak, it \pb{can} make \pb{us} sick.}{}%eng+spa trans
{}{}%rec - time

% 12
\gloexe{Glo4:Midiku}{}{amv}%
{Midiku hudiru\pb{wachuwan}mi.}%amv que first line
{\morglo{midiku}{doctor}\morglo{hudi-ru-wa-chuwan-mi}{screw-\lsc{urgt}-\lsc{1.obj}-\lsc{1pl.cond}-\lsc{evd}}}%morpheme+gloss
\glotran{Doctors \pb{can} screw \pb{us} up.}{}%eng+spa trans
{}{}%rec - time

\noindent
Ability~(\ref{Glo4:Vakatach}),~(\ref{Glo4:Yaku}), circumstantial~(\ref{Glo4:Kayanmi}), deontic~(\ref{Glo4:Chikitu}),~(\ref{Glo4:kuskanchik}) epistemic~(\ref{Glo4:kayakaya}) and teleological~(\ref{Glo4:Trabaha}) readings are all available. If a word ends with \phono{-chuwan}, stress is shifted to the antipenultimate syllable~(\ref{Glo4:kuskanchik}).\\

% 13
\gloexe{Glo4:Vakatach}{}{amv}%
{¿Vakata chuqamu\pb{waq}chu?}%amv que first line
{\morglo{vaka-ta}{cow-\lsc{acc}}\morglo{chuqa-mu-\pb{waq}-chu}{throw.stones-\lsc{cisl}-\lsc{2.cond}-\lsc{q}}}%morpheme+gloss
\glotran{\pb{Can you} throw stones at [herd] cows?}{}%eng+spa trans
{}{}%rec - time

% 14
\gloexe{Glo4:Yaku}{}{lt}%
{Yaku usun chaymi llaqtata rishaq. Manam riga\pb{chuwan}chu.}%lt que first line
{\morglo{yaku}{water}\morglo{usu-n}{waste.on.the.ground-3}\morglo{chay-mi}{\lsc{dem.d}-\lsc{evd}}\morglo{llaqta-ta}{town-\lsc{acc}}\morglo{ri-shaq}{go-\lsc{1.fut}}\morglo{mana-m}{no-\lsc{evd}}\morglo{riga-chuwan-chu}{irrigate-\lsc{1pl.cond}-\lsc{neg}}}%morpheme+gloss
\glotran{Water is spilling. So I’m going to go to town. \pb{We can’t} irrigate.}{}%eng+spa trans
{}{}%rec - time

% 15
\gloexe{Glo4:Kayanmi}{}{amv}%
{Kayanmi uniku qullqiyuqpaqyá ¿Maypam rigala\pb{wachuwan} runaqa?}%amv que first line
{\morglo{ka-ya-n-mi}{be-\lsc{prog}-\lsc{3}-\lsc{evd}}\morglo{uniku}{only}\morglo{qullqi-yuq-paq-yá}{money-\lsc{poss}-\lsc{ben}-\lsc{evd}}\morglo{may-pa-m}{where-\lsc{loc}-\lsc{evd}}\morglo{rigala-{wa-chuwan}}{give.as.a.gift-\lsc{1.obj}-\lsc{1pl.cond}}\morglo{runa-qa?}{person-\lsc{top}}}%morpheme+gloss
\glotran{There are some just for people with money. Where \pb{can} people give \pb{us} things as gifts?}{}%eng+spa trans
{}{}%rec - time

% 16
\gloexe{Glo4:Chikitu}{}{amv}%
{Chikitu llamachata apaku\pb{waq}.}%amv que first line
{\morglo{chikitu}{small}\morglo{llama-cha-ta}{llama-\lsc{dim}-\lsc{acc}}\morglo{apa-ku-waq}{bring-\lsc{refl}-\lsc{2.cond}}}%morpheme+gloss
\glotran{\pb{You could} bring a small little llama.}{}%eng+spa trans
{}{}%rec - time

% 17
\gloexe{Glo4:kayakaya}{}{amv}%
{Wañuypaqpis kaya\pb{chuwan}tri.}%amv que first line
{\morglo{wañu-y-paq-pis}{die-\lsc{inf}-\lsc{purp}-\lsc{add}}\morglo{ka-ya-chuwan-tri}{be-\lsc{prog}-\lsc{1pl.cond}-\lsc{evc}}}%morpheme+gloss
\glotran{\pb{We could be} also about to die.}{}%eng+spa trans
{}{}%rec - time

% 18
\gloexe{Glo4:Trabaha}{}{amv}%
{Trabaha\pb{waq}mi mikuyta munashpaqa.}%amv que first line
{\morglo{trabaha-waq-mi}{work-\lsc{2.cond}-\lsc{evd}}\morglo{miku-y-ta}{eat-\lsc{inf}-\lsc{acc}}\morglo{muna-shpa-qa}{want-\lsc{subis}-\lsc{top}}}%morpheme+gloss
\glotran{\pb{You have to} work if you want to eat.}{}%eng+spa trans
{}{}%rec - time

% 19
\gloexe{Glo4:kuskanchik}{}{ch}%
{Pul\pb{í}\pb{chuwan} kuskanchik.}%ch que first line
{\morglo{puli-chuwan}{walk-\lsc{1pl.cond}}\morglo{kuska-nchik}{together-\lsc{1pl}}}%morpheme+gloss
\glotran{\pb{We should} walk together.}{}%eng+spa trans
{}{}%rec - time

\subsubsection{Past conditional (irrealis)}\index[sub]{past conditional}
The past conditional is indicated by the combination --~as distinct words~-- of the conditional with \phono{ka-RQa}, the third person past tense form of \phono{ka-} ‘be’~(\ref{Glo4:hanaypaq}--\ref{Glo4:piyurtri}).\\

% 1
\gloexe{Glo4:hanaypaq}{}{amv}%
{Riru\pb{yman} \pb{karqa} ñuqapis yanga hanaypaq.}%amv que first line
{\morglo{ri-ru-y-man}{go-\lsc{urgt}-\lsc{1}-\lsc{cond}}\morglo{ka-rqa}{be-\lsc{pst}}\morglo{ñuqa-pis}{I-\lsc{add}}\morglo{yanga}{lie}\morglo{hanay-paq}{up.hill-\lsc{abl}}}%morpheme+gloss
\glotran{I, too, \pb{would have gone} in vain from up hill.}{}%eng+spa trans
{}{}%rec - time

% 2
\gloexe{Glo4:pachalla}{}{amv}%
{Chay pachalla~\dots{} ruwashi\pb{nkiman karqa}.}%amv que first line
{\morglo{chay}{\lsc{dem.d}}\morglo{pacha-lla}{date-\lsc{rstr}}\morglo{ruwa-shi-nki-man}{make-\lsc{acmp}-\lsc{2}-\lsc{cond}}\morglo{ka-rqa}{be-\lsc{pst}}}%morpheme+gloss
\glotran{That time, \pb{you could have} helped make it.}{}%eng+spa trans
{}{}%rec - time

% 3
\gloexe{Glo4:Mastam}{}{amv}%
{Mastam katrayku\pb{runman karqa}.}%amv que first line
{\morglo{mas-ta-m}{more-\lsc{acc}-\lsc{evd}}\morglo{katra-yku-ru-n-man}{release-\lsc{excep}-\lsc{urgt}-\lsc{3}-\lsc{cond}}\morglo{ka-rqa}{be-\lsc{past}}}%morpheme+gloss
\glotran{She \pb{should have let} more out.}{}%eng+spa trans
{}{}%rec - time

% 4
\gloexe{Glo4:piyurtri}{}{amv}%
{¿Imapis mas piyurtri ka\pb{nchikman karqa}?}%amv que first line
{\morglo{ima-pis}{what-\lsc{add}}\morglo{mas}{more}\morglo{piyur-tri}{worse-\lsc{evc}}\morglo{ka-nchik-man}{be-\lsc{1pl}-\lsc{cond}}\morglo{ka-rqa}{be-\lsc{pst}}}%morpheme+gloss
\glotran{What worse thing \pb{could we have} been?}{}%eng+spa trans
{}{}%rec - time

\noindent
The regular conditional form may be used in all dialects~(\ref{Glo4:Dimunyu}--\ref{Glo4:pushakarunki}); the alternative conditional forms may be used in those dialects in which they are available in the present tense~(\ref{Glo4:chawaru}--\ref{Glo4:rikisun}).\\

% 5
\gloexe{Glo4:Dimunyu}{}{ach}%
{Dimunyu chayqa kara. Mikurama\pb{nmantri kara} icha aparama\pb{nmantri kara}.}%ach que first line
{\morglo{Dimunyu}{Devil}\morglo{chay-qa}{\lsc{dem.d}-\lsc{top}}\morglo{ka-ra}{be-\lsc{pst}}\morglo{miku-ra-ma-n-man-tri}{eat-\lsc{urgt}-\lsc{1.obj}-\lsc{3}-\lsc{cond}-\lsc{evc}}\morglo{ka-ra}{be-\lsc{pst}}\morglo{icha}{or}\morglo{apa-ra-ma-n-man-tri}{bring-\lsc{urgt}-\lsc{1.obj}-\lsc{3}-\lsc{cond}-\lsc{evc}}\morglo{ka-ra}{be-\lsc{pst}}}%morpheme+gloss
\glotran{That was the devil. He \pb{could have} eaten me or he \pb{could have} taken me away.}{}%eng+spa trans
{}{}%rec - time

% 6
\gloexe{Glo4:Kundinakuru}{}{sp}%
{Kundinakuru\pb{nmantri kara}. Qullqi chay kasa.}%sp que first line
{\morglo{kundina-ku-ru-n-man-tri}{condemn-\lsc{refl}-\lsc{urgt}-\lsc{3}-\lsc{cond}-\lsc{evc}}\morglo{ka-ra}{be-\lsc{pst}}\morglo{qullqi}{money}\morglo{chay}{\lsc{dem.d}}\morglo{ka-sa}{be-\lsc{npst}}}%morpheme+gloss
\glotran{She \pb{would have} condemned herself [to being a zombie]. That was money.}{}%eng+spa trans
{}{}%rec - time

% 7
\gloexe{Glo4:nkimantri}{}{amv}%
{“Lusta paga\pb{nkimantri karqa} lusninta,” niniyá.}%amv que first line
{\morglo{lus-ta}{light-\lsc{acc}}\morglo{paga-nki-man-tri}{pay-\lsc{2}-\lsc{cond}-\lsc{evc}}\morglo{ka-rqa}{be-\lsc{pst}}\morglo{lus-ni-n-ta}{light-\lsc{euph}-\lsc{3}-\lsc{acc}}\morglo{ni-ni-yá}{say-\lsc{1}-\lsc{emph}}}%morpheme+gloss
\glotran{“\pb{You should have} paid the electric bill, his electric bill,” I said then.}{}%eng+spa trans
{}{}%rec - time

% 8
\gloexe{Glo4:pushakarunki}{}{lt}%
{Chayta pushakarunki\pb{man} \pb{kara}.}%lt que first line
{\morglo{chay-ta}{chay-\lsc{acc}}\morglo{pusha-ka-ru-nki-man}{bring.along-\lsc{passacc}-\lsc{urgt}-\lsc{2}-\lsc{cond}}\morglo{ka-ra}{be-\lsc{pst}}}%morpheme+gloss
\glotran{You \pb{should have} taken her.}{}%eng+spa trans
{}{}%rec - time

% 9
\gloexe{Glo4:chawaru}{}{amv}%
{Mastam chawaru\pb{waq} \pb{karqa}.}%amv que first line
{\morglo{mas-ta-m}{more-\lsc{acc}-\lsc{evd}}\morglo{chawa-ru-waq}{milk-\lsc{urgt}-\lsc{2.cond}}\morglo{ka-rqa}{be-\lsc{pst}}}%morpheme+gloss
\glotran{You \pb{could have} milked more.}{}%eng+spa trans
{}{}%rec - time

% 10
\gloexe{Glo4:rikisun}{}{amv}%
{¿Chay rikisun kayarachu? Rikushpatr miku\pb{chuwan kara}.}% que first line
{\morglo{chay}{\lsc{dem.d}}\morglo{rikisun}{cheese.curd}\morglo{ka-ya-ra-chu}{be-\lsc{prog}-\lsc{pst}-\lsc{q}}\morglo{riku-shpa-tr}{go-\lsc{subis}-\lsc{evc}}\morglo{miku-chuwan}{eat-\lsc{1pl.cond}}\morglo{ka-ra}{be-\lsc{pst}}}%morpheme+gloss
\glotran{Was there the cheese curd? We \pb{could have} gone and eaten it.}{}%eng+spa trans
{}{}%rec - time

\clearpage
\begin{landscape}
\small\centering
% TABLE 23a
\captionof{table}{Past conditional inflection}\label{Tab23a}\par
\begin{tabular}{lll}
\lsptoprule
Person		& Singular		& Plural	\\
\midrule
1 & -y-man karqa-\uo\tss{\AMV}		& -nchik-man karqa-\uo\tss{\AMV}		\\
 & -y-man kara-\uo\tss{\LT}		& -nchik-man kara-\uo\tss{\ACH,\LT,\SP}	\\
 & -:-man kara-\uo\tss{\ACH,\SP}	& -nchik-man kala-\uo\tss{\CH}		\\
 & -:-man kala-\uo\tss{\CH}		& -chuwan karqa-\uo\tss{\AMV}		\\
 & 								& -chuwan kara-\uo\tss{\ACH,\LT}		\\[2ex]
%\midrule
2 & -nki-man karqa-\uo\tss{\AMV}		& -nki-man karqa-\uo\tss{\AMV}		\\
 & -nki-man kara-\uo\tss{\ACH,\LT,\SP}	& -nki-man kara-\uo\tss{\ACH,\LT,\SP}		\\
 & -nki-man kala-\uo\tss{\CH}			& -nki-man kala-\uo\tss{\CH}		\\
 & -waq karqa-\uo\tss{\AMV}			& -waq karqa-\uo\tss{\AMV}		\\[2ex]
%\midrule
3 & -n-man karqa-\uo\tss{\AMV}			& -n-man karqa-\uo\tss{\AMV}		\\
 & -n-man kara-\uo\tss{\ACH,\SP,\LT}	& -n-man kara-\uo\tss{\ACH,\SP,\LT} 		\\
 & -n-man kala-\uo\tss{\CH}			& -n-man kala-\uo\tss{\CH}		\\
\lspbottomrule
\end{tabular}

\bigskip
% TABLE 23b
\captionof{table}{Past conditional inflection -- actor-object suffixes}\label{Tab23b}\par
\begin{tabular}{@{\hspace{1ex}}l@{\hspace{1.5ex}}l@{\hspace{1.5ex}}l@{\hspace{1.5ex}}l@{\hspace{1.5ex}}l@{\hspace{1ex}}}
\lsptoprule
2>1		& 3>1		& 3>1pl	& 1>2	& 3>2	\\
\midrule
-wa-nki-man ka-rqa\tss{\AMV}	& -wa-n-man ka-rqa\tss{\AMV}	&	-wa-nchik-man ka-rqa\tss{\AMV}	& -yki-man ka-rqa \tss{\AMV}	& -shu-nki-man ka-rqa\tss{\AMV} \\
-wa-nki-man ka-ra\tss{\LT}	& -wa-n-man ka-ra\tss{\LT}	&	-wa-nchik-man ka-ra\tss{\LT}	&	-yki-man ka-ra \tss{\LT}	&	-shu-nki-man ka-ra\tss{\LT} \\
-ma-nki-man ka-ra\tss{\ACH,\SP}	&	-ma-n-man ka-ra\tss{\ACH,\SP}	&	-ma-nchik-man ka-ra\tss{\ACH,\SP}	&	 	&	\\
-ma-nki-man ka-la\tss{\CH}	&	-ma-n-man ka-la\tss{\CH}	&	-ma-nchik-man ka-la\tss{\CH}	&	 	&	 	\\
\lspbottomrule
\end{tabular}
\end{landscape}

\subsection{Imperative and injunctive}\label{ssec:impinj}
\subsubsection{Imperative \phono{-y}}\index[sub]{imperative}
\phono{-y} indicates the second-person singular imperative~(\ref{Glo4:kullarnikitaqa}).\\

% 1
\gloexe{Glo4:kullarnikitaqa}{}{amv}%
{¡Chay kullarnikitaqa surquru\pb{y}!}%amv que first line
{\morglo{chay}{\lsc{dem.d}}\morglo{kullar-ni-ki-ta-qa}{necklace-\lsc{euph}-\lsc{2}-\lsc{acc}-\lsc{top}}\morglo{surqu-ru-y}{take.out-\lsc{urgt}-\lsc{imp}}}%morpheme+gloss
\glotran{That necklace of yours, \pb{take} it out!}{}%eng+spa trans
{}{}%rec - time

\noindent
\phono{-y} is suffixed to the verb stem, plus derivational suffixes, if any are present~(\ref{Glo4:Wanura}).\\

% 2
\gloexe{Glo4:Wanura}{}{ach}%
{¡Wañura\pb{chi}\pb{y} wakta!}%ach que first line
{\morglo{wañu-ra-chi-y}{die-\lsc{urgt}-\lsc{caus}-\lsc{imp}}\morglo{wak-ta}{\lsc{dem.d}-\lsc{acc}}}%morpheme+gloss
\glotran{\pb{Kill} that one!}{}%eng+spa trans
{}{}%rec - time

\noindent
When verb has a first-person singular direct or indirect object, \phono{-y} attaches to the 2>1 actor-object suffix \phono{-ma/wa}~(\ref{Glo4:qacha}), (\ref{Glo4:Samaykachilla}).\\

% 3
\gloexe{Glo4:qacha}{}{sp}%
{¡Ñuqamanpis qacha\pb{may}!}%sp que first line
{\morglo{ñuqa-man-pis}{I-\lsc{all}-\lsc{add}}\morglo{qacha-ma-y}{rip-\lsc{1.obj}-\lsc{imp}}}%morpheme+gloss
\glotran{\pb{Rip} it for \pb{me}, too!}{}%eng+spa trans
{}{}%rec - time

% 4
\gloexe{Glo4:Samaykachilla}{}{amv}%
{¡Samaykachilla\pb{way}, awilita!}%amv que first line
{\morglo{sama-yka-chi-lla-wa-y}{rest-\lsc{excep}-\lsc{caus}-\lsc{rstr}-\lsc{1.obj}-\lsc{imp}}\morglo{awilita}{grandmother}}%morpheme+gloss
\glotran{Just \pb{make (have/let)} me rest, grandmother!}{}%eng+spa trans
{}{}%rec - time

\noindent
The second-person plural imperative may be indicated by the joint action derivational suffix, \phono{-pa(:)kU}, immediately When it precedes \phono{-y}, and \phono{-ma/wa}, if present~(\ref{Glo4:pakuy}), (\ref{Glo4:Taki}).\\

% 5
\gloexe{Glo4:pakuy}{}{amv}%
{¡Lluqsi\pb{pakuy} (llapayki)!~\updag}%amv que first line
{\morglo{lluqsi-paku-y}{go.out-\lsc{jtacc}-\lsc{imp}}\morglo{(llapa-yki)}{all-\lsc{2}}}%morpheme+gloss
\glotran{Leave.\pb{\lsc{pl}!}}{}%eng+spa trans
{}{}%rec - time

% 6
\gloexe{Glo4:Taki}{}{ach}%
{¡Taki\pb{pakuy}!~\updag}%ach que first line
{\morglo{taki-paku-y}{sing-\lsc{jtacc}-\lsc{imp}}}%morpheme+gloss
\glotran{Sing \pb{\lsc{pl}}!}{}%eng+spa trans
{}{}%rec - time

\noindent
The first-person plural imperative is identical to the first person plural future: it is indicated by the suffix \phono{-shun}~(\ref{Glo4:Tushu}),~(\ref{Glo4:Kuskallam}).\\

% 7
\gloexe{Glo4:Tushu}{}{amv}%
{¡Tushu\pb{shun}!}%amv que first line
{\morglo{tushu-shun}{dance-\lsc{1pl.fut}}}%morpheme+gloss
\glotran{\pb{Let’s} dance!}{}%eng+spa trans
{}{}%rec - time

% 8
\gloexe{Glo4:Kuskallam}{}{lt}%
{¡Kuskallam wañuku\pb{shun}!}%lt que first line
{\morglo{kuska-lla-m}{together-\lsc{rstr}-\lsc{evd}}\morglo{wañu-ku-shun}{die-\lsc{refl}-\lsc{1pl.fut}}}%morpheme+gloss
\glotran{\pb{Let’s} die together!}{}%eng+spa trans
{}{}%rec - time

\noindent
Prohibitions are formed by suffixing the imperative with \phono{-chu} and preceding it with \phono{ama}~(\ref{Glo4:diharama}--\ref{Glo4:katraykanaku}).\\

% 9
\gloexe{Glo4:diharama}{}{ach}%
{“¡\pb{Ama}yá diharama\pb{y}chu!” nishpa lukuyakuyan.}%ach que first line
{\morglo{ama-yá}{\lsc{proh}-\lsc{emph}}\morglo{diha-ra-ma-y-chu}{leave-\lsc{urgt}-\lsc{1.obj}-\lsc{imp}-\lsc{neg}}\morglo{ni-shpa}{say-\lsc{subis}}\morglo{luku-ya-ku-ya-n}{crazy-\lsc{inch}-\lsc{refl}-\lsc{prog}-\lsc{3}}}%morpheme+gloss
\glotran{“\pb{Don’t} leave me!” he said, going crazy.}{}%eng+spa trans
{}{}%rec - time

% 10
\gloexe{Glo4:imanama}{}{ch}%
{¡\pb{Ama} ñuqaktaqa imanama\pb{y}pis\pb{chu}!}%ch que first line
{\morglo{ama}{\lsc{proh}}\morglo{ñuqa-kta-qa}{I-\lsc{add}-\lsc{top}}\morglo{ima-na-ma-y-pis-chu}{what-\lsc{vrbz}-\lsc{1.obj}-\lsc{imp}-\lsc{add}-\lsc{neg}}}%morpheme+gloss
\glotran{\pb{Don’t} do anything to me!}{}%eng+spa trans
{}{}%rec - time

% 11
\gloexe{Glo4:manchari}{}{amv}%
{¡\pb{Ama} manchari\pb{y}chu! ¡\pb{Ama} qawa\pb{y}chu!}%amv que first line
{\morglo{ama}{\lsc{proh}}\morglo{mancha-ri-y-chu}{scare-\lsc{incep}-\lsc{imp}-\lsc{neg}}\morglo{ama}{\lsc{proh}}\morglo{qawa-y-chu}{look-\lsc{imp}-\lsc{neg}}}%morpheme+gloss
\glotran{\pb{Don’t} be scared! Don’t look!}{}%eng+spa trans
{}{}%rec - time

% 12
\gloexe{Glo4:katraykanaku}{}{lt}%
{¡\pb{Ama}m nunka katraykanaku\pb{shun}chu!}%lt que first line
{\morglo{ama-m}{\lsc{proh}-\lsc{evd}}\morglo{nunka}{never}\morglo{katra-yka-naku-shun-chu}{release-\lsc{excep}-\lsc{recp}-\lsc{1pl.fut}-\lsc{neg}}}%morpheme+gloss
\glotran{\pb{Let’s never} leave each other!}{}%eng+spa trans
{}{}%rec - time

\noindent
\phono{¡Haku!} ‘Let’s go!’ is irregular: it cannot be negated or inflected~(\ref{Glo4:pakananpaq}), (\ref{Glo4:shunchu}), except, optionally, with the first-person plural -\phono{nchik}.\\

% 13
\gloexe{Glo4:pakananpaq}{}{amv}%
{¡\pb{Haku}ña, taytay, pakananpaq chay aychata!}%amv que first line
{\morglo{haku-ña,}{let’s.go-\lsc{disc}}\morglo{tayta-y}{father-\lsc{1}}\morglo{paka-na-n-paq}{hide-\lsc{nmlz}-\lsc{3}-\lsc{purp}}\morglo{chay}{\lsc{dem.d}}\morglo{aycha-ta}{meat-\lsc{acc}}}%morpheme+gloss
\glotran{\pb{Let’s go}, mate, so he can hide this meat!}{}%eng+spa trans
{}{}%rec - time

% 14
\gloexe{Glo4:shunchu}{}{amv}%
{¡\pb{Ama} ri\pb{shunchu} (*haku)!}%amv que first line
{\morglo{ama}{\lsc{proh}}\morglo{ri-shun-chu}{go-\lsc{1pl.fut}-\lsc{neg}}}%morpheme+gloss
\glotran{\pb{Let’s not go}!’ ‘\pb{We shouldn’t go}.}{}%eng+spa trans
{}{}%rec - time

\noindent
The second-person future tense, too, is often interpreted as an imperative~(\ref{Glo4:Diosninchikqa}), and prohibitions can be formed by preceding this with \phono{ama}~(\ref{Glo4:kutimu}).\\

% 15
\gloexe{Glo4:Diosninchikqa}{}{lt}%
{Diosninchikqa nin, “¡Iha, apa\pb{nki} pukatrakita, wamanripata!”}%lt que first line
{\morglo{Dios-ni-nchik-qa}{God-\lsc{euph}-\lsc{1pl}-\lsc{top}}\morglo{ni-n}{say-\lsc{3}}\morglo{iha}{daughter}\morglo{apa-nki}{bring-\lsc{2}}\morglo{pukatraki-ta}{pukatraki.flower-\lsc{acc}}\morglo{wamanripa-ta}{wamanripa.flower-\lsc{acc}}}%morpheme+gloss
\glotran{Our God said, “Daughter, \pb{bring} pukatraki plants and wamanripa plants!”}{}%eng+spa trans
{}{}%rec - time

% 16
\gloexe{Glo4:kutimu}{}{ch}%
{¡\pb{Ama} kutimu\pb{nki}chu! Qamqa isturbum kayanki.}%ch que first line
{\morglo{ama}{\lsc{proh}}\morglo{kuti-mu-nki-chu}{return-\lsc{cisl}-\lsc{2}-\lsc{neg}}\morglo{qam-qa}{you-\lsc{top}}\morglo{isturbu-m}{nuisance-\lsc{evd}}\morglo{ka-ya-nki}{be-\lsc{prog}-\lsc{2}}}%morpheme+gloss
\glotran{\pb{Don’t come} back! You’re being a nuisance.}{}%eng+spa trans
{}{}%rec - time

\subsubsection{Injunctive \phono{-chun}}\index[sub]{injunctive}
\phono{-chun} indicates the third person injunctive~(\ref{Glo4:Kukantaraq}--\ref{Glo4:Witrqa}), the suggestion on the part of the speaker as to the advisability of action by a third party.\\

% 1
\gloexe{Glo4:Kukantaraq}{}{amv}%
{¡Kukantaraq akuyku\pb{chun}!}%amv que first line
{\morglo{kuka-n-ta-raq}{coca-\lsc{3}-\lsc{acc}-\lsc{cont}}\morglo{aku-yku-chun}{chew-\lsc{excep}-\lsc{injunc}}}%morpheme+gloss
\glotran{\pb{Let her} take her coca still!}{}%eng+spa trans
{}{}%rec - time

% 2
\gloexe{Glo4:Uqusakuna}{}{amv}%
{¡Uqusakuna hinalla ka\pb{chun}!}%amv que first line
{\morglo{uqu-sa-kuna}{wet-\lsc{prf}-\lsc{pl}}\morglo{hina-lla}{thus-\lsc{rstr}}\morglo{ka-chun}{be-\lsc{injunc}}}%morpheme+gloss
\glotran{\pb{Let the} wet ones be like that!}{}%eng+spa trans
{}{}%rec - time

% 3
\gloexe{Glo4:Witrqa}{}{amv}%
{¡Witrqa\pb{chun} piliyaqkunata kalabusupi!}%amv que first line
{\morglo{witrqa-chun}{close.in-\lsc{injunc}}\morglo{piliya-q-kuna-ta}{fight-\lsc{ag}-\lsc{pl}-\lsc{acc}}\morglo{kalabusu-pi}{prison-\lsc{loc}}}%morpheme+gloss
\glotran{Let them shut the brawlers up in the prison!}{}%eng+spa trans
{}{}%rec - time

\noindent
There are no first or second person injunctive suffixes. \phono{-chun} attaches to the verb stem, plus derivational suffixes, if any are present~(\ref{Glo4:muchun}--\ref{Glo4:Hinallana}).\\

% 4
\gloexe{Glo4:muchun}{}{ach}%
{¡Kuti\pb{muchun}! Wañuchina:paq.}%ach que first line
{\morglo{kuti-mu-chun}{return-\lsc{cisl}-\lsc{injunc}}\morglo{wañu-chi-na-:-paq}{die-\lsc{caus}-\lsc{nmlz}-\lsc{1}-\lsc{purp}}}%morpheme+gloss
\glotran{\pb{Have him} come back -- so I can kill him!}{}%eng+spa trans
{}{}%rec - time

% 5
\gloexe{Glo4:Papaniy}{}{amv}%
{Papaniy wañu\pb{ku}\pb{chun}pis wamran kawsa\pb{ku}\pb{chun} ninshi. Chaykunata upyachiwaptinshi kawsakurqani.}%amv que first line
{\morglo{papa-ni-y}{father-\lsc{euph}-\lsc{1}}\morglo{wañu-ku-chun-pis}{die-\lsc{refl}-\lsc{injunc}-\lsc{add}}\morglo{wamra-n}{child-\lsc{3}}\morglo{kawsa-ku-chun}{live-\lsc{refl}-\lsc{injunc}}\morglo{ni-n-shi}{say-\lsc{3}-\lsc{evr}}\morglo{chay-kuna-ta}{\lsc{dem.d}-\lsc{pl}-\lsc{acc}}\morglo{upya-chi-wa-pti-n-shi}{drink-\lsc{caus}-\lsc{1.obj}-\lsc{subds}-\lsc{3}-\lsc{evr}}\morglo{kawsa-ku-rqa-ni}{live-\lsc{refl}-\lsc{pst}-\lsc{1}}}%morpheme+gloss
\glotran{\pb{Let} him die; \pb{let} his child live, my father said, they say. When they made me take those [cures], I lived.}{}%eng+spa trans
{}{}%rec - time

% 6
\gloexe{Glo4:Hinallana}{}{lt}%
{¡Hinallaña ka\pb{ya}\pb{chun}!}%lt que first line
{\morglo{hina-lla-ña}{thus-\lsc{rstr}-\lsc{disc}}\morglo{ka-ya-chun}{be-\lsc{prog}-\lsc{injunc}}}%morpheme+gloss
\glotran{\pb{Let it} be just like that!}{}%eng+spa trans
{}{}%rec - time

\noindent
It simultaneously indicates injunctivity and person, and is in complementary distribution with other inflectional suffixes. The negative injunctive is formed by suffixing \phono{-chu} to the injunctive and preceding it with \phono{ama}~(\ref{Glo4:lluqsilluqsi}),~(\ref{Glo4:palumaqa}).\\

% 7
\gloexe{Glo4:lluqsilluqsi}{}{ch}%
{¡\pb{Ama} lluqsi\pb{chunchu} tukuy puntraw!}%ch que first line
{\morglo{ama}{\lsc{proh}}\morglo{lluqsi-chun-chu}{go.out-\lsc{injunc}-\lsc{neg}}\morglo{tukuy}{all}\morglo{puntraw}{day}}%morpheme+gloss
\glotran{\pb{Don’t let him} leave all day!}{}%eng+spa trans
{}{}%rec - time

% 8
\gloexe{Glo4:palumaqa}{}{ach}%
{Ishkay palumaqa nin, “¡\pb{Ama} yantataqa apaya\pb{chunchu}!”}%ach que first line
{\morglo{ishkay}{two}\morglo{paluma-qa}{dove-\lsc{top}}\morglo{ni-n}{say-\lsc{3}}\morglo{ama}{\lsc{proh}}\morglo{yanta-ta-qa}{firewood-\lsc{acc}-\lsc{top}}\morglo{apa-ya-chun-chu}{bring-\lsc{prog}-\lsc{injunc}-\lsc{neg}}}%morpheme+gloss
\glotran{The two doves said, “\pb{Don’t let them} bring the firewood!”}{}%eng+spa trans
{}{}%rec - time

\noindent
The third-person future tense can sometimes be interpreted as an injunctive~(\ref{Glo4:pampankichu}).\\

% 9
\gloexe{Glo4:pampankichu}{}{ach}%
{Wañuchiptin, ‘¡\pb{Amam pampankichu}! ¡Hinam ismu\pb{nqa}!’ ninshi.}%ach que first line
{\morglo{wañu-chi-pti-n}{die-\lsc{caus}-\lsc{subds}-\lsc{3}}\morglo{ama-m}{\lsc{proh}-\lsc{evd}}\morglo{pampa-nki-chu}{bury-\lsc{2}-\lsc{neg}}\morglo{hina-m}{thus-\lsc{evd}}\morglo{ismu-nqa}{rot-\lsc{3.fut}}\morglo{ninshi}{say-\lsc{3}-\lsc{evr}}}%morpheme+gloss
\glotran{When they killed him, “\pb{Don’t bury} him! \pb{Let him rot} like that!” he said.}{}%eng+spa trans
{}{}%rec - time

\subsection{Aspect}\label{ssec:aspect}
In \SYQ, continuous aspect is indicated by \phono{-ya}. \phono{-ya} belongs to the set of derivational affixes. Unlike inflectional morphemes, \phono{-ya} can appear in subordinate clauses and nominalizations (\phono{puñu-ya-pti-n} ‘when he is sleeping’; \phono{ruwa-ya-q} ‘one who is making’) and can --~and, indeed, sometimes must~-- precede some derivational suffixes (\phono{miku-ya-chi-n} ‘he is making him eat’). Perfective aspect, generally indicated by \phono{-Ru}, may, in some cases, also be indicated by reflexive \phono{-kU}. §~\ref{ssec:progressive}--\ref{ssec:perfective} cover \phono{-ya} and \phono{-kU}, respectively.

\subsubsection{Continuous \phono{-ya}}\label{ssec:progressive}\index[sub]{progressive}
All dialects of \SYQ{} indicate continuous aspect with \phono{-ya}. \phono{-ya} marks both the progressive~(\ref{Glo4:Lliwmantriki}--\ref{Glo4:Uchuypis}) and durative components~(\ref{Glo4:Pipis}), (\ref{Glo4:Hitakaruyta}) of the continuous, indicating both actions and states continuing in time.\\

% 1
\gloexe{Glo4:Lliwmantriki}{}{amv}%
{Lliwmantriki invita\pb{ya}n payqa.}%amv que first line
{\morglo{lliw-man-tri-ki}{all-\lsc{all}-\lsc{evc}-\lsc{iki}}\morglo{invita-ya-n}{invite-\lsc{prog}-\lsc{3}}\morglo{pay-qa}{she-\lsc{top}}}%morpheme+gloss
\glotran{She must \pb{be inviting} everyone, for sure, her.}{}%eng+spa trans
{}{}%rec - time

% 2
\gloexe{Glo4:napanapa}{}{ch}%
{Kumunidadllañam napa:ku\pb{ya}: trabahapa:ku\pb{ya}:.}%ch que first line
{\morglo{kumunidad-lla-ña-m}{community-\lsc{rstr}-\lsc{disc}-\lsc{evd}}\morglo{na-pa:ku-ya-:}{\lsc{dmy}-\lsc{jtacc}-\lsc{prog}-\lsc{1}}\morglo{trabaha-pa:ku-ya-:.}{work-\lsc{jtacc}-\lsc{prog}-\lsc{1}}}%morpheme+gloss
\glotran{Just the community, \pb{we’re doing} it, \pb{we’re working}.}{}%eng+spa trans
{}{}%rec - time

% 3
\gloexe{Glo4:Walmikunaqa}{}{ch}%
{Walmikunaqa talpu\pb{ya}: allicha\pb{ya}: kulpakta maqa\pb{ya}:.}%ch que first line
{\morglo{walmi-kuna-qa}{woman-\lsc{pl}-\lsc{top}}\morglo{talpu-ya-:}{plant-\lsc{prog}-\lsc{1}}\morglo{alli-cha-ya-:}{good-\lsc{fact}-\lsc{prog}-\lsc{1}}\morglo{kulpa-kta}{clod-\lsc{acc}}\morglo{maqa-ya-:}{hit-\lsc{prog}-\lsc{1}}}%morpheme+gloss
\glotran{The women \pb{are planting}, \pb{improving}, \pb{hitting} big clumps of earth.}{}%eng+spa trans
{}{}%rec - time

% 4
\gloexe{Glo4:Imatatrik}{}{ach}%
{¿Imatatrik ruwa\pb{ya}n? Trabaha\pb{ya}ntriki.}%ach que first line
{\morglo{ima-ta-tri-k}{what-\lsc{acc}-\lsc{evc}-\lsc{k}}\morglo{ruwa-ya-n}{make-\lsc{prog}-\lsc{3}}\morglo{trabaha-ya-n-tri-ki}{work-\lsc{prog}-\lsc{3}-\lsc{evc}-\lsc{iki}}}%morpheme+gloss
\glotran{What \pb{is} he do\pb{ing}? He must \pb{be} work\pb{ing}.}{}%eng+spa trans
{}{}%rec - time

% 5
\gloexe{Glo4:Diosninchik}{}{lt}%
{Chayshi Diosninchik, “¿Imatam ashi\pb{ya}nki?” nin.}%lt que first line
{\morglo{chay-shi}{\lsc{dem.d}-\lsc{evr}}\morglo{Dios-ni-nchik}{God-\lsc{euph}-\lsc{1pl}}\morglo{ima-ta-m}{what-\lsc{acc}-\lsc{evd}}\morglo{ashi-ya-nki}{look.for-\lsc{prog}-\lsc{2}}\morglo{ni-n}{say-\lsc{3}}}%morpheme+gloss
\glotran{Then Our God said, “What \pb{are} you search\pb{ing} for?”}{}%eng+spa trans
{}{}%rec - time

% 6
\gloexe{Glo4:Uchuypis}{}{lt}%
{Uchuypis pasapasaypaqmi chakirun, uchuypis chakisham ka\pb{ya}n.}%lt que first line
{\morglo{uchu-y-pis}{chili-\lsc{1}-\lsc{add}}\morglo{pasa-pasaypaq-mi}{complete-completely-\lsc{evd}}\morglo{chaki-ru-n,}{dry-\lsc{urgt}-\lsc{3}}\morglo{uchu-y-pis}{chili-\lsc{1}-\lsc{add}}\morglo{chaki-sha-m}{dry-\lsc{prf}-\lsc{evd}}\morglo{ka-ya-n}{be-\lsc{prog}-\lsc{3}}}%morpheme+gloss
\glotran{The chilies completely dried out; the chilies \pb{are} dried out.}{}%eng+spa trans
{}{}%rec - time

% 7
\gloexe{Glo4:Pipis}{}{amv}%
{Pipis. Ñuqa ukupaw kaku\pb{ya}ni.}%amv que first line
{\morglo{pi-pis}{who-\lsc{add}}\morglo{ñuqa}{I}\morglo{ukupaw}{busy}\morglo{ka-ku-ya-ni}{be-\lsc{refl}-\lsc{prog}-\lsc{1}}}%morpheme+gloss
\glotran{No one. \pb{I’m busy}.}{}%eng+spa trans
{}{}%rec - time

% 8
\gloexe{Glo4:Hitakaruyta}{}{amv}%
{Hitakaruyta muna\pb{ya}ni.}%amv que first line
{\morglo{hita-ka-ru-y-ta}{fall-\lsc{passacc}-\lsc{urgt}-\lsc{inf}-\lsc{acc}}\morglo{muna-ya-ni}{wany-\lsc{prog}-\lsc{1}}}%morpheme+gloss
\glotran{I \pb{want} to fall.}{}%eng+spa trans
{}{}%rec - time

\noindent
\phono{-ya} may be used with or in place of \phono{-q} to mark habitual action~(\ref{Glo4:suliyasaka}--\ref{Glo4:Fwirsawan}) when such action is customary.\footnote{An anonymous reviewer points out that \phono{-ya} in Yauyos seems to resemble the cognate suffix \phono{-yka}: in Huallaga Q, which \citet{Weber89} calls a general imperfective. The cognate suffix in South Conchucos Q, \phono{-yka}, in contrast, does not appear in habitual contexts. \citet{Hintz} observes that while it is not a general imperfective, it is still much broader than a simple progressive; Hintz concludes that \phono{-yka}: in South Conchucos is continuous aspect.}\\

% 9
\gloexe{Glo4:suliyasaka}{}{amv}%
{Mana suliyasa kaptinqa wakta suliyachi\pb{ya}nchik.}%amv que first line
{\morglo{mana}{no}\morglo{suliya-sa}{sun-\lsc{prf}}\morglo{ka-pti-n-qa}{be-\lsc{subds}-\lsc{3}-\lsc{top}}\morglo{wak-ta}{\lsc{dem.d}-\lsc{acc}}\morglo{suliya-chi-ya-nchik}{sun-\lsc{caus}-\lsc{prog}-\lsc{1pl}}}%morpheme+gloss
\glotran{When [the oca] hasn’t been sunned, we \pb{sun} it.}{}%eng+spa trans
{}{}%rec - time

% 10
\gloexe{Glo4:Uyqapa}{}{amv}%
{Uyqapa millwantam kaypaq puchka\pb{ya}nchik.}%amv que first line
{\morglo{uyqa-pa}{sheep-\lsc{gen}}\morglo{millwa-n-ta-m}{wool-\lsc{3}-\lsc{acc}-\lsc{evd}}\morglo{kay-paq}{\lsc{dem.p}-\lsc{abl}}\morglo{puchka-ya-nchik}{spin-\lsc{prog}-\lsc{1pl}}}%morpheme+gloss
\glotran{We \pb{spin} sheep’s wool here.}{}%eng+spa trans
{}{}%rec - time

% 11
\gloexe{Glo4:Fwirsawan}{}{ach}%
{Fwirsawan wawaku\pb{ya}nchik.}%ach que first line
{\morglo{fwirsa-wan}{force-\lsc{instr}}\morglo{wawa-ku-ya-nchik}{give.birth-\lsc{refl}-\lsc{prog}-\lsc{1pl}}}%morpheme+gloss
\glotran{With effort, we \pb{give birth}.}{}%eng+spa trans
{}{}%rec - time

\noindent
\phono{-ya} can appear in subordinate clauses~(\ref{Glo4:Hinaptinshi}),~(\ref{Glo4:runaqawar}).\\

% 12
\gloexe{Glo4:Hinaptinshi}{}{ach}%
{Hinaptinshi iskinapa ka\pb{yapti}n baliyarun.}%ach que first line
{\morglo{Hinaptin-shi}{then-\lsc{evr}}\morglo{iskina-pa}{corner-\lsc{loc}}\morglo{ka-ya-pti-n}{be-\lsc{prog}-\lsc{subds}-\lsc{3}}\morglo{baliya-ru-n}{shoot-\lsc{urgt}-\lsc{3}}}%morpheme+gloss
\glotran{Then when he \pb{was} in the corner, they shot him.}{}%eng+spa trans
{}{}%rec - time

% 13
\gloexe{Glo4:runaqawar}{}{amv}%
{Wak runaqa warminta wañurachin maqa\pb{yashpa}lla.}%amv que first line
{\morglo{wak}{\lsc{dem.d}}\morglo{runa-qa}{person-\lsc{top}}\morglo{warmi-n-ta}{woman-\lsc{3}-\lsc{acc}}\morglo{wañu-ra-chi-n}{die-\lsc{urgt}-\lsc{caus}-\lsc{3}}\morglo{maqa-ya-shpa-lla}{beat-\lsc{prog}-\lsc{subis}-\lsc{rstr}}}%morpheme+gloss
\glotran{That man, turning jealous, killed his wife, when he \pb{was beating} her.}{}%eng+spa trans
{}{}%rec - time

\noindent
\phono{-ya} preceeds \phono{-mu} and \phono{-chi}~(\ref{Glo4:runata}),~(\ref{Glo4:Ladirankunapaq}) and precedes all inflectional suffixes.\\

% 14
\gloexe{Glo4:runata}{}{lt}%
{Limpu limpu runata firmaka\pb{yachi}n.}%lt que first line
{\morglo{limpu}{all}\morglo{limpu}{all}\morglo{runa-ta}{person-\lsc{acc}}\morglo{firma-ka-ya-chi-n}{sign-\lsc{passacc}-\lsc{prog}-\lsc{caus}-\lsc{3}}}%morpheme+gloss
\glotran{They’re \pb{making} all the people sign.}{}%eng+spa trans
{}{}%rec - time

% 15
\gloexe{Glo4:Ladirankunapaq}{}{ach}%
{Ladirankunapaq rumipis hinku\pb{yamu}ntriki.}%ach que first line
{\morglo{ladira-n-kuna-paq}{hillside-\lsc{3}-\lsc{pl}-\lsc{abl}}\morglo{rumi-pis}{stone-\lsc{add}}\morglo{hinku-ya-mu-n-tri-ki}{roll-\lsc{prog}-\lsc{cisl}-\lsc{evc}-\lsc{iki}}}%morpheme+gloss
\glotran{Stones, too, would \pb{be rolling} down the sides [of the mountain].}{}%eng+spa trans
{}{}%rec - time

\noindent
It forms the present~(\ref{Glo4:Suyaykamay}), past~(\ref{Glo4:saqaykurqa}), (\ref{Glo4:Antaylumata}) and future~(\ref{Glo4:mandaku}) progressive.\\

% 16
\gloexe{Glo4:Suyaykamay}{}{ch}%
{¡Suyaykamay! ¡Qarqaryam qipa:ta shamuku\pb{ya}n!}%ch que first line
{\morglo{suya-yka-ma-y}{wait-\lsc{excep}-\lsc{1.obj}-\lsc{imp}}\morglo{qarqarya-m}{zombie-\lsc{evd}}\morglo{qipa-:-ta}{behind-\lsc{1}-\lsc{acc}}\morglo{shamu-ku-ya-n}{come-\lsc{refl}-\lsc{prog}-\lsc{3}}}%morpheme+gloss
\glotran{Wait for me! A zombie \pb{is coming} behind me!}{}%eng+spa trans
{}{}%rec - time

% 17
\gloexe{Glo4:saqaykurqa}{}{amv}%
{¿Maypa saqaykurqa? Paypis wishtu ka\pb{yarqa}.}%amv que first line
{\morglo{may-pa}{where-\lsc{loc}}\morglo{saqa-yku-rqa}{go.down-\lsc{excep}-\lsc{pst}}\morglo{pay-pis}{she-\lsc{add}}\morglo{wishtu}{lame}\morglo{ka-ya-rqa}{be-\lsc{prog}-\lsc{pst}}}%morpheme+gloss
\glotran{Where did she fall? She, too, \pb{was limping}.}{}%eng+spa trans
{}{}%rec - time

% 18
\gloexe{Glo4:Antaylumata}{}{sp}%
{Antaylumata tarirushpaqa pallaku\pb{yara} hinaptinshi~\dots}%sp que first line
{\morglo{antayluma-ta}{antayluma.berries-\lsc{acc}}\morglo{tari-ru-shpa-qa}{find-\lsc{urgt}-\lsc{subis}-\lsc{top}}\morglo{palla-ku-ya-ra}{pick-\lsc{refl}-\lsc{prog}-\lsc{pst}}\morglo{hina-pti-n-shi}{then-\lsc{evr}}}%morpheme+gloss
\glotran{After finding some antayluma berries, she \pb{was} gather\pb{ing} them up. Then~\dots}{}%eng+spa trans
{}{}%rec - time

% 19
\gloexe{Glo4:mandaku}{}{amv}%
{Vakamik mandaku\pb{yanqa}.}%amv que first line
{\morglo{vaka-mi-k}{cow-\lsc{evd}-\lsc{ik}}\morglo{manda-ku-ya-nqa}{be.in.charge-\lsc{refl}-\lsc{prog}-\lsc{3.fut}}}%morpheme+gloss
\glotran{The cows are \pb{going to be giving} orders.}{}%eng+spa trans
{}{}%rec - time

\subsubsection{Durative \phono{-chka}}\label{ssec:durative}\index[sub]{durative}
\phono{-chka} is very rarely employed, occuring spontaneously in a non-quotative context only seven times in the corpus. Indeed, it is probably best qualified as non-productive in all but \SP. \phono{-chka} is in complementary distribution with continuative \phono{-ya}, but it is more semantically restricted than \phono{-ya}. A -\phono{chka} action or state is necessarily simultaneous with some other action or state, either expilicit in the dialogue~(\ref{Glo4:Kayllapam}),~(\ref{Glo4:kaytam}) or supplied by context~(\ref{Glo4:Aviva}),~(\ref{Glo4:Taqsachkay}).\\

% 1
\gloexe{Glo4:Kayllapam}{}{ach}%
{Kayllapam kwida\pb{chka}nki ñuqaqa aparamu:.}%ach que first line
{\morglo{kay-lla-pa-m}{\lsc{dem.p}-\lsc{rstr}-\lsc{loc}-\lsc{evd}}\morglo{kwida-chka-nki}{care.for	-\lsc{dur}-\lsc{2}}\morglo{ñuqa-qa}{I-\lsc{top}}\morglo{apa-ra-mu-:}{bring-\lsc{urgt}-\lsc{cisl}-\lsc{1}}}%morpheme+gloss
\glotran{\pb{You’ll go on taking} care of this here [\pb{while] I} bring it.}{}%eng+spa trans
{}{}%rec - time

% 2
\gloexe{Glo4:kaytam}{}{sp}%
{Mundum ñitiramashun kaytam sustininkiqa. Kayta sustini\pb{chka}nki ñuqañataqmi huk waklawpis siqaykayamun.}%sp que first line
{\morglo{mundu-m}{world-\lsc{evd}}\morglo{ñiti-ra-ma-shun}{crush-\lsc{urgt}-\lsc{1.obj}-\lsc{1pl.fut}}\morglo{kay-ta-m}{\lsc{dem.p}-\lsc{acc}-\lsc{evd}}\morglo{sustini-nki-qa}{sustain-\lsc{2}-\lsc{top}}\morglo{kay-ta}{\lsc{dem.p}-\lsc{acc}}\morglo{sustini-\pb{chka}-nki}{sustain-\lsc{dur}-\lsc{2}}\morglo{ñuqa-ña-taq-mi}{I-\lsc{disc}-\lsc{seq}-\lsc{evd}}\morglo{huk}{one}\morglo{wak}{\lsc{dem.d}}\morglo{law-pis}{side-\lsc{add}}\morglo{siqa-yka-ya-mu-n}{go.down-\lsc{excep}-\lsc{prog}-\lsc{cisl}-\lsc{3}}}%morpheme+gloss
\glotran{The world is going to crush us. Hold this! \pb{You go on} \pb{holding} this one. \pb{I, too} -- another is falling over there.}{}%eng+spa trans
{}{}%rec - time

% 3
\gloexe{Glo4:Aviva}{}{amv}%
{Aviva, tiya\pb{chka}nki chayllapa.}%amv que first line
{\morglo{Aviva}{Aviva}\morglo{tiya-chka-nki}{sit-\lsc{dur}-\lsc{2}}\morglo{chay-lla-pa}{\lsc{dem.d}-\lsc{rstr}-\lsc{loc}}}%morpheme+gloss
\glotran{Aviva, \pb{you’re going to be sitting} just right there [\pb{while the others} go looking].}{}%eng+spa trans
{}{}%rec - time

% 4
\gloexe{Glo4:Taqsachkay}{}{ch}%
{¡Taqsachkay!~\updag}%ch que first line
{\morglo{taqsa-chka-y}{wash-\lsc{dur}-\lsc{imp}}}%morpheme+gloss
\glotran{You go on wash\pb{ing} [while I play].}{}%eng+spa trans
{}{}%rec - time

\subsubsection{Perfective \phono{-ku}}\label{ssec:perfective}\index[sub]{perfective!\phono{-ku}}
\phono{-ku} may indicate completion of change of position with \phono{ri-} ‘go’ and other verbs of motion~(\ref{Glo4:kidalun}--\ref{Glo4:paqwash}); it also commonly occurs with \phono{wañu-} ‘die’~(\ref{Glo4:Baliyaptinqa}), (\ref{Glo4:Imanarunqatr}). \citet[135]{Adelaar06}\index[aut]{Adelaar, Willem F. H.} writes of Tarma Quechua: “This \phono{-ku-}, probably the result of a functional split of the ‘reflexive’ marker \phono{-ku-}, has acquired a marginal aspectual function and indicates the completion of a change of position.”\\

% 1
\gloexe{Glo4:kidalun}{}{ch}%
{Pashñalla kidalun. ¿Qaliqa \pb{liku}n maytataq?}%ch que first line
{\morglo{pashña-lla}{girl-\lsc{rstr}}\morglo{kida-lu-n}{stay-\lsc{urgt}-\lsc{3}}\morglo{qali-qa}{man-\lsc{top}}\morglo{li-ku-n}{go-\lsc{refl}-\lsc{3}}\morglo{may-ta-taq}{where-\lsc{acc}-\lsc{seq}}}%morpheme+gloss
\glotran{Just the girl stayed. The man \pb{went} where?}{}%eng+spa trans
{}{}%rec - time

% 2
\gloexe{Glo4:quykuptin}{}{amv}%
{Qullqita quykuptin~\dots{} \pb{pasaku}n.}%amv que first line
{\morglo{qullqi-ta}{money-\lsc{acc}}\morglo{qu-yku-pti-n}{give-\lsc{excep}-\lsc{subds}-\lsc{3}}\morglo{pasa-ku-n}{pass-\lsc{refl}-\lsc{3}}}%morpheme+gloss
\glotran{When he gave him the money, he \pb{went away}.}{}%eng+spa trans
{}{}%rec - time

% 3
\gloexe{Glo4:paqwash}{}{lt}%
{\pb{Ripuku}n paqwash llapa wawan tudu \pb{ripuku}n.}%lt que first line
{\morglo{ripu-ku-n}{go-\lsc{refl}-\lsc{3}}\morglo{paqwash}{completely}\morglo{llapa}{all}\morglo{wawa-n}{child-\lsc{3}}\morglo{tudu}{everything}\morglo{ripu-ku-n}{go-\lsc{refl}-\lsc{3}}}%morpheme+gloss
\glotran{Then, he \pb{left} for good -- all his children -- all \pb{left}.}{}%eng+spa trans
{}{}%rec - time

% 4
\gloexe{Glo4:Baliyaptinqa}{}{ach}%
{Baliyaptinqa \pb{wañuku}n.}%ach que first line
{\morglo{baliya-pti-n-qa}{shoot-\lsc{subds}-\lsc{3}-\lsc{top}}\morglo{wañu-ku-n}{die-\lsc{refl}-\lsc{3}}}%morpheme+gloss
\glotran{When they shot him, he \pb{died}.}{}%eng+spa trans
{}{}%rec - time

% 5
\gloexe{Glo4:Imanarunqatr}{}{ach}%
{¿Imanarunqatr? \pb{Wañuku}ntri.}%ach que first line
{\morglo{ima-na-ru-nqa-tr}{what-\lsc{vrbz}-\lsc{urgt}-\lsc{3.fut}-\lsc{evc}}\morglo{wañu-ku-n-tri}{die-\lsc{refl}-\lsc{3}-\lsc{evc}}}%morpheme+gloss
\glotran{What will happen? He must have \pb{died}.}{}%eng+spa trans
{}{}%rec - time

\subsection{Subordination}\label{ssec:subordination}\index[sub]{subordination}
\SYQ{} counts three subordinating suffixes --~\phono{-pti}, \phono{-shpa}, and \phono{-shtin}~-- and one subordinating structure --~\phono{-na-}\lsc{poss}\phono{-kama.} In addition, the nominalizing suffixes, \phono{-na}, \phono{-q}, \phono{-sa}, and \phono{-y} form subordinate relative and complement clauses (see~§~\ref{ssec:sdfv}).

\phono{-pti} is employed when the subjects of the main and subordinate clauses are different (\phono{Huk} \phono{qawa-\pb{pti-n-qa}}, \phono{ñuqa-nchik} \phono{qawa-nchik-chu} ‘Although \pb{others} see, we don’t see’); \phono{shpa} and \phono{-shtin} are employed when the subjects of the two clauses are identical (\phono{tushu\pb{-shpa}}/\phono{\pb{-shtin}} \phono{wasi-ta} \phono{kuti-mu-n} ‘Dancing they return home’). Cacra, but not Hongos, employs \phono{-r} (realized \textipa{[l]}) in place of \phono{-shpa} (\phono{traqna\pb{-l}} \phononb{pusha-la-mu-n} ‘binding his hands and feet, they took him along’). \phono{-pti} generally indicates that the event of the subordinated clause began prior to that of the main clause but may also be employed in the case the events of the two clauses are simultaneous (\phono{urkista-qa} \phono{traya-mu\pb{-pti}-n} \phono{tushu-rqa-nchik} ‘When the band arrived, we danced’). \phono{-shpa} generally indicates that the event of the subordinated clause is simultaneous with that of the main clause (\phono{Sapu-qa} \phono{kurrkurrya-\pb{shpa}} \phono{kurri-ya-n} ‘The frog is running going \emph{kurr-kurr!}’) but may also be employed when event of the subordinated event precedes that of the main clause. \phono{-shtin} is employed only when the main and subordinate clause events are simultaneous (\phono{awa-\pb{shtin}} \phono{miku-chi-ni} \phono{wamra-y-ta} ‘(By) weaving, I feed my children’). \phono{-pti} subordinates are suffixed with allocation suffixes (\phono{tarpu\pb{-pti}-\pb{nchik}} ‘when we plant’); \phono{-shpa} and \phono{-shtin} subordinates do not inflect for person or number (\phononb{*tarpu-\pb{shpa}-\pb{nchik}}; \phononb{*tarpu-\pb{shtin}-\pb{yki}}).\
{\phono{-shpa} appears 1432 times in the corpus; in three instances it is inflected for person. In elicitation, speakers adamantly reject the use of personal suffixes after \phono{-shpa}.} Subordinate verbs are never suffixed with any other inflectional morphemes, with the exception of \phono{-ya} (\phono{*tarpu\pb{-rqa}-\pb{shpa}}; \phono{*tarpu-\pb{shaq}-\pb{shpa}}). The evidentials, \phono{-mI}, \phono{shI}, and \phono{-trI} cannot appear on the interior of subordinate clauses, and the negative particle \phono{-chu} can neither appear on the interior nor suffix to subordinate clauses (\phono{mana-m rima-pti-ki} (\phono{*chu}) ‘if you don’t talk’). Subordinate verbs inherit tense, aspect and conditionality specification from the main clause verb (\phono{ri-shpa} \phono{qawa-\pb{y-man}} \phono{\pb{karqa}} ‘If I \pb{would have} gone, I \pb{would have} seen’). Depending on the context, \phono{-pti} and \phono{-shpa} can be translated by ‘when’, ‘if’, ‘because’, ‘although’, or with a gerund; \phono{-shtin} can be translated by a gerund only. This information is summarized in Table~\ref{Tab24}.

% TABLE 24
\begin{table}[!ht]
\small\centering
\caption{Subordinating suffixes}\label{Tab24}
\begin{tabularx}{\textwidth}{lLL}
\lsptoprule
&Subordinate-clause event begins \emph{before} main-clause event & Subordinate-clause event \emph{simultaneous} with main-clause event		\\
\midrule
Identical Subjects	& \phono{-shpa}		& \phono{-shpa}, \phono{-shtin}	\\
Different Subjects	& \phono{-pti}		& \phono{-pti}					\\
\lspbottomrule
\end{tabularx}
\end{table}

\phono{-na-}\lsc{poss}\phono{-kama} is limitative. It forms subordinate clauses indicating that the event referred to either (1)~is simultaneous with or (2)~limits the event referred to in the main clause (\phono{puñu\pb{-na-y-kama}} ‘while I was sleeping’; \phono{wañu-na-n-kama} ‘until she died’). 

\subsubsection{Different subjects \phono{-pti}}\index[sub]{different subjects}
\phono{-pti} is employed when the subjects in the main and subordinated clauses are different~(\ref{Glo4:Aruschata}), (\ref{Glo4:tiniynti}) and the event of the subordinated clause begins before~(\ref{Glo4:kundurqa}) or is simultaneous with~(\ref{Glo4:mumintu}) the event of the main clause.\\

% 1
\gloexe{Glo4:Aruschata}{}{amv}%
{¿Aruschata kumbida\pb{ptinchik} miku\pb{n}manchu?}%amv que first line
{\morglo{arus-cha-ta}{rice-\lsc{dim-\lsc{acc}}}\morglo{kumbida-pti-nchik}{share-\lsc{subds}-\lsc{1pl}}\morglo{miku-n-man-chu}{eat-\lsc{3}-\lsc{cond}-\lsc{q}}}%morpheme+gloss
\glotran{\pb{If we} share the rice, will \pb{she} eat it?}{}%eng+spa trans
{}{}%rec - time

% 2
\gloexe{Glo4:tiniynti}{}{ch}%
{\pb{Qusa:} tiniynti alkaldi ka\pb{ptin}, “Kumpañira, ¿maypim qusayki?” nima\pb{n}.}%ch que first line
{\morglo{qusa-:}{husband-\lsc{1}}\morglo{tiniynti}{lieutenant}\morglo{alkaldi}{mayor}\morglo{ka-pti-n}{be-\lsc{subds}-\lsc{3}}\morglo{kumpañira}{compañera}\morglo{may-pi-m}{where-\lsc{loc}-\lsc{evd}}\morglo{qusa-yki}{husband-\lsc{2}}\morglo{ni-ma-n}{say-\lsc{1.obj}-\lsc{3}}}%morpheme+gloss
\glotran{\pb{When my husband} was vice-mayor \pb{they} asked me, “Compañera, where is your husband?”}{}%eng+spa trans
{}{}%rec - time

% 3
\gloexe{Glo4:kundurqa}{}{sp}%
{Chay kundurqa qipi\pb{pti}n huk turuta pagaykun.}%sp que first line
{\morglo{chay}{\lsc{dem.d}}\morglo{kundur-qa}{condor-\lsc{top}}\morglo{qipi-pti-n}{carry-\lsc{subds}-\lsc{3}}\morglo{huk}{one}\morglo{turu-ta}{bull-\lsc{acc}}\morglo{paga-yku-n}{pay-\lsc{excep}-\lsc{3}}}%morpheme+gloss
\glotran{\pb{After} the condor carried her, she payed him a bull.}{}%eng+spa trans
{}{}%rec - time

% 4
\gloexe{Glo4:mumintu}{}{amv}%
{Huk mumintu puriya\pb{pti}ki imapis prisintakurushunki.}%amv que first line
{\morglo{huk}{one}\morglo{mumintu}{moment}\morglo{puri-ya-pti-ki}{walk-\lsc{prog}-\lsc{subds}-\lsc{2}}\morglo{ima-pis}{what-\lsc{add}}\morglo{prisinta-ku-ru-shu-nki}{present-\lsc{refl}-\lsc{urgt}-\lsc{2.obj}-\lsc{2}}}%morpheme+gloss
\glotran{One moment you’re walking \pb{and} something presents itself to you.}{}%eng+spa trans
{}{}%rec - time

\noindent
\phono{-pti} subordinates always inflect for person with allocation suffixes~(\ref{Glo4:Kalurniyuq}), (\ref{Glo4:plantaman}).\\

% 5
\gloexe{Glo4:Kalurniyuq}{}{amv}%
{Kalurniyuq ka\pb{pti}\pb{ki}qa \pb{yawarnin} yanash.}%amv que first line
{\morglo{kalur-ni-yuq}{fever-\lsc{euph}-\lsc{poss}}\morglo{ka-pti-ki-qa}{be-\lsc{subds}-\lsc{2}-\lsc{top}}\morglo{yawar-ni-n}{blood-\lsc{euph}-\lsc{3}}\morglo{yana-sh}{black-\lsc{evr}}}%morpheme+gloss
\glotran{\pb{When you} have a fever, its blood is black, they say.}{}%eng+spa trans
{}{}%rec - time

% 6
\gloexe{Glo4:plantaman}{}{ach}%
{Chay plantaman siqaru\pb{pti:}pis chay turuqa~\dots{} siqaramun qipa:paq plantaman.}%ach que first line
{\morglo{chay}{\lsc{dem.d}}\morglo{planta-man}{tree-\lsc{all}}\morglo{siqa-ru-pti-:-pis}{go.up-\lsc{urgt}-\lsc{subds}-\lsc{1}-\lsc{add}}\morglo{chay}{\lsc{dem.d}}\morglo{turu-qa}{bull-\lsc{top}}\morglo{siqa-ra-mu-n}{go.up-\lsc{urgt}-\lsc{cisl}-\lsc{3}}\morglo{qipa-:-paq}{behing-\lsc{1}-\lsc{abl}}\morglo{planta-man}{tree-\lsc{all}}}%morpheme+gloss
\glotran{\pb{When I} climbed up the tree, the bull~\dots{} climbed up the tree from behind me.}{}%eng+spa trans
{}{}%rec - time

\noindent
The structure is usually translated in English by ‘when’~(\ref{Glo4:Kundinawqa}),~(\ref{Glo4:Hinaptinshiwak}) or, less often, ‘if’~(\ref{Glo4:pagawa}),~(\ref{Glo4:karqa}), ‘because’~(\ref{Glo4:saqiru}--\ref{Glo4:ManaMana}), or ‘although’~(\ref{Glo4:qawanchikchu}).\\

% 7
\gloexe{Glo4:Kundinawqa}{}{sp}%
{Kundinawqa, witrqakuru\pb{pti}nqa, wasi utrkunta altukunapash \textup{[}yaykurun\textup{].}}%sp que first line
{\morglo{kundinaw-qa,}{zombie-\lsc{top}}\morglo{witrqa-ku-ru-pti-n-qa}{close-\lsc{refl}-\lsc{urgt}-\lsc{subds}-\lsc{3}-\lsc{top}}\morglo{wasi}{house}\morglo{utrku-n-ta}{hole-\lsc{3}-\lsc{acc}}\morglo{altu-kuna-pa-sh}{high-\lsc{pl}-\lsc{loc}-\lsc{evr}}\morglo{yayku-ru-n}{enter-\lsc{urgt}-\lsc{3}}}%morpheme+gloss
\glotran{\pb{When} they shut themselves in, the zombie [entered] through a hole in the attic.}{}%eng+spa trans
{}{}%rec - time

% 8
\gloexe{Glo4:Hinaptinshiwak}{}{ach}%
{Hinaptinshi “Wak turuta pagaykusayki,” ni\pb{pti}n asiptan.}%ach que first line
{\morglo{hinaptin-shi}{then-\lsc{evr}}\morglo{wak}{\lsc{dem.d}}\morglo{turu-ta}{bull-\lsc{acc}}\morglo{paga-yku-sayki}{pay-\lsc{excep}-\lsc{1>2.fut}}\morglo{ni-pti-n}{say-\lsc{subds}-\lsc{3}}\morglo{asipta-n}{accept-\lsc{3}}}%morpheme+gloss
\glotran{Then, they say, \pb{when} he said, “I’ll pay you that bull,” they accepted.}{}%eng+spa trans
{}{}%rec - time

% 9
\gloexe{Glo4:pagawa}{}{lt}%
{Manam pagawa\pb{pti}kiqa manam wamraykiqa alliyanqachu.}%lt que first line
{\morglo{mana-m}{no-\lsc{evd}}\morglo{paga-wa-pti-ki-qa}{pay-\lsc{1.obj}-\lsc{2}-\lsc{top}}\morglo{mana-m}{no-\lsc{evd}}\morglo{wamra-yki-qa}{child-\lsc{2}-\lsc{top}}\morglo{alli-ya-nqa-chu}{good-\lsc{inch}-\lsc{3.fut}-\lsc{neg}}}%morpheme+gloss
\glotran{\pb{If} you don’t pay me, your son isn’t going to get better.}{}%eng+spa trans
{}{}%rec - time

% 10
\gloexe{Glo4:karqa}{}{amv}%
{Wañuymantri karqa. Mana hampi\pb{pti}nqa.}%amv que first line
{\morglo{wañu-y-man-tri}{die-\lsc{1}-\lsc{cond}-\lsc{evc}}\morglo{ka-rqa}{be-\lsc{pst}}\morglo{mana}{no}\morglo{hampi-pti-n-qa}{cure-\lsc{subds}-\lsc{3}-\lsc{top}}}%morpheme+gloss
\glotran{I might have died. \pb{If} they hadn’t cured her.}{}%eng+spa trans
{}{}%rec - time

% 11
\gloexe{Glo4:saqiru}{}{amv}%
{Payqa rikunñash warmin saqiru\pb{pti}n.}%amv que first line
{\morglo{pay-qa}{he-\lsc{top}}\morglo{ri-ku-n-ña-sh}{go-\lsc{refl}-\lsc{3}-\lsc{disc}-\lsc{evr}}\morglo{warmi-n}{woman-\lsc{3}}\morglo{saqi-ru-pti-n}{abandon-\lsc{urgt}-\lsc{subds}-\lsc{3}}}%morpheme+gloss
\glotran{He left \pb{because} his wife abandoned him, they say.}{}%eng+spa trans
{}{}%rec - time

% 12
\gloexe{Glo4:Priykupaw}{}{amv}%
{Priykupaw puriyan siyrtumpatr warmin mal ka\pb{pti}n nin.}%amv que first line
{\morglo{priykupaw}{worried}\morglo{puri-ya-n}{walk-\lsc{prog}-\lsc{3}}\morglo{siyrtumpa-tr}{certainly-\lsc{evc}}\morglo{warmi-n}{woman-\lsc{3}}\morglo{mal}{bad}\morglo{ka-pti-n}{be-\lsc{subds}-\lsc{3}}\morglo{n-in}{say-\lsc{3}}}%morpheme+gloss
\glotran{Certainly, he’d be wandering around worried \pb{because} his wife is sick.}{}%eng+spa trans
{}{}%rec - time

% 13
\gloexe{Glo4:ManaMana}{}{ch}%
{Mana qusa: ka\pb{pti}n. Mana qali: ka\pb{pti}n trabahaya:.}%ch que first line
{\morglo{mana}{no}\morglo{qusa-:}{husband-\lsc{1}}\morglo{ka-pti-n}{be-\lsc{subds}-\lsc{3}}\morglo{mana}{no}\morglo{qali-:}{man-\lsc{1}}\morglo{ka-pti-n}{be-\lsc{subds}-\lsc{3}}\morglo{trabaha-ya-:}{work-\lsc{prog}-\lsc{1}}}%morpheme+gloss
\glotran{\pb{Because} I don’t have a husband. I’m working \pb{because} I don’t have a husband.}{}%eng+spa trans
{}{}%rec - time

% 14
\gloexe{Glo4:qawanchikchu}{}{amv}%
{Huk qawa\pb{pti}nqa, ñuqa-nchik qawanchikchu.}%amv que first line
{\morglo{huk}{one}\morglo{qawa-pti-n-qa}{see-\lsc{subds}-\lsc{3}-\lsc{top}}\morglo{ñuqa-nchik}{I-\lsc{1pl}}\morglo{qawa-nchik-chu}{see-\lsc{1pl}-\lsc{neg}}}%morpheme+gloss
\glotran{\pb{Although} others see it, we don’t see it.}{}%eng+spa trans
{}{}%rec - time

\noindent
Topic marking with \phono{-qa} does not generally disambiguate these readings. With \phono{-raq}, \phono{-pti} subordinates generally receive a ‘not until’ interpretation~(\ref{Glo4:lluqsirqachu}),~(\ref{Glo4:chawachikunqachu}).\\

% 15
\gloexe{Glo4:lluqsirqachu}{}{amv}%
{Hamu\pb{ptiyraq} ñuqaqa manam lluqsirqachu.~\updag}%amv que first line
{\morglo{hamu-pti-y-raq}{come-\lsc{subds}-\lsc{1}-\lsc{cont}}\morglo{ñuqa-qa}{I-\lsc{top}}\morglo{mana-m}{no-\lsc{evd}}\morglo{lluqsi-rqa-chu}{go.out-\lsc{pst}-\lsc{neg}}}%morpheme+gloss
\glotran{\pb{Not until} I came did she leave. (=‘\pb{Until} I came, she didn’t leave.’)}{}%eng+spa trans
{}{}%rec - time

% 16
\gloexe{Glo4:chawachikunqachu}{}{amv}%
{Manañam puntrawyaru\pb{pti}n vakay chawachikunqachu.}%amv que first line
{\morglo{mana-ña-m}{no-\lsc{disc}-\lsc{evd}}\morglo{puntraw-ya-ru-pti-n}{day-\lsc{inch}-\lsc{urgt}-\lsc{subds}-\lsc{3}}\morglo{vaka-y}{cow-\lsc{1}}\morglo{chawa-chi-ku-nqa-chu}{milk-\lsc{caus}-\lsc{refl}-\lsc{3.fut}-\lsc{neg}}}%morpheme+gloss
\glotran{\pb{Until it}’s day time, my cow won’t let herself be milked.}{}%eng+spa trans
{}{}%rec - time

\noindent
The first-person and second-person object suffixes, \phono{-wa/ma} and \phono{-sHu} precede \phono{-pti}~(\ref{Glo4:pasarushpa}).\\

% 17
\gloexe{Glo4:pasarushpa}{}{amv}%
{Chay pasarushpa sudarachi\pb{shupti}ki kapasmi surqurunman.}%amv que first line
{\morglo{chay}{\lsc{dem.d}}\morglo{pasa-ru-shpa}{pass-\lsc{urgt}-\lsc{subis}}\morglo{suda-ra-chi-shu-pti-ki}{sweat-\lsc{urgt}-\lsc{caus}-\lsc{2.obj}-\lsc{subds}-\lsc{2}}\morglo{kapas-mi}{perhaps-\lsc{evd}}\morglo{surqu-ru-n-man}{remove-\lsc{urgt}-\lsc{3}-\lsc{cond}}}%morpheme+gloss
\glotran{When you have it passed over you, when \pb{it makes you} sweat, it’s possible it could remove it.}{}%eng+spa trans
{}{}%rec - time

% TABLE 25a
\begin{table}[!ht]
\small\centering
\caption{\phono{-pti} inflection}\label{Tab25a}
\begin{tabular}{lll}
\lsptoprule
Person		& Singular		& Plural	\\
\midrule
1 & -pti-y\tss{\AMV,\LT} 	&-pti-nchik\\
 & -pti-:\tss{\ACH,\CH,\SP}&			\\[2ex]
%\midrule
2 &-pti-ki 					&-pti-ki\\[2ex]
%\midrule
3 &-pti-n 					&-pti-n\\
\lspbottomrule
\end{tabular}
\end{table}

% TABLE 25b
\begin{table}[!ht]
\small\centering
\caption{\phono{-pti} inflection -- actor-object suffixes}\label{Tab25b}
\resizebox{\textwidth}{!}{%
\begin{tabular}{@{\hspace{1ex}}l@{\hspace{2ex}}l@{\hspace{2ex}}l@{\hspace{2ex}}l@{\hspace{2ex}}l@{\hspace{1ex}}}
\lsptoprule
2>1		& 3>1		& 3>1pl	& 1>2	& 3>2	\\
\midrule
-wa-pti-ki\tss{\AMV,\LT}	&	-wa-pti-n\tss{\AMV,\LT}	& -wa-pti-nchik\tss{\AMV,\LT}	& -pti-ki	&	-shu-pti-ki \\
-ma-pti-ki\tss{\ACH,\CH,\SP}	&	-ma-pti-n\tss{\ACH,\CH,\SP}	&	-ma-pti-nchik\tss{\ACH,\CH,\SP}	&	 	&	 	\\
\lspbottomrule
\end{tabular}}
\end{table}

\subsubsection{Same-subjects \phono{-shpa}}\index[sub]{same-subjects}
\phono{-shpa} is employed when the subjects in the main and subordinated clauses are identical and the event of the subordinated clause is simultaneous with the event of the main clause~(\ref{Glo4:Chitchityaku}); the event of the subordinated clause may, however, precede that of the main clause~(\ref{Glo4:Familyanchikta}).\\

% 1
\gloexe{Glo4:Chitchityaku}{}{lt}%
{Chitchityaku\pb{shpa} rikullan kabrakunaqa.}%lt que first line
{\morglo{chitchitya-ku-shpa}{say.chit.chit-\lsc{refl}-\lsc{subis}}\morglo{riku-lla-n}{go-\lsc{rstr}-\lsc{3}}\morglo{kabra-kuna-qa}{goat-\lsc{pl}-\lsc{top}}}%morpheme+gloss
\glotran{Chit-chitt\pb{ing}, the goats just left.}{}%eng+spa trans
{}{}%rec - time

% 2
\gloexe{Glo4:Familyanchikta}{}{sp}%
{Familyanchikta wañurichi\pb{shpa}qa lliw partiyan.}%sp que first line
{\morglo{familya-nchik-ta}{family-\lsc{1pl}-\lsc{acc}}\morglo{wañu-ri-chi-shpa-qa}{die-\lsc{incep}-\lsc{caus}-\lsc{subis}-\lsc{top}}\morglo{lliw}{all}\morglo{parti-ya-n}{divide-\lsc{prog}-\lsc{3}}}%morpheme+gloss
\glotran{\pb{After} they killed our relatives, they distributed everything.}{}%eng+spa trans
{}{}%rec - time

\noindent
\phono{-shpa} subordinates do not inflect for person. \phono{-shpa} can generally be translated with a gerund~(\ref{Glo4:Traguwan}), as ‘when’~(\ref{Glo4:Kustumbrawku}) or, less often, ‘if’~(\ref{Glo4:kutimushaq}).\\

% 3
\gloexe{Glo4:Traguwan}{}{amv}%
{Traguwan, kukawan tushuchi\pb{shpa}llam kusichakuni.}%amv que first line
{\morglo{tragu-wan}{liquor-\lsc{instr}}\morglo{kuka-wan}{coca-\lsc{instr}}\morglo{tushu-chi-shpa-lla-m}{dance-\lsc{caus}-\lsc{subis}-\lsc{rstr}-\lsc{evd}}\morglo{kusicha-ku-ni}{harvest-\lsc{refl}-\lsc{1}}}%morpheme+gloss
\glotran{With liquor and coca, mak\pb{ing} them dance, I harvest.}{}%eng+spa trans
{}{}%rec - time

% 4
\gloexe{Glo4:Kustumbrawku}{}{amv}%
{Kustumbrawku\pb{shpa} hawkam yatrakunchik kaypahina.}%amv que first line
{\morglo{kustumbraw-ku-shpa}{accustom-\lsc{refl}-\lsc{subis}}\morglo{hawka-m}{tranquil-\lsc{evd}}\morglo{yatra-ku-nchik}{live-\lsc{refl}-\lsc{1pl}}\morglo{kay-pa-hina}{\lsc{dem.p}-\lsc{loc}-\lsc{comp}}}%morpheme+gloss
\glotran{\pb{When} we adjust, we live peacefully, like here.}{}%eng+spa trans
{}{}%rec - time

% 5
\gloexe{Glo4:kutimushaq}{}{amv}%
{Kuti\pb{shpa}qa kutimushaq kimsa tawa watata.}%amv que first line
{\morglo{kuti-shpa-qa}{return-\lsc{subis}-\lsc{top}}\morglo{kuti-mu-shaq}{return-\lsc{cisl}-\lsc{1.fut}}\morglo{kimsa}{three}\morglo{tawa}{four}\morglo{wata-ta}{year-\lsc{acc}}}%morpheme+gloss
\glotran{\pb{If} I come back, I’ll come back in three or four years.}{}%eng+spa trans
{}{}%rec - time

\noindent
Negated, V \phono{-shpa} can be translated ‘without’~(\ref{Glo4:likwarunchik}), ‘although’~(\ref{Glo4:Qullqita}) or ‘despite’.\\

% 6
\gloexe{Glo4:likwarunchik}{}{amv}%
{\pb{Mana} yanu\pb{shpa}llam likwarunchik.}%amv que first line
{\morglo{mana}{no}\morglo{yanu-shpa-lla-m}{cook-\lsc{subis}-\lsc{rstr}-\lsc{evd}}\morglo{likwa-ru-nchik}{liquify-\lsc{urgt}-\lsc{1pl}}}%morpheme+gloss
\glotran{\pb{Without} boiling it, we liquify it.}{}%eng+spa trans
{}{}%rec - time

% 7
\gloexe{Glo4:Qullqita}{}{ach}%
{Qullqita gana\pb{shpa}s bankuman ima trurakunki.}%ach que first line
{\morglo{qullqi-ta}{money-\lsc{acc}}\morglo{gana-shpa-s}{win-\lsc{subis}-\lsc{add}}\morglo{banku-man}{bank-\lsc{all}}\morglo{ima}{what}\morglo{trura-ku-nki}{put-\lsc{refl}-\lsc{2}}}%morpheme+gloss
\glotran{\pb{Although} you earn money and save it in the bank.}{}%eng+spa trans
{}{}%rec - time

\noindent
\phono{-shpa} may attach to coordinated verbs~(\ref{Glo4:Kulurchakunata}),~(\ref{Glo4:sigaruchakunata}).\\

% 8
\gloexe{Glo4:Kulurchakunata}{}{amv}%
{Kulurchakunata kayna trura\pb{shpa} qawa\pb{shpa} ñakarini.}%amv que first line
{\morglo{kulur-cha-kuna-ta}{color-\lsc{dim}-\lsc{pl}-\lsc{acc}}\morglo{kayna}{thus}\morglo{trura-shpa}{put-\lsc{subis}}\morglo{qawa-shpa}{look-\lsc{subis}}\morglo{ñaka-ri-ni}{suffer-\lsc{incep}-\lsc{1}}}%morpheme+gloss
\glotran{Look\pb{ing}, putt\pb{ing} the colors like this, I suffer.}{}%eng+spa trans
{}{}%rec - time

% 9
\gloexe{Glo4:sigaruchakunata}{}{amv}%
{Kukachakunata aku\pb{shpa} sigaruchakunata fuma\pb{shpa} richkan tutakama.}%amv que first line
{\morglo{kuka-cha-kuna-ta}{coca-\lsc{dim}-\lsc{pl}-\lsc{acc}}\morglo{aku-shpa}{chew-\lsc{subis}}\morglo{sigaru-cha-kuna-ta}{cigarette-\lsc{dim}-\lsc{pl}-\lsc{acc}}\morglo{fuma-shpa}{smoke-\lsc{subis}}\morglo{ri-chka-n}{go-\lsc{dur}-\lsc{3}}\morglo{tuta-kama}{night-\lsc{lim}}}%morpheme+gloss
\glotran{Chew\pb{ing} coca, smok\pb{ing} cigarettes, they go on until the night.}{}%eng+spa trans
{}{}%rec - time

\noindent
Only Cacra uses the \QI{} \phono{-r} in place of the \QII{} \phono{-shpa} (compare~(\ref{Glo4:harkanchik}--\ref{Glo4:Wiqawninchikman}) with~(\ref{Glo4:kampukta})).\\

% 10
\gloexe{Glo4:harkanchik}{}{amv}%
{Vakata harkanchik puchkashpa millwata \pb{puchkapuchkashpa}.}%amv que first line
{\morglo{vaka-ta}{cow-\lsc{acc}}\morglo{harka-nchik}{herd-\lsc{1pl}}\morglo{puchka-shpa}{spin-\lsc{subis}}\morglo{millwa-ta}{wool-\lsc{acc}}\morglo{puchka-puchka-shpa}{spin-spin-\lsc{subis}}}%morpheme+gloss
\glotran{We herd the cows spinning -- spinning and spinning wool.}{}%eng+spa trans
{}{}%rec - time

% 11
\gloexe{Glo4:baldillawan}{}{lt}%
{Kutimu\pb{shpa}qa kayna baldillawan apakushaq niwan.}%lt que first line
{\morglo{kuti-mu-shpa-qa}{return-\lsc{cisl}-\lsc{subis}-\lsc{top}}\morglo{kayna}{thus}\morglo{baldi-lla-wan}{bucket-\lsc{rstr}-\lsc{instr}}\morglo{apa-ku-shaq}{bring-\lsc{refl}-\lsc{1.fut}}\morglo{ni-wa-n}{say-\lsc{1.obj}-\lsc{3}}}%morpheme+gloss
\glotran{“\pb{When} I come back, I’ll bring them like this, with just a bucket,” he said to me.}{}%eng+spa trans
{}{}%rec - time

% 12
\gloexe{Glo4:Hinashpamaska}{}{ach}%
{Hinashpa maska\pb{shpa} puriya:.}%ach que first line
{\morglo{hinashpa}{then}\morglo{maska-shpa}{look.for-\lsc{subis}}\morglo{puri-ya-:}{walk-\lsc{prog}-\lsc{1}}}%morpheme+gloss
\glotran{Then I’m walk\pb{ing} around looking for them.}{}%eng+spa trans
{}{}%rec - time

% 13
\gloexe{Glo4:Wirtaman}{}{sp}%
{Wirtaman yaykuru\pb{shpa} klavilta lliw usharusa.}%sp que first line
{\morglo{wirta-man}{garden-\lsc{all}}\morglo{yayku-ru-shpa}{enter-\lsc{urgt}-\lsc{subis}}\morglo{klavil-ta}{carnation-\lsc{acc}}\morglo{lliw}{all}\morglo{usha-ru-sa}{waste.on.the.ground-\lsc{urgt}-\lsc{npst}}}%morpheme+gloss
\glotran{Enter\pb{ing} the garden, he left all the carnations discarded on the ground.}{}%eng+spa trans
{}{}%rec - time

% 14
\gloexe{Glo4:Wiqawninchikman}{}{ch}%
{Wiqawninchikman kayna katawan simillakta watakuru\pb{shpa} talpu:.}%ch que first line
{\morglo{wiqaw-ni-nchik-man}{waist-\lsc{euph}-\lsc{1pl}-\lsc{all}}\morglo{kayna}{thus}\morglo{kata-wan}{shawl-\lsc{instr}}\morglo{similla-kta}{seed-\lsc{acc}}\morglo{wata-ku-ru-shpa}{tie-\lsc{refl}-\lsc{urgt}-\lsc{subid}}\morglo{talpu-:}{plant-\lsc{1}}}%morpheme+gloss
\glotran{Like this, \pb{tying} it to our waists with a shawl we plant seeds.}{}%eng+spa trans
{}{}%rec - time

% 15
\gloexe{Glo4:kampukta}{}{ch}%
{Waqa\pb{l} likun atuq kampukta.}%ch que first line
{\morglo{waqa-l}{cry-\lsc{subis}}\morglo{li-ku-n}{go-\lsc{refl}-\lsc{3}}\morglo{atuq}{fox}\morglo{kampu-kta}{countryside-\lsc{acc}}}%morpheme+gloss
\glotran{Cry\pb{ing}, the fox went to the countryside.}{}%eng+spa trans
{}{}%rec - time

\subsubsection{Adverbial \phono{-shtin}}\index[sub]{adverbial}
\phono{-shtin} is employed when the subjects of the main and subordinated clauses are identical~(\ref{Glo4:Yatrakunchik}), (\ref{Glo4:Yantakunata}) and the events of the two clauses are simultaneous~(\ref{Glo4:pubripubri}).\\

% 1
\gloexe{Glo4:Yatrakunchik}{}{ach}%
{Yatrakunchik imaynapis~\dots{} waqaku\pb{shtin}pis~\dots{} asiku\pb{shtin}pis~\dots{} imaynapis.}%ach que first line
{\morglo{yatra-ku-nchik}{live-\lsc{refl}-\lsc{1pl}}\morglo{imayna-pis}{how-\lsc{add}}\morglo{maski}{maski}\morglo{waqa-ku-shtin-pis}{cry-\lsc{refl}-\lsc{subadv}-\lsc{add}}\morglo{asi-ku-shtin-pis}{laugh-\lsc{refl}-\lsc{subadv}-\lsc{add}}\morglo{imayna-pis}{how-\lsc{add}}}%morpheme+gloss
\glotran{We live however we can, although \pb{we’re crying}~\dots{} \pb{laughing}~\dots{} however we can.}{}%eng+spa trans
{}{}%rec - time

% 2
\gloexe{Glo4:Yantakunata}{}{amv}%
{Yantakunata qutu\pb{shtin} lliptakunata kañaku\pb{shtin},~\dots{} yatrana karqa.}%amv que first line
{\morglo{yanta-kuna-ta}{firewood-\lsc{pl}-\lsc{acc}}\morglo{qutu-shtin}{gather-\lsc{subadv}}\morglo{llipta-kuna-ta}{ash-\lsc{pl}-\lsc{acc}}\morglo{kaña-ku-shtin}{burn-\lsc{refl}-\lsc{subadv}}\morglo{yatra-na}{live-\lsc{nmlz}}\morglo{ka-rqa}{be-\lsc{pst}}}%morpheme+gloss
\glotran{\pb{Gathering} wood, \pb{burning} ash, we had to live [in the mountains].}{}%eng+spa trans
{}{}%rec - time

% 3
\gloexe{Glo4:pubripubri}{}{amv}%
{Wak pubri ubiha watra\pb{shtin} riyan.}%amv que first line
{\morglo{wak}{\lsc{dem.d}}\morglo{pubri}{poor}\morglo{ubiha}{sheep}\morglo{watra-shtin}{give.birth-\lsc{subadv}}\morglo{ri-ya-n}{go-\lsc{prog}-\lsc{3}}}%morpheme+gloss
\glotran{Those poor sheep are \pb{giving birth} even as they walk.}{}%eng+spa trans
{}{}%rec - time

\noindent
\phono{-shtin} subordinates do not inflect for person or number. \phono{-shtin} subordinates are adverbial and can generally be translated by ‘while’ or with a gerund~(\ref{Glo4:Pushayku}--\ref{Glo4:apayan}). While attested in spontaneous speech, \phono{-shtin} is rare. Speakers overwhelmingly employ \phono{-shpa} in place of \phono{-shtin}.\\

% 4
\gloexe{Glo4:Pushayku}{}{ach}%
{Pushayku\pb{shtin}qa wamrataqa makin yatapasha yantaman katran.}%ach que first line
{\morglo{pusha-yku-shtin-qa}{bring.along-\lsc{excep}-\lsc{subadv}-\lsc{top}}\morglo{wamra-ta-qa}{child-\lsc{acc}-\lsc{top}}\morglo{maki-n}{hand-\lsc{acc}}\morglo{yata-pa-sha}{feel-\lsc{repet}-\lsc{prf}}\morglo{yanta-man}{firewood-\lsc{all}}\morglo{katra-n}{release-\lsc{3}}}%morpheme+gloss
\glotran{\pb{Bringing} the boys [home], their hands held, she sent them for firewood.}{}%eng+spa trans
{}{}%rec - time

% 5
\gloexe{Glo4:iskwilapaq}{}{lt}%
{Chay iskwilapaq wamran miku\pb{shtin}.}%lt que first line
{\morglo{chay}{\lsc{dem.d}}\morglo{iskwila-paq}{school-\lsc{abl}}\morglo{wamra-n}{child-\lsc{3}}\morglo{miku-shtin}{eat-\lsc{subadv}}}%morpheme+gloss
\glotran{His child [came out] of school \pb{eating}.}{}%eng+spa trans
{}{}%rec - time

% 6
\gloexe{Glo4:Qarqaryam}{}{ch}%
{“¡Qarqaryam qipa:ta!” waqa\pb{shtin} shamukuyan.}%ch que first line
{\morglo{qarqarya-m}{zombie-\lsc{evd}}\morglo{qipa-:-ta}{behind-\lsc{1}-\lsc{acc}}\morglo{waqa-shtin}{cry-\lsc{subadv}}\morglo{shamu-ku-ya-n}{come-\lsc{refl}-\lsc{prog}-\lsc{3}}}%morpheme+gloss
\glotran{“A zombie is behind me!” he was coming \pb{crying}.}{}%eng+spa trans
{}{}%rec - time

% 7
\gloexe{Glo4:apayan}{}{ch}%
{Waqaku\pb{shtin} kayqa apayan waytakunakta.}%ch que first line
{\morglo{waqa-ku-shtin}{cry-\lsc{refl}-\lsc{subadv}}\morglo{kay-qa}{\lsc{dem.p}-\lsc{top}}\morglo{apa-ya-n}{bring-\lsc{prog}-\lsc{3}}\morglo{wayta-kuna-kta}{flower-\lsc{pl}-\lsc{acc}}}%morpheme+gloss
\glotran{\pb{Crying}, they are bringing flowers.}{}%eng+spa trans
{}{}%rec - time

% 8
\gloexe{Glo4:tristim}{}{sp}%
{Waqaku\pb{shtin} tristim ñuqanchikqa kidaranchik ñuqa mama:.}%sp que first line
{\morglo{waqa-ku-shtin}{cry-\lsc{refl}-\lsc{subadv}}\morglo{tristi-m}{sad-\lsc{evd}}\morglo{ñuqa-nchik-qa}{I-\lsc{1pl}-\lsc{top}}\morglo{kida-ra-nchik}{stay-\lsc{pst}-\lsc{1pl}}\morglo{ñuqa}{I}\morglo{mama-:-}{mother-\lsc{1}}}%morpheme+gloss
\glotran{\pb{Crying}, sad, we stayed, my mother and I.}{}%eng+spa trans
{}{}%rec - time

\subsubsection{Limitative \phono{-kama}}\index[sub]{limitative}
In combination with the nominalizer \phono{-na} and possessive inflection, \phono{kama} forms subordinate clauses indicating that the event referred to is either simultaneous with~(\ref{Glo4:vilakuranichu}) or limits~(\ref{Glo4:taksallapitaqa}--\ref{Glo4:naykama}) the event referred to in the main clause.\\

% 1
\gloexe{Glo4:vilakuranichu}{}{amv}%
{Mana vilakuranichu puñu\pb{naykama}m.}%amv que first line
{\morglo{mana}{no}\morglo{vila-ku-ra-ni-chu}{keep.watch-\lsc{refl}-\lsc{pst}-\lsc{1}-\lsc{neg}}\morglo{puñu-na-y-kama-m}{sleep-\lsc{nmlz}-\lsc{1}-\lsc{lim}-\lsc{evd}}}%morpheme+gloss
\glotran{I didn’t keep watch \pb{while I was sleeping}.}{}%eng+spa trans
{}{}%rec - time

% 2
\gloexe{Glo4:taksallapitaqa}{}{amv}%
{Taksalla taksallapitaqa tarpukuni, mana hat-hatunpichu. Yaku ka\pb{nankama}lla.}%amv que first line
{\morglo{taksa-lla}{small-\lsc{rstr}}\morglo{taksa-lla-pi-ta-qa}{small-\lsc{rstr}-\lsc{loc}-\lsc{acc}-\lsc{top}}\morglo{tarpu-ku-ni}{plant-\lsc{refl}-\lsc{1}}\morglo{mana}{no}\morglo{hat-hatun-pi-chu}{big-big-\lsc{loc}-\lsc{neg}}\morglo{yaku}{water}\morglo{ka-na-n-kama-lla}{be-\lsc{nmlz}-\lsc{3}-\lsc{lim}-\lsc{rstr}}}%morpheme+gloss
\glotran{I plant in just small, small [fields], not in really big ones. \pb{While/as long as} there’s water.}{}%eng+spa trans
{}{}%rec - time

% 3
\gloexe{Glo4:Chaytri}{}{amv}%
{Chaytri wañuq qarin wañu\pb{nankama}m maqarqa.}%amv que first line
{\morglo{chay-tri}{\lsc{dem.d}-\lsc{evr}}\morglo{wañu-q}{die-\lsc{ag}}\morglo{qari-n}{man-\lsc{3}}\morglo{wañu-na-n-kama-m}{die-\lsc{nmlz}-\lsc{3}-\lsc{lim}-\lsc{evd}}\morglo{maqa-rqa}{beat-\lsc{pst}}}%morpheme+gloss
\glotran{That’s why her\tss{1} late husband beat her\tss{2} \pb{until she\tss{2} died}.}{}%eng+spa trans
{}{}%rec - time

% 4
\gloexe{Glo4:pampamansa}{}{amv}%
{Almaqa wañu\pb{nankama} pampaman saqarun.}%amv que first line
{\morglo{alma-qa}{soul-\lsc{top}}\morglo{wañu-na-n-kama}{die-\lsc{nmlz}-\lsc{3}-\lsc{lim}}\morglo{pampa-man}{ground-\lsc{all}}\morglo{saqa-ru-n}{go.down-\lsc{urgt}-\lsc{3}}}%morpheme+gloss
\glotran{The ghost fell to the floor, \pb{to his death}.}{}%eng+spa trans
{}{}%rec - time

% 5
\gloexe{Glo4:naykama}{}{lt}%
{Traya\pb{naykama} ya hinalla kakun.}%lt que first line
{\morglo{traya-na-y-kama}{arrive-\lsc{nmlz}-\lsc{1}-\lsc{lim}}\morglo{ya}{\lsc{emph}}\morglo{hina-lla}{thus-\lsc{rstr}}\morglo{ka-ku-n}{be-\lsc{refl}-\lsc{3}}}%morpheme+gloss
\glotran{He’s like that \pb{until I arrive}.}{}%eng+spa trans
{}{}%rec - time

\section{Verb derivation}\label{sec:verbderivation}\index[sub]{verb derivation}
Five suffixes derive verbs from substantives: factive \phono{-cha}, reflexive \phono{-ku}, simulative \phono{-tuku}, inchoative \phono{-ya}. Additionally, two verbs can suffix to nouns to derive verbs: \phono{na-} ‘do, act’ and \phono{naya-} ‘give desire’.

A set of nineteen suffixes derives verbs from verbs. These are: \phono{-cha} (diminutive); \phono{-chi} (causative); \phono{-ka} (passive, accidental); \phono{-katra} (iterative); \phono{-kU} (reflexive, middle, medio-passive, passive, completive); \phono{-lla} (restrictive, limitative); \phono{-mu} (cislocative, translocative);\footnote{W. Adelaar (p.c.) points out that \phono{-mu} might also be treated as an inflectional suffix. An anonymous reviewer agrees: “the suffixes \phono{-ya}, \phono{-ru} and \phono{-ri} are all more derivational than \phono{-mu}, [which] never co-occurs with \phono{-ma} in QI,” they write. “Rather, \phono{-mu} and and \phono{-ma} seem to be in paradigmatic contrast, where \phono{-ma} essentially means ‘to ego,’ and \phono{-mu} means more generally ‘to any deictic center.”} \phono{-nakU} (reciprocal); \phono{-naya} (desirative); \phono{-pa} (repetitive); \phono{-pa(:)kU} (joint action); \phono{-pU} (benefactive); \phono{-ra} (uninterrupted action); \phono{-Ri} (inceptive); \phono{-RU} (action with urgency or personal interest, completive); \phono{-shi} (accompaniment); \phono{-ya} (intensifying); and \phono{-YkU} (exceptional performance). §~\ref{ssec:ADVS} and~\ref{VDfV} cover suffixes deriving verbs from substantives and from other verbs, respectively.

\subsection{Suffixes deriving verbs from substantives}\label{ssec:ADVS}
The suffixes deriving verbs from substantives are: factive \phono{-cha}, reflexive \phono{-ku}, simulative \phono{-tuku}, and inchoative \phono{-ya}. §~\ref{ssec:factive}--\ref{ssec:inchoative} cover each of these in turn. (Examples are fully glossed in the corresponding sections).

% TABLE 26 4.15
\begin{table}[!ht]
\small\centering
\caption{Suffixes deriving verbs from substantives, with examples}\label{Tab26}
\begin{tabularx}{\textwidth}{l@{~}lL@{~}L}
\lsptoprule
{\phono{-cha}}	& factive 		& {\phononb{Mama-n kanan qatra-\pb{cha-}ru-nqa.}} & ‘Now his mother is going \pb{to dirty} it.’		\\[0.5ex]
{\phono{-ku}}		& reflexive 	& {\phononb{Qishta-\pb{ku}-ru-n.}} & ‘They \pb{made a nest}.’		\\[0.5ex]
{\phono{-tuku}}	& simulative 	& {\phononb{Atrqray-shi huvin-\pb{tuku}-sa.}} & ‘The eagle \pb{disguised himself as} a young man.’		\\[0.5ex]
{\phono{-ya}}		& inchoative 	& {\phononb{Puntraw-\pb{ya}-ru-n.}} & ‘It dawned.’		\\[0.5ex]
{\phono{na-}}		& ‘do’			& {¿\phononb{\pb{Ima-na}-ku-shaq-taq mana kay pacha muna-wa-na-n-paq?}} & ‘\pb{What am I going to do} so that this earth won’t want me?’	\\[0.5ex]
{\phono{naya-}}	& ‘give desire’ & {\phononb{Pashña-\pb{naya}-shunki.}} & ‘You \pb{want} a girl.’	\\
\lspbottomrule
\end{tabularx}
\end{table}

\subsubsection{Factive \phono{-cha}}\label{ssec:factive}\index[sub]{factive}
\phono{-cha} suffixes to adjectives and nouns, deriving verbs with the meanings ‘to~make~A’ (\phono{qatra-cha-} ‘to make dirty’)~(\ref{Glo4:pawakatrashpa}--\ref{Glo4:rachimunki}), ‘to~make~N’ or ‘to~make~into~N’ (\phono{siru-cha-} ‘form a hill’)~(\ref{Glo4:Chayna}, (\ref{Glo4:ykuptinqa}), ‘to locate something in~N’ (\phono{kustal-cha-} ‘to put into sacks’)~(\ref{Glo4:papatam}), ‘to locate N in/on something’~(\ref{Glo4:turutaqa}), ‘to remove~N’ (\phono{usa-cha} ‘to remove lice’, \phono{qiwa-cha} ‘to remove weeds’).\\

% 1
\gloexe{Glo4:pawakatrashpa}{}{amv}%
{Maman kanan qatra\pb{cha}runqa pawakatrashpa.}%amv que first line
{\morglo{mama-n}{mother-\lsc{3}}\morglo{kanan}{now}\morglo{qatra-cha-ru-nqa}{dirty-\lsc{fact}-\lsc{urgt}-\lsc{3.fut}}\morglo{pawa-katra-shpa}{jump-\lsc{freq}-\lsc{subis}}}%morpheme+gloss
\glotran{Now his mother is going to \pb{make it dirty} jumping.}{}%eng+spa trans
{}{}%rec - time

% 2
\gloexe{Glo4:nqatri}{}{amv}%
{Hatun\pb{cha}nqatri kay.}%amv que first line
{\morglo{hatun-cha-nqa-tri}{big-\lsc{fact}-\lsc{3.fut}}\morglo{kay}{\lsc{dem.p}}}%morpheme+gloss
\glotran{This one is going to \pb{make it big}.}{}%eng+spa trans
{}{}%rec - time

% 3
\gloexe{Glo4:rachimunki}{}{lt}%
{Cañeteman alli\pb{cha}rachimunki kaypitr siguranaykipaqqa.}%lt que first line
{\morglo{Cañete-man}{Cañete-\lsc{all}}\morglo{alli-cha-ra-chi-mu-nki}{good-\lsc{fact}-\lsc{urgt}-\lsc{caus}-\lsc{cisl}-\lsc{2}}\morglo{kay-pi-tr}{\lsc{dem.p}-\lsc{loc}-\lsc{evc}}\morglo{sigura-na-yki-paq-qa}{insure-\lsc{nmlz}-\lsc{2}-\lsc{purp}-\lsc{top}}}%morpheme+gloss
\glotran{You’re going to have that \pb{fixed} in Cañete to be able to insure yourself here.}{}%eng+spa trans
{}{}%rec - time

% 4
\gloexe{Glo4:Chayna}{}{amv}%
{Chayna siru\pb{cha}kurun.}%amv que first line
{\morglo{chayna}{thus}\morglo{siru-cha-ku-ru-n}{hill-\lsc{fact}-\lsc{refl}-\lsc{urgt}-\lsc{3}}}%morpheme+gloss
\glotran{It \pb{formed} a hill like that.}{}%eng+spa trans
{}{}%rec - time

% 5
\gloexe{Glo4:ykuptinqa}{}{amv}%
{Parti\pb{cha}ykuptinqa chaki, chaki.}%amv que first line
{\morglo{parti-cha-yku-pti-n-qa}{parts-\lsc{fact}-\lsc{excep}-\lsc{subds}-\lsc{3}-\lsc{top}}\morglo{chaki}{dry}\morglo{chaki}{dry}}%morpheme+gloss
\glotran{When she \pb{breaks it into parts} -- dry, dry!}{}%eng+spa trans
{}{}%rec - time

% 6
\gloexe{Glo4:papatam}{}{amv}%
{Kustal\pb{cha}yan papatam.}%amv que first line
{\morglo{kustal-cha-ya-n}{sack-\lsc{fact}-\lsc{prog}-\lsc{3}}\morglo{papa-ta-m}{potato-\lsc{acc}-\lsc{evd}}}%morpheme+gloss
\glotran{She’s \pb{bagging} potatoes.}{}%eng+spa trans
{}{}%rec - time

% 7
\gloexe{Glo4:turutaqa}{}{amv}%
{Chay turutaqa llampu\pb{cha}ykun chay yubuchanman.}%amv que first line
{\morglo{chay}{\lsc{dem.d}}\morglo{turu-ta-qa}{bull-\lsc{acc}-\lsc{top}}\morglo{llampu-cha-yku-n}{llampu-\lsc{fact}-\lsc{excep}-\lsc{3}}\morglo{chay}{\lsc{dem.d}}\morglo{yubu-cha-n-man}{yoke-\lsc{dim}-\lsc{3}-\lsc{all}}}%morpheme+gloss
\glotran{They \pb{put llampu} on his little yoke.}{}%eng+spa trans
{}{}%rec - time

\subsubsection{Reflexive \phono{-ku}}\index[sub]{reflexive}
Suffixing to nouns referring to objects, \phono{-ku} may derive verbs with the meaning ‘to make/prepare N’ (\phono{qisha-ku-} ‘to make a nest’)~(\ref{Glo4:kasunchu}), (\ref{Glo4:Hiraku}); suffixing specifically to nouns referring to clothing and other items that can be placed on a person’s body, \phono{-ku} derives verbs with the meaning ‘to put on~N’ (\phono{kata-ku} ‘put on a shawl’)~(\ref{Glo4:ykurushaq}), (\ref{Glo4:nchikchu}); suffixing to adjectives referring to human states --~angry, guilty, envious~-- A\phono{-ku} has the meaning ‘to become A’ (\phono{piña-ku-} ‘to become angry’)~(\ref{Glo4:Kumudaku}),~(\ref{Glo4:Kurriy}).\\

% 1
\gloexe{Glo4:kasunchu}{}{amv}%
{Misa\pb{ku}n. Manam kasunchu misata.}%amv que first line
{\morglo{misa-ku-n}{mass-\lsc{refl}-\lsc{3}}\morglo{mana-m}{no-\lsc{evd}}\morglo{kasu-n-chu}{pay.attention-\lsc{3}-\lsc{neg}}\morglo{misa-ta}{mass-\lsc{acc}}}%morpheme+gloss
\glotran{She’s \pb{making [holding] mass}. They don’t pay attention to mass.}{}%eng+spa trans
{}{}%rec - time

% 2
\gloexe{Glo4:Hiraku}{}{ach}%
{\pb{Hiraku}run.}%ach que first line
{\morglo{hira-ku-ru-n}{herranza-\lsc{refl}-\lsc{urgt}-\lsc{3}}}%morpheme+gloss
\glotran{They \pb{made [held] an herranza}.}{}%eng+spa trans
{}{}%rec - time

% 3
\gloexe{Glo4:ykurushaq}{}{amv}%
{\pb{Walaku}ykurushaq.}%amv que first line
{\morglo{wala-ku-yku-ru-shaq}{skirt-\lsc{refl}-\lsc{excep}-\lsc{urgt}-\lsc{1.fut}}}%morpheme+gloss
\glotran{I’m going to \pb{put on} my skirt.}{}%eng+spa trans
{}{}%rec - time

% 4
\gloexe{Glo4:nchikchu}{}{amv}%
{Manash \pb{waytaku}nchikchu.}%amv que first line
{\morglo{mana-sh}{no-\lsc{evr}}\morglo{wayta-ku-nchik-chu}{flower-\lsc{refl}-\lsc{1pl}-\lsc{neg}}}%morpheme+gloss
\glotran{We don’t \pb{put flowers on} our hats [on All Saints’ Day], they say.}{}%eng+spa trans
{}{}%rec - time

% 5
\gloexe{Glo4:Kumudaku}{}{amv}%
{\pb{Kumudaku}run.}%amv que first line
{\morglo{kumuda-ku-ru-n}{comfortable-\lsc{refl}-\lsc{urgt}-\lsc{3}}}%morpheme+gloss
\glotran{He’s \pb{made himself comfortable}.}{}%eng+spa trans
{}{}%rec - time

% 6
\gloexe{Glo4:Kurriy}{}{lt}%
{¡Kurriy! \pb{Qillaku}yankitrari.}%lt que first line
{\morglo{kurri-y}{run-\lsc{imp}}\morglo{qilla-ku-ya-nki-tr-ari}{lazy-\lsc{refl}-\lsc{prog}-\lsc{2}-\lsc{evc}-\lsc{ari}}}%morpheme+gloss
\glotran{Run! You must be \pb{getting lazy}.}{}%eng+spa trans
{}{}%rec - time

\noindent
\phono{-ku} derivation is very productive and can be idiosyncratic (\phono{llulla-ku} ‘tell a lie’, \phono{midida-ku} ‘measure’)~(\ref{Glo4:mansuchu}), (\ref{Glo4:Karruwan}).\\

% 7
\gloexe{Glo4:mansuchu}{}{amv}%
{Manam mansuchu yatran waqra\pb{ku}yta.}%amv que first line
{\morglo{mana-m}{no-\lsc{evd}}\morglo{mansu-chu}{tame-\lsc{neg}}\morglo{yatra-n}{know-\lsc{3}}\morglo{waqra-ku-y-ta}{horn-\lsc{refl}-\lsc{inf}-\lsc{acc}}}%morpheme+gloss
\glotran{He’s not tame -- he can \pb{horn} [gore] people.}{}%eng+spa trans
{}{}%rec - time

% 8
\gloexe{Glo4:Karruwan}{}{sp}%
{Karruwan~\dots{} \pb{sillaku}ykushpam riyanchik.}%sp que first line
{\morglo{karru-wan}{bus-\lsc{instr}}\morglo{silla-ku-yku-shpa-m}{seat-\lsc{refl}-\lsc{excep}-\lsc{evd}}\morglo{ri-ya-nchik}{go-\lsc{prog}-\lsc{1pl}}}%morpheme+gloss
\glotran{In a car~\dots{} [it’s like] we’re \pb{riding horseback in a saddle}.}{}%eng+spa trans
{}{}%rec - time

\subsubsection{Simulative \phono{-tuku}}\index[sub]{simulative}
Suffixing to nouns, \phono{-tuku} derives verbs with the meaning ‘to pretend to be N’ or ‘to become N’ (\phono{maqta-tuku-} ‘pretend to be a young man’)~(\ref{Glo4:ukucha}--\ref{Glo4:asnuqa}).\\

% 1
\gloexe{Glo4:ukucha}{}{amv}%
{Chay ukucha ka\pb{sa} maqta\pb{tuku}shpa.}%amv que first line
{\morglo{chay}{\lsc{dem.d}}\morglo{ukucha}{mouse}\morglo{ka-sa}{be-\lsc{pst}}\morglo{maqta-tuku-shpa}{young.man-\lsc{simul}-\lsc{subis}}}%morpheme+gloss
\glotran{It was a mouse \pb{pretending to be} a man.}{}%eng+spa trans
{}{}%rec - time

% 2
\gloexe{Glo4:Sinvirgwin}{}{ch}%
{¡Sinvirgwinsa! ¡Qam ingañamalanki qali\pb{tuku}shpa!}%ch que first line
{\morglo{sinvirgwinsa}{shameless}\morglo{qam}{you}\morglo{ingaña-ma-la-nki}{trick-\lsc{1.obj}-\lsc{pst}-\lsc{2}}\morglo{qali-tuku-shpa}{man-\lsc{simul}-\lsc{subis}}}%morpheme+gloss
\glotran{Shameless bastard! You fooled me \pb{pretending to be a man}!}{}%eng+spa trans
{}{}%rec - time

% 3
\gloexe{Glo4:asnuqa}{}{amv}%
{Wak wañuq wañurun~\dots{} asnuqa wañuq\pb{tuku}run.}%amv que first line
{\morglo{wak}{\lsc{dem.d}}\morglo{wañu-q}{die-\lsc{ag}}\morglo{wañu-ru-n}{die-\lsc{urgt}-\lsc{3}}\morglo{asnu-qa}{donkey-\lsc{top}}\morglo{wañu-q-tuku-ru-n}{die-\lsc{ag}-\lsc{simul}-\lsc{urgt}-\lsc{3}}}%morpheme+gloss
\glotran{That “dead” one died~\dots{} the donkey had \pb{pretended to be dead}.}{}%eng+spa trans
{}{}%rec - time

The structure appears primarily --~indeed, almost exclusively~-- in the corpus in the context of a very popular genre of stories in which an animal dresses up, pretending to be a man, to trick a girl.

\subsubsection{Inchoative \phono{-ya}}\label{ssec:inchoative}\index[sub]{inchoative}
\phono{-ya} suffixes to nouns and adjectives to derive verbs meaning ‘to become~N’ (\phono{rumi-ya} ‘petrify’)~(\ref{Glo4:Puntrawya}),~(\ref{Glo4:Hukya}), ‘to become~A’ (\phono{alli-ya} ‘get well’)~(\ref{Glo4:Siyrtumpimik}--\ref{Glo4:qawashunki}), and ‘to perform a characteristic action with~N’ (\phono{kwahu-ya} ‘add curdling agent’).\\

% 1
\gloexe{Glo4:Puntrawya}{}{lt}%
{\pb{Puntrawya}ruptinqa.}%lt que first line
{\morglo{puntraw-ya-ru-pti-n-qa}{day-\lsc{inch}-\lsc{urgt}-\lsc{subds}-\lsc{3}-\lsc{top}}}%morpheme+gloss
\glotran{When it \pb{becomes day} [\pb{dawns}].}{}%eng+spa trans
{}{}%rec - time

% 2
\gloexe{Glo4:Hukya}{}{lt}%
{\pb{Hukya}runi.}%lt que first line
{\morglo{huk-ya-ru-ni}{one-\lsc{inch}-\lsc{urgt}-\lsc{1}}}%morpheme+gloss
\glotran{I \pb{join}ed them.}{}%eng+spa trans
{}{}%rec - time

% 3
\gloexe{Glo4:Siyrtumpimik}{}{amv}%
{Siyrtumpimik chay rumikunamik \pb{yanaya}sa kayan.}%amv que first line
{\morglo{siyrtumpi-mi-k}{certainly-\lsc{evd}-\lsc{ik}}\morglo{chay}{\lsc{dem.d}}\morglo{rumi-kuna-mi-k}{stone-\lsc{pl}-\lsc{evd}-\lsc{ik}}\morglo{yana-ya-sa}{back-\lsc{prog}-\lsc{prf}}\morglo{ka-ya-n}{be-\lsc{prog}-\lsc{3}}}%morpheme+gloss
\glotran{It’s true -- even the stones \pb{turn black} there.}{}%eng+spa trans
{}{}%rec - time

% 4
\gloexe{Glo4:wamraykiqa}{}{lt}%
{“Manam wamraykiqa \pb{alliya}nqachu”, nini.}%lt que first line
{\morglo{mana-m}{no-\lsc{evd}}\morglo{wamra-yki-qa}{child-\lsc{2}-\lsc{top}}\morglo{alli-ya-nqa-chu}{good-\lsc{inch}-\lsc{3.fut}-\lsc{neg}}\morglo{ni-ni}{say-\lsc{1}}}%morpheme+gloss
\glotran{“Your son isn’t going to \pb{get better},” I said.}{}%eng+spa trans
{}{}%rec - time

% 5
\gloexe{Glo4:Duruya}{}{amv}%
{\pb{Duruya}runña. \pb{Duruya}ruptin hurqunchik wankuman.}%amv que first line
{\morglo{duru-ya-ru-n-ña}{hard-\lsc{inch}-\lsc{urgt}-\lsc{3}-\lsc{disc}}\morglo{duru-ya-ru-pti-n}{hard-\lsc{inch}-\lsc{urgt}-\lsc{subds}-\lsc{3}}\morglo{hurqu-nchik}{remove-\lsc{1pl}}\morglo{wanku-man}{mold-\lsc{all}}}%morpheme+gloss
\glotran{It’s already \pb{hard}. When it \pb{get}s \pb{hard}, take it out [and put it] in the mold.}{}%eng+spa trans
{}{}%rec - time

% 6
\gloexe{Glo4:qawashunki}{}{ach}%
{Chay wañuruptikiqa, ¿pima qawashunki? ¿\pb{Yasqaya}ruptikiqa?}%ach que first line
{\morglo{chay}{\lsc{dem.d}}\morglo{wañu-ru-pti-ki-qa}{die-\lsc{urgt}-\lsc{subds}-\lsc{2}-\lsc{top}}\morglo{pi-m-a}{who-\lsc{evd}-\lsc{emph}}\morglo{qawa-shunki}{see-\lsc{3>2}}\morglo{yasqa-ya-ru-pti-ki-qa}{old-\lsc{inch}-\lsc{urgt}-\lsc{subds}-\lsc{2}-\lsc{top}}}%morpheme+gloss
\glotran{When you die, who’s going to see to you? Or when you \pb{get old}?}{}%eng+spa trans
{}{}%rec - time

\subsubsection{‘To do’ \phono{na-}}\label{ssec:todona}\index[sub]{to do}
\phono{na-}, following a demonstrative pronoun, yields a transitive verb meaning ‘to be thus’~(\ref{Glo4:hampichiptikiqa}),~(\ref{Glo4:puntrawch}) or ‘to do thus’~(\ref{Glo4:apuraw}).\\

% 1
\gloexe{Glo4:hampichiptikiqa}{}{amv}%
{Mana hampichiptikiqa \pb{chayna}nqam.}%amv que first line
{\morglo{mana}{no}\morglo{hampi-chi-pti-ki-qa}{cure-\lsc{caus}-\lsc{subds}-\lsc{2}-\lsc{top}}\morglo{chay-na-nqa-m}{\lsc{dem.d}-\lsc{vrbz}-\lsc{3.fut}-\lsc{evd}}}%morpheme+gloss
\glotran{If you don’t have her cured, it’s going to \pb{be like that}.}{}%eng+spa trans
{}{}%rec - time

% 2
\gloexe{Glo4:puntrawch}{}{amv}%
{Qayna puntraw \pb{chayna}n pararun tardi usyarirun.}%amv que first line
{\morglo{qayna}{previous}\morglo{puntraw}{day}\morglo{chay-na-n}{\lsc{dem.d}-\lsc{vrbz}-\lsc{3}}\morglo{para-ru-n}{rain-\lsc{urgt}-\lsc{3}}\morglo{tardi}{afternoon}\morglo{usya-ri-ru-n}{clear-\lsc{incep}-\lsc{urgt}-\lsc{3}}}%morpheme+gloss
\glotran{Yesterday \pb{it was like that} -- it rained and in the afternoon and it cleared up.}{}%eng+spa trans
{}{}%rec - time

% 3
\gloexe{Glo4:apuraw}{}{amv}%
{Mana apuraw alliyananchikpaqmi, qatra shakash \pb{chayna}n.}%amv que first line
{\morglo{mana}{no}\morglo{apuraw}{quickly}\morglo{alli-ya-na-nchik-paq-mi}{good-\lsc{inch}-\lsc{nmlz}-\lsc{1pl}-\lsc{purp}-\lsc{evd}}\morglo{qatra}{dirty}\morglo{shakash}{guinea.pig}\morglo{chay-na-n}{\lsc{dem.d}-\lsc{vrbz}-\lsc{3}}}%morpheme+gloss
\glotran{So that we don’t get better quickly, the filthy guinea pig \pb{goes like that}.}{}%eng+spa trans
{}{}%rec - time

\noindent
Following the interrogative indefinite \phono{ima} ‘what’ it yields a transitive verb, \phono{imana-}, meaning ‘to do something’~(\ref{Glo4:mamakuqa}),~(\ref{Glo4:imanashaykipischu}), ‘to happen to’~(\ref{Glo4:Wawayta}).\\

% 4
\gloexe{Glo4:mamakuqa}{}{ach}%
{Chay mamakuqa yataykun. ¿\pb{Imana}nqataq? Yataykachin.}%ach que first line
{\morglo{chay}{\lsc{dem.d}}\morglo{mamaku-qa}{grandmother-\lsc{top}}\morglo{yata-yku-n}{touch-\lsc{excep}-\lsc{3}}\morglo{ima-na-nqa-taq}{what-\lsc{vrbz}-\lsc{3.fut}-\lsc{seq}}\morglo{yata-yka-chi-n}{touch-\lsc{excep}-\lsc{caus}-\lsc{3}}}%morpheme+gloss
\glotran{The old woman touched [their arms]. \pb{What are they going to do}? They let her touch their arms.}{}%eng+spa trans
{}{}%rec - time

% 5
\gloexe{Glo4:imanashaykipischu}{}{ch}%
{\pb{Manam} ñuqaqa \pb{imanashaykipischu}. Kwirpu:mi hutrayuq.}%ch que first line
{\morglo{mana-m}{no-\lsc{evd}}\morglo{ñuqa-qa}{I-\lsc{top}}\morglo{ima-na-shayki-pis-chu}{what-\lsc{vrbz}-\lsc{1>2.fut}-\lsc{add}-\lsc{neg}}\morglo{kwirpu-:-mi}{body-\lsc{1}-\lsc{evd}}\morglo{hutra-yuq}{fault-\lsc{poss}}}%morpheme+gloss
\glotran{I’m \pb{not going to do anything to you}. My body is guilty.}{}%eng+spa trans
{}{}%rec - time

% 6
\gloexe{Glo4:Wawayta}{}{ach}%
{¿Wawayta \pb{imana}runtri?}%ach que first line
{\morglo{wawa-y-ta}{baby-\lsc{1}-\lsc{acc}}\morglo{ima-na-ru-n-tri}{what-\lsc{vrbz}-\lsc{3}-\lsc{evc}}}%morpheme+gloss
\glotran{What would have \pb{happened} to my son?}{}%eng+spa trans
{}{}%rec - time

\subsubsection{Sensual and psychological necessity \phono{naya-}}\label{ssec:senspsy}\index[sub]{sensual or psychological necessity}
\phono{naya-} --~‘to give desire’~-- suffixing to a noun derives a verb meaning ‘to give the desire for~N’~(\ref{Glo4:shunki}--\ref{Glo4:wanmi}).\\

% 1
\gloexe{Glo4:shunki}{}{amv}%
{Pashña\pb{naya}shunki.~\updag}%amv que first line
{\morglo{pashña-naya-shu-nki}{girl-\lsc{desr}-\lsc{2.obj}-\lsc{2}}}%morpheme+gloss
\glotran{You want a girl.}{}%eng+spa trans
{}{}%rec - time

% 2
\gloexe{Glo4:Mishki}{}{amv}%
{Mishki\pb{naya}ruwan.}%amv que first line
{\morglo{mishki-naya-ru-wa-n}{fruit-\lsc{desr}-\lsc{urgt}-\lsc{1.obj}-\lsc{3}}}%morpheme+gloss
\glotran{I want to eat fruit.}{}%eng+spa trans
{}{}%rec - time

% 3
\gloexe{Glo4:wanmi}{}{lt}%
{“Yaku\pb{naya}wanmi”, nin runaqa. Chayshi wamranta nin, “¡Yakuta apamuy!”}%lt que first line
{\morglo{yaku-naya-wa-n-mi}{water-\lsc{desr}-\lsc{1.obj}-\lsc{3}-\lsc{evd}}\morglo{ni-n}{say-n}\morglo{runa-qa}{person-\lsc{top}}\morglo{chayshi}{\lsc{dem.d}-\lsc{evr}}\morglo{wamra-n-ta}{child-\lsc{3}-\lsc{acc}}\morglo{ni-n}{say-3}\morglo{yaku-ta}{water-\lsc{acc}}\morglo{apa-mu-y}{bring-\lsc{cisl}-\lsc{imp}}}%morpheme+gloss
\glotran{The person said, “I’m thirsty.” So he said to his child, “Bring water!”}{}%eng+spa trans
{}{}%rec - time

\subsection{Verbs derived from verbs}\label{VDfV}
A set of eighteen suffixes derives verbs from verbs. They are: \phono{-cha}, \phono{-chi}, \phono{-ka}, \phono{-katra}, \phono{-kU}, \phono{-lla}, \phono{-mu}, \phono{-nakU}, \phono{-naya}, \phono{-pa}, \phono{-pa(:)kU}, \phono{-pU}, \phono{-Ra}, \phono{-Ri}, \phono{-RU}, \phono{-shi}, \phono{-tamu}, and \phono{-YkU}. Of the twenty, arguably only four --~causative \phono{-chi}, reflexive \phono{-ku}, reciprocal \phono{-nakU}, and desierative \phono{-naya}~-- actually change the root’s theta structure and derive new lexical items. The rest specify mode and/or aspect and/or otherwise function adverbally. The analyses of~§~\ref{modalsuffixes} identify some of the more common possible interpretations of these suffixes. That said, the interpretations given are hardly exhaustive or definitive, not least because each generally includes multiple vectors. 

\phono{-chi} (causative) derives verbs with the meaning ‘cause V’ or ‘permit V’ (\phononb{wañu-chi-} ‘kill’ (\lit~‘make die’)). Compounded with reflexive \phono{-ku}, \phono{-chi} derives verbs with the meaning ‘cause oneself to V’ or ‘cause oneself to be V-ed’ (\phononb{yanapa-chi-ku-} ‘get oneself helped’). 

\phono{-ka} (passive/accidental) indicates that the event referred to is not under the control either of a participant in that event or of the speaker (\phono{puñu-ka-} ‘fall asleep’).

\phono{-katra} (iterative) indicates extended or repetitive action (\phono{kurri-katra-} ‘to run around and around’).

\phono{-kU} (reflexive, middle, medio-passive, passive) derives verbs with the meanings ‘V oneself’ (\phono{mancha-ku-} ‘scare oneself’, ‘get scared’), ‘V for oneself/one’s own benefit (\phono{suwa-ku} ‘steal’) ‘be V-ed’ (\phono{pampa-ku-} ‘be buried’). 

\phono{-lla} (restrictive, limitative) indicates that the event referred to remains limited to itself and is not accompanied by other events (\phono{lluqsi-lla-} ‘just leave’). 

\phono{-mu} (cislocative, translocative) indicates --~in the case of verbs involving motion~-- motion toward the speaker or toward a place which is indicated by the speaker (\phono{apa-mu-} ‘bring here’).

\phono{-nakU} (reciprocal) derives verbs with the meaning ‘V each other’ (\phono{willa-naku-} ‘tell each other’); compounded with causative \phono{-chi}, \phono{-nakU} derives verbs with the meaning and ‘cause each other to V’ (\phono{willa-chi-naku-} ‘cause each other to tell’).

\phono{-naya} (desiderative) derives a compound verb meaning ‘to give the desire to V’ (\phono{miku-\pb{naya-}} ‘be hungry’ (\lit~‘gives the desire to eat’)).

\phono{-pa} (repetitive) indicates renewed or repetitive action (\phono{tarpu-pa-} ‘re-seed’, ‘repeatedly seed’); compounded with \phono{-ya} (intensive) \phono{-paya} derives verbs meaning ‘continue to V’ (\phono{trabaha-paya-} ‘continue to work’).

\phono{-pa(:)kU} (joint action)\index[sub]{joint action} indicates joint action by a plurality of individuals (\phononb{traba\-ha-pa:ku-} ‘work (together with others)’).

\phono{-pU} (benefactive) indicates that an action is performed on behalf --~or to the detriment~-- of someone other than the subject (\phono{pripara-pu-} ‘prepare (for s.o. else)’); compounded with \phono{-kU}, \phono{-pU} indicates that indicates the action is performed as a means or preparation for something else more important (including all remunerated labor) (\phono{awa-paku-} ‘weave (for others, to make money)’).

\phono{-Ra} (persistence) derives verbs with the meaning ‘continue to V’ (\phono{qawa-ra-} ‘look at persistently’); compounded with \phono{-ya} (intensive) \phono{-raya} derives passive from transitive verbs; that is, \phono{-raya} derives verbs meaning ‘be V-ed’ (\phono{wata-raya-} ‘be tied’).

\phono{-Ri} (inceptive) derives verbs meaning ‘begin to V’ (\phono{shinka-ri-} ‘begin to get drunk’).

\phono{-RU} (various) indicates action with urgency or personal interest (\phono{chaki-ru-} ‘dry out (dangerously)’); it is very frequently used with a completive interpretation (\phono{kani-ru-n} ‘bit’).

\phono{-shi} (accompaniment) derives verbs meaning ‘accompany in V-ing’ or ‘help V’ (\phono{harka-shi-} ‘help herd’).

\phono{-tamu} (irreversible) indicates an irreversible change of state (\phono{wañu}-\phono{-tamu-} ‘die’).

\phono{-YkU} (exceptional) is perhaps the derivative suffix for which is it hardest to identify any kind of central interpretation; with regard to cognates in other Quechuan languages, it is sometimes said that it indicates action performed in some way different from usual. 

Examples in Table~\ref{Tab27} are fully glossed in the corresponding sections.

% TABLE 27
\begin{table}[!ht]
\small\centering
\caption{Verb-verb derivational suffixes, with examples}\label{Tab27}
\begin{tabularx}{\textwidth}{l@{~}lL@{~}L}
\lsptoprule
{\phono{-cha}}	    & diminutive	&{\phononb{Wilka-y-ta puklla-\pb{cha}-ya-n.}}	&‘My grandson is playing’.\\[0.5ex]
{\phono{-chi}}	    & causative		&{\phononb{Ishpa-y-cha-ta tuma-ra-\pb{chi-}rqa-ni.}}	&‘I \pb{made him} drink urine.’\\[0.5ex]
{\phono{-ka}}		& passive/accidental &{\phononb{Puñu-\pb{ka}-ru-n-mi.}}	&‘She has fallen asleep’.\\[0.5ex]
{\phono{-katra}}	& iterative	&{\phononb{Pawa-\pb{katra-}shpa}}	&‘jumping and jumping’\\[0.5ex]
{\phono{-kU}}		& reflexive, passive	&{\phononb{Kikinpis Campiona\pb{ku}run.}}	&‘They themselves poisoned \pb{themselves} with Campión.’\\[0.5ex]
{\phono{-lla}}	 & restrictive	&{\phononb{Wak runa-qa~\dots{} piliya-ku-\pb{lla}-n.}}	&‘Those people~\dots{} \pb{just} fight.’\\[0.5ex]
{\phono{-mu}}		& cislocative	&{\phononb{Qati-\pb{mu}-shaq kay-man.}}	&‘I’m going to bring it over here.’\\[0.5ex]
{\phono{-nakU}}	& reciprocal		&{\phononb{Kay visinu-kuna-qa dinunsiya-\pb{naku}-n maqa-\pb{naku}-n.}}	&‘The neighbors denounce \pb{each other}, they hit \pb{each other}.’\\[0.5ex]
{\phono{-naya}}	& desiderative	&{\phononb{Ishpa-\pb{naya}-wa-n.}}	&‘I \pb{want to} urinate.’\\[0.5ex]
{\phono{-pa}}		& repetitive			&{\phononb{Qawa-\pb{pa}-yku-pti-n-ña-taq-shi.}}	&‘If he’s looking \pb{every second}.’\\[0.5ex]
{\phono{-pa(:)kU}}&joint action		&{\phononb{Tari-pa:ku-n-man-pis ka-rqa.}}	&‘\pb{They} might have found him.’\\[0.5ex]
{\phono{-pU}}		& benefactive		&{\phononb{Chay-lla-pa pripara-\pb{pu}-nki.}}	&‘Just there prepare it \pb{for me}.’\\[0.5ex]
{\phono{-Ra}}		& uninterrupted		&{\phononb{¿Ima-ta-m qawa-ra-ya-nki?}}	&‘What are you looking at (\pb{persistently})?’\\[0.5ex]
{\phono{-Ri}}		& inceptive			&{\phononb{Warmi-kuna-qa shinka-\pb{ri}-shpa~\dots{} waqa-n.}} 	&‘When the women [start to] get drunk~\dots{} they cry.’\\[0.5ex]
{\phono{-RU}}		& urgency, completive&{\phononb{Miku-\pb{ru}-shunki wak kundinaw-qa.}}	&‘(\pb{Careful}!) that zombie will eat you.’\\[0.5ex]
{\phono{-shi}}	& accompaniment		&{\phononb{“Harka-\pb{shi}-sa-yki-m”, ni-n.}}	&‘“I’m going to \pb{help} you pasture,” he said.’\\[0.5ex]
{\phono{-tamu}}	& irreversible 		&{\phononb{Wañu-\pb{tamu}-sha qari-qa.}}	&‘The man \pb{died}.’\\[0.5ex]
{\phono{-YkU}}	& exceptional		&{\phononb{Kay-lla-pi, Señor, tiya-\pb{yku}-y.}}	&‘Right here, Sir, \pb{please} have a seat.’\\
\lspbottomrule
\end{tabularx}
\end{table}

§~\ref{ssec:IDCA} looks at each of these suffixes in turn. \phono{-ya} (continuative), also VV derivative suffix, was treated above in~§~\ref{ssec:progressive}.

\clearpage
\subsubsection{Distribution of VV derivational suffixes}\label{modalsuffixes}
The default order of VV derivational suffixes is given in Table~\ref{Tab28}.

% TABLE 28
\begin{table}[!ht]
\small\centering
\caption{Default order of modal suffixes}\label{Tab28}
\begin{tabular}{*{17}{@{~}c@{~}}}
\lsptoprule
\phono{ka}	& \phono{pa}	& \phono{Ra}	& \phono{katra}	& \phono{cha}	& \phono{Ri}	& \phono{ykU}	& \phono{RU}	& \phono{chi}	 & \phono{shi}	& \phono{pU}	& \phono{na}	& \phono{kU}	& \phono{mu}	& \phono{lla}\\
\lspbottomrule
\end{tabular}
\end{table}

\noindent
Although this order is generally rigid, some suffixes show optional order when appearing consecutively. Causative \phono{-chi} is likely the most mobile; change in its placement results in a change in verb meaning (\phono{wañu-chi-naya-wa-n} ‘it makes me want to kill’ \phono{wañu-naya-chi-wa-n} ‘it makes me feel like I want to die’ (example from Albó~(1964), as cited in \citealt[284]{CerroP87}\index[aut]{Cerrón-Palomino, Rodolfo M.}). \phono{-chi} and continuative \phono{-ya} regularly commute~(\ref{Glo4:Llamputa}), (\ref{Glo4:suliyasawa}), as do exceptional \phono{-ykU} and reflexive \phono{-kU}~(\ref{Glo4:kuchilluwanpis}), (\ref{Glo4:paraptin}).\\

% 1
\gloexe{Glo4:Llamputa}{}{amv}%
{Llamputa mikuyka\pb{yachi}n shakashta.}%amv que first line
{\morglo{llampu-ta}{llampu-\lsc{acc}}\morglo{miku-yka-ya-chi-n}{eat-\lsc{excep}-\lsc{prog}-\lsc{caus}-\lsc{3}}\morglo{shakash-ta}{guinea.pig-\lsc{acc}}}%morpheme+gloss
\glotran{He’s making the guinea pig eat the \phono{llampu}.}{}%eng+spa trans
{}{}%rec - time

% 2
\gloexe{Glo4:suliyasawa}{}{amv}%
{Mana suliyasa kaptinqa wakta suliya\pb{chiya}nchik.}%amv que first line
{\morglo{mana}{no}\morglo{suliya-sa}{sun-\lsc{prf}}\morglo{ka-pti-n-qa}{subds-\lsc{3}-\lsc{top}}\morglo{wak-ta}{\lsc{dem.d}-\lsc{acc}}\morglo{suliya-chi-ya-nchik}{sun-\lsc{caus}-\lsc{prog}-\lsc{1pl}}}%morpheme+gloss
\glotran{When it hasn’t been sunned, we \pb{sun} it.}{}%eng+spa trans
{}{}%rec - time

% 3
\gloexe{Glo4:kuchilluwanpis}{}{ach}%
{Ima kuchilluwanpis imawanpis apunta\pb{ykuku}shpa kayhina kurriyamun.}%ach que first line
{\morglo{ima}{what}\morglo{kuchillu-wan-pis}{knife-\lsc{instr}-\lsc{add}}\morglo{ima-wan-pis}{what-\lsc{instr}-\lsc{add}}\morglo{apunta-yku-ku-shpa}{point-\lsc{excep}-\lsc{refl}-\lsc{subis}}\morglo{kay-hina}{\lsc{dem.p}-\lsc{comp}}\morglo{kurri-ya-mu-n}{run-\lsc{prog}-\lsc{cisl}-\lsc{3}}}%morpheme+gloss
\glotran{With a knife or whatever, \pb{taking aim} [at us] they’re running like this.}{}%eng+spa trans
{}{}%rec - time

% 4
\gloexe{Glo4:paraptin}{}{amv}%
{Ñuqanchikqa paraptin uvihanchik yatanpi puñunchik muntita mashta\pb{kuyku}shpam, ukunchikta yaku riptin.}%amv que first line
{\morglo{ñuqa-nchik-qa}{I-\lsc{1pl}-\lsc{top}}\morglo{para-pti-n}{rain-\lsc{subds}}\morglo{uviha-nchik}{sheep-\lsc{1pl}}\morglo{yata-n-pi}{side-\lsc{3}-\lsc{loc}}\morglo{puñu-nchik}{sleep-\lsc{1pl}}\morglo{munti-ta}{brush-\lsc{acc}}\morglo{mashta-ku-yku-shpa-m}{spread-\lsc{refl}-\lsc{excep}-\lsc{subis}-\lsc{evd}}\morglo{uku-nchik-ta}{below-\lsc{1pl}-\lsc{acc}}\morglo{yaku}{water}\morglo{ri-pti-n}{go-\lsc{subds}-\lsc{3}}}%morpheme+gloss
\glotran{When it rains, we spread out brush and sleep next to our sheep -- when the water goes below us.}{}%eng+spa trans
{}{}%rec - time

\noindent
Some combinations are not possible. Although some combinations are, arguably, precluded for pragmatic reasons (i.e., they would denote highly unlikely or even impossible states or events), the exclusion of others begs other accounts~(\ref{Glo4:shiku}).\\

% 5
\gloexe{Glo4:shiku}{}{amv}%
{*kumuda\pb{shiku}yan *kumuda\pb{kushi}yan}% que first line
{\morglo{*kumuda-shi-ku-ya-n}{comfortable-\lsc{acmp}-\lsc{refl}-\lsc{prog}-\lsc{3}}\morglo{*kumuda-ku-shi-ya-n}{comfortable-\lsc{refl}-\lsc{acmp}-\lsc{prog}-\lsc{3}}}%morpheme+gloss
\glotran{They \pb{accompanied getting} comfortable.} 
{}{}%rec - time

\subsubsection{Morphophonemics}\index[sub]{morphophonemics}
% TABLE 29
\begin{table}[!ht]
\small\centering
\caption{VV derivational suffixes -- morphophonemics}
\begin{tabular}{*{8}{l}}
\multicolumn{8}{l}{U represents an alternation between \textipa{[u]} and \textipa{[a]}.} \\
\lsptoprule
Morpheme & Realized as &\multicolumn{5}{l}{Before} & Elsewhere as\\
\midrule
-kU		& -ka	& -ma\tss{1.\lsc{obj}} & -mu	&	 & 	 &-chi &-ku \\
-pU		& -pa	& -ma\tss{1.\lsc{obj}} & -mu	& -kU 	& 	 & 	 & -pu \\
-RU		& -Ra	& -ma\tss{1.\lsc{obj}} & -mu	& -kU 	& -pU 	& -chi 	& -Ru \\
-ykU	& -yka	& -ma\tss{1.\lsc{obj}} & -mu	& 	 & -pU 	& -chi 	& -yku\\
\lspbottomrule
\end{tabular}
\end{table}

\noindent
In \SYQ, as in other Quechuan languages, the first-person-object suffix \phono{-ma}~(\ref{Glo4:gwardya}) and the cislocative suffix \phono{-mu}~(\ref{Glo4:Makiyta}) trigger the lowering of a preceding vowel \phono{-U-} to \phono{-a-}; causative suffix \phono{-chi} does so as well when it precedes \phono{-kU}, \phono{-RU}, or \phono{-ykU}~(\ref{Glo4:Wiraya}).\\

% 1
\gloexe{Glo4:gwardya}{}{sp}%
{Chay gwardya paqarinnintaq kaypaq traya\pb{ramu}n.}%sp que first line
{\morglo{chay}{\lsc{dem.d}}\morglo{gwardya}{police}\morglo{paqarin-ni-n-taq}{tomorrow-\lsc{euph}-\lsc{3}-\lsc{seq}}\morglo{kay-paq}{\lsc{dem.p}-\lsc{loc}}\morglo{traya-ra-mu-n}{arrive-\lsc{urgt}-\lsc{cisl}-\lsc{3}}}%morpheme+gloss
\glotran{The next day the police arrived here.}{}%eng+spa trans
{}{}%rec - time

% 2
\gloexe{Glo4:Makiyta}{}{amv}%
{Makiyta ñuqaqa paqa\pb{karamu}niñam.}%amv que first line
{\morglo{maki-y-ta}{hand-\lsc{1}-\lsc{acc}}\morglo{ñuqa-qa}{I-\lsc{top}}\morglo{paqa-ka-ra-mu-ni-ña-m}{wash-\lsc{refl}-\lsc{urgt}-\lsc{cisl}-\lsc{1}-\lsc{disc}-\lsc{evd}}}%morpheme+gloss
\glotran{I’ve already washed my hands.}{}%eng+spa trans
{}{}%rec - time

% 3
\gloexe{Glo4:Wiraya}{}{ach}%
{Wiraya\pb{ykachi}shpam qamtaqa mikushunki.}%ach que first line
{\morglo{wira-ya-yka-chi-shpa-m}{fat-\lsc{inch}-\lsc{excep}-\lsc{caus}-\lsc{subis}-\lsc{evd}}\morglo{qam-ta-qa}{you-\lsc{acc}-\lsc{top}}\morglo{miku-shunki}{eat-\lsc{3>2}}}%morpheme+gloss
\glotran{After she’s fattened you up, she’s going to eat you.}{}%eng+spa trans
{}{}%rec - time

\noindent
Additionally, in \SYQ, both \phono{-pU} and \phono{-kU} trigger vowel lowering, the first with \phono{-RU}~(\ref{Glo4:Tapumuptin}) and \phono{-ykU}~(\ref{Glo4:Chaytatrik}), and the second with \phono{-RU}~(\ref{Glo4:warmiqawa}) and \phono{-pU}~(\ref{Glo4:7}).\\

% 4
\gloexe{Glo4:Tapumuptin}{}{amv}%
{Tapumuptin traskirapamuway hinashpa allicha\pb{rapu}way.}%amv que first line
{\morglo{tapu-mu-pti-n}{ask-\lsc{cisl}-\lsc{subds}-\lsc{3}}\morglo{traski-ra-pa-mu-wa-y}{accept-\lsc{unint}-\lsc{ben}-\lsc{cisl}-\lsc{1.obj}-\lsc{imp}}\morglo{hinashpa}{then}\morglo{alli-cha-ra-pu-wa-y}{good-\lsc{fact}-\lsc{unint}-\lsc{ben}-\lsc{1.obj}-\lsc{imp}}}%morpheme+gloss
\glotran{When he asks, receive it for me then put it in order it \pb{for me}.}{}%eng+spa trans
{}{}%rec - time

% 5
\gloexe{Glo4:Chaytatrik}{}{amv}%
{Chaytatrik indika\pb{ykapu}wanki.}%amv que first line
{\morglo{chay-ta-tri-k}{\lsc{dem.d}-\lsc{acc}-\lsc{evc}-\lsc{ik}}\morglo{indika-yka-pu-wa-nki}{indicate-\lsc{excep}-\lsc{ben}-\lsc{1.obj}-\lsc{2}}}%morpheme+gloss
\glotran{You’re going to point that out \pb{to me}.}{}%eng+spa trans
{}{}%rec - time

% 6
\gloexe{Glo4:warmiqawa}{}{amv}%
{Wak warmiqa wawa\pb{paku}rusam.}%amv que first line
{\morglo{wak}{\lsc{dem.d}}\morglo{warmi-qa}{woman-\lsc{top}}\morglo{wawa-pa-ku-ru-sa-m}{give.birth-\lsc{mutben}-\lsc{urgt}-\lsc{npst}-\lsc{evd}}}%morpheme+gloss
\glotran{That woman gave birth to an illegitimate child.}{}%eng+spa trans
{}{}%rec - time

W.~Adelaar~(p.c.) points out that that “the morphophomemic vowel lowering presented [here] is not locally restricted.” In \phono{miku-yk\pb{a}-ya-chi-n}, for example, he writes, \phono{-ykU-} is apparently modified to \phono{-yka-} under the influence of a non-adjacent suffix \phono{-chi-}, and in \phono{ushtichi-k\pb{a}-la-\pb{mu}-y}, \phono{-kU} is apparently modified to \phono{-ka} under the influence of the non-adjacent \phono{-mu}. In these and similar cases, \SYQ{} patterns with the Central Peruvian QI, he writes. He suggests that this non-local vowel lowering may be an archaic feature since Southern Peruvian Quechua does not have it.

\subsubsection{Individual derivational and complementary suffixes}\label{ssec:IDCA}
\paragraph{\phono{-cha}}\index[sub]{diminutive}
Diminutive. \phono{-cha} indicates action performed by a child or in the manner of a child~(\ref{Glo4:willkayta}) or action of little importance.\\

% 1
\gloexe{Glo4:willkayta}{}{amv}%
{Chay willkayta uchuklla puklla\pb{cha}yan qawaykuni.}%amv que first line
{\morglo{chay}{\lsc{dem.d}}\morglo{willka-y-ta}{grandson-\lsc{1}-\lsc{acc}}\morglo{uchuk-lla}{small-\lsc{rstr}}\morglo{puklla-cha-ya-n}{play-\lsc{dim}-\lsc{prog}-\lsc{3}}\morglo{qawa-yku-ni}{look-\lsc{excep}-\lsc{1}}}%morpheme+gloss
\glotran{I look. My little grandson is playing.}{}%eng+spa trans
{}{}%rec - time

\noindent
It may also indicate an affectionate attitude on the part of the speaker~(\ref{Glo4:Imatataqru}), (\ref{Glo4:Kananna}). Not attested in the \CH{} dialect.\\

% 2
\gloexe{Glo4:Imatataqru}{}{amv}%
{¿Imatataq ruwayan pay? Graba\pb{cha}yan.}%amv que first line
{\morglo{ima-ta-taq}{what-\lsc{acc}-\lsc{seq}}\morglo{ruwa-ya-n}{make-\lsc{prog}-\lsc{3}}\morglo{pay}{she}\morglo{graba-cha-ya-n}{record-\lsc{dim}-\lsc{prog}-\lsc{3}}}%morpheme+gloss
\glotran{What is she doing? Recording.}{}%eng+spa trans
{}{}%rec - time

% 3
\gloexe{Glo4:Kananna}{}{amv}%
{Kanan nasi\pb{cha}ramunña.}%amv que first line
{\morglo{kanan}{now}\morglo{nasi-cha-ra-mu-n-ña}{be.born-\lsc{dim}-\lsc{urgt}-\lsc{cisl}-\lsc{disc}-\lsc{disc}}}%morpheme+gloss
\glotran{She’s already born now.}{}%eng+spa trans
{}{}%rec - time

\paragraph{Causative \phono{-chi}, \phono{-chi-ku}}\index[sub]{causative}
\phono{-chi} indicates that the subject causes or permits an action on the part of another participant; that is, \phono{-chi} derives verbs with the meaning ‘cause to V’~(\ref{Glo4:Ishpaychata}--\ref{Glo4:wanmiwanmi}).\\

% 1
\gloexe{Glo4:Ishpaychata}{}{amv}%
{Ishpaychata tumara\pb{chi}rqani.}%amv que first line
{\morglo{ishpay-cha-ta}{urine-\lsc{dim}-\lsc{acc}}\morglo{tuma-ra-chi-rqa-ni}{drink-\lsc{urgt}-\lsc{caus}-\lsc{pst}-\lsc{1}}}%morpheme+gloss
\glotran{I \pb{made/had} him drink urine.}{}%eng+spa trans
{}{}%rec - time

% 2
\gloexe{Glo4:Imash}{}{ach}%
{¿Imash waqa\pb{chi}shunki? ¿Ayvis waqankichu?}%ach que first line
{\morglo{ima-sh}{what-\lsc{evr}}\morglo{waqa-chi-shu-nki}{cry-\lsc{caus}-\lsc{2.obj}-\lsc{2}}\morglo{ayvis}{sometimes}\morglo{waqa-nki-chu}{cry-\lsc{2}-\lsc{q}}}%morpheme+gloss
\glotran{What \pb{makes} you cry, she asks? Do you cry sometimes?}{}%eng+spa trans
{}{}%rec - time

% 3
\gloexe{Glo4:shutuyka}{}{amv}%
{Ishchallataña shutuyka\pb{chi}yman, ¿aw?}%amv que first line
{\morglo{ishcha-lla-ta-ña}{a.little-\lsc{rstr}-\lsc{acc}-\lsc{disc}}\morglo{shutu-yka-chi-y-man}{drip-\lsc{excep}-\lsc{caus}-\lsc{1}-\lsc{cond}}\morglo{aw}{yes}}%morpheme+gloss
\glotran{I have to \pb{make} it drip just a little, right?}{}%eng+spa trans
{}{}%rec - time

% 4
\gloexe{Glo4:wanmiwanmi}{}{amv}%
{Ñakaya\pb{chi}wanmi.}%amv que first line
{\morglo{ñaka-ya-chi-wa-n-mi}{suffer-\lsc{prog}-\lsc{caus}-\lsc{1.obj}-\lsc{3}-\lsc{evd}}}%morpheme+gloss
\glotran{He’s \pb{making} me suffer.}{}%eng+spa trans
{}{}%rec - time

\noindent
Compounded with reflexive \phono{-ku,} \phono{-chi} indicates that the actor causes him/herself to act or causes or permits another to act on him/her~(\ref{Glo4:Chirirushpaqa}),~(\ref{Glo4:Yanapach}).\\

% 5
\gloexe{Glo4:Chirirushpaqa}{}{amv}%
{Chirirushpaqa manañam llushti\pb{chiku}nchu.}%amv que first line
{\morglo{chiri-ru-shpa-qa}{cold-\lsc{urgt}-\lsc{subis}-\lsc{qa}}\morglo{mana-ña-m}{no-\lsc{disc}-\lsc{evd}}\morglo{llushti-chi-ku-n-chu}{skin-\lsc{caus}-\lsc{refl}-\lsc{3}-\lsc{neg}}}%morpheme+gloss
\glotran{When it’s cold, it doesn’t \pb{let itself} be [=can’t be] skinned any more.}{}%eng+spa trans
{}{}%rec - time

% 6
\gloexe{Glo4:Yanapach}{}{amv}%
{Yanapa\pb{chiku}nki.}%amv que first line
{\morglo{yanapa-chi-ku-nki}{help-\lsc{caus}-\lsc{refl}-\lsc{2}}}%morpheme+gloss
\glotran{You’re going to \pb{get yourself} helped.}{}%eng+spa trans
{}{}%rec - time

\paragraph{Passive/accidental \phono{-ka}}\index[sub]{passive/accidental}
\phono{-ka} indicates that the event referred to is not under the control either of a participant in that event or of the speaker~(\ref{Glo4:runmi}--\ref{Glo4:Achka}).\\

% 1
\gloexe{Glo4:runmi}{}{amv}%
{Puñu\pb{ka}runmi.}% que first line
{\morglo{puñu-ka-ru-n-mi}{sleep-\lsc{passacc}-\lsc{urgt}-\lsc{3}-\lsc{evd}}}%morpheme+gloss
\glotran{She \pb{fell} asleep.}{}%eng+spa trans
{}{}%rec - time

% 2
\gloexe{Glo4:Pasaypaq}{}{amv}%
{Pasaypaq punkisa purirqa. Qapari\pb{ka}shtin rin ninmi.}%amv que first line
{\morglo{pasaypaq}{completely}\morglo{punki-sa}{swell-\lsc{prf}}\morglo{puri-rqa}{walk-\lsc{pst}}\morglo{qapari-ka-shtin}{shout-\lsc{passacc}-\lsc{subadv}}\morglo{ri-n}{go-\lsc{3}}\morglo{ni-n-mi}{say-\lsc{3}-\lsc{evd}}}%morpheme+gloss
\glotran{He was walking totally swollen. He was shouting [\pb{despite himself}].}{}%eng+spa trans
{}{}%rec - time

% 3
\gloexe{Glo4:yanchik}{}{ach}%
{Suyñu\pb{ka}yanchik runallata fiyullataña.}%ach que first line
{\morglo{suyñu-\pb{ka}-ya-nchik}{dream-\lsc{passacc}-\lsc{prog}-\lsc{1pl}}\morglo{runa-lla-ta}{person-\lsc{rstr}-\lsc{acc}}\morglo{fiyu-lla-ta-ña}{ugly-\lsc{rstr}-\lsc{acc}-\lsc{disc}}}%morpheme+gloss
\glotran{We’re \pb{having} terrible \pb{dreams} [nightmares] about the people.}{}%eng+spa trans
{}{}%rec - time

% 4
\gloexe{Glo4:Wakhina}{}{amv}%
{Wakhina lliw lliw tumba\pb{ka}rushpa~\dots}%amv que first line
{\morglo{wak-hina}{\lsc{dem.d}-\lsc{comp}}\morglo{lliw}{all}\morglo{lliw}{all}\morglo{tumba-ka-ru-shpa}{fall-\lsc{passacc}-\lsc{urgt}-\lsc{subis}}}%morpheme+gloss
\glotran{All of them, \pb{falling} down like that~\dots}{}%eng+spa trans
{}{}%rec - time

% 5
\gloexe{Glo4:Achka}{}{ch}%
{Achka luna huntalamusha. Taytalla:qa kallipa pulikusha ashi\pb{ka}yan tayta:taq.}%ch que first line
{\morglo{achka}{a.lot}\morglo{luna}{person}\morglo{hunta-la-mu-sha}{gather-\lsc{urgt}-\lsc{cisl}-\lsc{tk}}\morglo{tayta-lla-:-qa}{father-\lsc{rstr}-\lsc{1}-\lsc{top}}\morglo{kalli-pa}{street-\lsc{loc}}\morglo{puli-ku-sha}{walk-\lsc{refl}-\lsc{npst}}\morglo{ashi-ka-ya-n}{laugh-\lsc{passacc}-\lsc{prog}-\lsc{3}}\morglo{tayta-:-ta-qa}{father-\lsc{1}-\lsc{acc}-\lsc{top}}}%morpheme+gloss
\glotran{A lot of people had gathered. My father was walking in the street and they \pb{made fun} of him.}{}%eng+spa trans
{}{}%rec - time

\paragraph{Iterative \phono{-katra}}\label{par:freqkatra}\index[sub]{frequentive}
\phono{-katra} indicates extended~(\ref{Glo4:QawaQawayan}--\ref{Glo4:wayrakunaykipaq}), or repetitive~(\ref{Glo4:Killantin}--\ref{Glo4:Hinaptinqa}) action.\\

% 1
\gloexe{Glo4:QawaQawayan}{}{amv}%
{Qawa\pb{katra}yan.}%amv que first line
{\morglo{qawa-katra-ya-n}{look-\lsc{freq}-\lsc{prog}-\lsc{3}}}%morpheme+gloss
\glotran{She’s \pb{staring}’, ‘She’s \pb{looking around}.}{}%eng+spa trans
{}{}%rec - time

% 2
\gloexe{Glo4:wayrakunaykipaq}{}{amv}%
{Mana wayrakunaykipaq kaynacham apa\pb{katra}kunki.}%amv que first line
{\morglo{mana}{no}\morglo{wayra-ku-na-yki-paq}{wind-\lsc{refl}-\lsc{nmlz}-\lsc{2}-\lsc{purp}}\morglo{kayna-cha-m}{thus-\lsc{dim}-\lsc{evd}}\morglo{apa-katra-ku-nki}{bring-\lsc{freq}-\lsc{refl}-\lsc{2}}}%morpheme+gloss
\glotran{So that you don’t get bad air [sick], you’ll \pb{carry along} some just like this.}{}%eng+spa trans
{}{}%rec - time

% 3
\gloexe{Glo4:Killantin}{}{amv}%
{Killantin killantin maskani tapu\pb{katra}shpa.}%amv que first line
{\morglo{killa-ntin}{month-\lsc{incl}}\morglo{killa-ntin}{month-\lsc{incl}}\morglo{maska-ni}{search.for-\lsc{1}}\morglo{tapu-katra-shpa}{ask-\lsc{freq}-\lsc{subis}}}%morpheme+gloss
\glotran{I looked for him for months and months, \pb{asking and asking}.}{}%eng+spa trans
{}{}%rec - time

% 4
\gloexe{Glo4:maqtaqa}{}{amv}%
{Wak maqtaqa pukllayta atipanchu, qay. Yangam sayta\pb{katra}yan.}%amv que first line
{\morglo{wak}{\lsc{dem.d}}\morglo{maqta-qa}{young.man-\lsc{top}}\morglo{puklla-y-ta}{play-\lsc{inf}-\lsc{acc}}\morglo{atipa-n-chu}{be.able-\lsc{3}-\lsc{neg}}\morglo{qay}{hey}\morglo{yanga-m}{in.vain-\lsc{evd}}\morglo{sayta-\pb{katra}-ya-n}{kick-\lsc{freq}-\lsc{prog}-\lsc{3}}}%morpheme+gloss
\glotran{That boy can’t play [ball], eh. In vain, he’s \pb{kicking and kicking}.}{}%eng+spa trans
{}{}%rec - time

% 5
\gloexe{Glo4:Qunirichirqatriki}{}{amv}%
{Qunirichirqatriki. Qapari\pb{katra}rqa. Arruhaytash qallakuykun.}%amv que first line
{\morglo{quni-ri-chi-rqa-tri-ki}{warm-\lsc{incep}-\lsc{caus}-\lsc{pst}-\lsc{evc}-\lsc{iki}}\morglo{qapari-\pb{katra}-rqa}{shout-\lsc{freq}-\lsc{pst}}\morglo{arruha-y-ta-sh}{vomit-\lsc{inf}-\lsc{acc}-\lsc{evr}}\morglo{qalla-ku-yku-n}{begin-\lsc{refl}-\lsc{excep}-\lsc{3}}}%morpheme+gloss
\glotran{It must have heated him up. He \pb{shouted and shouted}. [Then] he starts to throw up, they say.}{}%eng+spa trans
{}{}%rec - time

% 6
\gloexe{Glo4:Hinaptinqa}{}{ch}%
{Hinaptinqa qaya\pb{katra}kun, “¡Abuelo Prudencio! ¡Suyaykamay! Qarqaryam qipa:ta shamukuyan.”}%ch que first line
{\morglo{hinaptin-qa}{then-\lsc{top}}\morglo{qaya-katra-ku-n}{shout-\lsc{freq}-\lsc{refl}-\lsc{3}}\morglo{abuelo}{grandfather}\morglo{Prudencio}{Prudencio}\morglo{suya-yka-ma-y}{wait-\lsc{excep}-\lsc{1.obj}-\lsc{imp}}\morglo{qarqarya-m}{zombie-\lsc{evd}}\morglo{qipa-:-ta}{behind-\lsc{1}-\lsc{acc}}\morglo{shamu-ku-ya-n}{come-\lsc{refl}-\lsc{prog}-\lsc{3}}}%morpheme+gloss
\glotran{Then he called \pb{several times}, “Grandfather Prudencio! Wait for me! A zombie is coming behind me!”}{}%eng+spa trans
{}{}%rec - time

\paragraph{Reflexive, middle, medio-passive, passive \phono{-kU}}\label{par:reflexive}\index[sub]{reflexive}\index[sub]{passive}
\phono{-kU} indicates that the subject acts on him/herself or that the subject of the verb is the object of the event referred to; that is, \phono{-kU} derives verbs with the meanings ‘V oneself’~(\ref{Glo4:Kikinpis}--\ref{Glo4:Kundina}), and ‘be V-ed’~(\ref{Glo4:huyahuya}).\\

% 1
\gloexe{Glo4:Kikinpis}{}{amv}%
{Kikinpis Campiona\pb{ku}run.}%amv que first line
{\morglo{kiki-n-pis}{self-\lsc{3}-\lsc{add}}\morglo{Campiona-ku-ru-n}{poison.with.Campion-\lsc{refl}-\lsc{urgt}-\lsc{3}}}%morpheme+gloss
\glotran{They themselves \pb{poisoned themselves with Campión}.}{}%eng+spa trans
{}{}%rec - time

% 2
\gloexe{Glo4:Kundina}{}{amv}%
{Kundina\pb{ku}rushpa chay pashña kaqta trayaramun.}%amv que first line
{\morglo{kundina-ku-ru-shpa}{condemn-\lsc{refl}-\lsc{urgt}-\lsc{subis}}\morglo{chay}{\lsc{demd}}\morglo{pashña}{girl}\morglo{ka-q-ta}{be-\lsc{ag}-\lsc{acc}}\morglo{traya-ra-mu-n}{arrive-\lsc{urgt}-\lsc{cisl}-\lsc{3}}}%morpheme+gloss
\glotran{\pb{Condemning himself} [becoming a zombie], he arrived at the girl’s place at night.}{}%eng+spa trans
{}{}%rec - time

% 3
\gloexe{Glo4:huyahuya}{}{ch}%
{Manam huya\pb{ku}:chu. Manam imapis manchachimanchu.}%ch que first line
{\morglo{mana-m}{no-\lsc{evd}}\morglo{huya-ku-:-chu}{scare-\lsc{refl}-\lsc{1}-\lsc{neg}}\morglo{mana-m}{no=\lsc{evd}}\morglo{ima-pis}{what-\lsc{add}}\morglo{mancha-chi-ma-n-chu}{scare-\lsc{caus}-\lsc{1.obj}-\lsc{3}-\lsc{neg}}}%morpheme+gloss
\glotran{I’m not \pb{scared}. Nothing scares me.}{}%eng+spa trans
{}{}%rec - time

\noindent
\phono{-kU} often functions as a dative of interest, indicating that the subject has some particular interest in the event referred to~(\ref{Glo4:inbidyusu}),~(\ref{Glo4:Mashwakuna}).\\

% 4
\gloexe{Glo4:inbidyusu}{}{lt}%
{Kay inbidyusu wawqin, “¡Suwa\pb{ka}muranki tuta!” nishpa.}%lt que first line
{\morglo{kay}{\lsc{dem.p}}\morglo{inbidyusu}{jealous}\morglo{wawqi-n}{brother-\lsc{3}}\morglo{suwa-ka-mu-ra-nki}{steal-\lsc{refl}-\lsc{cisl}-\lsc{pst}-\lsc{2}}\morglo{tuta}{night}\morglo{ni-shpa}{say-\lsc{subis}}}%morpheme+gloss
\glotran{His jealous brother said, “You stole those at night!”}{}%eng+spa trans
{}{}%rec - time

% 5
\gloexe{Glo4:Mashwakuna}{}{ch}%
{Mashwakuna ullukukunaktam ayvis talpu\pb{ku}nchik.}%ch que first line
{\morglo{mashwa-kuna}{mashwa-\lsc{pl}}\morglo{ulluku-kuna-kta-m}{ulluco-\lsc{refl}-\lsc{acc}-\lsc{evd}}\morglo{ayvis}{sometimes}\morglo{talpu-ku-nchik}{plant-\lsc{refl}-\lsc{1pl}}}%morpheme+gloss
\glotran{Sometimes we plant mashua and olluco and all.}{}%eng+spa trans
{}{}%rec - time

\noindent
\phono{-kU} is used with impersonal weather verbs~(\ref{Glo4:yanmari}); it can indicate completed action (a completed or more or less irreversible change of state)~(\ref{Glo4:Traputaqa}) (see~§~\ref{ssec:perfective} on perfective \phono{-ku}), and excess of action~(\ref{Glo4:Kashtu}),~(\ref{Glo4:Tilivisyunta}).\\

% 6
\gloexe{Glo4:yanmari}{}{amv}%
{Wayra{ku}yanmari. Wayra\pb{ku}yan, qasa\pb{ku}yan, rupa\pb{ku}yan.}%amv que first line
{\morglo{wayra-ku-ya-n-m-ari}{wind-\lsc{refl}-\lsc{prog}-\lsc{3}-\lsc{evd}-\lsc{ari}}\morglo{wayra-ku-ya-n}{wind-\lsc{refl}-\lsc{prog}-\lsc{3}}\morglo{qasa-ku-ya-n}{ice-\lsc{refl}-\lsc{prog}-\lsc{3}}\morglo{rupa-ku-ya-n}{burn-\lsc{refl}-\lsc{prog}-\lsc{3}}}%morpheme+gloss
\glotran{It’s windy. It’s windy, it’s freezing, it’s hot.}{}%eng+spa trans
{}{}%rec - time

% 7
\gloexe{Glo4:Traputaqa}{}{sp}%
{Traputaqa aparikushpa pasa\pb{ku}n.}%sp que first line
{\morglo{trapu-ta-qa}{rag-\lsc{acc}-\lsc{top}}\morglo{apa-ri-ku-shpa}{bring-\lsc{incep}-\lsc{refl}-\lsc{subis}}\morglo{pasa-ku-n}{pass-\lsc{refl}-\lsc{3}}}%morpheme+gloss
\glotran{Taking along the rag, she \pb{left}.}{}%eng+spa trans
{}{}%rec - time

% 8
\gloexe{Glo4:Kashtu}{}{amv}%
{Kashtu\pb{ku}yan.}%amv que first line
{\morglo{kashtu-ku-ya-n}{chew-\lsc{refl}-\lsc{prog}-\lsc{3}}}%morpheme+gloss
\glotran{He’s chewing \pb{a lot}.}{}%eng+spa trans
{}{}%rec - time

% 9
\gloexe{Glo4:Tilivisyunta}{}{ch}%
{Tilivisyunta lika\pb{ku}yan. Manam ñuqakunaqa gustamanchu chayqa tantu.}%ch que first line
{\morglo{tilivisyun-ta}{television-\lsc{acc}}\morglo{lika-ku-ya-n}{look.at-\lsc{refl}-\lsc{prog}-\lsc{3}}\morglo{mana-m}{no-\lsc{evd}}\morglo{ñuqa-kuna-qa}{I-\lsc{pl}-\lsc{top}}\morglo{gusta-ma-n-chu}{be.pleasing-\lsc{1.obj}-\lsc{3}-\lsc{neg}}\morglo{chay-qa}{\lsc{dem.d}}\morglo{tantu}{a.lot}}%morpheme+gloss
\glotran{They’re watching television [\pb{a lot}]. We don’t like that too much.}{}%eng+spa trans
{}{}%rec - time

\noindent
\phono{-ku} appears in reflexive verbs borrowed from Spanish, translating the Spanish pronouns \phono{me}, \phono{te}, \phono{se}, and \phono{nos}~(\ref{Glo4:kwinta}),~(\ref{Glo4:Iskapa}).\\

% 10
\gloexe{Glo4:kwinta}{}{lt}%
{Manañam kwinta\pb{ku}chuwanchu.}%lt que first line
{\morglo{mana-ña-m}{no-\lsc{disc}-\lsc{evd}}\morglo{kwinta-ku-chuwan-chu}{realize-\lsc{refl}-\lsc{1pl.cond}-\lsc{neg}}}%morpheme+gloss
\glotrannq{‘We can no longer realize it.’ \Sp~‘\spanish{Ya no podemos dar\pb{nos} cuenta}’.}{}%eng+spa trans
{}{}%rec - time

% 11
\gloexe{Glo4:Iskapa}{}{ch}%
{Iskapa\pb{ku}shaq maymanpis.}%ch que first line
{\morglo{iskapa-ku-shaq}{escape-\lsc{refl}-\lsc{1.fut}}\morglo{may-man-pis}{where-\lsc{all}-\lsc{add}}}%morpheme+gloss
\glotrannq{‘I’m going escape to wherever.’ \Sp~‘\spanish{\pb{Me} voy a escapar}’.}{}%eng+spa trans
{}{}%rec - time

\noindent
When it precedes either of the derivational suffixes \phono{-mu} or \phono{-chi} or the inflectional suffix \phono{-ma}, \phono{-kU} is realized as \phono{-ka}~(\ref{Glo4:inbidyusu}).

\paragraph{Restrictive, limitative \phono{-lla}}\index[sub]{limitative}
\phono{-lla} indicates that the event referred to remains limited to itself and is not accompanied by other events~(\ref{Glo4:runaqawam}),~(\ref{Glo4:Alkansaptin}).\\

% 1
\gloexe{Glo4:runaqawam}{}{amv}%
{Wak runaqa wama wamaqtam piliyaku\pb{lla}n.}%amv que first line
{\morglo{wak}{\lsc{dem.d}}\morglo{runa-qa}{person-\lsc{top}}\morglo{wama}{a.lot}\morglo{wamaq-ta-m}{a.lot-\lsc{acc}-\lsc{evd}}\morglo{piliya-ku-lla-n}{fight-\lsc{refl}-\lsc{rstr}-\lsc{3}}}%morpheme+gloss
\glotran{Those people fight too much, \pb{do nothing but} fight.}{}%eng+spa trans
{}{}%rec - time

% 2
\gloexe{Glo4:Alkansaptin}{}{amv}%
{Alkansaptin, “¡Suyayku\pb{lla}way!” nishpa.}%amv que first line
{\morglo{alkansa-pti-n,}{reach-\lsc{subds}-\lsc{3}}\morglo{suya-yku-lla-wa-y}{wait-\lsc{excep}-\lsc{restr}-\lsc{imp}}\morglo{ni-shpa}{say-\lsc{subis}}}%morpheme+gloss
\glotran{When he reached her, he said, “\pb{Just} wait for me!”}{}%eng+spa trans
{}{}%rec - time

\noindent
It may also express (a)~an affectionate or familiar attitude toward the event~(\ref{Glo4:Fiystapa}),~(\ref{Glo4:Aspirinakunata}), (b)~regret with regard to the event~(\ref{Glo4:Shunquy}),~(\ref{Glo4:pubrikunaqa}), or (c)~pity for event participants~(\ref{Glo4:wawakuna}).\\

% 3
\gloexe{Glo4:Fiystapa}{}{amv}%
{Fiystapa tushukunki. Kanan irransa kaku\pb{lla}nqatriki.}%amv que first line
{\morglo{fiysta-pa}{festival-\lsc{loc}}\morglo{tushu-ku-nki}{dance-\lsc{refl}-\lsc{2}}\morglo{kanan}{now}\morglo{irransa}{herranza}\morglo{ka-ku-lla-nqa-tri-ki}{be-\lsc{refl}-\lsc{rstr}-\lsc{3.fut}-\lsc{evc}-\lsc{iki}}}%morpheme+gloss
\glotran{You’ll dance at the festival. Now there’s going to be an herranza, for sure.}{}%eng+spa trans
{}{}%rec - time

% 4
\gloexe{Glo4:Aspirinakunata}{}{amv}%
{Aspirinakunata qayna puntraw apamu\pb{lla}wan qaquwan trakiyta.}%amv que first line
{\morglo{aspirina-kuna-ta}{aspirin-\lsc{pl}-\lsc{acc}}\morglo{qayna}{previous}\morglo{puntraw}{day}\morglo{apa-mu-lla-wa-n}{bring-\lsc{cisl}-\lsc{rstr}-\lsc{1.obj}-\lsc{3}}\morglo{qaqu-wa-n}{massage-\lsc{1.obj}-\lsc{3}}\morglo{traki-y-ta}{foot-\lsc{1}-\lsc{acc}}}%morpheme+gloss
\glotran{She brought me aspirin and everything yesterday and she rubbed my foot.}{}%eng+spa trans
{}{}%rec - time

% 5
\gloexe{Glo4:Shunquy}{}{amv}%
{Shunquy hunta llakiyuqtam saqi\pb{lla}sqayki; ñawiy hunta wiqiyuqtam diha\pb{lla}sqayki.}%amv que first line
{\morglo{shunqu-y}{heart-\lsc{1}}\morglo{hunta}{full}\morglo{llaki-yuq-ta-m}{sorrow-\lsc{poss}-\lsc{acc}-\lsc{evd}}\morglo{saqi-lla-sqayki}{leave-\lsc{rstr}-\lsc{1>2.fut}}\morglo{ñawi-y}{eye-\lsc{1}}\morglo{hunta}{full}\morglo{wiqi-yuq-ta-m}{tear-\lsc{poss}-\lsc{acc}-\lsc{evd}}\morglo{diha-lla-sqayki}{leave-\lsc{rstr}-\lsc{1>2.fut}}}%morpheme+gloss
\glotran{My heart full of sadness I’m going to abandon you, my eyes full of tears, I’m going to leave you.}{}%eng+spa trans
{}{}%rec - time

% 6
\gloexe{Glo4:pubrikunaqa}{}{ach}%
{Chay pubrikunaqa mana imatas yatranchu. Qullqitapis falsutapis traskillan.}%ach que first line
{\morglo{chay}{\lsc{dem.d}}\morglo{pubri-kuna-qa}{poor-\lsc{pl}-\lsc{top}}\morglo{mana}{no}\morglo{ima-ta-s}{what-\lsc{acc}-\lsc{add}}\morglo{yatra-n-chu}{know-\lsc{3}-\lsc{neg}}\morglo{qullqi-ta-pis}{money-\lsc{acc}-\lsc{add}}\morglo{falsu-ta-pis}{false-\lsc{acc}-\lsc{add}}\morglo{traski-lla-n}{accept-\lsc{rstr}-\lsc{3}}}%morpheme+gloss
\glotran{Those poor people don’t know anything. They accept counterfeit money [\pb{poor things}].}{}%eng+spa trans
{}{}%rec - time

% 7
\gloexe{Glo4:wawakuna}{}{ach}%
{Chay wawakuna kidan hukvida tristi sapan. Runapam makinpaña yatraku\pb{lla}n.}%ach que first line
{\morglo{chay}{\lsc{dem.d}}\morglo{wawa-kuna}{baby-\lsc{pl}}\morglo{kida-n}{stay-\lsc{3}}\morglo{hukvida}{a.lot}\morglo{tristi}{sad}\morglo{sapa-n}{alone-\lsc{3}}\morglo{runa-pa-m}{person-\lsc{gen}-\lsc{evd}}\morglo{maki-n-pa-ña}{hand-\lsc{3}-\lsc{loc}-\lsc{disc}}\morglo{yatra-ku-\pb{lla}-n}{live-\lsc{refl}-\lsc{rstr}-\lsc{3}}}%morpheme+gloss
\glotran{Those children remain really sad, alone. They live out of other people’s hands.}{}%eng+spa trans
{}{}%rec - time

\noindent
Other interpretations are also available~(\ref{Glo4:Qariqarillaraqchu}).\\

% 8
\gloexe{Glo4:Qariqarillaraqchu}{}{sp}%
{Qariqarillaraqchu qariqarillaraqmi niytaq niya\pb{lla}n hinashpa wañukun.}%sp que first line
{\morglo{qari-qari-lla-raq-chu}{man-man-\lsc{rstr}-\lsc{cont}-\lsc{q}}\morglo{qari-qari-lla-raq-mi}{man-man-\lsc{rstr}-\lsc{cont}-\lsc{evd}}\morglo{ni-y-taq}{say-\lsc{imp}-\lsc{seq}}\morglo{ni-ya-\pb{lla}-n}{say-\lsc{prog}-\lsc{rstr}-\lsc{3}}\morglo{hinashpa}{then}\morglo{wañu-ku-n}{die-\lsc{refl}-\lsc{3}}}%morpheme+gloss
\glotran{“Still brave and strong?” “Yes, still brave and strong!” he said for the sake of saying and died.}{}%eng+spa trans
{}{}%rec - time

\paragraph{\phono{-mu}}\index[sub]{cislocative}
In the case of verbs involving motion, \phono{-mu} indicates motion toward the speaker~(\ref{Glo4:killanta}),~(\ref{Glo4:Navidadninchik}) or toward a place which is indicated by the speaker~(\ref{Glo4:kaballuqa}--\ref{Glo4:wichayta}).\\

% 1
\gloexe{Glo4:killanta}{}{amv}%
{Ishkay killanta papaniy kartata pachimuwan wañukusanña.}%
{\morglo{ishkay}{two}\morglo{killa-n-ta}{month-\lsc{3}-\lsc{acc}}\morglo{papa-ni-y}{father-\lsc{euph}-\lsc{1}}\morglo{karta-ta}{letter-\lsc{acc}}\morglo{pachi-mu-wa-n}{send-\lsc{cisl}-\lsc{1.obj}-\lsc{3}}\morglo{wañu-ku-sa-n{}-ña}{die-\lsc{refl}-\lsc{prf}-\lsc{3}-\lsc{disc}}}%morpheme+gloss
\glotran{Two months later, my father sent me a letter that [the vicuña] had died.}{}%eng+spa trans
{}{}%rec - time

% 2
\gloexe{Glo4:Navidadninchik}{}{ch}%
{Navidadninchik traya\pb{mu}ptinqa tushukunchik.}%ch que first line
{\morglo{navidad-ni-nchik}{Christmas-\lsc{euph}-\lsc{1pl}}\morglo{traya-mu-pti-n-qa}{arrive-\lsc{cisl}-\lsc{subds}-\lsc{3}-\lsc{top}}\morglo{tushu-ku-nchik}{dance-\lsc{refl}-\lsc{1pl}}}%morpheme+gloss
\glotran{When our Christmas \pb{come}s, we dance.}{}%eng+spa trans
{}{}%rec - time

% 3
\gloexe{Glo4:kaballuqa}{}{amv}%
{Yuraq kaballuqa yuraq vakata arrastra\pb{mu}sa.}%
{\morglo{yuraq}{white}\morglo{kaballu-qa}{horse-\lsc{top}}\morglo{yuraq}{white}\morglo{vaka-ta}{cow-\lsc{acc}}\morglo{arrastra-mu-sa}{drag-\lsc{cisl}-\lsc{npst}}}%morpheme+gloss
\glotran{A white horse was dragging along a white cow.}{}%eng+spa trans
{}{}%rec - time

% 4
\gloexe{Glo4:Ladirank}{}{ach}%
{Ladirankunapaq rumipis hinkuya\pb{mu}ntriki.}%ach que first line
{\morglo{ladira-n-kuna-paq}{hillside-\lsc{3}-\lsc{pl}-\lsc{abl}}\morglo{rumi-pis}{stone-\lsc{add}}\morglo{hinku-ya-mu-n-tri-ki}{roll-\lsc{prog}-\lsc{cisl}-\lsc{3}-\lsc{evc}-\lsc{iki}}}%morpheme+gloss
\glotran{Stones, too, must be \pb{roll}ing down from the hillsides.}{}%eng+spa trans
{}{}%rec - time

% 5
\gloexe{Glo4:wichayta}{}{sp}%
{Kanan wichayta riya: uvihaman. Uviha:ta michi\pb{mu}shaq.}%sp que first line
{\morglo{kanan}{now}\morglo{wichay-ta}{up.hill-\lsc{acc}}\morglo{ri-ya-:}{go-\lsc{prog}-\lsc{1}}\morglo{uviha-man}{sheep-\lsc{all}}\morglo{uviha-:-ta}{sheep-\lsc{1}-\lsc{acc}}\morglo{michi-mu-shaq}{herd-\lsc{cisl}-\lsc{1.fut}}}%morpheme+gloss
\glotran{Now I’m going up hill to my sheep. I’m going to \pb{herd} my sheep.}{}%eng+spa trans
{}{}%rec - time

\noindent
In the case of verbs that do not involve motion, \phono{-mu} may have various senses. These may have in common that they all add a vector of movement to the action named by the V and, further, that such movement is away from ego, as an anonymous reviewer suggests~(\ref{Glo4:Lichita}),~(\ref{Glo4:Llushtichikala}).\\

% 6
\gloexe{Glo4:Lichita}{}{lt}%
{Lichita mañakara\pb{mu}y tiyuykipa.}%lt que first line
{\morglo{lichi-ta}{milk-\lsc{acc}}\morglo{maña-ka-ra-mu-y}{ask-\lsc{refl}-\lsc{urgt}-\lsc{cisl}-\lsc{imp}}\morglo{tiyu-yki-pa}{uncle-\lsc{2}-\lsc{loc}}}%morpheme+gloss
\glotran{\pb{Go ask} your uncle for milk.}{}%eng+spa trans
{}{}%rec - time

% 7
\gloexe{Glo4:Llushtichikala}{}{ch}%
{¡Llushtichikala\pb{mu}y hakuykikta!}%ch que first line
{\morglo{llushti-chi-ka-la-mu-y}{skin-\lsc{caus}-\lsc{refl}-\lsc{urgt}-\lsc{cisl}-\lsc{imp}}\morglo{haku-yki-kta}{jacket-\lsc{2}-\lsc{acc}}}%morpheme+gloss
\glotran{\pb{Go take off} your jacket!}{}%eng+spa trans
{}{}%rec - time

\paragraph{Reciprocal \phono{-nakU}}\label{par:reciprocal}\index[sub]{reciprocal}
\phono{-nakU} indicates that two or more actors act reciprocally on each other; that is, \phono{-nakU} derives verbs with the meaning ‘V each other’~(\ref{Glo4:pantyunpa}--\ref{Glo4:Kikinkunatrik}).\\

% 1
\gloexe{Glo4:pantyunpa}{}{amv}%
{¿Wakpaq pantyunpa pampa\pb{naku}nman?}%amv que first line
{\morglo{wak-paq}{\lsc{dem.d}-\lsc{abl}}\morglo{pantyun-pa}{cemetery-\lsc{loc}}\morglo{pampa-naku-n-man}{bury-\lsc{recp}-\lsc{3}-\lsc{cond}}}%morpheme+gloss
\glotran{Can people there bury \pb{each other} in the cemetery?}{}%eng+spa trans
{}{}%rec - time

% 2
\gloexe{Glo4:Kaypaqma}{}{ach}%
{Kaypaqmá kay visinukuna piliyakullan hukvidata dinunsiya\pb{naku}n maqa\pb{naku}n.}%ach que first line
{\morglo{kay-paq-m-á}{\lsc{dem.p}-\lsc{abl}-\lsc{evd}-\lsc{emph}}\morglo{kay}{\lsc{dem.p}}\morglo{visinu-kuna}{neighbor-\lsc{pl}}\morglo{piliya-ku-lla-n}{fight-\lsc{refl}-\lsc{rstr}-\lsc{3}}\morglo{hukvida-ta}{a.lot-\lsc{acc}}\morglo{dinunsiya-naku-n}{denounce-\lsc{recp}-\lsc{3}}\morglo{maqa-naku-n}{hit-\lsc{recp}-\lsc{3}}}%morpheme+gloss
\glotran{Around here, my neighbors fight a lot. They denounce \pb{each other}; they hit \pb{each other}.}{}%eng+spa trans
{}{}%rec - time

% 3
\gloexe{Glo4:Kikinkunatrik}{}{lt}%
{Kikinkunatrik ruwa\pb{naku}n wak pastuta kita\pb{naku}shpa.}%lt que first line
{\morglo{kiki-n-kuna-tri-k}{self-\lsc{3}-\lsc{pl}-\lsc{evc}-\lsc{ik}}\morglo{ruwa-naku-n}{make-\lsc{recp}-\lsc{3}}\morglo{wak}{\lsc{dem.d}}\morglo{pastu-ta}{pasture.grass-\lsc{acc}}\morglo{kita-naku-shpa}{take.away-\lsc{recp}-\lsc{subis}}}%morpheme+gloss
\glotran{They themselves do that to \pb{each other}, taking that pasture grass from \pb{each other}.}{}%eng+spa trans
{}{}%rec - time

\noindent
\phono{-na} never appears independently of \phono{-kU}. \phono{-chinakU} derives verbs with the meaning ‘cause each other to V’~(\ref{Glo4:chinaku}--\ref{Glo4:kimsam}). When it precedes either of the derivational suffixes \phono{-mu} or \phono{-chi} or the inflectional suffix \phono{-ma}, -(\phono{chi})\phono{nakU} is realized as -(\phono{chi})\phono{naka}.\\

% 4
\gloexe{Glo4:chinaku}{}{amv}%
{Yuyari\pb{chinaku}yan.}%amv que first line
{\morglo{yuya-ri-chi-naku-ya-n}{remember-\lsc{incep}-\lsc{caus}-\lsc{recp}-\lsc{prog}-\lsc{3}}}%morpheme+gloss
\glotran{They’re making \pb{each other} remember.}{}%eng+spa trans
{}{}%rec - time

% 5
\gloexe{Glo4:Kikinkamatr}{}{ach}%
{Kikinkamatr wañu\pb{chinaku}ra. Gwardyakunatr wañuchira.}%ach que first line
{\morglo{kiki-n-kama-tr}{self-\lsc{3}-\lsc{lim}-\lsc{evc}}\morglo{wañu-chi-naku-ra}{die-\lsc{caus}-\lsc{recp}-\lsc{pst}}\morglo{gwardya-kuna-tr}{police-\lsc{pl}-\lsc{evc}}\morglo{wañu-chi-ra}{die-\lsc{caus}-\lsc{pst}}}%morpheme+gloss
\glotrannq{‘They must have killed \pb{each other} themselves.’ (\lit~‘caused e.o. to die’)}{}%eng+spa trans
{}{}%rec - time

% 6
\gloexe{Glo4:kimsam}{}{ach}%
{Ishkay kimsam. Yatrachinakuykushpa misita watarun kunkanman.}%ach que first line
{\morglo{ishkay}{two}\morglo{kimsa-m}{three-\lsc{evd}}\morglo{yatra-chi-naku-yku-shpa}{know-\lsc{caus}-\lsc{recp}-\lsc{excep}-\lsc{subis}}\morglo{misi-ta}{cat-\lsc{acc}}\morglo{wata-ru-n}{tie-\lsc{urgt}-\lsc{3}}\morglo{kunka-n-man}{throat-\lsc{3}-\lsc{all}}}%morpheme+gloss
\glotrannq{‘Two or three. Teaching each other, they tied cats to their necks.’ (\lit~‘cause e.o. to know’)}{}%eng+spa trans
{}{}%rec - time

\paragraph{\phono{-naya}}\label{par:sensual}\index[sub]{sensual necessity}\index[sub]{psychological necessity}
In combination with a verb stem, V, it yields a compound verb meaning ‘to give the desire to V’~(\ref{Glo4:mikumiku}--\ref{Glo4:Hildapa}).\\

% 1
\gloexe{Glo4:mikumiku}{}{sp}%
{Tutakuykunña miku\pb{naya}n lliwña.}%sp que first line
{\morglo{tuta-ku-yku-n-ña}{night-\lsc{refl}-\lsc{excep}-\lsc{3}-\lsc{disc}}\morglo{miku-naya-n}{eat-\lsc{desr}-\lsc{3}}\morglo{lliw-ña}{all-\lsc{disc}}}%morpheme+gloss
\glotran{Night falls already and he \pb{is hungry} and everything already.}{}%eng+spa trans
{}{}%rec - time

% 2
\gloexe{Glo4:Mashwata}{}{amv}%
{Mashwata mikuptinchik ishpa\pb{naya}wanchik. Chay riñunninchikta limpiyanshi.}%amv que first line
{\morglo{mashwa-ta}{mashwa-\lsc{acc}}\morglo{miku-pti-nchik}{eat-\lsc{subds}-\lsc{1pl}}\morglo{ishpa-naya-wa-nchik}{urinate-\lsc{desr}-\lsc{1.obj}-\lsc{1pl}}\morglo{chay}{\lsc{dem.d}}\morglo{riñun-ni-nchik-ta}{kidney-\lsc{euph}-\lsc{1pl}-\lsc{acc}}\morglo{limpiya-n-shi}{wash-\lsc{3}-\lsc{evr}}}%morpheme+gloss
\glotran{When we eat mashua, it makes us \pb{want to} urinate. It cleans our kidneys, they say.}{}%eng+spa trans
{}{}%rec - time

% 3
\gloexe{Glo4:Chayta}{}{amv}%
{Chayta siguruta watanki Hilda icha tira\pb{naya}shpa iskaparunman.}%amv que first line
{\morglo{chay-ta}{\lsc{dem.d}-\lsc{acc}}\morglo{siguru-ta}{secure-\lsc{acc}}\morglo{wata-nki}{tie-\lsc{2}}\morglo{Hilda}{Hilda}\morglo{icha}{or}\morglo{tira-naya-shpa}{pull-\lsc{desr}-\lsc{subis}}\morglo{iskapa-ru-n-man}{escape-\lsc{urgt}-\lsc{3}-\lsc{cond}}}%morpheme+gloss
\glotran{Tie it up tight, Hilda, or else, \pb{wanting to} pull, it could escape.}{}%eng+spa trans
{}{}%rec - time

% 4
\gloexe{Glo4:Hildapa}{}{amv}%
{Hildapa turin maqta kay hanaypaq uraypaqa aritita ushtu\pb{naya}rachin.}%amv que first line
{\morglo{Hilda-pa}{Hilda-\lsc{gen}}\morglo{turi-n}{brother-\lsc{3}}\morglo{maqta}{young.man}\morglo{kay}{\lsc{dem.p}}\morglo{hanay-paq}{up.hill-\lsc{abl}}\morglo{uray-pa-qa}{down.hill-\lsc{loc}-\lsc{top}}\morglo{ariti-ta}{earring-\lsc{acc}}\morglo{ushtu-naya-ra-chi-n}{dress-\lsc{desr}-\lsc{urgt}-\lsc{caus}-\lsc{3}}}%morpheme+gloss
\glotran{Hilda’s brother from up here, down [on the coast] \pb{wanted to} have an earring put on.}{}%eng+spa trans
{}{}%rec - time

\noindent
Particularly with weather verbs, \phono{-naya} may indicate that the \lsc{E} named by the root V is imminent~(\ref{Glo4:Para}),~(\ref{Glo4:Shakashqa}).\\

% 5
\gloexe{Glo4:Para}{}{ach}%
{Para\pb{naya}mun.}%ach que first line
{\morglo{para-naya-mu-n.}{rain-\lsc{desr}-\lsc{cisl}-\lsc{3}}}%morpheme+gloss
\glotran{It’s \pb{about to} rain.}{}%eng+spa trans
{}{}%rec - time

% 6
\gloexe{Glo4:Shakashqa}{}{amv}%
{Shakashqa wañu\pb{naya}nña.}%amv que first line
{\morglo{shakash-qa}{giunea.pig-\lsc{top}}\morglo{wañu-naya-n-ña}{die-\lsc{desr}-\lsc{3}-\lsc{disc}}}%morpheme+gloss
\glotran{The guinea pig is \pb{about to} die already.}{}%eng+spa trans
{}{}%rec - time

\paragraph{Repetitive \phono{-pa}}\index[sub]{repetitive}
\phono{-pa} indicates repetitive action; that is, it derives verbs with the meaning ‘re-V’ or ‘V again’ or ‘repeatedly V’~(\ref{Glo4:mikusa}--\ref{Glo4:Imapaqay}) (\phono{yata} ‘touch’~→~\phono{yata-pa} ‘fondle’). It is unattested in the \CH{} dialect.\\

% 1
\gloexe{Glo4:mikusa}{}{amv}%
{Liyun mikusa. Tuqapaykun. ‘¿Wañusachu kayan?’ nishpa.}%amv que first line
{\morglo{liyun}{puma}\morglo{miku-sa}{eat-\lsc{npst}}\morglo{tuqa-pa-yku-n}{spit-\lsc{repet}-\lsc{excep}-\lsc{3}}\morglo{wañu-sa-chu}{dead-\lsc{prf}-\lsc{q}}\morglo{ka-ya-n}{be-\lsc{prog}-\lsc{3}}\morglo{ni-shpa}{say-\lsc{subis}}}%morpheme+gloss
\glotran{The puma [began to] eat it. He spit \pb{repeatedly}. “Is it dead?” he said.}{}%eng+spa trans
{}{}%rec - time

% 2
\gloexe{Glo4:puntrawhu}{}{ach}%
{Huk puntraw huk tuta nana\pb{pa}shunki.}%ach que first line
{\morglo{huk}{one}\morglo{puntraw}{day}\morglo{huk}{one}\morglo{tuta}{night}\morglo{nana-pa-shu-nki}{hurt-\lsc{repet}-\lsc{2.obj}-\lsc{2}}}%morpheme+gloss
\glotran{One day and one night it’s \pb{hurting and hurting} you [to give birth].}{}%eng+spa trans
{}{}%rec - time

% 3
\gloexe{Glo4:Imapaqtaqwa}{}{amv}%
{‘¿Imapaqtaq wak yawar yawar kayan?’ diciendo dice qawapaykun.}%
{\morglo{ima-paq-taq}{what-\lsc{purp}-\lsc{seq}}\morglo{wak}{\lsc{dem.d}}\morglo{yawar}{blood}\morglo{ka-ya-n}{be-\lsc{prog}-\lsc{3}}\morglo{qawa-pa-yku-n}{look{}-\lsc{repet}-\lsc{excep}-\lsc{3}}}%morpheme+gloss
\glotran{[They said,] “Why is there this blood, all this blood?” and stared at him.}{}%eng+spa trans
{}{}

% 4
\gloexe{Glo4:ykaramushpam}{}{lt}%
{Qawa\pb{pa}ykaramushpam.}%lt que first line
{\morglo{qawa-pa-yka-ra-mu-shpa-m}{look-\lsc{repet}-\lsc{excep}-\lsc{urgt}-\lsc{cisl}-\lsc{subis}-\lsc{evd}}}%morpheme+gloss
\glotran{Going to go \pb{check} it.}{}%eng+spa trans
{}{}%rec - time

% 5
\gloexe{Glo4:WarmiWarmi}{}{amv}%
{Warmi ka-pti-n-qa yata-pa{}-shpa-tr qaqu-ya-n.}%
{\morglo{warmi}{woman}\morglo{ka-pti-n-qa}{be-\lsc{subds}-\lsc{3}-\lsc{top}}\morglo{yata-pa-shpa-tr}{touch-\lsc{repet}-\lsc{subis}-\lsc{evc}}\morglo{qaqu-ya-n}{rub{}-\lsc{prog}-\lsc{3}}}%morpheme+gloss
\glotran{If it’s a woman he’ll be fondling her while he massages.}{}%eng+spa trans
{}{}%rec - time

% 6
\gloexe{Glo4:Imapaqay}{}{sp}%
{¿Imapaq aysa\pb{pa}maranki ñuqa hawka puñukupti:? ¡Manchachiman!}%sp que first line
{\morglo{imapaq}{what-\lsc{prup}}\morglo{aysa-pa-ma-ra-nki}{pull-\lsc{ben}-\lsc{1.obj}-\lsc{pst}-\lsc{2}}\morglo{ñuqa}{I}\morglo{hawka}{peaceful}\morglo{puñu-ku-pti-:}{sleep-\lsc{refl}-\lsc{subds}-\lsc{1}}\morglo{mancha-chi-ma-n}{scare-\lsc{caus}-\lsc{1.obj}-\lsc{3}}}%morpheme+gloss
\glotran{Why did you \pb{tug/yank} at me when I was sleeping peacefully? It scares me.}{}%eng+spa trans
{}{}%rec - time

\noindent
Compounded with intensive \phono{-ya}, \phono{-pa} indicates uninterrupted action; that is, \phono{-paya} derives verbs meaning ‘continue to V’~(\ref{Glo4:Puklla}).\\

% 7
\gloexe{Glo4:Puklla}{}{amv}%
{¿Puklla\pb{paya}nchu? ¿Kaniruytachu munayan?}%amv que first line
{\morglo{puklla-pa-ya-n-chu}{play-\lsc{repet}-\lsc{intens}-\lsc{3}-\lsc{q}}\morglo{kani-ru-y-ta-chu}{bite-\lsc{urgt}-\lsc{inf}-\lsc{acc}-\lsc{q}}\morglo{muna-ya-n}{want-\lsc{prog}-\lsc{3}}}%morpheme+gloss
\glotran{Is it \pb{still} playing? Or does it want to bite?}{}%eng+spa trans
{}{}%rec - time

\paragraph{\phono{-pU}}\index[sub]{translocative}
\phono{-pU} indicates that an action is performed on behalf~(\ref{Glo4:Chayllapa}),~(\ref{Glo4:Hinata}) --~or to the detriment~-- of someone other than the subject.\\

% 1
\gloexe{Glo4:Chayllapa}{}{amv}%
{Chayllapa pripara\pb{pu}nki.}%amv que first line
{\morglo{chay-lla-pa}{\lsc{dem.d}-\lsc{restr}-\lsc{loc}}\morglo{pripara-pu-nki}{prepare-\lsc{ben}-\lsc{2}}}%morpheme+gloss
\glotran{Just there prepare it [\pb{for her}].}{}%eng+spa trans
{}{}%rec - time

% 2
\gloexe{Glo4:Hinata}{}{lt}%
{“¡Hinata risara\pb{pu}way! Pagashaykim,” niwan.}%lt que first line
{\morglo{hina-ta}{thus-\lsc{acc}}\morglo{risa-ra-pu-wa-y}{pray-\lsc{unint}-\lsc{ben}-\lsc{1.obj}-\lsc{imp}}\morglo{paga-shayki-m}{pay-\lsc{1>2.fut}-\lsc{evd}}\morglo{ni-wa-n}{say-\lsc{1.obj}-\lsc{3}}}%morpheme+gloss
\glotran{He said to me, “Pray \pb{for me} like that! I’ll pay you.”}{}%eng+spa trans
{}{}%rec - time

\noindent
When it precedes either of the derivational suffixes \phono{-mu} or \phono{-chi} or the inflectional suffix \phono{-ma}, \phono{-pU} is realized as \phono{-pa}~(\ref{Glo4:Sigaru}),~(\ref{Glo4:Gwarda}).\\

% 3
\gloexe{Glo4:Sigaru}{}{amv}%
{Sigaru ranti\pb{pamu}wanki, Hilda, fumakushtin kutikamunanpaq.}%amv que first line
{\morglo{sigaru}{cigarette}\morglo{ranti-pa-mu-wa-nki}{buy-\lsc{ben}-\lsc{cisl}-\lsc{1.obj}-\lsc{2}}\morglo{Hilda}{Hilda}\morglo{fuma-ku-shtin}{smoke-\lsc{refl}-\lsc{subis}}\morglo{kuti-ka-mu-na-n-paq}{return-\lsc{refl}-\lsc{cisl}-\lsc{nmlz}-\lsc{3}-\lsc{purp}}}%morpheme+gloss
\glotran{Hilda, go and buy \pb{me} a cigarette so he can smoke while he’s coming back.}{}%eng+spa trans
{}{}%rec - time

% 4
\gloexe{Glo4:Gwarda}{}{ch}%
{“¡Gwarda\pb{pama}nki! ¡Gwarda\pb{pama}nki!” niman.}%ch que first line
{\morglo{gwarda-pa-ma-nki}{save-\lsc{ben}-\lsc{1.obj}-\lsc{2}}\morglo{gwarda-pa-ma-nki}{save-\lsc{ben}-\lsc{1.obj}-\lsc{2}}\morglo{ni-ma-n}{say-\lsc{1.obj}-\lsc{3}}}%morpheme+gloss
\glotran{He said to me, “Save it \pb{for me}! Save it \pb{for me}!”}{}%eng+spa trans
{}{}%rec - time

\paragraph{Joint action \phono{-pa(:)kU}}\label{par:joinaction}\index[sub]{joint action}\phono{-pa:kU} indicates action performed jointly by two or more (groups of) actors, i.e., it indicates a plurality of actors~(\ref{Glo4:Kutiramushpaqa}--\ref{Glo4:rupanta}). The long vowel may be dropped in those dialects where the first person is not indicated by vowel lengthening.\\

% 1
\gloexe{Glo4:Kutiramushpaqa}{}{amv}%
{Kutiramushpaqa kapastri tari\pb{pa:ku}nman karqa.}%
{\morglo{kuti-ra-mu-shpa-qa}{return-\lsc{urgt}-\lsc{cisl}-\lsc{subis}-\lsc{top}}\morglo{kapas-tri}{perhaps{}-\lsc{evc}}\morglo{tari-pa:ku-n-man}{find-\lsc{jtact}-\lsc{urgt}-\lsc{cond}}\morglo{ka-rqa}{be-\lsc{pst}}}%morpheme+gloss
\glotran{If \pb{they} had returned maybe \pb{they} would have found him.}{}%eng+spa trans
{}{}%rec - time

% 2
\gloexe{Glo4:Kayna}{}{ach}%
{Kayna hapi\pb{paku}nchik.}%ach que first line
{\morglo{kayna}{thus}\morglo{hapi-paku-nchik}{grab-\lsc{jtacc}-\lsc{1pl}}}%morpheme+gloss
\glotran{Like this. We hold on [to the woman to help her give birth].}{}%eng+spa trans
{}{}%rec - time

% 3
\gloexe{Glo4:Pasan}{}{sp}%
{Pasan. Lliw lliw ri\pb{pa:ku}yan. Sapalla: kashaq.}%sp que first line
{\morglo{pasa-n}{pass-3}\morglo{lliw}{all}\morglo{lliw}{all}\morglo{ri-pa:ku-ya-n}{go-\lsc{jtacc}-\lsc{prog}-\lsc{3}}\morglo{sapa-lla-:}{alone-\lsc{rstr}-\lsc{1}}\morglo{ka-shaq}{be-\lsc{be}-\lsc{1.fut}}}%morpheme+gloss
\glotran{They’re leaving. \pb{All [of them] are} going. I’m going to be all alone.}{}%eng+spa trans
{}{}%rec - time

% 4
\gloexe{Glo4:Chayshik}{}{amv}%
{Chayshik chay susiyukuna ruwa\pb{paku}rqa chay nichutanta.}%amv que first line
{\morglo{chay-shi-k}{\lsc{dem.d}-\lsc{evr}-\lsc{k}}\morglo{chay}{\lsc{dem.d}}\morglo{susiyu-kuna}{associate-\lsc{refl}}\morglo{ruwa-paku-rqa}{\lsc{make}-\lsc{mutben}-\lsc{pst}}\morglo{chay}{\lsc{dem.d}}\morglo{nichu-ta-n-ta}{\lsc{crypt}-\lsc{acc}-\lsc{3}-\lsc{acc}}}%morpheme+gloss
\glotran{That’s why, they say, before, the members made the crypts \pb{together}.}{}%eng+spa trans
{}{}%rec - time

% 5
\gloexe{Glo4:Kukakunata}{}{amv}%
{Kukakunata aku\pb{paku}nchik. Kustumbrinchikmi.}%amv que first line
{\morglo{kuka-kuna-ta}{coca-\lsc{pl}-\lsc{acc}}\morglo{aku-paku-nchik}{chew-\lsc{mutben}-\lsc{1pl}}\morglo{kustumbri-nchik-mi}{custom-\lsc{1pl}-\lsc{evd}}}%morpheme+gloss
\glotran{We chew coca [\pb{together}]. It’s our custom.}{}%eng+spa trans
{}{}%rec - time

% 6
\gloexe{Glo4:Uqaktam}{}{ch}%
{Uqaktam talpu\pb{pa:ku}ya:.}%ch que first line
{\morglo{uqa-kta-m}{oca-\lsc{acc}-\lsc{evd}}\morglo{talpu-pa:ku-ya-:}{plant-\lsc{jtacc}-\lsc{prog}-\lsc{1}}}%morpheme+gloss
\glotran{\pb{We}’re planting oca.}{}%eng+spa trans
{}{}%rec - time

% 7
\gloexe{Glo4:rupanta}{}{amv}%
{Kaña\pb{pa:ku}rqani rupanta. \emph{Comp}. Kaña\pb{paku}rqa\pb{nchik}.}%amv que first line
{\morglo{kaña-pa:ku-rqa-ni}{burn-\lsc{jtacc}-\lsc{pst}-\lsc{1}}\morglo{rupa-n-ta}{clothes-\lsc{-3}-\lsc{acc}}\morglo{kaña-paku-rqa-nchik}{burn-\lsc{jtacc}-\lsc{pst}-\lsc{1pl}}}%morpheme+gloss
\glotran{\pb{We}’ve been burning her clothes.’ ‘We have burned [for someone else].}{}%eng+spa trans
{}{}%rec - time

\paragraph{Mutual benefit \phono{-pakU}}\index[sub]{mutual benefit}
\phono{-pakU} indicates actions performed outside the scope of original planning~(\ref{Glo4:Sakristantam}--\ref{Glo4:warmiqa}) as well as actions performed as a means or preparation for something else more important (including all remunerated labor)~(\ref{Glo4:wamran}--\ref{Glo4:imatapis}).\\

% 1
\gloexe{Glo4:Sakristantam}{}{amv}%
{Sakristantam wañuchi\pb{paku}runi.}%amv que first line
{\morglo{sakristan-ta-m}{sacristan-\lsc{acc}-\lsc{evd}}\morglo{wañu-chi-paku-ru-ni}{die-\lsc{caus}-\lsc{mutben}-\lsc{urgt}-\lsc{1}}}%morpheme+gloss
\glotran{I killed the deacon [\pb{by accident}].}{}%eng+spa trans
{}{}%rec - time

% 2
\gloexe{Glo4:Urqupaqa}{}{amv}%
{Urqupaqa puchuka\pb{paku}nchikmiki.}%amv que first line
{\morglo{urqu-pa-qa}{hill-\lsc{loc}-\lsc{top}}\morglo{puchuka-paku-nchik-mi-ki}{finish-\lsc{mutben}-\lsc{1pl}-\lsc{evd}-\lsc{iki}}}%morpheme+gloss
\glotran{In the hills, we finish them [our matches] off [they run out \pb{on us}].}{}%eng+spa trans
{}{}%rec - time

% 3
\gloexe{Glo4:warmiqa}{}{amv}%
{Wak warmiqa wawa\pb{paku}rusam. Wawa\pb{paku}qtriki kidarqa.}%amv que first line
{\morglo{wak}{\lsc{dem.d}}\morglo{warmi-qa}{woman-\lsc{top}}\morglo{wawa-paku-ru-sa-m}{give.birth-\lsc{mutben}-\lsc{urgt}-\lsc{npst}-\lsc{evd}}\morglo{wawa-paku-q-tri-ki}{give.birth-\lsc{mutben}-\lsc{ag}-\lsc{evc}-\lsc{iki}}\morglo{kida-rqa}{remain-\lsc{pst}}}%morpheme+gloss
\glotran{That woman gave birth to an \pb{illegitimate child}. She must have stayed a \pb{single mother}.}{}%eng+spa trans
{}{}%rec - time

% 4
\gloexe{Glo4:wamran}{}{amv}%
{Tihi\pb{paku}shpalla wamran uywan.}%amv que first line
{\morglo{tihi-paku-shpa-lla}{weave-\lsc{mutben}-\lsc{subis}-\lsc{rstr}}\morglo{wamra-n}{child-\lsc{3}}\morglo{uywa-n}{raise-\lsc{3}}}%morpheme+gloss
\glotran{Just weaving [\pb{for pay}], she’s raising her son.}{}%eng+spa trans
{}{}%rec - time

% 5
\gloexe{Glo4:siyrapaqa}{}{sp}%
{Kay siyrapaqa pasiya\pb{paku}: michi\pb{paku}:.}%sp que first line
{\morglo{kay}{\lsc{dem.p}}\morglo{siyra-pa-qa}{mountain-\lsc{loc}-\lsc{top}}\morglo{pasiya-paku-:}{walk-\lsc{mutben}-\lsc{1}}\morglo{michi-paku-:}{herd-\lsc{mutben}-\lsc{1}}}%morpheme+gloss
\glotran{In these mountains, I pasture, I herd [\pb{for others}].}{}%eng+spa trans
{}{}%rec - time

% 6
\gloexe{Glo4:imatapis}{}{ach}%
{Karruwanñatr kanan imatapis ranti\pb{paku}yan chay llamayuqkuna alpakayuqkuna.}%ach que first line
{\morglo{karru-wan-ña-tr}{car-\lsc{instr}-\lsc{disc}-\lsc{evc}}\morglo{kanan}{now}\morglo{ima-ta-pis}{what-\lsc{acc}-\lsc{add}}\morglo{ranti-paku-ya-n}{buy-\lsc{mutben}-\lsc{prog}-\lsc{3}}\morglo{chay}{\lsc{dem.d}}\morglo{llama-yuq-kuna}{llama-\lsc{poss}-\lsc{pl}}\morglo{alpaka-yuq-kuna}{alpaca-\lsc{poss}-\lsc{pl}}}%morpheme+gloss
\glotran{Now the people with llamas and the people with alpacas must be buying everything [\pb{in order to sell it}] with a car.}{}%eng+spa trans
{}{}%rec - time

\noindent
When it precedes either of the derivational suffixes \phono{-mu} or \phono{-chi} or the inflectional suffix \phono{-ma}, \phono{-pakU} is realized as \phono{-paka}~(\ref{Glo4:Sibadata}).\\

% 7
\gloexe{Glo4:Sibadata}{}{amv}%
{Sibadata taka\pb{paka}ra\pb{mu}shaq waway machka mikunanpaq.}%amv que first line
{\morglo{sibada-ta}{barley-\lsc{acc}}\morglo{taka-paka-ra-mu-shaq}{beat-\lsc{mutben}-\lsc{cisl}-\lsc{1.fut}}\morglo{wawa-y}{baby-\lsc{1}}\morglo{machka}{cereal.meal}\morglo{miku-na-n-paq}{eat-\lsc{nmlz}-\lsc{3}-\lsc{purp}}}%morpheme+gloss
\glotran{I’m going to thresh barley [\pb{for someone else}] so my children can eat toasted barley.}{}%eng+spa trans
{}{}%rec - time

\paragraph{Uninterrupted action \phono{-Ra}}\index[sub]{uninterrupted action}
\phono{-Ra} --~realized as \phono{-la} in the \CH{} dialect and as \phono{-ra} in all others~-- indicates that the event referred to persists in time; that is, it derives verbs with the meaning ‘continue to V’~(\ref{Glo4:qaqaman}--\ref{Glo4:urata}).\\

% 1
\gloexe{Glo4:qaqaman}{}{sp}%
{Rinki qaqaman tiya\pb{ra}chishunki.}%sp que first line
{\morglo{ri-nki}{go-\lsc{2}}\morglo{qaqa-man}{cliff-\lsc{all}}\morglo{tiya-ra-chi-shu-nki}{sit-\lsc{unint}-\lsc{caus}-\lsc{2.obj}-\lsc{2}}}%morpheme+gloss
\glotran{You’ll go to the cliff and he’ll make you sit and sit [\pb{stay}] there.}{}%eng+spa trans
{}{}%rec - time

% 2
\gloexe{Glo4:mashta}{}{lt}%
{Durasnu~\dots{} llullu mashta\pb{ra}kuyan.}%lt que first line
{\morglo{durasnu}{peach}\morglo{llullu}{unripe}\morglo{mashta-ra-ku-ya-n}{spread.out-\lsc{unint}-\lsc{refl}-\lsc{prog}-\lsc{3}}}%morpheme+gloss
\glotran{Peaches~\dots{} They’re \pb{spread out} unripe.}{}%eng+spa trans
{}{}%rec - time

% 3
\gloexe{Glo4:urata}{}{lt}%
{Qawa\pb{raya}mun pashñaqa urata.}%lt que first line
{\morglo{qawa-ra-ya-mu-n}{look-\lsc{unint}-\lsc{intens}-\lsc{cisl}-\lsc{3}}\morglo{pashñaqa}{girl-\lsc{top}}\morglo{ura-ta}{hour-\lsc{acc}}}%morpheme+gloss
\glotran{The girl \pb{kept checking} the time.}{}%eng+spa trans
{}{}%rec - time

\noindent
In combination with intensive \phono{-ya}, \phono{-Ra} derives passive verbs from active verbs~(\ref{Glo4:kundurlla}--\ref{Glo4:sapallaka}).\\

% 4
\gloexe{Glo4:kundurlla}{}{amv}%
{Qaqapa ismu kundurlla warku\pb{raya}n.}%amv que first line
{\morglo{qaqa-pa}{cliff-\lsc{loc}}\morglo{ismu}{rotted}\morglo{kundur-lla}{condor-\lsc{rstr}}\morglo{warku-ra-ya-n}{hang-\lsc{unint}-\lsc{intens}-\lsc{3}}}%morpheme+gloss
\glotran{A rotten condor is \pb{hanging} from a cliff, they say.}{}%eng+spa trans
{}{}%rec - time

% 5
\gloexe{Glo4:Pwintikamatra}{}{amv}%
{Pwintikama trayaruptin huk mamakucha traqna\pb{raya}sa pwintipa.}%amv que first line
{\morglo{pwinti-kama}{bridge-\lsc{all}}\morglo{traya-ru-pti-n}{arrive-\lsc{urgt}-\lsc{subds}-\lsc{3}}\morglo{huk}{one}\morglo{mamakucha}{grandmother}\morglo{traqna-\pb{ra-ya}-sa}{bind.limbs-\lsc{unint}-\lsc{intens}-\lsc{npst}}\morglo{pwinti-pa}{bridge-\lsc{loc}}}%morpheme+gloss
\glotran{When he arrived at the bridge, an old woman \pb{was tied up} to the bridge.}{}%eng+spa trans
{}{}%rec - time

% 6
\gloexe{Glo4:tullatamqala}{}{sp}%
{“Qala tullatam aparun.” “¿Maypaqtaq chay aparusa?” “Ukllupam trura\pb{raya}sa.”}%sp que first line
{\morglo{qala}{dog}\morglo{tulla-ta-m}{bone-\lsc{acc}-\lsc{evd}}\morglo{apa-ru-n}{bring-\lsc{urgt}-\lsc{3}}\morglo{may-paq-taq}{where-\lsc{abl}-\lsc{seq}}\morglo{chay}{\lsc{dem.d}}\morglo{apa-ru-sa}{bring-\lsc{urgt}-\lsc{nspt}}\morglo{ukllu-pa-m}{store.house-\lsc{loc}-\lsc{evd}}\morglo{trura-\pb{ra-ya}-sa}{put-\lsc{unint}-\lsc{intens}-\lsc{npst}}}%morpheme+gloss
\glotran{“The dog took a bone.” “Where was it taken from?” “It \pb{was stored} in the store-house.”}{}%eng+spa trans
{}{}%rec - time

% 7
\gloexe{Glo4:sapallaka}{}{ach}%
{Kamallapaña sapalla: hita\pb{raya}pti: runa trayaramun.}%ach que first line
{\morglo{kama-lla-pa-ña}{bed-\lsc{rstr}-\lsc{loc}-\lsc{disc}}\morglo{sapa-lla-:}{alone-\lsc{rstr}-\lsc{1}}\morglo{hita-ra-ya-pti-:}{throw.out-\lsc{unint}-\lsc{intens}-\lsc{subds}-\lsc{1}}\morglo{runa}{person}\morglo{traya-ra-mu-n}{arrive-\lsc{urgt}-\lsc{cisl}-\lsc{3}}}%morpheme+gloss
\glotran{When I \pb{was layed out} in bed all alone, a person came.}{}%eng+spa trans
{}{}%rec - time

\paragraph{Inceptive \phono{-Ri}}\index[sub]{inceptive}
\phono{-Ri}, realized \phono{-li} in Cacra~(\ref{Glo4:manalaq}), indicates that the event referred to is in its initial stage, that it has not yet concluded~(\ref{Glo4:ParaPara}--\ref{Glo4:kalabasuy}).\\

% 1
\gloexe{Glo4:manalaq}{}{ch}%
{Nina:qa manalaq lupa\pb{li}yanchu. Manalaq shansha: kanchu.}%ch que first line
{\morglo{nina-:-qa}{fire-\lsc{1}-\lsc{top}}\morglo{mana-laq}{no-\lsc{cont}}\morglo{lupa-li-ya-n-chu}{burn-\lsc{incep}-\lsc{prog}-\lsc{3}-\lsc{neg}}\morglo{mana-laq}{no-\lsc{cont}}\morglo{shansha-:}{ember-\lsc{1}}\morglo{ka-n-chu}{be-\lsc{3}-\lsc{neg}}}%morpheme+gloss
\glotran{My fire still isn’t starting to burn. I still don’t have any embers.}{}%eng+spa trans
{}{}%rec - time

% 2
\gloexe{Glo4:ParaPara}{}{amv}%
{Para\pb{ri}runqañam.}%amv que first line
{\morglo{para-ri-ru-nqa-ña-m}{rain-\lsc{incep}-\lsc{urgt}-\lsc{3.fut}-\lsc{disc}-\lsc{evd}}}%morpheme+gloss
\glotran{It’s starting to rain already.}{}%eng+spa trans
{}{}%rec - time

% 3
\gloexe{Glo4:Warmikunaqa}{}{amv}%
{Warmikunaqa shinkarishpa takishpam waqan.}%
{\morglo{warmi-kuna-qa}{woman-\lsc{pl}-\lsc{top}}\morglo{shinka-ri-shpa}{get.drunk{}-\lsc{incep}-\lsc{subis}}\morglo{taki-shpa-m}{sing-\lsc{subis}-\lsc{evd}}\morglo{waqa-n}{cry-\lsc{3}}}%morpheme+gloss
\glotran{When the women start to get drunk and sing, they cry.}{}%eng+spa trans
{}{}

% 4
\gloexe{Glo4:kalabasuy}{}{lt}%
{Chaypa kalabasuy chinka\pb{ri}yanñam.}%lt que first line
{\morglo{chay-pa}{\lsc{dem.d}-\lsc{loc}}\morglo{kalabasu-y}{squash-\lsc{1}}\morglo{chinka-ri-ya-n-ña-m}{lose-\lsc{incep}-\lsc{prog}-\lsc{3}-\lsc{disc}-\lsc{evd}}}%morpheme+gloss
\glotran{My squash there are getting lost.}{}%eng+spa trans
{}{}%rec - time

\noindent
\phono{-ri} is common in apologetic statements and supplicatory commands~(\ref{Glo4:Pasakamuy}),~(\ref{Glo4:Kaytatr}). \phono{-li} is attested in Carcra but not in Hongos.\\

% 5
\gloexe{Glo4:Pasakamuy}{}{amv}%
{¡Pasakamuy! ¡Tiya\pb{ri}kuy!}%amv que first line
{\morglo{pasa-ka-mu-y}{pass-\lsc{refl}-\lsc{cisl}-\lsc{imp}}\morglo{tiya-ri-ku-y}{sit-\lsc{incep}-\lsc{refl}-\lsc{imp}}}%morpheme+gloss
\glotran{Come in! \pb{Please} sit down.}{}%eng+spa trans
{}{}%rec - time

% 6
\gloexe{Glo4:Kaytatr}{}{amv}%
{Kaytatr paqa\pb{ri}kushun.}%amv que first line
{\morglo{kay-ta-tr}{\lsc{dem.d}-\lsc{acc}-\lsc{evc}}\morglo{paqa-\pb{ri}-ku-shun}{pay-\lsc{incep}-\lsc{refl}-\lsc{1pl.fut}}}%morpheme+gloss
\glotran{\pb{Let’s} wash this.}{}%eng+spa trans
{}{}%rec - time

\paragraph{Urgency, personal interest \phono{-RU}}\label{par:urgpers}\index[sub]{urgency/personal interest}
\phono{-RU} is realized as \phono{-lU} in the \CH{} dialect~(\ref{Glo4:paqwapaqwa}) and as \phono{-rU} in all others. It has a variety of interpretations, all subsumed, in some grammars of other Quechuan languages, as “action with urgency or personal interest”~(\ref{Glo4:virdita}--\ref{Glo4:Sinvirgwinsa}).\\

% 1
\gloexe{Glo4:virdita}{}{amv}%
{“Mana virdita mikushpaqa lukiya\pb{ru}shaq”, nin.}%amv que first line
{\morglo{mana}{no}\morglo{virdi-ta}{green-\lsc{acc}}\morglo{miku-shpa-qa}{eat-\lsc{subis}-\lsc{top}}\morglo{luki-ya-ru-shaq}{crazy-\lsc{inch}-\lsc{urgt}-\lsc{1.fut}}\morglo{ni-n}{say-\lsc{3}}}%morpheme+gloss
\glotran{They say, “If I don’t eat green [pasture grass], I’m going to go crazy.”}{}%eng+spa trans
{}{}%rec - time

% 2
\gloexe{Glo4:rantikuptinqa}{}{amv}%
{Chay mana rantikuptinqa~\dots{} chaki\pb{ru}nqa.}%amv que first line
{\morglo{chay}{\lsc{dem.d}}\morglo{mana}{no}\morglo{ranti-ku-pti-n-qa}{buy-\lsc{refl}-\lsc{subds}-\lsc{3}-\lsc{top}}\morglo{chaki-ru-nqa}{dry-\lsc{urgt}-\lsc{3.fut}}}%morpheme+gloss
\glotran{If she doesn’t sell it [right away], it’s going to dry out [and be worthless].}{}%eng+spa trans
{}{}%rec - time

% 3
\gloexe{Glo4:Sinvirgwinsa}{}{amv}%
{“¡Sinvirgwinsa! ¡Ñuqaqa willaku\pb{ru}shaqmi gwardyanman tirruku kasaykita!”}%amv que first line
{\morglo{sinvirgwinsa}{shameless}\morglo{ñuqa-qa}{I-\lsc{top}}\morglo{willa-ku-ru-shaq-mi}{tell-\lsc{refl}-\lsc{urgt}-\lsc{1.fut}-\lsc{evd}}\morglo{gwardyan-man}{police-\lsc{all}}\morglo{tirruku}{terrorist}\morglo{ka-sa-yki-ta}{be-\lsc{prf}-\lsc{2}-\lsc{acc}}}%morpheme+gloss
\glotran{“Shameless bastard! I’m going to tell the police that you were a terrorist!”}{}%eng+spa trans
{}{}%rec - time

\noindent
It very often marks perfective aspect~(\ref{Glo4:paqwapaqwa}--\ref{Glo4:chichinanpaq}) (see~§~\ref{par:simplepast} on past-tense marker \phono{-RQa}).\footnote{An anonymous reviewer suggests that Yauyos \phono{-ru} is a “budding completive/perfective aspect marker, very similar to \phono{-rQu} in Cuzco and in Huallaga, but less well developed than perfective \phono{-ru} in Tarma. And far less developed than past tense/perfective \phono{-ru} in South Conchucos, where it has moved to the inflectional tense slot and is in paradigmatic relation with \phono{-rQa}, \phono{-shQa}, futures, conditional, \etc” The reviewer cites \citet{bybee1994}: the inference of recent past is not uncommon for derivational completive aspect markers.}\\

% 4
\gloexe{Glo4:paqwapaqwa}{}{ch}%
{Qali paqwa\pb{lu}n allicha\pb{lu}:.}%ch que first line
{\morglo{qali}{man}\morglo{paqwa-lu-n}{finish-\lsc{urgt}-\lsc{3}}\morglo{alli-cha-lu-:.}{good-\lsc{fact}-\lsc{urgt}-\lsc{1}}}%morpheme+gloss
\glotran{The men finish\pb{ed} and we fix\pb{ed} it up.}{}%eng+spa trans
{}{}%rec - time

% 5
\gloexe{Glo4:ChinkaChinka}{}{ach}%
{Chinka\pb{ru}n. Ni may risan yatrakunchu.}%ach que first line
{\morglo{chinka-ru-n}{lose-\lsc{urgt}-\lsc{3}}\morglo{ni}{nor}\morglo{may}{where}\morglo{ri-sa-n}{go-\lsc{prf}-\lsc{3}}\morglo{yatra-ku-n-chu}{know-\lsc{refl}-\lsc{3}-\lsc{neg}}}%morpheme+gloss
\glotran{They \pb{got lost}. We don’t know where they went.}{}%eng+spa trans
{}{}%rec - time

% 6
\gloexe{Glo4:chichinanpaq}{}{amv}%
{Mana chichinanpaq tardi wata\pb{ru}n mamanta wawanta kapacha\pb{ru}n.}%amv que first line
{\morglo{mana}{no}\morglo{chichi-na-n-paq}{nurse-\lsc{nmlz}-\lsc{3}-\lsc{purp}}\morglo{tardi}{late}\morglo{wata-ru-n}{tie-\lsc{urgt}-\lsc{3}}\morglo{mama-n-ta}{mother-\lsc{3}-\lsc{acc}}\morglo{wawa-n-ta}{baby-\lsc{3}-\lsc{acc}}\morglo{kapacha-ru-n}{muzzle-\lsc{urgt}-\lsc{3}}}%morpheme+gloss
\glotran{So that he wouldn’t nurse, she ti\pb{ed} up his mother and \pb{put} a muzzle on her baby.}{}%eng+spa trans
{}{}%rec - time

\noindent
When it precedes any of the derivational suffixes \phono{-mu}, \phono{-pU}, \phono{-kU}, \phono{-chi} or the inflectional suffix \phono{-ma}, \phono{-RU} is realized as \phono{-Ra}~(\ref{Glo4:Campionchata}),~(\ref{Glo4:ChaymiChaymi}).\\

% 7
\gloexe{Glo4:Campionchata}{}{amv}%
{Campionchata winarun aytrikurun qarinta miku\pb{rachi}n.}%
{\morglo{Campion-cha-ta}{\phono{Campion}.rat.poison-\lsc{dim}-\lsc{acc}}\morglo{wina-ru-n}{add.in-\lsc{urgt}-\lsc{3}}\morglo{aytri-ku-ru-n}{stir-\lsc{refl}-\lsc{urgt}-\lsc{3}}\morglo{qari-n-ta}{man-\lsc{3}-\lsc{acc}}\morglo{miku-ra-chi-n}{eat-\lsc{urgt}-\lsc{caus}-\lsc{3}}}%morpheme+gloss
\glotran{She threw in the rat poison, stirred it, and made her husband eat it.}{}%eng+spa trans
{}{}%rec - time

% 8
\gloexe{Glo4:ChaymiChaymi}{}{sp}%
{Chaymi, “¡Kaypaq hurqa\pb{rama}nki kay hawlapaq.”}%sp que first line
{\morglo{chay-mi}{\lsc{dem.d}-\lsc{evd}}\morglo{kay-paq}{\lsc{dem.p}-\lsc{abl}}\morglo{hurqa-ra-ma-nki}{remove-\lsc{urgt}-\lsc{1.obj}-\lsc{2}}\morglo{kay}{\lsc{dem.p}}\morglo{hawla-paq}{cage-\lsc{abl}}}%morpheme+gloss
\glotran{So, [he said,] “Take me out of this! [Let me out] of this cage here!”}{}%eng+spa trans
{}{}%rec - time

\paragraph{Accompaniment \phono{-sHi}}\index[sub]{accompaniment}
\phono{-sHi} is realized as \phono{-si} in the \SP{} dialect~(\ref{Glo4:AsnuqaAsnuqa}) and as \phono{-shi} in all others.\\

% 1
\gloexe{Glo4:AsnuqaAsnuqa}{}{sp}%
{Asnuqa nin, “Ñuqa tari\pb{si}sayki sugaykitaqa”.}%sp que first line
{\morglo{asnu-qa}{donkey-\lsc{top}}\morglo{ni-n,}{say-\lsc{3}}\morglo{ñuqa}{I}\morglo{tari-si-sayki}{find-\lsc{acmp}-\lsc{1>2.fut}}\morglo{suga-yki-ta-qa}{rope-\lsc{2}-\lsc{acc}-\lsc{top}}}%morpheme+gloss
\glotran{The donkey said, “I’m going to \pb{help} you find your rope.”}{}%eng+spa trans
{}{}%rec - time

\noindent
\phono{-sHi} indicates accompaniment for the purpose of aiding or protecting; that is, \phono{-sHi} derives verbs meaning ‘accompany in V-ing’~(\ref{Glo4:hamurqachu}) or ‘help V’~(\ref{Glo4:Harka}--\ref{Glo4:manchumanchu}).\\

% 2
\gloexe{Glo4:hamurqachu}{}{amv}%
{Manam hamurqachu tiya\pb{shi}q.}%
{\morglo{mana-m}{no-\lsc{evd}}\morglo{hamu-rqa-chu}{come-\lsc{pst}-\lsc{3}-\lsc{neg}}\morglo{tiya-shi-q}{sit-\lsc{acmp}-\lsc{ag}}}%morpheme+gloss
\glotran{She didn’t come to \pb{help} sit.}%eng+spa trans
{‘No vino a acompañar a sentar’.}%
{}{}%{Vinac\_JC\_Cure}{04:19--04:23}%

% 3
\gloexe{Glo4:Harka}{}{amv}%
{Harka\pb{shi}saykim nin huvin.}%
{\morglo{harka-shi-sayki-m}{herd-\lsc{acmp}-\lsc{1>2.fut}-\lsc{evd}}\morglo{ni-n}{say-3}\morglo{huvin}{young.man}}%morpheme+gloss
\glotran{“I’m going to \pb{help} you pasture,” the young man said.}%eng+spa trans
{‘“Te voy a ayudar a pastear”, le dijo el joven’.}%
{}{}%{Madean\_GH\_FourStories}{02:43--02:46}%

% 4
\gloexe{Glo4:Hampi}{}{amv}%
{Hampi\pb{shi}rqatrik. ¿Imataq kutichirqa?}%amv que first line
{\morglo{hampi-shi-rqa-tri-k}{heal-\lsc{acmp}-\lsc{pst}-\lsc{evc}-\lsc{ik}}\morglo{ima-taq}{what-\lsc{seq}}\morglo{kuti-chi-rqa}{return-\lsc{caus}-\lsc{pst}}}%morpheme+gloss
\glotran{She must have \pb{helped} cure. What did she offer?}{}%eng+spa trans
{}{}%rec - time

% 5
\gloexe{Glo4:manchumanchu}{}{ch}%
{Kwida\pb{shi}manchu. Hapalla: kwidaku: hapalla:.}%ch que first line
{\morglo{kwida-shi-ma-n-chu}{care.for-\lsc{acmp}-\lsc{1.obj}-\lsc{3}-\lsc{neg}}\morglo{hapa-lla-:}{alone-\lsc{rstr}-\lsc{1}}\morglo{kwida-ku-:}{take.care-\lsc{refl}-\lsc{1}}\morglo{hapa-lla-:}{alone-\lsc{rstr}-\lsc{1}}}%morpheme+gloss
\glotran{He didn’t \pb{help} take care [of the animals]. Alone, I took care of them. Alone.}{}%eng+spa trans
{}{}%rec - time

\paragraph{Irreversible change \phono{-tamu}}\index[sub]{irreversible change}
\phono{-tamu} indicates an irreversible change~(\ref{Glo4:mastakuyashpa}--\ref{Glo4:Atuqtaqa}). It is very frequently used in the \CH{} dialect but not often spontaneously attested in other dialects.\\

% 1
\gloexe{Glo4:mastakuyashpa}{}{ch}%
{Kaman mastakuyashpa kamanpa tiyakuykushpaqa wañu\pb{tamu}sha.}%ch que first line
{\morglo{kama-n}{bed-\lsc{3}}\morglo{masta-ku-ya-shpa}{spread.out-\lsc{refl}-\lsc{prog}-\lsc{subis}}\morglo{kama-n-pa}{bed	-\lsc{3}-\lsc{loc}}\morglo{tiya-ku-yku-shpa-qa}{sit-\lsc{refl}-\lsc{excep}-\lsc{subis}-\lsc{top}}\morglo{wañu-tamu-sha}{die-\lsc{irrev}-\lsc{npst}}}%morpheme+gloss
\glotran{When she was making the bed, when she sat on the bed, she \pb{died}.}{}%eng+spa trans
{}{}%rec - time

% 2
\gloexe{Glo4:qariqa}{}{lt}%
{Wañu\pb{tamu}sha qariqa; warmiqa kidarusha.}%lt que first line
{\morglo{wañu-tamu-sha}{die-\lsc{irrev}-\lsc{npst}}\morglo{qari-qa}{man-\lsc{top}}\morglo{warmi-qa}{woman-\lsc{top}}\morglo{kida-ru-sha}{remain-\lsc{urgt}-\lsc{npst}}}%morpheme+gloss
\glotran{The man \pb{died}; the woman remained.}{}%eng+spa trans
{}{}%rec - time

% 3
\gloexe{Glo4:Puchuka}{}{amv}%
{Puchuka\pb{tamu}n.}%amv que first line
{\morglo{puchuka-tamu-n}{finish-\lsc{irrev}-\lsc{3}}}%morpheme+gloss
\glotran{It \pb{finished off}.}{}%eng+spa trans
{}{}%rec - time

% 4
\gloexe{Glo4:Atuqtaqa}{}{amv}%
{Atuqtaqa ñiti\pb{tamu}n umapaq.}%amv que first line
{\morglo{atuq-ta-qa}{fox-\lsc{acc}-\lsc{top}}\morglo{ñiti-tamu-n}{crush-\lsc{irrev}-\lsc{3}}\morglo{uma-paq}{head-\lsc{abl}}}%morpheme+gloss
\glotran{They \pb{crushed} the fox from the head.}{}%eng+spa trans
{}{}%rec - time

\paragraph{Intensive \phono{-ya}, \phono{-raya}, \phono{-paya}}\index[sub]{intensive}
\phono{-ya} is dependent; it never occurs independent of \phono{-ra} or \phono{-pa.} (see~§~\ref{par:sensual} and~\ref{par:joinaction}).\\

\noindent
\phono{-raya} is a detransitivizer, deriving passive from transitive verbs; that is, \phono{-raya} derives verbs meaning ‘be V-ed’~(\ref{Glo4:Pwintikama}--\ref{Glo4:sapallahi}).\\

% 1
\gloexe{Glo4:Pwintikama}{}{amv}%
{Pwintikama trayaruptin huk mamakucha traqna\pb{raya}sa pwintipa.}%amv que first line
{\morglo{pwinti-kama}{bridge-\lsc{all}}\morglo{traya-ru-pti-n}{arrive-\lsc{urgt}-\lsc{subds}-\lsc{3}}\morglo{huk}{one}\morglo{mamakucha}{grandmother}\morglo{traqna-\pb{ra-ya}-sa}{bind.limbs-\lsc{unint}-\lsc{intens}-\lsc{npst}}\morglo{pwinti-pa}{bridge-\lsc{loc}}}%morpheme+gloss
\glotran{When he arrived at the bridge, an old woman \pb{was tied up} to the bridge.}{}%eng+spa trans
{}{}%rec - time

% 2
\gloexe{Glo4:tullatamap}{}{sp}%
{“Qala tullatam aparun.” “¿Maypaqtaq chay aparusa?” “Ukllupam trura\pb{raya}sa.”}%sp que first line
{\morglo{qala}{dog}\morglo{tulla-ta-m}{bone-\lsc{acc}-\lsc{evd}}\morglo{apa-ru-n}{bring-\lsc{urgt}-\lsc{3}}\morglo{may-paq-taq}{where-\lsc{abl}-\lsc{seq}}\morglo{chay}{\lsc{dem.d}}\morglo{apa-ru-sa}{bring-\lsc{urgt}-\lsc{nspt}}\morglo{ukllu-pa-m}{store.house-\lsc{loc}-\lsc{evd}}\morglo{trura-\pb{ra-ya}-sa}{put-\lsc{unint}-\lsc{intens}-\lsc{npst}}}%morpheme+gloss
\glotran{“The dog took a bone.” “Where was it taken from?” “It \pb{was stored} in the store-house.”}{}%eng+spa trans
{}{}%rec - time

% 3
\gloexe{Glo4:sapallahi}{}{ach}%
{Kamallapaña sapalla: hita\pb{raya}pti: runa trayaramun.}%ach que first line
{\morglo{kama-lla-pa-ña}{bed-\lsc{rstr}-\lsc{loc}-\lsc{disc}}\morglo{sapa-lla-:}{alone-\lsc{rstr}-\lsc{1}}\morglo{hita-ra-ya-pti-:}{throw.out-\lsc{unint}-\lsc{intens}-\lsc{subds}-\lsc{1}}\morglo{runa}{person}\morglo{traya-ra-mu-n}{arrive-\lsc{urgt}-\lsc{cisl}-\lsc{3}}}%morpheme+gloss
\glotran{When I \pb{was layed out} in bed all alone, a person came.}{}%eng+spa trans
{}{}%rec - time

\noindent
\phono{-raya} may also indicate persistent or repetitive action~(\ref{Glo4:QawaQawa}). (see~§~\ref{par:joinaction} for further examples).\\

% 4
\gloexe{Glo4:QawaQawa}{}{lt}%
{Qawa\pb{raya}mun pashñaqa urata.}%lt que first line
{\morglo{qawa-ra-ya-mu-n}{look-\lsc{unint}-\lsc{intens}-\lsc{cisl}-\lsc{3}}\morglo{pashñaqa}{girl-\lsc{top}}\morglo{ura-ta}{hour-\lsc{acc}}}%morpheme+gloss
\glotran{The girl \pb{kept checking} the time.}{}%eng+spa trans
{}{}%rec - time

\noindent
\phono{-paya} indicates uninterrupted action; that is, \phono{-paya} derives verbs meaning ‘continue to V’~(\ref{Glo4:Pukllapaya}) (see~§~\ref{par:sensual} for further examples).\\

% 5
\gloexe{Glo4:Pukllapaya}{}{amv}%
{¿Puklla\pb{paya}nchu? ¿Kaniruytachu munayan?}%amv que first line
{\morglo{puklla-pa-ya-n-chu}{play-\lsc{repet}-\lsc{intens}-\lsc{3}-\lsc{q}}\morglo{kani-ru-y-ta-chu}{bite-\lsc{urgt}-\lsc{inf}-\lsc{acc}-\lsc{q}}\morglo{muna-ya-n}{want-\lsc{prog}-\lsc{3}}}%morpheme+gloss
\glotran{Does it \pb{keep on} playing? Or does it want to bite?}{}%eng+spa trans
{}{}%rec - time

\paragraph{Exceptional \phono{-YkU}}\label{-yku}\index[sub]{exceptional}
\phono{-YkU} has a broad range of meanings; in early grammars of other Quechuan languages \phono{-YkU} is said to indicate ‘action performed in some way different from usual’~(\ref{Glo4:Pilata}--\ref{Glo4:Piluyta}).\\

% 1
\gloexe{Glo4:Pilata}{}{ach}%
{Pilata\pb{yka}chishpash baliyasa. Baliyayta munasa.}%ach que first line
{\morglo{pilata-yka-chi-shpa-sh}{lie.face.down-\lsc{excep}-\lsc{caus}-\lsc{subis}-\lsc{evr}}\morglo{baliya-sa}{shoot-\lsc{npst}}\morglo{baliya-y-ta}{shoot-\lsc{inf}-\lsc{acc}}\morglo{muna-sa}{want-\lsc{npst}}}%morpheme+gloss
\glotran{They \pb{made them lie face-down} on the ground and shot them. They wanted to shoot.}{}%eng+spa trans
{}{}%rec - time

% 2
\gloexe{Glo4:Chaypash}{}{amv}%
{Chaypash alma trayan hinashpash kurasunninta tapaku\pb{yku}n.}%amv que first line
{\morglo{chay-pa-sh}{\lsc{dem.d}-\lsc{loc}-\lsc{evr}}\morglo{alma}{soul}\morglo{traya-n}{arrive-\lsc{3}}\morglo{hinashpa-sh}{then-\lsc{evr}}\morglo{kurasun-ni-n-ta}{heart-\lsc{euph}-\lsc{3}-\lsc{acc}}\morglo{tapa-ku-yku-n}{knock-\lsc{refl}-\lsc{excep}-\lsc{3}}}%morpheme+gloss
\glotran{The souls arrive there, they say, then they \pb{knock} their hearts.}{}%eng+spa trans
{}{}%rec - time

% 3
\gloexe{Glo4:Hinashpachaypa}{}{amv}%
{Hinashpa chaypa lliw lliw qutunaku\pb{yku}shpa almata dispachashun.}%amv que first line
{\morglo{hinashpa}{then}\morglo{chay-pa}{\lsc{dem.d}-\lsc{loc}}\morglo{lliw}{all}\morglo{lliw}{all}\morglo{qutu-naku-yku-shpa}{gather-\lsc{recp}-\lsc{excep}-\lsc{subis}}\morglo{alma-ta}{soul-\lsc{acc}}\morglo{dispacha-shun}{dispatch-\lsc{1pl.fut}}}%morpheme+gloss
\glotran{Then, when we are all \pb{grouped} together, we’ll bid farewell to the souls.}{}%eng+spa trans
{}{}%rec - time

% 4
\gloexe{Glo4:karruwan}{}{sp}%
{Kay karruwan trayamuptinqa sillaku\pb{yku}shpam riyanchik.}%sp que first line
{\morglo{kay}{\lsc{dem.p}}\morglo{karru-wan}{car-\lsc{instr}}\morglo{traya-mu-pti-n-qa}{arrive-\lsc{cisl}-\lsc{subds}-\lsc{3}-\lsc{top}}\morglo{silla-ku-yku-shpa-m}{seat-\lsc{refl}-\lsc{excep}-\lsc{subis}-\lsc{evd}}\morglo{ri-ya-nchik}{go-\lsc{prog}-\lsc{1pl}}}%morpheme+gloss
\glotran{When they arrive with the car, we’re going \pb{galloping} in a saddle.}{}%eng+spa trans
{}{}%rec - time

% 5
\gloexe{Glo4:tirrimutukunapimik}{}{amv}%
{Chay tirrimutukunapimik kahun saqa\pb{yka}ramun chaykunawan.}%amv que first line
{\morglo{chay}{\lsc{dem.d}}\morglo{tirrimutu-kuna-pi-mi-k}{earthquake-\lsc{pl}-\lsc{loc}-\lsc{evd}-\lsc{ik}}\morglo{kahun}{box}\morglo{saqa-yka-ra-mu-n}{go.down-\lsc{excep}-\lsc{urgt}-\lsc{cisl}-\lsc{3}}\morglo{chay-kuna-wan}{\lsc{dem.d}-\lsc{pl}-\lsc{instr}}}%morpheme+gloss
\glotran{In that earthquake the coffins \pb{fell down} with those.}{}%eng+spa trans
{}{}%rec - time

% 6
\gloexe{Glo4:Piluyta}{}{amv}%
{Piluyta yupa\pb{yku}shpaqa wak duyñuytaqa mikukurunkitriki.}%amv que first line
{\morglo{pilu-y-ta}{hair-\lsc{1}-\lsc{acc}}\morglo{yupa-yku-shpa-qa}{count-\lsc{excep}-\lsc{subis}-\lsc{top}}\morglo{wak}{\lsc{dem.d}}\morglo{duyñu-y-ta-qa}{owner-\lsc{1}-\lsc{acc}-\lsc{top}}\morglo{miku-ku-ru-nki-tri-ki}{eat-\lsc{refl}-\lsc{urgt}-\lsc{2}-\lsc{evc}-\lsc{iki}}}%morpheme+gloss
\glotran{“If you \pb{count} my hairs,” [said the hairless dog to the zombie] “you can eat my mistress.”}{}%eng+spa trans
{}{}%rec - time

\noindent
It merits further analysis. \phono{-YkU} is common in polite imperatives~(\ref{Glo4:Sumbriruyta}),~(\ref{Glo4:Kayllapi}).\\

% 7 (8*)
\gloexe{Glo4:Sumbriruyta}{}{amv}%
{Sumbriruyta kumadricha quykamuway.}%amv que first line
{\morglo{sumbriru-y-ta}{hat-\lsc{1}-\lsc{acc}}\morglo{kumadri-cha}{comadre-\lsc{dim}}\morglo{qu-yka-mu-wa-y}{give-\lsc{excep}-\lsc{cisl}-\lsc{1.obj}-\lsc{imp}}}%morpheme+gloss
\glotran{Comadre, do me a favor and hand me my hat.}{}%eng+spa trans
{}{}%rec - time

% 8 (9*)
\gloexe{Glo4:Kayllapi}{}{amv}%
{Kayllapi, Señor. ¡Tiya\pb{yku}y!}%amv que first line
{\morglo{kay-lla-pi,}{\lsc{dem.p}-\lsc{rstr}-\lsc{loc}}\morglo{señor}{sir}\morglo{tiya-yku-y}{sit-\lsc{excep}-\lsc{imp}}}%morpheme+gloss
\glotran{Right here, Sir, \pb{please} have a seat.}{}%eng+spa trans
{}{}%rec - time

\noindent
\phono{-YkU} also occurs with nouns referring to a time of day~(\ref{Glo4:Chaypaq}).\\

% 9 (10*)
\gloexe{Glo4:Chaypaq}{}{amv}%
{Chaypaq tuta\pb{yku}run. Tuta\pb{yku}ruptin vilata prindirun.}%amv que first line
{\morglo{chay-paq}{\lsc{dem.d}-\lsc{abl}}\morglo{tuta-yku-ru-n}{night-\lsc{excep}-\lsc{urgt}-\lsc{3}}\morglo{tuta-yku-ru-pti-n}{night-\lsc{excep}-\lsc{urgt}-\lsc{subds}-\lsc{3}}\morglo{vila-ta}{candle-\lsc{acc}}\morglo{prindi-ru-n}{light-\lsc{urgt-}\lsc{3}}}%morpheme+gloss
\glotran{Later, \pb{night fell}. When it \pb{got dark}, he lit a candle.}{}%eng+spa trans
{}{}%rec - time

\noindent
When it precedes any of the derivational suffixes \phono{-mu}, \phono{-pU}, \phono{-chi}, \phono{-RU} or the inflectional suffix \phono{-ma}, \phono{-ykU} is realized as \phono{-yka}~(\ref{Glo4:Pilata}),~(\ref{Glo4:tirrimutukunapimik}).
