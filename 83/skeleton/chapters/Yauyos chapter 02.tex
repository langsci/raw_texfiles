% CHAPTER 2 PHONOLOGY AND MORPHOPHONEMICS
\chapter{Phonology and morphophonemics}\label{ch:Phono and morpho}
This chapter covers the syllable structure, stress pattern, phonemic inventory, and morphophonemics of Southern Yauyos Quechua. 

\section{Introduction and summary}\label{sec:phon intro}
The syllable structure, stress pattern, phonemic inventory, and morphophonemics of \SYQ{} are not extraordinary. Indeed, what is most extraordinary about them is precisely how unextraordinary they are: \SYQ{} is, phonologically, extraordinarily conservative,\footnote{Other phonologically conservative Quechuan languages include Sihuas, which, like Yauyos, retains contrasts between */ch/ and */tr/, */ll/ and */l/, as well as */sh/ and */s/. Thanks to an anonymous reviewer for pointing this out.} with four of its five dialects essentially instantiating the systems proposed for Proto-Quechua in \citet{Landerman91}\index[aut]{Landerman, Peter N.}, \citet[ch.4]{CerroP87}\index[aut]{Cerrón-Palomino, Rodolfo M.}. All \SYQ{} dialects retain contrasts between \textipa{[č]} and \textipa{[ĉ]};\footnote{In Ecuador, Columbia, Bolivia, Argentina, the east and south of Peru, as well as in Sihuas, Ambo-Pasco, Tarma, Wanka, Lambayeque, Chachapoyas and Cajamarca (thanks to an anonymous reviewer for calling my attention to the final examples here), \textipa{*/ĉ/} underwent deretroflection. \SYQ, however, retains Proto-Quechua forms like \phono{\pb{tr}ina} ‘female’, \phono{\pb{tr}upa} ‘tail’, \phono{ka\pb{tr}ka-} ‘gnaw’, and \phono{qu\pb{tr}a} ‘lagoon’. Thus, in \SYQ, \phono{\pb{tr}aki} ‘foot’ contrasts with \phono{\pb{ch}aki} ‘dry’.} \textipa{[k]}, \textipa{[q]}\footnote{\textipa{*/q/} was neither velarized nor glottalized in \SYQ{} (which is not to say that these processes are the norm). The language retains, for example, the \PQ{} forms \phono{\pb{q}usa} ‘husband’, \phono{\pb{q}asa-} ‘freeze’, \phono{wa\pb{q}a-} ‘cry’, \phono{a\pb{q}u} ‘sand’, \phono{u\pb{q}u-} ‘wet’, \phono{wi\pb{q}aw} ‘waist’, \phono{wa\pb{q}ra} ‘horn’, and \phono{atu\pb{q}} ‘fox’. \SYQ{} thus retains contrasts like those between \phono{\pb{q}iru} ‘stick’ and \phono{\pb{k}iru} ‘tooth’; \phono{\pb{q}illa} ‘lazy’ and \phono{\pb{k}illa} ‘moon’.} and \textipa{[h]}\footnote{\textipa{*/h/} appears in \SYQ, as in \PQ, principally word-initially, as in \phono{\pb{h}api-} ‘grab’, \phono{\pb{h}ampi-} ‘cure’, and \phono{\pb{h}aya-} ‘be bitter’.}; \textipa{[l]} and \textipa{[λ]}; \textipa{[n]} and \textipa{[ň]};\footnote{In \SYQ, \textipa{[ň]} did not undergo depalatalization as it did in the Quechuas of Central Peru. \textipa{[ň]} figures in the first-person personal pronoun \phono{ñuqa} as well as in lexemes such as \phono{ñaka-ri-} ‘suffer’, \phono{ñaña} ‘sister’, \phono{ñiti-} ‘crush’, \phono{ñawsa} ‘blind’, and \phono{ñañu} ‘thin’. Examples of \textipa{[n]/[ň]} minimal pairs include \phono{a\pb{n}a} ‘mole’ and \phono{a\pb{ñ}a-} ‘scold’; and \phono{na} \lsc{dmy} and \phono{ña} \lsc{disc}.} and \textipa{[s]} and \textipa{[š]};\footnote{\textipa{[š]} suffered depalatalization throughout the south. \SYQ, however, retains Proto-Quechua forms like \phono{\pb{sh}imi} ‘mouth’, \phono{\pb{sh}unqu} ‘heart’, \phono{\pb{sh}ipa\pb{sh}} ‘maiden’, \phono{wa\pb{sh}a} ‘back’, \phono{i\pb{sh}kay}, ‘two’, and \phono{mi\pb{sh}ki} ‘sweet’. \textipa{[s]/[š]} minimal pairs include \phono{\pb{s}uqu} ‘gray hair’ and \phono{\pb{sh}uqu-} ‘sip’. One also finds contrasts between the native-borrowed pairs \phono{a\pb{sh}ta-} ‘move’ and \phono{a\pb{s}ta} ‘until’; and \phono{a\pb{sh}a-} ‘yawn’ and \phono{a\pb{s}a-} ‘anger’.} none of the dialects includes ejectives or aspirates in its phonemic inventory. Vowel length is contrastive in the grammars but not the lexicons of the dialects of Azángaro-Chocos-Huangáscar, Cacra-Hongos and San Pedro. In these dialects, as in all the \QI{} (\QB) languages with the exception of Pacaraos, vowel length marks the first person in both the nominal (possessive) and verbal paradigms (\phono{wasi-:} ‘my house’ and \phono{puri-:} ‘I walk’). The Cacra-Hongos dialect is unique among the five in that, there, the protomorpheme \textipa{*/r/} is generally but not uniformly realized as \textipa{[l]}, and word-initial \textipa{*/s/} and \textipa{*/h/} are generally but not uniformly realized as \textipa{[h]}, and \textipa{[š]}, respectively.\footnote{W. Adelaar (p.c.)\index[aut]{Adelaar, Willem F. H.} writes that, at least with regard to the examples given here and below, the “Cacra-Hongos development of \textipa{*/s/} to \textipa{/h/} is found throughout Junín (with the exception of Jauja). These dialects also use \phono{shamu-}, instead of \phono{hamu-}. The first form [\dots] is typical for Quechua I, and also for Ecuador and San Martín. \phono{shamu-} may be older than \phono{hamu-},” he writes, “but the correspondence is largely unpredictable according to dialects.” An anonymous reviewer adds that Sihuas retains */s/ in \phono{sama-} ‘rest’, \phono{saru-} ‘step on’, \phono{sayta-} ‘kick’, and \phono{sita}- ‘hit’, among others.} The first of these mutations it has in common with neighboring Junín. 

A note on \textipa{*/l/} {Cerrón-Palomino\index[aut]{Cerrón-Palomino, Rodolfo M.} --~like \citep{torero1964dialectos},\index[aut]{Torero, Alfredo} but unlike \citet{Parker69}\index[aut]{Parker, Gary J.}~-- does not include \textipa{*/l/} in his catalogue of proto-phonemes. He admits, however, that the status of \textipa{*/l/} is controversial. While it does occur in a small number of proto-morphemes, and, indeed, both /l/ and /ll/ occur in all of the \QI{} contemporary varieties in Ancash and Huanuco, except for Humalies and Margos (thanks to an anonymous reviwer for pointing this out), he calls it “\spanish{Un elemento marginal y parasitario}” (“a marginal and parasitic element”). He admits, however, that the hypothesis that \PQ{} included palatal lateral (\textipa{/ll/}) but not a alveolar lateral (\textipa{/l/}) runs into the problem that the universal tendency is that the presence of \textipa{/ll/} depends on the presence of \textipa{/l/}, but not vice versa \citet[123]{CerroP87}. W. Adelaar (p.c.)\index[aut]{Adelaar, Willem F. H.} writes, “In support of the controversial status of \textipa{*/l/} which runs against the universal tendency that \textipa{/λ/} presupposes \textipa{/l/}, there is the case of Amuesha (Yanesha’). This language has a generalized palatal vs. non-palatal opposition in its consonant inventory, but precisely \textipa{*/l/} is missing (apparently an areal feature shared with Quechua).” I have postulated an \textipa{/l/} for \SYQ, as both \textipa{[λ]} and \textipa{[l]} appear in more than just a few marginal lexemes. \textipa{[λ]} appears in \SYQ{} lexemes like \phono{\pb{ll}aki} ‘sadness’, \phono{\pb{ll}uqsi-} ‘exit’, \phono{a\pb{ll}in} ‘good’, \phono{a\pb{ll}qu} ‘dog’, \phono{tu\pb{ll}u} ‘bone’, \phono{ay\pb{ll}u} ‘family’, \phono{wa\pb{ll}qa} ‘garland’, and \phono{ka\pb{ll}pa} ‘strength’, among many others. As for \textipa{[l]}, as noted in §~\ref{sec:phoinvmor}, it appears, first, as an allomorph of \textipa{/r/} in the \CH{} dialect. It also appears in exclamations like \phono{¡alaláw!} ‘how cold!’ and \phono{¡añaláw!} ‘how beautiful!’ (which occur in Jaqaru, a neighboring Aymara language, as well \citealt{Castro95}), as well as in onomatopoetic terms like \phono{luqluqluqya-} ‘make the sound of boiling’. Finally, crucially, \textipa{[l]} also appears in a non-negligible number of semantically contentful lexemes, including \phono{\pb{l}apu-} ‘slap’, \phono{\pb{l}apcha-} ‘touch’, \phono{\pb{l}aqatu} ‘slug’, \phono{\pb{l}ashta} ‘snow’, \phono{\pb{l}awka-} ‘feed a fire’, \phono{\pb{l}ayqa-} ‘bewitch’, \phono{\pb{l}ani} ‘penis’, \phono{\pb{l}umba} ‘without horns’, \phono{a\pb{l}paka} ‘alpaca’, \phono{a\pb{l}mi-} ‘forge a river’, and \phono{a\pb{l}qalli} ‘testicle’. \textipa{[l]/[λ]} minimal pairs can be found in contemporary \SYQ{} in the \CH{} dialect where \textipa{[l]} is an allomorph of \textipa{/r/}. These pairs include \phono{laki-} ‘separate’ and \phono{llaki-} ‘grieve’; \phono{tali-} ‘find’ and \phono{talli-} ‘pour’; \phono{lunku} ‘sack’ and \phono{llunku} ‘picky’; and \phono{lulu} ‘kidney’ and \phono{llullu} ‘unripe’.

§~\ref{sec:syllabe structure} treats syllable structure and stress pattern; §~\ref{sec:phoinvmor}, phonemic inventory and morphophonemics; §~\ref{sec:spanish loan}, Spanish loan words. 

\section{Syllable structure and stress pattern}\label{sec:syllabe structure}
Syllable structure in \SYQ, as in other Quechuan languages, is \pCpVpCp{} except in borrowed words. That is, syllables of the form \CCV{} and \VCC{} are prohibited. One vowel does not follow another without an intervening consonant, \ie,~sequences of the form \VV{} are prohibited. Only the first syllable of a word may begin with a vowel (\phono{a.pa-} ‘bring’; \phono{ach.ka} ‘a lot’). 

As in the overwhelming majority of Quechuan languages, primary stress falls on the penultimate syllable of a word (compare \phono{yanápa-n} ‘he helps’ and \phononb{yanapá-ya-n} ‘he is helping’; \phono{awá-rqa} ‘he wove’ and \phono{awa-rqá-ni} ‘I wove’). The first syllable of a word with more than four syllables generally receives weak stress. There are two exceptions to this rule. First, in all dialects, exclamations often receive stress on the ultimate syllable (\phono{¡Achachák!} ‘What a fright!’ \phono{¡Achachalláw!} ‘How awful!’). Second, in those dialects where vowel length indicates the first person, stress falls on the ultimate syllable just in case person marking is not followed by any other suffix (\phono{uyari-yá-:} ‘I am listening’, \phono{ri-rá-:} ‘I went’).\footnote{It is worth noting that this is phenomenon is far from universal: as an anonymous reviwer points out, “all of the Ancash Quechua varieties mark first person with vowel length, but stress never falls on the lengthened syllable in word-final position. The same is true for Huamalies in western Huanuco. The phenomenon [described here for Yauyos] does hold for Huallaga in central Huanuco, as described by \citet{Weber89}”.}

\section{Phonemic inventory and morphophonemics}\label{sec:phoinvmor}\index[sub]{phonemic inventory}\index[sub]{morphophonemics}
\SYQ{} counts three native vowel phonemes:\index[sub]{phonemic inventory!vowel} \textipa{/a/}, \textipa{/i/}, and \pb{\textipa{/u/}}. In words native to \SYQ, the closed vowels \textipa{/i/} and \pb{\textipa{/u/}} have mid and lax allomorphs \textipa{[e]}, \textipa{[ɪ]} and \textipa{[o]}, \textipa{[υ]}, respectively. That is, in words native to \SYQ, no member of either of the triples \{\textipa{[i]}, \textipa{[e]}, \textipa{[ɪ]}\} or \{\textipa{[u]}, \textipa{[o]}, \textipa{[υ]}\}, is contrastive with any other member of the same triple. The alternations \textipa{[i]}~\textasciitilde~\textipa{[e]} and \textipa{[u]}~\textasciitilde~\textipa{[o]} are conditioned by environment: the second member of each pair appears in a syllable including \textipa{/q/} (\textipa{/qilla/} \ ‘lazy’~→~\textipa{[qeλa]}, \textipa{/atuq/} ‘fox’~→~\textipa{[atoq]}).\footnote{An anonymous reviewer points out that “the most complete grammars of Quechuan languages show several lexemes with mid vowels that are not conditioned by /q/. See, for example, the discussions in \citet[46--51]{Cusihuaman76} on Cuzco and in \citet[xiv--xv]{swisshelm1972} on Ancash. Similar mid vowel data are found in Ayacucho, Santiago del Estero, Cajamarca, San Martin, Huallaga, and Corongo, among others. It would be surprising (and noteworthy!) if SYQ has no such lexemes, in contrast to other Quechuan languages across the family.” I cannot at this point confirm either that Yauyos does or does not have such lexemes.}

Vowel length is contrastive in the morphologies but not the lexicons of the dialects of \ACH, \CH{} and \SP. In these dialects --~as in all the \QI{} (\QB) languages with the exception of Pacaraos~-- vowel length marks the first person in both the substantive (possessive) and verbal paradigms (\phono{wawa-\pb{:}} ‘my house’ and \phono{puri-\pb{:}} ‘I walk’ (rendered ‘\phono{wawa-\pb{y}}’ and \phono{puri-\pb{ni}} in the \AMV{} and \LT{} dialects))\footnote{It is worth noting that in some \QI{} varieties --~Huaylas, South Conchucos and Huamalies among them~-- lengthened high vowels lower to mid vowels, \eg, /wayi-:/ [waye:], /puri-:/ [pure:]. Thanks to an anonymous reviwer for pointing this out.}. 

In all dialects, the consonant inventory\index[sub]{phonemic inventory!consonant} counts seventeen native and six borrowed phonemes. The native phonemes include voiceless plosives \textipa{/p/}, \textipa{/t/}, \textipa{/ch/}, \textipa{/tr/}, \textipa{/k/} and \textipa{/q/}; voiceless fricatives \textipa{/s/}, \textipa{/sh/} and \textipa{/h/}; nasals \textipa{/m/}, \textipa{/n/} and \textipa{/ñ/}; laterals \textipa{/l/} and \textipa{/ll/}; tap \textipa{/r/}; and approximants \textipa{/w/} and \textipa{/y}/. Borrowed from Spanish are voiced plosives \textipa{/b/}, \textipa{/d/} and \textipa{/g/};\footnote{In \SYQ, \textipa{*/p/} \textipa{*/t/} and \textipa{*/k/} were not sonorized. \SYQ{} retains \PQ{} forms like \phono{wam\pb{p}u} ‘boat’ and \phono{shimpa} ‘braid’; \phono{in\pb{t}i} ‘sun’ and \phono{anta} ‘copper-colored’; and \phono{punki} ‘swell’ and \phono{pun\pb{k}u} ‘door, entryway’.} voiceless fricative \textipa{/f/}; voiced fricative \textipa{/v/}; and trill \textipa{/rr/}. In the Cacra-Hongos dialect, the protomorpheme \textipa{*/r/} is generally but not uniformly realized as \textipa{[l]} (\textipa{*}\phono{\pb{r}una}~>~\phono{\pb{l}una} ‘person’, \textipa{*}\phono{\pb{r}i-y}~>~\phono{\pb{l}i-y} ‘go!’, \textipa{*}\phono{ha\pb{r}ka-}~>~\phono{ha\pb{l}ka-} ‘herd’); and word-initial \textipa{*/s/} and \textipa{*/h/} are generally but not uniformly realized as \textipa{[h]}\footnote{This is hardly unique to Yauyos, occuring in notably in the lects of Yauyos’ immediate neighbor to the north, Junín. In \CH, as in the \QB{} lects generally, many stems retain initial \textipa{/s/}: \phono{supay} ‘phantom’, \phono{sipi} ‘root’, \phono{siki} ‘behind’, \phono{supi} ‘fart’, \phono{suwa-} ‘to rob’, \phono{sinqa} ‘nose’, \phono{sasa} ‘hard’, and \phono{siqna} ‘wrinkle’. \CH{} also shares with Junín the mutation of r to l. \CH{} patterns with Huanca with regard to all but one of the phonological innovations common to the lects of other \QB{} regions. For example, \CH{} and Huanca retain ñ and ll, ch and tr.} and \textipa{[ʃ]}, respectively (\textipa{*}\phono{\pb{s}apa}~>~\phono{\pb{h}apa} ‘alone’, \textipa{*}\phono{surqu-}~>~\phono{hurqu-} ‘take out’, \textipa{*}\phono{\pb{h}amu-}~>~\phono{\pb{sh}amu-} ‘come’, \textipa{*}\phono{\pb{h}ampatu}~>~\phono{\pb{sh}ampatu} ‘frog’).\footnote{Further examples: \phono{saru-}~>~\phono{haru-} ‘trample’, \phono{sara}~>~\phono{hara} ‘corn’, \phono{siqa-}~>~\phono{hiqa-} ‘go up’, \phono{sira-}~>~\phono{hila-} ‘sew’, \phono{sama}~>~\phono{hama} ‘rest’.} Examples of native and borrowed lexemes that resist these mutations include \phono{\pb{r}iqsi-} ‘become acquainted’ and \phono{\pb{r}iga-} ‘irrigate’; \phono{\pb{s}iki} ‘behind’ and \phono{\pb{s}apu} ‘frog’; and \phono{\pb{h}api-} ‘grab’).\footnote{In Lincha and Tana --~Cacra and Hongos’ immediate neighbors to the north-east and south-west, respectively~-- speakers may realize word-initial \textipa{*/r/} and \textipa{*/s/} as \textipa{[l]} and \textipa{[h]}, respectively, in a few cases (\textipa{*}\phono{runku-}~>~\phono{lunku-} ‘bag’, \textipa{*}\phono{\pb{s}apa}~>~\phono{\pb{h}apa} ‘alone’). These substitutions are not systematic, however, and remain exceptions.}

Tables~\ref{Tab5},~\ref{Tab6}, and~\ref{Tab7} give the vowel inventory, consonant inventory, and morphophonemics of \SYQ. If the orthographic form differs either from the usual orthographic symbol among Andean linguists or from the IPA symbol, these are noted in square brackets. Parentheses indicate a non-indigenous phoneme.

% TABLE 5
\begin{table}
\small\centering
\caption{Vowel inventory}\label{Tab5}\index[sub]{phonemic inventory!vowel}
\begin{tabular}{lccc}
\lsptoprule
  & Front  & Central  & Back  \\
\midrule
Closed (High)  & \textipa{i}  &   & \textipa{u} \\
Open (Low)  &   & \textipa{a}  &   \\
\lspbottomrule
\end{tabular}
\end{table}

% TABLE 6
\newcommand{\tabrot}[1]{\begin{rotate}{50} #1 \end{rotate}}
\begin{table}
\small\centering
\caption{Consonant inventory}\label{Tab6}\index[sub]{phonemic inventory!consonant}
\begin{tabular}{lcccccccc}
\\[4ex]
 & \tabrot{Bilabial} & \tabrot{Labio-dental} & \tabrot{Alveolar} & \tabrot{Post-alveolar} & \tabrot{Retroflex} & \tabrot{Palatal} & \tabrot{Velar}  & \tabrot{Uvular} \\
\lsptoprule                                 
Voiceless plosive & \textipa{p} &   & \textipa{t} &   & \textipa{tr [ĉ][ʈ]} & \textipa{ch [č][c]} & \textipa{k} &  q  \\
Voiced plosive & \textipa{(b)} &    & \textipa{(d)} &    &    &    & \textipa{(g)} &    \\
Nasal & \textipa{m} &    & \textipa{n} &    &    & \textipa{ñ [ň][ɲ]} &    &    \\
Trill &    &    & \textipa{(rr)[r]} &    &    &    &    &    \\
Tap or Flap &    &    & \textipa{r [ɾ]} &    &    &    &    &    \\
Voiceless fricative &    & \textipa{(f)} & \textipa{s} & \textipa{sh [š][ʃ]} &    &    & \textipa{h} &    \\
Voiced fricative &    & \textipa{(v)} &    &    &    &    &    &    \\
Approximant & \textipa{w} &    &    &    &    & \textipa{y [j]} &    &    \\
Lateral approximant &    &    & \textipa{l} &    &    & \textipa{ll [λ][ʎ]} &    &    \\
\lspbottomrule
\end{tabular}
\end{table}

% TABLE 7
\begin{table}
\renewcommand*\arraystretch{1.3}
\small\centering
\caption{Morphophonemics}\label{Tab7}\index[sub]{morphophonemics}
\begin{tabularx}{\textwidth}{lL}
\lsptoprule
\textipa{/n/} & realized as \textipa{[m]} before \textipa{/p/}; in free alternation with nasalization of the preceeding vowel before \textipa{/m/}; \mbox{(\ie,~\phono{rina\pb{n}paq}~→~\textipa{[rina\textsubbar{m}paq]})} \\

\textipa{/m/} & \textipa{[m]} is in free alternation with \textipa{[n]} before \textipa{/w/} and \textipa{/m/} \mbox{(\ie,~\phono{qa\pb{m}man}~→~\textipa{[qa\textsubbar{n}man]})} \\

\textipa{/k/} & \textipa{[k]} is in free alternation with \textipa{[ø]} before \textipa{/k/} and \textipa{/q/} \mbox{(\ie,~\phono{wa\pb{k}qa}~→~\textipa{[waqa]})}\\

\textipa{/q/} & \textipa{[q]} is in free alternation with \textipa{[ø]} before \textipa{/q/} \mbox{(\ie,~\phono{ruwaqqa}~→~\textipa{[ruwaqa]})} \\

\textipa{/q/} & \textipa{[q]} is in free alternation with \textipa{[g]} after \textipa{/n/} \mbox{(\ie,~\phono{rin\pb{q}a}~→~\textipa{[rin\textsubbar{g}a]})} \\

\textipa{/-qa/} \lsc{top} & \textipa{[qa]} is in free alternation with \textipa{[aq]} after \textipa{[aj]} \mbox{(\ie,~\phono{ch\pb{ay-qa}}~→~\textipa{[tʃajaq]})} \\

\pb{\textipa{/u/}} & realized as \textipa{[o]} or \textipa{[υ]} when it figures in a syllable that either includes \textipa{/q/} or precedes one that does \mbox{(\ie,~\phono{\pb{u}rq\pb{u}}~→~\textipa{[\textsubbar{o}rq\textsubbar{o}]})} \\

\textipa{/i/} & realized as \textipa{[e]} or \textipa{[ɛ]} when it figures in a syllable that either includes \textipa{/q/} or precedes one that does \mbox{(\ie,~\phono{q\pb{i}llu}~→~\textipa{[q\textsubbar{e}ʎu]})} \\
\lspbottomrule
\end{tabularx}
\end{table}

\section{Spanish loan words}\label{sec:spanish loan}\index[sub]{loan words}
As detailed in §~\ref{sec:endangerment}, \SYQ{} is extremely endangered: all but the most elderly speakers are bilingual and, indeed, Spanish-dominant. As a result, individual speakers are not limited by the constraints of Quechuan phonology and generally pronounce loan words with something very close to their original syllable structure and phonemes, even where these do not conform to the constraints of Quechuan phonology. With that said, where restructuring does take place, it does so according to the rules detailed in §~\ref{ssec:spalw}.

\subsection{Spanish loan word restructuring}\label{ssec:spalw}
\textit{Syllable structure violations -- vowel sequences.} In cases where the loaned word includes the prohibited sequence \textipa{*}\VV, \SYQ, like other Quechuan languages, generally applies one of three strategies: (a) the elimination of one or the other of the two vowels (‘ac\pb{ei}te’~→~\phono{as\pb{i}ti} ‘oil’); (b) the replacement of one of the two vowels by a semiconsonant (‘c\pb{ue}rpo’~→~\phono{k\pb{wi}rpu} ‘body’, ‘s\pb{ue}ño’~→~\phono{s\pb{uy}ñu} ‘dream’); or (c) the insertion of a semiconsonant between the two vowels (‘c\pb{ua}lqu\pb{ie}ra’ → \phononb{k\pb{uwa}lk\pb{iye}ra} ‘any’).\\

\noindent
\textit{Syllable structure violations -- consonant sequences.} In case the loaned word includes a syllable of the prohibited form \textipa{*}\CCV{} or \textipa{*}\VCC, \SYQ, again, like other Quechuan languages, employs one of two strategies: (a) the elimination of one of the two consonants (‘\pb{gr}ingo’~→~\phono{\pb{r}ingu} ‘gringo’) or (b) the insertion of an epenthetic vowel (‘\pb{gr}oche’~→~\phono{\pb{gur}uchi} ‘hook’, ‘crochet’).\\

\noindent
\textit{Stress pattern violations.} Speakers vary in the extent to which they restructure borrowed Spanish terms to conform to Quechua stress pattern. Plentiful are examples of both practices:

\begin{table}[!ht]
\small\centering
\caption{Loan word restructuring}\label{tab:lw rest}
\begin{tabular}{llll}
\lsptoprule
\multicolumn{2}{l}{No restructuring} & \multicolumn{2}{l}{Restructuring} \\ 
\midrule
\phono{kan\pb{á}sta-wan} & \Sp~‘\spanish{can\pb{á}sta}’ ‘basket’ & \phono{tirrurist\pb{á}-wan} & \Sp~‘\spanish{terror\pb{í}sta}’ ‘terrorist’\\ 
\phono{fw\pb{í}ra-ta} & \Sp~‘\spanish{u\pb{é}ra}’ ‘outside’ & \itshape Kañití-ta & \Sp~‘\spanish{Cañéte}’ ‘Cañete’\\ 
\phono{m\pb{ú}tu-qa} & \Sp~‘\spanish{m\pb{ó}to}’ ‘motorcycle’ & \itshape vaká-qa & \Sp~‘\spanish{váca}’ ‘cow’\\ 
\lspbottomrule
\end{tabular}
\end{table}

Words of five or more syllables permit the preservation of the original Spanish stress pattern in the interior of a word that still adheres to the Quechua pattern of assigning stress to the penultimate syllable (\phono{timblúr-wan-ráq-tri} ‘with an earthquake, still, for sure’ (\Sp~‘\spanish{temblór}’ ‘earthquake’)).\\

\noindent
\textit{Phonemic inventory -- consonants.} Spanish loan words often feature consonants foreign to the \SYQ{} inventory: voiced plosives \textipa{/b/}, \textipa{/d/} and \textipa{/g/}; voiceless fricative \textipa{/f/}; voiced fricative \textipa{/v/}; and trill \textipa{/rr/}. It might be expected that \textipa{[b]} and \textipa{[d]} would be systematically replaced with their voiceless counterparts, \textipa{[p]} and \textipa{[t]}, and that trill \textipa{[r]} would, similarly, be replaced by tap/flap \textipa{[ɾ]}. Speakers of \SYQ, even the oldest, do not in fact regularly replace these or other non-native phonemes (‘\pb{b}alde’~→~\phono{\pb{b}aldi} ‘bucket’; ‘\pb{d}octor’~→~\phono{\pb{d}uktur} ‘doctor’; ‘ca\pb{rr}o’~→~\phono{ka\pb{rr}u} ‘car’; ‘\pb{f}iesta’~→~\phono{\pb{f}iysta} ‘festival’; ‘\pb{v}elar’~→~\phono{\pb{v}ilaku-} ‘watch’, ‘hold vigil’).\\

\noindent
\textit{Phonemic inventory -- vowels.} The inventory of Spanish vowels includes two foreign to \SYQ{}: \textipa{/o/} and \textipa{/e/} (‘Di\pb{o}s’ ‘God’; ‘l\pb{e}ch\pb{e}’ ‘milk’). As detailed in §~\ref{sec:phoinvmor}, in words native to \SYQ, \textipa{[o]} and \textipa{[e]} are allophones of \pb{\textipa{/u/}} and \textipa{/i/}, respectively. It is to be expected, then, that speakers would systematically replace the \textipa{[o]} and \textipa{[e]} of Spanish loan words with native correlates \textipa{[u]} and \textipa{[i]}, respectively (‘sap\pb{o}’~→~\phono{sap\pb{u}} ‘frog’; ‘c\pb{e}rv\pb{e}za’~→~\phono{s\pb{i}rb\pb{i}sa} ‘beer’). This does indeed occur. More commonly, however, \textipa{[o]} and \textipa{[e]} are either replaced by the \pb{\textipa{/u/}} and \textipa{/i/} allophones \textipa{[υ]} and \textipa{[ɪ]} (‘cosa’~→~\textipa{[kυsa]} ‘thing’, ‘tele’~→~\textipa{[tɪlɪ]} ‘tv’) or, even, not replaced at all. The realization of non-native vowels varies both among speakers and also among words: different speakers render the same word differently and individual speakers render the same phoneme differently in different words.\\

\noindent
\textit{Special case: ‘ado’.} Spanish loan words ending in ‘-ado’ --~with the non-native \textipa{/d/} and \textipa{/o/}~-- present a special case. ‘-ado’ is generally rendered \textipa{[aw]} in \SYQ{} (‘apur\pb{ado}’~→~\phono{apur\pb{aw}} ‘quick’; ‘l\pb{ado}’~→~\phono{l\pb{aw}} ‘place’). Interestingly, \footnote{An anonymous reviewer has brought it to my attention that “in many \QI{} languages, such as several varieties in Ancash,‘-ado’~→~/a:/, e.g, ‘apura:’, la:. In fact, -la: has become a case suffix ‘at, near’ that competes with the semantic territory of the native locative.”}\\

Finally, restructuring to accommodate any of the three --~stress pattern, syllable structure or phonemic inventory~-- does not depend on restructuring to accommodate any of the others. That is, stress pattern can be restructured to eliminate violations of \SYQ{} constraints, while violations of constraints on syllable structure or phonemic inventory are left unrestructured, and similarly for any of the six possible permutations of the three.

\subsection{Loan word orthography}
I have chosen an orthography\index[sub]{orthography} that makes use of all and only the letters appearing in Tables~\ref{Tab4} and~\ref{Tab5}, above. Orthography rather strictly follows pronunciation in the case of consonants in both indigenous and borrowed words; in the case of vowels in borrowed words, it is something of an idealization (\ie,~it should not in these cases be mistaken for phonetic transcription). 

This alphabet does not include the letters ‘c’, ‘j’, ‘z’, ‘e’ or ‘o’, all of which occur in the original Spanish spelling of many borrowed words. Spanish ‘c’, ‘j’ and ‘z’ have been replaced with their \SYQ{} phonetic equivalents: “hard” ‘c’ is replaced with ‘k’; “soft” ‘c’ with ‘s’; ‘j’ with ‘h’; and ‘z’ with ‘s’. Thus, the borrowed Spanish words ‘\pb{c}a\pb{j}a’ (‘box’, ‘coffin’) and ‘\pb{c}erve\pb{z}a’ (‘beer’) are rendered \phono{\pb{k}a\pb{h}a} and \phono{\pb{s}irbi\pb{s}a}, with no change in the pronunciation of the relevant consonants in either case. Spanish ‘e’ and ‘o’, appearing simply, are replaced with ‘i’ and ‘u’ (‘c\pb{o}mpadr\pb{e}’~→~\phono{k\pb{u}mpadr\pb{i}}). Spanish vowel sequences including ‘e’ and ‘o’ are replaced as shown in Table~\ref{tab:lw orth}.

In the special case where the sequence ‘ue’ or ‘ua’ is preceded by ‘h’ --~generally not not necessarily silent in Spanish~-- ‘h’ and ‘u’ together are replaced by the semiconsonant \textipa{[w]} (‘\pb{hué}rfano’ → \phono{\pb{wi}rfanu} ‘orphan’.

I have deviated from these practices only in the case of proper names, spelling these as they are standardly spelled in Spanish. Thus, Cañete and San Jerónimo, for example, are \emph{not} rendered, as they would be under the above conventions, \textit{Kañiti} and \textit{San Hirunimu}. ‘Dios’ (‘God’) is treated as a proper name.

\clearpage
\begin{table}[!ht]
\small\centering
\caption{Loan word orthography}\label{tab:lw orth}
\begin{tabular}{l@{~→~}ll@{~→~}ll}
\lsptoprule
ea & iya &‘sol\pb{ea}’   &\phono{sul\pb{iya}-} &‘sun’  \\
au & aw &‘\pb{au}toridad’ &\phono{\pb{aw}turidad} &‘official’ \\
ía & iya &‘polic\pb{ía}’  &\phono{pulis\pb{iya}} &‘police’ \\
ia & ya  &‘famil\pb{ia}’  &\phono{famil\pb{ya}} &‘family’ \\
ie & iy  &‘s\pb{ie}mpre’  &\phono{s\pb{iy}mpri} &‘always’ \\
io & yu  &‘invid\pb{io}so’  &\phono{inbid\pb{yu}su} &‘jealous’ \\
ío & iyu &‘t\pb{ío}’   &\phono{t\pb{iyu}} &‘uncle’ \\
ua & wa  &‘g\pb{ua}rdia’  &\phono{g\pb{wa}rdya} &‘guard’ \\
ue & wi  &‘c\pb{ue}nto’   &\phono{k\pb{wi}ntu} &‘story’ \\
ue & uy  &‘s\pb{ue}ño’   &\phono{s\pb{uy}ñu} &‘dream’ \\
\lspbottomrule
\end{tabular}
\end{table}

\vfill
\null
