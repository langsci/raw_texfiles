\addchap{Acknowledgments}
\begin{refsection}
It is a joy for me to be able to acknowledge all the people and institutions who have helped me in the course of this project. I owe thanks, first, to Willem Adelaar, who read the manuscript with extraordinary care and offered me invaluable comments which saved me from numerous, numerous errors. Many thanks are due, too, to Rodolfo Cerr\'on-Palomino for comments and advice, as well as to Andr\'es Chirinos Rivera for orientation. Also offering orientation as well as generous and very enjoyable hospitality were Carmen Escalante Guti\'errez and Ricardo Valderrama Fern\'andez. Paul Heggarty -- an intrepid Andean hiker -- joined me in the field in the course of his own research; he also found me much-needed support to complete this grammar as well as its accompanying lexicon. Three anonymous reviewers offered extensive, wise comments. Limitations on my time and abilities kept me from incorporating all the changes they suggested. Selfless proofreaders also offered advice for which I am very grateful. Teachers and consultants in Yauyos number more than one hundred; they are acknowledged -- insufficiently -- in \sectref{sec:fieldwork}. In addition to these, there are many, many people in Yauyos and especially in Vi\~nac who are owed thanks for all manner of help and, above all, for friendship. Requiring special mention among these are my principal teacher, Delfina Chullukuy, my principal translator, Esther Madue\~no, and my \textit{\~na\~na} and \textit{turi} Hilda Quispe and Ram\'on Alvarado. 

Thanks go, too, to Elio A. Farina for help with \LaTeX.

Finally, I honestly don't know how to express my gratitude to Sebastian Nordhoff and Martin Haspelmath, above all for their wisdom and patience.

The fieldwork upon which the grammar and dictionary are based enjoyed the support of several institutions. I am grateful to San Jose State University which offered support in the form of a faculty development that enabled me to initiate the project. Support at the conclusion came from the Max Planck Institute for Evolutionary Anthropology; it is thanks to the MPI that I was able to turn a ragged draft into a publishable manuscript. Finally, I benefited extensively from two Documenting Endangered Languages fellowships from the National Endowment for the Humanities and National Science Foundation (FN-50099-11 and FN-501009-12). Any views, findings, conclusions, or recommendations expressed here do not necessarily reflect those of the National Endowment for the Humanities or the National Science Foundation.

Errors remain, of course, for which I am entirely responsible.
% \printbibliography[heading=subbibliography]
\end{refsection}
