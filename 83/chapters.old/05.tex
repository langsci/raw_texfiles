% CHAPTER 5 PARTICLES
\chapter{Particles}\label{ch:particles}
This chapter covers particles\index[sub]{particles} in Southern Yauyos Quechua. In \SYQ, as in most other Quechuan languages, the class of particles can be sorted into seven sub-classes: interjections (\phono{¡Alaláw!} ‘How cold!’); assenters and greetings (\phono{aw} ‘yes’); prepositions (\phono{asta} ‘until’); adverbs (\phono{ayvis} ‘sometimes’); coordinators (\phono{icha} ‘or’); negators (\phono{mana} ‘no, not’); and prenumerals (\phono{la}, \phono{las}, occurring with expressions of time). Interjections, assenters and greetings, prepositions, and adverbs are covered in §~\ref{sec:interjections}--\ref{sec:adverbs}, respectively. Coordinators are discussed in §~\ref{sec:coord} on coordination; negators in §~\ref{sec:negation} on negation; and prenumerals in Sub §~\ref{ssec:timenum} on time numerals and prenumerals.

\section{Interjections}\label{sec:interjections}\index[sub]{interjections}
All spontaneously attested indigenous exclamations share a common pattern: they begin with \phono{a} and end in \phono{w} or, less commonly, in \phono{k} or \phono{y}, as in ~(\ref{TI:¡Atratráw!}-\ref{TI:¡Atratrák!}); with the exception of the final \phono{w}, they feature almost exclusively the alveolar and palatal consonants \phono{ch}, \phono{ll}, \phono{l}, \phono{n}, \phono{ñ}, \phono{t}, and \phono{y} (which accounts for the entire catalogue of \SYQ{} alveolars and palatals with the exception of voiceless fricatives \phono{s}, \phono{sh}, and retroflex \phono{tr}); they include no vowels except for \phono{a}; they consist, with few exceptions, of three or four syllables; and they bear stress on the final syllable. Syllable repetition is not uncommon. Non-exclamatory interjections do not follow this pattern, like in ~(\ref{TI:¡Hinata!}) and (\ref{TI:¡Pay!}). Curse words are freely borrowed from Spanish~(\ref{TI:¡Karay!}--\ref{TI:¡Miyrda!}). (\ref{Glo5:Primay}--\ref{Glo5:Achachaw}) give a few examples.\\

% Table
\newcounter{Tabinter}\setcounter{Tabinter}{0}\renewcommand{\theTabinter}{\alph{Tabinter}}%
\newcommand{\Tabinterection}[2]{\refstepcounter{Tabinter}(\theTabinter)\label{TI:#1} & \Qyell{\phono{#1}} & #2 \\}%
\begin{table}[!ht]
\small\centering
\caption{Interjections}\label{Tab:Inter}
\begin{tabular}{cll}
\lsptoprule
\Tabinterection{¡Atratráw!}{‘Yikes!’ ‘What a fright!’}
\Tabinterection{¡Achachalláw!}{‘How awful!’ ‘How ugly!’}
\Tabinterection{¡Achalláw!’}{‘How beautiful!’}
\Tabinterection{¡Alaláw!’}{‘How cold!’}
\Tabinterection{¡Atatacháw!}{‘How beautiful!’}
\Tabinterection{¡Ananáw!}{‘Ouch!’}
\Tabinterection{¡Añalláw!}{‘How delicious!’}
\Tabinterection{¡Atratrák!}{‘Yikes!’ ‘What a fright!’}
\Tabinterection{¡Hinata!}{‘So be it!’}
\Tabinterection{¡Pay!}{‘Enough!’ ‘Thanks!’}
\Tabinterection{¡Karay!}{‘Darn!’}
\Tabinterection{¡Karahu!}{‘Damn!’}
\Tabinterection{¡Miyrda!}{‘Shit!’}
\lspbottomrule
\end{tabular}
\end{table}

% 1
\gloexe{Glo5:Primay}{}{amv}%
{Primay Amaciatapis chayhinashiki intrigaykururqa. ¡\pb{Achachalláw}!}%amv que first line
{\morglo{prima-y}{cousin-\lsc{1}}\morglo{Amacia-ta-pis}{Amacia-\lsc{acc}-\lsc{add}}\morglo{chay-hina-shi-ki}{\lsc{dem.d}-\lsc{comp}-\lsc{evr}-\lsc{iki}}\morglo{intriga-yku-ru-rqa}{deliver-\lsc{excep}-\lsc{urgt}-\lsc{pst}}\morglo{achachalláw}{how.awful}}%morpheme+gloss
\glotran{They delivered my cousin Amacia, too [to the Devil], they say. \pb{How awful!}}{}%eng+spa trans
{}{}%rec - time

% 2
\gloexe{Glo5:Fiyu}{}{amv}%
{Fiyu fiyu qatram warmi kasa chay warmi. ¡\pb{Atatayáw}!}%amv que first line
{\morglo{fiyu}{ugly}\morglo{fiyu}{ugly}\morglo{qatra-m}{dirty-\lsc{evd}}\morglo{warmi}{woman}\morglo{ka-sa}{be-\lsc{npst}}\morglo{chay}{\lsc{dem.d}}\morglo{warmi}{woman}\morglo{atatayáw}{how.disgusting}}%morpheme+gloss
\glotran{That woman was a horrible, filthy woman. \pb{How disgusting}!}{}%eng+spa trans
{}{}%rec - time

% 3
\gloexe{Glo5:Ayayaw}{}{amv}%
{¡\pb{Ayayáw}! Yo me asusté.}%amv que first line
{\morglo{ayayáw}{yikes}\morglo{[Spanish]}{}}%morpheme+gloss
\glotran{\pb{Yikes}! I got scared.}{}%eng+spa trans
{}{}%rec - time

% 4
\gloexe{Glo5:Hinaptinshi}{}{amv}%
{Hinaptinshi chay katataqa tiyaykun ukuman “¡\pb{Achachá}!” qayakun.}%amv que first line
{\morglo{hinaptin-shi}{then-\lsc{evr}}\morglo{chay}{\lsc{dem.d}}\morglo{kata-ta-qa}{shawl-\lsc{acc}-\lsc{top}}\morglo{tiya-yku-n}{sit-\lsc{excep}-\lsc{3}}\morglo{uku-man}{inside-\lsc{all}}\morglo{achachá}{how.hot}\morglo{qaya-ku-n}{shout-\lsc{refl}-\lsc{3}}}%morpheme+gloss
\glotran{Then he sat on the shawl and [fell] in [the boiling water]. “\pb{It’s burning}!” he shouted.}{}%eng+spa trans
{}{}%rec - time

% 5
\gloexe{Glo5:Sapallaykitr}{}{ach}%
{¿Sapallaykitr hamuyankiyá? ¡\pb{Atratrák}!}%ach que first line
{\morglo{sapa-lla-yki-tr}{alone-\lsc{rstr}-\lsc{2}-\lsc{evc}}\morglo{hamu-ya-nki-yá}{come-\lsc{prog}-\lsc{2}-\lsc{emph}}\morglo{atratrák}{how.frightening}}%morpheme+gloss
\glotran{You’re coming all alone, then? \pb{Yikes}!}{}%eng+spa trans
{}{}%rec - time

% 6
\gloexe{Glo5:Dios}{}{amv}%
{¡Dios Tayta! ¿Imapaq kimawanchikman? ¡\pb{Achachalláw}!}%amv que first line
{\morglo{Dios}{God}\morglo{tayta}{father}\morglo{ima-paq}{what-\lsc{purp}}\morglo{kima-wa-nchik-man}{burn-\lsc{1.obj}-\lsc{1pl}-\lsc{cond}}\morglo{achachalláw}{how.awful}}%morpheme+gloss
\glotran{Good God! Why would they burn [cremate] us? \pb{How awful}!}{}%eng+spa trans
{}{}%rec - time

% 7
\gloexe{Glo5:Achachaw}{}{amv}%
{¡\pb{Achacháw}! Apuríman lapcharun kichkata.}%amv que first line
{\morglo{achacháw}{ouch}\morglo{Apurí-man}{Apurí-\lsc{all}}\morglo{lapcha-ru-n}{grab-\lsc{urgt}-\lsc{3}}\morglo{kichka-ta}{thorn-\lsc{acc}}}%morpheme+gloss
\glotran{\pb{Ouch}! She grabbed onto a thorn bush [going to] Apurí.}{}%eng+spa trans
{}{}%rec - time

\section{Assenters and greetings}
The list of assenters\index[sub]{assenters} includes three members: \phono{arí}, \phono{aw}, and \phono{alal}, exemplified in ~(\ref{Glo5:Pukapis}) and ~(\ref{Glo5:lavashuntriki}).\\

% 1
\gloexe{Glo5:Pukapis}{}{amv}%
{Pukapis kasa vakahina. \pb{Arí}, wak sintakusa kayan.}%amv que first line
{\morglo{puka-pis}{red-\lsc{add}}\morglo{ka-sa}{be-\lsc{npst}}\morglo{vaka-hina}{cow-\lsc{comp}}\morglo{arí}{yes}\morglo{wak}{\lsc{dem.d}}\morglo{sinta-ku-sa}{ribbon-\lsc{refl}-\lsc{prf}}\morglo{ka-ya-n}{be-\lsc{prog}-\lsc{3}}}%morpheme+gloss
\glotran{\spkr~1: “The colored one was like a cow.” \spkr~2: “\pb{Yes}, it has [its ears pierced with] ribbons.”}{}%eng+spa trans
{}{}%rec - time

% 2
\gloexe{Glo5:lavashuntriki}{}{amv}%
{\pb{Aw}, lavashuntriki, kaypis qatra qatra kayan.}%amv que first line
{\morglo{aw}{yes}\morglo{lava-shun-tri-ki}{wash-\lsc{1pl.fut}-\lsc{evc}-\lsc{iki}}\morglo{kay-pis}{\lsc{dem.p}-\lsc{add}}\morglo{qatra}{dirty}\morglo{qatra}{dirty}\morglo{ka-ya-n}{be-\lsc{prog}-\lsc{3}}}%morpheme+gloss
\glotran{\pb{Yes}, we’ll wash it. It’s really dirty.}{}%eng+spa trans
{}{}%rec - time

\noindent
The first and second are used in all dialects, while the the third is used only in \CH. \phono{arí} often carries the emphatic enclitic \phono{-yá}~(\ref{Glo5:Kutimushaq}).\\

% 3
\gloexe{Glo5:Kutimushaq}{}{amv}%
{“Kutimushaq,” nishpash chay pindihuqa manam warminman trayachinchu. ¡\pb{Ariyá} warmiyuq!}%amv que first line
{\morglo{kuti-mu-shaq}{return-\lsc{cisl}-\lsc{1.fut}}\morglo{ni-shpa-sh}{say-\lsc{subis}-\lsc{evr}}\morglo{chay}{\lsc{dem.d}}\morglo{pindihu-qa}{bastard-\lsc{top}}\morglo{mana-m}{no-\lsc{evd}}\morglo{warmi-n-man}{woman-\lsc{3}-\lsc{all}}\morglo{traya-chi-n-chu}{arrive-\lsc{caus}-\lsc{3}-\lsc{neg}}\morglo{ari-yá}{yes-\lsc{emph}}\morglo{warmi-yuq}{woman-\lsc{poss}}}%morpheme+gloss
\glotran{Although the bastard [had] said, “I’m going to return,” he never made it back to his wife. \pb{Yes}! He had a wife!}{}%eng+spa trans
{}{}%rec - time

\noindent
\phono{aw} is used to check for agreement from interlocutors and to form tag questions~(\ref{Glo5:Chay}),~(\ref{Glo5:Yapamik}).\\

% 4
\gloexe{Glo5:Chay}{}{amv}%
{Chay chaqla kinraytatr pasarurqa, ¿\pb{aw}?}%amv que first line
{\morglo{chay}{\lsc{dem.d}}\morglo{chaqla}{stone.outcropping}\morglo{kinray-ta-tr}{across-\lsc{acc}-\lsc{evc}}\morglo{pasa-ru-rqa}{pass-\lsc{urgt}-\lsc{pst}}\morglo{aw}{yes}}%morpheme+gloss
\glotran{He must have come by around that stone outcropping, \pb{no}?}{}%eng+spa trans
{}{}%rec - time

% 5
\gloexe{Glo5:Yapamik}{}{amv}%
{Yapamik kutinqa, ¿\pb{aw}?}%amv que first line
{\morglo{yapa-mi-k}{again-\lsc{evd}-\lsc{ik}}\morglo{kuti-nqa}{return-\lsc{3.fut}}\morglo{aw}{yes}}%morpheme+gloss
\glotran{She’s going to come back, \pb{isn’t she}?}{}%eng+spa trans
{}{}%rec - time

\noindent
Speakers of \SYQ{} make extensive use of the borrowed Spanish greetings\index[sub]{greetings}, \phono{buynus diyas} ‘good day’, \phono{buynas tardis} ‘good afternoon’ and \phono{buynas nuchis} ‘good evening’, ‘good night’~(\ref{Glo5:ganawniki}). \phono{¡Rimallasayki!} ‘I greet you!’ is the most common of the greetings indigenous to \SYQ. \phono{¡Saludallasayki!} is also used.\\

% 6
\gloexe{Glo5:ganawniki}{}{amv}%
{Mana ganawniki kanchu ni “\pb{Buynus diyas}” ni “\pb{Buynus diyas}, primacha”, nada nishunkichu.}%amv que first line
{\morglo{mana}{no}\morglo{ganaw-ni-ki}{cattle-\lsc{euph}-\lsc{2}}\morglo{ka-n-chu}{be-\lsc{3}-\lsc{neg}}\morglo{ni}{nor}\morglo{buynus}{good}\morglo{diyas}{day}\morglo{ni}{nor}\morglo{buynus}{good}\morglo{diyas}{day}\morglo{prima-cha}{cousin-\lsc{dim}}\morglo{nada}{nothing}\morglo{ni-shunki-chu}{say-\lsc{2.obj}-\lsc{2}-\lsc{neg}}}%morpheme+gloss
\glotran{When you don’t have cattle, they don’t even say “\pb{Good morning},” “\pb{Good morning}, cousin,” to you -- nothing.}{}%eng+spa trans
{}{}%rec - time

\section{Prepositions}
\SYQ{} makes use of some prepositions\index[sub]{prepositions} borrowed from Spanish. The preposition most frequently employed is \phono{asta} (‘up to’, ‘until’, ‘even’, \Sp~‘\spanish{hasta}’ ‘up to’, ‘until’)~(\ref{Glo5:wanukunay}). \phono{asta} is usually employed redundantly, in combination with the indigenous case suffix \phono{-kama}, apparently with the same semantics (\phono{asta aka-kama} ‘until here’).\\

% 1
\gloexe{Glo5:wanukunay}{}{lt}%
{\pb{Asta} wañukunay puntraw\pb{kama}triki chayna purishaq.}%lt que first line
{\morglo{asta}{until}\morglo{wañu-ku-na-y}{die-\lsc{refl}-\lsc{nmlz}-\lsc{1}}\morglo{puntraw-kama-tri-ki}{day-\lsc{lim}-\lsc{evc}-\lsc{iki}}\morglo{chayna}{thus}\morglo{puri-shaq}{walk-\lsc{1.fut}}}%morpheme+gloss
\glotran{\pb{Until} the day I die, I’m going to walk around like that.}{}%eng+spa trans
{}{}%rec - time

% 2
\gloexe{Glo5:Tinkuyani}{}{amv}%
{Tinkuyani ubihaywan ñuqa \pb{disdi} uchuychallay\pb{paq} kani.}%amv que first line
{\morglo{tinku-ya-ni}{find-\lsc{prog}-\lsc{1}}\morglo{ubiha-y-wan}{sheep-\lsc{1}-\lsc{instr}}\morglo{ñuqa}{I}\morglo{disdi}{since}\morglo{uchuy-cha-lla-y-paq}{small-\lsc{dim}-\lsc{rstr}-\lsc{1}-\lsc{abl}}\morglo{kani}{be-\lsc{1}}}%morpheme+gloss
\glotran{I’ve found myself with my sheep \pb{since} I was very small.}{}%eng+spa trans
{}{}%rec - time

\section{Adverbs}\label{sec:adverbs}
The class of adverbs\index[sub]{adverbs} native to \SYQ{} is rather small~(\ref{Glo5:Chafliwan}--\ref{Glo5:qaninpa}).\\

% 1
\gloexe{Glo5:Chafliwan}{}{amv}%
{Chafliwan pikarun, \pb{yapa} hapin, \pb{yapa} pikarun, \pb{yapa} hapin, \pb{yapa} pikarun.}%amv que first line
{\morglo{chafli-wan}{pick-\lsc{instr}}\morglo{pika-ru-n}{pick-\lsc{urgt}-\lsc{3}}\morglo{yapa}{again}\morglo{hapi-n}{grab-\lsc{3}}\morglo{yapa}{again}\morglo{pika-ru-n}{pick-\lsc{urgt}-\lsc{3}}\morglo{yapa}{again}\morglo{hapi-n}{grab-\lsc{3}}\morglo{yapa}{again}\morglo{pika-ru-n}{pick-\lsc{urgt}-\lsc{3}}}%morpheme+gloss
\glotran{He struck with a pick. \pb{Again}, [the zombie] grabs him. \pb{Again} he struck with the pick. \pb{Again} he grabs. \pb{Again} he struck.}{}%eng+spa trans
{}{}%rec - time

% 2
\gloexe{Glo5:Yaqa}{}{amv}%
{\pb{Yaqa} wañurqani chayshi tiyay.}%amv que first line
{\morglo{yaqa}{almost}\morglo{wañu-rqa-ni}{die-\lsc{pst}-\lsc{1}}\morglo{chay-shi}{\lsc{dem.d}-\lsc{evr}}\morglo{tiya-y}{aunt-1}}%morpheme+gloss
\glotran{I \pb{almost} died, then, [says] my aunt.}{}%eng+spa trans
{}{}%rec - time

% 3
\gloexe{Glo5:qaninpa}{}{lt}%
{Hinallatañam \pb{qaninpa} apakaramun wak yantata.}%lt que first line
{\morglo{hina-lla-ta-ña-m}{thus-\lsc{rstr}-\lsc{acc}-\lsc{disc}-\lsc{evd}}\morglo{qaninpa}{before}\morglo{apa-ka-ra-mu-n}{bring-\lsc{passacc}-\lsc{urgt}-\lsc{cisl}-\lsc{3}}\morglo{wak}{\lsc{dem.d}}\morglo{yanta-ta}{firewood-\lsc{acc}}}%morpheme+gloss
\glotran{Just like before already, they brought that firewood.}{}%eng+spa trans
{}{}%rec - time

\noindent
Verbal modification in \SYQ, as in other Quechuan languages, is accomplished primarily by derivatives and enclitics (\phono{-pa} ‘repeatedly’, \phono{-ña} ‘already’). \SYQ{} makes heavy use of the adoped/adapted Spanish adverbs \phono{apuraw} ‘quick’, \phono{pasaypaq} ‘completely,’ \phono{siympri} ‘always’ and \phono{ayvis} ‘sometimes’~(\ref{Glo5:apuraw}--\ref{Glo5:Ayvis}).\\

% 4
\gloexe{Glo5:apuraw}{}{ach}%
{Mana \pb{apuraw} hurquptinqa chayqa wañuchin.}%ach que first line
{\morglo{mana}{no}\morglo{apuraw}{quick}\morglo{hurqu-pti-n-qa}{remove-\lsc{subds}-\lsc{3}-\lsc{top}}\morglo{chay-qa}{\lsc{dem.d}-\lsc{top}}\morglo{wañu-chi-n}{die-\lsc{caus}-\lsc{3}}}%morpheme+gloss
\glotran{If [the placenta] is not taken out \pb{quickly}, it kills.}{}%eng+spa trans
{}{}%rec - time

% 5
\gloexe{Glo5:pasa}{}{lt}%
{Uchuypis \pb{pasa-pasaypaq}mi chakirun, uchuypis chakisham kayan.}%lt que first line
{\morglo{uchu-y-pis}{chile-\lsc{1}-\lsc{add}}\morglo{pasa-pasaypaq-mi}{comp-completely-\lsc{evd}}\morglo{chaki-ru-n}{dry-\lsc{urgt}-\lsc{3}}\morglo{uchu-y-pis}{chile-\lsc{1}-\lsc{add}}\morglo{chaki-sha-m}{dry-\lsc{prf}-\lsc{evd}}\morglo{ka-ya-n}{be-\lsc{prog}-\lsc{3}}}%morpheme+gloss
\glotran{My chiles, too, \pb{completely} dried out. My chiles, too, are dried out.}{}%eng+spa trans
{}{}%rec - time

% 6
\gloexe{Glo5:siympri}{}{amv}%
{Waqayaniyá \pb{siympri} yuyariyaniyá.}%amv que first line
{\morglo{waqa-ya-ni-yá}{cry-\lsc{prog}-\lsc{1}-\lsc{emph}}\morglo{siympri}{always}\morglo{yuya-ri-ya-ni-yá}{remember-\lsc{incep}-\lsc{prog}-\lsc{1}-\lsc{emph}}}%morpheme+gloss
\glotran{I’m crying. I’m \pb{always} remembering.}{}%eng+spa trans
{}{}%rec - time

% 7
\gloexe{Glo5:Ayvis}{}{ach}%
{\pb{Ayvis} lliw chinkarun \pb{ayvis} huklla ishkayllata tariru:.}%ach que first line
{\morglo{ayvis}{sometimes}\morglo{lliw}{all}\morglo{chinka-ru-n}{lose-\lsc{urgt}-\lsc{3}}\morglo{ayvis}{sometimes}\morglo{huk-lla}{one-\lsc{rstr}}\morglo{ishkay-lla-ta}{two-\lsc{rstr}-\lsc{acc}}\morglo{tari-ru-:}{find-\lsc{urgt}-\lsc{1}}}%morpheme+gloss
\glotran{\pb{Sometimes} all get lost; \pb{sometimes} I find just one or two.}{}%eng+spa trans
{}{}%rec - time

\noindent
Additionally, adverbs can sometimes be derived from adjectives with the suffixation of \phono{-lla}~(\ref{Glo5:pitapis}),~(\ref{Glo5:Kayta}); and adjectives may sometimes occur adverbally, in which case they are usually inflected with \phono{-ta}, as in ~(\ref{Glo5:Kanan}--\ref{Glo5:Tushuptiypis}).\\

% 8
\gloexe{Glo5:pitapis}{}{ach}%
{Ni pitapis kritika:chu dañukuruptinpis \pb{sumaqlla}m nikulla:.}%ach que first line
{\morglo{ni}{nor}\morglo{pi-ta-pis}{who-\lsc{acc}-\lsc{add}}\morglo{kritika-:-chu}{criticize-\lsc{1}-\lsc{neg}}\morglo{dañu-ku-ru-pti-n-pis}{damage-\lsc{refl}-\lsc{urgt}-\lsc{subds}-\lsc{3}-\lsc{add}}\morglo{sumaq-lla-m}{pretty-\lsc{rest}-\lsc{evd}}\morglo{ni-ku-lla-:}{say-\lsc{refl}-\lsc{rstr}-\lsc{1}}}%morpheme+gloss
\glotran{I don’t criticize anyone. When they do harm, I talk to them \pb{nicely}.}{}%eng+spa trans
{}{}%rec - time

% 9
\gloexe{Glo5:Kayta}{}{amv}%
{¡Kayta pasarachiy! Kargarayanñamiki. ¡\pb{Sumaqlla} winaruy!}%amv que first line
{\morglo{kay-ta}{\lsc{dem.p}}\morglo{pasa-ra-chi-y}{pass-\lsc{passacc}-\lsc{caus}-\lsc{imp}}\morglo{karga-ra-ya-n-ña-mi-ki}{carry-\lsc{unint}-\lsc{intens}-\lsc{3}-\lsc{disc}-\lsc{3}-\lsc{evd}-\lsc{iki}}\morglo{sumaq-lla}{pretty-\lsc{rstr}}\morglo{wina-ru-y}{add.in-\lsc{urgt}-\lsc{imp}}}%morpheme+gloss
\glotran{Have him come here! It’s being carried already. Add it in \pb{nicely}!}{}%eng+spa trans
{}{}%rec - time

% 10
\gloexe{Glo5:Kanan}{}{amv}%
{Kanan tutaqa suyñukuruni \pb{fiyuta}m. ¿Ima pasaruwanqa?}%amv que first line
{\morglo{kanan}{now}\morglo{tuta-qa}{night-\lsc{top}}\morglo{suyñu-ku-ru-ni}{dream-\lsc{refl}-\lsc{urgt}-\lsc{1}}\morglo{fiyu-ta-m}{ugly-\lsc{acc}-\lsc{evd}}\morglo{ima}{what}\morglo{pasa-ru-wa-nqa}{pass-\lsc{urgt}-\lsc{1.obj}-\lsc{3.fut}}}%morpheme+gloss
\glotran{Last night I dreamed \pb{horribly}. What’s going to happen to me?}{}%eng+spa trans
{}{}%rec - time

% 11
\gloexe{Glo5:Manachu}{}{amv}%
{¿Manachu chay Aliciawan risachiwaq? Aliciam \pb{sumaq sumaqta} risan.}%amv que first line
{\morglo{mana-chu}{no-\lsc{q}}\morglo{chay}{\lsc{dem.d}}\morglo{Alicia-wan}{Alicia-\lsc{instr}}\morglo{risa-chi-waq}{pray-\lsc{caus}-\lsc{2.cond}}\morglo{Alicia-m}{Alicia-\lsc{evd}}\morglo{sumaq}{pretty}\morglo{sumaq-ta}{pretty-\lsc{acc}}\morglo{risa-n}{pray-\lsc{3}}}%morpheme+gloss
\glotran{Can’t you have Alicia pray for her? Alicia prays \pb{really nicely}.}{}%eng+spa trans
{}{}%rec - time

% 12
\gloexe{Glo5:Tushuptiypis}{}{amv}%
{Tushuptiypis \pb{alli-allita} pigakuq.}%amv que first line
{\morglo{tushu-pti-y-pis}{dance-\lsc{subds}-\lsc{1}-\lsc{add}}\morglo{alli-alli-ta}{good-good-\lsc{acc}}\morglo{piga-ku-q}{stick-\lsc{refl}-\lsc{ag}}}%morpheme+gloss
\glotran{When I would dance, he would stick himself [to me] \pb{really well}.}{}%eng+spa trans
{}{}%rec - time

\noindent
Some nouns referring to time may occur adverbally without inflection, as in ~(\ref{Glo5:Kanallan})and ~(\ref{Glo5:Rinrilla}), others are inflected with \phono{-ta}, as  (see~§~\ref{ssec:timenouns})~(\ref{Glo5:Chaymi}) shows.\\

% 13
\gloexe{Glo5:Kanallan}{}{amv}%
{“¡\pb{Kanallan} intrigaway!” nishpash chay kundur trayarun.}%amv que first line
{\morglo{kanallan}{right.now}\morglo{intriga-wa-y}{deliver-\lsc{1.obj}-\lsc{imp}}\morglo{ni-shpa-sh}{say-\lsc{subis}-\lsc{evr}}\morglo{chay}{\lsc{dem.d}}\morglo{kundur}{condor}\morglo{traya-ru-n}{arrive-\lsc{urgt}-\lsc{3}}}%morpheme+gloss
\glotran{“Hand her over to me \pb{right now}!” said the condor [when] he arrived.}{}%eng+spa trans
{}{}%rec - time

% 14
\gloexe{Glo5:Rinrilla}{}{ach}%
{Rinrilla:pis uparura \pb{qayna wata}qa.}%ach que first line
{\morglo{rinri-lla-:-pis}{ear-\lsc{rstr}-\lsc{1}-\lsc{add}}\morglo{upa-ru-ra}{deaf	-\lsc{urgt}-\lsc{pst}}\morglo{qayna}{previous}\morglo{wata-qa}{year-\lsc{top}}}%morpheme+gloss
\glotran{My ears went deaf \pb{last year}.}{}%eng+spa trans
{}{}%rec - time

% 15
\gloexe{Glo5:Chaymi}{}{ch}%
{Chaymi shamula: \pb{qaspalpuqta}. Chaymi karkarya qipa:ta shamusha.}%ch que first line
{\morglo{chay-mi}{\lsc{dem.d}-\lsc{evd}}\morglo{shamu-la-:}{come-\lsc{pst}-\lsc{1}}\morglo{qaspalpuq-ta}{nightfall-\lsc{acc}}\morglo{chay-mi}{\lsc{dem.d}-\lsc{evd}}\morglo{karkarya}{zombie}\morglo{qipa-:-ta}{behind-\lsc{1}-\lsc{acc}}\morglo{shamu-sha}{come-\lsc{npst}}}%morpheme+gloss
\glotran{Then I came \pb{at nightfall}. Then a zombie came behind me.}{}%eng+spa trans
{}{}%rec - time

\section{Particles covered elsewhere}
Coordinators are discussed in §~\ref{sec:coord} on coordination, negators in §~\ref{sec:negation} on negation, and prenumerals in Sub §~\ref{ssec:timenum} on time numerals and prenumerals.
