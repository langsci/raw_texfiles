\addchap{Acknowledgements} 

The person I would like to express my deepest gratitude to is my supervisor Martine Grice. She was my most important advisor through my academic life and she constantly advanced my development with her unique and charming way of giving me feedback. From her, I learned how to ask the right questions and how to communicate these questions in a precise and diplomatic way. Martine always provided the right amount of support and pressure I needed, while leaving me great freedom in finding my way through the jungle of academia. She always had an ear for any kind of problem, be it overly dramatic doubts about the scientific method itself or be it being lost in the process of writing. She always believed in me and my development as a scientist and cheered me up on cloudy days. I truly believe she did not only make me a good linguist but will always be a personal role model. Thank you, Martine.

I would like to express my gratitude to Rachid Ridouane. He constantly advanced my understanding of Tashlhiyt and generously introduced me to his home in Morocco, to his culture, and to his heritage. He led me through the depths of the Agadir Souk, taught me the art of pouring Moroccan tea, and warmly welcomed me in the midst of his family and friends. I am very grateful to these memories. Thank you, Rachid.

I also thank all members of the faculty of Amazigh studies at the Ibn Zohr University in Agadir. Without the help of the faculty, none of the data could have been collected. In this context, one man deserves particular credit: Abderrahmane Charki helped us find consultants, organise our recordings, and conduct our experiments. Even though he constantly ate all the chocolate that was supposed to be for our consultants, he did a terrific job and made my life in the field much easier. I am very happy to have met him. Generally, I am grateful to all consultants that patiently participated in my tiresome experiments. It was a beautiful experience full of positivity and support. Tenemmirt, Agadir.

Bodo Winter has played a major role in my scientific and personal development. He was always two steps ahead of me and that way sparked my interest with his enthusiastic and positive way of looking at things. Over the many years, he has become my teacher and a dear friend.

The first person to incubate my early interest in language was Kay González-Vilbazo. He taught the introduction to linguistics at the department of German language and literature in Cologne. He told us the most interesting narratives about the mesmerising nature of Universal Grammar, genetically endowed language acquisition devices, and sign language development in Nicaragua. I was instantly hooked due to his convincing and inspiring way. While I have started questioning his view onto language, cognition, and evolution shortly after, I am eternally grateful to have had him as one of my first university teachers. He sparked the flame and made me want to become a linguist. 

Jan Menge has been an inspiring character in my development as a linguist. Unfortunately, Jan has long left the field of linguistics. I don’t think he is aware of the major impact he has had on my professional career. During my days as an undergraduate, he used to always put my know-it-all allures into perspective, challenged me, and showed me alternative views. I am very grateful to him.

I would like to thank Frank Domahs to put his faith in an overly-motivated undergraduate. I met Frank in 2007 during a research internship at the neurological department of Aachen. He believed in my abilities to conduct my first full-fledged psycholinguistics experiment. Despite me being terrified of writing in English, he persuaded me to write up the results. The write-up turned out to be my first journal publication and gave me important insights into the scientific machinery.

Alongside these amazing scholars, I had many great teachers that I encountered during my academic journey. Many thanks go to Beatrice Primus, who saw potential in a presumptuous undergraduate and hired me as a student assistant. I would like to thank Walter Huber and Richard Wiese to give me the opportunity to have a first peak into neurolinguistic methods in Aachen and in Marburg. I am grateful to Nikolaus Himmelmann who fortunately was my teacher for a couple of years. He taught me a lot about the linguistic craftsmanship and scientific integrity. Many thanks go to James Kirby who was my most important advisor during my year in Edinburgh. Patiently, he spent hours on discussing data with me and always gave me great advice. He continued to be a colleague and has become a friend. Besides the exchange with James, I cannot value my student experience in Edinburgh high enough. The School of Philosophy, Psychology and Language Science is a beautiful and inspiring place and hosts remarkable teachers. I did not only learn a lot about a multitude of different topics, but learned how communicating knowledge is done right. Thank you Alice Turk, Bob Ladd, Simon Kirby, Graeme Trousdale, Antonella Sorace, and James Hurford.

\newpage
Many other people in the Institute of Phonetics in Cologne have helped to make my work not only easier but, most notably, enjoyable. The exchanges with past and present colleagues and students over the now seven years were always inspiring, exciting and, most of all, fun. I have learned important lessons from many of you and I am grateful for your support: Anna Bruggeman, Anne Hermes, Aviad Albert, Bastian Auris, Christian Weitz, Christine Riek, Christine Röhr, Doris Mücke, Francesco Cangemi, Henrik Niemann, Janina Kalbertodt, Jessica Di Napoli, Martina Krüger, Mathias Stoeber, Simon Ritter, Simon Wehrle, Stefan Baumann, and Theo Klinker.

I would like to thank the a.r.t.e.s. graduate school of humanities in Cologne to award me with a scholarship that made this endeavour financially possible.\footnote{The work presented in this book is based on my doctoral dissertation which was  accepted by the Faculty of Arts and Humanities of the University of Cologne in 2016.} I am particularly thankful to Andreas Speer, the hard working motor behind a.r.t.e.s., who has enabled dozens of academic careers and will hopefully continue to do so.

Besides these people that walked next to me in my academic life, gratitude is long overdue to the people that were there for me in my personal life and without whom none of my professional achievements would have been possible. I would like to thank my mother, Andrea, and her husband Andreas as well as my father, Frank, and his partner Anna to keep believing in me. Even though unfamiliar with the world I entered, they always trusted my judgments and rewarded me with their support. 

I am lucky enough to have great friends, but I would like to thank two amazing men in particular. Dennis Köhler always challenged my view on this world and has had a huge impact on my desire to learn new things and look beyond my limited experiences. He was always an authority to me and a valuable advisor during my personal development. 

Christopher Arnold - nothing less of an authority in my life - showed me that faith in yourself and your goals can indeed move mountains. Despite being struck by fate again and again, despite forces that tried to break him down, he always remained unbroken. He is a constant inspiration for me. In times of discouragements, it was always his voice that had told me to get up and move on. 

\begin{flushright}
\textit{Cologne, May 2017}
\end{flushright}