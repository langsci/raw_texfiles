\chapter{Concluding remarks and future directions}
This book presented a first investigation into stress and intonation in Tashlhiyt Berber. The analysis contributes to the small but steadily growing body of available phonetic descriptions and phonological analyses of Tashlhiyt in particular and Berber in general. In addition to its descriptive contributions, the analysis of tonal events contributes to phonological typology by supplementing our knowledge of intonation systems with data from a Berber language. Moreover, Tashlhiyt proved to be a case study of particular relevance for intonational theory. The linguistic system of Tashlhiyt exhibits two structural properties that make it especially interesting in view of cross-linguistic prosody research: Tashlhiyt does not have word stress and it allows for phonotactic patterns that are adverse to the production and perception of pitch.

\section{Tonal placement without word stress}
In well-investigated word stress languages, such as West Germanic languages, tonal events often align with lexically stressed syllables. Tashlhiyt lacks word stress, which raises the question as to how intonational events are aligned with the segmental string. 

Most of the (few) languages that are postulated not to have word stress have been described in terms of predetermined tonal strings associating sequentially within a domain like the accentual phrase or phonological phrase (\citealt{Jun2005}, for Korean\il{Korean}; \citealt{JunFougeron2002}, for French\il{French}; \citealt{Karlsson2014}, for Mongolian\il{Mongolian}). To my knowledge, the only language which has been instrumentally studied with regard to tonal alignment and which exhibits comparable patterns to Tashlhiyt is Ambonese Malay (\citealt{MaskGussenhoven2016}). Maskikit-Essed and Gussenhoven reported on a sparse distribution of tonal events in Ambonese Malay. They investigated two tunes characterised by a rise-fall and a rise, respectively, both occurring at the end of a intonation phrase. The alignment of the high target of the rise-fall was prone to a large amount of variability. This variability appeared to be of gradual nature, i.e. tonal events were variably aligned with the phrase-final word. These tonal events were interpreted as edge tones that are approximately timed to occur with the phrase-final word, rather than associated to a specific syllable. \is{word stress}\is{tonal alignment}\il{Malay (Ambonese)} 

This variability resembles the tonal placement patterns in Tashlhiyt. Both systems exhibit a sparse distribution of tones across the utterance and have arguably no word stress. This raises the question as to whether the absence of a lexical determined metrical anchor is correlated with a certain flexibility in tonal placement. Answers to this question remain speculative until more instrumental studies are available. We hope that the used methods and analyses will spark further interest to investigate stress and intonation in other Berber languages. This would be particularly interesting with regard to the interaction of word stress and tonal placement. Other Berber languages such as Zwara Berber\il{Berber (Zwara)}, spoken in Libya, have been reported to exhibit word stress. Whether tonal placement exhibits less variability in these varieties is an interesting question. Preliminary results suggest that this is indeed the case (\citealt{Gussenhoven.accepted}). Therefore, descriptive and comparative explorations of stress and intonation in other Berber languages could be a very informative departure point from which possible typological generalisations can be made. 

As opposed to Ambonese Malay\il{Malay (Ambonese)}, tonal placement patterns in Tashlhiyt, as presented here (also in \citealt{Grice.etal2015tash}), exhibit variability that was referred to as discrete variability. Tonal events systematically align with one of two mutually exclusive positions that can be defined in discrete ways, rather than continuously varying between these positions. Thus, the observed variability exhibits a systematic structure. Most interestingly, large amounts of this structured variability can be related to Tashlhiyt’s phonotactic patterns. 

\section{Phonotactic restrictions on tonal placement}
In Tashlhiyt, many words can be comprised of consonants only. Consonants can be considered poorer carriers of pitch than vowels and even within consonants, some segment types are better suited to convey pitch information than others. In extreme cases, the phonetic opportunity afforded for the execution of pitch movements is exceptionally limited. In spite of these restrictions, communicatively relevant tones in Tashlhiyt show a preference to be located in certain prosodic positions. This creates a functional conflict between privileged prosodic locations and their inherent qualities to carry pitch information. This conflict can be resolved in many different ways. Recall that if the tone bearing word contains more than one sonorant, the tonal events described in this book associate either to the penult or to the final syllable. If the word does not contain a sonorant, the tones associate either to the vowel of the preceding word or to a schwa within the word. Alternatively, the tones may remain unrealised. With which syllables the tones associate is not only dependent on sentence modality but also on at least three competing factors orthogonal to that: (1) the tones are generally attracted to the right edge, i.e. the tones are more often located on the final syllable than on the penult and they are more often located on schwa than on the preceding word. (2) The tones are attracted to more sonorous elements, i.e. if there are two available sonorants within the word, the tones are preferably located on the more sonorous one. If there is only one sonorant in the word, the tones are almost exclusively located on it, and if there is no sonorant in the word, the tones may be realised on a sonorant of the preceding word. (3) The tone is attracted to heavy syllables, i.e. the tones are more often located on heavy syllables than on light syllables.\is{tone bearing unit}\is{syllable weight} 

These tendencies can all be conceived of as optimising the phonetic prerequisites for the manifestation of pitch. A pitch target or contour has a more salient percept on segments that are higher in intensity and have a richer harmonic structure (\citealt{Zhang2004,Barnes.etal2014}). In this sense, tonal events in Tashlhiyt are attracted to positions that are well suited to pitch perception.

With regard to (1), phrase-final elements are often spatio-temporally expanded (among other things, they are longer and louder, see Chapter 2 and 6), not only offering more time for the articulation of a complex pitch movement but also enabling a more salient pitch percept. With regard to (2), a pitch movement on a vowel is easier to perceive than one on a consonant. Even if the consonant is voiced – clearly a prerequisite for pitch – a pitch target is easier to perceive on nasals and liquids than on fricatives and stops. In terms of pitch perception and production, voiced obstruents are weak carriers of the signal. When airflow is inhibited during an articulatory constriction, the leeway for vocal fold modulation is limited. Moreover, voiced obstruents are acoustically manifested by low intensity and a restricted harmonic structure, diminishing the saliency of a pitch percept. Voiceless obstruents are at the extreme end of the scale being the the type of sounds that are least suited for carrying information cued by fundamental frequency. Finally, with regard to (3), a more complex rhyme may also be beneficial for mechanisms of pitch perception. A longer rhyme, i.e. more elements of high intensity and rich harmonic structure, will enable a more salient percept of pitch. This, however, is dependent on the identity of the coda consonant, since not all consonants are well suited to carry pitch. 

Interestingly, these individual factors not only correlate with pitch perceptibility, they also resonate with cross-linguistic asymmetries.\footnote{It is not implied that there is a causal relationship between the functional motivation of a speech pattern and observed cross-linguistic asymmetries, even though the author considers this hypothesis highly plausible (e.g. \citealt{Ohala1993,Blevins2004}, among many others).} For example, many tone languages restrict contour tones to syllables with rhymes that contain more sonorous elements (\citealt{Zhang2001,Gordon2004}). Tashlhiyt is a rare case with regard to sonority determining structures at higher levels of prosodic structure. However, there are some other reports of tone bearing units of intonational tones being defined by sonority. For example, Japanese allows secondary association of edge tones to sonorant moras but not to non-sonorant moras (\citealt{PierrBeck1988}). Moreover, there are many languages with weight-sensitive word stress or tone systems (cf. \citealt{DeLacy2002,DeLacy2007,Gordon2004,Gordon2006}, for overviews), in which syllables with relatively more complex rhymes exhibit more tonal contrasts and are more likely to be the metrically strong syllable of a constituent.\is{syllable weight}\is{secondary association}\il{Japanese}  

In languages discussed in the above-cited literature, the co-occurrence of a tone with a certain structure appears to be phonologised. In such cases, the quality of a unit for expressing pitch contrasts categorically determines its actual co-occurrence with phonological tones (e.g. a contour tone co-occurs with a certain syllable structure or not). Tashlhiyt, however, does not exhibit such a consistent mapping, but rather allows for several competing possibilities of how the segmental structure affects the realisation of tone.

The most frequently observed option is a temporal shift of the tonal event, i.e. tones move towards segmental material that is better suited to carrying pitch information. This is true for both words with sonorants and words containing no sonorants. In cases with multiple available sonorants, the tone can shift to a position that is segmentally most suitable for the realisation and perception of the pitch movement. In cases with no available sonorants, the tone can even shift to a position that is outside of the tone-bearing word. Similar temporal shifts of tonal events to less restricted material have been observed for Neapolitan Italian  (\citealt{CangemiGrice2016}). As opposed to Neapolitan Italian, Tashlhiyt does not exhibit gradual shifts but discrete shifts that can be analysed as changes in the association of the tones.\il{Italian (Neapolitan)}

In addition to temporal shifts, the tonal movement can be undershot. This resembles phenomena referred to as ‘truncation’, i.e. cases in which a tonal target is not fully reached. In words with available sonorants, truncation mainly applies to the final low tone following the high target. The high target is almost always realised in these cases. Similar observations have been made for other languages such as Catalan, English, German, Russian, Spanish, and Swedish (\citealt{EriksonAlstermark1972,Grabe1998,Grabe.etal2000,PrietoOrtega2009,Rathcke2009}).\il{Catalan}\il{English}\il{German}\il{Russian}\il{Spanish}\il{Swedish} Alternatively, Tashlhiyt allows the entire tonal complex to be absent altogether. To our knowledge, the latter case has never been reported on in the literature. From a functional perspective, this is arguably the least desirable pattern. The absence of the tone leaves the listener with no local cue to disambiguate an intonational contrast. All these patterns share a common relationship between segments and tones: the segmental material determines the realisation of tones, or in other words, ‘the text drives the tune’.\is{tonal shift} \is{truncation} 

Alternatively, the text can be adjusted to enable the realisation of the tune. Tashlhiyt allows vowel insertion to bear functionally relevant tonal movements. This is comparable to Italian word-final schwa in loan words, which has been shown to be dependent on the complexity of the tune that needs to be realised (\citealt{Grice.etal2015bari}). Similar observations have been made for Moroccan Arabic (\citealt{DE1985}) and Tarifiyt Berber (\citealt{DellTangi1992}), two languages also spoken in Morocco. The mirror image of these phenomena has been reported for Greek and other languages. In Greek, vowels are often deleted in certain prosodic positions. However, this deletion is less likely in prosodically privileged positions, namely those locations that bear complex intonational pitch movements (\citealt{Kaimaki2015}). Similarly, \citet[184]{Heath1987} described a process in which schwa deletion in word-final syllables is blocked by “list intonation” for Moroccan Colloquial Arabic. \citet[51]{Vance1987} claims that in Japanese high vowel devoicing is blocked “when a final syllable in the devoicing environment must carry a rising intonation.” These patterns constitute cases in which the requirement of the tonal realisation causes the segmental structure to adjust or in other words cases in which ‘the tune drives the text’ (\citealt{Grice.etal2015bari}).\il{Arabic (Moroccan)}\il{Berber (Tarifiyt)}\il{Greek}\il{Japanese}

All of the discussed patterns can be conceived of as solutions to a functional dilemma: the requirement to realise meaningful pitch movements in privileged prosodic positions and the extent to which segments lend themselves for a clear manifestation of these pitch movements. Most well documented languages such as West Germanic languages usually do not face such extreme conflicts of tune-text-association. The patterns observed for Tashlhiyt, thus, add to our understanding of how conflicting structural preferences can be resolved. Tashlhiyt demonstrates, more clearly than any other language before, that the tune and the text interact in a synergetic way. The text drives the tune and the tune drives the text.

\section{Future directions}
Future studies will have to complement the present studies in order to evaluate the proposed analyses and to increase our understanding of prosodic structure and intonation in Tashlhiyt. Fruitful avenues for future research are the investigation of tonal events and phrasing in longer utterances. While the present analysis has discussed the possibility of focus-induced phrasing, the data was restricted to short utterances. Impressionistic observations from semi-spontaneous tasks hint at the possibility of a richer prosodic structure in longer phrases than the patterns described in this book.\footnote{Some of the recordings involving semi-spontaneous tasks have been made available by \cite{cotass}.} The question arises as to whether there is robust evidence for prosodic constituents smaller than the intonation phrase and larger than the phonological word. Analyses of corpora of more natural speech may answer remaining questions. Only after gaining a broader understanding of prosodic structure and after considering the full range of tonal events, a complete analysis of intonation in Tashlhiyt can be proposed. 

Beyond descriptive and typological purposes, the analysis of Tashlhiyt has clearly demonstrated that presently available formalisms for the description of intonation systems have difficulties in modelling certain types of variability. Both the discrete variability of tonal events and the gradient variability of pitch scaling pose a challenge to the Autosegmental-Metrical model. These types of variability appear to be common in intonational systems and are rather difficult to formalise within the available symbolic representations. The problem appears to be rooted in the traditional assumption that speech can be represented as sequences of discrete symbols using discrete mathematics. Even though practical for description and comparison purposes, more and more evidence suggests that discrete mathematics are insufficient to account for numerous observations even within the segmental domain. For example, processes of assimilation and neutralisation are commonly represented via discrete formalisms resulting in categorical predictions. A segment is assimilated or not; it is neutralised or not. However, these alternations might in fact not always be categorical. Evidence suggests that the assimilation of a segment does not involve a discrete alternation but a continuous modulation of gestural overlap with different degrees of overlap dependent on the context (\citealt{BrowmanGoldstein1986}). Similarly, often-cited neutralisation processes have repeatedly been shown to be incomplete (e.g. \citealt{PortODell1985,Roettger.etal2014}).\is{gestural coordination}\is{incomplete neutralisation} 

\newpage 
In addition to the continuous nature of categorically formalised phenomena, certain alternations may be categorical but apply only probabilistically depending on several competing factors (cf. /t/ deletion in Chapter 7). Thus, phenomena traditionally formalised as discrete operations or entities are in fact more adequately represented in terms of statistical distribution expressed by continuous mathematics. This comprises both the probability of an event occurring as well as the phonetic variability the event exhibits when it occurs. Developing formalisms that capture these aspects of speech would enable a more detailed comparison of phonological systems, and in turn, more adequate models of human language. 
