\setcounter{chapter}{1}
\chapter{Hintergründe}
\setcounter{section}{3}
\setcounter{subsection}{1}
Zum einen wird Konversation \citet{Farkas2010} zufolge generell durch zwei Aspekte getrieben: Der erste ist, dass Teilnehmer dem Bedürfnis folgen, den cg zu erweitern. Weil sie danach streben, platzieren sie überhaupt Elemente auf dem Tisch. Der zweite ist, dass Teilnehmer danach streben, einen stabilen Kontextzustand zu erreichen, d.h. einen Zustand, in dem keine Frage offen ist oder (mit den Komponenten sprechend) nichts auf dem Tisch liegt. Aufgrund dieser Bestrebung entfernen die Teilnehmer die Elemente dann so vom Tisch, dass der cg erweitert wird. Dies sind kanonische Verhaltensweisen für Gespräche allgemein. \citet{Farkas2010} nehmen darüber hinaus konkreter an, dass es für einzelne Sprechakte kanonische Reaktionen gibt. Jeder Zug im Diskurs, der ein Element auf dem Tisch platziert, ist deshalb mit einem kanonischen Zug verbunden, um dieses  Element vom Tisch zu entfernen. Der kanonische Weg, ein Thema vom Tisch zu entfernen, ist, einen Diskurszustand zu erreichen, in dem das Thema \textit{entschieden} (\textit{decided}) \is{Entschiedenheit} ist. Eine Proposition p gilt relativ zum cg als entschieden, wenn p oder ¬p Teil des cg ist.  Züge, die Elemente auf dem Tisch platzieren, definieren Zielzustände der Konversation (oder ggf. Mengen solcher Zustände), die erreicht sind, wenn das betroffene Element auf die Art vom Tisch entfernt wird, die den cg erweitert. Diese Idee des kanonischen Zuges, um ein Element vom Tisch zu entfernen, wird mit der \textit{Projektionsmenge} (ps) \is{Projektionsmenge (projected set)} durch eine eigene Komponente aufgefangen. Ein Zug im Diskurs, der ein Element auf dem Tisch platziert, projiziert simultan eine Menge von zukünftigen cgs, relativ zu denen das Thema auf dem Tisch entschieden ist. Diese Mengen im ps sind Obermengen des aktuellen cg. 

\subsection{Illustration des Zusammenspiels der Komponenten: Der assertive Kontextwechsel}
Um die Rolle dieser Komponenten in der Modellierung von Diskurssituationen zu veranschaulichen, werden in diesem Abschnitt die Schritte des Kontextwechsels unter Äußerung einer Assertion \is{Assertion} durchgespielt. In Kapitel~\ref{chapter:hue} werde ich das Mo\-dell erweitern, um Direktive erfassen zu können. In \citet{Farkas2010} werden neben Assertionen auch Entscheidungsfragen behandelt.

Betrachtet wird im Folgenden der Effekt, den eine Standardassertion, die \citet{Farkas2010} durch einen V2-Deklarativsatz mit fallender Intonation realisiert sehen, auf den Kontext ausübt. 

Bevor die Assertion in (\ref{232}) getätigt wird, besteht der Kontextzustand K\textsubscript{1} (vgl. \ref{231}).

\begin{exe}
\ex\label{231} K$_1$: initialer Kontextzustand\\[-.5em]
\begin{tabular}[t]{B{6.65em}p{6.65em}p{6.65em}B{6.65em}}
\lsptoprule
 $\textrm{DC}_{\textrm{A}}$ & \multicolumn{2}{B{13.3em}}{Table} &  $\textrm{DC}_{\textrm{B}}$ \tabularnewline
\cmidrule(lr){1-1}\cmidrule(lr){2-3}\cmidrule(lr){4-4}
{} & \multicolumn{2}{p{13.3em}}{} & {}  \tabularnewline\cmidrule(lr){1-2}\cmidrule(lr){3-4}
\multicolumn{2}{p{13.3em}}{Common Ground $\textrm{s}_{1}$}&\multicolumn{2}{p{13.3em}}{Projected Set $\textrm{ps}_{1} = \lbrace \textrm{s}_{1} \rbrace$} \tabularnewline
\lspbottomrule
\end{tabular}
\end{exe}

\begin{exe}
	\exi{(1*)}{K$_1$: initialer Kontextzustand\\
	\begin{tabularx}{\linewidth}{B{7em}B{7em}B{7em}B{7em}}
		\lsptoprule
		$\textrm{DC}_{\textrm{A}}$ & \multicolumn{2}{B{12.8em}}{Table} &  $\textrm{DC}_{\textrm{B}}$ \tabularnewline
		\cmidrule(lr){1-1}\cmidrule(lr){2-3}\cmidrule(lr){4-4} 
		\tabularnewline
		\cmidrule(lr){1-2}\cmidrule(lr){3-4}
		\multicolumn{2}{p{12.8em}}{Common Ground $\textrm{s}_{1}$} & \multicolumn{2}{p{12.8em}}{Projected Set $\textrm{ps}_{1} = \lbrace \textrm{s}_{1} \rbrace$} \tabularnewline
		\lspbottomrule
	\end{tabularx}}
\end{exe}

\begin{exe}
	\ex\label{232}
	A: Astrid ist zu Hause.
\end{exe}
Der Effekt, den die Äußerung der Assertion aus (\ref{232}) auf den Kontext nimmt, ist, dass die ausgedrückte Proposition p $\textrm{DC}_{\textrm{A}}$ hinzugefügt wird, sowie, dass die syntaktische Struktur des Satzes und sein Denotat oben auf den Stapel auf dem Tisch abgelegt werden. Am Zustand des cg ändert sich nichts: Der neue Zustand $\textrm{s}_{2}$ ist identisch mit dem vorherigen (vgl. \ref{233}).

\begin{exe}
\ex\label{233} $\textrm{K}_{2}$: A hat relativ zu K\textsubscript{1} assertiert: \textit{Astrid ist zu Hause}.\\[-.5em]
\begin{tabular}[t]{B{6.65em}p{6.65em}p{6.65em}B{6.65em}}
\lsptoprule
 $\textrm{DC}_{\textrm{A}}$ & \multicolumn{2}{B{13.3em}}{Table} &  $\textrm{DC}_{\textrm{B}}$ \tabularnewline
\cmidrule(lr){1-1}\cmidrule(lr){2-3}\cmidrule(lr){4-4}
{p} & \multicolumn{2}{p{13.3em}}{$\langle \textrm{Astrid ist zu Hause} [\textrm{D}];\lbrace\textrm{p}\rbrace\rangle$} & {}  \tabularnewline\cmidrule(lr){1-2}\cmidrule(lr){3-4}
\multicolumn{2}{p{13.3em}}{Common Ground $\textrm{s}_{2}$ = $\textrm{s}_{1}$}&\multicolumn{2}{p{13.3em}}{Projected Set $\textrm{ps}_{2} = \lbrace \textrm{s}_{1} \cup \lbrace \textrm{p} \rbrace \rbrace$} \tabularnewline
\lspbottomrule
\end{tabular}
\end{exe}
\begin{exe}
	\exi{(3*)} $\textrm{K}_{2}$: A hat relativ zu K\textsubscript{1} assertiert: \textit{Astrid ist zu Hause}.\\
	\begin{tabularx}{\linewidth}{B{7em}B{7em}B{7em}B{7em}}
		\lsptoprule
		$\textrm{DC}_{\textrm{A}}$ & \multicolumn{2}{B{12.8em}}{Table} &  $\textrm{DC}_{\textrm{B}}$ \tabularnewline
		\cmidrule(lr){1-1}\cmidrule(lr){2-3}\cmidrule(lr){4-4}
		{p} & \multicolumn{2}{B{12.8em}}{$\langle \textrm{Astrid ist zu Hause} [\textrm{D}];\lbrace\textrm{p}\rbrace\rangle$} & {}  \tabularnewline
		\cmidrule(lr){1-2}\cmidrule(lr){3-4}
		\multicolumn{2}{p{12.8em}}{Common Ground $\textrm{s}_{2}$ = $\textrm{s}_{1}$}&\multicolumn{2}{p{12.8em}}{Projected Set $\textrm{ps}_{2} = \lbrace \textrm{s}_{1} \cup \lbrace \textrm{p} \rbrace \rbrace$} \tabularnewline
		\lspbottomrule
	\end{tabularx}
\end{exe}
Der Inhalt dieser Assertion kann jetzt cg-Inhalt werden, indem B ihn annimmt\slash bestätigt\slash akzeptiert, etwa durch Äußerungen bzw. non-verbale Handlungen wie in (\ref{234}).