% latex table generated in R 4.1.1 by xtable 1.8-4 package
% Sun Feb 06 15:32:09 2022
\begin{table}[ht]
\centering
\scalebox{0.80}{
\begin{tabular}{crr|c}
  \lsptoprule
\multirow{2}{*}{Par-initial Pronominals} & \multicolumn{2}{c}{Across paragraph referent prominence} & \multirow{2}{*}{Total}\\
 \cline{2-3} & \multicolumn{1}{c}{non-prominent} & \multicolumn{1}{c}{prominent} &   \\ 
  \midrule
%description & 1091 &  519 & 1610 \\ 
%  row \% & 67.8\% & 32.2\% & 37.5\% \\ 
%  col \% & 40.9\% & 31.9\% &  \\ 
%  table \% & 25.4\% & 12.1\% &  \\ 
%  \hline
%name & 1434 &  916 & 2350 \\ 
%  row \% & 61.0\% & 39.0\% & 54.8\% \\ 
%  col \% & 53.8\% & 56.3\% &  \\ 
%  table \% & 33.4\% & 21.3\% &  \\ 
%  \hline
count &  140 &  191 &  331 \\ 
  row \% & 42.3\% & 57.7\% & %7.7\% 
  \\ 
  col \% & 5.3\% & 11.7\% &  \\ 
%  table \% & 3.3\% & 4.5\% &  \\ 
%  \hline
%Total & 2665 & 1626 & 4291 \\ 
%   & 62.1\% & 37.9\% &  \\ 
   \lspbottomrule
\end{tabular}}
\caption[Cross-tabulated distribution of RFs (description, name, pronoun) by across-boundary prominence status of referents (hearafter, prominence conditions).]{Cross-tabulated distribution of RFs (description, name, pronoun) by cross-boundary prominence status of referents (hearafter, prominence conditions): nonprominent and prominent. For each RF, the first line shows the number of instances of each prominence condition. The second line (row\%) shows the row-wise distribution of each RF across both prominence conditions. The third line (col\%) shows the column-wise distribution of each prominence condition across different RFs.} 
\label{tab:crossboundaryAll}
\end{table}
