\begin{table*}
\fittable{
\begin{tabular}{l rrrr rrr}
\lsptoprule
Model & RE Acc.$\uparrow$ & SED$\downarrow$ & BLEU$\uparrow$ & Text Acc.$\uparrow$ & Precision$\uparrow$ & Recall$\uparrow$ & F1$\uparrow$ \\ \midrule
\modname{RREG-S} & \underline{54.60} & \textbf{3.65} & \underline{72.05} & 16.28 & \textbf{89.52} & \underline{77.57} & \underline{82.28} \\
\modname{RREG-L} & 53.43 & 3.77 & 71.27 & 15.49 & 73.94 & 73.96 & 73.95 \\
\modname{ML-S} & 54.35 & 3.70 & 70.89 & 15.43 & 71.70 & 63.52 & 66.39 \\
\modname{ML-L} & \textbf{56.69} & \underline{3.66} & \textbf{72.25} & \underline{16.36} & 81.66 & 63.62 & 68.36 \\
\modname{ATT+Copy} & 48.75 & 4.46 & 68.48 & 14.88 & 85.33 & 75.74 & 79.63 \\
\modname{ATT+Meta} & 53.34 & 4.22 & 70.82 & \textbf{16.54} & 86.32 & 75.56 & 79.81 \\
\modname{ProfileREG} & 40.96 & 7.40 & 61.04 & 11.39 & \underline{86.44} & \textbf{87.40} & \textbf{86.91} \\
\lspbottomrule
\end{tabular}}
\caption[Results of the automatic evaluation of \webnlg.]{Results of the automatic evaluation of \webnlg. The best results are \textbf{boldfaced}, while the second best are \underline{underlined}. $\uparrow$ means the higher the metric the better, while $\downarrow$ means the lower the better.}
\label{tab:webnlg_result}
\end{table*}
