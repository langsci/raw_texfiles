
\begin{table}
\begin{tabularx}{\textwidth}{lrYY}
  \lsptoprule
\multirow{2}{*}{ Current RF} & \multicolumn{3}{c}{Antecedent Type} \\
 \cline{2-4} &  {\val{ante-description}} &  {\val{ante-name}} &  {\val{ante-pronoun}} \\
  \midrule
%\val{description} & 4835 & 1235 &  900 \\ 
  description \% & \fbox{69.4\%} & 17.7\% & 12.9\% \\ 
  %col \% &  57.2\% & 14.1\% & 20.1\% \\ 

%\val{name} & 1010 & 5276 & 1416 \\ 
  name \% & 13.1\% & \fbox{68.5\%} & 18.4\% \\ 
  %col \% & 12.0\% & 60.4\% & 31.6\% \\ 

%\val{pronoun} & 2601 & 2227 & 2167 \\ 
  pronoun \% & 37.2\% & 31.8\% & \fbox{31.0\%} \\ 
  %col \% & 30.8\% & 25.5\% & 48.3\% \\ 
   \lspbottomrule
\end{tabularx}
\caption[The probability of different antecedent types for each referential form.]{The probability of different antecedent types (\val{ante-description}, \val{ante-name}, \val{ante-pronoun}) for each referential form. For instance, 69.4\%, 17.7\%, and 12.9\% of all the REs with the RF type \val{description} have an antecedent of the type \val{description}, \val{proper name}, and \val{pronoun}, respectively.}
\label{tab:prtype3}
\end{table}
