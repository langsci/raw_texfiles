\newcommand*{\orcid}[1]{}


\newcommand{\sur}[1]{\ensuremath{^{\textrm{#1}}}}
\newcommand{\sous}[1]{\ensuremath{_{\textrm{#1}}}}

%https://tex.stackexchange.com/questions/31091/space-after-latex-commands
%the use of xspace creates the needed space after the words
\newcommand{\val}[1]{\texttt{#1}}
\newcommand{\corpus}[1]{\textsc{\lowercase{#1}}}
\newcommand{\method}[1]{\textit{#1}}
\newcommand{\modname}[1]{\textsc{\lowercase{#1}}}
\newcommand{\term}[1]{\textit{#1}}
\newcommand{\intext}[1]{\textit{#1}}
\newcommand{\sharedtask}[1]{\textsc{\lowercase{#1}}}
\newcommand{\example}[1]{\textsc{\lowercase{#1}}}

\newcommand{\defen}[1]{\textcolor{gray}{\textsc{definition:}} #1}

\theoremstyle{remark}
\newtheorem{definition}{Definition}


%corpora
\newcommand{\grec}{\textsc{grec}\xspace}
\newcommand{\wsj}{\textsc{wsj}\xspace}
\newcommand{\webgnlg}{\textsc{webgnlg}\xspace}
\newcommand{\onto}{\textsc{ontonotes}\xspace}
\newcommand{\grectwo}{\textsc{grec-2.0}\xspace}
\newcommand{\grecp}{\textsc{grec-people}\xspace}
\newcommand{\msrcor}{\textsc{grec-2.0}\xspace}
\newcommand{\negcor}{\textsc{grec-people}\xspace}
%\newcommand{\msrcor}{\textsc{msr}\xspace}
%\newcommand{\negcor}{\textsc{neg}\xspace}
\newcommand{\webnlg}{\textsc{webnlg}\xspace}
\newcommand{\grecmsreight}{\textsc{grec-msr'08}\xspace}
\newcommand{\grecmsrnine}{\textsc{grec-msr'09}\xspace}
\newcommand{\grecnegnine}{\textsc{grec-neg'09}\xspace}
\newcommand{\grecnegten}{\textsc{grec-neg'10}\xspace}
\newcommand{\model}{\textsc{model}\xspace}
\newcommand{\bert}{\textsc{bert}\xspace}
\newcommand{\refe}{REF\xspace}
\newcommand{\ante}{ANTE\xspace}

\newcommand{\context}{REG-in-con\-text\xspace}
\newcommand{\shot}{one-shot REG\xspace}
\newcommand{\Shot}{One-shot REG\xspace}
\newcommand{\referent}[1]{\textsc{#1}}

%%%%chapters
\newcommand{\1}{\ref{chap1}\xspace}
\newcommand{\2}{\ref{chap2}\xspace}
\newcommand{\3}{\ref{chap3}\xspace}
\newcommand{\4}{\ref{chap4}\xspace}
\newcommand{\5}{\ref{chap5}\xspace}
\newcommand{\6}{\ref{chap6}\xspace}
\newcommand{\7}{\ref{chap7}\xspace}
\newcommand{\8}{\ref{chap8}\xspace}

%%%%Studies
\newcommand{\studA}{\hyperref[sec:reconsturction]{A}\xspace}
\newcommand{\studB}{\hyperref[sec:consensus]{B}\xspace}
\newcommand{\studC}{\hyperref[sec:recency]{C}\xspace}
\newcommand{\studD}{\hyperref[sec:corpuspar]{D}\xspace}
\newcommand{\studE}{\hyperref[sec:mlstudy]{E}\xspace}
\newcommand{\studF}{\hyperref[sec:modelcomparison]{F}\xspace}
\newcommand{\studG}{\hyperref[sec:probingexplainability]{G}\xspace}
\newcommand{\expone}{\hyperref[subsec:building]{1}\xspace}
\newcommand{\exptwo}{\hyperref[subsec:ffs]{2}\xspace}
\newcommand{\expthree}{\hyperref[subsec:optfeatures]{3}\xspace}


%% For Language Science Press changes
\newcommand{\scaps}[1]{\textsc{\lowercase{#1}}}

%% Replacing italics with underline 
\newcommand{\italunder}[1]{\textbf{#1}}

% For adding the publication footnotes.
\newcommand\blfootnote[1]{%
	\begingroup
	\renewcommand\thefootnote{}\footnote{#1}%
	\addtocounter{footnote}{-1}%
	\endgroup
}

%Roman numbers
\renewcommand{\UrlFont}{\ttfamily\small}
\newcommand{\RNum}[1]{\uppercase\expandafter{\romannumeral #1\relax}}


%\usepackage{amssymb}% http://ctan.org/pkg/amssymb
%\usepackage{pifont}% http://ctan.org/pkg/pifont
\newcommand{\cmark}{\ding{51}}%
\newcommand{\xmark}{\ding{55}}%
