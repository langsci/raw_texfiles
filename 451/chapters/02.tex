% !TeX spellcheck = en_US
\chapter{Choosing referring expressions in context: Linguistic studies}\label{chap2}


\section{Introduction}\label{chap2:sec:intro}

Referring is a fundamental aspect of communication and plays a crucial role in how we discuss and interact with the world around us. In conversations and written texts, we often employ a variety of forms to mention people, objects, and concepts. Intriguingly, the same referent can be addressed in multiple ways, depending on the context, the speaker's intention, and the level of formality or familiarity involved. Consider the following examples to illustrate this diversity in referential expression:

\begin{exe}
	\ex 
	\begin{xlist}
		\ex \italunder{A woman in a black dress} just walked past the building. \label{ex:woman1}
		\ex \italunder{The woman you were talking to the other day} just walked past the building. \label{ex:woman2}
		\ex \italunder{Emily Smith, the famous novelist,} just walked past the building. \label{ex:woman3}
		\ex \italunder{Emily Smith} just walked past the building. \label{ex:woman4}
		\ex \italunder{Emily} just walked past the building. \label{ex:woman5}
		\ex \italunder{The novelist} just walked past the building. \label{ex:woman6}
		\ex \italunder{That woman over there} just walked past the building. \label{ex:woman7}
		\ex \italunder{She} just walked past the building. \label{ex:woman8}
		\ex \italunder{She} entered the alley and \italunder{$\emptyset$} just walked past the building. \label{ex:woman9}
	\end{xlist} 
\end{exe}

The bold referring expressions can all be used in various contexts to refer to the fictional novelist, \intext{Emily Smith}. Below, I outline scenarios where each sentence might be appropriate:

\begin{description}
	\item[Scenario 1:]Sentence \ref{ex:woman1} describes a situation where the speaker is not familiar with the novelist and simply wants to report an event to the listener.
	\item[Scenario 2:]Sentence \ref{ex:woman3} could be used when the speaker knows Emily Smith but is not sure if the listener would recognize her without additional details.
	\item[Scenario 3:] If both the speaker and the listener know, or are friends with, the novelist, then sentence \ref{ex:woman5} is suitable.
	\item[Scenario 4:] When the speaker and the listener are talking about the novelist and she is the topic of their conversation, sentence \ref{ex:woman8} is appropriate.
\end{description}


This chapter presents various theories surrounding the choice of referential forms and the linguistic factors at play.  Section \sectref{sec:disctheory} introduces linguistic theories that provide insights into the reasons behind choosing specific RFs. These insights connect the choice of RF with the status of a referent in a given context, such as whether a referent is accessible, prominent, or salient. The empirical studies discussed in \sectref{sec:lingfactors} highlight several factors influencing a referent's status. This section does not differentiate between corpus-based analyses and psycholinguistic experiments, focusing instead on the factors themselves rather than the methods employed to study them.  This chapter lays the groundwork for Chapters \5 and \6, which further explore the importance of different linguistic features for the \context task.


\section{Theories about RF choice}\label{sec:disctheory}

From the provided examples, it becomes clear that multiple ways exist to refer to a specific entity, and the felicity of these REs varies depending on the context. For example, in \textsc{scenario 4} \REF{ex:woman8}, using a pronoun is entirely appropriate due to the established topic of conversation. In contrast, employing the RE from \REF{ex:woman1} would be inappropriate, and using the RE from \REF{ex:woman3} would come across as overinformative and peculiar.

What aspects of these scenarios make the use of different REs appear more or less felicitous? This question might be addressed from a cognitive standpoint. Another approach would be to answer it based on the characteristics of the text or the existing relationships between REs within the text. 

\term{Givenness Theory} \citep{gundel1993cognitive} and \term{Accessibility Theory} \citep{ariel1990accessing,ariel2001accessibility} offer cognitive explanations for the choice of referring expressions. Both theories also present a hierarchy that associates various RFs with the cognitive status or accessibility each form mediates. \term{Centering Theory} (CT) \citep{grosz1995centering} and \term{Prominence Theory} (PT) \citep{Heusinger2019} explain this choice by examining the properties of a text and existing relationships between the REs. I will first provide a brief overview of the two cognitive theories and then delve into centering and prominence perspectives on reference.

Givenness Theory addresses the cognitive status of a referent in an addressee's mind. According to \citet{gundel1993cognitive}, there are six distinct cognitive statuses that can explain the use of various RFs. These cognitive statuses and the forms that express them are organized in a hierarchy known as the \term{Givenness Hierarchy} (see \figref{fig:givenness}). The statuses are arranged in such a way that each status also entails all statuses beneath it. Therefore, if a referent is \term{uniquely identifiable}, it is also necessarily \term{referential} and \term{type identifiable}.

\begin{figure}
\tabcolsep=.5\tabcolsep
\begin{tabular}{@{}ccccccccccc@{}}
	\begin{tabular}[c]{@{}c@{}}in\\ focus\end{tabular} & \textgreater{} & activated              & \textgreater{} & familiar   & \textgreater{} & \begin{tabular}[c]{@{}c@{}}uniquely\\ identifiable\end{tabular} & \textgreater{} & referential           & \textgreater{} & \begin{tabular}[c]{@{}c@{}}type\\ identifiable\end{tabular} \\
	&                &                        &                &            &                &                                                                 &                &                       &                &                                                             \\
	\{it\}	&                & 
	$ \left\{\begin{array}{@{\,}c@{\,}}
			\text{that}\\
			\text{this}\\
			\text{this N}
		\end{array}\right\}$ &                & 
	\{that N\} &                & 
	\{the N\}  &                & 
	\begin{tabular}[c]{@{}c@{}} \{indefinite \\ this N\} \end{tabular} &                & 
	\{a N\}                                                        
\end{tabular}
	\caption{The Givenness Hierarchy of \citet{gundel1993cognitive}.}\label{fig:givenness}
\end{figure}

The least restrictive cognitive status is termed \term{type identifiable}. This means that upon hearing the expression, the addressee should be able to recognize the correct type of a referent. For instance, a sentence such as \intext{I bought a car} should provide the addressee with a representation of a car rather than any other object, such as \intext{a bag}. The linguistic form associated with this cognitive status in English is \val{indefinite determiner}\val{ + }\val{N}.

Conversely, at the opposite end of the spectrum, the most restrictive cognitive status is termed \term{in focus}. Being ``in focus" signifies that the referent is both in short-term memory and is the current center of attention. This status allows the use of pronominal forms. 

The following examples from \citet[129]{gundel2003} are all continuations of the sentence \intext{I could not sleep last night}. These continuations show that different REs are associated with different cognitive statuses according to the Givenness Hierarchy.

\begin{exe}
	\ex
	\begin{xlist}
		\ex \italunder{A train} kept me awake. \\
		Type Identifiable -- identify what kind of thing this is.
		\ex \italunder{This train} kept me awake.\\
		Referential -- associate a unique representation by the time the sentence is processed.
		\ex \italunder{The train} kept me awake.\\
		Uniquely identifiable -- associate a unique representation by the time the nominal is processed.
		\ex \italunder{That train} kept me awake.\\
		Familiar -- associate a representation already in memory.
		\ex \italunder{This train/this/that} kept me awake.\\
		Activated -- associate a representation from working memory.
		\ex \italunder{It} kept me awake.\\
		In focus -- associate a representation that your attention is currently focused on.
	\end{xlist}
\end{exe}

A significant limitation of this hierarchy is its failure to account for REs in the form of a proper name. To address this limitation, \citet{Mulkern1996} expanded the hierarchy to include proper names. She noted that a proper name might be employed if a referent meets the unique identifiability criterion. Typically, the most extended form of the proper name (for instance, the complete name with modification) is used when the referent is first introduced. \citeauthor{Mulkern1996} also observed that a single name (either first or last) is commonly used to reference a referent that satisfies, at a minimum, the familiarity criterion.

In combination with \citeauthor{Mulkern1996}'s extension, the Givenness Theory offers cognitive explanations for the use of various RFs. However, it does not account for the use of modifications, such as in the case of modified NPs. 

Accessibility Theory, a cognitive theory rooted in the concept of a referent's accessibility, introduces a more comprehensive hierarchy of referring expressions \citep{ariel1990accessing,ariel2001accessibility}. \citet{ariel1990accessing,ariel2001accessibility} posited that each RF encodes a distinct degree of mental accessibility. Furthermore, REs cue the addressee on ``how to retrieve the appropriate mental representation in terms of degree of mental accessibility" \citep[31]{ariel2001accessibility}. \figref{fig:accessibilityhierarchy} shows the Accessibility Hierarchy, as described in \citet{ariel2001accessibility}.

\begin{figure}
	\begin{tikzpicture}
		\node (A) at (0,0){Low Accessibility};
		\node (B) at (0,-10){High Accessibility};
		\draw[{Triangle[]}-{Triangle[]}] (A)--node[auto, align=left]{%
			(a) Full name + modifier \\
			(b) Full name \\
			(c) Long definite description \\
			(d) Short definite description \\
			(e) Last name \\
			(f) First name \\
			(g) Distal demonstrative + modifier \\
			(h) Proximal demonstrative + modifier \\
			(i) Distal demonstrative (+ NP) \\
			(j) Proximal demonstrative (+ NP) \\
			(k) Distal demonstrative (- NP) \\
			(l) Proximal demonstrative (- NP) \\
			(m) Stressed pronoun + gesture \\
			(n) Stressed pronoun \\
			(o) Unstressed pronoun \\
			(p) Cliticized pronoun \\
			(q) Verbal person inflection \\
			(r) Zero
		}(B);
	\end{tikzpicture}
	\caption{The Accessibility Hierarchy of \citet{ariel2001accessibility}.}\label{fig:accessibilityhierarchy}
\end{figure}

As illustrated in \figref{fig:accessibilityhierarchy}, the hierarchy provides information regarding the accessibility degrees of proper names, descriptions, demonstrative NPs, and pronouns. Each primary category is further divided into more fine-grained subcategories. \tabref{tab:accessibilitypropername} displays the degree of accessibility of proper names, arranged from the least to the most accessible.

\begin{table}[h!]
\fittable{
	\begin{tabular}{lllllll}
		\val{Full name+modifier} & \textless{} & \val{Full} \val{name} & \textless{} & \val{Last} \val{name} & \textless{} & \val{First} \val{name} \\
		\intext{Joe Biden, the US president} & \textless{} & \intext{Joe Biden} & \textless{} & \intext{Biden} & \textless{} & \intext{Joe}
	\end{tabular}}\caption{The degree of accessibility of proper names, arranged from the least to the most accessible.}\label{tab:accessibilitypropername}
\end{table}

According to \citet{ariel1990accessing}, three criteria -- \term{informativity}, \term{rigidity}, and \term{attenuation} -- control the linguistic coding of accessibility degrees. These criteria, which partially overlap, translate the cognitive concept of accessibility into RFs with varying degrees of accessibility.

Based on the informativity criterion, expressions containing more lexical information are employed to retrieve referents with a lower degree of accessibility.
Informativity serves as the primary criterion for determining the use of full names (e.g., \intext{Joe Biden}) versus partial names (e.g., \intext{Biden}). Nevertheless, it does not explain the distinction between first names (e.g., \intext{Joe}) and last names (e.g., \intext{Biden}), as both possess comparable amounts of lexical information.


The second criterion is rigidity, which pertains to the uniqueness of an expression. In Western countries, at least, last names (e.g., \intext{Biden}) are more distinctive than first names (e.g., \intext{Joe}). Thus, their difference in accessibility can be attributed to rigidity.


The final criterion is attenuation, which refers to the phonological size of an expression. As per \citet{ariel1991function}, attenuation and informativity significantly overlap. Nonetheless, this criterion distinguishes REs that are equally informative and rigid. For instance, consider the expressions \intext{the United States of America} and \intext{US}. Both expressions are equally unique and informative, but \intext{the United States of America} is less attenuated, suggesting a lower degree of accessibility. The Accessibility Theory posits that forms that are more informative, more rigid, and less attenuated are utilized to refer to discourse referents less accessible to the addressee.


The Givenness and Accessibility hierarchies provide cognitive explanations for the utilization of various REs, linking this use to the referent's cognitive status in the addressee's mind. A notable strength of these two theories is their comprehensive inventory of RFs, which they associate with cognitive explanations. However, a significant limitation of these theories is their evaluation of each referent's cognitive status or accessibility in isolation, overlooking potential competition within a given context.

Centering Theory, which I present next in this chapter, seeks to address the aforementioned two shortcomings by (1) considering the contextual competition and relational properties of referents and (2) offering concrete implementation rules for predicting RFs. CT models the relationship between the choice of RE, the focus of attention, and coherence within a discourse segment \citep{grosz1995centering}. This theory posits that the choice of RE is constrained by the centrality of the discourse referents to an utterance. 

Each utterance ($U_{i}$) in a discourse segment ($D$) possesses a \textit{ranked} set of \term{for\-ward-looking centers} (Cfs). These are the entities that are mentioned in an utterance and are potential candidates to become the center of the next utterance. They are ranked based on their grammatical roles, with subjects usually ranked higher than objects. The highest-ranked Cf in $U_{i}$ is termed the \term{preferred center} (Cp).

Furthermore, non-initial utterances have a \term{backward-looking center} (Cb). This is the entity that is the current center of attention in an utterance and the current utterance is ``about." The Cb of $U_{i}$ is the highest-ranked Cf of its antecedent $U_{i-1}$, which is realized in $U_{i}$. A list of various utterances is provided in \tabref{tab:ctexamples}, along with their Cf, Cp, and Cb.

\begin{table}
	\fittable{
		\begin{tabular}{llll}
			\lsptoprule
			Utterance & Cf& Cp& Cb\\
			\midrule
			{[}$U1${]} Lena is a passionate biker.        & \textsc{lena}                & \textsc{lena}  & -     \\
			{[}$U2${]} She bikes everyday to the university.        & \textsc{lena}, \textsc{university}                & \textsc{lena}  & \textsc{lena}     \\
			{[}$U3${]} Now, she wants to go on a cycling trip. & \textsc{lena}, \textsc{cycling trip} & \textsc{lena}  & \textsc{lena}  \\
			{[}$U4${]} Maria likes to join her.   & \textsc{maria}, \textsc{lena}         & \textsc{maria} & \textsc{lena}  \\
			{[}$U5a${]} She calls her to check the dates.     & \textsc{maria}, \textsc{lena}, \textsc{dates}         & \textsc{maria} & \textsc{maria} \\
			{[}$U5b${]} Nina told her not to.     & \textsc{nina}, \textsc{maria}         & \textsc{nina} & \textsc{maria} \\
			\lspbottomrule
		\end{tabular}\caption[Centering Theory and the instantiation of centers.]{Centering Theory and the instantiation of centers. Cf, Cp, and Cb stand for forward-looking, preferred, and backward-looking centers, respectively.}\label{tab:ctexamples}
	}
\end{table}


In addition, CT develops a typology for transitions from $U_{i-1}$ to $U_{i}$ based on the interaction between the centers. These transitions can be distinguished by two factors: (1) whether Cb and Cp of $U_{i}$ are the same, and (2) whether Cb of $U_{i-1}$ and $U_{i}$ are the same. 

\begin{table}
	\begin{tabular}{ccc}
		\lsptoprule
		Transition types & Cb($U_{i}$)=Cb($U_{i-1}$) & Cb($U_{i}$)=Cp($U_{i}$) \\
		\midrule
		Continue     & $+$               & $+$             \\
		Retain        & $+$               & $-$             \\
		Smooth shift     & $-$               & $+$             \\
		Rough shift           & $-$               & $-$         \\
		\lspbottomrule   
	\end{tabular} 
	\caption{Centering transitions \citep{Walker1996}.} \label{tab:centering_transitions}
\end{table}

\tabref{tab:centering_transitions} shows that in the \term{continue} transition, the speaker has talked about an entity in the previous utterance and now continues to talk about it. The Cbs in the previous and current utterances are identical, and the Cb appears to be in the subject position, the highest-ranked grammatical role. In \tabref{tab:ctexamples}, the transition from $U2$ to $U3$ is a continue transition.

In a \term{retain} transition, the speaker continues to discuss the same referent as in the previous utterance but intends to shift to a new referent in the subsequent sentence. The change is motivated by positioning the referent in a less preferred grammatical position in the utterance. The transition from $U3$ to $U4$ exemplifies a retain transition.


A \term{shift} transition causes the Cb of the current utterance to change. If the entity is realized as Cp, the transition is \term{smooth} and signals that the speaker is interested in continuing to discuss the current Cb. It is a \term{rough} transition if the entity is realized in a less preferred syntactic position. $U5a$ and $U5b$ represent two different continuations of $U4$. While the transition between $U4$ and $U5a$ is smooth, the transition between $U4$ and $U5b$ is rough.

According to CT, ``sequences of continuation are preferred over sequences of retaining and sequences of retaining are to be preferred over sequences of shifting" \citep[17]{grosz1995centering}. Therefore, one basis for achieving local coherence would be to shift toward new centers using retention.

The RF choice also plays a crucial role in maintaining coherent discourse through transitions. A key rule in CT is the \term{pronoun rule.} It posits that if any elements of the Cf of the previous utterance are realized as pronouns in the current sentence, the Cb of the current sentence must also be realized as a pronoun. This rule explains why a continuation like \REF{ex:ctpron1}, in which the Cb is pronominalized, reads better than \REF{ex:ctpron2}. Moreover, this rule suggests that pronouns are the preferred form of referring when the transition between the two utterances is a continue transition.

\begin{exe}
	\ex Lena invited Maria for dinner. 
	\ex
	\begin{xlist}
		\ex She asked her to be there at 7 p.m. \label{ex:ctpron1}
		\ex Lena asked her to be there at 7 p.m. \label{ex:ctpron2} 
	\end{xlist}
	
\end{exe}

Further supporting this, experimental studies have demonstrated that when Cb is realized as a non-pronominal form in a continuation scenario, a processing penalty (known as the \term{Repeated Name Penalty}) occurs \citep{Almor1999}. Moreover, the application of the pronoun rule extends beyond theoretical analysis. It has been implemented in various computational studies \citep{kibble1999using, poesio2004centering, poesio2004centering} to enhance the naturalness and coherence of generated texts.

As the previous paragraphs have shown, CT provides concrete rules for predicting the RF based on the coherence of the text and the available transitions between utterances. However, for the implementation to work, CT has reduced the dimension of the referential inventory to only two forms, unlike the previous two cognitive theories. Therefore, the predictive power of CT is limited to the pronominalization problem. In addition to its limited scope, CT can only explain the choice of RF in \term{local contexts}, that is, in two adjacent utterances. As \citet{Heusinger2019} point out, however, the ranking of referents in discourse is not limited to the local context; rather, they exhibit a global effect. In what follows, I present the Prominence Theory of \citet{Heusinger2019}. The advantage of this theory over the Accessibility and Givenness theories is that it considers the relational properties of referents. Compared to CT, PT does not limit its scope to the local context but extends its boundaries to a broader context. Moreover, it concretely provides three defining characteristics of a theory of reference.

Based on the concept of prominence discussed in \citet{himmelmann2015prominence}, PT \citep{Heusinger2019} addresses both the \term{dynamic} and \term{relational} aspects of referents. Utilizing the following criteria, this theory views prominence as a ``structure-building principle" to explain the representation of different referents in discourse (p. 119):

\begin{definition}[Singling-out]
	Prominence is a relational property that singles out one element from a set of elements of equal type and structure.
\end{definition}

\begin{definition}[Dynamicity]
	Prominence status shifts in time (as discourse unfolds).
\end{definition}

\begin{definition}[Structural attraction]
	Prominent elements are structural attractors; i.e., they serve as anchors for the larger structures they are constituents of, and they
	may license more operations than their competitors.
\end{definition}

The first rule (Def. 1) accounts for the relational nature of reference. It states that the prominence status of a referent is determined by comparison with other elements of similar type and structure, that is, other discourse referents. Thus, in contrast to Givenness and Accessibility, referents are not considered in isolation, but their prominence status is determined in relation to other competing referents.

The second definition (Def. 2) addresses the dynamic nature of referents. As discourse unfolds, the prominence status of referents changes dynamically. This means that the most prominent referent may lose prominence and regain it later in the discourse. Both PT and CT account for dynamicity; however, unlike CT, PT relates dynamicity to a broader context.

The third definition (Def. 3) states that more variation is observed when a referent is prominent. For less prominent entities, we use enriched forms with more semantic content; for more prominent entities, we can use various referential strategies. Thus, a broader inventory of forms is available for reference to prominent entities. Therefore, in line with Givenness Theory, PT can also explain the varied use of referring expressions in context.

Based on the aforementioned three principles, PT successfully combines different aspects of the previous theories of reference, namely, Givenness, Accessibility, and CT. The fact that it considers the relational properties of reference renders it more powerful than the two cognitive theories. With its second rule, the dynamicity principle, PT can explain the shifts that occur in discourse. Unlike CT, it also has the advantage of considering a broader context, which is more suitable for studying natural language.

Another important point is that PT does not introduce a rigid inventory of RFs, nor does it limit its applicability to a specific class of RFs. Instead, it proposes three principles as the structure-building elements of discourse. Consequently, the theory is more adaptable and applicable to a variety of cases. A notable advantage of CT, however, is its definitive rules and straightforward implementation. However, this aspect also introduces limitations. The initial pronominalization rule of CT heavily relies on a single factor, specifically the grammatical role. Nevertheless, as we will explore, numerous other factors and their interactions play a vital role in predicting RFs. 

For the reasons stated above, I employ the terminology of PT in the remainder of this book. Having presented various theoretical explanations, I now turn to the factors influencing the choice of RE. In their work, \citet{Heusinger2019} describe these features as ``prominence-lending cues" because ``they boost the prominence value of their respective referent to a certain extent" (p. 119).


\section{Prominence-lending cues}\label{sec:lingfactors}
\begin{sloppypar}
In the previous section, I mentioned that the RF choice reflects the prominence status of referents in context. The factors influencing prominence have frequently been discussed in both theoretical and empirical studies.
A reduced form, such as a pronoun, is often employed to refer to prominent entities, whereas a semantically richer expression is used for less prominent entities. Factors influencing prominence include grammatical function, animacy, recency, thematic role, the presence of competing referents, and coherence relations, among others \citep{Brennan1995, fukumura2011effect, arnold2007effect, ariel1990accessing}. The impact of these factors on the prominence of referents has been studied either in isolation \citep{arnold2007effect} or in combination \citep{ariel1990accessing}.
\end{sloppypar}

In the remainder of this section, I will discuss prominence-lending cues and their influence on the prominence status of referents in context. The cues under discussion in this section include syntax (\sectref{subsec:syntax}), thematic role (\sectref{subsec:thematicrole}), givenness (\sectref{subsec:givennness}), competition (\sectref{subsec:competition}), animacy (\sectref{subsec:animacy}), and recency (\sectref{subsec:recency}).


\subsection{Syntax}\label{subsec:syntax}

This section highlights the syntax-related factors that influence the prominence status of referents. Below, I discuss the effects of \term{grammatical role}, \term{first-mention bias}, and \term{syntactic parallelism}.

\subsubsection{Grammatical role} 
Psycholinguistic studies have shown that subjects are more prominent than objects or adjuncts \citep{Stevenson1994,Arnold2000a,Fukumura2010,}. There are several reasons for this, including
(1) subjects often acting as agents in a sentence,
(2) subjects being the topic of a sentence, and (3) subjects being mentioned first in a canonical structure in a language like English. These factors contribute to the prominence of a subject in a sentence. Therefore, the subsequent mention of the subject is more likely to be pronominal \citep{Brennan1995,arnold2008reference, Arnold2010}. For example, as a continuation of sentence \REF{ex:syn}, sentence \REF{ex:syn_subj} is more natural than sentence \REF{ex:syn_obj}. Since \intext{John} is the subject of sentence \REF{ex:syn}, it is more prominent than \intext{David}, which means that it is more likely to be pronominalized in the following sentence.

\begin{exe}
	\ex John$_{\textsc{subj}}$ invited David$_{\textsc{obj}}$ for dinner. \label{ex:syn}
	\ex
	\begin{xlist}
		\ex \italunder{He} asked David to be there at 7 p.m. \label{ex:syn_subj}
		\ex \italunder{John} asked David to be there at 7 p.m. \label{ex:syn_obj}
		\ex David asked \italunder{him/John} to cook pasta. \label{ex:syn_nonparal}
	\end{xlist}
\end{exe}




\subsubsection{First-mention bias} 
In a language like English, the subject is typically the first entity mentioned in a sentence. The entity's prominence may be attributed to being the first-mentioned entity in the sentence, rather than because of its role as subject. \citet{Gernsbacher1989} argues that the first-mention position confers an advantaged cognitive status on entities. The addressees construct a mental representation of the information they receive. The referent presented in the first position serves as the foundation for this mental representation. \citet{Kaiser2011} conducted a series of sentence-completion and eye-tracking experiments in Finnish, demonstrating that grammatical role and order of mention are two independent factors influencing the choice of RF.

\subsubsection{Syntactic parallelism} 
Syntactic parallelism may also increase the prominence of a referent and the likelihood of pronominalization. Pronouns, as reduced forms, are more likely to be used when the target referent occupies the same syntactic position as its coreferential antecedent. Therefore, if we choose to continue sentence \ref{ex:syn} by discussing \intext{John} in the subject position (sentence \ref{ex:syn_subj}), there is a strong preference for using a pronoun compared to a scenario where \intext{John} appears as the object of the sentence (sentence \ref{ex:syn_nonparal}).

\subsection{Thematic role}\label{subsec:thematicrole}
Several experimental studies have examined the effect of \term{thematic role} alternation on reference production \citep{Stevenson1994, Arnold2001, Fukumura2010, rosa2015semantic, vogels2019both}. 
These studies investigated (1) whether certain thematic roles increase the likelihood of a referent being mentioned again in a subsequent context, known as \term{the next mention bias}, and (2) whether certain thematic roles enhance the likelihood of pronominalization.


\citet{Stevenson1994} tested various thematic role pairs in two text--com\-ple\-tion experiments and discovered that after a goal--source sentence, people showed a preference for continuing the text with the goal referent rather than the source referent. In other thematic role pairs, the patient was preferred over the agent, and the stimulus was preferred over the experiencer. However, the study identified no significant effect of thematic role on the choice of RF. The critical factor in choosing the RF was the first-mention bias, that is, whether the antecedent referent was mentioned first or second in the sentence.

In the same vein, \citet{Arnold2001} examined the role of thematic roles in reference production, focusing specifically on the thematic roles goal and source.
The experiments employed \term{transfer to possession verbs}, such as \intext{send--receive}.
The advantage of using these verbs lies in the fact that for some, like \intext{send}, the subject is the source, while for others, such as \intext{receive}, the subject is the goal.
Consequently, both the source and the goal appeared in the subject position. The following examples from \citet{Arnold2001} demonstrate the source--goal and goal--source conditions.


\begin{exe}
	\ex
	\begin{xlist}
		\ex $[$Source--Goal$]$ The drama club was worried that no one would come to the opening performance
		of their play. Everyone agreed to try to get all their friends to come. \italunder{Erin}$_{\textsc{source}}$
		sent an invitation to \italunder{Bill}$_{\textsc{goal}}$.  \label{ex:source_goal}
		\ex $[$Goal--Source$]$ Getting a telegram always scares me. It has to be either great news or awful
		news. \italunder{Juan}$_{\textsc{goal}}$ received a telegram from \italunder{Claire}$_{\textsc{source}}$ when their mother died. \label{ex:goal_source}
	\end{xlist}
\end{exe}

Similar to \citet{Stevenson1994}, \citet{Arnold2001} demonstrated that speakers tend to refer to goal entities more frequently than to source entities. Both studies suggest that end-states, or what \citeauthor{Stevenson1994} termed \term{consequences}, are the most prominent elements in a sentence. Consequently, the most predictable next mentions are end-state entities, such as the goal in source--goal sentences and the patient in agent--patient constructions. 

Contrasting with \citeauthor{Stevenson1994}, \citeauthor{Arnold2001} observed an effect of thematic roles on referential form choice: speakers used pronouns for goal entities more often than for source entities. However, since the likelihood of continuing with the subject referent is much higher than that of continuing with the goal referent, the subject bias is stronger. In other words, thematic roles influence referent accessibility only in situations where other factors, such as subject effects, are less dominant \citep{Arnold2001}.

\citet{Fukumura2010} replicated this line of experiments, employing stimulus--experiencer verbs within implicit causality contexts. They observed a next-mention bias for stimulus entities but found no significant effect on pronominalization, aligning with the findings of \citet{Stevenson1994}. 

\ea
	\ea $[$stimulus--experiencer (SE)$]$ \italunder{Glen}$_{\textsc{stimulus}}$
	annoyed \italunder{July}$_{\textsc{experiencer}}$ when the two-minute silence
	took place in the yard. This was because...\label{ex:SE}
	\ex $[$experiences--stimulus (ES)$]$ \italunder{Glen}$_{\textsc{experiencer}}$
	despised \italunder{July}$_{\textsc{stimulus}}$ when the two-minute silence
	took place in the yard. This was because...\label{ex:ES}
	\z
\z

According to the experimental results outlined in this section, thematic role influences the likelihood of a referent appearing as the next mention, but its impact on the RF choice is open to debate.

\subsection{Givenness}\label{subsec:givennness}

\citet{gundel2003} defines \term{referential givenness} as ``a relation between a linguistic expression and a corresponding non-linguistic (conceptual) entity in (a model of) the speaker/hearer's mind" (p. 125). According to \citeauthor{gundel1993cognitive}'s Givenness Hierarchy, as presented in \sectref{sec:disctheory}, different RFs convey varied information about the presumed cognitive status of entities in the addressee's mind.
The speaker assesses whether the addressee has a mental representation of a referent and selects forms accordingly. If the speaker presumes the referent is new to the addressee or its mental representation is inactive, a full form, such as a proper name or description, is used. Conversely, if the entity is the focus of the current sentence or the speaker believes the addressee is familiar with it, a reduced form is employed for reference.

As implied in the previous paragraphs, the \term{newness--givenness} distinction is not binary and should be considered a gradient notion. Various studies have adopted parameters such as degree of salience, familiarity \citep{prince1992zpg}, accessibility \citep{ariel2001accessibility}, activation, and identifiability \citep{chafe1976givenness} as key characteristics of the newness-givenness distinction. \citet{chafe1976givenness} initially made a distinction based on the identifiability of referents, that is, whether or not the mental representations of referents are identifiable to the addressee. Subsequently, he assigned three activation states to each class: \val{given}, \val{accessible}, \val{new}. \citet{prince1992zpg}, on the other hand, distinguished two levels: (1) \term{hearer-old} versus \term{hearer-new}, and (2) \term{discourse-old} versus \term{discourse-new}. According to this distinction, an entity can be both new and given on two different dimensions: it can be new in discourse but old to the hearer. Suppose \REF{ex:messi} is the introductory sentence of a sports article. \intext{Lionel Messi} is the first mention of the Argentinian soccer player in the text, making it discourse-new. However, it is very likely that the reader of a sports magazine is already familiar with \intext{Lionel Messi}, making this referent hearer-old.

\begin{exe}
	\ex \italunder{Lionel Messi} joined Paris Saint-Germain. He ... \label{ex:messi}
\end{exe}

Although it is highly likely that the addressee is familiar with the target in the previous example, there is still the possibility that the referent is unknown to the addressee. According to \citet[127]{Baumann2012}, ``persons, places or other entities are rarely ever objectively known or unknown but only with respect to some intended recipient". Generally, speakers and writers do not have access to the thoughts of the addressee, especially when addressing a large audience. Since a simplified notion of givenness will be explored in the following chapters, I will not delve further into the intricacies of givenness and its various interpretations.

\subsection{Competition}\label{subsec:competition}
Generally, the stronger the competition between a referent and other referents, the lower the likelihood of using an attenuated RE for that referent. In this section, I discuss two forms of competition, namely \term{gender} and \term{additional character} effects.

\subsubsection{The gender effect} 
Studies have shown that the likelihood of employing attenuated REs diminishes when a referent of the same gender is present in the immediate context of the target referent. One possible explanation could be the desire to avoid ambiguity. In order to circumvent ambiguity in these situations, the speaker opts for more specific forms, such as proper names and descriptions \citep{Karmiloffsmith1985,arnold2007effect,Fukumura2013,rosa2015semantic}. Consider the following:

\begin{exe}
	\ex 
	\begin{xlist}
		\ex Mary had an appointment with John. \italunder{She} turned up half an hour late. \label{ex:gender1}
		\ex\label{ex:gender2} Mary had an appointment with Emily. \italunder{She} turned up half an hour late. 
	\end{xlist}
\end{exe}

The referents in \REF{ex:gender1} possess different genders, and the pronoun \intext{she} unambiguously refers to \intext{Mary}. Conversely, in \REF{ex:gender2}, both referents share the same gender. Therefore, the pronoun becomes ambiguous, potentially referring to either \intext{Mary} or \intext{Emily}.

\citet{arnold2007effect} consider semantic competition as another plausible explanation for the increased use of more specific forms in gender-congruent settings. According to this perspective, a same-gender competing referent is semantically more similar to the target than an opposite-gender referent. Consequently, it is likely that same-gender referents experience higher competition, leading to a reduced prominence status for the target. \citet{Fukumura2013} tested this theory in an experiment conducted in Finnish, a gender-neutral language where the same pronoun (\intext{hän}) is used for both males and females. They observed a lower frequency of pronouns when both referents shared the same gender.

Similar to \citet{arnold2007effect}, \citet{Fukumura2013} suggest that a competing referent, which is semantically very similar to the target referent, impairs the memory retrieval of the non-linguistic representation of the target. Consequently, speakers resort to using more specific forms to resolve this interference. ``The fact that gender congruence reduced the use of Finnish pronouns suggests that gender is one of the non-linguistic properties that speakers take into account, even when the language does not express the referent's gender and hence the presence of a same-gender competitor does not make the use of a pronoun ambiguous" \citep[p, 1017]{Fukumura2013}. In summary, the gender effect can be attributed to several factors, most notably ambiguity avoidance and semantic competition.


\subsubsection{The additional character effect} 

In addition to the gender effect, \citet{arnold2007effect} mention a second type of competition, hereafter referred to as the additional character effect: the presence of additional characters in the immediate context of the target, regardless of whether the target and competitors share the same gender, reduces the likelihood of pronominalization. In a storytelling experiment, participants observed two-panel cartoons and listened to the first sentence under two conditions: (1) the sentence included \textit{solely} the main referent [Condition A], and (2) alongside the main character, another character of a different gender was present [Condition B].

\begin{exe}
	\ex 
	\begin{xlist}
		\ex \italunder{Mickey} went for a walk in the hills one day. [Condition A]
		\ex \italunder{Mickey} went for a walk \italunder{with Daisy} in the hills one day. [Condition B]
	\end{xlist}
\end{exe}

Participants were instructed to recite the first sentence and continue the story with a second sentence. 
The results of the experiment showed that in condition A, the likelihood of using a pronoun for the main referent was greater than in condition B. Unlike the gender-related effect, the outcome observed in this experiment was not due to ambiguity avoidance. 
According to \citet{arnold2007effect}, participants were 30\% more inclined to use \intext{Mickey} instead of \intext{he} in Condition B as opposed to Condition A. In condition B, the two characters engaged in discourse share the available attentional resources, resulting in diminished activation for each within the speaker's internal representation. Thus, the speaker tends to use a more specific form to activate the representation of the target referent.

This explanation aligns with the semantic competition effect. Although the semantic similarity between the two gender-incongruent characters is lower than that of the same-gender referents, they still share identical animacy values. 
The similarity between the two characters might intensify competition and decrease the target referent's prominence status. Therefore, animacy can also be regarded as a factor associated with semantic competition.

\subsection{Animacy}\label{subsec:animacy}
Various linguistic theories propose that \term{animate} entities hold more prominence than \term{inanimate} entities \citep{comrie1989language,Aissen2003}. This prominence influences numerous linguistic choices. For instance, animate entities are more likely to be chosen as the subject or topic of a sentence compared to inanimate entities \citep{Givon1983,Dahl1996}. \citet{Dahl1996} examined the effect of animacy on the choice of RF in a Swedish text corpus, revealing that in 36\% of instances, pronouns were used to refer to third-person human referents. By contrast, only 8\% of pronominal instances referred to non-human referents. However, distinguishing between the effects of subjecthood and animacy in such a corpus study is challenging. To clarify, it is necessary to determine whether the increased frequency of pronominalization is attributable solely to animacy, or whether animate referents tend to occupy the subject position, which inherently carries a greater likelihood of pronominalization.

To unravel these effects, \citet{fukumura2011effect} investigated the impact of animacy on RF choice through a series of controlled story-completion experiments. 
Regarding \REF{ex:hiker1}, the study discovered that speakers tended to use pronouns for animate referents, specifically \intext{the hikers}, more frequently than for inanimate referents, like \intext{the canoes}.
This pattern persisted even when the conditions were reversed, as in \REF{ex:hiker2}, where the positions of the NPs were switched: speakers more frequently used pronouns for animate objects over inanimate subjects.
Consequently, the influence of animacy is distinct from that of the grammatical role.
The elevated rate of pronominalization for animate referents suggests their greater conceptual prominence in the speaker's mind. 
As a result, the speaker needs to retrieve less semantic content to refer to them \citep{fukumura2011effect,vogels2014referential}.

\begin{exe}
	\ex 
	\begin{xlist}
		\ex The hikers carried the canoes a long way downstream. Sometimes, ... \label{ex:hiker1}
		\ex The canoes carried the hikers a long way downstream. Sometimes, ... \label{ex:hiker2}
	\end{xlist}
\end{exe}

\citeauthor{fukumura2011effect} also investigated whether pronominalization rates were affected by the presence of animacy-congruent competitors in the previous sentence. 
They observed a decline in the likelihood of pronominalization when both the referent and its competitor were animate. However, this effect did not occur among inanimate referents.

The diminished use of pronouns when animate competitors are present may be partially attributed to the semantic competition described in \sectref{subsec:competition}.
This explanation is partial because animacy congruence affects only the pronominalization ratio for animate referents. The presence of an animate competitor diminishes the prominence of the animate target referent. Consequently, the speaker employs more explicit forms to activate the referent’s representation. This effect is absent in congruent pairs of inanimate referents.


\subsection{Recency}\label{subsec:recency}
Recency is defined as the distance between the current mention of a referent and its antecedent. The larger the distance between the two mentions, the more likely it is to use a full noun phrase anaphora \citep{vonk1992use,givon1992grammar,Arnold2010}. Conversely, the smaller the distance between the two mentions, the more likely it is to use pronouns. This section outlines the three most common interpretations of recency employed in linguistic and computational studies.

\subsubsection{Immediate context} 
Research focusing on the occurrence of pronominal forms typically defines \term{short distance} or \term{immediate context} as scenarios where the antecedent appears in the same sentence or is separated by just one sentence. On the other hand, if the antecedent is positioned more than one sentence away, it is categorized as \term{long distance} \citep{hobbs1978resolving, ariel1990accessing, Hitzeman1998, poesio2004centering}.

The corpus analysis conducted by \citet{hobbs1978resolving} revealed that in 98\% of instances, the antecedent of a pronoun anaphora is found in either the preceding or the same sentence. Similarly, \citet{ariel1990accessing} investigated the distribution of pronouns, proper names, descriptions, and demonstratives in a corpus analysis and found that in over 80\% of the cases, pronouns exhibit a preference for short distances, meaning, antecedents located within the same sentence or just one sentence away.

\subsubsection{Non-local context} 
Different lines of study examine recency in a broader context. These studies incorporate the concept of \term{non-local context}, that is, a larger span of text, in their definition of recency. 
In a comprehensive study of topic continuity in discourse by \citet{Givon1983}, the measurement of distance to the previous mention extended as far back as 20 clauses. This research represents one of the initial efforts to quantify the role of distance in discourse.
\citet{mccoy1999generating} proposed in a computational pronominalization study that ``when the last mention of an item is several sentences back in the text, a definite description is preferred" (p~64). Using a corpus of New York Times articles, the study revealed that definite descriptions were nearly always used in long distance situations.

\subsubsection{Unit boundary} While the distance patterns described in the previous paragraphs can explain many instances of pronominalization, \citet{Fox1987} argues that these patterns do not encompass all varieties of anaphoric references.  The study by \citeauthor{Fox1987} demonstrates that pronouns can be used to refer to a referent over long stretches of distance until the goal of a narrative changes \citep[cited in][]{Smith2003}. Building on this idea, \citet{ariel1990accessing} introduces the concept of \textit{unity}, defined as an antecedent existing within the same frame, segment, or paragraph. Additionally, \citet{vonk1992use} and \citet{Tomlin1987} highlight the significance of \term{episode} and \term{unit boundaries}, typically regarded as \term{paragraph boundaries} in written texts, as contributing factors to the principle of recency.
In summary, this section has elucidated three distinct interpretations of recency. The first two involve measuring distance in sentences or clauses, whereas the third interpretation extends beyond the sentence level, emphasizing paragraphs.


\section{Summary and discussion}
\begin{sloppypar}
This chapter introduced several theories concerning the choice of REs. The Givenness and Accessibility hierarchies offer cognitive explanations for how the cognitive status and accessibility of discourse referents' mental representations affect this choice. However, a limitation of these cognitive theories is their failure to account for the relational properties of discourse referents. 
\end{sloppypar}

In addition to these two dominant theories, Centering Theory elucidates various pronominalization decisions by associating the choice of REs with the coherence of discourse. However, CT in its initial form adopts a more localized perspective and overlooks the global context in the choice of referring expression forms. 

A more recent approach, Prominence Theory, seeks to delineate the choice of REs, incorporating the dynamicity and relational properties of referents. Furthermore, this theory extends the interpretation of REs beyond merely local contexts. Prominence Theory posits that the choice of referential form is shaped by multiple factors, also termed prominence-lending cues. 

In \sectref{sec:lingfactors}, I explored a variety of prominence-lending cues. These cues exhibit differences in multiple aspects. Notably, \citet{PratSala2000} identified two distinct types of accessibility: \intext{inherent} and \intext{derived}. Inherent accessibility originates from the inherent characteristics of a referent and remains unchanged throughout the discourse. This category includes cues like animacy or gender, as a referent's animacy status or gender remains unchanged in a text. Conversely, derived accessibility varies within the discourse and responds to contextual factors. This form of accessibility emerges from the prominence of a referent within the discourse context. An example of this is subjecthood, since a referent does not possess innate subjecthood but becomes a subject in a sentence. A valid question to ask is which interpretation holds greater significance in determining the form of referring expressions.

In this chapter, six important prominence-lending cues have been discussed, along with their various interpretations and implications. I have not discussed other prominence-lending cues such as coherence relations \citep{Hobbs1979,kehler2002coherence} and information status \citep{Lambrecht1994} because they are not discussed in the subsequent chapters. Although I presented and discussed these cues individually, it is likely that a combination of them plays a role in determining the prominence status of the referents. As noted, ``they might weigh in differently in different contexts" \citep[24]{de2015putting}. \chapref{chap5} examines different implementations of these factors in a feature-based \context experiment to assess their importance in predicting the form of REs in context.
