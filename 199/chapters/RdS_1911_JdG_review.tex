\begin{appendices}
\addcontentsline{toc}{chapter}{Reviews of de Saussure 1911}
\chapter*{Reviews of de Saussure 1911}

\lettrine[loversize=0.1, nindent=0.25em]{W}{e know of two reviews} of
René de Saussure's 1911 work, both brought to our attention by remarks
in its successor, the 1919 work. The first of these appeared in the 20
November, 1911 edition of the \textsl{Journal de Genève} on page 3,
signed ``A.O''\footnote{A reviewer suggests that this was probably the
  same individual as the ``A. Oltramare'' whose review of
  \citet{saussure16:cours-original} is cited by
  \pgcitet{r.desaussure19:structure.logique}{4; pg. 140 below}.  The
  scholar in question appears to have been A[ndré] Oltramare
  (1884--1947), a Latinist and Socialist politican in Geneva at the
  time.} and followed a week later in the same newspaper by a brief
exchange between that author and René de Saussure.  The second was a
short note by Antoine Meillet that appeared in volume 18/2 of the
\textsl{Bulletin} of the Société de linguistique de Paris for
1912-1913 (not as indicated by
\pgcitet{r.desaussure19:structure.logique}{5} as appearing in the
society's \textsl{Mémoires} for 1911). We reproduce those reviews
here, with translation.

\FrenchPage[no]{
\section*{Review in \textsl{Journal de Genève}}
\addcontentsline{toc}{section}{Review in the \textsl{Journal de Genève}}
\label{pg:JdG-review}
\textbf{Principes logiques de la formation des mots}, par René
\textsc{de Saussure}, privat-docent à l'Université de
Genève. --- Première partie. --- Genève, librairie Kündig,
1911.\largerpage

Des travaux récents nous ont fait connaître des «formules» de
psychologie, d'économie politique et même d'esthétique; il
fallait prévoir que tôt ou tard les lois de langage seraient de
même exprimées sous une forme mathématique. Pour M. René de
Saussure le problème de la formation des «dérivés» est un
problème de chimie: chaque mot est une molécule; on découvrira
sa nature en le décomposant en ses atomes (radicaux et affixes),
auxquels un axiome complaisant octroie une valeur
invariable. L'ensemble des significations atomiques donnera le sens de
la molécule.

C'est l'analyse traditionelle des laboratoires dans toute son
audacieuse simplicité: Soit le mot \emph{labourage} à étudier
dans l'éprouvette; il se décomposera aisément en un radical
\emph{labour-} et un suffixe \emph{-age}. Pour fixer chacun de ces
deux éléments, on les remplace par des atomes «types» qui
n'expriment que les idées grammaticales (celle d'adjectif: symbole
\emph{a}, celle de verbe: \emph{i}, ou de substantif: \emph{o})
contenues dans \emph{labour-} et dans \emph{-age}. On arrive ainsi à
la formule:


{\centering
  \emph{Labour-age} = \emph{ac-tion} (de l'espèce particulière
  \emph{labour-}) = (\emph{i-o})
\par}


La synthèse n'est guère plus compliquée. Étant donnée une
idée complexe comme \emph{action de labourer,} on cherche le mot qui
l'exprimera: il appartiendra au type moléculaire: (\emph{i-o}); je
connais le radical verbal à employer(\emph{i}); je choisis un
suffixe de substantif (\emph{o}) et je forme: \emph{labourage}. Si
l'idée à désigner était nouvelle, le mot serait naturellement
nouveau; tous les inventeurs qui voudront donner à leur découverte
un mot logiquement formé devront, conclut M. R. de Saussure, se
conformer à ces principes.

Il est difficile de porter un jugement sur un ouvrage dont la
première partie est seule publiée; je crois cependant pouvoir
déjà dire que si celui-ci contient dans quelques-unes de ses
digressions plusieurs idées intéressantes et même nouvelles
(principalement sur la position relative des éléments verbaux), il
est faussé dans son ensemble par une erreur très grave de
méthode: certes, la linguistique n'est plus une discipline
exclusivement historique; les ouvrages de M. Bally ont démontré
qu'elle devient en partie une science d'observation actuelle; on a le
droit désormais d'étudier les faits du langage contemporain en les
isolant du passé, d'expliquer par exemple, comment au moyen de
suffixes on crée maintenant les mots nouveaux; ceux-là sont
naturellement pour notre conscience de vrais dérivés: nous sentons
immédiatement que \emph{labourage} équivaut à \emph{action de
  labourer}, car du suffixe \emph{-age} notre génération a formé
\emph{dérap-age, sabot-age}, etc.

Faire la nomenclature des suffixes de ce genre, qui sont
«actuellement vivants», déterminer parmi les moyens de formation
quel sont  ceux qui peuvent être conçus comme des règles
logiques, c'est à cela qu'eût dû se borner l'effort de l'auteur
\ldots\ Hélas! son ambition a été plus haute; il a cru que les
observations qu'il faisait étaient valables pour toutes les
époques du français et nous a donné surtout en exemples des
mots créés dans les siècles antérieurs au moyen de suffixes
qui sont morts aujourd'hui; ces dérivés-là sont «sentis» par
nous comme des mots simples: \emph{beauté} n'est plus pour notre
conscience un dérivé, car l'élément \emph{-té} est remplacé
maintenant comme suffixe créateur de substantifs abstraits par
\emph{-ité}. 
}

\EnglishPage[no]{\section*{~}\textbf{Principes logiques de la formation des mots}, par
  René \textsc{de Saussure}, privat-docent à l'Université de
  Genève. --- Première partie. --- Genève, librairie Kündig,
  1911.

  Some recent works have brought to our attention the «formulas» of
  psychology, political economy and even {\ae}sthetics: it had to be
  expected that sooner or later the laws of language would be
  similarly expressed in mathematical form. For Mr. René de
  Saussure, the problem of the formation of ``derived words'' is a
  chemical problem: each word is a molecule; we will discover its
  nature by decomposing it into its atoms (roots and affixes), to
  which a convenient axiom allocates an invariant value. The
  collection of atomic meanings will give the sense of the molecule.

  This is the traditional laboratory analysis in all its audacious
  simplicity. Let us suppose that the word we have to analyze in our
  test tube is \emph{labourage} `plowing': it will be easily
  decomposed into a root \emph{labour-} `(to) plow' and a suffix
  \emph{-age}. To determine each of these two elements, we replace
  them by ``type'' atoms which express only grammatical ideas (that of
  the adjective: symbolized \emph{a}; that of the verb, \emph{i}; or
  the noun: \emph{o}) contained in \emph{labour-} and in
  \emph{-age}. We thus arrive at the formula:

  {\centering \emph{Labour-age} = \emph{ac-tion} (of the specific type
    \emph{labour-}) = (\emph{i-o})
    \par}
  
  Synthesis is hardly more complicated.  Given a complex idea such as
  \emph{action of plowing,} we look for the word that expresses it: it
  will belong to the molecular type (\emph{i-o}). I know the verbal
  root to use (\emph{i}); I choose a nominal suffix (\emph{o}) and I
  form \emph{labourage}. If the idea to be represented is novel, the
  word would naturally be new: all inventors who would wish to give to
  their discovery a logically formed word will, concludes Mr. René de
  Saussure, have to comply with these principles.

  It is difficult to judge a work of which only the first part has
  been published; I think, however, that I can say that although this
  contains in its digressions several ideas that are interesting and
  even novel (especially concerning the relative position of verbal
  elements), it is distorted overall by a very serious methodological
  error. Certainly, linguistics is no longer an exclusively historical
  discipline: the works of Mr. Bally have shown that it is becoming in
  part a science of observation of the here and now.  We have
  henceforth the right to study the facts of contemporary language in
  isolation from the past, to explain, for example, how by means of
  suffixes we now create new words.  These are naturally for our
  awareness true derivatives: we sense immediately that
  \emph{labourage} is equivalent to \emph{action of plowing,} because
  our generation has formed \emph{dérap-age} `skid-ing',
  \emph{sabot-age} `sabotage-ing', etc. with the suffix \emph{-age}.

  To make a taxonomy of sufixes of this type, which are ``currently
  living'', to determine by the means of formation which are the ones
  that can be conceived as logical rules, it is to this that the
  author's efforts ought to have been limited \ldots\ Alas! his
  ambition has been higher. He has believed that the observations he
  has made were valid for all epochs of French and has given us as
  examples words created in past centuries by means of suffixes which
  are dead today. These derivatives are ``felt'' by us as simple
  words: \emph{beauté} `beauty' is no longer a derivative for our
  intuition, because the element \emph{-té} is now replaced as the
  suffix creating abstract nouns by \emph{ité}. 


}

\FrenchPage[no]{
  
  Toutes les lois que M. René de Saussure découvre avec raison dans
  les formations actuelles tombent à faux quand on veut les
  appliquer à plusieurs époques. L'axiome même de
  l'invariabilité des éléments, sur lequel repose tous ses
  raisonnements, est contredit par le premier exemple venu: le suffixe
  \emph{-age}, dont j'ai parlé plus haut, a changé de sens et
  même de catégorie grammaticale dans l'histoire du français.
  Il était autrefois adjectif (\emph{a}); il formait par exemple
  l'expression de \emph{lait formage} (lait qui prend une forme), d'où
  est venu notre mot \emph{fromage}; il a pris ensuite la
  signification substantive (\emph{o}) de collectif (\emph{herbage,
    feuillage}) pour devenir enfin aujourd'hui un créateur de nom
  d'action (\emph{labourage}).

  Si son objet avait été plus logiquement délimité,
  l'opuscule de M. René de Saussure eût été très utile aux
  linguistes et aux inventeurs. Conçu comme il l'est, il semble
  avoir comme but de prouver qu'une langue naturelle comme le
  français forme ses mots de la même manière qu'un langage
  artificiel comme l'esperanto et ses succédanés.

\begin{flushright} {A. O.}
\end{flushright}
}

\EnglishPage[no]{ All of the laws which Mr. René de Saussure rightly
  discovers in current formations break down when we try to apply
  them to several epochs. Even the axiom of the invariability of
  elements, on which all of his reasoning rests, is contradicted by
  the first example that comes up: the suffix \emph{-age}, which I
  spoke of above, has changed its sense and even its grammatical
  category in the history of French. It was formerly an adjective
  (\emph{a}); it formed for example the expression \emph{lait formage}
  (milk which takes a shape), from which comes our word \emph{fromage}
  `cheese'; it subsequently took on the nominal meaning (\emph{o}) of
  collective (\emph{herbage} `pasture', \emph{feuillage} `foliage'),
  finally becoming today the creator of action nouns (\emph{labourage}
  `plowing').

  If his objective had been more logically delimited, the work of
  Mr. René de Saussure would have been very useful for linguists and
  inventors. Conceived as it is, it seems to have as its end to prove
  that a natural language like French forms words in the same way as
  an artificial language like Esperanto and its alternatives.

  \begin{flushright} {A. O.}
  \end{flushright}
}

  \FrenchPage[no]{
    \begin{center}
      {\Large Response by R. de Saussure,
        \textsl{Journal de Genève} 27 November, 1911}
    \end{center}\largerpage
    \noindent\textbf{Principes logiques de la formation des mots,} par
    \textsc{René de Saussure}. --- (Genève, Kündig).
    
    Sous la signature A. O. le \emph{Journal de Genève} a donné de
    cet ouvrage un compte rendu à propos duquel une courte réponse
    sera permise à l'auteur. Celui-ci peut estimer, en effet, que
    cette critique lui attribue des idées qu'il n'a point émises,
    ou qui sont même directement contraire au sens de sa
    brochure. J'ai peine à m'expliquer le malentendu, quoique ayant
    eu tort peut-être de ne pas prendre toutes les précautions
    pour qu'une méprise sur le point de vue choisi fût impossible.

    «Si cet ouvrage, dit M. A. O., contient plusieurs idées
    interessantes et même nouvelles, il est faussé dans son
    ensemble par une erreur très grande de méthode\ldots\  Toutes
    les lois que M. René de Saussure découvre avec raison dans les
    formations actuelles (de mots) tombent à faux quand on veut les
    appliquer à plusieurs époques. L'axiome même de
    l'invariabilité des éléments, sur lequel reposent tous ses
    raisonnements, est contredit par le premier exemple venu.»

    Le malentendu est flagrant, vu qu'à aucun moment les principes
    logiques que la brochure cherchait à poser n'ont prétendu
    s'appliquer à une succession d'époques, diverses dans le
    temps.  Déjà le mot de \emph{logique}, et le fait que l'essai
    se déroule dans le point de vue logique pur, excluraient une
    pareille supposition, qui est d'ailleurs écartée explicitement
    au premier paragraphe, à propos de la valeur étymologique des
    mots simples déclarée indifferente. L'\emph{invariabilité}
    des éléments n'est pas relative au temps, mais aux divers mots
    considérés comme coexistants à un moment donné. C'est d'un
    mot de la langue à l'autre, non d'une époque à l'autre,
    qu'elle pose un principe.

    Quant au mot de \emph{fromage}, dont le suffixe sert à donner
    des exemples de ma mauvaise méthode, il n'est pas cité dans ma
    brochure. Considéré au sein de l'époque présente, ce mot
    est évidemment un mot simple (atome substantif), et n'offre, par
    suite, aucune sorte de suffixe.

    Je saisis du reste comme un acquiescement précieux ce mot de
    M. A. O., que toutes les lois énoncés dans l'ouvrage sont
    valables pour les \emph{formations actuelles}. Il ne m'en faut
    davantage. Si ces lois sont valables pour les formations
    actuelles, c'est dire en effect qu'elles ont toute chance de
    l'être pour une époque donnée quelconque, chose qui reste
    éminemment distincte d'une pluralité d'époques, avec
    l'évolution qu'elle comporte. Ainsi, une fois levée
    l'équivoque initiale qui obscurcissait le début, je crois
    pouvoir dire que mon honorable critique se trouve plus près
    qu'il ne pense lui-même de donner raison à ce qui forme le
    vrai fond de l'ouvrage qu'il condamne.\\
      \hbox{}\hfill\hbox{R.S.}
    \begin{center}
      ---
    \end{center}
    L'auteur de l'article, auquel répond M. R. de Saussure, nous
    écrit:

    «M. René de Saussure déclare que ses «Principes logiques»
    n'ont jamais prétendu s'appliquer à diverses
    époques. J'avais parfaitement compris son intention: c'est
    pourquoi je lui ai reproché d'avoir emprunté des exemples à
    toutes les phases de l'évolution linquistique au lieu de se
    borner à étudier les procédés actuels de formation
    verbale.» }

  \EnglishPage[no]{\vspace*{2\baselineskip}\largerpage

     \textbf{Principes logiques de la formation des mots,} par
    \textsc{René de Saussure}. --- (Genève, Kündig).
    
    Under the name A. O. the \emph{Journal de Genève} has given this
    work a review with regard to which a short reply will be permitted
    to the author. The latter may consider, indeed, that this critique
    attributes to him ideas that he has not at all put forth, or which
    are even directly contrary to the sense of his booklet. I find it
    hard to explain the misunderstanding to myself, despite perhaps
    having been wrong not to take all precautions that a mistake
    concerning the point of view chosen should have been impossible.

    ``If this work'' says Mr. A. O., ``contains several ideas that are
    interesting and even novel, it is distorted overall by a very
    serious methodological error \ldots\ All of the laws which
    Mr. René de Saussure rightly discovers in the current formation
    (of words) are falsified when we wish to apply them to several
    epochs. Even the axiom of the invariability of elements, on which
    all of his reasoning rests, is contradicted by the first example
    that comes up.''

    The misunderstanding is blatantly obvious, since at no point were
    the logical principles that the booklet sought to present claimed
    to apply to a succession of epochs, diverse in time. Already the
    word \emph{logical}, and the fact that the essay is developed from
    the point of view of pure logic, would exclude such a supposition,
    which however is explicitly dismissed in the first paragraph,
    where the etymological value of simple words is declared
    irrelevant. The \emph{invariability} of elements is not relevant
    to time, but to the various words considered to coexist at a given
    moment. It is from one word of the language to another, not from
    one time period to another, that the principle applies.

    As for the word \emph{fromage,} whose suffix serves to provide
    examples of my bad method, it is not mentioned in my
    booklet. Considered within the present period, this word is
    obviously a simple word (nominal atom), and does not present, in
    consequence, any sort of suffix.

    I take as a valuable acknowledgement, however, Mr. A. O.'s remark
    that all of the laws set out in the work are valid for
    \emph{current formations.} No more is necessary for me. If these
    laws are valid for curent formations, that is to say indeed that
    there is every chance they will be so for any given period,
    something that remains entirely distinct from a plurality of
    periods, with the evolution that this involves. Thus, once the
    initial misunderstanding is removed which obscured the starting
    point, I think I can say that my honorable critic finds himself
    closer than he thinks to agreeing with that which forms the true
    basis of the work he condemns.

    \begin{flushright}
      {R.S.}
    \end{flushright}

    \begin{center}
      ---
    \end{center}

    The author of the article to which Mr. R. de Saussure replies
    writes to us:

    ``Mr. René de Saussure declares that his ``Logical principles''
    were never intended to apply to different periods. I have
    understood his intention perfectly well: that is why I have
    reproached him for having borrowed examples form all phases of
    linguistic development rather than limiting himself to the study
    of current processes of verbal formation.''
    
  }

    \FrenchPage[no]{
      \section*{Antoine Meillet's Review in the \textsl{Bulletin de la
          Société de linguistique de Paris} (vol. 18/2 {[1912-1913]}, pp. xxii-xxiii)}
    \addcontentsline{toc}{section}{Meillet's review in the
      \textsl{Bulletin de la Société de linguistique de Paris}}
    \noindent
    René \textsc{de Saussure}. — \emph{Principes logiques de la
      formation des mots}. Première partie. Genève (chez Kündig),
    1911, in-8 ; 122 p.\\

    M. René de Saussure n’est pas linguiste de profession, et c’est
    l’étude de l’espéranto qui l’a conduit à examiner les principes de
    la formation des mots. Il se préoccupe, non de ce qui est, mais de
    ce qui doit être. Presque chacun des principes qu'il pose est en
    quelque mesure en contradiction ou au moins en désaccord avec les
    faits positifs des langues naturelles. Un mot n’est pas proprement
    le symbole d’une idée, mais un signe phonique associé à un
    ensemble complexe de faits psychiques de toutes sortes. Ce qui
    fait qu’une notion est représentée par un mot simple ou un mot
    composé n’est pas le degré de complexité qu’elle présente, c’est
    le caractère plus ou moins familier de la notion : quand on a
    nommé pour la première fois le bateau à vapeur, on lui a donné une
    désignation complexe; maintenant, on l’appelle volontiers un
    vapeur. L’analyse de M. René de Saussure porte la plupart du temps
    sur le français, auquel elle s’applique souvent mal. Par exemple,
    il distingue, p. 60, dans \emph{agir}, un radical \emph{ag} et un
    suffixe \emph{ir}; or, jamais un Français ne saurait analyser
    ainsi \emph{agir}; rien ne lui permet d’isoler \emph{ag} , et
    -\emph{i}- se trouvant dans toutes les formes du verbe en fait
    partie intégrante; l’abstrait \emph{action} n’est pas dérivé de
    \emph{agir}; c’est un mot qui, en latin, appartenait au groupe de
    \emph{agere}, mais qui, au point de vue français, en est en somme
    indépendant. Si \emph{agir} a en français un dérivé abstrait,
    c’est \emph{agissement}. \emph{Action} n’est pas plus dérivé
    d'\emph{agir} que \emph{qualité} ne l’est de \emph{quel}. — Si
    l’on veut bien faire abstraction et des langues telles qu’elles
    sont et de leur passé, le petit livre de M. René de Saussure fera
    réfléchir utilement sur les rapports qui existent entre les mots
    simples et leurs dérivés ou leurs composés.
    \label{sec:meillet-review}

   
    \begin{flushright}
      \textsc{A. Meillet.}
    \end{flushright}

    \begin{figure}[b]
      \begin{center}
        \includegraphics*[0,0][1.758in,2.5in]{./Photos/Meillet.eps}

        {[Paul Jules]} Antoine Meillet (1866--1936)
      \end{center}
      
        
    \end{figure}


    }

    \EnglishPage[no]{
      \noindent
    René \textsc{de Saussure}. — \emph{Principes logiques de la
      formation des mots}. Première partie. Genève (chez Kündig),
    1911, in-8 ; 122 p.\\

    Mr. René de Saussure is not a linguist by profession, and it is
    the study of Esperanto that has led him to examine the principles
    of word formation. He is concerned not with what is, but with what
    ought to be. Nearly every one of the principles that he asserts is
    in some way in contradiction or at least in disagreement with the
    empirical facts of natural languages. A word is not properly the
    symbol of an idea, but a phonic sign associated with a complex set
    of psychic facts of all sorts. What causes a notion to be
    expressed as a simple word or as a compound word is not the degree
    of complexity that it presents; it is the more or less familiar
    character of the notion. When a name was given for the first time
    to the steam boat, it was given a complex designation; now, we
    happily call it a steam{[er]}.  Mr. René de Saussure's analysis
    bears most of the time on French, to which it is often poorly
    applicable. For example, he discerns, p. 60, in \emph{agir} `to
    act', a root \emph{ag} and a suffix \emph{ir}; now no Frenchman
    would analyze \emph{agir} in that way; nothing allows him to
    isolate \emph{ag}, and since -\emph{i}- is found all of the forms
    of the verb, it constitutes an integral part of it.  The abstract
    noun \emph{action} is not derived from \emph{agir}; it is a word
    which in Latin belonged to the family of \emph{agere}, but which,
    from the point of view of French, is quite independent. If
    \emph{agir} has a derived abstract noun in French, it is
    \emph{agissement}. \emph{Action} is no more derived from
    \emph{agir} than \emph{qualité} is from \emph{quel}. --- If one is
    willing to abstract away both from languages as they are and from
    their past, the little book of Mr. René de Saussure will provoke
    useful reflection on the relations that exist between simple words
    and their derived forms and compounds.

    \begin{flushright}
      \textsc{A. Meillet.}
    \end{flushright}

    
      }
\end{appendices}
