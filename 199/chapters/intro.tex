%%% -*- Mode: LaTeX -*-

\currentpdfbookmark{Contents}{name} % adds a PDF bookmark
{\sloppy\tableofcontents}
  
\newpage
\thispagestyle{empty}
\begin{figure}
  \includegraphics[width=.9\textwidth]{./Photos/Rene6.eps}
  \begin{center} {René de Saussure}
  \end{center}
\end{figure}
\newpage

\chapter{Introduction}
\label{ch.intro}
\begin{refsection}
\lettrine[loversize=0.1, nindent=0.25em]{I}{n August, 2014,} while
going through the library of his late father Antoine de Saussure (son
of Louis-Octave de Saussure, a younger brother of Ferdinand and Ren\'e
de Saussure), Louis de Saussure discovered a little book of 122 pages
entitled ``Principes logiques de la formation des mots,'' written in
1911 by his great-uncle René and obviously dealing in a general way
with morphology.  René de Saussure was (as discussed below) an
engineer and mathematician, not a linguist like his brother
Ferdinand. Although he was active in the Esperanto movement in the
early years of the 20th century, and wrote on issues concerning the
adoption of this proposed international language as discussed in
\sectref{sec:esperanto} below, he has not been known for the
relevance of his work to topics in general linguistics.  The book in
question seems in particular to have escaped the attention of
linguists of the time and later; and indeed, indications of its very
existence in the catalogs of major research libraries are quite rare.

While the 1911 book identified itself as the ``first part'' of a
projected work, no second part was ever written as such. In
1919,\footnote{Some confusion is produced by the fact that while this
  work identifies itself as published in 1919 by ``Librairie
  A. Lefilleul, Christoffelgasse, Berne,'' and the text is signed with
  the date 17 March, 1919, the title page shows it as having been
  printed in 1918 by ``Imprimerie B\"uchler \& Cie, Berne.''  As noted
  below on page~\pageref{page:date.1919}, the internal evidence argues
  that the book should be referred to by its 1919 publication date.}
however, René de Saussure published a further work of 68 pages, ``La
structure logique des mots dans les langues naturelles,
consid\'er\'ee au point de vue de son application aux langues
artificielles'' \citep{r.desaussure19:structure.logique}, including an
initial chapter on much the same topic.  While the 1911 work makes no
reference to other writings by linguists (such as the author's
brother), the 1919 book was composed after the appearance of Ferdinand
\posscitet{saussure16:cours-original} \emph{Cours de linguistique
  générale}, and cites Ferdinand de Saussure's views on general
linguistics in places, including a brief but illuminating passage
contrasting two possible theories of word structure
\pgcitep{r.desaussure19:structure.logique}{27--28} which will be
explored below in \sectref{sec:morph-theory}.\largerpage[-2]\pagebreak

Both volumes -- and especially the second -- must be seen
as motivated by Ren\'e's concerns for the design of Esperanto, but their
basic premise is that this can only be carried out rationally on a
foundation of understanding of the workings of natural languages. As a
result, the theoretical framework and general principles proposed
should be viewed as a contribution to general linguistics, and not
solely in terms of their implications for artificial languages.

\begin{figure}[p]
  \begin{center}
    \includegraphics[width=\textwidth]{./Photos/ExportRene8.jpg}
  \end{center}
  \caption{\label{fig:rene8}
  René de Saussure as a child, as painted by his uncle
    Théodore (1824--1903). Courtesy of the de Saussure family.}
%\end{center}
\end{figure}

\section*{René de Saussure}
\label{sec:bio}
\noindent
\lettrine[loversize=0.1, nindent=0.5em]{R}{en\'e de Saussure
  (1868--1943), the sixth child and fourth son} of Henri\linebreak and Louise de
Saussure (née de Pourtalès), was eleven years younger than his brother
Ferdinand. A mathematician and engineer, he is best known as a
prominent figure in the Esperanto movement in the early years of the
twentieth century (see \sectref{sec:esperanto} below).

He did his undergraduate studies at the École Polytechnique in Paris
from 1887 to 1889 before moving to the U.S.A. where he received a PhD
from John Hopkins University (Baltimore) in 1895. He was appointed
Professor of mathematics at the Catholic University of America at
Washington D.C. in 1896 and held this position until 1899 when he came
back to Switzerland. He then held positions at the Universities of
Geneva and Berne.

During his American years, while he studied mathematics, René de
Saussure ran a firm of architects in Virginia with a friend of his and
with the partnership of his older brother Horace, a painter. The firm
was successful enough to be awarded the building of a musical
auditorium, but the partnership did not last
\pgcitep{joseph12:saussure}{390,391}. In 1892 he married Jeanne Davin,
an American Roman Catholic woman, and obtained American
citizenship. The marriage was tragically ended by Jeanne’s death in
1896, at the age of only 24. In 1898, René married Catherine Maurice,
from Geneva, who came to live with him in the United States. But a new
tragedy was soon to occur: she died after giving birth to their son
Jean in April 1899. René then immediately resigned from his position
at the Catholic University and came back to Geneva with the baby. He
married later for the third time, to Violette Herr from Zurich, who
gave birth to another son, Maxime.

His interest in the development of science in America was at the time
an original move in the family -- his brother Ferdinand was himself
mostly connected to the German and French academic worlds -- but also
an indication of the openness of his intellectual environment towards
new horizons, already previously shown by various members of the
family.

Another brother of René and Ferdinand, Léopold, obtained French
citizenship (taking advantage of a right granted to members of
families who had emigrated during the wars of religion) and became an
officer in the French navy. This duty led him to sail in the far East
and in particular to China where he became interested in Chinese
astronomy and its relation with Western views, as well as the Chinese
language, eventually answering questions about Chinese that Ferdinand
would ask him in letters. René and Léopold were close to each other in
childhood and even `invented' their own `language', the grammar of
which their older brother Ferdinand tried to crack at the time.

Their father Henri de Saussure, himself a recognized entomologist,
went all the way to Mexico in his youth, participated in the
cartography of the country and studied traditional artifacts. Later
on, Ferdinand’s own son Raymond also lived in the USA during WWII
after having been in a close intellectual relationship with Sigmund
Freud. 

\begin{figure}[t]
  \begin{center}
    \includegraphics*[0,0][3.37in,4.0in]{./Photos/ExportRene5.jpg}
  \end{center}
  \caption{\label{fig:rene5}
  René de Saussure (bottom) with (left to right) Leopold
    de Saussure (brother, 1866--1925), Elizabeth Théodora (sister,
    1863--1944), Edmond de la Rive (her husband, 1847--1902), and
    Louise de Saussure, née de Pourtalès (mother, 1837--1906). Photo
    courtesy of the de Saussure family.}
%\end{center}
\end{figure}

René, Ferdinand and the other members of the family were raised in a
family with a solid scientific background, tracing back at least to
the geologist, meteorologist and alpinist Horace-Bénédict de Saussure
in the 18th century. Horace-Bénédict was among the major discoverers
of hercynian folding in geology \citep{carozzi89:saussure-theory-of-earth},
and his grand-son, the biochemist Nicolas-Théodore was a pioneer in
research on photosynthesis. The family provided an environment with a
strong incentive to creative thinking and adventurous exploration,
certainly qualities to be found in the works of both Ferdinand and
René, however opposite the directions they may seem to have taken.

After returning to Switzerland, René taught at the University of
Geneva from 1904 to 1910. During this time Ferdinand was also in
Geneva, appointed as Adjunct Professor in 1891 and as a full Professor
in 1896 following a long teaching career in Paris. Ferdinand gave his
famous three courses in general linguistics from 1907 to 1911, thus at
a time when the interaction with his younger brother was facilitated
by the circumstances. It is likely that René and his famous elder
brother pursued an ongoing interaction about language, in the
fundamental structure of which both were so much interested;
\pgcitet{joseph12:saussure}{539} for example speculates that René
discussed the notion of arbitrariness with Ferdinand in the context of
the invention of the Esperantist currency spesmilo and in relation to
Ferdinand’s famous analogy between language and money as social
institutions. It is also clear that René and Ferdinand had a number of
occasions to exchange views on Esperanto, in particular regarding the
question of whether its artificial nature as a non-native language
would preserve it from the usual movements of language diachronic
evolution.

Ferdinand was not always interested, however, in exchanging ideas with
his mathematician brother. In a letter of 1895 to Ferdinand, René
complains: ``I wish however that we could exchange sometimes some
ideas, even though our domains are so different from one
another. Sometimes not so bad ideas can be suggested by someone
working in a different domain hence conceiving of things from another
perspective.'' They had in fact already exchanged some intellectual
correspondence in a number of letters, but in them they discussed
mathematics and physics, not language, and Ferdinand seems rather to
be lecturing his younger brother about epistemology. A letter by René
dated 1890 shows him responding at length to criticisms by Ferdinand
about René’s hypotheses on a fourth dimension of matter. Whereas René
seems to take the discussion to the level of abstract thought
experiments, Ferdinand delivers more concrete, empirically anchored
arguments. For example, when René explains that a third dimension
would be unimaginable to a two-dimensional being, as an illustration
of why a 4th dimension may be unimaginable to us, Ferdinand replies
that no such being can actually exist.

One might venture to suggest that René’s book on morphology was
triggered by a desire to oppose Ferdinand’s holistic early
structuralist view with the help of mathematical, compositional
principles and formal arguments, so that their brotherly debate would
reach the scientific community outside the closed doors of family
discussion. It is noticeable that when René’s first book was published
in 1911, Ferdinand was just then concluding the delivery of his famous
lectures on General linguistics, before he became ill and passed away
in 1913. His \textsl{Course in General Linguistics}
\citep{saussure16:cours-original} was only posthumously reconstructed
and published in 1916, at which point it -- though not its author --
was available to René in the preparation of his 1919 continuation of
the 1911 work. Whether the debates were fierce between the brothers or
not is not known, but they are likely to have been so.

This being said, it might be that René’s knowledge about the then
recent developments in the Mathematical sciences in relation to
philosophy actually did influence Ferdinand’s conception of
language. In an 1890 letter, René mentions a new treatise on physics
\citep{Stallo1882:physics}, on which he comments in details in his own
works; it is noteworthy that Stallo develops a conception of physics
based on relations of `identity and differences' and a philosophy
where objects are known through their mutual relations only
\pgcitep{joseph12:saussure}{367}, all of which which will sound quite
familiar to anyone aware of Ferdinand’s theory of value.
	
René’s enduring involvement in the Esperantist movement even led him
to teach a course at the University of Geneva in 1910 on the ``History
of the international language movement from Descartes and Leibnitz to
Esperanto'' \pgcitep{joseph12:saussure}{566}.

From 1920 to 1925 René was a professor at the University of Berne. In
1934 he was nominated as the official representative of American
universities during the celebrations of the University of Berne’s
jubilee. The same year, René was awarded a doctorate \emph{honoris
  causa} from the Faculty of Sciences in Geneva for his contribution
to the geometry of movement, work which had also been recognized by a
prestigious French prize in geometry in 1917. According to
M. E. Briner, Dean of the Faculty at that time, René de Saussure
``addressed geometry of movement from a new, original and fruitful
perspective.''\footnote{\textsl{Journal de Genève}, 17 March, 1943.} A
review of his work on the geometry of movement \citep{bricard10:rene}
is enlightening in terms of method: just as in his treatment of
morphology, René de Saussure develops a novel theory where only a
limited number of parameters (actually, five parameters) may enter
into the calculation of the forms of an object in space, but more
importantly he proposes a number of ``conditions'' to which the solid
object in movement is subjected. As a result, his theory, developed in
the published version of his thesis on metageometry (in 1921), allows
for relatively simple calculations of movement based on a number of
dimensions besides mass, time, and energy.
\pgcitet{joseph12:saussure}{366} suggests that René’s research on the
boundaries of physics and geometry prefigures Einstein’s subsequent
Theory of Relativity.

It is apparent that René de Saussure’s work was very creative, even
though it did not lead to significant continuations.  At the same
time, he was very concerned with the aim of finding the commonalities,
and therefore the universality, of the various domains of geometry,
his specialty -- movement --– being conceived as a mere extension of
`classical' geometries. Perhaps the search for universal grounds,
i.e. the essentialist perspective, is what unites the two brothers'
remarkable minds, despite the clearly different perspectives they
adopt on language, one from a scholar originally specialized in the
history of languages and the other from a mathematician.

\section*{The present volume}

René de Saussure's works on word formation present a number of points
of interest, partly for general historical reasons and especially for
an understanding of the history of theorizing about the analysis of
words within modern linguistics. Neither has been made available
previously in English, and even the French originals are difficult to
obtain. The present volume contains the original French
texts\footnote{Pages 29--68 of
  \citet{r.desaussure19:structure.logique} are devoted to the
  application of de Saussure's ideas to artificial languages, followed
  by a description of the grammar of a variatnt of Esperanto written
  in that language, and these sections of the work are not included
  here.} and two reviews of the 1911 volume, with English translations
(by S. R. Anderson), preserving the original pagination and (so far as
possible) typography. These are followed by commentaries on some
interesting aspects of the work and its history: discussions of the
background of this work in René de Saussure's involvement with the
design of the international auxiliary language Esperanto (by Marc van
Oostendorp), and of the morphological and semantic theories (by
Stephen R. Anderson and Louis de Saussure, respectively) that underlie
the texts.\pagebreak

PDF copies of scanned images of the two original works (including the
portions of \citealt{r.desaussure19:structure.logique} not included
here) have been deposited in the Zenodo online archive, and can be
consulted at \url{https://doi.org/10.5281/zenodo.1217635}.

We are grateful to the de Saussure family for their permission to
reproduce the photographs used here as the frontispiece and
Figure~\ref{fig:rene5}, and the oil painting of René de Saussure as a
child in Figure~\ref{fig:rene8}. We are also grateful to Prof. David
Pesetsky for locating and photocopying the copy of
\citealt{r.desaussure19:structure.logique} from which the edition in
Part~\ref{ch.1919text} was prepared, and to Prof. S. Jay Keyser for
having donated this to the MIT library. Anonymous referees for
Language Science Press and also for other publishers who considered
early versions of our project provided useful comments which we have
attempted to incorporate, as did Prof. Thomas Leu, who read a more
recent version of the manuscript.

\sloppy
\printbibliography[heading=subbibliography,notkeyword=this] 
\end{refsection}

%%% Local Variables: 
%%% mode: latex
%%% TeX-master: "./RdS_Morphology.tex"
%%% End: 
