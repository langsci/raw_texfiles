\documentclass[output=paper]{langsci/langscibook} 
\author{Louis de Saussure\affiliation{Université de Neuchâtel}}
\title{The theory of meaning in René de Saussure's works} 
\ChapterDOI{10.5281/zenodo.1306472} %will be filled in at production
\abstract{\noabstract}
\maketitle

\begin{document} 
 \label{sec:semantics} 
\lettrine[loversize=0.1, nindent=0em]{F}{or a contemporary reader,} 
 the works of René de Saussure (henceforth simply ``René'') presented in
 the present volume are strongly suggestive of certain trends in
 modern syntactic theorizing.  Even morphologically simple words are
 resolutely decomposed into structured combinations, reminiscent of
 the approach of Generative Semantics in the 1970s and some more
 recent Minimalist work. The positing of component atoms of meaning
 with no content beyond that of the basic lexical categories suggests
 a parallel with the ``little \emph{v, n, a}'' elements of some
 current syntactic analyses. Pursuing these similarities, though,
 would be rather anachronistic, since René de Saussure's views on word
 structure are not articulated against the background of a theory of
 syntax above the word level. In the present commentary, therefore,
 the focus will be more narrowly on the relation between words (simple
 and complex) and their meanings.

In short, René de Saussure’s theory of lexical meaning is
compositional in a radical reductionist sense on which all words are
reducible to a composition of minimal ``atoms''.  These atoms are
themselves realizations of three overarching categories signifying
\emph{being}, \emph{property} (quality) and \emph{action}. These basic
notions are grammatical ones, as they map onto the three categories of
noun, adjective and verb; but they are ultimately semantic in essence,
and are labelled grammatical ``ideas''. This assumption that grammatical
categories are fundamentally meaningful in themselves, and not mere
abstract unsaturated slots to be filled with content, is a central
claim of the theory and quite an original one. Elements of different
types combine together to form new meanings: therefore, a word may be
about a substance but nonetheless incorporate a verbal notion, as in
\emph{couronnement} `coronation’ which is a noun but still embeds a
notion of action indicated by a particular morph or \emph{atom} (`e’)
which is verbal in essence. More precise semantic contents are various
add-ons which build upon the results of the combination of basic
semantic categories.

René’s approach can be linked to more classical views on language
tracing back to the Stoics and is opposed in many ways to his brother
Ferdinand’s systemic view.  However it is also reminiscent of
structuralist approaches to such semantics as
\posscitet{hjelmslev43:prolegomena} functional conception. The first
text \citep{r.desaussure11:formation} is the most developed but the
second one \citep{r.desaussure19:structure.logique} adds very valuable
elements of precision, elaboration and sometimes amendments to the
original theory.
  
René’s conception of language in general is oriented towards
reductionism and universality. Although his assertions in the 1911
book are not claimed to be universally valid (he admits implicitly
that his work might not apply to all languages), it has an overall
universalist orientation, and therefore we expect him to assume some
universalism in semantics in particular. In the 1919 text, he
clarifies his ambitions to adopt the ``international'' point of view on
natural languages in order to find generalities (later on to be put to
work in the construction of artificial languages like Esperanto)
\citep[5]{r.desaussure19:structure.logique} and he actually
concludes with a strong claim about the universality of the few basic
overarching semantic-grammatical categories that he identifies
\citep[26]{r.desaussure19:structure.logique}.

\section{The theory of meaning in the scientific context}
\label{sec:theory-of-meaning}

René claims at the beginning of his book: “the logical principles of
the formation of words are thus the same for all languages” but
further adds a nuance: ``\ldots or at least for all [languages] that
start from the same primitive elements”
\citep[4]{r.desaussure11:formation}. Unfortunately, he does not
elaborate any further on the distinction between types of languages in
this respect before looking at some (scarce) crosslinguistic data in
the 1919 text where he strengthens his universalist claim. This is a
conception that departs seriously from the more structuralist and
relativist views that were soon to emerge in the structuralist trend
based on some interpretations of his brother Ferdinand's
(\citeyear{saussure16:cours-original}) \emph{Course in General
  Linguistics}. This has implications at the level of semantics and
places René on the side of the ``naturality of meaning'' hypotheses which
relate categorial language to categorial thought without assuming the
prevalence of language but rather that of mind (thought) as bearing
universal patterns, while Ferdinand’s theory is rather that languages
have irreconcilable semantic systems --- all built, however, upon a
similar and thus natural mechanism, the ``faculty'' of language.

It is clear in the 1911 book that René chooses to ignore linguistic
variation as much as possible, and this attitude is also palpable in
his conception of meaning. His general approach is one of
simplification: he aims at reducing the principles in play to the
strictest possible minimum (in striking similarity with Ferdinand, who
goes as to posit only one overarching axiom, ‘\emph{langue} is a
system of signs’). It is noteworthy that Ferdinand’s views also
disregard linguistic variation, but only at the level of dialectal
differences. René chooses to overlook differences across languages in
general as much as possible. However, in the 1919 text, some contrasts
are made and he takes Ferdinand’s own examples (such as the famous
\emph{mouton} / \emph{sheep} / \emph{mutton}) in order to refine some
of the claims of the 1911 text. Doing so, he explains that general
principles have to be understood precisely as general, providing the
big picture, while all fine-grained differences in the empirical
reality (“in practice”:
\citealt[15]{r.desaussure19:structure.logique}) are to be purposely
ignored.

In a number of respects, his theory reminds one of the Port Royal
grammar \citep{arnauld-lancelot1660:port-royal} and more generally of
the formal, logicist, view of semantics that begins with the Stoics
and which his brother Ferdinand precisely opposed.

As a reminder: for the classical view, generally speaking, one ``sign''
(word or other morphological unit) means unambiguously one idea, and
one grammatical category expresses one and only one particular notion
--- a view generally labelled that of ‘logic-grammar
parallelism’. This conception traces back at least to Augustine’s very
influential works on language (in particular \textsl{De
  Dialectica}\footnote{\protect\citealt{augustine:de-dialectica}.} and
\textsl{De Doctrina
  Christiana}\footnote{\protect\citealt{augustine:de-doctrina-christiana}.})
which themselves refine the Stoics' semiotic theory. In short,
according to the classical approach initiated notably by Augustine,
language is a set of public signs which give a mental access to
otherwise private ``thoughts''.

The Stoics' semiotic theory already assumed that a sign in general is
a perceivable element that triggers an inference of a non-perceivable
one. On that view, the inference from the perceivable to the
non-perceivable is allowed either by causal rules, as when one infers
fire from smoke, or by conventions, as when one infers a thought on
the basis of a perceivable linguistic string. These conventions are
based on general principles that play out on each level of language
compositionality. To linguistic sentences correspond logical
propositions, to ``words'' -- not only words in the conceptual lexicon
but also non-inflectional affixes -- correspond concepts (``conceptions'')
of various kinds depending on the grammatical category. Finally,
functional categories and in particular ``grammatical words'', such as
conjunctions, correspond to types of mental operations.  With regard
to words proper, the logical-grammatical parallelism approach
classifies concepts, or ``conceptions'', mainly as substances (objects),
signified by ``substantive nouns'' (nouns) and as accidents, or
qualities, i.e. properties predicated about the substance
(adjectives).

In this tradition, another important dimension is the reductionist
conception of meaning, where each meaningful combination is actually a
synthesis from more general but also more developed notions. For
example, a verb such as \emph{briller} `to shine' is reducible to an
ontological hidden verb of being, and therefore carries
existentiality: \emph{briller} is a linguistic reduction of a notion
of \emph{être brilliant} `be shining', actually `be [sun-shining]', so
that \emph{Le soleil brille} ‘The sun shines’ actually means ‘The sun
is being sun-shining’.

René, knowingly or not, adopts a similar approach in several respects
and would probably be happy to be labelled a Stoic, which may not be
very surprising given that he was a mathematician. His brother
Ferdinand, on the contrary, fiercely rejected this logical tradition,
at least in the formulation of the \textsl{Course in General
  Linguistics}. There he famously says that the classical conception,
which “supposes ready-made thoughts pre-existing to words”, is a
“simplistic view” \citep[97]{saussure:cours}. Even though it is
usually assumed that Ferdinand de Saussure had no real semantic theory
of his own, he had at least a theory about the source of lexical
meaning: each word or ``sign'' is a mental pairing of a concept with an
\emph{image acoustique}, an imprint of sounds in memory, and the
meaning of that sign results from the complex relations of oppositions
and associations it enters into within the whole system of
interdependent signs -- to be later described as a ``structure'': the
\emph{langue}.\largerpage

For Ferdinand, meaning exists only as it results from ``value'', that
is, from differentiation within the system. This postulate derives
from his thorough knowledge of the history of languages, which he
views as involving breakings off of previous, more general signs, so
that new signs begin to exist through oppositive relations with one
another. On that view, whatever the degree of morphological
compositionality eventually involved (as in the famous example of
\emph{poirier} as composed of \emph{poire} `pear' and -\emph{ier}
`tree'; cf. \sectref{sec:rene-vs-ferdinand} above), the whole
lexical meaning stems from this dynamic of oppositive
relationships. For René, on the contrary, compositionality is central
to the meaning of words, in line with the former idea that words evoke
\emph{ex-ante} ideas and notions. Where Ferdinand insists on the
contrasts between languages, connecting with more cultural and
conventionalist views about meaning and reference, René insists on the
similarities between words across languages or at least across the
languages that he compares (mostly French and German, but also English
and a few others in the 1919 text).

Furthermore, René radically opposes his brother’s view that meaning
depends on the linguistic environment, whether real (``syntagmatic''
relations in Ferdinand’s terminology) or abstract (``paradigmatic'',
associative relations). The following quote, which uses some of his
brother’s own terminology, seems precisely directed against
Ferdinand’s view: “The logical analysis of words is only possible if
the symbols with which we work are invariant elements; thus the sense,
the value of an atom, must depend only on itself and not at all on the
sense or the value of the atoms that surround it”
\citep[8]{r.desaussure11:formation}. While Ferdinand has a top-down
view of meaning, imposed by the idea that the ``value'' of any element
itself involves the whole ``system of signs'' within a given language,
which then differs radically from those of other languages, René has a
bottom-up conception of meaning where blocks with fixed, invariant and
\emph{a priori} meanings are combined together in the making-up of
more complex meanings. He has a clearly analytic methodology which
consists in explaining “compound words by simple words, and simple
words by the basic words”
\citep[21]{r.desaussure19:structure.logique}.

\section{The elements of meaning}
\label{sec:elements-meaning}

His notion of ``atoms'' of meaning which combine together in order to
generate more complex ``molecules'' is interestingly comparable to what
Louis \citet{hjelmslev43:prolegomena} would later propose as a theory
of structural functional semiotics, although this was presented as a
continuation of Ferdinand de Saussure’s pre-structuralism. The
technical notions of structural semantics such as \textsc{semes} (atoms
of meaning or \emph{semantic traits}), \textsc{sememes} (aggregations of
semes corresponding to a lexical item), \textsc{archisememes} (those
particular semes that have overarching value in a semantic paradigm),
seem also to resonate with René’s approach of ``molecular'' meaning, even
though René’s notion of atom is about morphs whereas the structuralist
notion of semantic traits is about abstract concepts, not linguistic
realizations.

However, whereas structural semantics assumes that semantic elements
combine together within a system of oppositions, mirroring structural
phonology, René does not hold that ``atoms'' should exist because of some
system of oppositions relative to one particular language. What René
advocates is rather the idea that ``molecules'' are like sets of ideas,
each of which is realized linguistically by an ``atom''. When none of
these atoms is sufficient to fully express a particular idea, a
morphological composition of the basic atoms is needed. This,
interestingly, is again comparable to some extent to what mainstream
structural semantics would propose.

For René, there are two kinds of words: simple words and compound
words, the latter being either compounds of autonomous words or
morphological constructions involving affixes. Since simple words and
affixes fall within the same general notion of bearers of simple
meanings --- that is, both affixes and words have a single, atomic
meaning --- the two morphological categories reduce to the same
general semantic type. This approach is a typical reductionist
conception of language which ignores cross-linguistic disparities as
much as possible.

The only difference between simple words and affixes is that,
according to René, affixes tend to represent more general ideas than
lexical roots, since, as he ventures to suggest, the more general an
idea, the more frequent it is in discourse, and the more frequent a
meaning, the more likely it is to aggregate in an affix. This is of
course a debatable claim, but it does in a way prefigure some
approaches within the contemporary conceptions of grammaticalisation
and lexicalisation, which suggest a general pattern for the production
by a language of grammatical morphemes on the basis of archaic
conceptual lexical words. 

These atoms, in any case, are considered by René the ‘basic material
for word formation’ \citep[5]{r.desaussure11:formation} and are
assumed to be stable across contexts and also across
compositions. Their individual content is supposedly known and
identified. René insists in a footnote that this principle of
invariability of atoms is not about their form but about their
meaning, but his comments on the formal variation of morphemes shows
substantial ignorance both of the nature of allomorphic variation and
of linguistic diversity \citep[6]{r.desaussure11:formation}.

Just as for the Port-Royal grammarians and for most of the formal
tradition in semantics, meaning according to René is fundamentally
compositional, not inferential. He even dismisses the need for ‘rules
of derivation’ \citep[8]{r.desaussure11:formation} (but the picture
is somewhat different in the 1919 text, see below). However,
intellectually speaking, his approach is organized in a slightly
different way than what existed on the market at his time. Notably,
his view of the way in which meanings combine is an original one. This
is most of all because words, according to him, involve not only
hierarchical meanings (i.e. the meaning of a word includes the meaning
of the superordinate, as \emph{horse} means ‘horse’ but also ‘animal’
and so on\footnote{That René takes the example of \emph{horse}
  ironically echoes Ferdinand's mention \citep[97]{saussure:cours}{97}
  of the word \emph{horse} when he criticizes the classical
  tradition.}). Furthermore, the top-most superordinate meanings are
actually those of abstract grammatical categories \emph{themselves},
breaking into pieces the traditional distinction that opposes
conceptual-declarative and grammatical-functional meaning. More
precisely, for René de Saussure, the meaning of a word has several
layers: an ``explicit'' meaning (the particular idea) and a series of
implicit meanings (the more general ideas)
\citep[10]{r.desaussure11:formation}. Interestingly, the most
general, the most abstract idea is a \emph{grammatical} one. ‘Horse’
involves not only ‘mammal’, ‘vertebrate’, ‘animal’ (which all have to
be substantive, \citealt[13]{r.desaussure11:formation}) but even the
``substantive idea'' itself.

There are three such fundamental, overarching grammatical elements of
meaning: the nominal idea (ontologies), the adjective idea (qualities,
properties) and the verbal idea (states of affairs and (other)
actions). The ``nominal idea'' corresponds to the abstract notion of
``being'' itself \citep[10]{r.desaussure11:formation}; this resonates
again with the philosophical tradition of Port-Royal. Through this
abstract notion of being, which is semantically attached to the
grammatical category as a whole and to all its members, a word like
\emph{horse} entails therefore a meaning of ‘some being’. This
``nominal idea'' is implied not only by concrete real beings such as
‘horse’ but also for abstractions like ``theory'' or ``type''. Therefore,
it is not only the case that substantives indicate one class of
concepts (substances) and adjectives another one (accidents or
qualities), as in traditional grammar, but also, and much more
interestingly, that there is a grammatical notion that semantically
encodes the very notions of being (substance) and of quality, as
top-most overarching semantic elements. And of course, the same holds
for the ``verbal idea''.

He explains: “The nominal idea, the adjectival idea and the verbal
idea are thus ideas completely like other ideas: they are simply those
of our ideas that are the most general and as a consequence, the most
abstract. […] I will call them grammatical ideas”
\citep[12]{r.desaussure11:formation}.

These grammatical ideas are represented in language by a number of
linguistic items, the basic grammatical atoms, which bear only the
function of evoking the grammatical idea. There may be various
linguistics markers of one grammatical idea but still they all encode
that particular grammatical idea. An adjectival suffix, for example,
encodes the grammatical idea of quality (‘hum-\emph{an}’), since all
morphological indicators of a grammatical category encode the semantic
notion behind it. In some cases a particular grammatical item operates
a shift of categorization, as when \emph{riche} ‘rich’ becomes
\emph{un riche} ‘a rich [person]'
\citep[19]{r.desaussure11:formation}. In such cases, as with
\emph{ce} and \emph{le} (see \citealt[28]{r.desaussure11:formation}),
he explains that the determiner is the ``nominalizing atom'', the maker
of substantivity, which might recall a very contemporary view
(introduced in \citealt{abney:thesis}) on which noun phrases are
actually headed grammatically by the determiner, not the noun (in
René's theory the definite determiner is more abstract than the
indefinite: cf. \citealt[27--28]{r.desaussure11:formation}).

\pagebreak The ``adjective idea'' can be found not only in grammatical adjectives
(typically because of a variety of affixes which themselves bear the
``adjective idea''), but also, curiously, in some prepositions, which
René treats as ``dissociated suffixes'', such as French \emph{de} ‘of’ in
\emph{d’homme} ‘of man’ where it ``plays the same role'' as \emph{ain}
in \emph{humain} (i.e. as ‘an’ in ‘hum\emph{an}’). The same property
is attributed to some pronouns, such as \emph{qui} ‘that, which' in
\emph{qui diffère} ‘that differs’ as an equivalent to \emph{différent}
‘different’. Needless to say, the latter claim both recalls Port-Royal
again, but once again with differences: ultimately, in René’s approach,
\emph{qui diffère} and \emph{différent} have the same meaning (but
then what happens to the verbal idea in such a composition is
unclear).

René acknowledges the existence of other types of ``atoms'' which do not
belong to the nominal, adjectival or verbal type: adverbs,
prepositions, etc. However René considers any attempt to classify them
pointless because they do not, according to him, contain any general
ideas of their own, but only particular ones (e.g. time or place
etc. when considering adverbs)
\citep[117]{r.desaussure11:formation}.

\section{Meaning, composition of meaning, and grammar}
\label{sec:meaning-grammar}

These fundamental grammatical ideas have compositional properties. René
considers that “the noun cannot function alone any more than a body
without limbs'' \citep[75]{r.desaussure11:formation} and thus needs
verbal and adjectival atoms in order to construct not only sentences
but even ``molecules''. This suggests that René, unlike Frege, does not
have a referential conception of meaning, since nouns are unlike verbs
in that they can refer without need of anything else, whereas verbs
need arguments to refer to actual eventualities.\largerpage[1.5]

The relationship between grammar and semantics in René’s work is
somewhat ambivalent. On the one hand, all conceptual meanings are
subordinate to some grammatical notion, but, on the other hand, since all
grammatical notions are ideas, the grammar is ultimately a matter of
meaning, and therefore bears a semantic essence. This brings the two
brothers close: Ferdinand, looking at the ``system of signs'', does not
really discriminate between the world of the conceptual and that of
the functional. All these elements are signs with values. For example,
according to Ferdinand, systems with masculine and feminine genders
only differ from those that have a neuter gender because of the mutual
oppositions of these genders inside the system of gender that they
form together.  
 
However, contrary to Ferdinand’s views, for whom \emph{langue} resides
in the community of speakers,\footnote{However Ferdinand’s views are
  more complicated. \emph{Langue} as specific to one particular
  language exists in the corresponding community of speakers and thus
  is perhaps ``external'', but \emph{langue} as a principle of
  organisation is true of all languages and is thus internal, as a
  cognitive feature of the brain.} René clearly holds an
internalist conception, even though unfortunately he does not
elaborate on the ontology of language. Language, for René, is neither a
mere indicator of thoughts (with reference to external objects) as in
Port-Royal grammar, nor a thought-maker as it is in Ferdinand’s
account at the other end of the spectrum. Language for René is rather
an apparatus which imposes its principles on the referential lexicon
and thus belongs itself to the domain of thought, since the grammar
is, ultimately, conceptual.

When René discusses the notion of the ``verbal idea'', he faces a problem
of categorization, since it is not intuitive to unite stative and
dynamic notions in a single general type
\citep[17]{r.desaussure11:formation}. At first, he represents not
one but two quite distinct verbal ideas: ``ag'' for ``active verbs'',
i.e. verbs indicating dynamic aspect, and ``sta'' for stative verbs. But
further on, René elaborates a unifying solution which has the merits of
being already aspectual, years before the seminal works by Otto
Jespersen on that notion outside the Slavic domain in the twenties
(e.g. \citealt{jespersen24:philosophy-of-grammar}).

René's knowledge of physics is clearly recruited to discuss the status
of verbality and of eventualities. He claims that states are a subtype
of dynamic events: “the static idea is however only a special case of
the dynamic idea. The latter implies forces in activity: if the forces
are not in equilibrium, we have the proper dynamic idea (action),
which is the general case; if the forces are in equilibrium, we have
the static idea (state). And indeed as soon the state changes, we fall
back into action”
\citep[32--33]{r.desaussure11:formation}.\footnote{This view, which
  René imports from physics into linguistics, is problematic when it
  comes to the way language represents facts: that states are
  particular instances of non-states does not look right
  linguistically, and René seems to be lacking some more precise
  knowledge of sub-categories. What is more, the complements of an
  atelic verb may turn it into a telic verb phrase, a problem that
  could hinder René’s ``principle of invariability of atoms''
  \citep[21]{r.desaussure11:formation}.}  An issue with this
conception of the ``verbal idea'' could be that ``state'' looks pretty
much like a ``nominal idea'' since it also involves a notion of
being. However, whereas the nominal idea concerns the ontology of
substances, the verbal idea concerns some particular property that a
substance may bear at some particular time.\largerpage[1.5]

In the 1911 text, René seems to leave aside the possibility that words
do gain some semantic autonomy when they are incorporated into the
lexicon as units; he mentions the movement from compounding to units
in the 1919 book but only to suggest that this does not impact the
analysis because it is a historical, evolutive, problem
\citep[5]{r.desaussure19:structure.logique}. In a footnote
\citep[37]{r.desaussure11:formation} he acknowledges a suggestion by
Charles Bally to consider the autonomy of lexical items (Bally was the
successor of Ferdinand de Saussure at the University of Geneva and
editor of the \textsl{Cours}), but René abstains from a conclusion
here, saying that he only aims at developing the logical point of
view. Yet in the 1919 text, he actually moves on and recognizes that
each word, despite being possibly compositional, “is in itself its own
structure” \citep[27]{r.desaussure19:structure.logique}.

However, René does not always avoid such discussions, as when he
compares \emph{doux} ‘sweet’ and \emph{doucereux} ‘smooth, unctuous’,
saying that the meaning from the logical point of view is the same but
``superfluous atoms'' of meaning are added on top of the logical meaning
of \emph{doucereux} to express a nuance \citep[61--62]{r.desaussure11:formation}.

The 1919 text provides an interesting philosophical development of
these primitive categories of meaning (``grammatical ideas''). In just a
few words, which, however prefigure later conceptions of the relation
between language and cognition (``thought''), René relates natural
language to the nature of the human mind much more deeply than what
one finds in the canonical version of Ferdinand’s \textsl{Cours}. René
even places it in the context of human evolution: ``\,“grammatical”
categories correspond to “logical” categories, and since the former
are the same for all languages, language must therefore be the
expression of a certain philosophical conception of the world, the
popular conception if you will, but one that must have very deep
roots, because it emanates so to speak from within natural evolution.”
\citep[26]{r.desaussure19:structure.logique}.

Leaving aside the \textsl{Cours}, which does not do full justice to
Ferdinand’s actual thoughts on the matter, it is likely that such
considerations have their roots in the debates and discussions the two
brothers certainly used to have about language. There may even be a
trace here of Ferdinand’s own influence, since he is reported to have
clearly affirmed, notably during his third course of General
Linguistics in Geneva, that language is an ``instinct'' (a notion that
echoes that of ``faculty'' found in the \emph{Cours}) and that it has to
do with the ``folders of the brains''
\citep[80]{saussure93:troisieme-cours}.

René asserts two major principles with regard to the ‘synthesis of
words’, that is, the construction of words composed of more than one
atom of meaning \citep[99]{r.desaussure11:formation}; these
principles are very interesting inasmuch as they suggest an economical
view of meaning construction which goes well with many scholars’
positions across various linguistic domains of investigation: for
example, the Prague school in phonology, but also Martinet’s view on
linguistic change, and of course the well-known principles of economy
at play in pragmatics according to most authors in the Gricean
tradition.

On the one hand, René posits a \emph{principle of necessity} which
requires that a certain quantity of information is provided by the
word in order that clarity and completeness of meaning are
ensured. Conversely, a \emph{principle of sufficiency} requires that
an idea is expressed only once in a word, and that no idea foreign to
the meaning of the word is incorporated. These two principles limit
each other: the principle of sufficiency imposes a restriction on the
degree of linguistic precision of an item, avoiding superfluous
meanings or redundancy, whereas the principle of necessity imposes the
requirement to go as far as possible in the linguistic marking of
meaning.

The equilibrium between these principles results in the fact that “the
meaning of a word depends only on its own content, and on all of its
content and not on the manner in which one supposes this word to be
derived from another” \citep[99]{r.desaussure11:formation}. In
passing, we note once again René’s opposition to derivational theories,
even though he acknowledges that derivations take place in the history
of languages \citep[120]{r.desaussure11:formation}. In this respect,
the theory is perhaps made clearer in the 1919 text, as René (after
noticing the behavioural similarity between verbs and adjectives)
considers that there exist constructions like adjectivalizations,
which he recognizes as derivations. But still, this does not change
the whole picture inasmuch as this type of derivation is (probably) to
be understood as historically and not synchronically relevant.

In some cases, the two principles cannot both be met adequately. This
typically happens when the idea becomes too precise for expression as
the result only of the composition of more basic items. Two options
are possible in theory: \emph{excess}, if some notion is needed but
its linguistic marking would extend to additional extraneous meanings,
and \emph{rounding down}, if some necessary idea is however omitted
from expression for lack of appropriate atom; René claims that
languages always choose rounding down because it is economical since
it avoids more confusions than excess. \emph{Couronner} ‘to crown’ is
an example of a process by rounding down: in this word, there is a
notion of an object (the crown) and of an undetermined action, but
\emph{couronner} involves more than just any action performable with a
crown: a notion of a patient and of a particular positioning of the
crown \citep[109]{r.desaussure11:formation}.\largerpage[-2]

``Rounding down'' cases force René to open the theory of meaning to some
degree of contextual determination and of sematic underspecification;
unfortunately he does not elaborate on this important aspect of
meaning.\footnote{Also, René uses only his own lay semantic intuition,
  and fails sometimes to pursue further semantic investigation. The
  example of ‘coronation’ could for instance be challenged: first,
  isn’t the notion of an individual’s head already ``contained'' in some
  way in the object itself? In other words, isn’t the function of an
  object part of its meaning? After all, a crown is made in such a
  shape that it ought to fit specifically on a head. There could be
  room for a notion of typicality here: a typical action performed
  with a crown is to put it one one’s head in the way it is made to be
  employed. As a result, it is very hard to imagine different usages
  of ‘to crown’ than this one (except of course non-literal
  usages). If this is correct, then there is no need for a notion of
  usage or context here and René seems to have trapped himself into
  problematic consequences of his claims about ``rounding down'', at
  least with unambiguous words like the one he chooses to illustrate
  it.}

Some further elaborations of the theory concern ``synonyms'' (with
interesting ideas such as that the linguistic variation observable for
some meanings has the function of selecting the proper grammatical
idea for the former), ambiguous atoms (which he names ‘double-sense
atoms’ and views as phonetic accidents irrelevant for the logical
analysis), etc.

The 1919 text adds a few elements to the picture elaborated in 1911,
notably some nuances. His brother Ferdinand is now universally
acclaimed and the tone of the text is closer to reconciliation with
his recently deceased brother than the 1911 text was. This is true
even though the 1919 book seems to continue in some way the
discussions with Ferdinand: the \textsl{Cours} is abundantly cited and
René attempts to bridge a few elements of his approach to what he finds
there. For example, he tries to relate Ferdinand’s principle of the
arbitrariness of the sign \emph{vs}. motivation to the distinction
between simple and compound words in his own theory
\citep[6]{r.desaussure19:structure.logique}.  Although some very
interesting clarifications, elaborations and adaptations are provided
by the 1919 text, (for example about the ``law of reversal'' which
establishes the relationship between morphology and syntax,
\citep[11]{r.desaussure19:structure.logique}, the essence of the
1911 theory remains unchanged and still opposed to Ferdinand’s
systemic view.

René's compositional theory achieves a considerable degree of
descriptive adequacy. Some of the examples which he analyses show that
many subtleties of meaning composition can be accounted for within the
theory, as when he compares \emph{brossier} (someone who sells
brushes, in a somewhat archaic French) and \emph{brosseur} (someone
who brushes) \citep[112]{r.desaussure11:formation}: only the second
case contains the verbal idea of action. The assumptions driving this
analysis may push theorizing further: while we intuitively tend to
define the word \emph{brossier} as someone who \emph{sells} brushes,
thus using a verbal notion in the definition, it might more accurate
to define it as someone who \emph{has brushes} (for sale).

Similarly, his approach to word composition accounts for the fact that
verbs constructed on the basis of nouns (e.g. \emph{couronner}) are
semantically of a different type than ``simple verbs'' such as
\emph{frapper} ‘to beat’; however such compositions as
\emph{couronner} are not, he insists, derivations from a noun into a
verbal derivative but compositions of atoms of different kinds, even
though the anti-derivational stance is attenuated in some way in the
1919 text where he suggests that some words (like \emph{musique},
‘music’) “giving rise to new adjectives” such as \emph{musical} “where
the root \emph{music} plays the role of a simple element”
\citep[5]{r.desaussure19:structure.logique}.

\section{Conclusion}
\label{sec:semantics-conclusion}

All in all, the most important feature of René de Saussure’s theory of
meaning is the assumption that grammatical categories as such are not
only semantic types but are also meaningful, corresponding to the
overarching concepts (of being, or qualities or to actions) that serve
as the foundation of all possible particular lexical meanings. This is
perhaps a consequence of his particular interest in morphology and
its role in the constitution of meaning. In this perspective, syntax,
then, should perhaps be considered as a formal apparatus that governs
the arrangements of meanings, not of pure abstract labels.


\sloppy
\printbibliography[heading=subbibliography,notkeyword=this] 

\end{document}
 
