\documentclass[output=paper]{langsci/langscibook} 
\author{Stephen R. Anderson\affiliation{Yale University}}
\title{The morphological theory of René de Saussure's works}
\ChapterDOI{10.5281/zenodo.1306492}
\abstract{\noabstract}
\maketitle

\begin{document}
\label{sec:morph-theory}
\lettrine[loversize=0.1, nindent=0em] {T}{he morphological theory presented in the texts} 
reproduced as Parts~\ref{ch.1911text} and~\ref{ch.1919text} 
of the present work constitutes a sort of historic
landmark.\footnote{The material in this section has been presented in
  part to audiences at Mediterranean Morphology Meetings 10 (Haifa,
  2015) and 11 (Cyprus, 2017). Helpful comments from the audiences on
  those occasions is gratefully acknowledged.}  Although somewhat
familiar to the community of students and advocates of the construc\-ted
language Esperanto, as discussed in \sectref{sec:esperanto}, it
has remained essentially unknown to scholars in the broader community
of linguists. It represents a particularly explicit formulation of the
sort of view that would later be associated with the theory of the
Structuralist morpheme, and as such contrasts strikingly with what we
can conclude about the view of internally complex words in the work of
Ferdinand de Saussure.

\section{René de Saussure's conception of morphology}
\label{sec:basic-principles}

The retrospective importance of Ren\'e de Saussure's works on the
nature of word formation lies in the fact that they articulate, at the
very beginning of what we think of as the ``modern'' period in
linguistics, a clear version of one of the two poles that would come
to dominate discussion of this area within the field.\footnote{For a
  review of this history, largely from an American point of view, see
  \citet{anderson17:matthews,anderson17:ohmt}. Some of the discussion
  below is drawn from these sources.}  From the outset, after
distinguishing simple words (e.g. French \emph{homme} `man',
\emph{grand} `large, tall', etc.)  from compounds (e.g. French
\emph{porte-plume} `penholder', German \emph{Dampfschiff} `steamship',
etc.) and derived words (e.g. French \emph{grandeur} `size, height',
\emph{humanit\'e} `humanity'), he argues that
\begin{quotation}
  ``il y a deux sortes d'\'el\'ements primitifs: les mots-radicaux,
  tels que : «hom\-me», «grand», etc., et les affixes, tels que : «iste»
  (dans «violoniste»), «pr\'e» (dans «pr\'evenir»), etc.''

  {[However,]} ``[a]u point de vue logique, il n’y a pas de
  diff\'erence essentielle entre un radical et un affixe; {[\ldots]}
  On peut donc consid\'erer les affixes comme des mots simples, et les
  mots d\'eriv\'es au moyen d’affixes, comme de v\'eritables mots
  compos\'es. II n’y a plus alors que deux sortes de mots : les mots
  simples (radicaux, pr\'efixes, suffixes), et les mots compos\'es par
  combinaison de mots simples.''
  \pgcitep{r.desaussure11:formation}{4--5}\footnote{``There are two
    kinds of primitive element: root words, such as \emph{homme}
    ‘man’, \emph{grand} ‘tall’, etc., and affixes, such as
    -\emph{iste} (in \emph{violoniste} ‘violinist’), \emph{pr\'e} (in
    \emph{pr\'evenir} ‘precede’), etc. From the logical point of view,
    there is no essential difference between a root and an affix:
    {[\ldots]} We can therefore consider affixes as simple words, and
    words derived by means of an affix as real compound words. There
    are then only two sorts of word: \emph{simple words} (roots,
    prefixes, suffixes) and \emph{compound words} formed by combining
    simple words.''}
\end{quotation}

On this view, derived words are just a class of compounds.  As a
consequence, affixes are to be viewed as simple sound-meaning pairs,
just like simple words. Derived words are no different from compounds
in that both are composed of two or more atoms in structured
combination with one another.  The type example Saussure gives,
\emph{violoniste} `violinist', is thus analyzed as a compound composed
of two equally basic units, both nouns: \emph{violon} `violin' and
-\emph{iste} `person whose profession or habitual occupation is
characterized by the root to which it is attached' with the two parts
being parallel simple associations between sound and meaning.

\section{Differences between René's view and that of his brother}
\label{sec:rene-vs-ferdinand}

Readers familiar with the work of Ren\'e's brother Ferdinand de
Saussure may be struck by the difference between this position on the
structure of complex words and the one that pervades
\citealt{saussure:cours}.  What matters here is the fact that Ren\'e
\citet{r.desaussure11:formation} enunciates categorically the view
that all morphological elements, roots and affixes alike, constitute
parallel atomic sound-meaning pairings.  In this regard, such elements
are uniformly of the type Ferdinand \citet{saussure:cours} would
analyze as \emph{minimal signs}: arbitrary, irreducible associations
of expression (sound, gesture, orthography) with content. As pointed
out by \citet{matthews01:hist.struct.lx}, the observation that such
associations are a core characteristic of natural language was by no
means completely original with Saussure, but his importance lies in
having made them the center of attention in the study of language.

\begin{figure}[htb]
  \begin{center}
    \includegraphics*[0,0][2.833in,3.472in]{./Photos/Ferdinand1.eps}
  \end{center}
  \caption{Ferdinand de Saussure (1857--1913)}
\end{figure}

Where the brothers part company is in the more general analysis of
words.  For both, simple words (e.g. \emph{arbre} `tree') are minimal
signs, but where Ren\'e saw derived words like \emph{violoniste}
`violinist' as simply combinations of such units, Ferdinand presents a
rather different view.  For him, words that are not simple are also
signs --- of a type he refers to as \emph{relatively} or
\emph{partially motivated}. That is, the sign relation between form
and meaning obtains here, too, but in such cases it is not completely
arbitrary: part of the relation is motivated by the relation between
this sign and others:
\begin{quotation}
  ``Une unité telle que \emph{désireux} se compose en deux sous-unités
  (\emph{désir-eux}), mais ce ne sont pas deux parties indépendantes
  ajoutées simplement l’une à l’autre (\emph{désir}+\emph{eux}). C’est
  un produit, une combinaison de deux éléments solidaires, qui n’ont
  de valeur que par leur action réciproque dans une unité supérieure
  (\emph{désir}×\emph{eux}). Le suffixe, pris isolément, est
  inexistant; ce qui lui confère sa place dans la langue, c’est une
  série de termes usuels tels que \emph{chaleur-eux},
  \emph{chanc-eux}, etc. À son tour, le radical n’est pas autonome; il
  n’existe que par combinaison avec un suffixe; dans \emph{roul-is},
  l’élément \emph{roul}- n’est rien sans le suffixe qui le suit. Le
  tout vaut par ses parties, les parties valent aussi en vertu de leur
  place dans le tout, et voilà pourquoi le rapport syntagmatique de la
  partie au tout est aussi important que celui des parties entre
  elles'' (\pgcitealt{saussure:cours}{176--177}; cf. also
  \pgcitealt{saussure93:troisieme-cours}{85--90}).\footnote{``A unit
    such as \emph{desireux} is composed of two sub-units
    (\emph{desire}+\emph{ous}), but these are not two independent parts simply added
    to one another. It is a product, a combination of two linked
    elements which only have their value by their reciprocal relation
    within a larger unit. The suffix, taken in isolation, does not
    exist: what gives it its place in the language is a series of
    words like \emph{chaleureux} ‘warmth-ous, warm’, \emph{chanceux}
    ‘fortune-ous, lucky’, etc. In its turn, the root is not
    autonomous: it only exists through its combination with a
    suffix. In \emph{roulis} ‘rotation’, the element \emph{roul}-
    ‘roll’ is nothing without the suffix that follows it. The whole
    has value through its parts, the parts also have value through
    their place in the whole, and that is why the syntagmatic relation
    of the part to the whole is as important as that of the parts to
    one another.''}
\end{quotation}


The point to focus on here is that for Ferdinand, as opposed to
Ren\'e, a suffix in a derived word does not have value in and of
itself as a minimal sign, but rather it obtains its significance from
the fact that words of similar form are related to one
another. Elsewhere in the \emph{Cours} he gives the example of
\emph{poirier} `pear tree'. This is obviously related to \emph{poire}
`pear', but the meaning of \emph{poirier} is not just a compound of
two meanings `pear' and `tree bearing fruit specified by the root to
which it is attached' (in the way Ren\'e analyzes the meaning of
\emph{violoniste}). Rather, it arises because the relation between
\emph{poire} and \emph{poirier} is similar to that of other pairs in
the language: \emph{cerise} `cherry' / \emph{cerisier} `cherry tree',
\emph{pomme} `apple' / \emph{pommier} `apple tree', etc. Complex words
thus get their sense from their place in a constellation of relations
among words.

\begin{figure}[htb]
  \begin{center}
%     \includegraphics*[0,0][4.17in,3.5in]{./Photos/poirier.eps}
    \begin{tikzpicture}
\matrix [matrix of nodes, column sep=.75cm, nodes in empty cells] {
\draw[
    path picture={
\path[path picture={
        \node at (path picture bounding box.center) {
        \includegraphics[trim=0 0 0 35cm,scale=.4]{./Photos/peartree.jpg}
        };}]
    ($(path picture bounding box.north west)!0!(path picture bounding box.south west)$)
    rectangle
    ($(path picture bounding box.north east)!1/2!(path picture bounding box.south east)$);
    \path[fill=white,draw]
    ($(path picture bounding box.north west)!1/2!(path picture bounding box.south west)$)
    rectangle node {\sffamily POIRIER}
    ($(path picture bounding box.north east)!1!(path picture bounding box.south east)$);
    }
  ](0,0) ellipse (1 and 1.5) node [ellipse, minimum width=2cm, minimum height=3cm] (peartree) {}; & & 
  \draw[
    path picture={
\path[path picture={
        \node at (path picture bounding box.center) {
        \includegraphics[trim=0 0 0 55cm, scale=.9]{./Photos/cherrytree.jpg}
        };}]
    ($(path picture bounding box.north west)!0!(path picture bounding box.south west)$)
    rectangle
    ($(path picture bounding box.north east)!1/2!(path picture bounding box.south east)$);
    \path[fill=white,draw]
    ($(path picture bounding box.north west)!1/2!(path picture bounding box.south west)$)
    rectangle node {\sffamily CERISIER}
    ($(path picture bounding box.north east)!1!(path picture bounding box.south east)$);
    }
  ](0,0) ellipse (1 and 1.5) node[ellipse, minimum width=2cm, minimum height=3cm] (cherrytree) {}; & 
  \draw[
    path picture={
\path[path picture={
        \node at (path picture bounding box.center) {
        \includegraphics[trim=0 0 0 30cm, scale=.15]{./Photos/appletree.jpg}
        };}]
    ($(path picture bounding box.north west)!0!(path picture bounding box.south west)$)
    rectangle
    ($(path picture bounding box.north east)!1/2!(path picture bounding box.south east)$);
    \path[fill=white,draw]
    ($(path picture bounding box.north west)!1/2!(path picture bounding box.south west)$)
    rectangle node {\sffamily POMMIER}
    ($(path picture bounding box.north east)!1!(path picture bounding box.south east)$);
    }
  ](0,0) ellipse (1 and 1.5) node[ellipse, minimum width=2cm, minimum height=3cm] (appletree) {};\\
  \node{\sffamily `PEAR TREE'}; & & \node{\sffamily `CHERRY TREE'}; & \node{\sffamily `APPLE TREE'};\\
  & \node{\sffamily\huge ≈}; & & & \node{\sffamily\huge ...};\\
\draw[ 
    path picture={
\path[path picture={
        \node at (path picture bounding box.center) {
        \includegraphics[trim=0 0 0 35cm, scale=.075]{./Photos/pear.jpg}
        };}]
    ($(path picture bounding box.north west)!0!(path picture bounding box.south west)$)
    rectangle
    ($(path picture bounding box.north east)!1/2!(path picture bounding box.south east)$);
    \path[fill=white,draw]
    ($(path picture bounding box.north west)!1/2!(path picture bounding box.south west)$)
    rectangle node {\sffamily POIRE}
    ($(path picture bounding box.north east)!1!(path picture bounding box.south east)$);
    }
  ](0,0) ellipse (1 and 1.5) node [ellipse, minimum width=2cm, minimum height=3cm] (pear) {}; & & 
  \draw[ 
    path picture={
\path[path picture={
        \node at (path picture bounding box.center) {
        \includegraphics[trim=0 0 0 40cm, scale=.04]{./Photos/cherry.jpg}
        };}]
    ($(path picture bounding box.north west)!0!(path picture bounding box.south west)$)
    rectangle
    ($(path picture bounding box.north east)!1/2!(path picture bounding box.south east)$);
    \path[fill=white,draw]
    ($(path picture bounding box.north west)!1/2!(path picture bounding box.south west)$)
    rectangle node {\sffamily CERISE}
    ($(path picture bounding box.north east)!1!(path picture bounding box.south east)$);
    }
  ](0,0) ellipse (1 and 1.5) node [ellipse, minimum width=2cm, minimum height=3cm] (cherry) {}; & 
  \draw[ 
    path picture={
\path[path picture={
        \node at (path picture bounding box.center) {
        \includegraphics[trim =  0 0 .5cm 35cm, scale=.5]{./Photos/apple.png}
        };}]
    ($(path picture bounding box.north west)!0!(path picture bounding box.south west)$)
    rectangle
    ($(path picture bounding box.north east)!1/2!(path picture bounding box.south east)$);
    \path[fill=white,draw]
    ($(path picture bounding box.north west)!1/2!(path picture bounding box.south west)$)
    rectangle node {\sffamily POMME}
    ($(path picture bounding box.north east)!1!(path picture bounding box.south east)$);
    }
  ](0,0) ellipse (1 and 1.5) node  [ellipse, minimum width=2cm, minimum height=3cm] (apple) {}; & \\
  \node{\sffamily `PEAR'}; & & \node{\sffamily `CHERRY'}; & \node{\sffamily `APPLE'};\\
};

 \path[<->,Triangle-Triangle] (peartree)   edge [bend left=45] (pear.north east);
 \path[<->,Triangle-Triangle] (cherrytree) edge [bend left=45] (cherry);
 \path[<->,Triangle-Triangle] (appletree)  edge [bend left=45, looseness=1.15] (apple);
\end{tikzpicture}

  \end{center}
  \caption{\label{poirier}Relatively or partially motivated signs.\footnotesize\xspace Pear tree image CC-BY SA B137, Cherry tree image CC-BY SA Benjamin Gimmel, Apple tree image CC-BY SA Maseltov, Pear image in the public domain, Cherry image CC-BY SA Benjamint444, Apple image CC-BY SA PiccoloNamek.}
\end{figure}


The picture in Figure~\ref{poirier} is quite different from the view
presented by Ren\'e, who explicitly rejects such an account:
\begin{quotation}
  ``Il n’est donc pas besoin d’établir des \emph{règles de dérivation}
  reliant l’un à l’autre le sens des mots d’une même famille (comme
  «homme», «humain», «humanité»; «couronne», «couronner»,
  «couronnement»), car on crée ain\-si des liens artificiels entre des
  atomes qui devraient rester indépendants et interchangeables comme
  les différentes pièces d’une
  machine.'' \citep[8]{r.desaussure11:formation}.\footnote{``There is
    thus no need to establish \emph{rules of derivation} linking to
    each other the senses of words belonging to the same family (such
    as \emph{homme} `man', \emph{humain} `human', \emph{humanit\'e}
    `humanity'; \emph{couronne} `crown (n.)', \emph{couronner} `(to)
    crown', \emph{couronnement} `coronation'), because that would
    create artificial links between atoms that must remain independent
    and interchangeable like the different parts of a machine.''}
\end{quotation}

We can categorize the difference between the views of the two Saussure
brothers, at least roughly, in terms of two useful dimensions of
theories as distinguished by \pgcitet{stump:inflect_morph_book}{1}. On\largerpage[1]
the first of these, theories can be lexical, and treat all
form-content associations as listed; or they can be inferential, in
treating form-content relations in complex words as more holistic.
\begin{description}
\item[Lexical] theories are those where associations between
  (morphosyntactic) content and (phonological) form are listed in a
  lexicon. Each such association is discrete and local with respect to
  the rest of the lexicon, and constitutes a morpheme of the classical
  sort.
\item[Inferential] theories treat the associations between a word’s
  morphosyntactic\linebreak properties and its morphology as expressed by rules
  or formulas.
\end{description}

Independent of this distinction, theories can be incremental, with
elements of content associated in a one-to-one fashion with elements
of form, or realizational, in which the relation is less direct, such
that a single element of content can be associated with one element of
form, or several, or none at all, and vice versa.
\begin{description}
\item[Incremental] theories are the ones on which a word bears a given
  content property exclusively as a concomitant of a specific formal
  realization.
\item[Realizational] theories are the ones on which the presence of a
  given element of content licenses a specific realization, but does
  not depend on it. 
\end{description}

The two dimensions are logically independent, and Stump identifies
examples of all four possible combinations of values. By and large,
though, most theories are either lexical and incremental or
inferential and realizational. The first class sees the locus of
form-content relations as a set of something like Saussurean minimal
signs, identifiable generally with the classical understanding of the
morpheme.  Inferential/realizational theories, in contrast, see the
form-content relation as rather more diffuse, and in practice continue
the distinct classical tradition of ``word and paradigm'' analysis. 

In those terms, we can categorize the Saussure brothers’ views of the
nature of a complex word like \emph{poirier}: René sees this as the
combination of two independent lexical elements, where each part of
the meaning is associated uniquely with a specific, independently
listed element of form. His theory is thus a lexical/incremental view,
and the components of a complex word are essentially what would later
be called morphemes.  

Ferdinand, in contrast, sees the complex meaning as arising from a
relation that has a status in the language.  It is this rule relating
\emph{poire} and \emph{poirier} (and also
\emph{cerise}/\emph{cerisier}, etc.), not the suffix -\emph{ier}
itself, that yields the meaning ‘pear tree’. His is an inferential
theory, and while this example does not serve to make the point, from
other sources (such as discussions of Gothic, Greek and Latin
morphology in his courses on these languages) we can say that it is
realizational rather than incremental.

The contrast between these somewhat different views does not seem to
have attracted much attention at the time, although it represents what
has historically been the basic opposition in morphological theory.
René’s story is an early version of what we can call a morphemic
theory, one that takes internal components of complex words as the
basic locus of meaning. These are combined by an extension of the
syntax, and the resulting structures are compositional functions.
Ferdinand’s story, in contrast, is a version of what would later be
called a word and paradigm theory, where whole words are the locus of
meaning and an understanding of their content, as well as their form,
comes from an analysis of their place in a network of relations to
other words.

In the difference between the two, we can see the origins of a basic
contrast between theories of morphology.  In practice, however, this
contrast did not become a matter of theoretical discussion
immediately.  René saw the decomposition of complex words into
combinations of simplexes --- their analysis as structured
concatenations of minimal signs --- as transparently obvious, a simple
matter of logic, and his 1911 book and its 1919 continuation develop
this picture in some detail, discussing the logical/grammatical types
of simple words that we find, and the varieties of combination of
these atomic units that exist (in French, at least, and to a lesser
extent in German and in English).

Although the theoretical difference between these two views was not a
focus of attention in the work of either of the de Saussure brothers,
that does not mean that it went unnoticed. In the conclusion to the
portion of the later work included here
\citep[27--28]{r.desaussure19:structure.logique}, Ren\'e observes that
his theory attributes the same kind of structure to all languages, a
type that is nicely suitable for artificial languages, but that this
is not the only way to view the structure of derived words. Citing the
views of his late brother, which had become available in 1916 with the
publication by his students of the \textsl{Cours de linguistique
  g\'en\'erale},\footnote{Ren\'e cites pages 235 and 236 of the
  \textsl{Cours}, although the passage in question appears on pages
  228--230 in the edition we use here.} he notes that the structure of
e.g. the made-up word \emph{indécorable} could be analyzed in two
ways. On the one hand, we could isolate component elements (\emph{in},
\emph{d\'ecor}, \emph{able}) and combine these in quasi-syntactic
fashion along the lines of his theory.  Alternatively, we could regard
it as a unit whose content is revealed by an analogical proportion:

\begin{center}
  \emph{pardonner}: \emph{impardonnable} = \emph{décorer}: x\\
  x = \emph{indécorable}.
\end{center}

Discussing the contrast between these views, Ferdinand poses the
question of which of them ``correspond \`a la r\'ealit\'e'': that is,
in our terms, whether we ought to prefer a morphemic theory, which he
associates with ``Hindu grammar'' or the work of classical Sanskrit
grammarians, or an inferential-realizational one, which he identifies
with ``European'' grammatical presentation.  Ren\'e's answer to this
is clear and categorical: although nominally restricting himself to
artificial languages, his opinion is that ``the method of Hindu
grammar is the only satisfying one.''

In support of this opinion, he again invokes Ferdinand's opinion in
the \textsl{Cours}\footnote{Cited by Ren\'e as found on pages 196, 197
  and 198; in the edition used here, the relevant passages are on
  pages 190, 191 and 192. My translation: sra. } that ``\emph{abstract
  entities always rest, in the final analysis, on concrete
  entities.}'' It is unclear, however, why Ren\'e thought that this
should imply that the structure of complex words is to be regarded as
a structured concatenation of elementary atoms, rather than as
systematic associations among related words. In context, Ferdinand
invokes this principle in connection with abstractions such as
``genitive case'' in e.g. Latin, where a number of different
inflectional forms all support the same function. Positing an entity
``genitive case'' is here based on the fact that a number of diverse
concrete forms manifest it in their relation to other forms within the
paradigm of a Latin nouns, which would be illicit if there were no
such concrete support for the category, but this fact does not require
us to analyze case marking in Latin as exclusively a syntagmatic
rather than a paradigmatic matter (in Ferdinand's terms).

Similarly, in the passage in question, he notes that English
\emph{the man I have seen} has no overt relative pronoun, but
denies that in order to analyze this as a relative clause, there must
be an abstract unpronounced element to represent the missing object:
the material support here for the notion of relative clause structure
is said to be the word order of the concrete words involved. The point
at stake in this part of the \textsl{Cours} (chapter 8 of part II) is
not directly relevant to a choice between the ``Hindu'' and
``European'' conceptions of word structure.

Which is not to say that in opposing those two conceptions, Ferdinand
felt as strongly as Ren\'e that the choice between them in developing
a theory of language was clear and unambiguous. Most discussion of
complex words in the \textsl{Cours} is couched in terms of analogical
relations between whole words, and not as a sort of syntax of
morpheme-like simple signs. In various places, however, he also
recognizes that the linguistic conscience of speakers commonly
includes the identification of component roots, suffixes, desinences,
etc., as well as the order in which these appear within a word. 

Does this fact support the claim that word formation is fundamentally
a matter of arranging such pieces? It is worth noting that in the
section of the \textsl{Cours} where this issue is raised
\citep[228--230]{saussure:cours} Ferdinand notes the existence of
facts that cast doubt on such a conclusion. In a passage skipped over
in Ren\'e's quote, he notes that in German

\begin{quotation}
  ``[d]ans un cas comme \emph{Krantz}~: \emph{Kr\"antze} fait sur
  \emph{Gast}~: \emph{G\"aste}, la décomposition semble moins probable
  que la quatrième proportional, puisque le radical du modèle est
  tantôt \emph{Gast-} tantôt \emph{Gäst-}.''\footnote{``in a case like \emph{Krantz}
    : \emph{Kräntze} `wreath'/`wreaths' built on \emph{Gast} :
    \emph{Gäste} `guest' : `guests', decomposition seems less likely
    than proportional analogy, since the root of the model is
    sometimes \emph{Gast-} and sometimes \emph{Gäst-}.''}
\end{quotation}
Here part of the indication of plurality is through variation in the
shape of the stem, and not only by the presence of an atom
\textsc{Plural}, although he qualifies the significance of that
observation by noting, in effect, the existence of contextually
determined allomorphy.

Ferdinand's point here is that decomposition would only make sense if
it resulted in invariants; where there is non-segmentable variation in
shape associated with the formation of a morphologically complex form,
the analysis based on a rule of proportional analogy gives a more
satisfactory account. This is, perhaps, the first instance in the
literature of the use of non-concatenative morphology as the basis of
an argument against analysis into morphemes. We can note that the
argument might have been strengthened, especially given his reluctance
(just noted) to posit significant zero elements, if he had cited pairs
such as \emph{Tochter}~: \emph{T\"ochter} `daughter~: daughters' or
\emph{Mantel}~: \emph{M\"antel} `overcoat~: overcoats' on the analogy
of \emph{Mutter}~: \emph{M\"utter} `mother~: mothers' or
\emph{Vater}~: \emph{V\"ater} `father~: fathers' respectively, where
the stem alternation is the \emph{only} indication of the plural.

On the other hand, he did not regard the matter as entirely
settled. The argument that he sees as most favorable to the analysis
of complex words in terms of their components
\citep[229--230]{saussure:cours} is somewhat curious, and based on a
phenomenon in the history of Latin known as Lachmann's Law. As
described by \pgcitet{jasanoff04:lachmann}{405}, this is the rule
``according to which verbal roots ending in an etymological voiced
stop (*-\emph{b}-, *-\emph{d}-, etc.), but not a voiced aspirate
(*-\emph{bh}-, *-\emph{dh}-, etc.), lengthen their root vowel in the
past participle and its derivatives (e.g. \emph{ag\=o} `drive',
ptcp. \emph{\=actus} (+\emph{\=acti\=o}, etc.), \emph{cad\=o} `fall',
ptcp. \emph{c\=asus}< *\emph{c\=assus})'' while similar verbs whose
roots end in voiceless stops show no such lengthening
(e.g. \emph{faci\=o} `make', ptcp. \emph{f\u{a}ctus}, \emph{speci\=o}
`watch', ptcp. \emph{sp\u{e}ctus}). 

The participial forms in which the lengthening occurs were surely
inherited from Proto-Indo European with the stem-final voiced stops
devoiced (via regressive assimilation from the \emph{t} of the
participle ending), and thus should not contrast with root-final
voiceless stops where these are unchanged from the stem form. The
operation of Lachmann's Law thus requires a sensitivity on speakers'
parts to a difference between roots ending in voiced stops as opposed
to voiceless ones, even in inflected or derived forms where that
difference was not apparent in the word's phonetic realization. And
that, in turn, would seem to imply an awareness of the identity of
roots independent of the specific complex words in which they appear.
In \pgposscitet{saussure:cours}{230} words, ``[i]l n'a pu y arriver
qu'en prenant fortement conscience des unités radicales \emph{ag-
  teg-}'' \footnote{``[i]t is not possible to arrive at this without
  being clearly aware of the root units \emph{ag- teg-}''}{[in t\=ectus]}.

A full discussion of Lachmann's Law would take us well beyond our
concerns here; for a recent summary of the facts and analytic
controversies, see \citet{roberts09:lachmann}. To assess de
Saussure's argument, however, it is not necessary to go into such
detail. His position is that ``it is necessary to suppose that
\emph{\=actus} goes back to *\emph{\u{a}gtos} and to attribute the
lengthening of the vowel to the voiced consonant that follows''
\pgcitep{saussure:cours}{230 {[my translation: sra]}}.  Now we know
from comparative evidence that the stem final consonants in these
clusters were already assimilated to the following \emph{t} in early
Latin; and we also know that they were inherited as voiceless by the
Romance daughter languages. In most generative phonological analyses,
beginning with that of \citet{kiparsky:thesis}, the voicing
distinction is seen as preserved in the underlying form of the root
(e.g. \mbox{/ag-/}), with assimilation in the participle stem but only
after the required lengthening has taken place. Such an analysis was
not, of course, available to de Saussure, but another possibility was,
and in fact is evident in his positing of *\emph{\u{a}gtos} as the
form to which lengthening applies and his subsequent discussion.

In fact, in an earlier paper, \citet{saussure89:lachmann} argued that
the root-final voiced consonants in these words were in fact restored
in early Latin by analogy:

\noindent\parbox{\linewidth}{\begin{center}
  \emph{f\u{a}ci\=o} : \emph{f\u{a}ctus} = \emph{\u{a}g\=o} : x\\
  x = \emph{\u{a}gtus}
\end{center}}

\noindent
These forms with the restored (though admittedly problematic in
phonetic terms) voiced obstruents then underwent lengthening by
Lachmann's Law, after which voicing assimilation again operated to
produce voiceless clusters.

Ferdinand's claim that the root-final voiced obstruents were restored
in early Latin just long enough for Lachmann's Law to operate has
generally been rejected by subsequent scholarship, but it is far from
clear that, even if we were to accept it, his account would imply that
the relevant participial forms would need to be analyzed as
combinations of a root and an affix. After all, the crucial step is
the analogy just offered that led to the (temporary) restoration of
root-final \emph{g} in *\emph{\u{a}gtos}, but this analogy is a matter
of the relation between full word forms.  After the incorporation of
such forms with restored voiced stops into the language, purely
phonetic change could then effect the lengthening by Lachmann's Law,
followed by another purely phonetic change of regressive voice
assimilation. At no point in this scenario is it necessary to
recognize the root \mbox{/ag-/} as a structural unit, despite de
Saussure's claim to the contrary.\largerpage[1.5]

It seems, then, that there is no real argument to be found in
Ferdinand's discussion of the matter in the \textsl{Cours} that would
support Ren\'e's opinion that ``the method of Hindu grammar is the
only satisfying one.'' Indeed, while Ferdinand left open the
possibility that languages might differ from one another as to which
of the two analytic frameworks was most appropriate, his own practice
generally relied on analogical (rather than compositional) operations
as providing the basis for analyzing complex words. Note that it is
not necessary to deny that speakers can identify stems, affixes, and
desinences as these are found in complex words\footnote{Note that in
  cases where part of a word's content is signalled through
  ``non-concatenative'' means, the analysis of content in terms of
  combinations of atoms will not be exhaustive. See
  \citet{sra:morph_book} and much other literature for discussion.}
in order to suggest that complex words are formed on the basis of
relations between whole words and not fundamentally by combining these
elements syntagmatically.  Stems are plausibly present in the lexicon
on such a view, and affixal material, along with a variety of
non-affixal ways of signalling morphological content, is represented
in the structural changes of rules of word formation.  These aspects
of a grammar suffice to support the observed meta-linguistic awareness
on speakers' parts of components present in complex words without
requiring that the analysis of such words be grounded in the
quasi-syntactic combination of ``morphemes'' or the atoms of René's
theory.

The tension between the two views of complex words that we have been
discussing, and the clear contrasting of these views in the work of
Ren\'e and Ferdinand de Saussure, provides important motivation for
studying this neglected predecessor of the theories that would become
prominent in the structuralist morphology of the mid-20th
century. Significantly, \cite{saussure:cours} already saw the tension
between what we can call rule-based and morpheme-based accounts of the
structure of complex words as a significant theoretical issue, to be
resolved by empirical argument. If we are tempted to see this difference
as a matter of recent morphological theory, or perhaps one that
originates in \posscitet{hockett:2models} “Two models” paper, we
should see that in fact it has been with us since the very earliest
days of what we think of as scientific linguistics.

\section{Some specific points in René's theory}
\label{sec:specifics}

If the only reason to read
\citet{r.desaussure11:formation,r.desaussure19:structure.logique} were
that he largely anticipates the later position of structuralist
morphologists, these works might be written off as merely historical
curiosities. In fact, however, Ren\'e de Saussure's work on morphology
develops a rather fuller and more explicit theory of word structure
than just a commitment to a morpheme-like view, and a number of more
specific points that arise in this theory also have importance for
this area today.

\subsection{Binary branching}
\label{sec:binary-branching}

One aspect of this theory is a limitation on the internal complexity
of words to binary branching structure: ``Lorsqu'un mot composé
contient plus de deux éléments, son analyse peut toujours être ramenée
à celle de plusieurs mots ne contenant chacun que deux éléments.''
(1919:~13)\footnote{``When a compound word contains more than two
  elements, its analysis can always be reduced to that of several
  words each containing only two elements.'' In the remainder of this
  section, references to \cite{r.desaussure11:formation} and
  \citet{r.desaussure19:structure.logique} will be given simply as
  ``1911'' or ``1919'' with page reference.} Such a principle is
presented as a substantive component of the theory of morphology by a
number of writers
\citep{aronoff:book,booij77:dutch-morphology,lieber:thesis,scalise:text},
who offer arguments in favor of a binary-branching analysis of words
that might be presumed to have ternary or other structure.

René shows that the way in which a multi-element word is analyzed
(with flat structure or with one or another of alternative possible
binary branching structures) makes a difference: thus, he argues that
\emph{international} should be analyzed as {[[inter~nation]~al]}
`[that which is [between nations]]' and not as
{[inter~[nation~al]]}. The reason for treating the word in this way is
to provide an appropriate basis for the meaning of the complex form,
since he sees semantic structure as directly represented in the
morphological analysis.

René's semantically based analysis might be seen as at variance with
the structure apparently motivated by form alone. In the word
\emph{international} in English,\footnote{The same issue does not
  arise as obviously in French.} the suffix \emph{-al} is apparently
more closely related to the stem \emph{nation} than is the prefix
\emph{inter-}: cyclic analyses and their variants treat \emph{-al} as
a ``level I'' or stem-level affix, while \emph{inter-} is a ``level
II'' (or III) or word-level affix, which would seem to require the
structure {[inter~[nation~al]]}, contrary to René's account.  For a
somewhat more obvious example, consider the reading of English
\emph{criminal lawyer} as `practitioner of criminal law', where the
semantics seems to motivate {[[criminal~law]~(y)er]}, while the form
would seem to require {[criminal~[law~(y)er]]}. 

``Bracketing paradoxes'' of this sort received a great deal of
attention in the morphological and phonological literature of the
1980s (e.g. \citealt{williams:heads}; \citealt{pesetsky:bracketing};
\citealt{spencer88:bracketing-paradoxes}) and later.  It is clear,
however, that this issue could never arise for René, since he is quite
clear that semantic considerations alone are relevant to word
structure: ``[l]a signification de tout mot composé résulte alors
directement de l'analyse de son contenu, et non de la manière dont on
suppose ce mot dérivé d'un autre'' (1919:~27)\footnote{``The meaning of
  every compound word thus results directly from the analysis of its
  content, and not from the manner in which we suppose this word to be
  derived from another.''}.  Things could hardly be otherwise, since,
as will be discussed below, phonological considerations of the sort
that give rise to the conflict posed by ``bracketing paradoxes'' form
no part of his theory, in which morphological structure is conceived in
exclusively semantic terms.

\subsection{Category, headedness and the difference between words
  and phrases}
\label{sec:headedness}

Internal branching is by no means all there is to the internal
structure of words on René's theory.  He makes it clear (1911ff.) that each of
the atoms contained in a complex word is assigned to a major word
class: noun, verb or adjective, depending on its semantics.  Thus,
\emph{-iste} in \emph{violoniste} is a noun because it designates a
person; \emph{-able} in \emph{louable} `commendable' is an adjective,
because it contains a qualifying idea; and \emph{-is-} in
\emph{moderniser} and \emph{-ifi-} in \emph{béatifier} are verbs,
because they contain a dynamic idea.

Atoms can also impose restrictions on the structures into which they
enter based on these word classes. In an interesting passage
(1911:~36--37), René argues that the word \emph{couronnement}
`coronation' appears to contain only the atoms \emph{couronne} `crown'
and \emph{-ment}, but there must also be an additional verbal atom
(represented here by the medial \emph{-e-}, reduced from the
infinitive ending \emph{-er}), because nominalizing atoms such as
\emph{-ment} ``ne sont employés qu'après des atomes verbaux, comme les
atomes \emph{ité, esse} ne le sont qu'après des atomes
adjectifs.''\footnote{``are only used after verbal atoms, just as the
  atoms \emph{ité, esse} are only used after adjectival atoms.''
  (1911:37).}

The categorial identity of a complex structure is determined by that
of its constituent elements. Here René enunciates an important basic
principle of word structure: ``L'espèce grammaticale d'un mot est
déterminée par son \emph{dernier} élément; ainsi,
\textbf{Schreib'tisch} est un substantif, parce que \textbf{Tisch} est
un substantif.''\footnote{``The grammatical category of an element is
  determined by its \emph{final} element: thus {[German]}
  \textbf{Schreib'tisch} `writing table' is a noun because
  \textbf{Tisch} `table' is a noun.'' (1919:~16; cf. also
  1911:~42--43)}  This is of course the same as what
\citet{williams:heads} would propose (along with
\citealt{selkirk:wordsyntax}) as the ``Right Hand Head Rule'', a
supposedly general principle of word structure, at least for English
and French.

Just as the literature of the 1980s recognized the existence of
exceptions to the Right Hand Head Rule, René notes that there are
words like \emph{timbre-poste} {[stamp-postage]} `postage stamp' and
\emph{assurance-vieillesse} {[insurance-old age]} `old age insurance'
where the element that should be seen as the head from a semantic
point of view is the leftmost, not the rightmost constituent.
Accordingly, when words of this type are pluralized (as
e.g. \emph{timbres-poste} `postage stamps') the plural marker appears
inside the compound, on the first element.

René's account of these words is that they are not really compounds,
but an rather abbreviated form of phrases.  Within his theory, this is
a principled solution, grounded in a proposed regular relation between
complex words (compounds) and phrases. While the former, called by him
``condensed molecules'', conform to the principle of a structural head
on the right, syntactically formed phrases (called ``dissociated
molecules''), in contrast, have initial heads.  

This structural difference is claimed to be completely systematic, and
the basis of the fundamental rule invoked for the analysis of complex
words: the Law of Reversal.  In general, complex words have phrasal
paraphrases, and \emph{vice versa}. The relation between these is
governed by the principle that ``la dissociation d'une molécule
condensé (ou la condensation d'une molécule dissociée) renverse
l'ordre des atomes.''\footnote{``The dissociation of a condensed
  molecule (or the condensation of a dissociated molecule) reverses
  the order of the atoms.'' (1911:35; cf. also 1919:11--12)} This
equivalence underlies a great deal of the discussion in both of the
works under discussion in the present book. The procedure for deriving
a word corresponding to a complex idea is to express it in syntactic
(``dissociated'') form, and then reverse the order of the atoms
involved to produce a corresponding ``condensed'' form, a compound
word.  Conversely, to analyze a complex word, one takes the atoms of
which it is composed and reverses their order, which should yield a
phrasal paraphrase (with a certain amount of unexplained footwork to
provide or ignore the purely grammatical markers that may be
required). Thus, German (condensed) \emph{Schreib'tisch} `writing
table' corresponds to the dissociated form `table (for) writing'.  On
this basis, if e.g. \emph{timbre-poste} `postage stamp' is ``really'' a
sort of phrase, rather than a legitimate compound, the position of its
head on the left is just what we would expect.

The fundamental nature of this relation undersores the fact that for
René, syntax and morphology are fundamentally distinct (but
systematically related) domains within the overall theory of grammar.
That is, complex words are not just syntactic constructions with some
kind of phonological unity, as in some modern theories, but a
systematically different structure.  Morphology is not simply the
syntax of small domains.  On the other hand, since the relation
between morphology and syntax is a resolutely synchronic one (and René
was at pains on several occasions to stress that his theory was
intended to be purely synchronic), the difference is not to be seen as
a matter of morphology preserving the syntax of an earlier historical
stage of the language, as \posscitet{givon:yesterdaysyntax} aphorism
``today's morphology is yesterday's synax'' would have it.

\subsection{The principles of necessity and sufficiency}
\label{sec:necess-sufficient}

Central to René's view of the relation between content and
morphological form are two complementary principles (1919:~13):
\begin{quotation}
  \textbf{Principle of necessity}. In the construction of a compound
  word it is necessary to introduce (by means of the law of reversal)
  all of the simple elements (roots and affixes) necessary to evoke
  clearly the idea that the word is to express (in the given
  circumstances).\medskip\\
  \noindent\textbf{Principle of sufficiency}: We must also, in this
  construction, avoid the introduction of useless pleonasms, as well
  as ideas foreign to the idea that we wish to express.
\end{quotation}

To accord with these two principles, a complex word constructed to
express a given complex idea should contain all and only those
elements whose content is contained therein, without duplication,
superfluous elements or gaps. Of course, it may well be the case that
the linguistic resources of a given language are not such as to make
it possible to include precisely the required content: it may be that
some part of the required meaning can only be introduced in
conjunction with other, superfluous material. In that case it is
necessary either to undershoot or to overshoot what is desired: either
to leave out some component of meaning or to include some extraneous
material.  In this case, René invokes a Principle of Least Effort: it
is better to omit (the minimum of) content than to include something
irrelevant. Thus (1911:~103--104), to produce a word meaning ``to
put a crown on the head of someone'' there is no suffix in French
meaning exactly ``to take an object and place it on the head of
someone; the best we can do is to add a verbalizing suffix to get
\emph{couronner} `to perform an action with a crown'.  Any other
suffix including more of the required meaning would also introduce
additional, unmotivated material, so the Principle of Least Effort
requires us to be content with this.

Here and elsewhere (cf. 1919:~6, fn.~1) we see the notion of ranked,
violable constraints familiar from modern Optimality Theory
\citep{prince:smolensky04:ot} and related frameworks: a number of
principles are enunciated, but when two of these conflict, one is
satisfied at the expense of the other.  In the example just
considered, the Principle of Sufficiency is presumed to outrank the
Principle of Necessity. Elsewhere (1919:~24), with respect to the general
principle that the head of a dissociated molecule is on the left,
another principle is formulated to the effect that in structures with
a prepositional element and its complement, the prepositional element
precedes.  Where these two ordering principles conflict, the
regularity governing the more specific case (preposition plus
complement) outranks that governing the general case of dissociated
molecules. 

As a consequence of the Principle of Sufficiency, the addition of an
affixal element to a base is always required to introduce additional
content, and not just to duplicate content already present in the
base: ``tout suffixe doit introduire dans le mot auquel on l'accole
une idée (générale ou particulière) qui n'y était pas
contenue.''\footnote{``every suffix must introduce into the word to
  which it is attached an idea (general or specific) which was not
  already contained in it.'' (1911:~95)}  This might seem trivial and
obvious, but it allows René to get some of the same results as those
that fall under the heading of \textsc{blocking}.

Famously, \pgcitet{aronoff:book}{43--44} claimed that words like
*\emph{gloriosity} are excluded in English because of the prior
existence in the lexicon of (essentially synonymous) \emph{glory}. For
René, this would follow from the fact that *\emph{gloriosity} looks
like it is built on \emph{glory}, and the suffixes \emph{-os-ity}
would not add any content. Actually, the facts here are less than
clear: various online dictionaries include \emph{gloriosity} as a word
of English, but assign it a meaning distinct from that of
\emph{glory}: thus, Merriam-Webster's \emph{Word Central} defines
\emph{gloriosity} as ``a moment or experience of glory.''  This
specific example apart, though, René gives a better example (1911:~95)
from French. He notes that the word *\emph{matronine} is impossible in
that language, because the content of the suffix \emph{-ine} `female
person' is already included in the base \emph{matrone} `matron, older
respectable woman'.  *\emph{Matronine} would thus include a ``useless
pleonasm'', and is accordingly blocked.

The effects of the Principle of Sufficiency thus include some, but
surely not all of those that have been attributed to blocking in the
modern literature. Nonetheless, it does have some of the flavor of
principles enforcing disjunctive operation on rules in morphology of a
language.

\subsection{Some limitations of René's theory}
\label{sec:limitations}

While the sections above illustrate some of the ways in which René's
theory of word formation anticipates later views, it would not do to
exaggerate the extent to which he provides an account with coverage or
adequacy equivalent to those we entertain today. There are in fact
some major limitations to the works presented in this book as a
comprehensive view of this domain of grammar.

One severe limitation concerns the degree of empirical coverage of
René's picture. He claims that this should be universally applicable:
``Les principes logiques de la formation des mots sont donc les mêmes
pour toutes les langues, du moins pour toutes celles qui partent des
mêmes éléments primitifs,''\footnote{``The logical principles of the
  formation of words are thus the same for all languages, or at least
  for all those that begin from the same primitive elements.''
  (1911:~4)} where the limitation to languages with the same
``primitive elements'' simply means those that build word from roots
and affixes (perhaps thereby excluding, e.g., root-and-pattern
morphology of the Semitic type, which may have been familiar to him).
But in fact he does not take account of any languages outside the set
of very familiar ones.  His examples are virtually all from French,
German and English, plus a few forms from another Indo-European
language, Albanian. It seems likely that he had no serious knowledge
of any others, although his brother Leopold was an amateur sinologist,
and Ferdinand is known to have obtained some Chinese material from
him.  A broader experience of the world's languages would surely have
convinced René of the need to modify or abandon some of his claims.

Another important limitation of René's theory is the complete lack of
a theory of \textsc{allomorphy}. He generally assumes that variation in the
shape of elements is due to some unformulated principles of ``euphony'',
although he mentions various instances of variation in shape that
cannot plausibly be attributed to the phonology.  In fact, he had no
interest in phonological variation, since his goal is the analysis of
the content of complex words, and its relation to the basic elements
of which these are composed. In terms of the traditional sub-parts of
a theory of morphology, he is really only interested in
\textsc{morphotactics}, the combination of meaningful elements and their
relation to meaning. 

A full theory of morphology needs to take allomorphy into
account too, and this does not engage his interest.  An important
reason for this apparent deficit was surely the fact that René was
ultimately interested in the principles of word formation in natural
language as a guide to the design of artificial languages, like
Esperanto -- and one of the guiding principles in designing such
languages is to minimize or eliminate anything like allomorphic
variation, so that content and form are related in as uniform a way as
possible.

\section{Conclusion: ``Saussurean'' morphology}
\label{sec:morphology-conclusion}

So how much attention should we pay to what René, or indeed either of
the brothers de Saussure, has to say about morphological theory?
 Obviously neither of them presents us with a comprehensive theory of
this area of linguistic structure. Indeed, both address morphological
issues against the background of other interests.

René was interested in the morphology of natural language as forming
the background to the design of an artificial language like
Esperanto. The aspects of natural language morphology that engage him
are just those that might come into play in this other enterprise.
Ferdinand, on the other hand, was primarily occupied (at least in the
\textsl{Cours}) with bigger issues related to more foundational
distinctions: synchrony \emph{vs.} diachrony, \emph{langue
  vs. parole}, etc.  Actually we can derive much more information about his views
on morphology in other work devoted to the analysis of specific
linguistic material.

Both brothers, though, really do have some articulated views on the
subject of word formation, views that are interesting not only as a
sort of historical curiosity.  And it is especially interesting, it
would seem, to observe that the basic difference between views of
complex words grounded in the combinatorics of morphemes
\emph{vs}. those based on rule-governed relations is already
prominent, and explicitly discussed, in their work.


\sloppy
\printbibliography[heading=subbibliography,notkeyword=this] 
\end{document}
 
