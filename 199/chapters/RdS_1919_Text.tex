%%% -*- Mode: LaTeX -*-

\addcontentsline{toc}{chapter}{La structure logique des mots -- The logical structure of words}\addchapmark{René de Saussure}\addsecmark{The 1919 Text}
\chapter*{La structure logique des mots -- The logical structure of words}
\label{ch.1919text}

\lettrine[loversize=0.1, nindent=0em]{O}{n the last page} of René de
Saussure's 1911 book, this is indicated as the ``Première partie'' of
a larger project. The implicitly promised ``deuxième partie'' of such
a project is represented by the first chapter of the work that follows
here. We reproduce (and translate) only the initial chapter of this
little book; its second chapter concerns the application of René de
Saussure's theory of word structure to artificially constructed
languages, and this is followed by an appendix providing ``all the
grammar, the syntax and the formation of words in \emph{Esperantido}''
{[René de Saussure's revised version of Esperanto]}, along with a
short text in that language.  As this material is not directly
relevant to our concerns with René de Saussure's theory of word
structure in natural languages, we do not include it here.

The conventions for translation adopted are largely the same as for the
first book: French words cited as examples have been preserved, with
the first instance of a given word provided with an English
gloss. Since only a rather small set of French words are invoked,
however, and many of these are closely cognate with their English
equivalents, we have frequently dispensed with glosses after a word's
first appearance. French words cited as concepts or ideas have
generally been translated except where this would impair the sense of
the text (in which case they have been treated in the same manner as
examples). As before, our goal in the translation provided here is to
provide access to the French text for the English reader, rather than
a new English version of René de Saussure's work. The pagination of
the original text has been preserved and indicated at the top of the
page, although no attempt has been made to maintain the division of
pages into lines.  As in the case of the 1911 text, we have retained
the original typography as far as possible, with inserted obviously
missing material enclosed in square brackets.

We have not included the original cover, whose content and
organization are identical with the title page here except for
identifying the work as published by ``Librairie A. LEFILLEUL,
Christoffelgasse --- Berne, 1919.''  As noted in the introduction, we
assume the publication date of 1919 should take priority over the
printing date of 1918 in referring to the book. Although some
references to this work in the literature treat it as having appeared
in 1918, it is clear that 1919 is more accurate: this is confirmed by
\label{page:date.1919}
the closing at the end of chapter 2 (not included here), signed on
page 58 ``Berne, 17 mars 1919. ANTIDO'' (this last being René de
Saussure's \emph{nom de plume} in his esperantist writings).

% \newpage
\FrenchPage[no]{
  
\vspace*{2in}
\begin{center}
{\huge \textbf{LA STRUCTURE}}\\[2ex]

{\Large LOGIQUE}\\[2ex]

{\Huge \textbf{DES MOTS}}\\[3ex]

{\large DANS LES LANGUES NATURELLES,}\\[1ex]

CONSIDÉRÉE AU POINT DE VUE DE SON\\[1ex]

{\large APPLICATION AUX LANGUES ARTIFICIELLES}\\[2ex]

PAR\\[2ex]

{\huge RENÉ de SAUSSURE}\\[1ex]

Ancien élève de l’Ecole Polytechnique de Paris,

Ph. D. and Fellow by courtesy of the Johns Hopkins University,

Lauréat de l’Institut de France\\[2ex]

\rule{1in}{.4pt}

\vspace*{3ex}
\end{center}
{\raggedleft
  \emph{„Une analyse exacte de la signification\\
  des mots ferait mieux connaître que toute\\
  autre chose les opérations de l'entendement.“}
  
\textsi{leibnitz}.\\}

\vspace*{4ex}
\begin{center}
  IMPRIMERIE BÜCHLER \& C\sxsuper{IE}, BERNE\\
  1918
\end{center}
}

\EnglishPage[no]{

\vspace*{2in}
\begin{center}
{\Huge THE LOGICAL}\\[2ex]

{\Large STRUCTURE}\\[2ex]

{\Huge OF WORDS}\\[3ex]

{\large IN NATURAL LANGUAGES,}\\[1ex]

CONSIDERED FROM THE POINT OF VIEW OF ITS\\[1ex]

{\large APPLICATION  TO ARTIFICIAL LANGUAGES}\\[2ex]

BY\\[2ex]

{\huge RENÉ de SAUSSURE}\\[1ex]

Former student at the École Polytechnique de Paris,

Ph. D. and Fellow by courtesy of the Johns Hopkins University,

Lauréat of the Institut de France\\[2ex]

\rule{1in}{.4pt}

\vspace*{3ex}
\end{center}
{\raggedleft
  \emph{``An exact analysis of the meaning\\
  of words would make known better than any\\
  other thing the operations of understanding.''}
  
  
\textsi{leibnitz}.\\}

\vspace*{4ex}
\begin{center}
  IMPRIMERIE BÜCHLER \& C\sxsuper{IE}, BERNE\\
  1918
\end{center}
  }

\FrenchPage[no]{
\begin{center}
{\Huge LA STRUCTURE}\\[1ex]

{\Large LOGIQUE}\\[1ex]

{\Huge DES MOTS}\\[2ex]

{\large dans les langues naturelles,}\\[1ex]
considérée au point de vue de son\\[1ex]
application aux langues artificielles.\\[2ex]

\rule{1in}{.4pt}

\vspace*{3ex}
{\Large Préliminaire.}
\end{center}

Les idées développées dans le présent essai ont été en partie déjà
exposées dans une publication antérieure, parue sous le titre de
\emph{Principes logiques de la formation des mots}.\footnote{Genève,
  1911. Librairie Kündig.}

Au lieu de publier aujourd’hui la deuxième partie de ce travail, il
m’a paru préférable de refondre le tout en un seul article, plus
condensé et mieux ordonné.  Je profite de cette occasion pour faire
quelques remarques préliminaires, qui m’ont été suggérées par la
lecture des compte-rendus auxquels ma première brochure a donné lieu.

Les linguistes considèrent généralement les faits linguistiques au
point de vue historique, évolutif, tandis que mon but est d’étudier la
structure des mots dans les langues considérées à une époque donnée de
leur existence. Une telle étude appartient plutôt au domaine du
logicien qu’à celui du linguiste; seule la matière qui entre en jeu
est la même.

Ainsi, par exemple, lorsque j’admets que les éléments simples (racines
ou aflixes), qui entrent dans la composition des mots, sont des
éléments \emph{invariables}, cela ne signifie pas que ces éléments
sont invariables dans le temps,\footnote{Comme me le faisait dire
  l’auteur du compte-rendu de ma brochure, paru dans le \emph{Journal
    de Genève} du 20 novembre 1911.} mais que, dans une langue
considérée à une époque donnée, ces éléments restent les mêmes
lorsqu’on les transporte d’un mot dans un autre. Ceci revient à dire
que, par exemple, le mot \emph{grand} reste toujours le même mot-adjectif,
qu’on le considère soit comme mot autonome,}

\EnglishPage[no]{
  \begin{center}
{\Huge The Logical Structure}\\[1ex]

{\Huge of Words}\\[2ex]

{\large in natural languages,}\\[1ex]
considered from the point of view of its\\[1ex]
application to artificial languages.\\[2ex]

\rule{1in}{.4pt}

\vspace*{3ex}
{\Large Preliminaries.}
\end{center}

The ideas developed in the present essay have already been presented
in part in a previous publication, which appeared with the title
\emph{Logical Principles of Word Formation.}\footnote{Geneva, 1911,
  Librairie Kündig.}

Instead of publishing now the second part of that work, it has seemed
preferable to me to recast the whole in a single article, more
condensed and better organized. I take advantage of this occasion to
make some preliminary remarks, which have been suggested to me in
reading the reviews to which my first booklet gave rise.

Linguists have generally considered linguistic facts from the
historical, evolutionary point of view, while my aim is to study the
structure of words in languages considered at a given time in their
existence.  Such a study belongs to the domain of the logician rather
than that of the linguist; only the material that comes into play is
the same.

Thus, for example, when I suppose that the simple elements (roots or
affixes) that enter into the composition of words are
\emph{invariable} elements, that does not mean that these elements are
invariable over time,\footnote{As the author of the review which
  appeared in the \emph{Journal de Genève} for 20 November, 1911,
  would have me say. {[See pp. \pageref{pg:JdG-review}ff. of the
    present volume]}} but that, in a language considered at a given
point in time, these elements remain the same when one transports them
from one word into another. This amounts to saying, for example, that
the word \emph{grand} `large, tall' still remains the same adjectival
word whether we consider it as an autonomous word }

\FrenchPage{\protect\noindent soit comme faisant partie des mots
  \emph{grand'eur}, \emph{grand'ir}, \emph{s’a'grand'ir},
  \emph{grand'duc}, etc. Cette remarque est importante: elle montre
  par exemple, que des verbes comme \emph{couronn'er},
  \emph{bross'er}, \emph{clou'er}, etc., ne sont pas des verbes
  simples comme \emph{frapp'er}, \emph{écri're}, etc., mais de vrais
  mots composés, formés d’un substantif (\emph{couronne},
  \emph{brosse}, \emph{clou}, etc.) et d’un affixe verbal (\emph{er})
 ; en d’autres termes, dans le verbe \emph{frapper} l’idée verbale
  pénètre non seulement la désinence \emph{er}, mais aussi le radical
  \emph{frapp}, tandis que dans le verbe \emph{couronn'er} l’idée
  verbale est contenue exclusivement dans la désinence \emph{er} (tout
  comme elle est contenue uniquement dans le dernier élément
  \emph{essen} du mot allemand \emph{Abend'essen}). Telle est
  l’interprétation qu’il faut donner au principe de l’invariabilité
  des éléments, énoncé à la page 10 sous le n° 7. Ce principe, du
  reste, revient à considérer les langues naturelles (y compris le
  français) comme des langues où les mots composés et les mots dérivés
  sont formés par la soudure d’éléments \emph{invariables} et
  \emph{indépendants} les uns des autres, éléments qui sont de
  véritables mots, puisque chacun d’eux est le signe d’une idée qui
  lui est propre.

  On voit qu’il n’est pas question ici d’étymologie; du reste il
  semble qu’actuellement les linguistes eux-mêmes admettent
  l’existence de deux sortes de recherches en linguistique. C’est du
  moins ce qui ressort clairement de l’article écrit par Monsieur
  A. Oltramare\footnote{Voir la \emph{Semaine littéraire} du 27 mai
    1916, p. 258. Genève.} à propos du \emph{Cours de linguistique
    générale}\footnote{Oeuvre posthume, publiée par les soins de
    messieurs Ch. Bally, professeur, et A. Sechehaye, privat-docent, à
    l’Université de Genève. Librairie Payot, Lausanne, 1916.} de mon
  frère Ferdinand de Saussure:

  \begin{quotation}
    .... „les historiens du langage, dit l’auteur de cet article,
    n’ont fixé que l’évolution de certains faits isolés; les
    grammairiens se sont contenté de déterminer dans la langue ce qui
    est correct et ce qui ne l’est pas; les phonologues ont seulement
    observé le mécanisme de l’instrument vocal .... Comment découvrir
    ainsi les lois universelles du langage? — En divisant la
    difficulté, répond F. de Saussure; en étudiant la langue non
    seulement dans son histoire, mais surtout dans son état actuel;
    en coordonnant les données de faits linguistiques simultanés. Il
    faut donc distinguer deux sortes de recherches: l’étude de
    l'évolution et celle d’une période donnée; il y a deux
    linguistiques: l’une est \emph{diachronique} (évolution), l’autre
    est \emph{synchronique} (état). La première détermine comment les
    vocables se substituent les uns aux autres dans le temps; elle
    conditionne la seconde dialectique, qui décrit les rapports de
    termes contemporains les uns des autres.“
  \end{quotation}
\noindent Et Monsieur Oltramare ajoute:

\begin{quotation}
  „C’est dans le domaine de la linguistique synchronique que F. de
  Saussure innove radicalement. L’analyse doit ici être subjective:
  elle ne s’occupe que des faits perçus par la conscience de la
  moyenne des sujets parlants. Un mot comme \emph{enfant} doit y être
  considéré comme un bloc indivisible, alors que l’analyse objective,
  en usage dans la diachronique, eût décomposé le même terme
  (\emph{en'fant}) et l’eût rapproché de \emph{in'fans} (non doué de
  la parole)“.
\end{quotation}
} 

\EnglishPage{\noindent
  or as constituting part of the words \emph{grand'eur} `size',
  \emph{grand'ir} `to grow', \emph{grand'duc} `grand duke', etc. This
  remark is important: it shows, for example, that verbs like
  \emph{couronn'er} `to crown', \emph{bross'er} `to brush',
  \emph{clou'er} `to nail', etc. are not simple verbs like
  \emph{frapp'er} `to strike', \emph{écri're} `to write', etc., but
  real compound words, formed with a noun (\emph{couronne} `crown',
  \emph{brosse} `brush', \emph{clou} `nail', etc.) and a verbal affix
  (\emph{er} `\textsc{infinitive}'); in other terms, in the verb
  \emph{frapper} the verbal idea enters into not only the desinence
  \emph{er}, but also the root \emph{frapp}, while in the verb
  \emph{couronn'er} the verbal idea is exclusively contained in the
  desinence \emph{er} (just as it is uniquely contained in the final
  element \emph{essen} `eat' of the German word \emph{Abend'essen}
  `dinner'). This is the interpretation that must be given to the
  principle of the invariability of elements, set out on page 10 under
  number 7. This principle, besides, amounts to considering natural
  languages (including French) as languages where compound words and
  derived words are formed by joining \emph{invariable} and
  \emph{independent} elements with one another, elements which are
  genuine words, since each of them is the sign of an idea that is
  proper to it.

  We see that it is not a question here of etymology; besides, it
  seems that currently linguists themselves admit the existence of two
  kinds of investigation in linguistics. At least that is what clearly
  emerges from the article written by Mr. A. Oltramare\footnote{See
    the \emph{Semaine littéraire} for 27 May, 1916, p. 258, Geneva.}
  in connection with the \emph{Cours de linguistique
    générale}\footnote{Posthumous work, published by the good
    offices of Ch. Bally, professor, and A. Sechehaye, privat-docent
    {[roughly, Lecturer]}, at the University of Geneva. Librairie
    Payot, Lausanne, 1916.} of my brother Ferdinand de Saussure:
  \begin{quotation}
    \ldots ``historians of language, says the author of this article,
    have only determined the evolution of certain isolated facts;
    grammarians have been content to determine what is correct and
    what is not in the language; phonologists have only observed the
    mechanism of the vocal instrument \ldots\ How are the universal
    laws of language to be discovered in this way? --- In dividing up
    the problem, F. de Saussure responds; in studying language not
    only in its history, but especially in its current state; in
    coordinating the elements of simultaneous linguistic facts.  It is
    thus necessary to distinguish two sorts of investigation: the
    study of evolution, and that of a given period; there are two
    sorts of linguistics: one is \emph{diachronic} (evolution), and
    the other is \emph{synchronic} (state). The first determines how
    spoken words substitute for one another over time; this depends on
    the second approach, which describes the relations of
    contemporaneous terms to one another.''
  \end{quotation}

  \noindent
  And Mr. Oltramare adds:

  \begin{quotation}
    ``It is in the domain of synchronic linguistics that F. de
    Saussure makes radical innovations. The analysis here must be
    subjective: it is concerned only with facts perceived by the
    average speaker. A word like \emph{enfant} `child' must here be
    considered as an indivisible whole, while the objective analysis,
    as employed in diachrony, would have decomposed the same term
    (\emph{en'fant}) and would have compared it with \emph{in'fans}
    (not endowed with speech).''
  \end{quotation}
}

\FrenchPage{

  J’avais fait moi-même la même remarque à propos du mot
  \emph{musique}\footnote{Voir \emph{Formation des mots}, p. 120.}
  (\emph{mus'ique}, ancien adjectif de \emph{muse}) que l’on doit
  considérer actuellement comme un mot simple substantif, donnant
  naissance lui-même à de nouveaux adjectifs, tels que
  \emph{music'al}, \emph{music'ien}, etc., où le radical \emph{music}
  joue le rôle d’un élément simple.

  D’une manière générale, on peut dire que tous les mots composés
  tendent à devenir simples, car tout mot en évoluant tend à perdre sa
  signification primitive et à en acquérir une nouvelle, qui n’est par
  conséquent plus conforme à sa structure; mais cette évolution
  n’empêche pas l’analyse logique des mots en linguistique
  synchronique, parce qu’elle est très lente; on peut même dire,
  qu’elle est négligeable pour tous les mots composés qui rentrent
  dans un type général. Ainsi les mots tels que \emph{beau'té}, \emph{plén'itude},
  en français, \emph{equal'ity}, en anglais, \emph{Schön'heit}, en allemand, etc.,
  forment toute une catégorie de mots dont la structure est encore
  aujourd’hui exactement la même que celle des mots latins
  correspondants \emph{ver'itas}, \emph{pulchr’itudo}, etc.; leur signification,
  est bien restée conforme à leur structure, puisque les suffixes \emph{ité},
  \emph{itude}, \emph{heit}, etc., expriment tous l’idée substantive générale de
  „\emph{chose (en général)}“, „\emph{chose abstraite}“, et que tous les mots que
  nous venons de citer sont bien destinés à représenter, sous la forme
  d’une chose abstraite („\emph{beau-té}“), une idée adjective („\emph{beau}“), qui
  par elle-même n’est pas une chose.

  Dans un court compte-rendu\footnote{Voir les Mémoires de la Société
    de linguistique, 1911, Paris.} que Monsieur le professeur Meillet
  a bien voulu faire de mon premier travail, je trouve la remarque
  suivante: „M. R. de Saussure, dit-il, recherche non ce qui est,
  mais ce qui doit être.“

  Si c’est là une critique, je puis répondre que les grammairiens font
  à peu près la même chose, puisqu’ils déterminent dans la langue ce
  qui est correct et ce qui ne l’est pas. Mais tandis que le
  grammairien se place au point de vue de l’usage établi dans une
  langue particulière, nous nous plaçons au point de vue international
  des langues en général, et nous recherchons, parmi toutes les formes
  existantes, celles qui ont un caractère incontestable de
  généralité. En tout cas, la remarque faite par Monsieur Meillet est
  intéressante, et elle nous donne l’occasion de préciser l’objet que
  nous avons en vue: c’est par la constatation de ce qui \emph{est}
  général dans les langues naturelles que nous trouverons ce qui
  \emph{doit être} dans une langue artificielle pour que son mécanisme
  se rapproche le plus possible de celui des langues naturelles.

  Evidemment, les lois générales sont aussi difficiles à percevoir
  dans les faits linguistiques que les lois de la physique dans les
  phénomènes biologiques ou physiologiques, à cause de la complexité
  et de la variabilité des organismes vivants; mais cela ne veut pas
  dire que ces lois n’existent pas; le tout est de les découvrir sous
  l’apparente complexité des formes.

  Dira-t-on, par exemple, qu’il n’existe pas de loi de numération dans
  les langues naturelles, parce que certains noms de nombres, comme
  \emph{onze}, \emph{douze}, \emph{treize}, etc., n’ont pas leur forme
  régulière (\emph{dix-un}, \emph{dix-deux}, \emph{dix-trois}, etc.) ?
  ou parce que d’autres ont pris des formes excep-}

\EnglishPage{
   I had myself made the same comment with respect to the word
  \emph{musique} `music'\footnote{See \emph{Formation des mots},
    p.120} (\emph{mus'ique}, former adjective from \emph{muse}), which
  must now be considered a simple noun, itself giving rise to new
  adjectives such as \emph{music'al} `musical', \emph{music'ien}
  `musician', etc., where the root \emph{music} plays the role of a
  simple element.

  In general, we can say that all compound words tend to become
  simple, because every word tends in its evolution to lose its
  original meaning and to acquire a new one, which as a result no
  longer conforms to its structure; but this evolution does not
  prevent the logical analysis of words in synchronic linguistics,
  because it is very slow; we can even say that it is negligible for
  all compound words that fit into a general type. Thus, words like
  \emph{beau'té} `beauty', \emph{plén'itude} `fullness' in French,
  \emph{equal'ity} in English, \emph{Schön'heit} `beauty' in German,
  etc. all make up a category of words whose structure is still today
  exactly the same as that of corresponding Latin
  words\emph{ver'itas}, \emph{pulchr'itudo}, etc.; their meaning has
  remained conformant to their structure, since the suffies
  \emph{ité, itude, heit,} etc. all express the genral nominal idea
  of ``\emph{thing (in general)}'', ``abstract thing'', and all the
  words that we have just cited are well designed to represent, in the
  form of an abstract thing (``\emph{beau-té}'') an adjectival idea
  (``\emph{beau}'' `beautiful') which by itself is not a thing.

  In a short review\footnote{See the Mémoires de la Société de
    linguistique, 1911{[\emph{sic}]}, Paris. {[Reproduced on
      pp. \pageref{sec:meillet-review}f. of the present volume]}} that
  Professor Meillet has been so kind as to provide of my first work, I
  find the following remark: ``Mr. R. de Saussure, he says, studies
  not what is, but what must be.''

  If that is a criticism, I can reply that grammarians do almost the
  same thing, since they determine in language what is correct and
  what is not. But while the grammarian takes the point of view of the
  usage established in a particular language, we take the
  international point of view of languages in general, and we study,
  among all existing forms, those that have an incontestable character
  of generality.  In any case, the remark made by Mr. Meillet is
  interesting, and it gives us the opportunity to clarify the object
  we have in mind: it is by the investigation of what \emph{is}
  general in natural languages that we find what \emph{must be} in an
  artificial language for its mechanism to come as close as possible
  to that of natural languages.

  Obviously, general laws are as difficult to perceive in linguistic
  facts as are physical laws in biological or physical phenomena,
  because of the complexity and variability of living organisms; but
  that does not mean that these laws do not exist. The main thing is
  to uncover them under the apparent complexity of forms.

  Will we say, for example, that there does not exist a law of
  numeration in natural languages, because the names of certain
  numbers like \emph{onze} `eleven ', \emph{douze} `twelve',
  \emph{treize} `thirteen, etc. do not have their regular form
  (\emph{dix-un} `ten-one', \emph{dix-deux} `ten-two',
  \emph{dix-trois} `ten-three', etc.)?  Or because others take
  excep{[tional]} forms,
  }

\FrenchPage{\protect\noindent tionnelles, comme \emph{soixante-dix},
  \emph{quatre-vingt}, etc.?  Evidemment non. Il est clair que si
  quelques nombres font exception à la règle, cela vient uniquement de
  la fréquence de leur emploi, qui les a détériorés en vertu de la loi
  du moindre effort.\footnote{Ces exceptions n’infirment pas la loi
    générale; elles sont dues uniquement à l’intervention d’autres
    causes entrant en conflit avec cette loi. On peut comparer la loi
    générale à un système de tranchées défensives dans l’art
    militaire: un tel système est établi suivant un plan logique et ce
    plan logique subsiste alors même que d’autres causes, par exemple
    une attaque ennemie, en aurait détruit une partie.} Mais la loi de
  numération n’en existe pas moins et la preuve, c’est qu’on la trouve
  encore intacte dans certaines langues, comme l’albanais, où elle a
  conservé une forme absolument régulière:\\

  \noindent
  1 (\emph{nje}), 2 (\emph{d\'u}), 3 (\emph{tri}), 4 (\emph{kater}), 5
  (\emph{pés}), 6 (\emph{kjast}), 7 (\emph{st\'at}), 8 (\emph{tét}), 9
  (\emph{n\'ant}), 10 (diét\emph{}); 11 (\emph{diét e nje}), 12
  (\emph{diét e d\'u}), 13 (\emph{diét e tri}), 14 (\emph{diét e
    kater}), etc.; 20 (\emph{d\'u-diet}), 21 (\emph{d\'udiét e nje}),
  22 (\emph{d\'udiét e d\'u}), etc.; 30 (\emph{tri-diét}), etc.; 40
  (\emph{kater-diét}), etc., etc.\\

  Ainsi c’est bien ce qui „est“ généralement dans les langues
  naturelles, et en particulier en albanais, qui conditionne ce que
  „doit être“ le système de numération dans une langue artificielle.

  De même, pour juger de la structure des mots dans une langue
  artificielle, il est nécessaire d’étudier d’abord cette structure
  dans les langues naturelles. Mais cela ne signifie pas que dans ces
  dernières tous les mots composés suivent la loi générale, ou qu’ils
  aient tous une signification conforme à leur structure. Dans le
  \emph{Cours de linguistique générale} cité plus haut, l’auteur
  (parlant des langues naturelles) fait remarquer avec raison (p. 187)
  que les mots sont des signes linguistiques plus ou moins
  \emph{motivés}; entre le signe tout à fait arbitraire et le signe
  tout à fait motivé il y a des degrés. Qu’est-ce à dire, si ce n’est
  que les signes \emph{arbitraires}, ou \emph{immotivés}, sont les
  mots simples qui servent de point de départ à la formation des mots
  composés (comme en algèbre des lettres arbitraires \emph{a, b, x,
    y}, etc. servent de point de départ aux formules); que les mots
  \emph{complètement motivés} sont les mots composés qui ont une
  signification conforme à leur contenu, et que les mots
  \emph{partiellement motivés} sont ceux dont la signification n’est
  que partiellement expliquée par leur contenu.\footnote{L’expression
    \emph{complètement motivé} ne doit pas être prise dans un sens absolu,
    car un mot composé n’est jamais la description complète d’une
    idée; il exprime seulement, par une sorte de logique
    différenciative, ce en quoi cette idée diffère des autres de même
    espèce. On peut dire qu’un mot est \emph{complètement motivé} lorsqu’il
    satisfait aux deux principes de nécessité et de suffisance,
    exposés à la page 13.}

  Les différentes langues naturelles sont plus ou moins riches en
  mots complètement motivés. Pour dégager les lois générales de la
  formation des mots, on devra donc s’appuyer de préférence sur les
  langues qui, comme l’allemand, sont riches en mots de cette
  espèce. C’est ce que nous ferons dans le présent essai, parce que
  nous avons l’intention d’appliquer ensuite ces lois aux langues
  artificielles, et il est bien évident qu’une langue artificielle
  sera d’autant plus à la portée de tout le monde, qu’elle sera plus
  riche en mots motivés, car alors le}

\EnglishPage{\protect\noindent [excep]tional {[forms]}, such as \emph{soixante-dix}
  `sixty-ten: seventy', \emph{quatre-vingt} `four-\linebreak twenty: eighty',
  etc.?  Obviously not. It is clear that if some numbers constitute
  exceptions to the rule, that comes solely from the frequency of
  their use, which has caused them to deteriorate as a result of the
  law of least effort.\footnote{These exceptions do not weaken the
    general law; they are only the result of the intervention of other
    causes that are in conflict with that law. We can compare the
    general law to a system of defensive trenches in the art of war:
    such a system is established according to a logical plan, and this
    logical plan remains even when as a result of other causes, such
    as an enemy attack, a part of it is destroyed.} But the law of
  numeration exists nonetheless, and the proof is that we find it
  still intact in certain languages, such as Albanian, where it has
  been preserved in absolutely regular form:\\

  \noindent
  1 (\emph{nje}), 2 (\emph{d\'u}), 3 (\emph{tri}), 4 (\emph{kater}), 5
  (\emph{pés}), 6 (\emph{kjast}), 7 (\emph{st\'at}), 8 (\emph{tét}), 9
  (\emph{n\'ant}), 10 (diét\emph{}); 11 (\emph{diét e nje}), 12
  (\emph{diét e d\'u}), 13 (\emph{diét e tri}), 14 (\emph{diét e
    kater}), etc.; 20 (\emph{d\'u-diet}), 21 (\emph{d\'udiét e nje}),
  22 (\emph{d\'udiét e d\'u}), etc.; 30 (\emph{tri-diét}), etc.; 40
  (\emph{kater-diét}), etc., etc.\\

  Thus it is just what ``is'' in general in natural languages, and in
  particular in Albanian, that conditions what ``must be'' the system
  of numeration in an artificial language.

  Similarly, to evaluate the structure of words in an artificial
  language, it is necessary first to study this structure in natural
  languages. But that does not mean that in the latter all compound
  words will follow the general law, or that they all should have a
  meaning in conformance with their structure. In the \emph{Cours de
    linguistique générale} cited above, the author (speaking of
  natural languages) observes correctly (p. 187) that words are more
  or less \emph{motivated} linguistic signs: between the completely
  arbitrary sign and the completely motivated sign there are
  degrees. Which is to say that if it is not \emph{arbitrary} or
  \emph{unmotivated} signs, it is simple words that serve as the point
  of departure for the formation of compound words (as in algebra the
  arbitrary letters \emph{a, b, x, y}, etc. serve as the point of
  departure for formulas); that \emph{completely motivated} words are
  compound words that have a meaning in accord with their content, and
  that \emph{partially motivated} words are those for which the
  meaning is only partially explained by their content.\footnote{The
    expression \emph{completely motivated} must not be taken in an
    absolute sense, because a compound word is never the complete
    description of an idea; it only expresses, by a sort of
    differentiative logic, that in which this idea differs from others
    of the same sort.  We could say that a word is \emph{completely
      motivated} when it satisfies the two principles of necessity and
    sufficiency, presented on page 13 below.}

  Different natural languages are more or less rich in completely
  motivated words. To bring out the general laws of the formation of
  words, we will thus have to depend preferably on languages, like
  German, that are rich in words of that type. That is what we will do
  in the present essay, because we intend to apply these laws
  subsequently to artificial languages, and it is quite obvious that
  an artificial language will be more accessible to everyone the
  richer it is in motivated words, because then the

  }
  

\FrenchPage{\protect\noindent nombre des signes arbitraires de la langue, c’est-à-dire
  le vocabulaire des mots simples que l’on est obligé d’apprendre par
  cœur, sera réduit à un minimum. L’exemple, choisi plus haut, du
  système de numération dans les langues naturelles est tout à fait
  frappant: en albanais, tous les noms de nombres sont entièrement
  motivés, à l'exception des nombres 0, 1, 2, 3, 4, 5, 6, 7, 8, 9, 10,
  100, 1000, etc., qui sont immotivés; en français, outre les nombres
  précédents, le nombre 20 est immotivé; les nombres 13, 14, 15, 16,
  30, 40, 50, etc. ne sont que partiellement motivés; en allemand,
  la numération est encore moins bonne: certains nombres, comme
  \emph{dreizehn}, sont bien complètement motivés, mais il le sont à
  rebours de la loi générale, et il en résulte que des noms de nombre
  à structure semblable, comme \emph{dreizehn} et \emph{dreihundert}
  n’ont pas du tout des significations semblables. Cette remarque est
  importante, car elle montre qu’en étudiant la structure des mots
  dans les langues naturelles, on pourra rencontrer dans telle ou
  telle langue particulière des exceptions à la loi générale; mais
  ces exceptions n’infirment pas la règle, si l’on se place au point
  de vue des langues en général, ou, si l’on veut, au point de vue
  international.

  En résumé, si nous adoptons le terme de \emph{lexicologique} pour désigner
  les langues riches en mots immotivés et celui de \emph{grammatical} pour
  désigner celles qui sont riches en mots motivés,\footnote{Voir le
    \emph{Cours de ling. gén.} déjà cité, pag. 189.} nous arrivons à
  cette conclusion qu’une langue artificielle doit être construite sur
  le type „grammatical“ et qu’elle est capable de réaliser ce type
  infiniment mieux qu’aucune langue naturelle, puisque dans une langue
  artificielle rien n’empêche de réduire au minimum le nombre des mots
  immotivés, et de remplacer tous les mots qui ne sont que
  partiellement motivés par des mots qui le sont entièrement, comme
  nous le verrons au second chapitre.
  
  {\raggedleft R. de S.\\}}

\EnglishPage{\protect\noindent number of arbitrary signs in the language, that is, the
  vocabulary of simple words that one is obliged to learn by heart,
  will be reduced to a minimum. The example chosen above of the system
  of numeration in natural languages is quite striking: in Albanian,
  all of the names of numbers are entirely motivated, with the
  exception of the numbers 0, 1, 2, 3, 4, 5, 6, 7, 8, 9, 10, 100,
  1000, which are unmotivated; in French, besides the preceding
  numbers, the number 20 is unmotivated; the numbers 13, 14, 15, 16,
  30, 40, 50 etc. are only partially motivated; in German, the
  numeration is still less good: certain numbers, like \emph{dreizehn}
  `thirteen' are completely motivated, but they are so contrary to the
  general law, and the result is that number names with similar
  structure, like \emph{dreizehn} `thirteen' and \emph{dreihundert}
  `three hundred' do not at all have similar meanings. This remark is
  important, because it shows that in studying the structure of words
  in natural languages, we may encounter in one or another language
  exceptions to the general law; but these exceptions do not weaken
  the law, if we take the point of view of languages in general, or,
  if you like, the international point of view.

  In summary, if we adopt the term \emph{lexicological} to designate
  languages rich in unmotivated words, and the term \emph{grammatical}
  to designate those that are rich in motivated words,\footnote{See
    the \emph{Cours de linguistique générale}, cited above, page
    189.}  we will arrive at the conclusion that an artificial
  language must be constructed to be of the ``grammatical'' type and that
  it is capable of realizing this type infinitely better than any
  natural language, since in an artificial language, nothing prevents
  us from reducing to a minimum the number of unmotivated words, and
  replacing all the words that are only partially motivated with words
  that are entirely so, as we will see in the second chapter.

   {\raggedleft R. de S.\\}}
   
   \clearpage\chead{} \noindent\relax[Blank page in original]\cleardoublepage
   \setcounter{TextPageNo}{8}
 
\FrenchPage[no]{
  \begin{center}
    CHAPITRE PREMIER.\\[2ex]

    {\Large\textbf{La structure logique des mots\\[1ex]
        dans les langues naturelles.}}\\[2ex]

§ 1. \textbf{\textsf{Principes généraux.}}\\[2ex]

\textbf{A. Définitions.}\\[1ex]
  \end{center}

  1. \emph{Un mot est le signe usuel au moyen duquel on exprime une
    idée}. — Faire l'\emph{analyse} d’un mot, c’est rechercher l’idée
  exprimée par ce mot. Au contraire, faire la \emph{synthèse} d’un
  mot, c’est construire le mot qui doit évoquer une idée
  donnée. L’analyse est donc faite par le lecteur ou l’auditeur,
  tandis que la synthèse est faite par l’écrivain ou l’orateur.

  2. \emph{Un mot simple est un mot qui ne contient qu'un seul
    élément} (ex.: maison). — Il existe trois espèces de mots simples
  ou éléments: les mots \emph{racines} ou mots autonomes (ex.:
  \textbf{\textsf{homme}}); les mots \emph{préfixes} qui se placent
  avant une racine (ex.: \textsf{\textbf{re}} dans
  \textbf{\textsf{re'tirer}}); enfin les mots \emph{suffixes} et les
  mots\footnote{Nous prenons le mot „\emph{mot}“ dans le sens
    d'\emph{unite significative}. Voir la définition ci-dessus (N°
    1).} \emph{désinences}, qui se placent après une racine (ex.:
  \textsf{\textbf{iste}} dans \textsf{\textbf{violon'iste}};
  \textsf{\textbf{er}} dans \textsf{\textbf{couronn'er}}).

  En général, les affixes (préfixes, suffixes ou désinences) ne sont
  pas des mots autonomes; cependant les préfixes-prépositions, comme
  \textsf{\textbf{sous}} dans \textsf{\textbf{sou'tirer}}, et certains
  suffixes, comme \textsf{\textbf{full}} dans le mot anglais
  \textsf{\textbf{beauti'ful}}, sont des mots autonomes.

  3. \emph{Un mot composé est un mot formé par la soudure de plusieurs
    mots simples ou éléments} (ex.: \textsf{\textbf{hum'an'ité}},
  \textsf{\textbf{steam'ship}}, etc.). — Il n’y a pas de différence
  essentielle entre un mot „composé“ de plusieurs mots autonomes,
  comme \textsf{\textbf{Schreib'tisch}}, et un mot „dérivé“ d’un mot
  autonome par l’adjonction d’affixes, comme
  \textsf{\textbf{grand'eur}}, \textsf{\textbf{couronn'er}},
  \textsf{\textbf{hum'an'ité}}, etc. On peut donc considérer tous les
  mots dits „dérivés“, comme des mots „composés“ par soudure. Ainsi,
  dans l’analyse des mots, il ne faut pas dire que
  \textsf{\textbf{couronn}} est le}

\EnglishPage[no]{
  \begin{center}
    CHAPTER ONE.\\[2ex]

    {\Large\textbf{The logical structure of words\\[1ex]
        in natural languages.}}\\[2ex]

§ 1. \textbf{\textsf{General principles.}}\\[2ex]

\textbf{A. Definitions.}\\[1ex]
  \end{center}

  1. \emph{A word is the usual sign by means of which we express an
    idea.} --- To \emph{analyze} a word is to look for the idea
  expressed by that word. Conversely, to \emph{synthesize} a word is
  to construct the word that evokes a given idea. The analysis is thus
  performed by the reader or the hearer, while the synthesis is
  performed by the writer or the speaker.

  2. \emph{A simple word is a word that contains only a single
    element} (e.g. maison `house'). --- There are three types of
  simple words or elements: \emph{root} words or autonomous words
  (e.g. \textup{homme} `man'); \emph{prefix} words, which are placed
  before a root (e.g. \textup{re} in \textup{re'tirer} `remove'); and
  finally \emph{suffix} words and \emph{desinence} words\footnote{We
    take the word ``word'' in the sense of \emph{meaningful unit}. See
    the definition above (\# 1).} which are placed after a root
  (e.g. \textup{iste} in \textup{violon'iste} `violinist'; \textup{er}
  in \textup{couronn'er} `to crown').

  In general, affixes (prefixes, suffixes and desinences) are not
  autonomous words; however preposition-prefixes, like \textup{sous}
  `under' in \textup{sou'tirer} `extract', and certain suffixes, like
  \textup{full} in the English word \textup{beauti'ful}, are
  autonomous words.

  3. \emph{A compound word is a word formed by joining together
    several simple words or elements} (e.g. \textup{hum'an'ité}
  `humanity', \textup{steam'ship}, etc.). --- There is no essential
  difference between a word ``compounded'' of several autonomous
  words, like \emph{Schreib'tisch} `writing table: desk' and a word
  ``derived'' from an autonomous word by the addition of affixes, like
  \textup{grand'eur} `size', \textup{couronn'er} `to crown',
  \textup{hum'an'ité} `humanity' etc. We can thus consider all words
  called ``derived'' as words ``compounded'' by joining.  Thus, in the
  analysis of words, it is not necessary to say that \textup{couronn}
  `crown' is the

  }

\FrenchPage{\protect\noindent radical du verbe \textsf{\textbf{couronner}}, mais que
  \textsf{\textbf{couronn}} est un substantif, qui avec l’affixe
  verbal \textsf{\textbf{er}} forme le verbe
  \textsf{\textbf{couronner}}.\footnote{Cette remarque revient à dire
    que les langues dites à \emph{flexion} sont en réalité des langues
    \emph{à soudure}, lorsqu’on fait une analyse rationnelle de leurs
    mots.}

  4. \emph{On nomme pléonasme la répétition de la même idée au moyen
    de mots superflus}. — En général, les pléonasmes n’ont pas
  d’utilité et ne font qu’alourdir l’expression (ex.: le mot allemand
  \textsf{\textbf{Prinz'ess'in}} contient un pléonasme, puisque les
  suffixes \textsf{\textbf{ess}} et \textsf{\textbf{in}} expriment
  tous deux la même idée féminine). Toutefois, comme les pléonasmes ne
  modifient pas la signification du mot ou de la phrase qui les
  contient, on les emploie souvent pour renforcer l’idée à exprimer
  (ex.: \textsf{\textbf{non, non !}} est une expression plus forte
  que simplement \textsf{\textbf{non !}}).

  Réciproquement, si à un mot (ou à une phrase) on ajoute un ou
  plusieurs mots, et que cette addition ne change pas le sens du mot
  (ou de la phrase) donné, on peut en déduire que les mots ajoutés
  forment un simple pléonasme, c’est-à-dire que l’idée qu’ils
  expriment était déjà contenue dans les données
  primitives.\footnote{Cette remarque est importante et nous aurons
    plusieurs fois l’occasion de l’appliquer à l’analyse des mots.}

  5. \emph{On dit que deux mots sont synonymes lorsqu’ils évoquent la
    même idée, ou des idées presque identiques.} — La synonymie peut
  donc être traduite par le signe =, à condition de ne pas attribuer à
  ce signe une valeur aussi absolue qu’en mathématique (ex.:
  \textsf{\textbf{bonheur}} = \textsf{\textbf{félicité}}, signifiera
  simplement que ces mots sont des synonymes).

  6. \emph{La signification d’un mot est l'idée évoquée par ce mot,
    tandis que son sens n’est que l’un des aspects opposés sous
    lesquels ce mot peut être considéré.} — Ainsi par exemple, tout
  mot peut être pris au sens propre ou au sens \emph{figuré}; tout nom
  peut être pris au sens \emph{concret} ou au sens \emph{abstrait},
  etc.\\

  \begin{center}
    \textbf{B. Analyse des mots.}
  \end{center}


  7. \textsi{Principe de l'invariabilité les éléments.}
  \emph{Tout élément simple (racine ou affixe) forme un tout
    invariable, qui a sa signification propre.} — Par exemple, dans
  les mots \textsf{\textbf{grand, grandeur, agrandir,}} etc.,
  l’élément \textsf{\textbf{grand}} est toujours le même individu; le
  mot \textsf{\textbf{grandeur}} n’est donc pas un substantif simple
  comme \textsf{\textbf{maison}}; c’est un mot composé d’un élément
  adjectif, \textsf{\textbf{grand}}, et d’un élément substantif,
  \textsf{\textbf{eur}}.

  8. \emph{Un mot-racine exprime plutôt une idée particulière} (ex.:
  \textsf{\textbf{éléphant}}), \emph{tandis qu’un mot-affixe exprime
    toujours une idée générale} (ex.: le suffixe
  \textsf{\textbf{ine}}, dans \textsf{\textbf{héro'ïne}}, exprime
  l’idée}

\EnglishPage{\protect\noindent root of the verb \textup{couronner} `to crown', but that
  \textup{couronn} `crown' is a noun, which with the affix \textup{er}
  forms the verb \textup{couronner}.\footnote{This remark amounts to
    saying that languages called \emph{flexional} are in reality
    \emph{joining} languages, when we make a rational analysis of
    their words.}

  4. \emph{We use the term pleonasm to designate the repetition of the
    same idea by means of superfluous words.} --- In general,
  pleonasms are not useful and only serve to weigh down the expression
  (e.g.: the German word \textup{Prinz'ess'in} `princess' contains a
  pleonasm, since the suffixes \textup{ess} and \textup{in} both
  express the same idea of feminine).  Nonetheless, as pleonasms do
  not modify the meaning of the word or phrase that contains them, they
  are often used to reinforce the idea to be expressed
  (e.g. \textup{non, non!} `no, no!'  is a stronger expression than
  just \textup{non!} `no!).
  
  Conversely, if we add one or several words to a word (or to a
  phrase), and this addition does not change the sense of the given
  word (or phrase), we can deduce from this that the idea they express
  is already contained in the basic material.\footnote{This remark is
    important and we will have several occasions to apply it to the
    analysis of words.}

  5. \emph{We say that two words are synonyms when they evoke the same
    idea, or nearly identical ideas.} --- Synonymy can thus be
  translated by the sign =, provided we do not attribute to that sign
  a value as absolute as in mathematics (e.g. \textup{bonheur}
  `happiness' = \textup{félicité} `bliss' means simply that these
  words are synonyms).

  6. \emph{The meaning of a word is the idea evoked by that word,
    while its sense is only one of the opposed aspects from which the
    word can be considered.} --- Thus for example every word can be
  taken in the \emph{proper} sense or in the \emph{figurative} sense;
  every noun can be taken in the \emph{concrete} sense or in the
  \emph{abstract} sense, etc.\\

  \begin{center}
    \textbf{B. Analysis of words.}
  \end{center}

  7. \textsi{Principle of the invariability of elements.} \emph{Every
    simple element (root or affix) forms an invariable whole, which
    has its own meaning.} --- For example, in the words \textup{grand}
  `large', \textup{grandeur} `size', \textup{agrandir} `enlarge',
  etc., the element \textup{grand} is always the same individual; the
  word \textup{grandeur} is thus not a simple noun like
  \textup{maison} `house'; it is a word composed of an adjectival
  element \textup{grand} and a nominal element \textup{eur}.

  8. \emph{A root word generally expresses a particular idea}
  (e.g. \textup{éléfant} `elephant') \emph{while an affix word
    always expresses a general idea} (e.g.: the suffix \textup{ine} in
  \textup{héroïne} expresses the {[general]} idea

  }

  \FrenchPage{\protect\noindent générale du féminin). — Les affixes qui expriment les
    idées les plus générales sont ceux qui correspondent aux idées
    grammaticales de \emph{substantif}, d’\emph{adjectif} et de
    \emph{verbe} (ex.: l’affixe-désinence \textup{er} dans
    \textup{couronn'er} exprime l’idée \emph{verbale} générale; le
    suffixe \textup{ain} dans \textup{hum'ain} exprime l’idée
    \emph{adjective} générale; le suffixe \textup{eur} dans
    \textup{grand'eur} exprime l’idée \emph{substantive} générale);
    ces suffixes généraux peuvent donc aussi être considérés comme des
    désinences \emph{grammaticales}.

    \emph{Beaucoup de mots simples sont synonymes} (ex.:
    \textup{peur} = \textup{crainte}). — Il y a aussi des suffixes
    synonymes (ex.: les suffixes \textup{ité} dans \textup{égalité},
    \textup{esse} dans \textup{rich'esse}, \textup{eur} dans
    \textup{grand'eur}, etc., sont évidemment des synonymes;
    théoriquement ces suffixes sont interchangeables entre eux).

    Enfin \emph{les mots simples ne sont pas tous des éléments
      indépendants les uns des autres.} — Les idées qu’ils expriment
    forment des hiérarchies, qui procèdent du particulier au
    général. Ainsi, les mots \textup{pomme} et \textup{fruit} sont
    dépendants l’un de l’autre, en ce sens que l’idée particulière
    „pomme“ implique en elle-même l’idée plus générale de „fruit“; à
    son tour, l’idée évoquée par le mot \textup{fruit} implique l’idée
    encore plus générale d'„\emph{objet}“, de „\emph{chose}“,
    c’est-à-dire finalement l’idée générale de „\emph{substantif}“, de
    „\emph{substance}“; en résumé, dans tout mot simple évoquant une
    idée particulière, comme \textup{pomme}, se trouvent à l’état
    latent des idées plus générales, exprimables par d’autres mots,
    tels que \textup{fruit}, \textup{chose}, \textup{substantif}. On
    n’ajoute donc rien à l’idée „pomme“ en lui accolant l’idée de
    „fruit“ ou l’idée de „chose“; en anglais, par exemple, on
    pourrait écrire \textup{apple} = \textup{apple'fruit}, ce qui
    montre que l’addition du mot \textup{fruit} ne produit qu’un
    pléonasme (voir règle 4), c’est-à-dire que l’idée „fruit“ se
    trouvait déjà implicitement contenue dans le mot \textup{apple}.

    9. \emph{Avant de faire l’analyse d’un mot on doit le débarrasser
      des pléonasmes inutiles qu’il peut contenir.} — Ainsi, par
    exemple, avant d’analyser le mot allemand \textup{Prinz’ess'in},
    on supprimera l’un des deux suffixes synonymes \textup{ess} ou
    \textup{in}.

    10. \textsi{\scshape loi du renversement}. \emph{Pour faire l'analyse d’un
      mot composé de deux éléments, on sépare ces éléments et l’on
      renverse leur ordre} (ex.: \textup{survol} = „vol sur“; la forme
    \textup{survol} est la forme synthétique, tandis que la forme „vol
    sur“ est la forme analytique). — On peut aussi énoncer la loi du
    renversement en disant que: \emph{l’ordre analytique de deux
      éléments est inverse de leur ordre synthétique}; ou encore, que
    pour dessouder deux éléments réunis en un mot, il suffit de
    renverser leur ordre.

    \emph{Cas logique d’exception}. Il peut arriver toutefois (quand
    le premier élément du mot à analyser est une préposition, un
    nombre ou un verbe), que le second élément soit le
    \emph{complément direct} du premier; dans ce cas, l’analyse
    consiste en une simple séparation des deux
  }

  \EnglishPage{\protect\noindent general {[idea]} of the feminine.  The affixes that
    express the most general ideas are those that correspond to the
    grammatical ideas of \emph{noun, adjective} and \emph{verb} (e.g.:
    the desinence affix \textup{er} in \textup{couronn'er} `to crown'
    expresses the general \emph{verbal} idea; the suffix \textup{ain}
    in \textup{hum'ain} `human' expresses the general
    \emph{adjectival} idea; the suffix \textup{eur} in
    \textup{grand'eur} `size' expresses the general \emph{nominal}
    idea); these general suffixes can also be considered
    \emph{grammatical} desinences.

    \emph{Many simple words are synonyms} (e.g.: \textup{peur} `fear'
    = \textup{crainte} `fear'). --- there are also synonymous suffixes
    (e.g.: the suffixes \textup{ité} in \textup{égal'ité} `equality',
    \textup{esse} in \textup{rich'esse} `wealth', \textup{eur} in
    \textup{grand'eur} `size', etc., are of course synonyms;
    theoretically these suffixes are interchangeable with one
    another).

    Finally \emph{simple words are not all elements that are
      independent of one another.} --- The ideas that they express
    form hierarchies which proceed from the specific to the
    general. Thus, the words \textup{pomme} `apple' and \textup{fruit}
    `fruit' are dependent on one another, in the sense that the
    specific idea ``\emph{apple}'' implies in itself the more general
    idea of ``\emph{fruit}''; in turn, the idea evoked by the word
    \textup{fruit} implies the more general idea of ``\emph{object}'',
    of ``\emph{thing}'' and finally the general idea of
    ``\emph{noun}'', of ``\emph{substance}''; in sum, in every simple
    word evoking a particular idea, like \textup{pomme}, is to be
    found underlyingly more general ideas, expressible in other words
    such as \textup{fruit, thing, noun}.  We thus add nothing to the
    idea ``\emph{apple}'' by adjoining to it the idea of
    ``\emph{fruit}'' or the idea of ``\emph{thing}''; in English, for
    example, we could write \textup{apple} = \textup{apple'fruit},
    which shows that the addition of the word \textup{fruit} only
    produces a pleonasm (see rule 4). that is, the idea
    ``\emph{fruit}'' is already contained implicitly in the word
    \textup{apple}.

    9. \emph{Before analyzing a word it is necessary to clear away the
      useless pleonasms that it may contain.} --- Thus, for example,
    before analyzing the German word \textup{Prinz'ess'in} `princess',
    we eliminate one of the two synonymous suffixes \textup{ess} or
    \textup{in}.

    10. \textsi{Law of reversal.}  \emph{To analyze a word composed of
      two elements, we separate the elements and reverse their order}
    (e.g.: \textup{survol} `overflight' = ``\emph{flight over}''; the
    form \textup{survol} is the synthetic form, while the form
    ``\emph{flight over}'' is the analytic form). --- We can thus
    formulate the law of reversal in saying that: \emph{the analytic
      order of two elements is the inverse of their synthetic order};
    or else that to disconnect two elements brought together in one
    word, it suffices to reverse their order.

    \emph{Logically exceptional case.} It can sometimes happen (when
    the first element of the word to be analyzed is a preposition, a
    number or a verb) that the second element is the \emph{direct
      complement} of the first; in that case, the analysis consists of a
    simple separation of the two
    
  }

  \FrenchPage{\protect\noindent éléments, \emph{sans renversement} de leur ordre (ex.:
    \textup{inter'règne} — „\emph{entre règnes}“ parce qu’ici le mot
    \textup{règne} est le complément de la préposition \textup{entre};
    ce cas d’exception est logique parce que les deux éléments
    \textup{entre} et \textup{règne} formaient déjà un seul tout avant
    même d’être réunis; au contraire, dans l’exemple \textup{survol}
    = „vol sur“, il y a renversement parce qu’ici le mot \textup{vol}
    n’est pas le complément de la préposition \textup{sur}; les deux
    idées ,,\emph{vol}“ et „\emph{sur}“ représentent dans ce mot des
    idées autonomes et indépendantes l’une de l’autre.\footnote{Pour
      les autres cas d’exceptions, voir p. 16.}

    11. Le procédé d’analyse qui consiste à séparer les deux éléments
    d’un mot composé (avec ou sans renversement) n’est pas toujours
    suffisant. \emph{Pour pousser plus loin l’analyse, il faut mettre
      en évidence l’idée sous-entendue qui se cache dans la soudure
      entre les deux éléments du mot composé} (ex.:
    \textup{Schreib'tisch} — „\emph{table} [\emph{pour}]
    \emph{écrire}“, \textup{steam'ship} = „\emph{bateau} [\emph{mû par
      la}] \emph{vapeur}“ etc. — La nature de l’idée sous-entendue
    varie beaucoup d’un mot à un autre; toutefois, cette idée peut
    presque toujours être traduite par l’expression: „\emph{de
      l’espèce caractérisée par}“ (ex.: \textup{Schreib'tisch} =
    „\emph{table} [\emph{de l’espèce caractérisée par}]
    \emph{écrire}“).

    Dans le cas particulier des mots composés sans renversement, comme
    \textup{inter'règne}, il n’y a pas d’idée sous-entendue dans la
    soudure entre les deux éléments, puisque dans ce cas le second
    élément est complément direct du premier.\footnote{Par contre,
      dans la langue française, il y a toujours dans ce cas une idée
      sous-entendue \emph{avant} le mot composé. Ainsi:\textup{
        inter'règne} = [\emph{temps}] \emph{entre règnes},
      \textup{coupe-papier} = „[\emph{objet qui}] \emph{coupe
        papier}“, \textup{sous-pied} = „[\emph{objet qui est}]
      \emph{sous pied}“, \textup{tri'angle} = „[\emph{chose qui est}]
      \emph{trois angles}“, \textup{tri'corne} = „[\emph{objet qui
        est}] \emph{trois cornes}“, etc.}

    12. \emph{Pour faire une analyse encore plus complète, il faut
      mettre en évidence les idées générales qui existent à l’état
      latent dans toute idée particulière.}\footnote{Voir le n° 8.} —
    Ainsi pour analyser le mot \textup{cheval}, on peut mettre en
    évidence une idée plus générale, telle que „\emph{quadrupède}“,
    „\emph{vertébré}“, „\emph{animal}“, etc., qui est contenue
    implicitement dans l’idée „\emph{cheval}“; on a alors les diverses
    possibilités d’analyse:\\

    \begin{tabular}[b]{ll}
\textup{cheval} =	„\emph{quadrupède}&[\emph{espèce}]	\emph{cheval}“,\\
\textup{cheval} =	„\emph{vertébré}&[\emph{espèce}]	\emph{cheval}“,\\
\textup{cheval} =	„\emph{animal}&[\emph{espèce}]	\emph{cheval}“,
    \end{tabular} etc.\\

    On a en effet, par la loi de renversement: „\emph{animal cheval}“
    = \textup{horse'animal}; or, \textup{horse'animal} n’est qu’une
    forme pléonasmatique de \textup{horse}, comme \textup{apple'fruit}
    de \textup{apple}. On a donc bien le droit d’écrire:
    „\emph{animal cheval}“ — \textup{cheval}, ou réciproquement.

    \emph{Remarque}. Il faut bien faire la distinction entre les mots
    composés du type \textup{Pferd'tier}, qui contiennent un pléonasme
    et les

  }

  \EnglishPage{\protect\noindent elements, \emph{without reversing} their order (e.g.:
    \textup{inter'règne} `interregnum' --- ``\emph{between
      reigns}'' because here the word \textup{règne} `reign' is the
    complement of the preposition \textup{entre} `between'; this
    exceptional case is logical because the elements \textup{entre}
    and \textup{règne} would already form a unit before even being
    joined; on the contrary, in the example \textup{survol}
    `overflight' = ``flight over'', there is reversal because here the
    word \textup{vol} `flight' is not the complement of the
    preposition \textup{sur} `over'; the two ideas ``\emph{flight}''
    and ``\emph{over}'' represent in this word autonomous ideas
    independent of one another.\footnote{For other exceptional cases,
      see p. 16.}

    11. The process of analysis which consists of separating the two
    element of a compound word (with or without reversal) is not
    always sufficient. \emph{To carry the analysis further, it is
      necessary to highlight the underlying idea which is concealed
      under the juncture} between \emph{the two elements of the
      compound word} (e.g.: \textup{Schreib'tisch} =
    ``\emph{table}[\emph{for}] \emph{writing}'', \textup{steamship} =
    ``\emph{ship} [\emph{driven by}] \emph{steam}'', etc.[)] --- the
    nature of the idea that is understood varies considerably from one
    word to another; however, this idea can nearly always be
    translated by the expression ``\emph{of the type characterized
      by}'' (e.g.: \textup{Schreib'tisch} = ``\emph{table} [\emph{of
      the type characterized by}] \emph{writing}'').

    In the special case of words composed without reversal, like
    \textup{inter'règne}, there is no idea understood in the
    juncture between the two elements, since in this case the second
    element is the direct complement of the first.\footnote{On the
      other hand, in French, there is always an understood idea in
      this case \emph{before} the compound word. Thus
      \textup{inter'règne} = ``[\emph{period}] \emph{between
        reigns}'', \textup{coupe-papier} `paper cutter' =
      ``[\emph{object which}] \emph{cuts paper}'', \textup{sous-pied}
      `foot strap' = ``[\emph{object which is}] \emph{under foot}'',
      \textup{tri'angle} `triangle' = ``[\emph{thing which is}]
      \emph{three angles}'', \textup{tri'corne} `three-cornered hat' =
      ``[\emph{object which is}] \emph{three horns}'', etc.}

    12. \emph{To make an analysis still more complete, it is necessary
      to highlight the general ideas that exist in a latent state
      within each specific idea.}\footnote{See number 8.} --- Thus, to
    analyze the word \textup{cheval} `horse', we can bring out a more
    general idea, such as ``\emph{quadruped}'', ``\emph{vertebrate}'',
    ``\emph{animal}'', etc. which is implicitly contained in the idea
    ``\emph{horse}'': we thus have various possibilities of analysis:\\

    \begin{tabular}[b]{ll}
\textup{cheval} =	``\emph{quadruped}&[\emph{of the type}]	\emph{horse}'',\\
\textup{cheval} =	``\emph{vertebrate}&[\emph{of the type}]	\emph{horse}'',\\
\textup{cheval} =	``\emph{animal}&[\emph{of the type}]	\emph{horse}'',
    \end{tabular} etc.\\

   Indeed, by the law of reversal, we have: ``\emph{animal horse}'' =
   \textup{horse'animal}; now \textup{horse'ani\-mal} is only a
   pleonastic form of \textup{horse}, as \textup{apple'fruit} is of
   \textup{apple}. We thus are justified in writing: ``\emph{animal
     horse}'' = \textup{cheval}, or vice versa.

   \emph{Remark:} It is necessary to make the distinction between
   compound words of the type \textup{Pferd'tier}, which contain a
   pleonasm, and

  }

  \FrenchPage{\protect\noindent mots du type \textup{Fell'tier}, qui n’en contiennent
    pas. Dans le premier cas, on a simplement: \textup{Pferd'tier} =
    „\emph{animal} [\emph{de l’espèce}] \emph{cheval}“; dans le second
    cas, on a: \textup{Fell'tier} = „\emph{animal} [\emph{de l’espèce
      caractérisée par}] \emph{une fourrure}“.

    13. Enfin, \emph{dans les cas peu fréquents où les règles
      précédentes d’analyse sont encore insuffisantes, on mettra en
      évidence les idées cachées dans le contexte du mot à
      analyser}. — En particulier, on rétablira dans le contexte les
    mots sous-entendus (ex.: \textup{un riche} signifie „\emph{un
      homme riche}“, \textup{un mille-pied} signifie „\emph{un animal
      qui a mille pieds}“).

    C’est aussi le contexte qui décide si un mot doit être pris au
    sens \emph{propre} ou au sens \emph{figuré},\footnote{Ainsi dans
      le mot \textup{mille-pied}, le mot \textup{mille} est pris au
      sens figuré.} au sens \emph{concret} ou au sens
    \emph{abstrait}. Ainsi la signification d’un mot dépend aussi de
    son contexte, ou plus généralement des circonstances dans
    lesquelles ce mot est employé.

    14. \emph{Lorsqu'un mot composé contient plus de deux éléments,
      son analyse peut toujours être ramenée à celle de plusieurs mots
      ne contenant chacun que deux éléments.}

    Ainsi le mot \textup{Schrauben'dampfer'aktien'gesellschaft} se
    décompose d’abord en deux parties (\textup{Schraubendampfer} et
    \textup{Aktiengesellschalt}) auxquelles on appliquera la loi de
    renversement en considérant chacune des deux parties comme un mot
    simple; on analysera ensuite chaque partie séparément par le même
    procédé, et l’on répétera l’opération jusqu’à ce qu’il ne reste
    plus que des mots simples comme résidu.

    \begin{center}
      \textbf{C. Synthèse des mots}.
    \end{center}

    15. La synthèse, ou construction, des mots est basée sur les deux
    principes suivants, qui ne sont du reste que l’expression de la
    \emph{loi du moindre effort}:

    \textsi{Principe de nécessité}. \emph{Dans la construction d’un
      mot composé il faut introduire} (au moyen de la loi de
    renversement) \emph{tous les éléments simples} (racines et
    affixes) \emph{qui sont nécessaires pour évoquer clairement l’idée
      que ce mot doit exprimer} (dans des circonstances données).

    \textsi{Principe de suffisance:} \emph{On doit aussi, dans cette
      construction, éviter l’introduction de pléonasmes inutiles,
      ainsi que celle d’idées étrangères à l’idée que l’on veut
      exprimer.}

    En d’autres termes, si un mot est construit suivant les deux
    principes de nécessité et de suffisance, \emph{la signification de
      ce mot sera conforme à son contenu}, c’est-à-dire que le mot
    construit sera un mot \emph{entièrement motivé} (dans les
    circonstances données).

    16. \emph{Pour faire la synthèse d’un mot, on se servira de procédés
    exactement inverses, de ceux qui ont servi à en faire l’analyse.}

  }

  \EnglishPage{\protect\noindent words of the type \textup{Fell'tier} which do not. In
    the first case, we have simply: \textup{Pferd'tier} =
    ``\emph{animal} [\emph{of the type}] \emph{horse}''; in the
    second, we have: \textup{Fell'tier} = ``\emph{animal} [\emph{of
      the type characterized by}] \emph{a fur}''.

    13. Finally, \emph{in the infrequent cases where the preceding
      rules of analysis are still insufficient, we will bring out the
      ideas concealed in the context of the word to be analyzed.} ---
    In particular, we will restore the words understood in the context
    (e.g.: \textup{un riche} `a rich' means ``\emph{a rich man}'',
    \textup{un mille-pied} `a millipede' = ``\emph{an animal that has
      a thousand legs}'').

    It is also the context that decides whether a word should be taken
    in its \emph{proper} sense or in a \emph{figurative}
    sense,\footnote{Thus in the word \textup{mille-pied}, the word
      \textup{mille} `thousand' is taken in the figurative sense.} in
    the \emph{concrete} sense or the \emph{abstract} sense.  Thus, the
    meaning of a word also depends on its context, or more generally,
    on the circumstances in which the word is used.

    14. \emph{When a compound word contains more than two elements,
      its analysis can always be reduced to that of several words each
      containing no more than two elements.}

    Thus, the word \textup{Schrauben'dampfer'aktien'gesellschaft}
    `screw-steamer joint stock company' breaks down first into two
    parts (\textup{Schrauben'dampfer} and \textup{Aktiengesellschaft})
    to which we apply the law of reversal, considering each of the two
    parts as a simple word; we then analyze each part separately
    through the same procedure, and we repeat the operation until only
    simple words remain as the result.

    \begin{center}
      \textbf{C. Synthesis of words}
    \end{center}

    15. The synthesis, or construction, of words is based on the two
    following principles, which are only the expression of the
    \emph{law of least effort}:

    \textsi{Principle of necessity}. \emph{In the construction of a
      compound word it is necessary to introduce} (by means of the law
    of reversal) \emph{all of the simple elements } (roots and
    affixes) \emph{necessary to evoke clearly the idea that the word
      is to express} (in the given circumstances).

    \textsi{Principle of sufficiency}: \emph{We must also, in this
      construction, avoid the introduction of useless pleonasms, as
      well as ideas foreign to the idea that we wish to express.}

    In other terms, if a word is constructed according to the two
    principles of necessity and sufficiency, \emph{the meaning of the
      word will conform to its content,} that is, the constructed word
    will be an \emph{entirely motivated} word (in the given
    circumstances).

    16. \emph{To perform the synthesis of a word, we make use of
      procedures that are the exact inverses of those we make use of
      to perform its analysis.}
  
  }

  \FrenchPage{

    Prenons comme exemple le mot \textup{couronner} et
    faisons en d’abord l’analyse. Ce mot se compose de deux éléments:
    le substantif \textup{couronn} et l’affixe verbal \textup{er}. On a donc d’abord:
    \textup{couronner} = \textup{couronn'er}. Appliquant ensuite la loi de renversement
    (règle n° 10), on obtient:
    
    {\centering
      \textup{couronn'er} = „\emph{er couronn}“\footnote{La forme analytique „\emph{er
        couronn}“ n’existe pas en français, mais elle existe en
        anglais: „\emph{to crown}“, car dans cette expression c’est le mot \emph{to}
        qui exprime l’idée verbale.}\\[0.5ex]

    }
    
    La désinence \textup{er} exprime l’idée verbale générale; cette
    désinence est donc synonyme du mot \textup{faire}, dans le sens général de
    „faire une action“; on exprimera cette synonymie en posant

    {\centering
      \textup{er} = „\emph{faire l’action}“.\\[0.5ex]
    
    }

    D’autre part, en faisant sortir du mot \textup{couronne} l’idée plus
    générale d'„objet“, qui y est contenue (règle n° 12) et
    „appliquant encore une fois la loi de renversement (règle n° 10),
    on a:
    
    {\centering \textup{couronne} = \textup{couronne}(\textup{objet})
      = „\emph{objet couronne}“\\[0.5ex]

    }

    Reportant ces résultats dans la première égalité, il vient:

    {\centering\textup{couronn'er} = „\emph{er couronne}“ = „(\emph{faire
        l’action}) (\emph{objet couronne})“.\\[0.5ex]

      }

    Enfin, appliquant la règle n° 11, d’après laquelle la soudure d’un
    mot composé renferme à l’état latent l’idée générale „\emph{de
      l'espèce caractérisée par}“, on a comme résultat final de
    l’analyse:
    
    {\centering\textup{couronn'er} — „\emph{faire l’action} [de
      l’espèce caractérisée par]\\
      l'\emph{objet couronne}“.\\[0.5ex]

      }

      Réciproquement, si l’on veut faire la synthèse de l’idée
      „\emph{faire l’action caractérisée par l’objet couronne}“,
      c’est-à-dire construire le mot qui exprime cette idée, on
      effectuera en ordre inverse des opérations inverses:

      On remarquera d’abord que l’idée donnée contient deux idées
      indépendantes: „\emph{faire l’action}“ et „\emph{objet
        couronne}“, réunies par l’expression „\emph{caractérisée
        par}“, dont il n’y a pas lieu de tenir compte, puisqu’elle est
      destinée à disparaître dans la soudure du mot composé, d’après
      la règle 11. L’idée dont il faut faire la synthèse peut donc
      être réduite à la forme:
    
     {\centering „(\emph{faire l’action}) (\emph{objet couronne})“.\\[0.5ex]

       }

    Or, d’une part, on a par la règle de renversement (n° 10) et par
    suppression de pléonasme (n° 12):
    
      {\centering„\emph{objet couronne}“ = \textup{couronn}(\textup{objet}) =
      \textup{couronne};

      }

    \noindent d’autre part, l’idée „\emph{faire une action}“ est l’idée verbale
    générale, et cette idée peut être exprimée soit par les mots
    racines \textup{faire}, \textup{agir}, soit par des suffixes, tels
    que \textup{ir} (dans \textup{blanch'ir}), \textup{er} (dans
    \textup{clou'er}), etc. On doit évidemment mettre l’idée verbale
    sous la forme d’un suffixe, toutes les fois que l’élément qui la
    représente doit occuper en fin de synthèse la place d’un suffixe;
    c’est bien ce qui a lieu dans l’exemple choisi, car, en remplaçant
    l’idée verbale „\emph{faire l’action}“ par l’affixe \textup{er}, il ne reste plus
    qu’à faire la synthèse de l’expression analytique:
    
     {\centering „(\emph{er}) (\emph{couronne})“,

     }

  }

  \EnglishPage{

    Let us take as an example the word \textup{couronner} `to crown',
    and first perform the analysis.  This word consists of two
    elements: the noun \textup{couronn} and the verbal affix
    \textup{er}. Thus we have initially: \textup{couronner} =
    \textup{couronn'er}. Applying next the law of reversal (rule
    number 10), we obtain:

    {\centering \textup{couronn'er} = ``\emph{er
        couronn}''\footnote{The analytiic form ``\emph{er couronn}'' does not
        exist in French, but it does exist in English: „\emph{to crown}“,
        because in this expression it is the word \emph{to}
        that expresses the verbal idea.}\\[0.5ex]}

      The desinence \textup{er} expresses the general verbal idea;
      this desinence is thus synonymous with the word \textup{faire}
      `to do', in the general sense of ``\emph{perform an action}'';
      we express this synonymy by asserting

       {\centering
      \textup{er} = ``\emph{perform the action}''.\\[0.5ex]
    
    }

    On the other hand, in extracting from the word \textup{couronne}
    the more general idea of ``\emph{object}'' which is contained in
    it (rule number 12) and applying once more the law of reversal
    (rule number 10), we have:

    {\centering \textup{couronne} = \textup{couronne}(\textup{objet})
      = ``\emph{object crown}''\\[0.5ex]

    }

    Returning these results to the first equation, it becomes:

     {\centering\textup{couronn'er} = ``\emph{er crown}'' =
       ``(\emph{perform the action}) (\emph{object crown})''.\\[0.5ex]

      }

      Finally, applying rule number 11, according to which the
      juncture of a compound word contains latently the general idea
      ``\emph{of the type characterized by}'', we have as the final
      result of the analysis

      {\centering\textup{couronn'er} — ``\emph{perform the action} [of
        the type characterized by]\\
        the \emph{object crown}''.\\[0.5ex]

      }

      Conversely, if we wish to perform the synthesis of the idea
      ``\emph{perform the action characterized by the object crown}'',
      that is, construct the word that expresses that idea, we carry
      out the inverse operations in the inverse order:

      We observe first that the given idea contains two independent
      ideas: ``\emph{perform the action}'' and ``\emph{object
        crown}'', linked by the expression ``\emph{characterized
        by}'', which need not be taken into account because it is
      destined to disappear in the junction of the compound word,
      according to rule number 11. The idea whose synthesis must be
      performed can thus be reduced to the form:

      {\centering ``\emph{(perform the action) (object crown)}''\\[0.5ex]

      }

      Now on the one hand, we have by the rule of reversal (number 10)
      and by suppression of pleonasm (number 12):

      {\centering ``\emph{object crown}'' - \textup{couronn(objet)} =
        \textup{couronn};\\[0.5ex]

      }

      \noindent on the other hand, the idea ``\emph{perform an
        action}'' is the general verbal idea, and this idea can be
      expressed either by the roots \textup{faire} `to do',
      \textup{agir} `to act', or by suffixes such as \textup{ir} (in
      \textup{blanch'ir} `to whiten'), \textup{er} (in
      \textup{clou'er} `to nail'), etc.  We must obviously introduce
      the verbal idea in the form of a suffix every time the element
      that represents it occupies, at the end of the synthesis, the
      position of a suffix; this is what takes place in the example
      chosen, because, in replacing the idea ``\emph{perform the
        action}'' by the suffix \textup{er}, it only remains to
      perform the synthesis of the analytic expression:

      {\centering ``\emph{(er) (crown)}'',

        }

  }

  \FrenchPage{\protect\noindent qui, conformément à la loi de renversement (n° 10), se
    condense en un seul mot: \textup{couronn'er}. (On laissé tomber
    l’\emph{e} muet de \textup{couronne}, comme on laisse tomber une
    bavure, après que la soudure est effectuée.) La synthèse est ainsi
    terminée: le mot \textup{couronner} représente bien, par sa
    structure et son contenu, l’idée donnée.

    En résumé, \emph{analyser un mot, c’est tirer de l’intérieur même de ce
    mot} (au moyen des règles posées plus haut) \emph{toute une phrase
    explicative de la signification de ce mot}; réciproquement, \emph{faire
    la synthèse d'un mot, c’est recondenser cette phrase explicative
    en un seul mot par des opérations inverses.}

  {\small \emph{Remarques sur l'interprétation des principes
      généraux}. Dans l’application des principes généraux que nous
    venons de résumer, on devra procéder avec circonspection dans
    chaque cas particulier, car les mots (racines ou affixes), qui
    forment la matière régie par ces principes, ne sont pas des
    éléments aussi rigides et précis que des signes mathématiques:
    ceux-ci ont par leur nature même une valeur parfaitement définie,
    tandis que celle des signes linguistiques est toujours plus ou
    moins élastique. Ainsi, toutes les fois que l’on emploie le signe
    = on doit se rappeler que ce signe signifie simplement „synonyme
    de“, d’après la définition donnée au n° 5. Il est rare que deux
    mots synonymes aient exactement la même valeur; ce cas ne se
    présente guère que pour certain suffixes: par exemple, les
    suffixes \textup{ité} (dans \textup{égal'ité}), \textup{eur} (dans
    \textup{grand'eur}), \textup{esse} (dans \textup{rich'esse}),
    etc., sont exactement équivalents, car ces suffixes ont partout le
    même rôle et la même signification. On comprend du reste
    facilement pourquoi deux mots synonymes n’ont presque jamais une
    signification identique: la langue profite précisément de la
    différence de forme de deux mots synonymes pour établir entre ces
    mots une différence de sens, quoique leur signification générale
    soit la même. Par exemple, au point de vue logique, le mot
    allemand: \textup{süss'lich} = \textup{süss}, parce qu’en
    ajoutant l’idée adjective \textup{lich} à une racine adjective,
    comme \textup{süss}, on produit un simple pléonasme; mais cela ne
    veut pas dire, qu’il n’y ait, dans la pratique, aucune différence
    de valeur entre ces deux formes. De même, si l’on écrit
    \textup{mouton} = \textup{sheep}, cela signifie que le mot anglais
    \textup{sheep} est la traduction du mot français \textup{mouton},
    mais il n’en résulte pas que ces deux mots aient une valeur
    identique; il y a des cas par exemple où \textup{mouton} est
    traduit, non par \textup{sheep}, mais par \textup{mutton}.

    Il peut arriver, au contraire, que des mots ayant la même forme
    extérieure (\emph{homonymes}) aient des significations absolument
    différentes (Ex.: \textup{son}, pronom possessif; \textup{son},
    phénomène physique; \textup{son}, de grain\footnote{On voit que
      pour distinguer des homonymes, on recourt à la règle d’analyse
      n° 12, car en spécifiant un des sens du mot son au moyen de
      l’épithète „\emph{de grain}“, on ne fait que mettre en évidence une
      idée plus générale, qui était déjà implicitement contenue dans
      le mot son.}, etc.; de même, le suffixe \textup{eur} dans
    \textup{grandeur} n’a aucun rapport avec le suffixe \textup{eur}
    dans \textup{acheteur}, etc.). Il est à peine besoin d’insister
    sur ces distinctions, tant elles sont évidentes.

    Ce qui importe donc pour le logicien, c’est beaucoup moins la
    forme extérieure d’un mot que sa signification. Il peut arriver,
    par exemple, que la forme d’un élément se modifie pour rendre plus
    facile la prononciation du mot, dont cet élément fait
    partie. Ainsi dans la série \textup{homme}, \textup{hum'ain},
    \textup{hum'an'ité}, le mot \textup{homme} prend la forme
    \textup{hum}, et le suffixe \textup{ain} la forme \textup{an}; il
    n’y a pas de doute, cependant que nous avons à faire là aux mêmes
    éléments, puisqu’en allemand, par exemple, on retrouve la même
    série de mots encore intacte: \textup{Mensch},
    \textup{mensch'lich}, \textup{Mensch'lich'keit}. Il peut même
    arriver qu’un mot français ait conservé sa forme latine dans les
    mots dérivés (Ex.: \textup{père}, \textup{pater'nel}); peu}

}

\EnglishPage{\protect\noindent which, in conformity with the law of reversal (number
  10), condenses to a single word: \textup{couronn'er} `to crown'. (We drop the
  mute \emph{e} of \textup{couronne} `crown' as we drop a smudge,
  after the juncture is effected.) The synthesis is thus finished: the
  word \emph{couronner} represents well, by its structure and its
  content, the given idea.

  In sum, \emph{to analyze a word is to take from the very interior of
    that word} (by means of the rules presented above) \emph{an entire
    phrase explanatory of the the meaning of that word;} conversely,
  \emph{to synthesize a word is to condense that explanatory phrase
    back into a single word by the reverse operations.}\\

  {\small\emph{Remarks on the interpretation of the general
      principles.} In the application of the general principles that
    we have just summarized, we must proceed with circumspection in
    each particular case, because the words (roots or affixes that
    form the material governed by these principles are not elements as
    rigid and precise as mathematical signs: the latter have, by their
    very nature, a perfectly defined value, while that of linguistic
    signs is always more or less elastic.  Thus, every time we use the
    sign = we have to remember that that sign means simply
    ``\emph{synonym of}'', according to the definition given in number
    5. It is rare that two synonymous words have exactly the same
    value; this case hardly presents itself except for certain
    suffixes: for example the suffixes \textup{ité} (in
    \emph{égal'ité} `equality'), \textup{eur} (in \textup{grand'eur}
    `size'), \textup{esse} (in \textup{rich'esse} `wealth'), etc. are
    exactly equivalent, because the suffixes all have the same role
    and the same meaning. Otherwise, we understand easily why two
    synonymous words almost never have an identical meaning: language
    profits precisely from the difference in form of two synonymous
    words to establish a difference of sense between them, even though
    their general meaning may be the same. For example, from a logical
    point of view, the German word \textup{süsslich} `sweetish' =
    \textup{süss} `sweet', because by adding the adjectival idea
    \textup{lich} to an adjectival root like \textup{süss}, we produce
    a simple pleonasm, but that does not mean that there is, in
    practice, no difference in value between these two
    forms. Similarly, if we write \textup{mouton} = \textup{sheep},
    that means that the English word \textup{sheep} is the translation
    of the French word \textup{mouton}, but it does not follow that
    the two words have an identical value: there are circumstances,
    for example, where \textup{mouton} is translated not by
    \textup{sheep} but by \textup{mutton}.

    On the other hand, it can happen that words with the same external
    form (\emph{homonyms}) have absolutely different meanings (e.g.:
    \textup{son}, possessive pronoun; \textup{son} `sound' physical
    phenomenon; \textup{son} `bran' of grain\footnote{We see that to
      distinguish homonyms, we go back to rule of analysis number 12,
      because in specifying one of the senses of the word \textup{son}
      by means of the qualifier ``\emph{of grain}'', we are only
      bringing out a more general idea which was already implicitly
      contained in the word \textup{son}.}, etc.; similarly, the
    suffix \textup{eur} in \textup{grandeur} is unrelated to the
    suffix \textup{eur} in \textup{acheteur} `buyer', etc.).  It is
    hardly necessary to insist on these distinctions, since they are
    so obvious.

    What matters therefore for the logician is much less the external
    form of a word than its meaning.  It can happen for instance, that
    the form of an element can change to make the pronunciation of the
    word of which it is a part easier. Thus, in the series
    \textup{homme, hum'ain, hum'an'ité} `man, human, humanity', the
    word \textup{homme} takes the form \textup{hum} and the suffix
    \textup{ain} the form \textup{an}; there is, however, no doubt
    that we have to do there with the same elements, since in German
    for example we find the same series of words still intact:
    \textup{Mensch, mensch'lich, Mensch'lich'keit}. It can even happen
    that a French word may have preserved its Latin form in derived
    words (e.g.: \textup{père} `father', \textup{pater'nel}
    `paternal'); that does not


    }
  }

  \FrenchPage{{\small\protect\noindent importe, car le principe de l’invariabilité des
      éléments (n° 7) se rapporte plutôt au signifié qu’au
      signifiant. 

      Nous avons vu qu’il existe deux sortes de mots composés: ceux
      du type \textup{survol}, dont les éléments sont des mots qui
      expriment des idées indépendantes l’une de l’autre (cas
      général), et ceux du type \textup{inter'règne}, où l’un des
      éléments est le complément direct de l’autre (cas
      particulier). Il n’en faut pas conclure toutefois qu’il n’existe
      pas d’autres types dans les langues naturelles; ainsi, dans les
      langues latines, les mots composés contenant deux mots-racines
      (\textup{timbre-poste}, \textup{assurance-vieillesse}, etc.),
      sont formés à rebours de la loi générale. Ces formes ne sont pas
      vraiment synthétiques; ce sont de simples abréviations de la
      forme analytique: „\emph{timbre (de) poste}“, „\emph{assurance
        (pour la) vieillesse}“, etc. Si notre étude portait uniquement
      sur la langue française, il nous faudrait admettre deux lois de
      formation des mots: une loi de renversement pour les mots
      composés contenant une racine et un affixe (par ex.:
      \textup{sou'tenir} = „\emph{tenir sous}“, \textup{util'ité} =
      „\emph{ité} [espèce] \emph{util}“, etc.), et une loi de
      formation directe pour les mots composés contenant deux racines
      (\textup{timbre-poste}, etc.); mais comme nous nous plaçons au
      point de vue international des langues en général, nous sommes
      fondés à considérer la loi de renversement comme la seule loi
      générale de formation des mots composés, par le fait que cette
      loi est générale pour les langues germaniques, slaves, etc., et
      qu’elle existe même dans les langues latines pour composer une
      racine avec des affixes\footnote{Ainsi, par exemple, le mot
        \textup{pomm'ier} est aussi conforme à la loi de renversement
        que le mot allemand \textup{Apfel'baum}, ou le mot anglais
        \textup{apple'tree}, car le suffixe \textup{ier} signifie
        „\emph{un arbre}“, „\emph{un objet qui porte}“. Ex.:
        \textup{chandel'ier} = „\emph{ier chandel}“ = „\emph{objet qui
          porte} (des) \emph{chandelles}“.}. Au point de vue
      international, des mots composés à la manière de
      \textup{timbre-poste} sont anormaux; ils proviennent du manque
      d’habitude qu’ont les Latins de faire des mots composés, et en
      effet lorsqu’on veut former le pluriel de ces mots, le signe du
      pluriel tombe au milieu du mot (des \textup{timbres-poste}), ce
      qui est une anomalie au point de vue logique et pratique. Du
      reste, les mots du type \textup{timbre-poste},
      \textup{va-nu-pied}, etc. ne sont pas de vrais mots composés, en
      ce sens que leurs éléments ne sont pas \emph{soudés} (comme dans les
      mots \textup{survol}, \textup{Apfelbaum}, etc.), mais seulement
      réunis par un trait d’union. Or, d’après la définition (n° 3,
      p. 9) un mot composé est formé par \emph{soudure} de ses éléments.

      \emph{Remarque}. L’espèce grammaticale d’un mot est déterminée
      par son \emph{dernier} élément; ainsi \linebreak\textup{Schreib'tisch} est un substantif,
      parce que \textup{Tisch} est un substantif. Cette règle suppose
      évidemment que le mot considéré est construit conformément à la
      loi de renversement; si le mot composé a une structure anormale,
      la règle est naturellement inapplicable; ainsi dans le mot
      \textup{timbre-poste} c’est le premier élément \textup{timbre} qui détermine
      l’espèce grammaticale du mot. Du reste, il suffit pour lever le
      doute, de redonner au mot composé sa forme analytique;
      l’élément qui détermine l’espèce grammaticale occupe alors
      toujours la première place. Ex.: „\emph{table} à écrire“, „\emph{timbre} de
      poste“, etc.

      Enfin, nous avons vu que tout mot composé de plus de deux
      éléments est divisible en deux parties, analysables
      séparément. Mais cette division ne peut pas être effectuée d’une
      manière arbitraire. Ainsi, par exemple, le mot
      \textup{passenger'steam'ship} ne peut pas être divisé en
      (\textup{passenger'steam}) et \textup{ship}, mais seulement en
      \textup{passenger} et (\textup{steam'ship}). De même le mot
      \textup{hum'an'it'ar'isme} ne peut être analysé que par la série
      suivante de mots à deux parties:

      \begin{center}
        \textup{humanitarisme} = \textup{humanitar'isme},\\
        \textup{humanitaire} = \textup{humanit'aire},\\
        \textup{humanité} = \textup{human'ité},\\
        \textup{humain} = \textup{hum'ain}.
      \end{center}

    }


  }

  \EnglishPage{{\small\protect\noindent
      matter, because the principle of the invariability of elements
      (number 7) relates to the signified rather than to the
      signifier.

      We have seen that there are two sorts of compound words: those
      of the type \textup{survol} `overflight', where the elements are
      words that express ideas independent of one another (the general
      case), and those of the type \textup{inter'règne} `interregnum',
      where one of the elements is the direct complement of the other
      (the special case).  It should not be concluded from this that
      other types do not exist in natural languages; thus, in the
      Romance languages, compound words containing two root words
      (\textup{timbre-poste} `stamp-post: postage stamp',
      \textup{assurance-vieillesse} `insurance-old age', etc.), are
      formed in the wrong way from the general law. These forms are
      not really synthetic: they are simply abbreviations of the
      analytic form: ``\emph{stamp (of) post(age)}'',
      ``\emph{insurance (for) old age}'', etc. If our study bore
      uniquely on French, we would have to admit two laws of word
      formation: a law of reversal for compound words containing a
      root and an affix (for example: \textup{sou'tenir} `to support'
      = ``\emph{to hold under}'', \textup{util'i\'té} `usefulness' =
      ``\emph{ity} [of type] \emph{useful}''), and a law of direct
      formation for compound words containing two roots
      (\textup{timbre-poste}, etc.); but since we are taking the
      international perspective on languages in general, we are
      justified in taking the law of reversal as the only general law
      for the formation of compound words, by the fact that this law
      is general for the Germanic, Slavic, etc. languages, and that it
      exists also in the Romance languages for combining a root with
      affixes.\footnote{Thus, for example, the word \textup{pomm'ier}
        `apple tree' is as much in conformace with the law of reversal
        as the German word \textup{Apfel'baum}, or the English word
        \textup{apple'tree}, because the suffix \textup{ier} means
        ``\emph{a tree}'', ``\emph{an object that bears}''. E.g.:
        \textup{chandelier} = ``\emph{ier candle}'' = ``\emph{object
          that bears candles}''.} From the international point of
      view, words composed in the manner of \textup{timbre-poste} are
      abnormal: they come from the lack of a habit in Latin speakers
      to form compound words, and indeed when we want to form the
      plural of these words, the sign of the plural falls in the
      middle of the word (\textup{timbres-poste} `postage stamps'),
      which is anomalous from the logical and practical point of view.
      Besides, words of the type \textup{timbre-poste, va-nu-pied}
      `go-bare-foot: tramp', etc. are not true compound words, in the
      sense that their elements are not \emph{joined} (as in the words
      \textup{survol, Apfelbaum}, etc.), but just combined by a
      hyphen. Now according to the definition (number 3, p. 9) a
      compound word is formed by
      the \emph{juncture} of its elements.\\

      \emph{Remark.} The grammatical category of a word is determined
      by its \emph{final} element: thus,\linebreak \textup{Schreib'tisch}
      `writing table' is a noun because \textup{Tisch} `table' is a
      noun. This rule obviously presumes that the word under
      consideration is built in accord with the law of reversal: if
      the compound word has an abnormal structure, the rule is
      naturally inapplicable; thus, in the word \textup{timbre-poste}
      it is the first element \textup{timbre} that determines the
      grammatical category of the word. Besides, to remove doubt it
      suffices to give the compound word back its analytic form; the
      element that determines the grammatical category always occupies
      the first position. E.g.: ``\emph{table} for writing'',
      ``\emph{stamp} of postage'', etc.

      Finally, we have seen that every compound word composed of more
      than two elements is divisible into two parts, analyzed
      separately. But this division must not be made
      arbitrarily. Thus, for example, the word
      \textup{passenger'steam'ship} cannot be divided into
      \textup{(passenger'steam)} and \textup{ship}, but only into
      \textup{passenger} and \textup{(steam'ship)}. Similarly the word
      \textup{hum'an'it'ar'isme} `humanitarianism' can only be
      analyzed by the following series of two part words:
      \begin{center}
        \textup{humanitarisme} = \textup{humanitar'isme},\\
        \textup{humanitaire} = \textup{humanit'aire},\\
        \textup{humanité} = \textup{human'ité},\\
        \textup{humain} = \textup{hum'ain}.
      \end{center}
   
    }
  }  

  \FrenchPage{\noindent
    %
    \begin{center}
      § 2. — \textsf{\large Les mots fondamentaux.}
    \end{center}

    Les mots, ou plutôt les idées qu’ils expriment, ne sont pas tous
    indépendants les uns des autres; ils forment, nous l’avons vu (N°
    8) des hiérarchies. Plus une idée est générale, plus le mot qui la
    représente a une place élevée dans cette hiérarchie. Considérons
    par exemple les mots: \textup{chat}, \textup{chien},
    \textup{cheval}, \textup{lion}, \textup{corbeau}, \textup{fourmi},
    etc.; tous ces mots contiennent en eux-mêmes l’idée plus générale
    d'„\emph{animal}“. L’idée „\emph{animal}“ est donc en quelque
    sorte le chef de file auquel sont subordonnées les idées
    particulières: „\emph{chat}“, „\emph{chien}“, „\emph{cheval}“,
    etc.: c’est pourquoi le mot \textup{animal} peut être considéré
    comme ayant dans la hiérarchie des mots un grade plus élevé que
    les mots \textup{chat}, \textup{chien}, etc.

    Il est important de remarquer à ce propos, que c'est l’idée
    particulière „\emph{chat}“ qui contient en elle-même l’idée plus
    générale „\emph{animal}“ (en effet, tous les chats sont des
    animaux, et en adjoignant à l’idée „\emph{chat}“ celle
    d'„\emph{animal}“, on produit un simple pléonasme: \textup{cat} =
    \textup{cat'animal}); au contraire, on ne peut pas dire que l’idée
    générale „\emph{animal}“ contienne en elle-même l’idée
    particulière „\emph{chat}“, car les animaux ne sont pas tous des
    chats; l’idée „\emph{chat}“ est une spécialisation de l’idée
    „\emph{animal}“ (\textsl{chat} = „\emph{animal}, espèce
    \emph{chat}“).

    Au sommet de la hiérarchie se trouvent donc les idées les plus
    générales, les plus abstraites. Ces idées sont l’idée substantive
    (\emph{chose, substance}), l’idée adjective (\emph{qualité}) et
    l’idée verbale (\emph{action}), avec, si l’on veut, l’idée
    adverbiale (\emph{manière}). Les mots qui expriment ces idées sont
    donc les mots \emph{fondamentaux} de la langue. Ils en constituent
    les éléments les plus simples; en effet, tandis que tous les
    autres mots simples (par exemple: \textup{chat}) contiennent en
    eux-mêmes une série d’idées plus générales („\emph{mammifère}“,
    „\emph{vertébré}“, „\emph{animal}“, etc.), les mots fondamentaux,
    comme \textup{chose}, ne contiennent qu’une seule idée, puisque
    l’idée qu’ils expriment est déjà elle-même la plus générale
    possible.

    Les mots fondamentaux sont donc les éléments ultimes formant la
    base de l’analyse des mots, comme les atomes des corps simples
    forment la base de l’analyse chimique. Cherchons quels sont les
    mots fondamentaux de la langue française.

    I.\textsc{ Idée substantive}. L’idée substantive peut être
    exprimée par le mot \textup{chose}, pris dans son sens le plus
    général de „\emph{chose concrète}“ (vivante ou non-vivante) ou de
    „\emph{chose abstraite}“.

    L’idée substantive est souvent exprimée aussi par
    l'\emph{article}, placé devant un adjectif ou un verbe; s’il
    s’agit d’une chose concrète, on emploie l’article indéfini
    \textup{un} (Ex.: „\textup{un} \emph{blanc}“, „\textup{un}
    \emph{noir}“); s’il s’agit d’une chose abstraite, on emploie
    l’article défini \textup{le}, en allemand \textup{das} (Ex.: „\textup{le}
    \emph{boire} et \textup{le} \emph{manger}“,
    „\textup{l}\emph{'utile} et \textup{l}\emph{'agréable}“).

  }
  
  \EnglishPage{\noindent
    %
    \begin{center}
      § 2. — \textsf{\large Basic words.}
    \end{center}

    Words, or rather the ideas that they express, are not all
    independent of one another. They form hierarchies, as we have seen
    (number 8). The more general an idea is, the more the word that
    represents it has a high place in this hierarchy. Let us consider
    for example the words \textup{chat} `cat', \textup{chien} `dog',
    \textup{cheval} `horse', \textup{lion} `lion', \textup{corbeau} `crow',
      \textup{fourmi} `ant', etc. All of these words contain in
      themselves the more general idea of ``\emph{animal}''. The idea
      ``\emph{animal}'' is thus in a way the leading form to which the
      specific ideas ``\emph{cat}'', ``\emph{dog}'', ``\emph{horse}'',
      etc. are subordinated: this is why the word \textup{animal} can
      be considered to have a higher rank in the hierarchy of words
      than the words \textup{chat, chien,} etc.

      It is important to note in this connection that it is the
      specific idea ``\emph{cat}'' that contains in itself the more
      general idea ``\emph{animal}'' (actually, all cats are animals,
      and in adding to the idea ``\emph{cat}''  that of
      ``\emph{animal}'', we produce a simple pleonasm: \textup{cat} =
      \textup{cat'animal}. On the other hand, we cannot say that the
      general idea ``\emph{animal}'' contains in itself the specific
      idea ``\emph{cat}'', because not all animals are cats: the idea
      ``\emph{cat}'' is a specialization of the idea ``\emph{animal}''
      (\textup{cat} = ``\emph{animal}, type \emph{cat}'').

      At the top of the hierarchy are found all of the most general
      ideas, the most abstract. These ideas are the nominal idea
      (\emph{thing, substance}), the adjectival idea (\emph{quality}),
      and the verbal idea (\emph{action}), together with, if you like,
      the adverbial idea (\emph{manner}). The words that express
      these idea are thus the \emph{basic} words of the language. They
      constitute the simplest elements of it; indeed, while all other
      simple words (for example, \textup{chat}) contain in themselves
      a series of more general ideas (``\emph{mammal}'',
      ``\emph{vertebrate}'', ``\emph{animal}'', etc.), the basic words
      like \textup{chose} `thing' only contain a single idea, since
      the idea they express is itself already the most general
      possible.

      The basic words are thus the ultimate elements making up the
      analysis of words, just as atoms are the simple substances that
      form the basis of chemical analysis. Let us seek the basic words
      of the French language.

      I. \textsc{Nominal idea}. The nominal idea can be expressed by
      the word \textup{chose}, taken in its most general sense of
      ``\emph{concerete thing}'' (alive or not alive), or of
      ``\emph{abstract thing}''.

      The nominal idea is often expressed by the \emph{article},
      placed before an adjective or a verb; if it is a question of a
      concrete thing, we use the indefinite article \textup{un}
      (e.g. ``\textup{un} \emph{blanc}'' `a white (one)',
      ``\textup{un} \emph{noir}'' `a black (one)'); if it is a
      question of an abstract thing, we use the definite article
      \textup{le}, in German \textup{das} (e.g. ``\textup{le}
      \emph{boire} et \textup{le} \emph{manger}'' `drink and food',
      ``\textup{l}\emph{'utile} et \textup{l}\emph{'agréable}'' `the
      useful and the pleasant').}

    \FrenchPage{L’idée substantive peut aussi être exprimée par le pronom
      \textup{ce} (ceci, cela), dans le sens de „ce qui est“, „ce qui
      existe“.

      Enfin l’idée substantive générale („\emph{chose abstraite}“) est
      encore exprimable au moyen de suffixes, tels que \textup{ité} ou
      \textup{té} (dans \textup{beau'té}), \textup{eur} (dans
      \textup{grand'eur}), \textup{tion} ou \textup{ation} (dans
      \textup{prépar'ation}), \textup{ture} (dans \textup{écri'ture}),
      etc. En effet, \textup{beauté} signifie „la chose abstraite
      \emph{beau}“, „le beau“; or, par la loi de renversement:
      \textup{beau'té} = „\textup{té} \textup{beau}“, c’est-à-dire que
      le suffixe \textup{té} (ou \textup{ité}) exprime bien l’idée
      substantive générale de „\emph{la chose abstraite}“. Nous
      reviendrons du reste sur l’analyse des mots tels que
      \textup{beau'té} et \textup{écri'ture}.

      II. \textsc{Idée adjective}. On dit souvent que l’adjectif
      exprime la \emph{qualité}, la \emph{propriété}
      (\emph{Eigenschaft}). Mais il y a lieu de remarquer que les mots
      \textup{qualité}, \textup{propriété}, sont des substantifs; ils
      représentent donc, non l’idée adjective elle-même (qui n’est pas
      une chose), mais l’idée adjective substantifiée. Ainsi, ce ne
      sont pas les adjectifs \textup{égal}, \textup{grand},
      \textup{riche}, etc., qui expriment des „\emph{qualités}“, des
      „\emph{propriétés}“, mais ce sont les substantifs
      \textup{égal'ité}, \textup{grand'eur}, \textup{rich'esse},
      etc. Or, ceci signifie que hiérarchiquement les mots généraux
      \textup{qual'ité} et \textup{propri'été}, sont chefs de file des
      mots particuliers \textup{égal'ité}, \textup{grand'eur},
      \textup{rich'esse}, etc.; ou encore que les mots \textup{qual}
      et \textup{propre}, sont chefs de file des adjectifs
      \textup{égal}, \textup{grand}, \textup{riche}, etc.; autrement
      dit, tout adjectif contient en lui-même l’idée „\emph{qual}“ (ou
      l’idée „\emph{propre}“) à l’état latent; le radical
      \textup{qual}\footnote{Le mot-racine \textup{qual} n’existe pas
        en français comme adjectif autonome (car le mot français
        \textup{quel} n’a pas tout à fait la même signification), mais
        nous verrons plus loin (p. 22) que \textup{qual} est synonyme
        de \textup{qualitatif}.} (qui n’est autre que l’adjectif latin
      \textup{qualis}) et le mot \textup{propre} sont donc des
      éléments fondamentaux, qui expriment l’idée adjective générale,
      car ils expriment l’idée commune à tous les adjectifs.

      On peut arriver au même résultat d’une autre manière, en
      comparant les deux séries suivantes:

      \begin{center}
        \begin{tabular}[t]{ll@{ }l}
          FRANÇ'AIS, & QUAL'ITÉ & (PROPRI'ÉTÉ),\\
          \emph{Lyonn'ais}, & \emph{util'ité},\\
          \emph{Marseill'ais},  & \emph{égal'ité},\\
          \emph{Toulon'ais},  & \emph{médiocr'ité},\\
          \emph{Orléan'ais},  & \emph{van'ité},\\
          \multicolumn{1}{c}{etc.}  & \multicolumn{1}{c}{etc.}.
        \end{tabular}
      \end{center}

      La première colonne montre que le mot \textup{Franç'ais} est le
      chef de file des mots \textup{Lyonn'ais}, \textup{Marseill'ais},
      etc., et la seconde colonne, que le mot \textup{qual'ité} est le
      chef de file des mots \textup{égal'ité}, \textup{util'ité},
      etc. Or, si le mot \textup{Lyonn'ais} contient l’idée de
      „\emph{Franç'ais}“, c’est évidemment parce que le mot
      \textup{Lyon} contient l’idée de „\emph{France}“; de même, si
      le mot \textup{util'ité} contient l’idée de „\emph{qual'ité}“ ou
      de „\emph{propri'été}“ c’est parce que le mot \textup{util}
      contient l’idée „\emph{qual}“, ou l’idée „\emph{propre} [à]“.   

  }
  
  \EnglishPage{The nominal idea can also be expressed by the pronoun
    \textup{ce} (ceci `this', cela `that'), in the sense of ``that
    which is, that which exists''.

    Finally, the general nominal idea (``\emph{abstract thing}'') is
    also expressible by means of suffixes, such as \textup{ité} or
    \textup{té} (in \textup{beau'té}), \textup{eur} (in
    \textup{grand'eur}), \textup{tion} or \textup{ation} (in
    \textup{prépar'ation}), \textup{ture} (in \textup{écri'ture}),
    etc. Actually, \textup{beauté} means ``the abstract thing
    \emph{beautiful}'', ``the beautiful''; or thus by the law of
    reversal \textup{beauté} = ``\textup{té beau}'', that is the
    suffix \textup{té} (or \textup{ité}) expresses the general
    nominal idea of ``\emph{the abstract thing}''.  We will come back
    additionally to the analysis of words such as \textup{beau'té}
    and \textup{écri'ture}.

    II. \textsc{Adjectival idea}. We often say that the adjective
    expresses a \emph{quality}, a \emph{property}
    (\emph{Eigenschaft}). But it should be noted that the words
    \textup{qualité,  propriété} are nouns: they thus represent not
    the adjectival idea itself (which is not a thing) but the
    nominalized adjectival idea.  Thus it is not the adjectives
    \textup{égal, grand, riche} etc. that express ``qualities'' or
    ``properties'', but rather the nouns \textup{égal'ité,
      grand'eur, rich'esse} etc. Now that means that hierarchically
    the general words \textup{qualité} and \textup{propriété}
    are the leading forms for the specific words \textup{égal'ité,
      grand'eur, rich'esse} etc.; or again that the words
    \textup{qual} and \textup{propre} are the leading forms for the
    adjectives \textup{égal, grand, riche} etc. To put it another
    way, every adjective contains in itself the idea ``qual'' (or the
    idea ``propre'') in a latent state: the root
    \textup{qual}\footnote{The root word \textup{qual} does not exist
      in French as an autonomous adjective (for the French word
      \textup{quel} does not have at all the same meaning), but we
      will see below (p. 22) that \textup{qual} is a synonym of
      \textup{qualitatif}.} (which is nothing more than the Latin word
    \textup{qualis}) and the word \textup{propre} are thus basic
    elements which express the general adjectival idea, since they
    express the idea common to all adjectives.

    We can arrive at the same result in another way, by comparing the
    two series below:

    \begin{center}
        \begin{tabular}[t]{ll@{ }l}
          FRANÇ'AIS, & QUAL'ITÉ & (PROPRI'ÉTÉ),\\
          \emph{Lyonn'ais}, & \emph{util'ité},\\
          \emph{Marseill'ais},  & \emph{égal'ité},\\
          \emph{Toulon'ais},  & \emph{médiocr'ité},\\
          \emph{Orléan'ais},  & \emph{van'ité},\\
          \multicolumn{1}{c}{etc.}  & \multicolumn{1}{c}{etc.}.
        \end{tabular}
      \end{center}

      The first column shows that the word \textup{Fran\c{c}'ais}
      `French'man' is the leading form for the words
      \textup{Lyonn'ais}, \textup{Marseill'ais}, etc., and the second
      column that the word \textup{qual'ité} is the leading form for
      the words \textup{égal'ité}, \textup{util'ité}, etc. Now if the
      word \textup{Lyonn'ais} contains the idea of ``French'man'' it
      is obviously because the word \textup{Lyon} contains the idea of
      ``France'': similarly, if the word \textup{util'ité} contains
      the idea of ``qual'ité'' or of ``propri'été'', it is because the
      word \textup{util} `useful' contains the idea ``qual'' or the
      idea ``proper [to]''.
    }

    \FrenchPage{
      On voit maintenant pourquoi \textup{utilité} signifie „qualité
      \emph{util}“. En appliquant la loi de renversement, on a:
      „\emph{qualité util}“ = \textup{util'qual'ité}; or, l’idée
      „\emph{qual}“ existant déjà dans l’adjectif \textup{util}
      produit un pléonasme superflu, qu’on peut supprimer: le mot
      \textup{util'qual'ité} se réduit donc à
      \textup{util'ité}\footnote{Ainsi donc ce n’est pas le suffixe
        \textup{ité} qui apporte dans un mot l’idée de
        „\emph{qualité}“. Ce suffixe n’apporte que l’idée substantive,
        et l’idée „\emph{qual}“ est apportée implicitement par
        l’adjectif qui est accolé au suffixe \textup{ité}.}. On
      démontrerait de même que:

      \textup{Lyonn'ais} = „\emph{Franç'ais de Lyon}“, car, d’après la
      loi de renversement: „\emph{Franç'ais Lyon}“ =
      \textup{Lyon'Franç’ais}; mais l’idée „ \emph{France}“ existant
      déjà dans le mot \textup{Lyon} produit un pléonasme inutile,
      qu’on peut supprimer: le mot \textup{Lyon'Franç'ais} se réduit
      donc à \textup{Lyonn'ais}.

      En résumé, l’idée adjective générale doit être représentée, non
      par les substantifs \textup{qualité}, \textup{propriété}, mais
      par les adjectifs \textup{qual}, \textup{propre (à)}.

      L’idée adjective est en outre exprimable par de nombreux
      suffixes, tels que \textup{ain} (dans \textup{hum'ain}),
      \textup{ique} (dans \textup{symbol'ique}), \textup{eux} (dans
      \textup{chanc’eux}), \textup{al} (dans \textup{nation'al}),
      etc. Tous ces suffixes sont donc synonymes de l’idée adjective
      exprimée par les mots racines \textup{qual}, \textup{propre}
      (à), c’est-à-dire qu’ils sont théoriquement interchangeables
      avec ces racines. Ainsi, par exemple, on a par la loi de
      renversement:

      \textup{hum'ain} — „\emph{ain hom}“ — „\emph{propre [à]
        [l']homme}“; ou encore, en remplaçant le suffixe \textup{ain}
      par la racine \textup{qual}, dans le mot \textup{humanité}:

      \textup{hum'an’ité} = \textup{hom'qual'ité} = „\emph{qualité
        [d’]homme}“.

      L’idée adjective générale est aussi exprimable par la
      préposition \textup{de}. En effet:

      \begin{center}
        \begin{tabular}[t]{l}
        „pied hum'\emph{ain}“ = „pied \emph{d}’homme“;\\
        „amour pater'\emph{nel}“ = „amour \emph{de} père“;
        \end{tabular}
      \end{center}

      \noindent
      ces égalités montrent, en tenant compte de la loi de
      renversement, que la préposition \textup{de} est bien synonyme
      des suffixes \textup{ain}, \textup{el}, \textup{ique}, etc.

      Enfin l’idée adjective peut encore être exprimée par le mot
      \textup{qui} dans le sens de „\emph{qui est}“. Pour s’en rendre
      compte, il suffit de remarquer qu’on n’ajoute rien à un adjectif
      en lui adjoignant l’expression „\emph{qui est}“. Par exemple,
      „un homme grand“ = „un homme \emph{qui est} grand“. L’expression
      „\emph{qui est grand}“ contient donc un pléonasme, puis qu’elle
      est réductible à \textup{grand} (voir n° 4, p. 10); or, ceci
      revient à dire que l’idée „\emph{qui est}“ est implicitement
      contenue dans tout adjectif\footnote{Ne pas confondre l’idée
        „\emph{qui est}“ (idée adjective) avec l’idée „\emph{ce qui
          est}“ (idée substantive, voir plus haut).}.

      III. \textsc{Idée verbale}. Le verbe, dit-on généralement,
      exprime l'„\emph{action}“ ou l'„\emph{état}“. Mais les mots
      \textup{action} et \textup{état} sont des sub-
  }
  
  \EnglishPage{We see now why \textup{utilité} means ``quality
    \emph{util}''. Applying the law of reversal, we have:
    ``\emph{quality util}'' = \textup{util'qual'ity}; now since the
    idea ``\emph{qual}'' already present in the adjective
    \textup{util} produces an unnecessary pleonasm, we can remove it,
    and the word \textup{util'qual'ity} reduces to
    \textup{util'ity}\footnote{Thus it is not the suffix
      \textup{ité} that carries the idea of ``\emph{quality}'' in a
      word. This suffix only carries the nominal idea, and the idea
      ``\emph{qual}'' is implicitly borne by the adjective which is
      attached to the suffix \textup{ité}.} We can similarly
    demonstrate that:

    \textup{Lyonn'ais} = ``\emph{Fran\c{c}'ais} `French'man'
    \emph{from Lyon}'', since according to the law of reversal,
    ``\emph{Fran\c{c}'ais Lyon}'' = \textup{Lyon'Fran\c{c}'ais}, but
    since the idea ``\emph{France}'' exists already in the word
    \textup{Lyon}, this produces an unnecessary pleonasm which we can
    eliminate: the word \textup{Lyon'Fran\c{c}'ais} thus reduces to
    \textup{Lyonn'ais}.

    To summarize, the general adjective idea must be represented not
    by the nouns \textup{quality, property}, but by the adjectives
    \textup{qual, proper (to)}.

    The adjectival idea can also be represented by a number of
    suffixes, such as \textup{ain} (in \textup{hum'ain}),
    \textup{ique} (in \textup{symbol'ique}), \textup{eux} (in
    \textup{chanc’eux}), \textup{al} (in \textup{nation'al}), etc. All
    of these suffixes are thus synonyms of the adjectial idea
    expressed by the root words \textup{qual, proper (to)}; that is,
    they are theoretically interchangeable with these roots. Thus, for
    example, we have by the law of reversal:

    \textup{hum'ain} = ``\emph{ain hom}'' = ``\emph{proper [to]
      man}''; or again, replacing the suffix \textup{ain} by the root
    \textup{qual}, in the word \textup{humanité}:

    \textup{hum'an'ité} = \textup{hom'qual'ité} = ``\emph{quality
      [of] man}''.

    The general adjectival idea is also expressible by the preposition
    \textup{de} `of'. Thus:

    \begin{center}
        \begin{tabular}[t]{l}
        ``pied hum'\emph{ain}'' = ``pied \emph{d}’homme (`foot \emph{of} man')'';\\
        ``amour pater'\emph{nel}'' = ``amour \emph{de} père (`love \emph{of} father')'';
        \end{tabular}
      \end{center}

      \noindent
      These equations show, taking the law of reversal into account,
      that the preposition \textup{de} is indeed synonymous with the
      suffixes \textup{ain}, \textup{el}, \textup{ique}, etc.

      Finally, the adjectival idea can also be expressed by the word
      \textup{qui} in the sense of ``\emph{qui est}'' `who/which
      is'. To recognize this, it is sufficient to note that we add
      nothing to an adjective when we adjoin the expression
      ``\emph{who/which is'}'. For example, ``un homme grand'' `a tall
      man' = ``a man \emph{who is} tall''.  The expression ``\emph{who
        is tall}'' thus contains a pleonasm, since it can be reduced
      to \textup{tall} (see number 4, p. 10); now this comes down to
      saying that the idea ``\emph{who/which is}'' is implicitly
      contained in every adjective\footnote{Do not confuse the ides
        ``\emph{who/which is}'' (adjectival idea) with the idea
        ``\emph{that which is}'' (nominal idea, see above).}.

      III. \textsc{Verbal idea}. The verb, we usually say, expresses
      ``\emph{action}'' or ``\emph{state}''.  But the words
      \textup{action} and \textup{state} are no[uns]
    }

  \FrenchPage{\noindent
    %
    stantifs; ils représentent donc, non l’idée verbale elle-même
    (qui n’est pas une „chose“), mais l’idée verbale
    substantifiée. L’idée verbale ne peut être définie que par les
    verbes correspondant aux substantifs \textup{action} et
    \textup{état}, c’est-à-dire par les verbes \textup{agir},
    \textup{faire} (\emph{une action}) ou \textup{être} (\emph{dans un
      état}). Mais les mots \textup{ag'ir}, \textup{fai're},
    \textup{êt’re}, se composent encore de deux éléments: un
    mot-racine \textup{ag}, \textup{fai} ou \textup{êt}, et une
    désinence \textup{ir} ou \textup{re}. Cette désinence, qui sert à
    exprimer le temps du verbe, est évidemment superflue pour l’objet
    que nous avons en vue. Donc, en dernière analyse, les éléments
    fondamentaux qui expriment l’idée verbale générale sont les
    racines \textup{ag}, \textup{fai}, ou \textup{êt}.

    On peut arriver à ce résultat d’une autre manière: de même que le
    mot \textup{qualité}, ou le mot \textup{propriété}, représente
    l’idée commune à tous les adjectifs substantifiés
    (\textup{égalité}, \textup{utilité}, etc.), de même le mot
    \textup{action}, ou le mot \textup{état}, représente l’idée
    commune à tous les verbes substantifiés. Formons le tableau de ces
    substantifs:

    \begin{center}
      \begin{tabular}[t]{ll}
        AC'TION,&ÉT'AT\\        
        \emph{abdic'ation},&\emph{abond'ance},\\
        \emph{fabric'ation},&\emph{exist'ence}\\
        \emph{pénétr'ation},&\emph{suffis'ance},\\
        \multicolumn{1}{c}{etc.} &\multicolumn{1}{c}{etc.}
      \end{tabular}
    \end{center}

    Puisque le mot \textup{pénétr'ation}, par exemple, contient en
    lui-même l’idée d’„\emph{ac'tion}“, le radical \textup{pénétr}
    doit contenir l’idée „\emph{ac}“ (ou „\emph{ag}“, racine du verbe
    \textup{ag'ir}). Ainsi, toutes les racines verbales contiennent
    implicitement en elles-mêmes l’une des deux idées générales
    „\emph{ag}“ ou „\emph{êt}“. On retrouve bien ainsi le même
    résultat, et l’on comprend maintenant pourquoi on peut écrire:
    \textup{pénétr'ation} = „\emph{ac'tion pénétr}“; en effet, en
    vertu de la loi de renversement, le second membre de cette égalité
    peut s’écrire: \textup{pénétr'ac'tion}, mot composé qui se réduit
    à: \textup{pénétr'tion} puisque l’idée „\emph{ac}“ (ou
    „\emph{ag}“) est déjà contenue dans la racine verbale
    pénétr. Ainsi les deux égalités:

    \begin{center}
      \begin{tabular}[t]{l@{ = }l@{ [espèce] }l}
        \textup{pénétr'ation} &  „\emph{ac'tion} & \emph{pénétr}“\\
        \textup{util'ité} & „\emph{qual'ité} & \emph{util}“
      \end{tabular}
    \end{center}
    
    \noindent
    sont en tous points semblables à l’égalité:

    \begin{center}
      \begin{tabular}[t]{l@{ = }l@{ [espèce] }l}
        \textup{Lyonn'ais} & „\emph{Franç'ais} & \emph{Lyon}“.
      \end{tabular}
    \end{center}

    Considérons maintenant les verbes tels que \textup{couronn'er},
    \textup{clou'er} \textup{pâl'ir}, etc., dérivés d’un substantif ou
    d’un adjectif. Comme les substantifs \textup{couronne},
    \textup{clou}, etc., ne contiennent pas d’idée verbale, celle-ci
    ne peut être contenue que dans les suffixes verbaux \textup{er},
    \textup{ir}, etc., des verbes \textup{couronn'er},
    \textup{pâl'ir}, etc. On en conclut donc que l’idée verbale
    générale peut être exprimée aussi par les suffixes
  }
  
  \EnglishPage{\noindent
    %
    [no]uns; they thus represent not the verbal idea itself (which is
    not a ``thing'') but the nominalized verbal idea. The verbal idea
    can only be defined by the verbs corresponding to the nouns
    \textup{action} and \textup{state}, that is by the verbs
    \textup{agir} `to act', \textup{faire} `to do, perform (\emph{an
      action})' or \textup{être} `to be (\emph{in a state})'. But
    the words \textup{ag'ir}, \textup{fai're}, \textup{êt're} are
    again made up of two elements: a root word \textup{ag},
    \textup{fai}, or \textup{êt} and a desinence \textup{ir} or
    \textup{re}. This desinence, which serves to express the tense of
    the verb, is obviously superfluous for our object here. Thus, in
    the last analysis, the basic elements that express the general
    verbal idea re the roots \textup{ag},
    \textup{fai}, or \textup{êt}.

    We can reach this result in another way: just as the word
    \textup{quality} or the word \textup{property} represents the idea
    common to all nominalized adjectives, so the word \textup{action}
    or the word \textup{state} represents the idea common to all th
    nominalized verbs. Let us make up the table of these nouns:

    \begin{center}
      \begin{tabular}[t]{ll}
        AC'TION,&ST'ATE\\        
        \emph{abdic'ation},&\emph{abond'ance},\\
        \emph{fabric'ation},&\emph{exist'ence}\\
        \emph{pénétr'ation},&\emph{suffis'ance},\\
        \multicolumn{1}{c}{etc.} &\multicolumn{1}{c}{etc.}
      \end{tabular}
    \end{center}

    Since the word \textup{pénétr'ation}, for example, contains in
    itself the idea of ``\emph{ac'tion}'', the root
    \textup{pénétr} must contain the idea ``\emph{ac}'' (or
    ``\emph{ag}'', root of the verb \textup{ag'ir}). Thus all verbal
    roots implicitly contain in themselves one of the two genreal
    ideas ``\emph{ag}'' or ``\emph{êt}''.  We thus find again the
    same result, and we now understand why we can write:
    \textup{pénétr'ation} = ``\emph{ac'tion pénétr}''; actually, by
    virtue of the law of reversibility the second member of this
    equation can be written: \textup{pénétr'ac'tion}, a compound word
    that reduces to: \textup{pénétr'tion} since the idea ``\emph{ac}''
    (or ``\emph{ag}'') is already contained in the root
    \textup{pénétr}. Thus the two equations:

    \begin{center}
      \begin{tabular}[t]{l@{ = }l@{ [type] }l}
        \textup{pénétr'ation} &  ``\emph{ac'tion} & \emph{pénétr}''\\
        \textup{util'ité} & ``\emph{qual'ité} & \emph{util}''
      \end{tabular}
    \end{center}
    
    \noindent
    are in every respect similar to the equation:

    \begin{center}
      \begin{tabular}[t]{l@{ = }l@{ [type] }l}
        \textup{Lyonn'ais} & „\emph{Franç'ais} & \emph{Lyon}“.
      \end{tabular}
    \end{center}

    Let us now consider verbs such as \textup{couronn'er},
    \textup{clou'er} \textup{pâl'ir} `to turn pale', etc., derived
    from a noun or an adjective. As the nouns \textup{couronne},
    \textup{clou}, etc. do not contain a verbal idea, this must only
    be contained in the verbal suffixes \textup{er}, \textup{ir},
    etc. of the verbs \textup{couronn'er}, \textup{pâl'ir}, etc. We
    conclude from this that the general verbal idea can also be
    expressed by the suffixes
    }

  \FrenchPage{\noindent
    %
    \textup{er}, \textup{ir}, \textup{re}, etc., exactement comme
    l’idée adjective l’est par les suffixes \textup{ain},
    \textup{ique}, \textup{eux}, etc.\footnote{Dans tout cet essai
      nous ne considérons les verbes qu’à l’infinitif, car ce qui nous
      intéresse dans les désinences verbales, ce ne sont pas les
      différents temps du verbe, mais uniquement le fait que ces
      désinences expriment aussi l’idée verbale générale.}

    \emph{Résumé}. Les mots ou éléments fondamentaux, derniers résidus
    de l’analyse des mots dans la langue française, sont les suivants:
    1. le mot \textup{chose}, l’article \textup{un} ou \textup{le}, le
    pronom \textup{ce}, les suffixes \textup{ité}, \textup{eur}, etc.,
    \textup{tion}, \textup{ture}, etc., qui expriment l’idée générale
    \emph{substantive}; 2. le mot \textup{propre} (à), la racine
    \textup{qual}, le pronom-adjectif \textup{qui} (est), la
    préposition \textup{de}, les suffixes \textup{ain}, \textup{ique},
    \textup{al}, \textup{eux}, etc., qui expriment l’idée générale
    \emph{adjective}; 3. les racines \textup{ag}, \textup{fai},
    \textup{êt}, les suffixes verbaux \textup{er}, \textup{ir},
    \textup{re}, etc., qui expriment l’idée générale \emph{verbale};
    à ces mots fondamentaux on peut encore ajouter, comme n° 4, le
    suffixe \textup{ment} (dans \textup{agréable'ment}) qui exprime
    l’idée générale \emph{adverbiale}, et qui est synonyme de l’idée
    „\emph{à la manière}“.

    Comme on le voit ci-dessus, et comme il est naturel, tous les mots
    fondamentaux sont des mots ou éléments simples. En effet, les mots
    fondamentaux, en tant que derniers résidus d’analyse, doivent être
    non seulement des mots simples, mais parmi les mots simples ils
    doivent être ceux dont la constitution est la plus simple; ainsi,
    tout mot fondamental, comme \textup{chose} par exemple, ne
    contient en lui-même aucune autre idée plus générale, tandis qu’un
    mot simple, non fondamental, comme \textup{chat}, contient
    implicitement en lui une série d’autres idées plus générales, tels
    que „\emph{mammi-fère}“, \emph{vertébré}, „\emph{animal}“, etc. Le
    mot \textup{chose} est comparable en quelque sorte à une boule
    pleine et homogène, tandis que tout mot simple et non fondamental
    (comme \textup{chat}) peut être comparé à ces boules creuses, qui
    contiennent à leur intérieur une série de boules plus petites,
    emboîtées les unes dans les autres et correspondant aux idées plus
    générales (\emph{mammifère, vertébré, animal,} etc.)
    implicitement contenues dans ce mot.

    Or, le but de l’analyse des mots est d’expliquer la signification
    des mots à structure complexe, par celle des mots à structure
    simple; on explique donc les mots composés par les mots simples,
    et les mots simples par les mots fondamentaux. Il en résulte que
    ces derniers ne peuvent être définis autrement que par eux-mêmes;
    les mots fondamentaux sont les signes représentatifs de l’idée
    substantive, de l’idée adjective ou de l’idée verbale, et toute
    autre définition serait illusoire, car elle impliquerait l’emploi
    de mots plus complexes que les mots fondamentaux qu’il s’agit de
    définir; on tomberait dans un cercle vicieux, comme le chimiste
    qui après avoir expliqué les molécules des corps par les atomes,
    voudrait définir ces atomes à leur tour par des
    molécules\footnote{Voir à ce propos la note de la page 25.}.
  }
  
  \EnglishPage{\noindent
    %
    \textup{er}, \textup{ir}, \textup{re}, etc., exactly as the
    adjectival idea is by the suffixes \textup{ain}, \textup{ique},
    \textup{eux}, etc.\footnote{Throughout this essay we only consider
      verbs in the infinitive, because what interests us in the verbal
      desinences is not the different tenses of the verb, but only the
      fact that these desinences also express the general verbal
      idea.}

    \emph{Summary.} The basic words or elements, the final residue of
    the analysis of words in the French language, are the following:
    the word \textup{chose} `thing', the article \textup{un} or
    \textup{le}, the pronoun \textup{ce}, the suffixes \textup{ité},
    \textup{eur}, etc., \textup{tion}, \textup{ture}, etc., which
    express the general \emph{nominal} idea; 2. the word
    \textup{propre} `proper (to)', the root \textup{qual}, the
    pronoun-adjective \textup{qui} `who/which (is)', the preposition
    \textup{de}, the suffixes \textup{ain}, \textup{ique},
    \textup{al}, \textup{eux}, etc., which express the general
    \emph{adjectival} idea; 3. the roots \textup{ag}, \textup{fai},
    \textup{êt}, the verbal suffixes \textup{er}, \textup{ir},
    \textup{re}, etc., which express the general \emph{verbal} idea;
    to these basic words we can also add, as number 4, the suffix
    \textup{ment} (in \textup{agréable'ment}) which expresses the
    general \emph{adverbial} idea, and which is synonymous with
    ``\emph{in the manner}''.

    As we see above, and as is natural, all of the basic words are
    simple words or elements. Indeed, the basic words, as the final
    residue of the analysis, must be not only simple words, but among
    the simple words, they must be ones whose constitution is the
    simplest.  Thus, every basic word, like \textup{chose} `thing' for
    example, does not contain in itself any more general idea, while a
    simple but non-basic word like \textup{chat} `cat' implicitly
    contains in itself a series of other more general ideas, such as
    ``\emph{mammal}'', ``\emph{vertebrate}'', ``\emph{animal}'',
    etc. The word \textup{chose} is in a way comparable to a solid,
    homogenous ball, while a simple but non-basic word (like
    \textup{chat}) can be compared to those hollow balls, which contain
    within them a series of smaller balls, enclosed within one another
    and corresponding to the more general ideas (\emph{mammal,
      vertebrate, animal}, etc.) implicitly contained in this word.

    Now the goal of the analysis of words is to explain the meaning of
    structurally complex words by that of structurally simple words.
    We thus explain compound words by simple words, and simple words
    by the basic words. The result is that these last cannot be
    defined otherwise than by themselves: the basic words are the
    signs representative of the nominal idea, the adjectival idea or
    the verbal idea, and any other definition would be illusory, for
    it would imply the use of words more complex than the basic words
    which are to be defined; we would fall into a vicious circle, like
    the chemist who after explaining the molecules of a body by atoms,
    would wish to define these atoms in turn by molecules\footnote{In
      this connection, see the note on page 25.}.
    }

  \FrenchPage{\small
    
    \emph{Remarque sur les mots fondamentaux}. Mais ici une remarque
    s’impose: les mots fondamentaux, qui définissent les idées
    générales de substantif, d’adjectif ou de verbe, sont des éléments
    simples, dont quelques-uns (par exemple les éléments
    \textup{qual}, \textup{ag}, \textup{êt,} etc.) ne sont pas des
    mots autonomes, et ne peuvent pas par conséquent être employés
    tels quels dans le langage courant. C’est pourquoi, dans la
    pratique, les grammairiens définissent quelquefois les idées
    substantive, adjective et verbale par des mots en apparence non
    fondamentaux, voire même par des mots composés, ou des expressions
    encore plus complexes. Ainsi, on peut définir, par exemple, l’idée
    adjective comme étant l’idée exprimée par le mot
    \textup{qualitatif} ou par les expressions „\emph{de qualité}“,
    „\emph{qui est de qualité}“ (\textup{hum'ain} = „\textup{ain}
    \textup{homme}[“] = „\emph{de qualité} homme“, „\emph{qui est de
      qualité} homme“). Mais il est facile de voir que toutes ces
    expressions ne sont complexes qu’en apparence; elle sont toutes
    logiquement réductibles à l’adjectif fondamental \textup{qual}:
    en effet, si d’un adjectif comme \textup{beau} on dérive le
    substantif \textup{beau'té}, en anglais \textup{beauty}, et
    qu’ensuite du substantif \textup{beauty} on dérive l’adjectif
    \textup{beauti'ful}, on aura \textup{beau'ti’ful} = \textup{beau},
    parce que les deux opérations s’annulent réciproquement, l’une
    étant l’inverse de l’autre; de même, si de l’adjectif latin
    \textup{qual} on forme le substantif \textup{qual'itas}, en
    français \textup{qualité}, et qu’ensuite du subtantif
    \textup{qualité} on dérive l’adjectif \textup{qualita'tif}, on
    aura \textup{qual'ita'tif} = \textup{qual}, pour la même raison
    que \textup{beau’ti'ful} = \textup{beau}. Ainsi, le mot
    \textup{qualitatif} est bien un mot fondamental, qui représente
    l’idée adjective, et la complexité de sa structure n’est
    qu’apparente.

    Il en est de même des expressions d’apparence encore plus
    complexes: „\emph{de qualité}“ ou „\emph{qui est de qualité}“. En
    effet, nous savons que l’expression „\emph{qui est}“ équivaut à
    l’idée adjective (voir p. 19); en outre, on a par la loi de
    renversement: ,

    \begin{center}
      „\emph{de qualité}“ = „\emph{qualita'tif}“
    \end{center}

    \noindent
    puisque la préposition \textup{de} et le suffixe \textup{tif}
    expriment tous deux l’idée adjective; enfin, nous venons de voir
    que \textup{qualitatif} se réduit à \textup{qual}, donc en résumé
    l’expression „\emph{qui est de qualité}“ se réduit à „\emph{qui
      est qual}“, expression qui se réduit elle-même à
    „\emph{qual}“. On arriverait directement au même résultat, en
    remarquant que l’on n’ajoute rien à un adjectif en lui adjoignant
    l’expression „\emph{qui est}“ ou, „\emph{de qualité}“; ainsi:

    \begin{center}
      \begin{tabular}[t]{r@{ = }l}
        „un homme \emph{grand}“ & „un homme \emph{qui est grand}“\\
                                & „un homme \emph{qui est de qualité grand}“.
      \end{tabular}
    \end{center}

    On peut faire des remarques semblables à propos de l’idée verbale
    et des éléments fondamentaux \textup{ag}, \textup{fai} ou
    \textup{êt}, qui la représentent. Les grammairiens définissent
    généralement l’idée verbale au moyen des expressions en apparence
    complexes: „\emph{faire une action}“ ou „\emph{être dans un
      état}“; mais „\emph{faire une action}“ se réduit à
    \textup{faire}, le mot „action“ servant seulement à indiquer que
    le verbe \textup{faire} doit ici être pris dans le sens
    d’\textup{agir}; de même l’expression „\emph{être dans un état}“
    se réduit à \textup{être}, le mot „état“ servant seulement à
    indiquer que le verbe \textup{être} ne doit pas être pris ici dans
    le sens d'„\emph{exister}“. Finalement les verbes \textup{fai're},
    \textup{ag'ir}, \textup{êt're}, qui expriment l’idée verbale
    renferment encore un pléonasme, car les suffixes verbaux comme
    \textup{re}, \textup{ir}, etc., n’expriment eux-même que l’idée
    verbale; les expressions „\emph{faire une action}“ et
    „\emph{être dans un état}“ sont donc bien réductibles aux éléments
    fondamentaux \textup{fai}, \textup{ag} ou \textup{êt}.\\[1ex]

    {\large § 3. — \textsf{Exemples d’analyses et de synthèses.}}\\

    1. Faire l’analyse du mot: \textup{grandeur}.

    D’après la loi de renversement, \textup{grand'eur} = „\emph{eur
      grand}“, c’est-à-dire „\emph{la chose grand}“, puisque le
    suffixe \textup{eur} exprime l’idée substantive générale de
    „chose“. Cette analyse est insuffisante.}
  
  \EnglishPage{\small
    %
    \emph{Remark on the basic words}. But here a remark is necessary:
    the basic words, which define the general ideas of noun,
    adjective, or verb are simple elements, of which some (for example
    \textup{qual}, \textup{ag}, \textup{êt}, etc.) are not
    autonomous words, and cannot as a consequence be used as such in
    everyday language. This is why, in practice, grammarians sometimes
    define the basic nominal, adjectival, and verbal ideas with words
    that do not appear basic, even with compound words or expressions
    even more complex. Thus we can define, for example, the adjectival
    idea as as being expressed by the word \textup{qualitatif} or by
    the expressions ``\emph{of quality}'', ``\emph{which is of
      quality}'' (\textup{hum'ain} = ``\textup{ain} \textup{homme}'' =
    ``\emph{of quality} man'', ``\emph{which is of quality}
    man''). But it is easy to see that all of these expressions only
    appear complex; they are all logically reducible to the basic
    adjective \textup{qual}. Actually, if from an adjective like
    \textup{beau} we derive the noun \textup{beau'té}, in English
    \textup{beauty}, and if then from the noun \textup{beauty} we
    derive the adjective \textup{beautiful}, we will have
    \textup{beau'ti'ful} = \textup{beau}, because the two operations
    cancel each other, the one being the inverse of the other.
    Similarly, if from the Latin adjective \textup{qual} we form the
    noun \textup{qual'itas}, in French \textup{qualité}, and then
    from the noun \textup{qualité} we derive the adjective
    \textup{qualita'tif}, we will have \textup{qual'ita'tif} =
    \textup{qual}, for the same reason that \textup{beau'ti'ful} =
    \textup{beau}. Thus the word \textup{qualitatif} is indeed a basic
    word which represents the adjectival idea, and its structural
    complexity is only apparent.

    The same is true for the apparently even more complex expressions
    ``\emph{of quality}'' or ``\emph{which is of quality}''. Actually,
    we know that the expression ``\emph{who/which is}'' is equivalent
    to the adjectival idea (see p. 19). Besides, by the law of
    reversal we have:

    \begin{center}
      ``\emph{de qualité}'' = ``\emph{qualita'tif}''
    \end{center}

    \noindent
    since the preposition \textup{de} and the suffix \textup{tif} both
    express the adjectival idea.  Finally, we have just seen that
    \textup{qualitatif} reduces to \textup{qual}, and so in sum the
    expression ``\emph{qui est de qualité}'' reduces to ``\emph{qui
      est qual}'', an expression which itself reduces to
    ``\emph{qual}''.  We would arrive at the same result in noting
    that we add nothing to an adjective when we add to it the
    expression ``\emph{qui est}'' or ``\emph{de qualité}'', thus:

    \begin{center}
      \begin{tabular}[t]{r@{ = }l}
        ``un homme \emph{grand}'' & ``un homme \emph{qui est grand}''\\
                                & ``un homme \emph{qui est de qualité grand}''.
      \end{tabular}
    \end{center}
    
    We can make similar remarks concerning the verbal idea and the
    basic elements \textup{ag}, \textup{fai} or \textup{êt} which
    represent it.  Grammarians generally define the verbal idea by
    means of apparently complex expressions: ``\emph{to perform an
      action}'' or ``\emph{to be in a state}'', but ``\emph{faire une
      action}'' `to perform an action' reduces to \textup{faire} `to
    do', the word ``action'' serving only to indicate that the verb
    \textup{faire} must here be taken in the sense of \textup{agir}
    `to act'. Similarly the expression ``\emph{être dans un état}''
    `to be in a state' reduces to \textup{être}, the word ``état''
    `state' serving only to indicate that the verb \textup{être}
    must not here be taken in the sense ``to exist''. Finally, the
    verbs \textup{fai're}, \textup{ag'ir}, \textup{êt're} which
    express the verbal idea again contain a pleonasm, since the
    suffixes like \textup{re}, \textup{ir}, etc. themselves only
    express the verbal idea; the expressions ``\emph{faire une
      action}'' and ``\emph{être dans un état}'' are thus reducible to
    the basic elements \textup{fai}, \textup{ag}, or \textup{êt}.\\[1ex]

    {\large § 3. — \textsf{Examples of analyses and syntheses.}}\\

    1. Analyze the word: \textup{grandeur}

    According to the law of reversal, \textup{grand'eur} = ``\emph{eur
    grand}'', that is ``\emph{la chose grand}'' `the thing large', since
  the suffix \textup{eur} expresses the general nominal idea of
  ``thing''. This analysis is insufficient.}

  \FrenchPage{\small
    
    Pour pousser l’analyse plus à fond, il faut, d’après la règle 12,
    mettre en évidence les idées générales qui existent à l’état
    latent dans les divers éléments du mot à analyser. Ainsi, nous
    savons que tout adjectif contient en lui-même l’idée générale
    „\emph{qual}“; on peut donc écrire: \textup{grand} =
    \textup{grand(qual)}, comme nous avons écrit: \textup{cat} =
    \textup{cat(animal)}, ou \textup{apple} =
    \textup{apple(fruit)}. Par suite:

    {\centering
      \textup{grand'eur} = \textup{grand(qual)'ité},\par
    }

    \noindent
    puisque les suffixes \textup{eur} et \textup{ité} sont équivalents; enfin, en
    appliquant la loi de renversement:

    {\centering
      \begin{tabular}[t]{r@{ = }l}
        \textup{grandeur} & „\emph{qualité grand}“\\
                          & „\emph{qualité} [de l’espèce] \emph{grand}“
      \end{tabular}\par
    }

    2.	Faire la synthèse de l’idée: „\emph{qualité grand}“.
    
    Par la loi de renversement on a:
    
    {\centering
      „\emph{qualité grand}“ = \textup{grand'qual'ité}.\par
    }
    
    Or, le mot \textup{grandqualité} contient un pléonasme inutile
    puisque l’idée „\emph{qual}“ existe déjà dans l’adjectif
    \textup{grand}; ce mot se réduit donc à \textup{grand'ité},
    c’est-à-dire à \textup{grand'eur}, en remplaçant le suffixe
    \textup{ité} par son synonyme \textup{eur}.

    3.	Faire l’analyse du mot: \textup{écriture}.

    Le mot \textup{écriture} se compose du verbe \textup{écri} et du
    substantif \textup{ture}. D’après la loi de renversement,
    \textup{écri'ture} = „\emph{ture écri}“, c’est-à-dire: „\emph{la
      chose écri}“, puisque le suffixe \textup{ture} exprime l’idée
    substantive générale de „chose“. Or, \textup{écri} =
    \textup{écri're}, puisque l’idée verbale „\emph{re}“ est déjà
    contenue dans le verbe \textup{écri}. On peut donc dire que
    \textup{écriture} signifie „\emph{la chose écrire}“.

    Si l’on vent pousser plus loin l’analyse, il faut mettre en
    évidence l’idée verbale générale „\emph{ag}“ (ou „\emph{ac}“), qui
    existe dans tout verbe, c’est-à-dire que: \textup{écri} =
    \textup{écri(ag)}, tout comme \textup{grand} =
    \textup{grand(qual)}, ou \textup{apple} =
    \textup{apple(fruit)}. D’autre part, le suffixe \textup{ture} est synonyme
    du suffixe \textup{tion}, on a donc:

    {\centering
      \textup{écri'ture} = \textup{écri(ag)'tion} =
      \textup{écri(ac)'tion},\par
    }

    \noindent
    et enfin, en appliquant la loi de renversement:
    
    \noindent
    {\centering
      \textup{écriture} = „\emph{action écri}“ ou „\emph{action écrire}“
      = „\emph{action} [de l’espèce] \emph{écrire}“.\par
    }

    4. Faire la synthèse de l’idée: „\emph{action écrire}“: Le mot
    \textup{écrire} contient un pléonasme, car si l’on compare le
    verbe \textup{écri're} au verbe \textup{pâl'ir}, par exemple, on
    voit que dans ce dernier verbe l’élément \textup{pâl} est un
    adjectif, tandis que dans le premier, l’élément \textup{écri} est
    lui-même un verbe; l’idée verbale exprimée par le suffixe
    \textup{re} existe donc déjà dans l’élément \textup{écri},
    c’est-à-dire que \textup{écrire} est réductible logiquement à
    \textup{écri}, lorsqu’on ne se préoccupe pas du temps de la
    conjugaison. La loi de renversement donne ensuite la synthèse:

    {\centering
      „\emph{action écri}“ = \textup{écri'action} =
      \textup{écri'ac’tion}.\par
    }

    Or, l’idée „\emph{ac}“ (ou „\emph{ag}“) est encore l’idée verbale,
    laquelle est déjà contenue dans le verbe \textup{écri},
    c’est-à-dire que \textup{écri'ac} est une forme pléonasmatique
    réductible à \textup{écri}, de sorte que \textup{écri'ac'tion} est
    réductible à \textup{écri'tion}, ou encore \textup{écri'ture},
    puisque les suffixes \textup{tion} et \textup{ture} sont
    synonymes.

    5. Faire l’analyse du mot: \textup{humanité}.

    Pour faire cette analyse, il est bon de considérer la série
    \textup{homme}, \textup{hum'ain}, \textup{hum'an'ité}. On voit
    alors que dans le mot \textup{humanité} l’élément \textup{hum}
    n’est qu’une altération du substantif \textup{homme}, et l’élément
    \textup{an} une altération du suffixe adjectif \textup{ain} (ou
    vice-versa).

    Or, le suffixe \textup{ain}, exprimant l’idée adjective générale,
    est synonyme du mot-racine \textup{qual}; on a donc,
    \textup{hum'an'ité} = \textup{hom'qual'ité}, c’est-à-dire, par la
    loi de renversement: „\emph{qualité homme}“.

    6. Faire l’analyse du mot \textup{maniement}.

    Si l’on considère la série: \textup{main}, \textup{mani'er},
    \textup{mani'e'ment}, on voit que dans le mot
    \textup{mani'e'ment}, l’élément \textup{mani} n’est qu’une
    altération du substantif \textup{main}, et l’élément \textup{e}
    est une altération du suffixe verbal \textup{er}.
  }
  
  \EnglishPage{\small
    %
    To push the analysis deeper, it is necessary, according to rule
    12, to bring out the general ideas that exist in a latent state in
    the different elements of the word to be analyzed. Thus, we know
    that every adjective contains in itself the general idea
    ``\emph{qual}''; we can therefore write: \textup{grand} =
    \textup{grand(qual)} just as we have written \textup{cat} =
    \textup{cat(animal)}, or \textup{apple} =
    \textup{apple(fruit)}. Consequently:

    {\centering
      \textup{grand'eur} = \textup{grand(qual)'ité},\par
    }

    \noindent
    since the suffixes \textup{eur} and \textup{ité} are
    equivalent. Finally, applying the law of reversal:

    {\centering
      \begin{tabular}[t]{r@{ = }l}
        \textup{grandeur} & ``\emph{qualité grand}''\\
                          & ``\emph{qualité} [of the type] \emph{grand}''
      \end{tabular}\par
    }

    2.	Synthesize the idea ``\emph{qualité grand}''.

    By the law of reversal, we have:

    {\centering
      ``\emph{qualité grand}'' = \textup{grand'qual'ité}.\par
    }
    
   Now the word \textup{grandqualité} contains an unnecessary
   pleonasm since the idea ``\emph{qual}'' already exists in the
   adjective \textup{grand}; this word thus reduces to
   \textup{grand'ité}, that is \textup{grand'eur} on replacing the
   suffix \textup{ité} with its synonym \textup{eur}.

   3. Analyze the word: \textup{écriture}.

   The word \textup{écriture} is composed of the verb
   \textup{écri} and the noun \textup{ture}. According to the law of
   reversal, \textup{écri'ture} = ``\emph{ture écri}'', that is
   ``\emph{the thing write}'', since the suffix \textup{ture} expresses
   the general nominal idea ``thing''. And \textup{écri} =
   \textup{écri're} `to write' since the verbal idea ``\emph{re}''
   is already contained in the verb \textup{écri}. We can thus say
   that \textup{écriture} means ``\emph{the thing [---] to
     write}''.

   If we want to push the analysis further, it is necessary to bring
   out the general verbal idea ``\emph{ag}'' (or ``\emph{ac}'') that
   exists in every verb --- that is, \textup{écri} =
   \textup{écri(ag)}, just as \textup{grand} =
    \textup{grand(qual)}, or \textup{apple} =
    \textup{apple(fruit)}. Additionally, the suffix \textup{ture} is a
    synonym of \textup{tion}, and we thus have:

    {\centering
      \textup{écri'ture} = \textup{écri(ag)'tion} =
      \textup{écri(ac)'tion},\par
    }

    \noindent
    and finally, applying the law of reversal:
    
    \noindent
    {\centering
      \textup{écriture} = ``\emph{action écri}'' or ``\emph{action écrire}''
      = ``\emph{action} [of the type] \emph{écrire}''.\par
    }

    4. Synthesize the idea: ``\emph{action écrire}''. The word
    \textup{écrire} contians a pleonasm, for if we compare the verb
    \textup{écri're} with the verb \textup{p\^al'ir} `to turn pale'
    for example, we see that in this last verb the element
    \textup{p\^al} `pale' is an adjective, while in the first, the
    element \textup{écri} is itself a verb.  The verbal idea
    expressed by the suffix \textup{re} thus already exists in the
    element \textup{écri}, that is \textup{écrire} is logically
    reducible to \textup{écri}, since we are not concerned with the
    tense of the conjugation. The law of reversal thus gives the
    synthesis:

    {\centering
      ``\emph{action écri}'' = \textup{écri'action} =
      \textup{écri'ac’tion}.\par
    }

    Now the idea ``\emph{ac}'' (or ``\emph{ag}'') is again the verbal
    idea which is already contained in the verb \textup{écri}, that
    is \textup{écri'ac} is a pleonastic form reducible to
    \textup{écri}, so that \textup{écri'ac'tion} is reducible to
    \textup{écri'tion}, or rather \textup{écri'ture}, since the
    suffixes \textup{tion} and \textup{ture} are synonyms.

    5. Analyze the word: \textup{humanité}.

    To carry out this analysis, it is good to consider the series
    \textup{homme}, \textup{hum'ain}, \textup{hum'an'ité}. We see
    then that in the word \textup{humanité} the element \textup{hum}
    is merely a modification of the noun \textup{homme}, and the
    element \textup{an} is a modification of the adjective suffix
    \textup{ain} (or vice versa).

    Now the suffix \textup{ain}, expressing the general adjectival
    idea, is synonymous with the root word \textup{qual}.  We thus
    have \textup{hum'an'ité} = \textup{hom'qual'ité}, that is, by
    the law of reversal ``\emph{quality man}''.

    6. Analyze the word: \textup{maniement} `handling'

    If we consider the series \textup{main} `hand', \textup{mani'er}
    `handle', \textup{mani'e'ment}, we see that in the word
    \textup{mani'e'ment} the element \textup{mani} is just a
    modification of the noun \textup{main}, and the element \textup{e}
    is a modification of verbal suffix \textup{er}.
    }

  \FrenchPage{\small    
    Or, le suffixe \textup{er}, exprimant l’idée verbale générale, est
    synonyme du mot-radical \textup{ag} (agir); on a donc:
    \textup{mani'e(r)'ment}= \textup{main'ag'ment}, ou mieux:
    \textup{main'ag'tion}, à cause de la synonymie des suffixes
    \textup{ment} et \textup{tion}. Enfin, par la loi de renversement:
    \textup{main’ag'tion} = „\emph{agtion main}“, d’où:
    \textup{mani'e'ment} = „\emph{action} [de l’espèce caractérisée
    par] \emph{la main}“.

    7. Analyser les mots: \textup{moderniser}, \textup{béatifier},
    \textup{agrandir}, \textup{épurer}.

    Les suffixes \textup{is} et \textup{ifi} sont des suffixes verbaux
    synonymes du mot-racine \textup{rend}, c’est-à-dire qu’on peut
    écrire \textup{is'er} = \textup{ifi'er} = \textup{rend're}. On a
    donc, par la loi de renversement:

    {\centering
      \begin{tabular}[t]{l}
      \textup{modern'iser} = „\emph{iser modern}“ — „\emph{rendre moderne}“,\\
      \textup{béat’ifier} = „\emph{ifier béat}“ = „\emph{rendre béat}“.
      \end{tabular}
      \par}


    D’autre part, les préfixes \textup{a} (dans \textup{a'grand'ir})
    et \textup{é} (dans \textup{é'pur'er}) sont aussi synonymes des
    suffixes \textup{is} et \textup{ifi}; \textup{agrandir} signifie
    donc: „\emph{rendre grand}“, et \textup{épurer} = [„]\emph{rendre
      pur}“; on a donc \textup{épurer} = \textup{purifier}.

    \emph{Remarque}. Le mot \textup{qual'ifier} signifie „\emph{rendre
      qual}“, et comme le mot \textup{qual} sert de chef de file à
    tous les adjectifs, le mot \textup{qualifier} servira de chef de
    file à tous les verbes tels que \textup{béatifier},
    \textup{purifier}, \textup{moderniser}, etc.

    8. Analyser les mots: \textup{se moderniser,} \textup{s'agrandir}.

    D’après ce qui précède, \textup{se moderniser} signifie „\emph{se
      rendre modern}e“, c’est-à-dire „\emph{devenir moderne}“; de
    même: \textup{s'agrandir} = „\emph{se rendre grand}“ =
    „\emph{devenir grand}“.

    \emph{Remarque}. Les mots tels que \textup{pâlir} (devenir pâle),
    \textup{blanchir} (rendre, ou devenir blanc), etc., ne satisfont
    pas au principe de nécessité, autrement dit ces mots ne sont que
    partiellement motivés. Pour les rendre complètement motivés, il
    faudrait dire: \textup{blanchifier} pour „\emph{rendre blanc}“, et
    \textup{se blanchifier} pour „\emph{devenir blanc}“.

    9. Analyser le mot: \textup{international}.

    Tout mot composé est divisible en deux parties, mais non
    arbitrairement. Ainsi le mot \textup{international} = \textup{
      internation’al} (et non pas \textup{inter’national}). Le suffixe
    \textup{al} exprime l’idée adjective, idée que l’on peut traduire ici par
    l’expression „\emph{qui est}“. On a donc par la loi de renversement:

    {\centering
      \textup{internation’al} = „\emph{al internation}“ = „\emph{qui
        est internation}“.
      \par}
    \noindent
    Reste à analyser le mot \textup{internation}; ce mot se compose de
    la préposition \textup{inter}, ou \textup{entre}, et du substantif
    \textup{nation}, mais il faut remarquer que ce substantif est le
    \emph{complément} de la préposition \textup{entre}, c’est-à-dire
    que l’on se trouve dans le cas particulier où la loi de
    renversement n’est pas logiquement applicable (cas
    \textup{inter'règne}, n° 10, p. 11); on a donc par simple
    séparation des éléments: \textup{internation} = „\emph{entre
      nations}“. Donc en résumé: \textup{international} = „\emph{qui
      est entre nations}“.

    10. Analyser les mots: \textup{qualité}, \textup{propriété}.

    Le suffixe \textup{ité} est synonyme de l’idée substantive
    générale: „\emph{ce (qui est)}“. On a donc:

    {\centering
      \begin{tabular}[t]{l}
        \textup{qualité} = „\emph{ité qual}“ — „\emph{ce qui est
        qual}“\\
        \textup{propriété} = „\emph{ité propre}“ = „\emph{ce qui est propre [à]}“.
      \end{tabular}
      \par}


    On ne peut pas pousser l’analyse plus loin, car les éléments
    \textup{ité}, \textup{qual}, \textup{propre}, sont tous des
    éléments fondamentaux, c’est-à-dire des éléments simples qui ne
    contiennent pas en eux-mêmes d’idées plus générales. On peut
    seulement remarquer que l’idée „\emph{qual}“, ou „\emph{propre}“,
    est l’idée générale qu’exprime tout adjectif; on peut donc définir
    les mots \textup{qualité} et \textup{propriété} comme indiquant
    „\emph{ce qu'exprime l'adjectif}“, ou encore (puisque le suffixe
    \textup{ité} représente l’idée substantive), on peut dire que les
    mots \textup{qual'ité} et \textup{propri'été} sont les
    „\emph{adjectivo-substantifs}“ types, servant de chefs de file à
    tous les adjectivo-substantifs particuliers: \textup{util'ité},
    \textup{vér’ité}, \textup{grand'eur}, etc.
  }
  
  \EnglishPage{\small
    Now the suffix \textup{er}, expressing the general verbal idea, is
    a synonym of the root word \textup{ag} (agir); we thus have:
    \textup{mani'e(r)'ment}= \textup{main'ag'ment}, or better:
    \textup{main'ag'tion}, because of the synonymy of the suffixes
    \textup{ment} and \textup{tion}. Finally, by the law of reversal:
    \textup{main’ag'tion} = ``\emph{agtion main}'', from which:
    \textup{mani'e'ment} = ``\emph{action} [of the type characterized
    by] \emph{the hand}''.\vspace*{1ex}

    7. Analyze the words: \textup{moderniser}, \textup{béatifier},
    \textup{agrandir}, \textup{épurer}.

    The suffixes \textup{is} and \textup{ifi} are verbal suffixes
    synonymous with the root word \textup{rend} `make'; that is, we
    can write \textup{is'er} = \textup{ifi'er} = \textup{rend're} `to
    make'.  We thus have, by the law of reversal:

    {\centering
      \begin{tabular}[t]{l}
      \textup{modern'iser} = ``\emph{iser modern}'' — ``\emph{to make modern}'',\\
      \textup{béat’ifier} = ``\emph{ifier béat}'' = ``\emph{to make blessed}''.
      \end{tabular}
      \par}

    Furthermore, the prefixes \textup{a} (in \textup{a'grand'ir} `to
    enlarge') and \textup{é} (in \textup{é'pur'er} `to purify')
    are also synonymous with the suffixes \textup{is} and
    \textup{ifi}; \textup{agrandir} thus means ``\emph{to make
      large}'' and \textup{épurer} = ``\emph{to make pure}''; we
    thus have \textup{épurer} = \textup{purifier}.

    \emph{Remark}. The word \textup{qual'ifier} means ``\emph{to make
      qual}'', and since the word \textup{qual} serves as the leading form
    for all adjectives, the word \textup{qualifier} serves as the
    leading form for all of the verbs like \textup{béatifier},
    \textup{purifier}, \textup{moderniser}, etc.\vspace*{1ex}

    8. Analyze the words: \textup{se moderniser,} \textup{s'agrandir}.

    From the preceding, \textup{se moderniser} means ``\emph{to make
      self modern}'', that is ``\emph{to become modern}''; similarly
    \textup{s'agrandir} = ``\emph{to make self large}'' = ``\emph{to
      become large}''.

    \emph{Remark}. Words like \textup{p\^alir} (to become pale),
    \textup{blanchir} (to make or to become white), etc. do not
    satisfy the principle of necessity; in other words, these words
    are only partially motivated. To render them completely motivated,
    it would be necessary to say \textup{blanchifier} for ``\emph{to
      make white}'' and \textup{se blanchifier} for ``\emph{to become
      white}''.\vspace*{1ex}

    9. Analyze the word: \textup{international}.

    Every compound word is divisible into two parts, but not
    arbitrarily. Thus the word \textup{international} =
    \textup{internation'al} (and not \textup{inter'national}). The
    suffix \textup{al} expresses the adjectival idea, an idea which we
    can translate here by the expression ``\emph{which is}''. We thus
    have by the law of reversal:

    {\centering
      \textup{internation’al} = ``\emph{al internation}'' = ``\emph{which
        is internation}''.
      \par}
    \noindent
    It remains to analyze the word \textup{internation}.  This word is
    composed of the preposition \textup{inter}, or \textup{entre}
    `between', and the noun \textup{nation}, but it is necessary to
    note that the noun is the \emph{complement} of the preposition
    \textup{entre}: that is, we find ourselves in the specific case
    where the law of reversal is not logically applicable (case
    \textup{inter'règne}, number 10, p. 11).  We thus have by simple
    separation of the elements: \textup{inter'nation} =
    ``\emph{between nations}''. Thus in summary:
    \textup{international} = ``\emph{that which is between
      nations}''.     
    \vspace*{1ex}
    
    10. Analyze the words: \textup{qualité}, \textup{propriété}.

    The suffix \textup{ité} is a synonym of the general nominal idea
    ``\emph{that (which is)}''.  We thus have:

    {\centering
      \begin{tabular}[t]{l}
        \textup{qualité} = ``\emph{ité qual}'' — ``\emph{that which is
        qual}''\\
        \textup{propriété} = ``\emph{ité propre}'' = ``\emph{that
        which is proper  [to]}''.
      \end{tabular}
      \par}

    We cannot push this analysis further, since the elements
    \textup{ité}, \textup{qual}, \textup{propre} are all basic
    elements, that is, simple elements that do not contain more gneral
    ideas in themselves. We can simply note that the idea
    ``\emph{qual}'' or ``\emph{propre}'' is the general idea that
    every adjective expresses; or again (since the suffix
    \textup{ité} represents the nominal idea), we can say that the
    words \textup{qual'ité} and \textup{propri'été} are the
    \emph{adjectivo-noun} type, serving as leading forms for all of
    the specific adjectivo-nouns: \textup{util'ité},
    \textup{vér'ité}, \textup{grand'eur}, etc.}

  \FrenchPage{{\small
    
    11. Analyser les mots: \textup{action}, \textup{état}.

    Les suffixes \textup{tion}, \textup{at}, sont synonymes de l’idée
    substantive générale: „\emph{(ce qui est).}[“] On a donc
    \textup{ac'tion} = „\emph{tion ac}“ = „\emph{ce qui est ac}“, ou
    „\emph{ag}“, c’est-à-dire: „\emph{ce qui est agir}“. De même,
    \textup{ét'at} = „\emph{at ét}“ = „\emph{ce qui est ét}“,
    „\emph{ce qui est être}“ (en ayant soin de donner ici au verbe
    \emph{être} le sens „\emph{étre dans un état}“, et non le sens
    d'„\emph{exister}“).

    On ne peut pousser l’analyse plus loin, puisque les éléments
    \textup{tion}, \textup{at}, \textup{ac} (ou \textup{ag}),
    \textup{ét}, sont déjà tous des éléments fondamentaux. On peut
    seulement remarquer que l’idée „\emph{ag}“, ou „\emph{ét}“, est
    l’idée générale exprimée par un verbe, on peut donc définir les
    mots \textup{action} et \textup{état} comme indiquant „\emph{ce
      qu’exprime le verbe}“\footnote{C’est précisément la définition
      du mot \textup{action}, donnée par Larousse. Seulement Larousse
      fait d’autre part un cercle vicieux en définissant à son tour le
      mot \textup{verbe} par les mots composés \textup{action} et
      \textup{état}. On ne doit pas dire: „le verbe exprime l’action”,
      mais: „le verbe exprime l’idée „\emph{ag}“, ou si l’on veut:
      l’idée „\textup{agir}“, puisque \textup{agir} est réductible à
      \textup{ag}.}, ou encore (puisque les suffixes \textup{tion},
    \textup{at}, représentent l’idée substantive), on peut dire que
    les mots \textup{ac'tion} et \textup{ét'at} sont les
    „\emph{verbo-substantifs}“ types, servant de chefs de file à tous
    les verbo-substantifs particuliers: \textup{abdic'ation},
    \textup{écri'ture}, \textup{abond'ance}, etc.

    On voit que les quatre mots \emph{qual'ité}, \emph{propri'été},
    \emph{ac'tion}, \emph{ét'at} sont très généraux, puisque chacun de
    ces mots est composé de deux mots fondamentaux.}

    {\large
      \begin{center}
        § 4. — \textsf{Symétrie du verbe et de l’adjectif\\
          par rapport au substantif.}
      \end{center}
    }

    Les exercices précédents mettent en évidence une symétrie
    remarquable de l’adjectif et du verbe vis-à-vis du substantif. Non
    seulement les mots tels que \textup{qual'ité} et \textup{ac'tion},
    \textup{grand'eur} et \textup{écri'ture}, etc., ont des structures
    symétriques, mais aussi les mots tels que \textup{hum'an'ité} et
    \textup{mani'e'ment}, qui représentent des substantifs
    \emph{abstraits} tirés de substantifs \emph{concrets}, soit par
    adjectivation, soit par verbification; les séries:

     {\centering
      \begin{tabular}[t]{lll}
        \textup{homme}, &\textup{hum'ain} &\textup{hum’an’ité,}\\
        \textup{main}, &\textup{mani'er}, &\textup{mani'e'ment},
      \end{tabular}
      \par}\vspace*{1ex}
    
    \noindent
    sont tout-à-fait symétriques. On peut citer même les séries
    doubles:

    {\centering
      \begin{tabular}[t]{lllll}
        \textup{homme},& \textup{hum'ain},& \textup{hum'an’ité},&
                                                                  \textup{hum'an’it'aire},&
                                                                                            \textup{hum'an'lt'ar'isme};\\        
        \textup{règle}, &\textup{règl'er}, &\textup{règl'e'ment},&
                                                                   \textup{règl'e'ment'er},&
                                                                                             \textup{règl'e'ment'a'tion}.  
      \end{tabular}
      \par}\vspace*{1ex}


    Chacune de ces séries doubles est équivalente à deux séries\largerpage
    simples: en effet, de l’idée concrète „\emph{homme}“ on tire
    d’abord par adjectivation (\emph{hum'ain}) l’idée abstraite
    d'„\emph{humani\-té}“, dans le sens de „\emph{qualité d'homme}“ ou
    „\emph{homme en général}“; mais si l’on prend ce même mot
    \textup{humanité} dans le sens concret de „\emph{ensemble des
      hommes}“, „\emph{collectivité humaine}“ (groupe concret
    d’individus), alors de cette idée concrète, on peut tirer de
    nouveau une idée abstraite, „\emph{humanitarisme}“, par
    l’intermédiaire d’une seconde adjectivation
    („\emph{humanit'aire}“).

    De même de l’idée concrète „\emph{une règle}“ on tire par
    verbification (\emph{règl'er}) l’idée abstraite
    „\emph{règlement}“, dans le sens „\emph{action de régler}“, ou
    „\emph{la règle en général}“; prenant ensuite le mot
    \textup{règlement} dans le sens concret de „\emph{ensemble de
      règles}“, „\emph{groupe concret de règles}“, on peut de cette
    idée concrète tirer de nouveau une idée abstraite
    „\emph{réglementation}“, par l’intermediaire d’une seconde
    verbification („\emph{réglement'er}“).

    En résumé, on peut dire que le Substantif forme la substance, le
    corps du langage, tandis que le Verbe et l’Adjectif sont les deux
    membres symétriques qui permettent le fonctionnement de ce corps
    en y introduisant l’action et la qualité.
  }
  
  \EnglishPage{ {\small
     11. Analyze the words: \textup{action}, \textup{état}.

      The suffixes \textup{tion}, \textup{at} are synonyms of the
      general nominal idea: ``\emph{(that which is)}''. We thus have
      \textup{ac'tion} = ``\emph{tion ac}'' = ``\emph{that which is
        ac}'' or ``\emph{ag}'', that is ``\emph{that which is to
        act}''.  Similarly, \textup{ét'at} = ``\emph{at ét}'' =
      ``\emph{that which is ét}'', ``\emph{that which is to be}''
      (being careful to give the verb `to be' here the sense ``\emph{to be
        in a state}'' and not the sense of ``\emph{to exist}''.

      We cannot push the analysis further, since the elements
      \textup{tion}, \textup{at}, \textup{ac} (or \textup{ag}),
      \textup{ét} are all already basic elements. We can only note
      that the idea ``\emph{ag}'' or ``\emph{ét}'' is the general idea
      expressed by a verb.  We can thus define the words
      \textup{action} and \textup{état} as indicating ``\emph{that
        which the verb expresses}''\footnote{This is precisely the
        definition of the word \textup{action} given by the
        \textsl{Larousse}. But \textsl{Larousse} on the other hand
        gets into a vicious circle by defining in turn the word
        \textup{verb} by the compound words \textup{action} and
        \textup{state}. We must not say ``the verb expresses the
        action'' but rather ``the verb expresses the idea \emph{ag}''
        or if you like, the idea ``\textup{agir}'' since \textup{agir}
        is reducible to \textup{ag}.}, or again (since the suffixes
      \textup{tion}, \textup{at} represent the nominal idea) we can
      say that the words \textup{ac'tion} and \textup{ét'at} are of
      the ``\emph{verbo-noun}'' type, serving as leading forms for
      all of the specific verbo-nouns: \textup{abdic'ation},
      \textup{écri'ture}, \textup{abond'ance}, etc.


      We see that the four words \emph{qual'ité}, \emph{propri'été},
      \emph{ac'tion}, \emph{ét'at} are very general, since each of
      these words is composed of two basic words.
      }

      
      \begin{center}
        § 4. — \textsf{\large Symmetry of the verb and the adjective\\
          in relation to the noun}
      \end{center}

      The preceding exercises demonstrate a remarkable symmetry of the
      adjective and the verb with regard to the noun.  Not only have
      words like \textup{qual'ité} and \textup{ac'tion},
      \textup{grand'eur} and \textup{écri'ture}  symmetrical
      structures, but also words like \textup{hum'an'ité} and
      \textup{mani'e'ment}, which represent \emph{abstract} nouns
      derived from \emph{concrete} nouns either by adjectivalization
      or verbalization. The series:

    {\centering
      \begin{tabular}[t]{lll}
        \textup{homme}, &\textup{hum'ain} &\textup{hum’an’ité,}\\
        \textup{main}, &\textup{mani'er}, &\textup{mani'e'ment},
      \end{tabular}
      \par}\vspace*{1ex}

    \noindent
    are completely symmetrical. We can even cite the double series:\vspace*{1ex}

    {\centering
      \begin{tabular}[t]{lllll}
        \textup{homme},& \textup{hum'ain},& \textup{hum'an’ité},&
                                                                  \textup{hum'an’it'aire},&
                                                                                            \textup{hum'an'lt'ar'isme};\\        
        \textup{règle}, &\textup{règl'er}, &\textup{règl'e'ment},&
                                                                   \textup{règl'e'ment'er},&
                                                                                             \textup{règl'e'ment'a'tion}.  
      \end{tabular}
      \par}\vspace*{1ex}

    Each of these double series is equivalent to two simple series:
    indeed, from the concrete idea ``\emph{homme}'' we first derive by
    adjectivalization (\emph{hum'ain}) the abstract idea of
    ``\emph{humanité}'', in the sense ``\emph{quality of man}'' or
    ``\emph{man in general}''; but if we take the same word
    \textup{humanité} in the concrete sense of ``\emph{the set of
      men}'' or ``\emph{the human collectivity}'' (concrete group of
    individuals), then from that concrete idea, we can derive a new
    abstract idea, ``\emph{humanitarisme}'' by means of a second
    adjectivalization (``\emph{humanitaire}'').

    Similarly, from the concrete idea ``\emph{une règle}'' `a rule'
    we derive by verbalization (\emph{règl'er} `to rule') the
    abstract idea ``\emph{règlement}'' `ruling, regulation' in the
    sense ``\emph{action of ruling}'' or ``\emph{rules in general}'';
    taking next the word \textup{règlement} in the concrete sense of
    ``\emph{set of rules}'' or ``\emph{concrete group of rules}'', we
    can now derive from that concrete idea an abstract idea
    ``\emph{règlementation}'' `regulations' by means of a second
    verbalization (\emph{règlement'er} `to regulate').\largerpage[2]

    In summary, we can say that the Noun forms the substance, the body
    of language, while the Verb and the Adjective are the two
    symmetric limbs that allow this body to function by introducing
    action and quality.
    }

  \FrenchPage{\small
    \emph{Digression}. — Les remarques précédentes nous conduisent
    naturellement à l’examen des rapports qui existent entre la pensée
    et le langage. Mr. Couturat, dans un article sur la „Structure
    logique du langage“, dont je ferai la critique au chapitre II, a
    abordé la même question, et il dit avec
    raison\footnote{\emph{Revue de métaphysique et de morale}, n° 1,
      1912, Paris.}:

    \begin{quotation}
      „De toutes les manifestations de la pensée, le langage est la
      plus universelle et, malgré tout, la plus adéquate. Si imparfait
      qu’il soit comme mode d’expression, il est encore le plus
      commode et le plus complet. Il est impossible que l’esprit
      humain, qui le façonne et le transforme sans cesse pour son
      usage, n’y imprime pas la trace de ses tendances et de ses
      fonctions, et que les formes du langage ne reflètent pas, dans
      une certaine mesure, les formes de la pensée.“
    \end{quotation}


    J’exprimerai la même idée autrement, en disant que les catégories
    „grammaticales“ correspon\-dent aux catégories „logiques“, et comme
    les premières sont les mêmes pour toutes les langues, le langage
    serait ainsi l’expression d’une certaine conception philosophique
    du monde, conception populaire si l’on veut, mais qui doit avoir
    des racines très profondes, parce qu’elle émane pour ainsi dire du
    dedans de l’évolution naturelle.

    Il est donc très important de fixer le nombre des catégories ou
    classes fondamentales de mots.

    Mr. Couturat n’en reconnaît que deux fondamentales: les
    \emph{verbes} et les \emph{noms}\footnote{Voir la critique de ce
      système au chapitre II.}; le professeur Ostwald va plus loin:
    il ramène tous les concepts au concept de „\emph{chose}“ qu’il
    considère comme le plus général\footnote{Voir \emph{Esquisses
        d'une philosophie des sciences}, par W. Ostwald, p. 62 de la
      traduction française. Félix Alcan, Paris, 1911.}, ce qui revient
    à subordonner l'idée verbale et l’idée adjective à l’idée
    substantive, puisque l’idée de „chose“ est précisément l’idée
    substantive. Or, cette subordination est impossible, car „agir“
    par exemple n’est pas une chose.

    Il existe en réalité \emph{trois} classes fondamentales de mots
    (les substantifs, les adjectifs et les verbes) correspondant aux
    trois concepts les plus généraux de „\emph{chose}“, de
    „\emph{qualité}“ et d'„\emph{action}“, exprimés plus exactement par
    les mots fondamentaux \textup{chose}, \textup{qual} et
    \textup{ag}. Ces trois concepts sont indépendants les uns des
    autres, comme le sont par exemple en physique les trois concepts
    de temps, de force et d’espace; leur nombre est par conséquent
    irréductible.

    Cette division des concepts en trois catégories correspond
    évidemment à la manière dont nous concevons la réalité. Nous
    distingons en effet dans tout phénomène: 1° la \emph{chose en soi}
    qui nous semble être le support du phénomène; 2° le phénomène
    \emph{objectif}; 3° le phénomène \emph{subjectif}.

    Considérons par exemple un phénomène lumineux: nous expliquons un
    tel phénomène par des vibrations très rapides d’atomes;
    vibrations qui, se propageant à travers l’éther, viennent frapper
    notre œil et y produire des sensations de lumière, de couleur,
    etc. L’„atome“ qui vibre, c’est la chose en soi, c’est le concept
    évoqué par le \emph{substantif} „atome“; cet atome „vibre“, voilà
    le phénomène objectif, c’est-à-dire le phénomène mécanique
    (quantitatif), le seul que considère la science, et cette partie
    du phénomène, qui est un „agir“ est exprimée par le \emph{verbe}
    „vibrer“; enfin la vibration de cet atome produit sur notre œil
    l’impression subjective de couleur, par exemple de couleur
    „jaune“, et cette partie subjective (qualitative) du phénomène est
    traduite par \emph{l'adjectif} „jaune“. Telle semble être en gros
    la correspondance qui existe entre les classes de mots et les
    catégories de notre pensée.

    On peut dire qu’à certains égards la psychologie est la science de
    l'\emph{adjectif}, car tout ce qui est subjectif est qualitatif,
    et le „qualitatif“ est exprimé par
  }
  
  \EnglishPage{\small
    %
    \emph{Digression}. --- The preceding remarks lead us naturally to
    examine the relations that exist between thought and
    language. Mr. Coutourat, in an article on the ``Logical structure
    of language'', which I will criticize in chapter II, has addressed
    the same question, and he says correctly\footnote{\emph{Revue de
        métaphysique et de morale}, n° 1, 1912, Paris.}:

    \begin{quotation}
      ``Of all manifestations of thought, language is the most
      universal, and despite all, the most adequate. As imperfect as
      it may be as a mode of expression, it is still the easiest and
      most complete. It is impossible that the human spirit, which
      shapes and transforms it ceaselessly through usage, should not
      imprint on it the trail of its tendencies and functions, and
      that the forms of language should not reflect, to a certain
      extent, the forms of thought.''
    \end{quotation}

    I will express the same idea otherwise, in saying that
    ``grammatical'' categories correspond to ``logical'' categories,
    and since the former are the same for all languages, language must
    therefore be the expression of a certain philosophical conception
    of the world, the popular conception if you will, but one that
    must have very deep roots, because it emanates so to speak from
    within natural evolution.

    It is thus very important to establish the number of basic
    categories or classes of words.

    Mr. Coutourat recognizes only two basic ones: \emph{verbs} and
    \emph{nouns}\footnote{See the criticism of this system in chapter
      II. {[not included in the present edition]}}. Professor Ostwald
      goes further: he brings all concepts down to the concept of
      ``\emph{thing}'' which he considers the most
      general\footnote{See \emph{Esquisses d'une philosophie des
          sciences}, by W. Ostwald, p. 62 in the French
        translation. Félix Alcan, Paris, 1911.}, which comes down to
      subordinating the verbal idea and the adjectival idea to the
      nominal idea, since the idea of ``\emph{thing}'' is precisely
      the nominal idea. Now this subordination is impossible, since
      ``to act'' for example is not a thing.

    In reality, there exist \emph{three} basic classes of words
    (nouns, adjectives and verbs) corresponding to the three most
    general concepts of ``\emph{thing}'', ``\emph{quality}'' and
    ``\emph{action}'', expressed more exactly by the basic words
    \textup{thing}, \textup{qual} and \textup{ag}. These three
    concepts are independent of one another, as are for example in
    physics the three concepts of time, force and space: their number
    is thus irreducible.

    This division of concepts into three categories evidently
    corresponds to the way in which we conceive reality. We
    distinguish, indeed, in every phenomenon 1. the \emph{thing in
      itself} that seems to us to be the support of the phenomenon;
    2. the \emph{objective} phenomenon; 3. the \emph{subjective}
    phenomenon.

    Let us consider for example a phenomenon of light: we explain such
    a phenomenon by the very rapid vibration of atoms, vibrations that
    propagate across the ether, coming to strike our eye and produce
    there the sensations of light, of color, etc. The ``atom'' which
    vibrates is the thing in itself, it is the concept evoked by the
    \emph{noun} ``atom''; this atom ``vibrates'', and there is the
    objective phenomenon, that is the mechanical (quantitative)
    phenomenon, the only thing that science considers, and this part
    of the phenomenon, which is an ``acting'', is expressed by the
    \emph{verb} ``to vibrate''; finally the vibration of this atom
    produces on our eye the subjective impression of color, for
    example the color ``yellow'', and this subjective (qualitative)
    part of the phenomenon is translated by the \emph{adjective}
    ``yellow''. Such seems to be roughly the correspondence that exists
    between the classes of words and the categories of our thought.

    We can say that in certain respects psychology is the science of
    the \emph{adjective}, since everything that is subjective is
    qualitative, and the ``qualitative'' is expressed by
    }

    \FrenchPage{\noindent
      {\small
      %
        l’adjectif. Au contraire, les sciences physiques seraient la
        science du \emph{verbe} et du \emph{substantif}, car ces
        sciences ne considèrent que le phénomène objectif, lequel est
        essentiellement quantitatif et se réduit en dernière analyse à
        des masses en mouvement, c’est-à-dire à des grammes, des
        centimètres et des secondes; or, la masse est une „chose“, un
        substantif, et son mouvement est un „agir“, un verbe.}\\[1ex]

      \emph{Conclusion}. Lorsqu’on se place au point de vue
      international, on peut dire que le type le plus général de
      structure des mots dans les langues naturelles est le type
      \emph{par soudure}, c’est-à-dire que les mots composés sont
      formés par soudure de plusieurs mots simples (racines ou
      affixes), que l’on peut considérer comme invariables (de forme
      et de signification) et autonomes. La signification de tout mot
      composé résulte alors directement de l’analyse de son contenu,
      et non de la manière dont on suppose ce mot dérivé d’un autre;
      chaque mot est en soi un édifice propre.

      Il est vrai qu’on considère les langues latines plutôt comme des
      langues à flexion, que comme des langues à soudure, mais cela
      vient uniquement de ce que l’on regarde en général les affixes
      comme des éléments qui modifient le sens de la racine, au lieu
      de regarder la racine et les affixes comme autant de mots,
      invariables et autonomes, c’est-à-dire comme autant de signes
      exprimant chacun une idée qui lui est propre. Ainsi l’adjectif
      \textup{humain}, ou le verbe \textup{couronner}, ne seront plus
      considérés comme des mots simples dérivés respectivement de
      \textup{homme} et de \textup{couronne}, mais comme des mots
      composés du substantif \textup{homme} et du suffixe adjectif
      \textup{ain}, du substantif \textup{couronne} et de la désinence
      verbale \textup{er}. La différence est essentielle; dans le
      premier cas l’idée adjective pénètre le mot \textup{humain} tout
      entier, et l’idée verbale pénètre de même tout le mot
      \textup{couronner}, de sorte que ces deux mots sont des mots
      simples comparables, par exemple, à l’adjectif \textup{utile} ou
      au verbe \textup{frapper}; dans le second cas, l’idée adjective
      reste cantonnée dans le suffixe \textup{ain} sans pénétrer la
      racine \textup{hum}, et l’idée verbale reste cantonnée dans la
      désinence \textup{er} sans pénétrer la racine \textup{couronn},
      de sorte que les mots \textup{hum'ain}, \textup{couronn'er} sont
      des mots composés très différents des mots simples
      \textup{utile} et \textup{frapper}. Le mot \textup{couronn'er}
      est en réalité un substantivo-verbe, de structure semblable à
      celle du substantivo-verbe \textup{hand'schreiben} en allemand.

      En assimilant ainsi les mots dérivés à des mots composés, on
      réduit la structure générale des différents mots à un seul
      type, le même pour toutes les langues; ce type est aussi celui
      qui convient le mieux aux langues artificielles, puisqu’il
      permet de trouver la signification de chaque mot par la simple
      analyse de son contenu et sans se soucier de savoir si ce mot
      est, ou non, dérivé d’un autre.
  }
  
  \EnglishPage{\noindent
    {\small
      % 
      the adjective. In contrast, the physical sciences would be the
      science of the \emph{verb} and the \emph{noun}, for these
      sciences consider only the objective phenomenon, which is
      essentially quantitative and reduces in the last analysis to
      masses in movement, that is, to grams, centimeters and seconds;
      and mass is a ``thing'', a noun, and its movement is ``to act'',
      a verb.}\\

    \emph{Conclusion}. When we take the international point of view,
    we can say that the most general type of structure of words in
    natural languages is the type \emph{by juncture}, that is, that
    compound words are formed by joining several simple words (roots
    or affixes) which we can consider as invariable (in form and
    meaning) and autonomous. The meaning of every compound word thus
    results directly from the analysis of its content, and not from
    the manner in which we suppose this word to be derived from
    another; each word is in itself its own structure.

    It is true that we consider the Romance languages rather more as
    flexional languages than as languages by juncture, but that comes
    exclusively from the fact that we generally think of affixes as
    elements that modify the sense of the root, instead of regarding
    the root and the affixes as so many words, invariable and
    autonomous, that is as so many signs each expressing an idea which
    is proper to it. Thus the adjective \textup{humain} or the verb
    \textup{couronner} are no longer to be considered as simple words
    derived from \textup{homme} and \textup{ couronne}, respectively,
    but as words composed of the noun \textup{homme} and the
    adjectival suffix \textup{ain}, of the noun \textup{couronne} and
    the verbal desinence \textup{er}. The difference is essential: in
    the first case the adjectival idea fills the entire word
    \textup{humain}, and the verbal idea fills all of the word
    \textup{couronner} in the same way, such that these two words are
    simple words comparable, for example, to the adjective
    \textup{utile} or the verb \textup{frapper}.  In the second case,
    the adjectival idea remains confined to the suffix \textup{ain}
    without seeping into the root \textup{hum}, and the verbal idea
    remains confined to the desinence \textup{er} without penetrating
    the root \textup{couronn}, so that the words \textup{hum'ain},
    \textup{couronn'er} are compound words quite different from the
    simple words \textup{utile} and \textup{frapper}. The word
    \textup{couronn'er} is really a noun-verb, with structure
    comparable to that of the noun-verb \textup{hand'schreiben} in
    German.

    In thus assimilating derived words to compound words, we reduce
    the general structure of different words to a single type, the
    same for all languages.  This type is also the one that is most
    suitable for artificial languages, since it allows us to find the
    meaning of each word by the simple analysis of its content and
    without caring to know if this word is or is not derived from
    another.}

    \FrenchPage{Ce point de vue, du reste, n’est pas nouveau. Dans
      l’ouvrage déjà cité\footnote{\emph{Cours de ling. gén}., p. 235
        et 236.}, l’auteur admet qu’il y a deux manières de concevoir
      les mots dérivés, lorsqu’il dit:\\

      {\small
        %
        „Il y a conflit entre ces deux conceptions: pour former
      \emph{indécoràble}, nul besoin d’en extraire les éléments
      (\emph{in'décor’able}); il suffit de prendre l’ensemble et de le
      placer dans l'équation:

      \begin{center}
        \emph{pardonner}: \emph{impardonnable} = \emph{décorer}: x\\
        x = \emph{indécorable}.
      \end{center}


      Laquelle de ces théories correspond à la réalité? \ldots\ Nos
      grammaires européennes opèrent avec la quatrième
      proportionnelle; elles expliquent par exemple la formation d’un
      prétérit allemand en partant de mots complets; on dit à
      l’élève: sur le modèle de \emph{setzen}: \emph{setzte}, formez
      le prétérit de \emph{lachen}, etc. Au contraire la grammaire
      hindoue étudierait dans un chapitre déterminé les racines
      (\emph{setz}-, \emph{lach}-, etc.), dans un autre les
      terminaisons du prétérit (-\emph{te}, etc.); elle donnerait les
      éléments résultant de l’analyse, et on aurait à recomposer les
      mots complets .... Selon la tendance dominante de chaque groupe
      linguistique, les théoriciens de la grammaire inclineront vers
      l’une ou l’autre de ces méthodes.... Le latin avait à un haut
      degré le sentiment des pièces du mot (radicaux, suffixes, etc.),
      et de leur agencement. Il est probable que nos langues modernes
      ne l’ont pas de façon aussi aiguë, mais que l’allemand l’a plus
      que le français.“
      % 
    }\\

    Si nous ne craignions pas d’empiéter sur le domaine du linguiste,
    nous émettrions l’opinion que la méthode de la grammaire hindoue
    est la seule satisfaisante\footnote{A l’appui de cette opinion je
      crois pouvoir invoquer les lignes suivantes du \emph{Cours de
        linguistique générale} (p. 196): „\emph{les entités
        abstraites reposent toujours en dernière analyse, sur des
        entités concrètes}; aucune abstraction grammaticale n’est
      possible sans une série d’éléments matériels qui lui sert de
      substrat, et c’est toujours à ces éléments qu’il faut revenir en
      fin de compte.“  Et plus loin (p. 197 et 198): „Une unité
      matérielle n’existe que par le sens, la fonction dont elle est
      revêtue; ....  inversement un sens, une fonction n’existent que
      par le support de quelque forme matérielle.“}, mais afin de ne
    pas nous lancer dans un domaine qui nous est étranger, nous nous
    bornerons à constater que cette méthode est la seule qui convient
    pour les langues artificielles. En effet, „dans une langue
    artificielle les mots sont presque tous analysables; un
    espérantiste a pleine liberté de construire sur une racine donnée
    des mots nouveaux“. Dans ce jugement exprimé par l’auteur déjà
    cité,\footnote{\emph{Cours de ling. gé}n., p. 234.}  nous n’avons
    qu’un mot à supprimer: le mot „presque“.
  }
  
  \EnglishPage{This point of view, moreover, is not new. In the work
    already cited\footnote{\emph{Cours de ling. gén}., p. 235
        and 236.}, the author supposes that there are two ways of
      conceiving derived words when he says:\\

      {\small
        %
        ``There is a conflict between two conceptions: to form
        \emph{indécorable}, there is no need to extract its elements
        from it (\emph{in'décor'able}); it suffices to take the
        whole and place it in the equation:

      \begin{center}
        \emph{pardonner}: \emph{impardonnable} = \emph{décorer}: x\\
        x = \emph{indécorable}.
      \end{center}

      Which of these theories corresponds to reality? Our European
      grammars operate with the fourth proportional; they explain, for
      example, the formation of a German preterite starting from
      complete words. We tell the student: on the model of
      \emph{setzen} : \emph{setzte}, form the preterite of
      \emph{lachen}, etc. On the other hand, Hindu grammar would study
      in a particular chapter the roots (\emph{setz-, lach-,} etc.),
      in another the preterite endings (\emph{-te}, etc.). It would
      give the elements resulting from the analysis, and we would have
      to put the complete words together. \ldots\ According to the
      dominant tendency in each linguistic group, the theoreticians of
      grammar incline toward the one or the other of these
      methods. \ldots\ Latin had to a high degree a feeling for the
      pieces of a word (roots, suffixes, etc.) and for their
      arrangement. It is likely that our modern languages do not have
      this in as sharp a fashion, but German has it more than French.''
    }\\

    If we did not fear to impinge on the domain of the linguist, we
    would offer the opinion that the method of Hindu grammar is the
    only satisfying one\footnote{In support of this opinion I believe I
      can invoke the following lines from the \emph{Cours de
        linguistique générale} (p. 196): ``\emph{abstract entities
        are always based, in the final analysis, on concrete entities}; no
      grammatical abstraction is possible without a series of material
      elements which serve as its substrate, and it is always to these
      elements that it is necessary to return in the end.'' And later
      (p. 197 and 198): ``A material unit only exists by virtue of the
      sense, the function with which it is imbued; \ldots\ inversely,
      a sense, a function exists only with the support of some
      material form.''}, but in order not to plunge into a domain
    which is foreign to us, we will limit ourselves to noting that
    this method is the only one that is suitable for artificial
    languages. Indeed, ``in an artificial language the words are
    almost all analyzable; an Esperantist is fully at liberty to
    construct new words from a given root''. In this judgment
    expressed by the author already cited\footnote{\emph{Cours de
        ling. gé}n., p. 234.}, we have only one word to eliminate:
    ``almost''. 
    }
    

%%% Local Variables: 
%%% mode: latex
%%% TeX-master: "./RdS_Morphology.tex"
%%% End: 
