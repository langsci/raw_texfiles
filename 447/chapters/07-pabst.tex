\documentclass[output=paper]{langscibook}
\ChapterDOI{10.5281/zenodo.15006605}
\author{Katharina Pabst\orcid{}\affiliation{Radboud University; University of Toronto} and Sam Brunet\orcid{}\affiliation{University of Toronto} and Alison L. Chasteen\orcid{}\affiliation{University of Toronto} and Sali A. Tagliamonte\orcid{}\affiliation{University of Toronto}}
\title[Tracking language change in real time]{Tracking language change in real time: Challenges for community-based research in the 21\textsuperscript{st} century}

\abstract{In this paper, we discuss our experiences conducting a longitudinal research study involving trend and panel data in the 21\textsuperscript{st} century. Our goal is to document both challenges and opportunities for researchers engaging in real-time community-based research. The data for this project come from the \textit{Language in Later Life} project, an interdisciplinary research project investigating the language of healthy adults in Toronto during the transition from later life to retirement. We focus on strategies for recruiting panel speakers and finding matches for our trend sample. One key finding is that it is not just crucial to find ways to keep in touch with participants, but also with research assistants and fieldworkers, who were essential for (re-)locating participants almost twenty years after the first point of data collection. While our project was conducted in 2018--2019, i.e., before the COVID-19 pandemic, we argue that many of the obstacles we encountered still apply, and might even be exacerbated, making it more important than ever to reflect on our methods for tracking language change in real time.}

\IfFileExists{../localcommands.tex}{
  \addbibresource{../localbibliography.bib}
  \usepackage{langsci-optional}
\usepackage{langsci-gb4e}
\usepackage{langsci-lgr}

\usepackage{listings}
\lstset{basicstyle=\ttfamily,tabsize=2,breaklines=true}

%added by author
% \usepackage{tipa}
\usepackage{multirow}
\graphicspath{{figures/}}
\usepackage{langsci-branding}

  
\newcommand{\sent}{\enumsentence}
\newcommand{\sents}{\eenumsentence}
\let\citeasnoun\citet

\renewcommand{\lsCoverTitleFont}[1]{\sffamily\addfontfeatures{Scale=MatchUppercase}\fontsize{44pt}{16mm}\selectfont #1}
  
  %% hyphenation points for line breaks
%% Normally, automatic hyphenation in LaTeX is very good
%% If a word is mis-hyphenated, add it to this file
%%
%% add information to TeX file before \begin{document} with:
%% %% hyphenation points for line breaks
%% Normally, automatic hyphenation in LaTeX is very good
%% If a word is mis-hyphenated, add it to this file
%%
%% add information to TeX file before \begin{document} with:
%% %% hyphenation points for line breaks
%% Normally, automatic hyphenation in LaTeX is very good
%% If a word is mis-hyphenated, add it to this file
%%
%% add information to TeX file before \begin{document} with:
%% \include{localhyphenation}
\hyphenation{
affri-ca-te
affri-ca-tes
an-no-tated
com-ple-ments
com-po-si-tio-na-li-ty
non-com-po-si-tio-na-li-ty
Gon-zá-lez
out-side
Ri-chárd
se-man-tics
STREU-SLE
Tie-de-mann
}
\hyphenation{
affri-ca-te
affri-ca-tes
an-no-tated
com-ple-ments
com-po-si-tio-na-li-ty
non-com-po-si-tio-na-li-ty
Gon-zá-lez
out-side
Ri-chárd
se-man-tics
STREU-SLE
Tie-de-mann
}
\hyphenation{
affri-ca-te
affri-ca-tes
an-no-tated
com-ple-ments
com-po-si-tio-na-li-ty
non-com-po-si-tio-na-li-ty
Gon-zá-lez
out-side
Ri-chárd
se-man-tics
STREU-SLE
Tie-de-mann
}
  \togglepaper[1]%%chapternumber
}{}

\begin{document}
\maketitle
\label{chap:pabst}
\graphicspath{{figures/pabst}}

\section{Background}
\label{sec:pabst:1}
Ever since Labov’s foundational study of sound change in Martha’s Vineyard \citep{Labov1963}, many researchers have been using synchronic data from different age groups to study language change in progress. However, in recent years, there has been an increasing interest in supplementing this work with real time studies. The reason for this is twofold: For one thing, real time data can provide insights into how language change proceeds over time and within the same individuals. Moreover, there are more opportunities for this kind of work thanks to the increasing number of community-based studies that can be revisited as well as alternative data sources that have become available, both online and through archives \citep[298]{Sankoff2018}.


Generally, researchers distinguish two types of real time studies: panel studies, which follow the same individuals over time, and trend studies, which examine the behavior of different individuals who are matched for social factors such as age, gender, and education across two or more points in time \citep[1003]{Sankoff2005}.



Although real time studies may be more feasible than ever before, \citet[53]{CieriCieri2017} point out that they are still inherently ``difficult, time consuming, and expensive'' since it takes tremendous financial and organizational resources to track down the same individuals, especially after several years or decades, or to recruit participants that match the original sample in the case of trend studies. Despite these difficulties, a growing number of studies have succeeded in collecting real time data in a variety of settings, including Swedish in Eskilstuna \citep{Sundgren2001}, Brazilian Portuguese in Rio de Janeiro (\citealt{NaroScherre2003}), French in Montreal (\citealt{SankoffBlondeau2007, WagnerSankoff2011, SankoffWagner2020}), Danish in Copenhagen and various smaller communities (\citealt{Gregersen2009, GregersenEtAl2009}), English in Springville (\citealt{Cukor-AvilaBailey2011, Cukor-AvilaCukor-Avila2017}), English in Tyneside (\citealt{Buchstaller2016, BuchstallerEtAl2017, MechlerBuchstaller2019}), and Swabian in Stuttgart and Schwäbisch Gmünd  (\citealt{Beaman2021a, Beaman2021b, BeamanBaayen2021, BeamanTomaschek2021}). However, practical advice about how researchers have managed to accomplish this herculean task is rare. One notable exception is Gillian \citegen{Sankoff2017} paper on the origins of the Montreal French project, one of the first and most comprehensive longitudinal corpora in the field, which was conceptualized in the early 1970s. In this paper, Sankoff outlines many of the challenges in building the original corpus as well as the follow-up studies, including the development of sampling strategies and guidelines for ensuring confidentiality and research access. She also discusses many challenges that are more specific to longitudinal projects, including the difficulties in tracing participants many years after the original project, attrition due to mortality and changed life circumstances, and how lack of resources prohibited long-term planning for research in the future.


Based on our own experiences, we know that anticipating these kinds of challenges can make a world of difference for researchers hoping to engage in real time studies, which is why we decided to build on Sankoff’s paper by sharing our experiences in conducting our own longitudinal, real-time study. The goal of this paper is to document the challenges we faced, outline our strategies for overcoming them, and share practical advice for success. We note that our data was collected in 2018--2019, i.e., before the COVID-19 pandemic, but many of the same challenges still apply in the post-COVID era. In fact, we would argue that the ongoing public health situation likely exacerbates many of the obstacles we experienced.\\

\section{Our starting point: The Toronto English Archive (TEA)}
\label{sec:pabst:2}
The starting point for our longitudinal work is Sali Tagliamonte’s Toronto English project (\citealt{Tagliamonte2003b}). Like the Montreal French project, this project was not originally planned to be longitudinal. The foundations for this work were laid in the early 2000s: Having returned to Toronto after a 20-year hiatus, Tagliamonte noticed tremendous changes were taking place in the local variety of English and decided to document the evolving situation in a community-based study focused on language change in progress. Most of the data collection for the project took place in 2003--2004. Smaller data samples were completed in 2002, 2004 and 2006 in course-based experiential learning projects focused on adolescents and young adults. In order to be included in the study, participants had to be born and raised in Toronto and speak English as their first or dominant language.\\

\section{The Language in Later Life project}
\label{sec:pabst:3}
\subsection{Project goals}
\label{sec:pabst:3.1}
Over fifteen years later, after several years of discussion, two researchers, Sali Tagliamonte (a linguist) and Alison Chasteen (a social psychologist) developed a plan to investigate the language of healthy adults in the transition from retirement to later life (\citealt{ChasteenTagliamonte2018}). The research project entitled \textit{Sociolinguistic and Psychological Impacts of Language in Later Life} was designed specifically as both a window into language variation and change in later life, which is severely understudied, and as a way to learn more about socio-cognitive adjustments and experiences of individuals later in life. The data collection for this project mostly took place in 2018--2019. Together with our research team, which included Graduate Student Project Manager Katharina Pabst and Undergraduate Student Project Manager Samantha Brunet, we set out to collect both panel and trend data from people who were born and raised in Toronto. The project’s broad goal was to reinterview as many individuals as possible who were 30 years or older at the time of the original interviews in the early 2000’s. In practice, this meant we were looking for 99 individuals. Given previous accounts of how difficult it is to track people down after many years, we cautiously proposed to re-interview at least 14 individuals who were part of the original study.



The second goal was to conduct a trend study. To keep this part of the project manageable, we aimed to find matches for everyone who was 50 years old or older in 2002--2004 with a new sample in 2018--2020. This meant seeking out 51 new individuals who were 50 years or older. To provide a comparative perspective on our panel participants, we also aimed to recruit matches for them that were consonant in terms of age, gender, and education, both as they were in 2002--2004 and as they were at the time of data collection in 2018--2019.\\

\subsection{Project design}
\label{sec:pabst:3.2}

The project comprises several sub-components, including a panel study (henceforth abbreviated with P) and a trend study (henceforth abbreviated with T). In the following, we will briefly describe these projects in more detail. In this context, numeral 1 refers to data that was collected in 2002--2004 and numeral 2 refers to data that was collected in 2018--2019.



The goal of the project was to allow for four types of comparison (see \tabref{tab:pabst:1}):


\begin{itemize}
\item
The first comparison is a panel study, in which the same 14 individuals are interviewed in 2002--2004 and 2018--2019. During the second interview, panel participants were approximately 14--17 years older than in the original interview (P1--P2).  This comparison offers insight into lifespan change.
\item
The second comparison is a trend study, including different individuals that were matched for age, gender, and education at two points in time (T1--T2). This comparison mirrors change in time.
\item
The third comparison involves the panel participants as they were in 2002--2004 and a set of social twins that matches them in terms of age, gender, and education as they were in 2002--2004 (P1--MP1). This comparison offers insight into the panel participants and their same age cohort in 2002--2004.
\item
Finally, the fourth and last comparison comprises the panel participants as they were in 2018--2019 and a set of social twins that matches them in terms of age, gender, and education in 2018--2019 (P2--MP2). This comparison offers insight into how the panel participants in later life compare with their own age cohort in later life, in 2018--2019.
\end{itemize}


\begin{table}
\begin{tabularx}{.8\textwidth}{Qll}
	\lsptoprule
	Type of comparison & Time 1 (2002--2004) & { Time 2 (2018--2019)}\\
	\midrule
	Panel: & P1\tikzmark{p1} & \tikzmark{p2}P2\tikzmark{p2.2}\\
	Same individuals & & \\
	\midrule
	Trend: & T1\tikzmark{t1} & \tikzmark{t2}T2\\
	Dfferent individuals, & & \\
	same age, gender, & & \\
	and education & & \\
	\midrule
	Cross-sectional &  & \tikzmark{m1}MP1, MP2~\tikzmark{m2}\\
	comparisons: & & \\
	Different individuals,  & & \\
	same age, gender, & & \\
	and education  & & \\
	as P1 and P2 & & \\
	\lspbottomrule
\end{tabularx}
\begin{tikzpicture}
	[overlay, remember picture, shorten >=.5pt, shorten <=.5pt, transform canvas={yshift=.25\baselineskip}]
	\draw [->] ({pic cs:p1}) -- ({pic cs:p2});
	\draw [->] ({pic cs:p1}) to[out=0,in=180] ({pic cs:m1});
	\draw [->,rounded corners] ({pic cs:p2.2}) -| ++(5em,-10pt) |- ({pic cs:m2});
	\draw [->] ({pic cs:t1}) -- ({pic cs:t2});
\end{tikzpicture}
\caption{Project design}
\label{tab:pabst:1}
\end{table}

\subsection{Recruitment strategies}
\label{sec:pabst:3.3}

The biggest challenge for our project was to find the people who were to be interviewed a second time, which would enable us to probe lifespan change. \tabref{tab:pabst:2} provides an overview of our recruitment strategies, in order of their success. There were 99 individuals in the original sample who met the criteria for a second interview. We note that some participants were found using more than one method, which is why the numbers do not add up to 99.



First, we conducted obituary searches only to find that 22 individuals had died between the original interviews and the time of data collection. Given the advanced age of our target population and consistent with \citegen{Sankoff2017} reports of attrition due to mortality, this was not unexpected. However, we emphasize how significant this fact is in terms of limiting the number of possible participants in a real time study across many years.



Our most successful strategy was getting in touch with former participants by seeking out and contacting the original interviewers for more information and also for their help in recruiting the second interviews. Most of the original fieldworkers were students and research assistants in Sali Tagliamonte’s lab in the early 2000s and many of them had interviewed individuals in their social networks. Fortunately, they had often kept in touch with the same individuals over the years. Further, they had maintained connections with Sali Tagliamonte.\footnote{We do not mean to make light of how difficult this strategy was. Only a handful of the original interviewers were willing or able to help given their jobs and family situations. One former interviewer had to be flown in from a US city for the weekend. However, a side product was the pleasant reconnection and updates between Sali Tagliamonte and her former students, suggesting that maintenance of social ties with research team members is an asset for community-based research.} Using this method, we were able to find six participants, four of whom agreed to a second interview, for a success rate of 66.7\%.



Because the original sampling strategy targeted specific neighborhoods in Toronto, we had kept records of participants’ addresses. Therefore, we were able to use a reverse address lookup website named 411.ca where one can enter a Canadian address and determine if a name and/or phone number are attached to that address.\footnote{\url{https://411.ca} Data for this paper retrieved  Monday, January 9, 2023 6:44am.} In total, we found 23 participants this way. We successfully connected with twelve of them and managed to recruit six of them for the second interview, an overall success rate of 26.1\%.



Another successful strategy was using LinkedIn, a professional networking website. Samantha Brunet created a profile specifically for this purpose and sent potential candidates a contact request explaining the nature of our project. We found ten former participants this way, two of whom were successfully contacted. In the end, one agreed to a second interview, for a success rate of 10\%.


We also sent letters to all fifty-three individuals for whom we had valid addresses and who we believed to still be alive using their last known addresses from 2002--2004. We successfully got in touch with fourteen individuals this way, three of whom agreed to return, resulting in a success rate of 5.7\%.
A substantial number of letters were returned to us. Oftentimes, the returned envelopes included helpful information. For example, one letter stated that the addressee passed away in 2010. Another one mentioned that the addressee had moved years ago, indicating that the address we had on file was no longer usable. Taken together, the relatively low success rate of this approach leads us to conclude that it would be promising to collect email addresses for participants in future research projects since email contact information tends to remain more stable than geographic location.

Other strategies we used included looking for individuals on Facebook or searching for them on Google. These approaches allowed us to find five individuals each; however, not a single one was contacted successfully. For Facebook, this is likely due to the site’s privacy settings, which automatically route messages from individuals outside of one’s network into a spam folder, which most people rarely check. For the remainder, it is difficult to assess the poor outcome because we have no way of knowing if the communications were successfully received by the individuals intended.


Despite the extraordinary difficulties in tracking former participants, we managed to find and communicate with 34 of them. However, only 14 of them agreed to be part of our study. Those who declined to participate offered a wide variety of reasons, ranging from having very busy schedules to no longer living in the area. In many cases, the participants were quite old and felt unable to put in the requisite time and energy. In one case, a former participant had developed cancer and was undergoing intensive chemotherapy. One person was adamant that they had ``nothing to talk about,'' while another felt self-conscious about the messiness of their apartment. The latter concern is why we decided to offer participants the opportunity to come to the lab instead if they preferred, which gained us several participants.
\largerpage
Next, we turn to the strategies we used to recruit the trend participants and social twins for the panel participants. Our original goal was to recruit all individuals using the Adult Volunteer Pool, a database maintained by the Psychology Department at the University of Toronto, which has contact information for individuals 50 years or older who have also indicated interest in participating in paid research studies. We were able to recruit 22 individuals this way. However, one challenge that we did not anticipate was how difficult it was to find matches for the social characteristics. The psychology database does not include information on participants’ first language, or where they were born and raised – both of which were important selection criteria for our study. Given that Toronto has had tremendous in-migration in the past two decades, many of the participants in the Adult Volunteer Pool were not born in the area or moved to Toronto before the age of 10. Another complication was that most individuals in this database are over 55 years old, making it difficult to find matches for the younger panel speakers who were in early adulthood or middle-aged in 2002--2004.

\begin{table}
\small
\begin{tabularx}{\textwidth}{XY}
\lsptoprule
Success rate & Notes\\
\midrule
\multicolumn{2}{c}{\textbf{Connection to former interviewers}}\\
\midrule
Found: 6 & Former students who conducted \\
Successfully contacted: 6 & the interviews in 2002--2004 \\
Recruited: 4 & \\
Success rate: 66.7\% & \\
\midrule
\multicolumn{2}{c}{\textbf{411.ca}}\\
 \midrule
Reliable phone numbers: 23 & Reverse address lookup\\
Successfully contacted: 12 & \\
Recruited: 6 & \\
Success rate: 26.1\% & \\
\midrule
\multicolumn{2}{c}{\textbf{LinkedIn}}\\
\midrule
Found: 10 & Professional networking website,\\
Successfully contacted: 2 & created profile and sent\\
Recruited: 1 & participants a contact request\\
Success rate: 10\% & \\
\midrule
\multicolumn{2}{c}{\textbf{Letters to addresses from 2002--2004}}\\
\midrule
Letters sent: 53 & Used addresses we had \\
Successfully contacted: 14 & on file from 2002--2004 \\
Recruited: 3 & \\
Success rate: 5.7\% & \\
\midrule
\multicolumn{2}{c}{\textbf{Facebook}}\\
\midrule
Found: 5 & All five participants who were \\
Successfully contacted: 0 & found were closer to 30 during \\
Recruited: 0 & the original interviews; \\
Success rate: 0\% & privacy settings forbid direct messaging\\
\midrule
\multicolumn{2}{c}{\textbf{Google search}}\\
\midrule
Found: 5 & Last resort, only useful \\
Successfully contacted: 0 & in that we found one \\
Recruited: 0 & participant who had moved away\\
Success rate: 0\% & \\
\lspbottomrule
\end{tabularx}
\caption{Overview of recruitment strategies for panel participants.}
\label{tab:pabst:2}
\end{table}

% Next, we turn to the strategies we used to recruit the trend participants and social twins for the panel participants. Our original goal was to recruit all individuals using the Adult Volunteer Pool, a database maintained by the Psychology Department at the University of Toronto, which has contact information for individuals 50 years or older who have also indicated interest in participating in paid research studies. We were able to recruit 22 individuals this way. However, one challenge that we did not anticipate was how difficult it was to find matches for the social characteristics. The psychology database does not include information on participants’ first language, or where they were born and raised – both of which were important selection criteria for our study. Given that Toronto has had tremendous in-migration in the past two decades, many of the participants in the Adult Volunteer Pool were not born in the area or moved to Toronto before the age of 10. Another complication was that most individuals in this database are over 55 years old, making it difficult to find matches for the younger panel speakers who were in early adulthood or middle-aged in 2002--2004.


\begin{table}
\begin{tabularx}{\textwidth}{XYY}
\lsptoprule
{ Method} & { Number of individuals recruited} & { Notes}\\
\midrule
{ {Adult Volunteer Pool}} & { {22}} & { {Database used by Psychology Department at U of T, has contact info for individuals 50+}}\\
\midrule
{ {Forever Young magazine}} & { {32}} & { {Magazine for individuals 50+ who live in the GTA}}\\
\midrule
{ {Social media (Facebook, Reddit)}} & { {10}} & { {Toronto-specific Facebook groups, r/Toronto}}\\
\midrule
{ {Friends of friends}} & { {6}} & { {Research team reached out to friends who matched criteria of participants we did not have matches for}}\\
\lspbottomrule
\end{tabularx}
\caption{Overview of recruitment strategies for trend participants.}
\label{tab:pabst:3}
\end{table}

Therefore, we pursued additional avenues for recruitment. For example, we posted two ads in “Forever Young”, a magazine distributed across the Greater Toronto Area, aimed at older adults age 50+ and primarily read by individuals more than 70 years old. In both May and July 2019, we purchased advertising space. In May, we posted an ad seeking older adults aged 50--95 and received hundreds of calls and emails from interested locals. In July, we more specifically advertised to people aged 50--65, and received considerably less attention (only around 25 calls/emails). In total, we were able to secure 32 matches this way. We found 10 more matches by posting on Toronto-specific Reddit and Facebook groups. The remaining six participants were recruited by reaching out to friends who matched the criteria of participants we had not yet successfully matched.

\subsection{Data}
\label{sec:pabst:3.4}

\tabref{tab:pabst:4} shows our final participant sample. In the end, we were able to recruit 14 panel participants, exactly meeting our original goal. We were further able to recruit 50 out of 51 trend participants we were aiming for. The original sample included one participant who did not finish high school, which is much rarer nowadays due to laws governing minimum education requirements in Ontario and Canada more generally, preventing us from finding a suitable match. As for the social twins, we succeeded in finding matches for our panel participants as they were in 2018--2019, but did not find matches for all 14 of them as they were in 2002--2004. However, Sali Tagliamonte is currently engaged in another large-scale project in Toronto (\citealt{Tagliamonte2018}) that targets younger speakers, some of whom will hopefully be able to fill these gaps.

We should note that most individuals in our study are considered white according to census criteria (but they comprise a mix of British, European, and Asian backgrounds). There is only one biracial black and white speaker in the panel sample. In contrast, due to shifting demographics in Canada in the intervening 20 years or so, the trend sample has representation of other ethnic groups, but even here the overwhelming majority is also white (as defined above). The criterion of having been born and raised in the city of Toronto continues to restrict the sample to a predominantly white population.\footnote{In more recent research based in a high school, sampling included locally born and raised youth as well as those that had come from other countries to study the impact of immigrant populations on the local vernacular \citep{TagliamonteManuscript}.}  The sample is largely balanced between male- and female-presenting speakers, with female-presenting speakers being slightly overrepresented.


\begin{table}
\footnotesize
\begin{tabular}{lrrc}
\lsptoprule
Type of comparison & Time 1 (2002--2004) & \multicolumn{2}{c}{{ Time 2 (2018--2019)}}\\
\midrule
Panel: & n=14 & \multicolumn{2}{c}{n=14} \\
same individuals & Age 34--73 & \multicolumn{2}{c}{Age 49--86}\\
\midrule
Trend: & n=51& \multicolumn{2}{c}{n=50}\\
different individuals & Age 51--92 & \multicolumn{2}{c}{Age 50--89}\\
same age and gender &  & \multicolumn{2}{c}{}\\
\midrule
Cross-sectional&  & Panel Match 1 & \multicolumn{1}{l}{Panel Match 2: }\\
comparisons &  & n=6 & \multicolumn{1}{l}{n=14}\\
&  & Age 50--73 & \multicolumn{1}{l}{Age 47--90}\\
\lspbottomrule
\end{tabular}
\caption{Speaker sample}
\label{tab:pabst:4}
\end{table}

In sum, we used different recruitment methods for different participant groups, relying heavily on existing information about the participants we had collected during the original interviews from the early 2000s. The strategy that led to the most success was networking with former fieldworkers. Taken together, these two aspects of our methodology demonstrate two key messages for future research: 1) the critical need to document information in research, not simply about the participants, but also research assistants and fieldworkers; and 2) the importance of maintaining social ties with research assistants and even the participants themselves (see \citealt{WagnerTagliamonte2019}).

The least successful strategy for recruitment was social media, likely because of the age group we were targeting, i.e., the older adult sector. Given the dramatic changes in social media platforms, certain generations may not be as engaged on certain platforms (e.g., Facebook or Twitter) nor possibly as open to communication in one modality or the other. Note that recent work shows that platforms such as Instagram can be crucial for targeting younger speakers (\citealt{NesbittWatts2022}). For the older adults we were seeking, targeted ads (especially those in print media) were more successful. The key message here is to determine the regular communication habits of the generational sector being targeted for study.

Drawing on existing participant pools such as the Adult Volunteer Pool can be incredibly helpful. However, if these are not run by linguists, they may not include the macrosocial information linguists need to decide whether speakers match inclusion criteria, such as \textit{born and raised in Toronto}.

\section{Community-based research during and after COVID-19}
\label{sec:pabst:4}

As mentioned earlier, all our trend and panel data were collected in 2018--2019, before the COVID-19 pandemic put a sudden end to most in-person data collection over the period 2019--2021. During this time many projects shifted to online data collection. Researchers interested in conducting a longitudinal research project might wonder how the pandemic may have affected the data from longitudinal research covering that period. In the years immediately following the pandemic shut-down, several research studies have shown that online data collection is a viable alternative (\citealt{CarmichaelEtAl2022, Hall-LewEtAl2022, LeemanEtAl2020, Sneller2022, SnellerEtAl2022}). These studies confirm that online data collection was not disruptive to underlying linguistic patterns. Such research also demonstrates the feasibility of longitudinal work repeated sampling in communities that have been studied in the past.



In the summer of 2021, Zoom interviews with Toronto youth confirmed for us that at least when working with young people, ZOOM data is a viable alternative to in-person conversations. While interactional patterns in an online environment can be affected during speaker changes, when the participant talks, the language appears to be relatively unaffected by the modality (Tagliamonte, p.c. 31-8-2021). Further, \citegen{GardnerKostadinova2024} study demonstrates that there are no significant differences in the internal constraints on one of the most diagnostic stylistic variables, \textit{-ing} variation, in a consistent comparison of frequency of forms and linguistic constraints across modalities. The efficacy of online interviewing was further confirmed in interviews done with older adult participants on a recent fieldtrip to northwestern Ontario (Tagliamonte p.c. 30-7-2022). In fact, older people who have mobility issues, find it difficult to travel, or are immunocompromised, greatly enjoyed participating in interviews from the comfort of their own homes. However, it should be noted that some participants required the support of a younger family member to help them with the technology. These studies demonstrate online modalities offer certain sectors of the population an ideal option for participating in linguistic studies.


\section{Practical advice}
\label{sec:pabst:5}

Doing longitudinal community-based work is challenging, but there are ways to ensure greater success. In this project, Sali Tagliamonte had kept detailed records of interview content and documentation of names and addresses of the interviewees and research assistants who conducted the interviews from the previous project.\footnote{Our ethics protocol stipulated that we intended to contact previous participants from the 2003 study. While the original ethics protocol stipulated participant anonymity for research purposes, ongoing contact with the project director was encouraged. We acknowledge that data protection requirements may differ across countries and disciplines.} While the original purpose was to create a foundation for future research and to maintain contacts in the speech community, this information made it possible to seek out participants for the current project. However, the time span (18 years), mobility of individuals, and changes in life circumstances made it very difficult nonetheless to find and recruit them. On a related note, Katharina Pabst, who conducted most of the interviews for this project, found that reading through the transcripts of the first interviews was also an ideal way to get a better idea of what topics panel speakers might be interested in talking about during the second interviews. For example, one participant mentioned being an avid photographer during the first interview. When Katharina entered the participant’s house for the second interview, she noticed many gorgeous photos on the wall. Asking about these was an ideal way to break the ice and develop rapport with the participant.



Given how difficult it was to track down the original participants, we recommend adding the protocol of asking participants whether they can be contacted in future research studies and obtain information such as phone numbers, social media contacts, and email addresses.\footnote{In current research practise, permission for future contact is stipulated in the consent procedures.}



Many of the panel speakers we re-interviewed did not remember taking part in the original project, sometimes leading to suspicion on their part. While all the authors regularly give outreach talks to share the results of our work, a new aim of our research modus operandi will be to be more deliberate about finding ways to stay in touch with participants and share the results of unfolding research findings in our work. Public lectures, newsletters, and social media are an ideal means to make that happen.



Both for the sake of funding applications and your own mental health, we recommend setting realistic goals for the kind of community you are working with and the age group. \citet{Sankoff2017} warned that attrition was the highest for older speakers, and as our project has shown, there was indeed substantial attenuation of the target population, not only due to mortality but also pervasive health issues that made participating in the project difficult for many different reasons.



Joining forces with researchers in social psychology has been an outstanding success, allowing us to collect new types of data that will eventually shed light on the relationship between cognitive development and language use in a way that has never been possible before. It has also taught us how to scientifically study the experiences of aging and ageism through sociolinguistic data, demonstrating how fruitful combining methods and resources can be (see \citealt{ChasteenEtAl2022}). The success of this research relies as much on careful planning as it does on an open-minded scientific stance on the part of the researchers, flexibility, and resourcefulness. We also greatly benefitted from informal conversations with other researchers who have conducted longitudinal, community-based research. Our goal in this paper has been to pay forward our experiences and engage in continued conversations about methodological challenges and opportunities.


\section*{Acknowledgements}

We gratefully acknowledge the support of the Social Science and Humanities Research Council (SSHRC), including Research Grant \# 410-2003-0005 (Linguistic Changes in Canada Entering the 21\textsuperscript{st} Century) to Sali A. Tagliamonte and Research Grant \#430-2018-000026 (Sociolinguistic and Psychological Impacts of Language in Later Life) to Alison L. Chasteen and Sali A. Tagliamonte. Further, we thank the government of Ontario and the Department of Linguistics at the University of Toronto for an Ontario Trillium Scholarship to Katharina Pabst. Finally, we extend our gratitude to the many research assistants of the University of Toronto Variationist Sociolinguistics Lab and the Intergroup Relations Lab who collected and processed the data for this project.

\printbibliography[heading=subbibliography,notkeyword=this]
\end{document}
