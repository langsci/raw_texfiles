\addchap{Preface}
\begin{refsection}
%This preface has no abstract and no authors, but cites \citet{Nordhoff2018} nevertheless.
%content goes here 
This book offers an in-depth exploration of contemporary issues and methodologies in the fields of dialectology and sociolinguistics. Readers will find a diverse collection of studies that examine how language varies and changes across different regions, communities, and social contexts. The book covers a wide range of languages, including German, English, Yiddish, Russian, and Japanese, providing a global perspective on linguistic diversity.
Key themes include the use of modern data sources, such as social media, to study language patterns and the impact of digital communication on regional dialects. The book also addresses the dynamics of language contact in expatriate communities, revealing how speakers adapt and merge linguistic features from different dialects. Several chapters focus on the evolution of dialectological research, offering critiques and new approaches to studying regional language variations. Readers will also encounter innovative methods, such as cognitive geography, which uses mental representations of space to understand dialect variation, and tone distance measures, which are crucial for studying tonal languages.
Additionally, the book presents case studies on how non-experts perceive and categorize dialects, providing insights into the public's understanding of linguistic diversity. It also tackles challenges in selecting dialect speakers for research, especially in urban environments, where traditional criteria may no longer apply.

Part \ref{part:1}, \emph{Geolinguistic methods and big data in dialectology}, focusses on the application of geospatial analysis, big data, and innovative methods in the study of dialects and language variation.
\citeauthor{chapters/01-baxter} examines methodology in the use of Twitter in the corpus-based analysis of AAE syntax, focusing on geospatial mapping and data extraction methods. \citeauthor{chapters/02-baxterEtAl} focus on the creation of an atlas of AAE syntax using Twitter data, with a geospatial mapping of linguistic features across different U.S. regions. \citeauthor{chapters/03-blassnigg} discuss the application of geolinguistic methods in dialectology, using data from the Atlas of Colloquial German in Salzburg, and highlights cluster analysis of regional dialects. \citeauthor{chapters/04-sekeres} explore the use of cognitive geography in explaining dialect variation, comparing cognitive and geographic distances in the perception of dialect differences.

Part \ref{part:2}, \emph{Corpus-based studies and dialect change}, brings together studies that use corpora to analyze language change, dialect contact, and linguistic variation over time.
\citeauthor{chapters/05-hirano} analyses linguistic change in an English-speaking expatriate community in Japan, focusing on dialect contact and long-term accommodation. \citeauthor{chapters/06-nove} use archival data to study the acoustic correlates of vowel length contrasts in Central Yiddish, highlighting dialect change in the Transcarpathian region. \citeauthor{chapters/07-pabst} discuss longitudinal research on language change, focusing on methodological challenges in tracking linguistic variation over time. \citeauthor{chapters/08-siewert} investigate changes in Low Saxon dialects over time, using corpus-based methods to explore dialect similarities and divergence across the Dutch-German border. \citeauthor{chapters/09-burkette} review the evolution of the Linguistic Atlas Project’s fieldwork methods from mapping dialect boundaries to recording variation and sociolinguistic attitudes, highlighting changes in interview methods and goals.

Part \ref{part:3}, \emph{Dialectology, linguistic identity, and social factors}, explores how linguistic variation relates to social identity, language ideologies, and the intersection of language with social factors.
\citeauthor{chapters/10-jahns} introduces the linguistic-positioning task to study linguistic choices in the Jewish community in Berlin, emphasizing the impact of language ideologies on speaker choices. \citeauthor{chapters/11-post} investigates regional prosodic differences in modern urban Russian speech, focusing on how these differences persist despite standardization. \citeauthor{chapters/12-takemura} re-evaluates the criterion for selecting local dialect speakers in Japanese dialectology, emphasizing the role of parental origin in linguistic identity.

Part \ref{part:4}, \emph{Theoretical approaches and innovations in dialectology}, encompasses theoretical discussions and methodological innovations that challenge traditional dialectological frameworks.
\citeauthor{chapters/13-dollinger} critiques the anti-pluricentric perspectives in German dialectology, proposing a new framework for understanding linguistic standards and dialects. \citeauthor{chapters/14-kathrein} evaluates dialect collections from the Tyrol region, comparing entries with standard language to understand how laypersons conceptualize their dialects. \citeauthor{chapters/15-sung} compare tone distance measures in Sinitic dialects, proposing improvements to dialectometry methods, especially for tonal languages. \citeauthor{chapters/16-yurayong} examine tone paradigms in Tai dialects, applying a historical-comparative approach to dialectal classification and language contact.

Overall, this book is a valuable resource for linguists, researchers, and anyone interested in the complex and ever-changing landscape of human language. It highlights the importance of adapting research methods to keep pace with the evolving nature of language and offers fresh perspectives on how we study and understand dialects and language variation.

%{\sloppy\printbibliography[heading=subbibliography]}
\end{refsection}
