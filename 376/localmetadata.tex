\title{Referring to\newlineCover{} discourse participants\newlineCover{} in Ibero-Romance languages}
% \subtitle{Add subtitle here if it exists}
\BackBody{This volume brings together contributions by researchers focusing on personal pronouns in Ibero-Romance languages, going beyond the well-established variable of expressed vs. non-expressed subjects. While factors such as agreement morphology, topic shift and contrast or emphasis have been argued to account for variable subject expression, several corpus studies on Ibero-Romance languages have shown that the expression of subject pronouns goes beyond these traditionally established factors and is also subject to considerable dialectal variation. One of the factors affecting choice and expression of personal pronouns or other referential devices is whether the construction is used personally or impersonally. The use and emergence of new impersonal constructions, eventually also new (im)personal pronouns, as well as the variation found in the expression of human impersonality in different Ibero-Romance language varieties is another interesting research area that has gained ground in the recent years. In addition to variable subject expression, similar methods and theoretical approaches have been applied to study the expression of objects. Finally, the reference to the addressee(s) using different address pronouns and other address forms is an important field of study that is closely connected to the variable expression of pronouns. The present book sheds light on all these aspects of reference to discourse participants. The volume contains contributions with a strong empirical background and various methods and both written and spoken corpus data from Ibero-Romance languages. The focus on discourse participants highlights the special properties of first and second person referents and the factors affecting them that are often different from the anaphoric third person. The chapters are organized into three thematic sections: (i) Variable expression of subjects and objects, (ii) Between personal and impersonal, and (iii) Reference to the addressee.}
\author{Pekka Posio and Peter Herbeck} 

\renewcommand{\lsID}{376}
% \renewcommand{\lsDedicationFont}{\fontsize{14pt}{9.3mm}\selectfont}
\dedication{This book is dedicated to the memory of\\ Kimberly Geeslin (1971–2023), whose research and\\ kindness will continue to inspire linguists across the globe.}

\renewcommand{\lsISBNdigital}{978-3-96110-416-1}
\renewcommand{\lsISBNhardcover}{978-3-98554-074-7}
\BookDOI{10.5281/zenodo.8123675}
\typesetter{Helene A. von Martens}
\proofreader{Amir Ghorbanpour,
Elliott Pearl,
Georgios Vardakis,
Janina Rado,
Jeroen van de Weijer,
Lachlan Mackenzie,
Neneng Sri,
Rebecca Madlener}
\lsCoverTitleSizes{42.5pt}{14.2mm}% Font setting for the title page


\renewcommand{\lsSeries}{orl}
\renewcommand{\lsSeriesNumber}{4} 


%     1. Peter Herbeck & Pekka Posio: Introduction 
% I Variable expression of subjects and objects 
%     2. Bessett, Ryan (University of Texas Rio Grande Valley): “A cross-dialectic comparison of first person singular subject pronoun expression in Southern Arizona and Southeast Texas”
%     3. Brown, Esther & Javier Rivas (University of Colorado, Boulder): “Variable first and third person indirect object pronoun expression in Spanish and Galician: a usage-based approach” 
%     4. Kimberly Geeslin1, Bret Linford2, Tom Goebel-Mahrle1, & Jingyi Guo1 (Indiana University1 & Grand Valley State University2): “Variable subject expression in L2 Spanish: The role of perseveration in acquisition”
% II Between personal and impersonal 
%     5. Amaral, Eduardo (Universidade Federal de Minas Gerais) & Wiltrud Mihatsch (Universität Tübingen): Portuguese a pessoa, uma pessoa, as pessoas: emerging inclusive impersonals
%     6. Henriques, Yoselin (Universität Zürich): “Personal uses of impersonalizing strategies: hybrid constructions with se and a gente in rural Madeiran Portuguese varieties” 
%     7. Juanito Ornelas de Avelar (Universidade de Campinas): “The pronominal use of geral in Brazilian Portuguese: impersonal readings and syntactic distribution”
%     8. Orozco, Rafael (Louisiana State University), Luz Marcela Hurtado (Central Michigan University) & Marienne Dietz (Universidad de Antioquia): “Toward the prevalence of a personal use of impersonal uno in Colombian Spanish?” 
% III Referring to the speaker and the addressee
%     9. Marques, Aldina (Universidade do Minho) & Isabel Margarida Duarte (Universidade do Porto): “Referring to discourse participants in European Portuguese: address pronouns and construction of social relations”
%     10. Nogué Serrano, Neus (Universitat de Barcelona) & Lluis Payrató (Universitat de Barcelona): “Variation and change in reference to discourse participants in Catalan parliamentary debate, from 1932-38 to 1980-2013 (or 2020)”
%     11. Pierre, Emeline & Barbara De Cock (Université catholique de Louvain): “Discourse participants in impersonal constructions: the case of first and second person objects of non-anaphorical 3rd person plural forms” 
