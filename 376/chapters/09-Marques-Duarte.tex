\documentclass[output=paper]{langscibook}
\ChapterDOI{10.5281/zenodo.8124504}
\author{Isabel Margarida Duarte\orcid{0000-0001-7908-5649}\affiliation{Universidade do Porto} and Maria Aldina Marques\orcid{0000-0003-3263-1977}\affiliation{Universidade do Minho}}
\title[Referring to discourse participants in European Portuguese]
	  {Referring to discourse participants in European Portuguese: The form of address \textit{o senhor}}
\abstract{We will examine the uses of the noun phrase \textit{o senhor} (formal ‘you’), as well as its linguistic and discourse status. As a form of address, it has acquired features that are typical of pronominal forms of address, with bleaching of semantic traits that point to an ongoing process of grammaticalization. In European Portuguese, despite being an issue that has been addressed several times, a comparison of the existing theoretical explanations has yet to be accomplished. Furthermore, its usage has not been analysed in different discourse contexts so as to attest to these changes. It is therefore necessary to revive and broaden the discussion. The data we have employed in this analysis is taken from the corpus \textit{Perfil sociolinguístico da fala bracarense} (`Sociolinguistic profile of Braga speech'), consisting of sociolinguistic interviews. We also built an ad hoc corpus, comprising political debates and interviews. In addition, for specific questions, some data was obtained from the CETEMPúblico corpus, and from the Davies \& Ferreira corpus for diachronic data. The overall goal of this study is the analysis of the linguistic and discourse features of the address form \textit{o senhor}. It is a qualitative approach, complemented by quantitative analysis of the occurrences recorded. 

The results of our study show that \textit{o senhor} is a hybrid form of address, revealing features from the two categories, the nominal form of address and the pronominal form of address. The confrontation of diachronic and synchronic data shows that the semantic values of the noun affect the current pragmatic values of the forms of address (FA).}

% % % \smallskip \\
% % % \textbf{Keywords:} discourse uses, European Portuguese, forms of address, grammaticalization, \textit{o senhor} ‘Mr/you’}
\IfFileExists{../localcommands.tex}{
  \addbibresource{../localbibliography.bib}
  % add all extra packages you need to load to this file

\usepackage{tabularx,multicol}
\usepackage{url}
\urlstyle{same}

\usepackage{listings}
\lstset{basicstyle=\ttfamily,tabsize=2,breaklines=true}

\usepackage{langsci-basic}
\usepackage{langsci-optional}
\usepackage{langsci-lgr}
\usepackage{langsci-osl}
% \usepackage{./langsci/styles/langsci-lgr}
% \usepackage{./langsci/styles/langsci-osl}
% \usepackage{langsci-gb4e}

\usepackage{tikz}
\usetikzlibrary{patterns,calc}
\pgfdeclarepatternformonly{south east lines}{\pgfqpoint{-0pt}{-0pt}}{\pgfqpoint{3pt}{3pt}}{\pgfqpoint{3pt}{3pt}}{
    \pgfsetlinewidth{0.6pt}
    \pgfpathmoveto{\pgfqpoint{0pt}{3pt}}
    \pgfpathlineto{\pgfqpoint{3pt}{0pt}}
    \pgfpathmoveto{\pgfqpoint{.2pt}{-.2pt}}
    \pgfpathlineto{\pgfqpoint{-.2pt}{.2pt}}
    \pgfpathmoveto{\pgfqpoint{3.2pt}{2.8pt}}
    \pgfpathlineto{\pgfqpoint{2.8pt}{3.2pt}}
    \pgfusepath{stroke}}
    
\usepackage{stmaryrd}
\usepackage{wasysym}
\usepackage{multirow}
\usepackage{caption}
\usepackage{subcaption}
\usepackage{mathrsfs}
\usepackage{qtree}

\usepackage{linguex}


  %pminos do not split footnotes
% \interfootnotelinepenalty=10000 %Footnote in Laporte chapters has to be split SN


%\DeclareIndexNameFormat{default}{%
%\nameparts{#1}%
%\usebibmacro{index:name}%
%{\index[names]}%
%{\namepartfamily}%
%{\namepartgiveni}%
% {}% L1
% {}% L2
%{\namepartprefix}% generates spurious space L3
%{\namepartsuffix}% generates spurious space L4
%}

%  {\DeclareIndexNameFormat{default}{%
%     \usebibmacro{index:name}{\index[names]}{#1}{#3}{#5}{#7}}}

%\DeclareIndexNameFormat{default}{%
%  \usebibmacro{index:name}{\sindex[nom]}{#1}{#3}{#5}{#7}}

%\DeclareIndexNameFormat{default}{%
%  \usebibmacro{index:name}{\sindex[person]}{#1}{#3}{#5}{#7}}
%\DeclareIndexNameFormat{default}{%
%\nameparts{#1} \usebibmacro{index:name}{\sindex[person]]}{\namepartfamily}{‌​\namepartgiven}{\nam‌​epartprefix}{\namepa‌​rtsuffix}}

%\newcommand{\smiley}{:)}

%\renewbibmacro*{index:name}[5]{%
%\usebibmacro{index:entry}{#1}%
%{\iffieldundef{usera}{}{\thefield{usera}\actualoperator}\mkbibindexname{#2}{#3}{#4}{#5}}}

% \newcommand{\noop}[1]{}

%remove for final
%\overfullrule=1mm

\newcommand{\tobi}[2]}}
\renewcommand{\S}[1]{\tobi{#1}{\textsc{*}}}

% this volume references
% puts: [this volume]
% already defined: \citetv
%\newcommand{\citepv}[1]{(\citeauthor{#1} \citeyear*{#1} [this volume])}
\newcommand{\citealtv}[1]{\citeauthor{#1} \citeyear*{#1} [this volume]}

%parentheses around example number
\newcommand{\pref}[1]{(\ref{#1})}

% in-text examples

\newcommand{\lnex}[1]{\textit{#1}} %target lang word
\newcommand{\lnlit}[1]{(lit.: `#1')} %literal reading
\newcommand{\lnlat}[1]{(#1)} % latinization
\newcommand{\lntrans}[1]{`#1'} %translation
\newcommand{\lnexl}[2]%
{\lnex{#1}{} \lnlat{#2}} % ex with latinization
\newcommand{\lnexlat}[3]{\lnex{#1}{} \lnlat{#2}{} \lntrans{#3}} % ex with latinization and tranl.

%ch01
\newcommand{\co}[1]{\mbox{\textbf{#1}}}

%ch09

\newcommand{\cyrbulg}[1]{\begin{otherlanguage*}{bulgarian}#1\end{otherlanguage*}}


%ch10
\newcommand{\nlp}{{\small NLP}}
\newcommand{\mwe}{{\small MWE}}
\newcommand{\rae}{{\small RAE}}
\newcommand{\lvc}{{\small LVC}}
\newcommand{\pos}{{\small P}o{\small S}}
%\newcommand{\todo}[1]{ \textcolor{red}{#1} }

%\renewcommand{\labelenumi}{\theenumi}
%\ainamefmt{{vv}{ll}{, ff}{, jj}} % fullname

\newcommand{\biberror}[1]{{\color{red}#1}}

\newcommand{\osenovaitem}{--~} 
  %% hyphenation points for line breaks
%% Normally, automatic hyphenation in LaTeX is very good
%% If a word is mis-hyphenated, add it to this file
%%
%% add information to TeX file before \begin{document} with:
%% %% hyphenation points for line breaks
%% Normally, automatic hyphenation in LaTeX is very good
%% If a word is mis-hyphenated, add it to this file
%%
%% add information to TeX file before \begin{document} with:
%% %% hyphenation points for line breaks
%% Normally, automatic hyphenation in LaTeX is very good
%% If a word is mis-hyphenated, add it to this file
%%
%% add information to TeX file before \begin{document} with:
%% \include{localhyphenation}
\hyphenation{
    Beck-man
    Ngu-yen
    back-chan-nel
    back-chan-nels
    mo-not-o-nous
    ste-reo-typ-i-cal
}

\hyphenation{
    Beck-man
    Ngu-yen
    back-chan-nel
    back-chan-nels
    mo-not-o-nous
    ste-reo-typ-i-cal
}

\hyphenation{
    Beck-man
    Ngu-yen
    back-chan-nel
    back-chan-nels
    mo-not-o-nous
    ste-reo-typ-i-cal
}
 
  \togglepaper[9]%%chapternumber
}{}

\begin{document}
\maketitle 



\section{Introduction}\label{sec:marques:1}

Nominal structures that show signs of a process of grammaticalization (cases like \textit{a gente} ‘us’, \textit{o senhor} ‘Mr/you’, \textit{vossa excelência} ‘your excellence/your lordship/your grace’), changing from noun to pronoun, are a linguistic phenomenon that has been studied extensively in European Portuguese (EP) and Brazilian Portuguese (BP) (see, e.g, \citealt{Cintra1972}, \citealt{Nascimento1987}, \citealt{Cook1994}, \citealt{Faraco1996}, \citealt{Pereira2003}, \citealt{Sória2013}, among many others). Some of these processes certainly do not exhibit the same degree of grammaticalization in the two stabilized varieties of the Portuguese language. Although some of these structures in BP, such as \textit{a gente} ‘us’, have been the focus of more studies than in EP, this does not mean that identical processes have not also occurred in EP, and several studies have been published on the topic (\citealt{Nascimento1987}, \citealt{Pereira2003}, \citealt{Posio2021}, among others). 



In the case of EP, although this issue has been addressed several times, a comparison of the proposed theoretical explanations of the form \textit{o senhor} has yet to be accomplished, nor has its usage in different discourse contexts been analysed in order to identify the changes it has experienced. It is, therefore, necessary to revive and broaden the discussion, paying particular attention to the actual uses of \textit{o senhor}. 



In this chapter, we aim to gain new insights into the categorization and uses of the noun phrase (NP) \textit{o senhor}, as well as its linguistic and discourse status.\footnote{We will not analyse the morphosyntactic variants \textit{senhor/senhora}, \textit{senhores/senhoras} ‘Mr\slash Mrs\slash Ms, sir\slash madam, gentlemen\slash ladies’ according to gender and number categories. In fact, these variations are more complex and involve the morpho-phonetic, syntactic, semantic and pragmatic levels that would require a different approach.} As a form of address (FA), we intend to show that \textit{o senhor} is a hybrid form, whose usage is at times closer to nominal forms of address (NFA) and at others closer to pronominal forms of address (PFA).  



Our analysis is based on a qualitative approach, due to the theoretical need to consider the linguistic and extralinguistic contexts (situational, social, experiential) in the description and explanation of how the object under study works in discourse. This approach will be complemented by a quantitative examination of the occurrences recorded in the corpus and an analysis of the results.



This chapter is organised as follows. After this brief Introduction (\sectref{sec:marques:1}), the state of the art is presented in \sectref{sec:marques:2}, starting with an overview of the address system in Contemporary European Portuguese (\sectref{sec:marques:2.1}), followed by a review of the main theories and previous studies on \textit{o senhor} (\sectref{sec:marques:2.2}). The theoretical and methodological framework on which our analysis is based is described in \sectref{sec:marques:3}. The following \sectref{sec:marques:4} is devoted to the analysis of the form of address \textit{o senhor}, taking into account also the vocative form \textit{senhor}. This is a synchronic approach, completed by a diachronic comparison. Historical data is employed to explain the current functionalities of these forms. In \sectref{sec:marques:4.1}, we present the changes the form has experienced from a diachronic perspective and, in \sectref{sec:marques:4.2}, from a synchronic perspective, its occurrence and pragmatic-discursive features in different discourse genres, namely, in sociolinguistic interviews (\sectref{sec:marques:4.2.1}), as well as in television interviews and electoral political debates (\sectref{sec:marques:4.2.2}). The next section focuses on \textit{o senhor} as the only form of address (\sectref{sec:marques:4.3}) and the process of phonetic erosion that is currently taking place, namely in the terms \textit{sô, sor, se,} and \textit{sotor} (\sectref{sec:marques:4.4}). This section ends with a discussion of the results (\sectref{sec:marques:4.5}).  In \sectref{sec:marques:5}, we present the final considerations on the issues analysed, suggesting avenues for future research.


\section{The address system in European Portuguese: State of the art}\label{sec:marques:2}
\subsection{Proposals to categorize the forms of address in European Portuguese}\label{sec:marques:2.1}

The forms of address (FA) are a pragmatic category, central to the description and analysis of discourse organization, especially in the construction of interpersonal relations. In other words, they are “relationèmes” \parencites[37]{Kerbrat-Orecchioni1992}[8]{Kerbrat-Orecchioni2010}.  The importance of the role they play is evident in the bibliography on the topic, which encompasses a wide range of theoretical frameworks (\citealt{Cintra1972}, \citealt{Medeiros1985}, \citealt{Carreira1997,Carreira2004}, \citealt{Hammermüller2004}, \citealt{Duarte2010,Duarte2011}, \citealt{Kerbrat-Orecchioni2010}, and \citealt{ScherreMartins2015}).



The FA have been systematically organized \citep[8]{Kerbrat-Orecchioni2010} into two subcategories as pronominal forms of address (PFA) and nominal forms of address (NFA), taking into account the lexical or pronominal nature of their members, as well as their different semantic and pragmatic features and functions. Amongst these, there are the semantic-pragmatic functions performed by their constituent elements, namely a deictic (personal) pronominal function and a nominal definitory function concerning the identity and other features of the addressee, prototypically performed by PFA and NFA, respectively.\footnote{According to \citet{Johnen2014}, “La distinction entre «déictique» et «définitoire» a été introduite par \citet[114–120]{Bühler1982}, faisant lui-même référence à Apollonius Dyscole, pour saisir la différence entre pronoms (dont la fonction est déictique, car ils font partie du «système d’orientation subjective ici-maintenant-moi», \citealt[149]{Bühler1982}) et substantifs (dont la fonction est définitoire, car ils caractérisent sémantiquement leurs référents)”. “The distinction between ‘deictic' and ‘defining' was introduced by \citet[114–120]{Bühler1982}, himself referring to Apollonius Dyscolus, to grasp the difference between pronouns (whose function is deictic, because they are part of the “subjective orientation system here-now-me”, \citealt[149]{Bühler1982}) and nouns (whose function is defining, because they semantically characterize their referents)”.}  They are complementary functions, in that the deictic function is prototypical of the FA category as a whole and the identifying function may be present in varying degrees of prominence. However, as individualizing prototypical features of each category, the deictic function is specific to PFA, whereas the definitory function characterizes the NFA category (\citealt{Carreira1997,Carreira2004,Carreira2007,Kerbrat-Orecchioni2014}).



For European Portuguese, we highlight three major theoretical proposals. The first is the work of \citet{Cintra1972}, who identifies three FA categories. This is a particularity of the Portuguese address system in comparison to other Romance languages like French (\citealt{NascimentoMendesDuarte2018}, \citealt{DuarteMarquesForthcoming}). Diverging from the dichotomous model of \citet{BrownGilman1960}, and taking into account the subject function of FA, Cintra analyses and divides the address system in European Portuguese into three categories, pronominal (PFA), nominal (NFA) and verbal (VFA) forms of address (see \tabref{tab:marques:1} regarding Cintra’s tripartite morphosyntactic categorization). They are organized according to an individual or collective addressee and the interpersonal relations established, of greater or lesser intimacy, or of greater or lesser hierarchy and deference. The VFA, constituted only by the verbal form, marks in the 3\textsuperscript{rd} person the zero degree of deference, as a strategy of avoiding the specificities that govern the FA choices for EP speakers (\citealt{Carreira1997}, \citealt{Hammermüller2004}, \citealt{DuarteMarquesForthcoming}). 



\tabref{tab:marques:1}, Cintra’s tripartite morphosyntactic model, shows the complexity of the FAs in Portuguese, whose translation into English is reduced, in almost all cases, to the use of the form “you” (formal or informal), such as: \textit{Quer?/} Do you want?; \textit{O senhor quer?/} Do you want?; \textit{O António quer?} Do you want?; \textit{Queres?} Do you want? and so on.


\begin{table}
\begin{tabularx}{\textwidth}{XXX}

\lsptoprule

{ Pronominal FA} & { Nominal FA} & { Verbal FA}\\
\midrule
{\textit{tu (\textbf{Tu} queres?)}}

{\textit{vós (Vós quereis?)}}

{\textit{você (Você quer?)}}

{\textit{vocês (\textbf{Vocês} querem?)}}

{\textit{Vossa Excelência (\textbf{V. Ex.ª} quer?)}}

{\textit{Vossas Excelências (V.\textsuperscript{as}} \textit{Ex.\textsuperscript{as}} \textit{querem?}} & {\textit{O senhor, a senhora, os senhores, as senhoras (\textbf{o senhor} quer?)}}

{\textit{O senhor Dr., o senhor Ministro (\textbf{o senhor Dr.} quer?)}}

{\textit{O pai, a mãe, o avô (\textbf{o pai} quer?)}}

{\textit{O António, a Maria (\textbf{A Maria} quer?)}}

{\textit{O meu amigo, o patrão (\textbf{o meu amigo} quer?)}} & { \textit{Quer? Querem?}}\\
\lspbottomrule
\end{tabularx}
\caption{Cintra’s tripartite morphosyntactic categorization of EP \citep[11-12]{Cintra1972}}
\label{tab:marques:1}
\end{table}


In addition to this morphosyntactic categorization, the NFA category is broad and complex \citep[7]{Kerbrat-Orecchioni2010}, encompassing not only the vocative uses \REF{ex:marques:1} that are characteristic of English and French NFA, for instance \citep{Formentelli2009}, but also uses with the syntactic functions of subject and complement \REF{ex:marques:2}, exclusive to PFA in these two languages.


\ea\label{ex:marques:1}
 \gll Ana,  trouxe o livro?\\
        Ana \textsc{pst}.\GlossMarkup{3SG}   the  book?\\
\glt ‘Ana, did you bring the book?’\\
\ex\label{ex:marques:2}
 \gll A              Ana      trouxe      o livro?\\
        \textsc{art.def.f.sg}    Ana.\textsc{sbj}  \textsc{pst.3sg}      the book?\\
\glt ‘Did you bring the book?’
\z 


In turn, although maintaining Cintra’s tripartite morphosyntactic categorization, \citet{Carreira1997,Carreira2004,Carreira2007} proposes a verbal proxemic criterion that organizes the address forms into a continuum from proximity to social distance. She also develops a new definition of the address system in EP, which integrates locutive and delocutive forms, besides the traditional allocutive forms, concerning, respectively, the designative forms of the speaker and of others as objects of discourse \citep{Carreira1997}. 



Finally, Medeiros/Oliveira (\citeyear{Medeiros1985, Medeiros1992, Oliveira2004})\footnote{\textit{Medeiros} and \textit{Oliveira} refer to the same author.} is noteworthy in her reorganization of Cintra’s categories into \textit{pure pronouns, pro-pronouns} and \textit{zero forms}. The researcher brings to the discussion the sociolinguistic categories of power and solidarity established by \citet{BrownGilman1960}, aggregated to informal and formal forms of address (T/V, according to the Latin system), to propose a more comprehensive, psycho-sociolinguistic model of the forms of address. According to the author, Brown and Gilman’s theoretical model is only applicable to “conventionalized forms of address” and, therefore, unable to explain the complexity of the phenomenon, contrary to the model she proposes \citep[340]{Medeiros1992}.



Noting that there is a process of negotiation of address forms in verbal interactions, \citet[335/338]{Medeiros1992} emphasizes how idiosyncratic issues determine the choices made by speakers. In fact, she proposes a model that takes into account the contextual variability based on the idiosyncrasies of the participants in the interaction. Rooted in the concept of negotiation, her theoretical proposal is a fundamental contribution to the study of the FA, which forces us to consider an experiential, idiosyncratic dimension that governs the speakers’ choices. \citet{Formentelli2009} came to the same conclusion, after investigating the forms of address in an academic environment.


\subsection{Previous studies on the categorization of the form of address \textit{(o) senhor} in contemporary European Portuguese}\label{sec:marques:2.2}

In addition to the above{}-mentioned individualized categories according to their lexical, pronominal, or verbal nature, researchers have identified other features that underlie new categories, namely the syntactic distribution of the form of address. Thus, for instance, in relation to the English address system, \citet[182]{Formentelli2009} identifies NFA with the vocative category (which has its own intonational profile), highlighting the high productivity rate of this open category, with a very free distribution within the utterance. However, this transposition to EP raises some theoretical questions. The NFA in EP do not present the same syntactic restrictions as in English or in French. This is one of the reasons why the form \textit{o senhor} is subject to different categorizations.



As an address form, \textit{(o)} \textit{senhor} is a general appellative (\ref{ex:marques:3}–\ref{ex:marques:7}), which may occur aggregated to other forms of nominal address (anthroponyms, such as first/last name) (\ref{ex:marques:3}–\ref{ex:marques:4}); functional appellatives \REF{ex:marques:5}, such as academic titles, professional titles, positions, etc., and as the NP \textit{o senhor} \REF{ex:marques:7}, and as a single form or combined with the aforementioned nominal forms \REF{ex:marques:6}:


\ea\label{ex:marques:3}
 \gll Bem, \ExHighlight{senhor}  \ExHighlight{Vicente},    {ficamos por aqui.}\\
        Well, Mr.    Vicente,    {we can end here.}\\
\glt ‘Well, Mr. Vicente,  I think we can end here.’ [PSFB interview 12]\textsuperscript{}\footnote{These examples are taken from the corpora analysed (see \sectref{sec:marques:4.2}).}\\
\ex\label{ex:marques:4}
 \gll Olhe,  \ExHighlight{sr.}   \ExHighlight{Machado},    {acha que}   em Lisboa   se fala       {da mesma forma} que em Braga?\\
        Look, Mr.   Machado,    \textsc{prt.3sg}   in Lisbon   people speak   {in the same way} as   in Braga?\\
\glt ‘Look, Mr. Machado, do you think people in Lisbon speak in the same way as in Braga?’ [PSFB interview 22]\\
\ex\label{ex:marques:5}
\gll Eu?!  \ExHighlight{Ó}      \ExHighlight{senhor}   \ExHighlight{engenheiro}…\\
        Me?!  \textsc{voc}    Mr    Engineer…\\
\glt ‘Me?! Really, Mr Engineer…’ [debate, 2011]
\ex\label{ex:marques:6}
\textit{- Não foi o Partido Socialista. Foi o Eurostat.}\\
    \glt ‘It wasn’t the Socialist Party. It was Eurostat.’\\
\gll {- Foi}        \ExHighlight{o}              \ExHighlight{senhor}    \ExHighlight{engenheiro}    \ExHighlight{José Sócrates}…  \\
        \textsc{pst.3sg}  \textsc{art-def-m-sg}   Mr    Engineer    {José Sócrates}… \\
\glt ‘It was you (the Mr Engineer José Sócrates)...’ [debate, 2005] 
\ex\label{ex:marques:7}
\gll Se    \ExHighlight{o}              \ExHighlight{senhor}  {for eleito}    {o que é que}    \ExHighlight{o}                  \ExHighlight{senhor}    fará?\\
        If    \textsc{art-def-m-sg}    Mr    \textsc{pass.3sg}  what   \textsc{art-def-m-sg}   Mr         {will do}?\\
\glt ‘If you are elected, what will you do?’ [interview, 2010]
\z 


The address form \textit{(o) senhor} occurs as a vocative in \REF{ex:marques:3}, \REF{ex:marques:4} and \REF{ex:marques:5}, but also with an allocutive function in the syntactic position of subject in \REF{ex:marques:6} and \REF{ex:marques:7}, a trait associated with the address pronouns in English and French (\citealt{Formentelli2009, Kerbrat-Orecchioni2010, Johnen2014}).



It is on the basis of these classifications and the discussions generated around the status and features of the NFA and PFA that we should examine the categorizations proposed for the FA \textit{o senhor}. First, the distinction of syntactic contexts, which characterizes the functionalities of the FA (nominal and pronominal) in English and serves to distinguish between the PFA category and the syntactic function of the pronoun, is not valid for Portuguese (see \citealt[376]{Johnen2014}). In some Romance languages, including Portuguese, third person forms are used in allocution to refer to the second person, which Pountain calls a “third-as-second person form” \citep[149–150]{Pountain2003}. \citet{HeineSong2011} draw attention to other languages where the same phenomenon occurs. As Pountain rightly points out, this structural possibility leads to a very open pronominal system. In EP, third-as-second person forms can occur with either a delocutive or allocutive function.  In these contexts, forms that can occur both with allocutive and delocutive values intersect. The immediate context and the global context are essential for the disambiguation of the function. So, \textit{(O) senhor}, as a FA, is distinct from the uses of \textit{(el) señor} in Spanish, a language that otherwise shares many affinities with Portuguese. \citet[264]{CastilloLluch2014} is adamant when she states that the form \textit{(el) señor} in its allocutive function does not occur with a definite article, contrarily to its delocutive use.



The tripartite categorization of the FA in EP is accepted by researchers, despite minor differences in the delimitation and designation of each category. However, the inclusion of the FA \textit{o senhor} is more problematic, as we will see next (\sectref{sec:marques:4}). In fact, due to its traits, this FA shows the porosity of the categories, the continuum of values, and can, therefore, be framed in different categories.



\citet[292]{CelsoCunhaandCintraCintra1984} include the FA \textit{o senhor} in the category of address pronouns, which also includes the forms \textit{você}, \textit{vossa excelência} or \textit{vossa senhoria} ‘you, your grace, or your lordship’. The category is constituted by “... certain words and locutions that are equivalent to actual personal pronouns, such as \textit{você, o senhor, Vossa Excelência}”. Note that Cintra, in an earlier work (\citeyear{Cintra1972}), places \textit{o senhor} in the nominal address forms (saying that \textit{o senhor} and \textit{a senhora} are the most pronominalized of these forms, \citealt[12]{Cintra1972}). \citet{Medeiros1985} considers \textit{o senhor} a pro-pronoun, but \textit{você} a pure pronoun, whereas \citet[184]{Preti2004} distinguishes two pronoun subcategories for BP, pronoun forms and pronominalized forms, and includes the FA \textit{o senhor} in the latter: “...pronominalized forms, that is, with personal pronoun value (\textit{você, o senhor, Vossa Excelência, Vossa Senhoria} and its variations)”. \citet[184]{IlariEtAl1996} also argues that the “...set of personal pronouns in Portuguese (...) includes, in the second person, \textit{o senhor}\slash\textit{a senhora}”. In \citet[900]{Raposo2013}, the author speaks of “pronominal locution”, as it consists of two elements (unlike pronouns).



Diverging from these categorizations, \citet[248]{NascimentoMendesDuarte2018} consider \textit{o senhor} a nominal form, albeit “equivalent to the 3rd person paradigm of \textit{você} (\textit{Você quer? O senhor quer?}\slash Do you [−formal] want? Do you [+formal] want?)”. \tabref{tab:marques:2} summarizes the different terms used by the authors discussed here. 

\vfill
\begin{table}[H]
\tabcolsep=.5\tabcolsep
\begin{tabularx}{\textwidth}{Qcccccc}
        \lsptoprule
		 & \multicolumn{6}{c}{Author}\\\cmidrule(lr){2-7}
		 & Cintra &  Medeiros &  Preti &  Ilari &  Nascimento & Raposo\\\midrule
		Nominal form of address & \ding{51} &   &   &   & \ding{51} &   \\
		Pronominalized form &       &   & \ding{51} &   &   &   \\
		Pronominal locution &       &   &   &   &   & \ding{51} \\
		Pro-pronoun         &  & \ding{51}  &   &   &   &   \\
		Pronoun of address  &  \ding{51} &   &  & \ding{51}  &   \\
\lspbottomrule
\end{tabularx}
\caption{Categorization of the address form \textit{o senhor}\label{tab:marques:2}}
\end{table}
\vfill\pagebreak


For all the researchers cited, on a list that is far from exhaustive, there is an identification, approximation or parallelism of the address form \textit{o senhor} to the PFA category. Although the designations may obliterate this issue, the classification of the NP \textit{o senhor} as a pronoun does not derive from a purely decontextualized grammatical classification, but rather from one of its heterogeneous discourse uses. The central conclusion to be drawn from this literature review is that \textit{o senhor} is a fuzzy category.


\section{Theoretical and methodological framework}\label{sec:marques:3}

Taking into account the contributions discussed in \sectref{sec:marques:2}, we have adopted a prag\-mat\-ic-dis\-cur\-sive theoretical approach. We focus on discourse genres, as a nuclear research concept, with an emphasis on oral verbal interactions characterized by different degrees of formality, as a central factor to consider in the analysis of the variety of address forms in EP and the contexts in which they occur \citep{Marques2014}. Besides the authors mentioned previously, the works of \citet{Kerbrat-Orecchioni1992,Kerbrat-Orecchioni2005} constitute the basis for our approach to forms of address in the construction of interpersonal relations in discourse interactions. Assuming, therefore, an interdisciplinary perspective, and in order to explain some of the uses of the NP \textit{o senhor} as a form of address, we have also used the grammaticalization theory by \citet{TraugottHeine1991}, \citet{Lehmann2015}, and subsequent developments since then, such as \citet[17]{HeineKuteva2004}, who established four criteria of grammaticalization, as follows: “(a) desemanticization (or “semantic bleaching”) – loss in meaning content, (b) extension (or context generalization) – use in new contexts, (c) decategorialization – loss in morphosyntactic properties characteristic of lexical or other less grammaticalized forms, and (d) erosion (or “phonetic reduction”) – loss in phonetic substance.” 



As mentioned previously, the address forms are sensitive to local and global contexts: thus, a qualitative analysis of the data collected is required, so as to identify the pragmatic meanings they bring to discourses. However, we will combine it with a quantitative analysis, in a complementary perspective, which will serve to show the changes of their uses over time, and the predominance of certain forms of address in a certain genre.



Having examined different oral and written discourse genres that exemplify the diversity of address forms in Portuguese, we have selected the NP \textit{(o) senhor} in its pragmatic function of allocution as the object of our analysis. Phonetic and morphosyntactic issues will also be considered in the analysis of the occurrences and uses of this form.



The main questions of this research are related to the linguistic and discourse status of the address form \textit{(o) senhor}, the semantic and pragmatic features that stand out in its use, both diachronically and synchronically, and the contexts and frequency of occurrences of this form, according to the discourse genre. In order to answer these questions, our expected results are that: (1) \textit{O senhor} is a frequent form of address in contemporary Portuguese; (2) limited by its lexical origin, the form \textit{(o) senhor} includes a pragmatic trait of respect; (3) in contemporary Portuguese, the use of the form has become widespread, marking a relationship of social distance; (4) the nominal form of address \textit{o senhor} is in a process of grammaticalization in EP.



Thus, the goals of this chapter are (1) to determine the linguistic and discursive features and uses of \textit{o senhor} from both the diachronic and synchronic perspectives, in order to (2) identify and analyse its pragmatic{}-discursive functions; (3) to identify features of use that support its classification as an address form in a process of grammaticalization towards pronominalization. To perform the analysis, we have employed data from the corpus \textit{Perfil Sociolinguístico da Fala Bracarense} (PSFB), ‘Sociolinguistic Profile of Braga Speech’, built from sociolinguistic interviews. From this corpus, consisting of 80 interviews of about 60 minutes each, we have selected interviews with male informants ($N = 36$). Taking into account the occurrences of the forms \textit{tu} and \textit{o senhor}, the interviews were grouped into two categories (those that used the FA \textit{tu} and those in which the participants used the FA \textit{o senhor}) from which four interviews per category were randomly selected. These are interviews 1, 3, 5, 7, and 12, 22, 31 and 43, respectively. 



An ad hoc corpus was also built, composed of televised political debates and interviews. They are verbal interactions that took place during the legislative elections in the early 21st century in Portugal. The corpus has about five and half hours of recordings distributed over the following interactions: three debates, held in 2005 (c. 90 minutes), 2011 (c. 45 minutes), and 2015 (c. 60 minutes); and three interviews, held in 2005 (c. 41 minutes and c. 40 minutes), and in 2009 (c. 46 minutes). Additionally, for specific issues, some data from the CETEMPúblico\footnote{To account for occurrences in the written press as a source of endorsement, we also used, for three examples, the CETEMPúblico, a journalistic corpus.} corpus and from \textit{Corpus do Português} (\citealt{DaviesFerreira2016}) were used as sources of written and diachronic data, respectively. 



The variety of data selected provides both a diachronic and synchronic perspective on the uses of \textit{(o) senhor}: firstly, written texts, from the 13th century to the present day, in the \textit{Corpus do Português} (\citealt{DaviesFerreira2016}) provide evidence of semantic and pragmatic changes; secondly, from a synchronic perspective, sociolinguistic interviews (PSFB), which cover diverse social groups, provide evidence of informal orality in interactions with strong features of colloquialism; and finally, a corpus of political interviews and debates provides evidence of a more formal register. The diachronic perspective adds data that serve to better capture the synchronic functions, i.e., there are dominant semantic traits from other time periods that persist today in certain allocutive uses, although their meaning has changed significantly over time, as we will show in the next section. According to \citet[257]{Dickey1997}, this has also happened with French \textit{monsieur}, English \textit{Mister} (from ‘master’), and German \textit{Herr}, for example. These FA no longer mean ‘older’, ‘wiser’, ‘more respectable’, as in \textit{senior}, \textit{lord} or \textit{master}, nor ‘owner’, as in \textit{landlord} or \textit{proprietor}, as they once did, nor are they used exclusively to address the nobility, although they conserve traces of deference.


\section{Analysis: The form of address \textit{o senhor} in European Portuguese}\label{sec:marques:4}
\subsection {The \textit{(o) senhor} structure from a diachronic perspective}\label{sec:marques:4.1}

A diachronic perspective on how \textit{o senhor} has shifted from NP to FA serves to contextualize the current functions of the FA and is the basis for the synchronic analysis conducted in \sectref{sec:marques:4.2}. The data were collected from the \textit{Corpus do Português} (\citealt{DaviesFerreira2016}). The noun \textit{senhor} comes from the Latin, \textit{senex\slash senior > senhor} and its distinguishing value of deference and respect comes from this sense of ‘older, wiser, more respectable’, values that different societies attribute to a generational status. The occurrences available in the \textit{Corpus do Português} (\citealt{DaviesFerreira2016}), from which all pre-twentieth century examples were collected, point to the diachronic meaning of \textit{o senhor} as a lexical item endowed with a certain content, organized into two interrelated semantic dimensions. From the 13\textsuperscript{th} to the 18\textsuperscript{th} century, \textit{senhor} occurred mainly as a proper noun, in reference to God \REF{ex:marques:8}, and as a common noun, referring to someone of the male gender in a very high social position, owner of various types of assets \REF{ex:marques:9}.\footnote{\citet{Biderman1972} gives a detailed account of the forms of address in Portuguese from a diachronic perspective.} In this case, \textit{o senhor} is synonymous of \textit{dono} ‘owner/lord’: \textit{o senhor} do lagar, \textit{o senhor} da herdade, \textit{o senhor} do preito, \textit{o senhor} da terra ‘the lord of the press, the lord of the estate, the lord of the servants, the lord of the land’, as in example \REF{ex:marques:9}, taken from the \textit{Terceira} \textit{Partida de Afonso X} (1221–1284):

\ea\label{ex:marques:8}
{E ante que chegue ao logar hu diz por que \ExHighlight{o senhor} todalas cousas cria. am dofereçer os clerigos o Olio da hûã das enpolas que dissemos que he ûtar os enfermos.} [Corpus do Português]\\
\glt ‘And before reaching the moment when they say “because the Lord creates all things”, the clerics should offer the holy oils from one of the containers and anoint the sick.’\footnote{Primeira Partida de Afonso X [{Corpus do Português}].}\\
\ex\label{ex:marques:9}
{Outrossy dizemos que se \ExHighlight{o senhor} mãdasse ao seruo conprar algûã cousa} ... [Corpus do Português]\\
\glt ‘Also, we can say that if his grace were to send the servant to buy something...’\footnote{Example \REF{ex:marques:9} is taken from the 3\textsuperscript{rd} Partida de Afonso X [Corpus do Português]. As a vocative, \textit{senhor} appears only addressed to God, as in the {Crónica General de España} of 1344: “{Ó Senhor} Jhesu Cristo, cujo he o reyno e o inperio e todos os poderios som em tuas mããos!”/“O Lord Jesus Christ, whose is the kingdom and the empire, and all the powers are in your hands!”. The use in religious discourse as a form of addressing the divinity is systematic and has continued to the present day.}\\
\z 

From the 16\textsuperscript{th} century onwards, there are frequent examples of the NP \textit{o senhor} where it is used in honorific titles, preceding the designation of professions, anthroponyms, noble titles \REF{ex:marques:10} or the proper name of a member of the upper class \REF{ex:marques:11}, as in the examples:

\ea\label{ex:marques:10}
{\ExHighlight{O senhor} Rei D. Pedro tinha um couteiro em Alcântara}… [Corpus do Português]\\
\glt ‘His Royal Highness King Dom Pedro had a gamekeeper in Alcantara...’\footnote{\textit{Vida e feitos d’el-rey Dom João Segundo}, by Garcia de Resende (1533) [Corpus do Português].}\\
\ex\label{ex:marques:11}
{E {\ExHighlight{o senhor}} Dom Alvaro yrmão do duque, E o duque e {\ExHighlight{o senhor}} Dom Jorge postos a pee cada hum de sua parte levaram a princesa.} [Corpus do Português]\\
\glt ‘And his grace Dom Alvaro brother to the duke, And the duke and his grace Dom Jorge stood up and took the princess.’
\z 


These are all delocutive uses. In fact, in the \textit{Corpus do Português} (\citealt{DaviesFerreira2016}), very few occurrences of allocution are attested before the 19\textsuperscript{th} century. The first occurrence \REF{ex:marques:12} dates from the 17\textsuperscript{th} century, in a literary text by the writer Francisco Manuel de Melo: 

\ea\label{ex:marques:12}
\gll Vá-se        já        \ExHighlight{o}              \ExHighlight{senhor}  {muito embora, que},  sendo destes senhores,        {poucas saudades nos deixará}\\
{go-}\textsc{sbj.3sg.}=\textsc{rfl.3}   {right now}  \textsc{art.def.m.sg}    sire     because,           being {one of these} gentlemen,       {you will not be missed at all}\\
\glt ‘You should leave right now, sire, because, being one of these gentlemen, you will not be missed at all.’ \emph{[Corpus do Português]}
\z 

In the 18\textsuperscript{th} century, the epistolary genre seems to create room for allocutive uses. However, once again, there are very few occurrences and not without some ambiguity between allocutive or delocutive use \REF{ex:marques:13}, given that only short excerpts are available which do not fully contextualize the uses found:


\ea\label{ex:marques:13}
{Aqui perguntaria {\ExHighlight{o Senhor}} João se os arredores de Roma serão tão lindos como os do Porto? }
\glt ‘Here would Mr João ask\slash Here would you [Mr João] ask if the surroundings of Rome are as beautiful as those of Porto?’\footnote{{Cartas do Abade António da Costa}, 1744 [Corpus do Português].}
\z 


The occurrences of \textit{o senhor} increased and reached their peak in the 19\textsuperscript{th} century, as \tabref{tab:marques:3}  shows. 


\begin{table}
\begin{tabular}{l S[table-format=4.0] S[table-format=3.2]}
\lsptoprule
Cronology  &   {Occurrences} &  {Per million}\\\midrule
 {sXIII}   &   28       &   50.82 \\
 {sXIV}    &  101       &   78.44 \\
 {sXV}     & 794        &   279.12\\ 
 {sXVI}    &  867       &   200.08\\ 
 {sXVII}   &   424      &   129.58\\ 
 {sXVIII}  &    445     &   203.25\\ 
 {sXIX}    &  4642      &   476.76\\ 
 {sXX}     &  3920      &   193.44\\
\lspbottomrule
\end{tabular}
\caption{Occurrences of \textit{o senhor}, according to Corpus do Português (\citealt{DaviesFerreira2016})\label{tab:marques:3}}
\end{table}


In the 19\textsuperscript{th} century, the form started to appear in dialogues in novels \REF{ex:marques:14} and plays, as an allocutive form of address, not exclusive to the nobility, marking a formal relationship of social distancing:


\ea\label{ex:marques:14}
\gll {O seu} amigo é um canalha!… \ExHighlight{O}              \ExHighlight{senhor}   é        {um homem de bem.} \\
        Your friend is a scoundrel!…  \textsc{art.def.m.sg}   mister       be.\textsc{prs.3sg}  {a good man}\\
\glt ‘Your friend is a scoundrel! … You’re a good man.’\footnote{\textit{Singularidades de uma rapariga loira} by Eça de Queirós [Corpus do Português]}\\
\z 

Still in the 19\textsuperscript{th} century, there are many instances of \textit{o senhor} in a complex NFA, placed before a title, as in \textit{o senhor pároco}, \textit{o senhor cónego} (15), \textit{o senhor administrador}, \textit{o senhor doutor} literally, ‘Mr Parish Priest’, ‘Mr Canon’, ‘Mr Administrator’, ‘Mr Doctor’, etc.


\ea\label{ex:marques:15}
- E \ExHighlight{o senhor cónego} toma um copinho de geleia, sim? \\
‘- And you, Mr Canon, will you have a cup of jam?’ \footnote{\textit{O Crime do Padre Amaro} by Eça de Queirós [Corpus do Português]}\\
\z 

The use of \textit{o senhor} as a form of address was reinforced in the 20\textsuperscript{th} century \citep{Biderman1972}, providing thus more relevant contexts of use. For the first time, we are able to analyse what is explicitly described as registered uses in orality. The form \textit{o senhor} is much more frequent in orality than in writing, at least judging from the \textit{Corpus do Português} (\citealt{DaviesFerreira2016}). In these cases, it is almost always used as a form of address, and it is therefore understandable that it exists in oral dialogical interactions and in the fictional dialogues that seek to reproduce them. 



In this diachronic reading of the occurrences of \textit{o senhor}, a degree of semantic bleaching is noticeable. As from the 19\textsuperscript{th} century, not only does \textit{o senhor} reveal semantic features considered exclusive to nouns, but it is also used in allocution, as a way to address male addressees with whom the speaker does not have a close relationship. The restriction of FA use only to addressing members of the nobility disappears and the appellative becomes more common, directed at a wider range of addressees, while maintaining a dimension of respect that comes from its initial use.



Interpreting the change in address forms, \citet[370]{Biderman1972} considers that \textit{o senhor} fills a void in the former system occupied by the pronoun \textit{vós} (you-\GlossMarkup{2sg.deferential}). She refers to a tripartite system of pronoun forms: \textit{tu – você}\footnote{See \citet{NascimentoMendesDuarte2018}.} – \textit{o senhor} (\citealt[373]{Biderman1972}). This position in the address system causes \textit{o senhor} to be used as a pronominalized form, without the possibility of different pronominal choice.



Despite these changes, the semantic-pragmatic values of the lexeme of positive appreciation remains:

\ea\label{ex:marques:16}
A começar pelo princípio (…) e a acabar no fim (…), Kevin é \ExHighlight{um senhor}.\\
\glt ‘Starting at the beginning (...) and finishing at the end (...), Kevin is a gentleman.’ [CETEMPúblico] \\
\z 

This positive evaluation is also a part of the new meanings of \textit{o senhor} as ‘adult male person.’

\ea\label{ex:marques:17}
O meu pai conhece \ExHighlight{um senhor} que deita fogo de artifício.\\
\glt ‘My father knows a gentleman who sets fireworks.’ [PSFB interview 4]\\
\z 

If \textit{conhece um senhor} ‘knows a gentleman’ were substituted by \textit{conhece um homem}  ‘knows a man’, this would imply a decrease in the positive valuation of the object of discourse, even though it is perfectly acceptable to say ‘knows a man’. The FA \textit{o senhor} shares this positive value associated with deference and respect, as we shall see.  


\subsection{Contexts of the occurrence of \textit{o senhor} in contemporary European Portuguese: A synchronic perspective}\label{sec:marques:4.2}

The FA \textit{o senhor} is a challenge to the classical conception of a watertight categorization. As a lexical item with semantic content, this form is also addressed to an adult male, known or unknown. As the only form (\textit{o senhor}) or occasionally followed by other nominal forms of address (\textit{o senhor + Presidente}), \textit{o senhor} occurs in contexts that, according to the syntactic criterion adopted (to distinguish nominal forms of address from pronominal forms of address), are specific to the pronominal forms of address.



As this FA can occur with an allocutive or delocutive function, the ambiguity this may create is resolved by the linguistic or situational context, as in example \REF{ex:marques:6}. In fact, this usage is only apparently delocutive. \textit{O senhor engenheiro José Sócrates ‘Mr Engineer José Sócrates}’ is the locutor’s  addressee. The ambiguity may be reinforced by linguistic mechanisms, such as repeating the 3\textsuperscript{rd} person singular pronoun, and non-verbal mechanisms, namely, by eye contact. If politicians look directly at the moderator, a reorganization of relationships among all the addressees takes place. \citet[109]{Kerbrat-Orecchioni2010} emphasizes the importance of non-verbal mechanisms of address to identify the addressee. \citet[133]{Goffman1981} also defines addressee as “(…) the one to whom the speaker addresses his visual attention”. José Sócrates in \REF{ex:marques:6} may be shown as a secondary addressee by this gesture but is nevertheless the main target of the illocutionary act of criticism (\citealt{Goffman1981}, \citealt{Maury-Rouan2005}, \citealt{Rossano2013}, \citealt{ConstantindeChanayKerbrat-Orecchioni2017}).



\subsubsection{Occurrences of \textit{o senhor} in sociolinguistic {interviews} {in} {European} {Portuguese}}\label{sec:marques:4.2.1}



In interpersonal relationships, the choice of \textit{o senhor} underlines and simultaneously constructs a formal relationship of respect and deference, in contexts where the form \textit{você} is assessed by the speakers as inappropriate. In the corpora consulted, the sociolinguistic interviews are particularly productive in terms of this type of occurrence. As we mentioned in the methodological framework, we have used the \textit{Perfil Sociolinguístico da Fala Bracarense} corpus. It is a stratified sample, according to age, gender, and education. The interviewers (E) are young university students. The interviewees or informants (I) are organized into four age brackets (15–25; 26–59, 60–75 and +75). In the forms of address adopted, the intragenerational, interpersonal relationship determined the use of the informal second person address form (\textit{tu}) \REF{ex:marques:18}. Furthermore, because they may also belong to different generations, an intergenerational relationship determined the use of 3\textsuperscript{rd} person forms, with variations between use of the verbal form (\GlossMarkup{3SG}) and the use of \textit{o senhor} \REF{ex:marques:19}. \textit{O senhor} is the most frequent form, sometimes the only one, along with the occurrence of the 3\textsuperscript{rd} person verbal form. The interactional relationship that is established is one of reciprocity of address forms combined with proximity \REF{ex:marques:18} or social distance \REF{ex:marques:19}, as in the following examples:\largerpage[2]

\ea \label{ex:marques:18}
\gll E: \ExHighlight{Tu}          se    \ExHighlight{pudesses}  viver noutro sítio...\\
        E: \textsc{pron.2sg}    if     can-\textsc{subj.imp.2sg}   \textsc{inf} anywhere else\\
        \\
\gll I: Ao fim,     sempre em frente, já    {\ExHighlight{vês}}      {a minha} escola.\\
        I: At the end,  straight ahead, {} see-{\textsc{prs.2sg}}  my school\\
\glt ‘E: If you could live anywhere else...’\\
\glt ‘I: At the end, straight ahead, you can see my school’ [PSFB interview 1]
\ex \label{ex:marques:19}
\gll E: \ExHighlight{O}             \ExHighlight{senhor}    \ExHighlight{gosta}     de     viver  aqui? \\
        E: \textsc{art.def.m.sg} mister         like-\textsc{prs.3sg} \textsc{prep}  \textsc{inf} here?\\
        \\
\gll I: Desculpe,  mas perdi {o fio à meada},     {da pergunta que}                \ExHighlight{a}             \ExHighlight{menina}  fez.\\
        I: Sorry,  {}    {I’ve lost} {my train of thought}, {regarding the question}  \textsc{art.def.f.sg}  Miss  asked.\\
\glt ‘E: Do you like living here?\\
\glt `I: I’m sorry, I’ve lost my train of thought, regarding the question you asked’ [PSFB interview 31]\\
\z 

Given the characteristics of the interview genre (and specifically, of the sociolinguistic interview genre, which is aimed at getting the informants to talk about their personal lives, experiences and opinions), we find the forms of address mainly in the interviewer’s interventions.



The results of the analysis of the interview data in terms of absolute occurrences, presented in Tables~\ref{tab:marques:4} and~\ref{tab:marques:5}, corroborate the interpersonal relationship profile presented with regard to the FA used, and show how \textit{tu} (pronoun and\slash or verb form) and \textit{o senhor} (and\slash or verb form) are in complementary distribution:

\vfill
\begin{table}[H]
\begin{tabular}{cccc}
\lsptoprule
{FA} & {\itshape tu} & {{only verb in \GlossMarkup{2SG}}} & {{total}}\\
{Interviews} & {{+ verb in \GlossMarkup{2SG}}} &  & \\
\midrule
{1} & {67} & {70} & {137}\\
{3} & {65} & {88} & {153}\\
{5} & {36} & {238} & {274}\\
{7} & {15} & {31} & {46}\\
\midrule
{Total occurrences} & {183} & {427} & {610}\\
\lspbottomrule
\end{tabular}
\caption{Forms of address of the 2\textit{\textsuperscript{nd}} person singular in sociolinguistic interviews}\label{tab:marques:4}
\end{table}
\vfill
\begin{table}[H]
\begin{tabular}{ccccc}
\lsptoprule
{FA} {Interviews} & {\itshape o senhor} & {{(\textit{o}})} {{\textit{senhor}}}  & {only verb in}  & {total}\\
& {+ verb in 3SG} & {{name}} & {3SG} & \\
& & {+ verb in 3SG} & & \\
\midrule
12 & {46} & {2} & {57} & {125}\\
22 & {37} & {6} & {106} & {149}\\
31 & {16} & {0} & {6} & {22}\\
43 & {0} & {14} & {33} & {47}\\
\midrule
{Total occurrences} & {99} & {22} & {202} & {333}\\
\lspbottomrule
\end{tabular}
\caption{Forms of address of the 3\textit{\textsuperscript{rd}} \textit{person singular in sociolinguistic interviews}}
\label{tab:marques:5}
\end{table}
\vfill\pagebreak


Given the features of the discourse genre, the findings show that it is the interviewers who mostly use the forms of address as part of question acts. In some interviews, the interviewees hardly use any form of address, or indeed none at all.



In \tabref{tab:marques:4}, the FA \textit{tu} is the only FA used together with the \GlossMarkup{2SG} verbal form, corresponding to 30\% of all verbal forms of address used in the interviews (\textit{tu} + verb in \GlossMarkup{2SG} and only verb in \GlossMarkup{2SG}). It is important to stress the Pro-Drop nature of European Portuguese in order to understand those occurrences of the verbal form. It should be also noted that interview 5 stands out for the number of occurrences of second person verbal forms. The interviewee gives short answers, which leads to the occurrence of more than three hundred question acts in the course of the 60 minutes dedicated to each interview. This points out the interviewer’s need to provoke the informant to get him to talk.



In \tabref{tab:marques:5}, the VFA (\GlossMarkup{3SG}) is still prevalent (202 occurrences), and \textit{o senhor} is the most used FA (99 occurrences against 22 occurrences of \textit{(o) senhor} + F-L name). The divergence of occurrences that stands out in interview 43 stems from the social prestige of the interviewee (parish priest). The interviewer prefers the structure \textit{(o) senhor abade} ‘mister parish priest’ to \textit{(o)} \textit{senhor} as a more deferential FA. The higher prevalence of only verbal forms (verbal address, i.e., \GlossMarkup{3SG} without an expressed subject) in this interview may also be due to the fact that the use of FA is not exclusively determined by linguistic rules, but also involves idiosyncratic features.



\subsubsection{Occurrences of \textit{o senhor} in political interviews and debates on Portuguese television}\label{sec:marques:4.2.2}\largerpage
The debates and interviews in the corpus we compiled were collected according to the established criteria, namely, occurring in the 21st century during electoral campaigns with male political participants, given the objective of analysing the occurrences and characteristics of the form of address \textit{o senhor} in these discourse genres. Three debates between leaders of the two main parties were selected, the first in 2005, between José Sócrates (JS), leader of the Socialist Party (PS), and Pedro Santana Lopes (PSL), leader of the Social Democratic Party (PSD), lasting about 90 minutes; the second in 2011, between José Sócrates, leader of PS, and Pedro Passos Coelho (PPC), leader of PSD, lasting approximately 45 minutes. The third debate was held in 2015 between Pedro Passos Coelho, leader of PSD, and António Costa (AC), leader of PS, lasting 60 minutes.



Three interviews were also selected, two in 2005, one with Pedro Santana Lopes, leader of PSD, lasting about 41 minutes, conducted by the journalists Paulo Magalhães and Manuel Carvalho, and the other with Jerónimo de Sousa, leader of the Portuguese Communist Party (PCP), lasting approximately 40 minutes and conducted by the journalists Raquel Abecassis and Eduardo Dâmaso. The third interview was held in 2019, with Rui Rio, leader of PSD, and the journalists Carlos Daniel and António Esteves, lasting about 30 minutes.



In the debates analysed, the frequency of \textit{o senhor} and other forms of address is related to the degree of interaction in each debate. In debates with high interactivity like these ones, the politician’s question and confront each other and often disagree. Interruptions, overlapping turns, and the moderator’s difficulty in controlling the course of the debate are signs of this high level of interaction. In these contexts of confrontation and combative aggressiveness, features of colloquialism are frequent. In these highly interactive debates and as we can see in \tabref{tab:marques:6}, the frequency of FA occurrences with \textit{(o)} \textit{senhor} should be taken into account in the analysis of genre features and the interpersonal relationship built. They mark a formal relationship of respect, with varying degrees of distancing, building a polite relationship, despite the pragmatic dimension of confrontation that runs through the discourse genre.


\begin{table}
\fittable{\begin{tabular}{lcccc}

\lsptoprule
& {{2005}} & {{2011}} & {{2015}} & {{Total}}\\
& {{JS – PSL}} & {{JS – PPC}} & {{PPC – AC}} & {{occurrences}}\\
\midrule
{{\itshape o senhor}} & {21} & {98} & {51} & {170}\\
{{\textit{o} {\textit{senhor}} {(Title} {/} {F-L} {Name)}}} & {2} & {31} & {38} & {71}\\
{{\itshape os senhores}} & {8} & {1} & {16} & {25}\\
{{\textit{senhor} {(Title} {/} {F-L} {Name)}}} & {2} & {61} & {77} & {140}\\
{{\itshape meus senhores}} & {2} & {0} & {0} & {2}\\
{{Total occurrences}} & {35} & {181} & {182} & {408}\\
\lspbottomrule
\end{tabular}}
\caption{Occurrences of FA with \textit{senhor} in political debates}
\label{tab:marques:6}
\end{table}


The address form \textit{o senhor} is the most frequent, with 170 occurrences of the total 408 address forms in which this term is present. Moreover, delocutive forms with allocutive value predominate (241 occurrences), although forms with a vocative function seem to be growing (from 2 occurrences in 2005 to 77 in 2015). \textit{Meus senhores} ‘my [possessive] gentlemen’ is residual, because even though the term may be used by the moderators, it seems to have fallen into disuse. We believe this abandonment is related to possile changes in the forms of address used by journalists, who seem to opt more frequently for nominal forms of address, for example, first and last names. \textit{Os senhores} (‘gentlemen’) is also a form of address that is rarely used. The fact that these are debates between party leaders directs the allocution to the individual rather than to the group to which they belong. When the attack is directed at the political group, the participants in the debate choose to mention the name of the party alongside \textit{os senhores}. The contestation of the adversary through acts of criticism and accusation are preferential contexts of occurrence of the forms \textit{o senhor} and \textit{senhor} + NFA. The numbers found suggest a significant relationship between verbal aggressiveness and the occurrence of these forms (example \ref{ex:marques:6}).{\interfootnotelinepenalty=10000\footnote{In contrast, in debates with low interaction, politicians assign the moderator the role of direct addressee, marked verbally and non-verbally. The direction of their gaze is a key indicator. The debate assumes a question-answer structure, with fewer interruptions, less overlapping, and easier turn alternations. Allocution marks are infrequent.}}



It should be noted that we consider the genre to be of central importance for the study of these topics, and that the defining traits of electoral political debates should not by any means be confused with electoral interviews. However, the latter reveal interesting similarities with debates, considering that the journalist(s) generally ask the interviewees controversial and difficult questions whose answers are similar to those they would give a political opponent. Moreover, the journalists assume they are the spokespeople of the Portuguese people, and more than just question, they may actually confront their interviewees, coming closer to the relationship of conflict inherent to a debate. 



In this corpus of interviews, there are numerous occurrences of \textit{o senhor} (\ref{ex:marques:20}--\ref{ex:marques:23}). This seems to be the interviewers’ preferred FA to address electoral candidates, besides the use of their first/last name \REF{ex:marques:20}, which may combine with \textit{Doutor} (Doctor) \REF{ex:marques:23}. Together with these occurrences, the forms \textit{sotor} \REF{ex:marques:21} and \textit{o sotor} \REF{ex:marques:22}, contracted forms of \textit{senhor doutor\slash o senhor doutor} ((the) Mr Doctor), reveal a pattern of occurrence to whose analysis we will return:

\ea\label{ex:marques:20}
\gll Jerónimo de Sousa,  \ExHighlight{o} \ExHighlight{senhor}  disse       que o euro {não trouxe} o crescimento prometido…\\
        Jerónimo de Sousa,  \textsc{art.def.m.sg} mister  say-\textsc{prt.3sg} that the euro {has not brought} the promised growth…\\
\glt ‘Jerónimo de Sousa, you [the Mr] said that the euro has not brought the promised growth...’ [interview, 2005b]\\
\ex\label{ex:marques:21}
\gll Boa noite      \ExHighlight{sotor},                  \ExHighlight{o}            \ExHighlight{senhor}  {desde há duas semanas que anda a dizer...} \\
        Good evening,    {Mr doctor [contracted form]},  \textsc{art.def.m.sg} mister    {have been saying that for two weeks now…} \\
\glt ‘Good evening, sir, you have been saying that for two weeks now…’ [interview, 2005]\\
\ex\label{ex:marques:22}
\gll {Ou seja},         \ExHighlight{o}             \ExHighlight{sotor}                  {não se sente preso por este acordo com com com o PP...}\\
        {In other words},  \textsc{art.def.m.sg}  {Mr doctor [contracted form]}  {do not feel bound to this agreement with with PP...}\\
\glt ‘In other words, you [the Mr doctor, contracted form] do not feel bound to this agreement with with PP...’ [interview, 17/02/2005]
\ex\label{ex:marques:23}
\gll \ExHighlight{Doutor Rui Rio}, {o jornal Expresso revelou hoje que há} {uma conspiração do Ministério Público com envolvimento ...} \ExHighlight{O} \ExHighlight{senhor}   acha crível uma tese com estas características?\\
         {Doctor Rui Rio}, {the Expresso newspaper revealed today that there is} {a conspiracy in the Public Prosecutor’s Ministry with the involvement ...} \textsc{art.def.m.sg}    Mr      think credible a theory with these characteristics?\\
\glt ‘Mr Rui Rio, the Expresso newspaper revealed today that there is a conspiracy in the Public Prosecutor’s Ministry with the involvement [...]. Do you [the Mr] think a theory with these characteristics is credible?’ [interview, 2019]
\z 

The FA \textit{o senhor} occurs frequently as an anaphoric resumption of a NFA with a vocative function as in \REF{ex:marques:20}, \REF{ex:marques:21} and \REF{ex:marques:23}. It occurs in these cases with the function of pronominal deixis, preceding the verb as syntactic subject. Given that Portuguese is a null-subject language, the speaker could have opted for \textit{Jerónimo de Sousa, disse que o euro não trouxe o crescimento prometido} ‘Jerónimo de Sousa, [you] said that the euro didn’t bring the promised growth’. However, there is a change at the pragmatic level, which is fundamental. The occurrence of \textit{o senhor} stresses a relationship of politeness between the speakers. There is in fact a clear difference in the degree of politeness between \textit{o senhor + V} and the exclusive use of the verb, which is less empathetic and aloof, the “zero degree of politeness” that \citet{Carreira1997} refers to.



These data show strong idiosyncratic variability which is typical of the discourse genre but maintaining always a minimal relationship of respect, to which the use of the forms of address and \textit{o senhor}, in particular, contribute. The political interview genre (as well as the electoral debate) determines a formal relationship between the speakers, but with remarkable variability. Indeed, there are idiosyncratic traits that mark the speech of the journalists, as they systematically opt for certain variations of this form. \tabref{tab:marques:7} summarizes the occurrences of address forms in the three interviews considered.


\begin{table}
\begin{tabular}{lccc}
\lsptoprule
& {{Journalists}} & {{Interviewees}} & {{Total}}\\
\midrule

{\textit{o senhor}} & {35} & {0} & {35}\\
{\textit{sotor\slash o sotor}} & {16 / 15} & {0} & {31}\\
{\textit{os senhores}} & {2} & {3} & {5}\\
{\textit{doutor} + N + last name} & {3} & {0} & {3}\\
{first name + last name} & {11} & {0} & {11}\\
\midrule
{{Total}} & {82} & {4} & {85}\\
\lspbottomrule
\end{tabular}
\caption{FA occurrences in political interviews}
\label{tab:marques:7}
\end{table}


Some conclusions can be drawn from the figures obtained: it is mostly journalists who address the politicians using the FA, which is common in this journalistic genre. They have to take the initiative to ask the questions, which is why they address the interviewee using the FA. The most frequent forms are \textit{o senhor} (35 occurrences) and \textit{sotor\slash o sotor} (with 31 occurrences). The fact that \textit{sotor\slash o sotor} are so frequent in these records suggests the standardization or conventionalization of this form. The nominal address form first/last name appears in third place in the number of occurrences, but far below the others.



In conclusion, we can say that in 50 of the 82 occurrences involving the journalists, \textit{o senhor} and \textit{sotor} are found before a verb as a syntactic subject. These findings can be related to the results highlighted in \citet{Allen2019}, about the growth in productivity of these forms by the end of the 20\textsuperscript{th} century.


From the analysis of the occurrences in the different corpora, we further conclude that the NP \textit{senhor} does not occur as a vocative, contrary to medieval uses. We are aware, however, of its use in a religious context, addressed to God, and also in children’s speech and in popular registers, to call the attention of an unknown adult. In these last two cases, it is usually accompanied by the particle \textit{ó} ‘hey’ (as in \textit{hey mister}).

\subsection{\textit{O senhor} as the only form of address}\label{sec:marques:4.3}

As a form of address, \textit{o senhor} is used alone in the utterance, marking a systematic relationship of respect with the addressee. It is frequent in interviews, whether political or sociolinguistic interviews. In the PSFB corpus, as in examples \REF{ex:marques:24} and \REF{ex:marques:25}, \textit{o senhor} is the most frequent form of address used by the interviewers, who are young women, to address the interviewees, who are male and from an older generation, regardless of their social status:


\ea\label{ex:marques:24}
\gll E    \ExHighlight{o}              \ExHighlight{senhor}  não    sabe?\\
         And  \textsc{art.def.m.sg}    mister      \textsc{neg}    know-\textsc{prs.3sg}?\\
\glt ‘And don’t you know?’ [PSFB interview 25]
\ex\label{ex:marques:25}
\gll E,    por exemplo,    acha       {que os seus filhos estão a educar os seus} netos         da mesma maneira que          \ExHighlight{o}              \ExHighlight{senhor}    os educou?\\
        And, for example,    \textsc{prt.3sg} {that your children are raising your} grandchildren the same way that              \textsc{art.def.m.sg}   Mr    raised them?\\
\glt ‘And, for example, do you think that your children are raising your grandchildren the same way that you raised them?’ [PSFB interview 22]
\z 


The prevalence of this form of address in the interactions analysed, regardless of the social group to which the addressee belongs, seems to point to a more generalized use of \textit{o senhor}. This may in turn lead to the banalization of its use as it becomes more automatized, consequently decreasing the prominence of the semantic-pragmatic feature of deference found in the \textit{Corpus do Português} (\citealt{DaviesFerreira2016}). \textit{O senhor} thus seems to move into the se\-man\-tic-prag\-mat\-ic area of the form \textit{você}. This is a shift that signals a degree of instability and plasticity of the FA, which is reflected in uses like this one: 

\ea\label{ex:marques:26}
\gll Mas  \ExHighlight{o}              \ExHighlight{senhor}  acha       importante,      por exemplo, acha      importante    \ExHighlight{vocês}      irem      à missa?\\
         But  \textsc{art.def.m.sg}   mister    find-\textsc{prs.3sg}  {important},    for example, find-\textsc{prt.3sg}  important    \textsc{ppr.2pl} go-\textsc{inf.3pl}    to mass?\\
\glt ‘But do you [+deference] think it is important, for example, do you think it is important that you [-deference]  all go to mass....? [PSFB interview 22]
\z 

The forms of address \textit{o senhor} and \textit{vocês} (see Duarte \& Marques, accepted) participate in the construction of an anaphoric chain that brings together the interpersonal values of respect in the two FA. These are scalar uses of \textit{o senhor}. There is a difference in the pragmatic values of respect and deference between the use of \textit{o senhor} and \textit{o senhor} + NFA (see \citealt{HummelLopes2020}) on the traits of respect and deference).  Not only titles, but also proper name and family name convey deference to varying degrees. Using \textit{senhor} followed by the first name, last name or full name (\textit{sr. Joaquim, senhor Silva, senhor Joaquim Silva}) is a mark of respect and establishes a growing degree of deference. In the gradation established, the example below \REF{ex:marques:27} illustrates a form of respect, but not of deference. An addressee whom the speaker addresses with \textit{senhor} + \textit{first name} is not in a high interpersonal position relative to the speaker. \textit{O senhor} marks a relationship of respect, determined by a generational criterion, but not of deference.


\ea\label{ex:marques:27}
Bem, \ExHighlight{senhor Vicente}, ficamos por aqui.\\
\glt ‘Well, Mr. Vicente, I think we can end here.’ [PSFB interview 12]\\
\z


One of the contexts for the occurrence of \textit{o senhor} is as an anaphoric resumption of an immediately preceding NFA \REF{ex:marques:28}. It is a fundamental usage to determine the semantic-pragmatic adaptability of \textit{o senhor} as a scalar form of deferential address. As a case of anaphoric retaking by coreference, the interpersonal relationship of deference created by the NFA remains unchanged.

\ea\label{ex:marques:28}
\gll \ExHighlight{Senhor} \ExHighlight{Engenheiro}  \ExHighlight{José} \ExHighlight{Sócrates}, \ExHighlight{o}            \ExHighlight{senhor}  \ExHighlight{insiste}          \ExHighlight{na} \ExHighlight{co-incineração}.\\
         Mr. Engineer      José Sócrates, \textsc{art.def.m.sg} Mr      \textsc{prt.3sg}  on co-incineration.\\
\glt ‘Mr. Engineer José Sócrates, you insist on co-incineration.’ [debate, 2005]
\z 



In pragmatic terms, recourse to the nominal address form is a discourse strategy to ‘recognize’ the others, assigning them a specific social role in the interaction, which the form \textit{o senhor} does not do. The derogatory irony of the FA used with critical intention in the example below \REF{ex:marques:29} derives from the mismatch between the chosen form of address and the social status of the public figures mentioned.


\ea\label{ex:marques:29}
Assim, já poderia marcar mais um almoço, com \ExHighlight{o senhor} Alegre; um lanche, com \ExHighlight{o senhor} Machete e mais um jantar, com \textit{o senhor} Monjardino!\\
\glt ‘Thus, you could schedule another lunch with Mr. Alegre; tea with Mr. Machete, and another dinner with Mr. Monjardino!’ [CETEMPúblico]\\
\z 

\begin{sloppypar}
These public figures are usually referred to as \textit{Manuel Alegre\slash (senhor) doutor Manuel Alegre, Rui Machete\slash (senhor) doutor Rui Machete, Carlos Monjardino\slash (senhor) doutor Carlos Monjardino}, but never referred to as \textit{Senhor Alegre}, \textit{Senhor Machete} and \textit{Senhor Monjardino}. Thus, in this example, there is a downgrading of the referents’ image reducing them to the status almost of regular people. 
\end{sloppypar}


\subsection{Phonetic contraction of the address form \textit{(o) senhor}}\label{sec:marques:4.4}

The phonetic phenomenon of erosion is frequent in oral language uses, especially in more informal contexts. The NP \textit{o senhor}, in the subject position of V and performing an allocutive function, seems to be realized shorter in duration\footnote{This is our native speakers' perception, as we do not measure the duration of the elocution. However, we consider this situation is similar to \citegen{Posio2018} finding regarding the duration of \textit{a/uma pessoa} in grammaticalized vs. non-grammaticalized uses.}  and it even appears reduced to the forms \textit{seor} or \textit{sor}. The examples \REF{ex:marques:30} and \REF{ex:marques:31} illustrate a very frequent usage in the corpus analysed:

\ea\label{ex:marques:30}
\gll Rui Rio,   aceitaria      a leitura que     \ExHighlight{o}             \ExHighlight{seor}               foi       um melhor líder nas {duas últimas semanas que nos últimos dois anos?} \\
         Rui Rio, accept-\textsc{cond.3sg} the reading that  \textsc{art.def.m.sg}  {Mr [eroded form]} be-\textsc{prt.3sg}   a better leader {in the} {last two weeks than in the last two years?}\\
\glt ‘Rui Rio, (...). Would you accept the reading that you have been a better leader in the last two weeks than in the last two years?’ (interview, 2019)\\
\ex\label{ex:marques:31}
\gll O             \ExHighlight{sor}           {tem falado}      {muito do record da carg}a fiscal...\\
         \textsc{art.def.m.sg}  {Mr [eroded form]}  {has spoken}  {a lot about the record tax} burden…\\
\glt ‘You [the Mr, contracted form] have talked a lot about the record tax burden...’  (interview, 2019)
\z 


There are other reduced forms of \textit{senhor} that are equally documented in the PSFB, like the form \textit{se} in \textit{se Joaquim}.\footnote{A variant of this form is \textit{sô}, also as a mark of informality. It has occurred in literary texts since the 19th century, as documented in the Corpus do Português, which records 65 examples.} This form can be used with both males \REF{ex:marques:32} and females \REF{ex:marques:33}, as in \textit{se} \textit{Manel}, \textit{se} \textit{Maria}, preceding the proper name in an address form that is typical of popular varieties, with a clear generational dimension of politeness, being normally used for older addressees:\largerpage[2]

\ea\label{ex:marques:32}
\gll {O meu falecido pai andava pelas ruas}    vem      aí       o                     \ExHighlight{se}             Machadinho               {cos jornais} {e tal}\\
         {My late father walked the streets} come-\textsc{prs.3sg} \textsc{adv}    \textsc{art.def.m.sg}  {Mr [eroded form]} {Last Name [diminutive form]}    {with the newspapers} {and such}”.\\
\glt ‘My late father walked the streets “Here comes Mr Machadinho with the newspapers and such’ [PSFB, interview 22]
\ex\label{ex:marques:33}
\gll Ande       \ExHighlight{se}               Joaquina, você…    {Ela era Joaquina} a avó.     Você       {não vai},      {não lhe vai bater muitas vezes…}\\
         \textsc{sbj.3sg}   {Mrs [eroded form]}  Joaquina, \textsc{pron.p}…  {She was Joaquina}   the granny. \textsc{pron.p}   \textsc{neg.fut},    {you’re not going to beat her many times...}\\
\glt ‘Come on Mrs Joaquina, you… She was Joaquina the granny. You’re not going to beat her many times...’ [PSFB interview 22]
\z


The form \textit{sotor} \REF{ex:marques:34} and \textit{o sotor} \REF{ex:marques:35} has a different status, as a phonetic contraction of the address form \textit{o} \textit{senhor doutor} ‘(the) Mr. Doctor’. It presents a different distribution from that of the eroded forms analysed above, as it hardly ever occurs with any other nominal address form:\footnote{It may occur with light-hearted purposes or as a strategy to attenuate an act of criticism, for example. It is, however, different in status from the form \textit{stor/setor}, typical of primary and secondary school students when addressing their teachers.}

\ea\label{ex:marques:34}
\gll …\ExHighlight{sotor}, ... {espero que}  \ExHighlight{o}              \ExHighlight{sotor}                        não     {me fuja a uma resposta...}\\
         Mr {doctor [contracted form]}  {I hope}    \textsc{art.def.m.sg} {Mr doctor [contracted form]} \textsc{neg}   {\textsc{sbjv.3sg} to avoid my question…}\\
\glt ‘...Sir, ... I hope you’re not trying to avoid my question...’ [interview, 2005a]\\
\ex\label{ex:marques:35}
\gll O              \ExHighlight{sotor}                  tem      uma ótima relação {com António Costa.}\\
         \textsc{art.def.m.sg}  {Mr doctor [contracted form]}  \textsc{prt.3sg}  a very good {relationship with António Costa}\\
\glt ‘You have a very good relationship with António Costa.’ [interview, 2019]
\z 


The examples of \textit{sotor} are indicative of new uses, recorded in contemporary European Portuguese. They point to different degrees of a lexicalization process of the form \textit{o senhor doutor}. They are present in our corpus of electoral interviews and debates, despite having a formal language register as a parameter of genre. This contraction occurs also in the written press, even if only incipiently to mark orality with different degrees of informality \REF{ex:marques:36} and \REF{ex:marques:37}, which the CETEMPúblico corpus records. In \REF{ex:marques:37}, the artifice of placing the word between quotation marks indicates the not yet fully conventionalized nature of the form:\largerpage[-2]

\ea\label{ex:marques:36}
\gll \ExHighlight{Ó}      \ExHighlight{sotor}                  {está aqui um médico, quer que o} chame?\\
         \textsc{voc.}  {Mr doctor [contracted form]}  {there is a doctor here, do you want} {me to call him?}\\
\glt ‘Hey mister, there is a doctor here, do you want me to call him?’ [CETEMPúblico]\\
\ex\label{ex:marques:37}
\gll \ExHighlight{O}            “\ExHighlight{sotor}”                  teve      exactamente o mesmo tipo de estupefacção que eu tive, disse a juíza.\\
         \textsc{art.de.m.sg}  {Mr doctor [contracted form]}  \textsc{pst.3sg.}  exactly the same reaction of stupefaction as I did, said the judge.\\
\glt ‘You (Mr Doctor, contracted form) had exactly the same reaction of stupefaction as I did, said the judge.’ [CETEMPúblico]
\z 


From a pragmatic point of view, this contraction, frequent in political debates and interviews, may also indicate some kind of growing informality in the social relationships, specifically between journalists and politicians. The increasing use of the graphical form points to a relatively advanced state of integration of \textit{sotor} in the lexicon of the Portuguese language. 


\subsection{Discussion of the results}\label{sec:marques:4.5}

The categories of address forms are heterogeneous and porous. Far from a watertight delimitation, they rather configure a continuum of values and functions. They share the central deictic properties of the deictic category (a prototypical trait), but differ in social values, namely in definitional capacity (of idiosyncratic, social, and discourse features of the addressee). In each nominal and pronominal category, we must thus consider more or less prototypical forms with varied functionalities.\footnote{\citet{Kerbrat-Orecchioni2014} also recognizes the possibility that the address pronouns in French can convey social and relational values, not limited to the function of personal deictics.}



The analysis we conducted of the form \textit{(o) senhor} highlights its current complexity, which we approached from a synchronic perspective, complemented by a diachronic view on the matter. The result is a clarification of the uses and functionalities of \textit{(o) senhor} today. The frequency of occurrence and the semantic-pragmatic features of \textit{(o) senhor} have changed diachronically and are still changing in contemporary EP. These changes are limited to the allocutionary uses that have accompanied changes in the Portuguese address system over time.



There are a diversity of functions and syntactic positions occupied by the forms of address \textit{senhor} and \textit{o senhor}. \textit{Senhor} occurs in the vocative position, combined with other nominal forms of address, organized on a gradation according to the features of [±] formality and [±] deference. It may also occur in this context with delocutive value, functioning as a full word. In the allocutive function, there is some fixation of the structure, as it only has this function if it occurs with the definite, male, singular article characteristic of the nominal address form category. Other categories of determiners like \textit{um senhor, este senhor, aquele senhor, certo senhor} ‘a gentleman, this gentleman, that gentleman, a certain gentleman’, etc. always have delocutive uses. In syntactic terms, \textit{o senhor} performs the function of subject or complement, like the personal pronouns. It is integrated into the sentence structure. It is also in this context that it occurs as the only form of address and may accumulate an anaphoric function of linking to a previous nominal form of address (see \REF{ex:marques:20}). In this case, the degree of deference varies from context to context, depending on the NFA, not on the form of address \textit{o senhor}. 



The data we analysed also point to different uses and frequencies of occurrence, according to the discourse genres and the idiosyncrasies of the speakers. But there are also dimensions of change regarding the semantic-pragmatic characteristics of this FA, in connection with a semantic axis from deference to respect, originating from its lexical content as a full word. As a single FA, frequent especially from the last century onwards, the NP \textit{o senhor} is experiencing a process of semantic bleaching, conveying a general relational value of respect. Therefore, it occurs in situations of varying formality. It marks a social relationship of distance with regard to the addressee, identified as a `male, adult interlocutor'. These syntactic, semantic and pragmatic particularities are accompanied by a process of phonetic erosion (see \citealt[3]{HeineKuteva2004}),\footnote{The phenomenon is very similar to what has happened to the Portuguese address pronoun \textit{você}, where there has been a change/reduction of form that accompanies semantic change and the content.}  which gave rise to the eroded forms, \textit{se, sô, sor, seor}. The contracted \textit{sotor}\slash\textit{stor} (\textit{senhor + doutor}) is one of the most widespread forms of these eroded forms, with uses that are signs of the word's integration into the Portuguese lexicon, occurring particularly in written contexts. In the data collected in the \textit{Corpus do Português} (\citealt{DaviesFerreira2016}) for current use, only the forms \textit{sotor} and \textit{stor} occur with 8 and 14 occurrences, respectively. They are also the only eroded forms that have been introduced into dictionaries.\footnote{\url{https://www.infopedia.pt/dicionarios/lingua-portuguesa/sotor} and other online dictionaries.}



According to the four criteria established by \citet[17]{HeineKuteva2004}, the uses we identified in the data analysed suggest that there is an ongoing process of grammaticalization.\footnote{We use the term grammaticalization in a broad sense, encompassing processes also called pragmaticalization or pragmatization and discursivization.} The analysis carried out reveals processes of semantic bleaching, use in new contexts, syntactic fixation, recategorization (approximation to the pronoun category) and phonetic erosion. These characteristics of \textit{(o) senhor} are related to each other, as \citet{HeineKuteva2004} remind us. The semantic change highlights a more grammatical sense, although in \textit{o senhor} as a deictic element, some part of the semantic value of the NP is maintained. In pragmatic terms, \textit{o senhor} marks social distancing (social deixis) but retains NFA marks, like the combination with the 3rd person \citep{Carreira2009} and the occurrence with the definite article, a characteristic of NFA in this context.


\section{Final considerations}\label{sec:marques:5}

This paper has focused on the different discourse contexts in which the form of address \textit{o senhor} appears in EP. We consider that there is a generalization of the uses of \textit{o senhor} which defines a respectful form of address, regardless of whether the relationship between the speakers is asymmetrical or symmetrical.



Having analysed~different data, organized~according to different discourse genres, we found that the discourse genre interferes in the speakers’ choices of the FA, but further research is required to confirm these findings. The variability in usage that we identified, however, forces us to consider that other dimensions of verbal interaction interfere in the speakers’ choices, thus, an idiosyncratic dimension should be considered in the analysis. Finally, the nominal form of address \textit{o senhor} is in a process of grammaticalization in EP. In short, \textit{o senhor} is a hybrid form of address, with uses that sometimes bring it closer to a NFA and sometimes to a PFA; there is a synchronic convergence of both categories’ features, according to different contexts and usages. While it may be premature to speak of a stabilization of the grammatical category pronoun for \textit{o senhor} in allocutive use, it is safe to say that there are uses with a pronoun function, a deictic function, even though the semantic bleaching is not finished. We thus underline the instability of these usages, some more grammaticalized than others, as in examples \REF{ex:marques:33}, (\ref{ex:marques:20}--\ref{ex:marques:21}) and \REF{ex:marques:27}.



Taking into account our preliminary expectations, the results of the present analysis show that \textit{o senhor} is a widespread form of address marking a relationship of social distance in contemporary Portuguese. Finally, we have identified a set of characteristics that allow us to state that as a nominal form of address, \textit{o senhor} is in a process of grammaticalization in EP.



Some avenues for future research have become evident in the course of the analysis. We have established that the variation \textit{o senhor}\slash\textit{a senhora} deserves further investigation, as it is not limited to a mere morphosyntactic variation. The form \textit{(a) senhora} is used in specific contexts deserving more research work. Another topic that deserves future attention is the VFA category. The use of the 3\textsuperscript{rd} person singular of the verb without a subject, much more frequent in the data analysed than the NFA or PFA, is in line with the fact that EP is a null-subject language. But we also consider that it has discursive implications that have yet to be determined. As mentioned above, the duration of elocution, in previously published works, is associated with the process of grammaticalization. The analysis of the behaviour of \textit{o senhor} in this respect may bring more data to the current discussion. Finally, the functioning of this FA in different discourse genres, not only at the oral level, but also in certain written genres, like the epistolary one, deserves further analysis. Specifically, in future research, we intend to analyse the crystallized expression \textit{sim senhor}\slash\textit{sim senhora} (‘yes sir’\slash ‘yes ma’am’) which also presents unique features and functions that may bring new information on the process of grammaticalization.

\printbibliography[heading=subbibliography]

\end{document}
