\documentclass[output=paper]{langscibook}
\ChapterDOI{10.5281/zenodo.8124496}
\author{Yoselin {Henriques Pestana}\orcid{}\affiliation{Universität Zürich}}
\title[Personal uses of impersonalizing strategies]{Personal uses of impersonalizing strategies: Hybrid constructions with \textit{a gente} and \textit{se} in rural Madeiran Portuguese varieties}
\abstract{The varieties of Portuguese spoken in Madeira present a predominant use of \textit{a gente}, a grammaticalized first person plural pronoun, derived from the noun phrase ‘the people’, instead of the traditional pronoun \textit{nós}. They also exhibit constructions where \textit{a gente} cooccurs with the impersonal clitic \textit{se}. In a pioneering study, \citet{Martins2009} provides a detailed description of what she calls “double subject impersonal constructions” and proposes that \textit{a gente} restricts the generic interpretation of the clitic \textit{se}. Based on spoken data from semi-directed interviews and free-speech conversations with elderly speakers of rural Madeiran Portuguese, this chapter provides a quantitative and qualitative approach to the [(\textit{a gente})\,+\,\textit{se}] construction. The goal of this study is twofold. First, a depiction of the broad referential range of this hybrid structure is presented. Its possible interpretations cover a scope similar to that of first person plural pronouns reaching from indefinite readings to deictic ones (referring to participants of the speech act). Second, a description of the syntactic features of this innovative construction will show that the element \textit{se} is being reanalyzed as a dependent person marker in rural Madeiran Portuguese varieties.}
\IfFileExists{../localcommands.tex}{
  \addbibresource{../localbibliography.bib}
  \usepackage{langsci-optional}
\usepackage{langsci-gb4e}
\usepackage{langsci-lgr}

\usepackage{listings}
\lstset{basicstyle=\ttfamily,tabsize=2,breaklines=true}

%added by author
% \usepackage{tipa}
\usepackage{multirow}
\graphicspath{{figures/}}
\usepackage{langsci-branding}

  
\newcommand{\sent}{\enumsentence}
\newcommand{\sents}{\eenumsentence}
\let\citeasnoun\citet

\renewcommand{\lsCoverTitleFont}[1]{\sffamily\addfontfeatures{Scale=MatchUppercase}\fontsize{44pt}{16mm}\selectfont #1}
   
  %% hyphenation points for line breaks
%% Normally, automatic hyphenation in LaTeX is very good
%% If a word is mis-hyphenated, add it to this file
%%
%% add information to TeX file before \begin{document} with:
%% %% hyphenation points for line breaks
%% Normally, automatic hyphenation in LaTeX is very good
%% If a word is mis-hyphenated, add it to this file
%%
%% add information to TeX file before \begin{document} with:
%% %% hyphenation points for line breaks
%% Normally, automatic hyphenation in LaTeX is very good
%% If a word is mis-hyphenated, add it to this file
%%
%% add information to TeX file before \begin{document} with:
%% \include{localhyphenation}
\hyphenation{
affri-ca-te
affri-ca-tes
an-no-tated
com-ple-ments
com-po-si-tio-na-li-ty
non-com-po-si-tio-na-li-ty
Gon-zá-lez
out-side
Ri-chárd
se-man-tics
STREU-SLE
Tie-de-mann
}
\hyphenation{
affri-ca-te
affri-ca-tes
an-no-tated
com-ple-ments
com-po-si-tio-na-li-ty
non-com-po-si-tio-na-li-ty
Gon-zá-lez
out-side
Ri-chárd
se-man-tics
STREU-SLE
Tie-de-mann
}
\hyphenation{
affri-ca-te
affri-ca-tes
an-no-tated
com-ple-ments
com-po-si-tio-na-li-ty
non-com-po-si-tio-na-li-ty
Gon-zá-lez
out-side
Ri-chárd
se-man-tics
STREU-SLE
Tie-de-mann
} 
  \togglepaper[6]%%chapternumber
}{}

\begin{document}
\AffiliationsWithoutIndexing{}
\maketitle

\section{Introduction}

In some Portuguese varieties, the traditional first person plural (henceforth \GlossMarkup{1PL}) pronoun \textit{nós}, illustrated in example \REF{ex:henriques:1}, coexists with a newer \GlossMarkup{1PL} pronoun \textit{a gente,} as shown in example \REF{ex:henriques:2}. The latter originates from a noun phrase consisting of the definite feminine article \textit{a} (‘the’) and the generic noun \textit{gente} (‘people’). By losing its nominal properties, this noun phrase gave rise to an indefinite pronoun \textit{a gente} displaying generic readings (\citealt{Lopes2007,Lopes2003,Lopes1999, deOmena2003}, among others). This grammaticalization process first resulted in a referential shift from speaker-exclusive readings to speaker-inclusive ones. From the 19th century onward, \textit{a gente} became a new \GlossMarkup{1PL} pronoun with specific interpretation and has even replaced the pronoun \textit{nós} in some Portuguese varieties. The nominal origin and gradual grammaticalization of \textit{a gente} led to a mismatch between its semantic and syntactic properties resulting in mixed agreement patterns (cf. \sectref{sec:henriques:3}).\footnote{The glossing of the language examples follows the Leipzig Glossing Rules.}

\ea\label{ex:henriques:1}
 \gll Nós, nessa altura, não tínhamos luz. \\
      \textsc{pron.sbj.1pl} in-that time \textsc{neg} have-\textsc{ipfv.1pl} electricity \\
 \glt ‘Back then, we did not have electricity.’
\ex\label{ex:henriques:2}
 \gll Sabe onde é que {a gente} vai dar com ele?\\ 
      know.\textsc{prs.3sg} where is that \textsc{pron.sbj.1pl} go.\textsc{prs.3sg} give-\textsc{inf} with him\\ 
  \glt `Do you know where we find him?'
\z
 

In rural Madeiran Portuguese (hereafter MP) varieties, the \GlossMarkup{1PL} pronoun \textit{a gente} seems to have largely replaced the canonical \textit{nós.} However, there is another variant to these pronominal \GlossMarkup{1PL} expressions illustrated in \REF{ex:henriques:3}. In these constructions, the \GlossMarkup{1PL} pronoun \textit{a gente} cooccurs with what seems to be an impersonal \textit{se} (henceforth \textit{se}-\textsc{imp}\footnote{The label \textit{se}-\textsc{imp} is used in this chapter as an umbrella term for impersonal and passive \textit{se} constructions. Note that all examples found in the corpus are instances of non-agreeing and thus impersonal \textit{se} constructions.}), found in most Romance languages.

\ea\label{ex:henriques:3}
 \gll {A gente} contava-se os dias [...]\\
         \textsc{pron.sbj.1pl} count-\textsc{ipfv.3sg}=SE the days\\
 \glt `We counted the days […]'
 \z 

The element \textit{se} appears cliticized to the verb, whereas \textit{a gente} is identifiable as the subject. I will call it “hybrid construction” due to its nature of combining a personal pronoun with an impersonal marker. The qualitative analysis shows that these constructions present a referential scope reaching from indefinite to deictic readings which is thus congruous with the range of interpretations of the other \GlossMarkup{1PL} pronouns \textit{nós} and \textit{a gente} available in the varieties under study.\footnote{The fact that the pronoun \textit{a gente} is still frowned upon in the context of school education contributes to the existence of discrepancies on the stylistic level. Thus, the pronoun \textit{nós} is perceived as corresponding to more formal contexts by speakers with higher educational levels. A similar tendency is found in French in the use of \textit{nous} as opposed to the newer \GlossMarkup{1PL} pronoun \textit{on} \citep{Coveney2000}. Future research on MP based on corpora displaying different degrees of formality will allow to confirm or rule out this trend.}



So far, very little attention has been paid to these constructions, except for the research by \citet{Martins2003, Martins2005, Martins2009}. Based on the assumption that the two elements share subject features, the author refers to them as “impersonal subject doubling constructions” \citep{Martins2009}. However, the data analyzed in the present study bring to fore other aspects of the referential properties of these constructions. Thus, they provide evidence that these constructions may display interpretations that go far beyond the speaker-inclusive impersonality thoroughly described in \citet{Martins2009}, as illustrated in the example \REF{ex:henriques:4}.


\ea \label{ex:henriques:4} 
\gll Amanhã {a gente} vai-se limpar o escritório.\\
     Tomorrow \textsc{pron.sbj.1pl} go-\textsc{prs.}3\textsc{sg=SE} clean-\textsc{inf} the office\\
\glt ‘Tomorrow, we will clean the office.’
\z

In light of these new insights, this chapter aims to review the referential and syntactic properties of the hybrid construction. Its particular properties are derived under the hypothesis that \textit{se} might be reanalyzed as a person-marking item associated with the \GlossMarkup{1PL} pronoun \textit{a gente} in rural MP varieties. 

The present study is structured as follows: \sectref{sec:henriques:2} describes the data under survey. \sectref{sec:henriques:3} presents a brief overview of the variation of \GlossMarkup{1PL} pronominal subject expressions in Portuguese varieties. \sectref{sec:henriques:4} describes the referential range of the construction under focus and its constituting elements in Portuguese. \sectref{sec:henriques:5} analyzes some of the syntactic features the hybrid construction [(\textit{a gente})+\textit{se}] displays. Lastly, \sectref{sec:henriques:6} provides some conclusions along with observations for future research.

\section{Data and methodology}\label{sec:henriques:2}

Portuguese dialectology grapples with a scarcity of corpora and data that impedes a detailed and thorough analysis of the numerous morphosyntactic phenomena still little-known to linguists. So far, dialectal studies on Madeiran varieties have used either the dialectal corpus \textit{CORDIAL-SIN} (e.g., \citealt{Martins2021}, among others) or the \textit{Corpus de Concordância} (e.g., \citealt{Bazenga2019,Bazenga2015}) as their primary data sources. However, apart from presenting only few excerpts of spontaneous speech of the dialect under analysis, the \textit{CORDIAL-SIN} raises other problems, such as the communicative asymmetry between interviewer  -- a speaker of the standard Portuguese variety --  and a local informant, which results in auto-correction and neutralization of certain phenomena. Furthermore, the thematic domains addressed in the spontaneous and semi-directed speech samples of the available corpora  -- i.e., local traditions and customs --  do not enable a broad referential scope such as the one displayed by the hybrid construction under discussion. Therefore, we compiled a corpus composed of two sub-samples  -- semi-directed sociolinguistic interviews and free conversation samples --  to tackle these shortcomings.\footnote{The corpus analyzed in this study is, in comparison to the available corpora, in many respects broader in coverage. Firstly, it includes speech samples from different rural sites of Madeira Island, whereas the \textit{Corpus the Concordância} focuses exclusively on the variety spoken in the capital city of the island, Funchal. The \textit{CORDIAL-SIN} includes data from four localities, two of which are situated on the archipelago’s main island: Câmara de Lobos and Caniçal. Secondly, the sub-sample of semi-directed interviews contains, in addition to questions used in classic dialectological interviews, topics which allude to the personal and family lives of the informants. The inclusion of such topics enables the mention of more delimited groups. Thirdly, the sub-sample of free conversations between family members, neighbors or friends not only promotes the occurrence of \GlossMarkup{1PL} expressions with deictic interpretations, but also presents a highly natural communicative environment. Moreover, the fact that the interviewer is perceived by the informants as a member of their speech community, also contributes to counteracting the asymmetrical situation of sociolinguistic interviews.} 



The sociolinguistic profiles of the speakers chosen as informants roughly meet the standards introduced by \citet{ChambersTrudgill1980}, also known by the acronym NORM (“non-mobile, older, rural, males”), except for their gender.\footnote{Our corpus consists of language produced predominantly by female speakers. This is due to the extremely limited number of local male speakers of the target age group who have not emigrated for an extended period.} The informants have a low level of education ranging from zero to four years of primary school and represent an age scale ranging from 54 to 84.



Part of the data stems from 13 semi-directed interviews in different rural sites across Madeira Island.\footnote{The interviews were conducted in Estreito de Câmara de Lobos, Câmara de Lobos, Curral das Freiras, Maroços, Canhas, Santa, Tabúa, Campanário, Camacha (Santa Cruz), Santo António (Santana), and São Vicente.} To increase the occurrence of constructions displaying the broadest possible referential range, our data include classic dialectological interview questions and questions concerning the social environment and private lives of the informants. The latter category proved to be crucial for triggering more specific uses of the hybrid construction under analysis. Furthermore, although most of the interviews were conducted with one interviewer and one informant, the inclusion of several participants produced occurrences of completely deictic expressions, i.e., constructions that refer exclusively to the speech-act participants.



In addition to the 13 semi-directed interviews, the second part of our corpus contains free conversation samples. These latter samples complement the above interviews in two different aspects. Although the interviewer was present during the recording, by taking place in a familial context, the free conversations yield “immediate” speech samples (following \citealt{KochOesterreicher1990}). These circumstances are invaluable for the study of morphosyntactic variation. Secondly, since the discourse participants are familiar with each other and therefore share social networks, the free conversation samples include several specific uses of the hybrid construction as the informants often produce utterances that refer to particular groups to which they belong. This type of use is less common in semi-directed interviews in which speech act participants do not know each other. Conversely, informants of semi-directed interviews provide more clues so that the addressee correctly infers the intended \GlossMarkup{1PL} reference, which is a complex and challenging task for linguistic analysis.



In total, the data under survey contain 827 examples of \GlossMarkup{1PL} subjects. For the analysis, all occurrences were coded manually for an array of grammatical and referential properties, including expression or omission of the subject pronoun, verbal agreement, coreference, and referential range.


\section{First person plural (\GlossMarkup{1PL}) in Portuguese varieties}\label{sec:henriques:3}
%\label{bkm:Ref101794587}
%\label{bkm:Ref83806011}

This section provides a brief overview of the \GlossMarkup{1PL} pronominal subject expressions \textit{a gente} and \textit{nós} in Portuguese. It then describes the differences in the usage of \textit{a gente} and \textit{nós} in rural MP varieties.



Due to its nominal origins described in the introduction to this chapter, \textit{a gente} presents discrepancies between its semantic and syntactic properties, leading to varying agreement patterns in terms of verbal inflection, as can be observed in the contrast between examples \REF{ex:henriques:5} and \REF{ex:henriques:6}. Likewise, \textit{a gente} triggers varying adjectival (or participial) agreement as shown in example \REF{ex:henriques:7} (cf. \citealt{Pereira2003, CostaPereira2013,CostaPereira2005}). 

\ea\label{ex:henriques:5}
 \gll {A gente} fizemos uma fogueira.\\
         \textsc{pron.sbj.1pl} make-\textsc{pst.1pl} a bonfire\\
 \glt `We made a bonfire.'
\ex\label{ex:henriques:6}
 \gll {A gente} ia lavar a roupa aqui.\\
         \textsc{pron.sbj.1pl} go.\textsc{ipfv.3sg} wash-\textsc{inf} the clothes here\\
 \glt `We came here to wash our clothes.'
\ex\label{ex:henriques:7}
 \gll {A gente} era pequenos.\\
         \textsc{pron.sbj.1pl} be.\textsc{ipfv.3sg} small-\textsc{masc.pl}\\
 \glt `We were little.'
 \z 

A considerable number of studies show variation between \textit{a gente} and \textit{nós} in different varieties of Brazilian Portuguese (BP) (cf. \citealt{Vianna2011} for an overview). It is observable that the newer pronoun \textit{a gente} “is the more productive of the forms” (\citealt{TravisSilveira2009}: 22) and thus, appears to replace \textit{nós} progressively. This canonical pronoun and its associated \GlossMarkup{1PL} verbal marking  -- the desinence {}-\textit{mos} illustrated in example \REF{ex:henriques:8} --  are subject to a restricted distribution. \citet{TravisSilveira2009} observe the retention of these morphological forms in high-frequency verbs such as \textit{ter} (‘to have’) or \textit{ser} (‘to be’) and in cohortative constructions\footnote{There are numerous denominations for these constructions. \citet{TravisSilveira2009}, for instance, use the label “hortative constructions”. Others use a more transparent terminology such as “inclusive imperative” (\citealt{DobrushinaGoussev2005}). Following \citet{Posio2012} the term “cohortatives” will be used in this chapter to refer to these constructions.}  illustrated in \REF{ex:henriques:9}. According to the authors, these are some of the few domains to which \textit{a gente} has not extended. 

\ea\label{ex:henriques:8}
 \gll onde nós vivíamos\\
         where \textsc{pron.sbj.1pl} live-\textsc{ipfv.1pl}\\
 \glt `where we lived'
\ex\label{ex:henriques:9}
  \gll Então, vamos à minha casa!\\
         so go.\textsc{prs.1pl} to-the my house\\
 \glt `So, let's go to my place!'
 \z 

Limited studies have addressed the variation of \GlossMarkup{1PL} expression in European Portuguese (EP) varieties. Contrary to the traditional belief that the newer form \textit{a gente} is commonly found in central and southern EP varieties, recent studies have shown that this pronoun exhibits a high usage rate throughout continental EP varieties. For instance, while contrasting the use of the two \GlossMarkup{1PL} pronominal variants in Brazilian and European varieties, \citet{Sória2013} found that the pronominal expression \textit{a gente} is highly productive in most of the 31 local EP varieties accounted for. The author concludes that this pronoun is not only firmly established but also most commonly used throughout the observed EP varieties \citep{Sória2013}. Similarly, \citet{Posio2012} observes a relatively high application rate of \textit{a gente} in his contrastive study on \GlossMarkup{1PL} subject expression in EP and Peninsular Spanish. Regarding the referential scope of \textit{a gente,} the author acknowledges its speaker-inclusive impersonal traits attributing this fact to its impersonal origin.


\begin{quote}
While the construction with \textit{a gente} was included in the current study as a way to create first person plural reference, examining the use of \textit{a gente} in context reveals that in the EP data it is very seldom used in contexts where only a personal (i.e. inclusive or exclusive) reference is possible. In most cases, the referential range of \textit{a gente} can be described as speaker-inclusive impersonal or allowing both impersonal and first person plural interpretations. \citep[348]{Posio2012}    
\end{quote}

As for the varieties under survey, the data suggest that the newer pronoun \textit{a gente}, found in all but three examples, has largely replaced the canonical subject pronoun \textit{nós}. Given this markedly high use rate of \textit{a gente} in the data of rural MP varieties, it can be hypothesized that this newer \GlossMarkup{1PL} pronoun is more grammaticalized in some EP varieties than previously believed (cf. \citealt{Posio2012}) and thus may occur in less restricted referential contexts. As expected, the data show divergent agreement patterns triggered by \textit{a gente}, with a clear predominance of \GlossMarkup{3SG} verb forms. \tabref{tab:henriques:1} summarizes the quantitative analysis concerning the subject-verb agreement involving \GlossMarkup{1PL} subject pronoun variants.\footnote{Although the hybrid construction [(\textit{a gente})+\textit{se}] is considered a variant of the subject pronouns \textit{a gente} and \textit{nós}, it was excluded from the quantitative analysis resumed in \tabref{tab:henriques:1}. Moreover, 21 tokens of non-finite verbs occurring with the pronoun \textit{a gente} were excluded, due to their well-known distinctive person marking behavior.}


\begin{table}
\begin{tabular}{l rrrr}

\lsptoprule
~ & \multicolumn{2}{c}{\GlossMarkup{3SG}} & \multicolumn{2}{c}{\GlossMarkup{1PL}}\\\cmidrule(lr){2-3}\cmidrule(lr){4-5}
  & $n$ & \% & $n$ & \%\\\midrule
∅ & 5 & 9.1  & 50 & 90.9 \\
\textit{a gente} & 160 & 87.9 & 22 & 12.1  \\
\textit{nós} & 0 & 0  & 3 & 100\phantom{.0} \\
\lspbottomrule
\end{tabular}
\caption{Subject and verb agreement with \GlossMarkup{1PL} pronouns (\textit{a gente} and \textit{nós}) in rural MP varieties}
\label{tab:henriques:1}
\end{table}

As shown in \tabref{tab:henriques:1}, the data under survey contain 75 examples of \GlossMarkup{1PL} verb forms, the majority of which are found in clauses lacking an overt subject. The low occurrence of \GlossMarkup{3SG} verbs without an expressed subject is due to the ambiguity of this form, referring either to a \GlossMarkup{3SG} or to a \GlossMarkup{1PL} subject associated with the pronoun \textit{a gente}. Therefore, ambiguous examples such as \REF{ex:henriques:10} were excluded from the analysis.

\ea \label{ex:henriques:10}
 \gll {A gente}$_i$ vê ele$_j$ quando passa$_{i,j}$ por ali. \\
         \textsc{pron.sbj.1pl} see.\textsc{prs.3sg} to my pass-\textsc{prs.3sg} through there\\
 \glt `We see him when (he/we) pass(es) by.'
\z 

The only occurrences of verb agreement in \GlossMarkup{3SG} lacking an overt subject considered in the analysis are those displaying coreference with a preceding \GlossMarkup{1PL} pronoun.\footnote{There is an ongoing discussion on the existence of a null subject associated with \textit{a gente.} For instance, \citet{Pereira2003} and \citet{MartinsNunes2021} state that its mixed verbal agreement patterns hinder the existence of a null subject associated with this pronoun. \Citet{Sória2013}, in the same vein, argues that the omissions of \textit{a gente} cannot be considered proper cases of null subjects due to the fact that it is only permissible in restricted syntactic contexts in which there is an overt \textit{a gente} in the immediate discourse.} Example \REF{ex:henriques:11} illustrates the coreference between a \GlossMarkup{3SG} verb form and \textit{a gente}.

\ea\label{ex:henriques:11}
 \gll Mas {a gente} não deitava aquela [carne] fora;\\ 
         But \textsc{pron.sbj.1pl} \textsc{neg} throw-\textsc{ipfv.3sg} that [meat] away\\ 
 \glt `But we did not throw that meat away;'\\ 
 \gll tirava um bocadinho e cozia.\\
         take-\textsc{ipfv.3sg} a bit and cook-\textsc{ipfv.3sg}\\
 \glt `(we) took out a little bit and (we) cooked (it).'\\
 \gll    Depois     partia               um bocadinho a cada um. \\
         Afterwards cut-\textsc{ipfv.3sg} a bit     for each one\\
 \glt `Afterwards (we) cut a little bit for each one.'\\
\z

Interestingly, despite not displaying an overt subject pronoun \textit{a gente} in the immediate co-text, example \REF{ex:henriques:12} is not ambiguous. What establishes the \GlossMarkup{1PL} reference of the second verb form \textit{tinha} is its coreferentiality with the element \textit{se} cliticized to the first \GlossMarkup{3SG} verb \textit{tinha{}-se}. Furthermore, the coordination with the \GlossMarkup{1PL} form \textit{temos} also indicates coreference between the two preceding null subject verb forms. Based on their coreference, these three verb forms, including the first verbal form bearing the element \textit{se} (\textit{tinha-se}) in example \REF{ex:henriques:12} are variants of \GlossMarkup{1PL} expression lacking an overt subject.

\ea\label{ex:henriques:12}
 \gll Tinha-se uma fonte; tinha e temos.\\
         have-\textsc{ipfv.3sg=SE} a fountain have-\textsc{ipfv.3.sg} and have-\textsc{prs.1pl}\\
 \glt `(We) used to have a fountain, (we) still have (one).'
\z 

As discussed in the introduction to this chapter, MP varieties display  -- in addition to \textit{a} \textit{gente} and \textit{nós} --  what we call a hybrid construction. This \GlossMarkup{1PL} expression consists of the newer pronoun \textit{a gente} and the originally impersonal marker \textit{se}. In this paper, we argue that \textit{a gente} takes the role of a \GlossMarkup{1PL} subject, which is omissible in these Portuguese varieties.\footnote{Due to the possible omission of \textit{a gente} the hybrid construction under analysis will be represented as “[(\textit{a gente})+\textit{se}]” in this chapter.} The clitic \textit{se} associated with this pronoun is reanalyzed as a dependent person form encoding \GlossMarkup{1PL} marking. As a result, the omission of \textit{a gente} is more permissible in the varieties under focus than in continental Portuguese varieties (cf. \citealt{Pereira2003, Posio2012}), for the clitic \textit{se} manages to disambiguate \GlossMarkup{3SG} verb forms and establish \GlossMarkup{1PL} specific reference, as illustrated in example \REF{ex:henriques:12} above and in \REF{ex:henriques:13}.

\ea\label{ex:henriques:13}
 \begin{xlist}
 \exi{A:} \gll As mulheres trabalhavam na  fazenda ou bordavam?\\
     the women work-\textsc{ipfv.3.pl} in-the field or sew-\textsc{ipfv.3pl}\\
 \glt`Did the women work in the fields or sew?'\\
 \exi{B:}
    \gll Bordava-se.\\
         sew-\textsc{ipfv.3sg}=SE\\
 \glt  `(We) sewed.'
 \end{xlist}
\z 

In terms of frequency, the data indicate that within the different forms of \GlossMarkup{1PL} expression  -- e.g., pronominal forms \textit{a gente} and \textit{nós} as opposed to the hybrid construction [(\textit{a gente})\,+\,\textit{se}] --  the latter is far more frequent than the former, as the results in \tabref{tab:henriques:2} show.

\begin{table}
\begin{tabular}{l rrrr}

\lsptoprule
~ & \multicolumn{2}{c}{\GlossMarkup{3SG}} & \multicolumn{2}{c}{\GlossMarkup{1PL}}\\\cmidrule(lr){2-3}\cmidrule(lr){4-5}
  & $n$ & \% & $n$ & \%\\\midrule
{[(\textit{a gente})+\textit{se}]} & 566 & 98.95 & 6 & 1.05\\
\textit{a gente / nós}\footnote{The pronoun \textit{nós} only appears three times in the data under study. All of these occurrences occur with \GlossMarkup{1PL} verb forms.} & 160 & 86.49 & 25 & 13.51\\

\lspbottomrule
\end{tabular}
\caption{First person plural expression in rural MP varieties}
\label{tab:henriques:2}
\end{table}

Furthermore, the low rate of \GlossMarkup{1PL} verbal forms and the virtual substitution of the canonical pronoun form \textit{nós} are likely to be symptomatic of a possible ongoing realignment of the pronominal and verbal paradigm in the insular Portuguese variety under focus.


Although these hybrid constructions have received little attention in research on morphosyntactic variation so far, \citet{Martins2009} provided a seminal study in this area. Following a generative framework, \citet[179]{Martins2009} argues that these structures fall under the denomination of “double subject impersonal \textit{se} construction”, as a strong subject pronoun or determinant phrase (DP) appears with what the author considers to be an “impersonal subject” (\textit{se}). According to Martins, this construction is present in both insular and continental Portuguese dialects, particularly in the center-south region of continental Portugal (\citealt[180, fn. 2]{Martins2009}). In the Portuguese varieties taken into account by the author, \textit{se} can cooccur not only with \GlossMarkup{1PL} pronouns (\textit{nós} and \textit{a gente}) but also with \GlossMarkup{3PL} pronouns and “less commonly with full determinant phrases” \citep[179]{Martins2009}. The interpretation of these constructions depends on the semantics of the subject expression, in the sense that the “doubling strong pronoun” semantically restricts the denotative scope of \textit{se}. Thus, an inclusive interpretation  -- including the speaker in a non-specific group of humans --  arises when the strong pronoun or DP is \GlossMarkup{1PL}. In contrast, \GlossMarkup{3PL} strong pronouns in combination with \textit{se} usually trigger an exclusive interpretation, excluding the speaker from the referent group.



In contrast to these findings, our corpus shows a significant number of other uses that are not included in \citet{Martins2009}, i.e., denoting specific sets of referents. In a footnote, the author states that only two cases of a specific reading were attested and that it “appears to be infrequent” (\citealt[186, fn. 10]{Martins2009}). The lack of such data in \citet{Martins2009} is likely to be a corpus effect. The \textit{CORDIAL-SIN} includes classic dialectological interviews and short free speech samples. The topics addressed therein cover a thematic range of aspects of the language community's cultural life (e.g., traditions, customs, fishing and farming practices, etc.), thus favoring the mention of unspecific groups. The fact that our semi-directed interviews also include questions on the informant's personal life contributes to the allusion to specific referents and specific groups. Our data thus suggest that the analyzed construction manifests a broader referential scope than previously assumed. This wide scope of possible interpretations ranging from impersonal to personal can be observed in examples \REF{ex:henriques:14} and \REF{ex:henriques:15} respectively:

\ea\label{ex:henriques:14}
 \gll {A gente} diz-se assim: uma traçada.\\
         \textsc{pron.sbj.1pl} say-\textsc{pres.3sg}=SE so: a bundle\\
 \glt `We say it like this: a bundle.'
\ex\label{ex:henriques:15}
 \gll Mãe, {o que é que} {a gente} vai-se fazer? \\ 
         mother what \textsc{pron.sbj.1pl} go-\textsc{prs.3sg}=SE do-\textsc{inf}\\
 \glt `Mom, what are we going to do?'
\z 

While \REF{ex:henriques:14} is consistent with the impersonal and speaker-inclusive interpretative effects described by \citet[186–188]{Martins2009}, \REF{ex:henriques:15} illustrates a personal interpretation of the construction due to its use in a directive speech act addressed to the interlocutor. The readings conveyed by [(\textit{a gente})+\textit{se}] are further discussed in the following section.

\section{Referential scope}\label{sec:henriques:4}

This section provides an overview of the referential scope of the hybrid construction and its constituents, namely the pronoun \textit{a gente} and the clitic \textit{se.} This detailed description highlights the possible reference overlaps of both elements, which have made possible the conjoint construction found in rural MP varieties. As mentioned, I consider [(\textit{a gente})+\textit{se}] to be a variant of pronominal \GlossMarkup{1PL} expressions in the varieties under focus. As such, it shows a complex reference, which Posio considers as being able to “include any human beings from the addressee to a third person or persons, an institution, or even the whole humankind” (\citeyear[342]{Posio2012}).


The first part of this section will deal with the referential range of \textit{se}-\textsc{imp} constructions in Portuguese varieties. The second part is dedicated to the referential aspects of \textit{a gente}. Finally, the last section uses the observations of the first two sections to phrase possible interpretations of the hybrid construction under analysis.

\subsection{\textit{se}-\textsc{imp} and its referential properties}\label{sec:henriques:4.1}

In most Romance languages, \textit{se}-\textsc{imp} constructions are a common agent-de\-fo\-cus\-ing strategy in which the reference of the agent is interpreted as unspecific and human. Due to its properties of conveying a reduction in referentiality regarding the intended subject, recent studies on impersonalizing strategies have referred to these as “R-impersonals” \citep{Siewierska2011}. In terms of its formal characteristics, the clitic \textit{se} attaches to a verb in third person, singular or plural. With transitive verbs, a plural NP bearing the semantic role of patient can trigger plural agreement with the verb, thus manifesting both object and subject properties.\footnote{The patient NP is typically placed after the verb. Preverbal patient NPs, display topic status, and thus appear in canonical subject position. However, both preverbal as well as postverbal plural patient NPs may trigger agreement on the verb. \citet[187]{PosioVilkuna2013} state that the postverbal position is most commonly found in their dialectal data.} Following this possible agreement, traditional Portuguese grammars usually distinguish between two formally different \textit{se} constructions: an agreeing construction often referred to as “passive”, and a non-agreeing one often referred to as “impersonal” \citep{Naro1976}. \citet{Cinque1988}, who provides a detailed seminal description for \textit{si} constructions in Italian, proposes that depending on its agreement, the intended subjects have different interpretations: “quasi-existential” in agreeing constructions and “quasi-universal” in non-agreeing structures. While observing the same types of constructions in Portuguese, \citet[750]{RaposoUriagereka1996} adopt the labels “indefinite SE construction” and “generic SE construction” respectively.


The grammatical status of \textit{se}-\textsc{imp} has been prone to polemic in linguistic studies. Some authors have considered \textit{se} to display subject properties (cf. \citealt{Martins2009,Martins2005,Martins2003, RaposoUriagereka1996}). Others have highlighted its functionality in discourse and considered it a grammaticalized impersonality marker (\citealt{PosioVilkuna2013}).


As far as the referential properties of impersonal \textit{se} are concerned, few studies on Portuguese have dealt with it extensively. For instance, \citet{Naro1976} observes that in standard EP, \textit{se} might incidentally include the speaker in its referential scope. Regarding dialectal Portuguese varieties, \citet{PosioVilkuna2013} find that the default readings of \textit{se}-\textsc{imp} tend to be speaker-inclusive impersonal. They might even alternate with the \GlossMarkup{1PL} pronouns \textit{a gente} and \textit{nós} in impersonal contexts\footnote{Given the fact that the examples used by \citet[211--213]{PosioVilkuna2013} to illustrate this alternation stem exclusively from Madeiran informants, one might be tempted to hypothesize that this alternation is a possible hybrid construction used in coreferential contexts, without the expressed subject pronoun \textit{a gente}.}, according to the speaker-inclusive semantic properties they share. A description that ascribes a more specific reference property to the element \textit{se} can be found in \citet{Casteleiro1975}. Considering \GlossMarkup{1PL} expression items in nonstandard varieties of continental Portuguese, the author acknowledges that, apart from \textit{a gente}, \textit{se} frequently alludes to \GlossMarkup{1PL} referents \citep[65]{Casteleiro1975}. However, the examples proposed to support this idea do not present enough context to univocally infer a specific referent. Hence, according to these observations on the semantic properties of the element \textit{se}, there seems to be a consensus that  -- per its agent demotion properties --  \textit{se} manifests an overall indefinite interpretation that incidentally may include the speaker in its scope. 


These results regarding the predominance of speaker-inclusive readings of impersonal \textit{se} are consistent with the data under survey in this study. All of the 258 \textit{se}-\textsc{imp} constructions found in the corpus have speaker-inclusive readings. Furthermore, the impersonal interpretation is blocked in episodic clauses (e.g., featuring predicates anchored in time). The fact that the only verbs found in \textit{se}-\textsc{imp} constructions are imperfect or present tense underlines this aspect. The cooccurrence with perfective predicates would trigger a specific reading anaphorically associated to a \GlossMarkup{1PL} referent, as illustrates the difference between examples \REF{ex:henriques:16} and \REF{ex:henriques:17}. 

\ea\label{ex:henriques:16}
 \gll Deita-se sal na carne.\\
         put-\textsc{pres.3sg}=SE salt in-the meat\\
 \glt `One puts salt on the meat.'
\ex\label{ex:henriques:17}
 \gll Teve-se uma viagem maravilhosa.\\
         have-\textsc{pst.3sg}=SE a trip wonderful\\
 \glt `(We) had a wonderful trip.'
\z 

The preferably speaker-inclusive interpretations of \textit{se}-\textsc{imp} found in previous studies, in addition to the rural MP data analyzed in this study, might facilitate the specific, and even deictic, readings displayed by the hybrid construction under focus.

\subsection{The pronoun \textit{a gente} and \GlossMarkup{1PL} reference}\label{sec:henriques:4.2}

The nominal origin of \textit{a gente} (\citealt{Lopes2003,Lopes1999}) has resulted in a mismatch between semantic and syntactic properties, a phenomenon well studied in previous research (cf. for EP, \citealt{CostaPereira2013, CostaPereira2005}). Thus, the deviation between notional person (\GlossMarkup{1PL}) and the grammatical person (\GlossMarkup{3SG}) often incentivizes debates about the pronominal status of \textit{a gente} (e.g., \citealt{Taylor2009}). While its pronominal properties are still subject to ongoing discussion, there seems to be consensus on its referential properties ranging from impersonal readings to personal ones. As mentioned in \sectref{sec:henriques:3}, previous studies show that the variation between \textit{nós} and \textit{a gente} is present in both BP and EP varieties. However, it has been shown that EP varieties display lower usage rates of \textit{a gente} than BP varieties, where there is a notorious expansion of \textit{a gente} into more formal discursive contexts (\citealt{CallouLopes2004}). Along these lines, the research on \textit{a gente} indicates that increasing application rates correlate with increasing grammaticalization (cf.  \citealt{deOmena2003}). These empirical studies show that \textit{a gente} can be used as a subject pronoun with speaker-inclusive impersonal reference and specific reference alike, despite the generic origins of \textit{a gente} (cf.  \citealt{deOmena2003, TravisSilveira2009}). 

The data under analysis here suggest a high usage rate of the pronoun \textit{a gente} (cf. \tabref{tab:henriques:1}), which seems to have almost entirely replaced the canonical pronoun \textit{nós} in rural MP varieties. These findings might indicate a grammaticalized pronoun state comparable to the one found in BP varieties. Moreover, its referential scope is compatible with other \GlossMarkup{1PL} pronominal expressions, ranging from an unspecified group of persons (or people in general) including the speaker, to purely deictic uses, referring to speech-act participants.\footnote{Following \citet[342]{Posio2012}, this flexibility of reference “is what makes possible also the use of first person plural as an impersonalizing strategy”.} 

Considering that the speaker-inclusive impersonal reference constitutes the common denominator of both \GlossMarkup{1PL} pronouns and \textit{se}-\textsc{imp}, this intersection may have been the basis of the junction of \textit{a gente} and \textit{se} in the varieties under focus.

\subsection{The referential scope of the hybrid construction [(\textit{a gente})\,+\,\textit{se}]}\label{sec:henriques:4.3}


The reference of the hybrid constructions are generally consistent with the range of possible references displayed by \GlossMarkup{1PL} pronouns. Not only does [(\textit{a gente})+\textit{se}] display the impersonal speaker-inclusive references described by \citet[186–188]{Martins2009}, it also shows purely deictic interpretations referring to speech-act participants. The contrast between examples \REF{ex:henriques:18} and \REF{ex:henriques:19} spans the wide variety of possible interpretations.

\ea\label{ex:henriques:18}
 \gll Pega-me às costas! {A gente} chega-se lá num instante.\\
         pick-\textsc{imp.2sg=acc.1sg} to-the back \textsc{pron.sbj.1pl} arrive-\textsc{pres.3sg}=SE there in-an instant\\
\glt `Pick me up on your back! We'll get there in no time.'
\ex\label{ex:henriques:19}
 \gll {A gente} aqui chama-se abóbora moira.\\
         \textsc{pron.sbj.1pl} here call-\textsc{pres.3sg}=SE pumpkin <word>\\
 \glt `We call it moira-pumpkin here.'
\z 



The deictic reading of the hybrid construction in \REF{ex:henriques:18} is connected with the imperative  -- which refers to the addressee --  and the accusative clitic (-\textit{me}) in the first sentence. The group referred to (i.e., \GlossMarkup{2SG} imperative \textit{pega} and \GlossMarkup{1SG} clitic -\textit{me}) is construed simultaneously as the sentence is uttered, which leads to the deictic interpretation of [(\textit{a gente})+\textit{se}] in the following sentence. Contrary to that, \REF{ex:henriques:19} shows an impersonal interpretation, referring to the speaker community in general.



In between these two poles of the referential continuum  -- speaker-inclusive impersonality and hearer-inclusive reference\footnote{\citet[342]{Posio2012} utilizes clusivity as the differentiating factor between “hearer-inclusive” and “impersonal (speaker-inclusive)” readings. The two extremes of the continuum proposed here are defined following Posio's categorization.} --  there are several intermediate interpretations, highly dependent on various co-textual aspects, to which I will return later on in this chapter.



The results of the quantitative analysis (cf. \sectref{sec:henriques:5.1}) show that the majority of the constructions being analyzed lack an overt subject. Interestingly, the omission of \textit{a gente} materializes in a construction in which \textit{se} can establish personal interpretation, as illustrated in the affirmative verbal response\footnote{In Portuguese, there are several strategies to respond affirmatively to a polar question. One of these strategies consists of repeating the finite verb by adapting person and number features (cf. \citealt{Martins2013,Martins2016a})}  to the question in \REF{ex:henriques:20}.

\ea\label{ex:henriques:20}
 \begin{xlist}
 \exi{A:} \gll Vocês já estão em casa?\\
         \textsc{pron.sbj.2pl} already be-\textsc{pres.3pl} at home?\\
  \glt `Are you already home?'\\
 \exi{B:} \gll Está-se.\\
         be-\textsc{pres.3sg}=SE\\
 \glt `Yes, we are.'
 \end{xlist}
\z 

The specific interpretation in \REF{ex:henriques:20} is based on the contrast between the \GlossMarkup{2PL} pronoun \textit{vocês} and the verb \textit{estar} bearing the element \textit{se}. The speaker's utilization of this construction as an affirmative verbal answer (\citealt{Martins2013,Martins2016a}) further supports the hypothesis of \textit{se} functioning as a \GlossMarkup{1PL} person marker. 


In direct comparison, these \textit{se} interpretations differ strikingly from those of \textit{se}-\textsc{imp}s described in \sectref{sec:henriques:4.1}. The fact that an impersonal interpretation of \textit{se} is impossible in contexts where it refers to speech-act participants seems to confirm the hypothesis regarding the reanalysis of \textit{se}. This reanalysis becomes even more evident in cases like \REF{ex:henriques:21} below, where \textit{se}, controlled by another subject NP with \GlossMarkup{1PL} reference rather than \textit{a gente}, seems to add number- and person-marking to the \GlossMarkup{3SG} verb form.\footnote{Note that there are no occurrences of lack of agreement between a coordinate preverbal subject and the verb in the data under analysis.}

\ea\label{ex:henriques:21}
 \gll Eu mais meus primos ia-se buscar lenha. \\ 
         \textsc{pron.sbj.1sg} with my cousins go.\textsc{ipfv.3sg}=SE get-\textsc{inf} firewood\\
 \glt `Me and my cousins used to go get firewood.'
\z

There are at least two possible analyses for examples like \REF{ex:henriques:21}. One analysis would consider the preverbal coordinate NP (\textit{Eu mais meus primos)} to be a topic, which precedes the sentence displaying a null subject \textit{a gente} (\textit{Eu mais meus primos, {\normalfont∅} ia-se buscar lenha}). A second possibility is to assume the preverbal NP (\textit{Eu mais meus primos}) to be the subject. In light of the second analysis, example \REF{ex:henriques:21} could illustrate a further step in the grammaticalization path of \textit{se} as a \GlossMarkup{1PL} person marker. 

There are several possible intermediate interpretations of the hybrid construction located between the impersonal and deictic poles of the proposed referential scale. These specific interpretations of [(\textit{a gente})+\textit{se}] are determined by a vast array of co-textual factors. 

\ea\label{ex:henriques:22}
 \gll Ia-se as duas.\\
         go.\textsc{3sg}=SE the two\\
 \glt `The two of us used to go.'
\ex\label{ex:henriques:23}
 \gll Ia-se todos para lá.\\
         go.\textsc{3sg}=SE all to there\\
 \glt `All of us went there.'
\z 

Examples \REF{ex:henriques:22} and \REF{ex:henriques:23} illustrate the graduality of possible specific interpretations of \textit{se}. For instance, the specific reading of \REF{ex:henriques:22} relies on the cooccurrence with numerals, thus on the cardinality of the group. The interpretation of \REF{ex:henriques:23} is slightly less specific than the one triggered by \REF{ex:henriques:22}. The cooccurrence of \textit{todos} (‘all’) implies that there is a specific number of members in the set of referents which consequently evokes a specific rather than an impersonal reference.


The examples clearly illustrate the broad scope of references covered by the hybrid construction under analysis. It appears in contexts where possible references of \textit{a gente} and \textit{se}-\textsc{imp} converge. This suggests that the speaker-inclusive impersonality shared by both constituents is the common denominator and may be where this hybrid construction originated. The fact that this originally \textit{se}-\textsc{imp} occurs in specific or even deictic contexts shows that it no longer requires \textit{a gente} to establish personal reference, which could be a symptom of the reanalysis of \textit{se} as a person marker. A possible syntactic catalyst for this reanalysis is described by \citet{PosioVilkuna2013}. The authors observe that, while in EP varieties, the patient NP can be reanalyzed as the subject of \textit{se}-\textsc{imp} constructions, “in Madeira and Porto Santo dialects the Patient has been reanalyzed as a direct object” (2013: 213). This reanalysis of the patient NP as a direct object can be observed in examples of \textit{se}-\textsc{imp} constructions featuring accusative clitics,\footnote{Following \citet[214]{PosioVilkuna2013} the reanalysis of the patient NP as a direct object can be observed in their example (24) partially reproduced here as \REF{ex:henriques:fn15-1}.
\ea \label{ex:henriques:fn15-1}
\gll Em sendo para a latada, deixa-se-a crescer [...]\\
    in be-\textsc{ger} for the trellis leve-\textsc{3sg}=SE=\textsc{acc.3sg} {grow \textsc{inf}}\\
\glt `Being for the trellis, you let them grow [...].'
\z}
the combination of which is considered ungrammatical in standard EP \citep[786]{Naro1976}. The reanalysis of the patient NP as a direct object and thus eliminating it from the list of possible clausal subjects, might have served as a catalyst for the latter reanalysis of \textit{se} as a person marker. 

\section{Syntactic features of [(\textit{a gente})\,+\,\textit{se}]}\label{sec:henriques:5}

The new insights into the referential scope of the hybrid construction, briefly introduced in the previous section, have substantial repercussions on the analysis of the syntactic properties of this particular phenomenon. Given that the hybrid construction [(\textit{a gente})+\textit{se}] not only exhibits impersonal readings, in which they partially overlap with the semantics of canonical \textit{se}-\textsc{imp} constructions, but also allows for specific and even deictic readings (e.g., the reference to speech act participants), it is necessary to reconsider the syntactic and semantic properties of the formerly impersonal marker \textit{se} in the studied variety.

This section describes and discusses different syntactic properties displayed by the construction under study here in contrast with the findings of previous studies. \sectref{sec:henriques:5.1} deals with variable subject expression of \textit{a gente} in these contexts. \sectref{sec:henriques:5.2} describes verbal agreement patterns in clauses where [(\textit{a gente})+\textit{se}] accounts for subject person marking. \sectref{sec:henriques:5.3} provides evidence for the ability of \textit{se} in these contexts to trigger adjectival agreement. \sectref{sec:henriques:5.4} connects with the former two and adds a descriptive insight into the construction's behavior in coreferential contexts.

\subsection{\label{bkm:Ref83459778}Variable subject expression}\label{sec:henriques:5.1}


As it has already been stated in \sectref{sec:henriques:3}, the construction under analysis can occur with the subject pronoun \textit{a gente} \REF{ex:henriques:24}, with subject NPs as its antecedents \REF{ex:henriques:25}, or in clauses without an overt subject or antecedent \REF{ex:henriques:26}.

\ea\label{ex:henriques:24}
 \gll Mas {a gente} não {se fazia} bacalhau.\\
         But \textsc{pron.sbj.1pl} \textsc{neg} SE=make-\textsc{ipfv.3sg} codfish\\
 \glt `But we didn't make codfish.'
\ex\label{ex:henriques:25}
 \gll Eu e Alicinha, cada uma fazia a sua semana. Cosia-se uma semana inteira o almoço.\\
         \textsc{pron.sbj.1sg} and Alicinha each one make-\textsc{ipfv.3sg} the \textsc{poss.f-sg} week cook-\textsc{ipfv.3sg}=SE one week whole the lunch\\
 \glt `Alicinha and I each made her own week. (We) cooked lunch for a whole week.'
\ex\label{ex:henriques:26}
 \gll Se ele fosse preciso ser opearado, ficava-se lá.\\
         If \textsc{pron.sbj.3sg} be.\textsc{sbjv.ipfv.3sg} necessary be.\textsc{inf} operated stay-\textsc{ipfv.3sg}=SE there\\
 \glt `If it would be necessary for him to get surgery, (we) would stay there.'
\z 

The high frequency of hybrid constructions without an expressed subject, in which the clitic \textit{se} is the primary element encoding \GlossMarkup{1PL} reference on the verb, indicates the degree of grammaticalization of this expression in rural MP varieties. Indeed, the vast majority of the analyzed clauses in the data do not occur with the subject pronoun \textit{a gente}.

Out of the 566 clauses containing [(\textit{a} \textit{gente})+\textit{se}], 177 occur with the overt subject pronoun. The remaining 389 cases are occurrences of \GlossMarkup{3SG} verbal forms with the clitic \textit{se} displaying personal (i.e. specific or deictic) interpretations.

\begin{table}
\begin{tabular}{l rrrr}
\lsptoprule
~ & \multicolumn{2}{c}{overt} & \multicolumn{2}{c}{null}\\\cmidrule(lr){2-3}\cmidrule(lr){4-5}
  & $n$ & \% & $n$ & \%\\\midrule
{[\textit{a gente}+\textit{se}]} & 177 & 31.3 & 389 & 68.7\\
\textit{a gente/nós} & 208 & 79.1 & 55 & 20.9\\
\lspbottomrule
\end{tabular}\\
\caption{First person plural pronominal expression in rural MP varieties}
\label{tab:henriques:3}
\end{table}

Contrary to these findings, Posio states for peninsular EP varieties that \textit{a gente} “is usually expressed even in contexts that strongly favor the omission,” such as coreferential contexts within coordinated clauses (\citeyear[345]{Posio2012}). This is a straightforward consequence of the ambiguity conveyed by the omission of \textit{a gente} with \GlossMarkup{3SG} verb forms. Thus, as illustrated below, the presence of \textit{se} results in higher permissibility of the \GlossMarkup{1PL} subject omission. Example \REF{ex:henriques:27}, for instance, illustrates an occurrence of \textit{se} lacking an overt subject pronoun. Its specific interpretation is, again, determined by co-textual factors.

\ea \begin{xlist}
 \exi{A:} Quantos filhos é que a sua mãe teve? \label{ex:henriques:27}\\
 \glt `How many children did your mother have?'\\
 \exi{B:} \gll Era-se dez.\\
         be.\textsc{ipfv.3sg}=SE ten\\
 \glt (`We) were ten.' (=`There were ten of us.')
 \end{xlist}
\z

The fact that null pronominal subjects are so common in these contexts is relevant in two different respects. First, it shows that \textit{se} is able to disambiguate \GlossMarkup{3SG} verb forms while establishing \GlossMarkup{1PL} reference. Second, it suggests that the element \textit{se} is being reanalyzed as a \GlossMarkup{1PL} person marker.

\subsection{Variable verbal agreement patterns}\label{sec:henriques:5.2}

In her analysis of the verbal agreement in constructions comprising a strong pronoun and the clitic \textit{se}, \citet[185]{Martins2009} found that it is the former that triggers agreement on the verb due to the presumably person-less nature of \textit{se}. She draws this conclusion from the fact that the same variable agreement patterns  -- namely \GlossMarkup{3SG}, \GlossMarkup{1PL}, and \GlossMarkup{3PL} --  induced by the pronoun \textit{a gente} can also be found encoded in the verbal forms associated with these “double subject impersonal \textit{se} constructions” (\citealt[185–186]{Martins2009}). In our data, as shown in \tabref{tab:henriques:2}, 13.5\% of the clauses in which \textit{a gente} or \textit{nós} assumes the role of pronominal subject have the verb in \GlossMarkup{1PL}. However, in regard to verbal agreement induced by the hybrid construction [(\textit{a gente})+\textit{se}], the data display a clear predominance of \GlossMarkup{3SG} verb forms and only six cases of verbal agreement with \GlossMarkup{1PL} verb forms (1.05\%). It is worth noting that in all these cases \textit{se} appears in the proclitic position, thus occurring in finite subordinate clauses \REF{ex:henriques:28}, and in principal clauses featuring negative polarity items \REF{ex:henriques:29} or other proclisis-inducing elements such as focalizing \textit{já} \REF{ex:henriques:30}.\footnote{European Portuguese varieties display complex clitic positioning patterns. This issue, however, goes far beyond the scope of the present chapter. For a seminal description on this issue see \citet{Martins2016b}. Furthermore, a recent study by the same author \citep{Martins2021} highlights the clitic positioning in insular Portuguese varieties of the Madeira and Azores archipelagos.}

\ea\label{ex:henriques:28}
 \gll porque {se fomos} as mais velhas\\
         because SE=be.\textsc{pst.1pl} the more old\\
 \glt `because (we) were the eldest'
\ex\label{ex:henriques:29}
 \gll Não {se morremos} de fome.\\
         \textsc{neg} SE=die-\textsc{pst.1pl} of hunger\\
 \glt `We didn't die of hunger.'
\ex\label{ex:henriques:30}
 \gll Já {se criámos} dois de cada vez.\\
         even SE=raise-\textsc{pst.1pl} two at each time\\
 \glt `We even raised two at a time.'
\z 


The proclitic position of \textit{se} could explain the preferable omission of \textit{a gente} and the use of the person marking morpheme {}-\textit{mos} in these contexts. The combination of the three person marking items  -- \textit{a gente}, morphological \GlossMarkup{1PL} marking \textit{{}-mos} and the element \textit{se}\footnote{There are no examples manifesting all three person marking items in the corpus under study. For illustrative purposes, consider the following fabricated example:
\ea[*]{
 \gll {A gente} trabalhamos-se muito.\\
      \textsc{pron.sbj.1pl} work-\textsc{prs.1pl}=SE {a lot}\\
 \glt `We work {a lot}.'}
\z 
}
\textit{–} could lead to over-specification of \GlossMarkup{1PL} person-encoding on the verb.\footnote{Despite the overall high productivity of [(\textit{a gente})+\textit{se}], reflexive/reciprocal verbs seem to restrict the use of this construction. It must be noted that \textit{se} is homonymous to the reflexive/reciprocal \GlossMarkup{3SG} and \GlossMarkup{3PL} clitic in Portuguese. Moreover, in some varieties with the pronoun \textit{a gente}  -- including those mentioned here --  it formally coincides with the \GlossMarkup{1PL} reflexive/reciprocal clitic \citep[185]{Martins2009}. Thus, in our data, reflexive/reciprocal verbs tend to block the use of [(\textit{a gente})+\textit{se}], due to the unacceptability of the sequence *\textit{se-se} discussed in \citet[footnote 18]{Martins2009}. The ungrammaticality of the sequence *\textit{se-se} also accounts for the well-attested incompatibility of reciprocal/reflexive verbs in \textit{se}-\textsc{imp} constructions. In light of these observations, further analyses are required to understand the use of [(\textit{a gente})+\textit{se}] with other clitic pronouns.} 

\subsection{Adjectival agreement}\label{sec:henriques:5.3}

For standard EP, \citet[191-192]{Martins2009} states that the element \textit{se} in impersonal constructions cannot establish adjectival agreement in predicative contexts. Comparing with the adjectival agreement properties found in dialectal EP varieties, the author differentiates between two types of “impersonal \textit{se}”: one found in standard EP varieties, whose number feature corresponds to “singular”; the other one, found in EP dialects, manifesting the construction under analysis, displays the number feature “plural” and therefore allows “plural agreement between \textit{se} and an adjectival predicate” \citep[192]{Martins2009}. Additionally, the author proposes two examples illustrating the ungrammaticality of plural adjectival agreement in \textit{se}-\textsc{imp} constructions in standard EP, reproduced here as \REF{ex:henriques:31} and \REF{ex:henriques:32}:

\ea[]{\label{ex:henriques:31}
 \gll Quando {se é} novo...\\
      When SE=be.\textsc{prs.3sg} young-\textsc{m.sg}\\
 \glt `When one is young...'}
\ex[*]{\label{ex:henriques:32}
 \gll Quando {se é} novos...\\
      When SE=be.\textsc{prs.3sg} young-\textsc{m.pl}\\
 \glt `When one is young...'}
\z 

In terms of the hybrid construction [(\textit{a gente})+\textit{se}], the data under analysis here confirm the tendencies described in \citet{Martins2009}. Thus, the hybrid construction exclusively triggers plural agreement in predicative contexts. Moreover, our data include cases of agreement reflecting the gender of the intended referents, as shown in examples \REF{ex:henriques:33} and \REF{ex:henriques:34}: 

\ea\label{ex:henriques:33}
 \gll Quando {se era} pequenos?\\
         when \textsc{se=be.ipfv.3sg} little-\textsc{m.pl}\\
 \glt `When (we) were little?'
\ex\label{ex:henriques:34}
 \gll Era-se pequenas.\\
         be.\textsc{ipfv.3sg}=SE little-\textsc{f.pl}\\
 \glt `(We) were little.'
\z 

The agreement contrast between these two examples stems from the fact that the group alluded to in example \REF{ex:henriques:34} is exclusively female (the informant is referring to herself and the neighbor's daughter). Example \REF{ex:henriques:33}, however, relates to the informant’s brothers and sisters, thus displaying default masculine and plural adjectival agreement.


The data under analysis suggest that the hybrid construction [(\textit{a gente})\,+\,\textit{se}] displays non-variable plural adjectival agreement in predicative contexts. However, in terms of gender agreement, variable patterns can be found. These variable gender agreement patterns are consistent with those attested for other \GlossMarkup{1PL} person marking items in Portuguese (\citealt{CostaPereira2013,CostaPereira2005, Pereira2003}). Furthermore, the fact that the element \textit{se} of hybrid constructions exclusively triggers plural agreement on the predicate might further endorse its status as a \GlossMarkup{1PL} person marker.


\subsection{Coreference}\label{sec:henriques:5.4}

Coreference has been identified as an essential contributing factor for the expression or omission of subject pronouns in pro-drop languages. There is a broad consensus that coreference with a previous subject favors the omission of subject pronouns (\citealt{Silva-Corvalán1982}, among others). This section focuses on the role of coreference regarding the structural and semantic features of [(\textit{a gente})+\textit{se}]. Even though this study does not claim to contemplate all the factors that enable subject omission in these contexts, the examples clearly illustrate that, even in contexts lacking an overt subject and those without an immediate \GlossMarkup{1PL} subject antecedent, \textit{se} establishes \GlossMarkup{1PL} reference.



Previous studies on the variation of \GlossMarkup{1PL} pronominal expression in Portuguese varieties have shown that the newer pronoun \textit{a gente} occurring with \GlossMarkup{3SG} verb forms can only be omitted in a restricted number of contexts. One requirement is coreference with the overt antecedent \textit{a gente} \citep{Sória2013}. The quantitative analyses on rural MP varieties (cf. \sectref{sec:henriques:3}) confirm the tendencies found in previous research: only five examples of null subject \textit{a gente} were found in the corpus. This is in line with Posio’s findings for continental EP varieties, where overt \textit{a gente} even appears “in contexts that strongly favor the omission” (\citeyear[345]{Posio2012}).



In the context of the hybrid construction [(\textit{a gente})+\textit{se}], the omission rate is much higher, occurring in 68.7\% of the cases displaying \GlossMarkup{1PL} reference (cf. \tabref{tab:henriques:3}). The data under analysis suggest that coreference is a key factor affecting the omission of the subject pronoun in hybrid constructions, resulting in a \GlossMarkup{3SG} verb form and the clitic \textit{se.} Furthermore, discourse connectedness (in terms of \citealt{ParedesSilva1993}) appears to determine whether a given occurrence of \textit{se} is to be interpreted as personal rather than impersonal. Example \REF{ex:henriques:35} shows the ability of the clitic \textit{se} to maintain \GlossMarkup{1PL}-specific reference when the pronoun \textit{a gente} is omitted.

\ea\label{ex:henriques:35}
 \gll {A gente} era-se costumadas ambas.\\
         \textsc{pron.sbj.1pl} be.\textsc{ipfv.3sg}=SE accostumed both\\
 \glt `We were used to each other.'\\
 \gll Ia-se para a escola,\\
         go.\textsc{ipfv.3sg}=SE to the school\\
 \glt `(We) used to go to school'\\
 \gll ia-se as duas passava-se ali...\\
         go.\textsc{ipfv.3sg}=SE the two.\textsc{fpl} pass-\textsc{ipfv.3sg}=SE there\\
 \glt `(we) used to go together, (we) would pass by...'
\z 

\citet{Cameron1995} proposes that a \GlossMarkup{1PL} expression is usually introduced into discourse only after its reference  -- or parts of the referent set --  has previously been established in the discourse. This can be seen in example \REF{ex:henriques:36} where the informant starts the utterance with \textit{eu} and then goes on to refer to herself and her spouse, which can be inferred from the semantics of the verb \textit{casar}. Partial coreference is thus established between the clitic \textit{se} and the first person singular personal pronoun \textit{eu}.\footnote{This partial coreference is connected to what \citet[522–524]{Gelbes2008} calls “correferencia inclusiva” (‘inclusive correferentiality’).}

\ea\label{ex:henriques:36}
 \gll Eu quando casei, criava-se dois [porcos].\\
         \textsc{pron.sbj.1sg} when marry-\textsc{pst.1sg} raise-\textsc{ipfv.3sg}=SE two [pigs]\\
 \glt `When I got married, (we) raised two [pigs].'
\z 

The importance of coreference between a \GlossMarkup{1PL} antecedent and the element \textit{se} becomes even more evident when discourse-initial contexts are considered. 

\ea\label{ex:henriques:37}
 \gll Matava-se um porco, era tudo salgado.\\
         kill-\textsc{ipfv.3sg}=SE a pig be.\textsc{ipfv.3sg} everything salted\\
 \glt `One used to kill pigs, everything had to be salted.'\\
 \gll Comprava-se uma salga para salgar o porco.\\
         buy-\textsc{ipfv.3sg}=SE a <name> to salt-\textsc{inf} the pig.\\
 \glt `One bought a salting vessel to salt the pig.'
\z 

In discourse-initial contexts, the absence of \textit{a gente} or another \GlossMarkup{1PL} referent renders the personal interpretation of \textit{se} improbable or at least impossible to determine. Consider the contrast between the previous example \REF{ex:henriques:37} and example \REF{ex:henriques:38} below:

\ea\label{ex:henriques:38}
 \gll Em princípio, {a gente} foi-se bebés.\\
      in beginning \textsc{pron.sbj.1pl} be.\textsc{pst.3sg}=SE babies\\
 \glt `First, we were babies.'\\
 \gll Depois cresceu-se, foi-se para a {Escola das Irmãs}.\\
      then grow.\textsc{pst.3sg}=SE go.\textsc{pst.3sg}=SE to the {<name of the school>}\\
 \glt `Then (we) grew up, (we) went to Escola das Irmãs.'\\
       Mas como os meus pais não tinham a possibilidade de pôr a gente a estudar,\\
 \glt `But, since my parents did not have the possibility to let us go to school,\\
 \gll apenas {se deu} a terceira classe.\\
      only \textsc{se=}give.\textsc{pst.3sg.} the third class\\
 \glt `(we) only completed the third grade.' 
\z 

Example \REF{ex:henriques:38} shows another extract of a discourse-initial context. The informant answers a question on how many siblings she has and how they were brought up. She retrieves the set of referents  -- the informant and her siblings --  by using the pronoun \textit{a gente} in the first sentence. What follows is a chronological depiction of the events with coreferential null subjects. Interestingly, in \REF{ex:henriques:38} the pronoun \textit{a gente} is omitted in the fourth sentence (\textit{apenas se deu a terceira classe}), even though there is discontinuity regarding the previous subject \textit{os meus pais}. This shows, in part, that coreference cannot fully account for the variation between expression and omission of the subject in hybrid constructions. The correct interpretation of the  \GlossMarkup{1PL} subject, in this case, is most probably established by the perfective past (\textit{deu}) describing an event anchored in time, thus favoring a personal interpretation \citep{Siewierska2011}.


As anticipated in the introductory lines to this section, there are particular contexts in which the omitted subject of the hybrid construction is not coreferential to the previous subject. The following examples, for instance, show contexts that strongly favor the omission of the subject. Hence, the element \textit{se} is used to encode \GlossMarkup{1PL} person marking on its own. 

\ea\label{ex:henriques:39}
 \gll Teresinha, vai-se brincar!\\
         <name> go.\textsc{pres}=SE play-\textsc{inf}\\
 \glt `Teresinha, let's go play!'
\ex
 \begin{xlist}\label{ex:henriques:40}
 \exi{A:}\gll A senhora brincava com os seus irmãos?\\
         the Mrs play-\textsc{ipfv.3sg} with the your siblings\\
 \glt `Did you play with your siblings?'\\
 \exi{B:}\gll Brincava-se ao domingo.\\
         play-\textsc{ipfv.3sg}=SE on-the Sunday\\
 \glt `Yes, (we) played on Sundays.'
 \end{xlist}
\z 

Example \REF{ex:henriques:39} shows a cohortative construction that “expresses the exhortation to the addressee to carry out an action together with the speaker” (\citealt{DobrushinaGoussev2005}: 179). Hence, the interpretation of the referent is inherently personal (i.e., deictic). Another context in which \textit{se} establishes personal \GlossMarkup{1PL} reference is found in affirmative verbal responses, such as \REF{ex:henriques:40}. In these contexts, the element \textit{se} retrieves the set of referents defined in the question (i.e., the informant and her siblings) thus assuming the role of a \GlossMarkup{1PL} person marker. These examples can be considered crucial evidence for the reanalysis of \textit{se} of the hybrid constructions in rural MP varieties. 


\section{Conclusion}\label{sec:henriques:6}
Rural Madeiran Portuguese varieties manifest two predominant \GlossMarkup{1PL} pronominal expressions: \textit{a gente} and the more common variant [(\textit{a gente})+\textit{se}]. Furthermore, the quantitative analysis indicates that the presence of \textit{se} allows for substantial variation in terms of the presence and absence of \textit{a gente} or other subjects displaying \GlossMarkup{1PL} reference (cf. \tabref{tab:henriques:3}). To account for this fact, a hypothesis was anticipated that the clitic \textit{se} seems to display \GlossMarkup{1PL} marking features in the absence of other person markers. There is evidence within the syntactic properties of [(\textit{a gente})+\textit{se}] outlined in this study that seems to support this tentative hypothesis:


\begin{enumerate}
 \item There is a meager rate of \GlossMarkup{1PL} verb forms in the context of these constructions, which might instigate that \textit{se} suffices to establish \GlossMarkup{1PL} reference.
 \item The hybrid construction (with and without the overt subject) can trigger variable gender agreement according to the constellation of the alluded group. However, in terms of number adjectival agreement in predicative contexts it exclusively triggers plural agreement.
 \item Independently of its coreference with a \GlossMarkup{1PL} antecedent, \textit{se} can trigger personal interpretations in contexts lacking an overt subject. Thus, it can be found in verbal affirmative answers and cohortative constructions, both of which favor the omission of the subject. 
\end{enumerate}

In terms of its referential properties, several observations can be made. The hybrid construction, whose constituents originate from impersonalizing strategies, might imply not only specific interpretations but also deictic ones, even in the contexts mentioned above lacking an overt subject. The fact that \textit{se} can refer to speech-act participants is the most straightforward argument supporting the initial tentative hypothesis. However, more research is needed to determine whether or not discourse-initial antecedentless contexts are the only restriction for the occurrence of \textit{se} referring to a specific \GlossMarkup{1PL} subject. Moreover, the analysis of diachronic data could offer more substantial insights into the possible origins of this hybrid construction.

\section*{Acknowledgments}
I would like to express my gratitude to all the informants who agreed to participate. Moreover, I wish to thank Carlota de Benito Moreno, Johannes Kabatek, two anonymous reviewers and the editors of this volume for their helpful comments, suggestions and insights. 

\printbibliography[heading=subbibliography]

\end{document}
