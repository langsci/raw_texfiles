\documentclass[output=paper]{langscibook}
\ChapterDOI{10.5281/zenodo.8124502}
\author{Émeline Pierre\orcid{0000-0003-0073-9633}\affiliation{Université catholique de Louvain} and Barbara {De Cock}\orcid{0000-0001-6968-3880}\affiliation{Université catholique de Louvain}}
\title[Discourse participants in impersonal constructions]
	  {Discourse participants in impersonal constructions: The case of first and second person object pronouns with Spanish non-anaphoric third person plural subjects}
\abstract{This paper offers a quantitative and qualitative analysis of object pronouns referring to discourse participants used with Spanish non-anaphoric third person plural subjects, e.g., \textit{me han criticado por largas respuestas} (‘they have criticized me for long answers’). Non-anaphoric third person plurals being agent-defocusing mechanisms which are sometimes considered to be impersonal, the discourse participant objects have a higher referentiality than the subject. We will show how this impacts the construction and conceptualization as a whole. By combining data from oral and written mode and informal and formal register, we examine the syntactic and semantic roles of discourse participants as well as the verb types they are used with. Our fine-grained study highlights that the use of discourse participant object pronouns and their roles are closely associated with the presence of relational verbs and communication verbs. Moreover, we show how the higher referentiality of the discourse participant object with respect to the subject is reflected through its being the main anchoring point for topic continuity. Finally, results indicate that the combination of a discourse participant object pronoun with a non-anaphoric third person plural is highly associated with the informal register and, whereas within oral conversations discourse participants tend to be oriented towards the speaker, in written interactions from a digital forum they rather involve the interlocutor.}

% % % \smallskip \\
% % % \textbf{Keywords}: agent-defocusing structure, discourse participant, referentiality, register, mode}
\IfFileExists{../localcommands.tex}{
  \addbibresource{../localbibliography.bib}
  % add all extra packages you need to load to this file

\usepackage{tabularx,multicol}
\usepackage{url}
\urlstyle{same}

\usepackage{listings}
\lstset{basicstyle=\ttfamily,tabsize=2,breaklines=true}

\usepackage{langsci-basic}
\usepackage{langsci-optional}
\usepackage{langsci-lgr}
\usepackage{langsci-osl}
% \usepackage{./langsci/styles/langsci-lgr}
% \usepackage{./langsci/styles/langsci-osl}
% \usepackage{langsci-gb4e}

\usepackage{tikz}
\usetikzlibrary{patterns,calc}
\pgfdeclarepatternformonly{south east lines}{\pgfqpoint{-0pt}{-0pt}}{\pgfqpoint{3pt}{3pt}}{\pgfqpoint{3pt}{3pt}}{
    \pgfsetlinewidth{0.6pt}
    \pgfpathmoveto{\pgfqpoint{0pt}{3pt}}
    \pgfpathlineto{\pgfqpoint{3pt}{0pt}}
    \pgfpathmoveto{\pgfqpoint{.2pt}{-.2pt}}
    \pgfpathlineto{\pgfqpoint{-.2pt}{.2pt}}
    \pgfpathmoveto{\pgfqpoint{3.2pt}{2.8pt}}
    \pgfpathlineto{\pgfqpoint{2.8pt}{3.2pt}}
    \pgfusepath{stroke}}
    
\usepackage{stmaryrd}
\usepackage{wasysym}
\usepackage{multirow}
\usepackage{caption}
\usepackage{subcaption}
\usepackage{mathrsfs}
\usepackage{qtree}

\usepackage{linguex}


  %pminos do not split footnotes
% \interfootnotelinepenalty=10000 %Footnote in Laporte chapters has to be split SN


%\DeclareIndexNameFormat{default}{%
%\nameparts{#1}%
%\usebibmacro{index:name}%
%{\index[names]}%
%{\namepartfamily}%
%{\namepartgiveni}%
% {}% L1
% {}% L2
%{\namepartprefix}% generates spurious space L3
%{\namepartsuffix}% generates spurious space L4
%}

%  {\DeclareIndexNameFormat{default}{%
%     \usebibmacro{index:name}{\index[names]}{#1}{#3}{#5}{#7}}}

%\DeclareIndexNameFormat{default}{%
%  \usebibmacro{index:name}{\sindex[nom]}{#1}{#3}{#5}{#7}}

%\DeclareIndexNameFormat{default}{%
%  \usebibmacro{index:name}{\sindex[person]}{#1}{#3}{#5}{#7}}
%\DeclareIndexNameFormat{default}{%
%\nameparts{#1} \usebibmacro{index:name}{\sindex[person]]}{\namepartfamily}{‌​\namepartgiven}{\nam‌​epartprefix}{\namepa‌​rtsuffix}}

%\newcommand{\smiley}{:)}

%\renewbibmacro*{index:name}[5]{%
%\usebibmacro{index:entry}{#1}%
%{\iffieldundef{usera}{}{\thefield{usera}\actualoperator}\mkbibindexname{#2}{#3}{#4}{#5}}}

% \newcommand{\noop}[1]{}

%remove for final
%\overfullrule=1mm

\newcommand{\tobi}[2]}}
\renewcommand{\S}[1]{\tobi{#1}{\textsc{*}}}

% this volume references
% puts: [this volume]
% already defined: \citetv
%\newcommand{\citepv}[1]{(\citeauthor{#1} \citeyear*{#1} [this volume])}
\newcommand{\citealtv}[1]{\citeauthor{#1} \citeyear*{#1} [this volume]}

%parentheses around example number
\newcommand{\pref}[1]{(\ref{#1})}

% in-text examples

\newcommand{\lnex}[1]{\textit{#1}} %target lang word
\newcommand{\lnlit}[1]{(lit.: `#1')} %literal reading
\newcommand{\lnlat}[1]{(#1)} % latinization
\newcommand{\lntrans}[1]{`#1'} %translation
\newcommand{\lnexl}[2]%
{\lnex{#1}{} \lnlat{#2}} % ex with latinization
\newcommand{\lnexlat}[3]{\lnex{#1}{} \lnlat{#2}{} \lntrans{#3}} % ex with latinization and tranl.

%ch01
\newcommand{\co}[1]{\mbox{\textbf{#1}}}

%ch09

\newcommand{\cyrbulg}[1]{\begin{otherlanguage*}{bulgarian}#1\end{otherlanguage*}}


%ch10
\newcommand{\nlp}{{\small NLP}}
\newcommand{\mwe}{{\small MWE}}
\newcommand{\rae}{{\small RAE}}
\newcommand{\lvc}{{\small LVC}}
\newcommand{\pos}{{\small P}o{\small S}}
%\newcommand{\todo}[1]{ \textcolor{red}{#1} }

%\renewcommand{\labelenumi}{\theenumi}
%\ainamefmt{{vv}{ll}{, ff}{, jj}} % fullname

\newcommand{\biberror}[1]{{\color{red}#1}}

\newcommand{\osenovaitem}{--~} 
  %% hyphenation points for line breaks
%% Normally, automatic hyphenation in LaTeX is very good
%% If a word is mis-hyphenated, add it to this file
%%
%% add information to TeX file before \begin{document} with:
%% %% hyphenation points for line breaks
%% Normally, automatic hyphenation in LaTeX is very good
%% If a word is mis-hyphenated, add it to this file
%%
%% add information to TeX file before \begin{document} with:
%% %% hyphenation points for line breaks
%% Normally, automatic hyphenation in LaTeX is very good
%% If a word is mis-hyphenated, add it to this file
%%
%% add information to TeX file before \begin{document} with:
%% \include{localhyphenation}
\hyphenation{
    Beck-man
    Ngu-yen
    back-chan-nel
    back-chan-nels
    mo-not-o-nous
    ste-reo-typ-i-cal
}

\hyphenation{
    Beck-man
    Ngu-yen
    back-chan-nel
    back-chan-nels
    mo-not-o-nous
    ste-reo-typ-i-cal
}

\hyphenation{
    Beck-man
    Ngu-yen
    back-chan-nel
    back-chan-nels
    mo-not-o-nous
    ste-reo-typ-i-cal
}
 
  \togglepaper[11]%%chapternumber
}{}

\begin{document}
\maketitle 

\section{Introduction}

In this chapter, we want to focus on non-anaphoric third person plural subjects. Non-anaphoric third person plurals consist of a verb agreeing with a third person null subject which, contrary to their anaphoric counterparts, lack an antecedent in the linguistic context. This is why we resort to the term “non-anaphoric third person plurals”, following \citet{CabredoHofherr2006}. More specifically, we will look into the impact of the presence of a discourse participant object, that is, an object in first or second person (be it direct or indirect), on the use of this form. We will do this through a comparison with non-discourse participant object pronouns (third person forms), focusing particularly on the ones referring to an animate entity, since those are most likely to have similar functions to the (by definition animate) discourse participants. 


Example \REF{ex:pierre:1} combines a non-anaphoric third person plural (\textit{han criticado} ‘have criticized’) with a first person pronoun object pronoun \textit{(me} ‘me’) and illustrates the phenomenon we are interested in. 


\ea\label{ex:pierre:1}
Los sistemas eléctricos son una de las características que los diferencian y también otros importantes, pero alargaría mucho esto y ya \ExHighlight{me han criticado} por largas respuestas, pero si quieres aprender tienes que leer. Saludos (YAHOO)
\glt ‘Electrical systems are one of the features that differentiate them and also other important ones, but I would lengthen this a lot and \ExHighlight{they} already \ExHighlight{have} \ExHighlight{criticized} \ExHighlight{me} for long answers, but if you want to learn you have to read. Regards’
\z 

Non-anaphoric third person plural forms have mainly been studied as impersonal constructions and have been considered to have a “non-referential human subject which excludes the speaker and the addressee” (\citealt[577]{SiewierskaPapastathi2011}). Given the above, non-anaphoric uses of the third person plural are considered “to be impersonal under the functional, agent defocusing view of impersonality which associates defocusing of an agent with loss of subject status and/or lack of full referentiality” \citep[74]{Siewierska2010}. As such, this use can be considered functionally similar to passive structures. Such agent-defocusing structures, among which we can find constructions such as the periphrastic passive (auxiliary with a past participle), the \textit{se}-construction or the numeral-based \textit{uno}, defocus or demote the agent of a verbal clause through under-elaboration, non-elaboration or both processes (see \citealt{Siewierska2008}). As Spanish is a pro-drop language, the presence of a subject pronoun is not obligatory, the subject being retrievable from the verbal morphology. Traditionally, it has been considered that the presence of a subject pronoun even cancels the non-anaphoric reading of the Spanish third person plural form \citep[1739]{FernándezSoriano1999Construcciones}. However, \citet{LapidusOtheguy2005} have shown that in some varieties of Spanish, a non-anaphoric reading is possible with overt \textit{ellos} ‘they’. Such cases were not found in the data studied for the present analysis.


Focusing on colloquial English, \citet[35--36]{WeinerLabov1983} show that the non-anaphoric third person plural can be considered a major alternative for the agentless passive. \citet[194]{DeCock2014} has shown previously that it is also much more frequent in Spanish informal spoken interaction than in more formal spoken interactions, such as TV debates or parliamentary debates. In the more formal interactions, other agent-defocusing structures such as the \textit{se}-passive and the periphrastic passive are more frequent, suggesting that in some cases the non-anaphoric third person plural can function as an alternative for Spanish agentless passive constructions. While there is a wider variety of agent-defocusing constructions in Spanish, we focus particularly on the alternation between a periphrastic passive and a non-anaphoric third person plural form, which has been described more amply for English. We will occasionally refer to the alternation between the non-anaphoric third person plural form and other agent-defocusing constructions as well, but refer the reader to \citet{Pierre2021} for a more elaborate analysis of the interaction between all agent-defocusing constructions.


Different uses of the non-anaphoric third person plural form have been identified among others by \citet{CabredoHofherr2006} and brought together in a semantic map by \citet{SiewierskaPapastathi2011}, based on criteria such as the delimitation through a locative, predicative, time or inference. \citet{Posio2015Alcance} looks into the applicability of this proposal for Spanish. Through a study of a translation corpus of some Harry Potter chapters, \citet{SiewierskaPapastathi2011} point out the importance of analyzing the role of speech act verbs with non-anaphoric third person plural forms (\citeyear[606]{SiewierskaPapastathi2011}). In order to investigate the importance of speech act verbs (and other verbs), we will analyze with which verb types the non-anaphoric third person plurals occur in various genres. We then expect these speech act verbs to be frequently used with non-anaphoric third person plurals. We will furthermore look into the impact of the presence of an object pronoun, particularly discourse participant object pronouns, in order to offer a more detailed account of the impact of the verb type and the presence of a discourse participant on the use of non-anaphoric third person plural subjects. 


Indeed, we wish to focus on the presence of discourse participant object pronouns with these constructions since they are often the most referential element of the utterance, given that the human subject is non-referential \citep[577]{SiewierskaPapastathi2011}. We thus expect this highly referential element to have an impact on the utterance as a whole and to fulfill a more crucial role for the discursive progression than the non-specific third person plural subjects.


The article is structured as follows. Data and methodology are presented in \sectref{sec:pierre:2}.  Using comparable corpora from different situations, we start by offering a quantitative overview of the data (\sectref{sec:pierre:3}). Then, in \sectref{sec:pierre:4}, we analyze with which semantic verb types and with which semantic roles these (non)-discourse participant pronouns occur (\sectref{sec:pierre:4.1}). The analysis then focuses on the pragmatic-discursive effects of the construction choice on the presentation of the first and second person (\sectref{sec:pierre:4.2}). Finally, we examine the presence of the phenomenon in function of genre and register variation (\sectref{sec:pierre:4.3}). \sectref{sec:pierre:5} synthesizes our main observations and presents the general conclusions drawn from this corpus-based study.



\section{Data and methodology}\label{sec:pierre:2}

In order to obtain a fine-grained description of the presence of direct object pronouns with non-anaphoric third person plurals, the present investigation looks into the presence of this phenomenon in both formal and informal registers and oral and written modes. Formal data consist of excerpts from the Wikipedia corpus \citep{ReeseRigau2010} for the written part and from European parliamentary debates (PROCEP) for the oral part. Only two occurrences of a non-anaphoric third person plural with a discourse participant were found in formal data. We will thus not provide further information on these corpora (for more details see \citealt[74--78]{Pierre2021}) but the low frequency of the phenomenon in formal data is in itself a relevant finding. The informal written part consists of extracts from the Spanish version of the Yahoo-based Contrastive Corpus Questions \& Answers (see \citealt{DeSmet2009}) compiled from the digital forum Yahoo Questions and Answers. A total of 46,603 words from daily life topics such as means of transportation and food habits has been analyzed. The informal oral part of the data comes from the corpus Español Lengua Oral (ESLORAv.2) (see \citealt{BarcalaVillapol2018} or  \citealt{VázquezRozasEtAl2020} for a detailed description of the corpus) and consists of spontaneous conversations between two or more interlocutors, usually friends or relatives, recorded between 2007 and 2015. The metadata available with the corpus allowed us to select only the participants whose main language is Castilian Spanish and to ensure that they use it on a daily basis. Participants whose native and main language was Galician or another official language of Spain were discarded. Similarly to the data selected from the digital forum, conversations mainly address topics from day-to-day life. The oral corpus totals 76,749 words. This greater size of the oral corpus compared to the written one is explained by the fact that the data used in the present work are part of a larger investigation on structures that are functionally similar to the passive, which are, overall, less frequent in oral data \citep[80]{Pierre2021} and required a larger corpus to collect sufficient material. All the examples used in this Chapter are presented in their original form and, thus, may contain non-standard language.


The degree of formality of language production can be measured following various methods such as the heuristic one suggested by \citet{HeylighenDewaele1999} or the continuum offered by \citet{Briz2010} which considers features such as the relation between the interlocutors, the degree of shared knowledge, the setting and the topic of the interaction and the planning and the tone of the discourse \citep{Briz2010}. We rely on this latter methodology to classify the data used in the present study and thus consider the interactions collected from the Yahoo and ESLORA corpora as informal language, whereas the fragments extracted from Wikipedia articles and parliamentary debates exhibit a higher degree of formality and are classified as samples of formal language.


All occurrences of non-anaphoric third person plurals were extracted by a combination of automatized searches. This process was carried out in the corpus processing system Unitex,\footnote{Unitex/GramLab retrieved from \url{https://unitexgramlab.org/fr}.} a text searching program that allows the automatic retrieval of linguistic phenomena. This process was followed by a manual revision to ensure the agent-defocusing character of each example. A manual annotation of the data was then performed on all the non-anaphoric third person plurals. Two parameters have been annotated: (i) the verb types, following the taxonomy developed in the \textit{Alternancias de Diátesis y Esquemas Sintáctico-Semánticos del Español} (ADESSE) project (\citealt{García-MiguelAlbertuz2005}), designed by the University of Vigo (Spain) and (ii) the syntactic and semantic role of the (non)-discourse participants.



\section{Quantitative overview}\label{sec:pierre:3}

Following the extraction methodology presented above, a total of 498 occurrences of non-anaphoric third person plurals was extracted, which gives a normalized frequency of 23.85 structures per ten thousand words. Out of this set, a manual revision and annotation made it possible to identify non-anaphoric third person plurals used with a discourse participant (first and second person object pronouns) and/or a non-discourse participant object pronoun (third person object pronoun). No cases of \textit{usted/ustedes}, the polite form of address of the pronoun \textit{you}, were found in our data. This amounts to 274 occurrences\footnote{Nineteen occurrences of non-anaphoric third person plurals combine two discourse participants, one in direct object position and one an indirect object position. This phenomenon is illustrated and discussed in \sectref{sec:pierre:3}.} of (non)-discourse participant object pronouns, which represents more than half of all the occurrences of the non-anaphoric third person plurals. These occurrences constitute the final data set of the present study. The next two examples illustrate cases of non-anaphoric third person plurals with a discourse participant (the second person singular \textit{te} ‘you’ in Example \REF{ex:pierre:2}) and with a non-discourse participant, illustrated in \REF{ex:pierre:3} with a third person plural pronoun \textit{les}.\footnote{We have slightly modified the transcriptions of the ESLORA corpus in order to distinguish the pauses more clearly from the remainder of the production.}


\ea\label{ex:pierre:2}
\begin{xlist}[Speaker 2]
\exi{H2} no se me ocurrí[a tampoco pero es que] si hacía esas cosas mi padre me daba un / so[papo]
\exi{Speaker 2} ‘I didn't thin[k of it either but] if I did those things my father would give me a / s[lap]’

\exi{H1}{[ni se te ocurría]}
\exi{Speaker 1} ‘[you wouldn't even think of it]’~

\exi{H1}{[sí oh] <inint> <Pausa>}
\exi{Speaker 1} ‘[yes oh] <inint> <Pause>’

\exi{H2} y me decía / <cita>[no vuelves] a salir en [tu vida]</cita>
\exi{Speaker 2} ‘and he would tell me / <quote>[you'll never again] go out in [your life] </quote>’

\exi{H1}{[lo que pasa que]}
\exi{Speaker 1} ‘[the thing is]’

\exi{H1}{[ahora no se les] puede pegar que que= es maltrato infantil / y \ExHighlight{te pueden denunciar} <Pausa> ¿sabes?}
\exi{Speaker 1} ‘[now you can't] hit them that= it's child abuse / and \ExHighlight{they} \ExHighlight{can} \ExHighlight{report} \ExHighlight{you} <Pause> you know?’

\exi{H2} pues te voy a decir una [cosa]
\exi{Speaker 2} ‘Well, I'm going to tell you [something]’~~

\exi{H1}{[fff]  <Pausa>}
\exi{Speaker 1}{[fff]  <Pause>}

\exi{H2} unas hostias a tiempo solucionan muchas cosas
\exi{Speaker 2} ‘A few punches in time will solve a lot of things.’ (ESLORA)
\end{xlist}
\ex\label{ex:pierre:3}
\begin{xlist}[Speaker 2]
\exi{H2} que una vez se quejaron los peregrinos que= que \ExHighlight{les habían cobrado} por= / meterlos en un pabellón porque normalmente <Pausa>
\exi{Speaker 1} ‘that pilgrims once complained that= \ExHighlight{they} \ExHighlight{had} \ExHighlight{been}   \ExHighlight{charged} for= / putting them in   a pavilion because they were normally <Pause>’

\exi{H1} es gratis
\exi{Speaker 1} ‘it’s free’ (ESLORA)
\end{xlist}
\z 

Third person pronouns represent participants who are not directly involved in the interaction but rather participants about whom certain facts or events are reported.


\begin{table}
\small
\begin{tabular}{l *5{S[table-format=2.0]} S[table-format=3.0]}
\lsptoprule
& \multicolumn{3}{c}{Raw number direct object} & \multicolumn{2}{c}{Raw number indirect objcet} & \\\cmidrule(lr){2-4} \cmidrule(lr){5-6}
Type of     &  {Ani-} &  {Inani-} &  {Indetermi-} &  {Animate} &  {Inanimate} & {Total}\\
participant & {mate}  & {mate}    & {nate}         & {} & {}    &  \\
\midrule
\GlossMarkup{1SG} & 16 &  0 &  0 &  34 &  0 &  50\\
\GlossMarkup{1PL} &  1 &  0 &  1 &  4  &  0 &  6\\
\GlossMarkup{2SG} & 12 &  0 &  0 &  54 &  0 &  66\\
\GlossMarkup{2PL} &  0 &  0 &  0 &  3  &  0 &  3\\
\midrule
 Sub-total         &  29 &  0 &  1 &  95 &  0 &  125\\
 \midrule
\GlossMarkup{3SG}  &  14 &  65 &  2 &  23 &  3 &  107\\
\GlossMarkup{3PL}  &  17 &  13 &  1 &   9 &  2 &  42\\
\midrule
Sub-total         &  31 &  78 &  3 &  32 &  5 &  149\\
\midrule
 Total &  &  &  &  &  & 274\\
\lspbottomrule
\end{tabular}
\caption{Distribution of (non-discourse) participants according to their syntactic role and animate nature (raw frequencies).}
\label{tab:pierre:1}
\end{table}


As indicated in \tabref{tab:pierre:1}, 125 non-anaphoric third person plurals appear with a discourse participant object pronoun and 149 with a third person object pronoun. When used with a discourse participant, results show that it is preferably a singular pronoun since, out of the 125 discourse participants, first and second person singular pronouns total 50 and 66 occurrences, respectively. The overall dominance of singular pronouns can partly be due to first and second person singular pronouns being generally more frequent than their plural counterparts in our data. These two object pronouns are illustrated in \REF{ex:pierre:4} and \REF{ex:pierre:5}, respectively.

\ea\label{ex:pierre:4}
\begin{xlist}[Speaker 2]
\exi{H2} el ciclo este que quiero hacer después de= de lo de Forestales que es de= / también de Forestales de lo de= <Pausa>
\exi{Speaker 2} ‘the cycle that I want to do after= the Forestry course which is from= / also from Forestry from= <Pause>’

\exi{H1} sí  <Pausa>
\exi{Speaker 1} ‘yes <Pause>’

\exi{H2} <\#> ¿cómo se llama? / gestor del= // de recursos forestales o algo así <Pausa> al  tener  la carrera {\ExHighlight{{me} \ExHighlight{convalidan}}}{ {en primero / dos asignaturas <Pausa>}} 
\exi {Speaker 2} {‘<\#> what is it called? / manager of the= // of forestry resources or something like that Pause as I have the degree} {\ExHighlight{they} \ExHighlight{validate}} {me in the first / two subjects  <Pause>’ (ESLORA)}
\end{xlist}
\ex\label{ex:pierre:5}
\begin{xlist}[Speaker 2]
\exi{H2} pero no la llamaron de ningún lado y [empez]ó ahí porque necesitaba dinero
\exi {Speaker 2} {‘but they didn't call her from anywhere and she [started] there because she needed money’}

\exi{H1} {[claro] claro <Pausa>}
\exi {Speaker 1} ‘[of course] of course <Pause>’

\exi{H2} pues es lo que hay que hacer <Pausa>  tú has estudiado para <Pausa>  pues para trabajar [si]
\exi{Speaker 2} ‘so that's what you have to do <Pause> you have studied for <Pause> so to work [if]’

\exi{H1}{[<inint>] pues ya \ExHighlight{{te} \ExHighlight{llamarán}}{= [<inint>]} }
\exi{Speaker 1} ‘[<inint>] then {\ExHighlight{they} \ExHighlight{will} \ExHighlight{call} \ExHighlight{you}}{= [<inint>]’} 

\exi{H2}{[si hay suerte] <Pausa>}
\exi{Speaker 2} ‘[if there is luck] <Pause>’ (ESLORA)
\end{xlist}
\z

Though in substantially more limited proportions, it is interesting to note that plural discourse participants are not absent from non-anaphoric third person plurals.\largerpage[2]


\eanoraggedright\label{ex:pierre:6}
\begin{otherlanguage}{spanish}
{Me gustaría hacer un par de puntualizaciones en este sentido. Está bien que hablemos de limitaciones de la producción, pero estas limitaciones de la producción tienen que venir asociadas al concepto de soberanía alimentaria, porque lo que no puede ser es que} {\ExHighlight{{nos} \ExHighlight{hagan}}}{ {lo mismo que con la leche:} }{\ExHighlight{{no} \ExHighlight{nos} \ExHighlight{digan}}} — a España — que tenemos que producir menos leche para que se invada nuestro mercado de leche extranjera, porque eso no nos hace más sostenibles, nos hace más pobres. {(PROCEP})
\end{otherlanguage}\pagebreak
\glt ‘I would like to make a couple of clarifications in this regard. It is right that we talk about production limitations, but these production limitations must be associated with the concept of food sovereignty, because what cannot be is that \ExHighlight{they} \ExHighlight{do} the same to us as with milk: \ExHighlight{they} \ExHighlight{do} \ExHighlight{not} \ExHighlight{tell} \ExHighlight{us} – Spain – that we have to produce less milk so that our market is invaded by foreign milk, because that does not make us more sustainable, it makes us poorer.’
\z 

Example \REF{ex:pierre:6}, an excerpt from parliamentary debates, shows an instance with the first person plural object pronoun \textit{nos} (‘us’). This ties in with a higher overall presence of first person plural forms in parliamentary debates, where speakers typically are spokespersons for a whole group, typically their political party (\citealt{Gelabert-Desnoyer2006a,Gelabert-Desnoyer2006b}, \citealt{DeCock2014}: 35). In this excerpt from the European Parliament, the speaker specifies that she speaks for her whole national group (by referring to \textit{a España}), thus resolving a possible ambiguity between reference to the political or national identity.


While first and second person pronouns, that is, pronouns encoding participants directly involved in the interaction, are rarely used in the plural in our data, third person plural pronouns, illustrated in \REF{ex:pierre:7} with the direct object pronoun \textit{las} ‘them’, tend to be more commonly employed.\largerpage

\ea\label{ex:pierre:7}
\begin{xlist}[Speaker 2]
\exi{H1}{[parece ser que es bas]tante timo el bufé<@@> <Pausa>}
\exi {Speaker 1} ‘[it seems to be qui]te a fraud the buffet<@@> <Pause>’

\exi{H2} ¿sí? <Pausa>

\exi {Speaker 2} ‘Yes? {{<Pause>’} }

\exi{H1} creo que son= para entrar diez euros <Pausa> pero tienes que tomar una consumición <Pausa> y las consumiciones dicen que {\ExHighlight{{las} \ExHighlight{clavan}}}{{= bien clavadas <Pausa>}}

\exi {Speaker 1} ‘I think it's= ten euros to get in {{<Pause>} }{but you have to have a drink} {{<Pause>} }{and they say that the drinks} {\ExHighlight{they} \ExHighlight{ripped} }{them off = well ripped off} {{<Pause>’}}

\exi{H2} ¡bua! pues yo cuando fui a Barcelona Pausa pagué diez euros / con= bebida incluida
\exi {Speaker 2 ‘Wow! well when I went to Barcelona} {{<Pause>} }{I paid ten euros / with= drink included’ (ESLORA)}
\end{xlist}
\z 

We also analyzed the syntactic role of the discourse participant, that is, direct or indirect object and identified its nature (animate versus inanimate), as it interacts with the verb type used in the construction.


A look at the syntactic role of the object pronouns reveals that discourse participants are preferably used as indirect objects, that is, as the receiver or the beneficiary of the action (95 occurrences out of 125), and especially the second person singular pronoun (54 occurrences). However, a larger set of data would be necessary to confirm the tendency. A comparison with non-discourse participants shows that, while discourse participants are preferably used as indirect objects, non-discourse participants are rather direct objects, reflecting more general tendencies of indirect objects often having animate referents. The relatively high proportion of inanimate third person singular pronouns (65 occurrences out of 107 third person singular pronouns) largely contributes to this result.  


The analysis also highlights 19 cases where both a direct and indirect object pronoun co-occur with the non-anaphoric third person plural. In our data, these cases always combine a discourse with a non-discourse participant, as illustrated in \REF{ex:pierre:8}, which includes a third person direct object \textit{lo} ‘it’ and a first person indirect object \textit{me} ‘me’.

\ea\label{ex:pierre:8}
\begin{xlist}[Speaker 2]
\exi{H2} que sí que= había visto algún{}- disfraces muy chulos y todo / lo [que pasa que] yo no entiendo nada de lo de= ya te digo de Comadres y todas estas historias porque nunca {\ExHighlight{{me} \ExHighlight{lo} \ExHighlight{[explicaron}}}{ {tampoco]} }
\exi{Speaker 2} ‘that yes that= I had seen some very cool costumes and everything / what [happens that] I don’t understand anything about= I’m telling you about Comadres and all these stories because {\ExHighlight{they}} {never [}{\ExHighlight{explained} \ExHighlight{it} \ExHighlight{to}} {me either].’}
\exi{H1}{[claro][pues eso / Comadres] y Compadres es así creo que es un día que se disfrazan todos de ho- mujeres y otro día también pero es más de postureo [¿sabes?]}
\glt{Speaker 1} ‘[of course] [so that / Comadres] and Compadres is like that I think it’s a day when they all dress up as me- women and another day too but it’s more like pretending [you know?]’  (ESLORA)
\end{xlist}
\z 

As will be discussed in \sectref{sec:pierre:4.1}, the syntactic role of the object pronouns is closely linked to the semantics of the verb. 


Regarding the nature of the object pronouns, that is, animate versus inanimate,\footnote{Three occurrences where categorized as \textit{indeterminate}. In these cases it was not possible to determine the nature of the pronoun due to a lack of context.} results indicate that animate pronouns dominate, whether the object discourse participant acts as a direct complement or an indirect one. The tendency is not observed when the object pronoun does not act as a discourse participant, in other words, when it is a third person pronoun \REF{ex:pierre:9}. 


\ea\label{ex:pierre:9}
\begin{xlist}[Speaker 2]
\exi{H1} bueno yo mañana sin falta lo pago sí porque yo creo que estoy fuera de plazo total= <Pausa>
\exi{Speaker 1} ‘well tomorrow without fail I will pay it yes because I think I am out of schedule total= <Pause>’

\exi{H1} y si \ExHighlight{{la} \ExHighlight{cortan}} {por lo menos que no sea de noche <Pausa>}
\exi{Speaker 1} ‘and if \ExHighlight{they} \ExHighlight{cut} \ExHighlight{it}, at least not at night {<Pause>’}

\exi{H2} sí / toma to[ma] H1 vete a saber si mañana nos levantamos    sin luz ¡eh!
\exi{Speaker 2} ‘yes / take ta[ke] H1 you never know if tomorrow we'll wake up without electricity, eh!’ (ESLORA)
\end{xlist}
\z 

Third person pronouns display a clear trend towards the substitution of direct object nouns referring to inanimate entities \REF{ex:pierre:9} or part of the discourse \REF{ex:pierre:10}. The alternative for \REF{ex:pierre:9} in a periphrastic passive, namely \textit{[la electricidad]} \textit{es cortada} ‘the electricity is cut off’ is imaginable but unlikely in colloquial language, underlining again the importance of the non-anaphoric third person plural as an agent-defocusing strategy in this register.

\ea\label{ex:pierre:10}
\begin{xlist}[Speaker 2]
\exi{H1} y si faltas a la práctica / estás suspenso [/ tú eliges o ha]cer huelga o ir a la prácti[ca]
\exi{Speaker 1} ‘and if you miss practice / you are suspended [/ you choose whether to] strike or go to prac[tice]’

\exi{H2}{[ho]y lo dije[ron {\ExHighlight{{lo} \ExHighlight{dijeron}}}] en la manifestación rollo que había profesores}
\exi{Speaker 2} ‘[to]day [{\ExHighlight{they} \ExHighlight{said} \ExHighlight{it}} said it]’ in the strike that they were teachers’

\exi{H1}{[<dud>hay dos</dud>]  <Pausa>}
\exi{Speaker 1} `[<doubt>there are two</doubt>] <Pause>’

\exi{H2} incluso ponían exámenes a propósito estos tres días para que la gente no faltara
\exi{Speaker 2} ‘they even put exams on purpose these three days so that people wouldn't be absent’ (ESLORA)
\end{xlist}
\z 


It is interesting to note that when third person direct object pronouns are used to substitute an animate entity, they do not stand for humans but also frequently refer to an animal (18 occurrences out of 31). In our data, this result does not apply to indirect object pronouns since animals account for 2 occurrences out of 32. 


In addition to the variation in number and person, in syntactic role and nature of the object, the analysis has revealed that discourse participants included in non-anaphoric third person plurals fulfil different semantic roles, as will be discussed in more detail in the qualitative analysis.

\section{Exploring the data: A qualitative analysis}\label{sec:pierre:4}

\subsection{Types of processes}\label{sec:pierre:4.1}

The analysis of the \textit{type of process} involved in a non-anaphoric third person plural with and without an object pronoun provides valuable insights into the understanding of the use of the agent-defocusing mechanism and the subsequent semantic roles of discourse participants.


Experience “consists of a flow of events, or ‘goings-on’” \citep[170]{HallidayMatthiessen2014}. In \citegen{Halliday1994} terminology, this flow of events is divided into different \textit{process types} which reflect experiences of happening, doing, sensing, saying, being or having (see \citealt{HallidayMatthiessen1999}). They unfold through time and have participants being directly involved in this process in some way; and, in addition, there may be circumstances of time, space, cause, manner or one of a few other types \citep[170]{HallidayMatthiessen2014}. Building on this systemic functional approach and the adapted classification offered in the ADESSE database (\textit{Alternancias de Diátesis y Esquemas Sintáctico-Semánticos del Español}), designed by the University of Vigo (Spain) (\citealt{García-MiguelAlbertuz2005}), verbs will be classified according to the type of process they reflect: \textit{mental}, \textit{relational}, \textit{material}, \textit{verbal}, \textit{existential} and \textit{modulation}\footnote{Modulation processes include verbs expressing causation (e.g., \textit{to help} or \textit{to allow}), acceptation (e.g., \textit{to accept} or \textit{to reject}), disposition (e.g., \textit{to dare} or \textit{to try}) and verbs of support. This latter category gathers verbs that combine with nominal clauses to form semantically complex phrases which confer another meaning on them than the one conveyed by the base verb.} (labelled according to \citealt{García-MiguelAlbertuz2005}’s work). 


\begin{table}
\begin{tabular}{lllll}

\lsptoprule

 Main &  Subtype of &  Non- &  Non- &  Examples\\
{category of} & {process} & {anaphoric} & {anaphoric} &\\
{process} & & \GlossMarkup{3PL} & \GlossMarkup{3PL} &\\
& & {with} & {without} & \\
& & {object} & {object} &\\
& & {pronouns} & {pronouns} &\\
\midrule
{Mental}  &  Sensation  &  2 (0.8) &  3 (3.5) & \textit{querer} (‘to\\
& & & & love, to\\
& & & & want’)\\
% \cmidrule{2-5}
&  Perception  &  12 (4.7) &  0 & \textit{ver} (‘to see’)\\
% \cmidrule{2-5}

&  Cognition  &  3 (1.2) &  5 (5.8) & \textit{enseñar} (‘to\\
& & & & teach’)\\
\midrule
Relational &  Attribution  &  9 (3.5) &  2 (2.3)  & \textit{asignar} (‘to\\
& & & & assign’)\\
% \cmidrule{2-5}
&  Possession  &  81 (31.9) &  7 (8.1)  & \textit{cobrar} (‘to\\
& & & & charge’),\\
& & & & \textit{pagar} (‘to\\
& & & & pay)\\
\midrule
{Material}  &  Space  &  31 (12.2) &  23 (26.7) & \textit{guardar} (‘to\\ 
& & & & keep’),\\
& & & & \textit{aislar}, (‘to\\
& & & & isolate')\\
% \cmidrule{2-5}
&  Change  &  23 (9.1) &  5 (5.8) & \textit{reconstruir}\\
& & & & (‘to rebuild’) \\
% \cmidrule{2-5}
&  Other facts  &  16 (6.3) &  3 (3.5) & \textit{utilizar} (‘to\\
& & & & use’)\\
% \cmidrule{2-5}
&  Behaviour  &  5 (2) &  1 (1.2) & \textit{violar} (‘to\\
& & & & rape’)\\

\lspbottomrule
\end{tabular}
\caption{Verb types in non-anaphoric third person plurals with direct and indirect object pronouns vs.  without object pronouns (raw frequencies (\%)).}
\label{tab:pierre:2}
\end{table}

\begin{table}
\begin{tabular}{lllll}

\lsptoprule
 Main &  Subtype of &  Non- &  Non- &  Examples\\
{category of} & {process} & {anaphoric} & {anaphoric} &\\
{process} & & {3PL} & {3PL} &\\
& & {with} & {without} & \\
& & {object} & {object} &\\
& & {pronouns} & {pronouns} &\\
\midrule
{Verbal}  & Communi- & 50 (19.7) & 30 (34.9) & \textit{explicar} (‘to\\
& cation & & & explain’),\\
& & & & \textit{decir} (‘to say’)\\
&  Assessment  &  2 (0.8) &  0 & \textit{criticar} (‘to\\
& & & &  critize’)\\
% \cmidrule{2-5}
\midrule
{Modulation}  &  Causation  &  4 (1.6) &  0 & \textit{ayudar} (‘to\\
& & & & help’)\\
&  Acceptation &  3 (1.2) &  2 (2.3) & \textit{aceptar} (‘to\\
& & & & accept’)\\
&  Verbs of &  9 (3.5) &  4 (4.7) & \textit{darse cuenta}\\
& support & & &  (‘to realize’)\\
% \cmidrule{2-5}
\midrule
Existential &  Phase-Time  &  2 (0.8) &  1 (1.2) & \textit{desencadenar}\\
& & & & (‘to trigger’)\\

&  Life  &  2 (0.8) &  0 & \textit{embarazar} (‘to\\
& & & & get [sb]\\
& & & & pregnant’)\\
% \cmidrule{2-5}
\midrule
{Not}  &  &  1 (0.4) &  0 & \\
{codifiable} & & & &\\
{due} {to} & & & &\\
{truncated} & & & &\\
{utterance} & & & &\\
\midrule
&  Total  &  255 (100) &  86 (100) & \\
\lspbottomrule
\end{tabular}
\caption{Verb types in non-anaphoric third person plurals with direct and indirect object pronouns vs.  without object pronouns (raw frequencies (\%)).}
\label{tab:pierre:3}
\end{table} % table 2 and 3!! change

Let us first compare the behaviour of non-anaphoric third person plurals with and without object pronouns. Tables~\ref{tab:pierre:2} and \ref{tab:pierre:3} reveal that when used with object pronouns, verbs of \textit{possession} largely predominate (31.9\%, 81 occurrences), followed by verbs of \textit{communication} and verbs denoting \textit{spaces}. These findings will be deepened in what follows. A different picture emerges when non-anaphoric third person plural subjects do not co-occur with object pronouns, since it appears that \textit{possession} verbs are hardly used (8.1\%) while \textit{communication} verbs (34.9\%) and, to a smaller extent, \textit{space} verbs (26.7\%) are the only two prevalent categories. These results point towards a different use of the non-anaphoric third person plurals when accompanied by object pronouns.


Interestingly, when non-anaphoric third person plurals are used with object pronouns, the study indicates that the results can be influenced by the frequent use of specific verbs. It is the case for the semantic category \textit{communication} where the verb \textit{decir} (‘to say’) predominates, followed by the verb \textit{llamar} (‘to call’). A similar pattern occurs with \textit{possession} verbs, where the verb \textit{dar} (‘to give’) largely dominates, followed by \textit{pagar} (‘to pay’) and \textit{vender} (‘to sell’). The tendency for one or a restricted number of verbs to appear frequently is not noticed in the semantic category that includes \textit{space} verbs. These observations lead us to assume that, depending on the semantic category, it is either a specific verb within the category or the semantics of the category itself which plays a decisive role in explaining the behaviour of non-anaphoric third person plurals. A brief review of the structure without object pronouns reveals a comparable phenomenon, the only difference being that \textit{tener} (‘to have’) is the most frequent verb in the category \textit{possession} verbs. As figures are lower when non-anaphoric third persons do not occur with an object pronoun, this result should be considered with caution. 


In what follows, we will examine the potential association of object pronouns with the different types of verbs.

\begin{table}
\begin{tabular}{llll}

\lsptoprule

 {Main} {category} &  {Subtype} {of} &  {Discourse} & {Non-discourse}\\
{of} {process} &  {process} & {participant} & {participant}\\
& & {Raw} {number} & {Raw} {number}\\
& & {(\%)} & {(\%)}\\
 \midrule
 {Mental} &  Sensation  &  1 (0.8) &  1 (0.7)\\
&  Perception  &  7 (5.6) &  6 (4)\\
&  Cognition  &  2 (1.6) &  1 (0.7)\\
% \cmidrule{2-4}
\midrule
 {Relational} &  Attribution  &  4 (3.2) &  6 (4)\\
&  Possession  &  \textbf{{47} {(37.6)}} &  \textbf{{43} {(28.9)}}\\
 % \cmidrule{2-4}
 \midrule
{Material}  &  Space  &  \textbf{{17} {(13.6)}} &  \textbf{{18} {(12.1)}}\\
&  Change  &  2 (1.6) &  {23} {(15.4)}\\
&  Other facts  &  2 (1.6) &  {15} {(10.1)}\\
&  Behaviour  &  2 (1.6) &  3 (2)\\
% \cmidrule{2-4}
\midrule
 {Verbal}  &  Communication  &  \textbf{{29} {(23.2)}} &  \textbf{{22} {(14.8)}}\\
 % \cmidrule{2-4}
 \midrule
&  Assessment  &  2 (1.6) &  0\\
 % \cmidrule{2-4}
 \midrule
 {Modulation}  &  Causation  &  3 (2.4) &  1 (0.7)\\
&  Acceptation  &  2 (1.6) &  1 (0.7)\\
&  Verbs of support &  4 (3.2) &  5 (3.4)\\
% \cmidrule{2-4}
\midrule
 {Existential} &  Phase-Time  &  0 &  2 (1.3)\\
&  Life  &  0 &  2 (1.3)\\
% \cmidrule{2-4}
\midrule
 {Not} {codifiable}  &  &  1 (0.8) &  0\\
 \midrule
& Total  &  125 (100) &  149 (100)\\
\lspbottomrule
\end{tabular}
\caption{Verb types in non-anaphoric third person plurals occurring with object pronouns referring to discourse participants vs. not referring to discourse participants (raw frequencies (\%)).}
\label{tab:pierre:4}
\end{table}



We now focus on discourse participants. It can be seen from \tabref{tab:pierre:4}\footnote{The total of \tabref{tab:pierre:4} amounts to 274 as it includes the 19 cases that combine both a direct and an indirect object.}  that two categories largely predominate: \textit{possession} verbs, which pertain to the head category \textit{relational} verbs, and \textit{communication}, which is one of the two subtypes of \textit{verbal} types of processes. The dominance of these types ties in with the dominance of indirect object forms among discourse participant object pronouns shown in \tabref{tab:pierre:1}, since these verbs privilege an indirect object. This high frequency of communication verbs further underpins the suggestion of \citet{SiewierskaPapastathi2011} to create a specific category of these verbs in their non-episodic use in the analysis of non-anaphoric third person plural forms.


Verbs of \textit{possession} are the most common verbs used with discourse participants (37.6\%, 47 occurrences). This type of verb traditionally receives two semantic labels: belonging, where an entity owns (part of) another entity, and transfer, which implies a change of owner of an entity, where a transfer from an agent (initial owner) to a recipient (final owner) takes place (\citealt{García-MiguelAlbertuz2005}). Example \REF{ex:pierre:11} illustrates this category with the verb \textit{pagan} (‘they pay’) accompanied by the discourse participant \textit{te} ‘you’, in a discussion between pilots and trainee pilots. 

\ea\label{ex:pierre:11}
el descanso es depende del trabajo que tengas si vuelas una aerolinea o si vuelas un privado si es de una aerolinea es por horas son ciertas horas al mes si vuelas mas \ExHighlight{te pagan} mas y si vuelas menos te pagan lo mismo (…). (YAHOO) \\
\glt {‘the rest depends on the job you have if you fly an airline or if you fly private if it is an airline it is by hours it is a certain number of hours per month if you fly more} {\ExHighlight{they} \ExHighlight{pay} \ExHighlight{you}} {more and if you fly less they pay you the same (…).’}\\
\z 

There is a transfer, of money in this case, from the agent (the person in charge of salaries in the airline) to the recipient (the pronoun \textit{te} ‘you’), which would refer to a person working as a pilot or a flight attendant. Based on this deictic use of \textit{te}, this second person singular form can be interpreted in the context of the forum as entailing a broader reference possibly including other forum members and the speaker’s own experience. In her scalar interpretation of reference of second person singular forms, \citet[91]{Kluge2012} qualifies this use as “anyone but addressee as a typical representative”. The discourse participant object occupies a recipient role, becoming the ultimate possessor. \textit{Relational} verbs serve to characterize and to identify (\citealt[210]{HallidayMatthiessen2004}). This type of verb thus helps to establish the relationship among different entities \citep{Um-eAmmaraJaved2019}. \textit{Relational} processes, and more particularly verbs of \textit{possession}, are perfect candidates for the presence of an object since these verbs usually require reference to more than one entity, the main one being the subject. As explained by  \citet[213]{HallidayMatthiessen2004}, while clauses including a \textit{material} process can appear with only one participant, \textit{relational} verbs imply at least two participants. Our results indicate that in the examined construction these participants are mainly indirect objects. 


\textit{Verbal} types of processes express acts of saying, within which participants function either as sayer, verbiage, receiver or addressee (\citealt{HallidayMatthiessen2004}). \textit{Communication} verbs are the second most frequent type of verb process with 23.2\% (29 occurrences). This type of verb is a typical resource to transfer information from one participant to another. As explained by \citet{Pierre2021}, verbal types of processes “engage the speaker on the cognitive and communication level” (\citeyear[146]{Pierre2021}). Example \REF{ex:pierre:12} illustrates a case of a non-anaphoric third person plural used with a verbal process \textit{dicen} ‘tell’ and including a discourse participant, which is, in this case, the second person singular \textit{te} ‘you’. The discourse participant pronoun has then a recipient role. 

\ea\label{ex:pierre:12}
{{entre otras cosas, en el curso te} }{\ExHighlight{{dicen}}}{ {a ke velocidad debes despegar y aterrizar el avion ke te asignaron, a cuantos grados debes girar los flaps segun el clima y la pista, o ke velocidad de crucero debes mantener. en un determinado viaje.. osea la informacion basica suerte!} }{(YAHOO)}{ }\\
\glt {‘among other things, in the course} {\ExHighlight{they} \ExHighlight{tell}} {you at what speed you should take off and land the plane you have been assigned, how many degrees you should turn the flaps depending on the weather and the runway, or what cruising speed you should maintain on a given trip... that is the basic information, good luck!’} \\
\z 

Thus, the two most frequent verb types place the discourse participant object in the recipient role, configured as an indirect object. This shows that the use of these object pronouns with an agent-defocusing strategy, leaving the subject underdetermined, typically configures roles in a context of transfer to a recipient, and not in an inversion of the agent-patient scheme as may be the case with some other agent-defocusing structures, such as the periphrastic passive.  


In addition to processes of \textit{possession} and \textit{communication,} verbs denoting \textit{space} also occur with discourse participants (17 occurrences out of 125, which represent 13.6\% of our data).  

\ea\label{ex:pierre:13}{[speaking about cleaning service]}
\begin{xlist}[Speaker 2]
\exi{H1} porque= / yo // en plan / estaba en= // en la habitación <Pausa>
\exi{Speaker 1} ‘because I was in the room <Pause>’

\exi{H2} hm <Pausa>
\exi{Speaker 2} ‘hm <Pause>’

\exi{H1} y= aún no ac- / no aún no había acabado ¿sabes? // de= [de] // estaba tomando el desayuno / y \ExHighlight{{me} \ExHighlight{echaron}} {básicamente ¿sabes? me dijeron} 
\exi{Speaker 1} ‘and not yet / no I wasn't done yet you know? // of= [of] // I was having breakfast / and \ExHighlight{they} basically \ExHighlight{kicked} \ExHighlight{me} \ExHighlight{out} you know? they said to me’ 

\exi{H2}{[hm]}
\exi{Speaker 2} ‘[hm] <Pause>’

\exi{H1} <cita>bueno vamos a limpiar la otra habitación <Pausa>
\exi{Speaker 1} ‘<quote>well we're going to clean the other room <Pause>’ 

\exi{H1} y venimos</cita> // ¿sabes? <Pausa>
\exi{Speaker 1} ‘and we're coming</quote> // you know // <Pause>’ 

\exi{H1} de esto que <Pausa> no habían pasado ni cinco minutos ¿sabes? // y ya habían venido entonces me fui para vues- / para= vuestra habitación 
\exi{Speaker 1} ‘from this that <Pause> it wasn't even five minutes later you know // and they had already come so I went to your- / to= your room.’  (ESLORA)
\end{xlist}
\z 

In Example \REF{ex:pierre:13}, the object discourse participant, here the first person singular expressed through the pronoun \textit{me} ‘me’, undergoes a change of location: an unspecified agent causes Speaker 1 to leave the room, thus representing the speaker as undergoing a (hostile) action by a group of agents that remains underdetermined. It then underlines the power relation between the patient and agent.  


It has to be noted that the other types of processes present a low frequency of use. \textit{Existential} verbs, which express processes linked to life, existence and phase-time relations, are even absent from non-anaphoric third person plural with discourse participants. This may seem logical since \textit{existential} verbs typically have only one argument, Moreover, a low frequency of use of \textit{existential} verbs was already found in non-anaphoric third person plurals, regardless of the presence of discourse participants \citep[204]{Pierre2021} (see also Tables~\ref{tab:pierre:2} and \ref{tab:pierre:3}).


A closer look at the type of verb processes engaged in non-anaphoric third person plurals co-occurring with a third person object pronoun reveals a relatively different picture. While \textit{possession} verbs remain the dominant types of verb processes, they occur in a smaller proportion (28.9\%) in comparison with non-anaphoric third person plurals with discourse participant objects (37.6\%). In addition, \tabref{tab:pierre:4} indicates that \textit{communication} verbs are slightly less used with non-discourse participants. This category of verb presents a similar frequency of use as verbs expressing \textit{spaces} (14), \textit{changes} (15) or \textit{other facts} (16). The latter two categories appear as rather typical of verbs occurring with non-discourse participants.\largerpage

\eanoraggedright\label{ex:pierre:14}
\begin{otherlanguage}{spanish}
viaje por Air Canada, cuando llegue a CYYZ me no encontre mi maleta, informe en el aeropuerto y me dieron un numero, llame y a los dos dias \ExHighlight{la} \ExHighlight{vinieron} \ExHighlight{a} \ExHighlight{dejar} a la casa… creo que podrias hacer lo mismo o denunciar a la empresa… (YAHOO)
\end{otherlanguage}\pagebreak
\glt {‘I travelled by Air Canada, when I arrived at CYYZ I couldn't find my suitcase, I informed the airport and they gave me a number, I called and two days later} {\ExHighlight{they} \ExHighlight{came}} {\ExHighlight{to} \ExHighlight{drop} \ExHighlight{it} \ExHighlight{off}} {at the house... I think you could do the same or report the company..}{{.’}}
\ex\label{ex:pierre:15}
\begin{otherlanguage}{spanish}
estube viendo un especial del A380, en NatGeo Channel de como lo hicieron desde el diseño hasta su construcción etc, está super interesante ese programa. hablaron de todo, lo de los wingets, como acortaron sus alas acortaron las medidas, donde \ExHighlight{lo} \ExHighlight{construyen}, todo lo que recorre para ir a la linea de ensamblaje final etc. (YAHOO)
\end{otherlanguage}
\glt{‘I was watching a special of the A380, on NatGeo Channel about how they made it from design to construction etc, it was very interesting this program. they talked about everything, the winglets, how they shortened the wings, where} {\ExHighlight{they} \ExHighlight{build} \ExHighlight{it}}{, everything it goes through to go to the final assembly line, etc}{{.’}}
\ex\label{ex:pierre:16}
{{Sin embargo en otras paginas encontre que era un sistema que se diseño en los 60 para los modulos Apolo de la NASA y de ahi} }{\ExHighlight{{lo utilizaron}}}{ {en algunos tipos de misiles a principios de los 70 Honeywell lo ofrecio a Douglas y fue ahi que surgio primeramente montarlo en simuladores, IBM fabricaba los chips y tenian gran variedad de fallas comenzando con sobrecalentamiento.} }{(YAHOO)}
\glt {‘However on other pages I found that it was a system that was designed in the 60's for Apollo modules of the NASA and from there} {\ExHighlight{they used it}} {in some types of missiles in the early 70's Honeywell offered it to Douglas and it was there that it was first mounted in} {simulators, IBM manufactured the chips and they had a variety of failures starting with overheating.’}
\z 

The verb phrase \textit{vinieron a dejar} (‘came to drop off’) in \REF{ex:pierre:14} illustrates the category \textit{space}, where the pronoun \textit{la} which substitutes the suitcase (\textit{maleta}) is moved from one place to another. In \REF{ex:pierre:15}, the verb \textit{construyen} ‘build’ implies a change of state, from non-existence to creation, including the necessary steps to be created. Example \REF{ex:pierre:16} illustrates a verb classified as ‘others’. This rather heterogeneous category includes verbs referring to a physical type of action that does not meet the criteria to be related to changes, space or behaviour. In \REF{ex:pierre:16}, the verb \textit{utilizaron} (‘used') denotes a physical manipulation but the patient, here the pronoun \textit{lo,} does not suffer any modification. 


Verbs expressing \textit{spaces, changes} and \textit{other facts}, pertain to the head category \textit{material verbs}. Our analysis points, thus, towards the use of \textit{material} verbs, which refer to physical actions, as a key characteristic of occurrences with non-dis\-course participants. This ties in with the high presence of inanimate non-dis\-course participant object pronouns, which are more likely to be the object of a material verb.


As a preliminary conclusion, this leads us to assume that the type of verb and the type of participant involved (or not) in the discourse are closely associated. The results have shown that the use of discourse participant objects is favoured by the presence of \textit{relational} verbs, followed by \textit{communication} verbs. Non-discourse participant objects also appear with these categories, though in smaller proportions. Our analysis has indicated that this type of object clearly links to \textit{material} verbs (\textit{changes}, \textit{spaces} and \textit{other facts}), a characteristic not observed in the behaviour of discourse participant objects. Finally, non-anaphoric third person plurals used without any object pronouns exhibit a marked tendency for \textit{communication} verbs and, to a lesser extent, for \textit{space} verbs. The analysis has, thus, helped us highlight the importance of the type of verb in the variation of the presence and the specificities of object pronouns, as well as the types of verbs that are most used with the non-specific third person plural form.



\subsection{Impact of the construction on the representation: A pragmatic-discursive approach}\label{sec:pierre:4.2}


In this section, we will adopt a more pragmatic-discursive analysis and focus on the impact of the construction of a non-anaphoric third person plural with a discourse participant object on the conceptualization of the event. We will also discuss contrasts with non-discourse participant objects and the construction without an object.


Let us first focus on the utterances with a communication verb, one of the most frequent verb types with this construction. \citeauthor{SiewierskaPapastathi2011} consider that the non-episodic uses with \textit{say} should be considered a separate type, rather than being considered as falling under the vague type proposed by \citet{CabredoHofherr2003, CabredoHofherr2006}, which is linked to a specific moment in time (\citealt{SiewierskaPapastathi2011}: 585). Other cases with communication verbs are, however, episodic. In many cases, the discourse participant object pronoun continues the reference of a deictic form in the preceding utterances (or is continued in what follows), showing the central position of the deictic forms for the development of topic continuity. This analysis of the broader discursive context shows indeed that, although the non-anaphoric third person plural form is the subject, the object pronoun with reference to a discourse participant is actually the form that anchors the utterance in the interaction, by referring to the speaker or addressee, and that ensures the topic continuity with regard to the preceding and following parts of the interaction. Thus, from a discursive perspective, focusing on the discourse participant object is key to analyzing the utterance in the broader context. Example \REF{ex:pierre:17}, an episodic use, illustrates this with a conversation concerning surgery on the broken leg of Speaker 1 (H1). 


\ea\label{ex:pierre:17}
\begin{xlist}[Speaker 3]
\exi{HI} (…) me operé en enero mes y medio {\ExHighlight{ \ExHighlight{me} \ExHighlight{llamaron}}}{ {[con la historia d]el seguro} }
\exi{Speaker 1} (…) ‘I got surgery in January a month and a half later {\ExHighlight{they} \ExHighlight{called} \ExHighlight{me}} {[about] the insurance’}
\exi{H3} claro.
\exi{Speaker 3} ‘of course.’
\exi{HI} ¿qué pasa? / que como me la había roto antes, 
\exi{Speaker 1} ‘What happens? That, since I had broken it previously,’ 
\exi{H3}{[sí]}
\exi{Speaker 3} ‘Yes’ 
\exi{HI} era un seguro de fractura
\exi{Speaker 1} ‘it was a fracture insurance (…)’ (ESLORA)
\end{xlist}
\z 

The first person singular object pronoun continues the narrative about being operated on (\textit{me operé} ‘I got surgery’) and is further taken up when recounting a previous fracture (\textit{me la había roto antes} ‘I had broken it previously’). We see then a clear configuration where the object pronoun refers to one of the discourse participants and constitutes the main thread of the narrative through co-reference with previous mention of the speaker. The non-anaphoric third person plural can be interpreted through contextual information (\textit{con} \textit{la historia del seguro} ‘about the matter of the insurance’) as referring to the insurance company and its actions towards the discourse participant, showing the importance of the agent-defocusing strategy for the discursive development.


However, not all 1\textsuperscript{st} and 2\textsuperscript{nd} person object pronouns combine deictic anchoring with establishing topic continuity through coreference. Indeed, in various cases, the discourse participant object does not establish a coreference with preceding or following references, as in \REF{ex:pierre:18}. It is then the main element that anchors the utterance in the interaction.

\ea\label{ex:pierre:18}
{{No es una planta inteligente. Tiene formas de supervivencia, pero de ninguna manera inteligencia. O sea} }{\ExHighlight{{te} \ExHighlight{mintieron}}}{!!! }{(YAHOO)}\\
\glt ‘It is not an intelligent plant. It has forms of survival but by no means intelligence. So \ExHighlight{they} \ExHighlight{lied} \ExHighlight{to} \ExHighlight{you}!!!’\\
\z 

The third person object pronouns that appear with communication verbs are mainly the direct object, that is the topic of communication. Only 11 are indirect objects, typically anaphorical, and have an antecedent that is linked to the discourse participants via a possessive pronoun, e.g., \textit{mi amigo} in \REF{ex:pierre:19}. Thus, while the third person object pronoun is ensuring topic continuity, the presence of a deictic form in the wider context establishes a link with the speaker, thus showing again that, while the non-anaphoric subject is underdetermined, the contribution of Speaker 1 as a whole is clearly tied into the ongoing interaction.

\ea\label{ex:pierre:19}
\begin{xlist}[Speaker 2]
\exi{H1} bueno / entonces / mi amigo {\textasciitilde}Diego y su amiga {\textasciitilde}Nuria Salgado decidieron presentarse
\exi{Speaker 1} ‘well / then / my friend Diego and his friend Nuria Salgado decided to be candidates.’
\exi{H2} ¿y los cogen? 
\exi{Speaker 2} ‘and do they take them?’

\exi{H1} por qué / nadie lo sabe // pero en plan=  <Pausa> hizo= / o sea / se- llamaron por teléfono {\ExHighlight{{les} \ExHighlight{hicieron}}}{ {una entrevista rápida por teléfono // y les dijeron que ya les avisarían y esa misma tarde les mandaron un correo // con un cuestionario de noventa preguntas cada u- / noventa y cuatro preguntas cada uno}} 
\exi{Speaker 1} ‘why / no one knows // but as a way of… <Pause> he did / so / they- they called by phone {\ExHighlight{they} \ExHighlight{interviewed} \ExHighlight{them}} {quickly by phone // and they told them that they would inform them and that same afternoon they sent them a mail // with a questionnaire of ninety questions each / ninety-four questions each’ (ESLORA)}
\end{xlist}
\z 

If we contrast these uses with non-anaphoric third person plural forms with communication verbs but without an object pronoun, as in \REF{ex:pierre:20}, we see a different picture. This is in addition a non-episodic use.


\ea\label{ex:pierre:20}
\begin{xlist}[Speaker 2]
\exi{H1}{[la se]rie}
\exi{Speaker 1} ‘[the ser]ies’

\exi{H2}{[ah] <Pausa>}
\exi{Speaker 2} ‘[ah] <Pause>’

\exi{H2} no sé  <Pausa>
\exi{Speaker 2} ‘I don’t know  <Pause>’

\exi{H1} \ExHighlight{dicen} que es muy buena también
\exi{Speaker 1} ‘\ExHighlight{they} \ExHighlight{say} that it’s really good as well.’ (ESLORA)
\end{xlist}
\z 

In those cases where there is no topic continuity nor deictic anchoring, the non-specific nature of the subject pronoun becomes the dominant feature. It functions then as a kind of evidential strategy, referring to hearsay but without further information. This hearsay meaning is also present in the uses with an object pronoun, but the presence of the concrete object pronoun puts the focus on the (highly specific) recipient of the message.


When investigating the non-anaphorical third person plurals overall, it should be noted that some of the second person singular object pronouns allow a reading that is not merely deictic. Indeed, as pointed out by \citet[4]{Posio2016}, in a so-called impersonal use of the second person singular “the speaker may be included or excluded and the reference may concern either a group of people or an individual”. Following \citet[89]{Kluge2012}, we refer to the generic use of the second person singular and opt for “a scalar model of referentiation of the second person singular, with five more or less well-defined focal points”. This proposal ranges from a speaker reference \textit{I} hiding behind \textit{you} over anyone (a generic use) to \textit{you} as term of address, with intermediary forms where respectively \textit{I} or \textit{you} are representative of a larger entity. While most cases included in our data deictically refer to the hearer, some cases include other uses on the scale, where the position of the discourse participant merits further discussion. Thus, \REF{ex:pierre:21} illustrates a reference to ‘anyone’, clearly not anchored in the speaker or hearer’s personal experience, since the interlocutors conclude they will have to go one day to this club, revealing that they do not have a concrete experience yet. However, the link to the discourse participants remains present in that they are discussing their own options to go there. 


\ea\label{ex:pierre:21}
\begin{xlist}[Speaker 2]
\exi{H1} me dijo eso que que {\ExHighlight{{te} }}{[}{\ExHighlight{{cobraban}}}{ {bastan]te} }
\exi{Speaker 1} ‘he told me that that {\ExHighlight{they} [\ExHighlight{charged} \ExHighlight{you}} {quite a lo]t’} 

\exi{H2}{[cucadas]  <Pausa>}
\exi{Speaker 2} ‘[cute things] <Pause>’

\exi{H1} pero no sé <Pausa> habrá que ir un [día]
\exi{Speaker 1} ‘but I don’t know <Pause> we’ll have to go one [day].’ (ESLORA)
\end{xlist}
\z 

Example \REF{ex:pierre:22}, by contrast, refers to the speaker as a representative of a larger entity, which may include the hearer. Indeed, the speaker narrates a personal experience as advice for the hearer. The generic reading then does not at all preclude a reference to the discourse participants. Quite the contrary, it often involves both speaker and hearer. 

\ea\label{ex:pierre:22}
\begin{xlist}[Speaker 2]
\exi{H1} un trenecito turístico de Monforte a Orense que te costaba= // (no sé) // veinte euros / creo que era
\exi{Speaker 1} ‘A tourist train from Monforte to Orense that costed you I don’t know twenty euros I think it was’ 


\exi{H2}{[hm / hm / hm / hm] hm / hm}
\exi{Speaker 2} ‘hm hm hm hm hm hm’ 

\exi{HI} te subías al tren / {\ExHighlight{{te} \ExHighlight{llevaban}}}{ {de Monforte a Orense / antes de llegar a Orense te hacían un recorr- / hacían un recorrido por to da= la = / ciudad} }
\exi{Speaker 1} ‘you got on the train {\ExHighlight{they} \ExHighlight{took} \ExHighlight{you}} {from Monforte to Orense before reaching Orense they took you for a tour through the town.’ (ESLORA)}
\end{xlist}
\z 

In Example \REF{ex:pierre:23} Speaker 1 fears electricity will be cut off due to late payment. Speaker 2 explains that advance warning is given, addressing this to Speaker 1 but also representing more general information concerning how electricity companies work. 

\ea\label{ex:pierre:23}
\begin{xlist}[Speaker 2]
\exi{H1}{[no si / me puedes dar para la fac]tura de la luz [que no sé cuánto será] y que mañana a primera hora [tengo que pa]gar}
\exi{Speaker 1} ‘[not if you / can give me for the light bill, since I don’t know how much it will be] and tomorrow first thing [I have to p]ay’

\exi{H2}{[<inint>] H2 [<inint>] <Pausa>}
\exi{Speaker 2} ‘(not understandable)  <Pause>’

\exi{H1} ¿qué hago? llamo= / y digo que no me llegó el recibo / que me den algo pa- / un código para [pagar o]= en /en internet ¿no? 
\exi{Speaker 1} ‘What do I do? I call and say that the receipt didn’t arrive that they give me a code to [pay or] on internet, didn't they?’ 

\exi{H2}{[claro] <Pausa> H2 <inint> <Pausa>}
\exi{Speaker 2} ‘Indeed <Pause> (not understandable) <Pause>’

\exi{H1} antes de que salgan a cortarla porque no te digo yo que no vengan mañana [a cortarla ¿eh? // de] hecho no te aseguro yo que no hayan venido ya y que no hayan encontrado el por[tal]</dud> 
\exi{Speaker 1} ‘before they go out to cut it off because I don’t tell you that they won't come tomorrow [to cut it off] actually I can’t assure you that they haven’t come already and that they haven’t found the por[tal]’ 

\exi{H2}{[<inint>] H2 [no] no / te tienen / \ExHighlight{te} \ExHighlight{tienen} que dar un aviso / te dan un aviso}

\exi{Speaker 2} ‘(not understandable) no no \ExHighlight{they} \ExHighlight{have} \ExHighlight{to} \ExHighlight{give} \ExHighlight{you} a notification they give you a notification’ (ESLORA)
\end{xlist}
\z 

Again, though the reference is larger than a strictly deictic one, it does involve one of the discourse participants concretely and thus maintains a deictic anchoring. It falls under the use described by \citet[89]{Kluge2012} as “anyone, but addressee as a typical representative”, since the addressee’s concrete situation is the starting point for a reference that can cover more people but in which the addressee remains included.  


Thus, these cases where a second person with a not exclusively deictic use appears still are to be considered as references to discourse participants and by no means make the whole construction impersonal. Overall, the agent-defocusing effect of the non-anaphoric third person plural forms then entails a more prominent position for the (discourse participant) object pronoun as compared to the less prominent non-anaphoric third person plural form, rather than a low referentiality for the utterance as a whole. Indeed, these discourse participant pronouns are then the main reference in the ongoing interaction, relating to one of the interaction participants, and as such occupy a crucial position.



\subsection{Mode and register variation}\label{sec:pierre:4.3}

The literature shows that the use of non-anaphoric third person plurals, regardless of the presence of a discourse participant object, is specific to spontaneous interactions (\citealt{SiewierskaPapastathi2011}: 585). \citet[118]{Pierre2021} indicates that this non-referential mechanism appears more typically in informal oral mode, though the mechanism is still relatively frequent in written informal productions. A closer look at the mechanism used with object (non)-discourse participants confirms the tendency to appear in discursive situations considered informal (see \tabref{tab:pierre:5}). However, it seems that the preference for appearing in oral rather than written types of data is less marked when the non-anaphoric third person plurals co-occur with object pronouns. Indeed, the results reveal a normalized frequency of 21.9 occurrences per ten thousand words in the oral data used for this study and 18.2 in the written data used in this study. 


\begin{table}
\fittable{\begin{tabular}{lllllllll} 
\lsptoprule

& \multicolumn{4}{c}{Discourse  participant} & \multicolumn{2}{c}{Non discourse} & &\\
& \multicolumn{4}{c}{} & \multicolumn{2}{c}{participant (only)} & &\\
\cmidrule(lr){2-5}\cmidrule(lr){6-7}
& \multicolumn{2}{c}{1\textsuperscript{st} person}& \multicolumn{2}{c}{2\textsuperscript{nd} person} & \multicolumn{2}{c}{3\textsuperscript{rd} person} & \multicolumn{2}{c}{{Total}}\\
\midrule
& {Freq.} & \% & {Freq.} & \% & {Freq.} & \% & {Freq.} & \%\\
\midrule 
\multicolumn{9}{l}{\itshape Informal data}\\
{Oral} & \textbf{43} {(5.6)} & {\textbf{25.6}} & {38 (4.9)} & {22.6} & \textbf{87 (11.3)} &  \textbf{51.8} & 168(21.9) & 100\\
 {(ESLORA)} & & & & & & & &\\

{Written} & 11 (2.4) & {12.9} & \textbf{31 (6.6)} & {\bfseries 36.5} & 43 (9.2) & 50.7 & 85 (18.2) & 100\\
(Yahoo Q\&A) & & & & & & & &\\
\midrule\multicolumn{9}{l}{\itshape Formal data}\\
{Oral} & 2 (0.5) & 100 & 0 & 0 & 0 & 0 & 2 (0.5) & 100\\
{(PROCEP)} & & & & & & & &\\
{Written} & 0 & 0 & 0 & 0 & 0 & 0 & 0 & 0\\
{(Wikipedia)} & & & & & & & &\\
\lspbottomrule
\end{tabular}}
\caption{Distribution of (non)-discourse participants in formal and informal data (raw frequencies (normalized per ten thousand words)). Frequencies (“freq.”) are given twice, as both raw and normalized.}
\label{tab:pierre:5}
\end{table}



\tabref{tab:pierre:5} provides more details on the distribution of object pronouns across the four types of language production used to collect the data of the present study (two different modes and two different registers). Since, as previously explained, nineteen occurrences combine two participants, where at least one of them is a non-discourse participant, \tabref{tab:pierre:5} only includes non-discourse participants occurring alone, that is, without the simultaneous presence of another participant. This prevents counting twice those utterances that contain a combination of participants and allows us to focus on how the register and the mode impact the presence of discourse versus non-discourse participants.


As shown in \tabref{tab:pierre:5}, it is especially the first person singular form that appears in the informal spoken mode (25.6\% of all object pronouns occurring in informal oral data compared to 12.9\% in informal written data), whereas the second person singular form appears more in the informal written mode (36.5\% and 22.6\%, respectively). This finding thus points towards a more interlocutor-oriented approach in written productions than what is observed in conversational situations. This finding can be related to the specific advice-giving function of the Yahoo \mbox{Q\&A} forum, where addressing the interlocutor is a crucial feature. As shown in the previous section, this second person form is not always to be interpreted in a purely deictic way, though. The dominance of the first person object pronoun in informal oral mode ties in with overall research results concerning the use of deictics in informal oral interactions (see \citealt{DeCock2014}: 35) and with the literature concerning person reference in different modes and registers. They thus highlight the impact of the nature of the language production.\largerpage


The higher presence of first person pronouns in informal oral conversations suggests that this type of conversation engages the speaker much more than communicative exchanges achieved through the written mode. When looking at the research body on the use of reference to discourse participants in English, deictic forms have been associated with spoken mode, and non-deictic and canonical passive constructions with written mode \citep{Biber1988}. The presence of deictics in spoken mode has also been pointed out by \citet{Chafe1982}, who formulates this tendency in terms of involvement, as opposed to the detachment reflected e.g. in the use of passives in the written mode. Ochs pointed out the higher use of passives in planned discourse vs. the preference for active constructions in unplanned discourse \citep[76]{Ochs1979}. The constructions we focus on in this paper can be explained partly through these findings. Indeed, the use of a non-anaphoric third person plural form as an alternative to a passive form in informal conversation can be explained in part through the preference for active constructions in unplanned discourse, leading to the Spanish periphrastic passive being even less frequent in unplanned spoken discourse than it already is in other genres. When adapting \citegen{Chafe1982} and \citegen{Biber1988} ideas to Spanish, we have to take into account, however, that the periphrastic passive is much less used in Spanish than in English, since Spanish also has impersonal and passive constructions formed with the third-person reflexive clitic \textit{se}, the latter being much more frequent than the periphrastic passive (see e.g., \citealt{LaslopDíaz2010} or \citealt{Pierre2021}), though in some, mainly informal, spoken genres, the non-anaphoric third person plural form is more frequent than the \textit{se}-passive (\citealt{DeCock2014}: 194, \citealt{Posio2015Alcance}, \citealt{Pierre2021}: 117--118). The non-anaphoric third person plural form also competes to some extent with these \textit{se}-constructions \citep[86]{Siewierska2011}. With regard to our informal written data, they seem to behave differently from the written mode commented upon by \citet{Biber1988} and \citet{Chafe1982}, who looked into English formal written data. The written data from the Yahoo Q\&A forum fall into Ochs’ description of unplanned discourse and are informal (as can be seen also by the lexical choices and spelling), though, which explains the presence of deictics and the use of active constructions, rather than canonical passive ones or \textit{se}-constructions. Indeed, research has shown \citep{Pierre2021} that periphrastic passives and \textit{se}-constructions occur more frequently in formal written texts than in informal written texts, whereas the latter contain non-anaphoric third person plural forms, which the former lack. In addition, \citet[384]{Posio2015Alcance} shows that there is no link between the degree of formality of the discourse and the presence of the \textit{se}-con\-struc\-tions, whereas a high degree of contextuality favours the non-anaphoric third person plurals (Posio contrasts formality with contextuality, following \citealt{HeylighenDewaele2002}).


Finally, \tabref{tab:pierre:5} reveals that the use of non-discourse participants remains stable as they total 51.8\% of object pronouns in the informal oral corpus and 50.7\% of object pronouns in the informal written corpus. These results seem to indicate that the use of pronouns referring to discourse participants is more influenced by the type of language production than the use of pronouns referring to non-discourse participants. 


Summarizing the behavior of non-anaphoric third person plurals occurring with vs. without pronouns referring to discourse participants in two registers and modes of productions, the following results can be put forward. Typically, the non-anaphoric third person plural is associated with informal situations, which confirms what is reported in the literature. Within oral data, discourse participants tend to be oriented towards the speaker whereas in written data they rather involve the interlocutor. It can thus be suggested that the distribution of participants is considerably impacted by the degree of formality and the mode of production of the language, but also by the specificities of the genre (informal oral conversations and written exchanges on a digital forum). However, as the figures are low, these tendencies need to be confirmed.  

\section{Conclusions}\label{sec:pierre:5}\largerpage

In this study, we have looked into the Spanish non-anaphoric third person plural form. Following earlier research, we focus on the verb types with which these forms are used. Given the non-referential nature of the subject, we have paid particular attention to the cases in which they appear with a referential object pronoun, be it a discourse participant or non-discourse participant. 

Through an analysis of corpora representing informal and formal oral and written genres, we have shown that the non-anaphoric third person plural form is virtually absent in formal genres, which is in line with previous findings by, e.g., \citet{DeCock2014, Pierre2021}. \citet[606]{SiewierskaPapastathi2011} also argue that the structure is particularly related to spontaneous conversations. A more detailed analysis of the occurrences found in our datasets has shown that references to discourse participants tend to occur as indirect object in the roles of receiver or beneficiary, whereas the non-discourse participant objects are rather used as direct objects, in line with more general tendencies of indirect objects being typically animate. The non-discourse participant objects refer mainly to inanimate entities (altogether another type of referent than the necessarily animate discourse participants). Note that the animate non-discourse participant objects  mainly refer to animals, rather than humans.

The analysis of the verb types has shown that the use of discourse participant objects in the examined construction is favoured by the presence of relational verbs, followed by communication verbs. This furthermore ties in with the use of discourse participant object pronouns as indirect objects in a receiver or beneficiary role. As such, both the verb semantics and the thematic role of the object pronoun play a role. Also, non-anaphoric third person plurals without any object pronoun are frequently used with communication verbs, regularly in a non-episodic use. These results support \citegen{SiewierskaPapastathi2011} suggestion to consider non-episodic uses of speech act verbs as a separate category in the study of non-anaphoric third person plural forms. 

Our pragmatic-discursive analysis sheds light on the impact of using a referential discourse participant object pronoun with a non-referential subject pronoun. In the absence of a referential subject, it is above all these deictic object forms that ensure the anchoring in the ongoing interaction and frequently also ensure topic continuity. The non-discourse participants, third person objects, on the other hand, tend to be anaphoric but about one third of the occurrences in our corpus include a deictic reference by means of a possessive pronoun in the object or in a coreferential object to which it refers. Some pronouns referring to discourse participants are not used with a merely deictic reference, but also allow for a generic reading. However, even in such cases the link with one or more discourse participants remains present and such utterances are then not to be considered entirely impersonal. 

The specificities of the genres analyzed explain the preference for first person object pronouns in informal conversation and second person singular object pronouns in the Yahoo Q\&A data, where participants answer questions. These results also show that it is the informal nature of the data, rather than their being written or spoken, that influences the presence of non-anaphoric third person plural forms, since the written and spoken informal datasets present similar frequencies of non-anaphoric third person plural forms.

Through this study of non-anaphoric third person plural forms with particular attention to their use with discourse participant object pronouns, we have aimed to contribute to the literature, which has hitherto focused mainly on the non-referential subject. By examining referential objects and the verb types with which the non-anaphoric third person plural forms appear, we hope to have contributed to a more complete image of how discourse participant objects are used with these forms, as well as to the place they hold in the development of interaction. 

\section*{Corpora}

\begin{description}
\sloppy
\item[Corpus para el estudio del español oral:] \url{http://eslora.usc.es}, versión 2.0 de septiembre de 2020, ISSN: 2444-1430.
\item[Proceedings from European Parliamentary debates (PROCEP):] \url{https://www.europarl.europa.eu/plenary/en/debates-video.html}
\item[Wikicorpus V.1.0:] Catalan, Spanish and English portion of Wikipedia, \url{https://www.cs.upc.edu/~nlp/wikicorpus/}
\item[Yahoo Contrastive Corpus of Questions and Answers:] Compiled by Hendrik De Smet at the Department of Linguistics, University of Leuven, 2009
\end{description}

\printbibliography[heading=subbibliography]

\end{document}
