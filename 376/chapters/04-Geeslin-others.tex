\documentclass[output=paper]{langscibook}
\ChapterDOI{10.5281/zenodo.8124492}
\author{Kimberly L. Geeslin\orcid{}\affiliation{Indiana University} and Thomas Goebel-Mahrle\orcid{}\affiliation{Indiana University} and Jingyi Guo\orcid{}\affiliation{Indiana University} and Bret Linford\orcid{}\affiliation{Grand Valley State University}}
\title [Variable subject expression in second language acquisition] {Variable subject expression in second language acquisition: The role of perseveration}
\abstract{The current study examines the development of variable subject expression in Spanish across multiple proficiency levels of second language learners, and compares their patterns with a group of native speakers from the same speech community. A cross-sectional design and a written contextualized preference task are employed to explore the differences in rates of subject form selection, the degree to which the linguistic and psychological construct known as perseveration, constrains the acquisition of subject expression, as well as the potential interaction between perseveration and other linguistic factors. Our analysis examines null and overt pronominal subjects as well as full lexical noun phrase verbal subjects. The results show that as proficiency level increases, learners’ selection rates of subject forms and rates of perseveration become gradually more native-like, and an increasing number of linguistic factors (prime form, gender continuity, tense mood aspect continuity) predict the occurrence of perseveration. In addition, for learners from the two most advanced levels and native speakers, feminine primes are more likely to perseverate than masculine primes, suggesting the effects of psychological processes (i.e., surprisal) on perseveration.}

% % % \smallskip \\
% % % {\textbf{Keywords:\ }}\textrm{second language acquisition, Spanish, subject expression, language variation, perseveration, referent cohesion, referent gender}}
\IfFileExists{../localcommands.tex}{
  \addbibresource{../localbibliography.bib}
  % add all extra packages you need to load to this file

\usepackage{tabularx,multicol}
\usepackage{url}
\urlstyle{same}

\usepackage{listings}
\lstset{basicstyle=\ttfamily,tabsize=2,breaklines=true}

\usepackage{langsci-basic}
\usepackage{langsci-optional}
\usepackage{langsci-lgr}
\usepackage{langsci-osl}
% \usepackage{./langsci/styles/langsci-lgr}
% \usepackage{./langsci/styles/langsci-osl}
% \usepackage{langsci-gb4e}

\usepackage{tikz}
\usetikzlibrary{patterns,calc}
\pgfdeclarepatternformonly{south east lines}{\pgfqpoint{-0pt}{-0pt}}{\pgfqpoint{3pt}{3pt}}{\pgfqpoint{3pt}{3pt}}{
    \pgfsetlinewidth{0.6pt}
    \pgfpathmoveto{\pgfqpoint{0pt}{3pt}}
    \pgfpathlineto{\pgfqpoint{3pt}{0pt}}
    \pgfpathmoveto{\pgfqpoint{.2pt}{-.2pt}}
    \pgfpathlineto{\pgfqpoint{-.2pt}{.2pt}}
    \pgfpathmoveto{\pgfqpoint{3.2pt}{2.8pt}}
    \pgfpathlineto{\pgfqpoint{2.8pt}{3.2pt}}
    \pgfusepath{stroke}}
    
\usepackage{stmaryrd}
\usepackage{wasysym}
\usepackage{multirow}
\usepackage{caption}
\usepackage{subcaption}
\usepackage{mathrsfs}
\usepackage{qtree}

\usepackage{linguex}


  %pminos do not split footnotes
% \interfootnotelinepenalty=10000 %Footnote in Laporte chapters has to be split SN


%\DeclareIndexNameFormat{default}{%
%\nameparts{#1}%
%\usebibmacro{index:name}%
%{\index[names]}%
%{\namepartfamily}%
%{\namepartgiveni}%
% {}% L1
% {}% L2
%{\namepartprefix}% generates spurious space L3
%{\namepartsuffix}% generates spurious space L4
%}

%  {\DeclareIndexNameFormat{default}{%
%     \usebibmacro{index:name}{\index[names]}{#1}{#3}{#5}{#7}}}

%\DeclareIndexNameFormat{default}{%
%  \usebibmacro{index:name}{\sindex[nom]}{#1}{#3}{#5}{#7}}

%\DeclareIndexNameFormat{default}{%
%  \usebibmacro{index:name}{\sindex[person]}{#1}{#3}{#5}{#7}}
%\DeclareIndexNameFormat{default}{%
%\nameparts{#1} \usebibmacro{index:name}{\sindex[person]]}{\namepartfamily}{‌​\namepartgiven}{\nam‌​epartprefix}{\namepa‌​rtsuffix}}

%\newcommand{\smiley}{:)}

%\renewbibmacro*{index:name}[5]{%
%\usebibmacro{index:entry}{#1}%
%{\iffieldundef{usera}{}{\thefield{usera}\actualoperator}\mkbibindexname{#2}{#3}{#4}{#5}}}

% \newcommand{\noop}[1]{}

%remove for final
%\overfullrule=1mm

\newcommand{\tobi}[2]}}
\renewcommand{\S}[1]{\tobi{#1}{\textsc{*}}}

% this volume references
% puts: [this volume]
% already defined: \citetv
%\newcommand{\citepv}[1]{(\citeauthor{#1} \citeyear*{#1} [this volume])}
\newcommand{\citealtv}[1]{\citeauthor{#1} \citeyear*{#1} [this volume]}

%parentheses around example number
\newcommand{\pref}[1]{(\ref{#1})}

% in-text examples

\newcommand{\lnex}[1]{\textit{#1}} %target lang word
\newcommand{\lnlit}[1]{(lit.: `#1')} %literal reading
\newcommand{\lnlat}[1]{(#1)} % latinization
\newcommand{\lntrans}[1]{`#1'} %translation
\newcommand{\lnexl}[2]%
{\lnex{#1}{} \lnlat{#2}} % ex with latinization
\newcommand{\lnexlat}[3]{\lnex{#1}{} \lnlat{#2}{} \lntrans{#3}} % ex with latinization and tranl.

%ch01
\newcommand{\co}[1]{\mbox{\textbf{#1}}}

%ch09

\newcommand{\cyrbulg}[1]{\begin{otherlanguage*}{bulgarian}#1\end{otherlanguage*}}


%ch10
\newcommand{\nlp}{{\small NLP}}
\newcommand{\mwe}{{\small MWE}}
\newcommand{\rae}{{\small RAE}}
\newcommand{\lvc}{{\small LVC}}
\newcommand{\pos}{{\small P}o{\small S}}
%\newcommand{\todo}[1]{ \textcolor{red}{#1} }

%\renewcommand{\labelenumi}{\theenumi}
%\ainamefmt{{vv}{ll}{, ff}{, jj}} % fullname

\newcommand{\biberror}[1]{{\color{red}#1}}

\newcommand{\osenovaitem}{--~} 
  %% hyphenation points for line breaks
%% Normally, automatic hyphenation in LaTeX is very good
%% If a word is mis-hyphenated, add it to this file
%%
%% add information to TeX file before \begin{document} with:
%% %% hyphenation points for line breaks
%% Normally, automatic hyphenation in LaTeX is very good
%% If a word is mis-hyphenated, add it to this file
%%
%% add information to TeX file before \begin{document} with:
%% %% hyphenation points for line breaks
%% Normally, automatic hyphenation in LaTeX is very good
%% If a word is mis-hyphenated, add it to this file
%%
%% add information to TeX file before \begin{document} with:
%% \include{localhyphenation}
\hyphenation{
    Beck-man
    Ngu-yen
    back-chan-nel
    back-chan-nels
    mo-not-o-nous
    ste-reo-typ-i-cal
}

\hyphenation{
    Beck-man
    Ngu-yen
    back-chan-nel
    back-chan-nels
    mo-not-o-nous
    ste-reo-typ-i-cal
}

\hyphenation{
    Beck-man
    Ngu-yen
    back-chan-nel
    back-chan-nels
    mo-not-o-nous
    ste-reo-typ-i-cal
}
 
  \togglepaper[4]%%chapternumber
}{}

\begin{document}
\AffiliationsWithIndexing{}
\maketitle

\section{Introduction}\label{sec:geeslin:1}
The current study examines the patterns of the perseveration of variable subject forms in Spanish across multiple proficiency levels of second-language (L2) learners, and for a group of native speakers (NS) from the same speech community. In a general sense, perseveration refers to the tendency for a particular form to appear again (i.e., to persevere) in subsequent discourse. We note that this is sometimes referred to as linguistic priming, but priming may also refer to the cognitive explanation for the effect known as perseveration (see \citealt{Otheguy2015}). Thus, we will employ the term perseveration throughout, and we will take this to describe a distribution of subject forms attested in our dataset, such that a given form (i.e., a prime) is followed by a second subject of the same type (e.g., an overt subject pronoun is followed by a subsequent overt subject pronoun). What is particularly interesting about perseveration is that it may occur even in the absence of a discourse-based or functional explanation. For example, \citet{Poplack1980}, studying Puerto Rican Spanish which tends to delete word-final -\textit{s}, found that plural marking was more likely to occur if a previous element in an NP was already -\textit{s} marked for plural (e.g., \textit{la}{[∅]} \textit{mujere}{[∅]} \textit{bonita}{[∅] vs.} \textit{la}\textit{s} \textit{mujere}\textit{s} \textit{bonita}\textit{s} ‘the beautiful women’). This marking cannot be accounted for by appealing to discourse constraints or a disambiguation strategy; rather, overt plural marking with -\textit{s} is favored when a previous element is already marked, even though a functional hypothesis would consider this marking redundant. In short, the phenomenon can be described as the distribution that results when one form begets another subsequent like form, even when this is unnecessary for reasons such as disambiguation. We will return to a more detailed discussion of this distribution later.

Our selection of subject forms as the object of study is based in large part on the wealth of research available on their variable use. Collectively, sociolinguistic studies show that variation between null subjects and overt subject pronouns (SPs) is constrained by factors such as person/number, verbal tense, mood, and aspect (TMA), reflexivity of the verb, lexical content of the verb, and specificity of the referent as well as by discourse-related factors beyond the verb phrase, such as perseveration, discourse genre, referent cohesiveness and clause type  (e.g., \citealt{CarvalhoShin2015, OtheguyLivert2007, ShinOtheguy2009, Silva-Corvalán1994, TravisCacoullos2012}). Although there is a tendency to limit the study of subject forms to null and overt SPs, there has recently been expansion to study the patterns that influence the realization of subject forms as full lexical noun phrases (NPs), even in contexts where they have been mentioned previously. These studies have demonstrated that many of the same independent linguistic constraints influence use of these forms relative to null and overt pronouns (see \citealt{Bentivoglio1993, Dumont2006, GudmestadGeeslin2022, GudmestadGeeslin2013, Silva-Corvalán2015}). In general, we see that rates of use of null forms vary dialectally, but the constraints on these rates of use are often steady across studies and speech communities (\citealt{CarvalhoShin2015, GudmestadGeeslin2022}). Similarly, there are ample studies of subject form expression in L2 Spanish, examining variability in learner-directed input as compared to patterns of second language development (\citealt{Gurzynski-Weiss2018}), the subtle differences attested between highly-advanced non-native speakers (NNSs) and NSs of Spanish \citep{GeeslinGudmestad2011,GeeslinGudmestad2016}, stages of development across multiple levels of proficiency (\citealt{GeeslinLinfordFafulas2015}), and differences between group and individual patterns for variable structures (\citealt{GeeslinEtAl2013}), to name only a few key issues. As with the studies of NS patterns of use, we find that the constraints on the expression of subject forms are relatively stable. This makes subject form expression a particularly good test case for a variety of theoretical questions.

The current study adopts a variationist framework, which is characterized by its attention to the many factors that simultaneously influence patterns of use. An advantage of this approach is that it allows researchers to determine the constraints that influence the realization of specific (socio)linguistic variables in the interlanguage, how these constraints develop over time, and if learners approximate native-like usage. For NSs of Spanish, the wealth of research available has shown the multiple independent linguistic factors, as well as social factors such as regional location of a speech community come into play in studying the patterns of subject from use. Likewise, there now exists a significant body of L2 research showing that sensitivity to the factors that constrain NS variation can be acquired by NNSs of the language (for overviews see \citealt{Geeslin2014, Kanwit2018}). The native-like variation between two or more grammatical forms that perform the same function (i.e., variable structures) is guided probabilistically by the semantic, morphosyntactic, and discourse-level features of the linguistic context as well as the social features of the extra-linguistic context and is often referred to as Type II variation (\citealt{BayleyPreston1996, MougeonDewaele2004, Young1991}).  In Spanish, research has examined the SLA of the variation between forms used to express copulas, mood contrasts, the progressive aspect, future and past-time marking, as well as grammatical subjects and objects (see \citealt{Geeslin2018} for a review). Together these studies demonstrate that the variationist perspective can be applied profitably to the study of second language development.\footnote{For a generative approach to the study of L2 subject expression, see \citet{Lozano2002}, who utilizes experimental methodology to investigate the acquisition of universal properties of subject expression versus Spanish language-specific parameters. Additionally, \citet{Lozano2016} uses corpus data to argue that advanced learners are pragmatically redundant in their subject expression and may struggle with the syntax-discourse interface.}

In this brief introduction, we have established that subject form expression is a well-studied structure in both first and second language contexts and, that it lends itself to the study of perseveration. We close this introduction by highlighting the contributions of the current investigation to the larger whole. While there are studies that analyze the relative influence of many morphosyntactic factors on subject form expression, there is still a need for careful examinations of the more complex patterns that are exhibited in extended discourse. For example, we know that the form of the previous mention of the referent predicts rates of production of subject forms for highly advanced NNSs (e.g., \citealt{GeeslinGudmestad2011}). Moreover, this particular factor tends to be one that is controlled in elicitation tasks designed to study L2 development, rather than the focus of the analysis (e.g., \citealt{GeeslinLinfordFafulas2015}). To date, less is known about the developmental path that L2 learners follow in their acquisition of sensitivity to these more complex factors and whether this sensitivity might lead to similar patterns of perseveration in learner language. Finally, within the variationist framework, studies of subject expression by L2 learners have been based primarily on oral production data. Thus, the current study contributes to the growing body of research on the acquisition of variable structures by using a cross-sectional design to examine the L2 development of subject expression and sensitivity to the form of the previous mention of the referent. We accomplish this through an analysis of subject form selection on a controlled preference task, which allows us to ensure that each participant responds to the same confluence of independent variables. In so doing, we also continue the cross-disciplinary dialogue between research on language variation and second language acquisition and provide common ground to move both fields forward.



\section{Spanish subject expression}\label{sec:geeslin:2}


The syntax of Spanish allows for grammatical subjects to be expressed overtly as a personal pronoun (overt SP; example \ref{itm:geeslin:1}), a lexical noun phrase (lexical NP; example \ref{itm:geeslin:2}), demonstrative pronoun (example \ref{itm:geeslin:3}), indefinite pronoun (example \ref{itm:geeslin:4}), interrogative pronoun (example \ref{itm:geeslin:5}), as well as allowing phonetically unexpressed, or null subjects (example \ref{itm:geeslin:6}).


\ea Él habla español.\\
\glt `\textit{He} speaks Spanish.’ \label{itm:geeslin:1}

\ex Juan habla español.\\
\glt ‘\textit{Juan} speaks Spanish.’\label{itm:geeslin:2}

\ex Ese habla español.\\
\glt ‘\textit{That} one speaks Spanish.’ \label{itm:geeslin:3}

\ex Alguien habla español.\\
\glt `\textit{Someone} speaks Spanish.’\label{itm:geeslin:4}

\ex ¿Quién habla español?\\
\glt ‘\textit{Who} speaks Spanish?’ \label{itm:geeslin:5}

\ex ∅ Habla español.\\
\glt ‘\textit{(He/she-}{\textit{null}}{)} speaks Spanish.’\label{itm:geeslin:6}

\z



With few exceptions, variationist research on subject expression in Spanish has focused on the variation between null and overt SPs and the analyses are limited to contexts that are determined to permit variation between the two forms (\citealt{OtheguyZentella2007}). Nonetheless, there are studies that suggest that these two forms are in variation with others, most notably full lexical NPs which have been shown to occur even following an adjacent previous mention of the same subject (\citealt{Bentivoglio1993, Dumont2006, GudmestadGeeslin2022, GudmestadGeeslin2013, Silva-Corvalán2015}).\footnote{While it may seem counterintuitive given the general rules of use for full lexical NPs, recent research has begun to provide two strong arguments for their consideration within the same envelope of variation as other subject forms. Firstly, their use can be constrained by similar factors to other subject forms (e.g., \citealt{GudmestadGeeslin2022, Dumont2006}) and secondly, the forms have been shown to appear in interview speech with some regularity in contexts where they would not be expected, such as those where they are mentioned previously and where there is no need to disambiguate from other referents.} Previous variationist research on subject expression in Spanish has found that variation between null and overt SPs is constrained by morphosyntactic factors such as person/number, tense, mood, and aspect (TMA), lexical frequency and reflexivity of the verb, semantic factors such as lexical content of the verb and specificity of the referent, discourse-level factors such as switch reference\footnote{Also known as “continuity of reference” (\citealt{ShinOtheguy2009}), “coreferentiality” \citep{Silva-Corvalán1994}, “subject continuity” (\citealt{TorresTravis2010}) and “discourse connection” (\citealt{CarvalhoChild2011}).}, referent cohesiveness, the form of the previous mention of the subject (i.e., perseveration), discourse genre, clause type, as well as some extra-linguistic factors (\citealt{Ávila-Jiménez1995, BayleyPease-Alvarez1996, BayleyPease-Alvarez1997, Bentivoglio1987, Cameron1994,Cameron1995, CameronFlores-Ferrán2004, Enríquez1984, ErkerGuy2012, Flores-Ferrán2005, Hochberg1986, Morales1986, OtheguyLivert2007, OtheguyZentella2012, Shin2006, Shin2012, ShinCairns2009, ShinOtheguy2009, Silva-Corvalán1994, Travis2007, TorresTravis2010}; inter alia). The discourse-level factors previously mentioned have been found to be crucial in explaining subject expression in Spanish and are the focus of the current study. Hence, the following review will be limited specifically to the impact of discourse-level factors on subject expression and how these findings have influenced the goals of the present investigation.



Perhaps the most widely studied discourse-level factor is switch reference (\citealt{BayleyPease-Alvarez1997, Bentivoglio1987, Cameron1994, Cameron1995, CameronFlores-Ferrán2004, ErkerGuy2012, OtheguyEtAl2007, Silva-Corvalán1994, ShinCairns2009, ShinOtheguy2009, TorresTravis2010, Travis2007}). Contexts where the subject referent is different from the referent of the previous tensed verb are known as “switch reference” while situations where these two referents are the same are known as “same reference”. Results across studies based on oral speech of speakers from a variety of linguistic backgrounds demonstrate that overt SPs are more frequent in contexts of switch reference than same reference. Moreover, several studies find that this factor is one of the most important factors influencing subject form variation (see \citealt[28]{Cameron1994}).



Some studies have expanded the analysis of switch reference (\citealt{BayleyPease-Alvarez1997, Cameron1995, Travis2007}). For example, \citet{BayleyPease-Alvarez1997} analyzed degrees of “discourse connectedness” in Mexican-descent children’s oral and written narratives. The categories of the discourse connectedness variable accounted for the continuity of TMA between tensed verbs in the discourse, the clause distance to the previous mention of the referent, the previous mentions of the referent in different syntactic functions, and changes in narratives. By combining these factors, \citet{BayleyPease-Alvarez1997} identified five degrees of discourse connection: from the most connected discourse (where the referent and TMA of the verb were the same as the referent and TMA in the preceding tensed verb) to the least connected discourse (where the narrative section or discourse topic changed). Their findings showed that there was a steady increase in the probability of using overt SPs as the discourse became less connected. They also found that the effect of the discourse connectedness variable was a more robust predictor of subject use than switch reference alone. As we turn our attention to the focus of the current study, the role of perseveration in subject form acquisition and use, we will see that discourse connectedness must remain in view as well.



\section{Research on perseveration}\label{sec:geeslin:3}



Interest in perseveration\footnote{Also known as “parallelism” (e.g., \citealt{CarvalhoChild2011}), “linguistic priming” and “structural priming” (\citealt{PickeringFerreira2008}) among other terms.}, the focus of the current study, stems from research findings that showed that patterns of deletion could not be explained using discourse constraints alone.  For example, \citet{Poplack1980} analyzed the factors that constrain overt plural -\textit{s} marking (in variation with -\textit{s} aspiration and deletion) in Puerto Rican Spanish. From a corpus of naturalistic productions, 6439 tokens of words in plural NP strings were extracted (e.g., \textit{las nenas bonitas} ‘the pretty girls’) and were coded for grammatical category, following phonological segment, following stress, presence of disambiguating plural information, position of the word in the NP string, and presence of preceding plural marking. Apart from following stress, the functional and discourse-related factors did not account for the realization of final -\textit{s}. In fact, Poplack found that “[p]resence of a plural marker before the token favors marker retention on that token, whereas absence of a preceding marker favors deletion (…) (Additionally,) [t]he most favorable context for marker deletion is precisely when the two preceding markers have already been deleted" (pp. 63–64). Thus, her findings do not fit a functional explanation and, instead, are consistent with \citet[4]{TorresTravis2010}, who state that “the use of a certain structure in one utterance functions as a prime on a subsequent utterance, such that that same structure is repeated”.



With respect to subject form perseveration, most studies distinguish between what \citet{TravisCacoullos2018} dub \textit{co-referential subject priming}, which focuses on the previous expression of the same referent regardless of clausal distance, and \textit{adjacent clause subject priming}, which analyzes the subject forms in adjacent clauses regardless of co-referentiality. Variations of these analyses were carried out by \citet{TorresTravis2010}, \citet{CarvalhoChild2011}, \citet{Abreu2012}, \citet{Flores-Ferrán2002}, \citet{GeeslinGudmestad2011} and \citet{Travis2007}.\footnote{We recognize that there are additional nuances within the various coding schemes employed in these studies. For example, some studies examine only the form of the preceding co-referential subject (\citealt{CarvalhoChild2011, TorresTravis2010}), others examine the form of the preceding subject, even when it is not co-referential (\citealt{Cameron1994, CameronFlores-Ferrán2004}), while others examine the form of the preceding mention of the referent in subject position regardless of the distance between mentions (\citealt{Abreu2012, Flores-Ferrán2002, GeeslinGudmestad2011, Travis2007}).} Additional studies focus solely on first- and third-person singular forms (\citealt{dePradaPérez2020}) and still others have examined perseveration’s role in the expression of second person singular pronouns (\citealt{CallaghanTravis2021}). The diverse coding schemes employed in these studies may well stem from the fact that perseveration has been shown to happen on various levels of linguistic representation including syntactic, semantic, structural, and lexical (see \citealt{PickeringFerreira2008} for a review, \citealt{Travis2007}). For example, while lexical repetition appears to boost the strength of perseveration, perseveration also appears to happen between linguistic structures in the absence of lexical repetition (\citealt{PickeringFerreira2008, Travis2007}) and even when speakers switch between languages (\citealt{dePradaPérez2018, GriesKootstra2017, Sodaci2018, TorresTravis2010}).


In general, this research has shown perseveration to occur through findings such as higher rates of overt SPs in contexts where they are preceded by overt SPs than when they are preceded by nulls and vice versa. In fact, the form of the previous mention of a referent has been found to be one of the strongest predictors of subject form variation in some studies (e.g., \citealt{CarvalhoChild2011, TorresTravis2010}). A functional hypothesis, on the other hand, would predict that repeated marking of forms (such as overt subjects) would be unnecessary once the information was clearly established in accordance with \citegen{Grice1975} maxim of quantity, which states that speakers should “not make a contribution more informative than is required” (p. 45). In contrast to the use of subject forms for functional reasons, perseveration is unique because it appears to happen involuntarily without any pragmatic or functional motivation (\citealt{CameronFlores-Ferrán2004}). Thus, as \citet{CameronFlores-Ferrán2004} describe, it can be considered the part of language that is expressive, that is, where the message may be more spontaneous and less carefully planned. There are accounts that address the relative importance of perseveration vis-a-vis function, shifting the predictive importance toward one or the other (e.g., see \citealt{Otheguy2015} for discussion of the importance of function). It is our view that allowing for a role for functional factors does not diminish the apparent importance of perseveration. Instead, the current study recognizes a role for each.



\section{Subject expression in L2 Spanish}\label{sec:geeslin:4}

Research on subject expression in L2 Spanish began with a focus on the L2 acquisition of the null subject parameter and its associated properties (\citealt{Al-KaseyPérez-Leroux1998, Bini1993, Emberson1987, Galvan1999, Isabelli2004, Liceras1989, LicerasEtAl1997, Phinney1987, White1985}). Some work, informed by generative and optimality theory, explored discourse-pragmatic features as well (\citealt{LaFond2002, LafondEtAl2000, MontrulRodríguezLouro2006, Rothman2007}), although studies that investigated the role of pragmatics on subject expression generally focused on the acquisition of subject expression in \textit{obligatory} rather than \textit{variable} contexts (\citealt{BlackwellQuesada2012, QuesadaBlackwell2009, Rothman2009}). Both generative and discourse-pragmatic approaches to this issue continue to be of interest today (e.g., \citealt{Lozano2002,Lozano2016}).



In contrast to the aforementioned approaches, the variationist method allows us to measure and track patterns of use over the course of development without relying on an assessment of accuracy of a single form in a given context. This is especially helpful for charting acquisition in contexts where more than one subject form is allowable. Existing variationist research on subject expression in L2 Spanish has sought to determine the various linguistic and extra-linguistic factors that guide the use of subject forms. Through a series of studies, \citet{GeeslinGudmestad2008, GeeslinGudmestad2010, GeeslinGudmestad2011} and \citet{GudmestadGeeslin2010} showed that advanced L2 learners appear to reach a native-like sensitivity to the predictors of subject form variation in sociolinguistic interviews. They analyzed all forms produced in the subject position, including null, overt SPs and lexical NPs. Like \citealt{BayleyPease-Alvarez1997}, \citet{GeeslinGudmestad2011} examined referent cohesiveness and found that NSs and NNSs used more overt subjects as distance between mentions of the referent increased or functions of the referent changed. Relevant to the current study, they found that perseveration occurred for these speakers given that “null subjects were followed by a higher frequency of null subjects and overt forms were followed by a higher frequency of overt forms” (\citealt[10]{GeeslinGudmestad2011}).



Continuing this line of work, \citet{LinfordGeeslinForthcoming} studied the L2 acquisition of sensitivity to referent cohesiveness on variable subject expression in Spanish. For their study, 125 NNSs across five levels of proficiency (beginner to highly-advanced) and a group of 25 NSs completed a written contextualized preference task (WCPT) in which aspects of referent cohesiveness were manipulated. Specifically, the distance to the previous mention of the referent, the syntactic function of the previous mention of the referent and the TMA of the verb with the previous mention of the referent were manipulated. For the WCPT, participants selected either a null subject, an SP or a lexical NP to complete phrases that were embedded into a written dialogue. Each item was categorized into one of eight referent cohesiveness categories, from most to least cohesive based on the manipulated factors. The results showed that native-like rates of selection of the three forms did not occur until the highest level of proficiency, suggesting that acquiring the rates of variation across subject forms occurs rather late in the acquisition process. In addition, unlike previous research on oral production data that found a consistent decrease in the use of null subjects as the discourse became less cohesive, they did not find this to be a case between all categories for any group – even the native speakers – including after they reorganized the categories based on the varying degrees of importance of the sub-factors. Hence, even in this highly controlled and structured task, we see that referent cohesiveness and its associated properties, despite their clear importance, cannot be taken as the lone explanatory factor. 



In connecting these findings across studies, we hypothesize that better understanding the role of the form of the previous mention of subject referents and the resulting perseveration is a key step in understanding the limitations of previous findings. In addition, other associated properties of referent cohesiveness not explored yet, such as referent gender, might be related to perseveration as well. In fact, there is complementary research on the acquisition of gender as well as research on the psychological notion of \textit{surprisal,} which suggests that the gender of the referent and the cohesiveness between referents in terms of gender may further play a role in understanding perseveration. For example, \citet{Malovrh2014} found that even the most advanced learners performed less accurately on a written and oral short film retell when producing feminine clitics (i.e., \textit{la}[\textit{s}]) versus masculine clitics (i.e., \textit{lo}[\textit{s}]). He further posited that “masculine forms are used as defaults under conditions in which access to working memory is more restricted” (p. 66), such as experimental tasks. Earlier studies such as \citet{Klee1989} also found that, with respect to object clitics, learners tend to acquire feminine clitics last, and default to \textit{lo} as an archmorpheme in all accusative contexts. Relatedly, studies on noun and adjective agreement such as \citet{Alarcón2010} have found that learners are typically more accurate at producing gender agreement between masculine nouns and adjectives and are often guided by semantic notions such as animacy. Taken together, these disparities in learner performance between masculine and feminine referents suggest that masculine and feminine gender are activated differently in psychological representation. It is possible that, if masculine gender acts as a default, perseveration may obtain less in cases where the prime gender is masculine. Conversely, overt feminine referents may prime a preference for overt feminine referents. This relates to the notion of \textit{surprisal}, as described by \citet{Jaeger2007}. These authors analyzed the English ditransitive construction which varies between the more frequent double NP construction (e.g., \textit{I gave him the book}) and the less frequent NP PP construction (e.g., \textit{I gave the book to him}). The authors found that the less frequent construction was more likely to result in perseveration on the following ditransitive structure. They interpret this result as a product of the \textit{surprisal} caused by the less frequent construction. According to this surprisal-sensitive persistence hypothesis (\citealt[3]{Jaeger2007}), “less expected prime structures are predicted to prime more (i.e., to lead to a bigger increase in the probability of repetition) than more expected prime structures”. In other words, less frequent variants exert a stronger priming effect due to their salience in the discourse. If we assume that masculine is the unmarked gender (and indeed, it has been argued that feminine is the marked gender in Spanish, see \citealt{Beatty-MartínezDussias2019,Harris1991}), then it is possible for a feminine form to more strongly activate an underlying feminine representation, leading to higher rates of perseveration.


\section{The current study}\label{sec:geeslin:5}



The current project was designed to examine the development of subject expression in L2 Spanish, looking specifically at perseveration in referential third person singular contexts\footnote{Here, we describe third person pronouns as “referential” since they refer to persons who are not actively participating in the discourse between interlocutors, as opposed to first and second person pronouns which are deictic in nature (see \citealt{dePradaPérez2020} for further discussion of the referential nature of third person pronouns).} in order to deepen our understanding of the relationship between perseveration of referential pronouns and other discourse-related factors, such as referent cohesion. To meet these goals, the current study answers the following questions:


\begin{enumerate}
 \item What is the overall frequency of subject form selection by native speakers and L2 learners across different levels of proficiency on a written contextualized preference task?

 \item Do L2 learners across different levels of proficiency and native speakers perseverate subject forms on a written contextualized preference task?

 \item If perseveration is attested for a speaker group, is it constrained by independent factors such as the prime form (null, overt or lexical NP), and factors related to referent cohesiveness, such as, switch reference, TMA continuity, gender continuity, and/or gender of the referent?

\end{enumerate}

\subsection{Participants}\label{sec:geeslin:5.1}

The participants were 125 L2 learners of Spanish and 25 NSs. All L2 learners were native speakers of English and ranged in age from 18 to 47 years (average = 21.9 yrs.). There were 70 female and 55 male L2 learners. The L2 learners were divided into five groups of Spanish proficiency (split into five percentile ranges) based on their scores on a 24-item grammar proficiency test (see \sectref{sec:geeslin:5.2} for additional information about the grammar test). \tabref{tab:geeslin:1} summarizes these participant characteristics.

\begin{table}
\begin{tabular}{lr *4{r} c c }
\lsptoprule
      &     & \multicolumn{5}{c}{Year} & \\\cmidrule(lr){3-7}
Level & $N$ & 1  & 2 & 3 & 4 & G  & Test mean (\%)\\\midrule
    1 &  25 & 19 & 6 &   &   &      &  30.8\\
    2 &  18 & 6  & 11& 1 &   &      &  41.7\\
    3 &  22 &    & 8 & 13 & 1&      &  51.3\\
    4 &  31 &    &   & 11 & 20 &    &  68.0\\
    5 &  29 &    &   &    & 4  & 25 &  89.9\\
NS\footnote{Two native speaker participants did not complete the grammar test.} & 25 & \multicolumn{5}{c}{n/a} & 97.3\\
%\href{https://usc-word-edit.officeapps.live.com/we/wordeditorframe.aspx?ui=en%2DUS & rs=en%2DUS & wopisrc=https%3A%2F%2Findiana-my.sharepoint.com%2Fpersonal%2Fkgeeslin_iu_edu%2F_vti_bin%2Fwopi.ashx%2Ffiles%2Fd821b39815614d21828e7eaec5d4ba97 & wdenableroaming=1 & mscc=1 & wdodb=1 & hid=3901F968-6C6C-4192-BBC4-0C0CE3F069C8 & wdorigin=Sharing & jsapi=1 & jsapiver=v1 & newsession=1 & corrid=657c2955-f735-e81f-1fb6-ac119a6eab80 & usid=657c2955-f735-e81f-1fb6-ac119a6eab80 & sftc=1 & mtf=1 & sfp=1 & instantedit=1 & wopicomplete=1 & wdredirectionreason=Unified_SingleFlush & preseededsessionkey=9748705b-4059-89af-b4a0-f3194c0613c5 & preseededwacsessionid=657c2955-f735-e81f-1fb6-ac119a6eab80 & rct=Medium & ctp=LeastProtected#_ftn1}{{\textsuperscript{[1]}}}\\
\lspbottomrule
\end{tabular}\\
\caption{Participant characteristics by year of university Spanish course enrollment and mean score in a grammar proficiency test (“G”: Graduate).}
\label{tab:geeslin:1}
\end{table}



% \href{https://usc-word-edit.officeapps.live.com/we/wordeditorframe.aspx?ui=en%2DUS & rs=en%2DUS & wopisrc=https%3A%2F%2Findiana-my.sharepoint.com%2Fpersonal%2Fkgeeslin_iu_edu%2F_vti_bin%2Fwopi.ashx%2Ffiles%2Fd821b39815614d21828e7eaec5d4ba97 & wdenableroaming=1 & mscc=1 & wdodb=1 & hid=3901F968-6C6C-4192-BBC4-0C0CE3F069C8 & wdorigin=Sharing & jsapi=1 & jsapiver=v1 & newsession=1 & corrid=657c2955-f735-e81f-1fb6-ac119a6eab80 & usid=657c2955-f735-e81f-1fb6-ac119a6eab80 & sftc=1 & mtf=1 & sfp=1 & instantedit=1 & wopicomplete=1 & wdredirectionreason=Unified_SingleFlush & preseededsessionkey=9748705b-4059-89af-b4a0-f3194c0613c5 & preseededwacsessionid=657c2955-f735-e81f-1fb6-ac119a6eab80 & rct=Medium & ctp=LeastProtected#_ftnref1}{{\textsuperscript{[1]}}} {Two native speaker participants did not complete the grammar test.}




The results of a One-Way ANOVA revealed that differences in grammar test scores across participant groups were significant $[F(5,142) = 280.6,\allowbreak p < 0.0001]$, and Games-Howell post hoc tests\footnote{Games-Howell post hoc tests were employed since the test of homogeneity of variances was significant.} revealed significant differences between all participant groups.



The NS group included 19 females and 6 males. Their countries of origin were Argentina (1), Bolivia (1), Colombia (3), Costa Rica (1), Mexico (6), Nicaragua (1), Peru (3), Puerto Rico (1) and Spain (8). Their ages ranged from 22 to 44 (mean = 31.24 years). All were university-educated instructors of Spanish residing in the U.S. at the time of data collection. In addition, all were bilingual in at least Spanish and English. Similar to \citet{GeeslinGudmestad2008}, this group of native speakers was chosen precisely because it is these speakers with whom the L2 learners in our study interact and as such constitute a reasonable target for acquisition.



\subsection{Elicitation tasks}\label{sec:geeslin:5.2}\largerpage
All participants completed three tasks in the following order: a written contextualized preference task (WCPT), a grammar proficiency test, and a background questionnaire. The tasks were administered to the first four groups of L2 participants either on paper or online by means of \textit{Quia Web}\footnote{\url{http://www.quia.com}} during the participants’ regularly scheduled class time. The graduate-level L2 learners and the native speakers completed the tasks online at a location of their choice.


Participants first completed the WCPT, which consisted of 20 items embedded within a fictional dialogue in Spanish. Each item had three response choices, which were identical except for the subject forms: null, overt SP, or lexical NP.\footnote{The reason for limiting the options to these three forms was that \citet{GeeslinGudmestad2008} found that the other forms (e.g., demonstrative pronouns, indefinite pronouns, and interrogative pronouns) represented a very small portion of the data (only 4.6\% for highly-advanced L2 learners and 8.4\% for native speakers).} Participants were instructed to read the dialogue and select the phrase with the form that sounded most natural in each context. The preceding mention of the referent was varied throughout the WCT in order to provide a means for examining perseveration. Additionally, the instrument presented contexts with same and switch reference, same and switch TMA, and also varied the gender of the referent, making it possible to study the way these factors conspire to influence perseveration.



Various linguistic features of the context were controlled in the instrument to avoid potential confounding factors. All referents in each item were animate third person singular referents, and all clauses were independent clauses.  Furthermore, the verb forms in each item were divided evenly between the simple present indicative and the imperfect indicative, allowing for a balance between clearly defined and potentially ambiguous verb forms. Although at first glance this task may resemble those used under other theoretical approaches (e.g., generative), it falls within variationist framing because of (1) its attention to the many independent linguistic factors that come to bear on form selection and (2) the response format that allows participants to indicate a preference without reference to accuracy or acceptability. While not always the case, another defining feature is that it creates extended narrative context rather than eliciting sentence-level judgements. Example \ref{itm:geeslin:7} is an excerpt taken from the written contextualized task followed by a translation into English:\largerpage

\ea
\begin{xlist}
\exi{\textbf{\textit{Jorge:}}} \textit{¿De verdad? Pues ya nunca llego tarde porque cada vez que yo llegaba tarde, ella siempre se ponía muy seria.} \label{itm:geeslin:7}
 \begin{xlista}
 \ex \textit{Decía que le daba igual…}
 \ex \textit{Ella decía que le daba igual…}
 \ex \textit{Juanita decía que le daba igual…}
 \end{xlista}
\exi{\textbf{Jorge:}} Really? Well, now I never arrive late because every time I arrived late, she always got really serious.  
 \begin{xlista}
 \ex ∅ Said that it didn’t matter…
 \ex She said that it didn’t matter…
 \ex Juanita said that it didn’t matter…
 \end{xlista}
 \end{xlist}
\z

The second task, a grammar test, consisted of a fictional narration in Spanish that contained 25 contextualized items in which the participants were instructed to choose between three possible options to complete the sentences grammatically.\footnote{As mentioned earlier, only 8 of the native speaker participants scored 100\% on the proficiency test. Even so, it is important to note that the results on the same grammatical proficiency test of over 500 native and L2 learners of Spanish were submitted to a reliability test using SPSS. The Cronbach’s Alpha for the proficiency test was 0.868, well above the minimum requirement of 0.70 for a test to be reliable \citep{GeorgeMallery2012}.} In the current study, we excluded from the analysis one item due to high levels of variability among native speakers,\footnote{The item that was removed asked test-takers to select between options that contained the Spanish copulas.} specifically, the item that included Spanish copulas and is marked with preterit or imperfect aspect. As both aspectual marking and copula contrast in Spanish tend to vary, the variability on this item is not unusual, leaving a total of 24 items.


The background questionnaire for the L2 learners consisted of 33 questions in English that gathered demographic information (e.g., age, gender, etc.) as well as determined the participants’ current and previous experience with Spanish and other languages. Another questionnaire was created for the NSs which contained 10 items to gather information regarding demographics, time spent in the U.S., and experience with other languages. The data collected through these tasks yielded the description of the participants provided in \sectref{sec:geeslin:5.1}.


\subsection{Coding and analysis}\label{sec:geeslin:5.3}



In this study, we examined contexts in which the previous mention of the referent was in subject position regardless of the distance to the previous mention, thereby following the operationalizations of \citet{Abreu2012}, \citet{Flores-Ferrán2002}, \citet{GeeslinGudmestad2011}, and \citet{Travis2007}. This operationalization allowed us to examine the potential relationship to perseveration of continuity of reference and/or gender of the referent. However, three of the total twenty contexts were excluded from this analysis because there is no previous mention of the referent in subject position or this mention falls in a previous item, where the subject depends on the participant response to the previous item. In addition, one context was excluded from the analysis because the potential previous mention of the same referent has an ambiguous reference. These exclusions left us with 16 items for analysis and a total of 2,396 tokens.  

Our dependent variable in the current analysis is whether perseveration occurred, that is, whether the form selected by the participant was the same or different from the previous form of the referent. Additionally, we coded our data for several independent variables. It will be recalled that previous research indicates that referent cohesiveness, which has been operationalized with varying degrees of distinction, sometimes based on TMA continuity, position of the referent, or other factors, plays an important role in understanding patterns of subject form use. To this end, we explored multiple factors related to referent cohesiveness. From studies such as \citet{GeeslinGudmestad2011} and \citet{LinfordGeeslinForthcoming}, we know that TMA continuity (which distinguishes adjacent tensed verbs that contrast in TMA from those that do not) adds dimension to our understanding of the relationship between referent cohesiveness and perseveration and, thus, we included this factor in our coding. We also examined the variable \textit{gender continuity}, which captures whether a referent has the same gender as the referent of the subject of the previous third person singular verb. As noted earlier, this factor is particularly relevant for third person subjects because third person pronouns are referential. For this factor, we disregarded intervening referents that were not third person singular because we assume that gender continuity is most likely a relevant influence for perseveration of third person singular subject forms. In cases of same reference, there is by definition, also gender continuity and, thus, the key contrast occurs in contexts of switch reference. This interaction is reflected in our coding scheme (details below).\footnote{There is only one item that has an intervening referent that is not third person singular. This item was coded as switch reference for the “same/switch reference” factor, since the intervening verbal subject is first person singular. The same item was coded as same reference, same gender for the “gender continuity” factor, since the closest preceding third person singular verbal subject has the same reference as the subject of this item.} This variable sheds light on the relationship between perseveration and the function that forms, such as overt SPs or lexical NPs, play in distinguishing the current referent from a previous one.


In addition to gender continuity, we also coded for the gender (masculine vs. feminine) of the prime.\footnote{There is only one item where the prime gender and the current referent gender do not correspond. In this item, the current referent is singular feminine [∅\textit{/Ella/Olivia no siente nada por él} ‘null/She/Olivia does not feel anything for him’], while the prime is a plural masculine referent null subject \textit{ellos} ‘they’ which includes the current referent.} Our reasons for including this variable stem from the research reviewed previously on the acquisition of various other L2 Spanish structures that reflect learners’ differing patterns of acquisition with feminine referents (e.g., \citealt{Alarcón2010, Klee1989, Malovrh2014}) as well as the possible role that a non-default form may play in processes such as perseveration (\citealt{Jaeger2007}). We summarize our coding of these independent variables in Tables~\ref{tab:geeslin:2}–\ref{tab:geeslin:6}.\largerpage[-1]\pagebreak


\begin{table}
\begin{tabularx}{\textwidth}{lQ}
 \lsptoprule
 {Categories} & {Example (prompt with response options)}\\\midrule
 Null &  \textit{El único problema es que hace cinco meses que} \EmphEmptySet{∅} \textit{rompió con su exnovio tras una relación de dos años y todavía está un poco triste.}
 \newline
 ‘The only problem is that it has been five months since she (\EmphEmptySet{∅}) broke up with her ex{}-boyfriend after a two-year relationship and she is still a little sad.’\newline
 \textit{∅/Ella/Juanita salía con Paco García.}\newline
 ‘∅/She/Juanita was dating Paco García.’\\
 \midrule
 {Overt pronoun} & {\textit{Sé que} }{\textbf{\textit{ELLA}}}{ \textit{no siente nada por él ahora…Antes…}}\newline
 {‘I know that} {\textbf{\textit{SHE}}} {doesn’t feel anything for him now…Before…’}\newline
 {\itshape ∅/Ella/Olivia sentía algo por él, ¿no?}\newline
 {‘∅/She/Olivia felt something for him, right?’}\\
 \midrule
 {Lexical NP} & {\textit{Sí, sí. Me voy.} }{\textbf{\textit{JUANITA}}}{ \textit{ya me está esperando.}}\newline
 {‘Yes, yes. I’m leaving.} {\textbf{JUANITA} }{is already waiting for me.’}\newline
 {\itshape ∅/Ella/Juanita se irrita un poco cuando llego tarde}\newline
 {‘∅/She/Juanita becomes a little irritated when I arrive late.’}\\
 \lspbottomrule
\end{tabularx}
\caption{Analysis of independent variable: Prime form (The form of the preceding mention of the referent in subject position, regardless of the distance)}
\label{tab:geeslin:2}
\end{table}


\begin{table}
\begin{tabularx}{\textwidth}{lQ}
 \lsptoprule
 Categories & {Example (prompt with response options)}\\\midrule
 Same reference & { Pablo: \textit{Pues, en primer lugar, Olivia no} \textbf{\textit{tiene}} \textit{exnovios.} }\newline
 {‘Pablo: Well, first, Olivia doesn’t \textbf{\textit{have}} ex{}-boyfriends.’}\newline
 {Jorge: \textit{∅/Ella/Olivia salía con Enrique el año pasado, ¿no?}}\newline
 {‘Jorge: ∅/She/Olivia dated Enrique last year, right?’}\\
 \midrule
  Switch reference & {\textit{Ok, ok…la verdad es que} \textbf{\textit{tengo}} \textit{una cita con mi novia.}}\newline
 {‘Ok, ok…the truth is that I \textbf{\textit{have}} a date with my girlfriend.’}\newline
 {\itshape ∅/Ella/Mi novia quiere comer en \pagebreak un restaurante elegante así que...}\newline
 {‘∅/She/My girlfriend wants to eat at an elegant restaurant so …’}\\
 \lspbottomrule
 \end{tabularx}
\caption{Analysis of independent variable: Same vs. switch reference (Whether the subject referent of the immediately preceding tensed verb is the same)}
\label{tab:geeslin:3}
\end{table}

\begin{table}
\begin{tabularx}{\textwidth}{lQ}
 \lsptoprule
 Categories & {Example (prompt with response options)}\\\midrule
 Same TMA & {\textit{Tal vez, pero también él le} }{\textbf{\textit{dice}}}{ \textit{a Juanita que no pasa nada si ella sale con otro chico. Si…}}\newline
 {‘Possibly, but he also tells Juanita that it’s ok if she dates another guy. If…’}\newline
 {\itshape ∅/él/Paco tiene problema conmigo...}\newline
 {‘∅/he/Paco has a problem with me…’}\\
 \midrule
  Different TMA & {\textit{Antes Paco me} }{\textbf{\textit{trataba}}}{ \textit{como a un amigo pero ahora que Juanita sale conmigo,…}}\newline
 {‘Before Paco treated me as a friend but now that Juanita dates me,…’} \newline
 {\itshape ∅/él/Paco ni me mira.}\newline
 {‘∅/he/Paco doesn’t even look at me.’}\\
 \lspbottomrule
\end{tabularx}
\caption{Analysis of independent variable: TMA continuity (Is TMA of previous mention of the referent in subject position the same?)}
\label{tab:geeslin:4}
\end{table}


\begin{table}
\caption{Analysis of independent variable: Gender continuity (Is the gender of the 3\textsuperscript{rd} person subject referent of the immediately preceding tensed verb the same?)}
\begin{tabularx}{\textwidth}{>{\raggedright}p{\widthof{\ different referents}}Q}
\lsptoprule
Categories & {Example (prompt with response options)}\\\midrule
 Same gender, same referents & {Pablo: \textit{Pues, en primer lugar,} \textbf{\textit{OLIVIA}} \textit{no tiene exnovios.} }\newline
 {‘Pablo: Well, first, \textbf{\textit{OLIVIA}} doesn’t have ex{}-boyfriends.’}\newline
 {Jorge: \textit{∅/Ella/Olivia salía con Enrique el año pasado, ¿no?}}\newline
 {‘Jorge: ∅/She/Olivia dated Enrique last year, right?’}\\
\midrule
 {Same gender, different referents} & {Jorge: \textit{¿Seguro que no la conoces? Pues, es la chica con quien hablaba Ana Ramírez después de la clase de biología todos los días el semestre pasado. De hecho,} \textbf{\textit{ANA} }\textit{es su mejor amiga.}}\newline
 {‘Jorge: Are you sure you don’t know her? Well, she is the girl with whom Ana Ramírez chatted after the biology class everyday las semester. In fact, \textbf{\textit{ANA}} is her best friend.’}\newline
 {Pablo: \textit{Ah, ok...sí...}}\newline
 {‘Pablo: Ah, ok…yes…’}\newline
 {\itshape ∅/Ella/Juanita es muy guapa entonces.}\newline
 {‘∅/She/Juanita is very beautiful then.’}\\
\midrule
 {Different gender, different referents} & {\textit{Antes Paco me trataba como a un amigo pero ahora que} }{\textbf{\textit{JUANITA}}}{ \textit{sale conmigo,…}}\newline
 {‘Before Paco treated me as a friend but now that} {\textbf{\textit{JUANITA} }}{dates me,…’}\newline
 {\itshape ∅/él/Paco ni me mira.}\newline
 {‘∅/he/Paco doesn’t even look at me.’}\\
 \lspbottomrule
\end{tabularx}
\label{tab:geeslin:5}
\end{table}

\begin{table}
\caption{Analysis of independent variable: Prime gender}
\begin{tabularx}{\textwidth}{lQ}
\lsptoprule
 Masculine & \textit{Tal vez, pero también él le dice a Juanita que no pasa nada si ella sale con otro chico. Si…}\newline
 {‘Possibly, but he also tells Juanita that it’s ok if she dates another guy. If…’}\newline
 {\itshape ∅/él/Paco tiene problema conmigo...}\newline
 {‘‘∅/he/Paco has problem with me…’}\\
\midrule
 {Feminine} & {\textit{¿De verdad? Pues ya nunca llego tarde porque cada vez que yo llegaba tarde,} }{\textbf{\textit{ELLA}}}{ \textit{siempre se ponía muy seria.} }\newline
{‘Really? Well now I never arrive late because every time I arrived late,} {\textbf{\textit{SHE}}} {always got very serious.’}\newline
 {\itshape ∅/Ella/Juanita decía que le daba igual.}\newline
 {‘∅/She/Juanita said that she didn’t care.’}\\
\lspbottomrule
\end{tabularx}
\label{tab:geeslin:6}
\end{table}

The initial step in our analysis was to examine\largerpage{} the distribution of the subject forms selected by each participant group. Although this is not the dependent variable for the remaining analyses, it is important to provide this distribution as a backdrop for comparison to other studies. Following the reporting of the distribution of the forms selected on our WCPT, we provide a similar report of the distribution of the perseveration attested by each participant group. Although we do provide the overall rates of perseveration by group, we note that it is the rate of perseveration within the prime form that is more meaningful in answering our research questions and contextualizing our findings within the existing research on this subject. Our analysis then turns to a statistical examination of the degree to which the independent factors in our analysis are related to perseveration for each group. We present the findings of a binary logistic regression analysis{\interfootnotelinepenalty=10000\footnote{We used the Generalized Estimating Equations tool in SPSS 27 with participant as an exchangeable subject variable for the regressions.}} for each participant group as a means for answering our third research question. The independent variables included in the regression are prime form, TMA continuity, and gender continuity. We included prime form in order to examine how this variable affects perseveration when other factors are considered in the same statistical model. Gender continuity rather than same vs. switch reference was included since the former further specifies the degrees of discourse cohesion captured in the switch reference variable and it allows us to tease apart the effects of selecting each form for functional reasons (e.g., contrast/clarity) and psychological or expressive reasons (e.g., priming). Specifically, one would expect based on functional use of subject forms that null subjects would be perseverated most often in same reference contexts, overt SPs would be perseverated most often in switch reference contexts where the gender is contrastive with the previous subject referent, and lexical NPs would be perseverated most often in switch reference contexts, especially without contrastive gender of the referents. The degree to which these functional predictions (do not) account for the patterns attested, indicates a role for psychological effects, such as perseveration. We did not include the prime gender variable in the regression because there were unintended correlations between the prime gender and switch reference in the instrument design. The final step of our analysis is to focus more directly on the prime gender and its relationship to perseveration. By looking at the prime gender only in contexts of switch reference, we are able to test the hypotheses put forth earlier regarding the degree to which perseveration is differential for default vs. non-default forms.



\section{Results}\label{sec:geeslin:6}
\begin{sloppypar}
As described previously, we begin our presentation of the results with an overview of the distribution of the forms selected according to proficiency level. \tabref{tab:geeslin:7} shows the overall distribution of the subject forms selected by each participant group and the percentage each form selected constitutes within each participant group. The note below the table provides details of ANOVA tests of differences between groups for each subject form.
\end{sloppypar}

\begin{table}
\begin{tabular}{l *{6}{r} c}
 \lsptoprule
         & \multicolumn{2}{c}{***} & \multicolumn{2}{c}{***} & \multicolumn{2}{c}{***} & \\
 {Level} & \multicolumn{2}{l}{Null subjects} & \multicolumn{2}{l}{Overt SPs} & \multicolumn{2}{l}{Lexical NPs} & {Total}\\\cmidrule(lr){2-3}\cmidrule(lr){4-5}\cmidrule(lr){6-7}
 & {\#} & {\%} & {\#} & {\%} & {\#} & {\%} & {$N$}\\
 \midrule
 1 &  133 &  33.5 &  151 &  38.0 &  113 &  28.5 &  397\\
 2 &  99 &  34.5 &  106 &  36.9 &  82 &  28.6 &  287\\
 
 3 &  127 &  36.1 &  155 &  44.0 &  70 &  19.9 &  352\\
 4 &  249 &  50.2 &  179 &  36.1 &  68 &  13.7 &  496\\
 5 &  334 &  72.0 &  90 &  19.4 &  40 &  8.6 &  464\\
 NS &  307 &  76.8 &  63 &  15.8 &  30 &  7.5 &  400\\
 \lspbottomrule
\end{tabular}
\caption{Distribution of forms selected by group. Note: ANOVA tests between groups $\text{***}=p<0.001$,   Nulls [$F (5,144) = 36.13,\allowbreak p<0.001$], Overt SPs [$F (5,144) = 8.817,\allowbreak p<0.001$], Lexical NPs [$F (5,144) = 18.433,\allowbreak p<0.001$].\label{tab:geeslin:7}}
\end{table}

\tabref{tab:geeslin:7} demonstrates that the L2 learners select a relatively large proportion of overt SPs and lexical NPs at lower proficiency levels and gradually select a higher percentage of null subjects as L2 proficiency increases. This trend is especially noticeable at level 4 where the selection rates of overt SPs and lexical NPs decrease, accompanied by a sharp increase in the selection of null subjects. Results from three one-way ANOVAs comparing the selection rates of each of the forms between groups were significant (see note below \tabref{tab:geeslin:7}). Post-hoc Tukey tests showed that for null subjects, the selection rates for learners in levels 1-3 were not significantly different from each other, level 4 learners’ selection rate of null subjects was significantly different from all other levels, and level 5 and NSs were not significantly different from each other. For overt SPs, post hoc Tukey tests showed that the selection rates from level 1 to 4 were not significantly different from one another and level 5 and NSs were not significantly different from each other. Finally, post hoc Games-Howell tests showed that for lexical NPs, levels 1–3 were not significantly different from each other, and level 4, 5 and the NSs were not significantly different from one another. In sum, we see that these apparent shifts, at level 4 for null subjects and lexical NPs, and at level 5 for overt SPs represent significant shifts in rates of selection.



We now turn our analysis toward the overarching question of whether we find perseveration, for all groups and for all forms, and what other independent variables help us understand the patterns of perseveration attested in our dataset. \tabref{tab:geeslin:8} shows the rates of perseveration within participant groups for each form and the overall rates of perseveration. For example, for level 3 learners, when the prime form was null, the form selected was also null in 34 tokens, which represent 30.9 percent of the cases; and in total, they selected the same form as the prime with 90 tokens (25.6 percent of all the cases). The use of the asterisks (*) indicates the results of chi-square tests that measured the degree to which these patterns were significantly different by form, within the group. Building on the previous example, this means that for level 3, the $p$-value for a test measuring the degree to which rates of perseveration differed by form was smaller than 0.001.

\begin{table}
\begin{tabular}{l *8{r}}
 \lsptoprule
 {Level}  & \multicolumn{2}{c}{Null} & \multicolumn{2}{c}{Overt SP} & \multicolumn{2}{c}{Lexical NP} & \multicolumn{2}{c}{OP}\\
 \cmidrule(lr){2-3}\cmidrule(lr){4-5}\cmidrule(lr){6-7}\cmidrule(lr){8-9}
 & \#  & \%  & \#  & \%  & \#  & \%  &  \# &  \%\\
 \midrule
 1  & 27  & 21.8  & 39  & 31.5  & 36  & 24.2  &  102 &  25.7\\
 2  & 27  & 30.0  & 28  & 31.5  & 22  & 20.4  &  77 &  26.8\\
 3***  & 34  & 30.9  & 41  & 37.3  & 15  & 11.4  &  90 &  25.6\\
 4***  & 78  & 50.3  & 50  & 32.3  & 13  & 7.0  &  141 &  28.4\\
 5***  & 102  & 70.3  & 22  & 15.2  & 6  & 3.4  &  130 &  28.0\\
 NS***  & 93  & 74.4  & 22  & 17.6  & 2  & 1.3  &  117 &  29.3\\
 \lspbottomrule
\end{tabular}
\caption{Number and percentage of perseveration by prime form and overall perseveration (OP). Note: $\chi^2$ tests, $\text{***} = p<0.001$.\label{tab:geeslin:8}}
\end{table}



\tabref{tab:geeslin:8} demonstrates that starting at level 3 and continuing for each of the more advanced groups, and the NSs, the perseveration rates differ significantly across the three forms examined. This result is highly anticipated as we would not expect similar rates of perseveration for null subjects and lexical NPs in natural discourse. On the contrary, we might expect that because the perseveration of lexical NPs is functionally unnecessary for content recovery, it would occur at lower rates. A more interesting question is whether the rates of perseveration for a given form differ by participant group, as this would demonstrate the path of acquisition, and whether our learners arrive at native-like patterns of perseveration. To assess this, we compared perseveration rates between groups by means of One-Way ANOVAs for each prime form (i.e., three separate ANOVAs). The results of the ANOVAs show that there are significant differences in perseveration rates between groups for null primes [$F (5, 144) = 25.003,\allowbreak p<0.001$], overt SP primes [$F (5,144) = 5.187,\allowbreak p<0.001$] and for lexical NP primes [$F (5,144) = 10.519,\allowbreak p<0.001$]. Post hoc Tukey tests revealed that for perseveration of null primes, levels 1–3 were not significantly different from each other, level 4 was significantly different from all other groups and level 5 and NSs were not significantly different from each other. For overt SPs, post hoc Games-Howell tests showed no significant differences in perseveration between levels 1-4; level 5 was significantly different from levels 1, 3 and 4, and NSs were only significantly different from level 3. For lexical NP primes, levels 1–3 were not significantly different from each other, levels 3–5 were not significantly different from each other, and levels 4-NSs were not significantly different from one another. As a whole, this indicates that level 5 was the only group that demonstrated consistent native-like patterns of perseveration across primes. Adding detail, for null primes it appears that there is a shift at level 4 that leads to more native-like patterns by level 5, whereas this happens somewhat sooner for lexical NPs, showing the transition between levels 3 and 4. Patterns for overt SPs are not as linear as for null subjects and lexical NPs and, thus, the patterns attested by the ANOVA are more complex, but show a general trend toward reduction in perseveration over time, with a dramatic shift between levels 4 and 5.



Thus far, our analysis demonstrates that perseveration does in fact vary by form and proficiency level. To understand these patterns, we conducted separate binary logistic regressions for each level to examine other factors that may influence perseveration (see \sectref{sec:geeslin:5.3} for independent variable details).  \tabref{tab:geeslin:9} summarizes the results of each of these six statistical models. 

\begin{table}
\robustify\bfseries
\begin{tabular}{l *2{S[table-format=2.3] S[table-format=<1.3]} *2{S[table-format=1.3,text-series-to-math]}}
\lsptoprule
 {Level} & \multicolumn{2}{c}{{Prime form}} & \multicolumn{2}{c}{Gender continuity} & \multicolumn{2}{c}{TMA continuity}\\\cmidrule(lr){2-3}\cmidrule(lr){4-5}\cmidrule(lr){6-7}
  & {Wald $\chi^2$} & {$p$} & {Wald $\chi^2$} & {$p$} & {Wald $\chi^2$} & {$p$}\\
\midrule
 1 &  2.399            &   0.301 &  0.738 &   0.691 &  0.942 & 0.332\\
 2 &  3.907            &   0.142 &  0.733 &   0.693 &  \bfseries 7.335 & \bfseries 0.007\\
 3 &  \bfseries 32.731 &  \bfseries <0.001 &  7.107 &   0.029 &  0.012 & 0.912\\
 4 &  \bfseries 52.368 &  \bfseries <0.001 &  2.221 &   0.329 &  1.642 & 0.200\\
 5 &  \bfseries 59.282 &  \bfseries <0.001 &  \bfseries 22.687 &  \bfseries <0.001 & \bfseries 8.655 & \bfseries 0.003\\
 NS & \bfseries 51.358 &  \bfseries <0.001 &  \bfseries 11.163 &  \bfseries 0.004 &  1.900 &  0.168\\
\lspbottomrule
\end{tabular}
\caption{Results of binary logistic regressions for perseveration by level (significant results bolded)}
\label{tab:geeslin:9}
\end{table}

As is demonstrated in \tabref{tab:geeslin:9}, prime form is a significant predictor of perseveration for level 3 and above. In contrast, the factors related to referent cohesiveness, such as gender continuity (which includes a measure of switch reference) and TMA continuity do not seem to play a significant role in patterns of perseveration until level 5. Additionally, the role of gender continuity is apparent for highly advanced learners and for NSs. We did find one anomalous result in that TMA continuity is significant at level 2, but then not for other levels until level 5. With regard to the direction of these effects, for prime form, we find that for nearly all groups for which this variable was significant, perseveration was most common with null primes, followed by overt SP primes, then lexical NP primes. The only exception to this trend was Level 3 in which perseveration occurred more with overt SP than null primes. For gender continuity, for level 5 and the NSs, perseveration occurred least often in same reference contexts and more often in switch reference contexts, with little apparent difference with and without switches in gender. Finally, for continuity of TMA, level 3 demonstrated more perseveration when there was a switch in TMA whereas level 5 showed the opposite trend.  We will discuss this result further in the section that follows. To summarize the overall patterns in terms of development, we see that learners first perseverate at different rates by prime form and then, at much higher levels of development, begin to demonstrate patterns of perseveration that are sensitive to independent variables related to referent cohesiveness. Such factors are indicators of functional explanations for (lack of) perseveration and it is to be expected that the interplay between functional patterns and psychological ones, such as perseveration, requires advanced ability in a language. We will return to these results in the discussion section.

The final step in our analysis was to examine whether the prime gender plays a role in perseveration. This analysis focused on switch reference contexts for which the nearest previous mention of the referent was in subject position. This narrower scope was selected because we identified interactions with switch reference as well as the distance of the mention of the reference, on the one hand, and the prime gender, on the other. These interactions were artifacts of the instrument rather than indications of how these might operate in naturally-occurring language. For example, on the WCPT we found that items with male primes were found in significantly more contexts of switch reference (57.2\%) than those with female primes (33.4\%) [Pearson $\chi^2 = 135.391, \text{df}=1, p<0.001$]. We further noted that although our original analysis included previous referents in subject position, regardless of intervening mentions of the referent, that this might influence the role that gender of the prime played. Consequently, we further narrowed our analysis to those contexts where the previous mention of the referent was in subject position (e.g., there were no intervening mentions of the referent as verbal objects). \tabref{tab:geeslin:10} presents the results of Chi-square tests to determine if the prime gender significantly correlated with perseveration for each participant group in contexts of switch reference where the previous mention of the referent was in subject position. 

\begin{table}
\robustify\bfseries
\begin{tabular}{l S[table-format=2.3] S[table-format=<1.3] *2{r S[table-format=2.1,text-series-to-math]}}
\lsptoprule
 {Level} & {Pearson $\chi^2$} & {$p$} & \multicolumn{2}{c}{{\textsc{m} prime}} & \multicolumn{2}{c}{{\textsc{f} prime}}\\\cmidrule(lr){4-5}\cmidrule(lr){6-7}
&   & & {$N$} & {\%} & {$N$} & {\%}\\
\midrule
 1  & 0.350  & 0.554  & 20 &  26.7 & 11 &  22.0\\
 2  & 0.854  & 0.355  & 9  &  17.0 & 9  &  25.0\\
 3  & 0.287  & 0.582  & 18 &  27.3 & 10 &  22.7\\
 4  & \bfseries 6.596  & \bfseries 0.010  & \bfseries 21 & \bfseries 22.6 & \bfseries 26 & \bfseries 41.9\\
 5  & \bfseries 13.647 & \bfseries <0.001 & \bfseries 18 & \bfseries 20.7 & \bfseries 29 & \bfseries 50.0\\
 NS & \bfseries 13.539 & \bfseries <0.001 & \bfseries 13 & \bfseries 17.3 & \bfseries 24 & \bfseries 48.0\\
\lspbottomrule
\end{tabular}
\caption{Perseveration by prime gender in switch reference contexts with previous mention of referent in subject position (significant results bolded)}
\label{tab:geeslin:10}
\end{table}

As shown in \tabref{tab:geeslin:10}, prime gender was found to significantly correlate to perseveration for levels 4, 5 and NSs. In every case, when this factor was found to be significant, there was significantly more perseveration for feminine primes than for masculine ones. As with the results from the regression analysis, we will further explore this finding in the discussion that follows.


\section{Discussion}

The current study was designed to expand our understanding of how perseveration operates with third person referential subjects for second language learners of Spanish and how these patterns change as proficiency increases. Our first research question examined the distribution of subject forms selected on the WCPT and how this differed by group. Our analysis showed that in the early stages of acquisition (levels 1–3), L2 learners select each form at similar rates, suggesting that their selection is not (heavily) influenced by contextual factors. At levels 3–4, however, learners begin showing differences in selection rates for each form. As L2 proficiency level increased, patterns indicated a shift toward greater selection rates of null forms and lowered selection rates of overt SPs and lexical NPs. For null and lexical NPs, the shift toward native-like rates occurred at level 4, whereas this occurred later (only at level 5) for overt SPs. These results confirm previous research employing both experimental and spontaneous oral data (e.g., \citealt{GeeslinGudmestad2010, GeeslinGudmestad2011}) and suggest that acquiring native-like patterns of subject form variation occurs relatively late in the acquisition process. As previous research has suggested, native-like variation of this structure is guided by a myriad of semantic and discursive attributes of the linguistic context, as well as psychological processes such as priming, and this complexity likely contributes to acquisition rates of these patterns. 


Our second research question examined the rates of perseveration. Our analysis showed that from level 3 onward the rates of perseveration differed by prime form, and from level 4 the perseveration rate was highest for null primes and lowest for lexical NP primes. We note that this difference across forms, and the direction of those differences, is not surprising, but it is important from a developmental standpoint to understand when L2 learners begin to reflect these patterns in their own use. Adding further depth to these developmental patterns, our analysis of patterns within prime forms showed that for null subjects, only level 5 learners reached native-like patterns (i.e., did not differ significantly from the native speaker group). For overt SPs the trend over levels was not linear, but level 5 did reach native-like rates of perseveration. Finally, for lexical NPs this occurred slightly earlier in the process and level 4 learners were shown not to differ significantly from level 5 or from the NSs. The gradual differentiation of patterns by form as proficiency increases is consistent with previous research, regardless of the additional factors under examination (e.g., \citealt{GeeslinLinfordFafulas2015}). However, the current study is the first to our knowledge to look at changes in perseveration rates by level.


\begin{sloppypar}
The remainder of our analysis (research question 3 and its sub-questions) sought to explore the role of additional factors in understanding perseveration among our L2 learner groups. Henceforth our dependent variable is whether perseveration occurred, and the goal of the analysis is to determine which independent factors contribute to the occurrence of perseveration. The regression analyses conducted for each level indicate stable patterns of change across proficiency levels, with the exception of the effect of TMA continuity, which was a significant predictor of perseveration at levels 2 and 5, but in opposite directions. We hypothesize that the results for the role of TMA continuity for level 5 are in line with the general trends and the level 2 results are anomalous. This may reflect the pattern of acquisition of the two morphosyntactic forms in alternation on our instrument, rather than a fact related to perseveration. Specifically, the acquisition of the two forms that mark past tense in Spanish is likely in progress for level 2 learners (\citealt{Salaberry2011}) and the forms are likely to draw additional attention until they are incorporated into the learners’ grammar (\citealt{VanPatten1990}). Lower rates of perseveration with switches in TMA for level 2 may indicate the attention required to process these forms at this particular level, whereas by level 5 we see the direction of effect that explanations based on referent cohesiveness would predict. Returning then to overall patterns of development attested by the regression analyses, we see that as proficiency increases, learner grammars move toward patterns that are predicted by an increasing number of factors. The importance of the prime form is attested for level 3 and above and the role of gender continuity becomes apparent at level 5, as does TMA continuity. The reader will recall that because of the overlap between gender continuity and a dichotomous switch reference distinction, whereby same reference contexts are also, by definition, contexts where gender is also continuous, we combined these factors into a single, three-part distinction. Consequently, this variable represents a level of complexity that a simpler switch vs. same reference distinction would not. Our hypothesis is supported by earlier studies that show a relatively earlier effect for switch reference when not combined with gender continuity (\citealt{GeeslinLinfordFafulas2015}). It is likely that this complexity explains why the variable is significant only for the highest proficiency level and for NSs, whereas \citet{LinfordGeeslinForthcoming} found that referent cohesiveness alone constrained subject expression as early as level 3 among their learners.
\end{sloppypar}


The final step in our analysis was to look not only at whether there were shifts in gender of the referent, a reflection of referent cohesion that offers a functional explanation of perseveration, but also whether the prime gender is related to perseveration. This final variable speaks to hypotheses related to surprisal or default forms and is based on psychological processes rather than functional ones (\citealt{Jaeger2007}). Our results do, in fact, demonstrate an effect for prime gender beginning at level 4 and continuing for level 5 and the NS group. Specifically, for these groups we see that feminine primes are more likely to perseverate than masculine ones. This finding is consistent with our predictions as the feminine form is described in the literature as less frequent, and the less likely form to serve as the default, both for learners and NSs (\citealt{Alarcón2010, Klee1989, Malovrh2014}). In terms of L2 development, it is reasonable to expect that learners must reach a fairly advanced level of proficiency in order to demonstrate these sophisticated patterns of language processing. Although our findings are as expected, our study is the first of its kind to demonstrate a role for both functional and psychological processes as they relate to L2 perseveration.



Throughout the discussion, we have provided an account of the differences between levels in an effort to use our cross-sectional design to illustrate the path of L2 acquisition. In order to provide a snapshot of development that captures all of our findings, we summarize the results in \tabref{tab:geeslin:11} in terms of developmental trends by level of proficiency.
\vfill
\begin{table}[H]
\begin{tabularx}{\textwidth}{lQ}
\lsptoprule
 {Level} & {Summary of patterns}\\
 \midrule
 {1} & {Subject form selection rates are near chance; Perseveration rates are not influenced by prime form, gender or TMA continuity or prime gender} \\
 \midrule
 {2} & {Similar to level 1, except perseveration rates for null SPs increase, and TMA continuity has a level-specific relationship to perseveration}\\
 \midrule
 {3} & {Subject form selection rates are similar to levels 1 and 2; Perseveration rates begin to show differentiation by prime form (null, overt SP and lexical NP); Perseveration rates for null primes and lexical NP primes begin to shift toward native-like patterns}\\
 \midrule
 {4} & {Rates of selection for null and lexical NP subjects are significantly different from lower levels (overt SPs are not); Perseveration rates for null primes are significantly different from lower levels and those for overt SP and lexical NP primes are like NSs; Perseveration rates are significantly influenced by prime form, but not by other discourse factors; prime forms denoting feminine referents start to be more likely perseverated and this continues through higher levels}\\
 \midrule
 {5} & {Subject selection rates are like NSs for all forms and uniquely so, for null and overt SPs; Perseveration rates for all prime forms are like NSs; perseveration rates are significantly related to prime form, gender continuity and TMA continuity; prime forms denoting feminine referents continue to be perseverated at higher rates}\\
 \midrule
 {NS} & {Subject form selection rates are significantly different by form; perseveration rates are significantly related to prime form and gender continuity; prime forms denoting feminine referents are perseverated at higher rates}\\
\lspbottomrule
\end{tabularx}
\caption{Developmental stages for subject form perseveration}
\label{tab:geeslin:11}
\end{table}\vfill\pagebreak

As seen in \tabref{tab:geeslin:11}, there is little difference in form selection and rates of perseveration between levels 1 and 2 and patterns at this level do not appear to be influenced by the independent factors in the current study. By level 3, however, rates of selection remain similar to earlier levels but rates of perseveration for null subjects and lexical NPs are shifting towards native-like tendencies. Nevertheless, we do not see a marked influence on rates of perseveration by the referent cohesiveness variables, nor by the prime gender. The learners at level 4 show the sharpest differentiation from earlier levels. They exhibit significantly different rates of selection of null and lexical NP subjects, and they have reached native-like rates of perseveration for lexical NPs and also differ significantly from lower levels in their rates of perseveration of null subjects. However, not all independent variables in the current study have begun to demonstrate a significant relationship to perseveration given that gender continuity nor TMA continuity was not significant for this group. The patterns documented for learners at level 5 are similar to the native speaker group in several ways. First, level 5’s rates of selection of all forms are comparable to that of native speakers. Additionally, their rates of perseveration for null, overt SP and lexical NP primes do not significantly differ from native speakers. It is also at level 5 that we begin to see the native-like influence of gender continuity on patterns of perseveration.  However, TMA continuity was significant for level 5 but not for the NSs. 


In sum, the current study adds to the body of literature on the L2 acquisition of subject expression by exploring factors related to perseveration.  Furthermore, our study is the first to demonstrate the role that the prime gender plays in perseveration. Finally, and perhaps most importantly, the results provide evidence for a developmental path in which learners move towards nativelike subject expression as they gain proficiency in Spanish. It is noteworthy that differences remain even between highly advanced learners and native speakers as patterns that reflect this complex interplay of factors are likely to develop only with extensive exposure to the language. Taken together, our findings suggest a role for functional factors that proceed from the discourse at hand, as well as the psychological factors, such as priming, that lead to perseveration. Indeed, the study stresses the importance of considering functional accounts (see \citealt{Otheguy2015}) as well as the effect of perseveration in order to account for subject expression in both native and L2 Spanish. Our findings further suggest that controlled instruments such as the WCPT in the present study can be effective means for teasing out these subtle differences (see also \citealt{GeeslinLinfordFafulas2015}).


\section{Conclusion and future directions}

The present study has shown that perseveration occurs among L2 learners of Spanish and its study provides important information about the acquisition of subject forms. Specifically, we see that learners first come to differentiate rates of selection of subject forms and then, at higher levels of proficiency, demonstrate varying rates of perseveration by form. Additionally, we see that learners do come to make use of other related factors, such as gender continuity and TMA continuity. We also showed that for learners as well as for native speakers, the examination of prime gender adds dimension to our knowledge of the perseveration of subject forms. These findings are consistent with existing literature but also provide new insights related to L2 acquisition and the role of prime gender in particular.


Benefits of our findings notwithstanding, there are limitations to the current project that provide impetus for future investigations. As is often the case, the benefits of using a highly controlled elicitation instrument were appropriate given the goals of our investigation and, at the same time, it is important to take what we have learned and explore these same patterns in more freely produced samples of language. Specifically, it will be important to expand these findings in contexts where there are a greater number of referents in play and the narrative structure is more complex. We further recognize that learner populations differ, and these results should be expanded to include learners in other contexts and with other first languages. Similarly, adding additional native speaker groups to the study of these factors is essential to corroborate and build on our findings. Clearly our native speaker group serves as an example of the input our learners receive, but they do not represent the diverse speech communities throughout the Spanish-speaking world. 

\printbibliography[heading=subbibliography]

\end{document}
