\documentclass[output=paper]{langscibook}
\ChapterDOI{10.5281/zenodo.8124486}
\author{Peter Herbeck\affiliation{University of Vienna \& University of Wuppertal} and Pekka Posio\orcid{}\affiliation{University of Helsinki}}
\title[Introduction]
      {Referring to discourse participants in Ibero-Romance languages: Introduction}
\abstract{}
\IfFileExists{../localcommands.tex}{
  \addbibresource{../localbibliography.bib}
  \usepackage{langsci-optional}
\usepackage{langsci-gb4e}
\usepackage{langsci-lgr}

\usepackage{listings}
\lstset{basicstyle=\ttfamily,tabsize=2,breaklines=true}

%added by author
% \usepackage{tipa}
\usepackage{multirow}
\graphicspath{{figures/}}
\usepackage{langsci-branding}

  
\newcommand{\sent}{\enumsentence}
\newcommand{\sents}{\eenumsentence}
\let\citeasnoun\citet

\renewcommand{\lsCoverTitleFont}[1]{\sffamily\addfontfeatures{Scale=MatchUppercase}\fontsize{44pt}{16mm}\selectfont #1}
   
  %% hyphenation points for line breaks
%% Normally, automatic hyphenation in LaTeX is very good
%% If a word is mis-hyphenated, add it to this file
%%
%% add information to TeX file before \begin{document} with:
%% %% hyphenation points for line breaks
%% Normally, automatic hyphenation in LaTeX is very good
%% If a word is mis-hyphenated, add it to this file
%%
%% add information to TeX file before \begin{document} with:
%% %% hyphenation points for line breaks
%% Normally, automatic hyphenation in LaTeX is very good
%% If a word is mis-hyphenated, add it to this file
%%
%% add information to TeX file before \begin{document} with:
%% \include{localhyphenation}
\hyphenation{
affri-ca-te
affri-ca-tes
an-no-tated
com-ple-ments
com-po-si-tio-na-li-ty
non-com-po-si-tio-na-li-ty
Gon-zá-lez
out-side
Ri-chárd
se-man-tics
STREU-SLE
Tie-de-mann
}
\hyphenation{
affri-ca-te
affri-ca-tes
an-no-tated
com-ple-ments
com-po-si-tio-na-li-ty
non-com-po-si-tio-na-li-ty
Gon-zá-lez
out-side
Ri-chárd
se-man-tics
STREU-SLE
Tie-de-mann
}
\hyphenation{
affri-ca-te
affri-ca-tes
an-no-tated
com-ple-ments
com-po-si-tio-na-li-ty
non-com-po-si-tio-na-li-ty
Gon-zá-lez
out-side
Ri-chárd
se-man-tics
STREU-SLE
Tie-de-mann
} 
  \togglepaper[1]%%chapternumber
}{}

\begin{document}
\maketitle

\section{Introducing reference to discourse participants} 

The current volume aims at presenting a panoramic view of recent advances in the study of reference to discourse participants in Ibero-Romance languages and to search for connections between phenomena that have previously been studied in isolation. It brings together contributions on person reference in Ibero-Romance languages that go beyond the well-established field of study focusing on the expression vs. non-expression of subject pronouns. Several corpus studies on Ibero-Romance languages have shown that the phenomena affecting the expression of subject pronouns transcend the traditionally established factors like morphological ambiguity of the verb, emphasis, contrast, and topic continuity. Besides additional factors like tense/aspect/mood marking, subjectivity, and the degrees of fixation of subject-verb combinations with particular verb forms (see, e.g., \citealt{CarvalhoShin2015} and \citealt{Posio2018} for an overview), it has been demonstrated that grammatical person is one of the most significant variables affecting subject expression, and that the factors influencing the expression of anaphoric third person subjects differ from those conditioning deictic first- and second-person subjects. The question of overt/covert alternations in syntax has also been addressed in research focusing on the encoding of impersonal reference in languages with variable subject expression, such as new pronoun-like referential devices emerging from noun phrases, and the uses of impersonal constructions like passives and impersonals formed with the reflexive clitic \textit{se}  to express personal reference. These devices allow speaker-inclusive and/or addressee-inclusive interpretations to different degrees, depending on the variety and the type of nominal expression encoding impersonality. Moreover, while approaches to variable pronoun expression have traditionally focused on nominative subjects, recent research has opened new avenues to studying to what extent the same or different factors affect the expression of other forms such as direct (accusative) objects and indirect (dative) objects encoding experiencers and recipients.



This introductory chapter provides an overview of the topics of this volume and emphasizes the relevance of the studies included against the background of previous research on pronouns and other referential devices. We use this latter term, as suggested by \citet{Kibrik2011}, to subsume bound person marking morphemes, pronouns and noun phrases used to establish reference, in order to highlight the fact that the boundaries of the traditional categories “pronoun” and “noun phrase” are flexible and some semi-grammaticalized items display properties of both (e.g., Portuguese \textit{a gente}  ‘the people; we’ and  \textit{a/uma pessoa}  ‘the/a person; one’). By focusing on discourse participants, i.e., the speaker(s) and the addressee(s), this volume takes into account the findings of previous research regarding the similarities between these categories as well as differences with regard to the third person.\largerpage

\begin{sloppypar}
For instance, while third-person referents need to be introduced into the discourse and the choice of referential devices (null pronouns, agreement morphemes, weak pronouns, demonstratives, strong pronouns, definite and indefinite NPs, lexical expressions) referring to them is therefore heavily influenced by in\-for\-ma\-tion-struc\-tur\-al categories, such as referentiality, topic continuity, and accessibility (\citealt{Givón1983, Levinson1987, Ariel1990, GundelZacharski1993}, among many others), discourse participants are, in principle, always accessible by virtue of being present in the communicative situation. To use the file card metaphor (\citealt{Reinhart1981, Heim1982}), the cards for the speaker and addressee are always on top of the file and, thus, available as topics (\citealt[45--46]{Erteschik-Shir2007}). For first-person pronominal forms, it has therefore been argued that, apart from topic and referential continuity, factors in determining subject pronoun expression include subjectivity and the epistemic stance of the speaker (see, e.g., \citealt{Enríquez1984, AijónOlivaSerrano2010, Posio2011, Herbeck2021}), probably depending on the type of verb lexeme or individual verb forms. For address forms in morphological second and third person, use of a subject pronoun or noun phrase is governed by factors related to formality of the discourse, (positive and negative) politeness and (inter)personal relations between the interlocutors (see \citealt{DeJongeNieuwenhuijsen2012} and \citealt{Uber2016} for an overview and references). 
\end{sloppypar}


However, it is also important to note that grammatical and referential categories do not always coincide: canonically, speakers refer to themselves with first-person pronouns and to their addressees by second-person pronouns, but in practice, there is ample variation in the choice of referential devices. The use of noun phrases formally in the third person to refer to first and second person entities is a well-known case, as is the emergence of formally third-person expressions (like the Spanish impersonal pronoun \textit{uno} ‘one’ or the Portuguese noun phrases \textit{a pessoa}  ‘the person’; \citetv{chapters/08-Orozco-others, chapters/05-Amaral-Mihatsch}) that establish reference to the speaker, either through generalization involving the speaker or directly as a first-person reference. Another example of such a development is the personal use of \textit{a gente}  ‘the people’ that has developed into an impersonalization strategy and, subsequently, into a first-person plural pronoun, in particular in Brazilian Portuguese (e.g., \citealt{Lopes2004, Zilles2005}), or the appearance of “new” address forms based on noun phrases in European Portuguese (see \sectref{sec:intro:4}).\largerpage



This volume is structured into three thematic blocks addressing the beforementioned topics: Part~\ref{part:1}, \textit{Variable expression of subjects and objects} contains the contributions from Ryan Bessett examining first-person singular (\GlossMarkup{1SG}) subject expression in two varieties of Spanish spoken in the United States, Esther Brown and Javier Rivas studying the expression of first- and third-person indirect object pronouns in Spanish and Galician, and Kimberly Geeslin, Tom Goebel-Mahrle, Jingyi Guo, and Bret Linford, whose study focuses on the role of perseveration in the acquisition of variable subject expression in Spanish as a second language. 



Part~\ref{part:2}, \textit{Between personal and impersonal},  contains the papers by Eduardo Amaral and Wiltrud Mihatsch who study emerging impersonal constructions with the lexeme \textit{pessoa}  ‘person’ in Portuguese, Yoselin Henriques Pestana's paper on personal uses of impersonal constructions in rural Madeiran Portuguese, Juanito Ornelas de Avelar’s paper on the pronominal uses of the word \textit{geral}  ‘general’ in Brazilian Portuguese, the contribution by Rafael Orozco, Luz Marcela Hurtado and Marianne Dieck who study the personal uses of the impersonal pronoun \textit{uno}  ‘one’ in Colombian Spanish, and the paper by Émeline Pierre \& Barbara de Cock on the use of object discourse participant pronouns in a third person plural impersonal construction. 



Part~\ref{part:3}, \textit{Referring to the addressee},  contains Aldina Marques and Isabel Margarida Duarte’s study of the address form \textit{senhor}  and its variants in European Portuguese and Neus Nogué Serrano and Lluís Payrató’s paper on changing address forms in Catalan parliamentary discourse. 


With this selection of papers, our intention is not only to shed light on the different ways of referring to discourse participants in Ibero-Romance languages, but also to open new perspectives to phenomena related to person reference and inspire future research on reference and discourse. This introductory chapter provides an overview about the state-of-the-art of the three thematic blocks and outlines the research questions addressed in the contributions of the current volume. 


\section{Variable expression of subjects and objects}

The variable expression of subjects has received considerable attention in generative syntax as well as functional and constructional approaches. While in generative studies of the so-called \textit{pro}-drop parameter (\citealt{Chomsky1981,Chomsky1982, Rizzi1982, Solà1992, Barbosa1995}), the focus has been on the formal properties of a language system that make null pronouns possible, functional approaches have focused on the conditions under which subjects are expressed in actual language use (see, e.g., \citealt{Enríquez1984, Bentivoglio1987, Cameron1993, Morales1997, OtheguyEtAl2007, TravisCacoullos2012, Posio2011,Posio2018}, among many others). Thus, the following represent some of the leading questions in studies approaching variable subject expression: 

\begin{enumerate}
    \item Which factors determine subject pronoun use and what is the internal ranking of these factors? 
    \item How does usage frequency (e.g., of verbs and verb forms) relate to the expression of pronominal subjects? 
    \item Is subject expression governed by same or different factors across languages and language varieties? 
\end{enumerate}


With respect to the first question, there is a vast body of research examining subject expression from various theoretical and methodological perspectives. The following non-exhaustive list presents the most prominent factors affecting subject expression that have been singled out in previous studies:

\begin{enumerate}
    \item (morphological) ambiguity (e.g., \citealt{Hochberg1986})
  emphasis and contrast (e.g., \citealt{Rigau1989, Luján1999, Mayol2010})
    \item referential and/or topic continuity (e.g., \citealt{Bentivoglio1983})
    \item switch reference (e.g., \citealt{Cameron1993, Silva-Corvalán2001, TravisCacoullos2012})
    \item person/number; tense, aspect, mood; verb type (e.g., \citealt{Enríquez1984, Morales1997, Silva-Corvalán2001, OtheguyEtAl2007, OrozcoGuy2008, Posio2011})
\end{enumerate}


A common trend observed in most studies is that topic and referential continuity favor null subjects, while shifting topics and switch reference favor the presence of overt subject pronouns. With respect to morphological ambiguity, one hypothesis that has been discussed is whether syncretic verb forms trigger higher subject expression rates than non-syncretic ones, i.e., the so-called “functional hypothesis” \citep{Hochberg1986}. However, the results of different studies are not homogenous, some studies supporting and others not providing direct evidence for the relevance of syncretism between first, second and third person (see \citealt{Posio2018} for discussion). With respect to emphasis and contrast, several studies claim that strong subject pronouns encode different flavors of these notions. Thus, \citet{Rigau1989} makes a distinction between weakly and strongly emphasized strong pronouns and \citet{Mayol2010} claims that strong pronouns encode different types of contrastive topics. 



Tense, aspect and mood marking on verbs is closely related to the factor of morphological ambiguity: certain verb forms have ambiguous person-marking morphology in first-person singular (\GlossMarkup{1SG}) and third-person singular (\GlossMarkup{3SG}), e.g., in Catalan, European Portuguese, and Spanish. Thus, imperfect, conditional, and subjunctive paradigms display a syncretism between \GlossMarkup{1SG} and \GlossMarkup{3SG} in these languages. Furthermore, phonological processes lead to a higher number of ambiguous verb endings in some varieties of the Ibero-Romance languages, which in turn might influence subject expression rates. \citet{Hochberg1986} investigated how word-final /s/ deletion correlated with subject expression in Puerto Rico Spanish, observing that ambiguity between second-person singular (\GlossMarkup{2SG}) and \GlossMarkup{3SG} verb forms, e.g., in the indicative present, or between \GlossMarkup{1SG}, \GlossMarkup{2SG}, and \GlossMarkup{3SG}, e.g., in indicative imperfect correlated with higher subject expression frequencies. More recently, \citet{HerbeckForthcoming} has observed that some Valencian Catalan varieties display high frequency of \GlossMarkup{1SG} subject pronouns in the present perfect, in which the \GlossMarkup{1SG} and \GlossMarkup{3SG} auxiliary has the same form. However, the relevance of the functional hypothesis is not confirmed by some studies with different type of data (\citealt{Ranson1991, Cameron1992,Cameron1993}, cf. \citealt[290]{Posio2018}). \citet{Silva-Corvalán2001}, on the contrary, argues that the discourse function of TAM markings (i.e., event foregrounding vs. backgrounding), rather than their surface ambiguity, is the decisive factor explaining variable subject expression in different tenses and moods. However, Brazilian Portuguese provides strong evidence in favor of the role of morphological syncretism. Brazilian Portuguese has developed from a consistent to a partial \textit{pro}-drop language (in the sense of \citealt{Holmberg2005}; see \citealt{Kato1999, Barbosa2009}; collected works in \citealt{KatoNegrão2000}). Interestingly, the loss of some properties of a (consistent) null subject language and syncretism in verb morphology has consequences, not only for the expression of referential subjects, but also for the encoding of impersonal (see, e.g., \citealt{Carvalho2019} for impersonal \GlossMarkup{3SG} null subjects) and personal subjects (e.g. \textit{a gente}  referring to first-person plural [\GlossMarkup{1PL}]; see \citealt{Lopes2004}).



In the case of person/number, one important question is whether subject expression behaves similarly in different grammatical persons and numbers. Thus, the difference between deictic first and second person and discourse anaphoric third person might affect the relevance or weight of factors such as topic continuity, switch reference and (morphological) ambiguity. In fact, the study of only one grammatical person at a time has turned out to be a very fruitful approach (see  \citealt{LapidusOtheguy2005, TravisTorresCacoullos2012, LapidusShin2014}, \citetv{chapters/02-Bessett} for Spanish; \citealt{HerbeckForthcoming} for Valencian Catalan). This issue takes us to one of the main issues of the present volume: the question of what the factors governing subject expression are in the case of reference to discourse participants and whether they differ from those that have been observed to hold for discourse anaphoric persons. In fact, for devices expressing reference to the speaker, it has been argued that, rather than continuity, subjectivity is a major factor for the use of an implicit or explicit subject pronoun (see, e.g., \citealt{Enríquez1984, AijónOlivaSerrano2010, Posio2011,Posio2018, Hennemann2016, Herbeck2021}).



However, the influence of grammatical person cannot be considered in isolation but must be examined together with verb type and specific verb forms, taking into account that these might not behave uniformly in different varieties and languages. This brings us to the points (ii) and (iii) above: Subject expression has been argued to show considerable variation if specific verb forms (and not only verb types) are considered. With respect to verb type and semantics, it has frequently been observed that verbs of cognition have particularly high \GlossMarkup{1SG} subject expression rates in Spanish (e.g., \citealt{Enríquez1984, Morales1997, Posio2011}). However, the issue is complex because the group of verbs of mental activity is not homogenous, some verbs of cognition favoring \GlossMarkup{1SG} subject expression (e.g., [\textit{yo}] \textit{creo} ‘I think’), others disfavoring it in Peninsular Spanish (e.g. [\textit{yo}] \textit{sé}  ‘I know’; \citealt{Herbeck2021}). Furthermore, \citet{AijónOlivaSerrano2010} argue that [\textit{yo}] \textit{creo}  has higher subject expression rates when used as a verb expressing the personal opinion of the speaker and lower rates if it has a mere epistemic function. This mirrors \citegen{Enríquez1984} classification of verbs of cognition into verbs expressing a mental state/activity and those expressing opinions: the latter class has the highest \GlossMarkup{1SG} subject expression rates in her study. The question thus arises whether verb semantics, the function of a verb in context (expression of epistemicity, evidentiality, subjectivity, opinions, etc.), the concrete verb form, or a combination of all these factors affects the expression of pronouns referring to discourse participants. 


The question of semantic factors influencing expression of pronouns is especially interesting if a cross-linguistic perspective is integrated into the picture. As \citet{Posio2018} notes, there is considerable variation with respect to which verbs and verb forms trigger high and low subject pronoun rates in different Ibero-Romance languages and varieties. For example, with \textit{decir}  ‘say’, subject expression has been observed to have a low frequency in Peninsular Spanish (e.g., \citealt{Posio2013,Posio2014}), but a high frequency in Colombian Spanish \citep[25]{Orozco2015}. Furthermore, \citet{Posio2013,Posio2014} observes in his study of Peninsular Spanish and European Portuguese that \GlossMarkup{1SG} subject expression is particularly high with the verb form  \textit{creo}  ‘I think’   in the former language, while the verb form \textit{digo}  ‘I say’ has low to average expression rates. In European Portuguese, on the contrary, the equivalent form \textit{digo}  ‘I say’ has high subject expression rates, but the verb form \textit{acho}  ‘I think’ is associated with low rates. In a similar vein, \citet{HerbeckForthcoming} observes that \GlossMarkup{1SG} subject pronouns are frequently expressed with the verb \textit{dir}  ‘say’ in Valencian Catalan varieties, while \textit{creure}  ‘think’ has only average rates of \GlossMarkup{1SG} subject expression, unlike in Peninsular Spanish. \citet{Posio2013,Posio2014,Posio2015} argues that certain verb forms and their subject pronouns have become formulaic sequences due to high frequency of use in determined discourse context and correlated routinization. The degree of fixation of a \GlossMarkup{1SG} or \GlossMarkup{2SG} subject pronoun\,+\,verb sequence might differ across languages and varieties, the sequence [\textit{yo}] \textit{creo}  showing a higher degree of grammaticalization and conventionalization in Spanish than the corresponding sequence in EP. This raises the question of the relation between frequency, the expression of reference to discourse participants, and verb forms encoding epistemic stance, personal opinions and speech acts – notions that are particularly relevant in speaker/hearer interactions.

Lastly, while subject expression (as null or strong pronouns) is a topic that has received considerable attention from both theoretic and data-based perspectives, the question of whether variable object pronoun expression as weak or strong pronouns is governed by the same or different factors as subject pronoun expression still awaits further research. This issue is particularly relevant for dative experiencers that have been argued to display several subject properties (see \citealt{Masulló1993}). Thus, with psych-verbs selecting a dative experiencer, the question is whether the same factors govern expression of strong speaker/addressee pronouns vs. clitic ones as with nominative subjects and with dative objects (see \citetv{chapters/03-Brown-Rivas}  for related discussion). The issue is interesting in the light of functional approaches positing that certain scales, such as topicality, animacy, and grammatical function (see \citealt{Givón1983} and related work) affect the choice of referential device. Thus, it has been argued that topic continuity favors subject omission (\citealt{Givón1983, Bentivoglio1983}, among many others). Topicality is in turn favored with animate referents (if compared with inanimates) and with nominative subjects if compared with dative and accusative NPs (see \citealt{Givón1983}). While dative experiencers are highly animate and, thus, predicted to be high on the scale of topicality, they display case marking typically associated with lower topicality than nominative subjects. 



The thematic part \textit{Variable expression of subjects and objects} of the present volume deals with several questions discussed above. First, the paper by \textcitetv{chapters/03-Brown-Rivas} addresses the issue whether object expression is affected by same or different factors as subject expression. They analyze which factors influence weak (e.g., \textit{me, te}) and strong (e.g., \textit{a mí, a ti}) indirect object pronoun expression in Spanish and Galician, offering insights into two closely related Ibero-Romance languages. The authors show that while expressing the object as a strong pronoun is more frequent in Spanish than in Galician, it is affected by the same syntactic, discourse, and interactional factors in the two languages. Thus, their study indicates that expression of \GlossMarkup{1SG} object pronouns is favored in utterance initial position, in constructions with \textit{gustar}-type verbs, when primed by previous mention, and in non-continuous contexts.



The next paper of this part by \textcitetv{chapters/02-Bessett} asks whether reference to the speaker presents differences between two geographically close varieties of Spanish. Bessett examines \GlossMarkup{1SG} subject expression in the Spanish spoken in two communities located on the border between Mexico and the US: Southern Arizona and Southeast Texas. The quantitative results of the study show that \GlossMarkup{1SG} subject expression occurs with similar rates in the two samples representing the two varieties and is conditioned by similar factors (switch reference; clause type; tense, mood, and aspect; and whether the verb is reflexive or not). In addition to contributing to the general discussion on factors affecting subject expression, the paper provides new data from these borderland varieties of Spanish.



Lastly, while the role of perseveration and priming has been studied in the context of subject expression (see, e.g., \citealt{TravisCacoullos2018}), its role in the acquisition of subject expression by L2 speakers of Spanish still awaits wider research. The paper by \textcitetv{chapters/04-Geeslin-others} investigates the acquisition of subject expression in Spanish as a second language by native speakers of English. In particular, their study focuses on how far perseveration, i.e., the use of the same subject form that has been used in the context preceding the target form, influences subject expression across several proficiency levels. This factor is examined in interaction with other linguistic factors (form of the prime – null, overt, or lexical noun phrase – switch reference, gender continuity, TAM continuity) in a written contextualized preference task. Focusing on third person subjects, the study sheds new light on the role that the form of the previous mention of the referent of a (null or overt) subject plays in conditioning subject form in second language acquisition.


\section{Between personal and impersonal}

During the last decade there has been an increasing interest towards constructions expressing impersonal, generic, or vague reference to human participants within functional-typological linguistics (see, e.g., the papers in \citealt{MalchukovSiewierska2011}). The notion of impersonality is used in various ways in linguistics. Traditionally, Ibero-Romance linguistics has distinguished between syntactic and semantic impersonality (\citealt{FernándezSoriano1999Construcciones}: 1723). Syntactically impersonal constructions can be defined as those where the verb does not agree with a subject or the overt subject is lacking completely, as is the case with meteorological verbs (e.g., Spanish \textit{llueve}  ‘it rains’) or existential verbs (e.g., Portuguese \textit{há ovos no frigorífico}  ‘there are eggs in the fridge’). Impersonal constructions with expletive or “dummy” subjects are rare in Ibero-Romance varieties, although they are found in varieties of Portuguese \citep{Carrilho2005} and Dominican Spanish \citep{Toribio2000}. 



Semantically impersonal constructions can be defined as those where the subject argument is reduced in referentiality: there is a subject in the verbal construction, but it is either non-canonical in the sense that it does not coincide with the agent of the depicted action (e.g., Spanish “passive” \textit{se}-constructions like \textit{se venden coches}  ‘cars are sold’ where \textit{coches}  ‘cars’ is formally the subject) or it is referentially vague in the sense that it does not point at any particular participant (e.g., Spanish \textit{uno}  ‘one’, as in \textit{uno no sabe qué hacer}  ‘one doesn’t know what to do’). In the latter case, typical sources of impersonality are personal pronouns used for non-canonical reference: for instance, \GlossMarkup{2SG} pronouns and verb forms are used in many languages to refer vaguely to ‘anyone in general or in a given situation’, and the third person plural is often used to refer vaguely to ‘people in general’. A further distinction can be made between generic and episodic readings of impersonal subjects: for instance, impersonal second person singulars can, in most cases, only be used in generic sentences that are not anchored into any specific point in time or place, while third person plurals are found in both generic and episodic sentences, i.e., sentences referring to actions taking place in a given time and space. In formal approaches, a similar distinction is made between “generic” and “arbitrary” pronouns, i.e., those referring to ‘anyone’.



The subtype of impersonal constructions that is of interest to the current volume has been referred to as reference impersonals or R-impersonals (\citealt{SiewierskaPapastathi2011, MalchukovOgawa2011, MalchukovSiewierska2011}) or human impersonals (\citealt{CabredoHofherr2008}), highlighting the fact that the source of impersonality is a reduction in the referentiality of the subject argument in these constructions. For a more elaborate typology of human impersonals, see \citet{GastvanderAuwera2013}. Crucially, the subject (or agent) of the depicted action is always human (or at least construed as human), and non-animate or non-human animate participants are not acceptable without very specific context (e.g., Portuguese ?/*\textit{ladra-se muito à noite}  ‘one barks a lot at night’, where the intended referent would be the neighborhood dogs). What makes these constructions interesting for the current volume is the two-way relationship between personal pronouns and other referential devices used to refer to discourse participants: not only do deictic pronouns like the second-person singular acquire generic and impersonal uses, but originally impersonal forms like the \textit{se}-constructions or Portuguese \textit{a gente}  and \textit{a pessoa}  (see \citetv{chapters/06-Henriques}, and \citetv{chapters/05-Amaral-Mihatsch}) also develop uses where their primary referential range is the speaker or a group including the speaker.


The reduction of referentiality found in human impersonal constructions does not mean that their reference is completely arbitrary: rather, the choice of the human impersonal construction, as well as contextual elements such as locative expressions, typically restrict the scope of possible referents of the constructions (i.e., their referential range; \citealt{PosioVilkuna2013}). Thus, human impersonals deriving from personal pronouns such as \GlossMarkup{2SG} or the third person plural (\GlossMarkup{3PL}) typically maintain part of their “original” referential properties. For instance, \GlossMarkup{2SG} used impersonally implies that the intended referent is singular and may coincide with the speaker or the addressee, whereas impersonal \GlossMarkup{3PL}, at least in most cases, exclude both speech act persons from their referential range. 



Since Ibero-Romance languages display a wide range of human impersonal constructions – including reflexive-based \textit{se}-constructions, \textit{one}-impersonals like Spanish \textit{uno},  pronoun-based like \GlossMarkup{2SG} and \GlossMarkup{3PL}, and noun-based like Portuguese \textit{a pessoa} – the constructions are specialized to express different kinds of referential range. For instance, non-anaphoric \GlossMarkup{3PL} with no expressed subjects – the topic of \citeauthor{chapters/11-Pierre-De-Cock}’s paper (\citeyear{chapters/11-Pierre-De-Cock} [this volume]) – generally expresses a referential range that excludes the speaker and the addressee. However, \GlossMarkup{1SG} and \GlossMarkup{2SG} pronouns can occur in these constructions as direct or indirect objects. \citeauthor{chapters/11-Pierre-De-Cock} argue that the referential vagueness of the subject of these constructions makes the object arguments more prominent. 


\begin{sloppypar}
In the European language area, there is a widespread construction type known as \textit{man}-impersonals, i.e., human impersonal pronouns derived from the word meaning ‘man’ that are found in most Germanic languages and French. As pointed out by  \citet{GiacaloneRamatSansò2007} and \citet{SiewierskaPapastathi2011}, these constructions are found in languages with obligatory subject expression, whereas so-called null subject or \textit{pro}-drop languages are less prone to develop such constructions. Thus, \textit{man}-impersonals are not found in present-day Spanish, Portuguese, or Italian, although they have existed in earlier stages of these languages (\citealt{GiacaloneRamatSansò2007}). Portuguese is an interesting exception to this typological tendency, as it does present a range of constructions based on the noun \textit{pessoa}  ‘person’ that have developed uses akin to \textit{man}-impersonals (\citealt{DuarteMarques2014, Posio2017,Posio2021, AmaralMihatsch2019}, \citeyear{chapters/05-Amaral-Mihatsch} [this volume]). The Portuguese \textit{a gente}  construction deriving from the noun phrase meaning ‘the people’ is another example of a human impersonal construction similar to \textit{man}- constructions. This construction has now become the predominant choice of referential device used for the first-person plural in varieties of Portuguese (in particular Brazilian Portuguese), while in European Portuguese it remains ambiguous between personal and impersonal readings \citep{Posio2012}. 
\end{sloppypar}


The development of noun-based constructions like \textit{a gente}  and the \textit{pessoa}  constructions has been previously attributed to the high number of expressed pronominal subjects in Portuguese (in particular Brazilian, but also European variety; \citealt{Posio2021}). Interestingly, Portuguese seems to be particularly disposed to develop “new” impersonal constructions and referential devices from noun phrase constructions.



The thematic part \textit{between impersonal and personal}  addresses the above-men\-tioned research questions from different angles. \textcitetv{chapters/05-Amaral-Mihatsch}, \textcitetv{chapters/06-Henriques}, and \textcitetv{chapters/07-Avelar} examine dialectal and/or informal data from different varieties of Portuguese to study the emergence of “new” impersonal pronouns. Another interesting question is why some impersonal constructions tend to acquire personal uses in discourse: this question is addressed in the papers by \textcitetv{chapters/06-Henriques} and \textcitetv{chapters/08-Orozco-others}. A case in point about the emerging impersonal pronouns in Brazilian Portuguese is the \textit{geral} construction discussed by Ornelas de Avelar, where the adjective meaning ‘general’ has been repurposed as a human impersonal subject. \textcitetv{chapters/06-Henriques} explores the \textit{a gente}  construction which is used in Madeiran Portuguese together with the morpheme \textit{se}  that also expresses impersonality in what Henriques calls hybrid constructions. These constructions, Henriques argues, are an example of impersonal constructions developing personal uses. A similar development can be observed with the \textit{pessoa}  constructions studied by Amaral \& Mihatsch that are often used to refer to the speaker, although simultaneously expressing generalizations or mitigation. \textcitetv{chapters/08-Orozco-others} also look into a development from impersonal to personal, but in Colombian Spanish, where the human impersonal pronoun \textit{uno}  has developed personal uses to the extent that it can be considered a variant of the \GlossMarkup{1SG} pronoun \textit{yo} ‘I’. 



Lastly, the paper by \textcitetv{chapters/11-Pierre-De-Cock} investigates a configuration where an impersonal third-person plural subject co-occurs with a referential object pronoun. Their paper investigates the use of object discourse participant pronouns in impersonal third person plural constructions (e.g. \textit{me han criticado}  ‘they have criticized me’). Given that third-person plurals have been analyzed as an agent-defocusing mechanism, the authors examine to what extent the higher referentiality of the first or second person object pronoun, contrasting with the lower referentiality of the subject, affects the conceptualization of the whole construction. The authors offer a quantitative and qualitative analysis of different types of corpus data by means of which they investigate different factors influencing the use of the examined construction, such as topic continuity, verb type, and different types of register.


\section{Reference to the addressee}\label{sec:intro:4}

The variable use of address forms is a widely studied topic in Ibero-Romance linguistics (see, e.g., \citealt{HummelLaslop2010} for Spanish), and different geographical and social varieties of Ibero-Romance languages display a wide range of address systems ranging from only one address pronoun referring to singular addressees, e.g., \textit{ustedeo}  in different regions of Central America (see \citealt{Moser2006, QuesadaPacheco2010}), to tripartite pronominal systems, e.g., the use of \textit{tú}, \textit{vos} and \textit{usted}  in Uruguayan Spanish (\citealt{Steffen2010}), or the use of \textit{tu}, \textit{vós},  and \textit{vostè}  in Catalan (see \citealt{Robinson1980, Todolí2006, Nogué2022}, \citealt[8.2.2]{GIEC2022}), and complex nominal and pronominal systems comprising address pronouns as well as the use of proper names and honorific nominal forms of address, as is the case in European Portuguese (see, e.g., \citealt{Allen2019}).



Since \citegen{BrownGilman1960} seminal work, the use of different forms of address has been related to the notions of power and solidarity that hold between two (or more) interlocutors to different degrees. For example, in the \GlossMarkup{2SG} pronoun \textit{tú}  in Peninsular Spanish expresses intimacy and solidarity in reciprocal uses in which the level of power is equal, while in non-reciprocal uses it may express condescendence (cf. \citealt[622]{Uber2016}). \citet{BrownLevinson1972} and \citet{García1992} use the notions of positive and negative politeness to account for the choice of address forms. Thus, use of informal address forms like \textit{tú}  in Peninsular Spanish can be considered a form of positive politeness, showing affection and approval, while use of formal address forms such as Peninsular Spanish \textit{usted} correlates with negative politeness, i.e., showing respect and keeping distance (cf. \citealt[622]{Uber2016}). 



\textrm{On a morpho-syntactic level, address forms show interesting patterns of agreement mismatches: the pronoun \textit{usted} and its plural \textit{ustedes} in Spanish are mor\-pho-syn\-tac\-tic\-ally third person forms used to refer to the addressee. However, there are varieties of Spanish in which \GlossMarkup{3PL} \textit{ustedes}  is used with second person plural (\GlossMarkup{2PL}) inflection (\citealt[254]{DeJongeNieuwenhuijsen2012}). Furthermore, in several varieties that use the \GlossMarkup{2SG} pronoun \textit{vos}, different systems of verb inflection can be found: the use of \textit{vos}  with the \GlossMarkup{2SG} inflection (called “mixed pronominal \textit{voseo}”, e.g., \textit{vos no puedes}  ‘you cannot’) or the use of \textit{tú} with the verbal inflection of \textit{vos}  (“mixed verbal \textit{voseo}”, e.g,. \textit{tú no podés}  ‘you cannot’; cf. \citealt[256--257]{DeJongeNieuwenhuijsen2012}). In some systems that make use of \textit{tú}}, \textit{vos}, as well as \textit{usted}, a functional partition can be observed: for example, in Uruguay, \textit{vos}  is used as an intimate and confidential address form and \textit{tú} as an informal but less intimate address form (cf. \cites[258]{DeJongeNieuwenhuijsen2012}[329]{HualdeTravis2010}). Furthermore, historical, political and social factors may intervene in the choice of address forms in different varieties (as, e.g., in Nicaragua; see \citealt{Lipski1994}: 159ff).


In several varieties of Catalan, a tripartite address system can also be found: \textit{tu}  which is morpho-syntactically a \GlossMarkup{2SG} pronoun, refers to the addressee and it agrees with a \GlossMarkup{2SG} verb (\textit{tu ho saps}  ‘(you) know it’). Use of this form indicates a degree of intimacy (cf. \citealt{Robinson1980}) between the interlocutors, similarly to \textit{tú} in Spanish. The form \textit{vostè}  in singular and \textit{vostès} in plural are interpreted referentially as second person, referring to the addressee(s), but they are morpho-syntactically third person, agreeing with a third person verb form. The address pronoun can also be omitted (e.g., (\textit{Vostè}) \textit{ho sap millor}  ‘You know it best’, \citealt[8.2.2]{GIEC2022}). This pattern stems from the fact that \textit{vostè}  originates from the noun phrase \textit{vostra mercè}  ‘your grace’ and thus behaves like a third person noun phrase with respect to agreement patterns. Use of these forms is related to a lower degree of intimacy between the interlocutors \citep{Robinson1980}, to politeness and a certain degree of social distancing and formality of the speech act \citep[8.2.2]{GIEC2022}). Apart from these second person and third person address forms, there is a third form, \textit{vós}, which is used to refer to a 2SG addressee but triggers 2PL agreement on the verb, e.g., \textit{Què en penseu} (\textit{vós})? ‘What do (you [\GlossMarkup{2SG}]) think [\GlossMarkup{2PL}] of it?’ (\citealt[8.2.2]{GIEC2022}). As mentioned in the \citet[8.2.2]{GIEC2022}, traditionally the use of \textit{vós} indicated “cordial and friendly respect” [our translation] and was used to address elderly interlocutors. In colloquial speech, the use of \textit{vós}  is decreasing, but it is very common in juridical and administrative language (\citealt[8.2.2]{GIEC2022}; see also \citealt{Nogué2022} for discussion). This form is associated with a lower degree of distancing than the form \textit{vostè}. Thus, in the varieties in which the tripartite system is used, there seems to be a functional partitioning not only between [+/\textminus{}intimate] forms, but furthermore, between the two [\textminus{}intimate] forms \textit{vós}  and \textit{vostè}. However, some varieties have abandoned the use of a tripartite system, as for example, in Nothern Catalonia, where \textit{vostè}  is not used anymore, in the Comunitat Valenciana, where \textit{vós} has fallen into disuse; in other regions of Catalonia, the tripartite system is characterized as unstable (cf. \citealt{Robinson1980, Nogué2022}).


Within personal pronoun paradigms, address forms are most open to variation and the introduction of new forms deriving from nominal sources. Some well-known cases of such “new” pronouns are the development of \textit{vuestra merced > usted} ‘you-singular/formal’ in Spanish as well as the creation of plural forms like \textit{vos}  ‘you-PL’\,+\,\textit{otros}  ‘others’  \textit{> vosotros}  ‘you-plural/informal’ through univerbation (\citealt{Lapesa1981} [1942]: 259, 392). As is the case with impersonal constructions, Portuguese is particularly prone to the apparition of “new” address forms based on noun phrases such as \textit{o senhor}  ‘the sir’, \textit{a doutora}  ‘the doctor’ or combinations of nominal forms of address and proper names (e.g., \textit{a doutora Maria} ‘the doctor Maria’). The complexity of the European Portuguese address system is described in terms of a tripartite categorization into pronominal, nominal and verbal address forms (\citealt{Cintra1972}, \citetv{chapters/09-Marques-Duarte}). These forms are used to encode different levels of proximity, familiarity and deference that are difficult to formalize or describe in terms of T/V distinctions (\citealt{BrownGilman1960}), as the interpretation of each form depends heavily on sociolinguistic and socio-situational factors. \citet{Carreira2005} considers that the European Portuguese address system provides various ways to encode indirectness and negative politeness. It is also interesting to note that the avoidance of nominal and pronominal address forms is a common strategy: the use of \GlossMarkup{3SG} verb forms without expressed subjects is a way to avoid the choice of address form and can be considered a “zero degree of politeness” \citep[313]{Carreira2005}.


Research on address forms has traditionally focused on accounting for variation between competing address forms in different discourse types and between different types of interlocutors, as well as explaining the diachronic development of address forms. The variation is affected by sociolinguistic and sociosituational factors such as age, sex, profession, social rank, personal relation, time of acquaintance, place and type of conversation, among others (cf. \citealt[627]{Uber2016} for an overview and references). 

The third thematic part of the current volume, \textit{Reference to the addressee},  presents two studies focusing on Catalan and European Portuguese address forms from both synchronic and diachronic perspectives. The paper by \textcitetv{chapters/10-Nogue-Serrano-Payrato} discusses interesting data and sheds new light on the use and role of different address forms in Catalan parliamentary debates. The authors examine reference to the participants in comparing two time periods: from 1932 to 1938, and from 1980 to 2020, using qualitative as well as quantitative methods. The authors show that the study of reference to discourse participants in parliamentary debates needs to go beyond the study of first and second-person forms and include several third-person forms as well. Furthermore, they observe a development of address forms, specifically vocatives, from more complex to less complex forms. Likewise, they detect an increase of the use of \textit{vostè} and a loss of \textit{vós} and \textit{vostra senyoria}  and other third-person forms. Lastly, on a general level, the study shows a move towards more informality on a continuum. The chapter by \textcitetv{chapters/09-Marques-Duarte} examines the address form \textit{o senhor}  in a wide variety of data from an interactional perspective. Their analysis also covers contracted forms like \textit{sotor},  deriving from the contraction of the complex address form \textit{senhor doutor} ‘mister doctor’ and other innovations. Similarly to the \textit{pessoa}  constructions analyzed by \textcitetv{chapters/05-Amaral-Mihatsch}, the address forms based on \textit{senhor}  form a network of partially variable constructions with different grades of productivity and variability, occupying a position between noun phrases and pronouns in the paradigm of referential devices of European Portuguese. 

\printbibliography[heading=subbibliography]
\end{document}
