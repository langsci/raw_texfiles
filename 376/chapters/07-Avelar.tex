\documentclass[output=paper]{langscibook}
\ChapterDOI{10.5281/zenodo.8124498}
\author{Juanito {Ornelas de Avelar}\orcid{}\affiliation{University of Campinas}}
\title {On pronominal uses of \textit{geral} in Brazilian Portuguese}
\abstract{This paper analyzes the impersonal use of {\textit{geral}} ‘general’ in Brazilian Portuguese, in the light of investigations dealing with impersonalization strategies in the generative literature. I will show that {\textit{geral}} behaves as a φ-fea\-tureless impersonal pronoun with regard to agreement patterns and to generic/arbitrary interpretation, but as a pronoun with φ-fea\-tures if we take its syntactic distribution into consideration. Despite this incongruity, I will argue that {\textit{geral}} must be analyzed as an item that is devoid of φ-fea\-tures, similarly to {\textit{man}} {in Swedish,} {\textit{si}} in Italian and {\textit{on}} in French, according to \citegen{Egerland2003} proposal. The analysis provides evidence in favor of the hypothesis that the distribution of impersonal pronouns in different sentential positions is better captured in terms of case marking instead of syntactic function \citep{Fenger2018}. I will also show that {\textit{geral}} can be used as a first-person plural pronoun, which seems to depend on strictly pragmatic factors, as a result of the lack of φ-fea\-tures.}
\IfFileExists{../localcommands.tex}{
  \addbibresource{../localbibliography.bib}
  % add all extra packages you need to load to this file

\usepackage{tabularx,multicol}
\usepackage{url}
\urlstyle{same}

\usepackage{listings}
\lstset{basicstyle=\ttfamily,tabsize=2,breaklines=true}

\usepackage{langsci-basic}
\usepackage{langsci-optional}
\usepackage{langsci-lgr}
\usepackage{langsci-osl}
% \usepackage{./langsci/styles/langsci-lgr}
% \usepackage{./langsci/styles/langsci-osl}
% \usepackage{langsci-gb4e}

\usepackage{tikz}
\usetikzlibrary{patterns,calc}
\pgfdeclarepatternformonly{south east lines}{\pgfqpoint{-0pt}{-0pt}}{\pgfqpoint{3pt}{3pt}}{\pgfqpoint{3pt}{3pt}}{
    \pgfsetlinewidth{0.6pt}
    \pgfpathmoveto{\pgfqpoint{0pt}{3pt}}
    \pgfpathlineto{\pgfqpoint{3pt}{0pt}}
    \pgfpathmoveto{\pgfqpoint{.2pt}{-.2pt}}
    \pgfpathlineto{\pgfqpoint{-.2pt}{.2pt}}
    \pgfpathmoveto{\pgfqpoint{3.2pt}{2.8pt}}
    \pgfpathlineto{\pgfqpoint{2.8pt}{3.2pt}}
    \pgfusepath{stroke}}
    
\usepackage{stmaryrd}
\usepackage{wasysym}
\usepackage{multirow}
\usepackage{caption}
\usepackage{subcaption}
\usepackage{mathrsfs}
\usepackage{qtree}

\usepackage{linguex}


  %pminos do not split footnotes
% \interfootnotelinepenalty=10000 %Footnote in Laporte chapters has to be split SN


%\DeclareIndexNameFormat{default}{%
%\nameparts{#1}%
%\usebibmacro{index:name}%
%{\index[names]}%
%{\namepartfamily}%
%{\namepartgiveni}%
% {}% L1
% {}% L2
%{\namepartprefix}% generates spurious space L3
%{\namepartsuffix}% generates spurious space L4
%}

%  {\DeclareIndexNameFormat{default}{%
%     \usebibmacro{index:name}{\index[names]}{#1}{#3}{#5}{#7}}}

%\DeclareIndexNameFormat{default}{%
%  \usebibmacro{index:name}{\sindex[nom]}{#1}{#3}{#5}{#7}}

%\DeclareIndexNameFormat{default}{%
%  \usebibmacro{index:name}{\sindex[person]}{#1}{#3}{#5}{#7}}
%\DeclareIndexNameFormat{default}{%
%\nameparts{#1} \usebibmacro{index:name}{\sindex[person]]}{\namepartfamily}{‌​\namepartgiven}{\nam‌​epartprefix}{\namepa‌​rtsuffix}}

%\newcommand{\smiley}{:)}

%\renewbibmacro*{index:name}[5]{%
%\usebibmacro{index:entry}{#1}%
%{\iffieldundef{usera}{}{\thefield{usera}\actualoperator}\mkbibindexname{#2}{#3}{#4}{#5}}}

% \newcommand{\noop}[1]{}

%remove for final
%\overfullrule=1mm

\newcommand{\tobi}[2]}}
\renewcommand{\S}[1]{\tobi{#1}{\textsc{*}}}

% this volume references
% puts: [this volume]
% already defined: \citetv
%\newcommand{\citepv}[1]{(\citeauthor{#1} \citeyear*{#1} [this volume])}
\newcommand{\citealtv}[1]{\citeauthor{#1} \citeyear*{#1} [this volume]}

%parentheses around example number
\newcommand{\pref}[1]{(\ref{#1})}

% in-text examples

\newcommand{\lnex}[1]{\textit{#1}} %target lang word
\newcommand{\lnlit}[1]{(lit.: `#1')} %literal reading
\newcommand{\lnlat}[1]{(#1)} % latinization
\newcommand{\lntrans}[1]{`#1'} %translation
\newcommand{\lnexl}[2]%
{\lnex{#1}{} \lnlat{#2}} % ex with latinization
\newcommand{\lnexlat}[3]{\lnex{#1}{} \lnlat{#2}{} \lntrans{#3}} % ex with latinization and tranl.

%ch01
\newcommand{\co}[1]{\mbox{\textbf{#1}}}

%ch09

\newcommand{\cyrbulg}[1]{\begin{otherlanguage*}{bulgarian}#1\end{otherlanguage*}}


%ch10
\newcommand{\nlp}{{\small NLP}}
\newcommand{\mwe}{{\small MWE}}
\newcommand{\rae}{{\small RAE}}
\newcommand{\lvc}{{\small LVC}}
\newcommand{\pos}{{\small P}o{\small S}}
%\newcommand{\todo}[1]{ \textcolor{red}{#1} }

%\renewcommand{\labelenumi}{\theenumi}
%\ainamefmt{{vv}{ll}{, ff}{, jj}} % fullname

\newcommand{\biberror}[1]{{\color{red}#1}}

\newcommand{\osenovaitem}{--~} 
  %% hyphenation points for line breaks
%% Normally, automatic hyphenation in LaTeX is very good
%% If a word is mis-hyphenated, add it to this file
%%
%% add information to TeX file before \begin{document} with:
%% %% hyphenation points for line breaks
%% Normally, automatic hyphenation in LaTeX is very good
%% If a word is mis-hyphenated, add it to this file
%%
%% add information to TeX file before \begin{document} with:
%% %% hyphenation points for line breaks
%% Normally, automatic hyphenation in LaTeX is very good
%% If a word is mis-hyphenated, add it to this file
%%
%% add information to TeX file before \begin{document} with:
%% \include{localhyphenation}
\hyphenation{
    Beck-man
    Ngu-yen
    back-chan-nel
    back-chan-nels
    mo-not-o-nous
    ste-reo-typ-i-cal
}

\hyphenation{
    Beck-man
    Ngu-yen
    back-chan-nel
    back-chan-nels
    mo-not-o-nous
    ste-reo-typ-i-cal
}

\hyphenation{
    Beck-man
    Ngu-yen
    back-chan-nel
    back-chan-nels
    mo-not-o-nous
    ste-reo-typ-i-cal
}
 
  \togglepaper[7]%%chapternumber
}{}

\begin{document}
\AffiliationsWithoutIndexing{}
\maketitle 

\section{Introduction}

This study approaches the occurrence of \textit{geral} ‘general’ in Brazilian Portuguese in cases in which its use can be analyzed as an impersonalization strategy (i.e., denoting an indefinite human referent). \textit{Geral} is originally an adjective, but the examples in \REF{ex:avelar:1} below illustrate its use as a generic pronoun (corresponding to generic ‘you’ in English – see \ref{ex:avelar:1a}--\ref{ex:avelar:1b}) or as an arbitrary pronoun (corresponding to referentially undetermined ‘they’ in English – see \ref{ex:avelar:1c}--\ref{ex:avelar:1d}).\footnote{Please see Appendix~\ref{appendix:avelar} for links to the internet sources for all examples.}

\ea\label{ex:avelar:1} 
 \ea\label{ex:avelar:1a}
  \gll \ExHighlight{Geral} sabe que esporte gera renda.%\footnote{https://votolegal.com.br/em/brunoramos} %(accessed on 8 %\citealtSep 2018)}\\
  \\
         \textsc{geral} knows.\textsc{pres} that sport generates money\\
   \glt ‘Everyone knows that sports generate income.’

 \ex\label{ex:avelar:1b}
   \gll Bom mesmo era na {idade média} que \ExHighlight{geral} morria de sífilis e     ninguém tava {nem aí}. % \footnote{ \url{https://twitter.com/bbcbrasil/status/1020326263335079936} {(accessed on 8 \citealtSep 2018)}}\\
   \\
          good really was in-the {age middle} that \textsc{geral} died.\textsc{3sg} of syphilis and nobody was unconcerned %concerned??\\
          \\
   \glt ‘Things went really well in the Middle Ages, when everyone/you died from syphilis and nobody cared.’

 \ex\label{ex:avelar:1c} 
  \gll de noite fomos pro baile e depois \ExHighlight{geral} foi chegando e curtindo pra caramba % \footnote{ \url{https://vk.com/topic-73988417_37659003?offset=1380} {(accessed on 4 \citealtJul 2019)}}
 \\
         at night went.\textsc{past.1pl} to-the prom and after \textsc{geral} went   arriving and enjoying very much\\
  \glt‘in the evening we went to the dance and many people/some people kept on coming and enjoyed it a lot’

 \ex\label{ex:avelar:1d}
  \gll Olha aí a galera dos comes e bebes da festa que \ExHighlight{geral} ficou   alucinado! %\footnote{ \url{https://www.facebook.com/watch/?v=1722931431109561} {(accessed on 1 \citealtJul 2019)}}
  \\
         look there   the people   of-the foods   and drinks   of-the   party   that \textsc{geral} was crazy\\
  \glt ‘These are the guys who took care of the party’s foods and beverages that got many people/lots of people crazy!!!’
\z 
\z\largerpage

\begin{sloppypar}
Based on the Minimalist version of the Principles and Parameters Theory \citep{Chomsky1995}, this study aims at analyzing the behavior of \textit{geral} in the light of investigations dealing with impersonal pronouns in the generative literature. I approach in greater detail the works of \citet{Egerland2003} and \citet{Fenger2018}, who pursue the hypothesis that impersonal pronouns fall into two groups with regard to the presence or absence of the so-called φ-fea\-tures (which codify information related to grammatical categories such as gender, number and person): those that exhibit φ-fea\-tures and those that are devoid of φ-fea\-tures. The presence or absence of φ-fea\-tures has syntactic and pragmatic implications, since they determine the syntactic positions in which impersonal pronouns may occur and, at least in part, also condition their readings in a given context. Although \textit{geral} behaves in a way that is apparently inconsistent with what would be expected in terms of φ-fea\-tures, I will argue that it must be analyzed as an item that is devoid of such features. I shall further argue that, under specific pragmatic circumstances, \textit{geral} may gain a referentially definite reading, equivalent to the personal pronouns \textit{nós} and \textit{a gente} ‘we’, which refer to the first-person plural in Brazilian Portuguese.
\end{sloppypar}

This study will be carried out from an exclusively qualitative perspective. In addition to resorting to my own intuition (a common expedient in generative investigations), the analysis was based on data collected from webpages, informal writing on blogs, social networks, forums and commercial advertisement. The database currently consists of around 150 occurrences of \textit{geral} as an impersonalization strategy and has been gathered since 2018. In most cases, it is not possible to identify the author of each utterance nor their regional provenance, which at this point prevents me from engaging in more detailed sociolinguistic considerations regarding the relevant use of \textit{geral}.

This chapter is structured as follows: in \sectref{sec:avelar:2}, I present other uses of \textit{geral}, i.e., as an adjective, a noun and an adverb, in which this item also conveys the meanings of indetermination or intensification; in \sectref{sec:avelar:3}, I present the proposals of \citet{Egerland2003} and \citet{Fenger2018} for dividing impersonal pronouns into two large groups – those that exhibit φ-fea\-tures and those that are devoid of φ-fea\-tures; in \sectref{sec:avelar:4}, I analyze the behavior of \textit{geral} with regard to the expected properties of each of those types of pronouns, with the purpose of establishing in which group it belongs; in \sectref{sec:avelar:5}, I present some occurrences of \textit{geral} {in which it refers to the first-person plural; in \sectref{sec:avelar:6}, I summarize the conclusions of this study.}

\section{Adjectival, nominal and adverbial uses of \textit{geral}}\label{sec:avelar:2}

{The use of} \textit{geral} as an impersonalization strategy has been associated with the speech of younger individuals (below 30 years of age) living in urban areas in different regions of Brazil. Nevertheless, a systematic and in-depth sociolinguistic study on the distribution of \textit{geral} in Brazilian Portuguese remains to be conducted so as to enable a precise mapping of occurrences according to geographic, social and age criteria. Although it is commonly associated with spontaneous utterances of younger individuals, \textit{geral} may be observed in different age groups, including the author of this paper, who is currently in his forties. 

As an adjective, the \textit{Dicionário Houaiss da Língua Portuguesa} gives the following definitions of \textit{geral}: ‘that which applies to an array of cases or individuals’ \REF{ex:avelar:2a}, ‘that which embraces the totality or the majority of a group of persons or things’ \REF{ex:avelar:2b} and ‘universal, widespread’ \REF{ex:avelar:2c}, among others. The examples are presented in the online version of \textit{Dicionário Houaiss}.

\ea\label{ex:avelar:2}
 \ea\label{ex:avelar:2a}
    {lei {\ExHighlight{geral}}} \hspace{3,3cm} ‘general statute’\\
    {assembleia {\ExHighlight{geral}}} \hspace{2,1cm}‘general assembly’\\
    {busca o bem \ExHighlight{geral}} \hspace{1,88cm}‘(s)he seeks the general good’\\
  \ex\label{ex:avelar:2b} 
    {{tendências}} {\ExHighlight{{gerais}}}{ }\hspace{1,99cm}{‘overall tendencies’}\\
    {o desejo} {\ExHighlight{{geral}}}{ {da população}}\hspace{0.5cm} {‘the population’s overall desire’}\\
    {greve} {\ExHighlight{{geral}}}{ }\hspace{3cm}{‘general/mass strike’}
  \ex\label{ex:avelar:2c}
    o conceito   de animal   é   mais {\ExHighlight{{geral}}}{ {do que o de inseto.}}\\ {‘the concept   of animal   is   more general   than that of insect’}
\z 
\z


The \textit{Dicionário} also has entries for this item as a noun, with definitions such as ‘the greater number; the main part; generality, majority’ \REF{ex:avelar:3a}, ‘that which is ordinary, usual; common’ \REF{ex:avelar:3b} and ‘a part of the stadium, without benches nor roofing, from which a game or show can be watched standing, at the same level as the arena; the audience in such places’ \REF{ex:avelar:3c}. In the first two cases, \textit{geral} is a masculine noun, whereas it is a feminine noun in the third entry.

\ea\label{ex:avelar:3} 
 \ea\label{ex:avelar:3a}
  \gll o {\ExHighlight geral} da população não {se alimenta}  adequadamente\\
         {the.\textsc{masc} general of-the population not eat properly}\\
  \glt ‘the majority of the population does not eat properly’

 \ex\label{ex:avelar:3b} 
  \gll o \ExHighlight{geral} é a casa possuir duas entradas\\
         the.\textsc{masc} general   is the house   to-possess   two   entrances\\
  \glt ‘in most cases, a house has two entrances’
 \ex\label{ex:avelar:3c} 
  \gll a {\ExHighlight{geral}} fez     uma   algazarra     ensurdecedora\\
         the.\textsc{fem} general   made   a     hullabaloo   deafening\\
  \glt ‘the audience made a deafening hullabaloo’
\z
\z

The \textit{Dicionário} further mentions the use of \textit{geral} in composite expressions, such as \textit{dar uma geral} (4a) and \textit{em geral} \REF{ex:avelar:4b}, respectively meaning ‘to clean thoroughly’ and ‘in most cases’.

\ea\label{ex:avelar:4}
 \ea\label{ex:avelar:4a}
  \gll o sábado     é   o dia     em que   a faxineira dá     uma \ExHighlight{geral} na casa\\
         the Saturday   is   the day   in which   the cleaner gives   a general in-the   house.\\
  \glt ‘on Saturdays, the maid cleans the house thoroughly’
 \ex\label{ex:avelar:4b} 
  \gll {em} \ExHighlight{geral} vai   ao sítio     duas   vezes   por mês\\
         {in general  goes   to-the ranch   two   times   by month}\\
  \glt  ‘(s)he usually goes to the ranch twice a month’
\z 
\z

The \textit{Dicionário} does not register two fairly common uses of \textit{geral}: its occurrence as an impersonal pronoun, with which this study is directly concerned, and cases such as those in \REF{ex:avelar:5}, in which \textit{geral} may be classified as an adverb. In the latter case, \textit{geral} {works as an intensifier, as} {\textit{muito}} {‘very, much’,} {\textit{demais}} {‘too much’,} {\textit{tudo}} {‘everything’,} {\textit{todo lugar}} {‘everywhere’, among other possibilities. The adverbial use of} \textit{geral} {is quite common even among speakers that find the impersonal use of} \textit{geral} {odd and/or do not resort to it.}

\ea\label{ex:avelar:5}
 \ea\label{ex:avelar:5a} 
  \gll {Bebeu} \ExHighlight{geral} e     o   que   serviam     ele  tomava %\footnote{ \url{http://armazemdoseubrasil.blogspot.com/2011_07_10_archive.html} {(accessed on 30 \citealtSep 2021)}}
\\
         drank.\textsc{3sg} general   and   the  that   served.\textsc{3pl} he   took.\textsc{3sg}\\
  \glt ‘He drank a lot and took whatever was being served.’

 \ex\label{ex:avelar:5b} 
  \gll Choveu \ExHighlight{geral} nos   últimos  sete   dias, mas   em volumes   diferenciados\\
         rained     general   in-the   last     seven   days but   in   volumes   different\\
 %\footnote{ \url{http://diariogaucho.clicrbs.com.br/rs/noticia/2009/01/parana-tem-mais-da-metade-das-lavouras-com-qualidade-media-e-ruim-2357367.html} {(accessed on 30 \citealtSep 2021)}}
    \glt ‘It has rained heavily in the past seven days, but in different volumes.’

 \ex\label{ex:avelar:5c} 
  \gll A coisa   não funcionou   legal   e     [ele]   {se machucou} \ExHighlight{geral}\\
  % \footnote{ \url{https://www.diariodecuiaba.com.br/ilustrado/sobras/416243} {(accessed on 30 \citealtSep 2021)}}
         the thing   not worked     well   and   he     injured     general\\
  \glt ‘It didn’t work out well and he was badly injured.’
\z 
\z

The cases this study is more directly concerned with are those presented in \REF{ex:avelar:1}, in which \textit{geral} {is used as an impersonalization strategy. It is not clear how this item came to be used as a pronoun, but it is possible that this usage represents an advanced stage of its grammaticalization in certain varieties of Brazilian Portuguese. However, I will not concern myself with this question in the present study.}

\section{Impersonal pronouns and φ-fea\-tures}\label{sec:avelar:3}
\subsection{Agreement patterns, interpretation and syntactic distribution}\label{sec:avelar:3.1}

{In this section, I approach the studies of \citet{Egerland2003} and \citet{Fenger2018}, who explore the idea that impersonal pronouns may be described with regard to the presence or absence of φ-fea\-tures. Other generative studies adopt similar perspectives or introduce different approaches (\citealt{Cinque1988, D’AlessandroAlexiadou2003, Hoekstra2010}, among others), but I restrict myself to these two contributions because the analyses they put forward deal with aspects that are more directly pertinent to a formal comparison with properties observed in the use of} {\textit{geral}}. 

{Based on these authors’ proposals, the structures of impersonal pronouns with and without φ-fea\-tures may be represented as in \REF{ex:avelar:6a} and \REF{ex:avelar:6b} respectively: in (a), we see what I will term \textsc{imp} pronouns here, i.e., impersonal pronouns devoid of φ-fea\-tures; in (b), we see what I will term φ-\textsc{imp} pronouns, i.e., impersonal pronouns with φ-fea\-tures.}

\ea\label{ex:avelar:6}
 \ea\label{ex:avelar:6a}
  \textsc{imp}:    [{\textsubscript{NP}} N ]
 \ex\label{ex:avelar:6b}
  φ-\textsc{imp}:    [{\textsubscript{φP}} {φ [}{\textsubscript{NP}} {N ] ]}
\z 
\z

In short, \textsc{imp} pronouns are bare, in the sense that there is no functional projection associated with N, whereas φ-\textsc{imp} pronouns have at least one projection (φP), a phrase headed by specified φ-fea\-tures. \citet{Egerland2003} analyzes impersonal pronouns in Romance and Scandinavian languages to show that {\textit{man} }{in Swedish,} {\textit{on} }{in French and} {\textit{si} }{in Italian are of the \textsc{imp} type, whereas} {\textit{maður} }{in Icelandic and} {\textit{du} }{in Swedish (as} {\textit{you} }{in English) are of the φ-\textsc{imp} type. The author seeks to derive some grammatical and pragmatic properties from the opposition in \REF{ex:avelar:6} above. I will address three such properties here, which shall be relevant for an analysis of the behavior observed in} {\textit{geral}}{: (i) agreement patterns, (ii) generic and/or arbitrary readings, and (iii) its syntactic position.}

\subsubsection{Agreement patterns}

According to Egerland, φ-\textsc{imp} pronouns display previously specified φ-fea\-tures and therefore always trigger the same agreement mark. The Icelandic pronoun {\textit{maður}}{, exemplified in \REF{ex:avelar:7}, is just such a pronoun: adjectives related with it must be marked as singular, as} {\textit{stoltur} }{‘proud’ in \REF{ex:avelar:7a} and} {\textit{sannfærður}} {‘convinced’ in \REF{ex:avelar:7b}; the use of the plural form of these adjectives (}{\textit{stoltir}} {and} {\textit{sannfærðir} }{respectively) is ungrammatical.}

\ea\label{ex:avelar:7} Icelandic \citep[78]{Egerland2003}
 \ea\label{ex:avelar:7a}
  \gll {Í hernum}     er  maður   stoltur {/} *stoltir   af henni.\\
         {in the army}   is \textsc{maður}   proud.\textsc{sg} /   proud.\textsc{pl}   of her\\
  \glt ‘People in the army/they are proud of her.’
 \ex\label{ex:avelar:7b} 
  \gll {þrátt fyrir} sannanirnar   var   maður   ekki alveg {sannfærður / *sannfærðir}    {um sekt hans.}\\
         {in spite of} {the evidence}   was   \textsc{maður}   not completely {convinced.\textsc{sg}/convinced.\textsc{pl}} {about guilt his}\\
  \glt ‘in spite of the evidence, people/they were not convinced.’
\z
\z

{In contrast with φ-\textsc{imp}, \textsc{imp} may be associated with different agreement markings. According to Egerland, this is due precisely to the fact that this kind of pronoun has no φ-fea\-ture, which makes it possible for it to occur along with items bearing different agreement markings. The Swedish cases in \REF{ex:avelar:8} exemplify this property: in \REF{ex:avelar:8a} the pronoun} {\textit{man} }{occurs with the singular and plural forms of the adjective corresponding to} {\textit{proud} }{in English (}{\textit{stolt} }{and} {\textit{stolta} }{respectively); the same occurs with the form corresponding to} {\textit{convinced} }({\textit{övertygad}} {and} {\textit{övertygade} }{respectively) in \REF{ex:avelar:8b}.}

\ea\label{ex:avelar:8} Swedish \citep[78]{Egerland2003}
 \ea\label{ex:avelar:8a} 
  \gll Inom armén     är   man   {stolt/stolta}         över henne. \\
         within {the army}   is   \textsc{man}   {proud.\textsc{sg}/proud.\textsc{pl}}   of her\\
 \ex\label{ex:avelar:8b} 
  \gll Trots     bevisföringen   var   man   inte   helt  {övertygad / övertygade}       om   hans skuld.\\
         {in spite of}  {the evidence}     was  \textsc{man}   not   completely {convinced.\textsc{sg} / convinced.\textsc{pl}}   about   his guilt\\
\z 
\z

{Egerland shows that the lack of uniformity in the agreement patterns may also be observed in the case of gender markings. In Italian, for example, the pronoun} {\textit{si} }{in copular constructions usually requires the third-person singular to be marked in the verb, with the adjective in the masculine plural, as in \REF{ex:avelar:9a}. However, the adjective may be used in the feminine if} {\textit{si} }{refers to a group of women, as in \REF{ex:avelar:9b}.}

\ea\label{ex:avelar:9} Italian \citep[79]{Egerland2003}
 \ea\label{ex:avelar:9a}
  \gll Quando   si   è     giovani,...\\
         when \textsc{si} is.\textsc{sg} young.\textsc{pl.masc}\\
  \glt ‘When people are young…’

 \ex\label{ex:avelar:9b} 
  \gll Quando si è donne, si è disposte a rinunciare a   molte  cose   per i {propri figli}.\\
         when \textsc{si} is women \textsc{si} is ready.\textsc{pl.fem} to renounce to many  things  for   the children\\
\z 
\z

In short, the relevant distinctions between Icelandic \textit{maður}, on the one hand, and \textit{man} in Swedish and \textit{si} in Italian, on the other, are a result of the presence or absence of φ-fea\-tures: whereas \textit{man} {and} {\textit{si}} {are \textsc{imp} pronouns (allowing them to occur with items bearing different φ-fea\-ture specifications),} {\textit{maður}} {is φ-\textsc{imp} (limiting their occurrence to items agreeing with their φ-fea\-tures).}

\subsubsection{Generic and arbitrary readings}

Another relevant distinction between \textsc{imp} and φ-\textsc{imp} is their compatibility with generic and/or arbitrary interpretations. Egerland takes pronouns that have a generic reading to refer to “a quasi-universal set of individuals” whereas those that have an arbitrary reading describe “a non-specific group of individuals”. The author argues that \textsc{imp} can have generic and arbitrary readings, as in the examples of Swedish {\textit{man} }{in \REF{ex:avelar:10a} and \REF{ex:avelar:10b} respectively. In contrast, the φ-\textsc{imp} Icelandic pronoun} {\textit{maður} }{only licenses the generic reading, as can be seen in the grammaticality contrast in \REF{ex:avelar:11}: in \REF{ex:avelar:11a} the meaning is generic, but the only possible reading in \REF{ex:avelar:11b} is that it is arbitrary because of the choice of verbal tense (a form of past simple that is usually not compatible with the generic interpretation).}

\ea\label{ex:avelar:10:} Swedish (\citealt{Egerland2003}: 76; 80)
 \ea[]{\label{ex:avelar:10a}
  \gll Man   måste   arbeta  till 65.\\
         \textsc{man} must   work  until 65\\
  \glt `People have to work until the age of 65.’}
  
 \ex[]{\label{ex:avelar:10b}
  \gll Man   arbetade   i två månader   {för att lösa}   problemet.\\
         \textsc{man} worked   for two months   {to solve}     {the problem}\\
  \glt ‘Some people\slash they worked for two months to solve.’}
\z 

\ex\label{ex:avelar:11} Icelandic \citep[81]{Egerland2003}
 \ea[]{\label{ex:avelar:11a}
  \gll  Maður   vinur   til 65     ára aldurs.\\
         \textsc{maður} works   until 65   years age\\}
 \ex[*]{\label{ex:avelar:11b}
 \gll  {Maður}   hefur unnið   {að því í tvo mánuði}   að leysa vandamálið.\\
         \textsc{maður} has worked   {for two months}     to solve {the problem}\\}
\z
\z

Taking into consideration this contrast between Swedish \textit{man} {(as well as Italian} {\textit{si} }{and French} {\textit{on}}{) and Icelandic} {\textit{maður}}, Egerland comes to the generalization that φ-\textsc{imp} pronouns can only be generic, whereas \textsc{imp} pronouns can have both generic and arbitrary readings. In order to account for this distinction, the author assumes that the generic reading is defined by the presence of a generic operator (see \citealt{KrifkaEtAl1995} and \citealt{Chierchia1995}) that can have either a φ-\textsc{imp} or an \textsc{imp} pronoun under its scope. In contrast, the arbitrary reading is only triggered when a pronoun is devoid of lexical content (which the author sees as equivalent to not having φ-fea\-tures) beyond the trait [+human] and is not under the scope of a generic operator. According to \citet[89]{Egerland2003},

\begin{quote}
By and large, the meaning of such an element amounts to nothing but a [+human] entity in an episodic context. Whether the subject is understood as a single individual or a group of people is entirely determined by the discourse context and is not restrained by any syntactic restrictions. Essentially, this amounts to saying that the notion “arbitrary” […] has no theoretical status and that there is no natural class of “arbitrary pronouns”. Also, there is no “arbitrary” feature to be appealed to in syntactic derivations.
\end{quote}

{Therefore, according to Egerland’s proposal, a pronoun’s ability to bear both readings (generic and arbitrary) or only one of them depends on whether it has or does not have φ-fea\-tures: those that are devoid of φ-fea\-tures (\textsc{imp}) are naturally interpreted as arbitrary in the absence of a generic operator; those that have φ-fea\-tures (φ-\textsc{imp}) will only be interpreted as impersonal if they are under the scope of this operator and, since they have lexical content, they will never be interpreted as arbitrary.}

\subsubsection{Syntactic function}

\citet{Egerland2003} also notes that a distinction between \textsc{imp} and φ-\textsc{imp} lies in their syntactic functions: \textsc{imp} can only appear as syntactic subjects, whereas φ-\textsc{imp} can appear syntactically as both subjects and objects. This distinction may be observed in comparing \REF{ex:avelar:12} and \REF{ex:avelar:13}: in \REF{ex:avelar:12}, the syntactic object is an \textsc{imp} pronoun ({\textit{man}}, {\textit{si} }{and} {\textit{on} }{in Swedish, Italian and French, respectively) and the resulting sentences are ungrammatical; in \REF{ex:avelar:13}, the object pronouns are the non-nominative} {versions of} {\textit{maður} }{and} {\textit{du}} {(which can work as a φ-\textsc{imp} pronoun in Swedish) and, in this case, sentences are grammatical.}

\ea\label{ex:avelar:12}\citep[91]{Egerland2003}
 \ea Swedish\\
 \gll * Det har sett man.\\
       {}  they have seen \textsc{man}\\
 \ex French\\
 \gll * Ils ont vu on. / *Ils on ont vu.\\
       {}  they have seen \textsc{on}\\
 \ex Italian\\
 \gll  * Loro si hanno visto.\\
       {}  they have seen \textsc{si}.\\
\z
\ex\label{ex:avelar:13}\citep[91]{Egerland2003}
 \begin{xlist}
  \exi{a.} Icelandic\\
   \gll Svona tölur   segja   manni     að eitthvað {sé í} ólagi.\\
         such figures   tell   \textsc{manni} that something is wrong\\
  \exi{b.} Swedish\\
   \gll  Om de   litar   på   dig   får     du     inte göra   dem   besvikna.\\
         if they     rely   on   \textsc{you} must \textsc{you} not make  them  disappointed\\
\end{xlist}  
\z 

According to Egerland, \textsc{imp} cannot be syntactic objects because, in the absence of φ-fea\-tures, their semantic role as an internal argument cannot be properly evaluated. In favor of this analysis, \citet{Egerland2003} draws attention to the distinction between Swedish nominative \textit{man} {and accusative/oblique} {\textit{en}}{, in sentences such as \REF{ex:avelar:14}: the relation between these two pronouns is the same that can be observed between English} {\textit{he} }{and} {\textit{him}}{, and the morphological distinction is due to case-marking.}

\ea\label{ex:avelar:14}
 \gll Om de   litar   på   {en \textsubscript{i}} får {{man}\textsubscript{i}} inte göra   dem   beskivna.\\
         if they     rely   on   one   must   \textsc{man}   not make   them   disappointed\\
\z 

{Because} {\textit{en}} {is a numeral, it bears inherent number marking, which makes it possible to classify it as φ-\textsc{imp} and thus allows it to occur in a non-nominative position. From an interlinguistic perspective, Egerland notes that} {\textit{man}}{{}-impersonals may not occur in an object position, whereas} {\textit{one}}{{}-impersonals suffer no such restriction.}

{Egerland’s analysis thus establishes a relation between the presence or absence of φ-fea\-tures and a set of grammatical and pragmatic properties in the use of impersonal pronouns: \textsc{imp} pronouns, precisely because they lack φ-fea\-tures, have different agreement patterns and are only possible in nominative position, where they license both a generic and an arbitrary reading; φ-\textsc{imp} pronouns, in turn, have a fixed agreement pattern and may present different syntactic functions, but only license a generic reading.}

\subsection{Case-marking, φ-fea\-tures and syntactic distribution}\label{sec:avelar:3.2}\largerpage[-1]

{\citet{Fenger2018} analyzes the behavior of the so-called “dedicated impersonal pronouns” (i.e., those that are exclusively used as impersonal pronouns) in eight Germanic languages, and argues that, differently from what \citet{Egerland2003} had stated, it is not the syntactic function that restricts the distribution of these pronouns, but rather the presence of a case projection in their internal configuration. The author assumes that only pronouns with φ-fea\-tures may have case projection (KP), as represented in \REF{ex:avelar:15b}, whereas pronouns devoid of φ-fea\-tures cannot have such projection \REF{ex:avelar:15a}. As a result, the distinction between \textsc{imp} and φ-\textsc{imp} presented in the previous section can be translated as a distinction between NP (noun phrase) and KP (Kase phrase), the latter of which has φ-fea\-tures in its internal constitution.}

\ea\label{ex:avelar:15}
%  \begin{xlist}
%   \exi{a.}
  \ea\label{ex:avelar:15a} 
  \textsc{imp}: {[\textsubscript{NP}} {N ]}\\ 
%   \exi{b.}
  \ex\label{ex:avelar:15b} 
  φ-\textsc{imp}:    [{\textsubscript{KP}} {K [}{\textsubscript{φP}} {φ [}{\textsubscript{NP}} {N ] ] ]}\\
%  \end{xlist}
\z 
\z 

From a theoretical perspective, in the light of the Minimalist Program \citep{Chomsky1995}, Fenger’s proposal (\citeyear{Fenger2018}) is easily motivated by the assumption that case-marking results from the agreement between the (interpretable) φ-fea\-tures of a noun constituent and the (non-interpretable) φ-fea\-tures of a given functional head: the agreement with φ-fea\-tures present, for example, in T(ense), V(erb) and P(reposition) results in nominative, accusative and oblique markings, respectively, on a noun constituent.

In line with other studies on case-marking (for instance, \citealt{BittnerHale1996, NeelemanWeerman1999}, among others), Fenger assumes nominative to be a non-case, which implies that nominative constituents bear no KP, differently from, e.g., accusatives and obliques. Fenger derives the syntactic distribution of \textsc{imp} and φ-\textsc{imp} pronouns from this property: since they bear no KP, \textsc{imp} pronouns may only occur in nominative positions; φ-\textsc{imp} pronouns may in turn occur in positions associated with other cases, precisely because they bear a KP.\largerpage[-1]

One advantage of \citegen{Fenger2018} over \citegen{Egerland2003} proposal is related to the distribution of \textsc{imp} and φ-\textsc{imp} in ECM-constructions: the latter can be a subject of such constructions (provided its reading is generic), whereas the former cannot, irrespective of their generic or {arbitrary reading. The distinction is illustrated in \REF{ex:avelar:16} and \REF{ex:avelar:17} below, with Icelandic and Swedish examples presented by \citet[299--300]{Fenger2018}. If the distribution of impersonal pronouns depended exclusively on syntactic function, the Swedish sentences in \REF{ex:avelar:16b} and \REF{ex:avelar:17b} should be grammatical, since} {\textit{man}}{, an \textsc{imp} pronoun, is the subject in both cases; because the sentence is an ECM-construction, the position of} {\textit{man} }{within the embedded clause is marked with the accusative, not the nominative case, which is easily explained if we assume that this pronoun has no KP, a mandatory projection for enabling the occurrence of noun constituents in positions other than the nominative. In Icelandic, as expected,} {\textit{mann}}{, the accusative form of} {\textit{maður}}{, can appear in this position because it is a φ-\textsc{imp} pronoun, as can be seen in \REF{ex:avelar:16a}. The ungrammaticality of \REF{ex:avelar:17a} results from the fact that} \textit{mann} cannot be read arbitrarily, irrespective of the position in which it occurs.\largerpage

\ea\label{ex:avelar:16} Context: He is a station master.\\ {Intended: ‘Therefore he always sees} {\ExHighlight{people}} {leave for the holidays.’}\\
 \ea\label{ex:avelar:16a} Icelandic (φ-\textsc{imp}, generic)\\
  \gll {} Þess vegna   sér  hann   mann         alltaf     fara í frí.\\
        {} that because   see  he \textsc{impersonal} always   leave in holiday\\
 \ex\label{ex:avelar:16b} Swedish (\textsc{imp}, generic)\\
 \gll * {Därför} see han man alltid åka på semester.\\
       {}  therefore   see he \textsc{impersonal}   always    go     on holydays\\
\z 
\ex\label{ex:avelar:17} Context: I lay awake all night.\\ {Intended: ‘I heard} {\ExHighlight{someone}} {work on the road.’}\\
 \ea\label{ex:avelar:17a} Icelandic (φ-\textsc{imp}, arbitrary)\\
  \gll * Ég heyrði   mann         vinna   vegavinnu.\\
       {}  I   heard     \textsc{impersonal} work   road.construction\\
 \ex\label{ex:avelar:17b} Swedish (\textsc{imp}, arbitrary)\\
  \gll  * Jag   hörde   man         arbetade ute   på gatan.\\
        {} I     heard   \textsc{impersonal}   work out     in the.street\\
\z 
\z

Fenger believes that the generic reading is achieved through the presence of a generic operator [GEN] that can bind both \textsc{imp} and φ-\textsc{imp} pronouns. As for φ-\textsc{imp}, the author states that “its feature specification includes the speaker and the addressee, and this is not contradictory to the requirements of [GEN]. It does not mean that [GEN] always needs to have an element which necessarily includes [speaker] and [addressee], but the element cannot have features that are contradictory with [GEN]” \citep[310]{Fenger2018}.

{As for the arbitrary reading, Fenger departs from Egerland regarding the idea that the arbitrary reading stems naturally from the absence of both lexical content (φ-fea\-tures in this case) and a generic operator. For Fenger, at least two possibilities can be entertained: (i) the presence of an existential operator or (ii) the local relation between the pronoun and the Asp(ect) head. The author does not commit herself to either proposal, but regards both as superior to Egerland’s hypothesis because they predict the occurrence of φ-\textsc{imp} pronouns that may be read arbitrarily, such as German} {\textit{wer} }{and English} {\textit{they}}{, as in \REF{ex:avelar:18} below.}

\ea\label{ex:avelar:18}
 \ea\label{ex:avelar:18a}  
  \gll Ich  habe   die ganze   Zeit {\ExHighlight{wen}} auf/an der Strasse  arbeiten hören.\\
         I   have   the whole   time \textsc{indef} on the road       work hear\\
 \glt ‘I heard someone work on the road.’

 \ex\label{ex:avelar:18b} \textit{\ExHighlight{They} {have called for you, but I don’t know what is about.}}\\
\z 
\z

{I shall here assume Fenger’s proposal regarding the presence of KP in φ-\textsc{imp} and the absence of this projection in \textsc{imp}. As I will argue, this proposal is quite advantageous in the attempt to locate the occurrences of} \textit{geral} {within the set of properties of the Brazilian Portuguese pronominal system. As for the conditions for the generic and/or arbitrary reading, I shall not commit myself to any approach here, since, from a purely formal perspective, there are no relevant consequences for the properties of} \textit{geral} {that I will be considering.}


\subsection{Comparative perspective}\label{sec:avelar:3.3}


{The table below summarizes the set of properties that have been addressed in this section according to the distinction of impersonal pronouns as \textsc{imp} and φ-\textsc{imp}. In the next section, I will analyze} \textit{geral} {according to these properties in an attempt to determine the best   characterization of this item when used as an impersonal pronoun.}


\begin{table}
\begin{tabular}{lcc}

\lsptoprule

{Properties} & \textsc{imp} & {{φ-\textsc{imp}}}\\
\midrule
{Is associated with a single form of agreement.} & { no} & { yes}\\
{May have either a generic or an arbitrary reading.} & { yes} & { no}\\
{Only occurs in nominative positions.} & { yes} & { no}\\
\lspbottomrule
\end{tabular}
\caption{Properties of \textsc{imp} and φ-\textsc{imp} pronouns\label{tab:avelar:1}}
\end{table}

\section{{Properties} {of} \textit{geral} {in} {Brazilian} Portuguese}\label{sec:avelar:4}
 \subsection{Agreement patterns}\label{sec:avelar:4.1}

{As far as agreement is concerned,} \textit{geral} {behaves as an \textsc{imp} pronoun, since it does not trigger fixed agreement marking in verbs and adjectives. Particularly in the case of verbal agreement,} \textit{geral} {can occur with verbs both in the third-person singular, as in \REF{ex:avelar:19}, and in the third-person plural, as in \REF{ex:avelar:20}, although the former is more frequent.}


\ea\label{ex:avelar:19}
 \ea\label{ex:avelar:19a} 
  \gll eu   não  consigo   entender pq \ExHighlight{geral} não   gosta de mim\\
  %  }\footnote{ \url{https://twitter.com/isasalviattii/status/928063135046660097} {(accessed on 1 \citealtJul 2019)}}
         I   not   can     understand why   \textsc{geral} not like.\textsc{pres.3sg} of mine \\
  \glt ‘I cannot understand why nobody likes me.’ \\
 \ex\label{ex:avelar:19b}
  \gll Alguém   postou  no Facebook     uma  lista   com  curiosidades sobre a   série Vaga-Lume, aquela   que {\ExHighlight{geral}} {conhece}\\
  %\footnote{ \url{https://sonhandocomdarcy.wixsite.com/sonhandocomdarcy/single-post/2015/12/10/Top-5-Autores-Que-Eu-Nunca-Li} {(accessed on 1 \citealtJul 2019)}}
         somebody  posted  in-the Facebook   a     list   with   curiosities about the   series Vaga-Lume {that one}   that \textsc{geral} know.\textsc{pres.3sg}\\
  \glt ‘Someone has posted on Facebook a list of curious facts about the book series Vaga-Lume, the one everybody knows.’
\z 
\ex\label{ex:avelar:20}
 \ea\label{ex:avelar:20a} 
  \gll nao   gosto         de la,     me  sinto     como   se \ExHighlight{geral} tivessem         me     observando \\ %\footnote{ \url{https://answers.yahoo.com/question/index?qid=20170211204407AA4anTs} {(accessed on 30 \citealtSep 2021)}}\\
         {not   like.\textsc{pres.1sg} of there   me  feel.\textsc{1sg} as     if \textsc{geral} had.\textsc{past.snj.3pl} me    observing}\\
  \glt ‘I don’t like that place, I feel like everyone was observing me.’
 \ex\label{ex:avelar:20b}
  \gll sabemos       que \ExHighlight{geral} curtem os bonés da     nossa   coleção\\ %\footnote{ \url{https://gramho.com/explore-hashtag/bonesjs} {(accessed on 30 \citealtSep 2021)}}
         know.\textsc{pres.1pl} that \textsc{geral} like.\textsc{pres.3pl} the caps of-the   our   collection\\
\glt ‘we know everyone likes the caps in our collection’
\z 
\z

As for adjectives, variation can be observed both in gender, with masculine \REF{ex:avelar:21a} and feminine \REF{ex:avelar:21b} forms, and in number, with singular \REF{ex:avelar:21} and plural \REF{ex:avelar:22}.

\ea\label{ex:avelar:21}
 \ea\label{ex:avelar:21a} 
  \gll que delícia \ExHighlight{geral} deixando   {de   ser}   otário e     respeitando   a   opinião   alheia \\ %\footnote{ \url{https://twitter.com/Mandy_Baessa} {(accessed on 1 \citealtJul 2019)}}
        what delight   \textsc{geral} leaving   {of   to-be}   douchebag.\textsc{sg.masc} and   respecting   the  opinion   of-other\\
 \glt ‘what a delight to see everyone quitting being a douchebag and respecting each other’s opinions’
 \ex\label{ex:avelar:21b} 
  \gll mas   o que \ExHighlight{geral} ficou   interessada mesmo foi     na receita   do     meu   bolo   de morango \\ %\footnote{ \url{https://blogqueideia.wordpress.com/2017/03/20/bolo-de-morango-a-receita/} {(acessado em 1 \citealtJul 2019)}}
         but   the what   \textsc{geral}   was   interested.\textsc{sg.fem} actually was   in-the recipe   of-the   my   cake   of strawberry\\
  \glt ‘[...] but what everyone was actually interested in was my strawberry cake recipe’
\z 
\ex\label{ex:avelar:22}
 \ea\label{ex:avelar:22a}
  \gll CAPCOM  sabe   que \ExHighlight{geral} ficaram no mínimo frustrados\\ %\footnote{ \url{https://beta2.gamevicio.com/noticias/2021/09/resident-evil-3-deve-receber-atualizacao-em-breve/} {(accessed on 30 \citealtSep 2021)}}
         C.       knows  that  \textsc{geral} were in-the minimum   frustrate.\textsc{masc.pl}\\ 
  \glt ‘CAPCOM knows that everyone was at least frustrated’
 \ex\label{ex:avelar:22b}
  \gll [estou]   chocada   que \ExHighlight{geral} tao   passados pq       {a médica [...]} combinou       as perguntas com   os   governistas\\ %\footnote{ \url{https://twitter.com/Neni66576183} {(accessed on 30 \citealtSep 2021)}} \\
         be.\textsc{1.sg} shocked   that   \textsc{geral} are   astonished.\textsc{masc.pl} because   {the doctor}   combined-\textsc{3sg} the questions with   the  governmentists\\
  \glt ‘[I’m] shocked that everyone is astonished that the doctor [...] previously agreed on the questions with government supporters’
\z 
\z

{Therefore, with regard to agreement patterns,} \textit{geral} {behaves as an \textsc{imp} pronoun, with no φ-fea\-tures demanding a fixed agreement marking.}



\subsection{Generic and arbitrary readings}\label{sec:avelar:4.2}

{As already noted in the introduction,} \textit{geral} {may occur with both the generic and the arbitrary readings. Occurrences in \REF{ex:avelar:23} exemplify the generic reading. More particularly, in \REF{ex:avelar:23a}} \textit{geral} {occurs with other pronominal forms (}{\textit{todo mundo}} {‘everyone’,} {\textit{todos}} {‘all’,} {\textit{você}} {‘you’) that are also interpreted as generic in Brazilian Portuguese. In \REF{ex:avelar:24}, we find some cases of} \textit{geral} {taking on arbitrary interpretation.}

\ea\label{ex:avelar:23}
 \ea\label{ex:avelar:23a}
  \gll {Todo mundo}   tem   aquele  autor       ou   autora que \ExHighlight{geral} conhece   bem   e     todos  falam         bem, mas  você   nunca  chegou     a pegar   qualquer   uma de  suas   obras   pra   ler\\ %\footnote{ \url{https://sonhandocomdarcy.wixsite.com/sonhandocomdarcy/single-post/2015/12/10/Top-5-Autores-Que-Eu-Nunca-Li} {(accessed on 1 \citealtJul 2019)}}
         everyone     has   that   author.\textsc{masc} or   author.\textsc{fem} that   \textsc{geral}   knows   well   and  all     speak.\textsc{pres.3pl} well but   you   never   arrived.\textsc{3sg} to take   any     one of   their   books   for     to-read\\
  \glt ‘Everyone has a male or female author that everyone knows well and praises, but you never actually got to reading one of their works.’
  \ex\label{ex:avelar:23b}
   \gll Quem   vai   perder    o mercado     muito   {em breve}   é a Samsung, que   acha   que \ExHighlight{geral} não  acompanha   a evolução\\ %\footnote{ \url{https://www.tudocelular.com/samsung/noticias/n142917/analise-samsung-galaxy-a10-review.html} {(accessed on 4 \citealtJul 2019)}}
         who   goes   to-read    the market   very   soon     is   the Samsung that  thinks that   \textsc{geral} not    follows     the evolution\\
  \glt ‘Samsung will soon be out of the market, for they think people do not keep up with innovation.’
\z 
\z 
\ea\label{ex:avelar:24}
 \ea\label{ex:avelar:24a}
  \gll Nem   preciso   dizer   que \ExHighlight{geral} ficou   boquiaberto ao     ver   nós dois   juntos \\ %\footnote{ \url{https://www.wattpad.com/590285161-visão-de-cria-cap\%C3\%ADtulo-35/page/2} {(accessed on 1 \citealtJul 2019)}}\\
         not   need.\textsc{1.sg} to-say  that   \textsc{geral}  was   agape when   to-see   we two   together\\
 \glt ‘I don’t even have to say that everyone was agape when they saw us together.’
  \ex\label{ex:avelar:24b}
   \gll rolou     um pipoco   [e] \ExHighlight{geral} correu achando   que  era   o   bope\\ %\footnote{ \url{https://twitter.com/boombapx/status/772478713963372544} {(accessed on 4 \citealtJul 2019)}}
         happened   an   uproar   and   \textsc{geral}   ran thinking   that   was   the  bope\\
  \glt ‘there was an uproar, everyone ran thinking it as the BOPE [a division of the police]’
\z
\z 

{Therefore, as regards generic and arbitrary readings,} \textit{geral} {also behaves as an \textsc{imp} pronoun, exhibiting patterns that are similar to Swedish} {\textit{man}}{, French} {\textit{on} }{and Italian} {\textit{si}}{, in line with the properties presented in \citet{Egerland2003}.}

\subsection{Syntactic distribution}\label{sec:avelar:4.3}

As far as its distribution within the sentence is concerned, \textit{geral} {is compatible with different syntactic functions and may occur in positions associated with the nominative, accusative and oblique cases. In addition to the cases hitherto presented, in which} \textit{geral} occurs in a nominative position, it may also appear in an accusative position, as in \REF{ex:avelar:25}, and in oblique positions, as in (\ref{ex:avelar:26}--\ref{ex:avelar:28}).

\ea\label{ex:avelar:25}
 \ea\label{ex:avelar:25a}
  \gll falaram   que   ele   é   uma simpatia   e   atendeu \ExHighlight{geral} com   o   maior     carinho \\ %\footnote{ \url{https://www.facebook.com/PaparazzoRN/photos/a.517262558616376/773477566328206/?type=3 & theater} {(accessed on 3 \citealtAug 2020)}}
  said.\textsc{3pl} that  he   is   a {nice person}   and received.\textsc{3sg} \textsc{geral} with   the  biggest   gentleness\\
  \glt ‘they said he is really nice and received everyone with the utmost gentleness’
  \ex\label{ex:avelar:25b}
   \gll Quando   aquela pessoa  que   elogia \ExHighlight{geral}, vem   e     te     elogia,   não   rola     emoção\\ %\footnote{ \url{https://twitter.com/nadinerv/status/766251351244414976} {(Accessed on 3 \citealtAug 2020)}}
         when     that person    that   praises   \textsc{geral} comes  and   you   praises not   happen   emotion\\
  \glt ‘When someone who praises everybody comes and praises you, you can’t feel touched.’
 \ex\label{ex:avelar:25c}
  \gll ensinei \ExHighlight{geral} a   como   jogar R6 \\
         taught.\GlossMarkup{1SG}   \textsc{geral}   to   how   play R6\\
  \glt `I have taught everyone how to play R6.’
\z %\footnote{ \url{https://www.youtube.com/watch?v=xgWk4Oo7TUU} {(Accessed on 3 \citealtAug 2020)}}
 \ex\label{ex:avelar:26}
  \gll ele  já       tirou   print   e     já       enviou  pra \ExHighlight{geral} \\ %\footnote{ \url{https://pandlr.com/forum/22-pan/forum/topic/off-alguem-que-entende-de-twitter-help/?cache=1} {(Accessed on 3 \citealtAug 2020)}}
        he   already   took   print  and   already   sent   to     \textsc{geral}\\
  \glt ‘he has already taken a screenshot and sent it to everyone’
\ex\label{ex:avelar:27}
 \gll Hoje   não me  interessa   a  aprovação  de \ExHighlight{geral} apenas   a   minha  felicidade \\ % \footnote{ \url{https://www.picuki.com/tag/quandovivideverdade} {(Accessed on 30 \citealtSep 2021)}}
         today   not me   interest   the  approval     of \textsc{geral} only     the  my   happiness\\
 \glt ‘Currently I don’t care about being approved by everyone, but only about my happiness.’
\ex\label{ex:avelar:28} 
 \gll \ExHighlight{Geral} sabe   que   ela   fica   cm \ExHighlight{geral} pega   o   bonde  todoooooo\\ %\footnote{ \url{https://curiouscat.me/Caralhouuuuuuuuuuuuu} {(Accessed on 7 \citealtSep 2021)\\
         \textsc{geral}   knows  that   she   stays   with   \textsc{geral} takes   the  tram   entire\\
 \glt{‘Everyone knows she picks up everybody, she fools around with everybody!!!’}
\z


{As for its syntactic distribution, \textit{geral} }{therefore behaves as a φ-\textsc{imp} pronoun, therefore contradicting what has been established concerning its agreement pattern and its reading, criteria that would rather align} \textit{geral} {with \textsc{imp} pronouns.}

\tabref{tab:avelar:2} illustrates the behavior of \textit{geral} {in comparison with that of \textsc{imp} and φ-\textsc{imp} pronouns.} 

\begin{table}
\begin{tabular}{l ccc}
\lsptoprule
{Properties} & \textsc{imp} & {{φ-\textsc{imp}}} & {{\textit{geral}}}\\
\midrule
Is associated with a single form of agreement. &  no &  yes &  no\\
May have either a generic or an arbitrary reading. &  yes &  no &  yes\\
Only occurs in nominative positions. &  yes &  no & no\\
\lspbottomrule
\end{tabular}
\caption{Properties of \textsc{imp}, φ-\textsc{imp} and \textit{geral}\label{tab:avelar:2}}
\end{table}

At first sight, we are thus facing a problem for the precise characterization of \textit{geral}, since, if it were an \textsc{imp} (i.e., devoid of φ-fea\-tures and of Case projection), it ought to be licensed only in nominative positions. In the next section, however, I shall argue that this apparently contradictory behavior of \textit{geral} regarding case marking is to be expected in view of the properties of the Brazilian Portuguese pronominal system.

\section{Placing \textit{geral} within the Brazilian Portuguese pronominal system}\label{sec:avelar:5}

{We have seen that} \textit{geral} {behaves as \textsc{imp} with regard to agreement patterns and to interpretation, but as φ-\textsc{imp} with regard to the syntactic positions in which it may occur. As I will argue for, this inconsistent behavior is expected if we take into consideration that the impersonal version of} \textit{geral} {is integrated into the Brazilian Portuguese pronominal system.}

{The paradigm of Brazilian Portuguese personal pronouns, especially in its vernacular varieties, licenses the occurrence of typically nominative pronouns in accusative and oblique positions (cf. \citealt{Carvalho2008, GalvesEtAl2016}, among others). This property is widely observed for the third-person nominative pronouns} {\textit{ele/ela}} {(‘he/she’), which are frequently used in the object position, instead of} {\textit{o/a} }{(‘him/her’), both in the singular and in the plural (\ref{ex:avelar:29}--\ref{ex:avelar:30}). The forms} {\textit{você} }{‘you’ \REF{ex:avelar:31} and} {\textit{a gente} }{‘we’ \REF{ex:avelar:32}, which are typically nominative position pronouns, are also frequent in the accusative position, instead of} {\textit{te} }{(‘you’) and} {\textit{nos} }{(‘us’) – cf. \citealt{Lopes2003, ViannaLopes2012, LopesVianna2013, LopesRumeu2015}, among others.}\footnote{One of the reviewers of this chapter made the following remarks: “The fact that {\textit{a gente} }{(and other ``pronouns'' that derived from NPs, such as Colloquial Brazilian Portuguese} {\textit{o pessoal}}, {\textit{as pessoas}} {and} {\textit{o povo} }{but also Standard European and Brazilian Portuguese}{ \textit{o senhor}}{) may appear in object position could be due to them still being felt as NPs (or NP-like).} {\textit{Geral}} {as a pronoun can be related to a noun as well; this would be an alternative explanation”. Even if this alternative explanation is correct, it does not exclude the need for an additional explanation, given that, unlike} {\textit{o pessoal}}, {\textit{as pessoas}} and {\textit{o povo}}{, the items} {\textit{geral}} {and} {\textit{a gente}} {do not trigger any specific mark of agreement with verbs and adjectives (cf. \sectref{sec:avelar:4}). In Brazilian Portuguese (at least in my variety, spoken in the metropolitan region of Rio de Janeiro),} {\textit{o pessoal}} {and} {\textit{o povo} }{always trigger the masculine singular mark, whereas} {\textit{as pessoas}} {triggers the feminine plural. These items have previously specified φ-features and, if analyzed as impersonal pronouns, should be treated as φ-\textsc{imp}. It is not surprising that they can occur as direct objects, regardless of whether they are nouns or pronouns. What is surprising in this picture is that} {\textit{geral}} {and} {\textit{a gente}} {occur as direct objects (even though they derive historically from nouns), as they behave like items without φ-features and, as such, should occur only as subjects. The explanation I propose in this chapter is that the Brazilian Portuguese pronominal system, in contrast to the system of the languages exemplified in \citet{Egerland2003} and \citet{Fenger2018}, allows typically nominative pronouns (including φ-featureless impersonal pronouns) in any syntactic function/position, regardless of such pronouns still being felt as NPs (or NP-like). Therefore, my proposal is not incompatible with the alternative explanation suggested by the reviewer.}}


\ea\label{ex:avelar:29}
 \gll testemunhas   confirmam   que  viram \ExHighlight{ele} no local   do crime \\ % \footnote{ \url{https://www.campograndenews.com.br/cidades/capital/rapaz-suspeito-de-matar-a-mulher-tem-habeas-corpus-negado-pela-justica} {(Accessed on 30 \citealtSep 2021)}}
         witnesses     confirm.\textsc{3pl} that   saw.\textsc{3pl} he   in-the   place   of-the crime\\
 \glt ‘witnesses confirm that they saw him in the crime scene’  
\ex\label{ex:avelar:30}
 \gll eles  levaram \ExHighlight{ela} pro   veterinário\\ % \footnote{ \url{https://www.vakinha.com.br/vaquinha/ajudem-a-luna-igor-romero-dos-santos} {(Accessed on 30 \citealtSep 2021)}}
         they   took.\textsc{3pl} she   for-the  vet\\
 \glt ‘they took her to the vet’ 
\ex\label{ex:avelar:31} 
 \gll eu   conheço     \textit{você}   desde   os   seis   anos   de idade \\ % \footnote{ \url{https://www.intrinseca.com.br/paratodososgarotosquejaamei/carta/3195/tentando-me-despedir} {(Accessed on 30 \citealtSep 2021)}}
         I   know.\GlossMarkup{1SG}   you   since   the  six     years   of age\\
 \glt ‘I have known you since you were six years old’
\ex\label{ex:avelar:32}
 \gll todos   cumprimentaram {\ExHighlight{a} \ExHighlight{gente}} \\ % \footnote{ \url{https://www.dgabc.com.br/Noticia/2803060/jovens-viajam-por-dez-horas-como-representantes-unicos-de-cidades} {(Accessed on 30 \citealtSep 2021)}}
         all.\textsc{pl} greeted.\textsc{3pl} we\\
 \glt ‘everyone greeted us’
\z 

{Although less frequently, the first-person singular (}{\textit{eu}}{, ‘I’) and first-person plural (}{\textit{nós}}{, ‘we’) nominative forms also occur in typically accusative positions instead of} {\textit{me} }{(‘me’) and} {\textit{nos} }{(‘us’), either as subjects of ECM-constructions (see \ref{ex:avelar:33}) or as direct objects (see \ref{ex:avelar:34}).}

\ea\label{ex:avelar:33} 
 \gll muitas que  não   viram \ExHighlight{eu} jogar falam     como   se   me     acompanhassem\\  %\footnote{ \url{https://www.palmeiras.com.br/pt-br/noticias/torcedores-idolatram-evair-em-loja-oficial-do-palmeiras-de-catanduva/} {(Accessed on 30 \citealtSep 2021)}} \\
         many\textsc{fm}  that   not   saw.\textsc{3pl} I   to-play talk.\textsc{3pl} as     if   me   follow.\textsc{past.sbj.3pl}\\
 \glt ‘many [fem.] who didn’t see me play talk as if they had been following my career’
\z 
\ea\label{ex:avelar:34} 
 \gll os  {funcionários [...]} atenderam \ExHighlight{nós} com   muita gentileza \\ %\footnote{ \url{https://behappyviagens.com.br/depoimento/131} {(Accessed on 30 \citealtSep 2021)}}
         the  employees     helped.\textsc{3pl}   we     with   much attention\\
 \glt ‘the employees helped us with the utmost attention’
\z 

{These data make it clear that any typically nominative pronoun in Brazilian Portuguese may also occur in non-nominative positions, in contrast with European Portuguese. It is not yet clear how to account for the licensing of nominative pronouns in non-nominative positions. \citet{AvelarGalves2016}, for example, argue that noun constituents in Brazilian Portuguese, including personal pronouns, may or may not exhibit case marking; in the specific situation of pronouns, one of the consequences of this variation would be precisely the use of the morphological nominative when the pronoun is not marked for case and, as such, is licensed for any syntactic position.}

{Irrespective of the formal explanation that may eventually account for these observations, the fact is that, in Brazilian Portuguese, typical nominative pronouns are licensed in positions associated with different cases. This must be the exact behavior of} \textit{geral} {if it has already been integrated to the pronominal system of the language, at least in the grammar of those speakers that resort to it as an impersonalization strategy. If this analysis is on the right track, then} \textit{geral} {must be characterized as an \textsc{imp} pronoun (that is to say, a typically nominative pronoun devoid of φ-fea\-tures), but that, differently from} {\textit{man}}, {\textit{on}}, {\textit{si}} {and other \textsc{imp} pronouns, occurs in different syntactic positions as a result of the peculiarities of the personal pronoun system in Brazilian Portuguese. The properties of} \textit{geral} {therefore provides indirect evidence in favor of \citegen{Fenger2018} perspective, for whom the distribution of personal pronouns in different positions in the sentence is better captured in terms of case marking, and not as a result of the syntactic function of the pronoun.}

\section{Occurrences of \textit{geral} with specific and inclusive readings}\label{sec:avelar:6}
{\citet[82--83]{Egerland2003} mentions situations in which Swedish \textit{man} may refer to the first-person singular, as in \REF{ex:avelar:35a}. French \textit{on} }{and Italian} {\textit{si}} {may also occur with a specific reading, but referring to the first-person plural, as in \REF{ex:avelar:35b} and \REF{ex:avelar:35c} respectively. Egerland terms these occurrences in Swedish, on the one hand, and in French and Italian, on the other,} {\textit{specific reading} }{and} {\textit{inclusive reading}}{, respectively.}

\ea\label{ex:avelar:35}
 \ea\label{ex:avelar:35a} Swedish (Arbitrary/Specific)\\
  \gll Man   arbetade   i två månader   för att lösa   problemet.\\
         \textsc{man} worked   for two months   to solve     the problem\\
 \ex\label{ex:avelar:35b} Italian (Arbitrary/Inclusive)\\
  \gll Si è lavorato     per due mesi     per risolvere   il problema.\\
        \textsc{si} has worked   for two months   to solve     the problem\\
 \ex\label{ex:avelar:35c} French (Arbitrary/Inclusive)\\
  \gll On a travaillé   pour deux mois   pour résoudre   le problème.\\
       \textsc{on} has worked   for two months   to solve       the problem\\
\z 
\z

{\textit{Geral} }{may also occur in a reading that refers to a group of individuals that includes the speaker, as in the case of pronouns} {\textit{nós} }{and} {\textit{a gente}} {‘we’. In \REF{ex:avelar:36a}, for example,} \textit{geral} {refers to a specific group of students in which the speaker is included; in \REF{ex:avelar:36b},} \textit{geral} {also refers to a specific group of people (the speaker’s family) that may or may not include the speaker.}

\ea\label{ex:avelar:36}
 \ea\label{ex:avelar:36a} 
  \gll Ter     aulas     no sábado     é horrivel!    Eu preferia ter       aulas     até dezembro!  Minha escola ainda não {se pronunciou}   quanto   a isso   só     falou que \ExHighlight{geral} vai  voltar   dia 10.\\ % \footnote{ \url{http://praverpralereouvir.blogspot.com/2009/07/atim-sera-que-estou-com-gripe-suina.html} {(Accessed on 30 \citealtSep 2021)}}
         to-have   classes    in-the Saturday   is horrible    I prefer.\textsc{past.1sg} to-have   classes    until December   My school   yet   not pronounce.\textsc{3sg} cregarding   to that  only   said.\textsc{3sg} that   \textsc{geral}   go   return day 10\\
  \glt {‘Having classes on Saturday is awful! I would rather have classes into December! My school has still not made an announcement about this and only said that we students are back on the 10}{\textsuperscript{th}}.’
 \ex\label{ex:avelar:36b} 
  \gll Minha mãe     ta internada,     ae  ontem     tive     que   fazer a janta,   fiz       um  macarrão  que \ExHighlight{geral} comeu   horrores, só     não   falo     como   fazer   porque   to   com pressa\\ %\footnote{ \url{https://hangarnet.com.br/showthread.php?tid=107803 & pid=1340162} {(Accessed on 30 \citealtSep 2021)}}
         my   mother   is hospitalized   so   yesterday   had.\textsc{1sg}  that   to-do the dinner  did.\textsc{1sg} a   pasta     that   \textsc{geral}   ate.\textsc{3sg} horrors only   not   say.\textsc{1sg}  how   to-do   because   am  with hurry\\
 \glt ‘My mother is hospitalized, so I had to cook dinner yesterday. I made pasta that everyone ate until they were full. I only won’t teach you how to make it because I’m in a hurry.’
\z 
\z

These occurrences raise the question of determining what licenses the specific and inclusive readings, since \textsc{imp} pronouns do not exhibit φ-fea\-tures. In other words, in the absence of φ-fea\-tures, these pronouns ought to exhibit either the generic or the arbitrary readings, since they lack the necessary ingredient for establishing a reference to the first-person. Discussing this kind of data, Egerland argues that there are reasons for believing that inclusiveness may not be predictable:

\begin{quote}
    [...] whereas there are principled reasons behind the restrictions on generic and “arbitrary” readings, it is more doubtful whether there are principles deciding whether impersonal pronouns are interpreted as including or excluding the speaker. Depending on the function of the message and the communicative strategies of the speaker, the impersonal construction is pragmatically open to a variety of uses. Some such uses will be inclusive, others exclusive, not for syntactic reasons, but due to contextual (extra-linguistic) factors.\hfill\hbox{\citep[96--97]{Egerland2003}}
\end{quote}

In other words, the factors at stake to determine the specific and inclusive readings of impersonal pronouns are not of a grammatical, but rather of a pragmatic nature, and it is altogether impossible to predict one or another reading without taking extralinguistic factors into account.

{Egerland acknowledges, however, that there is interlinguistic variation in the conditions for one or another reading to be triggered. The author exemplifies by comparing the Icelandic} {\textit{maður} }{and the Swedish} {\textit{man} }{impersonal pronouns, as in \REF{ex:avelar:37} and \REF{ex:avelar:38}. In Icelandic,} {\textit{maður} }{and} {\textit{eg} }{(‘I’) may be coreferential in \REF{ex:avelar:37a}, but not in \REF{ex:avelar:37b}; in contrast,} {\textit{man} }{and} {\textit{jag} }{(‘I’) may be coreferential as much in \REF{ex:avelar:38a} as in \REF{ex:avelar:38b}.}\pagebreak 


\ea\label{ex:avelar:37}  Icelandic (\citealt{Jónsson1992} \textit{apud} {\citealt{Egerland2003}: 98)}
 \ea\label{ex:avelar:37a}
  \gll Eg vona   að     maður   verdði     ekki of seinn.\\
        I hope     that   \textsc{maður}  will-be   not too late\\
  \glt ‘I hope I won’t be late.’

 \ex\label{ex:avelar:37b} 
  \gll Maður   vona   að eg   verdði     ekki of seinn.\\
         \textsc{maður} hopes   that I   will-be   not too late\\
  \glt ‘People hope I won’t be late.’
\z 
\ex\label{ex:avelar:38}
 \ea\label{ex:avelar:38a} Swedish \citep[98--99]{Egerland2003}\\
  \gll Jag  får     hopas   att     man   inte   kommer   för sent.\\
         I     may   hope   that   \textsc{man}  not   comes     too late\\
  \glt ‘I hope I won’t be late.’ / ‘I hope they won’t be late.’\\
 \ex\label{ex:avelar:38b} 
  \gll Man   får     hoppas   att     jag    inte   kommer   för sent.\\
        \textsc{man}  may   hope     that   I     not   come     too late\\
 \glt ‘I hope I won’t be late’\\ ‘They hope I won’t be late.’\\ ‘Let’s hope I won’t be late.’
 \z 
\z

{In his discussion of this difference, \citet[98--99]{Egerland2003} points to a suggestion by \citet{Jónsson1992}, according to which “a lower ranked feature cannot bind a higher ranked one”. If we consider that the first-person is ranked higher than the third-person, it becomes possible to explain why} {\textit{eg} }{may bind} {\textit{maður} }{in \REF{ex:avelar:37a}, producing the specific reading of the impersonal pronoun; the opposite, as in \REF{ex:avelar:37b}, preserves the generic reading of} {\textit{maður} }{(which is a third-person), since, by the same rule, the impersonal pronoun cannot bind a first-person pronoun. In Swedish, however,} {\textit{man} }{has no φ-fea\-tures and therefore escapes the condition set forth by Jónsson, thus enabling it to take on a specific reading referring to the first-person, whether it is or not bound by} {\textit{jag}}.

{What has been noted for} {\textit{man} }{is precisely what happens with} {\textit{geral}}{, as can be seen in \REF{ex:avelar:39}, with the exception of the type of readings (specific in Swedish and inclusive in Brazilian Portuguese).} {\textit{Geral} }{may be coreferential with the pronoun} {\textit{a gente} }{(‘we’, which would also apply to the form} {\textit{nós}}{, ‘we’), both in \REF{ex:avelar:39a} and in \REF{ex:avelar:39b}.}

\ea\label{ex:avelar:39}
 \ea\label{ex:avelar:39a} 
  \gll {A gente}   tá esperando   que \ExHighlight{geral} não   chegue   atrasado\\
         we       are hoping   that   \textsc{geral}   not   arrive     late\\
  \glt ‘We hope we won’t be late.’\\ ‘We hope they won’t be late.’
 \ex\label{ex:avelar:39b} 
  \gll \ExHighlight{Geral} tá esperando   que  {a gente}   não   chegue   atrasado.\\
         \textsc{geral}   are hoping   that   we       not   arrive     late\\
  \glt ‘We hope we won’t be late.’ \\ ‘They hope we won’t be late.’
\z 
\z

If Egerland’s proposal is on the right track, the possibility that \textit{geral} presents an inclusive reading in some contexts is precisely due to the fact that it bears no φ-fea\-tures (that is to say, it is an \textsc{imp} pronoun, and not a φ-\textsc{imp} pronoun). The factors that will determine such a reading are, however, of a pragmatic nature, since they depend on contextual (extralinguistic) factors that interfere in the reading of the sentence.

\section{Conclusion}\label{sec:avelar:7}

Although the grammatical and/or pragmatic factors that have triggered (or have been triggering) the use of \textit{geral} as an impersonal pronoun demand further study, there is evidence that the process of its impersonalization resulted in the emergence of an \textsc{imp} pronoun, with a behavior similar to that of {\textit{man}}, {\textit{on}} {and} {\textit{si}} as described by \citet{Egerland2003}. The apparent inconsistency with regard to the syntactic positions it can fill, approximating the pattern of a φ-\textsc{imp} pronoun, may be easily explained if we take into account the case-marking properties of the Brazilian Portuguese pronominal system, in which nominative pronouns are licensed in non-nominative positions. In other words, \textit{geral} has been gaining ground as an inherently nominative \textsc{imp} pronoun that is also licensed in accusative and oblique positions, as other nominative pronominal forms in Brazilian Portuguese. Besides the generic and arbitrary readings, \textit{geral} can also have an inclusive reading (when it is the referential equivalent of a first-person plural, just as pronouns \textit{nós} {and} {\textit{a gente}}). The inclusive reading seems to depend on strictly pragmatic factors, as a result of the lack of φ-fea\-tures.\largerpage

\appendixsection{Source documents used in the examples}\label{appendix:avelar}

\begin{exe}\sloppy
     \exr{ex:avelar:1}
     \ea {\url{https://votolegal.com.br/em/brunoramos} (accessed on 8 Sep 2018)}
     \ex {\url{https://twitter.com/bbcbrasil/status/1020326263335079936} {(accessed on 8 Sep 2018)}}
     \ex {\url{https://vk.com/topic-73988417_37659003?offset=1380} {(accessed on 4 Jul 2019)}}
     \ex {\url{https://www.facebook.com/watch/?v=1722931431109561} {(accessed on 1 Jul 2019)}}
    \z
    \exr{ex:avelar:5}
    \ea \url{http://armazemdoseubrasil.blogspot.com/2011_07_10_archive.html} {(accessed on 30 Sep 2021)}
        \ex \url{http://diariogaucho.clicrbs.com.br/rs/noticia/2009/01/parana-tem-mais-da-metade-das-lavouras-com-qualidade-media-e-ruim-2357367.html} {(accessed on 30 Sep 2021)}
        \ex \url{https://www.diariodecuiaba.com.br/ilustrado/sobras/416243} {(accessed on 30 Sep 2021)}
    \z
    \exr{ex:avelar:19}
      \ea \url{https://twitter.com/isasalviattii/status/928063135046660097} {(accessed on 1 Jul 2019)}
      \ex \url{https://sonhandocomdarcy.wixsite.com/sonhandocomdarcy/single-post/2015/12/10/Top-5-Autores-Que-Eu-Nunca-Li} {(accessed on 1 Jul 2019)}
    \z
    \exr{ex:avelar:20}
        \ea \url{https://answers.yahoo.com/question/index?qid=20170211204407AA4anTs} {(accessed on 30 Sep 2021)}
        \ex \url{https://gramho.com/explore-hashtag/bonesjs} {(accessed on 30 Sep 2021)}
    \z
    \exr{ex:avelar:21}
        \ea \url{https://twitter.com/Mandy_Baessa} {(accessed on 1 Jul 2019)}
        \ex \url{https://blogqueideia.wordpress.com/2017/03/20/bolo-de-morango-a-receita/} {(acessado em 1 Jul 2019)}
    \z
    \exr{ex:avelar:22}
        \ea \url{https://beta2.gamevicio.com/noticias/2021/09/resident-evil-3-deve-receber-atualizacao-em-breve/} {(accessed on 30 Sep 2021)}
        \ex \url{https://twitter.com/Neni66576183} {(accessed on 30 Sep 2021)}
    \z
    \exr{ex:avelar:23}
        \ea \url{https://sonhandocomdarcy.wixsite.com/sonhandocomdarcy/single-post/2015/12/10/Top-5-Autores-Que-Eu-Nunca-Li} {(accessed on 1 Jul 2019)}
        \ex \url{https://www.tudocelular.com/samsung/noticias/n142917/analise-samsung-galaxy-a10-review.html} {(accessed on 4 Jul 2019)}
    \z
    \exr{ex:avelar:24}
        \ea \url{https://www.wattpad.com/590285161-visão-de-cria-cap\%C3\%ADtulo-35/page/2} {(accessed on 1 Jul 2019)}
        \ex \url{https://twitter.com/boombapx/status/772478713963372544} {(accessed on 4 Jul 2019)}
    \z
    \exr{ex:avelar:25}
        \ea \url{https://www.facebook.com/PaparazzoRN/photos/a.517262558616376/773477566328206/?type=3 & theater} {(accessed on 3 Aug 2020)}
        \ex \url{https://twitter.com/nadinerv/status/766251351244414976} {(Accessed on 3 Aug 2020)}
        \ex \url{https://www.youtube.com/watch?v=xgWk4Oo7TUU} {(Accessed on 3 Aug 2020)}
    \z
    \exr{ex:avelar:26} \url{https://pandlr.com/forum/22-pan/forum/topic/off-alguem-que-entende-de-twitter-help/?cache=1} {(Accessed on 3 Aug 2020)}
    \exr{ex:avelar:27} \url{https://www.picuki.com/tag/quandovivideverdade} {(Accessed on 30 Sep 2021)}
    \exr{ex:avelar:28} \url{https://curiouscat.me/Caralhouuuuuuuuuuuuu} {(Accessed on 7 Sep 2021)}
    \exr{ex:avelar:29} \url{https://www.campograndenews.com.br/cidades/capital/rapaz-suspeito-de-matar-a-mulher-tem-habeas-corpus-negado-pela-justica} {(Accessed on 30 Sep 2021)}
    \exr{ex:avelar:30} \url{https://www.vakinha.com.br/vaquinha/ajudem-a-luna-igor-romero-dos-santos} {(Accessed on 30 Sep 2021)}
    \exr{ex:avelar:31} \url{https://www.intrinseca.com.br/paratodososgarotosquejaamei/carta/3195/tentando-me-despedir} {(Accessed on 30 Sep 2021)}
    \exr{ex:avelar:32}  \url{https://www.dgabc.com.br/Noticia/2803060/jovens-viajam-por-dez-horas-como-representantes-unicos-de-cidades} {(Accessed on 30 Sep 2021)}
    \exr{ex:avelar:33}  \url{https://www.palmeiras.com.br/pt-br/noticias/torcedores-idolatram-evair-em-loja-oficial-do-palmeiras-de-catanduva/} {(Accessed on 30 Sep 2021)}
    \exr{ex:avelar:34}  \url{https://behappyviagens.com.br/depoimento/131} {(Accessed on 30 Sep 2021)}
    \exr{ex:avelar:36}
        \ea\url{http://praverpralereouvir.blogspot.com/2009/07/atim-sera-que-estou-com-gripe-suina.html} {(Accessed on 30 Sep 2021)}
        \ex \url{https://hangarnet.com.br/showthread.php?tid=107803 & pid=1340162} {(Accessed on 30 Sep 2021)}
    \z
\end{exe}
    
\printbibliography[heading=subbibliography]
\end{document}
