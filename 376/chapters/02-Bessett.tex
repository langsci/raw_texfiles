\documentclass[output=paper]{langscibook}
\ChapterDOI{10.5281/zenodo.8124488}
\author{Ryan M. Bessett\orcid{}\affiliation{University of Texas Rio Grande Valley}}
\title[First person singular subject pronoun expression in Arizona and Texas]
	  {A cross dialectal comparison of first person singular subject pronoun expression in Southern Arizona and Southeast Texas}
\abstract{This study provides a cross-dialectic comparison of first person singular subject pronoun expression in the Spanish varieties of two US-Mexico borderland communities, Southern Arizona and Southeast Texas. Using data collected from sociolinguistic interviews of 32 Spanish/English bilingual speakers, this analysis further explores the impact that trans-frontier practices have on the realization of subject pronouns in border communities and demonstrates the similarities in the variable grammar of the Spanish spoken in the US Southwest. The results show that both Arizona and Texas express first person singular pronouns at a similar rate (19.3\% and 18.7\%, respectively). Additionally, the linguistic factors that condition the variable (switch reference; clause type; tense, mood, and aspect; and whether or not the verb is reflexive) are very similar within each group.}
\IfFileExists{../localcommands.tex}{
  \addbibresource{../localbibliography.bib}
  \usepackage{langsci-optional}
\usepackage{langsci-gb4e}
\usepackage{langsci-lgr}

\usepackage{listings}
\lstset{basicstyle=\ttfamily,tabsize=2,breaklines=true}

%added by author
% \usepackage{tipa}
\usepackage{multirow}
\graphicspath{{figures/}}
\usepackage{langsci-branding}

  
\newcommand{\sent}{\enumsentence}
\newcommand{\sents}{\eenumsentence}
\let\citeasnoun\citet

\renewcommand{\lsCoverTitleFont}[1]{\sffamily\addfontfeatures{Scale=MatchUppercase}\fontsize{44pt}{16mm}\selectfont #1}
   
  %% hyphenation points for line breaks
%% Normally, automatic hyphenation in LaTeX is very good
%% If a word is mis-hyphenated, add it to this file
%%
%% add information to TeX file before \begin{document} with:
%% %% hyphenation points for line breaks
%% Normally, automatic hyphenation in LaTeX is very good
%% If a word is mis-hyphenated, add it to this file
%%
%% add information to TeX file before \begin{document} with:
%% %% hyphenation points for line breaks
%% Normally, automatic hyphenation in LaTeX is very good
%% If a word is mis-hyphenated, add it to this file
%%
%% add information to TeX file before \begin{document} with:
%% \include{localhyphenation}
\hyphenation{
affri-ca-te
affri-ca-tes
an-no-tated
com-ple-ments
com-po-si-tio-na-li-ty
non-com-po-si-tio-na-li-ty
Gon-zá-lez
out-side
Ri-chárd
se-man-tics
STREU-SLE
Tie-de-mann
}
\hyphenation{
affri-ca-te
affri-ca-tes
an-no-tated
com-ple-ments
com-po-si-tio-na-li-ty
non-com-po-si-tio-na-li-ty
Gon-zá-lez
out-side
Ri-chárd
se-man-tics
STREU-SLE
Tie-de-mann
}
\hyphenation{
affri-ca-te
affri-ca-tes
an-no-tated
com-ple-ments
com-po-si-tio-na-li-ty
non-com-po-si-tio-na-li-ty
Gon-zá-lez
out-side
Ri-chárd
se-man-tics
STREU-SLE
Tie-de-mann
} 
  \togglepaper[2]%%chapternumber
}{}

\begin{document}
\maketitle 

\section{Introduction}\label{sec:aztx:1}
Subject pronoun expression (SPE) is a variable context in Spanish, in which a speaker can express a pronoun, as in \REF{ex:aztx:1} or not express a pronoun, as in \REF{ex:aztx:2}, both from the Corpus del Español del Valle (CoBiVa) (\citealt{ChristoffersenBessett2019}).\largerpage

\ea\label{ex:aztx:1}Pues, \ExHighlight{yo} me fui pensando porque para mí- a mí- \ExHighlight{yo} fui- \ExHighlight{yo} era criado con la intención de que a un padrecito nunca se dice que no. Entonces, \ExHighlight{yo} me fui para la casa. (CoBiVa005)\pagebreak
 \glt ‘Well, \ExHighlight{I} went about thinking because for me- for me- \ExHighlight{I} was- \ExHighlight{I} was raised with the intention that one to a parent never says no. So, \ExHighlight{I} went home.’ (CoBiva005)
\ex\label{ex:aztx:2}No, me gusta aquí el Valle. {\ExHighlight{(∅)}} Estoy a gusto. Ya me {\ExHighlight{(∅)}} acostumbré aquí, ya {\ExHighlight{(∅)}} sé cómo – cómo es la gente. (CoBiVa006)
 \glt ‘No, I like it here in the Valley. {\ExHighlight{(∅)}} am happy. {\ExHighlight{(∅)}} already got accustomed here, {\ExHighlight{(∅)}} already know what – what the people are like.’ (CoBiVa006)
\z 

\begin{sloppypar}
An abundance of research has been conducted on this variable and in a wide a range of communities in the United States, including: Arizona (\citealt{Cerrón-Palomino2016, Bessett2018}), California (\citealt{Silva-Corvalán1994, BayleyPease-Alvarez1997}), Florida (\citealt{Hurtado2001}, \citealt{Abreu2009}), Georgia (\citealt{Limerick2017}), New Mexico (\citealt{TorresCacoullosTravis2010a,TorresCacoullosTravis2010b}), New Jersey (\citealt{Flores-Ferrán2007}), and New York (\citealt{Flores-Ferrán2004, OtheguyLivert2007, OtheguyZentella2012}, among others). The center focus of SPE studies in the US has been to examine the possible effects of language contact on SPE in US Spanish, as witnessed in the summary of the previous research that follows.
\end{sloppypar}


One of the most debated features of SPE in US Spanish is the overall frequency with which bilinguals express pronouns, specifically if they do so at a higher rate than monolinguals, since in English, pronouns are expressed at nearly categorical rates. There are some studies that show a higher rate (\citealt{LapidusOtheguy2005, OtheguyLivert2007, OtheguyZentella2012}, among others), and there are others that do not (\citealt{Silva-Corvalán1994, BayleyPease-Alvarez1997, Hurtado2001, Flores-Ferrán2004,Flores-Ferrán2007, TorresCacoullosTravis2010b, TorresCacoullosTravis2010a, Cerrón-Palomino2016, Limerick2017, Bessett2018}, among others).



In order to determine the significance of overall frequency, the extralinguistic factor of community or group has been used as a predictor variable among different groups. For example, in New York, bilinguals who were born in the US produce more expressed pronouns than speakers who were born outside the US (\citealt{OtheguyLivert2007, OtheguyZentella2012}). However, in Arizona, when comparing Mexican vs Arizona born speakers, community was not a significant factor (\citealt{Bessett2018}). Another way to group participants to test frequency differences is through proficiency level and in Arizona English-dominant bilinguals are shown to disfavor expressed pronouns, the opposite of what would be expected if the result were due to contact with English (\citealt{Cerrón-Palomino2016}).



In addition to overall frequency, several linguistic factors have been found to condition SPE.  One of the most robust linguistic factors is the tense, mood, and aspect (TMA) of the verb. On the whole, it is noted that in Spanish, in general, TMA forms that are morphologically ambiguous (those whose form does not expressly provide the subject, like \textit{tuviera} ‘I/(s)he had’ or \textit{hablaba} ‘I/(s)he spoke’) tend to favor expressed pronouns while those forms that do not present morphological ambiguity (like \textit{tengo} ‘I have’ or \textit{hablé} ‘I spoke’), tend to disfavor expressed pronouns (\citealt{CarvalhoShin2015}). This tendency is also borne out in the context of US Spanish (\citealt{OtheguyZentella2012, ShinMontes-Alcalá2014, Cerrón-Palomino2016, Limerick2017}, among others), although there do exist a few exceptions (\citealt{LapidusShin2014, Limerick2019}).



A second linguistic factor is known as the switch reference constraint. This factor group examines the relationship in subject between a verb and the verb that comes before it. Commonly, three levels are considered: same subject, a change in the subject but same object, and a complete change in subject. In Spanish, when there is a complete change in subject in the present verb as compared to the previous one, pronouns are expressed at a higher rate than when the subject stays the same (\citealt{CarvalhoShin2015}, among many others). In bilingual communities in the US, some studies note that difference between same and change in subject happens to a lesser degree than among monolingual speakers (\citealt{OtheguyZentella2012, Limerick2017}). However, in other US bilingual communities, the switch reference constraint shows similar results to monolingual speakers (\citealt{Silva-Corvalán1994, Hurtado2001, TorresCacoullosTravis2018, Cerrón-Palomino2016, Bessett2018}).



Another factor group considered in previous studies is whether the verb is produced with a reflexive pronoun. When speakers produce a verb with a reflexive pronoun, this context has been found to disfavor the use of an expressed pronoun (\citealt{OtheguyLivert2007, Abreu2009, OtheguyZentella2012, Cerrón-Palomino2016}, among others). 



Clause type has also been noted to condition SPE in Spanish. In general, main clauses tend to favor expressed pronouns, while subordinate clauses and coordinate clauses disfavor expressed pronouns (\citealt{Flores-Ferrán2009, OtheguyZentella2012, ShinMontes-Alcalá2014}). However, in some communities, clause type is not a significant predictor variable (\citealt{TorresCacoullosTravis2010a}).



To summarize the overall findings of the realization of SPE in the context of Spanish in the US, the literature is divided into those that show evidence of contact (\citealt{Silva-Corvalán1994, LapidusOtheguy2005, OtheguyLivert2007, OtheguyZentella2012, Limerick2017}, among others) and those that do not (\citealt{Hurtado2001, Flores-Ferrán2004, TorresCacoullosTravis2010b, TorresCacoullosTravis2010a, Cerrón-Palomino2016, Bessett2018}, among others). Interestingly, in general, a pattern emerges, in which the studies on speech communities in the US Southwest tend to show lack of influence of contact for this variable including in New Mexico (\citealt{TorresCacoullosTravis2010b, TorresCacoullosTravis2010a}) and Arizona (\citealt{Cerrón-Palomino2016, Bessett2018}). This brings to the forefront of the discussion the trans-frontier practices from residents on both sides of the US–Mexico border (\citealt{Jaramillo1995, Bessett2015} and \citealt{Cerrón-Palomino2016}) that help to facilitate connectivity between the Spanish spoken on both sides of the border, especially in terms of subject pronoun expression. These results bring about the focus of the present study, which seeks to document this continuity of linguistic behavior along the border, through the analysis of first person-singular subject pronoun expression in Southern Arizona and Southeastern Texas. Previous SPE research on US Spanish includes different variables and different coding practices, making it difficult to compare clearly between communities. In the present study, by comparing two communities with the same factor groups, within group factors and the same researcher (coder), commonalities and differences noted in the results can be taken to better represent the realization of SPE in the communities in general.


\section{Methodology}\label{sec:aztx:2}
\subsection{Participants}\label{sec:aztx:2.1}



The participants in this study consist of 32 bilinguals (Spanish–English) from two bilingual communities along the US–Mexico border, one in Southern Arizona (Tucson and Nogales) and the other in Southeastern Texas (along the border from the McAllen metropolitan area through Brownsville). The participants are equally divided by community with 16 Arizona participants from the \citet{Bessett2012} corpus and 16 from the Corpus Bilingüe del Valle (\citealt{ChristoffersenBessett2019}). They are further divided equally by gender (16 men and 16 women). \tabref{tab:aztx:1} shows the distribution of the participants by community and gender.


\begin{table}
\begin{tabular}{lrr}
\lsptoprule
Gender & \multicolumn{2}{c}{Community}\\\cmidrule(lr){2-3}
	 & Arizona &  Texas\\\midrule
Male &  8 &  8\\
Female &  8 &  8\\
Total &  16 &  16\\
\lspbottomrule
\end{tabular}
\caption{Participants by community and gender\label{tab:aztx:1}}
\end{table}

Both participant groups and corpora represent data from sociolinguistic interviews of about an hour in length. All participants self-evaluated their proficiency. For the Arizona group, the scale was 0 (low proficiency) to 10 (high proficiency) and all participants rated themselves as 6 or higher in both Spanish and English. For the Texas group, the scale was from 0–6 and all participants rated their Spanish and English as 4 or higher. The speakers also demonstrated their bilingual abilities by participating in a conversation in Spanish for the duration of the approximately one hour-long interview, as well as a post interview conversation in English. The participants all also attend(ed) school in the United States (with English instruction), and they live and work in bilingual communities where they are called upon to speak and interact in both languages.


\subsection{Data collection}\label{sec:aztx:2.2}



The first 100 instances of verbs conjugated for the first person singular subject pronoun \textit{yo} were extracted from each of the 32 sociolinguistic interviews.  Limiting the study to one grammatical person is a practice based off the knowledge that each grammatical person is conditioned differently in terms of SPE and has been applied in previous studies (e.g., \citealt{LapidusOtheguy2005, TorresCacoullosTravis2010b, TorresCacoullosTravis2010a, LapidusShin2014, Bessett2018}). The decision to limit the number of tokens to 100 was to provide as equal a sample as possible among the 32 participants. Some participants use first person singular \textit{yo} quite often, while others do not.



Some specific tokens that refer to structures in which the participant does not make the decision on whether or not to express the subject pronoun \textit{yo} were excluded from the analysis. For example, reported speech (see \ref{ex:aztx:3}), where the person who produced the original utterance and not the participant made the decision to express, or not, the subject pronoun.




\ea\label{ex:aztx:3} …y un día me dijo- mi mamá dijo… dijo “{\ExHighlight{({∅}) Tengo}} que ir.” (CoBiVa005)
 \glt ‘…and one day (she) told me- my mom said… (she) said “{\ExHighlight{({∅}) have}} to go.”’ (CoBiVa005)
\z 


Another such case is with set phrases (as in example \ref{ex:aztx:4}), where again the participant does not make the decision to express or not a pronoun.

\ea\label{ex:aztx:4}¿Cómo te {\ExHighlight{({∅}) diré}}? (CoBiVa039)
 \glt ‘How will {\ExHighlight{({∅}) tell you}}?’(CoBiVa039)
\z 

The tokens were coded for the dependent variable of subject pronoun expression (expressed, unexpressed) as well as the four linguistic factors that have proven to condition the variable, as outlined in the introduction. For the factor group of “tense, mood, and aspect” (TMA), the categories were simplified to two, ambiguous TMA morphology (see example \ref{ex:aztx:5}) and unambiguous TMA morphology (see example \ref{ex:aztx:6}), following \citet{Cerrón-Palomino2016}, based on the well attested pattern of TMA outlined in the introduction. This decision also makes a comparison between the results of the present study and previous US Spanish communities more straightforward. 

\ea\label{ex:aztx:5} …para cuando {\ExHighlight{yo} \ExHighlight{estaba}} chiquillo en la escuela… (CoBiVa005)
 \glt `…for when {\ExHighlight{I} \ExHighlight{was}} little in school…' (CoBiVa005)
\ex\label{ex:aztx:6} …y vino un señor ahí a la casa a c- a curar la vaca y lo {{\ExHighlight{(∅)}} \ExHighlight{miré}} como le hizo… (CoBiVa005)
 \glt `…and a man came there to the house to c- cure the cow and {{\ExHighlight{(∅)}} \ExHighlight{saw}} how he did it…' (CoBiVa005)
\z 

Next, the factor group of “switch reference” documented the relationship of the current verb to the previous verb and consisted of three categories, coreference with subject (no switch, see example \ref{ex:aztx:7}), switch in subject but coreference with object (see example \ref{ex:aztx:8}), and a switch with the subject and all objects (complete switch, see example \ref{ex:aztx:9}). 

\ea\label{ex:aztx:7}…no puedo comer comida de México cuando {{\ExHighlight{(∅)}} \ExHighlight{estoy}} acá. (CoBiVa006)
 \glt ‘…I can’t eat Mexican food when {{\ExHighlight{(∅)}} \ExHighlight{am}} here.’ (CoBiVa006)
\ex\label{ex:aztx:8}No, me gusta aquí el Valle. {{\ExHighlight{(∅)}} \ExHighlight{estoy}} a gusto. (CoBiVa006)
\glt ‘No, I like it here in the Valley. {{\ExHighlight{(∅)}} \ExHighlight{am}} content.’ (CoBiVa006)
\ex\label{ex:aztx:9}No es una mala opción, ¿verdad?, pero {\ExHighlight{yo} \ExHighlight{creo}} que… (CoBiVa006)\\
 \glt ‘It’s no a bad option, right?, but {\ExHighlight{I} \ExHighlight{think}} that…’ (CoBiVa006)
\z 

A third factor group, “reflexive verb”, coded for if the verb was conjugated with a reflexive pronoun (see example \ref{ex:aztx:11}) or without a reflexive pronoun (see example \ref{ex:aztx:10}).\largerpage

\ea\label{ex:aztx:10}…ya concordamos más las opciones y de que {{\ExHighlight{(∅)}} \ExHighlight{voy} \ExHighlight{a} \ExHighlight{hacer}} eso… (T01)
\glt ‘…we already agreed the options more and that {{\ExHighlight{(∅)}} \ExHighlight{am} \ExHighlight{going} \ExHighlight{to} \ExHighlight{do}} that…’ (T01)
\ex\label{ex:aztx:11}…si no era que me \underline{{\ExHighlight{(∅)}} \ExHighlight{quedaba}} dormida en una casa de una amiga… (T01)
 \glt ‘…if it wasn’t that {{\ExHighlight{(∅)}} \ExHighlight{fell}} asleep at a friend’s house…’ (T01)
\z 

Lastly, “clause type” coded for the type of clause in which the verb was located. This factor group was divided into the following three categories: main clause (defined as an isolated clause, a clause between pauses, or one that had a subordinate clause) as seen in example \REF{ex:aztx:12}, coordinate clause (which included a string of main clauses) as seen in example  \REF{ex:aztx:13}, and subordinate clauses (see example \ref{ex:aztx:14}).

\ea\label{ex:aztx:12}Pues, {{\ExHighlight{(∅)}} \ExHighlight{tengo}} un hermano mayor que yo… (T02)
 \glt ‘…Well, {{\ExHighlight{(∅)}} \ExHighlight{have}} and older brother…’ (T02)
\ex\label{ex:aztx:13}De chiquita no prestaba mucha atención a cómo era Tucson y luego como {{\ExHighlight{(∅)}} \ExHighlight{he} \ExHighlight{notado}} muchos cambios… (T02)
 \glt ‘…As a little girl I didn’t pay much attention to what Tucson was like and later like {{\ExHighlight{(∅)}} \ExHighlight{have} \ExHighlight{noticed}} a lot of changes…’ (T02)
\ex\label{ex:aztx:14}Yo siempre era muy floja, entonces no es que no {{\ExHighlight{(∅)}} \ExHighlight{pudiera} \ExHighlight{hacer}} el trabajo… (T02)
 \glt ‘…I was always very lazy, so it wasn’t that {{\ExHighlight{(∅)}} \ExHighlight{couldn’t} \ExHighlight{do}} the work…’ (T02)
\z 


Once the data was coded, it was analyzed through a multivariate model using the statistical program GoldVarb for Mac (\citealt{SankoffSmith2018}).


\section{Results}\label{sec:aztx:3}

This section presents the results of the comparison in the realization of first person singular SPE between the Arizona and Texas communities. To this end, \sectref{sec:aztx:3.1} provides a summary of the overall frequency of SPE by community and compares these findings to other US Southwest communities. Next, \sectref{sec:aztx:3.2} presents an overview of the constraint hierarchy for the linguistic factor groups between the Arizona and Texas communities. Then, \sectref{sec:aztx:3.3} explores a detailed analysis of the factors that condition SPE in the Arizona and Texas and communities and discusses the similarities and differences between the two.



\subsection{Overall frequency of SPE in Arizona and Texas}\label{sec:aztx:3.1}\largerpage


The overall frequency with which the Arizona and Texas speakers produce an expressed pronoun is roughly the same, 19.3\% (274 of a total of 1,423 tokens) for Arizona bilinguals and 18.7\% (299 of a total of 1,600 tokens) for the Texas bilinguals. \tabref{tab:aztx:2} shows this pattern. 

\begin{table}
\caption{Overall frequency of first person singular SPE in Arizona and Texas Spanish\label{tab:aztx:2}}
\begin{tabular}{lcc}
	\lsptoprule
	        & {\%} & {$n$}\\\midrule
	Arizona & 19.3 & 274/1423\\
	Texas   & 18.7 & 299/1600\\
	\lspbottomrule
\end{tabular}
\end{table}

These results indicate a first parallel between the two communities as well as to the Phoenix, Arizona community whose speakers produce expressed pronouns at a rate of 23.2\% (\citealt{Cerrón-Palomino2016}). While there are clear parallels between Arizona and Texas, in terms of overall frequency, the speakers in New Mexico (\citealt{TorresCacoullosTravis2010b}) as well as California (\citealt{Silva-Corvalán1994}) show higher rates of 32\% and 34.7\%, respectively, diverging from the pattern. However, these rates all fall well within the range of monolingual Mexican communities which run from 16.7\% in Sonora (\citealt{Bessett2018}) to 33\% in Xalapa (\citealt{Orozco2016}), showing similar patterns of subject pronoun expression in the Mexican Spanish spoken on both sides of the US-Mexico border. It is also important to mention, as \citet{Travis2007} warns, overall frequency can be misleading, and so it is crucial to determine whether or not this finding is significant. One way to establish this is to use community as a factor group in the logistic regression and determine if the factor is selected by the model. \tabref{tab:aztx:3} shows the factor groups and their ranges for the factors that were selected by GoldVarb when using community (Arizona vs Texas) as a factor group.

\begin{table}
\caption{Constraint hierarchy of the linguistic factor groups with community (Arizona/Texas) as a factor for the probability of an expressed \textit{yo}\label{tab:aztx:3}}
\begin{tabular}{ll}
	\lsptoprule
	{Factor group} & {Range}\\\midrule
	{Switch reference} & {38}\\
	{Clause Type} & {32}\\
	{TMA} &  {21}\\
	{Reflexive} & {10}\\
	& \\
	\multicolumn{2}{c}{{Log likelihood = $-1242.309$}}\\
	\multicolumn{2}{c}{{$p < 0.05$}}\\
	\lspbottomrule
\end{tabular}
\end{table}

Community (Arizona/Texas) is not among the factors that condition SPE expression in this data set. This suggests that the realization of the variable in these two communities is similar. In order to further explore this idea, \sectref{sec:aztx:3.2} and \sectref{sec:aztx:3.3} discuss the constraint hierarchies of the factors that condition first person subject pronoun expression in the Arizona and Texas communities in separate regression models.



\subsection{Constraint hierarchy of the linguistic factor groups}\label{sec:aztx:3.2}

Having discussed the overall frequency, it is important to compare the linguistic factors that condition SPE in Arizona and Texas. The first measure is to evaluate the hierarchy of the factor groups, in accordance with comparative sociolinguistics which maintains that similarities in the constraint ranking can indicate a common origin for the pattern of the given structure in the communities (\citealt{Tagliamonte2003}). \tabref{tab:aztx:4} shows the hierarchy of factor groups and range for both communities.


%%please move \\begin{table} just above \\begin{tabular
\begin{table}
\begin{tabular}{lc c lc}
\lsptoprule

\multicolumn{2}{c}{Arizona} &  & \multicolumn{2}{c}{Texas}\\\cmidrule(lr){1-2}\cmidrule(lr){4-5}
{Factor group} & {Range} &  & {Factor group} & {Range}\\\midrule
{Switch/Reference} & {38} & {=} & {Switch Reference} & {39}\\
{Clause Type} & {35} & {=} & {Clause Type} & {31}\\
{TMA} & {17} & {=} & {TMA} & {24}\\
{Reflexive} & {16} & {=/${\neq}$} & {Reflexive} & {[7]}\\
&  &  &  & \\
\multicolumn{2}{c}{Log likelihood = $-582.725$} &  & \multicolumn{2}{c}{Log likelihood= $-654.674$}\\
\multicolumn{2}{c}{$p < 0.05$} & {~} & \multicolumn{2}{c}{$p < 0.05$}\\
\lspbottomrule
\end{tabular}
\caption{Factor groups that condition first person singular SPE in Arizona and Texas}
\label{tab:aztx:4}
\end{table}

Both the Arizona and Texas groups show switch reference, followed by clause type and then TMA as the significant factor groups. There is a discrepancy between the two data sets in that reflexive is significant for Arizona, but not Texas. However, in both cases, this factor group is the lowest ranking factor. In general, the two groups are strikingly similar in terms of the constraint hierarchy that conditions first person singular SPE. When comparing this to other US Southwest communities, we see more similarities. In Phoenix, Arizona (\citealt{Cerrón-Palomino2016}), while clause type was not included, switch reference was followed by TMA and then reflexive. Additionally, in New Mexico (\citealt{TorresCacoullosTravis2010a}), switch reference was followed by TMA (ambiguity of verb morphology). However, for this community clause type was not significant. Again, there seem to be strong similarities between the Arizona and Texas communities of the current study, and possibly with the previously studied Arizona (\citealt{Cerrón-Palomino2016}) and New Mexico (\citealt{TorresCacoullosTravis2010a}) communities. To better understand this relationship, \sectref{sec:aztx:3.3} will explore the constraint ranking within each factor group.



\subsection{Linguistic factors that condition first person SPE}\label{sec:aztx:3.3}\largerpage

\begin{sloppypar}
While \sectref{sec:aztx:3.2} suggests parallels between the factor groups that condition SPE among the Arizona and Texas speakers of this study, this section now turns to the more detailed analysis of the within-group factors and the directionality of their effects. \tabref{tab:aztx:5} outlines these results.
\end{sloppypar}


%%please move \\begin{table} just above \\begin{tabular
	\begin{table}
	\begin{tabular}{l rrc rrc}
		\lsptoprule
		{Factor} & \multicolumn{3}{c}{{Arizona}}  & \multicolumn{3}{c}{{Texas}}\\
		\cmidrule(lr){2-4}\cmidrule(lr){5-7}
		   & {FW} & {\%} & {$n$} &  {FW} & {\%} & {$n$}\\\midrule
        {Switch Reference} \\
		{Complete switch} & {0.70} & {32.1} & {207/645} &  {0.70} & {30.8} & {208/676}\\
		{Same object} & {0.35} & {8.7} & {8/92} &  {0.56} & {20.8} & {33/159}\\
		{Same subject} & {0.32} & {8.6} & {59/686} &  {0.31} & {7.6} & {58/765}\\
		& \multicolumn{2}{c}{{Range = 38}} &   & \multicolumn{2}{c}{{Range = 39}} & \\
		
		{Clause type}\\
		{Main} & {0.75} & {39.3} & {127/323} & {0.71} & {33.2} & {153/461}\\
		{Coordinate} & {0.42} & {12.9} & {112/870} & {0.40} & {11.2} & {94/836}\\
		{Subordinate} & {0.40} & {15.2} & {35/230} & {0.46} & {17.2} & {52/303}\\
		& \multicolumn{2}{c}{{Range = 35}} &  & \multicolumn{2}{c}{{Range = 31}} & \\
		
		{TMA}\\
		{Ambiguous} & {0.63} & {23.4} & {81/346} & {0.69} & {27.2} & {84/309}\\
		{Not ambiguous} & {0.46} & {17.9} & {193/1077} &{0.45} & {16.7} & {215/1291}\\
		& \multicolumn{2}{c}{{Range = 17}} &  & \multicolumn{2}{c}{{Range = 24}} & \\
		
		{Reflexive}\\
		{Not reflexive} & {0.52} & {20.3} & {253/1244} & {[0.51]} & {19.4} & {266/1372}\\
		{Reflexive} & {0.36} & {11.7} & {21/179} & {[0.44]} & {14.5} & {33/228}\\
		& \multicolumn{2}{c}{{Range = 16}} &  & \multicolumn{2}{c}{{Range = [7]}} & \\\tablevspace
		
		& \multicolumn{3}{c}{{Log likelihood = $-582.725$}} & \multicolumn{3}{c}{{Log likelihood = $-654.674$}}\\
		{~} & \multicolumn{3}{c}{{$p < 0.05$}} &  \multicolumn{3}{c}{{$p < 0.05$}}\\
		\lspbottomrule
	\end{tabular}
	\caption{Comparison of the linguistic factors that condition the expression of first person singular pronouns in Arizona and Texas. FW = factor weight.\label{tab:aztx:5}}
\end{table}

First, for both groups, switch reference is the highest ranking factor group. A complete switch from the previous subject highly favors an expressed pronoun in both communities and to the same degree (FW = 0.70), while a coreferential subject disfavors an expressed pronoun to a similar degree in both communities with a factor weight of 0.32 for Arizona and 0.31 for Texas. In both communities, a switch in the subject but with a coreferential object is ranked second, but this context disfavors an expressed pronoun in Arizona (FW = 0.35) while it slightly favors an expressed pronoun in Texas (FW = 0.56). These results, in addition to following the well-attested pattern of Spanish in general (\citealt{CarvalhoShin2015}), are also seen in New Mexico (\citealt{TorresCacoullosTravis2010a}) as well as in Phoenix, Arizona (\citealt{Cerrón-Palomino2016}). 



The second highest ranked factor group, clause type, shows further parallels between the two groups. Main clauses favor expressed pronouns for both Arizona (FW = 0.75) and Texas (FW = 0.71) speakers, while coordinate and subordinate clauses disfavor expressed pronouns. However, in Texas coordinate clauses rank lower (FW = 0.40) than subordinate clauses (FW = 0.46), while in Arizona the pattern is flipped, and subordinate clauses rank lower (FW = 0.40) than coordinate clauses (FW = 0.42). In terms of percent of expressed pronouns in coordinate and subordinate clauses however, the pattern is the same for the two communities. In Arizona, first person singular verbs are produced with an expressed pronoun in subordinate clauses at a rate of 15.2\% (35/230) while in coordinate clauses pronouns are expressed at a rate of 12.9\% (112/870). In Texas, the rates are 17.2\% (52/303) in subordinate clauses and 11.2\% (94/836) in coordinate clauses. Overall, in this context there are apparent similarities between the two communities.



The next factor group is TMA which is separated into ambiguous and not ambiguous verb morphology. Among Arizona speakers, ambiguous verb morphology favors expressed pronouns at a rate of 0.63, similar to the 0.69 rate for Texas speakers, while unambiguous verb morphology disfavors expressed pronouns in both Arizona (FW = 0.46) and Texas (FW = 0.45). These results conform to the general pattern noted for Spanish in general (\citealt{CarvalhoShin2015}) and, more specifically, to the Phoenix, Arizona (\citealt{Cerrón-Palomino2016}) and New Mexico (\citealt{TorresCacoullosTravis2010a}) communities. 



The last factor group is the presence or absence of a reflexive pronoun. The factor group is only significant for the Arizona community. However, the distribution of the effect is the same for both communities, the absence of a reflexive pronoun favors expressed subject pronouns in Arizona (FW = 0.52) as well as in Texas (FW = [0.51]), while the presence of a reflexive pronoun disfavors expressed subject pronouns in Arizona (FW = 0.36) and Texas (FW = [0.44]). In Phoenix, Arizona, much like the Arizona speakers in the current study, reflexive is a significant factor group (\citealt{Cerrón-Palomino2016}) and the absence of a reflexive pronoun favors expressed subject pronouns. With this factor group we again see continuities in the realization of SPE within the US Southwest.



\section{Conclusion}



This study provided a cross-dialectal comparison of first person singular subject pronoun expression in two border communities in the US Southwest, Southern Arizona and Southeastern Texas. Overall, the results indicated clear ties between the Arizona and Texas speech communities under consideration in this study. The overall frequency was nearly identical for the two groups, with the Arizona participants expressing first person singular pronouns at a rate of 19.3\%, while the rate for Texas participants is 18.7\%. This rate was also similar to the Phoenix, Arizona (\citealt{Cerrón-Palomino2016}) community as well as to the slightly higher rates of New Mexico (\citealt{TorresCacoullosTravis2010b}) and California (\citealt{Silva-Corvalán1994}) bilinguals. When taking into account the rates of expressed pronouns in Monolingual Mexican communities from 16.7\% in Sonora (\citealt{Bessett2018}) to 33\% in Xalapa (\citealt{Orozco2016}), US border communities fit well within the range, demonstrating a similarity. Additionally, the constraint hierarchy was identical for both Arizona and Texas speakers, switch reference, clause type, TMA and then reflexive (although reflexive was not significant for Texas bilinguals). The within-group factors were also ordered in the same way, with the exception of the inverse order of coordinate and subordinate clauses (although in both communities the two contexts disfavor expressed pronouns). First person singular subject pronoun expression appears to be conditioned in the same way in both Arizona and Texas. These results demonstrate continuity in the Spanish spoken in the two communities (Arizona and Texas) and when compared to previous studies, a pattern of continuity emerges among US Southwest bilinguals and within the Spanish spoken on both sides of the US-Mexico border. By examining these patterns with other variables, future studies may be able to demonstrate an overall pattern of cohesion in the Spanish spoken along the US-Mexico border.


\printbibliography[heading=subbibliography]
\end{document}
