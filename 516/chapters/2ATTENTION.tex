\chapter{Attention in the auditory realm}
\label{ch:2}

\textit{Attention} is one of the most prevalent concepts in (neuro)cognition, yet also one of the most diverse and deceptive terms, as researchers have attributed a plethora of phenomena to this concept \citep[for discussion, see][]{hommel_no_2019}. According to \citet[403--404]{james_principles_1890}, attention ``is the taking possession of the mind, in clear and vivid form, of one out of what seem several simultaneously possible objects or trains of thoughts. Focalisation, concentration of consciousness are of its essence. It implies withdrawal from some things in order to deal effectively with others, and is a condition which has a real opposite in the confused, dazed, scatterbrained state...''. \citet{petersen_attention_2012}, revisiting their \citeyear{posner_attention_1990} paper, propose that attention is an independent system, neuroanatomically different from other cognitive systems, such as incoming stimulus management, decision making, and output production, among others, which, of course, interact with attention and can potentially affect it. As stated in \citet{petersen_attention_2012}, the attention system consists of three neural networks, each one responsible for distinct attentional processes. These are:

\begin{enumerate}[label=\roman*.]
	\item the \textit{alerting} network, controlling arousal and vigilance,
	\item the \textit{orienting} network, regulating sensory processing, and
	\item the \textit{detection} network, executing target detection, thus also known as the \textit{executive} network.
\end{enumerate}
Although these three networks are responsible for different processes, they depend on each other and cooperate, as the activation of one might lead to the activation of another. 

\citet{naatanen_attention_1992}, in his book \textit{Attention and Brain Function}, describes some of the principal uses of attention, aiming to arrive at a better understanding of this broad construct. He begins by writing about \textit{selective} attention, which, as he mentions, occurs when we, based on our interests and goals, select to attend one of the many available input sources. To illustrate selective attention, he invokes the well-known cocktail party example, in which many people are talking simultaneously, but listeners can select to attend to the voice of one specific speaker. Another example of selective attention is a listener's ability to attend to the linguistic meaning transmitted by a speaker's voice, or the ability to attend to a speaker's voice just as a simple sound event, while disregarding the linguistic message that it is carrying. Thus, given listeners' interests, selective attention can be oriented to a variety of aspects, from simple physical properties, to complex stimuli, as well as to meaningful aspects. \citeauthor{naatanen_attention_1992} further describes \textit{divided} attention, i.e., the process of simultaneously attending to different streams of input which are usually dichotically presented: one stream of input is presented to one ear and another stream of input is presented to the other ear. Another phenomenon that has been attributed to the construct of attention is \textit{sustained} attention, which emerges when listeners are asked to detect, through time, infrequent, unpredictable, or weak events. In all these cases described in \citet{naatanen_attention_1992}, a form of \textit{active} or \textit{voluntary} attention is involved. Active/voluntary attention is the listeners' ability to choose their object of attention. Thus, voluntary attention shields the processing of information that we have chosen to process from other available but currently irrelevant information. Yet, some events, for example unexpected, abrupt, loud, or rare sounds, outside of the current attentional focus, may break through this shield, involuntarily drawing our attention away from the current focus. This switch of attentional resources has been called \textit{passive attention} \citep{james_principles_1890}, \textit{orienting reflex} \citep{sokolov_higher_1963, sokolov_perception_1963}, or \textit{involuntary attention orienting} \citep[e.g.,][]{naatanen_mismatch_2019}, among others. Involuntary attention usually lasts until the stimulus, which initiated the attention switch, is recognised and evaluated, after which attention is drawn back to the initial focus. However, after stimulus evaluation, it can very well be the case that involuntary attention changes to voluntary attention, thus bringing the stimulus to conscious processing. The distinct nature of involuntary and voluntary attention is at the core of this book, and specifically, the neural path from involuntary to voluntary attention orienting is investigated and discussed extensively in Chapter \ref{ch:4}. 

This book is concerned with the concept of attention as an orienting response, involving different processing stages. It is important to clarify that attention orienting is regulated by all three networks proposed by \citet{petersen_attention_2012}. Over the course of this chapter, we will see that attention orienting is a composite concept whose individual parts pertain to the distinct neural networks. \sectref{sec:2.1} elaborates on the origins of the attention orienting concept, and presents the two main accounts on the mechanism deemed to underpin attention orienting. \sectref{sec:2.2} elucidates the difference between the involuntary and voluntary nature of attention shifts through a review from very early to the most recent literature. Subsequently, \sectref{sec:2.3} is concerned with the auditory processing system and the stages in which involuntary and voluntary attention emerge, reviewing a model of the brain indices that underlie these processing stages. \sectref{sec:2.4} continues with a review of studies on the relevance of the acoustic properties of the sensory input for attention orienting, demonstrating the special role of signals involving a rise in amplitude or pitch, the latter being the focus of the present book. Finally, \sectref{sec:2.5} presents methods with which researchers have studied attention orienting, concentrating on event-related potentials and pupillometry as the two methods that are employed in the current work (see Chapters \ref{ch:4} and \ref{ch:5}, respectively).

\section{What is attention orienting?}
\label{sec:2.1}

In everyday life, we are confronted with an influx of sounds from several different sources. Some of these sounds might be unexpected, rare, or new. Our brains are equipped with mechanisms that can detect unexpected, rare, or new sounds, i.e., sounds that deviate in some property from the current auditory environment. Such deviant sounds may capture attention, prompting shifts of attentional resources in the baseline neural activity. These attentional shifts result from the operation of the orienting network, and can consequently elicit automated motor actions or appropriate adaptive responses if necessary \citep[e.g.,][]{bach_rising_2008}. Therefore, in a stimulus-rich environment, sensitivity to auditory changes and evaluation of their importance are essential properties of the neurocognitive system for both survival and communication. For instance, a failure to respond to an important sound change might be life-threatening, and responding to every sound change might exhaust mental resources which are needed for other purposes. In spoken communication, this could entail orienting a listener’s attention towards the most important part of the message, which is crucial for effective interpretation and speech planning. 

The concept of \textit{attention orienting} has its roots in the seminal work of Ivan \citeauthor{pavlov_conditioned_1927}. In \citeyear{pavlov_conditioned_1927}, \citeauthor{pavlov_conditioned_1927} described the concept of an \textit{orienting response} (OR), referring to it as a response with two stages: a reflex-like response towards a change in the current environment, which is followed by a perceptual/conscious processing of this change. Specifically, \citet[][12]{pavlov_conditioned_1927} defines attention orienting as follows: ``it is this reflex which brings about the immediate response in man and animals to the slightest changes in the world around them, so that they immediately orientate their appropriate receptor organ in accordance with the perceptible quality in the agent bringing about the change, making a full investigation of it''. \citeauthor{pavlov_conditioned_1927} thus proposed an association between the initiation of the OR and an early processing stage. During this processing stage, the organism is warned to process an unexpected or new event quickly and efficiently, but the properties of that event are not yet fully processed. At a later processing stage, it is possible to obtain full awareness of this event.

Over the years, many scholars have contributed to the development of \citeauthor{pavlov_conditioned_1927}'s OR concept. For example, in the latter half of the 20\textsuperscript{th} century, \citet[4]{posner_orienting_1980} defined attention orienting as the “alignment of attention with a source of sensory input or an internal semantic structure stored in memory”. \citeauthor{posner_orienting_1980} further differentiates orienting from \textit{detecting}, a distinct cognitive act, in that the former allows the listener to link responses to the input before it has been consciously processed, while the latter refers to conscious processing of the input. As discussed in \citet{naatanen_attention_1992}, views like \citeauthor{pavlov_conditioned_1927}'s and \citeauthor{posner_orienting_1980}'s suggest that the OR is evoked by early processes which lead to attention switch, also known as passive attention, or involuntary attention orienting. \sectref{sec:2.3} elaborates on the different processing stages of attention, and Chapter \ref{ch:4} provides experimental evidence for the link between the pre-attentive processing stage and involuntary attention orienting.

Many studies have further attempted to unravel the mechanism that underpins OR. This work paved the way for the development of many accounts, among them the \textit{novelty detection}, and the \textit{expectancy violation} accounts. The novelty detection account suggests that attention orienting is driven by the detection of a novelty in the current environment, and is rooted in theories which advocate for the role of habituation in OR \citep[e.g.,][]{sokolov_higher_1963, sokolov_perception_1963, cowan_attention_1998}. The OR leading theory by \citet{sokolov_neuronal_1960} substantially advanced \citeauthor{pavlov_conditioned_1927}'s concept. Briefly, \citeauthor{sokolov_neuronal_1960} conceives of orienting as a reflexive response,\footnote{Although \citeauthor{sokolov_neuronal_1960}'s reflex relates to \citeauthor{posner_orienting_1980}'s cognitive act of orienting, it does not allow for a distinction between involuntary and voluntary attention alignment.} reflected in physiological activities, such as heart and respiratory rates; electrodermal, vasoconstrictive, pupillary, and muscular reactions; eye, lid, ear, head, and torso movements; or neural changes in the autonomic and/or central nervous system (CNS), among others. He further assumes that the repetition of an originally new event gradually fabricates a memory of its physical properties, resulting in habituation of the OR. This memory has been referred to as the \textit{neuronal model} and is fundamentally important for the novelty detection system: \textit{only} an event that does not match the already-formed neuronal model, i.e., a novel event, has the power to elicit an OR \citep[e.g.,][]{reisberg_overcoming_1980, gati_novelty_1990, elliott_habituation_2001}; otherwise, OR is inhibited. 

Despite the simplicity of the novelty detection account in explaining OR activation, this account has received a lot of criticism due to its limitation in addressing and explaining all cases of an OR activation. For instance, an OR activation caused by a familiar but highly relevant (i.e., self-relevant) stimulus, such as one's name in an unattended conversation \citep[e.g.,][]{lynn_attention_1966}, cannot be justified by the novelty detection mechanism: a stimulus is rendered familiar precisely because its memory trace (i.e., neuronal model) has already been formed. Thus, according to the novelty detection account, its presentation should not lead to an OR activation, given that \textit{only} a stimulus that does not match the already-formed neuronal model has the potential to elicit an OR. Nor can this mechanism explain why an OR is not elicited by a sequence consisting of completely different but sequentially ordered stimuli such as  ``1 2 3 4 5 6 7 8...'', although in this case a neuronal model for each new digit is missing. Nonetheless, an OR is elicited by the stimulus ``5'' in a sequence like ``1 2 3 \textcolor{red}{5} 6 7 8...'' \citep{unger_habituation_1964}. Lastly, the novelty detection account cannot explain an OR activation by a stimulus which is neither new nor different or relevant, such as for example the stimulus ``7'' in a sequence like ``1 2 3 4 5 6 7 \textcolor{red}{7}...'' \citep{velden_necessary_1978}. In these two last examples, an OR activation appears to arise from interruptions of a sequence generating expectations as to the upcoming item. The ``novelty" here, if you like, lies in the unexpected items, which are not the next digits in the sequentially ordered numeric lists.

Given the limitations of the novelty detection mechanism to account for the aforementioned examples, another strand of research suggests that the mechanism underlying the attention orienting response is driven by expectancy violations \citep[e.g.,][]{naatanen_role_1990, sussman_dynamic_2001, vachon_broken_2012}. The expectancy violation account thus claims that the sensory processing system develops expectations by detecting regularities (local or global) in the environment. When a deviant sound occurs instead of an anticipated event, it attracts attention, initiating an orienting response. Thus, in this view, the precondition for an OR is the violation of rule-based expectancies and not just the absence of a memory. A large number of experimental studies have provided evidence in favour of the expectancy violation account as being the mechanism that underlies attention orienting \citep[for review, see][]{paavilainen_mismatch-negativity_2013}. Violations of rule-based expectancies have also been referred to as \textit{prediction errors} \citep[e.g.,][]{friston_free-energy_2010}. According to \citet{friston_free-energy_2010}, prediction errors arise because we build an internal model of the world and rely on this model to predict upcoming sensory events. These prediction errors drive top-down expectations towards mental representation updating \citep{friston_does_2018}. As mentioned in \citet{friston_does_2018}, sensory attention has been associated with noteworthy prediction errors. The expectancy violation basis of attention orienting is further discussed in \sectref{sec:2.3}.

Despite the different mechanisms thought to underlie OR in the two accounts outlined above, many studies tend to conflate the concepts of novelty and deviance \citep[see][]{tiitinen_attentive_1994}. This might be because the concept of novelty is not really independent from the concept of deviance, as a new or unfamiliar event typically also violates expectations, for example the stimulus ``2'' in  a sequence such as ``1 1 1 1 1 \textcolor{red}{2}''. Overall, as \citet{naatanen_orienting_1979} summarises, it appears that an OR may actually be initiated by very different aspects of stimuli, such as neuronal mismatches, task relevance, levels of significance, and anticipatory mismatches. \citet{corbetta_control_2002} add to the idea that attention is regulated by different stimulus aspects by stating that both cognitive and sensory cues (as well as their interaction) affect attention. Novelty and unexpectedness are two factors impacting attention that reflect the interaction between cognitive and sensory cues. More specifically, \citet[][201]{corbetta_control_2002} posit that ``unexpected, novel, salient and potentially dangerous events take high priority in the brain, and are processed at the expense of ongoing behaviour and neural activity".

A discrepant stimulus, or, in other words, a stimulus that deviates in some way from the recent past, can thus evoke an orienting response. This attention orienting response finds expressions in various (psycho)physiological, neural, motoric, and even behavioural reactions \citep[e.g.,][]{hughes_disruption_2007, liao_human_2016, marois_eyes_2018, marois_is_2019, naatanen_mismatch_2019}. These reactions have been reported to be indices of the OR. They may occur in different post stimulus stages, thereby reflecting qualitatively distinct types of OR. Going back to \citeauthor{pavlov_conditioned_1927} and to later work by \citeauthor{posner_orienting_1980}, we can see that both scholars linked the OR with two distinct stages: the pre-attentive stage, related to early processes, and the conscious stage, related to awareness and longer-latency processes. \citeauthor{maltzman_orienting_1971} (\citeyear[e.g.,][]{maltzman_orienting_1971, maltzman_orienting_1979}) also proposed a distinction between involuntary and voluntary ORs, a distinction rooted in the work by \citeauthor{james_principles_1890} in \citeyear{james_principles_1890}. In the next section, I review early theoretical work on the distinction between involuntary and voluntary attention orienting as well as some experimental work that lends credence to this distinction.

\section{Involuntary and voluntary attention orienting}
\label{sec:2.2}

One of the central topics of the attention literature has been the qualitative difference between the involuntary and voluntary nature of attention shifts. According to \citet{hatfield_attention_1998}, speculations about the nature of attentional shifts go back to ancient Greek philosophers and writers of the 4\textsuperscript{th} century who observed that an object or a sound can involuntarily attract attention due to sensory interest. \citet{wright_orienting_2008} mention that Descartes also differentiated between the case of paying voluntary attention to an event of cognitive interest and the case of an event attracting attention involuntarily by causing ``admiration''. Descartes further talked about an inverse proportionality between the ability to withhold voluntary attention directed towards an object or a task within the current focus and the intensity of a distracting event outside the current focus, like a loud sound or a pronouncedly novel event: the more intense a distracting event is, the more difficult it is to sustain the processing of the currently attended event.

The literature review of early work on attention reported in \citet{braunschweiger_lehre_1899} categorised attentional shifts into two domains: intellectual and sensory. \citet{james_principles_1890} discussed attentional processing in these two domains, linking the intellectual domain with voluntary shifts of attention, and the sensory one with involuntary attention shifts. As mentioned in \citet[10]{wright_orienting_2008}, for \citeauthor{james_principles_1890}, voluntary attention entails intentional and conscious preparation of the cognitive processing system, while involuntary attention features an accommodating or adjusting response of sensory organs. \citeauthor{james_principles_1890} also used the term \textit{passive attention} to describe involuntary shifts, as he reasoned that an unattended stimulus, by virtue of it being intense or sudden, can capture attention automatically. Similarly to Descartes, \citeauthor{james_principles_1890} wrote about an inverse relation between interest and attention, in that subsiding interest towards an event renders attention more vulnerable to a switch towards unexpected, rare, salient, or novel events.

Introspective and \textit{Gestalt} psychologists also distinguished between involuntary and voluntary attention. \citet[][]{wertheimer_untersuchungen_1923}, for example, claimed that voluntary attention emerges from voluntary concentration, while involuntary attention arises from stimulus-driven properties. Other scholars described this distinction by suggesting that voluntary attention is directed from one's self towards an event, while involuntary attention writes in the opposite direction: from an event towards one's self \citep[e.g.,][]{koffka_principles_1935}.

Let us now move to some more recent work. \citeauthor{maltzman_orienting_1979} reports in his \citeyear{maltzman_orienting_1979} paper that conditioning of the galvanic skin response (GSR) functions as one proxy of the OR. A basic hypothesis in the classical conditioning of GSR in humans has been the distinction of involuntary and voluntary ORs, resulting from  different conditions. The idea is that an involuntary OR is elicited by the initial presentation of an unpredicted stimulus, while a voluntary OR is evoked by task instructions, leading to problem solving or goal-oriented thinking \citep{maltzman_orienting_1968}. In a series of studies, Maltzman and colleagues \citep[e.g.,][]{maltzman_effects_1965, maltzman_orienting_1971}, using habituation experiments and measuring GSR as a proxy for OR, provided evidence for the distinction between involuntary and voluntary attention. Briefly, in these studies, two groups of participants were engaged. One group was instructed to respond to a particular word, while the other group did not receive any instructions, serving as controls. The results show that both groups responded to the initial presentation of the unpredicted stimulus with the same GSR effect magnitude. However, the group that received the instructions exhibited increased GSRs in relation to the multiple presentations of the stimulus compared to the controls. According to \citet{maltzman_orienting_1971}, these results suggest that task instructions, leading to self-regulation, modulated the magnitude of the GSR effect in relation to the repeated presentations of the stimulus, although the stimulus had become highly predictable. Based on this finding, \citeauthor{maltzman_orienting_1979} proposed that GSRs to the initial unpredictable stimulus and GSRs to the subsequent presentations of the increasingly predictable stimulus reflect two different OR types, probably initiated by distinct cortical processes. 

As reported in \citet{maltzman_orienting_1979}, the hypothesis about the distinction of OR into involuntary and voluntary components originates from \citeauthor{pavlov_selected_1955}'s (\citeyear{pavlov_selected_1955}) signal distinction into first and second systems. The assumption was thus that the first signal controls the involuntary OR because it is a response initiated by some new, unexpected, or intrinsic sensory properties of a stimulus. In turn, the second signal system should regulate the voluntary OR, as the factors that evoke such an OR are instructions, elements of speech, or complex thought processes.

The involuntary and voluntary nature of attention orienting has also been investigated in experimental research using event-related potentials (ERPs).\footnote{\sectref{sec:2.5} elaborates on the ERP method.} For instance, \citet{naatanen_orienting_1979} reports a series of studies which shed light on the distinction between involuntary and voluntary OR. In this series of studies using selective dichotic listening tasks, participants were presented with streams of frequent signals (standards) including some slightly discrepant stimuli (deviants). The discrepant stimuli were slightly higher in intensity or pitch than standards, and participants were instructed to silently detect and count the deviant stimuli \citep{naatanen_early_1978, naatanen_brain_1980}. In the first two experiments \citep{naatanen_early_1978}, participants received the same standard and deviant stimuli in both (attended and unattended) ears, while in the \citeyear{naatanen_brain_1980} experiment, participants received two different streams per ear, as standard and deviant stimuli were of different properties in the two streams (left ear: 1000 Hz standard and 1150 Hz deviant; right ear: 500 Hz standard and 575 Hz deviant). This series of studies showed that the deviant stimuli always evoked a negative displacement in the signal compared to the standard stimuli, regardless of whether they were presented in unattended or attended conditions. This negative shift was followed by a positive deflection only in the attended condition. On the basis of these results, the authors proposed that involuntary and voluntary ORs are reflected in different ERP components (see \sectref{sec:2.5}). Since the 20\textsuperscript{th} century, a large number of contemporary ERP studies have focussed on the neurophysiological underpinnings of attention orienting (for literature review, see \sectref{sec:2.3}, \sectref{sec:2.4}, and \sectref{sec:2.5}) and provided evidence for distinct neurophysiological responses reflecting the activation of involuntary and voluntary OR mechanisms. Specifically, as discussed in the following sections, the sequence of brain events related to attention orienting, depending on the cues that led to their activation, follow a route from pre-attentive to attentive processes, reflecting involuntary and voluntary ORs, respectively.

In summary, the fact that attention can be involuntarily or voluntarily oriented was established in the literature early on. Whereas voluntary (active) attention is initiated by a listener's conscious choice, involuntary (passive) attention is activated directly by a stimulus without the listener's will or control. Stimuli can attract attention because of their sensory or meaning properties. It is evident that there is an interplay between the nature of attention shifts and low-level sensory cues or high-level cognitive operations: sensory cues appear to initiate involuntary attention orienting, while meaningful aspects of stimuli (and thus their derived properties) lead to voluntary attention orienting. It has further been suggested that involuntary and voluntary ORs are controlled by different systems/mechanisms and consequently occur at different processing stages. This proposal foreshadows the contemporary distinction between signal-driven and context-driven attentional processes. Various terms have been used in the literature to refer to this distinction, for example, bottom-up vs. top-down \citep[e.g.,][]{gibson_senses_1966, gregory_intelligent_1970}, extrinsic vs. intrinsic \citep[e.g.,][]{milner_model_1974}, exogenous vs. endogenous \citep[e.g.,][]{posner_chronometric_1978}, and involuntary vs. voluntary \citep[e.g.,][]{muller_reflexive_1989}, among other terms. The essence is that the nature of a cue initiates different mechanisms, activating distinct attentional processing stages (pre-attentive or conscious), which results in involuntary or voluntary attention orienting. \tabref{tab:table2.1} summarises some of the related properties of involuntary and voluntary attention orienting based on both the previous literature and the experimental work reported in Chapters \ref{ch:4} and \ref{ch:5}.

\begin{table}
	\caption{Properties of involuntary and voluntary attention orienting \citep[adapted from][27, on visual attention]{wright_orienting_2008}.}
	\label{tab:table2.1}
	\begin{tabular}{l l} % add l for every additional column or remove as necessary
		\lsptoprule
		Involuntary attention       & Voluntary attention \\ %table header
		\midrule
		associated with signal-based cues      & associated with meaning-based cues \\
		initiated by sensory events                 & initiated by cognitive events  \\
		rapid \& transient response                       & late \& gradual response  \\
		pre-attentive stage                               & conscious stage  \\
		\lspbottomrule
	\end{tabular}
\end{table}

In the following, I will discuss the different processing stages in the auditory system and their relation to involuntary and voluntary attention.
 
\section{Auditory processing stages and their relation to attention}
\label{sec:2.3}

As briefly discussed in Sections \ref{sec:2.1} and \ref{sec:2.2}, involuntary and voluntary attention arise at two distinct perceptual stages, the pre-attentive and conscious stage, respectively. One branch of modern cognitive neuroscience has been concerned with the question of which brain processes underlie pre-attentive and conscious perception, resulting in the development of various models of auditory attention and processing. One of the most prevalent models is the \textit{``conscious and unconscious processes in audition''} model by \citet{naatanen_auditory_2011}. 

According to this model, immediately after sound presentation, the auditory system first encodes the sensory features of the sound, then forms its sensory-memory representation \citep[also known as central sound representation, CSR;][]{naatanen_concept_1999} by integrating feature and temporal information, and finally shifts to a slowly decaying phase that represents the sensory-memory trace of that particular sound. Based on the sensory-memory traces of the preceding sound sequences, the auditory system detects regularities between successive sound events, extrapolates a pattern, and predicts the properties of the prospective sound events. When these anticipatory predictions are not attested, in other words when they are violated in that the properties of the upcoming sound event deviate from what is expected, attention-calling mechanisms in relation to this unexpected event are activated \citep[for details on the model of central auditory processing, see][]{naatanen_auditory_2011,naatanen_mismatch_2019}. 

It has been shown that the first attention-calling mechanism that is activated in the auditory cortex after the presentation of an unexpected auditory event is the automatic change-detection signal, which evokes a neurophysiological response called mismatch negativity (MMN) \citep[e.g.,][]{naatanen_attention_1992}. MMN is defined as a regularity-violation response elicited by any perceptible change in auditory stimulation, even during unattended listening \citep[for reviews, see][]{naatanen_mismatch_2007, naatanen_mismatch_2019}. MMN elicitation in turn alerts the executive mechanisms, which might result in a physiological or behavioural response. MMN thus represents an automatic, pre-attentive response, activating an involuntary attention switch towards an unexpected auditory change \citep[e.g.,][]{naatanen_role_1990, naatanen_attention_1992, naatanen_perception_2001, naatanen_mismatch_2019}.

After MMN elicitation, and thus an involuntary switch of attentional resources towards the unexpected auditory deviance, this deviance can be brought into awareness. Conscious perception can be initiated either by a strong attention-calling process \citep{naatanen_attention_1992} or through the formation of an attentional trace. The formation of the attentional trace depends on the conscious, voluntary maintenance of the deviance's features by the attentional control mechanisms. As discussed by \citet{naatanen_mismatch_2019}, when the attentional trace is voluntarily maintained (this voluntary maintenance of the attentional trace is also called \textit{rehearsal}), the deviance (and its properties) is, first, available to top-down processes, and second, has the potential to reach long-term memory, with both cases leading to its conscious perception and semantic activation. Studies have shown that the route from pre-attentive to conscious perception of an unexpected auditory change or deviance is reflected through the MMN-P3 complex. This complex consists of the MMN activation, followed by a positive deflection. This positive deflection has been identified within the P3 family. The P3 response has been reported to reflect conscious processes of novel or salient events \citep[e.g.,][]{donchin_surprise_1981, polich_normal_1986, duncan_event-related_2009}. 

Therefore, one could argue that after the presentation of an unexpected deviance in the current auditory environment, the MMN plays a fundamental role in the sequence of brain processes that underlie attention. First, the MMN generator automatically shifts involuntary attention towards the unexpected auditory deviance, and subsequently, depending on the attentional trace of this deviance, activates the mechanisms of conscious attentional processes. In other words, attention is voluntarily oriented towards the deviance, reflected in the presence of a P3 response, by which this change is brought into awareness. In \sectref{sec:2.5}, I expand on the properties of these two neurophysiological correlates, as they are of particular interest for the research reported in Chapter \ref{ch:4}. Prior to that, \sectref{sec:2.4} focuses on the relevance of the physical properties of the sensory input for attention orienting. In particular, the aim of the following section is to demonstrate the special role of signals involving a rise, for example in amplitude or pitch, for attention orienting, with the latter being the focus of the experimental Chapters \ref{ch:4} and \ref{ch:5}.

\section{The role of rises in attention orienting}
\label{sec:2.4}

As we have seen in \sectref{sec:2.2}, the idea of an inverse relation between a listener's capacity to voluntarily sustain attention towards an event and the physical intensity of a distraction has been discussed in the literature for a very long time. One branch of modern neurocognitive empirical research has also been concerned with the question of how attention orienting towards unexpected sound events is conditioned by different auditory cues. The relevant studies manipulate the sound change direction by modifying a variety of acoustic features such as fundamental frequency (f0; perceived as pitch), duration (perceived as length), and intensity (perceived as relative loudness), and have reported an attentional bias towards unexpected sounds with rising as opposed to falling acoustic properties, indexed by a greater MMN amplitude or elicitation of an MMN-P3 complex \citep[for review, see][]{naatanen_mismatch_2019}. The last point deserves special attention, as the contribution of pitch rises to attention orienting is at the core of this book. Therefore, in what follows, I review some of the neuroscientific studies which have contributed empirical evidence to this idea in the domain of cognition.

In one of the earliest studies using ERPs, \citet{naatanen_early_1978} presented participants with sequences of frequent sounds, occasionally interrupted by sounds deviating in loudness (experiment 1) or pitch (experiment 2). Specifically, deviants were of higher loudness and pitch compared to the frequent sounds. In both experiments, deviant sounds (with higher loudness and pitch), elicited a negative deflection in the signal, similar to the MMN response which was later discovered. \citet{naatanen_event-related_1989} investigated ERP responses to task-irrelevant infrequent tone pips occurring in a stream of frequent tone pips. In particular, the infrequent sound events were characterised by increasing or decreasing intensity compared to the frequent sounds. The analysis was focused only on the decreasing deviants, reporting elicitation of an MMN response, with larger amplitude and shorter latency, as the deviants' intensity decreased. Further, in an MMN study, \citet{alain_brain_1994} presented participants with sequences of two tones, say A and B, alternating repetitively, e.g., ``A B A B A B...''. This repetitive alternation was interrupted from time to time by the same tone being presented twice, e.g.,  ``A B A B A \textcolor{red}{A} B...''. The tones differed in frequency by one, six, or twelve semitones, but all the remaining properties (intensity, duration, etc.) were kept the same. Participants were instructed to ignore the auditory stimuli and their task was to read a book of their choice. The authors report that all tones deviating from the pattern elicited an MMN response. Nevertheless, MMN amplitude was larger when the difference between the two tones was an increase in frequency. 

Subsequently, \citet{rinne_superior_2005} used functional magnetic resonance imaging (fMRI) to investigate the processing of infrequent large, medium, and small duration decreases in a stream of repetitive sound duration during unattended listening. The authors' prediction was that brain activation, as reflected in haemodynamic responses, would increase with a larger decrease in duration. This prediction was based on previous fMRI studies on auditory detection of pitch changes in which the magnitude of the brain response was positively correlated with the magnitude of the sound change, although these studies explored sound changes with pitch rises \citep[e.g.,][]{doeller_prefrontal_2003}. In contrast to their expectations, \citet{rinne_superior_2005} find that brain activation tended to be greater with small and medium duration decreases than with large ones. The authors tested the same contrasts in a separate ERP session, which also showed no differences in MMN amplitude among the different magnitudes of the sound decreases. Next, \citet{rinne_two_2006} used ERP measures to study the neural responses to intensity changes during unattended listening. Specifically, participants were presented with a repetitive stream of sine waves with a standard intensity of 60 dB interspersed with infrequent intensity falls and rises. The results show that both infrequent intensity falls and rises evoked an MMN, but also that rising intensity elicited an additional P3 response. Likewise, \citet{bach_rising_2008}, using fMRI, measured brain activity in relation to infrequent rising and falling sound intensity. Their results show that rising sound intensity (as opposed to falling sound intensity) led to activation of the right amygdala and left temporal areas of the brain, accelerating the activity response time. 

In an ERP study, \citet{macdonald_effects_2011} tested whether a \textit{psychological} rise in intensity would be processed differently from a \textit{psychological} fall in intensity, in a similar way to physical intensity rises and falls. Before I move to the results of this study, let me elaborate on what the authors mean by referring to \textit{psychological} rising and falling intensity. The stimuli of this study consisted of sequences of sine waves following a rule-based alternating low-high intensity pattern, as in the following sequence: L H L H L H. Now, this rule-based pattern was occasionally interrupted either by a repetition of the high-intensity sound (e.g., L H L H \textcolor{red}{H} H L H), or by a repetition of the low-intensity sound (e.g., L H L H L \textcolor{red}{L} L H). According to the authors, the repetition of the high-intensity stimulus in the first example sequence created the percept of a relative, psychological, intensity rise because the rule would have predicted the low-intensity sound. Likewise, the repetition of the low-intensity stimulus in the second example sequence created the percept of a relative, psychological, fall because the rule would have predicted the high-intensity sound. The results of this study show that MMN peak latency and amplitude to psychological rising sound intensity were significantly earlier and larger, respectively, compared to psychological falling sound intensity. The authors conclude that a deviant representing a rise in intensity, compared to what the auditory past would have predicted, is processed in a different way compared to a deviant representing an intensity fall, even when this rising/falling property is a controlled construct and not a physical attribute.

\citet{chobert_deficit_2012} investigated the pre-attentive auditory processing in children with dyslexia compared to matched controls. The authors used syllables with Consonant-Vowel (CV) structure as stimuli and tested ERP responses to f0 and vowel duration deviants. Specifically, f0 deviants were of higher pitch, while duration deviants were of shorter length, compared to standards. The authors report that all deviants evoked an MMN response, in both groups. Specifically, for the f0 deviants, the higher the pitch, the larger the MMN amplitude. However, for duration deviants the magnitude of duration decrease did not affect the magnitude of the MMN response. 

In a more recent study, \citet{hsu_brain_2015} explored the sensitivity of pitch change direction in relation to speech. It is important to note that, although some of \citeauthor{hsu_brain_2015}’s stimuli were produced with a human voice, they were not designed to convey any linguistic meaning. Specifically, the authors tested pitch changes produced on the phoneme [ɑ] with the Mandarin level tone, starting from habitual pitch and increasing f0 for as long as modal phonation was possible. For a second set of stimuli, they synthetically elevated the pitch above the speaker’s normal range. Pitch changes in non-speech stimuli (i.e., pure tones) at the same frequencies were also included in the study. Using an oddball paradigm, \citeauthor{hsu_brain_2015} investigated whether unexpected small and large rises in spoken pitch attract listeners’ attention to a greater extent than unexpected small and large pitch falls. In addition, they asked whether brain responses to spoken pitch rises are different from similar rising pitch changes in pure tones. They found that whereas MMN to changes at normal and synthetically elevated spoken pitch height did not differ as a function of pitch direction (falling vs. rising), it did differ as a function of the size of the change (small vs. large), as large pitch changes evoked a greater MMN. MMN to pure tones equivalent to the speaker’s normal pitch height was evoked only by large falling changes in pitch, while at elevated levels, it was evoked by small and large pitch rises, and by large pitch falls. Additionally, P3 was sensitive to the direction of the change, since only rising pitch changes (both small and large) at a normal spoken pitch height evoked a P3. With pure tones, large pitch changes at an elevated pitch height also gave rise to a P3 which, however, was not sensitive to pitch direction. Based on the above results and, specifically, the P3 sensitivity to rising pitch changes in speech, the authors suggest that sudden pitch rises in speech demand more attentional resources than sudden falls, since their presence activates additional conscious processing mechanisms.

Considering the results reported in the studies summarised above, it is evident that the saliency of the physical properties of the acoustic signal is essential for attracting attention, highlighting the orienting function of signals involving a rise in intensity, duration, or pitch. \citet{bach_rising_2008} mention that rising acoustic signals prompt auditory looming effects, indicating that the sound source is approaching. Some of the aforementioned studies further show that falling acoustic signals may also attract attention, but it appears that they are experienced mostly as fading (indicating a receding sound source). Falling signals are thus processed differently, activating attentional resources in a way that is distinct from the processing of rising signals \citep[e.g.,][]{rinne_superior_2005, macdonald_effects_2011}. Rises can then essentially be seen as intrinsic warning cues, activating attentional resources, which in turn elicit a series of neural, psychological, or physical reflexive responses \citep{bach_rising_2008}. Chapters \ref{ch:4} and \ref{ch:5} revisit the idea of an attentional bias towards rising sounds, focusing on pitch rises (comparing them to pitch falls), and extend this idea from general auditory cognition to spoken language by investigating the neurophysiological and pupillary underpinnings of attention orienting in German intonation.

\section{Approaches to studying attention orienting}
\label{sec:2.5}

The scope of this section is the review of methods employed by researchers to study attention orienting. Special focus is placed on two techniques, event-related potentials and pupillometry, as these techniques are highly relevant for the experimental work presented in Chapters \ref{ch:4} and \ref{ch:5}.

\subsection{Behavioural measures}
As discussed above, the orienting response has behavioural, physiological, psychological, and neural manifestations. Within the cognitive-behavioural tradition, a strand of research has been concerned with auditory attention capture and its behavioural correlates \citep[e.g.,][]{cowan_attention_1998, naatanen_primitive_2001}. This body of research investigates attention orienting towards an auditory distraction using tasks like visual serial recall \citep[e.g.,][]{hughes_negative_2003, hughes_impact_2005}, categorisation of visual targets \citep[e.g.,][]{escera_neural_1998}, and detection of auditory targets in dichotic listening \citep[e.g.,][]{naatanen_attention_1993}, among other tasks. In these experimental set-ups, participants are simultaneously presented with sound sequences including some deviances, but are explicitly instructed to ignore them, as they are irrelevant for the task they have to perform. Such auditory deviances draw attention away from the ongoing task, inducing disengagement. Attention orienting towards the deviances is expressed through the disruption of the ongoing activity, leading to longer processing or response time, and/or deterioration in task performance, even if the task is in a different modality than the deviant sound. Some studies have also shown that physiological responses to auditory deviances and performance deficiencies often co-occur, highlighting an association between them \citep[e.g.,][]{naatanen_primitive_2001}.

\subsection{Traditional psychophysiological methods} 
A vast array of methods investigating OR has been employed in traditional psychophysiology, including electrodermal \citep[e.g.,][]{maltzman_orienting_1979}, electromyographic \citep[e.g.,][]{dimberg_facial_1990}, vascular \citep[e.g.,][]{unger_habituation_1964}, and cardiac measures \citep[e.g.,][]{graham_heart-rate_1966}, among others. In these experimental set-ups, an auditory deviant stimulus activates a cognitive process which in turn evokes a physiological response. In other words, indices of changes in the body are tracked which are evoked by certain events in the information stream. Most of these psychophysiological measures exhibit long latencies (i.e., emerge more than 2 seconds after stimulus onset) relative to the rapidity of the corresponding cognitive processes, and they therefore provide a sum response at the end of the physiological change. They serve thus as indirect measures of the activated processes, complementing more direct measures. As a result, these measures do not allow for an analytical investigation of the different processing stages associated with the orienting response through the course of the observed physiological changes, as they can lead to confounds in the interpretations of the orienting response \citep[for a more elaborate discussion, see][]{naatanen_attention_1992}. Nevertheless, it is crucial to clarify that most of these measures may be sluggish in relation to the cognitive processes, but they are rapid in relation to pre-attentive/activational and emotional processes \citep[][]{naatanen_attention_1992}. In this sense, whereas these measures do not allow for a distinction between involuntary and voluntary OR mechanisms, they do provide correlates of passive attention, and thus the involuntary component of OR. This is one of the reasons why the relevant measures were fundamental for early research on the concept of the OR, and why these methods are still of great importance for OR and for other fields of psychophysiology.

\subsection{Brain measurements: The event-related potentials technique}
Neurocognitive studies have approached attention orienting through measurements of brain activity. This line of research has been particularly concerned with the neurocognitive processes and mechanisms that underlie attention orienting. Nowadays, many methods are available for measuring brain activity, such as electroencephalography (EEG), magnetoencephalography (MEG), positron emission tomography (PET), single photon emission computed tomography (SPECT), thermoencephaloscopy, and functional magnetic resonance imaging (fMRI), among others. EEG measures the electrical activity in the brain recorded simultaneously from many sites on the scalp, MEG measures neural activity in specific brain areas via a magnetic field, and the remaining techniques measure brain activity via changes in the blood flow or metabolic rate \citep[for review of the different brain measurements, see][]{naatanen_attention_1992}. Whilst the EEG method offers a high temporal resolution of electrical changes in the brain, all other measures provide high spatial resolutions of brain activity in locating the generating area. Given that this book is concerned with the temporal unfolding of the neurophysiological underpinnings of attention orienting, I focus on the EEG method, which offers a high resolution of temporal information. This technique is implemented in the experimental investigation reported in Chapter \ref{ch:4}.

Traditional psychophysiology has also used EEG α-blocking to measure the alpha-rhythmic activity of the brain in response to auditory stimuli, intending to reveal possible task-specific patterns in this activity \citep[see][]{naatanen_attention_1992}. However, borrowing \citeauthor{luck_introduction_2014}'s (\citeyear[][4]{luck_introduction_2014}) words, ``EEG is a very coarse measure of brain activity'', as the raw signal consists of merged neural activities arising from many different sources, making it almost impossible to identify specific neurocognitive responses, for example those related to attention orienting. Thus, the EEG method was not considered effective in traditional psychophysiology \citep[for a critical review, see][]{beaumont_eeg_1983}. However, the neural responses related to specific sensory or cognitive events, which co-exist in the EEG signal along with neural activity emerging from other sources, can be identified within the overall signal using the EEG-based method of \textit{event-related potentials} \citep[ERPs;][]{luck_introduction_2014}.

The ERP method is a simple averaging technique, and, as inferred by its name, refers to electrical potentials, namely, small changes in brain activity, which are temporally related to specific sensory or cognitive events \citep[e.g.,][]{luck_introduction_2014, bornkessel-schlesewsky_towards_2016}. ERPs are recorded via electrodes mounted in an elastic EEG cap and placed on the scalp. As previously discussed, ERPs are extracted from the neural activity by averaging across multiple instances of the same stimulus, separately for each electrode site. Specifically, averaging is applied to a specific time window, starting at stimulus onset and lasting up to a time point defined by the researcher. This produces the averaged ERP waves consisting of positive and negative electrical deflections over time, which are called \textit{peaks}, \textit{waves}, or \textit{components} \citep{luck_introduction_2014}. \figref{fig:2.1ERPtechnique} presents a graphic illustration of the ERP technique.

\begin{figure}[b]
	\centering
	\includegraphics[width=\textwidth]{figures/ERPtechnique.png}
	\caption[Graphic illustration of the ERP technique]{The continuous raw EEG signal (in microvolts, positivity upward) is shown at the top of the figure over a period of time. The pink rectangles illustrate EEG segments corresponding to presentations of stimulus A, and the purple rectangles depict EEG segments corresponding to presentations of stimulus B. The individual EEG segments are extracted and averaged as a function of stimulus, in order to obtain the two averaged ERP signals shown at the bottom of the figure. \citep[Adapted from][7]{luck_introduction_2014}.}
	\label{fig:2.1ERPtechnique}
\end{figure}

ERP components are characterised in terms of the following attributes:
\largerpage
\begin{enumerate}[label=\roman*.]
	\item \textit{polarity}, a deflection's positive or negative configuration,
	\item \textit{latency}, a deflection's peak point from stimulus onset,
	\item \textit{topography}, a deflection's scalp distribution over electrode sites, and
	\item \textit{amplitude}, a deflection's area under the curve, signifying its magnitude.
\end{enumerate}

ERP components are mainly classified into three categories: exogenous, endogenous, and motor components \citep[e.g.,][]{donchin_cognitive_1978, luck_introduction_2014}, although the boundaries between these categories are often blurred. The exogenous ERP components are defined by the external, i.e., sensory, attributes of a stimulus. The endogenous ERP components are triggered by cognitive processes. Lastly, the motor ERP components are tied to a given motor response.

Further, based on their classification and attributes, ERP components receive a specific label. Traditional ERP labels consist of a letter, \textit{P} or \textit{N}, indicating positive or negative polarity of the component, and a number, reflecting either the peak's position in the signal (i.e., if it is first, second, third peak etc.), or the peak latency in milliseconds.\footnote{Usually, when the number is greater than 5, it refers to peak latency \citep{luck_introduction_2014}.} For instance, an ERP component with positive deflection that occurs 300 ms after stimulus onset, which also happens to be the third positive peak in the signal, is labelled either \textit{P3} or \textit{P300} (see \figref{fig:2.1ERPtechnique} for an illustration). ERP components may also be linked to particular functions, and are then named after their functional relevance. One example is the \textit{error-related negativity}, a component which is elicited by an error in the stimulus stream \citep[for more on the naming conventions of ERP components, see][]{luck_introduction_2014}. 

The ERP technique is widely used in cognitive neuroscience, sensory processing, and language processing (of both spoken and written stimuli), among other domains. It is a powerful tool compared to behavioural or traditional psychophysiological measures, as it provides a high resolution of temporal information, in turn making it possible to continuously measure brain activity in response to a stimulus with millisecond accuracy \citep[e.g.,][]{naatanen_attention_1992, luck_introduction_2014}. This allows for an analytical investigation of the neurocognitive processes before, after, or during a specific stimulus event, in other words, as time unfolds. Further, as discussed by other scholars, ERPs offer a measure of online processing without demanding a behavioural response, i.e., during unattended stimulation \citep[e.g.,][]{naatanen_attention_1992, luck_introduction_2014}. This aspect is particularly useful for the study of attention orienting, as it allows for the investigation of the path from pre-attentive to attentive processes during unattended stimulation.

A large body of ERP research on automatic sensory processing has reported a number of early and late exogenous ERP components related to audition, for example, the Auditory Brainstem Responses (ABRs), the Middle Latency Responses (MLRs), and the Late Transient Responses (N1, P2, N2, P2). All these exogenous auditory ERP components, which range from 10 ms to 150 ms post stimulus, are related to the physical characteristics of the stimuli \citep[for review, see][]{naatanen_attention_1992}. In other words, they are signal-evoked potentials. Moreover, many ERP studies on attention orienting have been instrumental in elucidating the association between specific ERP components and attentional processes, namely the mismatch negativity (MMN) and the MMN-P3 complex, as already introduced in \sectref{sec:2.3}. The MMN and P3 components (in contrast to the exogenous auditory ERPs) are endogenous, as they are generated by a regularity-change-detection system, reflecting cognitive processing. In what follows, I elaborate on these two ERP components, which are central to the study reported in Chapter \ref{ch:4}. 

\subsection{ERP correlates of attention orienting: MMN and P3}
The mismatch negativity, known as MMN, is a neurophysiological response recorded in the EEG,\footnote{MMN can also be recorded in MEG, called MMNm. In addition to the auditory modality, MMN can be recorded in the visual (vMMN), somatosensory (sMMN), and olfactory (oMMN) modalities \citep[for more, see][]{naatanen_mismatch_2019}.} and discovered by \citet{naatanen_early_1978} in the late 1970s. \citet[1]{naatanen_mismatch_2019} describe MMN as ``a window to central auditory processing'' which enables us to better understand ``the brain processes forming the biological substrate of central auditory processing and the different forms of auditory memory, as well as the attentional processes controlling the access of sensory input to conscious perception and higher forms of memory''. As mentioned in \sectref{sec:2.3}, MMN is elicited by any perceptible change in the auditory stream. It is well established that MMN elicitation is independent from attentive processing \citep[for reviews on MMN in passive recordings, sleep, patients in coma, and anaesthetised humans \& animals, see][]{naatanen_mismatch_2007, naatanen_mismatch_2019}. Accordingly, it has received the label \textit{automatic regularity-violation response}.

MMN can be recorded in different stimulus paradigms, depending on the aspects of the central auditory system under study. One of the most frequently used paradigms is the classic \textit{steady-state} oddball paradigm in passive recordings \citep[e.g.,][]{naatanen_mismatch_2019}, which is also employed in the experimental study presented in Chapter \ref{ch:4}. In this paradigm, listeners are presented auditorily with sequences of repetitive sounds (called standards) which are occasionally interspersed with a sound that deviates in some property from the steady-state stream (called deviant) while they are watching a film with no sound, or reading a book of their choice.\footnote{Oddball paradigms have been also used in active recordings, or dichotic listening tasks.} Another commonly used paradigm is the \textit{changing-state} oddball paradigm, in which listeners are presented with sequences of changing auditory stimuli leading to a pattern (standards), occasionally replaced with a stimulus that disrupts the pattern (deviant). The changing-state oddball paradigm is used in the experimental study presented in Chapter \ref{ch:5}. Finally, some studies have utilised the \textit{multi-feature} oddball paradigm, in which every second stimulus is a standard, while the rest are deviants of different types \citep[for reviews of the different MMN paradigms, see][]{paavilainen_mismatch-negativity_2013, naatanen_mismatch_2019}. In all oddball cases, the deviant stimuli evoke an MMN activity in relation to their standard stimulation (see \sectref{sec:2.3} for more on the MMN generation process). 

\figref{fig:2.2MMNillustration} depicts an MMN to small frequency deviants in a stream of repetitive 1000 Hz tone stimuli, during unattended processing. The MMN is a negative potential which is traditionally obtained as a difference wave by subtracting the ERPs to standard from those to deviant stimuli \citep[i.e., deviant ERPs – standard ERPs;][]{naatanen_mismatch_2019}. MMN has a fronto-central maximum and a peak latency between 150 and250 ms after the stimulus onset. Nevertheless, there is considerable variability in the definition of this time window, as peak latency has usually been defined on the basis of difference waves.\footnote{As \citet[][349]{luck_introduction_2014} mentions ``any amplitude or latency measurement procedure that involves finding a peak is non-linear". Therefore, measuring peak latency before computing difference waves (or averaging waveforms together) leads to different results than performing the same measurement after computing difference waves. Further, latency differences may arise from amplitude changes. Although difference waves are a great tool for isolating specific components from the parent waveform, latency estimates can be ``distorted by changes in the amplitude of temporally overlapping components" \citep[][57]{luck_introduction_2014}.} Moreover, latency may vary with the degree of deviation \citep[e.g.,][]{emmendorfer_erp_2020}. There are other transient brain responses that indicate some kind of change or deviance detection (e.g., P3a/b, N400, P600), although their elicitation requires some level of saliency or directed attention. These can, for instance, be triggered by task demands \citep{sussman_five_2014}.

\begin{figure}
	\includegraphics[width=\textwidth]{figures/MMN.png}
	\caption[MMN to frequency deviants]{ERPs to standard and deviant stimuli are depicted on the left-hand side. Difference waves between ERP to deviants and ERP to standards, indicating the MMN component, are illustrated on the right-hand side. Both ERP and difference waves are shown at the Fz\footnote{The Fz electrode site refers to the placement of an electrode at the frontal (F) brain region on the midline (z). See Chapter \ref{ch:4} for more information on electrode placement according to the 10--20 system.} electrode site \citep[figure taken from][2]{naatanen_mismatch_2019}.}
	\label{fig:2.2MMNillustration}
\end{figure}

MMN has been described as a correlate of involuntary attention switch towards deviant sounds in the current auditory environment, as its elicitation reflects the automatic activation of the first attention-calling mechanism in the auditory cortex. As mentioned above, MMN is elicited rapidly after stimulus onset, revealing attentional processes that emerge at the pre-attentive stage. As outlined in \sectref{sec:2.3}, MMN generation can trigger an ``attention leak'' towards the unattended deviant, bringing it into awareness. In other words, MMN can initiate subsequent attentional processes at the attentive stage after activating an involuntary attention switch. This is usually reflected in the activation of additional ERP components such as the MMN-P3 complex. The MMN-P3 complex consists of MMN activation accompanied by a positive deflection within the P3 family \citep[for review, see][]{naatanen_mismatch_2019}. The P3 is a broad, positive-going component with peak latency around 300 ms (or more) after stimulus onset \citep[P3 latency depends on the complexity of processing -- the more complex the processing, the longer the latency -- with ranges from approximately 250 to 1000 ms; see][]{duncan_event-related_2009}. P3 has been divided into two subcomponents, P3a and P3b \citep[for review, see][]{polich_updating_2007}. P3a has a fronto-central scalp distribution with shorter and sharper peak latency, and has been related to attentive processes. P3b has a posterior scalp distribution with longer peak latency, and has been related to memory during discourse updating \citep[for reviews, see][]{polich_updating_2007, duncan_event-related_2009}. The current work is specifically concerned with the P3a, as it has been reported to reflect conscious attentional-related processes of novel, less probable, or salient events \citep[in terms of intrinsic significance or self-relevance and not physical properties; e.g.,][]{donchin_surprise_1981, polich_updating_2007, duncan_event-related_2009}. Thus, MMN and P3 responses have been identified as correlates of attention orienting, reflecting the route from pre-attentive to conscious processing of an unexpected auditory change or deviance \citep[e.g.,][]{escera_neural_1998, naatanen_perception_2001, hsu_mismatch_2023}. 
\newpage
Except for a correlate of attention orienting, clinical studies have also outlined MMN as an indicator of general brain plasticity \citep[e.g.,][]{kreegipuu_mismatch_2009}, with wide clinical use. For example, MMN deficiency has been found to reflect various neuropsychiatric, neurological, and neurodevelopmental cognitive and functional profiles \citep[for an elaborate review of MMN applications, see][]{naatanen_mismatch_2019}. The MMN component has been further used to study linguistic processes, from simple phoneme discrimination \citep[for review of studies, see][]{naatanen_mismatch_2019} to high-order processes such as syntax \citep[e.g.,][]{hasting_setting_2007}, revealing MMN as an index of language-specific detection, reflecting long-term traces for linguistically meaningful elements in the brain. Likewise, P3 has been associated with attentional processes during online comprehension of spoken language \citep[e.g.,][]{rohr_signal-driven_2021}. Certainly, there are various ERP components relevant for language processing (e.g., P200, N400, Late Positivity, among others), which, however, are beyond the scope of this book, and thus will not be discussed here \citep[for a brief review on language-related ERP components, see][]{luck_introduction_2014}. MMN and P3 responses in relation to speech processing will be further reviewed in \sectref{sec:3.4} of Chapter \ref{ch:3}. In Chapter \ref{ch:4}, MMN/P3 responses are revisited and explored as neurophysiological underpinnings of attention orienting in spoken language.

\subsection{Indexing attention orienting using pupillometry}
Another psychophysiological method that has been employed to study attention orienting is \textit{pupillometry}. Pupillometry is a widely used method, applied in various scientific domains, measuring changes in pupil size as a proxy for distinct neurocognitive processes \citep[e.g.,][]{amiez_chapter_2019}.

The black aperture at the centre of the iris (the coloured part of the eye) is called the pupil. Light passes through the pupil, with the iris controlling the amount of light that passes through. From there, light hits the retina, and is converted into an electrical signal sent to the brain via the optic nerve. Luminance changes evoke changes in pupil size. Specifically, high levels of light result in pupil contraction, while low levels of light lead to pupil dilation \citep[e.g.,][]{strauch_pupillometry_2022}. The pupil light reflex is mediated by two antagonistic networks: pupil contraction is controlled by the parasympathetic system, and pupil dilation is regulated by the sympathetic system \citep[for more on the pupil control pathways, see][]{amiez_chapter_2019, strauch_pupillometry_2022}. 

As \citet{johansson_computational_2018} point out, it has been established that whilst large changes in pupil size emerge as a result of luminance changes, pupillary responses also occur as a function of distinct brain processes. Specifically, the \textit{pupil dilation response} (PDR) has mainly been correlated with processes such as cognitive effort (in the sense of deliberate allocation of mental resources during a cognitive task), emotional processing, heightened attention, expectancy violation, and memory consolidation, among others \citep[for review, see][]{winn_best_2018}. Many of these studies have measured pupillary responses in relation to auditory stimuli \citep[for review, see][]{zekveld_pupil_2018}. In particular, a growing number of pupillometric studies have employed pupillometry to investigate auditory distractions and proposed the PDR as a valid psychophysiological index of auditory attention orienting, similar to the neural MMN/P3 responses \citep[e.g.,][]{nieuwenhuis_anatomical_2011, wang_circuit_2015, johansson_computational_2018, marois_eyes_2018, alamia_pupil-linked_2019}.  

\citet{strauch_pupillometry_2022} suggest pupillometry as a valuable psychophysiological tool for investigating the differential neural networks of the attention system as proposed by \citeauthor{petersen_attention_2012} (see beginning of this chapter). The authors further propose a classification of pupillary responses as a function of the factor that led to their elicitation, and rank these factors as: 

\begin{enumerate}[label=\roman*.]
	\item \textit{low-level}, including luminance changes and focal distance,
	\item \textit{intermediate-level}, involving alerting and orienting, and
	\item \textit{high-level} factors, encompassing an interaction between conscious attention and sensory processing.
\end{enumerate}
Each of these factors activates different neural circuits, which in turn elicit distinct types of pupillary responses. In particular, as already mentioned above, pupillary responses triggered by low-level factors are stimulated by the parasympathetic and sympathetic systems. Pupillary responses evoked by intermediate-level factors are mediated by two distinct pathways in the subcortical areas: arousal-related PDRs result from the locus coeruleus system (LC), and orienting-related PDRs are controlled by the superior colliculus system (SC) \citep[e.g.,][]{wang_circuit_2015, joshi_pupil_2020}. Finally, pupillary responses to high-level factors are regulated by the executive network located in the frontoparietal areas. 

The pupil is characterised by two states: a steady-state with a baseline size, and a transient-change state indicating temporary deviations from the baseline, expressed in sudden dilations or contractions \citep{joshi_pupil_2020}. Low-level factors lead to size changes in the pupil steady-state. Orienting factors trigger transient-changes in the baseline pupil size. Finally, alerting and executive factors can result in both steady-state and transient pupil size changes depending on whether the effect is sustained or temporary. 

Orienting-related PDRs are usually momentary, meaning they have short latencies: their onset typically occurs between 200 and 500 ms post stimulus, peaking approximately 1 second later, and ending rapidly just after stimulus completion \citep[for review on pupillary latencies, see][]{beatty_task-evoked_1982}. Further, orienting-related PDRs are sensitive to the saliency of the physical properties characterising auditory deviances. Specifically, the greater the acoustic saliency of a deviant (i.e., rising sounds), the larger the PDR amplitude \citep[e.g.,][]{liao_human_2016, wetzel_infant_2016, marois_eyes_2018, strauch_pupillometry_2022}. Certainly, high-level factors can further impact orienting-related PDRs \citep[e.g.,][]{joshi_pupil_2020, strauch_pupillometry_2022}. This is usually reflected in the PDR curve shape, manifesting as a more sustained response \citep[for more, see][]{strauch_pupillometry_2022}. This means that the latency of attention-related PDRs is indicative of the awareness level or processing depth: whereas transient responses occur pre-attentively, more prolonged responses emerge in full awareness \citep[for discussion, see][]{strauch_pupillometry_2022}. In other words, orienting-related PDRs, initiated by an auditory deviant event, can further activate the executive network, making the deviance available to top-down processes or long-term memory. Thus, orienting-related PDRs lead to conscious perception through their transformation into executive-related PDRs. Therefore, the route from pre-attentive to conscious perception of a deviance is also reflected in the time course of pupil responsiveness \citep[an integrated readout of attentional networks in the words of][]{strauch_pupillometry_2022}. One could hence argue that transient, signal-sensitive PDR is an instantiation of involuntary attention switch, while a more prolonged PDR, usually elicited by more complex stimulus properties, is a manifestation of voluntary attention alignment.

These patterns in pupil dilation response align with the findings in neurocognitive research reviewed so far in this chapter, showing that signal-based properties are essential for attracting involuntary attention, while voluntary attention is further activated by more complex top-down processes. In this context, we see a striking resemblance in the elicitation of MMN/P3 responses and PDR: the orienting-related PDR amplitude and latency mirrors MMN activation, while the executive-related PDR latency resembles the P3 elicitation. Many brain studies on the localisation of neural activity in specific brain structures support a direct link between the neurophysiological and pupillary underpinnings of attention orienting. Specifically, evidence from multiple research disciplines proposes the locus coeruleus (LC) system as one of the areas underlying P3 generation \citep[for reviews of studies on the relationship between LC, P3, and the orienting response, see][]{nieuwenhuis_decision_2005, nieuwenhuis_anatomical_2011, joshi_pupil_2020}. In this light, it is worth recalling that the LC system also regulates the PDR elicitation that is related to attention. Further, \citet{alamia_pupil-linked_2019}, in a recent study investigating the link between pupillary and neural responses to unattended deviances, found a correlation between PDR and central potentials. Besides pupil size dilation in response to deviant events, they also report deviances evoking the classic MMN neural response. As discussed by the authors, MMN is generated (partially) in the anterior cingulate cortex (ACC), a brain structure also responsible for evoking pupillary responses which is closely connected with LC. It appears thus that in addition to the temporal correlation between the neural and pupillary activity in relation to attentional processes, there is also a spatial relation between the two activities. The regulation of attention-related neural and pupillary responses by the same or connected neural circuits supports the idea that the OR involves a wide range of physiological changes occurring simultaneously across multiple indices. Nevertheless, the localisation of brain activity is beyond the scope of this work, and thus will not be discussed further. The pupillary response, and particularly the PDR, as a psychophysiological underpinning of attention orienting is of particular interest for Chapter \ref{ch:5}, where it is revisited and investigated as a proxy of attention orienting in speech.

\section{Summary}
\label{sec:2.6}

This chapter focused on the composite construct of attention orienting towards auditory deviances by exploring its cognitive and neural underpinnings. First, the origins of the attention orienting concept were reviewed. The orienting response (OR) is defined as a shift in attention initiated by an event deviating in some property from the recent past, finding expressions in various psychological, physiological, neural, motoric, or behavioural indices. In this context, it was further shown that the nature of the deviating property that initiates an OR response elucidates whether the shift in attention is involuntary or voluntary. In this respect, empirical studies concerned with the role of auditory cues in attracting attention were further reviewed, establishing the fundamental role of rising acoustic signals in involuntary attention orienting. In addition, it was shown that involuntary and voluntary orienting emerge at different processing stages (pre-attentive and attentive, respectively), as they are regulated by different mechanisms. 

Various different methods used to investigate attention orienting were outlined with a focus on ERPs and pupillometry, two techniques that provide a high resolution of temporal information, and thereby enable the measurement of brain and cognitive activity related to attentional processes over time. Special emphasis was also placed on the MMN/P3 ERP components and the PDR as reflexes of attention orienting. Specifically, MMN and P3 components were shown to be rigorous correlates of attention orienting, reflecting the route from involuntary to voluntary attention. Similarly, PDR was shown to be a valid index of attention orienting, highlighting a striking resemblance between MMN/P3 elicitation and PDR composites. The link between neural and pupillary activity in relation to attentional processes was illustrated, reinforcing the idea of the OR being expressed in multiple physiological exponents at the same time. In keeping with the focus of this book on attention orienting in spoken language and the relevance of intonation for its function, the next chapter introduces fundamental concepts related to prosody and intonation, and reviews important findings in relation to speech perception and processing.
