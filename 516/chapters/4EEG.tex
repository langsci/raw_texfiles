\chapter{Unravelling the role of intonation in attention orienting using ERPs}
\label{ch:4}

One of our fundamental abilities is that, based on our interests and goals, we can voluntarily select our object of attention among the many available input sources. Voluntary attention thus shields the processing of information that we have chosen to process from other available but currently irrelevant information. However, unexpected sound changes outside of the current attentional focus may break through this shield, causing an involuntary switch of attentional resources towards these unexpected changes (see Chapter \ref{ch:2}). One of the auditory cortex change detection mechanisms that triggers involuntary attention is reflected in the neurophysiological response called mismatch negativity \citep[MMN; e.g.,][]{naatanen_attention_1992}. The present study is concerned with the processing of intonational rises and falls in German when presented unexpectedly in a stream of repetitive auditory stimuli. In an EEG study, using an auditory passive oddball paradigm, I investigate whether rising pitch in speech is special in attention orienting by measuring MMN responses to intonational changes.\footnote{The research presented in this chapter has been reported in \citet{lialiou_auditory_2024}.}

\section{Motivation}
\label{sec:4.1}

Chapter \ref{ch:2} provided a comprehensive review of studies attesting that the physical properties of the acoustic signal are crucial for the activation of the orienting response, showcasing the orienting function of rising signals (see \sectref{sec:2.4}). Specifically, rises in the amplitude or pitch of sine waves form intrinsic warning cues with ripple effects, prompting auditory looming (i.e., the percept of an approaching sound source), which activates attentional resources and in turn leads to a series of neural, psychological, or physiological reflexive responses. However, it has also been shown that falling acoustic signals, too, may attract attention when presented unexpectedly, even though they are experienced mostly as a receding sound source and thus processed differently than rises. In a similar fashion, Chapter \ref{ch:3} highlighted the linguistic importance of intonational rises in speech as a decisive cue to prominence (see \sectref{sec:3.2}), indicating their crucial role in speech processing (see \sectref{sec:3.3}). Speakers make use of prominent intonational contours to highlight important elements in their utterances, endeavouring to orient listener attention to specific information or a specific idea. At the same time, prominent intonational events like rising contours attract and allocate listener attention towards constituents bearing these intonational events, facilitating their processing. However, Chapter \ref{ch:3} showed that, in addition to signal-driven cues (such as rises in pitch), context-driven cues take on a crucial role in speech processing, and sometimes overwrite the typical acoustic cues that signal prominence. 

In the current chapter, I revisit the idea of an attentional bias (essentially serving as a warning cue) towards sounds with rising as opposed to falling acoustic properties and extend this idea from general cognition to spoken language. In the context of previous neurocognitive studies suggesting that unexpected sound events with rising acoustic properties attract more attention than sound events with falling acoustic properties in a stream of repetitive non-linguistic auditory events (see \sectref{sec:2.4}), the current study explores the role of intonational rises and falls (attributable to accents and boundary tones) in attention orienting, using speech stimuli that convey linguistic meaning. Extending previous work on simple sine waves, I introduce rising and falling intonation on sequentially presented lexical items, which potentially gives rise to a list context and the concordant, language-specific expectations. In addition, I investigate the role of the phonological status of the rise (pitch accent or edge tone)\footnote{As already stated in the introductory chapter, the term edge tone is used interchangeably with the term boundary tone.} in attention orienting. Crucially, intonational theories such as the autosegmental-metrical theory \citep[AM; e.g.,][]{ladd_intonational_2008} postulate a strict link between the association (head/edge) and the functional (prominence/phrasing) properties of tonal events. Pitch accents are thereby related to prominence and edge tones to phonological phrasing, excluding a prominence-cueing function of the latter (see \sectref{sec:3.2} for more detail). However, recent research has provided some evidence against such strict postulations (for review, see Chapter \ref{ch:3}). Therefore, the question I address in relation to the phonological status of rises is whether only rises on pitch accents play a role, or whether rising edge tones have a similar effect, as in the serial recall results reviewed in \sectref{sec:3.3}. 

For this study, EEG recordings of listeners were collected and event-related brain potentials (ERPs; for more on this method, see \sectref{sec:2.5}) related to unexpected auditory deviances in a repetitive auditory stream were measured. Of particular interest for this study are the MMN and the subsequent positivity (P3). Previous studies have shown that these two brain responses appear to be the neurophysiological underpinnings of involuntary and voluntary attention, indexing the path from pre-attentive to conscious processes related to an unexpected auditory change (see \sectref{sec:2.5}). The MMN can be recorded in different stimulus paradigms, depending on the different aspects of the central auditory processing under study. One of the most frequently used paradigms is the classic oddball paradigm in passive recordings, utilised in the current study \citep[e.g.,][]{naatanen_mismatch_2019}. In this paradigm, participants are presented auditorily with sequences of repetitive sounds, called standards, occasionally interspersed with a rare sound, called deviant, during a film with no sound. This allows for the elicitation of brain responses in the absence of attentive listening, and can thus shed light on the nature of the underlying neural mechanisms.

The main question this study puts forward is whether intonational rises attract more attention than falls (RQ1 [rises]). Two central hypotheses are tested. Following the auditory looming literature \citep[e.g.,][]{bach_rising_2008, macdonald_effects_2011}, Hypothesis 1 states that unexpected rises in a sequence of repetitive falls should attract more attention than unexpected falls in a stream of repetitive rises, evoking a more pronounced MMN potentially followed by a P3 response. Furthermore, given the focus of the current work on the processing of intonational rises and falls in both accentual and boundary positions, I subsequently ask whether the phonological status of the rise and fall affects the underlying processing (RQ2 [edges]), or in other words whether accentual contours attract more attention than boundary contours. Hypothesis 2 thus states that the extent of the effect may vary as a function of the phonological association (pitch accent vs. boundary tone) of the tonal event \citep[e.g.,][]{ladd_intonational_2008}. If the standard AM account holds, postulating pitch accents as the primary markers of prominence, a greater MMN effect for accentual over boundary contours is expected.

\section{Methods}
\label{sec:4.2}

Employing the classic oddball paradigm in passive recordings and using the material described in \sectref{subsec:4.2.2}, four different oddball conditions were designed, in which rising and falling f0 contours alternated as standard/deviant sounds. In two conditions, standard/deviant sounds are composed of accentual contours (condition1: standard accentual rise/deviant accentual fall; condition2: standard accentual fall/deviant accentual rise), whereas in the other two conditions, standard/deviant sounds are composed of boundary contours (condition3: standard boundary rise/deviant boundary fall; condition4: standard boundary fall/deviant boundary rise). While listening to the stimuli, participants watched a nature documentary film with no sound \citep[Deep Blue;][]{fothergill_deep_2003}.

\subsection{Participants}
\label{subsec:4.2.1}

ERP data from 32 right-handed participants were recorded. Handedness was assessed by the Edinburgh Handedness Inventory \citep{oldfield_assessment_1971}. All participants were monolingual native speakers of German (28 female, 4 male), aged between 19 and 33 years old (mean age = 24.5 years, \textit{SD} = 3.5). Participants provided written informed consent in accordance with the Declaration of Helsinki and in compliance with the ethics clearance from the Ethics Board of the \textit{Deutsche Gesellschaft für Sprachwissenschaft} (DGfS). Participants received reimbursement for their participation (either course credit or monetary compensation). None of them reported any speech, hearing, or neurological impairment. 

\subsection{Speech materials}
\label{subsec:4.2.2}

The auditory stimuli used in the oddball paradigm comprise rising and falling f0 contours realised either on the stressed syllable (accentual contours), or at the right boundary, that is, on the last syllable, (boundary contours) of four different lexical items (\textit{Banane} ‘banana’, \textit{Limone} ‘lime’, \textit{Marone} ‘chestnut’, \textit{Melone} ‘watermelon’), resulting in a total of 16 tokens (4 contours x 4 lexical items). For segmental comparability of the lexical items, I selected common trisyllabic German nouns with simple segmental structure and primary lexical stress on the penultimate syllable (CV.\textbf{CV}.CV)\footnote{Bold face indicates lexical stress.} which were mainly composed of voiced sounds to enable continuous f0 tracking. Additionally, to control for potential word frequency effects on prominence perception \citep[see][]{cole_signal-based_2010}, all items were of approximately the same frequency class.\footnote{The frequency distributions of the words were extracted from the Leipzig Deutscher Wortschatz corpus \citep{quasthoff_projekt_2005}.} 

All stimuli were produced by a phonetically trained female native speaker of German and recorded with a sampling rate of 44100 Hz and 16-bit resolution (mono). To ensure natural speech production of the items, the speaker was asked to spontaneously produce all items in isolation, with all intonational contours. To circumvent inconsistencies in the realisation of the same intonation contour across the different items, the most natural-sounding item in terms of speech rate, duration, pitch range, and meaning\footnote{That is, a falling contour marking finality, a rising contour marking continuity.} was selected for each contour and presented repeatedly to the speaker as a prompt for the production of the same contour type across lexical items. The speaker repeated the prompt with as little delay as possible, resulting in a natural but very consistent production of the different contours across items, as confirmed by acoustic analysis. The original production of all stimuli was used in the experiment, normalised at −23 LUFS. \figref{fig:f0contours} illustrates the mean f0 contours as well as the individual f0 contours of all four items superimposed on each other. The similarity of the f0 contours indicates that they were produced in a very consistent manner across items (see bottom panels). 

\begin{figure}
	\includegraphics[width=\textwidth]{figures/f0contours.png}
	\caption[Mean \& individual f0 contours for each condition] {Top panel: mean f0 contours across items. Bottom panel: individual f0 contours of all items, superimposed for each condition.}
	\label{fig:f0contours}
\end{figure}

For the acoustic characterisation of the stimuli, the \textit{ProPer} toolbox was used. ProPer is an open source toolbox for the acoustic analysis of prosodic phenomena based on continuous measurements of periodic energy and f0 \citep{albert_using_2018, cangemi_modelling_2019, albert_proper_2020, albert_model_2023}. Prior to the ProPer analysis, f0 contours were extracted and corrected manually in Praat \citep{boersma_praat_2024}, using a customised version of \textit{mausmooth} \citep{cangemi_mausmooth_2015}. To acoustically describe the current stimuli, the relative periodic energy mass (henceforth, Mass) and the relative Delta f0 (henceforth, Δf0) metrics were employed. All analyses were conducted on the basis of syllabic units. Scripts and data tables of the current analysis have been made available on the Open Science Framework (OSF) platform (\url{https://osf.io/nhy4c}).

Mass is the area under the periodic energy curve between two syllabic boundaries. It is calculated as the integral of duration and power, accounting for these two cues together in one variable, capturing the overall prosodic strength of the corresponding syllable \citep[for more on the components of Mass and its improved acoustic characterisation of prominence, see][]{albert_model_2023}. Here, relative Mass is used, indicating the prosodic strength of one syllable relative to the other syllables in the word. Mass values are calculated as the area under the periodic energy curve of the entire word divided by the number of syllables in the word (Mass = raw Mass of the word/n of syllables). The average Mass is centred around one, so that weak syllables exhibit values lower than one, and strong syllables exhibit values higher than one (weak < 1 < strong). 

The measure of Δf0 describes the f0 trajectory across syllables, using both f0 and periodic energy. First, Δf0 measures the f0 at the centre of Mass within each syllabic interval, and then the difference in f0 between two subsequent syllables is calculated. Thus, Δf0 indicates the f0 change from syllable to syllable by calculating the difference from the previous one. For the first syllable, Δf0 is calculated relative to the speaker’s f0 median. The raw Δf0 is measured in Hz; in this analysis relative Δf0 values (relative Δf0 = raw Δf0/speaker’s f0 range) are used \citep[for more on Δf0, see][]{albert_model_2023}. 

\figref{fig:periogram} shows a representative example of a \textit{periogram}, a visual representation of the f0 curve, for the item \textit{Melone} realised with a rising pitch accent. The f0 curve is enriched with periodic energy, such that the thicker the line, the greater the periodic energy. The lower part of the figure depicts the periodic energy curve itself. Mass and Δf0 values are illustrated on the same figure.

\begin{figure}
	\includegraphics[width=\textwidth]{figures/periogram.png}
	\caption[A representative example of a periogram]{A representative example of a periogram. In the upper half of the figure, the blue line illustrates the f0 curve, with thickness reflecting its strength. The red line in the lower half part of the figure depicts the periodic energy curve. The Hz values on the y-axis correspond only to the f0 curve. The red vertical dashed lines indicate the position of the centre of Mass (CoM) within syllabic intervals. Mass values on the relative scale are given in numbers below each interval. The short vertical blue lines on the f0 curve indicate the centre of gravity (CoG), with Δf0 values displayed on the f0 curve.}
	\label{fig:periogram}
\end{figure}

The consistent production of the f0 contours across items is also evident in the ProPer analysis. \figref{fig:deltaFzero} depicts relative Δf0 values per syllable as a function of contour, across items as well as per item. In accentual falling contours (see depiction in blue in the top panels), Δf0 starts H(igh) on the first syllable, falls to a L(ow) on the stressed syllable, and levels (or slightly rises) towards the last syllable in the word. In accentual rising contours (see depiction in red in the top panels), Δf0 starts L on the first syllable, rises to a H on the subsequent stressed syllable, and levels towards the last syllable. In both falling and rising boundary contours (see bottom panels), Δf0 remains on the same level from the first to the subsequent stressed syllable, and then for the former contour (see depiction in green), falls to a L towards the last syllable (boundary) of the word, while for the latter (see depiction in yellow), it rises to a H.

\begin{figure}[t]
	\includegraphics[width=\textwidth]{figures/deltaFzero.png}
	\caption[Relative Δf0 values across conditions \& items]{Relative Δf0 values per syllable (x-axis) across accentual contours (top panels; accentual falls in blue, accentual rises in red) and boundary contours (bottom panels; boundary falls in green, boundary rises in yellow) presented across words (first and third panels) as well as separately for each word (second and fourth panels).}
	\label{fig:deltaFzero}
\end{figure}

Moving on to Mass, \figref{fig:mass} presents Mass values per syllable of each item across accentual and boundary contours. Visual inspection of Mass in both accentual and boundary contours shows a slight variability on the values across items. This is expected considering the different phonetic makeup of each syllable across the different lexical items. Despite these slight differences, the second syllable, which is stressed and accented, exhibits the greatest Mass across items, in the accentual contours, indicating that it is prosodically the strongest syllable in the word, while its overall prosodic strength is similar across rising/falling contours. In the boundary contours, prosodic strength does not differ greatly between the stressed/accented (second) and the last syllable, which carries the boundary tone. Mass on the stressed syllable is slightly higher in boundary falls than in boundary rises, while Mass on the last syllable is subtly higher in boundary rises than in boundary falls.

\begin{figure}[t]
	\includegraphics[width=\textwidth]{figures/mass.png}
	\caption[Relative Mass values across items]{Relative Mass values per syllable (x-axis) as a function of item and intonational contour. Top panels illustrate Mass values in accentual contours (rises in red, falls in blue). Bottom panels present Mass values in boundary contours (rises in yellow, falls in green).}
	\label{fig:mass}
\end{figure}
\largerpage
\tabref{tab:4.1} presents means and standard deviations for (i) relative Δf0 and (ii) relative Mass values per syllable across items for each contour, as well as (iii) the total duration of items.

\begin{table}
	\centering
	\caption[Δf0, Mass, and duration across items]{Mean and standard deviation (in brackets) for relative Δf0 and Mass values, per syllable, across items for each contour, as well as for the total duration of items.}
	\label{tab:4.1}
	\begin{tabular}{l r r r r} % add l for every additional column or remove as necessary
		\lsptoprule
		syllable & accentual fall & accentual rise & boundary fall & boundary rise\\ %table header
		\midrule
		&     & \textit{relative Δf0} &        &            \\
		\midrule       
		1 & 76.20 (9.99) & --76.13 (6.05) & --1.10 (3.5) & 1.20 (1.2)\\
		2 & --64.50 (18.5) & 53.50 (9.5) & 14.50 (9.2) & --7.40 (1.5)\\
		3 & --25.50 (13.5) & 33.70 (7.8) & --57.80 (5.8) & 42.70 (4.7)\\
		\midrule
		&           & \textit{relative Mass} &          &            \\
		\midrule
		1 & 0.88 (0.13) & 0.57 (0.14) & 0.83 (0.10) & 0.81 (0.14)\\
		2 & 1.42 (0.13) & 1.42 (0.07) & 1.20 (0.16) & 1.09 (0.15)\\
		3 & 0.70 (0.13) & 1.01 (0.09) & 0.97 (0.13) & 1.10 (0.08)\\
		\midrule
		&           & \textit{duration [ms]} &          &            \\
		\midrule
		item & 0.600 (0.01) & 0.680 (0.02) & 0.630 (0.02) & 0.620 (0.02)\\
		\lspbottomrule
	\end{tabular}
\end{table}

\subsection{EEG recording}
\label{subsec:4.2.3}

The electroencephalogram (EEG) was recorded from 32 Ag/AgCl electrodes, amplified with the Brain Vision amplifier, and digitised at a sampling rate of 1000 Hz. The electrodes were mounted in an elastic EEG cap (EasyCap, EasyCap GmbH, Herrsching, Germany) and placed on the scalp according to the standard international 10--20 system. The electrical contact between scalp and electrodes was achieved by applying a conductible electrolyte. As the MMN is well-documented to have a frontocentral topography \citep{naatanen_mismatch_2007}, a distribution of mostly frontal electrodes (AF3/4/7/8, F3/4/7/8, Fz, FC1/2/5/6, FCz, C3/4, Cz, CP1/2, CPz, P3/4/7/8, Pz, POz, Oz) was selected. The AFz electrode position served as the ground electrode, and additional electrodes were placed on the left and right mastoids for referencing (left) and re-referencing (right) of the EEG channels. To control for eye-movement artifacts, the electrooculogram (EOG) was also recorded with electrodes placed to left and right mastoids at the level of the external canthus of each eye, as well as to the supra- and infra-orbital foramens of the right eye. Electrode impedances were kept below 3 kΩ. \figref{fig:electrodes} illustrates the distribution of the electrodes used.

\begin{figure}[t]
	\includegraphics[height=.6\textheight]{figures/electrodes_cropped.jpg}
	\caption[2D layout of electrode placement]{2D layout of electrode placement according to the 10--20 system. In this system, each electrode is named by one or two letters to indicate the general brain region (Fp = frontal pole, F = frontal, C = central, P = parietal, O = occipital, T = temporal). Each letter is followed by a number representing the hemisphere (odd numbers = left hemisphere, even numbers = right hemisphere). The lowercase z stands for the number zero, which represents the electrodes on the midline. The higher the number, the greater the distance from the midline.}
	\label{fig:electrodes}
\end{figure}


\subsection{Procedure}
\label{subsec:4.2.4}
After electrode application, participants were seated in a booth with sound insulation on a comfortable chair, in front of a monitor. As there was no active task for this experiment, participants were asked to relax and watch a nature documentary film with no sound \citep[Deep Blue;][]{fothergill_deep_2003}. Participants were informed that during the film they would hear some audio unrelated to the film over the loudspeakers, but they were asked to ignore it.

\largerpage
Participants were presented with standard/deviant f0 contours realised on the same item within condition, but the items across conditions always differed (Latin Square Design: 4 items x 4 oddball conditions). To control for systematic order and frequency effects potentially induced by the exposure to the oddball condition and/or item order, 16 fully-counterbalanced lists were created with different oddball condition order and item, so that each list presented all items and oddball conditions but never the same item across conditions and never the same condition order. Each participant heard only one of the lists (the exact distribution of the lists have been made available on OSF [\url{https://osf.io/nhy4c/}]).
\newpage
Each oddball condition consisted of 1000 trials in total (850 standards and 150 deviants; the inter-stimulus interval was jittered between 450--545 ms in order to achieve the same stimulus onset asynchrony for all items, which was 1147 ms), resulting in a presentation time of approximately 20 min per condition and approximately 1 hour and 20 min in total. The order of the trials was fully randomised, with at least two consecutive standards between deviants in order to avoid deviant stimuli developing their own memory trace. The initial 15 standards of each condition were excluded from all subsequent analyses, as their presentation served the sensory-memory trace formation \citep[e.g.,][]{naatanen_mismatch_2019}, resulting in a total of 3940 trials (985 trials x 4 conditions). \figref{fig:oddball} depicts a schematic illustration of the oddball conditions.

\begin{figure}
	\includegraphics[width=\textwidth]{figures/oddball_cropped.png}
	\caption[Schematic illustration of the oddball condition]{Schematic illustration of the oddball conditions. The four rows present the four different oddball conditions. From top to bottom, the first two rows illustrate the oddball conditions in which accentual rising (in red) and accentual falling (in blue) intonation alternate as standard/deviant within condition; the last two rows depict the boundary rising (in yellow) and boundary falling (in green) oddball alternation within conditions. Each condition started with the presentation of 15 standard sounds to serve the creation of the memory trace. }
	\label{fig:oddball}
\end{figure}

\subsection{Data pre-processing}
\label{subsec:4.2.5}

The data were pre-processed using the Matlab-based toolbox EEGLAB \citep{delorme_eeglab_2004}, which was developed at the Swartz Center for Computational Neuroscience. To reduce computational demands, the first step was to resample the data to 250 Hz. Afterwards, the data were re-referenced to linked mastoids. Next, an independent component analysis (hereafter, ICA) for artifact correction was performed. For ICA decomposition, the EEG was filtered with a 1 Hz high-pass filter to approach stationarity, and a 45 Hz low-pass filter to remove line noise. Subsequently, artifact components (muscle and eye components above 80\%; heart components above 90\%) were automatically detected and removed from the raw EEG data. After artifact rejection, the raw EEG data were filtered with a 0.3 Hz high-pass and a 30 Hz low-pass filter, instead of baseline correction \citep[see][]{friederici_localization_2000, wolff_neural_2008, widmann_digital_2015, maess_high-pass_2016-1, maess_high-pass_2016} Thereafter, the data were epoched from −200 to 1000 ms post stimulus onset. For reproducibility, the pre-processing script can be found on OSF (\url{https://osf.io/nhy4c/}).

\subsection{Post-processing and statistical data analysis}
\label{subsubsec:4.2.6}

Post-processing and statistical analyses were conducted in R, version 4.1.2 \citep{r_core_team_r_2023}. For data processing and visualisation, the R packages \textit{tidyverse} 1.3.1 \citep{wickham_welcome_2019}, and \textit{ggplot2} 3.3.5 \citep{wickham_ggplot2_2016} were used. ERP amplitude was analysed by fitting Bayesian hierarchical regression models using the \textit{brms} 2.17.0 package \citep{burkner_brms_2023}. Data and scripts can be found on OSF.

\subsubsection{{Post-processing}}
\label{subsubsec:4.2.6.1}

After data epoching, to avoid effects of the repeated number of standards (recall that the deviant trials formed only 15\% of total trials), an equal number of standard and deviant trials entered the statistical analyses. To achieve this, only the standards that appeared directly before a deviant were selected, yielding a total of 300 trials (150 standards/150 deviants) per electrode site. 

The current analysis focuses on two ERP effects that are claimed to index activation of pre-attentive and conscious attentional mechanisms, respectively: the MMN and the P3 responses (see \sectref{sec:2.5}). The MMN is a negative auditory ERP component which is traditionally obtained as a difference wave by subtracting the ERPs to standard from those to deviant stimuli \citep[i.e., deviant ERPs – standard ERPs;][]{naatanen_mismatch_2019}. However, this approach requires the use of the grand averaged signal, leading to a great loss of variance in the data. In the current analysis, ERP amplitude was averaged by time window for every participant, electrode site, and trial. This allows for fitting the models on single-trial data and at the same time model variance associated to each participant. 

As discussed in \sectref{sec:2.5}, MMN is reported to typically peak between 100 and 250 ms after stimulus onset \citep[e.g.,][]{duncan_event-related_2009}. MMN is usually followed by a positive ERP component, the P3 response, around 300 ms or later after stimulus onset \citep[P3 latency depends on the complexity of the processing: the more complex the processing, the longer the latency, varying approximately from 250 to 1000 ms; see][]{duncan_event-related_2009}. Nevertheless, there is considerable variability in the definition of the time windows that have been used to analyse these effects, as peak latency has been usually defined on the basis of visual inspection of the difference waves. For example, there have been studies defining time windows with MMN peak latency at around 350 ms \citep[e.g.,][]{emmendorfer_erp_2020}. For this reason, ERP amplitude is here analysed from 0 to 700 ms after stimulus onset in steps of 100 ms, resulting in seven time windows (0--100 ms, 100--200 ms, 200--300 ms, 300--400 ms, 400--500 ms, 500--600 ms, 600--700 ms). Further, as the MMN is well-documented to have a frontocentral topography, and the P3 a frontal distribution \citep[e.g.,][]{duncan_event-related_2009}, for the current analyses a spatial region of interest\footnote{At first, I intended to run the models using all registered electrodes including sagittality and laterality as predictors in the models. However, besides the theoretical reasons I report here, there were also practical reasons that led to the decision of defining one spatial region of interest~-- the models including all scalp electrode sites were highly computationally demanding, to the point of being intractable.} was defined consisting of the AF3, AF4, F3, Fz, F4, FC1, FCz, FC2, and Cz electrode sites.

\subsubsection{{Inference criteria}}
\label{subsubsec:4.2.6.2}

ERP amplitude (in microvolt) was modelled from 0 to 700 ms after stimulus onset by fitting separate Bayesian hierarchical regression models per oddball condition, in steps of 100 ms. Treatment contrast was used to code the predictor \textsc{sound} (levels; standard/deviant) with the level \textit{standard} serving as the reference level. Random effects for \textsc{subjects} included full variance--covariance matrices \citep[e.g.,][]{barr_random_2013}. Weakly informative priors were used for all the parameters (the full prior specification can be found in the script provided on OSF), as they allow for a wide range of effect sizes but control for unreasonably large effects. All models ran with four chains and 4000 iterations with a warm-up period of 2000 iterations. There were no divergent transitions and all Ȓs were close to 1, showing that chains mixed without issues. Model fits were also visually inspected using the posterior predictive check function. 

In the following section, inferences are drawn from the posterior distributions of the parameters. For this, posterior estimates, the low and high boundaries of the 90\% credible interval (CrI) of the estimate, and the posterior probability that the estimate falls on one side of zero (e.g., \textit{P}(\textit{β} < 0) = 0.95) are reported. When almost all of the posterior mass for an estimate lies on one side of zero, zero is not included in the 90\% CrI (by a reasonably clear margin), and the posterior probability \textit{P} is close to one, the effect is considered reliable.  

\section{Results}
\label{sec:4.3}

\figref{fig:erp} illustrates the grand averaged ERP waves per oddball condition time-locked to the onset of stimulus, as depicted by the vertical dashed line (see also \figref{fig:diff} for the grand averaged difference waves obtained by subtracting standards from deviants per oddball condition). Left panels show ERPs to the two accentual oddball conditions in which accentual falls (in blue) and rises (in red) alternate as standard/deviant sounds. Right panels present ERPs to the two boundary oddball conditions in which boundary falls (in green) and rises (in yellow) alternate as standard/deviant sounds. All panels depict ERPs to standards in black, and EPRs to deviants in colour. Visual inspection of the waves reveals that all contour types, when presented as deviants, evoked an MMN activity relative to their corresponding standard stimulation with an onset around 200 ms. For all deviants except the accentual fall (left top panel), the MMN activity (coloured area between ERPs to standards and ERPs to deviants) appears to last for two successive time windows (200--400 ms), with the accentual rising deviant (left bottom panel) exhibiting the most pronounced effect. MMN to falling deviants (both accentual and boundary; top panels) appears to be followed by an additional P3, at the 400--500 ms time window (shaded area between waves) for the accentual falls, and at a later time window (500--600 ms) for boundary falls.

\begin{figure}
	\includegraphics[width=\textwidth]{figures/ERPs.png}
	\caption[Grand averaged ERP waves per oddball sequence] {Grand averaged ERP waves (per oddball sequence) recorded to the onset of stimulus (illustrated by the vertical dashed line) over time (x-axis) at the AF3, AF4, F3, Fz, F4, FC1, FCz, FC2, and Cz electrode sites. Negative voltage is plotted upwards. Left panels show ERPs to the two accentual oddball conditions in which accentual falls (in blue) and rises (in red) alternate as standard/deviant sounds. Right panels present ERPs to the two boundary oddball conditions in which boundary falls (in green) and rises (in yellow) alternate as standard/deviant sounds.}
	\label{fig:erp}
\end{figure}

\begin{figure}
	\includegraphics[width=\textwidth]{figures/diffplot.png}
	\caption[Grand averaged difference waves across conditions]{Grand averaged difference waves time-locked to the onset of stimulus (illustrated by the vertical dashed line) and obtained in AF3, AF4, F3, Fz, F4, FC1, FCz, FC2, and Cz electrode sites. Negative voltage is plotted upwards. Difference waves were obtained by subtracting standard waves from deviant waves per oddball condition. Left side panels depict difference waves of accentual falling deviants (in blue) and accentual rising deviants (in red). Right side panels show difference waves of boundary falling deviants (in green) and boundary rising deviants (in yellow).}
	\label{fig:diff}
\end{figure}

In what follows, I will first report results on the difference between deviants and standards (i.e., standard sound vs. deviant sound) per oddball condition to detect whether MMN and P3 responses were elicited by deviant sounds relative to their standard stimulation. A brain activity is referred to as an MMN  when a reliable negative difference between deviant and standard sounds is detected during the 100–200 ms, 200–300 ms, and 300–400 ms time windows. A brain activity is identified as a P3 response when a reliable positive difference between deviant and standard stimulation is observed in the 300–400 ms window or at a later time window. Subsequently, I will summarise MMN and/or P3 effects (if any) in the presence of rises vs. falls, regardless of their position (accentual/boundary), aiming to establish whether the direction of the pitch movement affected the evoked brain response. Lastly, I will sum up MMN and/or P3 effects in the light of the position of the rise and fall to find whether the phonological association (i.e., status) of the tonal event affects the elicitation of MMN and/or P3 responses. 

\subsection{Accentual contours}
\label{subsec:4.3.1}

\figref{fig:acc} illustrates the posterior distributions of the estimated effects for the difference between standard and deviant sounds in the two accentual oddball conditions per time window, in ascending order. The estimated differences between accentual falling deviants and accentual rising standards (oddball condition 1) are depicted in blue; the estimated differences between accentual rising deviants and accentual falling standards (oddball condition 2) are depicted in red.  

\begin{figure}
	\centering
	\includegraphics[width=\textwidth]{figures/acc_posteriors.png}
	\caption[Posterior distributions of the estimated effects for the difference between standard and deviant sounds in the two accentual oddball conditions]{Posterior distributions of the estimated effects for the difference between standard and deviant sounds in the two accentual oddball conditions. Accentual falling deviants vs. accentual rising standards are depicted in blue. Accentual rising deviants vs. accentual falling standards are shown in red. Time windows are presented in ascending order. Error bars around the posterior means represent 66\% (thick) and 90\% CrI.}
	\label{fig:acc}
\end{figure}

\subsubsection{{Accentual falling deviant vs. accentual rising standard}}

\tabref{tab:4.2} presents model details on ERP amplitude differences between deviant accentual falling contours and standard accentual rising contours per time window. In short, the model revealed no reliable differences between deviant accentual falling and standard accentual rising contours in the 0--100 ms, 100--200 ms, and 500--600 ms time windows. By contrast, between 200 and 300 ms, the model very strongly favours the interpretation of a negative-going difference in amplitude, indicating an MMN activity, although the 90\% CrI includes zero on the margin. The model also estimated a reliable positive difference between 400 and 500 ms, indicating that the MMN to falling deviants was followed by an additional P3, and a further negative-going difference between 600 and 700 ms.

\begin{table}
	\small
	\caption[Overview of model's results for condition 1]{Tabular overview of modelling results for accentual falling deviants vs. accentual rising standards per time window.}
	\label{tab:4.2}
	\begin{tabular}{l r r r r r r} % add l for every additional column or remove as necessary
		\lsptoprule
		Time & \textit{β} & SE & Low & High & Evid. & Post.\\ %table header
	  Window &            &    & CrI & CrI & Ratio & Prob \\ %table header
		\midrule
		0--100 ms & 0.24 & 0.24 & --0.16 & 0.63 & 5.46 & \textit{P}(\textit{β} > 0) = 0.85\\
		100--200 ms & 0.30 & 0.21 & --0.06 & 0.65 & 10.89 & \textit{P}(\textit{β} > 0) = 0.92\\
		200--300 ms & --0.46 & 0.27 & --0.90 & 0.00 & 19.78 & \textit{P}(\textit{β} < 0) = 0.95\\
		300--400 ms & --0.39 & 0.43 & --1.10 & 0.33 & 4.61 & \textit{P}(\textit{β} < 0) = 0.82\\
		400--500 ms & 0.95 & 0.46 & 0.18 & 1.72 & 44.20 & \textit{P}(\textit{β} > 0) = 0.98\\
		500--600 ms & --0.48 & 0.45 & --1.20 & 0.28 & 6.04 & \textit{P}(\textit{β} < 0) = 0.82\\
		600--700 ms & --0.98 & 0.33 & --1.54 & --0.43 & 532.33 & \textit{P}(\textit{β} < 0) = 1.00\\
		\lspbottomrule
	\end{tabular}
\end{table}

\subsubsection{{Accentual rising deviant vs. accentual falling standard}}

\tabref{tab:4.3} outlines model details for ERP amplitude contrasts between accentual rising deviants and accentual falling standards per time window. For these contrasts, the model estimated a reliable positive difference in amplitude in the 0 to 100 ms and 100 to 200 ms time windows. For the next two successive time windows, 200--300 ms and 300--400 ms, compelling evidence for a negative difference in amplitude was found, suggesting the presence of MMN activity. For the remaining time windows, the model did not suggest any reliable amplitude differences.

\begin{table}
	\small
	\caption[Overview of model's results for condition 2]{Tabular overview of modelling results for accentual rising deviants vs. accentual falling standards per time window.}
	\label{tab:4.3}
	\begin{tabular}{l r r r r r r} % add l for every additional column or remove as necessary
		\lsptoprule
		Time & \textit{β} & SE & Low & High & Evid. & Post.\\ %table header
		Window &            &    & CrI & CrI & Ratio & Prob \\ %table header
		\midrule
		  0--100 ms & 0.38 & 0.22 & 0.03 & 0.74 & 25.58 & \textit{P}(\textit{β} > 0) = 0.96\\
		100--200 ms & 0.40 & 0.21 & 0.06 & 0.75 & 34.56 & \textit{P}(\textit{β} > 0) = 0.97\\
		200--300 ms & --0.46 & 0.29 & --0.93 & 0.00 & 18.51 & \textit{P}(\textit{β} < 0) = 0.95\\
		300--400 ms & --3.49 & 0.39 & --4.13 & --2.83 & Inf & \textit{P}(\textit{β} < 0) = 1.00\\
		400--500 ms & --0.55 & 0.36 & --1.12 & 0.05 & 14.78 & \textit{P}(\textit{β} < 0) = 0.94\\
		500--600 ms & 0.52 & 0.42 & --0.16 & 1.22 & 8.22 & \textit{P}(\textit{β} > 0) = 0.89\\
		600--700 ms & --0.08 & 0.38 & --0.71 & --0.53 & 1.42 & \textit{P}(\textit{β} < 0) = 0.59\\
		\lspbottomrule
	\end{tabular}
\end{table}

Overall, these results show that both accentual rises and falls, when presented as deviant sounds, elicited an MMN activity, starting in the 200 to 300 ms time window. For the accentual rising deviants, the MMN activity appears to have a longer duration than it does for the accentual falling deviants, as in this condition a negative difference is present for two successive time windows (i.e., from 200 to 400 ms). For the accentual falling deviants, the results show that their MMN activity is followed by an additional P3 response in the 400 to 500 ms time window, yet accentual rising deviants did not engender such a brain response.

\subsection{Boundary contours}
\label{subsec:4.3.2}

\figref{fig:bound} presents posterior distributions of the estimated effects for the differences between standard and deviant sounds in the two boundary oddball conditions per time window, in ascending order. The estimated differences between boundary falling deviants and boundary rising standards (oddball condition 3) are depicted in green; the estimated differences between boundary rising deviants and boundary falling standards (oddball condition 4) are depicted in yellow.

\begin{figure}
	\centering
	\includegraphics[width=\textwidth]{figures/bound_posteriors.png}
	\caption[Posterior distributions of the estimated effects for the difference between standard and deviant sounds in the two boundary oddball conditions]{Posterior distributions of the estimated effects for the differences between standard and deviant sounds in the two boundary oddball conditions. Boundary falling deviants vs. boundary rising standards are depicted in green. Boundary rising deviants vs. boundary falling standards are shown in yellow. Time windows are presented in ascending order. Error bars around the posterior means represent 66\% (thick) and 90\% CrI.}
	\label{fig:bound}
\end{figure}

\subsubsection{{Boundary falling deviant vs. boundary rising standard}}

Model details on the ERP amplitude differences between boundary falling deviant and boundary rising standard contours per time window are presented in \tabref{tab:4.4}. Briefly, the model did not provide reliable differences in the 0 to 100 ms, 100 to 200 ms, 400 to 500 ms, and 600 to 700 ms time windows, but showed compelling evidence for a negative difference in amplitude between 200--300 ms and 300--400 ms, indexing the elicitation of an MMN activity. Although the model estimated a positive difference between 400 and 500 ms, this difference was not reliable. Nonetheless, the model provided compelling evidence for another positive-going difference in the 500 to 600 ms time window, suggesting the presence of a P3 response.

\begin{table}
	\centering
	\caption[Overview of model's results for condition 3]{Tabular overview of modelling results for boundary falling deviants vs. boundary rising standards per time window.}
	\label{tab:4.4}
	\begin{tabular}{l r r r r r r} % add l for every additional column or remove as necessary
		\lsptoprule
		Time & \textit{β} & SE & Low & High & Evid. & Post.\\ %table header
		Window &            &    & CrI & CrI & Ratio & Prob \\ %table header
		\midrule
		0--100 ms & 0.25 & 0.24 & --0.15 & 0.65 & 5.93  & \textit{P}(\textit{β} > 0) = 0.86\\
		100--200 ms & 0.17 & 0.23 & --0.20 & 0.54 & 3.68 & \textit{P}(\textit{β} > 0) = 0.79\\
		200--300 ms & --0.60 & 0.25 & --1.02 & --0.19 & 107.11 & \textit{P}(\textit{β} < 0) = 0.99\\
		300--400 ms & --1.68 & 0.46 & --2.43 & --0.92 & 1599 & \textit{P}(\textit{β} < 0) = 1.00\\
		400--500 ms & 0.39 & 0.49 & --0.43 & 1.20 & 3.82 & \textit{P}(\textit{β} > 0) = 0.79\\
		500--600 ms & 0.85 & 0.44 & 0.14 & 1.58 & 34.4 & \textit{P}(\textit{β} > 0) = 0.97\\
		600--700 ms & --0.05 & 0.37 & --0.66 & 0.55 & 1.24 & \textit{P}(\textit{β} < 0) = 0.55\\
		\lspbottomrule
	\end{tabular}
\end{table}

\subsubsection{{Boundary rising deviant vs. boundary falling standard}}

Model details on the ERP contrasts between boundary rising deviants and boundary falling standards per time window are shown in \tabref{tab:4.5}. No evidence was found for a difference in the 0 to 100 ms, 100 to 200 ms, and 400 to 500 ms time windows, but the model provided compelling evidence for a negative difference in amplitude for the 200 to 300 ms and 300 to 400 ms time windows, indicating an MMN activity. For the last two time windows (500--600 ms and 600--700 ms), the model estimated a negative-going difference. Here, although the 90\% CrI includes zero on the margin, it could still favour the interpretation of a (marginal) difference.

\begin{table}
	\centering
	\caption[Overview of model's results for condition 4]{Tabular overview of modelling results for boundary rising deviants vs. boundary falling standards per time window.}
	\label{tab:4.5}
	\begin{tabular}{l r r r r r r} % add l for every additional column or remove as necessary
		\lsptoprule
		Time & \textit{β} & SE & Low & High & Evid. & Post.\\ %table header
		Window &            &    & CrI & CrI & Ratio & Prob \\ %table header
		\midrule
		0--100 ms & 0.11 &  0.18 & --0.19 & 0.40 & 2.75 & \textit{P}(\textit{β} > 0) = 0.73\\
		100--200 ms & 0.05 &  0.21 & --0.29 & 0.39 & 1.49 & \textit{P}(\textit{β} > 0) = 0.60\\
		200--300 ms & --0.46 &  0.16 & --0.72 & --0.19 & 379.95 & \textit{P}(\textit{β} < 0) = 1.00\\
		300--400 ms & --2.39 &  0.43 & --3.10 & --1.68 & Inf & \textit{P}(\textit{β} < 0) = 1.00\\
		400--500 ms & --0.65 &  0.53 & --1.51 & 0.23 & 8.36 & \textit{P}(\textit{β} < 0) = 0.89\\
		500--600 ms & --0.73 &  0.45 & --1.48 & 0.00 & 18.75 & \textit{P}(\textit{β} < 0) = 0.95\\
		600--700 ms & --0.77 &  0.48 & --1.57 & 0.03 & 17.43 & \textit{P}(\textit{β} < 0) = 0.95\\
		\lspbottomrule
	\end{tabular}
\end{table}

I have shown that rising and falling boundary contours highlight an MMN activity relative to their corresponding standard stimulation, lasting from 200 to 400 ms. Furthermore, brain responses to boundary rises and falls differ in that the MMN to boundary falls is followed by a P3 response, while there is no evidence for boundary rises eliciting a subsequent positivity. 

\subsection{Interim summary}
\label{subsec:4.3.3}

\figref{fig:all} summarises all posterior distributions of the estimated differences per oddball condition in the time windows of interest (i.e., MMN and P3 time windows). \tabref{tab:4.6} presents an overview of the estimated differences between rising and falling contours as well as between accentual and boundary contours. These differences point towards a distinct relevance of contour direction (i.e., rise or fall) and linguistic context for the attentional mechanisms.

\begin{figure}
	\centering
	\includegraphics[width=\textwidth]{figures/all_posteriors.png}
	\caption[Posterior distributions of the estimated effects for the difference between standard and deviant sounds across all oddball conditions in the time windows of interest]{Posterior distributions of the estimated effects for the differences between standard and deviant sounds across oddball conditions in the time windows of interest (MMN time windows: 200--300 ms and 300--400 ms; P3 time windows: 400--500 ms and 500--600 ms). From top to bottom, boundary rising deviants vs. boundary falling standards are depicted in yellow, boundary falling deviants vs. boundary rising standards are depicted in green, accentual rising deviants vs. accentual falling standards are depicted in red, and accentual falling deviants vs. accentual rising standards are depicted in blue. Error bars around the posterior means represent 66\% (thick) and 90\% CrI.}
	\label{fig:all}
\end{figure}
\largerpage
Comparing accentual rising to falling deviants, it was found that both contours elicit an MMN activity with an onset in the 200 to 300 ms time window. Whereas there is no quantitative difference at the beginning of the MMN elicitation between accentual rises and falls, the MMN to accentual rises is prolonged over the subsequent time window (300--400 ms), reflecting a longer-lasting effect compared to the accentual falls. What is more, accentual falls after MMN elicitation engender a subsequent P3 in the 400 to 500 ms time window, a response that is not evoked by accentual rises.

\begin{table}[t]
	\caption[Overview of the estimated differences between rising and falling contours as well as between accentual and boundary contours]{Tabular overview of estimated differences between rising and falling contours as well as accentual and boundary contours in the time windows of interest.}
	\label{tab:4.6}
	\begin{tabular}{l r r r r r r} % add l for every additional column or remove as necessary
		\lsptoprule
		Time & \textit{β} & SE & Low & High & Evid. & Post.\\ %table header
		Window &            &    & CrI & CrI & Ratio & Prob\\ %table header
		\midrule
		Rises \textit{vs} falls & & & & & &\\
		\textit{Accentual} & & & & & &\\
		\midrule
		MMN (200--300 ms) & 0.00 & 0.39 & --0.65 & 0.64 & 1.02 & \textit{P}(\textit{β} < 0) = 0.50\\
		MMN (300--400 ms) & --3.10 & 0.59 & --4.07 & --2.15 & Inf & \textit{P}(\textit{β} < 0) = 1.00\\
		P3 (400--500 ms) & 1.50 & 0.59 & 0.54 & 2.47 & 132.33 & \textit{P}(\textit{β} > 0) = 0.99\\
		\midrule
		Rises \textit{vs} falls & & & & & &\\
		\textit{Boundary} & & & & & &\\
		\midrule
		MMN (200--300 ms) & 0.14 & 0.30 & --0.35 & 0.64 & 0.46 & \textit{P}(\textit{β} < 0) = 0.31\\
		MMN (300--400 ms) & --0.71 & 0.62 & --1.74 & --0.28 & 7.11 & \textit{P}(\textit{β} < 0) = 0.88\\
		P3 (500--600 ms) & 1.58 & 0.63 & 0.56 & 2.61 & 144.45 & \textit{P}(\textit{β} > 0) = 0.99\\
		\midrule
		Accent \textit{vs} Boundary & & & & & &\\
		\textit{Rises} & & & & & &\\
		\midrule
		MMN (200--300 ms) & 0.00 & 0.33 & --0.56 & 0.53 & 1.01 & \textit{P}(\textit{β} < 0) = 0.50\\
		MMN (300--400 ms) & --1.10 & 0.58 & --2.05 & --0.15 & 36.21 & \textit{P}(\textit{β} < 0) = 0.97\\
		\midrule
		Accent \textit{vs} Boundary & & & & & &\\
		\textit{Falls} & & & & & &\\
		\midrule
		MMN (200--300 ms) & --0.14 & 0.37 & 0.37 & 0.46 & 1.89 & \textit{P}(\textit{β} < 0) = 0.65\\
		MMN (300--400 ms) & --1.29 & 0.63 & --2.36 & --0.25 & 53.79 & \textit{P}(\textit{β} < 0) = 0.98\\
		\lspbottomrule
	\end{tabular}
\end{table}

Let me now turn to the rising vs. falling comparison in the boundary conditions. The findings here are quite similar. Both boundary rises and falls evoke an MMN activity, lasting for two successive time windows (200--400 ms), with a tendency for the boundary rises to show a stronger negative effect in the 300--400 ms time window. The evidence from the model is not reliable enough to claim with confidence that the MMN to boundary rises is more pronounced than the MMN to boundary falls. However, the MMN to boundary falls, similarly to the MMN to accentual falls, is followed by an additional P3. There is no such evidence for boundary rises.

Considering the phonological status of the rise and fall, namely their position in the prosodic structure (accent/boundary), the MMN to falling contours is of a similar magnitude in the starting time window (200--300 ms), regardless of phonological status, while the MMN to boundary falls is prolonged over the subsequent time window (300--400 ms) compared to the activity evoked by the accentual falls. Moreover, the MMN to both falling deviants is followed by an additional P3, although at different time windows, at the 400 to 500 ms window for accentual falls, and at a later time window (500--600 ms) for boundary falls. Zooming in on rising contours, although the onset of the MMN activity does not differ as a function of the phonological status of the rise (accentual/boundary), accentual rises exhibit the most pronounced effect in the 300 to 400 ms time window. Finally, no evidence is found for a P3 brain response in the presence of either accentual or boundary rises.

I return to these findings in \sectref{sec:4.4}, where I will argue that the observed brain responses in relation to the tested contours could indicate different neurocognitive processes for intonational rises and falls, due to both speech sound complexity and linguistic or context-specific interpretation.

\section{Discussion}
\label{sec:4.4}

Unlike previous neurophysiological research on unexpected auditory changes in a stream of repetitive stimulation, which focused primarily on non-linguistic stimuli, the current work explored the neural responses to rising and falling pitch attributable to accents and boundary tones of sequentially presented lexical items in German. The main aim of this study was twofold. Using ERP data in an oddball paradigm, this study investigated the relevance of intonational rises for attention orienting, as well as to what extent the phonological status of the rise plays a role. It is important to note that I discuss only MMN- and P3-related effects, as these are the brain responses that relate to my research questions. In this discussion, I will first put the findings in the broader context of rising vs. falling intonation, irrespectively of the prosodic structure, and their relevance for attention orienting (Hypothesis 1). Subsequently, I will discuss whether the phonological status of the rise and the fall (accent vs. boundary) modulates attentional resources differently (Hypothesis 2). Finally, I will briefly discuss some implications of the current findings for the neural architecture of speech perception.

\subsection{{The processing of rises and falls and their contribution to attention orienting}}

The obtained results provide evidence for distinct neurocognitive processes for intonational rises and falls in the context of an oddball list. This is reflected in the different brain responses observed during the online processing of the pitch contours: rising pitch contours engendered MMN activity, while falling pitch contours evoked an MMN-P3 complex. Whilst MMN indexes automatic (i.e., pre-attentive) processes related to involuntary attention switch towards the deviant pitch contour, the MMN-P3 complex indicates that processes at the pre-attentive level subsequently activate processes at the conscious level, bringing the perception of the deviant contour into awareness and to voluntary attention.

These processing patterns might reflect the presence of different mechanisms or routes for signal-driven (i.e., bottom-up) and context-driven (i.e., top-down) processes. Although the former have been the target of previous research utilising the oddball paradigm, the presentation of lexical items in the present study has the potential to give rise to contextual expectations for a particular list intonation. The results of this study indicate that, in a linguistic context, the acoustic signal is not the only source for attention orienting, as would be expected in a ``meaningless" context of pure sine waves. Pitch, like any other acoustic property, is processed as the sensory input unfolds, while at the same time, expectations for the forthcoming input are formed incrementally \citep[e.g.,][]{rohr_signal-driven_2021}. In a linguistic context, the generation of expectations can be based both on the sensory information of the input (signal-driven) and on contextual meaning \citep[context-driven; e.g.,][]{rohr_signal-driven_2021}. It is worth elaborating on how the present results elucidate this last point.

In two of the oddball conditions, falling pitch contours were presented as deviant sounds, with their corresponding standard stimulation consisting of repetitive rises. In such sequences, the listener can potentially build predictions derived from two different sources. First, the auditory processing system is able to predict prospective sounds by detecting regularities in the sensory input \citep[signal-driven expectations; e.g.,][]{naatanen_role_1990, sussman_dynamic_2001, friston_free-energy_2010, vachon_broken_2012}. Thus, the listener, after being exposed to a repetitive sequence of stimuli with rising pitch, predicts and anticipates that the next auditory event will again be a rising contour. Predictions can also arise from the linguistic interpretation of the context \citep[context-driven expectations; for predictions in language, see, among others,][]{schumacher_backward-_2015, rohr_signal-driven_2021}. The oddball paradigm resembles a list context because it presents (auditory) events repeatedly and sequentially. On top of the sequential presentation of the stimuli in this study, naturalistic pitch contours realised on real words were used, approximating an even more natural list context.\footnote{The words all refer to food items that could conceivably be used in, e.g., a shopping list.} List intonation in German, as mentioned in \sectref{sec:3.5}, typically involves rising pitch on non-final and penultimate items (indicating continuity) followed by a fall on the final item \citep[indicating finality; e.g.,][]{baumann_prosody_2001, peters_phonological_2018}. Thus, the repetitive rising stimulation, as a natural and appropriate contour on non-final list items, denotes that the list is not over yet. The repetitive/standard rises, therefore, might have elicited additional predictions driven by the contextually created meaning. Such an expectation is the anticipation that the list will come to an end at some point. Therefore, the listener, first, anticipates a rising contour on the basis of the sensory information that is available in the repetitive auditory stimulation (signal-driven expectation) and, second, given the available contextual meaning, expects that at some point the list will be over, thus anticipating a falling contour (context-driven expectation). Recall now that when the deviant fall was presented, an MMN-P3 complex was elicited, indicating that, first, the violation of the anticipated rising contour activated a pre-attentive response of an involuntary attention switch to the unexpected falling contour. Subsequently, a conscious or voluntary attention orientation towards this falling contour was observed, potentially induced by the validation of the context-driven expectation, that is, the anticipation that the list would at some point come to an end. Similar findings are reported in a study by \citet{liu_online_2016} on the neural processing of tone and intonation in Mandarin Chinese. Crucially, the authors observed that P3 is modulated by the context (question vs. statement) for the falling contour Tone4. The present findings are also in accord with \citet{rohr_signal-driven_2021} who showed, in an ERP study, that signal- and context-driven cues consume attentional resources at different processing stages: signal-driven cues attract attention at an early processing stage, while context-driven cues attract attention at a later processing stage (for more on these studies, see \sectref{sec:3.4}).

Now let us move to the other two oddball conditions in this study, where the standard stimulation consisted of repetitive falling pitch, occasionally interspersed with rising pitch. Such sequences give the feeling of the presentation of isolated events, as the repetitive/standard falling intonation is contextually an inappropriate/unnatural pitch contour for non-final items of a sequence. Hence, such sequences might allow the listener to only build signal-driven expectations. Put differently, based on the sensory input, the listener anticipates that the next sound will be a falling contour again, but cannot generate a prediction over and above the purely signal-based one, as this repetitive signal is already unexpected in the context of a list (i.e., incongruency in the prosodic realisation on non-final items in the list context). Recall now that the presentation of the deviant rises evoked an MMN activity, showing that the prediction generated from the sensory input was violated, activating an automatic involuntary attention switch to the unexpected rising contour. Brain responses related to conscious/voluntary attentional processes, such as the P3, were not observed for rising deviants.

To better understand how these distinct processes for the rising and falling deviants, as reflected in the observed brain responses, link to attention orienting and prominence-cueing, a joint consideration of these findings is required. The starting observation is that all pitch contours, when presented as deviants, evoked an MMN activity, indexing orientation towards the violation in the acoustic signal. In particular, rising pitch deviants, either on a stressed syllable or at the boundary of a word, engendered the most pronounced MMNs. This shows that pitch rises attract more involuntary attention. Falling pitch deviants elicited less pronounced MMNs, followed by a pronounced P3. This indicates that pitch falls also attract some involuntary attention which, however, ultimately leads to conscious attention orientation. This suggests that rising pitch, as an acoustically salient cue, causes an auditory looming effect at the pre-attentive stage, whereas falling pitch appears to be interpreted as linguistically prominent information within the list context. Thus, its processing is affected by the contextual meaning, which activates conscious attentional mechanisms. The results of this study appear to differ from \citet{hsu_brain_2015} (for a detailed review of this study, see \sectref{sec:2.4} and \sectref{sec:3.3}) in that \citeauthor{hsu_brain_2015} found a P3 response to rising pitch changes at a speaker’s normal pitch level (i.e., not elevated).

In what follows, I suggest that the current findings do not, in fact, contradict those of \citeauthor{hsu_brain_2015}. Crucially, the stimuli used in the current study were different from those used in \citeauthor{hsu_brain_2015}, as the current study used real words as opposed to a simple /ɑ/ (at normal and resynthesised elevated pitch levels) and sine waves. An additional difference (also described in \sectref{sec:3.3}) is that the rise or fall in the \citeauthor{hsu_brain_2015} stimuli involved a change in pitch from one stimulus to the other, i.e., across stimuli. By contrast, in this study, the rise or fall appears within stimuli. \citeauthor{hsu_brain_2015} suggest that sudden pitch rises in speech demand more attentional resources than sudden falls, and their presence in speech (in comparison to simple sine waves) activates additional conscious processing mechanisms. In the present study, where the speech signal is the only signal that listeners encounter, I conjecture that it is the linguistic/contextual meaning that draws voluntary attention, rather than speech per se. In other words, one could argue that in both \citeauthor{hsu_brain_2015} and in the current study, P3 appears to be elicited by some kind of available ``meaning", and not so much by pitch direction. In \citeauthor{hsu_brain_2015}, ``meaning" emerges from speech in comparison to the meaningless sine waves. It is the sudden rising and not the sudden falling pitch that evokes the P3 because rising pitch is acoustically more salient, but if it was only the direction of pitch and not the additional information of ``meaning", then pitch rises in sine waves would have elicited a P3 as well. In the current study, ``meaning" arises from the context. The rising deviant condition, although acoustically salient~-- and linguistically prominent, as in German prominence is often encoded through intonational rises~-- happens to be contextually inappropriate, and thus contextually ``meaningless", just like the sine waves in \citeauthor{hsu_brain_2015}. By contrast, the falling deviant condition, although acoustically less salient, is contextually appropriate as it transmits the linguistic meaning of the list context. Such a list context may not be as available to speakers of Mandarin, where local f0 changes are affected by lexical tone.

It has further been shown that different degrees of prosodic prominence trigger signal-driven processes to a different extent \citep[e.g.,][]{rohr_signal-driven_2021}. This is also evident in the current results. Specifically, a positive relation between prosodic prominence and MMN activity is observed: as prosodic prominence increases (rises being more prominent than falls), MMN activity intensifies. Nonetheless, the P3 response appears to be unaffected by the prosodic prominence level of the deviant. Instead, it is context-induced. Specifically, the context of the list in the case of the falling deviant (i.e., a sequence of rises) appears to trigger an anticipation of the end of the list, typically marked by a falling contour. Although the falling deviant is acoustically less salient than the rising deviant, it activates conscious attentional mechanisms as reflected in the P3. Thus, in real-time processing of sequentially repeated stimuli, the amount and level of attention allocated to the deviant stimulus appear to be determined by a combination of signal- and context-based properties, when contextual meaning is available. In turn, when contextual meaning is unavailable or inappropriate, signal-inherent properties guide attention orienting. It thus appears that the meaning of the sequence shifts the prominence status of the contours which, in turn, shifts the stage of attention orientation, activating different routes in processing (pre-attentive/conscious, involuntary/voluntary).

Overall, the current results show that the processing of rising and falling pitch contours produces distinct brain responses, which are claimed to be related to two core attentional processes: the automatic involuntary attention switch at the pre-attentive stage (reflected by MMN generation) and the voluntary attention orientation at the conscious stage (reflected in the P3 signature). Pre-attentively, signal-based cues appear to be fundamentally important. At this processing stage, pitch rises, as the most acoustically salient events, attract the most attention. This highlights the pivotal role of pitch rises not only in cognition, as previous MMN studies have shown \citep[e.g.,][]{naatanen_early_1978, alain_brain_1994, doeller_prefrontal_2003}, but also in spoken language, suggesting that there is something intrinsic in their acoustic signal, regardless of whether rises take the form of a pure sine wave, a speech sound in isolation or speech in context. This is likely due to the rising acoustic properties being so salient that they are able to warn and prepare the listener's nervous system about an important event happening in the environment, activating basic attentional resources that in turn elicit automated or appropriate adaptive responses (these adaptive responses have been described by \citeauthor{sokolov_higher_1963} \citeyear{sokolov_higher_1963} as reflexes; for more see \sectref{sec:2.1}). In spoken communication, this entails orienting a listener’s attention towards the most important part of the uttered message, which is crucial for effective interpretation and speech planning in drawing listener attention to an upcoming turn. The essential nature of signal-based cues for the pre-attentive processing stage has also been highlighted by studies reporting that not only salient acoustic contrasts, but also the timing of the acoustic cues are decisive for evoking attention-related brain responses pre-attentively \citep[e.g.,][]{ren_early_2009, tsang_erp_2011, li_process_2018, rohr_signal-driven_2021}.

Further, the findings of the current study manifest that conscious, voluntary attention is modulated by the meaning that intonation encodes in a given context, and not by pitch direction itself \citep[for similar attentional processes modulated by context-driven cues, see][]{liu_online_2016}. Here, the conscious processing stage is activated by pitch falls. Although pitch falls are not salient enough and are less prominent in the prosodic prominence hierarchy \citep[e.g.,][]{baumann_perceptual_2015}, it is evident that (in this case) language experience and context-driven expectations overwrite the signal-induced properties \citep[see][]{bishop_information_2013}. In other words, acoustic saliency can be overridden by expectations that emerge from context, making pitch falls highly relevant and thus linguistically prominent in the list context. This finding is in line with studies showing that signal-driven cues can be overwritten at later processing stages by prior linguistic- or discourse-based expectations, as well as by recent speech experience \citep[e.g.,][]{bishop_information_2013, kakouros_making_2018, roettger_listeners_2020, ventura_attention_2020, bornkessel-schlesewsky_rapid_2022}.


\subsection{{The role of the phonological status of the rise (and the fall) in attention orienting}}

Another question this study has advanced concerns the phonological status of the rises and falls, that is, their position in the prosodic structure and the contribution of this status to the observed attentional processes. Crucially, the results of this study demonstrated differences in the magnitude and latency of the MMN activity when comparing accentual to boundary contours (see \figref{fig:all}). Specifically, when comparing accentual to boundary rises, accentual rises exhibit the more pronounced MMN. In turn, when comparing accentual to boundary falls, boundary falls show a prolonged MMN latency as opposed to the very short activity evoked by accentual falls. I argue that during online processing, intonation contours, as a complex part of the speech signal, are processed holistically, meaning that attention is not oriented necessarily towards a specific point in the f0 trajectory. Rather, it is the pitch contour direction that modulates attention orienting (drawing attention to a higher level of linguistic representation such as a~-- putative~-- phrase).

To illustrate this point, consider the rising/falling contours investigated: both accentual and boundary contours were realised on sequentially presented trisyllabic lexical items (like \textit{Banane} `banana'). Hence, the domain of the realisation of the pitch contour in this study is the word; thus, each word is an intonational phrase on its own. Therefore, a complete pitch contour on every word consists of both a pitch accent and a boundary tone. Specifically, the pitch configuration which in this study is referred to as  \textit{accentual rise} is followed by a high boundary at the end of the word, whereas the pitch configuration which is referred to as \textit{accentual fall} is followed by a low boundary at the end of the word. Likewise, the \textit{rising boundary} is preceded by a low accent, whereas the \textit{falling boundary} is preceded by a high accent. Hence, by considering the entire pitch configuration as the current word stimuli unfold, we can better understand whether a specific part of the contour is relevant for attracting (more) attention or not. The current results indicate that the structural position of the pitch event (accent vs. boundary) has a secondary role in attentional processes. Specifically, it was found that 

\begin{enumerate}[label=\roman*.]
	\item a rising pitch contour is globally more successful in attracting attention than a falling one, regardless of the position of the rise or the fall, while secondarily,
	\item a rise on the stressed syllable leads to more attention than a rise on the final unstressed syllable, and
	\item high pitch on the stressed syllable in falling contours (H*) attracts more attention than falling pitch on the stressed syllable (H + L*).
\end{enumerate}

The finding that accentual rises induce a greater MMN effect, attracting more attention than boundary rises, is not surprising. It has already been claimed in previous work on prosodic prominence that accentual rising contours (and especially steep rising pitch accents) are the most prominent contours in the prosodic prominence hierarchy and that they demand more attentional resources than falling pitch accents \citep[e.g.,][]{baumann_perceptual_2015, rohr_signal-driven_2021}. Let us now zoom in on the finding that accentual rises attract more attention than boundary rises, considering it in light of the entire contour and the periodic energy that characterises these rising contours (for results on periodic energy, see \sectref{subsec:4.2.2}). In the accentual rising contours, pitch starts low, already rises quite steeply during the stressed syllable, and remains high towards the last syllable, and thus at the end of the word (an appropriate analysis following the German Tones and Break Indices (GToBI) annotation scheme, would be: L + H* H-\%; see \citet{grice_german_2005}). The rising pitch movement takes place on the lexically stressed syllable, which has high periodic energy, making the pitch strongly transmitted on that syllable. See \figref{fig:f0contours} for mean and individual intonation contours, \figref{fig:deltaFzero} and \figref{fig:mass} for measures related to periodic energy. Now, in boundary rising contours, pitch starts mid-level, remains at this level or falls slightly during the stressed syllable, and rises towards the end of the word. In GToBI, this would be annotated as L* L-H\%. The rising part of the contour is restricted to the final unstressed syllable, which has considerably lower periodic energy than the stressed syllable. Thus, the f0 in the boundary rise condition is transmitted more weakly than in the accentual rise condition. Therefore, although both rising contours attract more attention than the falling ones, within the rising category, the accentual rising contours are produced with more periodic energy and thus attract more attention than the boundary rising contours.

Turning to the comparison of accentual and boundary falling conditions, a more sustained MMN effect is evoked by boundary falls rather than accentual falls. Consider that, in the realisation of the accentual falling contours, pitch starts high at the beginning of the word (enabling it to fall) and falls throughout the stressed syllable. It then continues on the same low level until the end of the word (in GToBI, H + L* L-\%). In the realisation of the boundary falling contours, pitch starts on a relatively mid-level and rises slightly towards the stressed syllable, enabling it to fall at the end of the word (in GToBI, H* L-\%). The high accent on the stressed syllable (and before the fall at the boundary) leads to higher periodic energy on this part of the signal in comparison to the accentual falling contour, where pitch is already falling during the stressed syllable (this is conversant with the finding that H* is more prominent than H + L*; see \citeauthor{baumann_perceptual_2015} \citeyear{baumann_perceptual_2015}). Therefore, the high pitch accent and the amount of periodic energy in the signal before the boundary fall potentially contribute to the perception of the boundary falling contour condition as more prominent, leading to greater attention as compared with the accentual fall.

One could argue that this constitutes evidence for the attention orienting function and the prominence value of the accent, and thus the structural importance of the stressed syllable. However, remember that in comparing boundary rising to boundary falling contours, boundary rises were found to attract more attention than falls, although the stressed syllable in the former contour bears a low pitch accent, as opposed to the boundary fall that is preceded by a high pitch accent. It appears, therefore, that what takes place (in terms of f0 movement) at discrete prosodic positions (head/edge) is not sufficient to orient attention on its own. In contrast, it is the entire contour that guides attention. Prosodic positions appear to have a supplementary/secondary role in the modulation of attentional resources. This becomes evident when the investigated pitch contours are arranged according to their elicited MMN effects, assuming a decrease in MMN effect (and thus attention attraction) from left to right:


\begin{center}
 \begin{itemize}
	\item[] \textsuperscript{RISE}[accentual rises (\textcolor{red}{L + H*} H-\%) > boundary rises (L* L-\textcolor{dan}{H\%})]
	\centering
	\item[]	>
	\item[] \textsuperscript{FALL}[boundary falls (H* \textcolor{forgreen}{L-\%}) > accentual falls (\textcolor{blue}{H + L*} L-\%)]
 \end{itemize}
\end{center}
 
First and foremost, a rising pitch configuration globally attracts more attention than a falling one. Second, and within rising categories, when the rise occurs in different structural positions, it attracts more attention when it coincides with the head/stressed syllable (in the accentual rising condition) compared with when it occurs at the boundary (in the boundary rising condition). Now, when the rise occurs at the same structural position, it attracts more attention if it is a steep rise (L + H* in the accentual rising condition) as opposed to a shallow rise or just (level) high pitch (H* in the boundary falling condition). Finally, within falling categories, high pitch preceding a fall on the accented syllable (in the boundary falling condition) attracts more attention than a simple fall (in the accentual falling condition). These results are in line with what has previously been reported on prominence marking in German \citep[e.g.,][]{baumann_perceptual_2015, baumann_what_2018}. Crucially, it was shown here that these subtle prosodic differences are reflected in the pre-attentive MMN response.

\subsection{{Some implications for the neural architecture of language}}

The neural architecture of language perception is complex and dynamic \citep[for discussions, see among others,][]{gandour_hemispheric_2004, hickok_cortical_2007, assaneo_lateralization_2019}. It involves two fundamentally different neural mechanisms, a signal-based mechanism and a meaning-based one, expressed at distinct processing stages. On the one hand, the current results show that the intrinsic properties of the sensory input, that is, signal-driven cues (\citeauthor{assaneo_lateralization_2019} \citeyear{assaneo_lateralization_2019} refers to this as the \textit{intrinsic auditory mechanism}), are essential for speech perception at the early pre-attentive processing stage, with rises taking priority over falls precisely because of their acoustic saliency. The fundamental role of signal properties, and thus the special role of rises at the pre-attentive processing stage (feeding the signal-based mechanism) is also shown in previous MMN studies to unexpected sound changes \citep[for a review of studies, see Chapter \ref{ch:2}, and among others,][]{naatanen_mismatch_2019} but also in studies investigating the neural processing of linguistically meaningful variations, both at the lexical and the postlexical level \citep[e.g.,][]{ren_early_2009, tsang_erp_2011, li_unattended_2018, rohr_signal-driven_2021}. On the other hand, the current findings suggest that at a later, conscious, processing stage, the linguistic functions of the stimuli (called \textit{top-down/externally driven mechanism} by \citeauthor{assaneo_lateralization_2019} \citeyear{assaneo_lateralization_2019}) modulate speech perception (feeding the meaning-based mechanism). Specifically, the construction of meaning was found to attract voluntary attention towards meaningful aspects, here reflected by the use of words and the minimal context of list intonation. This is in line with findings from previous studies showing that top-down activities are decisive for the activation of later processing stages \citep[see, among others,][]{ren_early_2009, liu_online_2016, assaneo_lateralization_2019, rohr_signal-driven_2021}.

Overall, it appears that the neural architecture of the auditory cognition and spoken language share a common pool of mechanisms that are crucial for the orienting response in both auditory and speech processing. Chapter \ref{ch:6} brings together the results from this and the following chapter in discussing implications for the neural architecture of general cognition and spoken language. 

\section{Summary}
\label{sec:4.5}

The present study is a novel attempt to unravel the neural mechanisms that underlie attention orienting towards unexpected linguistic intonational changes by revisiting the idea of an attentional bias towards pitch rises (as opposed to pitch falls) and extending it from a general cognitive level (auditory looming) to a linguistic one. This study shows that, in a linguistic context, the amalgamation of different cues evokes qualitatively and quantitatively distinct neural responses tied to two core attentional mechanisms:

\begin{enumerate}[label=\arabic*.]
\item the involuntary attention switch, a mechanism at the pre-attentive processing stage which is reflected in MMN elicitation, and
\item the voluntary attention orienting, a mechanism at the conscious processing stage which is reflected in the P3 signature.
\end{enumerate}

In its most concise form, the main finding of this study is that, in spoken language, rising intonation takes on a special role in attracting involuntary attention, whereas contextual meaning is essential for voluntary attention orienting. Rising pitch evokes the largest MMN, indicating that it leads to a greater involuntary and automated attention switch compared with falling pitch. This holds regardless of the phonological association of the rise in the prosodic structure (head vs. edge). It thus appears that there is a biological basis for the cross-linguistic use of rises for attracting attention towards informative parts of the message~-- even though rises are grammaticalised differently across languages in being encoded as either pitch accents or edge tones. Furthermore, the appropriateness of the intonational pattern in a given context and, specifically in this study, in the list context, is decisive for voluntary attention orienting. Here, falling pitch, although acoustically less salient than rising pitch, engenders an additional P3, indicating that the contextual meaning prevails over or even “cancels out” the signal properties \citep[for a discussion on the interplay between sensory input and top-down activities in language processing, see][]{bornkessel-schlesewsky_rapid_2022}.

Overall, the role of rises is fundamentally important, not only for auditory cognition but also for (successful) language communication, because of their intrinsic acoustic properties that activate involuntary attentional resources, pre-attentively, regardless of whether they are carried by a simple sine wave, a speech sound in isolation, or meaningful speech. However, in spoken language, the acoustic signal is not the only source of information; hence, the importance of rises is mitigated by contextual meaning, which appears to activate voluntary attention as it is required for modulating/updating conscious processing stages and mental representations during language comprehension. Capitalising on the current findings, the next chapter delves further into the relevance of domain-final (the reflex of phrase-final edge tones) intonational rises (and falls) for attention orienting in spoken language by measuring pupil dilation response (PDR), a rigorous psychophysiological index of auditory attention orienting (see Chapter \ref{ch:2}).
