\chapter{General discussion}
\label{ch:6}

The present work made a novel attempt to unravel the cognitive and functional relevance of intonational prominence for attention orienting in the linguistic domain. This was achieved by focusing on the temporal unfolding of the neurophysiological and pupillary underpinnings of the orienting response (OR) towards intonational rises and falls embedded in oddball sequences, along with the relevant language-specific expectations. Attention orienting has mainly been investigated within cognitive domains like vision and audition, uncovering a sensitivity of the orienting response towards the physical properties of events (visual/acoustic saliency). The novelty of the current work is that, departing from neurocognitive studies on auditory perception, it raised the question of how the function of the orienting response extends from acoustic saliency to prominence, and hence from auditory cognition to speech. In what follows, I will first summarise the main results of the two studies presented in this book (\sectref{sec:6.1}). Next, I will discuss the general implications of the main findings (\sectref{sec:6.2}), and point towards future research directions (\sectref{sec:6.3}). The chapter closes with concluding remarks in \sectref{sec:6.4}.
 

\section{Summary and main observations}
\label{sec:6.1}

The aim of the study reported in Chapter \ref{ch:4} was to examine the neurophysiological correlates (ERPs) of attention orienting towards intonational rises and falls when presented unexpectedly in a stream of repetitive auditory stimuli. Specifically, the processing of rises and falls was investigated through the use of an oddball paradigm, where sequences of repetitive stimuli were occasionally interspersed with a deviant stimulus, allowing for elicitation of a mismatch negativity (MMN). Whereas previous oddball studies on attention towards unexpected sounds involving pitch rises were conducted on non-linguistic stimuli, this study used lexical items with natural intonation contours as stimuli. The main findings of this study can be summarised as follows:

\begin{enumerate}[]
	\item The direction of the pitch contour (the signal-driven cue) is primarily relevant for involuntary attention orienting (RQ1 [rises], RQ4 [architecture]).
	 
	\item Similarly, the appropriateness of the contour in the linguistic context (the context-driven cue) is primarily relevant for voluntary attention orienting (RQ3 [expectations], RQ4 [architecture]).
	
	\item By contrast, the effect of the rise holds regardless of the phonological status of the pitch event (RQ2 [edges]). This is not predicted by standard intonational theory, where only rises on heads (i.e., pitch accents) are expected to play a role in attention orienting.
\end{enumerate}

More specifically, the obtained results indicate that intonational rises play a special role in attention orienting at a pre-attentive processing stage, whereas contextual meaning (here a list of items) is essential for activating attentional resources at a conscious processing stage. This is reflected in the activation of distinct brain responses: rising intonation evokes the largest MMN; falling intonation elicits a less pronounced MMN followed by a P3 (reflecting a conscious processing stage). The data further reveal a complex interplay between the phonological status (i.e., accent/head marking vs. boundary/edge marking) and the direction of pitch change in their contribution to attention orienting. Attention is not necessarily oriented towards a specific position in prosodic structure (head or edge). Rather, it was found that the intonation contour itself, and the appropriateness of the contour in the linguistic context, are the primary cues to the two core stages of attention orienting, pre-attentive and conscious orientation respectively, whereas the status of pitch accent or edge tone plays only a minimal role. This could be either because of holistic processing, or because rises in either phonological position serve as attention orienters, in that both are meaningful rises, and not just transitions from one phonological event to another.

Chapter \ref{ch:5} delved into the pupillary underpinnings of attention orienting towards domain-final (the reflex of phrase final edge tones) intonational rises and falls. Using a \textit{changing-state} oddball paradigm, in which auditory sequences of sequentially ordered (seriatim) ascending numbers (standards) were occasionally interspersed with an out-of-the-sequence number (deviant), this study investigated whether domain-final rising pitch in speech takes on a special role in attention orienting by measuring listeners’ pupil dilation response (PDR). Further, this study placed a special focus on the contribution of individual cognitive variability to attention orienting. The main findings of this study can be summarised as follows:

\begin{enumerate}[]
	\item Intonational rises take on a special role in attracting attention, also at a more conscious level, and even when this rise is attributable to edge tones (RQ1 [rises], RQ2 [edges], RQ3 [expectations], RQ4 [architecture]).
	
	\item The cognitive bandwidth deployed by individuals in processing auditory deviances is critical for the effective activation of voluntary attention and protection of the attentional system from potential overloading (RQ4 [architecture]).
\end{enumerate}

Elaborating on the results, intonational rises are salient, prominent, and contextually relevant (within the chosen experimental set-up), and therefore activate both involuntary and voluntary attention, which is reflected in listeners' PDRs. Given that domain-final rises and falls function in a similar way in the context of a list, namely signalling the end of a sub-sequence in a longer list, rises take priority over falls precisely because of their prominence. This finding highlights the attention orienting function of intonational rises, suggesting that domain-final rises on deviant stimuli also enhance the ability of the deviant to attract attention. Further, this finding strengthens the argument of Chapter \ref{ch:4} that the direction of the pitch contour and its appropriateness in the linguistic context are fundamental for attention orienting, whereas the phonological status of the pitch event is not of primary relevance. Finally, this study shows that individual cognitive variability contributes to the function of attention orienting. Cognitive variability was measured on the basis of three cognitive skills: processing speed, inhibitory ability, and working memory capacity (WMC). The results reveal that individuals differ in the effective mechanisms they tend to use, which in turn affects how they process deviances.

\section{General implications of main findings}
\label{sec:6.2}

I now turn to the synthesis of the obtained results, offering a unified view on the cognitive and functional relevance of intonation for attention orienting, demonstrating that auditory cognition and spoken language share a common pool of mechanisms that pertain to the orienting response.

\subsection{{The relevance of intonation for the architecture of attention orienting in spoken language}}

Going back to the conceptual origins of OR, the hypothesis of two OR stages~-- involuntary and voluntary~-- and their attribution to different mechanisms~-- bottom-up and top-down~-- was already discussed by early philosophers, theorists, and psychologists (for more, see Chapter \ref{ch:2}). Ivan \citeauthor{pavlov_selected_1955}, the father of the OR concept, distinguished two signal systems. The first signal regulates the involuntary OR, initiated by the sensory properties of a stimulus, and the second signal system regulates the voluntary OR, evoked by complex thought processes. Importantly, the interplay between the mechanisms pertaining to OR and either low-level sensory cues or high-level cognitive operations was also attested by later psycho- and neurophysiological studies on auditory attention \citep[for reviews, see][]{naatanen_attention_1992, naatanen_mismatch_2019}. Essentially, the nature of the cue determines which mechanism will be fed. The different mechanisms, in turn, activate distinct attentional processing stages, resulting in involuntary or voluntary orienting. 

One of the questions that the present research has raised is whether the neurocognitive architecture of attention orienting is similar across auditory cognition and spoken language (RQ4 [architecture]). The experimental findings reported in this work show that there are commonalities across the two, as the mechanisms that pertain to the different stages of the orienting response are subserved by similar neural and pupillary underpinnings. 

In its most concise form, the main finding of this work is that signal-driven cues are decisive for activating an involuntary orienting response, while context-driven\footnote{Context-driven and meaning-based terms have been used interchangeably throughout this book, as they refer to the semantics of a respective stimulus, as well as to meaningful aspects licensed by pragmatics and/or contextually created expectations.} cues are essential for activating voluntary attention orienting. In the following, I will break this finding down by jointly discussing results from the two experimental studies. \figref{fig:6.1} presents a schematic illustration of how signal- and context-driven cues pertain to OR activation according to the present findings. Signal- and context-driven cues are represented by squares and triangles, respectively. The ``square" violation represents signal-driven violations that do not fit a given contextual meaning; the ``triangle" violation represents contextually meaningful signal violations that fit the surrounding context. For example, in the context of the experimental investigation reported in Chapter \ref{ch:4}, rising deviants are linked to the square violation, while falling deviants relate to the triangle violation. In the context of the experimental investigation reported in Chapter \ref{ch:5}, rising deviants pertain to the triangle violation as well.

\begin{figure}
	\centering
	\includegraphics[width=\textwidth]{figures/schematic illustration.png}
	\caption[Schematic illustration of signal- and context-driven cues pertaining to attention orienting]{Schematic illustration of the processing routes of the OR activations, as a function of signal- or meaning-based cues. The square represents signal-driven violations; the triangle represents contextually meaningful signal violations \citep[inspired from][12]{naatanen_auditory_2011}.}
	\label{fig:6.1}
\end{figure}

   
The qualitatively and quantitatively distinct ERP and PDR responses show that, on the one hand, the physical properties of the sensory input (signal-driven cues) are essential for speech processing at an early pre-attentive stage, attracting involuntary attention (RQ1 [rises]). On the other hand, the meaning-based properties of the same input (context-driven cues) are crucial for speech processing at a conscious stage (RQ3 [expectations]), leading thus to voluntary attention. Both pitch rises and falls, when presented as deviants, evoked an MMN, the neural correlate that indexes pre-attentive processes related to involuntary attention orienting \citep[e.g.,][]{naatanen_mismatch_2019}, with intonational rises evoking the largest MMN, owing to the acoustic saliency of the signal (Chapter \ref{ch:4}). Similarly, intonational rises and falls marking deviant numbers evoked increased PDR effects compared to deviants that do not differ intonationally from the standard numbers (Chapter \ref{ch:5}). These results indicate that the prosodic marking of deviants (i.e., signal-driven cues) activates pre-attentive attentional resources towards the relevant violation (see \figref{fig:6.1}, where both square and triangle violations initiate an involuntary OR). The saliency level of these signal-driven cues modulates the robustness of the OR, such that the more salient the properties of the speech signal, the more robust the involuntary OR. This is in line with findings from neurophysiological and pupillometric studies on auditory perception showing that the OR is sensitive to the physical saliency of the signal, such that the more salient the physical properties of a sound, the more enhanced the OR towards it \citep[e.g.,][]{naatanen_early_1978, alain_brain_1994, doeller_prefrontal_2003, rinne_superior_2005, rinne_two_2006, bach_rising_2008, macdonald_effects_2011, hsu_brain_2015, liao_human_2016, wetzel_infant_2016, marois_eyes_2018}. 

Evidence reinforcing the fundamental importance of signal-based cues at the pre-attentive level also comes from linguistic research showing that the elicitation of attention-related brain responses is sensitive both to the salient acoustic contrasts in the signal and the timing of the acoustic cues \citep[e.g.,][]{ren_early_2009, tsang_erp_2011, li_unattended_2018, rohr_signal-driven_2021}. The present results are compatible with the aforementioned studies on different languages like Mandarin Chinese, Cantonese, and German, suggesting a cross-linguistic similarity (for more on these studies, see Chapter \ref{ch:3}). 

Furthermore, deviant falls were shown in Chapter \ref{ch:4} to elicit an additional P3 response, reflecting that its processing has been brought into awareness \citep[e.g.,][]{polich_normal_1986}. Although falls are acoustically less salient than rises, they are contextually more appropriate (within the given experimental setting), and thus contextually more prominent, when presented as deviants after repetitive rises in the context of a list. This is because this whole intonational pattern reflects one of the most natural list intonation patterns in German (for more on list intonation in German, see Chapter \ref{ch:3}). In the experimental setting of Chapter \ref{ch:5}, although both deviant domain-final rises and falls share a similar contextual meaning, that of signalling the end of a sub-sequence in a list, rises are linguistically more prominent. This is evidenced by the fact that deviant domain-final rises evoked more prolonged PDRs than falls, leading to voluntary attention orienting, and thus conscious processing of the relevant deviant.\footnote{As discussed in Chapter \ref{ch:2}, the latency of attention-related PDRs indicates the processing stage during which the OR is activated: transient dilations reflect pre-attentive attentional processes; more prolonged responses show that subsequent conscious processing has been activated \citep[see][]{strauch_pupillometry_2022}.} These results indicate that the linguistically meaningful aspects of intonation licensed by contextually created expectations are decisive for the elicitation of voluntary OR (see illustration in \figref{fig:6.1} where only the triangle violation that fits the surrounding context leads to a voluntary OR). This is compatible with linguistic research on the interaction between signal- and context-driven processing \citep[e.g.,][]{liu_online_2016, rohr_signal-driven_2021} showing that voluntary attention is modulated by meaning-based expectations derived from the pragmatic context.

Therefore, just as in auditory cognition, so too in spoken language, a number of cues can activate the orienting response, with distinct cues tied to the two core attentional stages. Pre-attentively, attention is driven by the acoustic saliency of intonation, feeding bottom-up mechanisms like signal-driven expectations. Consciously, attention is driven by the meaning and the prominence relations that intonation encodes in a given context, feeding top-down mechanisms based on, e.g., language experience and context-driven expectations. Nonetheless, spoken language entails a higher level of complexity than auditory cognition, as signal-driven and context-driven cues~-- or in other words saliency and prominence~-- may be combined in speakers' intonational choices.

\subsection{{The cognitive \& functional contribution of intonational rises to attention orienting}}

The attention orienting function of rising intonation has been of particular interest for this work, due to its inherent association with prominence. As introduced in Chapter \ref{ch:1} (and as shown elsewhere in this book), there are many similarities between the notions of linguistic prominence and cognitive saliency. Departing from neurocognitive studies on auditory perception reporting an attentional bias towards salient rising as opposed to falling sounds, I posed the question whether this attentional bias towards rising signals extends from acoustic saliency to prominence, and hence from general cognition to spoken language (RQ1 [rises]). The present work answers this question in the positive. Concisely, the main finding is that intonational rises indeed take on a \textit{special} role in attention orienting, particularly~-- but not only~-- at the pre-attentive stage. 

That intonational rises are crucial for speech processing pre-attentively, attracting involuntary attention, is shown in the large MMN response rises evoke when presented as deviants (Chapter \ref{ch:4}). This indicates the advantage of rises in mismatch detection, causing a greater involuntary switch of the attentional resources towards deviant rises compared to deviant falls. This holds regardless of the phonological status of the rise as a pitch accent or an edge tone. This finding highlights the pivotal role of pitch rises not only in cognition, as previous MMN studies have shown \citep[e.g.,][]{naatanen_early_1978, naatanen_brain_1980, alain_brain_1994, doeller_prefrontal_2003, hsu_brain_2015}, but also in spoken language. I argue that this finding further shows that it is precisely the intrinsic acoustic saliency of rises that renders them fundamentally important for involuntary attention-related processes. In auditory perception, sounds with rising acoustic properties have been attributed a robust warning function, related to life-threatening or emergency situations, precisely because of their acoustic saliency, as they prepare the nervous system by activating attentional resources \citep[e.g.,][]{bach_rising_2008} which in turn elicit automated or appropriate adaptive responses \citep[called reflexes by][]{sokolov_higher_1963, sokolov_perception_1963}. In speech perception, this equates to intonational rises alerting and orienting listeners' attentional resources towards, e.g., important linguistic entities. This process is crucial for effective interpretation and speech planning.

The contribution of intonational rises to attention orienting in spoken language is not only restricted to the pre-attentive stage. As we have seen, meaning-based cues are decisive for voluntary orienting, and thus for attentional processes at a conscious stage. I showed in Chapter \ref{ch:4} that in speech, the acoustic saliency of intonational rises (and thus their attention orienting function) is mitigated by contextual meaning: when intonational rises are contextually inappropriate, they do not attract attention at a conscious level. By contrast, falls, which are not salient but contextually relevant, and thus prominent, do attract voluntary attention. Further, Chapter \ref{ch:5} showed that when both intonational rises and falls are contextually relevant, rises trump falls. I argue that this finding highlights the attention orienting function of intonational rises, even when they are attributable to edge tones, suggesting that when rises and falls function similarly in a given context, rises on deviant stimuli enhance the ability of these deviants to attract attention also at a more conscious level. Rises in this case are not only acoustically more salient, but also linguistically more prominent than falls. This is in line with studies on the neural processing of linguistically meaningful pitch variations in different languages \citep[e.g.,][]{tsang_erp_2011, ventura_attention_2020, rohr_signal-driven_2021}. For example, \citet{tsang_erp_2011} showed in a study on the pre-attentive processing of Cantonese tones, that MMN is sensitive to pitch contour direction and height. \citet{ventura_attention_2020} found in an EEG study that rising pitch accents guide attention towards semantic incongruencies, as they lead to more elaborate processing of the focused information. Likewise, \citet{rohr_signal-driven_2021}, investigating the signal- and context-driven effects of German prosodic prominence in an EEG study, reported a P3 elicitation, a neural correlate of conscious attention, to intonational rises on the stressed syllable of the critical words, and claimed that it is the linguistically ``\textit{meaningful} rises that orient attention, rather than rises \textit{per se}" (\citeyear[][44]{rohr_signal-driven_2021}).

Another question I raised in this work is whether rises at the edges of constituents attract attention to the same extent as accentual rises do (RQ2 [edges]). The present findings also answer this question in the positive. The experimental investigations reported in Chapters \ref{ch:4} and \ref{ch:5} both provide evidence for the attention orienting function of intonational rises, even when they are attributable to edge tones. That both accentual and boundary intonational rises have been found to function as attention orienting devices in the present research challenges the typological prediction posited by the autosegmental-metrical (AM) theory of intonational phonology \citep[e.g.,][]{ladd_intonational_2008, barnes_autosegmental-metrical_2022, grice_autosegmental-metrical_2022}, where pitch accents are strictly associated with a prominence-cueing function, and edge tones are associated with a mere phrasing function. Serial recall studies \citep{savino_intonation_2020, rohr_effect_2022, grice_rises_2024} have already provided first indications against this strict dichotomy, in reporting that rising edge tones marking the final item of non-final triplets boost recall accuracy of the whole triplet, orienting attention to the whole domain. The two present experimental investigations corroborate these findings by showing that not only accentual but also edge tone-related intonational rises can cue prominence, by orienting attention towards a deviant bearing them. All these findings posit a challenge for important aspects of prosodic theory and typology in pointing towards a prominence-cueing function of rising edge tones. 

Overall, the current results contribute to our understanding of how intonation affects listeners' detection of prediction-violations or mismatches, and thus the orienting of attention towards them by showing i) that intonational rises indeed make information more salient, attracting attention pre-attentively (serving like warning cues), and ii) that the attention orienting function of rises is mediated both by prominence and the expectations within a given context, leading to voluntary attention orienting.

\subsection{{The role of expectations in attention orienting}}

It appears that the auditory and linguistic processing systems share common features relating to attention orienting, with expectations (i.e., predictions) being a driving force. A question that this work has put forward in relation to expectations is whether language- and/or context-specific expectations affect attention orienting (RQ2 [expectations]). As Chapter \ref{ch:2} thoroughly reviewed, one of the mechanisms deemed to underlie attention orienting is the \textit{expectancy violation} mechanism, suggesting that the precondition for an OR is the violation of rule-based expectancies \citep[e.g.,][]{naatanen_role_1990, sussman_dynamic_2001, vachon_broken_2012}. 

According to \citeauthor{naatanen_auditory_2011}'s (\citeyear{naatanen_auditory_2011}) prevalent model, the auditory processing system develops sensory-memory representations related to the regularities between successive sound events in the environment (see the features of the auditory input integrated in the formation of representations in \figref{fig:6.1}). On the basis of these representations, the processing system forms expectations by predicting the properties of upcoming sound events. When these anticipatory predictions are violated because the properties of the upcoming event deviate from what is expected, attention-calling mechanisms are activated. As we already saw, the first attention-calling stage is the involuntary attention switch, activated by signal-driven cues. After the involuntary OR towards the auditory deviance, this deviance can be brought into awareness, either by a strong attention-calling process or through the formation of an attentional trace. This trace depends on voluntary maintenance by the control mechanisms (voluntary OR). When it is voluntarily maintained, it is available to top-down processes as well as to long-term memory, leading to conscious perception and semantic activation (see \sectref{sec:2.3}). The findings in the present work confirm that the same expectancy violation mechanism activates the subsequent attention-calling mechanisms in spoken language. 

Let me exemplify this last point by referring to the data from Chapter \ref{ch:4}. Here, listeners were presented with classic oddball lists, namely sequences of repetitive sounds (standards) occasionally interspersed with a rare sound (deviant). Specifically, rising and falling intonation alternated as standard/deviant sounds across different conditions. Rising and falling intonation was introduced on real lexical items in German referring to food items, giving rise to a list context and the relevant language-specific expectations. In this study, two layers of expectations were formed. First, our auditory processing system is able to predict prospective sounds by detecting regularities in the input. Therefore, listeners, after being exposed to the repetitive sequence of stimuli, either with rising or falling pitch, predict that the next event would bear the same pitch contour (signal-driven expectations). When this was not the case, an MMN was elicited, indicating an involuntary attention switch towards the deviant. Remember that all deviants elicited an MMN~-- this means that all deviants, regardless of their acoustic saliency, were able to activate the first attention-calling stage,\footnote{However, the more salient the intonational contour, the more robust the MMN response.} simply by violating signal-driven expectations. Predictions can, however, also arise from the linguistic interpretation of the context (context-driven expectations): the oddball paradigm resembles a list, and, additionally, the natural pitch contours used on real words enhanced the list context. Thus, when rising stimulation was presented repetitively (standard stimuli), as a natural and appropriate intonation on non-final list items, this appeared to elicit additional contextual predictions, such as the anticipation of the end of the list. Therefore, when the falling deviant was presented, it validated the context-driven expectations, as it is a cue to finality~-- even though it violated the signal-driven expectations. This is reflected in the MMN-P3 complex elicited by this deviant. Pre-attentively, the falling deviant, violating the signal-driven expectations, activates an involuntary OR, shown in MMN, but subsequently, manages to be voluntarily maintained and move into a conscious processing stage by validating the context-driven expectations, shown in the P3. These context-driven expectations were not formed when falling pitch was repetitively presented and occasionally interrupted by rising deviants, probably because the repetitive standard falling stimulation is already less expected on non-final items in the context of a list, blocking the generation of predictions over and above the pure signal.
\newpage
Let us now move to the data from Chapter \ref{ch:5}. In this study, listeners were presented with seriatim (i.e., sequentially ordered) ascending numeric lists (standards), occasionally interspersed with an out-of-the-sequence number (deviant). This deviant number featured either the same intonation as the standard numbers (neutral), or either final rising or final falling intonation. The pupillometric data show that all deviant numbers did elicit a PDR, regardless of intonation. This finding further corroborates the claim that attention orienting is underpinned by an expectancy violation mechanism. The numeric lists inherently give rise to expectations about forthcoming numbers. The processing system forms memory representations by detecting the regularity in the environment, that is, the seriatim pattern, and predicts that the next number will follow the same pattern. This pattern was violated, once the deviant number was presented, causing an OR as reflected in the PDRs. Listeners were not only exposed to the seriatim pattern, but they were also exposed to the signal-driven cues with which this pattern was produced. Therefore, listeners not only anticipated a following stepwise ascending number, but they also expected this number to feature neutral intonation, given that all previous numbers were produced with this realisation. When the deviant violated both the seriatim and the intonational pattern, the OR response was stronger, as reflected in the increased PDRs. This result shows that signal-driven cues garner more robust attentional resources towards the violation. Besides the seriatim and the signal-driven expectations, the listeners appeared to have formed a third layer of expectation, which arose from the linguistic meaning of intonation (context-driven expectations). More specifically, the neutral intonation of the standard stimulation is a natural list intonational pattern for non-final items in German. Thus, the expectations regarding a forthcoming number were enhanced by the linguistic meaning of the neutral intonation, indicating that the list was not over yet. Numeric lists are not infinite, however, and listeners therefore also expected that the list would end at some point, and would have listened out for the corresponding intonational cue. Now, both final rises and falls can mark the end of a smaller or larger unit. Thus, when final rises and falls were presented, they, on the one hand, validated the anticipation of the end of the list. On the other hand, the deviant was included in the same unit as the standard numbers, thereby violating the anticipated seriatim and signal-driven patterns. The partial validation of the context-driven expectation, along with the violations of the other two expectation layers, led to a conscious OR.

These findings are compatible with previous linguistic research which has shown that the processing or the perception of prosodic mismatches is shaped by expectations licensed both by signal properties and meaningful aspects of prosody within a pragmatic context \citep[e.g.,][]{liu_online_2016, rohr_perceptual_2020, rohr_signal-driven_2021, rohr_influence_2022}. Studies have also shown that the processing of such mismatches can be affected by language-specific expectations \citep[e.g.,][]{chandrasekaran_sensory_2009, ventura_attention_2020, grice_perception_2024}. The present results are also in line with studies reporting that prior linguistic- or discourse-based expectations, or even recent speech experience, can overwrite signal-driven expectations at later processing stages \citep[e.g.,][]{bishop_information_2013, kakouros_making_2018, roettger_listeners_2020}. \citet{rohr_signal-driven_2021} have claimed that attention orienting related to predictive processing is reflected in discrete stages in the construction of a mental representation during real-time comprehension, based on the fact that the amount of attention drawn to a stimulus is defined by signal-based cues combined with context-driven expectations.

The present findings contribute to the argument that the OR in speech is indeed underpinned by the expectancy violation mechanism proposed for the auditory domain. This is in accord with what has been suggested about the general functions of the linguistic processing system by \citet{schumacher_backward-_2015}. As reviewed in Chapter \ref{ch:3}, \citet{schumacher_backward-_2015} claim that the linguistic processing system, like the auditory processing system, constantly develops mental representations, related to entities in the linguistic environment. Based on these representations, predictions for the upcoming entities are derived. Prediction errors \citep{friston_free-energy_2010} in language, which occur from item-context mismatches, have further been found to elicit detection and attention-calling processes \citep[for more details on the architecture of the prediction-updating cycle, see also][]{rohr_signal-driven_2021, repp_what_2023}. The more enhanced the violation of a prediction, the higher the processing cost for an entity \citep[or, the more difficult its reconciliation with prior discourse; see][]{rohr_signal-driven_2021}. After the detection of violations, the mental model needs to update its representations. More specifically, as reported in \citet{rohr_signal-driven_2021}, if listeners allocated their attention towards some prosodic mismatch, topic shift, or novel information after detecting it, they need to incorporate the new knowledge in their representations by repairing, reorganising, and updating their mental model \citep[see also,][]{toepel_catching_2007, dimitrova_less_2012, brouwer_time_2013, schumacher_backward-_2015}.

\subsection{{Neural and pupillary underpinnings of attention orienting}}

Let us now zoom in further on evidence for commonalities in the architecture of auditory cognition and spoken language with regard to attention orienting (RQ4 [architecture]). More specifically, I will discuss the neural and pupillary underpinnings of attention orienting across the two domains. In general, the literature review in Chapter \ref{ch:2} elucidated that OR is expressed in multiple physiological, psychological, motoric, and neural exponents. It has emerged that the MMN and P3 ERP components are rigorous correlates of attention orienting in auditory perception, reflecting the route from involuntary to voluntary orienting \citep[e.g.,][]{escera_neural_1998, naatanen_perception_2001, hsu_mismatch_2023}. Similarly, PDR was shown to be a valid index of auditory attention orienting, highlighting a striking resemblance between MMN/P3 components and PDR composites \citep[e.g.,][]{nieuwenhuis_anatomical_2011, wang_circuit_2015, johansson_computational_2018, marois_eyes_2018, alamia_pupil-linked_2019, strauch_pupillometry_2022, contier_are_2024}. 

The present work has indicated that attention orienting in spoken language is manifested in the same neural and pupillary exponents. This is a crucial finding for two reasons. First, it reinforces the credibility of these OR correlates by attesting their presence in a different domain. Second, it shows that the function, or the architecture, of the orienting response is subserved by the same neural pathways and pupillary underpinnings across domains.

Overall, the common pool of mechanisms crucial for the orienting response in both the auditory realm and in speech is reflected in a number of ways, such as

\begin{enumerate}[]
	\item the involuntary \& voluntary OR stages, which are initiated by a similar interplay between distinct mechanisms (bottom-up \& top-down) and different cues (signal-driven \& context-driven);
	
	\item the expectancy violation mechanism, which is responsible for activating attention-calling mechanisms like OR; and 
	
	\item the neural and pupillary underpinnings that subserve the orienting network.
\end{enumerate}

\subsection{{Individual variability and other top-down operations}}

A final point I will discuss is the role of individual variability in the orienting response. Chapter \ref{ch:5} explored the impact of listener-specific cognitive variability on attention orienting. The aim was to better understand how other cognitive operations interact with attention orienting across individuals. In its most concise form, the main finding is that individuals do vary in the use or control of their attentional resources towards incoming information, utilising different top-down operations to different extents. \figref{fig:6.2} presents a schematic illustration of other cognitive operations involved in the process of involuntary attention transitioning to voluntary orienting as a function of individual variability.

\begin{figure}[H]
	\centering
	\includegraphics[width=0.8\textwidth]{figures/cognitive-variability-attention-schema.png}
	\caption[Schematic illustration of other top-down operations pertaining to attention orienting]{Schematic illustration of other top-down operations pertaining to attention orienting. The figure presents sets of possible activations of other cognitive operations leading from involuntary to voluntary orienting, as a function of individual cognitive profiles.}
	\label{fig:6.2}
\end{figure}

That the cognitive bandwidth deployed by individuals in processing auditory deviances in spoken language is critical for an effective activation of voluntary attention (and protection of the attentional system from potential overloading) is reflected in the observed PDR variation across individual cognitive profiles. In particular, Chapter \ref{ch:5} reports that all deviants elicited PDR responses. This indicates that, when deviants were presented, listener attention was involuntarily oriented towards them, regardless of intonation. Nonetheless, the three intonational patterns marking the deviants evoked different PDR modulations as a function of individual cognitive variability. 

The participants in this study tended to form two subgroups with regard to cognitive variability, operationalised with measures of inhibitory ability, processing time, and WMC. The first group of individuals (see paths represented by arrows on the right side of \figref{fig:6.2}) appeared to have both an efficient processing system, which evaluates the importance of deviants successfully, and sufficient suppressing mechanisms, which were activated when needed. That is, when a deviant event, e.g., an intonational rise, is salient and/or prominent, the processing system renders it important. Subsequently, suppressing operations, including inhibition and WMC resistance, are withdrawn and involuntary attention gives way to voluntary attention, bringing the deviant in conscious processing. When the saliency and/or prominence status of the event is low, the processing system renders it unimportant. Consequently, suppressing operations are activated, blocking irrelevant distractions from later stages of orienting and thereby protecting the attentional system from unnecessary overloading. 

For the second group of individuals (see paths represented by arrows on the left side of \figref{fig:6.2}), it is not as clear whether their processing system is efficient in evaluating the importance of deviants. Nevertheless, it is evident that these individuals do not have strong suppressing mechanisms. As a result, deviants garner additional conscious attentional resources regardless of whether their importance has been established or not, thereby maximising attentional processing load. 

On the basis of these results I propose that a deviant’s presentation causes, at first, an involuntary attention switch across all prosodic conditions. From this moment onwards, other top-down operations, like processing speed, inhibitory ability, and WMC, cooperate and continuously interact with the attentional system. The attentional system, in turn, determines whether the pre-attentive involuntary attention switch will lead to full awareness of the deviant, or whether the processing system will continue with the processing of the following input. As we saw, the successful evaluation of the deviant, and thus the activation of voluntary attention, depends on the individual's cognitive bandwidth. A deviant's evaluation by the processing system constitutes the first crucial step in determining which further operations will follow. Individuals may or may not have an effective processing system, leading to either successful or unsuccessful evaluation. The next step involves the activation of suppressing mechanisms like inhibition and WMC resistance. Inhibitory abilities underlie what are perhaps the most crucial operations in protecting the attentional system from potential overloading. Likewise, as we saw, individuals may or may not have efficient suppressing mechanisms to protect their attentional system. The combination of an effective processing system and sufficient suppressing mechanisms is essential, as failure in responding to an important sound change might be life-threatening, but responding to every sound change might exhaust mental resources that are needed elsewhere. In spoken communication this could entail a listener's failure to orient attention towards the most important part of a message, which in turn can affect successful interpretation and speech planning.

As mentioned in Chapter \ref{ch:5}, research on individual variability in auditory attention is scattered, and, to my knowledge, no previous study has directly tested the effect of inhibition or processing speed on auditory attention. With regard to WMC, the present results are compatible with \citet{berti_working_2003} and \citet{sanmiguel_when_2008} in reporting that increased working memory (WM) load, and thus low WMC, attenuates or even prevents auditory distraction. Nonetheless, the present findings stand in contrast with other studies reporting that listeners with high WMC are less susceptible to auditory deviances \citep[for review, see][]{sorqvist_high_2013, hughes_auditory_2014}. \citet{sanmiguel_when_2008} claim that WM load effects are influenced by the type of the auditory deviant or distraction. In this respect, \citet{sanmiguel_when_2008} argue that, in studies where WM load (i.e., low WMC) was found to increase attention towards auditory deviants, those deviants actually were ``competing" stimuli generating task-related conflicts, as in a Stroop task. However, in \citet{sanmiguel_when_2008}, the deviant stimuli were task-irrelevant, orienting attention away from the task. In a similar vein, \citet{sorqvist_high_2013}, showed in a meta-analysis that high WMC and attenuated distraction do not correlate when deviants are task-irrelevant. Given that deviants were also task-irrelevant in the present work, with no conflict being generated, I take these findings as further evidence in support of the present results. Finally, with regard to processing speed and WMC correlation, our results are in line with a study on visual attention by \citet{heitz_focusing_2007} reporting that individuals with high WMC had faster responses than individuals with low WMC.

Overall, the present findings on PDR contribute to research on individual variability by exploring three different cognitive skills: inhibition, processing speed, and WMC. The findings highlight the importance of taking individual cognitive variability into consideration in processing studies, and offer a starting point for further work through the cognitive test battery used for assessing the aforementioned cognitive operations.

\section{Limitations and directions for future work}
\label{sec:6.3}

The present work provided empirical findings that contribute to a better understanding of the relevance of intonation for attention orienting in spoken language. Thus, the current results form a solid basis for future research. Nevertheless, like many other studies, the current experimental designs have some limitations. To this end, in what follows, I propose directions for future work that could deepen our understanding of attention orienting in spoken language and the impact of intonational prominence on it. 

The present work investigated the temporal unfolding of the neural and pupillary underpinnings of attention orienting towards intonational rises and falls embedded in oddball sequences. The construction of the oddball sequences in the two present studies included real words, food items in Chapter \ref{ch:4} and numbers in Chapter \ref{ch:5}, simulating a more natural speech context (that of a list) compared to previous neurocognitive studies. Nonetheless, this context was still very much controlled. To this end, I propose that future research should focus on the investigation of the link between intonation and attention orienting in more ecologically valid contexts. One idea could be the use of semi-spontaneous memory games. For instance, listeners could be presented auditorily with people playing a game like \textit{I packed my bag} \citep{brandreth_everymans_1986}. In the most common version of this game, a speaker starts by saying ``I packed my bag and in it I put..." and refers to an object. The next speaker repeats the sentence with the original object and adds one more. The very next speaker repeats the sentence with the original object, the object named by the second speaker, and adds another one, and so forth. The structure of this game offers an excellent test case, as it creates memory traces of patterns which, of course, can be violated by introducing deviant items in the narrative, which, in turn, can be manipulated prosodically. Scenarios other than packing a bag can, of course, also be used, such as ``I went to the zoo and saw..." \citep{wallace_teaching_1973}, ``I went on a picnic and brought..." \citep{dynes_memory_2017}, ``I went to the shops and bought..." \citep{lauchlan_improving_2013}, or any other context. A more radical idea would be to test even more naturalistic settings, in presenting listeners with a podcast topic or an audio book and manipulate the intonation of some utterances in expected (congruent) and unexpected (incongruent) ways. The unexpected manipulations could serve as deviants, investigating whether or how attention is oriented towards such manipulations compared to the expected manipulations which could serve as standards.

Considering a different type of limitation, this work unravelled the cognitive and functional contribution of intonational rises, attributable both to pitch accents and edge tones, to attention orienting in German, a West Germanic language. In order to test the biological basis of the attention orienting function of rises, as well as the role of edge tones in the orienting response, non-West Germanic languages also need to be investigated. Maltese and/or Maltese English (MaltE) offer two excellent test cases for future research. Due to historical and demographic shifts, Maltese and MaltE have been in contact for many years, and they co-exist in the repertoire of most Maltese speakers \citep{vella_languages_2013}. What makes Maltese and/or MaltE particularly interesting is their prosodic characteristics. First, both Maltese and MaltE are reported to have weight-sensitive lexical stress \citep[e.g.,][]{vella_prosodic_1995, fabri_maltese_2009}. Most importantly, although both languages are described as head prominence languages \citep{jun_prosodic_2014-1}, meaning that they feature regular pitch accents to cue prominence, they also exhibit frequent H(igh) tones on initial unstressed syllables \citep[e.g.,][]{vella_phonetics_2007, vella_alignment_2011}. Thus, unlike German where lexical (stress) and postlexical (intonation) prominence co-occur, the two levels of prominence do not appear to be straightforwardly linked in Maltese and MaltE \citep{lialiou_word-level_2023}. This enables words in Maltese and MaltE to feature more than one prominence simultaneously: lexical stress prominence as well as intonational prominence on the initial syllable. This potentially raises questions about the perceptual status of prominence in Maltese and MaltE, as lexical stress and intonational prominence might serve as competing cues \citep{lialiou_word-level_2023}. Therefore, the investigation of the attention orienting function of initial rises in Maltese and MaltE, and its comparison to the attention orienting function of regular accentual rises, could shed light not only on the general importance of rises in attention, but more importantly on the functional contribution of rises at the edges of constituents to prominence. 

Another interesting test case could be Greek. Typologically, Greek, too, is a head prominence language \citep[e.g.,][]{jun_prosodic_2014-1}. Nonetheless, recent results on the perception of prominence in Greek show that L(ow) accents are equally likely to be judged as prominent as other rising accents when they appear in a nuclear position \citep{orrico_prosodic_2024}. \citet{orrico_prosodic_2024} suggest that their results indicate that it is the strong metrical position that drives the prominence perception in Greek, rather than the mere acoustic properties of the intonational events. It has been shown in Chapter \ref{ch:4} that listener attention is not always attracted towards the most acoustically salient entity. Although German is a language where prominence is very often encoded in the form of intonational rises, listener attention was voluntarily allocated to deviant falls. This voluntary orienting towards deviant falls was not driven by acoustic saliency, but rather by linguistic prominence and pragmatic criteria, in the form of language- and context-driven expectations. \citet{orrico_prosodic_2024} mention that their results on Greek show precisely that Greek listeners' prominence judgments are not necessarily driven by acoustic saliency, but rather by higher order phonological and pragmatic factors. Therefore, testing the relevance of intonation for the orienting response in Greek could potentially shed more light on whether the biological importance of rises as an attentional cue still holds in a language that may not prioritise the acoustic signal to cue prominence, and could therby elucidate the role of language-specificity in interpreting cues and in attention orienting.

A final point that deserves further investigation is the role of individual variability in attention orienting. As already mentioned in the previous section, Chapter \ref{ch:5} made a novel attempt to unravel the importance of individual variability in attention orienting with regard to other cognitive operations such as processing speed, inhibitory ability, and WMC. Nonetheless, other cognitive operations may also be relevant for a better understanding of the orienting response across different listeners. In this regard, a further shortcoming of the current work is that Chapter \ref{ch:4} did not look at all into individual variability. Given the promising findings of Chapter \ref{ch:5}, I propose that future research in the direction of individual variability is essential, especially when group level results are driven by a specific subgroup of the registered sample. The present work provides a basis for future research by offering a cognitive test battery that includes assessments for the aforementioned cognitive operations. Certainly, future work further testing the relevance of these cognitive operations is warranted. Given that the bigger picture on individual variability remains blurry, researchers who endeavour to better understand individual processing variability should consider investigating further cognitive operations and other skills, such as pragmatic skills.   

\section{Conclusion}
\label{sec:6.4}
\largerpage
To conclude, the present work made a novel attempt to unravel the relevance of intonation for attention orienting in the linguistic domain. It provided experimental data using ERPs and pupillometry, which were utilised to examine the route from involuntary to voluntary attention. Taken together, the findings from the experiments reported in this book indicate that both saliency and prominence, encoded through intonational choices, impact the orienting response by modulating different stages of attention. The current work increases our knowledge in relation to the orienting response in the linguistic domain, and takes us one step closer to a full understanding of how linguistic complexity, with its interactions between signal- and context-driven cues, pertains to the attentional network. Lastly, the present research corroborates the idea of an attentional bias towards intonational rises, associated with both heads and edges, and extends it, for the first time, from acoustic saliency to prominence, and hence from general cognition to spoken language.
