\chapter[Indexing attention orienting of domain-final rises using pupillometry] % short title needed for page header
		{Indexing the attention orienting function of domain-final rises using pupillometry}
\label{ch:5}

A fundamental characteristic of the human cognitive system is its ability to withdraw attention from a currently important activity and orient it towards a source of information that may be currently irrelevant, but valuable enough for further assessment. For example, an unexpected, rare, or new sound that deviates in some property from the current auditory environment may capture attention, prompting an involuntary switch of the attentional resources towards this deviant auditory event. In \chapref{ch:4}, I demonstrated the special role of rising intonation in attention orienting towards unexpected changes in spoken language in an event-related potential (ERP) study. More specifically, I showed that the attention orienting function of rising intonation in speech holds regardless of whether the rise was accentual or associated with an edge tone. Capitalising on these findings, the current study delves further into the relevance of domain-final\footnote{Hereafter, I use the terms \textit{domain-final}, \textit{edge}, and \textit{boundary tone} interchangeably.} intonational rises and falls (which are the reflex of phrase final edge tones) for attention orienting in German. Using a \textit{changing-state} oddball paradigm, in which auditory sequences of sequentially ordered (i.e., seriatim) ascending numbers (\textit{standards}) are occasionally interspersed with an out-of-the-sequence number (\textit{deviant}), this study investigates whether domain-final rising intonation takes on a special role in attention orienting by measuring listeners' pupil dilation response (PDR).\footnote{A part of the data reported on this chapter has been presented at the 46\textsuperscript{th} Annual Meeting of the Cognitive Science Society (CogSci 2024) and published in the conference proceedings \citep{lialiou_attention_2024}.}

\section{Motivation}
\label{sec:5.1}

Over the years, many studies have been concerned with the mechanism underpinning attention orienting. This work has paved the way for the development of many accounts. As reviewed in \chapref{ch:2}, one of the mechanisms that has been suggested as underlying auditory attention orienting is \textit{expectancy violation}: the auditory system is able to develop expectancies by detecting regularities in the sound environment, and thus predict upcoming sound events. When a deviant sound occurs instead of an anticipated event, it attracts attention, initiating an orienting response (OR; for more on the mechanisms of attention orienting, see \sectref{sec:2.1} and \sectref{sec:2.3}). It has been shown that this OR finds expression in multiple physiological indices, such as heart and respiratory rates, and electrodermal, vasoconstrictive, neural, as well as pupillary reactions (for more, see \sectref{sec:2.5}). 

As discussed in \chapref{ch:2}, pupillary responses, and more specifically, PDR has mainly been used as a correlate of cognitive processes like listening effort, heightened attention, expectancy violation, and memory consolidation \citep[for review, see][]{winn_best_2018}. PDRs have also been used in linguistic studies assessing cognitive resource allocation towards mismatches between prosody and focus in discourse processing \citep[e.g.,][]{zellin_eye_2011}; attention and linguistic processing difficulty towards grammatical, syntactic, and semantic mismatches \citep[e.g.,][]{demberg_pupillometry_2013, demberg_frequency_2016}; affective arousal towards expectancy violations such as rhyme violations in poetry processing \citep[e.g.,][]{scheepers_listening_2013}; attention and cognitive load towards mismatches between prosodic and syntactic grouping \citep[e.g.,][]{jesse_using_2019}; and attention allocation towards unpredictable sentence continuation and age-effects \citep[e.g.,][]{hauser_effects_2019}, among others. More importantly, PDR has also been identified, by a growing number of studies, as a rigorous (psycho)physiological index of both involuntary and voluntary auditory attention orienting, highlighting a striking resemblance between the PDR composites and the neural MMN/P3 responses observed using ERP measures (for more on PDR as a proxy for attention orienting as well as on the link between neural and pupillary activity, see \sectref{sec:2.5}). Specifically, as discussed in \chapref{ch:2}, pupillometry provides a high resolution of temporal information, making it possible to measure cognitive activity related to attentional processes over time. The latency of attention-related PDRs has been reported to be indicative of the processing stage in which the attentional mechanism is activated: transient dilations reflect pre-attentive processes, while more prolonged responses indicate processes at the conscious level \citep[see][]{strauch_pupillometry_2022}. 

Pupillometric studies on auditory attention, using oddball paradigms, have further reported that PDR is not only affected by the presence of an auditory deviant, but is also sensitive to the intensity of the physical characteristics of that deviant sound. In other words, the more salient the acoustic properties of a deviant sound, the larger the magnitude of the change in the PDR \citep[e.g.,][]{liao_human_2016, wetzel_infant_2016, marois_eyes_2018}. These findings align with neurophysiological and linguistic research on auditory and speech perception. In auditory perception, the saliency of an event is essential in attracting attention: the greater the rise in amplitude or pitch on a deviant sound, the greater the orienting response. Likewise, in speech perception, intonational rises are used for attracting interlocutors’ attention when asking questions, directing listeners’ attention towards the most important part in an utterance, and even guiding attention in serial recall tasks (for a review of auditory cognition, see \sectref{sec:2.4}; for a review of language perception, see \sectref{sec:3.3} and \sectref{sec:3.4}). 

The current study examines how attention orienting in response to deviances in numeric sequences is modulated by the intonation pattern that these deviances feature. Listeners’ pupil size, and specifically, PDR, is used as a proxy for attention orienting. As discussed in \chapref{ch:3}, intonational events are phonologically anchored to specific positions in the prosodic structure, forming either pitch accents (anchored to stressed syllables), or edge tones (anchored to edges of constituents). In the autosegmental-metrical theory of intonational phonology \citep[e.g.,][]{ladd_intonational_2008}, edge tones have been strictly associated with a phrasing function, while pitch accents have been associated with a prominence-cueing function. Serial recall studies \citep[e.g.,][]{savino_intonation_2020, rohr_effect_2022, grice_rises_2024} have provided some evidence against this strict dichotomy in reporting that rising edge tones attract attention by lending prominence to the whole domain they delimit, as reflected in listener recall performance. Building on these serial recall studies as well as the finding that rises associated with the edges of constituents induce an attention orienting response in a similar way to accentual rises (reported in \chapref{ch:4}), the present study focuses on the processing of domain-final rising and falling intonation in German (RQ1 [rise]), and further investigates the attention orienting function of rises at the edges of constituents (RQ2 [edges]). 

Moreover, considering the relevance of meaning for attention orienting in spoken language reported in \chapref{ch:4}, the present study maintains and refines the minimal context of lists employed in the previous chapter, this time using auditory numeric sequences consisting both of seriatim ascending numbers (\textit{standards}; e.g., 21 22 23 24 25 26 27 28...) and occasional out-of-the-sequence numbers (\textit{deviants}; e.g., 25 in 21 22 23 \textcolor{red}{25} 26 27 28...). Numeric sequences were selected for this study as they inherently give rise to the linguistic context of a list, allowing for expectancy formations as well as expectancy violations, which are both related to the forthcoming numbers. This provides an optimal context for studying attention orienting. Moving to the language-specificity of this context (RQ3 [expectations]), standard numbers feature shallow falling intonation (hereafter, neutral intonation), a pattern that can also be found on non-final items of a sequence or list in German, while deviant numbers are produced with one of the following three intonational patterns: neutral intonation, domain-final rises, or domain-final falls. Domain-final rises and falls are functionally distinct: rises indicate continuity; falls denote finality \citep[e.g.,][]{grabe_comparative_1998, baumann_prosody_2001, chen_language_2003, peters_phonological_2018}. However, both types of edge tone can mark the end of either small or large units in a sequence, and they thus fulfil a similar function (for more on German list intonation, see \sectref{sec:3.5}). Therefore, the naturalistic pitch contours used in this study for the realisation of both standard and deviant numbers simulate a natural linguistic context.

Assuming that attention orienting is indexed, at least in part, by PDR excursions, the prediction is that the presentation of a deviant number (as 27 in 23 24 25 \textcolor{red}{27} 28...) will disrupt the anticipated pattern. This prediction is based on the claim that the attention orienting response is underpinned by an expectancy violation mechanism (for more, see \chapref{ch:2}). The disruption of the anticipated pattern caused by the deviant number will thus capture attention, which in turn will induce an excursion in PDR. Based on the enhanced orienting function of rising pitch compared to falling pitch (for reviews, see \sectref{sec:2.4}, \sectref{sec:3.3} and \sectref{sec:3.4}, and for current results, see \chapref{ch:4}), it is further predicted that a domain-final rise in deviants will result in greater disruption, attracting more attention and thus inducing a more robust PDR compared to deviants produced with domain-final falls or with neutral intonation.

Now, individuals may vary in the use or control of attentional resources towards incoming information. These individual differences may arise from the use and activation of different cognitive functions interacting with the attentional system. Studies endeavouring to better understand individual variability during cognitive or language processing usually examine cognitive functions like working memory capacity (WMC), processing speed, and executive processes (e.g., inhibition, shifting, updating), among others \citep[for review, see][]{frischkorn_process-oriented_2022}. At a more exploratory level, this study attempts to unravel whether listeners’ individual cognitive profiles affect the processing of the introduced arithmetic deviances produced with different intonational patterns, as a way of better understanding the attention orienting function towards them. In this endeavour, selective attention, in terms of inhibitory ability, processing speed, and WMC are selected as proxies for measuring individual cognitive variability. Inhibition may be one of the most crucial cognitive operations for better understanding the mechanism of attention orienting towards deviant events across individuals. Inhibition reflects a listener's ability to suppress irrelevant or unimportant information from breaking into the current attentional focus. Thus, whereas individuals with better inhibitory ability might be better at suppressing unimportant auditory deviances, thus saving attentional resources, individuals with lower inhibitory capacity might be more susceptible to switching attention towards auditory deviances. Further, there is a consensus in the literature on individual variability that processing speed \citep[although a diverse term; for discussion see][]{frischkorn_process-oriented_2022} is essential in explaining varying processing patterns among individuals. Therefore, processing speed might also contribute to a better comprehension of the orienting response by exploring whether processing speed aids or impedes the evaluation of deviants, for example, on the basis of signal-driven cues like different intonation patterns (as in this study). Lastly, WMC is one of the most frequently investigated cognitive measures of individual variability. While WMC has been found to be a strong factor in individual variability in cognition, its contribution to attentional control is less clear \citep[for discussion, see e.g.,][]{keye_individual_2009, sorqvist_high_2013}. However, some studies have reported a link between working memory (WM) load and mitigation of distractions at the later stages of orienting, such that the greater the WM load, the less susceptible a person is to distractions \citep[e.g.,][]{sanmiguel_when_2008}. Thus, in the context of the current study, WMC measures might shed more light on the link between attention-related mechanisms and WMC resistance operations.

Lastly, this study puts forward another exploratory question concerning the contribution of sequence length (if any) in attention orienting. This question was born out of the design of the items, whereby sequence length served the purpose of making the deviant position less predictable throughout the experiment (see \sectref{subsec:5.2.2}). Given the exploratory nature of this question, no hypothesis is formed. The present study thus has two aims, a confirmatory one and an exploratory one. The main aim is to test the attention orienting function of domain-final rising intonation, using PDR as a rigorous proxy for attention orienting. In addition, this study puts forward two exploratory questions, one related to the effect of sequence length in attention orienting, and another exploring individual-specific cognitive variability in the activation of attentional resources towards deviances featuring different intonational patterns.\footnote{Prior to pupil data collection, the design of the stimuli was tested in a pilot study, ensuring that the introduced deviances were indeed perceptible. This pilot study is reported in Appendix~\ref{sec:pilot}.}

\section{Methods}
\label{methods}
\subsection{Participants}
\label{subsec:5.2.1}

Sixty native speakers of German (54 female, 6 male), aged between 19 and 38 years (mean age = 22.6 years, \textit{SD} = 3.3) with normal or corrected-to-normal vision participated in this study. Participants provided written informed consent in accordance with the Declaration of Helsinki and in compliance with the ethics clearance from the Ethics Board of the \textit{Deutsche Gesellschaft für Sprachwissenschaft} (DGfS). Participants received reimbursement for their participation (either course credit or monetary compensation). None of them reported any speech, hearing, or neurological impairments.

\subsection{Speech materials}
\label{subsec:5.2.2}

The auditory stimuli comprise sequentially ascending ordered lists of numbers, consisting of a set of 17 numbers (medium-length sequence) or 22 numbers (long sequence). They were combined with three different prosodic realisations (neutral, rise, fall) on the deviant number. In total, 36 unique experimental numeric sequences were constructed for this study, with 6 different numeric sequences per prosodic condition and sequence length. The experimental items were combined with 36 unique filler items, which did not include a deviant number. \figref{fig:5.1items} illustrates instantiations of filler and experimental materials.

\begin{figure}
	\includegraphics[width=\textwidth]{figures/Items.png}
	\caption[Exemplars of filler and experimental materials] {Example filler and experimental materials illustrating the different lengths of the numeric sequences used, as well as the two different deviant positions in the experimental items.}
	\label{fig:5.1items}
\end{figure}

The experimental items introduced arithmetic deviances, i.e., a number out of sequence, called deviant. To achieve this, one or two consecutive numbers were omitted from the sequence. Specifically, 18 out of the 36 numeric sequences (i.e., 3 out of the 6 numeric sequences, per prosodic condition and sequence length) introduced the deletion of 1 consecutive number, and the remaining 18 numeric sequences introduced the deletion of 2 consecutive numbers. The controlled variation in deletion of one or two consecutive numbers served to increase the difficulty of the task. To control for potential effects of deviant position in the sequences, the deviant was introduced at two different positions: position 11 in the medium sequence length and position 16 in the long sequence length, as shown in the second panel of \figref{fig:5.1items}. Sequence length served the purpose of making the position of the deviant less predictable throughout the experiment.

The rise and fall prosodic conditions involved domain-final pitch movements, reflecting phrase final High and Low edge tones, respectively. The prosodically neutral condition served as a baseline. The experimental sequences included numbers between 22 and 99 which consisted of either two (e.g., 50  \textit{fünfzig} [ˈfʏnftsɪç]), four (e.g., 52 \textit{zweiundfünfzig} [ˈtsvaɪuntfʏnftsɪç]) or five syllables (e.g., 57 \textit{siebenundfünfzig} [ˈzi:bənuntfʏnftsɪç]), always with primary stress on the first syllable, allowing enough time for the different intonation contours to unfold. For the deviant numbers, 32 of 36 consisted of four syllables, and the 4 remaining numbers consisted of five syllables. 

An example of a numeric sequence for each of the three prosodic conditions is depicted in \figref{fig:5.2stimuli}. In the rising condition, all standards in the sequence were produced with the same intonation, that is, with neutral intonation, typically used on non-final items of a sequence or list in German. The deviant was realised with a boundary rising intonational contour. In the falling condition, similarly to the rising one, all standards in the sequence were produced with the same neutral intonation, and the deviant was realised with a boundary falling intonational contour. A neutral condition, in which both standards and deviants in the sequence were produced with the shallow falling intonation, served as the baseline condition. While both boundary rises and falls can mark the end of small or large units in a sequence, they are functionally distinct: rises indicate continuity; falls indicate finality. Across all three conditions, the last number of the entire sequence of each trial was realised with a boundary falling intonational contour in order to mark the end of that trial.

\begin{figure}
	\includegraphics[width=\textwidth]{figures/stimuli.png}
	\caption[Speech waveform and f0 contours] {Speech waveform \& f0 contour of a sample experimental trial sequence per condition.}
	\label{fig:5.2stimuli}
\end{figure}

The filler items were constructed without a deviant in the sequence, enhancing the creation of the memory trace formation of the sequentially ascending ordered numbers. The filler items consisted of a different range of numbers compared to the experimental ones to ensure variability in sequence construction. Numbers ranged between 2 and 99, and consisted of either one, two, four, or five syllables, always with primary stress on the first syllable. Filler items were comparable to the experimental items with regard to the prosodic conditions in order to ensure that participants could not identify the deviant in the experimental items just by tuning into the prosodic marking of deviants. Out of the filler items, 12 highlighted a number in the sequence with a boundary rising intonational contour (comparable to the rising condition), another 12 items highlighted a number in the sequence with a boundary falling intonational contour (comparable to the falling condition), and a further 12 items were comparable to the neutral condition. Fillers differed from experimental items in that the position of the prosodically distinct number in the sequence was fully randomised, in order to ensure that participants would not be able to identify a particular position in the sequence which differed prosodically from the rest.

Participants were presented with all 72 items (36 experimental and 36 fillers; 12 for each of the three prosodic conditions) in a counterbalanced design. Specifically, items were distributed in three lists. Each list contained all items and conditions, but never the same item across conditions. The 72 items were further distributed across three blocks with 24 items each (12 experimental and 12 fillers). The order of the items in each block was pseudo-randomised, with at most three consecutive experimental items but never from the same condition. To control for systematic order and frequency effects potentially induced by the exposure to block and/or item order, the fully counterbalanced lists were created with different block order and item, so that each list presented all items and blocks but never the same item across blocks and never the same block order. Each participant heard only one of the lists (the exact distribution of the lists as well as all items have been made available on the Open Science Framework (OSF) platform [\url{https://osf.io/nhy4c}]). All stimuli were produced by a phonetically trained 38-year-old female native speaker of German and recorded with a sampling rate of 44.100 Hz and 16-bit resolution (mono). To ensure natural speech production of the items, first, the speaker produced all numbers from 0 to 100 in separate blocks as a function of prosodic condition (e.g., neutral prosody: \textit{0, 1, 2, 3,…100}; rising prosody: \textit{0, 1, 2, 3,…100}; falling prosody: \textit{0, 1, 2, 3,…100}). Subsequently, all number renditions were cut from each block, saved as individual audio files, and concatenated into the different numeric sequences using Praat \citep{BoersmaWeenink2023}. The inter-stimulus interval between spliced numbers was 100 ms. The average duration of the medium-length and long sequences was 24.86 sec and 25.21 sec, respectively. All stimuli used in the experiment were normalised at −23 LUFS but not manipulated further.

\subsection{Acoustic characterisation of speech material}
\label{subsec:5.2.3}

For the acoustic characterisation of the deviant numbers in the sequences, the relative Delta f0 (Δf0) metric from the \textit{ProPer} toolbox was employed \citep{albert_using_2018, cangemi_modelling_2019, albert_proper_2020, albert_model_2023}, as described in \chapref{ch:4}. Prior to the ProPer analysis, f0 contours were extracted and corrected manually in Praat \citep{boersma_praat_2024}, using a customised version of \textit{mausmooth} \citep{cangemi_mausmooth_2015}. The ProPer analysis was conducted on the basis of syllabic units. Scripts and data tables of the current analysis have been made available on OSF (\url{https://osf.io/nhy4c}). The measure of Δf0 traces the f0 trajectory across syllables, using both f0 and periodic energy, indicating f0 changes from syllable to syllable by calculating the difference from the previous one. For more details about the metric please refer to \sectref{subsec:4.2.2}. \tabref{tab:5.1} presents means and standard deviations for Δf0 per syllable across target numbers for each prosodic condition, as well as the total duration of the deviant numbers.

\begin{table}
	\centering
	\caption[Δf0 and duration across items]{Mean and standard deviation values (in brackets) for relative Δf0 values per syllable across items for each condition as well as for the total duration of items.}
	\label{tab:5.1}
	\begin{tabular}{c c c c c} % add l for every additional column or remove as necessary
		\lsptoprule
		measurement & & rising & falling & neutral\\ %table header
		\midrule
		relative Δf0  & \textit{quadrisyllabic} & & &\\
		\midrule
		& 1 & 0.97 (4.74) & 5.59 (3.92) & 2.82 (8.90)\\
		& 2 & 9.33 (3.36) & --17.9 (7.79) & --8.23 (4.93)\\
		& 3 & 7.04 (3.18) & --28.2 (12.4) & --11.5 (11.3)\\
		& 4 & 21.7 (7.12) & --15.8 (7.34) & --14.2 (15.0)\\
		\midrule
		relative Δf0  & \textit{pentasyllabic} & & &\\
		\midrule
		& 1 & 0.88 (1.38) & 3.88 (1.89) & 1.47 (2.07)\\
		& 2 & 3.00 (0.79) & --4.33 (0.87) & --1.16 (2.09)\\
		& 3 & 4.75 (3.13) & --14.7 (2.99) & --8.23 (1.19)\\
		& 4 & 5.24 (0.65) & --23.7 (7.41) & --4.13 (1.67)\\
		& 5 & 27.2 (13.3) & --15.9 (7.32) & --10.8 (13.8)\\
		\midrule
		duration [ms] & item & 1126 (59.64) & 1127 (86.06) & 1098 (61.64)\\
		\lspbottomrule
	\end{tabular}
\end{table}

\figref{fig:5.3df0} depicts relative Δf0 values per syllable as a function of prosodic condition across quadrisyllabic deviant numbers (the pattern is the same in pentasyllabic deviant numbers, as can be seen in \tabref{tab:5.1}). In the rising condition (see depiction in yellow), mean Δf0 starts at a mid-level and rises shallowly from the first to the second syllable, then remains on the same level until syllable three, and steeply rises towards the last syllable, i.e., the right edge of the word. \figref{fig:5.4GToBI} presents instantiations of the deviant f0 contours per prosodic condition, analysed in the autosegmental-metrical model of intonation laid out in the German Tones and Breaks Indices annotation scheme \citep[GToBI;][]{grice_german_2005}. In GToBI, the aforementioned rising contour is H*  $^\wedge{}$H-\%. In both falling and neutral conditions (see \figref{fig:5.3df0}, depiction in green and black, respectively), mean Δf0 starts somewhat higher than in the rising condition, and gradually falls from the first to the last syllable. The difference between the falling and neutral condition is that in the falling condition, the first syllable is slightly higher, and the drop from the first to the second, and from the second to the third syllable, is steeper than in the neutral condition. The drop towards the last syllable is smaller in the falling than in the neutral condition. In GToBI, the neutral and the falling conditions are labelled as H* L- and H* L-\%, respectively (see \figref{fig:5.4GToBI}, depiction in green and black, respectively). The larger boundary in the falling condition explains the gradient difference in the meaning between the two contours. Specifically, the neutral contour, because it does not fall as steeply as the falling contour, marks the end of a unit less clearly, and thus can be featured on non-final items in a sequence. The falling contour, in contrast, unambiguously marks the end of a unit indicating finality due to its steep fall and the extra-low pitch towards the end. Lastly, the rising contour can mark the end of a unit as well, but it is functionally different from the falling contour in that it indicates that more is expected (in a following unit).

\begin{figure}[h]
	\includegraphics[width=\textwidth]{figures/df05.png}
	\caption[Relative Δf0 values across items] {Mean relative Δf0 values per syllable (x-axis) across prosodic conditions in quadrisyllabic deviant numbers.}
	\label{fig:5.3df0}
\end{figure} 

\begin{figure}
	\includegraphics[width=\textwidth]{figures/GToBIdeviants.png}
	\caption[Instantiations of deviant f0 contours per prosodic condition in GToBI] {Instantiations of deviant f0 contours per prosodic condition in GToBI.}
	\label{fig:5.4GToBI}
\end{figure}

\subsection{Experimental procedures}
\label{subsec:5.2.4}

The experimental sessions of the main study followed the exact same structure as the pilot study (see \sectref{subsec:experimentalprocedures}). That is, the pupillometric task was performed always at the beginning of the session, followed by a battery of cognitive tests, always in the following order: a version of the flanker task (measuring inhibitory ability), a version of the odd-man-out task (measuring processing speed), and a version of the digit span task (measuring WMC; see below for more details on these tasks).\footnote{The button press, the flanker, and the odd-man-out tasks were implemented in \textit{OpenSesame} \citep{mathot_opensesame_2012}. The digit span task was implemented in \textit{SoSci} Survey \citep{leiner_sosci_2024}. In all these tasks, participants could take an optional short break between the blocks, and during the practice phase they received immediate feedback on the screen.}

\subsubsection{{Pupillometry}}
	
For the pupillometric task, the auditory stimuli were presented using the SR Researcher Experiment Builder (v. 4.595) via loudspeakers, with pupil data time-locked to the onset of each sequence. Pupil size was sampled at 1000 Hz, using an Eyelink 1000 eye-tracker (SR Research Ltd.). Prior to the beginning of the task, the system was calibrated to the dominant eye of each participant, using a 9-point calibration procedure. For all participants, the average luminance measured at the dominant eye was 50 lx.
	
Participants were seated in the eye-tracking lab in front of a computer monitor and a keyboard. Participants were informed that they would be presented auditorily with numeric sequences but they were naive to the deviances included. In order to keep them engaged with a task, participants had to answer a comprehension yes/no question related to the numeric sequence they heard in 35\% of the trials,\footnote{For example, one of the questions was \textit{War die Aufzählung in Zehnerschritten?} ``Was the enumeration in steps of ten?" The full set of the questions is provided on OSF. Participants' mean response accuracy to these questions was 95\% for questions related to the experimental items, and 100\% for questions related to the fillers (across individuals, response accuracy ranged between 86--100\%), indicating high engagement.} by pressing a button indicated on the keyboard. Written instructions were also provided. The experiment started with a practice phase of 5 items consisting of seriatim numbers. Two of them were followed by a comprehension question for which participants received immediate feedback on the screen. The experiment consisted of three blocks of 24 items each. Participants could take an optional short break between the blocks. The experiment lasted approximately one hour.
	
Every trial started with a drift correction. With the onset of the auditory sequence, a black fixation cross appeared on a grey background and remained on the screen during the whole trial. Participants were instructed to fixate on the black cross on the screen. Twenty-five of the trials were followed by a comprehension question presented onscreen, and participants had to press one of two buttons on the keyboard to indicate their yes/no response. Following the offset of each trial, a grey screen with a black dot was displayed, providing enough time for the pupil dilation to subside. Specifically, participants were instructed that during this screen, they could take time to rest their eyes. Once they were ready to continue, they had to press SPACE to start the next trial. \figref{fig:5.5pupil} depicts a schematic illustration of an experimental trial.
	
\begin{figure}
	\includegraphics[width=\textwidth]{figures/PUPIL-Trial-Illustration.png}
	\caption[Schematic exemplar of a pupillometric trial] {Schematic illustration of an experimental trial in the pupillometric task.}
	\label{fig:5.5pupil}
\end{figure}

\subsubsection{{Cognitive test battery}}

Various cognitive functions may give rise to individual differences in language processing. In the literature, a large number of tasks has been employed to investigate individual differences in the performance of cognitive and linguistic tasks. The cognitive battery employed in this study consists of modified versions of the flanker \citep{eriksen_effects_1974}, odd-man-out \citep{frearson_intelligence_1986}, and digit span \citep{wechsler_wms-r_1987} tasks, endeavouring to measure selective attention in terms of inhibitory ability, processing speed, and WMC, respectively. The current cognitive battery has been made freely available online for use in other studies (repository link: \url{https://osf.io/muh9t/}).

\subsubsubsection{{Flanker task}}

To measure inhibitory ability, a modified version of the flanker task was employed, using arrow stimuli \citep{eriksen_effects_1974, anwyl-irvine_gorilla_2020}. Participants were informed that a sequence of five arrows would appear horizontally in the middle of the screen. Participants were asked to indicate whether the middle arrow in this sequence pointed to the left or to the right by pressing the ``S'' or ``K'' button on the keyboard accordingly. They were instructed to place their fingers on the above-mentioned buttons to reduce cognitive load. The task consisted of two conditions: a congruent and an incongruent one. As depicted in the first two rows of \figref{fig:5.6flankeritems}, in the congruent condition, all arrows pointed in the same direction as the middle arrow. In the incongruent condition, the middle arrow pointed in the opposite direction compared to the other arrows in the sequence (see last two rows). Participants were asked to be as fast and as accurate as possible in their responses.

\begin{figure}
	\centering
	\includegraphics[width=0.5\textwidth]{figures/flankeritems.png}
	\caption[Item illustration in the flanker task] {Item illustration as a function of condition in the flanker task.}
	\label{fig:5.6flankeritems}
\end{figure}

In total, 96 experimental items were constructed for this task (48 congruent and 48 incongruent). Half of the congruent items (i.e., 24 items) presented all arrows pointing towards the right direction (see first row in \figref{fig:5.6flankeritems}), while the other half presented all arrows pointing to the left (see second row in \figref{fig:5.6flankeritems}). Half of the incongruent items (i.e., 24 items) displayed all arrows pointing to the right direction, while the middle arrow pointed to the left (see last row in \figref{fig:5.6flankeritems}). The remaining 24 incongruent items exhibited the reverse pattern, i.e., all arrows were pointing to the left, while the middle arrow pointed to the right (see third  row in \figref{fig:5.6flankeritems}). The task started with a practice phase of 12 items. The main experimental phase consisted of four blocks with 24 items each (25\% congruent-right direction, 25\% congruent-left direction, 25\% incongruent-right direction, 25\% incongruent-left direction). Item and block order were fully randomised. \figref{fig:5.7flanker} depicts a trial structure: for each trial, a fixation cross was presented on the screen for 1700 ms. The fixation cross was followed by the presentation of the arrow stimuli. The stimuli remained on the screen until participants pressed one of the two valid buttons (``S'' or ``K''). Each trial ended with a blank screen for 400 ms. The task lasted approximately 5 minutes.

\begin{figure}
	\includegraphics[width=0.7\textwidth]{figures/FLANKER-Trial-Illustration.png}
	\caption[Schematic exemplar of a flanker trial] {Schematic illustration of an experimental trial in the flanker task.}
	\label{fig:5.7flanker}
\end{figure}

\subsubsubsection{{Odd-man-out task}}

To measure processing speed, a modified version of the odd-man-out task was employed \citep{frearson_intelligence_1986}. In the original task, participants were presented with a disposition of three stimuli and were asked to decide whether the first or the third stimulus is more distant in relation to the second one, i.e., which is the ``odd-one-out'' in terms of proximity. Here, following \citet{diascro_odd-man-out_1994}, the original task was adapted by increasing the complexity of the paradigm. Specifically, two levels of difficulty were manipulated: the distance of stimuli disposition (hereafter, spatial condition) as well as the shape (hereafter, gap condition) of the ``odd-one-out'' stimulus. The stimuli consisted of three hexagons arranged in two odd-man-out conditions. In the spatial condition, the stimuli were arranged in a line, with the odd-man-out stimulus at a spatial distance from the other two hexagons. As depicted in the three top panels of \figref{fig:5.8item-oddman}, three spatial ``odd-one-out'' configurations were possible in this condition. In the gap condition, stimuli were arranged at equal distance from each other in a line, but they differed in that the ``odd-one-out'' hexagon had a gap on its top side. Again there were three possible locations for the missing side (the right, left, or middle hexagon). The three bottom panels of \figref{fig:5.8item-oddman} display the possible ``odd-one-out" configurations in the gap condition.

\begin{figure}
	\centering
	\includegraphics[width=0.8\textwidth]{figures/items-oddmanout.png}
	\caption[Item illustration in the odd-man-out task] {Item illustration as a function of condition in the odd-man-out task. The three top panels show the possible locations of the ``odd-one-out" stimulus’s disposition in the spatial condition; the three bottom panels illustrate the possible locations of the missing side of the ``odd-one-out" stimulus in the gap condition.}
	\label{fig:5.8item-oddman}
\end{figure}

Participants were informed that they would see sequences of three hexagons on the screen, and were asked to indicate the ``odd-one-out'' hexagon, namely the one that differed from the others. They were instructed to press the following buttons on the keyboard as fast and as accurately as possible: ``1'' when the ``odd-one-out'' stimulus appeared on the left side of the sequence, ``2'' when it appeared in the middle, and ``3” when it appeared on the right side. As before, participants were instructed to place their fingers on the relevant keys in advance, in order to reduce cognitive load. The task started with a practice phase of six items. The main experimental phase consisted of four blocks with 30 items each (5 items x 3 positions (right, left, middle) x 2 conditions (spatial, gap), resulting in a total of 120 experimental items). Item and block order were fully randomised. \figref{fig:5.9oddman} depicts a trial structure: for each trial, a fixation cross was presented on the screen for 2000 ms. The fixation cross was followed by the presentation of a stimulus trial. The stimuli remained on the screen for 100 ms and were immediately followed by a blank screen. The blank screen remained until participants pressed one of the three valid buttons (``1'', ``2'', or ``3'') or until time-out (5000 ms). The task lasted approximately 5 minutes.

\begin{figure}
	\includegraphics[width=0.7\textwidth]{figures/ODDMANOUT-Trial-Illustration.png}
	\caption[Schematic exemplar of an oddman-out trial] {Schematic illustration of an experimental trial in the odd-man-out task.}
	\label{fig:5.9oddman}
\end{figure}

\subsubsubsection{{Digit span task}}

Lastly, to measure WMC, a modified version of a digit span task was conducted, based on the WAIS-R Digit Span test \citep{wechsler_wms-r_1987}. For this modified version, stimuli recorded for a different experiment \citep[][]{rohr_effect_2022, grice_rises_2024} were used. Participants were instructed to wear headphones as they would be presented auditorily with sequences that progressively increased in length, containing digits from 1 to 9. After every two sequences of the same length, starting with a three-digit sequence, the length of the next two sequences was extended by one digit, up to two sequences of nine digits, resulting in a total of 14 experimental sequences. Participants were asked to recall all digits of a sequence immediately after the presentation of the last digit, in the exact same order they were presented. Participants could hear each sequence only once. After sequence presentation, a numeric keypad appeared on the screen. Responses were entered by clicking each digit on the numeric keypad in the appropriate order. No digit could be omitted from the response. At the top left of the screen, above the keypad, a counter showed how many digits had already been entered. As soon as the participants entered the last digit of a sequence, they could proceed to the next sequence by clicking a ``next'' button. All stimuli were initiated by a beep tone of 890 ms, followed by 500 ms of silence, before the digit sequence started. The task lasted approximately 5 minutes. \figref{fig:5.10digit} illustrates a trial structure.

\begin{figure}
	\includegraphics[width=0.8\textwidth]{figures/DIGITSPAN-Trial-Illustration_cropped.png}
	\caption[Schematic exemplar of a digit span trial] {Schematic illustration of an experimental trial in the digit span task.}
	\label{fig:5.10digit}
\end{figure}

\subsection{Data processing and statistical analyses}
\label{subsec:5.2.5}

Data processing and statistical analyses were conducted in R, version 4.1.2 \citep{r_core_team_r_2023}, using the R packages \textit{ggplot2} 3.3.5 \citep{wickham_ggplot2_2016}, \textit{itsadug} 2.4.1. \citep{van_rij_itsadug_2022}, \textit{mgcv} 1.9-1 \citep{wood_mgcv_2023}, \textit{PupilPre} 0.6.2. \citep{kyrolainen_pupilpre_2020}, and \textit{tidyverse} 1.3.1 \citep{wickham_welcome_2019}. For reproducibility, data and scripts have been made available at \url{https://osf.io/nhy4c}.

\subsection{Data pre-processing}

\subsubsection{{Pupil data}}
\largerpage
Pupil data were exported using SR Research Data Viewer (v.4.3.210), and were further processed using the R-package \textit{PupilPre} \citep{kyrolainen_pupilpre_2020}. Pupil data were re-aligned to 100 ms prior to the onset of the deviant number, and then continued for 3000 ms. Blink components were automatically detected and removed from the raw pupil data. The data were then manually checked, and the remaining blink artifacts were removed by hand. Subsequently, trials including more than 20\% of missing data because of blink artifacts were completely removed from further analyses, yielding 3.52\% loss of the total dataset. After artifact rejection, the raw data were interpolated using cubic spline interpolation and then filtered with a Butterworth 0.1 Hz low-pass filter. Skipped trials due to missing values and artifacts created by the filter were removed using the \textit{trim\_filtered} function. Thereafter, the raw data were baseline-normalised by trial (subtractive correction) using the average of 100 ms preceding the onset of the deviant. Finally, normalised data were down-sampled to a rate of 100 Hz (10 ms time bins). 

\subsubsection{{Cognitive test battery}}

For the flanker and odd-man-out tasks, accuracy (correct/incorrect responses, coded as 1/0) as well as response times (in ms) were recorded for each trial. To measure participants’ inhibitory ability, the efficiency measure by \citet{SpilsburyEtAl1990} was employed. Specifically, the number of the correct incongruent trials was divided by the median response time (inhibition score = [number of correct answers in incongruent trials] / [median response time]) per participant. To measure processing speed, a similar efficiency measure was calculated per participant. This measure was computed by dividing the number of correct trials by the median response time. Lastly, in the digit span task, digit responses were recorded in the recalled order by participants. The digit span of each participant was calculated as the length of the last correctly recalled sequence before that participant’s failure on two consecutive sequences.

\tabref{tab:5.2} presents individual cognitive profiles in terms of inhibition, processing speed, and WMC. The table shows raw mean scores per individual participant across the three cognitive tasks implemented. Higher scores in the flanker task indicate better inhibitory skills;  higher scores in the odd-man-out task indicate slower processing speed; and higher scores in the digit span task indicate larger WMC.

\begin{longtable}{c c c c}
	\caption{Mean performance per participant across the three cognitive tasks.\label{tab:5.2}}\\
	\lsptoprule
	Participant ID & Flanker & Odd-man-out & Digit Span\\ %table header
	\midrule\endfirsthead
	\midrule
	Participant ID & Flanker & Odd-man-out & Digit Span\\ %table header
	\midrule\endhead
		01 & 0.98 & 3.77 & 8\\ 
		02 & 0.98 & 5.97 & 6\\ 
		03 & 0.92 & 5.08 & 9\\ 
		04 & 0.98 & 4.15 & 6\\ 
		05 & 0.98 & 3.30 & 6\\ 
		06 & 0.98 & 7.00 & 5\\ 
		08 & 0.90 & 4.06 & 6\\ 
		09 & 0.98 & 7.14 & 5\\ 
		10 & 1.00 & 6.26 & 7\\ 
	    11 & 0.92 & 4.47 & 5\\ 
	    12 & 0.96 & 2.89 & 7\\ 
	    13 & 0.98 & 3.14 & 6\\ 
	    14 & 0.96 & 5.41 & 7\\ 
	    15 & 0.98 & 4.45 & 6\\ 
	    17 & 0.98 & 4.01 & 7\\ 
	    18 & 0.94 & 5.18 & 6\\ 
	    19 & 0.96 & 5.71 & 7\\ 
	    20 & 0.98 & 5.53 & 9\\ 
	    21 & 1.00 & 5.80 & 7\\ 
	    22 & 0.96 & 4.87 & 8\\ 
	    23 & 0.98 & 4.21 & 6\\ 
	    24 & 1.00 & 5.71 & 5\\ 
		25 & 0.98 & 4.14 & 6\\ 
		26 & 1.00 & 5.01 & 6\\ 
		27 & 0.94 & 5.27 & 6\\ 
		30 & 0.98 & 4.90 & 5\\ 
		31 & 0.85 & 3.71 & 8\\
		32 & 0.98 & 5.98 & 6\\ 
		33 & 0.98 & 5.54 & 6\\ 
		34 & 1.00 & 2.96 & 6\\ 
		35 & 1.00 & 4.64 & 7\\ 
		36 & 0.94 & 4.86 & 6\\
		37 & 0.96 & 4.43 & 8\\ 
		38 & 0.96 & 3.23 & 5\\ 
		39 & 0.96 & 3.55 & 7\\ 
		40 & 0.90 & 5.85 & 6\\ 
		41 & 0.98 & 5.14 & 7\\ 
		42 & 1.00 & 3.92 & 5\\ 
		44 & 0.96 & 4.79 & 6\\ 
		45 & 0.98 & 4.01 & 5\\ 
		46 & 1.00 & 8.05 & 6\\ 
		47 & 0.98 & 4.97 & 4\\ 
		48 & 0.92 & 5.12 & 7\\ 
		49 & 0.94 & 3.83 & 6\\ 
		50 & 0.92 & 4.39 & 8\\ 
		51 & 0.96 & 7.72 & 7\\ 
		52 & 1.00 & 4.10 & 5\\
		53 & 0.90 & 3.76 & 6\\ 
		54 & 1.00 & 6.46 & 6\\ 
		55 & 1.00 & 4.95 & 6\\ 
		57 & 0.98 & 4.98 & 6\\ 
		59 & 0.96 & 2.94 & 8\\ 
		60 & 0.96 & 4.56 & 7\\ 
		61 & 0.92 & 5.49 & 6\\ 
		62 & 0.88 & 5.04 & 7\\ 
		63 & 0.98 & 5.30 & 5\\ 
		64 & 1.00 & 4.71 & 8\\ 
		65 & 0.98 & 3.76 & 6\\ 
		66 & 0.96 & 5.28 & 9\\ 
		67 & 0.92 & 3.58 & 6\\ 
		\lspbottomrule
\end{longtable}

The next processing step consisted of standardising (z-scoring) participants' scores across the flanker, odd-man-out, and digit span tasks. Although participants could be classified on the basis of their performance on one of the three cognitive tasks, the combinations of the skills measured in the three tasks is considered crucial for a better understanding of the different cognitive pathways involved in the process of OR activation. Prior to statistical modelling, correlation tests among the three cognitive scores were employed. As in the pilot study using the button press paradigm (see \sectref{exp}), the correlation tests revealed a weak positive correlation between odd-man-out and flanker scores (\textit{r}(572278) = 0.34, \textit{p} < 0.0001), indicating that the slower the processing speed, the better the inhibitory skills (and vice versa). A weak negative correlation between digit span and flanker scores was also found (\textit{r}(572278) = --0.23, \textit{p} < 0.0001), indicating that the greater the span, the lower the inhibitory skills.

\subsubsection{{Inference criteria}}

The statistical analysis of the pupil data is divided into two parts, \textit{confirmatory} and \textit{exploratory}, similarly to the evaluation of the pilot results (see \sectref{subsec:data}). The confirmatory analysis tested the prediction that deviant numbers produced with rising intonation, due to its attention orienting function, will induce a stronger PDR compared to numbers realised with falling intonation, or when the intonation does not differ from that of the standard numbers, that is, deviants produced with neutral intonation. 

In the exploratory analyses, I examined whether sequence length affects PDR and/or whether it interacts with the different prosodic deviant realisations. Besides sequence length, I also explored whether individual cognitive profiles interact with the prosodic conditions and thus affect the processing of the introduced deviances produced with different intonational patterns. The full model specifications can be found in the script provided on OSF.

The PDR was normalised and modelled using Generalised Additive Mixed Modelling (GAMM), which has been shown to be effective for analysing pupil data, as it accounts for nonlinear patterns and interactions, nonlinear random effects, and the inherent autocorrelation of time-course data \citep[see][]{van_rij_analyzing_2019}. In particular, GAMM is appropriate for modelling nonlinear time-series patterns, capturing variation in two trajectories: \textit{height} and \textit{shape}. These two trajectories are captured by different terms: parametric terms allow for mean differences in the overall height of the curves, and smooth terms allow for differences in the shape of the curves. GAMM also accounts for random effect structures by using random smooths \citep[e.g.,][]{winter_how_2016, wood_generalized_2017, soskuthy_generalised_2017}. Random smooths expand the principle of smooth terms to the random effects by fitting separate smooths at each value of a grouping variable, thereby allowing different curve shapes for different subjects and/or items.

\subsubsection{{Confirmatory analysis}}

PDR was modelled as a function of the ordered factor\footnote{Ordered factors allow for testing whether the curves of each level of the factor differ not only in height (parametric coefficients) but also in shape (difference smooth terms).} \textsc{prosodic condition}. Treatment contrast was used to code \textsc{prosodic condition} (levels: rise/fall/neutral), with rise serving as the reference level. This coding allows for testing the following contrasts: 
\newpage
\begin{enumerate}[label=\arabic*.]
	\item rise vs. fall, and
	\item rise vs. neutral.
\end{enumerate}

The model included \textsc{prosodic condition} both as a parametric term, testing for overall \textit{height} differences in PDR curves between prosodic conditions, and as a smooth reference term, capturing the \textit{shape} effects in the reference level of the \textsc{prosodic condition} (i.e., rise) over \textsc{time}. The model also included a difference smooth term by \textsc{prosodic condition}, testing \textit{shape} differences of PDR curves between prosodic conditions (i.e., rise vs. fall, and rise vs. neutral) over \textsc{time}. Further, the model included a random smooth by \textsc{subject}, and a reference-difference random smooth for each \textsc{subject} by \textsc{prosodic condition} which captures \textit{shape} differences by \textsc{subject}. Lastly, autocorrelation within trajectories was controlled via the inclusion of an AR1 residual model. Given that this model could not test for the contrast between fall vs. neutral, the simultaneous confidence interval (CI) test\footnote{The simultaneous confidence interval (CI) test can be used as a proxy for a post-hoc test: when testing two whole curves simultaneously, if any point in the CI does not include zero, then the difference between them can be treated as significant.} was implemented to examine this contrast.

\subsubsection{{Exploratory analyses}}

Moving to the exploratory part of this study, for both \textit{sequence length} and \textit{individual cognitive variability}, I fitted three separate models to examine their interaction with \textsc{prosodic condition} in the three following contrasts: 

\begin{enumerate}[label=\arabic*.]
	\item \emph{Model 1}: neutral vs. rise,
	\item \emph{Model 2}: neutral vs. fall, and
	\item \emph{Model 3}: rise vs. fall.
\end{enumerate}

Results are presented on the basis of model summaries.

\subsubsubsection{{Sequence length}}

For the investigation of the \textit{sequence length} effect, PDR was modelled as a function of the ordered factors \textsc{prosodic condition} (treatment coded) and \textsc{length} (treatment coded, reference level: long), as well as their interaction. \textsc{prosodic condition}, \textsc{length}, and their interaction were included in the models as parametric terms and smooths over \textsc{time}. In addition, random smooths by \textsc{subject}, and reference-difference random smooths per condition for the individual levels of \textsc{subject}, were included in all models. Lastly, an AR1 residual model was included in all models to control for autocorrelation within trajectories. Model 1 was fitted such that the smooth over \textsc{time} for the condition [neutral, long] represented the reference smooth. The model further included difference smooths for the conditions [rise, long], [neutral, medium], and [rise, medium]. This approach allows for assessing the following differences: the difference smooth for [rise, long] reflects the effect of rising vs. neutral deviants in long sequences; the difference smooth [neutral, medium] shows the effect of neutral deviants in medium-length vs. long sequences; and the difference smooth [rise, medium] highlights the effect of rising deviants in medium-length vs. long sequences. Model 2 was fitted such that the smooth over \textsc{time} for the condition [neutral, long] represented the reference smooth. The model further included difference smooths for the conditions [fall, long], [neutral, medium], and [fall, medium]. Specifically, the difference smooth for [fall, long] reflects the effect of falling vs. neutral deviants in long sequences; the difference smooth [neutral, medium] shows the effect of neutral deviants in medium-length vs. long sequences; and the difference smooth [fall, medium] highlights the effect of falling deviants in medium-length vs. long sequences. Finally, model 3 was fitted such that the smooth over \textsc{time} for the condition [rise, long] represented the reference smooth. The model further included difference smooths for the conditions [fall, long], [rise, medium], and [fall, medium]. Specifically, the difference smooth for [fall, long] reflects the effect of falling vs. rising deviants in long sequences; the difference smooth [rise, medium] shows the effect of rising deviants in medium-length vs. long sequences; and the difference smooth [fall, medium] highlights the effect of falling deviants in medium-length vs. long sequences.\footnote{These models do not allow for estimating the following comparisons: [rise, medium] vs. [neutral, medium]; [fall, medium] vs. [neutral, medium]; [rise, medium] vs. [fall, medium]. Given that these differences are not of primary interest, I describe them based on the illustration of GAMM smooths.}

\subsubsubsection{{Cognitive variability}}

For the exploration of the \textit{individual cognitive variability} effect, and given the aforementioned correlations among the cognitive predictors, only one predictor was selected for the subsequent analyses. Due to its high relevance for attention orienting, the \textit{flanker scores}, indicating inhibitory skills, were chosen. Therefore, PDR was modelled as a function of the ordered factor \textsc{prosodic condition} (treatment coded) and its interaction with the continuous \textsc{flanker scores}. Specifically, \textsc{prosodic condition} was included in the models as a parametric term, with a reference smooth over \textsc{time} and a difference smooth over \textsc{time} by \textsc{prosodic condition}. Further, the models included a reference and a difference smooth over \textsc{flanker scores} by \textsc{prosodic condition}. The reference smooth captured non-linear changes (if any) in the average PDR as a function of the \textsc{flanker scores} for the reference prosodic level (neutral in model 1; falling in model 2; and rising in model 3), while the difference smooth reflected shape differences in the non-reference prosodic level (rising in model 1; neutral in model 2; falling in model 3), compared to the reference level. A tensor product reference smooth and a tensor product difference smooth were also parts of the model, capturing the two-way interaction between \textsc{prosodic condition} and \textsc{flanker scores} over \textsc{time}. The tensor product reference smooth modelled the effect of \textsc{flanker scores} (if any) on the curve shape of the prosodic reference level (neutral in model 1; falling in model 2; and rising in model 3). The tensor product difference smooth reflected whether this effect changed in the non-reference prosodic level (rising in model 1; neutral in model 2; and falling in model 3). In addition, random smooths by \textsc{subject} and reference-difference random smooths per condition for the individual levels of \textsc{subject} were included in all models. Lastly, an AR1 residual model was included in all models to control for autocorrelation within trajectories.

\section{Results}
\label{subsec:5.2.6}

This section presents results reporting model predictions and plotting GAMM smooths which depict the different estimated effects on the PDR within the time window of interest. The presentation of the results is structured as follows. First, I concentrate on the confirmatory analysis, that is, I report on PDR differences among prosodic conditions, and thus focus on results related to the hypothesis that rising deviants may induce stronger PDRs due to the attention orienting function of rising intonation. Subsequently, I report on the exploratory analyses, presenting the effects of sequence length and individual cognitive variability on PDRs, and hence on the processing of the introduced deviances produced with different intonational patterns.

\figref{fig:5.11raw} illustrates grand averaged changes in pupil diameter from the baseline average over time (sampled in 10 ms bins), as a function of prosodic condition, time-locked to the onset of the deviant stimulus (zero ms; as depicted by the vertical dashed line). The yellow point-range curve depicts pupillary responses to deviants produced with the rising edge tone (rising condition), the green point-range curve shows pupillary responses to deviants featuring the falling edge tone (falling condition), and the black point-range curve illustrates pupillary responses to deviants with neutral intonation (the baseline prosodic condition). Visual inspection of the curves reveals that deviant numbers elicit a PDR regardless of the intonation they feature, but the different intonational contours seemed to affect the PDR in distinct ways. 

\begin{figure}
	\includegraphics[width=0.8\textwidth]{figures/RAW.png}
	\caption[Grand averaged pupil dilation response per prosodic condition] {Grand averaged pupil dilation response per prosodic condition over time (sampled in 10 ms bins), time-locked to the onset of deviant stimulus (illustrated by the vertical dashed line), and estimated from the baseline average.}
	\label{fig:5.11raw}
\end{figure}

\subsection{{Confirmatory analysis}}

GAMM smooths\footnote{GAMM smooths illustrate height, shape, and latency properties of the effects.} (including 95\% CIs) of PDR to deviant numbers as a function of prosodic condition are depicted in \figref{fig:5.12confirmatory}. Colour coding of the prosodic conditions' estimated smooths corresponds to the colour coding in \figref{fig:5.11raw}.

\begin{figure}
	\centering
	\includegraphics[width=0.8\textwidth]{figures/curves.png}
	\caption[GAMM smooths of PDR (normalised) across prosodic conditions] {GAMM smooths of PDR (normalised) with 95\% CIs, across prosodic conditions. PDRs are time-locked to the onset of the deviant (zero ms), as indicated by the vertical dashed line.}
	\label{fig:5.12confirmatory}
\end{figure}

Comparing PDR curves between rising and neutral conditions, the model revealed that rising intonation elicited a more robust PDR, in terms of both the \textit{overall height} (parametric difference: β = --37.06, t = --4.380, \textit{p} < 0.001) as well as the \textit{shape} of the PDR curve (smooth difference: EDF = 3.198, F = 7.038, \textit{p} < 0.001), reflecting an increased and long-lasting effect. For the contrast between falling and neutral intonation, the simultaneous confidence interval test indicates that the PDR curve associated with falling edge tones was significantly different from the PDR curve associated with neutral intonation (t = 2.357), in that the amplitude is larger and prolonged. Finally, when comparing rising to falling intonation, rises are differentiated from falls by the \textit{shape} of the PDR curve (smooth difference: EDF = 1.048, F = 3.632, \textit{p} = 0.05), meaning that rising intonation exhibits a more sustained effect over time.

In sum, both rising and falling edge tones evoked greater pupillary dilations, both in magnitude and duration, than the baseline neutral intonation. At the same time, PDR to rises showed a more sustained effect over time compared to the falling condition.

\subsection{{Exploratory analyses}}

\subsubsection{{Sequence length}}

GAMM smooths (including 95\% CIs) of PDR to deviant numbers, as a function of prosodic condition and sequence length are shown in \figref{fig:5.13length}. As before, black depicts the estimated smooths of deviants neutrally produced, yellow depicts the estimated smooths of deviants produced with rising intonation, and green depicts the estimated smooths of deviants produced with falling intonation. Solid and dashed lines represent long and medium-length sequences, respectively.

\begin{figure}
	\centering
	\includegraphics[width=\textwidth]{figures/LENGTH_CURVES.png}
	\caption[GAMM smooths of PDR (normalised) across prosodic conditions and sequence lengths] {GAMM smooths of PDR (normalised) with 95\% CIs, across prosodic conditions and sequence lengths. PDRs time-locked to the onset of the deviant (zero ms), as indicated by the vertical dashed line.}
	\label{fig:5.13length}
\end{figure}

The models revealed an interaction between prosodic condition and sequence length. Specifically, when comparing rising to neutral deviants (see solid lines in middle and left panels of \figref{fig:5.13length}), and falling to neutral deviants (see solid lines in right and left panels of \figref{fig:5.13length}) in long sequences, the results showed that both rising and falling deviants elicited greater PDRs in terms of both the \textit{overall height} ([rise vs. neutral] parametric difference: β = 42.96, t = 3.259, \textit{p} = 0.001; [fall vs. neutral] parametric difference: β = 35.35, t = 2.904, \textit{p} = 0.001) and \textit{shape} of the PDR curve ([rise vs. neutral] smooth difference: EDF = 5.375, F = 5.602, \textit{p} < 0.001; [fall vs. neutral] smooth difference: EDF = 4.291, F = 5.699, \textit{p} < 0.001), demonstrating an increased and long-lasting effect. Nevertheless, no differences were found between rising and falling deviants in long sequences (see solid lines in middle and right panels of \figref{fig:5.13length}; [rise vs. fall] parametric difference: β = --6.984, t = --0.622, \textit{p} = 0.53; smooth difference: EDF = 1.011, F = 0.103, \textit{p} = 0.76).\footnote{Although the current models could not test for the same comparisons in medium-length sequences, it can be seen from the GAMM smooths in  \figref{fig:5.13length} that the differences in medium-length sequences follow the same trends as in long sequences. That is, on the one hand both rising and falling deviants in medium-length sequences appear to evoke more robust PDRs than neutral deviants (see dashed lines in \figref{fig:5.13length}), as they do in long sequences. On the other hand, it appears that rising deviants do not differ from falling deviants in medium-length sequences, similarly to the long sequences.}

When comparing rising and falling deviants in long sequences to rising and falling deviants in medium-length sequences (see middle and right panels of \figref{fig:5.13length}, respectively), the models show that both edge tones evoked PDRs of smaller amplitude and shorter latency in medium-length sequences as opposed to long sequences ([rise, long vs. rise, medium] parametric difference: β = --32.20, t = --2.747, \textit{p} = 0.001; smooth difference: EDF = 4.514, F = 3.210, \textit{p} = 0.001;  [fall, long vs. fall, medium] parametric difference: β = --43.86, t = --2.795, \textit{p} = 0.001; smooth difference: EDF = 3.271, F = 5.017, \textit{p} < 0.001). However, when comparing neutral deviants in long and medium-length sequences (see left panel of \figref{fig:5.13length}), PDRs to medium-length sequences differ only in \textit{shape}, in that they exhibit a less curvy line, indicating a faster response ([neutral, long vs. neutral, medium] smooth difference: EDF = 1.034, F = 5.380, \textit{p} = 0.01), while no \textit{height} differences are observed ([neutral, long vs. neutral, medium] parametric difference: β = --17.86, t = --1.235, \textit{p} = 0.21).

In sum, rising and falling edge tones evoked PDRs of greater magnitude and latency than the baseline neutral condition, in long (and medium-length) sequences. Within both rising and falling edge tones, the long sequences evoked PDRs of larger amplitude and prolonged latency as opposed to medium-length sequences. This was not the case for neutral intonation. For this prosodic condition, long sequences elicited longer pupillary responses, which, however, did not differ in amplitude compared to those elicited by medium-length sequences.

\subsubsection{{Cognitive variability}}
\largerpage
GAMM smooths (including 95\% CIs) of PDR to deviant numbers, as a function of prosodic condition and flanker scores, are shown in \figref{fig:5.14cogn}. Flanker scores are illustrated from left (\textit{lower inhibition}) to right (\textit{higher inhibition}) panels. \figref{fig:5.15cogn} illustrates difference smooths (calculated as smooth A -- smooth B) with 95\% CIs for the comparisons among prosodic conditions as a function of flanker scores (x-axis). The red shaded areas depict the points in the 95\% CI that do not include zero, which are regarded as the windows of significant differences.

\begin{figure}
	\centering
	\includegraphics[width=\textwidth]{figures/COGN.png}
	\caption[GAMM smooths of PDR (normalised) across prosodic conditions and flanker scores] {GAMM smooths of PDR (normalised) with 95\% CIs, across prosodic conditions and flanker scores. PDRs are time-locked to the onset of the deviant (zero ms), as indicated by the vertical dashed line. Yellow depicts the estimated smooths of deviants produced with rising intonation, green depicts the estimated smooths of deviants produced with falling intonation, and black depicts deviants produced neutrally.}
	\label{fig:5.14cogn}
\end{figure}

\begin{figure}
	\centering
	\includegraphics[width=\textwidth]{figures/diffplotCOGN.png}
	\caption[Difference smooths of PDR (normalised) for the comparisons among prosodic conditions as a function of flanker scores] {Difference smooths of PDR (normalised) with 95\% CIs for the comparisons among prosodic conditions as a function of flanker scores. The red shaded areas depict where the 95\% CI does not include zero, indicating the windows of significant differences.}
	\label{fig:5.15cogn}
\end{figure}

The models indicated an interaction between prosodic condition and flanker scores. More specifically, the results show that PDR changes as a function of inhibitory ability across prosodic conditions. For neutral intonation (horizontal top [and middle] panels in \figref{fig:5.14cogn}), it was shown that inhibitory ability modulated PDR \textit{shape}, such that the lower the flanker score, the longer the PDR (tensor product smooth: EDF = 8.442, F = 2.758, \textit{p} = 0.001). Similarly, for falling intonation (horizontal middle [and bottom] panels in \figref{fig:5.14cogn}), inhibitory ability also affected PDR \textit{shape}. From low to average to high flanker scores, a decrease in the duration of the PDR was observed (smooth difference: EDF = 1.900, F = 4.657, \textit{p} = 0.01; tensor product smooth: EDF = 10.322, F = 4.581, \textit{p} < 0.001). For rising intonation (horizontal [top and] bottom panels in \figref{fig:5.14cogn}), the results also show that PDR \textit{shape} changed as a function of inhibitory ability, such that the better the flanker score, the longer the PDR (tensor product smooth: EDF = 8.866, F = 4.476, \textit{p} < 0.001). 

Comparing neutral to rising intonation (horizontal top panels in \figref{fig:5.14cogn} and left panel in \figref{fig:5.15cogn}), PDRs differed in both \textit{height} and \textit{shape}, as a function of inhibitory skills: from low to high flanker scores, the higher the score, the stronger the rising PDRs, showing an increased and long-lasting effect (parametric difference: β = 37.471, t = 4.294, \textit{p} < 0.001; smooth difference EDF = 5.453, F = 6.164, \textit{p} < 0.001; tensor product smooth difference: EDF = 6.162, F = 3.346, \textit{p} = 0.001). Likewise, comparing falling to neutral intonation (horizontal middle panels in \figref{fig:5.14cogn} and middle panel in \figref{fig:5.15cogn}), PDRs differed in both \textit{height} and \textit{shape} as a function of low inhibitory skills, such that the lower the flanker scores, the weaker the neutral PDRs, showing a decreased and subtly faster effect (parametric difference: β = --23.014, t = --2.337, \textit{p} = 0.01; smooth difference EDF = 5.170, F = 3.422, \textit{p} = 0.001; tensor product smooth difference: EDF = 1.003, F = 2.757, \textit{p} = 0.05). Finally, comparing rising to falling intonation (horizontal bottom panels in \figref{fig:5.14cogn} and right panel in \figref{fig:5.15cogn}), PDRs differed only in \textit{shape} as a function of high inhibitory skills, such that the higher the flanker score, the faster the falling PDRs (smooth difference: EDF = 1.898, F = 2.427, \textit{p} = 0.05; tensor product smooth difference: EDF = 7.762, F = 2.887, \textit{p} = 0.001).  

In sum, individuals' inhibitory skills modulated PDR to deviants differently across prosodic conditions. Individuals with higher flanker scores, and thus better inhibitory skills, showed sustained PDRs only to rising deviants, as opposed to neutral and falling ones, with no difference between the latter two ([rise] > [fall = neutral]). In contrast, individuals with lower flanker scores, and hence weaker inhibitory skills, exhibited sustained PDRs to both rising and falling PDRs, but not to the baseline neutral ([rise = fall > neutral]).

\subsection{Interim summary}
\label{subsec:5.2.7}

To briefly summarise the current findings, deviants produced with rising and falling intonation (high and low edge tones, respectively) elicit a greater pupillary response (both in magnitude and duration) than deviants featuring the baseline neutral intonation. This holds for both long and medium numeric sequences. Within the edge tone conditions, rising intonation resulted in a greater pupillary response than falling intonation, manifesting as a more sustained effect over time. In the next section, I will argue that the gradient effects of intonation on PDR response indicate different modulations of attention orienting in each prosodic case.

Turning to the sequence length effect, the results show that both rising and falling edge tones evoked stronger PDRs, showing an increased and long-lasting effect in long sequences compared to medium-length sequences. However, neutral intonation elicited only subtly longer pupillary responses in long sequences, which, in turn, did not differ in height amplitude compared to those elicited in medium-length sequences. In the next section, I will reason that the findings we observe here reflect the temporal conditions under which our auditory processing system develops sensory-memory traces of the stimuli and their regularities, which are crucial representations for the mechanism which underlies the attention orienting response.

Finally, considering individual variability, it is evident that individuals' inhibitory skills affect PDR modulation across prosodic conditions. For individuals with strong inhibitory skills, only rising intonation led to more sustained PDRs, while falling and neutral ones evoked transient pupillary responses. In contrast, for individuals with weaker inhibitory skills, both rising and falling prosodic conditions caused prolonged PDRs, as both led to stronger responses than the neutral baseline. In the following, I will argue that the observed PDR responses in relation to inhibition indicate that the successful evaluation of the deviant, and thus the activation of the different stages of the orienting response, depends on individual-specific cognitive bandwidth. 

\section{Discussion}
\label{sec:discussion}

In this chapter, I investigated the relevance of domain-final rises and falls (the reflexes of phrase final edge tones) for attention orienting in German. Domain-final rising intonation has been shown to attract attention in serial recall experiments \citep[e.g.,][]{savino_intonation_2020, rohr_effect_2022, grice_rises_2024}. Here, I tested the attention orienting function of domain-final rises in a different way, namely by utilising a changing-state oddball paradigm in a pupillometry study. The aim of the study was trifold. Using PDR as a reflex of attention orienting, the main objective was to test whether deviant numbers in seriatim ascending sequences can capture more attention, thereby evoking more robust PDRs, when realised with a final rise compared to when they are produced with a final fall or when their intonation does not differ from that of the standards (the neutral case). In addition, this study took an explorative direction in the examination of sequence length effect on PDR elicitation, and the contribution of individual cognitive variability to the processing of deviances, with the aim of better understanding the function of attention orienting.

Prior to the pupillometric study, the design of the stimuli was piloted using a button press paradigm in an explicit deviant-detection task (see Appendix~\ref{sec:pilot}). It was confirmed that the design of the stimuli was successful, in that listeners indeed perceived and detected the deviances across numeric sequences. The pilot data provided some further interesting results which are discussed in \sectref{subsec:discAppendix}. The structure of the current section is the following. First, I put the findings in the broader context of intonational rises vs. falls and their relevance for attention orienting, and consequently, I elaborate on the implications of these findings for the function of rising intonation at the edges of constituents as an attention orienting device. Next, I highlight the relevance of sequence length for the orienting response in passive listening. Finally, I discuss the contribution of individual cognitive variability in the activation of attentional mechanisms.

\subsection{{Rising intonation at the edges of constituents as an orienting device}}
\largerpage
Starting with the group level findings from the confirmatory part of this study, my predictions were the following. First, given that attention orienting is indexed by dilations in pupil size \citep[e.g.,][]{liao_human_2016, marois_eyes_2018, marois_is_2019}, and given the claim that attention orienting is underlined by a mechanism driven by expectancy violations \citep[e.g.,][]{naatanen_auditory_2011, vachon_broken_2012}, I predicted that the presentation of a deviant number in a seriatim ascending numeric sequence would disrupt the anticipated pattern, causing an attention switch expressed in a PDR. In addition, given the association between attention orienting and rising pitch reported in previous work \citep[e.g.,][]{naatanen_early_1978, naatanen_brain_1980, alain_brain_1994, doeller_prefrontal_2003, chobert_deficit_2012, hsu_brain_2015, ventura_attention_2020, rohr_signal-driven_2021, lialiou_auditory_2024}, the second prediction was that deviants produced with the rising edge tone would result in greater disruption, expressed in a more robust PDR compared to deviants produced with the falling edge tone or with neutral intonation.    

The present pupillometric results provide evidence in favour of the first prediction: deviant numbers do indeed elicit a PDR, regardless of intonation. These results corroborate previous findings on auditory cognition claiming that attention orienting is underpinned by an expectancy violation mechanism \citep[e.g.,][]{hughes_disruption_2007, vachon_broken_2012, paavilainen_mismatch-negativity_2013, hughes_auditory_2014}. Now, when the intonation of deviants differed from the intonation of standards, exhibiting either rising or falling edge tones, deviants elicited an increased pupil dilation effect compared to the neutral condition, in which the intonation of the deviant did not differ from that of the standards. This finding indicates that prosodic marking on deviants results in more robust attentional resources being allocated towards the violation. This finding is compatible not only with the idea from the early literature that attention orienting is sensitive to the physical properties of auditory deviances \citep[for more, see][]{wright_orienting_2008}, but also with results from neurocognitive studies reporting that signals with both rising and falling acoustic properties attract attention when presented as deviants (for literature review, see \sectref{sec:2.4}).

Moving to the second prediction, and hence the comparison between rising and falling edge tones, a subtle PDR \textit{shape} difference was found between the two tones. This difference indicates that rising tones evoked more long-lasting changes to PDRs than falling tones. These results potentially reflect qualitatively different cognitive processes. As discussed in \sectref{sec:2.5}, the latency of attention-related PDRs is indicative of the processing stage in which the attentional mechanism is activated: transient dilations reflect pre-attentive attentional processes, while more prolonged responses indicate processes at the conscious level \citep[see][]{strauch_pupillometry_2022}. As discussed in \chapref{ch:2}, attention-related PDRs consist of different subcomponents, two of which are the orienting-related and the executive-related PDR. Orienting-related PDRs are usually momentary, and typically evoked by sensory events such as an instance of auditory deviance, while executive-related PDRs are prolonged and evoked by top-down processes. Thus, an orienting-related PDR, evoked by an auditory deviant, reflects an involuntary attention switch, which in turn, and after some evaluation of the deviant in the processing system, can further activate the executive network, thereby bringing the deviant into awareness. In other words, the route from pre-attentive to conscious perception of a deviant is reflected in the time course of pupillary responses. The findings of this study indicate that, on the one hand, falling edge tones induce an involuntary attention switch towards the deviant pre-attentively, as reflected in the observed transient pupil dilation. On the other hand, rising edge tones initially cause a similar involuntary switch of attention, and then, potentially due to their prominence, bring the perception of the deviant into awareness, thereby activating voluntary attention orienting. This finding can be related to previous neurocognitive studies \citep[e.g.,][]{rinne_superior_2005, macdonald_effects_2011} suggesting a differential processing of rises and falls. Whilst deviant sounds with both rising and falling acoustic properties have been found to attract attention in auditory cognition, rises have been claimed to serve as intrinsic warning cues due to their saliency \citep[e.g.,][]{bach_rising_2008}. Specifically, rises have been reported to prompt auditory looming effects, which indicate an approaching sound source, whereas falls are experienced as receding sound sources, activating fewer attentional resources than rises (for literature review, see \sectref{sec:2.4}). Let me now elaborate on how the results of this study elucidate the aforementioned point in the linguistic domain.

In the two oddball conditions discussed above, the deviant numbers are realised with rising and falling edge tones, whereas their corresponding standard stimulation is produced with neutral intonation (shallow falling intonation; for the acoustic characterisation of the intonational contours see \sectref{subsec:5.2.3}). In such sequences, the listener potentially builds predictions both on the basis of the sensory input and on the linguistic meaning this sensory input transmits. First, listeners detect regularities in the sensory input, then they extrapolate a pattern, and finally, they predict upcoming events (for more on the auditory system operations, see \sectref{sec:2.3}). Thus, the listener, after being exposed to the seriatim ascending numeric pattern produced with neutral intonation, not only predicts and anticipates the following stepwise ascending number, but also anticipates that this number will be produced with the same intonational pattern. Now, as shown in \chapref{ch:4}, predictions can also arise from linguistic meaning. The items used in this study are numeric sequences, which inherently give rise to expectations about forthcoming numbers. On top of this natural prediction context created by the (semantics of the) numeric sequences, the standard/seriatim numbers are produced with the aforementioned neutral intonational contour, a contour which can be featured on non-final items of a sequence or a list in German (for more on German list intonation, see \sectref{sec:3.5}). The expectations about a forthcoming number are therefore enhanced by the linguistic meaning of the intonation with which the standard numbers are realised, that is, a natural and appropriate contour on non-final list items which denotes that the sequence is not over yet. Now, the intonation of standard stimulation might have elicited additional meaning-based predictions, specifically, expectations about the forthcoming end of the numeric sequence. Therefore, the listener, on the one hand, anticipates a forthcoming ascending number, on the basis of both the seriatim pattern and the intonational meaning. On the other hand, the listener expects that the sequence will end at some point, and therefore seeks an intonational cue that signals this end. Both rising and falling edge tones, although they denote different things (rises indicate continuity, whereas falls indicate finality), can mark the end (or the boundary) of a smaller (or larger) unit within the utterance. Thus, domain-final rises and falls mark the end of smaller units within sequences and are contextualised in a similar way in the items of this study: the deviant number, produced with either of the two tones, is the last item (but still part) of the same sequence unit~-- but not the last item of the entire list~-- thereby violating the anticipated seriatim pattern. Furthermore, the presentation of the edge tones violates the expectation that the next number should be produced with neutral intonation. This means that, on the one hand, rising and falling edge tones validate the anticipation of a forthcoming sequence end, and on the other hand, they violate the anticipated seriatim pattern produced with neutral intonation, since they mark a deviant number included in the same unit with the standard numbers. 

The initial observation is that the presentation of a deviant, produced with either of the two edge tones, evokes an increased PDR (compared to a neutral deviant), indexing a greater switch of attentional resources towards the violation of the seriatim pattern produced with neutral intonation. Nonetheless, when the deviant number is produced with rising pitch, this elicited subtly prolonged pupil dilations, indicating that, in this case, the violation first causes an involuntary attention switch, followed by voluntary attention orienting and thus conscious processing. In a linguistic context where deviants produced with either of the two tones generate and violate the same signal-driven and meaning-based expectations, the ability of rising deviants to ultimately lead to conscious attention orientation, in contrast to falling (and neutral) deviants, can be attributed to their prominence. This is in line with \citet[][]{rohr_signal-driven_2021} showing that the amount of attention oriented to a stimulus is defined by signal-based cues combined with meaning-based expectations derived from the context. This is also manifested in the results reported in \chapref{ch:4}, where I showed that involuntary attention orienting is signal-driven, while voluntary attention is driven by meaningful aspects of intonation licensed by contextually created expectations (see \sectref{sec:4.4}). 

We can argue that the present results highlight the attention orienting function of rising pitch, suggesting that domain-final rises on deviant stimuli also enhance the ability of the deviant to attract attention at a more conscious level. This finding thus strengthens the case for the role of rising intonation at the edges of constituents in attention orienting. As discussed in \chapref{ch:3}, intonational events are phonologically anchored to specific positions in the prosodic structure, that is, they are either associated with the stressed syllable (pitch accents), or with the edges of constituents (edge tones). In the autosegmental-metrical theory of intonational phonology \citep[e.g.,][]{ladd_intonational_2008} pitch accents are associated strictly with a prominence-cueing function, while edge tones are associated with a phrasing function. In that sense, it has been claimed that accentual rises are better in directing listener attention than rises at edges (for more, see \sectref{sec:3.1}). This has already been called into question by results from serial recall studies \citep[e.g.,][]{savino_intonation_2020, rohr_effect_2022, grice_rises_2024} which report that rising edge tones marking the final item of non-final triplets boost the recall accuracy of the whole triplet, hence orienting attention to the whole domain (see also \sectref{sec:3.3}). The work reported in \chapref{ch:4} adds to those first indications of domain-final rising intonation attracting attention in showing that, during on-line processing, rising intonation takes on a special role in involuntary attention orienting, regardless of whether the rise played the role of accentual or boundary contour. Therefore, the results of the current study contribute further to this novel discovery that rises associated with the edges of constituents can also direct listener attention at a more conscious level. Building on the aforementioned serial recall results, one speculation is that domain-final rises attract more attention than falls because rising edge tones attract attention to the entire domain. It is hence possible that rising edge tones encourage the listener to more quickly integrate the deviant with previous items in the same domain, and that listener awareness of the violation thereby increases.

The privileged status of the rising edge tones is also shown in the pilot results on speed--accuracy trade-off and individual variability (see \sectref{exp}). In the rising condition, no trade-off between speed and accuracy was observed, whereas such a trade-off was found in both the neutral and falling conditions. Specifically, faster responses in the neutral condition and slower responses in the falling condition lead to worse performance. Further, by taking into account the cognitive profiles of the listeners in the pilot study, we can see that listeners with better inhibitory skills were more accurate in identifying the rising deviants than the average, jolting them out of their inhibition. These results show that, even in a relatively easy explicit task where listeners are aware of the deviant numbers, neutral and falling intonation can be somewhat challenging for the listeners, whereas rising intonation appears to capture listener attention more readily. 

\subsection{{The relevance of sequence length for attention orienting in language}}

Moving to the more exploratory part of this study, I first focus on the relevance of sequence length for attention orienting in language. The pilot data show no difference between the two sequence lengths across all conditions, meaning that listeners were equally good and equally fast in detecting the deviant in both medium-length and long sequences, when explicitly asked to do so (see \sectref{exp}). In contrast, the pupil results show that deviants in long sequences evoked more robust PDRs than deviants in the medium-length sequences across all prosodic conditions, with the neutral condition exhibiting this difference more subtly. The interpretation put forward here is that in passive listening, long sequences allow the auditory processing system to develop stronger memory traces, i.e., representations, of regularities in stimulus sequences, which are crucial for the expectancy violation mechanism. To illustrate the aforementioned point, consider how our sensory systems operate. According to \citet{naatanen_auditory_2011}, the sensory systems develop representations, called memory traces, related to the stimuli and their regularities. The perception of a stimulus occurs at the transient phase of memory trace formation. Our brain continuously updates the representations in order to maintain a correspondence with the environment. During this updating process, the system extrapolates a pattern and forms certain predictions. When these predictions are not attested in the input that the system perceives, violating the representations formed, attention-calling mechanisms are activated \citep[for more on the operations of the auditory system, see \sectref{sec:2.3}, and among others,][]{friston_free-energy_2010, friston_does_2018, naatanen_mismatch_2019}.

Neurocognitive studies using simple sine waves have shown that for the memory trace of standard stimulation in passive auditory oddball paradigms to be formed, the presentation of standard auditory events, in the beginning as well as during the experimental session, needs to exceed a threshold of about 200 ms, also called the temporal window of integration \citep[TWI; see][]{naatanen_mismatch_2019}. The presentation of the standard numbers before the deviant in both sequence lengths of this study's items undoubtedly exceeded this TWI, simply because the items consisted of quadrisyllabic and pentasyllabic numbers, meaning that each number itself had a duration of 1000 ms or more. Both sequence lengths presented a sufficient amount of standard numbers before the deviant, specifically, 10 standard numbers (TWI of ca. 10 seconds) in the medium-length and 15 standard numbers (TWI of ca. 15 seconds) in the long condition, since deviants in both length sequences activated attentional mechanisms as reflected in the elicited PDR responses. The fact that PDRs were more robust in the long sequences shows that the longer TWI allows for a better and stronger formation of the memory trace corresponding to the standard stimulation, in turn leading to a greater switch of attention when this memory trace is violated. Therefore, these findings attest that in language, by virtue of the stimuli being of longer duration than the simple sine waves used in studies of auditory cognition, a TWI of at least 10 seconds is needed to achieve the memory trace formation of standards and the concordant pattern extraction and prediction build-up. A longer TWI may help the processing system to build stronger memory traces, in turn leading to a more robust switch of attentional resources towards a violation. Nevertheless, for explicit deviant-detection tasks such as in the pilot study (see Appendix~\ref{sec:pilot}), it appears that the longer sequences do not provide any advantage over the medium-length sequences.

\subsection{{The contribution of individual variability to attention orienting}}

A final, exploratory objective that this study put forward concerns the contribution of individual cognitive variability to attention orienting. Participants' individual variability was measured on the basis of three cognitive skills: processing speed, inhibitory ability, and WMC. The cognitive measurements yielded weak correlations (see \sectref{subsec:5.2.5}), and thus the pupillary statistical analysis modelled only an interaction between flanker scores, reflecting participants' inhibitory ability, and prosodic conditions. The data show that the presentation of deviants attracts listeners' involuntary attentional resources, regardless of intonation. This is manifested in the dilations of participants' pupil size across all prosodic conditions. The data further indicate different PDR modulations for the three intonational patterns marking the deviants, as a function of inhibitory ability. More specifically, individuals with stronger inhibitory ability exhibited sustained PDRs to deviants produced with rising intonation, in contrast with more rapid PDRs to deviants produced with falling or baseline neutral intonation. Conversely, individuals with weaker inhibitory ability responded to both rising and falling deviant prosodic realisations with equally sustained PDRs, causing a more increased and prolonged effect than the neutral baseline. 

As mentioned previously in this chapter, the time course of pupillary response is indicative of the different stages of the orienting response: transient dilations indicate pre-attentive processes; more prolonged dilations reflect processes at the conscious level. Therefore, the present data indicate that the presentation of the deviant first causes an involuntary attention switch, across all prosodic conditions. From this moment, a constant tripartite interaction starts among signal-based cues from the input (bottom-down mechanism), top-down operations like contextual expectations driven by the meaning-based cues of the input, and other top-down executive mechanisms like processing speed, inhibition and WM, which inform the processing system about prediction errors (i.e., the presence of deviants). All mechanisms together feed the attentional system, which in turn determines whether the pre-attentive involuntary attention switch will lead to the deviant reaching full awareness, or whether the auditory processing system will continue with the processing of the following input. The successful evaluation of the deviant, and thus the activation of voluntary attention, depends on individual cognitive bandwidth. Inhibitory ability is a crucial cognitive factor to consider in understanding the observed PDR variability across individuals. Nonetheless, it is worth considering the three cognitive skills comprising cognitive profiles~-- inhibitory ability, processing speed, and WMC~-- together in an endeavour to better comprehend which cognitive operations interact with attention orienting across individuals (and how). As a general trend, correlation tests indicated that, in the current sample, individuals with better inhibitory abilities tended to be characterised by a slower processing speed and smaller WMC. Conversely, individuals with poorer inhibitory abilities tended to show a faster processing speed and larger WMC. Let us now consider the aforementioned PDR findings in the light of individual cognitive variability.

Individuals with high flanker scores, and hence strong inhibitory skills, exhibited prolonged PDRs only towards rising deviants, manifesting as a long-lasting effect, whereas they responded with quite rapid PDRs to falling and baseline neutral deviants. Despite a good ability to inhibit irrelevant information outside the current attentional focus, rising deviants broke the shield of voluntary attention, not only momentarily, but for longer time, bringing the perception of the relevant deviant to a more conscious level. Such individuals tended to exhibit a slower processing time, which gave them enough time to properly evaluate the importance of the deviant. During this time, the bottom-up mechanism interacts with the processing system and feeds it with cues coming from the signal. In the case of a rising deviant, the processing system gives it high importance, due to the deviant's high acoustic saliency and linguistic prominence. Given the established importance of the deviant, inhibition is blocked. The deviant thus enters WMC and subsequently activates voluntary attention. In the case of a falling deviant, or a deviant with baseline neutral intonation, the processing system renders the deviant's importance low, due to its (very) low acoustic saliency and linguistic prominence. Given the established low importance of the deviant, inhibition is activated, and attention is thus drawn back to the initial focus. In this latter case, the deviant's withholding from further processing is also achieved by WM resistance. Some studies on attentional control have shown that increased WM load prevents or attenuates auditory distraction by utilising a top-down control \citep[e.g.,][]{sanmiguel_when_2008}. Here, individuals' low WMC potentially leads to high WM load, resulting in WMC resistance towards the unimportant deviant in order to minimise the disruption of storage processes \citep[e.g.,][]{berti_working_2003, sanmiguel_when_2008}. In the case of rising deviants, such an operation is absent potentially due to the established significance of the deviant, rendering further processing necessary. 

In contrast to individuals with high flanker scores, individuals with low flanker scores, and thus weaker inhibitory skills, responded with prolonged PDRs to both rising and falling deviants, but with transient PDRs to baseline neutral deviants. This means that deviants, regardless of their prosodic marking (rising vs. falling), broke through the shield of voluntary attention. However, when a deviant did not differ prosodically from the standard numbers, owing to its low acoustic saliency and linguistic prominence, it only evoked an involuntary switch, drawing attention back to the initial focus. As the data reported in this chapter show, these individuals exhibited lower inhibitory skills and further tended to be characterised by faster processing speed and larger WMC. One potential scenario is that fast processing, in the current study, leads to inadequate evaluation of the importance of the acoustic cues characterising the deviant event. In other words, the lack of time for deviant evaluation may result in shallow processing, and thus an ineffective judgement of a deviant's importance. In this case, the deviant's inefficient evaluation, in conjunction with a weaker inhibitory mechanism and small WMC resistance,\footnote{Remember that these individuals have high WMC, which according to \citet{berti_working_2003} and \citet{sanmiguel_when_2008} prevents WM load, in turn making WMC more susceptible to distractions.} enables the deviant event to move to later stages of orienting (regardless of prosodic marking). Another possible scenario is that these individuals, despite their fast processing speed, engage in deep processing and establish deviant importance. In this scenario, the importance of rising deviants blocks the available suppressing mechanisms and activates voluntary attention. In contrast, falling deviants, due to their attenuated auditory cues, may lead to increased listening effort. Increased listening effort, in turn, may interfere with the operations of the~-- already weak~-- suppressing mechanisms, resulting in increased (auditory) attention demands and thereby voluntary attention orienting. The disentangling of the two plausible scenarios needs further investigation; nonetheless it appears that the weaker the repressive capacity, the easier the attentional overloading.

Previous research on individual variation in auditory attention has yielded sparse and contradictory evidence. Starting with the role of WMC in auditory distractions (i.e., deviations), the present results contradict studies reporting that individuals with high WMC are less susceptible to attentional switches towards auditory distractions or deviations \citep[for review, see][]{sorqvist_high_2013, hughes_auditory_2014}. As mentioned in \citet{sorqvist_high_2013}, the exact nature of the mechanism that WMC taps into is still under debate, with one of the views claiming that high WMC attenuates distractions because individuals with high WMC have excellent inhibitory skills. It is important to note that in the relevant studies, a link between WMC and inhibition was presumed, but not tested directly. The present work differs from the aforementioned studies, as WMC and inhibition were measured separately, showing that high WMC does not necessarily imply high inhibition \citep[although WM constitutes a control system; see][]{baddeley_working_2003}. Instead, we have seen that individuals with high WMC are characterised by lower inhibitory skills than individuals with low WMC.

Whereas some studies investigating the role of WM in attention control have shown that higher WMC leads to fewer distraction effects, other studies have shown the opposite. The present results are compatible with the latter. More specifically, they are in line with \citet[][]{berti_working_2003} and \citet[][]{sanmiguel_when_2008}, who show that increased WM load (i.e., low WMC) attenuates or even prevents auditory distraction. \citet[][]{sanmiguel_when_2008} argue that the type of distraction influences WM load effects. Specifically, the authors mention that in the studies where WM load was found to increase distraction, distractions were actually ``competing" stimuli generating task-related conflicts, as in a Stroop task. However, the distractor stimuli in the aforementioned study~-- or better, the deviant stimuli~-- were task-irrelevant, orienting attention away from the task. Likewise, in the present work, deviants were task-irrelevant, with no conflict being generated. Further, \citet[][]{sorqvist_high_2013} showed in a meta-analysis that high WMC does not correlate with attenuated distraction when deviants are task-irrelevant. I am taking these findings as further evidence in support of the results reported in the present work.

To my knowledge, no previous study has directly tested the effect of inhibition or processing speed on auditory attention. The study by \citet[][]{keye_individual_2009} investigated the relationship among WM, inhibition, and processing speed. Nonetheless, it is important to note that \citet[][]{keye_individual_2009} were concerned with \textit{visual} selective attention. \citet[][]{keye_individual_2009} found no support for a relation between WM and attention control or inhibition (conflict reduction). Further, the authors found a relation between high WMC and slow speed, as opposed to the present results. However, the findings of the present study are in line with the \citet[][]{heitz_focusing_2007} study, again on visual attention, in reporting that individuals with high WMC had faster responses than individuals with low WMC. 

Considering the present findings holistically, it appears that individuals differ in the effective mechanisms they have at their disposal. The evaluation of deviants by the processing system constitutes the first crucial step in determining which further operations will be activated. Individuals may or may not have an effective processing system, leading to successful or unsuccessful evaluation, respectively. The next step involves the activation of suppression mechanisms such as inhibition and WMC resistance. Likewise, individuals may or may not have efficient suppressing mechanisms to protect their attentional system from potential overloading. In this study, some individuals appear to have at their disposal both an efficient processing system, which evaluates the importance of the deviants successfully, and sufficient suppression mechanisms, activated when needed. When a deviant event is rendered important, suppressing operations~-- including inhibition and WMC resistance~-- are withdrawn and involuntary attention changes to voluntary attention, bringing the deviant event to conscious processing. When the event is rendered unimportant, suppressing operations are activated, blocking irrelevant distractions from later stages of orienting and protecting the attentional system from unnecessary overloading. For the remaining individuals, it is not as clear whether the processing system is efficient in evaluating the importance of deviants, but it is evident that they do not have at their disposal strong suppressing mechanisms. As a result, deviants, regardless of whether their importance has been established or not, command additional attentional resources, which maximises attentional processing load.

\section{Summary}
\label{sec:summary}

Pupillary data from 60 native German listeners were analysed for the present study. The findings
\begin{enumerate}[leftmargin=*, label=\arabic*.]
	\item corroborate the claim that attention orienting is underpinned by an expectancy violation mechanism,
	\item confirm that pupillometry is a rigorous technique for studying attention orienting,
	\item index the attention orienting function of domain-final rises in speech,
	\item show the relevance of temporal window of integration (TWI) length for attention orienting in language, and
	\item highlight the important contribution of individual cognitive variability.
\end{enumerate}

With respect to the attention orienting function of domain\hyp final rises in German, the current results support the idea of an attentional bias towards pitch rises, and extend the scope of this bias from a general cognitive level to the linguistic domain. In spoken language, rising pitch takes on a special role in attracting attention, also at a more conscious level, and even when this pitch rise is not attributable to a pitch accent, but rather to an edge tone (or edge tone complex). This finding strengthens the argument of \chapref{ch:4} that the phonological status of the pitch event (head- or edge-associated) is not of primary relevance for attention orienting owing to a prevalence for holistic processing. In contrast, both the direction of the pitch contour (signal-based cue) and the appropriateness of the contour in the linguistic context (meaning-based cue) are fundamental for involuntary and voluntary attention mechanisms, respectively. In this study, rising edge tones, specifically, feed both signal- and meaning-based mechanisms, activating both pre-attentive and conscious attentional stages. 

In the current data, rises and falls function in a similar way: they mark the end of a smaller sequence unit, including deviants which violate an expected pattern. Thus, rises take priority over falls precisely because of their prominence. This finding points towards an argument that the neural architecture of auditory cognition and spoken language shares a common pool of mechanisms which are crucial for the attention orienting network. This argument is revisited in \chapref{ch:6}.

As for the role of sequence length, this study shows that language benefits from a longer TWI than cognition. This is due to the complex nature of the speech signal~-- stimuli are longer in duration than simple sine waves. Based on the findings of this study, a TWI of at least 10 seconds is necessary for the formation of a memory trace related to regularities in standard stimulation. At the same time, a longer TWI allows the processing system to build stronger memory traces, which in turn evoke a stronger orienting response when violated. 

Last but not least, the findings of this study show that the cognitive bandwidth deployed by individuals to process auditory deviances is critical both for an effective activation of voluntary attention and protection of the attentional system from potential overloading. Both mechanisms (voluntary attention and cognitive offloading) are crucial for successful speech interpretation. Importantly, individuals differ in the effective mechanisms they make use of, which affects the processing of deviances introduced in the environment. Individuals lacking such effective cognitive mechanisms may end up exhausting mental resources that might have been needed for other purposes. In spoken communication, this could entail a listener's failure to orient attention towards the most important part of the uttered message, which in turn could affect interpretation and speech planning. Individual cognitive variability is thus a crucial factor to take into consideration for processing studies, especially when the general trend in the data represents individuals who can effectively use their attentional resources. The contribution of individual cognitive bandwidth to attention orienting is revisited in \chapref{ch:6}, in which I will bring together the findings of this and previous chapters and suggest a unified view of the cognitive and functional relevance of intonation for attention orienting in spoken language.
