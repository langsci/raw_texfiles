\chapter{General introduction}
\label{ch:1}

The world we live in today is filled with so much noise. We are overloaded every day with information coming from several sources. Nonetheless, we are somehow capable of selecting our object of attention. This is achieved through our cognitive system's ability to shield the processing of information we have chosen to process, protecting it from other available, but currently irrelevant, information. However, in a given sensory environment, unexpected, abrupt, intense, rare, or simply intrinsically interesting sensory events may occur, which in turn may break through this shield, attracting our attention away from our current focus. Switching attention towards sensory events outside of the current focus is a testament to our cognitive ability for detecting, processing, and evaluating such events. This attention switch has been referred to as the \textit{orienting response}, pertaining to different processing stages \citep[e.g.,][]{james_principles_1890, pavlov_conditioned_1927, sokolov_higher_1963, sokolov_perception_1963, posner_orienting_1980, naatanen_mismatch_2019}. Attention orienting has mainly been investigated within cognitive domains like vision and audition, but it is a crucial mechanism for the linguistic domain too. This is because, as discourse unfolds during spoken communication, interlocutors have to deal with a constantly changing influx of information coming from the speech stream as well as from other channels. The present book therefore explores the role of attention orienting in spoken language by focusing on the relevance of intonation for the orienting response, specifically by investigating the temporal unfolding of its neuro- and psychophysiological underpinnings.

\section{Topic, research questions, and main findings}
\label{sec:1.1}

Spoken communication necessarily involves \textit{intonation}, the \textit{melody} or the so-called \textit{tune} of our utterances. Intonation conveys meaning over and above the words bearing it. It is encoded primarily through modulations in fundamental frequency (f0), perceived as pitch variation, but other acoustic properties such as amplitude and intensity (perceived as loudness) or duration (perceived as length), among others, may also play a role. Intonation can differentiate between a number of meanings, and thereby serves multiple functions. For instance, intonation can serve to locate fixed attributes of words, like lexical stress in West Germanic languages, it can indicate whether an utterance is a question rather than a statement, or that an utterance is complete or incomplete \citep[for review on the different functions of intonation, see][]{grice_linguistic_2023}. Crucially for the current work, speakers may also make use of intonation to highlight important parts of an intended message, attempting to draw listeners' attention towards a specific idea they wish to express. Simultaneously, as listeners process their interlocutor's message, their attention is attracted and allocated to information rendered \textit{prominent} through intonation. This is often in the form of a \textit{rise} in pitch, referred to as an \textit{intonational rise}.

Prominence, a notion pertaining to language, is defined as the property of a linguistic unit ``standing out" from its neighbouring environment \citep[e.g.,][]{terken_perception_2000, streefkerk_prominence_2002, himmelmann_prominence_2015, von_heusinger_discourse_2019, cangemi_integrating_2020, grice_prosodic_2021}. Work on linguistic prominence has developed a tripartite definition. Following this definition, prominence is: 
\begin{enumerate}[label=\roman*.]
	\item a \textit{relational} property, meaning that one can render an element prominent only in comparison to other elements of equal type \citep[see, among others,][]{cangemi_integrating_2020}, 
	\item a \textit{structural} property, in that the prominent element functions as a structural attractor, licensing more linguistic operations than its competitors \citep[e.g.,][]{himmelmann_prominence_2015}, and 
	\item a \textit{dynamic} property, meaning that the prominence status of an entity can shift as time, or discourse, unfolds \citep[see, among others,][]{von_heusinger_discourse_2019}.
\end{enumerate}
The notion of prominence applies to all linguistic domains, from phonetics and phonology to morphology and syntax, semantics and pragmatics. This means that speakers are able to use a wide variety of linguistic cues to signal prominence, including prosodic and non-prosodic elements \citep[e.g.,][]{baumann_what_2018}.

Although prominence pertains to the linguistic domain, it features similarities with cognitive notions such as, for example, the notion of \textit{saliency}. More specifically, saliency refers to a sensory event being noticeable within a given sensory environment, by virtue of its physical (also known as \textit{signal-} or \textit{sensory-based}) characteristics, or its intrinsic (also called \textit{meaning-based}) properties\footnote{Throughout this book, I make a consistent distinction between \textit{signal-based} and \textit{meaning-based} properties of stimuli. The signal-based properties, as can be inferred from the term used, refer to the physical, acoustic attributes of a respective stimulus. The meaning-based properties refer to the semantics of a respective stimulus, as well as to meaningful aspects licensed by pragmatics and/or contextually created expectations. For this reason, the terms meaning-based, semantic-based, and context-driven are used interchangeably throughout this book.} (for review, see \chapref{ch:2}). It thus appears that prominence and saliency are much alike, given that they place focus on one entity at the cost of the other entities in an equal set of options. Throughout the remainder of this book, I will show how both prominence and saliency are closely related to the orienting response. For instance, (neuro)cognitive research has shown that stimuli can attract attention because of their physical saliency. More specifically, \chapref{ch:2} shows that in auditory perception, the orienting response is sensitive to the physical properties of the signal, such that the greater the amplitude or pitch rise on a sound, the more enhanced the orienting response towards it \citep[see, among others,][]{naatanen_early_1978, rinne_superior_2005, rinne_two_2006, macdonald_effects_2011, liao_human_2016, wetzel_infant_2016}. In a similar vein, work on prosodic prominence in West Germanic languages has shown that modulations in pitch direction, excursion, and scaling are some of the acoustic dimensions of intonation, also known as \textit{signal-driven} cues, which are indicative of different prominence values. The higher or the more rising the pitch, the greater the prominence value. This makes intonational rises decisive for prominence perception \citep[e.g.,][]{rietveld_relation_1985, gussenhoven_fundamental_1988, gussenhoven_perceptual_1997, ladd_perception_1997, baumann_perceptual_2015, knight_shape_2008, niebuhr_f0-based_2009, baumann_what_2018, rohr_influence_2022}. In particular, \chapref{ch:3} shows that in speech processing, attentional resources are allocated to prosodically prominent constituents, facilitating their processing or their recall performance \citep[e.g.,][]{savino_intonation_2020, ventura_attention_2020, rohr_signal-driven_2021, rohr_effect_2022}. Nonetheless, in addition to signal-driven cues like intonational rises, research has shown that \textit{context-driven} cues, like prior linguistic structure or discourse expectations (or even recent speech experience) have the potential to overwrite the typical acoustic cues that signal prominence, leading to different processes \citep[e.g.,][]{cole_signal-based_2010, bishop_information_2013, kakouros_making_2018}. Overall, as we will see in \chapref{ch:3}, prosodic prominence is expressed in discrete phonological choices, in continuous physical parameters in the phonetic record, and in interactions among acoustic, phonological, and contextual cues \citep[e.g.,][]{baumann_what_2018}. Therefore, intonation takes on a special role in prominence precisely because the link between signal- and context-driven cues intrinsically relates to the discrete phonological forms of intonational choices.

Building on previous (neuro)cognitive and linguistic research, the overarching goal of this book is to understand the relevance of intonation for the orienting response in spoken language. This goal can be divided into two central points of interest. Firstly, the attention orienting function of intonational rises is of particular interest for this work, given their inherent association with prominence. I will focus on the following three aspects: the direction of the intonational event (rise vs. fall); the position or status of a pitch rise (and a fall) in relation to prosodic structure (head vs. edge association; pitch accent vs. edge tone); and the role of language- and context-specific expectations for the processing of rises and falls. Accordingly, the following three questions are addressed:

\begin{description}[font=\normalfont]
	\item[RQ1 {[rises]}:] Do intonational rises attract more attention than falls by virtue of their acoustic prominence?
	\item[RQ2 {[edges]}:] Does the phonological status of the rise (and the fall) affect the underlying processes? In other words, do rises at the edges of constituents attract attention to the same extent as accentual rises?
	\item[RQ3 {[expectations]}:] Do language- and/or context-specific expectations affect attention orienting?
\end{description}

Secondly, the present work is also concerned with the neurocognitive mechanisms underlying attention orienting in the linguistic domain. Literature on audiovisual attention has established that the orienting response can be involuntary or voluntary. In particular, it has been shown that there is an interplay between the nature of a cue that evokes the orienting response and the attentional processing stage in which this response occurs (for more detail, see \chapref{ch:2}). Therefore, the question that this work posits in relation to this second point of interest is the following:

\begin{description}[font=\normalfont]
	\item[RQ4 {[architecture]}:] To what extent is the neurocognitive architecture of attention orienting similar across the cognitive and linguistic domains?
\end{description}

The main claim of this work is that a plethora of cues can activate the orienting response in spoken language, just as in cognition more generally. Nonetheless, distinct cues are tied to the two core attentional stages. Signal-driven cues appear to be crucial for activating an involuntary attention switch, emerging at a pre-attentive processing stage, while context-driven cues are essential for voluntary attention orienting, occurring at a conscious processing stage. This claim can be broken down into three parts. 

First, I propose that intonational rises indeed take on a special role in activating the orienting response, especially~-- but not only~-- pre-attentively. The experimental work reported in this book shows that it is the inherent acoustic properties of rises that make them fundamentally important for involuntary attention-related processes, regardless of whether we are concerned with a simple sine wave, a speech sound in isolation, or meaningful speech. More specifically, the data from the two studies comprising the present work support the previously established idea of an attentional bias towards pitch rises (essentially serving as warning cues), extending it from general cognition to language. Nonetheless, the current work also highlights that, in spoken language, the acoustic signal is not the only source of information. Specifically, the sensory importance of rises is mitigated by the contextual meaning, activating voluntary attention, which in turn is required for updating conscious processing stages and mental representations. Thus, as shown in \chapref{ch:4}, when intonational rises are contextually inappropriate, their importance is overwritten at the conscious stage by acoustically less prominent, but contextually appropriate intonational cues, despite that rises take priority over intonational falls at the early pre-attentive stage. However, when all available intonational cues convey the same contextual meaning, then, as \chapref{ch:5} shows, rises take priority over the other available cues, both pre-attentively and consciously. 

I further contend that some of the cues pertaining to the activation of attention orienting in spoken language are of primary relevance, while others play only a secondary role. For instance, this work shows that both the \textit{direction} of an intonational contour (rise vs. fall) and the \textit{linguistic meaning} are pivotal for the orienting response. In contrast, I will show that the role of the \textit{phonological status} of the pitch event (accent/edge) is only supplementary. An illustrative example of the primary role of the direction of the intonational contour and the secondary role of its phonological status comes from the current results showing that rising pitch, attributable to both accents and edge tones, is globally more successful in attracting attention than falling pitch, reflected in the neurophysiological and pupillary correlates investigated. Similarly, a contextually appropriate intonational rise,\footnote{In the present work, appropriate in the list context.} regardless of its phonological status, can activate both pre-attentive and attentive stages of attention orienting. Yet, a contextually incongruent or inappropriate rise evokes attention only pre-attentively, while attentively its orienting function is ``cancelled out" by other, appropriate intonational cues.  

Finally, I argue that cognition and language appear to share a common pool of mechanisms, crucial for modulating the orienting response both in auditory and speech processing. As \chapref{ch:2} discusses, attention orienting has been linked to both bottom-up (involuntary orienting) and top-down (voluntary orienting) mechanisms, subserved by distinct neural pathways. More importantly, there is an interplay between the mechanisms and low-level sensory cues or high-level cognitive operations: sensory-based cues appear to initiate involuntary attention, feeding bottom-up mechanisms, while meaning-based cues lead to voluntary attention, feeding top-down mechanisms. The current research attests that signal-driven and context\hyp driven prominence cues feed the two core mechanisms related to attention in a similar way. Further, based on the current data, both linguistic prominence and cognitive saliency appear to pertain to similar neural and pupillary underpinnings, providing further evidence for commonalities in the architecture of the cognitive and linguistic domains. Nonetheless, it is important to highlight that, in spoken language, signal-driven and context-driven cues interact with each other in more complex ways than in general cognition, which creates expectations at different layers. The investigation of the incremental processing of linguistic expectations in the current work thus allows for a better understanding of how cues are implemented at different levels (i.e., phonetic, phonological, discourse\slash context levels) as well as how bottom-up and top-down mechanisms interact with each other during speech comprehension.    

\section{Research rationale}
\label{sec:1.2}

This book investigates the relation between intonation and attention orienting by employing neurophysiological and pupillary data from German listeners. I first elaborate on the rationale behind choosing German as the language under investigation in the present work, and then I turn to the methodologies used.

First off, apart from considerations such as availability of large numbers of participants needed for (neuro)physiological experimentation while conducting the current work in Germany, the choice of German was also motivated by its prosodic characteristics. German is a well-studied West Germanic language with regard to its prosodic system. Among other characteristics, it has lexical stress and postlexical tonal events with specific association and functional properties \citep[e.g.,][]{grice_german_2005}. More specifically, in terms of prosodic typology \citep[e.g.,][]{jun_prosodic_2014}, in languages like German, there are two basic ways postlexical tonal events can be associated with positions in the prosodic structure. They can either associate as \textit{pitch accents}, docking onto a stressed syllable (head of the word or foot), or as \textit{edge tones}, pertaining to initial or final edges of smaller or larger constituents. At the same time, pitch accents have been considered to cue prominence, serving a highlighting function, while edge tones have been attributed the function of phonological phrasing. Now, except for the aforementioned functions, intonation also conveys meaning over and above the words or sentences bearing it. This means that different postlexical tonal events convey distinct meanings (without there being a one-to-one mapping), which in turn renders some of them appropriate in certain contexts, while making others inappropriate or surprising. A second reason why German was chosen as the test language here is that prominence-related phenomena have been extensively researched in this language (see \chapref{ch:3}). Given that prominence pertains to attention-related processes, previous work on the prosodic marking of prominence and prominence perception in German allows me to build and test specific hypotheses and predictions in the present work. In particular, the attested prominence value of intonational rises offers a useful starting point for further experimental testing of the cognitive and functional contribution of rises to attention orienting, and consequently to prominence. Therefore, the prosodic and prominence properties attributed to the German language offer an excellent test case for investigating the research questions addressed in the present work.

With regard to the methodologies used in the current research, the two experimental studies comprising this book endeavour to answer the research questions introduced in \sectref{sec:1.1}, by investigating the neural (\chapref{ch:4}) and pupillary (\chapref{ch:5}) underpinnings of attention orienting in spoken language. More specifically, the study reported in \chapref{ch:4} used the method of event-related potentials (ERPs), while the study reported in \chapref{ch:5} employed the method of pupillometry. Both techniques offer a high resolution of temporal information, allowing for the precise measurement of brain and cognitive activity related to attentional processes over time. Previous neurocognitive research has approached the study of attention orienting through measurements of brain activity, focusing on the neurocognitive processes and mechanisms that underlie attention orienting. Specifically, studies using ERPs have revealed specific neural responses, such as the mismatch negativity (MMN) and the subsequent positivity belonging to the P3 family, rigorous correlates of attention orienting. These neural responses have been claimed to reflect the route from pre-attentive to conscious processes \citep[for review, see][]{naatanen_mismatch_2019}. 

More recent psychophysiological research has approached attention orienting using pupillometry. In particular, a growing body of pupillometric research has proposed pupil dilation response (PDR) as a valid index of auditory attention, with similarities to neural MMN/P3 responses \citep[see, among others,][]{alamia_pupil-linked_2019}. For instance, \citet{strauch_pupillometry_2022} have recently shown that the different composites of a PDR can also index distinct processing pathways. Overall, building on these two methodologies used in previous (neuro)cognitive research on attention orienting offers a natural path towards extending the use of correlates which have been shown to be valid indices of the orienting response in general auditory cognition to the study of attention orienting in spoken language. In turn, this extension allows for a more accurate comparison of the function of the orienting response across the two domains.

\section{Outline and preview of the next chapters}
\label{sec:1.3}

This book is structured as follows:

 {\chapref{ch:2}} sheds light on the concept of attention as an orienting response, elaborating on its origins and the mechanisms proposed to underlie it. After a thorough review of the literature on the involuntary and voluntary nature of the orienting response from early studies up to the modern day, the chapter turns to the function of the auditory processing system. In this context, \chapref{ch:2} discusses the processing stages in which involuntary and voluntary attention emerge, demonstrating the brain indices that underlie these stages. Next, the crucial role of the acoustic properties of the sensory input for attention orienting is discussed, demonstrating the special role of signals involving a rise in amplitude or pitch, the latter being the scope of this book. Finally, this chapter places a special focus on showcasing the methods with which research has approached the study of attention orienting, elaborating on event-related potentials and pupillometry, the two methods employed in the present work.

 {\chapref{ch:3}} offers a condensed introduction of fundamental concepts related to spoken language. Its primary goal is to illustrate the association between intonation and its association to the notion of prominence. After elaborating on the phonological organisation of intonation, the chapter turns to the role of intonation, and especially of rising intonation, in prosodic prominence. To this end, it offers a literature review on the linguistic importance of intonational rises in prominence perception and speech processing. Subsequently, \chapref{ch:3} discusses the interplay between signal-driven and context-driven cues in prominence and speech processing. The chapter closes with a brief note on German list intonation as the linguistic context of the two experimental studies reported in this work.

 {\chapref{ch:4}} reports on the first experimental study. This study examines the processing of intonational rises and falls when presented unexpectedly in a stream of repetitive auditory stimuli. It examines the neurophysiological correlates (ERPs) of attention to these unexpected stimuli through the use of an oddball paradigm. In this paradigm, sequences of repetitive stimuli are occasionally interrupted by a deviant stimulus, allowing for elicitation of the mismatch negativity (MMN). Whereas previous oddball studies on attention towards unexpected sounds involving pitch rises were conducted on non-linguistic stimuli, the present study uses lexical items in German with naturalistic intonation contours as stimuli, potentially giving rise to a list context and the corresponding language-specific expectations. This study mainly relates to RQ1 [rises] and RQ2 [edges], but the results also provide important evidence related to the other key research questions (i.e, RQ3 [expectations] and RQ4 [architecture]).

 {\chapref{ch:5}} introduces the second experimental study. This study addresses the research questions from a different angle. Capitalising on the findings reported in \chapref{ch:4}, the study delves further into the attention orienting function of rising edge tones in German, by measuring listeners’ pupil dilation response (PDR) in a \textit{changing-state} oddball paradigm. Here, auditory sequences of sequentially ordered (i.e., seriatim) ascending numbers (\textit{standards}) are occasionally interspersed with an out-of-the-sequence number (\textit{deviant}). This study mainly relates to RQ1 [rises], RQ2 [edges], and RQ3 [expectations], but certainly, the obtained results inform RQ4 [architecture] as well. Further, this pupillometric study places a special focus on individual variability as a means of better understanding the attention orienting function. In this endeavour, selective attention (in terms of inhibitory ability), processing speed, and working memory capacity (WMC) are selected as proxies for measuring individual cognitive variability. Lastly, the study further explores the role of sequence length in attention orienting. The contribution of both individual cognitive variability and sequence length in attention orienting are particularly important for RQ4 [architecture].

 {\chapref{ch:6}} completes the book by bringing together the findings of the two experimental studies in a general discussion. This final chapter offers a unified view on the cognitive and functional relevance of intonation for attention orienting in the linguistic domain.
