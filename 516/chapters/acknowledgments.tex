\addchap{\lsAcknowledgementTitle} 

This book is a revised version of my doctoral dissertation which was accepted by the Faculty of Arts and Humanities of the University of Cologne in 2024. The original acknowledgements are reprinted below with minor updates.

First and foremost, I would like to express my sincere gratitude to my two advisors, Petra Schumacher and Martine Grice, for being models of two fierce female researchers who guided me and inspired me all along. Petra and Martine are two unique individuals, admirable scientists, and excellent professors with a profound interest in their students. I would like to particularly thank both of them for embracing me in the team, persistently supporting me, questioning me, challenging me, protecting me, believing in me, and granting me freedom in pursuing my own scientific ideas! Petra and Martine have inspired me not only with their research ideas and their inquisitive characters, but also with their constant effort to push against boundaries and obstacles that early researchers are facing. I wish to become as fierce in pushing for support and equality in academia and throughout life. I would also like to thank Stefan Baumann, my third examiner, for always giving me detailed and valuable feedback. 

Further, I want to acknowledge support from the DFG-funded CRC 1252 ``Prominence in Language" which provided the institutional structure and funding for conducting my PhD. This work would not have been possible without the doctoral research position I have had the privilege to acquire in the project A01. Further, I thank the a.r.t.e.s. Graduate School for the Humanities Cologne for awarding me the Katharina Niemeyer Grant for language editing.

I would also like to thank all my wonderful colleagues from the ``House of Prominence" for creating a very friendly environment, for sharing ideas and experiences, and for the fruitful~-- as well as bizarre~-- discussions we often had during lunch break. Big thanks goes to my office\hyp buddy Janne Lorenzen, Malin Spaniol, Magdalena Repp, and Guendalina Reul who except for amazing colleagues and peers have become truly caring friends, supporting me in various aspects of my life. Moreover, I would like to thank Sandra Vella and Elina Savino for their support, mentoring, help, and advice. Additionally, I thank Christine Röhr and Aviad Albert for their invaluable input regarding my research, their support, and collaborative spirit. Further, I want to express my truthful gratitude to Simon Wehrle for editing this entire dissertation. Big thanks goes to Ingmar Brilmayer as well, for teaching me how to run EEG experiments and how to process EEG data. I am also thankful to Jesse Harris for our collaboration, and for teaching me about pupillometry. Finally, I would like to thank Solveigh Janzen and Nadia Pelageina who supported me with data annotations and participant recruitment.

I would also like to thank all the other people (current and former members) from UoC who have influenced my work in many different, direct or indirect, ways. First of all, I want to thank my a.r.t.e.s. class 8 cohort. From the Department of German Language and Literature I, I want to thank Ina Bornkessel\hyp Schlesewsky, Hanna Buhl, Paul Compensis, Claudia Kilter, Anne Lützeler, Umesh Patil, Clare Patterson, Brita Rietdorf, Clarissa Selegrad, Sophie Sprengel, Caterina Ventura, and Robert Voigt. Brita and Claudia deserve a special thank you for their enormous help with recruiting participants and running my experiments. From the Institute of Phonetics, I thank Florence Baills, Anna Bruggeman, Francesco Cangemi, T. Mark Elison, Alicia Janz, Constantijn Kaland, Theo Klinker, Lena Pagel, Simon Roessig, and Simona Sbranna.

Outside of Cologne, I would like to thank Shravan Vasishth, Bruno Nicenboim, Anna Laurinavichyute, and Bodo Winter for introducing me to the world of Bayesian thinking, and Márton Sóskuthy for teaching me the beauty of GAMMs. They have all shaped my profound interest in data analysis and open science. I am deeply grateful for the opportunity they gave me to consult with them about the analyses reported in this dissertation, acquiring invaluable feedback. I would also like to thank Yiya Chen for all the stimulating discussions we had about my EEG investigation, and Kalliopi Katsika for helping me navigate academia.

Besides academia, I would like to express my gratitude to my beloved parents, Theofilos and Tonia, and to my two amazing brothers, Stefanos and Pericles, for being there for me in every step of my life! Σας ευχαριστώ για όλα! Further, a big thanks to my beloved friends Maria and Evi for their endless support despite the geographical distance. Finally, to my wonderful partner and best friend: Bill, thanking you here is definitely not representative of the role you have played in my PhD endeavour and in my life in general. I cannot express enough how grateful I am for your never-ending patience and support. Thank you for agreeing to move in Cologne with me, thank you for encouraging me, believing in me, taking care of me when I had no time for anything else than working on this dissertation, listening to my ideas regardless of how bored you got, and cheering me up when I got overwhelmed. Thank you for loving me! Ως τον ουρανό και πάλι πίσω!
