\chapter{Intonation and its association with prominence}
\label{ch:3}

\textit{Intonation} is a fundamental characteristic of spoken language and is usually described as the \textit{melody} of speech, or tone of voice. Intonation and \textit{prosody} are quite often used interchangeably~-- not unreasonably~-- as the two terms are closely interrelated. On the one hand, one could think of prosody as an umbrella term, pertaining broadly to the suprasegmental properties of speech used by speakers and listeners to encode and decode a range of interrelated phenomena \citep[e.g.,][]{barnes_autosegmental-metrical_2022, grice_linguistic_2023}, such as:

\begin{enumerate}[label=\roman*.]
	\item \textit{linguistic} phenomena, like metrical and rhythmic structure, prosodic phrasing and prosodic prominence,
	\item \textit{paralinguistic} and \textit{extralinguistic} phenomena, like communicative intentions, attitude, emotions, sociolinguistic identity, gender, sex and age.
\end{enumerate}
On the other hand, intonation is more narrowly related to the melodic part of speech, conveying meaning in a ``\textit{linguistically structured way}" \citep[][4]{ladd_intonational_2008}. This encompasses intonational tones being associated to particular positions in the prosodic structure, organised in discrete phonological categories related to intonational phonology \citep[e.g.,][]{ladd_intonational_2008}. Although intonational phonology separates the phonological organisation of intonation from its phonetic substance,\footnote{It is proposed that this separation enables intonational models to form descriptions and set requirements of intonational analysis at a more abstract level, so that similar levels of analysis can apply across languages \citep[for more, see][]{ladd_intonational_2008, barnes_autosegmental-metrical_2022}.} speech is a continuous signal, meaning that intonation is also expressed in the phonetic record. Fundamental frequency (hereafter, f0), perceived as pitch, and its modulations have been taken to be the main phonetic exponent of intonation \citep[e.g.,][]{barnes_autosegmental-metrical_2022}. Nonetheless, f0 modulations interact with other phonetic properties such as intensity (perceived as loudness), segmental durations (perceived as length), as well as spectral and articulatory characteristics \citep[e.g.,][]{ladd_intonational_2008}. In addition, f0 also serves other prosodic purposes, such as the encoding of lexical contrasts in Mandarin or Cantonese \citep[e.g.,][]{barnes_autosegmental-metrical_2022}, or paralinguistic and extralinguistic information \citep[e.g.,][]{grice_linguistic_2023}. Therefore, it is important to clarify that while f0 constitutes the central phonetic manifestation of intonation, f0 and intonation are not synonymous \citep[e.g.,][]{barnes_autosegmental-metrical_2022}.

When we speak, our intonation conveys meaning over and above the meaning of the syllables, words, and sentences bearing it. This means that the intonation of an utterance, also known as the \textit{tune}, can differentiate between a number of meanings, and serve several functions. For example, an intonational tone can indicate that an utterance is incomplete and that there is more to come, rather than being complete, or that it is a question rather than a statement. In addition, speakers make use of intonation to highlight particular information in their utterances or to chunk information into smaller units. Both of these functions, highlighting and chunking, serve to orient listeners’ attention to informative parts of utterances which are perceived as \textit{prominent}. As already discussed in the introductory chapter of this book, \textit{prominence} is the property of a linguistic unit ``standing out" from its neighbouring environment \citep[e.g.,][]{terken_perception_2000, streefkerk_prominence_2002, himmelmann_prominence_2015, von_heusinger_discourse_2019, cangemi_integrating_2020}. 

In the domain of prosody, prominence is referred to as \textit{prosodic prominence} and pertains to entities like syllables, feet, prosodic words, and intonational phrases \citep[e.g.,][]{grice_prosodic_2021}. Prosodic prominence is manifested both in discrete phonological choices as well as in continuous physical parameters. Phonologically, prominence is related to elements appearing in ``privileged" positions in the prosodic structure \citep[for more on the structural notion of prominence, see][]{cangemi_integrating_2020, grice_prosodic_2021}. Phonetically, prominence marking is associated with an enhancement of the physical properties (i.e., signal-based variables) characterising the production of an entity, e.g., f0, intensity, duration, periodic energy, and other spectral characteristics \citep[e.g.,][]{grice_introdution_2007, baumann_what_2018, grice_prosodic_2021, albert_model_2023}. In West Germanic languages, a lexically stressed syllable stands out from the rest of the syllables within the same word due to its metrical and/or phonetic characteristics \citep[e.g.,][]{cangemi_integrating_2020}. The prominent syllable at the word level is pivotal in higher prosodic structures like feet or prosodic words, due to the fact that the prominent syllable forms the head of these higher prosodic constituents, which in turn enables a structure to be built around them \citep[e.g.,][]{grice_prosodic_2021}. At the utterance level (also called \textit{phrase level}, or \textit{postlexical} level), for instance, a constituent in focus stands out from the remaining constituents in the background due to the type of pitch accent (i.e., postlexical tonal movements associated with the stressed syllable [head]) as well as the enhancement of the phonetic exponents of the particular pitch accent it bears \citep[e.g.,][]{bolinger_intonation_1986, bolinger_intonation_1989, baumann_-accentuation_2012, lorenzen_information_2022}. Work on prosodic prominence has shown that intonational rises, both in the sense of discrete phonological categories and continuous f0 shapes, are pivotal cues to prominence perception, and consequently for attracting attention \citep[e.g.,][]{baumann_perceptual_2015, knight_shape_2008, niebuhr_f0-based_2009, baumann_what_2018}. The fundamental importance of intonational rises in speech perception and processing is at the core of this book and is further discussed in the remainder of this chapter.

The current chapter offers a concise exploration of concepts related to spoken language, focusing especially on intonation and its association to the notion of prominence. Exploring this association is particularly important for the overarching goal of this book, which is to better understand the relevance of intonation for the orienting response in spoken language, given that the notion of prominence pertains to concepts defining general cognition (including the orienting response; see \chapref{ch:1}). The structure of this chapter is the following: \sectref{sec:3.1} elaborates on the phonological organisation of intonation and its relation to prosodic structure, and discusses one of the most influential accounts of intonational phonology. \sectref{sec:3.2} discusses the role of intonation in the prosodic marking of prominence and shows that intonational rises are a decisive cue to prominence. Subsequently, \sectref{sec:3.3} reviews studies supporting the linguistic importance of intonational rises in speech processing. Next, \sectref{sec:3.4} elucidates the interplay between signal-driven and context-driven processing, in pointing out that signal-based cues can potentially be overwritten by contextual cues, leading to different processes. The chapter closes with \sectref{sec:3.5}, in which list intonation is discussed as the linguistic context of the two experimental studies presented in Chapters \ref{ch:4} and \ref{ch:5}.

\section{Intonational phonology and prosodic structure}
\label{sec:3.1}

\largerpage

\citet[][4]{ladd_intonational_2008} defines intonation as ``the use of \textit{suprasegmental} phonetic features to convey `postlexical' or \textit{sentence-level} pragmatic meaning in a \textit{linguistically structured way}". The idea that intonation is organised in phonological categories, and is separated from phonetic substance, has been postulated in many theories of intonational analysis, with one of the most influential being the \textit{autosegmental-metrical} framework \citep[henceforth, AM; e.g.,][]{liberman_intonational_1975, pierrehumbert_phonology_1980, ladd_intonational_2008}.\footnote{AM is not the only framework positing the idea of intonational categoriality. For example, the British School, the IPO model, and other experimental approaches have proposed categorical uses of intonation \citep[for a comprehensive review, see][]{roessig_categoriality_2021}.} As discussed in \citet{grice_autosegmental-metrical_2022}, the AM framework has been influential for the development of many phonological models of intonation \citep[e.g.,][]{beckman_intonational_1986, shattuck-hufnagel_prosody_1996, gussenhoven_phonology_2004} as well as psychological models of speech production and perception \citep[e.g.,][]{frazier_prosodic_2006, watson_interpreting_2008, wagner_experimental_2010, shattuck-hufnagel_role_2020}. It has also been central in the advancement of the Tones and Break Indices (ToBI) intonational transcription systems \citep[for a summary, see][]{beckman_original_2005} and has formed the basis of prosodic typology \citep[e.g.,][]{jun_prosodic_2005, jun_prosodic_2014-1}.

AM theories assume a hierarchical organisation of prosodic structure, wherein the intonational contour, that is the tune of an utterance, consists of distinct units (referred to as \textit{tonal} or \textit{intonational events}) which are independent from the words bearing them, that is, they are postlexical. These distinct units (or else primitives) constitute sequences of H(igh) and L(ow) \textit{tones} which pertain to nodes on the prosodic tree. A prosodic tree involves nodes related to the word level, that is morae, syllables, feet, prosodic words, as well as higher nodes, above the word level, including Accentual Phrases, Intermediate Phrases, and Intonational Phrases. The phonological identity of a tone is mapped onto the continuous acoustic features of its realisation \citep[e.g.,][]{barnes_autosegmental-metrical_2022}. The specifics of the phonological representation and phonetic manifestation of tones are language-specific \citep[e.g.,][]{jun_introduction_2005}. Given that the current work focuses on German, let us zoom in on West Germanic languages.

West Germanic languages, such as English, Dutch, and German, are characterised by similar prosody and intonation. For instance, they all have lexical stress and use postlexical tonal events with specific association and functional properties \citep[e.g.,][]{grice_german_2005}. In what follows, I will focus only on postlexical tonal events. According to AM theory and prosodic typology \citep[e.g.,][]{ladd_intonational_2008, jun_prosodic_2014}, there are two basic ways in which postlexical tonal events can be associated with positions in the prosodic structure in West Germanic languages: as \textit{pitch accents} or as \textit{edge tones}.\footnote{Hereafter, the terms edge/boundary tone are used interchangeably.} Pitch accents and edge tones have distinct association properties. Whereas pitch accents are associated with stressed syllables, that is, with the head of a constituent, edge tones are associated with initial or final boundaries of smaller or larger constituents. Pitch accents always contain one starred tone, which is the primary tone associated with a tone-bearing unit and marked by an asterisk, or ``star" (*). Pitch accents may further contain an additional tone (e.g., H* or L+H*). Edge tones are marked either by the percentage symbol \% (e.g., H\%) or by the hyphen symbol - (e.g., H-), depending on whether they associate to a larger or smaller unit, respectively. \figref{fig:3.1prosodictree} illustrates the association properties of an utterance's tonal events with respect to the nodes of its prosodic tree \citep[example and figure are taken from][394]{grice_autosegmental-metrical_2022}. From top to bottom, \figref{fig:3.1prosodictree} illustrates the utterance's Intonational Phrase node (IP), consisting of two intermediate phrases (ip), each of which contains one or two accentual phrases (φ). The smallest level of phrasing here consists of one or more prosodic words (ω). In turn, a prosodic word consists of one or more feet (Σ), and a foot consists of one or more syllables (σ). In the bottom of \figref{fig:3.1prosodictree}, we can see the tonal structure associated with the utterance. Specifically, there are four starred tones (see orange boxes), i.e., pitch accents, associated with the stressed monosyllabic words \textit{too}, \textit{cooks}, \textit{spoil}, and \textit{broth}. These four starred tones dock onto the head of a foot, respectively. Three further tonal events, marked in green boxes, pertain to the two ip and the highest IP nodes, respectively, forming the edge tones which mark both smaller and larger units in the structure. 

\begin{figure}
	\includegraphics[width=0.7\textwidth]{figures/prosodic tree.png}
	\caption[Prosodic hierarchy]{Prosodic hierarchy of the utterance ``Too many cooks spoil the broth". Figure reproduced from \citet[][394]{grice_autosegmental-metrical_2022}.}
	\label{fig:3.1prosodictree}
\end{figure}

In AM, functional properties also play a role in the distinction between pitch accents and edge tones. According to AM and prosodic typology \citep[e.g.,][]{ladd_intonational_2008, jun_prosodic_2014}, pitch accents not only associate with stressed syllables, but they are also considered to primarily have a highlighting function related to prominence-cueing in West Germanic languages, which make use of both pitch accents and edge tones.\footnote{One of the three criteria for language classification in \citeauthor{jun_prosodic_2014}'s typology is the \textit{prominence type}. Languages which cue prominence by marking the head of a constituent are known as \textit{head prominence} languages. Languages which cue prominence by marking both the head and the edge of a prosodic constituent are known as \textit{head/edge prominence} languages. Finally, languages that lack both lexical and postlexical heads and cue prominence by marking the edge of a constituent are known as \textit{edge prominence} languages.} By contrast, edge tones not only associate with initial or final edges of smaller or larger constituents, but it is claimed that they are mainly used for phrasing and juncture-cueing. Thus, it appears that the association properties of tones (accent/boundary) are pre-packaged with distinct functions, as reported in \citet{grice_autosegmental-metrical_2022}: pitch accents cue prominence; boundary tones cue phrasing. Nonetheless, a growing number of research on both speech production and perception challenge this pre-defined relation between association and functional properties of tones postulated in AM models. For example, as discussed in \citet[][]{grice_autosegmental-metrical_2022, grice_commentary_2022} tones in Tashlhiyt Berber have no strict association properties, neither to heads nor to edges of constituents. Nevertheless, these tones appear to have a prominence-cueing function, since they are associated with words in focus. The loose association properties of tones in Tashlhiyt Berber implies that they are not pitch accents, given that they do not align with a prosodic head, although they do serve the highlighting function attributed to pitch accents \citep[for more on Tashlhiyt Berber, see][]{bruggeman_question_2017, bruggeman_lexical_2020}. Further, \citet{grice_autosegmental-metrical_2022, grice_commentary_2022} discusses the case of Maltese: whilst Maltese is described as an intonational language having regular pitch accents (in terms of AM), it commonly exhibits additional initial peaks on unstressed syllables. There is evidence that these initial peaks do not serve a demarcating function, although they are associated with the left edge of the word. It mostly appears that words in Maltese can have two types of prominence, that is, lexical stress prominence, and pitch prominence on the initial syllable. It seems to be the case, then, that tones in Maltese can associate with both edges and heads of words, but that their association properties do not necessarily lead to the assumed AM functions of demarcating or prominence-cueing \citep[for more on Maltese initial peaks, see][]{vella_prosodic_1995, vella_alignment_2011, grice_stress_2019, lialiou_periodic_2021, lialiou_word-level_2023}.

This evidence calls into question the link between the association (head/edge) and functional properties (prominence/phrasing) of tonal events postulated by AM, and highlights the need for a dissociation of the two. Such a dissociation will allow for more flexibility in describing diverse phenomena. This is necessary not only for ``new" or under-studied languages but also for (re)describing phenomena in already well-studied languages. \sectref{sec:3.3}, reviewing work on well-studied head prominence languages, shows that edge tones can, in fact, also affect prominence perception. Prior to that, \sectref{sec:3.2} discusses the link between intonational rises and prominence.

\section{Intonational rises as a prominence-cueing device}
\label{sec:3.2}

As already mentioned at the beginning of this chapter, intonation serves multiple functions, from helping to locate fixed attributes of words~-- such as lexical stress in West Germanic languages~-- to prominence marking at the utterance level \citep[for a comprehensive review, see][]{grice_linguistic_2023}. Speakers may use a wide assortment of linguistic means for signalling prominence, including prosodic and non-prosodic factors \citep[e.g.,][]{baumann_what_2018}. Nonetheless intonation is a pivotal cue to prominence in spoken communication. Specifically, speakers make use of intonation to highlight important parts of their message, endeavouring to orient listeners' attention to specific information. At the same time, intonation is crucial during speech processing, since prominent intonational events attract and allocate listeners' attention to the constituents bearing them, facilitating their processing. The encoding and decoding of prosodic prominence relates to discrete phonological choices, continuous phonetic modulations reflected in the speech signal, and to meaning derived from the context (see \chapref{ch:1}, and \sectref{sec:3.4} in the current chapter).

Whereas the specifics of prominence marking using intonation depend on a language's prosodic system \citep[e.g.,][]{rosenberg_cross-language_2012, smith_dialectal_2020}, cross-linguistic research has revealed f0 to be one of the primary cues \citep[e.g.,][]{liberman_intonational_1975, pierrehumbert_phonology_1980, ladd_intonational_2008, beckman_stress_2012}. Specifically, in West Germanic languages, modulations in f0 direction, excursion, scaling, and timing are some of the acoustic dimensions found to be indicative of different degrees of perceived prominence, such that f0 peaks are generally perceived as more prominent than f0 valleys \citep[e.g.,][]{rietveld_relation_1985, gussenhoven_perceptual_1997, ladd_perception_1997}. What is more, the size of the excursion has been found to be a decisive cue to prominence: the steeper the f0 movement, the more prominent the word associated with it \citep[e.g.,][]{hart_perceptual_1990, gussenhoven_fundamental_1988, gussenhoven_phonology_2004}. Further, it has been shown that the general shape of the f0 contour (e.g., the steepness of a rise or fall as well as its close [early/late] alignment with a stressed syllable) is pivotal to prominence \citep[e.g.,][]{kohler_perception_1991, niebuhr_f0-based_2009, knight_shape_2008}.
\newpage
Before discussing the phonological expressions of prominence marking, it is important to note that although f0 modulations are central to prominence marking postlexically, modulations of other phonetic exponents are also relevant. Perceived prominence has also been found to correlate with enhanced intensity, increased duration, enhanced spectral characteristics, articulatory effort, and, in more recent work, enhanced periodic energy \citep[e.g.,][]{beckman_articulatory_1994, cole_signal-based_2010, baumann_what_2018, roessig_modeling_2019, albert_model_2023}. It appears therefore that a multiplicity of cues interact with each other in complex constellations to express postlexical prominence. Most importantly, research on individual variability has shown that speakers encode and decode prominence relations using the aforementioned features in different combinations and with different degrees of strength \citep[e.g.,][]{baumann_what_2018, lorenzen_redundancy_2023, lorenzen_paradigmatic_2024}.

With respect to the phonological choices of postlexical prominence marking, in languages like German (a West Germanic \textit{head} prominence language according to \citeauthor{jun_prosodic_2014}'s typology), prominence is encoded (and decoded) through the use of pitch accents. That is, a highlighting function is attributed to tonal movements associated with the stressed syllable (head; for more, see \sectref{sec:3.1}). As mentioned in the previous section, the identity of a pitch accent (and any other postlexical tone) is defined by its phonetic substance. Specifically, the realisation of pitch accents in the acoustic record results in different pitch accent types. Accent types are defined on the basis of the aforementioned f0 dimensions; f0 direction (rise/ fall), f0 scaling (steep/shallow), f0 height (peak/valley), and f0 alignment (timing of f0 peak or valley relative to a stressed syllable). Research on prominence perception in West Germanic languages has shown that the type of pitch accent impacts the degree of perceived prominence.\footnote{Pitch accent placement within an utterance (prenuclear, nuclear, postnuclear) is also an important factor in prominence perception. Many schools of intonational analysis have claimed that the last accent in an utterance, the \textit{nuclear accent}, is the most prominent one. For German, the following prominence hierarchy of pitch accent placement has been proposed: nuclear > prenuclear > postnuclear \citep[e.g.,][]{baumann_perceptual_2015, grice_integrating_2017}. Pitch accent placement is beyond the scope of this book, and will therefore not be discussed further.} For instance, studies on Dutch \citep[e.g.,][]{rietveld_relation_1985, gussenhoven_fundamental_1988}, English \citep[e.g.,][]{ladd_perception_1997, knight_shape_2008, cole_sound_2019}, and German \citep[e.g.,][]{kohler_perception_1991, niebuhr_f0-based_2009, baumann_perceptual_2015, baumann_what_2018} have found that the overall shape of an accent/f0 contour affects prominence perception, revealing the importance of intonational rises, as higher f0 peaks and steeper f0 rises are perceived as more prominent.

Let us now zoom in further on intonational rises and their prominence status in German. \citet{baumann_perceptual_2015}, in a prominence rating task, investigated the degree of accent type prominence using the accent inventory of the German Tones and Break Indices model \citep[GToBI; see \tabref{tab:3.1};][]{grice_german_2005}. The items of the study consisted of utterances featuring proper names. The authors manipulated the pitch accent contour on the proper names, and instructed the listeners to rate how prominent a given proper name sounded using a visual analogue scale. The results of this study show that the accent type on a given proper name affected its perceived prominence: when proper names were produced with pitch rises, they were perceived as more prominent than when they were produced with pitch falls (f0 direction). Crucially, the rise or fall occurred on the stressed syllable of the proper name. In a similar fashion, steeper rises and falls on proper names were rated as more prominent than shallow rises and falls (f0 excursion), and finally, pitch peaks on proper names were judged as more prominent than pitch valleys (f0 height). \tabref{tab:3.1} presents the perceived prominence hierarchy of the tested German accent types proposed by \citet{baumann_perceptual_2015}. 

\begin{table}
	\caption{GToBI accent type prominence hierarchy based on the three dimensions: pitch direction, excursion, and value of starred tone. Increasing prominence is ranked from bottom to top. Table reproduced and slightly adapted from \citet[][2]{baumann_perceptual_2015}.}
	\label{tab:3.1}
	\begin{tabular}{c c c c} 
		\lsptoprule
		Accent type & Pitch direction & Pitch excursion & Value of starred tone\\ %table header
		\midrule
		L+H* & rise & steep & H\\
		L*+H & rise & steep & L\\
		H* & rise & shallow & H\\
		!H* & rise & shallow & !H\\
	 	H+!H* & fall & (relatively) steep & !H\\
		H+L* & fall & steep & L\\
		L* & level & shallow & L\\
		Ø & n/a & n/a & n/a\\
		\lspbottomrule
	\end{tabular}
\end{table}

In a more recent study, \citet{baumann_what_2018}, using a Rapid Prosody Transcription (RPT) task, tested the impact of continuous prosodic (e.g., f0, duration, intensity), discrete prosodic (pitch accent type and placement), as well as non-prosodic (e.g., lexical, syntactic, semantic) variables on prominence perception. Specifically, in this RPT task, untrained German listeners were asked to underline words regarded as prominent during or right after listening to an utterance. The results show that discrete prosodic variables, including pitch accent type, were the best predictors of prominence. More specifically, the particular shape characteristics of the pitch contour, with rising contours being pivotal, were found to be the primary cue to prominence perception. This study also highlighted a new~-- at the time~-- perspective related to prominence perception. In particular, the authors mention that despite the overall trend found at the group level, prominence perception in their sample was partly determined by individual differences, with some of the listeners attending mostly to pitch-driven cues, and others exhibiting a reverse pattern in attending mostly to lexical and semantic cues. 

So far, it has been shown for West Germanic languages, including German, that discrete (phonological) and continuous (phonetic) prosodic cues pattern together in flagging postlexical prominence, reflecting the fact that the shape of a pitch contour is defined by its f0 dimensions. Summarising, 

\begin{enumerate}[label=\roman*.]
	\item rises are more prominent than falls,
	\item steep rises are more prominent than shallow rises and falls, and
	\item peaks are more prominent than flat pitch and/or valleys.
\end{enumerate}
The results discussed in this section strongly suggest that intonational rises constitute a decisive cue in prominence perception. The next section elaborates further on the linguistic importance of intonational rises in speech processing.

\section{Intonational rises in speech processing}
\label{sec:3.3}

It was pointed out in \sectref{sec:2.4} that in auditory processing, the physical properties of the acoustic signal are essential for attracting attention. For instance, rises in the amplitude or pitch of sine waves prompt auditory looming effects, indicating that the sound source is approaching. Acoustic saliency thus serves as a warning cue that activates attentional resources \citep[e.g.,][]{bach_rising_2008}. Falling acoustic signals may also attract attention, but it has been shown that these are experienced mostly as fading (indicating a receding sound source), and are thus processed differently than rises \citep[e.g.,][]{macdonald_effects_2011}.

In a similar vein, related research has shown that prosodic cues, such as intonational rises, are pivotal in speech perception and processing. In West Germanic languages, rising pitch accents typically mark important and informative constituents in an utterance, such as new topics, new referents, or different types of focus \citep[especially contrastive or corrective focus; e.g.,][]{bolinger_intonation_1986, bolinger_intonation_1989, baumann_-accentuation_2012, grice_integrating_2017, lorenzen_information_2022, lorenzen_paradigmatic_2024}. Rising intonation allocates attention towards the constituents bearing it, which facilitates their processing. Further, rising pitch accents, when used as a focus-cueing device, also guide attention towards semantic incongruencies, because they lead to more elaborate processing of the focused information \citep[e.g.,][]{ventura_attention_2020, wang_influence_2011}. Moreover, intonational rises guide attention in memory and recall tasks, facilitating recall performance \citep[e.g.,][]{fraundorf_recognition_2010, koch_contrastive_2021, rohr_effect_2022}. Last but not least, information highlighted by rising intonation is perceived as more prominent, which leads to more attentional resources being allocated to information highlighted in this way than when it is highlighted by falling intonation \citep[e.g.,][]{sullivan_effects_1983, hsu_brain_2015, rohr_perceptual_2020, rohr_signal-driven_2021}. Nonetheless, not all rising pitch accents attract attention. For instance, Bari Italian postfocal rising pitch accents, which are used to mark interrogatives, do not attract attention \citep[i.e., postfocal rising pitch accents do not facilitate detecting incongruencies in the postfocal domain;][]{ventura_attention_2020, ventura_production_2020}. It thus appears essential to take the structure of the language under investigation into account. In West Germanic languages, prominence is very often encoded in the form of intonational rises. Therefore, the use of rising accents to mark important constituents in such languages may be the reason for their attention orienting potential. 

Let us now focus on the triplet intonation-attention-speech processing. It appears that the auditory and linguistic processing systems share many common features, especially with regard to prediction and attention orienting. These two mechanisms appear to be driving forces for information processing in general, including for linguistic input \citep[see][]{bornkessel-schlesewsky_towards_2016}. As laid out in \sectref{sec:2.3}, the auditory system develops sensory-memory representations related to the auditory environment, on the basis of which it predicts the properties of the upcoming sound events. When predictions are not attested, attention-calling mechanisms are activated. Likewise, according to \citet{schumacher_backward-_2015} the linguistic processing system constantly develops mental representations from which predictions for the upcoming entities are derived \citep[for more detail on the general architecture of the prediction-updating cycle, see][]{rohr_signal-driven_2021, repp_what_2023}. As already mentioned in \chapref{ch:2}, according to \citet{friston_free-energy_2010} our brains build an internal model of the world on which we rely to predict upcoming sensory events. When these predictions are violated, prediction errors~-- to use \citeauthor{friston_free-energy_2010}'s term~-- arise. Prediction errors in language, namely ``mismatches between context and target items" \citep[][36]{rohr_signal-driven_2021}, have been found to elicit detection and attention-calling processes as well. In the previous chapter, we saw that after an auditory deviation/mismatch, the first attention-calling mechanism that is activated in the auditory cortex is the mismatch negativity (MMN) response, oftentimes called the regularity-violation response \citep[e.g.,][]{naatanen_attention_1992}. Mismatches in language function in a similar way, eliciting negative neurophysiological deflections related to such violations \citep[for a general account on negativities as reflexes of prediction errors, see][]{bornkessel-schlesewsky_toward_2019}. The most well-studied prediction error deflection in language is the N400. The N400 ``represents processes that affect an early discourse linking stage and derive from expectations for an entity raised by the discourse context. The higher the processing cost for an entity or the more difficult its reconciliation with prior discourse, the more enhanced is the N400 amplitude" \citep[][36]{rohr_signal-driven_2021}. \citet{rohr_signal-driven_2021} point towards ERP studies on pitch/intonation processing, reporting that as pitch unfolds in time, listeners incrementally develop expectations for the upcoming entities and their prosodic realisation. These studies report that when a mismatch between the prosodic input and the upcoming information occurs, an N400, indicative of a prediction-violation (for more, see \chapref{ch:2}), is elicited in the brain \citep[e.g.,][]{toepel_catching_2007, baumann_-accentuation_2012}.
\largerpage
As I reviewed in the previous chapter, attention orienting towards an auditory mismatch in cognition can also elicit a positive deflection~-- identified as part of the P3 family~-- which involves a conscious perception of this mismatch. Likewise, \citet{rohr_signal-driven_2021} investigated the signal-driven effects of prosodic prominence in an EEG study and found that intonational rises, by virtue of being acoustically prominent, attracted more attention than less prominent intonational patterns, eliciting a P3 response (for a detailed review of this study, see \sectref{sec:3.4}). Further, \citet[][]{rohr_signal-driven_2021} mention that attention orienting in language can also be reflected in a Late Positivity, also identified within the P3 family \citep[e.g.,][see also \chapref{ch:2}]{ruchkin_multiple_1990}. \citet{rohr_signal-driven_2021} propose that Late Positivities in language reflect an update of the mental model. More specifically, according to \citet{rohr_signal-driven_2021}, listeners, after orienting their attention towards a prosodic mismatch, topic shift, or novel information, need to incorporate the new knowledge in their mental representations by updating their mental model. Late Positivity has also been considered as an index of attention orienting in studies investigating prosodic cues marking new information and focused constituents \citep[e.g.,][]{toepel_catching_2007}, as well as unexpected accentuation \citep[e.g.,][]{schumacher_pitch_2010, baumann_-accentuation_2012}, redundant accentuation on constituents in the background \citep[e.g.,][]{li_process_2018}, and deviant or unexpected prosodic patterns \citep[e.g.,][]{kakouros_making_2018}. According to \citet{rohr_signal-driven_2021}, these are cases where a listener perceives a conflict between prosodic and information structural cues. This conflict can be resolved through the updating processes, whereby the mental model is repaired and reorganised \citep[see also,][]{toepel_catching_2007, dimitrova_less_2012, brouwer_time_2013, schumacher_backward-_2015}.

Evidence supporting the privileged position of pitch rises in attracting attention also comes from studies on the neural processing of linguistically meaningful pitch variations at both the lexical and postlexical level in Mandarin Chinese \citep[e.g.,][]{ren_early_2009, tsang_erp_2011, liu_online_2016, li_unattended_2018}. These studies show that brain responses are not only activated by the acoustic contrasts in the signal, but they are also sensitive to the timing of the acoustic cues. The study by \citet{li_unattended_2018} is of particular relevance for the current work, as it investigated Tone2 (that is, a rising tone) and Tone4 (that is, a falling tone), similarly to Chapter's \ref{ch:4} intonational rising and falling contours. Using the MMN paradigm (for more on this paradigm, see \chapref{ch:2}), the authors compared the timing of the cues by contrasting Tone2/Tone4 with Tone3 as a reference tone, and they found that MMN is sensitive to the timing of the acoustic cues (i.e., the time point when the tone contours start to diverge; cue of divergence point). It is particularly interesting that the MMN time window they found for the contours with the early divergence cue is the same time window observed in the current study reported in \chapref{ch:4}. The contours used in the study reported in \chapref{ch:4} also have an early divergence point (i.e., divergence from the beginning of the word), pointing at a cross-linguistic similarity.  

\largerpage
As discussed in \sectref{sec:3.1}, AM theory and prosodic typology postulate that tonal events come pre-packaged with specific associations and functional properties (in the case of English, German, and other languages with a similar prosodic structure). More specifically, pitch accents are assumed to cue prominence, while edge tones serve to chunk utterances into smaller units \citep[e.g.,][]{ladd_intonational_2008, jun_prosodic_2014}. It is therefore proposed that accentual rises constitute a better cue in directing listener attention than rises at prosodic boundaries. Nevertheless, studies investigating prominence perception and processing call into question the strict dichotomy proposed by the AM theory. Specifically, it has been claimed that prosodic phrasing actually can affect prominence perception \citep[for discussion, see][]{grice_prosodic_2021}. Some evidence that rising edge tones can also cue prominence comes from serial recall tasks of nine-digit sequences in Italian \citep{savino_intonation_2020} and German \citep{rohr_effect_2022, grice_rises_2024}. These studies found that boundary rises marking the last item of non-final triplets facilitated the recall accuracy of the digit in the boundary position that bears the edge tone. This facilitation was also observed over the whole triplet. These results reveal that edge tone rises appear to cue prominence on the whole domain that is delimited. Other evidence for edge tones cueing prominence comes from processing studies on the domain of the word \citep[for discussion, see][]{grice_prosodic_2021}, showing that rising boundary tones facilitate, for example, word segmentation \citep[e.g.,][]{ou_language-specific_2021} and word recognition \citep[e.g.,][]{kember_processing_2021}.

Therefore, regardless of the functions with which pitch accents and boundary tones are pre-defined within AM theory, the prominence patterns in a language can be affected or modulated by different structural positions \citep[for the notion of structural prominence, see][]{streefkerk_prominence_2002, himmelmann_prominence_2015, cangemi_integrating_2020}. Put differently, AM postulates that prominence in West Germanic languages is mostly expressed in the form of pitch accents, directing the listener's attention to the highlighted information. The stressed syllable, as the docking site for the pitch accent, is the head of the word, and therefore occupies an essential position in the linguistic structure. Yet, prosodic boundaries appear to also guide attention towards important elements. This could be so because flagging prominent information at privileged positions, such as in the beginning or at the end of an utterance, is crucial for speech processing and planning \citep[e.g.,][]{seidl_infant_2006, ou_language-specific_2021}. Hence, one could argue that in a complex signal such as speech, prominence might not necessarily be encoded by cues on one specific structural position, but rather by the combination of cues on different privileged positions. Both experimental studies reported in this book (Chapters \ref{ch:4} and \ref{ch:5}) investigate the role of the phonological status of the rise, namely as pitch accent or edge tone, in attention orienting. The key, and novel, discovery is that rises associated with the edges of constituents can also direct the listener's attention towards the constituents bearing them; the evidence comes from ERP as well as pupillometric data. 

Although linguistic research has attested that rising pitch is pivotal in spoken communication, previous research on the processing of unexpected/deviant auditory events has focused on pure tones only. To my knowledge, the only oddball study that has explored the sensitivity of pitch change direction in relation to speech is \citet{hsu_brain_2015}. However, although some of \citeauthor{hsu_brain_2015}’s stimuli were produced with a human voice, they were not designed to convey any linguistic meaning (see the review of this study in \sectref{sec:2.4}). Chapters \ref{ch:4} and \ref{ch:5} delve into the role of intonational rises in attention orienting, using speech stimuli that convey linguistic meaning. Specifically, in extending previous neurophysiological work on simple sine waves, the study reported in \chapref{ch:4} introduces rising and falling intonation (attributable to both accents and boundary tones) on sequentially presented lexical items, yielding a list context and the concordant, language-specific expectations. Likewise, \chapref{ch:5} uses domain-final rises and falls (i.e., edge tones) within a numerical list context. Crucially, the rise or fall for the stimuli used in the \citeauthor{hsu_brain_2015} paper involved a change in pitch from one stimulus to the next (i.e., across stimuli), whereas in the current work the rise or fall takes place within the stimulus itself.

\section{Signal- and context-driven processing}
\label{sec:3.4}

The processing of information is highly affected by listener expectations \citep[e.g.,][]{clark_are_2013, huettig_four_2015, grice_integrating_2017, friston_does_2018, roessig_dynamics_2019}. These expectations are in part driven by pure acoustic properties, i.e., the prosodic context, but can also be driven by context- and/or language-specific expectations, i.e., expectations raised by the pragmatic context or by language structure. For example, a particular accent on a constituent or an inserted pause can create expectations as to the upcoming information. 

Previous research on acoustic mismatch detection has targeted signal-based expectations, but context may also shape attentional orientation. For instance, \citet{rohr_signal-driven_2021} investigated the role of signal- and expectation-driven effects of prosodic prominence in two EEG studies. First, four different German accentual contours were tested in isolated sentences (steep rise, shallow rise, [steep] fall, and no accent) with intonation the only factor influencing attention orienting. Second, the most prominent steep rising accent and the less prominent falling accent were tested with regard to expectations of how exciting/unusual the content of an utterance is, by relating the stimuli to an exciting/unusual and neutral (negligible/ordinary) pre-context. Results indicate that attentional cues, both signal-driven and context-driven, engender positivities of varying latency: (i) a prominent rise on the stressed syllable (and not a rise elsewhere) consumes attentional resources at an early processing stage (reflected in an Early Positivity), and (ii) highlighting induced by the context (i.e., the exciting context) consumes attentional resources at a later processing stage (reflected in a Late Positivity). Moreover, results show that prior context builds up expectations for the upcoming prosodic input, reflected in N400 prediction errors engendered by acoustically unexpected (here, prominent) accents as well as contextually inappropriate prosodic realisations. These studies hence suggest that attentional orientation and predictive processing reflect discrete stages in the construction of a mental representation during real-time comprehension. Another example of the interaction between signal- and context-driven processing of pitch variation is the study by \citet{liu_online_2016}. Liu and colleagues investigated the attentive processing of tone and intonation in Mandarin Chinese and found that conscious attentional neural responses (i.e., the P300 response) were modulated by the pragmatic context (question vs. statement) in which Tone4 appeared.

Following up on \citet{rohr_perceptual_2020, rohr_signal-driven_2021}, a recent study by \citet{rohr_influence_2022} on German investigated the effect of context-driven expectations on tonal cues to prominence. Specifically, the authors tested contexts where the speaker needs to mark information in an exciting way (exciting context) as opposed to contexts where the speaker provides information in an ordinary way (neutral context). For the former context, the authors found that speakers made the target information prominent, by using mostly rising accents characterised by large rising tonal onglides (measuring the pitch excursion between the pre-stressed syllable and the accentual peak). In contrast, speakers marked ordinary information with less prominent accents, mostly by using small rising or falling onglides. Listeners interpreted large rising onglides (as in L+H*) as unusual/exciting/prominent information, as opposed to small rising onglides (as in H*) and falling onglides (as in H+L*). This study provides further evidence not only for the intrinsic association between rising intonation and prominence, but also for the interplay between signal-driven and context-driven cues.

In addition, studies have shown that the processing and perception of prosodic prominence can also be shaped by language-specific expectations, such that attention is allocated to different information based on a language's prosodic structure \citep[e.g.,][]{chandrasekaran_sensory_2009, ventura_attention_2020}. For example, the degree of prosodic attenuation in postfocal material in Italian is different than in German. In parallel collaborative work with \citet{grice_perception_2024}, we conducted a web-based experiment investigating how native Italian speakers and German learners of Italian perceive the prominence of words with different focus structures (i.e., in broad and narrow focus, and postfocally) in Italian. In this study, we were particularly interested in the question of how listeners' native language shapes the perception of postfocal prominence. The results show that both groups of listeners perceived words in narrow focus as more prominent than words in broad focus, but that they differed in how they perceived postfocal words. Specifically, native Italian listeners perceived words in both broad focus and postfocal position equally prominent, despite the differences in the acoustic realisation of the target words in the two focus conditions. In contrast, the German learners perceived postfocal words as less prominent than words in broad focus. These findings indicate that native Italian speakers used both signal- and language-driven expectations in their perception of postfocal prominence, whereas the prominence perception of German learners appeared to be driven only by the signal, i.e., the acoustic saliency of the broad focus words compared to the reduced acoustic properties of the postfocal words. Both \citet{grice_perception_2024} and \citet{ventura_attention_2020} show that flat pitch appears to have a different status in Italian and German, as in Italian attention to postfocal material is enhanced by flat pitch. Therefore, German learners in \citet{grice_perception_2024} appeared to also use their L1 expectations regarding the deaccentuation of postfocal elements.

In sum, it appears that prior expectations driven by the linguistic or discourse context, or even by recent speech experience, can overwrite the typical acoustic cues to prominence \citep[e.g.,][]{cole_signal-based_2010, bishop_information_2013, kakouros_making_2018, roettger_listeners_2020}, leading to different processes: whilst signal-driven cues are decisive mainly for pre-attentive processes, context-driven cues mainly influence attentive processes. This point is revisited and discussed in depth in \chapref{ch:4}.

\section{A brief note on German list intonation}
\label{sec:3.5}

The previous sections of this chapter established the relevance of intonation for prominence. It is worth keeping in mind, however, that intonation serves multiple other functions as well. Crucially for the present work, intonation can differentiate between continuity and finality in utterances and~-- most importantly for this work~-- in lists. Thus, before closing this chapter, let us touch upon German list intonation, the linguistic context of the two experimental studies reported in this work. 

\begin{sloppypar}
German list intonation involves different pitch contours depending on whether the list items are non-final or final. More specifically, rising and level\footnote{With level contours, I refer to contours that are flat or shallow falling towards the boundary. The pre-boundary tonal specification can vary from rise-fall to shallow falling or shallow high.} contours are typically used by native speakers to recite elements of a ``multi-part unit such as a list..." \citep[][98]{peters_phonological_2018}, as rising and/or level contours on non-final and penultimate items in lists (or in utterances in general) denote continuity. In the context of a list, these contours indicate that the list is open or incomplete \citep[e.g.,][]{grabe_comparative_1998, baumann_prosody_2001, chen_language_2003, peters_phonological_2018}. The final element in a (closed) list, by contrast, is typically marked by a falling contour, indicating finality and marking the end of the list \citep[see][]{baumann_prosody_2001, peters_phonological_2018}. Nonetheless, final rises (the reflex of H\% boundary tones) can also mark the end of units, such as a sublist, as there are different \textit{types} of continuation. Continuation can be signalled both across smaller or larger domains \citep[e.g.,][]{chen_language_2003}. In discourse, this translates to sentence-internal continuation and continuation at clause-boundaries \citep[e.g.,][]{chen_language_2003}, while in the case of a list, this corresponds to list-internal (i.e., non-final items) and list-boundary (i.e., end of a sublist where a subsequent list or another utterance is yet to come) continuation. 
\end{sloppypar}

\figref{fig:3.2intonationlist} presents stylised examples of some possible intonational contours marking non-final and final items in a numeric list. In blue, we see instantiations of different rising (first and second row), high (third row), and low level (final row) contours that can mark non-final items in a list, indicating list-internal continuity. In red, we see contours that can mark final items in a list. More specifically, whilst in the first, second, and last row, we see instantiations of different falling contours denoting the end of a closed list, in the the third row, we see, in pink, an example of a final rise (i.e., a boundary rising contour), marking the end of a list and indicating list-boundary continuation. 

\begin{figure}
	\includegraphics[width=0.8\textwidth]{figures/ListIntonation.png}
	\caption[Illustrative examples of list intonation in German]{Illustrative examples of list intonation in German on non-final and final elements in lists using stylised contours \citep[adapted from examples (14), (15), and (16) in][98--99]{peters_phonological_2018}.}
	\label{fig:3.2intonationlist}
\end{figure}

\section{Summary}
\label{sec:3.6}

The current chapter served as a condensed introduction to prosody, intonation and prosodic structure. Central for this chapter has been the relevance of intonation for prominence. The studies reviewed and discussed show that rising intonation not only constitutes a decisive cue to prominence, but that it also plays a crucial role in speech processing. However, we saw that, in addition to signal-driven cues (like intonational rises) context-driven (i.e., meaning- or language-specific) cues are also of great importance in speech processing, as they can overwrite the importance of the signal at times, which in turn can lead to the activation of different processes. Here we note an interesting parallel between speech and auditory processing. As we saw in \chapref{ch:2}, saliency, both signal-driven and meaning-based, is the mainspring for attention orienting in auditory cognition. In a similar vein, this chapter showed that prominence, both signal-driven and context-driven is a driving force for speech processing. As the introductory chapter put forward, prominence pertains to the concept of attention (introduced in \chapref{ch:2}). We saw in this chapter that the linguistic domain uses a pool of mechanisms shared with general cognition, mechanisms which are crucial for the orienting response in both auditory and speech processing. I will revisit this argument in the upcoming chapters. More specifically, the experimental work presented in Chapters \ref{ch:4} and \ref{ch:5} builds on both neurocognitive and linguistic research in an endeavour to unravel the relevance of intonation as a prominence-cueing device for attention orienting in spoken language.
