\addchap{Prefácio}
\begin{refsection}
\title{Linguística Clínica:


Modelos, Avaliação e Intervenção} 
\author{Maria João freitas \and Marisa Lousada \and Dina Caetano Alves}  

\maketitle
\section*{Apresentação do volume}
A área de trabalho que designamos como Linguística Clínica reúne profissionais, investigadores e estudantes de diferentes graus académicos cujo foco de trabalho é a exploração da ponte entre a Linguística e a Fonoaudiologia (na tradição brasileira) ou a Terapia da Fala (na tradição portuguesa). Tem como objetivos centrais a construção de conhecimento sobre a natureza dos desempenhos linguísticos atípicos e a preocupação constante em tornar os processos de avaliação e de intervenção em contexto clínico cada vez mais rigorosos e eficazes. Nas últimas décadas, terapeutas da fala / fonoaudiólogos e linguistas têm colaborado em equipas multidisciplinares, que envolvem, também, psicólogos, professores e educadores de infância, no sentido de conjugarem teoria e prática na construção de instrumentos de avaliação e de programas de intervenção robustos e eficientes. Múltiplos modelos linguísticos têm sido usados em várias comunidades responsáveis por investigação e por formação de especialistas neste domínio, em vários pontos do mundo. Destacamos as equipas coordenadas por Barbara Bernhardt e Joseph Stemberger (University of British Columbia, Canadá), Daniel Dinnsen e Judith Gierut (Indiana University, EUA), Jane Marshall (University of London, UK), Naama Friedmann (Tel-Aviv University, Israel) ou Susan Ebbels (University College London, UK), entre muitos outros.

Os estudos seminais desenvolvidos para o português na área da Linguística Clínica surgem no Brasil e remontam aos anos 80 do século passado, seguindo as tendências internacionais à época: \citet{Lamprecht1986} e \citet{Yavas1988} aplicam à fonoaudiologia os princípios da Fonologia Generativa Natural; \citet{MatzenauerHernandorena1988} fá-lo através do Modelo de Traços Distintivos da Fonologia Generativa Clássica. Estes autores inauguraram, no sul do Brasil, uma linha de trabalho que tem vindo a explorar, desde então, a eficácia dos modelos linguísticos na avaliação e na intervenção clínicas e que envolve muitos fonoaudiólogos e linguistas brasileiros, alguns deles autores neste volume. Herdeiros desta tradição, vários terapeutas da fala e linguistas portugueses reu\-niram-se, mais tarde, em equipas sedeadas em vários centros de investigação brasileiros e portugueses, no sentido de observar especificidades do desenvolvimento linguístico atípico em português e de construir ferramentas capazes de auxiliar os terapeutas da fala na sua prática clínica, em diálogo constante com os seus pares de renome internacional na área da Linguística Clínica.

O volume que agora publicamos na coleção \emph{Textbooks in Language Science} da editora \emph{Language Science Press} foi programado tendo como base a área disciplinar que acima apresentámos. Destina-se a quem estuda ou investiga aspetos relacionados com desempenhos linguísticos atípicos, decorrentes de diferentes quadros clínicos, em países lusófonos. Os vários capítulos visam fornecer informação sobre o que tem sido feito em Linguística Clínica no Brasil e em Portugal, contribuindo para uma reflexão sobre o impacto da Linguística formal na investigação, na formação e na prática de terapeutas da fala / fonoaudiólogos, e inspirando linhas de investigação futuras nesta área do conhecimento humano. Esta publicação tem, como público alvo, estudantes e docentes de cursos de graduação e pós-graduação nas áreas da Terapia da Fala/ Fonoaudiologia, da Linguística, da Psicologia e da Educação. Reúne autores brasileiros e portugueses de várias universidades de referência nas áreas da Linguística e da Terapia da Fala / Fonoaudiologia, abordando vários domínios da Linguística. 

Os capítulos que integram o presente volume estão organizados em quatro blocos temáticos, de que a seguir damos conta. 

No bloco inicial, encontram-se dois capítulos que têm como objetivo central fornecer uma perspetiva histórica dos estudos em Fonologia Clínica e em Sintaxe Clínica. O capítulo 1, de Sharynne McLeod, Sarah Verdon e Nicole McGill, faz uma revisão da definição do conceito de \emph{Speech Sound Disorders} (Perturbações dos Sons da Fala (PSF)), seus tipos e causas, com referência à sua distribuição na população mundial e ao seu impacto no quotidiano da criança. Evoca modelos fonológicos usados no trabalho com crianças com PSF, fazendo referência ao diagnóstico em situação de bilinguismo e terminando com considerações sobre avaliação e intervenção no domínio das PSF. O capítulo 2 é da responsabilidade de João Costa, sendo dedicado à Sintaxe Clínica. Caracteriza-se o conhecimento sintático, sendo dada ênfase à necessidade do conhecimento aprofundado da teoria sintática para um trabalho mais especializado com indivíduos com perturbações de linguagem, desde a avaliação à intervenção terapêutica. 

O segundo bloco de capítulos centra-se em aspetos fonológicos e sua relação com a dimensão fonética da língua. O capítulo 3, de Haydée Wertzner, é dedicado à fundamentação e apresentação da metodologia de trabalho de avaliação das PSF adotada no Laboratório de Investigação Fonoaudiológica (LIFFON), da Faculdade de Medicina da Universidade de São Paulo, com grande impacto na comunidade de fonoaudiólogos brasileiros e acompanhada de perto pela comunidade de terapeutas da fala portugueses. Nele é referido um conjunto de aspetos cruciais de natureza metodológica para a avaliação clínica. Conceitos fonológicos referidos neste capítulo 3 são retomados nos capítulos seguintes, cuja sequência evoca a organização hierárquica dos constituintes propostos no âmbito da Fonologia não linear. Assim, o capítulo 4 e o capítulo 5 centram-se na avaliação e na intervenção, respetivamente, de aspetos segmentais da estrutura sonora. De seguida, o capítulo 6, o capítulo 7 e o capítulo 8 focam-se na avaliação e intervenção de aspetos prosódicos da estrutura sonora.

Numa perspetiva segmental, o capítulo 4, de Cristiane Lazzarotto-Volcão, Aline Oliveira e Carmen Matzenauer, revê os princípios gerais dos modelos da Fonologia Natural, da Fonologia Autossegmental e da Fonologia Gestual, dando ênfase à Teoria de Traços Distintivos e à sua operacionalização em contexto de avaliação, nomeadamente através do Modelo Padrão de Aquisição de Contrastes. O capítulo 5, de Helena Bolli Mota, Fernanda Wiethan e Vanessa Giacchini, define intervenção e refere aspetos de natureza metodológica a ela subjacentes, revendo posteriormente várias teorias fonológicas (Fonologia Natural, Teoria de Traços Distintivos, Teoria Autossegmental, Geometria de Traços, Teoria da Marcação e das Leis Implicacionais) já testadas e com eficácia comprovada clinicamente, mostrando que as mesmas subjazem a diversas abordagens terapêuticas, apresentadas ao longo do capítulo, para o tratamento da PSF de base fonológica. 

Numa perspetiva prosódica, o capítulo 6, de Ana Margarida Ramalho, Susana Rodrigues e Maria João Freitas, centra-se na definição e operacionalização de unidades da Fonologia não linear na avaliação do desenvolvimento fonológico infantil. As autoras exploram as relações entre os segmentos e os constituintes prosódicos \emph{sílaba}, \emph{pé} e \emph{palavra prosódica}, apresentando-se dados da avaliação fonológica de crianças com perfis linguísticos típico e atípico. É ainda dado relevo à forma como devem ser recolhidos e transcritos os dados de fala que servirão de base à análise fonológica. O capítulo 7, de Marisa Lousada, Ana Margarida Ramalho e Tânia Reis, recruta os mesmo aspetos linguísticos focados no capítulo 6, desenvolvendo uma reflexão sobre a interface entre estrutura segmental e estrutura prosódica em contexto de intervenção terapêutica. Focando-se previamente em aspetos associados à promoção da eficácia na intervenção, identificam e definem objetivos terapêuticos e fornecem instruções sobre intervenção, sugerindo o uso de tarefas específicas de natureza (psico)linguística, revendo, por fim, estudos de intervenção junto de crianças com desenvolvimento atípico em idade pré-escolar. Por fim, o capítulo 8, de Sónia Frota, Marisa Filipe, Marisa Lousada, Maria Manuel Vidal e Marina Vigário, centra-se nos constituintes mais altos da hierarquia prosódica, que permitem representar fenómenos como a entoação, o fraseamento e a proeminência. Definem estes constituintes e explicitam as suas funções nas línguas naturais. As autoras fornecem uma visão do desenvolvimento prosódico, da palavra prosódica ao enunciado, fazendo uma revisão da sua observação em diferentes quadros clínicos e centrando-se em instrumentos de avaliação prosódica usados internacionalmemte nos últimos 40 anos. Apresentam instrumentos de avaliação prosódica desenhados para português europeu, tecendo, por fim, considerações sobre intervenção terapêutica no domínio da prosódia.

Ainda no bloco dedicado à estrutura sonora, o capítulo 9, de Inês Catarino e Letícia Almeida, centra-se na importância dos testes de repetição de pseudopala\-vras para o diagnóstico. As autoras demonstram a relevância deste tipo de tarefa na avaliação clínica, em particular, no caso da Perturbação do Desenvolvimento da Linguagem (PDL). Listam, em seguida, com base na Fonologia  não linear, as variáveis com impacto no processamento fonológico de pseudopalavras. Por fim, apresentam instrumentos de repetição de pseudopalavras atualmente disponíveis para o português europeu. No capítulo 10, Dina Caetano Alves dá conta dos modelos teóricos fonológicos aplicados à avaliação e intervenção em consciência fonológica. A autora chama a atenção para a importância da seleção cuidadosa de estímulos na avaliação e na intervenção, considerando a Fonologia não linear como quadro de referência. Na secção dedicada à intervenção em consciência fonológica, são sugeridos os modelos e níveis de intervenção nos contextos educacional e clínico. O capítulo 11, de Ana Catarina Baptista, Catarina Oliveira e Jorge Humberto Martins, encerra o segundo bloco temático, sobre a estrutura sonora, com uma reflexão sobre processamento auditivo na avaliação e na intervenção clínicas. Os autores definem processamento auditivo, fornecem detalhe sobre a fisiologia da audição, centrando-se depois na perturbação do processamento auditivo, etiologia e fatores de risco, destacando, para a população infantil, as otites médias. Em seguida, dão informação sobre rastreio e avaliação da perturbação do processamento auditivo, terminando com instruções para o planeamento da intervenção.

O terceiro bloco temático deste volume integra capítulos que exploram o módulo gramatical da sintaxe, a interface gramática-pragmática e o discurso. O capítulo 12, de Anabela Gonçalves e Sónia Vieira, centra-se na avaliação da sintaxe. As autoras apresentam os conceitos e as estruturas sintáticas mais relevantes para avaliação do conhecimento sintático infantil e os diferentes métodos de avaliação do conhecimento sintático. O capítulo expõe ainda dados do desenvolvimento atípico em crianças com PDL. No capítulo 13, da autoria de Ana Lúcia Santos e Alexandrina Martins, é salientada a relevância da subordinação para a avaliação da compreensão e da produção do desenvolvimento linguístico atípico, particularmente em crianças com PDL e em crianças com Perturbação do Espetro do Autismo. A última secção do capítulo apresenta ainda algumas orientações para a intervenção na área da sintaxe.  O capítulo 14 é da responsabilidade de Letícia Corrêa, Marina Augusto e Tatiana Bagetti, sendo dedicado à intervenção no processamento de estruturas de alto custo (e.g. orações passivas, orações relativas, particularmente as de objeto), que estão usualmente comprometidas em crianças com PDL. As autoras terminam o capítulo com a apresentação de um estudo desenvolvido com base na aplicação de procedimento de intervenção vocacionado para a apresentação de estruturas de alto custo, que envolve a compreensão e a produção. 

Os dois últimos capítulos do terceiro bloco temático centram-se em interfaces com a Sintaxe. No capítulo 15, Letícia Corrêa, Vanessa Ribeiro e Jacqueline Long\-champs exploram a interface gramática-pragmática e os resultados obtidos nesta área com crianças, falantes do português do Brasil, com dificuldade de aprendizagem e crianças com Perturbação/Transtorno do Espetro do Autismo. Na secção 3, as autoras apresentam dados de crianças com dificuldades de aprendizagem que sugerem um comprometimento seletivo na interface gramática-pragmática, o qual pode constituir uma característica importante no diagnóstico de Perturbação/Transtorno do Desenvolvimento da Linguagem. O capítulo 16, da autoria de Stéphanie Vaz, Maria Lobo e Marisa Lousada, debruça-se sobre a avaliação com base em narrativas orais. A partir de modelos discursivos disponíveis na literatura, este capítulo explora as diferentes dimensões que podem ser avaliadas nas narrativas orais infantis, em termos de características macro e micro estruturais, apresentando instrumentos usados para várias línguas, incluindo o português europeu, na avaliação dos desempenhos verbais infantis neste modo comunicativo. São ainda apresentados alguns dados sobre o desenvolvimento de competências narrativas por crianças com desenvolvimento típico e atípico, falantes do português europeu, recolhidos a partir de um instrumento que contempla tarefas de produção e de compreensão de narrativas. 

O último bloco temático deste volume reúne um conjunto de investigações sobre populações específicas. No capítulo 17, Larissa Berti foca a atenção na importância de avaliar produção e perceção em crianças com perturbações fonológicas. Numa perspetiva segmental, a autora parte do modelo teórico da Fonologia Gestual para investigar a construção de representações fonoló\-gicas, assumindo-se uma correspondência gestual nos erros de produção e de percepção em crianças com o diagnóstico de distúrbio fonológico. Apresenta uma classificação dos erros de produção e perceção da fala com base no gesto articulatório. Como implicação clínica, a autora destaca a necessidade de considerar, simultaneamente, aspetos da produção e da percepção, tanto na avaliação quanto na intervenção. No capítulo 18, Aline Lorandi e Débora Mattos Marques debruçam-se sobre a consciência fonológica e a escrita numa população de pessoas com síndrome de Down. Fazendo uma breve introdução histórica e uma caracterização da síndrome, as autoras centram-se, de seguida, na avaliação e na intervenção com base em tarefas de consciência fonológica junto de pessoas com esta condição clínica. Por fim, refletem sobre uma proposta holística, integrativa e multidisciplinar para indivíduos com síndrome de Down, desenhada a partir da Teoria da Complexidade. O capítulo 19, de Ana Mineiro, Sofia Lynce de Faria e Mara Moita, centra-se nas principais dificuldades identificadas no desenvolvimento linguístico atípico em crianças surdas gestuantes de Língua Gestual Portuguesa e na aquisição e desenvolvimento de uma língua oral em crianças surdas com implante coclear. Por fim, o último capítulo, 20, de Joana Alves, Mafalda Cardoso, Mariana Morgado e Luís Jesus, mostra de que modo a avaliação linguística deta\-lhada pré-operatória e durante a cirurgia (craniotomia) de pessoas com tumores cerebrais pode contribuir para delimitar melhor a(s) área(s) a remover sem causar défices pós-cirúrgicos adicionais.


Maria João Freitas, Marisa Lousada, Dina Caetano Alves

\subsection*{Agradecimentos}Agradecemos à Language Science Press, aos seus editores e avaliadores, a oportunidade de publicação e a forma como contribuíram para a revisão deste trabalho. Agradecemos, igualmente, aos autores dos vários capítulos, a disponibilidade para contribuírem para este volume, enquanto autores e revisores do trabalho dos colegas. Agradecemos, ainda, a Sybil Vachaudez, a dedicação e paciência na preparação do texto para a sua edição.

\section*{Notas biográficas das organizadoras do volume}
\textbf{Maria João Freitas}, linguista, é Professora Associada com Agregação da Faculdade de Letras da Universidade de Lisboa (FLUL), membro do Centro de Linguística da Universidade de Lisboa (financiamento FCT pelo projeto estratégico UID/LIN/00214/2020) e colaboradora do Centro de Linguística da Universidade do Porto. É diretora do Curso de Doutoramento em Linguística da FLUL. Tem lecionado unidades curriculares nas áreas da Fonologia e da Aquisição e Desenvolvimento Linguísticos, em cursos de graduação e de pós-graduação na área da Linguística. Tem (co)orientado projetos de pós-doutoramento, teses de doutoramento e dissertações de mestrado nas áreas da Aquisição e Desenvolvimento Linguísticos, Terapia da Fala e Fonologia. Recebeu o grau de Doutoramento em Letras (1997), com uma tese sobre aquisição da estrutura silábica no português europeu, e o grau de Agregação em Linguística Portuguesa (2013), ambos atribuídos pela Universidade de Lisboa. A sua investigação tem-se centrado nas áreas da Aquisição da Fonologia, Linguística Clínica e Linguística Educacional. Tem colaborado em vários projetos nacionais e internacionais, com ou sem financiamento, como investigadora e/ou como coordenadora. É autora e coautora de artigos em revistas e livros nas suas áreas de investigação, tendo ainda organizado publicações de divulgação sobre fonologia e sobre aquisição e desenvolvimento linguísticos.
\linebreak

\textbf{Marisa Lobo Lousada}, terapeuta da fala, é Professora Adjunta na Escola Superior de Saúde da Universidade de Aveiro (ESSUA), membro do Center for Health Technology and Services Research (CINTESIS.UA; financiamento FCT pelo projeto UIDB/4255/2020) e colaboradora do Centro de Linguística da Universidade de Lisboa. É diretora do Mestrado em Terapia da Fala da ESSUA. Leciona unidades curriculares nas áreas da avaliação e intervenção nas perturbações dos sons da fala e nas perturbações do desenvolvimento linguagem, em cursos de graduação e pós-graduação na área da Terapia da Fala. Tem (co)orientado teses de doutoramento e dissertações de mestrado nas áreas da terapia da fala e linguística. É doutorada em Ciências e Tecnologias da Saúde (2012) pela Universidade de Aveiro, com uma tese sobre as alterações fonológicas em crianças com perturbação de linguagem. As suas principais áreas de investigação exploram a avaliação e intervenção das perturbações da linguagem e dos sons da fala. Tem colaborado como investigadora em vários projetos nacionais e internacionais. É autora e coautora de artigos em revistas internacionais e capítulos de livro nas suas áreas de interesse. É ainda (co)autora de instrumentos de avaliação e programas de intervenção na área da linguagem. 
\linebreak

\textbf{Dina Caetano Alves}, linguista e terapeuta da fala, é Professora Adjunta na Escola Superior de Saúde do Instituto Politécnico de Setúbal (ESS-/IPS), membro do Centro de Linguística da Universidade de Lisboa (financiamento FCT pelo projeto estratégico UID/LIN/00214/2020) e do Centro Interdisciplinar de Investigação Aplicada em Saúde do IPS. Tem lecionado unidades curriculares nas áreas da Aquisição e Desenvolvimento Linguísticos e das Perturbações da Fala e da Linguagem (oral e escrita), em cursos de graduação e de pós-graduação na área da Terapia da Fala, colaborando pontualmente em unidades curriculares destes domínios noutras instituições. Tem (co)orientado teses de doutoramento e dissertações de mestrado nas áreas da Aquisição e Desenvolvimento Linguísticos e Terapia da Fala. Recebeu o grau de Doutoramento em Psicolinguística atribuído pela Faculdade de Letras da Universidade de Lisboa (2012), com uma tese sobre o efeito das propriedades segmentais em tarefas de consciência segmental, de leitura e de escrita. Os seus trabalhos de investigação têm-se centrado nas áreas da Psicolinguística e Terapia da Fala. Tem colaborado em vários projetos de investigação, com ou sem financiamento, como investigadora. É autora e coautora de artigos e capítulos de livro nestas áreas. É ainda (co)autora de instrumentos de avaliação e recursos de intervenção na área da linguagem (oral e escrita) e fala, com especial enfoque no domínio fonológico.

{\sloppy\printbibliography[heading=subbibliography,notkeyword=this]}


\end{refsection}

