\documentclass[output=paper,colorlinks,citecolor=brown,booklanguage=portuguese]{langscibook} 
\ChapterDOI{10.5281/zenodo.7233235}
\title{Avaliação do Conhecimento Sintático}
\author{Anabela Gonçalves \affiliation{Universidade de Lisboa, Faculdade de Letras; Centro de Linguística da Universidade de Lisboa} and  Sónia Vieira \affiliation{University of Washington, Department of Speech and Hearing Sciences; Centro de Linguística da Universidade de Lisboa}}

\abstract{A investigação em aquisição da sintaxe permite determinar o percurso de desenvolvimento sintático das crianças, identificando-se os seus diferentes estádios. Apesar de já existirem diversos estudos sobre desenvolvimento sintático, ainda não há, para o português europeu, um mapeamento completo dos estádios de aquisição sintática para diferentes estruturas. Um levantamento exaustivo das diferentes etapas de aquisição típicas de uma criança, desde o momento de emergência da estrutura até à sua estabilização, é de crucial importância para a identificação de uma perturbação da linguagem sintática, uma vez que só conhecendo um comportamento sintático típico se pode diagnosticar um comportamento sintático atípico. Assim, ter em conta, na prática clínica, os resultados da investigação linguística e vice-versa é cada vez mais importante para uma avaliação precoce de perturbações linguísticas. Um maior conhecimento clínico de fenómenos sintáticos que caracterizam uma perturbação não só permite estabelecer um plano de intervenção orientado para as necessidades da criança como também é fundamental para a criação de testes robustos e fiáveis que incluam estruturas sintáticas que se revelem marcadores clínicos. Neste sentido, o presente capítulo tem dois objetivos principais: (i) apresentar, de forma sucinta, conceitos e algumas estruturas sintáticas relevantes para avaliação do conhecimento sintático infantil; (ii) refletir sobre métodos de avaliação do conhecimento sintático e respetivos desafios.


\textbf{Palavras-chave}: PDL sintática; marcadores clínicos; técnicas de avaliação sintática; testes de avaliação sintática.}

\IfFileExists{../localcommands.tex}{
   \addbibresource{../localbibliography.bib}
   \usepackage{langsci-optional}
\usepackage{langsci-gb4e}
\usepackage{langsci-lgr}

\usepackage{listings}
\lstset{basicstyle=\ttfamily,tabsize=2,breaklines=true}

%added by author
% \usepackage{tipa}
\usepackage{multirow}
\graphicspath{{figures/}}
\usepackage{langsci-branding}

   
\newcommand{\sent}{\enumsentence}
\newcommand{\sents}{\eenumsentence}
\let\citeasnoun\citet

\renewcommand{\lsCoverTitleFont}[1]{\sffamily\addfontfeatures{Scale=MatchUppercase}\fontsize{44pt}{16mm}\selectfont #1}
  
   %% hyphenation points for line breaks
%% Normally, automatic hyphenation in LaTeX is very good
%% If a word is mis-hyphenated, add it to this file
%%
%% add information to TeX file before \begin{document} with:
%% %% hyphenation points for line breaks
%% Normally, automatic hyphenation in LaTeX is very good
%% If a word is mis-hyphenated, add it to this file
%%
%% add information to TeX file before \begin{document} with:
%% %% hyphenation points for line breaks
%% Normally, automatic hyphenation in LaTeX is very good
%% If a word is mis-hyphenated, add it to this file
%%
%% add information to TeX file before \begin{document} with:
%% \include{localhyphenation}
\hyphenation{
affri-ca-te
affri-ca-tes
an-no-tated
com-ple-ments
com-po-si-tio-na-li-ty
non-com-po-si-tio-na-li-ty
Gon-zá-lez
out-side
Ri-chárd
se-man-tics
STREU-SLE
Tie-de-mann
}
\hyphenation{
affri-ca-te
affri-ca-tes
an-no-tated
com-ple-ments
com-po-si-tio-na-li-ty
non-com-po-si-tio-na-li-ty
Gon-zá-lez
out-side
Ri-chárd
se-man-tics
STREU-SLE
Tie-de-mann
}
\hyphenation{
affri-ca-te
affri-ca-tes
an-no-tated
com-ple-ments
com-po-si-tio-na-li-ty
non-com-po-si-tio-na-li-ty
Gon-zá-lez
out-side
Ri-chárd
se-man-tics
STREU-SLE
Tie-de-mann
}
   \boolfalse{bookcompile}
   \togglepaper[12]%%chapternumber
}{}

\begin{document}
\maketitle

\section{Introdução}
A sintaxe é a componente da gramática que diz respeito às “propriedades das combinações de palavras que constituem expressões legítimas da língua” \citep[119]{Duarte2000}.\footnote{A norma adotada na escrita deste capítulo foi a do português europeu.}
% \footnote{Na sequência de restrições decorrentes do template usado para publicação na editora Language Science Press, os capítulos que integram o volume surgem com a referência [this volume], em inglês.}
Qualquer falante que não apresente perturbações de linguagem adquire de forma espontânea, desde muito cedo \citep{Guasti2002}, o conjunto de princípios que lhe permitem construir e compreender um conjunto potencialmente infinito de combinações livres de palavras. É, portanto, capaz de produzir e de interpretar adequadamente combinações linguísticas que poderá nunca ter produzido nem ouvido anteriormente, o que torna visível o aspeto criativo do uso da língua.

A par da evolução gradual do seu conhecimento sintático intuitivo, a criança vai desenvolvendo, também desde cedo, capacidades metalinguísticas de natureza sintática – a chamada \emph{consciência sintática} –, que lhe permitem identificar e manipular as unidades sintáticas da língua (frases, sintagmas e palavras), bem como as relações que entre elas se estabelecem, e, de acordo com as regras do seu sistema linguístico, elaborar juízos sobre a gramaticalidade das combinações linguísticas \citep{Perdigao2015, Costa2017}.  

Um dos aspetos em que o conhecimento sintático intuitivo e a consciência sintática dos falantes são reconhecíveis é a capacidade que estes têm de, por um lado, organizar as palavras em sequências bem formadas e, por outro, distinguir sequências que são possíveis na língua – ou seja, gramaticais – das que não o são – ou seja, agramaticais. Os falantes são, assim, capazes de emitir juízos de gramaticalidade mesmo que não consigam explicitar as regras da gramática que foram, ou não, respeitadas. 

O conhecimento sintático intuitivo dos falantes permite-lhes, ainda, atribuir a interpretação adequada às expressões linguísticas. Por essa razão, qualquer falante sem perturbações no que respeita à linguagem é capaz de atribuir interpretações distintas a frases como \emph{O Zé adora a Maria} e \emph{A Maria adora o Zé}, em que as mesmas palavras são organizadas de forma diferente, estabelecendo entre si diferentes relações gramaticais.

Embora o conhecimento linguístico intuitivo – e o conhecimento sintático não é exceção – seja desenvolvido espontaneamente, sem instrução específica, nem todas as estruturas são adquiridas em simultâneo. Sabe-se, por exemplo que, mesmo em contexto de desenvolvimento típico, e no que diz respeito à produção e à compreensão, as frases passivas são adquiridas depois das ativas e que passivas curtas (\emph{o livro foi lido}) são adquiridas antes de passivas longas (\emph{o livro foi lido pelo João}) – e.g., para o português europeu (PE), \citet{Estrela2014} e \citet{Agostinho2020}; para o português brasileiro (PB), \citet{LimaJunior2016} e \citet{Correa2017}. Se houver uma perturbação linguística, prevê-se que estruturas de aquisição tardia em situações de desenvolvimento típico sejam ainda mais tardias ou adquiridas de forma incompleta, ou não sejam de todo adquiridas, dependendo da severidade dessa perturbação. 

Assim, é fundamental que, na criação de instrumentos de avaliação, sejam considerados os resultados experimentais obtidos através de estudos sobre desenvolvimento linguístico típico e atípico, de forma a incluir estruturas que se sabe que podem revelar-se como marcadores clínicos, identificadores de perturbações ao nível da sintaxe (\citealp{Costa2016, Martins2017}, \citetv{chapters/02}, \citetv{chapters/13}). 

O presente capítulo tem dois objetivos principais: (i) apresentar, de forma sucinta, conceitos e estruturas sintáticas mais relevantes para avaliação do conhecimento sintático infantil; (ii) refletir sobre métodos de avaliação do conhecimento sintático e respetivos desafios.

\section{Sintaxe e conhecimento sintático}\label{sec:in_second_part}
\subsection{As unidades \emph{frase} e \emph{constituinte}}
Como referimos anteriormente, a sintaxe é a componente da gramática que diz respeito aos princípios e às operações que permitem construir expressões linguísticas bem formadas na língua. O objetivo último é a construção de uma expressão linguística a que chamamos \emph{frase}, pelo que este conceito é central para a teoria sintática.

As chamadas gramáticas tradicionais tendem a definir \emph{frase} com base no significado. É isso, por exemplo, que se verifica na seguinte definição, apresentada em \citet[119]{Cunha1984}: “FRASE é um enunciado de sentido completo, a unidade mínima de comunicação.” Esta definição não é, aparentemente, problemática para identificar como frase uma sequência como \emph{Os alunos leram o livro durante o fim-de-semana}.

Definições de \emph{frase} que privilegiem o significado como critério para a identificação deste conceito colocam, no entanto, alguns problemas. Por exemplo, tais definições permitiriam considerar como frase uma sequência como (1), à qual os falantes atribuem um “sentido completo”, ainda que não respeite a regra sintática de concordância sujeito-verbo:

\begin{enumerate}[align=left]
    \item [(1)] *A gente vimos o polícia.\footnote{O asterisco indica que a sequência é agramatical, ou seja, não respeita regras da língua.}
\end{enumerate}

Assim, o conceito de frase não se pode definir exclusivamente com base no significado. Consideremos, antes, que uma frase é o resultado de combinações de palavras que estabelecem relações entre si, respeitando os padrões de ordem possíveis na língua. São exemplos dessas relações os processos de concordância (entre o sujeito e o verbo, por exemplo) e as relações de dependência (por exemplo, a obrigatoriedade de ocorrência de um complemento locativo no contexto do verbo \emph{ir}, que explica o contraste entre \emph{Os turistas foram a Lisboa.} e *\emph{Os turistas foram}). Às frases que respeitam os padrões da língua, os falantes são capazes de atribuir um significado.\footnote{Note-se, porém, que, por um lado, é possível atribuir significados a sequências sintaticamente mal formadas, como a de (1) ou, ainda, as produzidas por falantes estrangeiros não proficientes, caso em que é crucial cooperação de natureza pragmática entre falantes; por outro lado, é possível reconhecer a boa-formação sintática de frases às quais não é possível atribuir significado. Encontra-se nesta última situação a famosa frase \emph{Colorless green ideas sleeep furiously.} (Ideias verdes sem cor dormem furiosamente.), de \citet{Chomsky1957}.} 

Numa definição como a que se apresenta no parágrafo anterior, são de particular importância as expressões \emph{combinações de palavras} e \emph{relações}. Na verdade, se, por um lado, os falantes conhecem as combinações lineares possíveis (por exemplo, sabem que um determinante precede um nome, mas não um verbo), por outro lado, reconhecem palavras e grupos de palavras que estabelecem relações entre si e que se comportam como unidades sintáticas: estes grupos, que se vão organizando e complexificando, chamam-se \emph{constituintes}. Tome-se, como exemplo, a frase (2):

\begin{enumerate}[align=left]
    \item [(2)] O estudante de linguística deixou os livros de sintaxe na biblioteca da escola.
\end{enumerate}

Perante esta frase, os falantes são capazes de identificar pequenas unidades sintáticas que funcionam como constituintes. Assim, todos reconhecerão que existe uma relação entre \emph{biblioteca} e \emph{da escola}, que não existe entre \emph{livros} e \emph{da escola}; ou que não existe qualquer relação entre \emph{linguística} e \emph{deixou}, ainda que estas duas unidades se encontrem adjacentes. 

Um dos argumentos mais claros a favor de uma teoria sintática que considere a frase como o resultado de combinações de palavras ou de grupos de palavras em constituintes é a ambiguidade sintática. Considere-se a frase (3):

\begin{enumerate}[leftmargin=*]
    \item [(3)] Os convidados trouxeram pastéis de Belém.
\end{enumerate}

À frase (3) podem ser atribuídas duas interpretações, parafraseáveis da seguinte forma: (i) os convidados foram a Belém e de lá trouxeram pastéis (não sendo explicitado o tipo de pastéis); (ii) os convidados trouxeram um tipo de pastéis a que se chama “pastéis de Belém” (não sendo explicitado o local de onde os trouxeram). A razão por que esta frase recebe duas interpretações é sintática ou estrutural, pelo que este tipo de ambiguidade se designa como \emph{ambiguidade estrutural}. Isto significa que as duas interpretações decorrem da forma como as palavras se organizam em constituintes, o que nos permite concluir que a diferentes estruturas sintáticas correspondem interpretações distintas. Assim, a primeira interpretação de (3) obtém-se no caso de \emph{pastéis} funcionar como um constituinte independente do constituinte \emph{de Belém}; nesse caso, é possível deslocar apenas um destes constituintes (cf. (4a, b)), deixando as frases de ser ambíguas:\footnote{Veja-se \citet{Duarte2000} para os testes que permitem identificar constituintes.}

\begin{enumerate}[align=left]
    \item [(4)]
    \begin{enumerate}
        \item [a.] [De Belém], os convidados trouxeram pastéis.
\item [b.] [Pastéis], os convidados trouxeram(-nos) de Belém.
\end{enumerate}
\end{enumerate}

Por sua vez, a segunda leitura resulta de uma estrutura de constituintes diferente: neste caso, \emph{pastéis de Belém} comporta-se como um constituinte, pelo que pode ser deslocado na sua totalidade, o que elimina igualmente a ambiguidade:

\begin{enumerate}[align=left]
    \item [(5)] [Pastéis de Belém], os convidados trouxeram.
\end{enumerate}

Geralmente, a ambiguidade é apenas potencial, visto que, muitas vezes, o contexto linguístico anterior ou o próprio contexto situacional permitem resolvê-la, ou seja, permitem ao falante escolher a interpretação adequada.

Os constituintes desempenham funções sintáticas na frase em que ocorrem. Retomemos dois exemplos já apresentados anteriormente. 

\begin{enumerate}[align=left]
    \item [(6)]
    \begin{enumerate}
        \item [a.] O Zé adora a Maria.
    \item	[b.] A Maria adora o Zé.
    \end{enumerate}
\end{enumerate}

Qualquer falante sem perturbações ao nível da linguagem consegue atribuir uma interpretação distinta às frases em (6), uma vez que, nestas frases, os constituintes \emph{o Zé} e \emph{a Maria} têm diferentes funções sintáticas e essa diferença é crucial para o significado: o constituinte \emph{o Zé} é sujeito em (6a), mas objeto direto em (6b); o constituinte \emph{a Maria} é objeto direto em (6a), mas sujeito em (6b). 

Note-se que combinações de palavras com uma ordem diferente, mas com as mesmas funções sintáticas, dão origem a uma interpretação, no essencial, idêntica:\footnote{A diferença entre as frases de (7) é que, em (a), toda a informação é nova para o interlocutor, enquanto, em (b), apenas o sujeito, em posição final, constitui informação nova.}

\begin{enumerate}[align=left]
    \item [(7)]
    \begin{enumerate}
        \item [a.] [\textsc{sujeito} O Zé] adora [\textsc{objeto direto} a Maria].
      	\item [b.] Adora [\textsc{objeto direto} a Maria], [\textsc{sujeito} o Zé].
    \end{enumerate}
\end{enumerate}

As funções sintáticas desempenham um papel importante não só na interpretação das frases (veja-se os exemplos em (7)) como também nos processos de concordância. Assim, em PE, por exemplo, o sujeito desencadeia concordância no verbo em pessoa e número (cf. (8)):

\begin{enumerate}[align=left]
    \item [(8)]
    \begin{enumerate}
        \item [a.] O Manel comprou um carro desportivo.
     	\item[b.] *O Manel comprei um carro desportivo.
    \end{enumerate}
\end{enumerate}

\subsection{Propriedades dos itens lexicais relevantes para a sintaxe}
A sintaxe estabelece uma relação estreita com o léxico, a componente que integra as unidades lexicais, a que se encontram associadas propriedades de natureza diversa (fonológica, morfológica, sintática e semântica). Algumas destas propriedades determinam de forma relevante as possibilidades de organização das palavras em constituintes. Uma dessas propriedades é a classe a que o item pertence, ou seja, a sua categoria sintática. O conhecimento da categoria da palavra é importante do ponto de vista sintático, visto que determina as posições que essa palavra pode ocupar, isto é, determina a sua distribuição. Considere-se, por exemplo, a frase em (9):

\begin{enumerate}[align=left]
    \item [(9)] Os meninos leram as revistas.
\end{enumerate}

Se solicitarmos a um falante que substitua a palavra \emph{meninos} por outra, certamente que esse falante escolherá palavras como \emph{alunos}, \emph{rapazes}, \emph{pais}, entre muitas outras possibilidades; dito de outra forma, selecionará palavras da classe dos nomes. Se, por sua vez, lhe solicitarmos que substitua a palavra \emph{leram} por outra, selecionará, de entre várias alternativas possíveis, palavras como \emph{compraram}, \emph{rasgaram}, \emph{sujaram}, ou seja, procederão a substituições por outros verbos. O que os falantes não farão é substituir a palavra \emph{meninos} por \emph{compraram} ou a palavra \emph{leram} por \emph{rapazes}, o que significa que fazem as suas escolhas dentro da mesma classe.

O conhecimento dos itens lexicais da língua implica, ainda, saber que restrições é que eles impõem ao contexto sintático em que ocorrem. Com efeito, palavras da mesma categoria sintática podem não ser substituíveis entre si. Por exemplo, as palavras \emph{ler} e \emph{gostar} são verbos; no entanto, não podem ocorrer no mesmo contexto sintático: \emph{ler} seleciona um complemento nominal em posição de objeto direto, \emph{gostar} seleciona um complemento preposicionado, que corresponde ao complemento oblíquo (veja-se (10a, b) por oposição a (10c, d)):

\begin{enumerate}[align=left]
    \item [(10)]
    \begin{enumerate}
        \item [a.] Os meninos leram as revistas.
	\item[b.] Os meninos gostaram das revistas.
	\item[c.] *Os meninos gostaram as revistas.
	\item[d.] *Os meninos leram das revistas.
    \end{enumerate}
\end{enumerate}

Estas condições sobre a categoria dos constituintes que são selecionados como complementos por um item lexical designam-se como \emph{propriedades de seleção categorial} ou \emph{propriedades de subcategorização}.

Outro aspeto importante na determinação dos contextos em que os itens lexicais podem ocorrer resulta diretamente do seu significado enquanto predicadores. Consideremos verbos como \emph{matar} e \emph{assassinar} (cf. (11), ver discussão em \citealp[71-72]{Duarte2000}). Estes verbos, de significado próximo, selecionam o mesmo número de argumentos (dois), da mesma categoria (ambos sintagmas nominais) e com a mesma função sintática (sujeito e objeto direto):\footnote{Os argumentos correspondem aos constituintes que um predicador seleciona para que a predicação fique completa, isto é, para que a situação seja descrita de forma completa.} 

\begin{enumerate}[align=left]
    \item [(11)]
    \begin{enumerate}
        \item [a.] O incêndio matou três moradores do edifício.
\item	[b.] O suspeito assassinou três moradores do edifício.
    \end{enumerate}
\end{enumerate}

Apesar de, aparentemente, os dois verbos ocorrerem no mesmo contexto sintático, o verbo \emph{matar}, em (11a), não pode ser substituído pelo verbo \emph{assassinar} (veja-se: *\emph{O incêndio assassinou três moradores do prédio.}). Com efeito, este verbo seleciona, para a posição de sujeito, um argumento que denote uma entidade capaz de desencadear voluntariamente a situação, o que não se verifica com a expressão o \emph{incêndio}. Assim, os itens lexicais impõem ainda restrições de natureza semântica aos seus argumentos, aos quais atribuem funções semânticas (ou papéis temáticos). Se considerarmos os verbos das frases de (11), verificamos que o verbo \emph{matar} seleciona, para a posição de sujeito, um argumento com a função semântica de Causador (involuntário, como em (11a), ou voluntário, como em \emph{O suspeito matou deliberadamente três moradores do edifício}), enquanto o verbo \emph{assassinar} seleciona, para a posição de sujeito, um argumento com o papel semântico de Agente, designando uma entidade capaz de desencadear voluntariamente a situação.\footnote{Para uma lista de papéis semânticos, consulte-se, por exemplo, \citet[188-190]{Duarte2003}.} A estas condições quanto aos papéis semânticos atribuídos por um item lexical aos seus argumentos chama-se \emph{propriedades de seleção semântica}. 

Em função do seu significado e dos papéis semânticos que atribuem aos seus argumentos, os predicadores impõem ainda restrições quanto aos traços semânticos que estes argumentos possuem. Considerando, de novo, os verbos das frases em (11), verificamos que o verbo \emph{matar} seleciona, como Causador, um argumento que designe uma entidade não humana (\emph{o incêndio}) ou humana (\emph{o suspeito}, em \emph{o suspeito matou deliberadamente três moradores do edifício}), ao passo que o argumento Agente do verbo \emph{assassinar} designa obrigatoriamente uma entidade humana. Os traços semânticos dos argumentos decorrem das \emph{restrições de seleção} impostas pelo predicador a esses argumentos.

As propriedades de seleção semântica mantêm-se nas frases em que o predicador ocorre. Considere-se, por exemplo, o par de frases ativa-passiva:

\begin{enumerate}[align=left]
    \item [(12)]
    \begin{enumerate}
        \item [a.] O João leu o livro.
	\item[b.] O livro foi lido pelo João.
	\item[c.]  O livro foi lido.
    \end{enumerate}
\end{enumerate}

Ainda que, nas frases (12a, b), se alterem as funções sintáticas dos constituintes (\emph{o João} é sujeito em (12a), mas agente da passiva em (12b); \emph{o livro} é objeto direto em (12a), mas sujeito em (12b)), o significado destas frases é, no essencial, idêntico, visto que, em ambas, \emph{o João} recebe o papel semântico de Agente, designando a entidade que desencadeia a ação, e \emph{o livro}, o de Tema, designando a entidade que é afetada pela situação descrita. É de notar que, na passiva, o agente pode ser explícito (cf. (12b)) ou implícito (cf. (12c)), facto que está na base da distinção entre passiva longa e passiva curta.

\subsection{Ordem de palavras}\label{sec:cap12sec2.3}
Nas línguas, é possível determinar a ordem básica (ou canónica) dos constituintes. Esta ordem, menos marcada quer do ponto de vista sintático quer do ponto de vista discursivo, é identificada a partir de frases que sejam, cumulativamente, simples, declarativas e afirmativas, sem marcação prosódica de constituintes (isto é, sem que se atribua um destaque especial a constituintes da frase). Em português, a ordem básica é sujeito-verbo-objeto (SVO), como em (13):

\begin{enumerate}[align=left]
    \item [(13)] O Zé adora a Maria.
\end{enumerate}

Nesta língua, a ordem de palavras é importante para a interpretação das combinações de palavras, uma vez que é usada para a identificação de funções sintáticas, como anteriormente mostrámos.

Dizer que o português é uma língua SVO não significa, no entanto, que todas as frases manifestam este padrão de ordenação dos constituintes. Com efeito, construções sintáticas particulares ou razões de natureza discursiva podem determinar alterações à ordem básica dos constituintes.

Nos exemplos (14) a (16), apresentam-se algumas construções sintáticas em que a alteração à ordem básica é obrigatória: 

\begin{enumerate}[align=left]
    \item [(14)]
    \begin{enumerate}
        \item [a.] Os estudantes [[a quem]\textsubscript{i} o professor deu os livros [-]\textsubscript{i}] são muito traba\-lhadores. \\
     		(OSV na subordinada relativa)
	\item[b.] *Os estudantes [o professor deu os livros [a quem]] são muito traba\-lhadores.
    \end{enumerate}
    \item[(15)]
    \begin{enumerate}
        \item [a.] [O que]\textsubscript{i} é que o João leu [-]\textsubscript{i}? (OSV)
	\item[b.] *O João leu [o que]?
    \end{enumerate}
    \item[(16)]
    \begin{enumerate}
        \item [a.] O João disse [que a Teresa [te]\textsubscript{i} viu [-]\textsubscript{i} na escola].\\ 
		(SOV na subordinada completiva)
	\item[b.] *O João disse [que a Teresa viu-[te] na escola].
    \end{enumerate}
\end{enumerate}

Em (14), apresenta-se um exemplo de uma frase relativa em que que o constituinte relativo \emph{a quem} tem a função sintática de objeto indireto, embora preceda o verbo;\footnote{Classicamente, os elementos que introduzem as frases relativas são classificados como pronomes, advérbios ou adjetivos relativos.} (15) é um caso de uma interrogativa direta parcial introduzida por um constituinte interrogativo (interrogativa Qu-), tendo o pronome interrogativo a função sintática de objeto direto, ainda que em posição pré-verbal; a frase em (16) exemplifica a alteração à posição neutra (a ênclise) do clítico, que tem de ocupar a posição de próclise (isto é, uma posição pré-verbal) na presença de elementos que se designam como proclisadores (no caso de (16), a conjunção subordinativa completiva \emph{que}).

Para interpretarem estas frases, é necessário que os falantes reconheçam a relação entre o constituinte deslocado, através de uma operação de movimento, e a sua posição básica (assinalada como [-] nos exemplos (a) anteriores), identificando, assim, a relação de dependência entre esse constituinte e o verbo. Assim, no caso da frase (14a), por exemplo, é necessário relacionar a posição em que o constituinte relativo é realizado e aquela em que, como argumento do verbo \emph{dar}, é interpretado. 

Há outros casos em que a alteração à ordem não é obrigatória para garantir a boa-formação das frases, mas é desencadeada por fatores de natureza discursiva, permitindo, por exemplo, identificar o tópico da frase (ou seja, “o assunto sobre o qual se afirma ou se nega alguma coisa”, \citealp[149]{Duarte2000}). Considerem-se as frases em (17):

\begin{enumerate}[align=left]
    \item [(17)]
    \begin{enumerate}
        \item [a.] Os alunos conhecem a professora. (SVO)
	\item[b.] [A professora]\textsubscript{i}, os alunos conhecem [-]\textsubscript{i}.  (OSV) 
    \end{enumerate}
\end{enumerate}

A frase (17b) exemplifica uma estrutura de topicalização, tendo o constituinte \emph{a professora} a função discursiva de tópico. Este constituinte, que corresponde ao objeto direto do verbo \emph{conhecer}, foi deslocado para a posição inicial de frase, sendo necessário recuperar a relação com aquele verbo. Note-se que as frases (17a) e (17b) não são totalmente equivalentes: em (17a) o locutor está a afirmar sobre os alunos que eles conhecem a professora; por sua vez, em (17b), o locutor está a afirmar sobre a professora que os alunos a conhecem. 

\subsection{Frases simples e frases complexas}\label{sec:doispontoquatro}
O conhecimento sintático dos falantes permite-lhes formar diferentes frases, que exprimem diferentes atitudes proposicionais do falante que as produz (o locutor): as frases declarativas exprimem, na generalidade, uma asserção (\emph{cf}. (18)), as frases interrogativas exprimem um pedido de informação ou de ação (\emph{cf}. (19)), as frases imperativas exprimem uma ordem (\emph{cf}. (20)), as frases exclamativas exprimem uma avaliação (\emph{cf}. (21)) e as frases optativas exprimem um desejo (\emph{cf}. (22)).\footnote{Estas definições simplificadas dos diferentes tipos de frases são retiradas de \citet{Duarte2000}. Para uma tipologia mais completa, veja-se \citet[433-506]{Brito2003} e \citet{Barbosa2020}.}

\begin{enumerate}[align=left]
    \item [(18)]
    \begin{enumerate}
        \item [a.] Os deputados votaram essa lei. (declarativa afirmativa ativa)
	\item[b.] Os deputados não votaram essa lei. (declarativa negativa ativa)
	\item[c.] Essa lei foi votada pelos deputados. (declarativa afirmativa passiva)
    \end{enumerate}
    \item[(19)]
    \begin{enumerate}
        \item [a.] Vais ao cinema amanhã? (interrogativa global, com resposta \emph{sim / não})
	\item[b.] Quem é que vai ao cinema amanhã? (interrogativa direta parcial, Qu-)
	\item[c.] Viste quem? (interrogativa eco)
    \end{enumerate}
    \item[(20)]  Fecha já a janela! (imperativa)
    \end{enumerate}
    \begin{enumerate}[align=left]
    \item[(21)]
    \begin{enumerate}
        \item [a.] Esta paisagem é maravilhosa! (exclamativa total)
	\item[b.] Que maravilhosa é esta paisagem! (exclamativa parcial)
    \end{enumerate}
    \item[(22)] Oxalá tenhas sucesso no teu novo emprego! (optativa)
\end{enumerate}

As frases acima apresentadas são simples, uma vez que nelas ocorre um só domínio de predicação (formado pelo predicador e pelos seus argumentos) dotado de tempo gramatical. No entanto, é possível combinar recursivamente frases com outras frases, de modo a formar frases complexas, caracterizadas pela ocorrência de mais do que um domínio de predicação dotado de tempo gramatical. A formação de frases complexas resulta essencialmente de dois processos: a coordenação e a subordinação.\footnote{Nesta secção, faz-se apenas uma descrição sucinta das frases complexas. Para uma descrição mais detalhada sobre subordinação e coordenação em PE, vejam-se os capítulos relevantes em \citet{Mateus2003} e \citet{Raposo2013}, bem como as referências aí citadas.}

As estruturas de subordinação são estruturas de encaixe, ou seja, a oração subordinada é um constituinte da frase matriz, desempenhando uma função sintática (sujeito, objeto direto, complemento oblíquo, modificador). A frase na sua totalidade designa-se como frase matriz. 

Em português, existem três tipos de orações subordinadas:
 
(i) Substantivas – ocupam posições que podem ser preenchidas por expressões nominais; são deste tipo as orações completivas, preposicionadas ou não, selecionadas por verbos, nomes ou adjetivos (\emph{cf}. (23)) e as orações relativas sem antecedente expresso ou livres (isto é, aquelas em que o antecedente do constituinte relativo não se encontra realizado; \emph{cf}. (24)).\footnote{No caso das subordinadas relativas sem antecedente expresso introduzidas pelo advérbio \emph{onde}, a relativa ocorre em posições típicas de constituintes preposicionais e não nominais.}

\begin{enumerate}[align=left]
    \item[(23)]
    \begin{enumerate}
        \item [a.] A Maria \emph{prefere} [que lhe contem a verdade]. \\          (completiva de verbo, com a função de objeto direto)
	\item[b.] É um \emph{facto} [que a Maria apenas contou a verdade].\\ (completiva de nome, com a função de sujeito)
	\item[c.] O Manel está \emph{orgulhoso} por [a filha ter conseguido um bom emprego].\\ (completiva de adjetivo).
    \end{enumerate}
    \item[(24)]
    \begin{enumerate}
        \item [a.] [Quem fez esse disparate] não pensou.   \\                        (relativa sem antecedente expresso com a função de sujeito)
	\item[b.] A Maria adora [o que o Pedro diz]. \\
	(relativa sem antecedente expresso com a função de objeto direto)
\item[c.] Entreguei o livro [a quem mo pediu]. \\                                      (relativa sem antecedente expresso com a função de objeto indireto)
    \end{enumerate}
\end{enumerate}

(ii) Adjetivas - ocupam posições típicas de expressões adjetivais, como se verifica se compararmos os exemplos (a) e (b) de (25) e (26). São deste tipo as relativas com antecedente expresso.

\begin{enumerate}[align=left]
    \item[(25)]
    \begin{enumerate}
        \item [a.] O rapaz [que chegou] é meu irmão.
	\item[b.] O rapaz [loiro] é meu irmão.
	\end{enumerate}
\item[(26)]
\begin{enumerate}
\item[a.] Os meninos, [que estavam nervosos], fizeram o teste com dificuldade.
\item	[b.] Os meninos, [nervosos], fizeram o teste com dificuldade.
    \end{enumerate}
\end{enumerate}
As orações adjetivas são tipicamente relativas, com a função de modificadores (ou seja, não são argumentos) de expressões nominais (como em (25a) e em (26a)) ou de frases, como em (27):
\begin{enumerate}[align=left]
    \item [(27)] O João não cumpriu o prazo estabelecido, [o que me surpreendeu].
\end{enumerate}

Isto significa que o antecedente pode ser uma expressão nominal, (como em (25a) e (26a)) ou frásico (\emph{o João não cumpriu o prazo estabelecido}, em (27)). 

As subordinadas relativas adjetivas podem corresponder a um modificador restritivo ou a um modificador apositivo, conforme o papel que desempenham na construção da referência da expressão que modificam. Assim, as subordinadas relativas restritivas, como o nome indica, restringem a referência da expressão nominal que modificam, ou seja, definem um subconjunto de indivíduos a partir de um conjunto prévio; são, portanto, modificadores restritivos. Veja-se o exemplo em (28):

\begin{enumerate}[align=left]
    \item [(28)] As crianças [que tinham os calções vestidos] foram à praia.
\end{enumerate}

A partir de (28), podemos inferir que existem dois subconjuntos de crianças, as que tinham os calções vestidos e as que não tinham, e que apenas aquelas que pertencem ao primeiro subconjunto foram à praia.

Por sua vez, as subordinadas relativas apositivas (ou explicativas) são modificadores apositivos, sendo o antecedente uma frase (\emph{cf}. (27)) ou uma expressão nominal, como em (26a) e em (29): 

\begin{enumerate}[align=left]
    \item [(29)] As crianças, [que tinham os calções vestidos], foram à praia.
\end{enumerate}

Em (29), ao contrário do que acontece em (28), a subordinada relativa não restringe a referência da expressão nominal que modifica, ou seja, não define um subconjunto de indivíduos a partir de um conjunto prévio, dando apenas informação adicional. Assim, é definido apenas um conjunto – o das crianças –, sobre o qual se dá uma informação (a de que tinham os calções vestidos); não é, portanto, possível inferir que algumas crianças não tinham os calções vestidos.

Classicamente, considera-se que o constituinte relativo tem uma função sintática dentro da oração subordinada a que pertence, como, por exemplo, sujeito em (30a), objeto direto em (30b), complemento oblíquo em (30c):

\begin{enumerate}[align=left]
    \item [(30)]
    \begin{enumerate}
        \item [a.] O rapaz [\emph{que} encontrou os meus filhos no cinema] é filho da Teresa.
	\item[b.] O rapaz [\emph{que} o Zé encontrou no cinema] é filho da Teresa.
	\item[c.] O rapaz [\emph{com quem} a Ana partilhou as pipocas] é filho da Teresa.
    \end{enumerate}
\end{enumerate}

No quadro da gramática generativa, as orações relativas têm sido objeto de análises diversas. Por exemplo, para o PE, \citet{Brito1991} e \citet{Alexandre2000} consideram que o elemento que introduz relativas restritivas de sujeito e de objeto direto (\emph{que}) é um complementador, ou seja, é idêntico ao elemento que introduz as completivas finitas. Para o mesmo tipo de relativas, autores como \citet{Kayne1994} e \citet{Bianchi2000} consideram, ainda, que a expressão nominal antecedente se desloca do interior da relativa. Esta análise é assumida em \textcitetv{chapters/14} e \textcitetv{chapters/13}. Outras análises defendem que, independentemente do tipo de relativa, é o elemento que as introduz - o chamado constituinte relativo - que é movido para a posição inicial, a partir da posição correspondente à sua função sintática no interior da relativa.

Qualquer que seja a análise, os falantes têm de estabelecer corretamente uma relação de dependência entre a posição em que um constituinte é realizado e aquela em que é interpretado, como se referiu na secção \ref{sec:cap12sec2.3}. 

Nos exemplos em (30), as subordinadas relativas encontram-se encaixadas ao centro, mas as frases deste tipo podem também ocorrer encaixadas à direita, como em (31):

\begin{enumerate}[align=left]
    \item [(31)] Comprei o livro [que o meu irmão me recomendou].
\end{enumerate}

(iii) Adverbiais – ocorrem em posições típicas de expressões adverbiais e não são selecionadas, tendo, portanto, a função sintática de modificador. O valor semântico destas subordinadas (temporal, causal, condicional, concessivo, entre outros) é, geralmente, determinado pelo significado das conjunções e locuções que as introduzem. As subordinadas adverbiais têm alguma mobilidade na frase (\emph{cf}. (32)), embora nem sempre isso aconteça (\emph{cf}. (33)):

\begin{enumerate}[align=left]
    \item [(32)]
    \begin{enumerate}
        \item [a.] O João não saiu de casa [porque estava constipado].
	\item[b.] [Porque estava constipado], o João não saiu de casa.
	\item[c.] O João, [porque estava constipado], não saiu de casa.
	\end{enumerate}
\item[(33)]
\begin{enumerate}
    \item[a.] [Como estava constipado], o João não saiu de casa.
	\item[b.] *O João não saiu de casa [como estava constipado].
	\item[c.] O João, [como estava constipado], não saiu de casa.
    \end{enumerate}
\end{enumerate}

Um outro processo de formação de frases complexas é a coordenação. Ao contrário do que acontece com a subordinação, os termos coordenados são sintaticamente independentes, pois não desempenham funções sintáticas um relativamente ao outro. Tal como acontece com as subordinadas adverbiais, o valor semântico das conjunções e de outros conectores de coordenação determina diferentes tipos de frases coordenadas; por exemplo, \emph{e} tem o valor básico de adição (coordenação copulativa); \emph{mas} tem o valor básico de contraste (coordenação adversativa); \emph{ou} tem o valor básico de alternativa (coordenação disjuntiva).

\begin{enumerate}[align=left]
    \item [(34)] A Maria fez o jantar e o filho lavou a louça.
\item[(35)] A Maria fez o jantar, mas o filho não lavou a loiça.
\item[(36)] Faço uma sopa ou tu estrelas uns ovos?
\end{enumerate}

A coordenação permite formar constituintes complexos de categorias distintas. Assim, é possível coordenar frases, como em (34) a (36), mas também expressões nominais (\emph{cf}. (37a)), adjetivais (\emph{cf}. (37b)), preposicionais (\emph{cf}. (37c)), entre outras:

\begin{enumerate}[align=left]
    \item [(37)]
    \begin{enumerate}
        \item [a.][O João e o irmão] foram a Paris no Natal passado.
	\item[b.] A Maria é [simpática e conversadora].
	\item[c.] Os estudantes leram os livros [de sintaxe e de fonologia].
    \end{enumerate}
\end{enumerate}

\section{Métodos de avaliação do conhecimento sintático}
Nas secções anteriores, descreveram-se alguns conceitos e estruturas relevantes para a caracterização do conhecimento sintático dos falantes com desenvolvimento típico. Ora, na avaliação da linguagem, pretende-se obter, através das respostas dadas pelos sujeitos, uma medida de seu conhecimento. Assim, um dos primeiros passos para descrever o conhecimento linguístico de um falante é encontrar uma abordagem de avaliação para a população em estudo que quantifique índices de desempenho válidos (ver também \citetv{chapters/02}). Atualmente, ao nível da prática clínica, há três tipos de procedimentos que permitem avaliar o conhecimento sintático de um falante:

\begin{enumerate}[align=left]
\item[1)] Procedimentos informais ou não estandardizados, em que se pode incluir a avaliação do discurso espontâneo;
\item[2)] Escalas de desenvolvimento, em que são avaliados marcos de progressão, de acordo com o desempenho da criança (avaliação que pode ser efetuada pelo examinador ou pela observação dos pais/cuidadores);
\item[3)] Testes formais ou estandardizados.
\end{enumerate}

A análise do discurso espontâneo providencia uma recolha de dados mais informal, uma vez que não é afetada por exigências adicionais ligadas à tarefa de avaliação. O uso do discurso espontâneo pode ser vantajoso comparativamente com o recurso a testes mais formais e estandardizados, pois a recolha de dados é efetuada num ambiente mais natural e menos descontextualizado. 

Apesar de ser uma fonte rica de informação relativamente ao conhecimento sintático da criança, o discurso espontâneo é, no entanto, limitado por diferentes razões. Por um lado, há evidência de que as amostras recolhidas através do discurso espontâneo poderão subestimar o verdadeiro conhecimento sintático da criança, visto que certas estruturas sintáticas poderão não ocorrer numa determinada recolha de dados devido a motivos pragmáticos ou cognitivos (por exemplo, as frases passivas são raramente produzidas durante o discurso espontâneo). Assim, a amostra recolhida durante o discurso espontâneo da criança pode não ser representativa do seu conhecimento linguístico \emph{per se}, mas sim das estruturas que a criança escolhe usar naquele contexto em particular. O discurso espontâneo não só poderá não revelar o que a criança sabe, mas não teve oportunidade de usar, como também não fornece dados de compreensão da linguagem \citep{TagerFlushberg2000}.

O uso de técnicas mais estruturadas e específicas, como as técnicas de elicitação, é vantajoso quando há limite de tempo durante o processo de avaliação e quando a ocorrência de uma determinada estrutura sintática no discurso espontâneo é pouco frequente ou está mesmo ausente, sem que isso signifique necessariamente que a criança não a tenha disponível na sua gramática (\emph{e.g.}, durante o discurso espontâneo de uma criança, é mais frequente a produção de frases simples do que de frases complexas). O recurso a técnicas mais estruturadas implica, em geral, colocar os participantes em situações comunicativas que são especificamente delineadas de forma a que a produção de uma estrutura-alvo particular seja a resposta mais adequada em detrimento de outras. Quando um contexto de elicitação é cuidadosamente construído, há uma maior probabilidade de se observar comportamentos sintáticos que não ocorrem com frequência no discurso espontâneo da criança. As hipóteses de tentativa e erro em estruturas raramente produzidas poderão revelar o seu grau de dificuldade para a criança e levar o examinador a refletir sobre os fatores gramaticais que são especialmente problemáticos \citep{Zukowski2009}.

Note-se, ainda, que o facto de a análise do discurso espontâneo ser uma medida informal e que não possui dados normativos tem como consequência o aumento da subjetividade na avaliação dos comportamentos sintáticos entre os profissionais da linguagem, o que levará, necessariamente, a inconsistências no diagnóstico e na interpretação dos resultados obtidos. Na ausência de estudos alargados e pormenorizados sobre os diferentes estádios de desenvolvimento sintático de crianças portuguesas, é difícil para os terapeutas da fala avaliarem se a produção ou não de uma determinada estrutura sintática pela criança está conforme com o desenvolvimento típico. Essa consistência com os dados de aquisição pode ser alcançada fazendo com que as variáveis que estão sob o controlo do examinador sejam o mais uniformes possível e usando normas, ou seja, um desempenho coletivo de uma amostra representativa da população-alvo com a qual o desempenho de outros indivíduos é comparado. Isto significa que, se um instrumento de diagnóstico é referenciado à norma para a população em questão, o terapeuta é capaz de avaliar as competências linguísticas de forma mais fiável e válida. 

\largerpage[-1]
Atualmente, os terapeutas da fala têm uma possibilidade ampla de escolha no que diz respeito a baterias de avaliação que apresentam dados normativos de avaliação sintática para o PE: o Teste de Avaliação da Linguagem Oral (ALO, \citealp{SimSim1997}), o Teste de Identificação de Competências Linguísticas (TICL, \citealp{Viana2004}), o Teste de Avaliação da Linguagem na Criança (TALC, \citealp{SuaKay2006}), o Teste de Avaliação da Linguagem Pré-escolar  (TL-ALPE, \citealp{Mendes2014}), a Grelha de Observação da Linguagem – Nível escolar revista (GOL-E, \citealp{SuaKay2014}) e o Teste Schlichting: Teste de Avaliação da Competência Sintática (Sin:TACS, \citealp{Vieira2018}).\footnote{Refira-se, ainda, a bateria de testes MABILIN (Módulos de Avaliação de Habilidades Linguísticas), que inclui um módulo sintático concebido originalmente para o PB e adaptado para o PE por uma equipa do Centro de Linguística da Universidade Nova de Lisboa. Veja-se \textcitetv{chapters/14}.}  

Os testes existentes para a avaliação da sintaxe do PE em crianças utilizam diferentes medidas de competência sintática ao nível da compreensão, da produção e da consciência sintática. Para a recolha dos dados, utilizam-se diversas técnicas, de forma a avaliar o conhecimento sintático das crianças em idades pré-escolar e escolar.\footnote{Para uma descrição mais detalhada de cada subteste, de tipos de tarefas e de exemplos que compõem as diversas baterias de testes para o PE, ver \citet{Martins2017}.}

Relativamente à avaliação da compreensão sintática, e para estes testes em particular, as tarefas utilizadas variam entre selecionar a imagem correspondente à frase produzida pelo examinador (\emph{cf}. (38)) e responder a questões (\emph{cf}. (39)).

\begin{enumerate}[align=left]
    \item [(38)] O homem que está a escovar o cão é magro. (TALC)
\item[(39)] 	Hoje ou vamos à feira ou vamos ao jardim. Onde vamos hoje? (ALO)
\end{enumerate}

De forma a medir produção sintática, são utilizadas, nos testes acima mencionados, diferentes técnicas de elicitação, como: (a) completamento de frases (\emph{cf}.(40, 41)); (b) produção de frases complexas a partir de duas frases simples que são apresentadas à criança (\emph{cf}. (42)); (c) resposta a perguntas (\emph{cf}. (43, 44)); (d) imitação exata, sendo pedido à criança que repita a estrutura sintática que foi verbalizada momentos antes pelo examinador (\emph{cf}. (45)) e (e) imitação com variação, caso em que a criança produz a mesma estrutura sintática que o examinador, mas com variação lexical (\emph{cf}. (46)). 

\begin{enumerate}[align=left]
    \item [(40)] 	Aqui está uma \_\_\_ (\emph{bola}), aqui estão três \_\_\_ (\emph{bolas}). (TL-ALPE)
\item[(41)] 	Este bolo não é bom. Este bolo é bom. Este bolo é ainda \_\_\_. (TICL)
\item[(42)] 	O João caiu. Fez uma ferida. (GOL-E)
\item[(43)] 	O que é que a menina está a fazer? \\
	Resposta esperada: A menina lava/está a lavar os dentes. (TL-ALPE)
\item[(44)] 	O que é que o menino tem? \\
	Resposta esperada: O menino tem um chocolate e um sumo. (TL-ALPE)
\item[(45)] 	Examinador: Este vai comer. \\
	Resposta esperada: Este vai comer. (Sin:TACS)
\item[(46)] 	Examinador: Eu acho que levo o carro. \\
	Resposta esperada: Eu acho que levo o avião. (Sin:TACS)
\end{enumerate}

Por sua vez, a avaliação da consciência sintática, nos testes que a incluem, é realizada, na sua maioria, através de tarefas de juízos de gramaticalidade, tendo a criança de identificar se uma determinada estrutura-alvo é gramatical e, em caso negativo, de a corrigir (\emph{cf}. (47, 48)), ou através de tarefas de manipulação de palavras desorganizadas de forma a criar uma frase gramatical (\emph{cf}. (49)).

\begin{enumerate}[align=left]
    \item [(47)] 	* Ele se penteia-se sozinho (GOL-E)
\item[(48)] 	* Os meninos joga à bola. (TICL)
\item[(49)] 	chora bebé o (GOL-E)
\end{enumerate}

Os testes de avaliação acima mencionados avaliam a produção e/ou a compreensão de diversos fenómenos (morfos)sintáticos enunciados na secção \ref{sec:in_second_part} do presente capítulo, tais como: frases complexas (subordinadas e coordenadas), diferentes tipos de frases, processos de concordância (em género e número, bem como em pessoa e número), frases simples integrando verbos com diferentes propriedades de seleção, frases passivas, alterações à ordem básica dos constituintes e clíticos.

\section{Considerações sobre os desafios da avaliação da sintaxe}
Com exceção do teste Sin:TACS, os restantes testes referidos neste capítulo são delineados para uma avaliação global (quanto à compreensão e à produção) de vários módulos da gramática.  Uma grande vantagem de estes testes serem abrangentes por natureza é a de fornecerem uma descrição mais geral do comportamento linguístico de uma criança, apontando para as áreas da gramática em que há melhores ou piores desempenhos. Paradoxalmente, essa vantagem é também uma desvantagem, visto que esses testes não são construídos de forma a examinar de forma mais detalhada questões de âmbito lexical, fonológico, morfológico, semântico, sintático, morfossintático ou pragmático. Relativamente à sintaxe, o facto de nem sempre estas baterias de avaliação incluírem um número relevante de estruturas consideradas como marcadores clínicos (entendidos como comportamentos linguísticos que permitem distinguir o desenvolvimento típico do atípico) poderá ser particularmente problemático ao avaliarmos crianças com perturbações da linguagem \citep{Afonso2011}. 

De entre diferentes tipos de perturbações da linguagem, podemos referir a Perturbação do Desenvolvimento da Linguagem (PDL) como uma das que pode caracterizar-se por défices significativos ao nível da sintaxe.\footnote{Perturbação do Desenvolvimento da Linguagem (PDL) é o termo atualmente utilizado para referir as perturbações de linguagem que ocorrem sem causa aparente \citep{Bishop2017}. Os termos anteriormente utilizados para referir estas perturbações de linguagem eram Perturbação Específica da Linguagem (PEL) ou Perturbação Específica do Desenvolvimento Linguístico (PEDL). Sobre a evolução da classificação e do termo, veja-se \textcitetv{chapters/13}. Em PB, é usado o termo Transtorno do Desenvolvimento da Linguagem (TDL).}  A PDL é uma perturbação no desenvolvimento linguístico, apesar de não existir evidência de outras condições que expliquem tal perturbação, tais como défice de audição ou visão, alterações neurológicas, défices de socialização ou cognitivos. A identificação de marcadores clínicos em crianças com PDL poderá permitir um diagnóstico mais preciso desta perturbação, desde já complexa de identificar pela ausência de critérios subjacentes que a justifiquem. Crianças com PDL mostram limitações distintas, dependendo do(s) módulo(s) afetado(s) – fonológico, lexical, pragmático ou (morfos)sintático \citep{Friedmann2008}. No entanto, este capítulo foca-se apenas na PDL sintática, que se caracteriza por alterações (morfos)sintáticas ao nível da compreensão e da produção. Relativamente à PDL sintática, um marcador clínico revelará défices ou comportamentos sintáticos que ocorrem em crianças com essa condição mas não em crianças com desenvolvimento típico ou com outras perturbações da linguagem. 

Para se identificar os marcadores clínicos, deverão existir estudos extensivos que indiquem os estádios de aquisição de uma determinada estrutura sintática. Um conhecimento aprofundado do processo típico de aquisição permite estabelecer de forma mais eficaz um diagnóstico diferencial de uma criança com perturbações da linguagem, como é o caso da PDL sintática. O maior problema deste pressuposto é que a descrição, para o PE, de dados de investigação robustos e extensos sobre desenvolvimento sintático em crianças com desenvolvimento típico é ainda um trabalho em progresso, embora em franco crescimento.  Há que ter em consideração, no entanto, que estudos linguísticos para diferentes línguas têm apresentado resultados semelhantes, dado que o desenvolvimento linguístico é regulado por marcas e etapas universais \citep{SimSim1998}, o que nos permite avançar com algumas conclusões relativamente ao desenvolvimento típico de uma criança, independentemente da sua língua materna. Esse consenso não é tão evidente relativamente à natureza das alterações linguísticas em crianças com PDL sintática, devido à escassez de estudos existentes, embora se possam também encontrar desempenhos semelhantes em diferentes línguas.  No que diz respeito à produção, há estudos que indicam um maior número de erros sintáticos (\emph{cf}. (50)) e reduzida produção de palavras funcionais, tais como determinantes (\emph{cf}. (51)), preposições e conjunções, fazendo com que o enunciado de crianças com PDL sintática seja menos complexo e tenha menor extensão (\emph{cf}. (52)) \citep{Araujo2007, Silveira2011}. Os exemplos (50) a (52) foram retirados de dados de \citet{Martinsempreparacao} relativos à aplicação do teste Sin:TACS a uma criança com PDL de 7 anos e 11 meses.

\begin{enumerate}[align=left]
    \item [(50)] 	Avaliador: O botão que eu tenho na mão é pequeno.\\
	Criança: *O botão que eu tenho da mão é grande.
\item[(51)] 	Avaliador: A menina que partiu o braço tem um livro\\
	Criança: *A menina que partiu perna tem um balão.
\item[(52)] 	Avaliador: Antes de a menina dormir, ela abre a janela e vê as estrelas.\\
	Criança: Quando está a dormir vê a lua.
\end{enumerate}

Parece existir consenso entre estudos que apontam para alterações nas construções que envolvem movimento, tais como orações relativas \citep{Novogrodsky2006}, passivas \citep{Lely1996a} e interrogativas diretas parciais \citep{Friedmann2011}.\footnote{Sobre a relevância da subordinação no desenvolvimento atípico, ver \textcitetv{chapters/13}.}

No que diz respeito às relativas e às interrogativas, há estudos que referem assimetrias de desempenho em tarefas de compreensão e produção consoante as propriedades específicas deste tipo de estruturas, havendo maiores alterações nas relativas e nas interrogativas de objeto do que nas de sujeito (\citealp{Friedmann2011, Costa2009, Martins2018}, \citetv{chapters/14}). É de referir que crianças com desenvolvimento típico apresentam esta mesma assimetria de desempenho em frases subordinadas relativas, em tarefas de compreensão e produção (ver, entre outros, para o PE, \citealp{Vasconcelos1991, Costa2011}); no entanto, a aquisição destas estruturas é mais tardia em crianças com PDL, podendo nunca ser completa, o que faz deste fenómeno um possível e importante marcador clínico. 

Numa situação clínica e em contexto formal, especialmente com baterias de testes mais globais, é difícil testar todas as estruturas que se espera que a criança possua. Por esse motivo, será importante incluir as estruturas mais relevantes que permitam identificar de forma mais precisa o tipo de perturbação e as alterações que a caracterizam.  Para esse efeito, a sincronização do conhecimento crescente que temos acerca dos fenómenos sintáticos que caracterizam uma determinada perturbação da linguagem com a integração desse conhecimento em itens de avaliação que testam esses mesmos fenómenos é essencial.

Para além da importância de se conciliar teoria sintática e estudos sobre perturbação da linguagem em instrumentos de avaliação \citep{Costa2016}, há também que garantir que a ausência de uma resposta ou uma produção desviante por parte da criança face a um determinado estímulo se deve, acima de tudo, a desenvolvimento sintático comprometido e não à influência de outros fatores. Talvez não seja possível retirar conclusões de resultados obtidos através de métodos que não são convergentes, mas esses resultados serão, naturalmente, mais fiáveis se forem coincidentes com os de outros estudos, no sentido em que é mais provável que reflitam conhecimento sintático e não outras variáveis quaisquer.

Tal como referido anteriormente, os testes de avaliação sintática para o PE utilizam diversas técnicas, desde tarefas de juízos de gramaticalidade a técnicas de produção elicitada, como o completamento de frases e a imitação (sem e com variação). 

As técnicas de elicitação requerem que a criança seja bastante verbal. De forma a encorajar a criança a produzir a estrutura-alvo, é essencial que ela tenha uma razão para comunicar com o avaliador ou corre-se o risco de a tarefa não ser suficientemente motivadora para elicitar a estrutura pretendida, por ausência de um contexto comunicativo.

Apesar de, por um lado, os testes formais serem de rápida aplicação e cotação, disponibilizando informação estandardizada, eles são, por outro lado, e devido precisamente a essas características, descontextualizados, requerendo, por vezes, um esforço de reflexão sobre o conhecimento gramatical, o que poderá ser um desafio para crianças com perturbações da linguagem \citep{Marinellie2006}. Além disso, crianças mais novas (mesmo com desenvolvimento típico) têm dificuldade em produzir frases a que não conseguem atribuir qualquer significado. Esta última observação é importante, visto que os testes que avaliam, por exemplo, consciência sintática, através de tarefas de juízos de gramaticalidade, mas não apresentam um contexto comunicativo adequado podem levar a que a criança se desmotive e não revele aquilo que é capaz de fazer. De acordo com \citet{Vinther2002}, estabelecer um contexto com a ajuda de imagens ou objetos poderá dar a conhecer aos sujeitos a necessária intenção comunicativa, levando-os a esquecerem-se da sua própria condição de “sujeitos de avaliação” para entrarem no jogo de “alguém dizer alguma coisa a alguém”. O teste Sin:TACS, por exemplo, que utiliza, na maioria dos itens, a técnica de imitação elicitada, depende do contexto e dos materiais usados, permitindo que as crianças se foquem no conteúdo dos itens e garantindo que a produção da estrutura-alvo reflete uma reconstrução do estímulo e não apenas uma imitação passiva do mesmo. 

Num teste de avaliação, uma estrutura sintática não pode existir por si só, sendo necessário garantir a interação com outras componentes da gramática e com aspetos extralinguísticos. Assim, e para além do contexto comunicativo, há a considerar que o conhecimento lexical, o fonológico e o semântico, bem como o comprimento da frase (medida em palavras, sílabas ou morfemas), a atenção, a memória e até mesmo o método utilizado na avaliação poderão influenciar a resposta da criança. Isto significa que, na construção de um instrumento de avaliação sintática, é crucial reduzir a influência desses fatores, garantindo-se, tanto quanto possível, que a não produção de uma estrutura se deve exclusivamente a um défice de conhecimento sintático. Por exemplo, durante as fases iniciais de construção de itens para o teste Sin:TACS, e de forma a testar a produção de um clítico acusativo em próclise, foi criado, através de uma imagem, um contexto em que os pais estavam a brincar às escondidas com o filho, mas não o tinham encontrado. Desse modo, quando confrontada com o estímulo entre parênteses retos no exemplo (53), a criança teria de completar uma frase, produzindo o segmento em itálico. No entanto, a produção do clítico masculino (-\emph{o}) desencadeou o contacto entre dois segmentos sonoros próximos: a semivogal do ditongo nasal [ɐ̃w̃] do advérbio de negação e a vogal [u] correspondente ao clítico, fazendo com que fosse difícil para o examinador determinar se a criança teria omitido o clítico.

\begin{enumerate}[align=left]
    \item [(53)] [A mãe parou de procurar porque também] \emph{não o viu}.  
\end{enumerate}

Nas etapas seguintes de construção do teste, a imagem foi alterada: os pais passaram a procurar uma filha. Assim, a opção por um clítico do género feminino tornou-o saliente, sendo claramente percetível se a criança o omitiu ou não (\emph{cf}. (54)).

\begin{enumerate}[align=left]
    \item [(54)] [A mãe parou de procurar porque também] \emph{não a viu}.
\end{enumerate}

Como referido anteriormente, questões cognitivas subjacentes ao processamento linguístico, como a memória a curto prazo ou a memória de trabalho, podem ter influência no desempenho sintático. \citet{TagerFlushberg2000} refere que, com frequência, os testes estandardizados aumentam o comprimento das frases para as tornar mais complexas ou para apresentar mais opções de resposta. Como consequência, a não produção ou a não compreensão de uma determinada estrutura-alvo podem dever-se não necessariamente à sua complexidade linguística, mas sim a uma sobrecarga da memória ou da atenção. Este aspeto torna-se ainda mais relevante considerando que crianças com perturbações da linguagem não só têm um défice de conhecimento linguístico como também de memória a curto prazo, o que as coloca numa posição duplamente desvantajosa \citep{Vance2008}. É de notar que, na aplicação do teste Sin:TACS, nem sempre se verificou uma correlação entre o comprimento de uma frase e o número de respostas corretas, pois alguns itens com menor extensão obtiveram um maior número de respostas incorretas do que itens com maior extensão. Por exemplo, o item 16, com 14 sílabas, que testa uma completiva verbal no modo conjuntivo (\emph{cf}. (55)), foi mais fácil de produzir do que o item 23, com quatro sílabas, que testa a produção do grau superlativo relativo do adjetivo (\emph{cf}. (56)). 

\begin{enumerate}[align=left]
    \item [(55)] Eu quero que a bola seja uma vaca.
\item[(56)] [Este] é o maior.
\end{enumerate}

Por último, “importa salientar que os avaliadores devem ser sempre melhores que os instrumentos que usam” \citep[215]{Almeida2008}. A vantagem de se obter um valor quantitativo em tipos de avaliação formal é a de que se pode comparar, rapidamente e de uma forma fiável e válida, desempenhos sintáticos intra e entre grupos etários. No entanto, um desafio mais complexo será traduzir esse valor numérico em considerações sobre os mecanismos linguísticos que estão subjacentes a esses desempenhos, o que implica interpretar os resultados não olhando exclusivamente para o somatório das cotações obtidas. Uma análise qualitativa dos erros produzidos pelas crianças permite obter uma descrição pormenorizada do tipo de respostas dadas, revelando-se informação valiosa para a constituição de um perfil sintático e, consequentemente, para a elaboração de um diagnóstico mais preciso.

\section{Conclusão}
Um dos principais objetivos de investigação em aquisição da sintaxe tem sido identificar o percurso de desenvolvimento de uma criança, desde o momento de emergência das primeiras estruturas até à sua estabilização. Na prática clínica, têm sido utilizadas diferentes técnicas de recolha de dados, desde as mais informais, como a análise do discurso espontâneo, até às mais estruturadas, incluídas em testes mais formais e/ou estandardizados e delineadas de forma a avaliar estruturas sintáticas específicas. 

Apesar de os testes estandardizados permitirem obter um perfil sintático de forma relativamente rápida, são vários os desafios que colocam. De forma a estabe\-lecer-se um diagnóstico, é essencial que os instrumentos estejam validados entre sujeitos sem e com as perturbações que se pretende diagnosticar, pelo que a interação entre teoria linguística e linguística clínica é fundamental. No entanto, e para que essa interação funcione, é necessário não só um conhecimento robusto da língua que permita identificar comportamentos linguísticos associados a um perfil de desenvolvimento típico e construir instrumentos linguisticamente mais controlados, como também o conhecimento de marcadores clínicos que caracterizam diferentes tipos de perturbação sintática \citep{Costa2016}. Apesar de o diálogo entre linguistas, terapeutas da fala e educadores se encontrar numa fase de grande abertura, há ainda um caminho a percorrer para a realização de mais estudos que permitam estabelecer diferenças entre desenvolvimento típico e patologias da linguagem, condição primordial para que um diagnóstico diferencial seja produzido e para que o respetivo planeamento terapêutico seja delineado com maior probabilidade de sucesso.


\section*{Agradecimentos}
Este trabalho é financiado por fundos nacionais através da FCT – Fundação para a Ciência e a Tecnologia, I.P., no âmbito do projeto UIDB/00214/2020 (Centro de Linguística da Universidade de Lisboa).



{\sloppy\printbibliography[heading=subbibliography,notkeyword=this]}
\end{document}
