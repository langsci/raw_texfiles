\documentclass[output=paper,colorlinks,citecolor=brown]{langscibook}
\ChapterDOI{10.5281/zenodo.7233213}
\title{Assessment and intervention for children with speech sound disorders}
\author{Sharynne McLeod\affiliation{Charles Sturt University, Australia} and Sarah Verdon\affiliation{Charles Sturt University, Australia} and  Nicole McGill\affiliation{Charles Sturt University, Australia}}
\abstract{Speech-language pathologists frequently work with children with speech sound disorders (SSD) to increase their perception and production of speech sounds and intelligibility so that they enhance their participation in society. Most children with SSD have no known cause. There are a number of theories regarding children’s speech acquisition. Children’s speech acquisition begins at birth and researchers have demonstrated that most children’s speech is intelligible by 5 years of age. A range of assessment and intervention tools are available for monolingual speakers of languages such as English and Portuguese. Bilingual speakers may demonstrate cross-linguistic transfer, which should not be interpreted as SSD.}

\IfFileExists{../localcommands.tex}{
   \addbibresource{../localbibliography.bib}
   \usepackage{langsci-optional}
\usepackage{langsci-gb4e}
\usepackage{langsci-lgr}

\usepackage{listings}
\lstset{basicstyle=\ttfamily,tabsize=2,breaklines=true}

%added by author
% \usepackage{tipa}
\usepackage{multirow}
\graphicspath{{figures/}}
\usepackage{langsci-branding}

   
\newcommand{\sent}{\enumsentence}
\newcommand{\sents}{\eenumsentence}
\let\citeasnoun\citet

\renewcommand{\lsCoverTitleFont}[1]{\sffamily\addfontfeatures{Scale=MatchUppercase}\fontsize{44pt}{16mm}\selectfont #1}
  
   %% hyphenation points for line breaks
%% Normally, automatic hyphenation in LaTeX is very good
%% If a word is mis-hyphenated, add it to this file
%%
%% add information to TeX file before \begin{document} with:
%% %% hyphenation points for line breaks
%% Normally, automatic hyphenation in LaTeX is very good
%% If a word is mis-hyphenated, add it to this file
%%
%% add information to TeX file before \begin{document} with:
%% %% hyphenation points for line breaks
%% Normally, automatic hyphenation in LaTeX is very good
%% If a word is mis-hyphenated, add it to this file
%%
%% add information to TeX file before \begin{document} with:
%% \include{localhyphenation}
\hyphenation{
affri-ca-te
affri-ca-tes
an-no-tated
com-ple-ments
com-po-si-tio-na-li-ty
non-com-po-si-tio-na-li-ty
Gon-zá-lez
out-side
Ri-chárd
se-man-tics
STREU-SLE
Tie-de-mann
}
\hyphenation{
affri-ca-te
affri-ca-tes
an-no-tated
com-ple-ments
com-po-si-tio-na-li-ty
non-com-po-si-tio-na-li-ty
Gon-zá-lez
out-side
Ri-chárd
se-man-tics
STREU-SLE
Tie-de-mann
}
\hyphenation{
affri-ca-te
affri-ca-tes
an-no-tated
com-ple-ments
com-po-si-tio-na-li-ty
non-com-po-si-tio-na-li-ty
Gon-zá-lez
out-side
Ri-chárd
se-man-tics
STREU-SLE
Tie-de-mann
}
   \boolfalse{bookcompile}
   \togglepaper[1]%%chapternumber
}{}

\begin{document}
\maketitle

\begin{otherlanguage}{english}
\section{Introduction}
Children are the future leaders, citizens, and shapers of our world. Underpinning children’s current and future ability to participate effectively in daily life are the foundational skills of communication. To communicate is a basic human right \citep{McLeod2018} underlying children’s ability to learn and become literate, interact with others within their communities, obtain employment and build a career, and contribute effectively to society into the future.


\subsection*{Who are children with SSD?}
Early childhood is a time of rapid growth and development when children learn to intelligibly speak the sounds and words of the languages they are exposed to, enabling them to become competent communicators. For some children, however, this process does not come easily, and they have difficulty learning to pronounce sounds and words. The umbrella term used to describe the speech difficulties experienced by these children is speech sound disorders (SSD).


\subsection*{What is SSD?}
SSD is a type of communication impairment, where children present with a combination of difficulties in any of the following areas of their speech \citep{MultilingualChildren’sSpeech2012}:

\begin{itemize}
    \item Perception: the ability to perceive differences between speech sounds and words (e.g., \emph{pin} vs \emph{bin}).
    \item	Production: the articulation or motor production of speech sounds (e.g., interdental lisp /s/ \ding{213} [θ]).
\item	Phonological representation: the ability to organise sounds in their minds to produce groups of speech sounds.
\item	Phonotactics: the ability to produce syllable and word shapes, such as multisyllabic words (e.g., \emph{hospital}).
\item	Prosody: the use of appropriate stress, tones, rhythm, and intonation during speech.
\end{itemize}
SSD may impact children’s intelligibility and listeners’ perceptions regarding the acceptability of children’s speech productions \citep{MultilingualChildren’sSpeech2012}. For most children with SSD, there is no known cause or origin for their speech difficulties. Some sources such as the Diagnostic and Statistical Manual of Mental Disorders (DSM-5) \citep{APA2013} exclude children with a known cause from the definition of SSD. For the purposes of this chapter, these children are included under the umbrella definition of SSD. Some children may only have SSD and others might experience challenges in other areas of their development, such as language or motor skills.


\subsection*{Types of SSD}
According to \citet{McLeodBaker2017}, types of SSD stem from two main etiologies: (1) impairment of the phonological system (2) impairment of the motor speech and there are five different types of SSD, each with their own unique characteristics. Table \ref{tab:cap1tab1} provides a description of these categories.



\begin{tableeng}
    \centering
    \caption{Types of speech sound disorders}
    \begin{tabularx}{\textwidth}{QQQQ}
    \lsptoprule
 \textbf{Basis} & \textbf{Name of}&\textbf{Description}&\textbf{Example}\\
 & \textbf{SSD}&&\\
 \midrule
 %\addlinespace
 \multirow{2}{*}{\shortstack[l]{Phonological\\ (difficulties\\ with\\ organizing\\ speech into\\ patterns of\\ sounds)}} & Phonological
impairment & Multiple
{production} errors that follow predicable patterns&	Replacement of one group of sounds with another (e.g., replacing velars with alveolars)\\%\cline{2-4}
% \addlinespace
\tablevspace
 &  Inconsistent speech disorder&	Inconsistent phonological errors&	Inconsistent
{productions} of words\\
 \midrule
 %\addlinespace
 \multirow{3}{*}{\shortstack[l]{Motor speech\\ 
(difficulties\\ with\\ coordinated\\ movements of\\ the lips,\\ tongue, teeth,\\ palate and\\ respiratory\\ system)}} & Articulation impairment & Difficulty with production of
{specific} consonants &
%\begin{minipage}[t]{\linewidth}\begin{itemize}[nosep,after=\strut] %\item interdental lisp {/s/ \ding{213} [θ]}; 
%\item lateral lisp /s/ \ding{213} [ɬ] \item	/ɹ/ \ding{213} [w]
%\end{itemize}\end{minipage}
\makecell[tl]{\textbullet\, interdental lisp\\\;{/s/ \ding{213} [θ]}\\
\textbullet\, lateral lisp\\\;/s/ \ding{213} [ɬ]\\
\textbullet\, /ɹ/ \ding{213} [w]}\\%\cline{2-4}
%\addlinespace
\tablevspace
& Childhood apraxia of speech (CAS)&	Difficulty
{planning} and
{programming}
{movement} sequences and controlling prosody	&Difficulty with motor
{planning} and stress,
{imitation}, and production of consonants and vowels\\%\cline{2-4}
%\addlinespace
\tablevspace
& Childhood dysarthria&	Difficulty with sensorimotor
control in planning and executing speech &	Weakness,
incoordination, difficulty with phonation, stress, accuracy. \\
\lspbottomrule
    \end{tabularx}
    \label{tab:cap1tab1}
\end{tableeng}


For most children, SSD is of unknown origin \citep{Shriberg2005}. However, for some SSD the cause is known. This includes SSDs that accompany hearing loss \citep{Crowe2014}, congenital conditions and craniofacial anomalies such as cleft lip ± palate \citep{Cronin2019}, and SSD that result from syndromes such as Fragile X syndrome and Down syndrome \citep{Barnes2009}. Children may possess characteristics of multiple SSD types of both known and unknown origin. 


\subsection*{Prevalence of childhood SSD}
SSD is a prevalent condition in childhood \citep{Law2000}. Prevalence rates vary among international research studies due to differences in participant samples (e.g., age, population vs probability sampling), data collection methods, definitions, and testing cut point/criteria. However, the reported prevalence of childhood SSD ranges from 1.06\% \citep{McKinnon2007} to 20.5\% \citep{Tuomi1977}. Children with SSD form a high proportion of speech-language pathologists (SLPs’) caseloads throughout the world \citep{Broomfield2004, Mullen2010, Oliveira2015}.


\subsection*{Impact of SSD on children’s lives}
Children often demonstrate resilience and persistence in social interactions and become skilled in using nonverbal forms of communication (e.g., gestures) to help express their wants and needs. They can make friends and successfully and happily participate in daily life \citep{McLeodBaker2017}. However, SSD can have negative impacts on children especially if timely SLP support is not received in early childhood. For example, children may have difficulty with social interaction and making and keeping friends, experience bullying/teasing, and feel sad and frustrated with their SSD \citep{McCormack2011, McLeod2013}. For example, BJ stated: “I often felt left out because I wasn’t able to talk with other people, I wasn’t able to tell other people my thoughts or if I needed something. It was heartbreaking because I knew what I wanted to say, but I couldn’t say it’’ \citep[46]{Carrigg2015}.

Children with SSD may also experience difficulties in literacy and mathematical thinking, are more likely than their typically developing peers to drop out of school, and may go on to have difficulty acquiring and maintaining employment \citep{Anthony2011, Felsenfeld1994, Robertson1998, Ruben2000, Tambyraja2020}. 

SSD can also impact the lives of family members. Siblings may act as interpreters for children with SSD in the community, and parents may “battle” to obtain services and support for their children \citep{Barr2008, McCormack2012, McGill2020}. However, all children and families are unique with individual strengths, values, and interests. It is important to listen to and consider their experiences and perspectives when working with children with SSD and their families.

\section{Theoretical models underpinning work with children with SSD}
There are multiple theories that have been proposed to explain how children develop their sound systems and thus what may be occurring when children experience SSD. Many theories draw upon the following concepts in relation to sounds, and features of speech such as syllable shapes and stress patterns:

\begin{itemize}
    \item \emph{Naturalness}: Easier, earlier developing sounds or features that are more common across languages. Sometimes referred to as unmarked. 
\item	\emph{Markedness}: More difficult, later acquired sounds or features that are less common across languages. 
\end{itemize}

\subsection*{Generative phonology}
First proposed by \citet{Chomsky1968}, generative phonology is a theory of human sound structures that suggests that there is an abstract underlying representation of the target phonology of the ambient language in the brain and that the surface representation, that is, the realization of phonemes in speech is influenced by two key concepts:

\begin{itemize}
    \item Underlying phonological rules (e.g., voice, manner, place, naturalness, mar\-kedness)
\item	Information from other linguistic levels (i.e., the context of the production such as semantics and syntax)
\end{itemize}

The way phonological representations are stored in the brain may be different from how these phonemes are physically realized in speech. The theory of generative phonology has led to the common notation of placing the target phoneme in slashes / / while the realization of the phoneme in speech is represented using square brackets [ ]. For example, when a child is fronting the word \emph{keep} in English, this production would be written as: /kip/ \ding{213} [tip]

\begin{tableeng}
\small
    \caption{Phonological patterns in speech acquisition}
    \begin{tabularx}{\textwidth}{QlQQ}
    \lsptoprule
    \textbf{Phonological pattern}&	\textbf{Example}& 	\makecell[tl]{\textbf{Typical/}\\\textbf{atypical}}	&\textbf{Age when resolves}\\
\midrule
%\addlinespace
Fronting
(e.g., velars produced as alveolars) &	\makecell[tl]{\emph{casa}\\ /kazɐ/ \ding{213} [tazɐ]}&	Typical in Pt
\& En &	3.5 years\\
\tablevspace
Backing (e.g., alveolars produced as velars)&	\makecell[tl]{\emph{doce}\\/dos/ \ding{213} [ɡos]}&	Atypical in Pt
\& En	&Uncommon, may be indicative of SSD\\
\tablevspace
Stopping (e.g., fricatives produced as plosives/stops)&	\makecell[tl]{\emph{sumo}\\/sumu/ \ding{213} [tumu]}&	Typical in Pt \& En&	3--5 years depending on sound\\
\tablevspace
Cluster reduction (1--2 elements are deleted)&	\makecell[tl]{\emph{flor}\\/floɾ / \ding{213} [foɾ]}&	Typical in Pt \& En &	4--5 years\\
\tablevspace
Reduplication (repetition of a syllable)&	\makecell[tl]{\emph{bolo}\\/bolu/ \ding{213} [bobo]}&
	Typical in Pt \& En& 	3 years\\
%	\hline
%	\addlinespace
\tablevspace
Weak syllable deletion (unstressed syllable(s) are deleted from a polysyllabic word)&	\makecell[tl]{\emph{banana}\\/bɐˈnɐnɐ/ \ding{213} [nɐnɐ]}&	Typical in Pt \& En&
	4 years\\
%	\hline
%	\addlinespace
\tablevspace
Final consonant deletion (the final consonant is deleted)&	\makecell[tl]{\emph{dog}\\/doɡ/ \ding{213} [do]}&	\makecell[tl]{Typical in En\\
Uncommon in\\ Pt}&
	3 years in English\\
%	\hline
%\addlinespace
\tablevspace
Initial consonant deletion (the initial consonant is deleted)&	\makecell[tl]{\emph{bola}\\/bɔlɐ/ \ding{213} [ɔlɐ]}&	Atypical in En \&
Pt&	May be indicative of SSD\\
\lspbottomrule
    \end{tabularx}
    \label{tab:cap1tab2}
\end{tableeng}


\clearpage

\subsection*{Natural phonology}
\citet{Stampe1979a} proposed that children follow similar natural developmental patterns of producing sounds in easier ways during early speech development. This means that children may substitute more difficult sounds with easier sounds or omit sounds altogether while they are still learning their language. This theory is called natural phonology because processes are observed to be a common feature of typical development among children and often resolve naturally as children’s speech matures. These patterns are often referred to as phonological processes or phonological patterns \citep{Bowen2011, Hodson2007}. Many speech-language pathologists use these patterns as a way to classify speech sound errors and to diagnose SSD (e.g., \citealp{Hodson2007}). These patterns can vary depending on the language(s) that the child speaks. Some phonological processes are not typical within certain languages and may be an indicator of SSD. Additionally, if a phonological process persists past a certain age it may be an indicator of SSD. See Table \ref{tab:cap1tab2} for a description of some common phonological patterns, whether they are typical/atypical in Portuguese and English and the age at which these patterns typically resolve. Data for Portuguese children are based on \citet{Jesus2015} and data for English speaking children are based on \citet{Bowen2011}. These data are based on monolingual children, not bilingual speakers of Portuguese and English.




\subsection*{Nonlinear phonology}
Generative and natural phonology are considered to be theories of linear phonology. In contrast, nonlinear phonology, as described by \citet{Bernhardt2000} considers the phonological system as a whole, examining the hierarchical nature of the relationships between phonological units in the sound system rather than just looking at speech sounds. The hierarchy consists of the \emph{prosodic tier} (encompassing the phrase tier, the word tier, the foot tier, the syllable tier, the onset-rime tier, the skeletal tier, the timing tier) and the \emph{segmental tier} (encompassing features of the different segments being produced). Difficulties can occur at any one of these tiers. Analysis of speech using nonlinear phonology looks at patterns of errors that occur on each tier in the hierarchy to identify targets for intervention.

In nonlinear phonology, unmarked features of speech including sounds, syllable shapes and stress patterns are considered default elements of the sound system. This means they are likely to remain intact. Whereas marked features such as sounds, syllable shapes and stress patterns are considered non-default and are more likely to differ from the target production. The theory of nonlinear phonology recognizes that children’s speech development is additive (that children increase accuracy of target features over time), as opposed to other theories that suggest children learn speech in an errored way using less marked sounds and then unlearn these errors to achieve target accuracy as their speech systems mature.
\subsection*{Optimality theory}
\citet{Prince1993a} proposed optimality theory on the basis that there are two levels of sound representation: input (underlying representation) and output (surface realization). In contrast to generative phonology where rules determine surface representation, in optimality theory, constraints determine surface representation:

Type 1: \emph{Markedness constraints} -- Marked features may be excluded from the surface realization or deviate from the underlying representation as they are less common or more difficult to produce.

Type 2: \emph{Faithfulness constraints} -- Unmarked sound features are preserved in speech and remain faithful to the target production.

Optimality theory proposes that as the speech system of typically developing children matures their faithfulness to the adult target production increases.


\section{Typical speech acquisition}
Children’s speech acquisition begins at birth and most aspects of speech production are mastered by 5 years of age. Many elements are considered when describing children’s speech acquisition: “oral mechanism, perception, intelligibility, phonetic inventory, syllable and word shape inventory, mastery of consonants and vowels, percentage of consonants/consonant clusters/vowels correct, common mismatches, phonological processes, syllable structure, prosody, metalinguistic and phonological awareness skills” \citep[191]{McLeodBaker2017}. Children’s speech acquisition is influenced by age, and some studies have shown that sex (gender), maternal education, socio-economic status, and language spoken, may have an influence (whereas, other studies have shown these factors do not have an influence) on the rate of speech acquisition \citep{McLeodBaker2017}. There is cross-linguistic evidence that input frequency, functional load, and articulatory complexity influence children’s speech acquisition across a range of languages \citep{Ingram2012}. Consequently, consonants that are more difficult to pronounce (e.g., fricatives) that are rarely produced in a language may be later to be acquired. The emergence approach to speech acquisition \citep{Davis2013} provides a theory to summarize the influences of (1) biologically-motivated intrinsic capabilities (maturation, cognition, perception and production), (2) socially-motivated interactional capabilities with communication partners (e.g., turn taking and intention) and (3) the influence of the extrinsic context (e.g., ambient phonology).

One of the most commonly researched areas of children’s speech acquisition is the age of acquisition of consonants. \citet{McLeod2018a} reviewed typical consonant acquisition across 27 languages and more than 26,000 children. They found that regardless of the language spoken, 5-year-old children had acquired the majority of the consonants within their ambient language and produced at least 93 percent of consonants correctly. Difficult consonants were trills, flaps, fricatives, and affricates; whereas plosives and nasals, typically were acquired earlier. However, “there was an interaction between place and manner where plosives and nasals produced with anterior tongue placement were acquired earlier than anterior trills, fricatives, and affricates” \citep[1546]{McLeod2018a}. Their cross-linguistic review included two studies of Brazilian Portuguese. The first study was by Salviano Santini (1995) who researched consonant and consonant cluster production of 192 monolingual Brazilian Portuguese speakers aged 2;0-6;8 years. The second study was by \citet{Silva2012} who studied two groups of 240 children in Brazil aged 3;0 to 7;9 months. Both of these studies also provided normative data for the percentage of consonants correct (PCC). Recently, two additional studies of Brazilian Portuguese consonant acquisition have been published \citep{Ceron2017, Ceron2020}. These papers reported on different aspects of a study of 733 typically developing Brazilian Portuguese-speaking children (3;0-8;11 years), using the spontaneous naming task from the Phonological Assessment Tool (Instrumento de Avaliação Fonológica INFONO). They indicated that common phonological processes in younger children were: “cluster reduction, liquid gliding, fricative deletion–coda, and weak-syllable deletion”; and that /ɾ/ and complex onsets were acquired later than other consonants. They also confirmed the findings of \citet{McLeod2018a} that nasals and stops/plosives were acquired before fricatives and liquids. 


\subsection*{Multilingual acquisition}
Children’s language background plays an important role in the accurate identification of SSD. When children speak more than one language their speech production will be different from a monolingual speaker. This is because different language systems influence each other. For example, if a child learns to speak Portuguese first, and then learns to speak English as a second language, their speech development and production will likely be different from what is observed in a monolingual Portuguese-speaking child. In particular, sounds that are not shared between the languages, such as /ʎ/, may be more difficult for the multilingual child to produce or may develop later than by a monolingual child \citep{McLeod2018a}. Children may also use different stress patterns in their speaking if these are different between their first and additional languages. These differences are known as \emph{cross-linguistic transfer} and are not indicative of SSD, but rather a language difference arising from being a multilingual speaker. Multilingual children’s speech acquisition is influenced by:


\begin{enumerate} [align=left]
\item [] extrinsic factors from their language environment
\begin{itemize}
    \item ambient phonology
\item	cross-linguistic/cross-dialectal transfer
\end{itemize}
\item [] intrinsic factors
\begin{itemize}
    \item development/Maturation
\item	assimilation
\item	error productions (related to SSD) \citep{McLeod2021}
\end{itemize}
\end{enumerate}


Therefore, when working with multilingual children, differential diagnosis is needed when determining whether children’s productions are a result of SSD or a (typical) language difference. 


\section{Assessment and analysis of children’s speech}

When a child is suspected to have SSD there are many different aspects to consider during an assessment. A speech assessment typically includes: referral, case history, assessment of speech production, intelligibility, stimulability, speech perception, phonological processing, language, hearing, oral structure and function, nonverbal intelligence, participation and can also include pre-literacy skills \citep{McLeodEtAl2017}. After the assessment, the speech-language pathologist undertakes analysis of the speech sample, discusses the results with the child, family and others (e.g., teachers) to formulate a diagnosis, and if relevant sets goals for intervention. Considering families’ worldviews and cultural frames of reference is essential for service provision that is meaningful and relevant.

Assessments should target all of the child’s languages because languages differ phonologically and phonotactically, and children may not exhibit the same skills in each language due to differences in use, proficiency and cross-linguistic transfer \citep{McLeodEtAl2017}. Across the world, there are many different assessments that have been created to consider children’s speech production. Two websites contain lists of speech assessments for children in many different languages:

\begin{itemize}
    \item Multilingual Children’s Speech: \\
    \url{https://www.csu.edu.au/research/multilingual-speech/speech-assessments}
\item	Crosslinguistic Project Assessment Tools:
\\
\url{http://blogs.ubc.ca/crosslinguisticprojectmaterials/}
\end{itemize}

\citet{McLeod2014} considered children’s speech assessments in 17 languages and found that most were developed for monolingual children, particularly English-speaking children. Some of the speech assessments and analyses available in Portuguese that have been created for Portuguese-speaking children in Portugal or Brazil include:

\begin{itemize}
    \item Teste de Linguagem infantil: Nas áreas de fonologia, vocabulário, fluência e pragmática (ABFW) \citep{Andrade2000};
    \item	Avaliação Fonológica da criança (AFC) \citep{Yavas1991a};
\item	Teste Fonético-Fonológico-ALPE (TFF-ALPE) \citep{Mendes2009}; 
\item	Automatic Phonological Analysis Tools (APAT) \citep{Saraiva2017} for use with ALPE;
\item	Portuguese Phonology Test plus Nonlinear Scan Analysis form \citep{Bunney2012}.
\end{itemize}


The Intelligibility in Context Scale (ICS) (\url{https://www.csu.edu.au/research/multilingual-speech/ics}) is a 7-item parent-report scale that can be used as a screening tool to consider children’s intelligibility with different communication partners. Four-to-five-year-old children who achieve a score below 4 out of 5 may be considered for further speech assessment. The ICS has been translated into over 60 languages including European Portuguese and Brazilian Portuguese (Escala de Inteligibilidade em Contexto). The validity and reliability of the ICS has been reported in 18 studies of over 4235 children from 14 countries and 14 languages \citep{McLeod2020} including Portuguese-speaking children in Portugal \citep{Lousada2019}.


\section{Interventions for children with SSD}
\largerpage[2]
There are a range of intervention (or therapy) approaches that SLPs use when working with children with SSD. A survey of 88 Portuguese SLPs \citep{Oliveira2015} found that commonly used intervention approaches include: 


\begin{itemize}
    \item Articulation interventions; 
\item	Minimal pairs intervention;
\item	Phonological awareness intervention.
\end{itemize}


Other intervention approaches used by some Portuguese SLPs include: nonlinear intervention, cycles therapy and the core vocabulary approach. A number of intervention approaches are summarized in Table \ref{tab:cap1tab3}. Information is included about suitable sub-types of SSD for each approach. References for further reading are also included, reflecting the highest level of evidence available supporting the intervention (e.g., meta-analysis, systematic review, randomized controlled trial). Where multiple studies were available with the same level of evidence, more recent studies were referenced. More details about these and other intervention approaches for children with SSD can be found in \citep{Williams2021}.

The use of evidence (evidence-based practice) is important when selecting appropriate intervention approaches to use. \citet{Lousada2013} indicated that SLPs in Portugal favoured articulation-based approaches, so undertook a randomized controlled trial to compare phonological and articulation-based ap\-proa\-ches for 14 Portuguese children (4;0–6;7 years) with phonologically based SSD. Both intervention approaches resulted in increased (PCC); however, greater gains were achieved from the phonological intervention approach that combined expressive phonological tasks, phonological awareness, listening and discrimination activities.

\citet{Baker2018} developed a taxonomy of the different elements of phonological interventions. Similarities and differences exist across intervention approaches regarding goals, teaching moments, contexts, and procedural issues \citep{Baker2018}. There is flexibility in the implementation of some interventions, and it remains unclear exactly which elements actively contribute to treatment effects for children with SSD \citep{Baker2018}. However, intensity is an important factor and higher intensity intervention is generally recommended (e.g., regular sessions and high doses of teaching moments per session; \citealp{Baker2012}). Parents are often involved in intervention with children with SSD \citep{Oliveira2015}. If parents are trained to undertake home practice with children, this can increase intervention intensity and may improve children’s outcomes \citep{Sugden2018}. 




\begin{tableeng}
\small
\caption{Intervention approaches for children with SSD}
    \label{tab:cap1tab3}
\begin{tabularx}{\textwidth}{QlQ}
\lsptoprule
\textbf{Intervention approach (example)}&	\textbf{Sub-type of  SSD}& 	\textbf{Suggested reference}  \\
\midrule
Articulation Interventions& 	Articulation impairment&	\citet{Skelton2004}\\
\tablevspace
Biofeedback (ultrasound, electropalatography)&	Articulation impairment&	\citet{Preston2014}\\
\tablevspace
Complexity Approach (maximal oppositions)&	Phonological impairment&	\citet{Rvachew2001}\\
\tablevspace
Core Vocabulary	&Inconsistent SSD&	\citet{Broomfield2011}\\
\tablevspace
Cycles Approach&	Phonological impairment&	\citet{Almost1998}\\
\tablevspace
Digital Tools (T2T Speech and Language Therapy)&	All&	\citet{Jesus2019}\\
\tablevspace
Phonological Awareness Intervention&	All&	\citet{Hesketh2007}\\
\tablevspace
Minimal Pairs Intervention&	Phonological impairment&	\citet{Dodd2008a}\\
\tablevspace
Multiple Oppositions Intervention&	Phonological impairment&	\citet{Ceron2013}\\
\tablevspace
Nuffield Centre Dyspraxia Programme	&CAS&	\citet{Murray2015}\\
\tablevspace
Prompts for Restructuring Oral Muscular Phonetic Targets &	CAS&	\citet{Namasivayam2020a}\\
\tablevspace
Psycholinguistic Intervention&	All&	\citet{Speake2012}\\
\tablevspace
Speech Motor Programming (Rapid Syllable Transition Treatment ReST)&	CAS&	\citet{Murray2015}\\
\tablevspace
Speech Perception (Speech Assessment and Interactive Learning System SAILS)&	Phonological impairment&	\citet{Rvachew2004}\\
\tablevspace
Stimulability Approach&	Phonological impairment&	\citet{Miccio1996}\\
\lspbottomrule
\end{tabularx}
    \parbox{\textwidth}{\raggedright\footnotesize\noindent{Note: CAS, childhood apraxia of speech}}
\end{tableeng}
\clearpage

There is a lot of evidence to indicate that intervention for children with SSD can be successful resulting in intelligible speech \citep{Baker2011} enabling children to be the future leaders, citizens, and shapers of our world. As one child stated:

\begin{quote}
  I believe that even though we are only children we can still really make a difference in the world {\ldots} \citep[147]{McCormack2018}
\end{quote}


\end{otherlanguage}
{\sloppy\printbibliography[heading=subbibliography,notkeyword=this]}
\end{document}
