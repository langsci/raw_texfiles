\documentclass[output=paper,colorlinks,citecolor=brown,booklanguage=portuguese]{langscibook} 
\ChapterDOI{10.5281/zenodo.7233227}
\title{Desenvolvimento da prosódia infantil: Avaliação e intervenção}
\author{Sónia Frota\affiliation{Universidade de Lisboa, Faculdade de Letras, Centro de Linguística} and  Marisa Filipe \affiliation{Universidade de Lisboa, Faculdade de Letras, Centro de Linguística} and  Marisa Lousada\affiliation{Universidade de Aveiro, Escola Superior de Saúde, CINTESIS.UA} and  Maria Manuel Vidal\affiliation{Universidade de Lisboa, Faculdade de Letras, Centro de Linguística} and  Marina Vigário\affiliation{Universidade de Lisboa, Faculdade de Letras, Centro de Linguística}}


\abstract{Este capítulo é dedicado ao desenvolvimento prosódico e à avaliação e intervenção prosódicas em contexto clínico. Partindo da caracterização do desenvolvimento prosódico típico, a par do papel da prosódia no processo de aquisição da língua, descrevem-se formas de prosódia atípica em diferentes populações clínicas. De seguida, apresenta-se uma revisão crítica dos instrumentos de avaliação prosódica disponíveis, nomeadamente os desenvolvidos para avaliar a prosódia do Português, aplicáveis a populações típicas e atípicas, desde os momentos iniciais da aquisição prosódica até à idade adulta. Finalmente, discutem-se programas de intervenção prosódica propostos na literatura. O capítulo termina com a identificação de áreas lacunares, bem como de desenvolvimentos recentes promissores no domínio do rastreio de alterações prosódicas e da avaliação de estratégias de intervenção prosódica.


\textbf{Palavras-chave}: aquisição prosódica fraseamento prosódico; proeminência; entoação; prosódia atípica; PEPS-C; APC; Proso-Quest.}

\IfFileExists{../localcommands.tex}{
   \addbibresource{../localbibliography.bib}
   \usepackage{langsci-optional}
\usepackage{langsci-gb4e}
\usepackage{langsci-lgr}

\usepackage{listings}
\lstset{basicstyle=\ttfamily,tabsize=2,breaklines=true}

%added by author
% \usepackage{tipa}
\usepackage{multirow}
\graphicspath{{figures/}}
\usepackage{langsci-branding}

   
\newcommand{\sent}{\enumsentence}
\newcommand{\sents}{\eenumsentence}
\let\citeasnoun\citet

\renewcommand{\lsCoverTitleFont}[1]{\sffamily\addfontfeatures{Scale=MatchUppercase}\fontsize{44pt}{16mm}\selectfont #1}
  
   %% hyphenation points for line breaks
%% Normally, automatic hyphenation in LaTeX is very good
%% If a word is mis-hyphenated, add it to this file
%%
%% add information to TeX file before \begin{document} with:
%% %% hyphenation points for line breaks
%% Normally, automatic hyphenation in LaTeX is very good
%% If a word is mis-hyphenated, add it to this file
%%
%% add information to TeX file before \begin{document} with:
%% %% hyphenation points for line breaks
%% Normally, automatic hyphenation in LaTeX is very good
%% If a word is mis-hyphenated, add it to this file
%%
%% add information to TeX file before \begin{document} with:
%% \include{localhyphenation}
\hyphenation{
affri-ca-te
affri-ca-tes
an-no-tated
com-ple-ments
com-po-si-tio-na-li-ty
non-com-po-si-tio-na-li-ty
Gon-zá-lez
out-side
Ri-chárd
se-man-tics
STREU-SLE
Tie-de-mann
}
\hyphenation{
affri-ca-te
affri-ca-tes
an-no-tated
com-ple-ments
com-po-si-tio-na-li-ty
non-com-po-si-tio-na-li-ty
Gon-zá-lez
out-side
Ri-chárd
se-man-tics
STREU-SLE
Tie-de-mann
}
\hyphenation{
affri-ca-te
affri-ca-tes
an-no-tated
com-ple-ments
com-po-si-tio-na-li-ty
non-com-po-si-tio-na-li-ty
Gon-zá-lez
out-side
Ri-chárd
se-man-tics
STREU-SLE
Tie-de-mann
}
   \boolfalse{bookcompile}
   \togglepaper[08]%%chapternumber
}{}

\begin{document}
\lehead{S. Frota, M. Filipe, M. Lousada, M. Manuel Vidal \& M. Vigário}
\maketitle

\section{Desenvolvimento prosódico e aquisição da língua}\label{sec:cap8sec1}
A prosódia é a estrutura que organiza a dimensão sonora da linguagem.\footnote{A norma adotada na escrita deste capítulo foi a do português europeu.}
% \footnote{Na sequência de restrições decorrentes do template usado para publicação na editora Language Science Press, os capítulos que integram o volume surgem com a referência [this volume], em inglês.}
Esta estrutura realiza-se através de variações de duração, amplitude e frequência fundamental no sinal de fala que não dependem da natureza das consoantes e vogais presentes nas expressões linguísticas. Tais variações estabelecem padrões de tempo, intensidade e melodia, que são utilizados nas línguas para veicular múltiplas funções e significados. Uma das grandes funções da prosódia é a demarcação de unidades ou constituintes sonoros, dividindo os enunciados em sequências que facilitam o processamento da informação e a interação comunicativa. Uma outra grande função da prosódia é a transmissão de significados comunicativos, incluindo significados linguísticos e paralinguísticos \citep{Gussenhoven2004, Ladd2008}. Entre os primeiros, temos as distinções de tipos de frase (afirmação, pergunta, pedido, ordem, etc.), o salientar de informação importante (ênfase, foco), ou mesmo contrastes de significado ao nível da palavra (como no caso particular das línguas tonais, em que a melodia faz parte da especificação de palavras ou morfemas). Entre os segundos, destacam-se significados afetivos (como (in)felicidade, surpresa), as intenções do falante (como cooperação comunicativa, agressividade), ou a caracterização do falante (idade, género, perfil socioletal ou dialetal).

A função demarcativa é obtida através do fraseamento prosódico, sinalizado predominantemente pela entoação, ritmo, proeminência e pausas. Por exemplo, a sequência \emph{uma vez chegado o professor começou o espetáculo} é potencialmente ambígua: se \emph{uma vez chegado} formar um constituinte prosódico caracterizado pela presença de um contorno entoacional, teremos o significado de que o professor começou o espetáculo; já se o constituinte prosódico for \emph{uma vez chegado o professor}, então interpretamos que alguém, mas não o professor, terá começado o espetáculo. No caso deste exemplo, o constituinte prosódico relevante é o sintagma entoacional. Na hierarquia de constituintes prosódicos que define a organização sonora de uma língua, apresentada em \REF{ex:cap8ex1}, este é um dos constituintes mais altos e, juntamente com o enunciado e o sintagma fonológico, faz parte das categorias de interface entre informação sintática e semântico-discursiva e informação prosódica. Já os constituintes palavra prosódica e grupo de palavra prosódica constituem categorias de interface entre informação morfossintática e informação prosódica. Os constituintes prosódicos abaixo do nível da palavra resultam de informação puramente fonológica. Neste capítulo, ocupamo-nos fundamentalmente dos constituintes desde a palavra prosódica ao enunciado, que correspondem ao que designamos como fraseamento prosódico \citep{Frota2000, Vigario2003, Frota2003}. O fraseamento prosódico e a (morfos)sintaxe apresentam uma relação estreita, apesar de estas estruturas serem independentes \citep{Nespor2007}. 

\eabox{\label{ex:cap8ex1}
	\begin{forest}
		[{Enunciado (U)}
			[{Sintagma entoacional (IP)}
				[{Sintagma fonológico (PhP)}
					[{Grupo de palavra prosódica (PWG)}
						[{Palavra prosódica (PW)}
							[{Pé (Σ)}
								[{ Sílaba (σ)}]
							]
						]
					]
				]
			]
		]
	\end{forest}
}

No domínio da transmissão de significados comunicativos linguísticos, as distinções entre tipos frásicos são fundamentalmente veiculadas pela entoação, enquanto a proeminência tende a desempenhar um papel crucial na expressão da ênfase e foco. Também a estrutura entoacional e a cadeia de proeminências fazem referência à estrutura de fraseamento prosódico \citep{Frota2000, Ladd2008}. A cadeia de proeminências estabelece um elemento proeminente para cada constituinte prosódico. A entoação consiste em sequências de eventos tonais ligados a elementos prosódicos proeminentes (os acentos tonais) ou a fronteiras de constituintes prosódicos (os tons de fronteira). Assim, os enunciados que resultam da sequência \emph{uma vez chegado o professor começou o espetáculo} apresentam diferente fraseamento prosódico, com consequências na cadeia de proeminências e na entoação, como ilustrado em \REF{ex:cap8ex2} e \REF{ex:cap8ex3}. Estes vários aspetos da prosódia estabelecem relações com a sintaxe, a semântica e a pragmática \citep{Wagner2010}.  


\ea\label{ex:cap8ex2}{}
          [uma  vez   chegado{\normalfont ]\textsubscript{IP}}  {\normalfont [}o   professor   começou   o  espetáculo{\normalfont ]}\textsubscript{IP}\footnotemark\\
 \hphantom{[uma  vez   cheg}*\hspace{5.4cm}*\\
 \hphantom{[uma  vez   che}L*+H H \%\hspace{3.5cm}H+L*~~~~~~L\%\\
\z 

% \eabox{
% 	\begin{tikzpicture}
% 		\matrix [matrix of nodes, inner sep=.25ex, ampersand replacement=\&, row 1/.style={font=\itshape}]
% 		{
% 			{\normalfont [}uma \& vez \& chegado{\normalfont ]\textsubscript{IP}} \& {\normalfont [}o \& professor \& começou \& o \& espetáculo{\normalfont ]\textsubscript{IP}\footnotemark} \\
% 			\& \& $^*$ \& \& \& \& \& $^*$ \\
% 			\& \& L$^*$+H H\% \& \& \& \& \& H+L$^*$L\%\\
% 		};
% 	\end{tikzpicture}
% }

\footnotetext{O elemento proeminente, ou cabeça do constituinte prosódico, é assinalado com asterisco ($*$). Os acentos tonais, ligados a elementos proeminentes, apresentam um asterisco no tom associado à sílaba tónica. Indicam-se apenas os contornos nucleares.}


\ea\label{ex:cap8ex3}{}
    uma vez chegado  o professor{\normalfont ]\textsubscript{IP}} {\normalfont [}começou  o  espetáculo{\normalfont ]\textsubscript{IP}}\\
\hphantom{uma vez chegado  o pro}*\hspace{4cm}*\\
\hphantom{uma vez chegado  o pr}L*+H H\%\hspace{2.2cm}{H+L*}~~~~~L\%
\z 


% \eabox{
% 	\begin{tikzpicture}
% 		\matrix [matrix of nodes, inner sep=.25ex, ampersand replacement=\&, row 1/.style={font=\itshape}]
% 		{
% 			{\normalfont [}uma \& vez \& chegado  \& o \& professor{\normalfont ]\textsubscript{IP}} \& {\normalfont [}começou \& o \& espetáculo{\normalfont ]\textsubscript{IP}} \\
% 			\& \& \& \& $^*$  \& \& \& $^*$ \\
% 			\& \& \& \& L$^*$+H H\% \& \& \& H+L$^*$L\% \\
% 		};
% 	\end{tikzpicture}
% }

Apesar de as grandes funções da prosódia tenderem a ser gerais entre as línguas, a sua implementação varia consideravelmente de língua para língua (ou mesmo entre variedades da mesma língua, como no Português Europeu e Português do Brasil, cf. \citealp{Frota2016}). As várias teorias das estruturas prosódicas, incluindo as estruturas entoacional e rítmica, sublinham a relação entre prosódia e outros domínios da linguagem, bem como a variação prosódica existente entre as línguas \citep{Frota2012, Frota2018a}. As frases interrogativas, por exemplo, são assinaladas por diferentes elementos entoacionais em diferentes línguas ou variedades (melodia alta ou ascendente no Inglês ou no Português Europeu, melodia baixa ou descendente no Sueco ou Português do Brasil), tal como a expressão de informação importante ou foco (altura do pico melódico no Inglês, tipo de melodia utilizada no Português, suspensão da descida melódica no Japonês). A relação entre prosódia e outros domínios da linguagem, juntamente com a saliência percetiva das propriedades prosódicas, conferem à prosódia um papel central no processo de aquisição da língua. Por outro lado, a especificidade prosódica inerente a cada língua ou variedade determina a necessidade de estudos particulares do desenvolvimento prosódico e sua avaliação \citep{Prieto2018}. 

Os bebés apresentam uma sensibilidade precoce às propriedades prosódicas da linguagem (entoação, ritmo, proeminência, fraseamento prosódico). Dada a relação entre prosódia e outras estruturas linguísticas, nomeadamente ao nível da palavra e da frase, as propriedades prosódicas poderão fornecer pistas para a aprendizagem de outros aspetos da língua. Por exemplo, o ritmo, a proeminência e o fraseamento prosódico contribuem para a descoberta das palavras no contínuo de fala \citep{Nazzi2006, Butler2018}. Diferenças de proeminência constituem pistas relevantes para a categorização de palavras em palavras lexicais (com proeminência, logo percetivamente salientes) e palavras funcionais (sem proeminência). A distribuição destas diferenças de proeminência, a par dos padrões de proeminência entoacional ao nível da frase, correlaciona-se com a ordem de palavras e fornece, assim, pistas para a aquisição deste importante aspeto da sintaxe \citep{Christophe2008, Bernard2012}.

O papel desempenhado pela prosódia no processo de aquisição da língua, juntamente com a sua função comunicativa, faz da sensibilidade à prosódia e do desenvolvimento prosódico uma área-chave do desenvolvimento linguístico infantil. A perceção de propriedades prosódicas da língua – ritmo, entoação, acento de palavra, proeminência entoacional, fraseamento prosódico –  desenvolve-se precocemente, antes do final do primeiro ano de vida, demonstrando uma perceção orientada para a prosódia da língua materna. Rapidamente, o bebé mostra conseguir explorar estas propriedades prosódicas no desenvolvimento de outras capacidades linguísticas. No que respeita à produção, durante o segundo ano de vida registam-se desenvolvimentos importantes no sistema entoacional (como a prosódia característica de diferentes tipos frásicos), nos padrões acentuais e rítmicos e no fraseamento prosódico \citep{Frota2016b, Frota2017, Carvalho2018, Frota2018, Post2018, Frota2020}. Em todos estes domínios da prosódia, traços específicos da língua materna emergem nas produções da criança ao longo do segundo ano de vida. Dependendo das características prosódicas da língua materna, a aquisição poderá ser mais rápida, ou mais lenta, com produção semelhante à língua-alvo antes dos dois anos de idade ou ainda divergente da língua-alvo após os seis anos de idade.

Entre as aquisições mais precoces encontra-se o desenvolvimento do sistema entoacional \citep{Frota2018}. Este inclui a discriminação de vários contrastes entoacionais, como os contrastes que permitem a distinção, no Português, entre frases declarativas e interrogativas, ou entre foco amplo, em que toda a informação é nova, e foco contrastivo, em que a informação dada contrasta com informação pressuposta ou já conhecida. O desenvolvimento do sistema entoacional inclui também o estabele\-cimento da relação entre forma e significado das melodias produzidas. Os padrões fonológicos tendem a ser adquiridos antes dos detalhes fonéticos e categorias entoacionais fonologicamente mais simples precedem a aquisição de categorias mais complexas. Interessantemente, a aquisição de muitos dos aspetos centrais do sistema entoacional, antes do final do segundo ano de vida, precede a aquisição do inventário segmental e da estrutura silábica. Também a aquisição dos padrões rítmicos da língua se desenvolve cedo, por volta dos dois anos de idade, em particular em línguas com ritmo de tipo silábico, ou com estruturas rítmicas menos complexas (Francês, Castelhano, Húngaro). Em línguas ritmicamente mais complexas, a produção do ritmo-alvo é apenas atingida por volta dos quatro (Alemão) ou mesmo 6 anos de idade (Inglês; \citealp{Post2018}). A comunicação de informação nova ou contrastiva (foco) através da prosódia pode envolver tanto a entoação quanto a proeminência. O desenvolvimento desta competência depende também de diferenças tipológicas entre línguas, com línguas com sistemas mais complexos no mapeamento entre forma e função prosódica (como o Holandês) a apresentarem uma aquisição mais tardia \citep{Chen2018}.

Dado o papel da prosódia no processo de aquisição da língua, desvios no desenvolvimento prosódico podem ser informativos quanto à aquisição de outros domínios da linguagem e constituir indicadores para perturbações de linguagem.

\section{Prosódia atípica}\label{sec:cap8sec2}
Quando o desenvolvimento prosódico se desvia do esperado, estamos perante formas de prosódia atípica. As características da prosódia atípica são muito variáveis, podendo as suas consequências ser tão evidentes que chegam a perturbar a comunicação. O estudo das alterações prosódicas em populações clínicas é particularmente relevante, pois a prosódia não só é importante para a aquisição da língua materna, como também para a comunicação em geral, com implicações a nível social e profissional \citep{Paul2005}.

A investigação tem vindo a mostrar que diferentes perturbações se encontram associadas a alterações prosódicas (ver \citealp{Peppe2018}, para uma revisão recente). A título de exemplo, encontram-se documentadas alterações prosódicas nas lesões cerebrais adquiridas \citep{Adamaszek2013}, na esclerose múltipla \citep{Bunton2000}, na doença de Parkinson \citep{Penner2008, Frota2021}, na deficiência auditiva \citep{Parker1990}, na síndrome de Down \citep{Heselwood1995}, na afasia \citep{Seddoh2004}, na esquizofrenia \citep{Pascual2005}, na epilepsia \citep{SanzMartin2006}, na depressão \citep{Alpert2001}, na síndrome de Williams \citep{Catterall2006}, na dislexia \citep{Goswami2010}, ou nas perturbações do espectro do autismo \citep{Baltaxe1985}. As alterações prosódicas podem afetar aspetos expressivos ou recetivos da proeminência, entoação e/ou fraseamento. Por exemplo, têm sido descritas alterações na produção do acento de palavra, padrões rítmicos atípicos, nomeadamente devido a alterações na duração silábica, padrões entoacionais atípicos, bem como perturbações no fraseamento prosódico. A presença de alterações nos parâmetros acústicos de duração, amplitude e/ou frequência fundamental é comummente referida. Todavia, a ligação entre uma alteração prosódica particular e o seu eventual impacto numa dada função prosódica, como a função demarcativa ou as funções comunicativas (por exemplo, de distinção de tipos frásicos ou identificação do foco), pode ser complexa e variável de língua para língua. Alterações da frequência fundamental, ou dos padrões entoacionais, podem ter impacto não só na distinção entre tipos frásicos, mas também na identificação da informação importante ou foco e ainda nos padrões de fraseamento. Alternativamente, tais alterações podem manifestar-se como desvios acústicos (maior variabilidade nos contornos de frequência fundamental, ou frequência fundamental mais elevada), sem consequências para as funções prosódicas em que a frequência fundamental desempenha um papel.

As competências prosódicas de crianças com perturbação da linguagem parecem estar relativamente preservadas, de uma forma geral, quando comparadas com outras áreas da linguagem. Por exemplo, em comparação a pares com desenvolvimento típico emparelhados segundo medidas de compreensão da linguagem, crianças com perturbação da linguagem não se diferenciam de forma significativa em tarefas de avaliação prosódica, parecendo não existir uma alteração específica da prosódia nesta população clínica \citep{Wells2003}. No entanto, alguns estudos referem padrões rítmicos atípicos, especialmente devido à omissão de sílabas átonas \citep{McGregor1994}. Dificuldades com o ritmo nesta população clínica têm sido associadas a \emph{deficits} no processamento auditivo de pistas prosódicas \citep{Corriveau2007}. A caracterização da prosódia na perturbação da linguagem permanece uma questão em aberto, sendo particularmente relevante determinar se eventuais alterações prosódicas resultam desta condição clínica, ou constituem um factor que contribui para o desenvolvimento da perturbação da linguagem.

Em perturbações do neurodesenvolvimento de base genética, como a síndrome de Williams ou a síndrome de Down, têm sido reportadas várias alterações prosódicas. Diversos estudos mostram que indivíduos com síndrome de Williams apresentam um desempenho significativamente inferior em tarefas de avaliação prosódica, quando comparados a pares da mesma idade com um desenvolvimento típico \citep{PlesaSkwerer2007, Stojanovik2007, MartinezCastilla2011}. Em particular, as diferenças encontradas na compreensão e produção do fraseamento prosódico e na perceção do acento de palavra têm sido consistentes. O desenvolvimento da capacidade de segmentação de palavras parece ser tardio em bebés com Síndrome de Williams \citep{Nazzi2003}, o que poderá relacio\-nar-se com as dificuldades quanto ao fraseamento prosódico, bem como com um desenvolvimento lexical também tardio. Tais resultados sugerem que a aquisição de pelo menos alguns aspetos da prosódia é atípica nesta síndrome. Também na síndrome de Down foram descritos desempenhos inferiores tanto na compreensão como na produção da prosódia, sendo todavia a produção mais afetada do que a compreensão \citep{Stojanovik2011, Kent2013}. Bebés com esta síndrome apresentam um percurso atípico de desenvolvimento da capacidade de segmentação de palavras, caracterizado pela emergência tardia desta capacidade \citep{MasonApps2011, MasonApps2018} e pela dificuldade em utilizar as pistas de fraseamento prosódico que facilitam a segmentação por bebés com desenvolvimento típico \citep{Frota2020a}. 

Tanto na síndrome de Williams como na síndrome de Down, e ao contrário do que tem sido reportado para o desenvolvimento típico, as competências prosódicas parecem não estar relacionadas com o desenvolvimento de outros aspetos da linguagem \citep{Stojanovik2007, Stojanovik2011, MasonApps2018}. Todavia, um estudo recente aponta uma correlação entre as capacidades de utilizar informação prosódica para segmentar palavras e o desenvolvimento do vocabulário em bebés com síndrome de Down \citep{Frota2020a}.

Um número substancial de estudos salienta a existência de alterações prosódicas nas perturbações do espectro do autismo (ver \citealp{McCann2003, Peppe2018}, para uma revisão). Estas alterações têm sido descritas tanto do ponto de vista da produção como da percepção/compreensão. Ao nível da produção, foram identificadas dificuldades no ritmo, na velocidade de fala e nos padrões de entoação (e.g., \citealp{McCann2003, Shriberg2001, Paul2005}), bem como no estabelecimento da proeminência acima do nível da palavra \citep{McCann2007}. Outros estudos observaram também um aumento da altura da frequência fundamental em crianças com autismo de alto-funcionamento, sugerindo que a entoação exagerada poderá ser um marcador prosódico do autismo em crianças \citep{Nadig2012, Filipe2014}. Também têm sido descritas dificuldades na produção da prosódia emocional \citep{Filipe2016}. Todavia, é igualmente referido que muitos indivíduos com autismo (especialmente crianças com síndrome de Asperger) apresentam um bom desempenho em tarefas prosódicas, como no caso da expressão de diferentes tipos frásicos \citep{Paul2005, Peppe2011, Filipe2014}. Do ponto de vista da perceção/compreensão, as crianças com autismo mostraram-se capazes de recordar palavras acentuadas (palavras prosódicas) melhor do que as não acentuadas \citep{Fine1991} e de discriminar pares de palavras que diferem nos padrões de acento \citep{Grossman2010}. No entanto, foram encontradas alterações na perceção da proeminência relacionada com ênfase ou foco \citep{Paul2005}, na perceção de padrões entoacionais \citep{Peppe2007, Filipe2016} e na compreensão de funções comunicativas veiculadas pela prosódia, como a distinção entre frases declarativas e interrogativas \citep{Paul2005, Peppe2007, Peppe2011, Filipe2016}.

A caracterização do desenvolvimento prosódico atípico encontra-se ainda fragmentada. São necessários estudos que abordem os diferentes aspetos nucleares dos sistemas prosódicos, designadamente o fraseamento, a proeminência e a entoação, tendo em conta as dimensões de produção e compreensão, bem como a forma (parâmetros acústicos) e as funções prosódicas. Dada a especificidade prosódica característica de cada língua ou variedade, importa estudar as alterações prosódicas associadas a condições clínicas em diferentes línguas. A comparabilidade das abordagens metodológicas, ou dos instrumentos de avaliação utilizados, na descrição das diversas populações clínicas constituirá um passo importante para a compreensão do desenvolvimento prosódico atípico.

\section{Avaliação da prosódia}\label{sec:cap8sec3}
A avaliação da prosódia infantil é habitualmente desencadeada pela suspeita de perturbação de linguagem ou pela presença de fala atípica. Atualmente, encon\-tram-se disponíveis alguns instrumentos para a avaliação formalizada do desenvolvimento prosódico. Todavia, nem todos assentam em pressupostos teóricos claros de análise da prosódia, consideram a especificidade da língua em avaliação, ou têm aplicabilidade clínica.

O \emph{Prosody Profile} (PROP, \citealp{Crystal1982}) avalia apenas prosódia expressiva, centrando a sua análise na entoação. O PROP requer a gravação de uma amostra de fala que é posteriormente analisada e anotada através de um sistema de transcrição que incorpora elementos como acento, velocidade, ritmo, e várias categorias tonais. Utilizando fundamentalmente noções da escola britânica de entoação, o PROP supõe utilizadores treinados neste tipo de transcrição prosódica e não tem sido muito usado quer em contextos de investigação, quer em contextos clínicos.

Na mesma linha, o \emph{Prosody-Voice Screening Profile} (PVSP, \citealp{Shriberg1990}) parte de uma amostra de fala espontânea para quantificar aspetos prosódicos e características vocais. Encontra-se dividido em sete áreas, três das quais avaliam dimensões prosódicas: acento, fraseamento e velocidade de fala. A codificação do PVSP requer um avaliador treinado na sua cotação, de forma a que seja possível avaliar se as características de voz e de prosódia do falante são as esperadas, tendo como base um conjunto de critérios. Este instrumento dispõe de dados normativos dos 3 aos 19 anos de idade, para a língua inglesa. Todavia, a cobertura de aspectos prosódicos do PVSP é muito limitada, deixando de fora o vasto conjunto de significados comunicativos veiculados pela prosódia, bem como a perceção da prosódia.

O \emph{ Voice Assessment Protocol for Children and Adults} (VAP, \citealp{Pindzola1987}) é um procedimento para indivíduos com idades compreendidas entre os 4 e os 18 anos que avalia propriedades da fala, entre as quais a frequência fundamental, o ritmo e a velocidade. Já o \emph{Frenchay Dysarthria Assessment (FDA) – Second Edition} \citep{Enderby2008} é um teste validado de 10 itens, dos quais oito se focam na observação de estruturas orais e de fala. Este instrumento avalia alguns aspetos prosódicos desde os 12 anos de idade até à idade adulta. Os aspetos avaliados centram-se em características laríngeas (como a frequência fundamental) e medidas de inteligibilidade (palavras, frases e conversação). Estão disponíveis dados normativos para adultos sem disartria, bem como para pacientes com disartria específica. Tanto o VAP como o FDA são instrumentos clinicamente orientados, mas que não avaliam nem as vertentes de perceção e produção, nem as grandes funções da prosódia, limitando-se a alguns parâmetros acústicos e seus efeitos na inteligibilidade.

O \emph{Profiling Elements of Prosody in Speech - Communication} (PEPS-C, \citealp{McCann2003}) avalia competências prosódicas a partir dos 5 anos de idade até à idade adulta. O PEPS-C é um instrumento de grande relevância a nível clínico e científico, ao avaliar as vertentes de perceção e produção, bem como as dimensões formal e funcional da prosódica. Enquanto o nível formal avalia competências para percecionar e produzir sequências com determinadas propriedades prosódicas acústicas, independentemente do significado, o nível funcional avalia competências para percecionar e produzir significados comunicativos veiculados pela prosódia. Entre estes encontram-se as grandes funções da prosódia: a segmentação de enunciados através do fraseamento prosódico (segmentação); a distinção entre tipos frásicos (frases declarativas e interrogativas); a identificação da informação relevante na frase (foco), nomeadamente através da proeminência entoacional; a distinção dos significados afetivos de agrado e desagrado (afeto). As três primeiras funções examinam significados linguísticos, enquanto a quarta avalia um significado paralinguístico, cobrindo assim um amplo espectro de funções comunicativas. O instrumento utiliza sequências sonoras fixas, previamente definidas, facilitando a comparação e análise das respostas. O avaliador não necessita de treino específico de transcrição prosódica. Existem alguns dados normativos disponíveis para o PEPS-C, em várias línguas, mas não se trata de um instrumento estandardizado.

\section{Avaliar a prosódia do Português}\label{sec:cap8sec4}
Recentemente, têm sido desenvolvidos instrumentos de avaliação da prosódia do Português, teoricamente informados quanto às características prosódicas da língua e o percurso de desenvolvimento prosódico apontado por vários estudos experimentais. O perfil prosódico do Português Europeu, em particular a sua variedade \emph{standard} falada na região da grande Lisboa, é singular ao combinar propriedades típicas de línguas românicas e de línguas germânicas \citep{Frota2000, Frota2014, Frota2001, Vigario2003, Frota2015, Vigario2019}. O ritmo apresenta uma natureza mista, com características acentuais e silábicas. O fraseamento prosódico salienta a palavra prosódica e o sintagma entoacional. A entoação caracteriza-se por eventos tonais esparsos, o que destaca o papel do acento tonal nuclear na expressão de significados comunicativos. Para além da posição do acento tonal nuclear, também o tipo de acento tonal é relevante na expressão do foco, contribuindo para a proeminência entoacional característica do foco restrito ou contrastivo. Os três instrumentos de avaliação prosódica descritos abaixo, apesar das suas diferenças, cobrem os principais aspetos estruturadores do sistema prosódico – fraseamento, proeminência e entoação – tendo em atenção a sua funcionalidade na língua. Estudos experimentais de desenvolvimento prosódico têm apontado para uma perceção diferenciada de contrastes de acento, uma discriminação dos contrastes entoacionais envolvidos na distinção entre declarativa e interrogativa, ou entre foco amplo e foco contrastivo, bem como uma sensibilidade à presença de fronteira prosódica alta (a fronteira do sintagma antoacional), ainda durante o primeiro ano de vida \citep{Frota2014a, Butler2016, Butler2018, Frota2019, Frota2020}. O estabelecimento de relações forma-significado das melodias produzidas próximas da língua-alvo, tal como a produção de fraseamento prosódico semelhante ao do adulto, desenvolvem-se durante o segundo ano de vida \citep{Frota2016b}. Todavia, uma compreensão do desenvolvimento prosódico requer instrumentos que possibilitem uma avaliação sistemática e robusta das competências prosódicas, aplicável em grande escala, a populações típicas e atípicas, desde os momentos iniciais do desenvolvimento até à idade adulta. Os três instrumentos de avaliação da prosódia do Português pretendem responder a estes desafios, constituindo, até onde sabemos, um caso único no panorama da avaliação prosódica. 

\subsection{PEPS-C}\label{sec:cap8sec4.1}
O PEPS-C, já referido na \sectref{sec:cap8sec3}, encontra-se disponível para o Inglês Britânico, Escocês, Norte-americano e Australiano, estando também adaptado ao Castelhano, Francês, Irlandês, Holandês, Norueguês \citep{Peppe2006} e Português Europeu \citep{Filipe2017}. A versão Portuguesa avalia as mesmas dimensões da versão inglesa original. No entanto, foram introduzidas modificações de forma a traduzir e adaptar o instrumento atendendo às especificidades prosódicas da língua. A utilização de um mesmo instrumento, adaptado a várias línguas, abre a possibilidade de comparação de trajetórias de desenvolvimento prosódico entre línguas. Do ponto de vista clínico, o PEPS-C introduz no contexto português um instrumento para a avaliação das competências prosódicas recetivas e expressivas a partir dos 5 anos de idade, que era inexistente.

Como descrito na \sectref{sec:cap8sec3}, o PEPS-C avalia a capacidade de perceção e produção da prosódia nos níveis formal e funcional. Este instrumento é constituído por um total de 6 provas, sendo que cada prova inclui dois tipos de tarefas, uma tarefa recetiva e uma tarefa expressiva (ver Tabela \ref{tab:cap8tab1}). Cada tarefa apresenta um total de 16 itens, precedidos de 2 itens para treino e 2 exemplos.

Os estudos realizados com a versão portuguesa do PEPS-C em crianças monolingues falantes de Português Europeu, entre os 5 e os 12 anos de idade, mostraram importantes desenvolvimentos das competências prosódicas entre os 5 e os 8 anos de idade em todas as tarefas recetivas. Constitui exceção a tarefa Afeto, em que a performance aos 5 anos apresenta já um efeito de teto \citep{Filipe2017}. Quanto às tarefas expressivas, as competências prosódicas podem estar ainda em desenvolvimento aos 12 anos de idade. Assim, as competências recetivas precedem as expressivas.  Os resultados sugerem que as competências prosódicas relacionadas com o Afeto são adquiridas em primeiro lugar, seguindo-se as competências de Interação (tipo frásico) e depois as de segmentação. O foco parece ser a última competência prosódica adquirida no Português Europeu.



\begin{Tabela}
\caption{Descrição das provas do PEPS-C em função do nível (Forma e Função) e do modo (Recetivo e Expressivo).}
\label{tab:cap8tab1}

\small
\begin{tabularx}{\textwidth}{llp{2cm} Q}
\lsptoprule
 \textbf{Nível}& \textbf{Tipo} &\textbf{Provas}& \textbf{Descrição}  \\
 \midrule
 Formal & Recetivo &Discriminação de itens curtos    & Competências para perceber mudanças de entoação em estímulos curtos {deslexicalizados} (filtrados a 500Hz). \\
 \tablevspace
 & Expressivo & Imitação de & Competências de voz requeridas para \\
  & & itens curtos & imitar várias formas de entoação em {estímulos} curtos. \\
 \tablevspace
 & Recetivo & Discriminação de itens longos & Competências de discriminação de {formas} prosódicas em estímulos longos deslexicalizados. \\
 \tablevspace
 & Expressivo & Imitação de itens longos &	Competências de imitação de formas de prosódia em estímulos longos.\\
 \midrule
 Funcional & Recetivo	&Interação	&Competências para distinguir a entoação interrogativa e declarativa em palavras isoladas. \\
 \tablevspace
 &Expressivo&	Interação&	Competências para produzir a entoação interrogativa e declarativa em palavras isoladas.\\
\tablevspace
&Recetivo&	Afeto&	Competências para compreender a {entoação} que exprime afeto (agrado vs. desagrado) em palavras isoladas.\\
\tablevspace
 & Expressivo & Afeto & Competências para produzir entoação afetiva (agrado vs. desagrado) em palavras isoladas.\\
\tablevspace
& Recetivo&	Segmentação	&Competências de compreensão de frases ambíguas com recurso ao fraseamento prosódico. \\
\tablevspace
& Expressivo&	Segmentação&	Competências para produzir frases {ambíguas} de forma não ambígua com recurso ao fraseamento prosódico.\\
\tablevspace
& Recetivo &	Foco&	Competências para identificar o foco. \\
\tablevspace
& Expressivo&	Foco&	Competências para produzir o foco.\\
\lspbottomrule
\end{tabularx}
\end{Tabela}


A versão Portuguesa do PEPS-C tem também sido utilizada com populações clínicas, como no caso das perturbações do espectro do autismo. Estes estudos mostram que alterações prosódicas são frequentes, mas que variam ao longo deste espectro clínico. Crianças com autismo de grau mais ligeiro, nomeadamente com síndrome de Asperger, podem não ter problemas com distinções prosódias funcionais e categóricas (por exemplo, podem produzir e compreender frases declarativas e interrogativas), mas as suas produções são percebidas como atípicas ou não comuns, revelando ainda alterações em diferentes parâmetros acústicos \citep{Filipe2014}. Por sua vez, crianças com autismo de alto-funcionamento (menos funcionais do que crianças com síndrome de Asperger) podem manifestar problemas em perceber ou imitar padrões prosódicos e não compreendem algumas das funções comunicativas veiculadas pela prosódia \citep{Filipe2016}.

\subsection{APC -- Avaliação da prosódia na criança}
O teste Avaliação da Prosódia na Criança -- APC \citep{Vidal2018} foi desenvolvido de raiz, parcialmente inspirado no PEPS-C, com o objetivo de avaliar o desenvolvimento prosódico em crianças com idades compreendidas entre os 3 e os 6 anos. Em quatro provas, são avaliadas as competências de produção e compreensão de contornos entoacionais associados a emoções (afecto -- agrado e desagrado), de produção e compreensão da entoação de tipos frásicos distintos (declarativa neutra e chamamento), de produção de foco contrastivo e de discriminação de acento na palavra. Estas provas pretendem avaliar o nível funcional da prosódia, fundamental na transmissão de significados comunicativos linguísticos e paralinguísticos. O instrumento pode ser aplicado em populações com desenvolvimento típico ou atípico. 

O teste é apresentado em \emph{PowerPoint}, em formato de jogo. As quatro provas centrais que compõem o instrumento são construídas tendo por base uma pequena narrativa, sempre com a mesma menina como personagem central. A forma de apresentação das provas pretende motivar a criança e levá-la a responder de uma forma natural, em interação com as personagens da narrativa. Os estímulos foram selecionados a partir do vocabulário usado por crianças na faixa etária a que o teste se destina, de acordo com a informação disponível na base de dados informatizada PLEX5 \citep{Frota2012b}, nomeadamente quanto à idade de emergência e frequência de utilização do vocabulário infantil. Uma prova de nomeação do vocabulário antecede a realização das provas, permitindo ao avaliador verificar se a criança conhece todas as palavras utilizadas e nomear aquelas que ela mostre desco\-nhecer. 

Algumas provas incluem duas tarefas, uma recetiva e outra expressiva.  Cada tarefa apresenta um total de 8 itens de teste, precedidos de um exemplo e 2 itens para treino, sendo cada tarefa cotada com 0 a 8 pontos (um ponto por resposta certa). Existem ainda duas provas adicionais, que avaliam o nível formal da prosódia (\emph{Discriminação de itens curtos} e \emph{Imitação de itens curtos}) e que apenas devem ser realizadas perante um erro sistemático nas provas centrais do APC. Nesse caso, a avaliação da capacidade de percecionar e produzir estímulos com determinadas propriedades prosódicas, independentemente do seu significado, poderá ser um passo relevante para determinar a origem das dificuldades da criança. 

Até à data o teste foi aplicado apenas a um grupo de crianças entre os 3 e os 4 anos de idade, com desenvolvimento típico: 31 crianças, do concelho de Aveiro, com uma média de idades de 3 anos e 6 meses (Tabela \ref{tab:cap8tab2}). As provas relativas à compreensão e produção de contornos entoacionais para expressar afeto, bem como à compreensão e expressão de tipos frásicos, são as que obtiveram resultados mais altos (74 a 92\% de taxa de acerto). Estes resultados mostram-se já em linha com os obtidos por crianças mais velhas testadas com o PEPS-C (versão Portuguesa) -- ver \sectref{sec:cap8sec4.1}, apontando para um desenvolvimento precoce destas competências prosódicas. Pelo contrário, na prova de produção de foco contrastivo a taxa de acerto foi a mais baixa (50 a 60\%), também em linha com os resultados do PEPS-C para a tarefa expressiva de foco que apontam para um desenvolvimento mais tardio desta competência. Estando o foco relacionado não apenas com aspetos prosódicos, mas também sintático-semânticos e pragmático-discursivos, poderemos estar perante dificuldades que resultam de outros domínios da língua, sem excluir a possibilidade de dificuldades de origem metodológica na construção da própria prova de foco para as faixas etárias mais novas (dados de produção espontânea apontam para a produção de foco antes dos 2 anos de idade -- \citealp{Frota2016b}). Quanto à prova de discriminação do acento na palavra, a taxa de acerto supera os 72\%, o que convida a uma reflexão sobre os resultados da perceção do acento no Português tanto por bebés de 5-6 meses de idade, em que os padrões acentuais são distinguidos, como por adultos, que manifestam algumas dificuldades no processamento do acento consoante o tipo de tarefa experimental \citep{Lu2018, Frota2020}. Não tendo sido verificadas dificuldades no desempenho generalizado das 4 provas centrais do APC, não existem ainda dados acerca do desempenho nas provas do nível formal.



\begin{Tabela}
\caption{Resultados médios, desvio padrão e percentagem de respostas corretas no APC aos 3 anos e 6 meses de idade.}
\label{tab:cap8tab2}


\begin{tabular}{lrrr}
\lsptoprule
 \textbf{Provas}& \textbf{Média} &\textbf{DP} & \textbf{\%}  \\
 \midrule
Afeto -- Expressiva &	6,84&	1,73&	85,50\% \\
Afeto -- Recetiva&	6,71&	2,02&	83,88\%\\
Tipo frásico -- Expressiva&	7,39&	1,38&	92,38\%\\
Tipo frásico -- Recetiva	&5,93&	2,00&	74,12\%\\
Foco -- Expressiva (NV)&	4,77&	2,23&	59,62\%\\
Foco -- Expressiva (Nadj)&	4,24&	1,35&	53\%\\
Discriminação de acento de palavra&	5,81&	1,83&	72,62\%\\
\lspbottomrule
\end{tabular}
\end{Tabela}


O APC encontra-se ainda em avaliação, estando prevista a inclusão de três tarefas adicionais -- a tarefa recetiva associada à prova de foco e a prova de segmentação prosódica nas suas tarefas recetiva e expressiva, bem como a aplicação do teste em larga escala com vista à obtenção de dados de referência para a população portuguesa.

\subsection{Proso-Quest}
Dada a sensibilidade precoce para a prosódia e a centralidade do desenvolvimento prosódico no processo de aquisição da língua, é fundamental avaliar a prosódia o mais cedo possível no desenvolvimento. Não existindo anteriormente instrumentos para avaliar o desenvolvimento prosódico nos primeiros anos da criança, foi criado de raiz um novo questionário parental, o \emph{Proso-Quest} \citep{Frota2012a}. Este instrumento permite avaliar o conhecimento prosódico das crianças nas fases iniciais da aquisição, ao nível da compreensão e da produção. O instrumento foi desenvolvido para o Português Europeu, mas pode ser adaptado a outras variedades ou línguas e foi já aplicado a mais de 429 crianças com desenvolvimento típico e 20 crianças com síndrome de Down. O estudo de validação do instrumento encontra-se em fase de conclusão \citep{Filipeempreparacao}. Nesta secção descrevemos os resultados apresentados em \citet{Vigario2015} e \citet{Filipeempreparacao}.

O \emph{Proso-Quest} é um questionário de uma página, com 6 itens que se desdobram em 18 questões, destinado a ser respondido pelos pais ou cuidadores, com base na sua lembrança de momentos do desenvolvimento da criança. Os questionários parentais constituem um método reconhecido de avaliação de competências linguísticas (veja-se o caso dos questionários CDI, \citealp{Frota2016a}). O recurso a recolha de dados retrospetiva é também considerado um método válido e confiável \citep{Cameirao2010}. Até cerca dos 24 meses da criança, pede-se ao cuidador que indique a altura em que a criança já demonstra um conjunto de competências prosódicas na compreensão ou na produção. A fim de cobrir os domínios centrais da entoação, fraseamento e proeminência, são avaliados os seguintes aspetos: (i) compreensão/produção de diferentes tipos frásicos; (ii) compreensão/produção predominante ou mais favorável de palavras isoladas, em posição proeminente ou em interior de frase, e (iii) compreensão/produção de foco contrastivo e contexto de uso apropriado (Tabela \ref{tab:cap8tab3}).



\begin{Tabela}
\caption{Descrição dos itens do Proso-Quest.}
\label{tab:cap8tab3}


\begin{tabularx}{\textwidth}{p{2.5cm}p{2.5cm}Q}
\lsptoprule
\textbf{Componentes}&	\textbf{Tipo de item} &	\textbf{Descrição} \\
\textbf{prosódicos}&&\\
\midrule
Entoação&	Compreensão	Produção &Tipos frásicos: declarativa, {interrogativa}, ordem, pedido, chamamento\\
\midrule
Fraseamento	&Compreensão&	Palavra isolada\\
&&Palavra no final de constituinte\\ && Palavra no interior de constituinte\\
& 	Produção&	Palavra isolada\\
&&Palavras sem integração prosódica\\
&&Palavras prosodicamente {integradas}\\
\midrule
Proeminência&	Compreensão	&Foco contrastivo em contexto\\
& Produção &\\
\lspbottomrule
\end{tabularx}
\end{Tabela}


Os dados normativos para crianças com desenvolvimento típico (de diversas regiões de Portugal, continental e ilhas) mostram uma relação de precedência sistemática da compreensão face à produção. Em relação aos tipos frásicos, os resultados indicam que as crianças compreendem todos os tipos frásicos entre os 9 e os 13 meses, pela seguinte ordem de emergência: cham(amento) > decl(arativa) > ord(em), inter(rogativa), ped(ido). A produção dos tipos frásicos desenvolve-se entre os 14 e os 18 meses, pela ordem cham > decl > ped > ord, inter. Quanto às competências de fraseamento e segmentação de palavras, as posições proeminentes são facilitadoras da compreensão aos 10 meses, com a compreensão de palavras no interior da frase, posição não proeminente, a partir dos 14 meses. Na produção, a integração prosódica acontece mais tardiamente (18--19 meses), havendo anteriormente produção de palavras isoladas (14--15 meses) e produção de palavras em sucessão não integradas prosodicamente no mesmo constituinte (16-17 meses). O foco contrastivo é compreendido a partir dos 13 meses e produzido aos 18, havendo, uma vez mais, precedência da compreensão relativamente à produção. Em linha com resultados prévios de trabalhos experimentais \citep{Frota2018}, as competências entoacionais associadas a tipos frásicos desenvolvem-se mais cedo e as de segmentação/fraseamento e foco mais tarde. Notavelmente, todas estas competências se desenvolvem antes do final do segundo ano de vida. 

O \emph{Proso-Quest} foi também aplicado a um grupo clínico de crianças com síndrome de Down. Os resultados obtidos indicam diferenças significativas em relação à população com desenvolvimento típico em todos os domínios avaliados na compreensão, com um atraso de cerca de 4-5 meses. Contrariamente à população típica, o grupo com síndrome de Down, com uma média de idade de 20 meses, não mostra ainda aquisição dos domínios avaliados na produção.

O \emph{Proso-Quest} é de aplicação fácil e rápida, apresentando uma boa fiabilidade, consistência interna e relação com outras medidas de desenvolvimento linguístico. 

\largerpage
\section{Intervenção prosódica}
As secções anteriores deste capítulo mostram como a prosódia é o elemento organizador das sequências de fala, veiculando um conjunto relevante de significados comunicativos. A presença de padrões prosódicos atípicos ou perturbados, tanto na compreensão quanto na produção, conduz a dificuldades de fala e de linguagem com impactos no processamento e estruturação do discurso, nos significados transmitidos e percebidos, ou mesmo na aquisição de outros domínios da língua. Tais alterações prosódicas têm implicações evidentes em termos sociais, educativos e profissionais (e.g., \citealp{Paul2005}). Assim, como as consequências podem ser tão evidentes e ter um impacto tão negativo na comunicação, bem como no processo de aquisição da língua, é importante o desenvolvimento de programas de intervenção que permitam fazer face a estas dificuldades.

Na sua revisão crítica da literatura sobre intervenções prosódicas, \citet{Hargrove2009} e \citet{Hargrove2013} salientam que a prosódia pode ser modificada por intervenção. A maior parte das estratégias de intervenção recorre a \emph{feedback} visual ou verbal e à imitação, centrando-se em parâmetros prosódicos (como a frequência fundamental, a amplitude, a velocidade), mas menos nas funções da prosódia. As medidas utilizadas são fundamentalmente expressivas e muito poucos estudos incluem crianças. Destacamos abaixo dois programas de intervenção com características e resultados promissores.

O \emph{Prosody Treatment Program} \citep{Rothstein2013} constitui um recurso útil para a intervenção nas alterações de prosódia recetiva e/ou expressiva. Este programa é composto por dois níveis: pré-escolar e escolar. O nível pré-escolar destina-se a crianças dos três aos cinco anos e as atividades têm como objetivo trabalhar o ritmo, a frequência fundamental, o tom de voz, a manipulação de voz, ou os padrões entoacionais. O nível escolar é apropriado para crianças dos 5 aos 18 anos, inclui um teste de rastreio (\emph{Prosody Screening Test}) e atividades com o objetivo de melhorar as competências prosódicas quanto à amplitude, frequência fundamental e ritmo. O teste de rastreio examina, entre outros aspetos, os domínios do acento de palavra, da entoação das perguntas, do fraseamento e da prosódia emocional. No programa de intervenção, entre as atividades focadas na amplitude, encontra-se o treino de utilização da proeminência ao nível da sílaba e da palavra. No âmbito da frequência fundamental, são treinados padrões entoacionais associados a diferentes tipos frásicos, bem como usos paralinguísticos da entoação (na expressão de emoções). No domínio do ritmo, é treinada a velocidade de fala e o fraseamento. 

Para a dislexia do desenvolvimento, em particular, tem sido desenvolvido um programa de intervenção baseado no ritmo. O seu pressuposto teórico de base consiste na existência de um \emph{deficit} neurológico de processamento temporal da fala, com dificuldades na deteção das subidas de amplitude que constituem pistas para a segmentação do sinal acústico em sílabas e fonemas e que são especialmente salientes nas sílabas tónicas \citep{Goswami2011}. A intervenção rítmica multimodal recorre a música e poesia, incluindo um conjunto de tarefas motoras, expressivas e de processamento auditivo \citep{Bhide2013}. Face ao sucesso desta abordagem, a intervenção prosódica está a ser aplicada precocemente em bebés com risco familiar para dislexia. 

Apesar de desenvolvimentos recentes promissores, estão ainda por estabelecer intervenções eficazes para as alterações prosódicas, havendo ainda escassez de estudos empíricos e falta de ligação clara entre as atividades propostas e os tipos de alterações ou dificuldades prosódicas. À exceção do programa de intervenção rítmica de Goswami e colegas, a generalidade das propostas existentes não está ancorada em abordagens teóricas claras que fundamentem a natureza do problema prosódico, a sua relação com o desenvolvimento linguístico e a motivação das estratégias de intervenção.

\section{Conclusão}
\largerpage
Neste capítulo, apresentámos uma visão do desenvolvimento prosódico da palavra prosódica ao enunciado, do seu papel na aquisição da língua, bem como da relevância da avaliação prosódica, nomeadamente em idade precoce, e das oportunidades de intervenção prosódica. A afirmação de abordagens teóricas integradas de descrição da prosódia \citep{Nespor2007, Ladd2008} e a sua produtividade na aplicação a diversas línguas conduziu a avanços significativos nas últimas décadas na descrição e compreensão dos sistemas prosódicos das línguas e suas variedades (e.g., \citealp{Jun2005, Jun2014, Frota2015}). Apesar destes avanços, os estudos sobre o desenvolvimento prosódico infantil, típico e atípico, apenas recentemente têm beneficiado das abordagens teóricas e do conhecimento das tipologias prosódicas. Particularmente no caso da prosódia atípica, os estudos têm sido mais orientados no sentido da descrição de \emph{outputs}, em populações clínicas particulares, sem atender a pressupostos teóricos claros de análise prosódica, ou à especificidade da língua. Na reflexão apresentada neste capítulo, partimos da visão da prosódia como a estrutura organizadora da dimensão sonora da linguagem, destacando as dimensões cruciais do fraseamento, proeminência e entoação. Ao articular forma (propriedades acústicas) e função (categorias e significados), pretendemos destacar o essencial do que se conhece sobre o desenvolvimento prosódico. Parece-nos evidente que esta é uma área em que a aplicação de um modelo comum para a descrição e análise da prosódia típica e atípica traria desenvolvimentos promissores, com consequências para uma compreensão mais aprofundada das interfaces entre a prosódia e outras componentes da linguagem e do seu papel na aquisição da língua.

Também a avaliação prosódica beneficiaria de instrumentos teoricamente informados e devidamente adaptados às características prosódicas da língua em avaliação. Descrevemos os principais instrumentos de avaliação prosódica disponíveis, destacando aqueles que assentam em pressupostos teóricos claros de análise da prosódia e que apresentam aplicabilidade clínica. Para a avaliação da prosódia do Português, têm sido recentemente desenvolvidos instrumentos que possibilitam uma avaliação sistemática e robusta das competências prosódicas nos domínios estruturadores do sistema prosódico da língua. Estes instrumentos, no seu conjunto, são aplicáveis a populações típicas e atípicas, desde os momentos iniciais da aquisição prosódica até à idade adulta. Constituem, assim, um desenvolvimento importante na área da avaliação formalizada da prosódia, que aprofundará o nosso entendimento do desenvolvimento prosódico, bem como a caracterização da prosódia atípica associada às diversas condições clínicas com alterações de linguagem, e oferecerá meios objetivos de rastreio de alterações prosódicas ou de avaliação de estratégias de intervenção prosódica.

\clearpage
\section*{Agradecimentos}
A investigação que conduziu a este capítulo foi parcialmente financiada pelos projetos UIDB/00214/2020, UIDB/4255/2020, PTDC/MHCLIN/3901/2014 e PTDC{\slash}LLT-LIN/29338/2017, por fundos nacionais através da Fundação para a Ciência e Tecnologia e por fundos europeus através do programa FEDER, Portugal 2020 e Lisboa 2020.


{\sloppy\printbibliography[heading=subbibliography,notkeyword=this]}

\end{document}
