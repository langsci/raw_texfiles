\documentclass[output=paper,colorlinks,citecolor=brown,booklanguage=portuguese]{langscibook} 
\ChapterDOI{10.5281/zenodo.7233247}
\title{Consciência fonológica e escrita em pessoas com Síndrome de Down: Avaliação e intervenção}
\author{Aline Lorandi \affiliation{Universidade Feevale, Programa de Pós-graduação em Diversidade Cultural e Inclusão Social } and  Débora Mattos Marques\affiliation{União Espírita Bageense Caminho da Luz, SOMA - Serviço e Orientação Multidisciplinar de Aprendizagem}}

\abstract{Este trabalho discorre sobre aspectos de avaliação e intervenção ligados à consciência fonológica, à leitura e à escrita em pessoas com Síndrome de Down. A partir de análises quanti e qualitativas, destacamos a importância de uma abordagem multidisciplinar e holística em todo trabalho que busque tanto avaliar, quanto estimular habilidades, uma vez que todo ser humano, bem como o próprio processo de aprendizagem constituem-se como fenômenos complexos.}

\IfFileExists{../localcommands.tex}{
   \addbibresource{../localbibliography.bib}
   \usepackage{langsci-optional}
\usepackage{langsci-gb4e}
\usepackage{langsci-lgr}

\usepackage{listings}
\lstset{basicstyle=\ttfamily,tabsize=2,breaklines=true}

%added by author
% \usepackage{tipa}
\usepackage{multirow}
\graphicspath{{figures/}}
\usepackage{langsci-branding}

   
\newcommand{\sent}{\enumsentence}
\newcommand{\sents}{\eenumsentence}
\let\citeasnoun\citet

\renewcommand{\lsCoverTitleFont}[1]{\sffamily\addfontfeatures{Scale=MatchUppercase}\fontsize{44pt}{16mm}\selectfont #1}
  
   %% hyphenation points for line breaks
%% Normally, automatic hyphenation in LaTeX is very good
%% If a word is mis-hyphenated, add it to this file
%%
%% add information to TeX file before \begin{document} with:
%% %% hyphenation points for line breaks
%% Normally, automatic hyphenation in LaTeX is very good
%% If a word is mis-hyphenated, add it to this file
%%
%% add information to TeX file before \begin{document} with:
%% %% hyphenation points for line breaks
%% Normally, automatic hyphenation in LaTeX is very good
%% If a word is mis-hyphenated, add it to this file
%%
%% add information to TeX file before \begin{document} with:
%% \include{localhyphenation}
\hyphenation{
affri-ca-te
affri-ca-tes
an-no-tated
com-ple-ments
com-po-si-tio-na-li-ty
non-com-po-si-tio-na-li-ty
Gon-zá-lez
out-side
Ri-chárd
se-man-tics
STREU-SLE
Tie-de-mann
}
\hyphenation{
affri-ca-te
affri-ca-tes
an-no-tated
com-ple-ments
com-po-si-tio-na-li-ty
non-com-po-si-tio-na-li-ty
Gon-zá-lez
out-side
Ri-chárd
se-man-tics
STREU-SLE
Tie-de-mann
}
\hyphenation{
affri-ca-te
affri-ca-tes
an-no-tated
com-ple-ments
com-po-si-tio-na-li-ty
non-com-po-si-tio-na-li-ty
Gon-zá-lez
out-side
Ri-chárd
se-man-tics
STREU-SLE
Tie-de-mann
}
   \boolfalse{bookcompile}
   \togglepaper[18]%%chapternumber
}{}

\shorttitlerunninghead{Consciência fonológica e escrita em pessoas com Síndrome de Down}
\begin{document}
\shorttitlerunninghead{Consciência fonológica e escrita em pessoas com Síndrome de Down}
\maketitle

\section{Síndrome de Down}

\subsection{Histórico e breve caracterização da síndrome}

Estima-se que a Síndrome de Down (SD) sempre tenha existido, embora ela tenha sido descrita como uma condição genética pela primeira vez pelo médico britânico John Langdon Down, em 1962.\footnote{A norma adotada na escrita deste capítulo foi a do português brasileiro.}
% \footnote{Na sequência de restrições decorrentes do template usado para publicação na editora Language Science Press, os capítulos que integram o volume surgem com a referência [this volume], em inglês.}
Down foi o primeiro a caracterizar a relação entre os aspectos físicos comuns entre pessoas com essa síndrome e atrasos no desenvolvimento intelectual. Apenas em 1959, soube-se que a síndrome era causada pela presença de um cromossomo extra. Essa descoberta foi feita pelo geneticista Jérôme Lejeune \citep{Schwartzman1999}.

Há registros de crianças com características físicas semelhantes às de crianças com SD em pinturas antigas, como as de Andrea Mantega (1431-1506) \citep{Schwartzman1999}. O caso mais antigo de análise genética que identificou uma criança com SD data do ano de 5.500 a.C. \citep{RTENews2020}. É sabido, contudo, que crenças socioculturais que remontam à antiguidade grega \citep{Aranha2001} estimularam preconceito e ceticismo em relação ao potencial de desenvolvimento de crianças que nasciam com SD, o que pode ter contribuído para que essas crianças ficassem sem amparo e cuidados multidisciplinares necessários por muitos anos na história.

Apesar de haver poucos registros e estudos mais antigos sobre a SD, ela é atualmente a mais estudada causa de desenvolvimento atípico. Embora vastamente investigada, são muitas as questões que ainda permanecem sem resposta, incluindo a relação entre genótipo e fenótipo, uma vez que não se conhece completamente a síndrome em nível molecular \citep{Lanfranchi2019}.

A SD é entendida como uma alteração cromossômica, pela presença de um cromossomo a mais. Isso significa dizer que, enquanto as pessoas que não têm a síndrome apresentam 46 cromossomos em suas células, as pessoas com SD apresentam 47. O cromossomo que está triplicado é o de número 21.

A trissomia mais comum é causada pela completa triplicação do cromossomo 21, advindo, em 88\% dos casos, da mãe, em função de um erro de divisão celular chamado não-disjunção. Em outros casos da síndrome, apenas uma parte do cromossomo 21 é triplicada (trissomia parcial). Em outros, ainda, acontece a triplicação do cromossomo todo, mas apenas uma porção das células são trissômicas, enquanto outras células apresentam-se normais (mosaicismo) \citep{Schwartzman1999, KarmiloffSmith2016}. O mosaicismo costuma acontecer em 1,5-5\% dos casos. Há também a translocação como causa da SD, situação em que uma parte do material genético do cromossomo 21 é movida para outro cromossomo, em geral o 14 ou o 22. A translocação acontece em 4\% dos casos de SD \citep{KarmiloffSmith2016}. Embora de origem genética, não há indícios que apontem para causas hereditárias.

A SD acontece aproximadamente em 1 a cada 700\footnote{Este número varia em diferentes estudos, mas este parece ser o mais consensual.} nascimentos no mundo inteiro, em todos os grupos étnicos e em ambos os sexos \citep{Sommer2008, Herault2017}.

Dentre as características mais comumente relatadas e registradas nos estudos científicos, destacam-se dismorfia facial, língua desproporcionalmente larga, hipotonia muscular, baixa estatura, deficiência intelectual, problemas cardíacos congênitos e deficiências no sistema imunológico, além de um aumentado risco de desenvolvimento de leucemia na infância, bem como aparecimento precoce da doença de Alzheimer \citep{Sommer2008, KarmiloffSmith2016, Herault2017}. Destaca-se como aspecto importante a imensa variabilidade atribuída a diferenças individuais nos estudos que tratam da síndrome, que pode ser entendida tanto em relação a questões genéticas, quanto às ambientais \citep{Lanfranchi2019, KarmiloffSmith2016, Lorandi2019}.

Em uma revisão de estudos, \citet{Sommer2008} descrevem a SD em nível molecular e informam que uma proteína chamada RCAN1 (de “Regulator of Calcineurin 1”)\footnote{Anteriormente chamada proteína DSCR1 (“Down Syndrome Critical Region 1”).} é super expressa no cérebro dos fetos com SD. Essa superexpressão, em função das características dessa proteína, parece estar envolvida na patogênese da SD, de modo especial nos defeitos cardíacos e no atraso intelectual \citep{Fuentes1995}. \citet{Herault2017} consideram que os fenótipos observados na SD se devem a sensibilidade à dosagem dos genes do cromossomo 21, associados a interações entre genes e ambiente, e/ou a efeitos globais do cromossomo extra na regulação cromatina e metilação.

Embora seja comum caracterizar a SD em função dessa alteração cromossômica, especialmente porque é sua causa e a distingue de outras síndromes/transtornos/distúrbios do desenvolvimento, é importante registrar que não apenas a genética tem papel no fenótipo característico da síndrome, mas sua relação com aspectos ambientais \citep{KarmiloffSmith1998, Bishop2004a, Sommer2008, KarmiloffSmith2016, Herault2017}.

\subsection{Desenvolvimento de linguagem na Síndrome de Down}

Um dos aspectos afetados na SD é a linguagem e, neste âmbito, também é possível encontrar imensas diferenças individuais, que variam entre alguma inteligibilidade até a ausência de fala. A severidade dos problemas relativos à linguagem é, portanto, variável e atinge diferentemente os diversos aspectos linguísticos \citep{Laws2004}.

Pesquisas diversas com crianças e adolescentes com SD têm identificado perfis gerais relativos ao desenvolvimento linguístico, caracterizados por déficits na linguagem expressiva, especialmente com relação à inteligibilidade de fala, à sintaxe e à morfologia, acompanhados de déficits na memória de trabalho fonológica, assim como por vantagens relativas à compreensão de vocabulário em comparação com a idade mental não verbal \citep{Chapman2000}. Também a memória visual supera a memória fonológica \citep{KarmiloffSmith2016}.

\citet{Chapman2001} assinalam as evidências principais desses perfis, de acordo com a idade cronológica, sumarizado em \citet{Lorandi2019}. De modo geral, a compreensão é melhor que a produção, as habilidades cognitivas não verbais são superiores às de linguagem expressiva, e alguns subsistemas linguísticos podem apresentar muitas dificuldades, como é o caso da morfossintaxe (especialmente pela omissão de palavras gramaticais e de flexão verbal), da sintaxe (expressiva, em particular) e da articulação verbal \citep{Chapman2000, Chapman2001, Laws2004}.\footnote{Para outras referências sobre o assunto, sugerimos consultar \citet{Lorandi2019}.}

Seu vocabulário é maior em relação a grupos pareados por idade mental, porém parecem apresentar menor MLU (\emph{Mean Length of Utterance}), tanto em conversas informais quanto em narrativas, o que evidencia suas dificuldades em relação aos aspectos morfológicos da língua.

Um aspecto destacado em muitos dos estudos sobre a SD refere-se aos déficits na memória fonológica, o que parece subjazer a diversas dificuldades de linguagem expressiva \citep{Chapman2001, KarmiloffSmith2016}.

Para autores como \citet{KarmiloffSmith1998, Chapman2001, Laws2004} e \citet{KarmiloffSmith2016}, a variação individual no desenvolvimento da linguagem na SD parece não ser explicada apenas pela anormalidade cromossômica, uma vez que essa alteração é insuficiente para explicar os distúrbios de linguagem. Nesse sentido, supõem que as trajetórias de desenvolvimento da linguagem podem ser ligadas a uma série de efeitos genéticos “em cascata”, juntamente com a variação nos ambientes de aprendizagem da língua.

\subsection{Consciência fonológica e escrita na Síndrome de Down: Avaliação}

É bastante comum que crianças com SD apresentem severas dificuldades em relação à alfabetização, sendo que algumas delas sequer chegarão a desenvolver essas habilidades. Em função disso, muitos são os estudos que se debruçaram tanto sobre as dificuldades de leitura e escrita, como sobre possibilidades de intervenção e estimulação dessas habilidades em pessoas com SD.\footnote{Para revisões sobre este assunto, remetemos o/a leitor/a interessado/a aos textos de \citet{Lorandi2019} e \citet{Marques2018}.}

Ler e escrever em uma língua alfabética, como são o português ou o inglês, por exemplo, exige compreender, em algum nível, como funciona o sistema alfabético de escrita, em que cada um dos elementos - letras e grafemas - estarão relacionados a um som/fonema, que, por sua vez, compõe outro sistema - o fonológico \citep{Soares2017, Lorandi2021}. A própria noção de “sistema”, tal como a elabora \citet{Saussure[1916]1969}, remete à importante noção de relação de elementos. Nesse sentido, não é possível entender qualquer um desses sistemas sem levar em consideração o papel de cada elemento na composição do todo para dele fazer uso.

Assim, muitos estudos, de modo especial ao longo dos últimos trinta anos, têm se dedicado a estudar essa relação entre grafemas e fonemas como base da alfabetização - para a população com desenvolvimento típico. Para essa população, parece ser consenso que crianças com maior capacidade de pensar sobre os sons da língua e de manipulá-los intencionalmente (habilidade que chamamos de “consciência fonológica”) desenvolvem-se melhor e mais rapidamente em relação à escrita. Esse resultado é relativo a línguas alfabéticas.\footnote{Para uma revisão sobre estudos brasileiros que destacam a importância da consciência fonológica para a alfabetização, remetemos o/a leitor/a ao trabalho de \citet{Konrad2019}, que faz uma revisão integrativa desses estudos.} 

Tendo por base os estudos sobre o desenvolvimento típico, essa relação também passou a ser investigada em crianças e adolescentes com SD. Os resultados, contudo, não parecem apontar para uma relação de causalidade ou de reciprocidade tão clara como no desenvolvimento típico \citep{Lorandi2019}. As razões para esses resultados divergentes em relação ao desenvolvimento típico certamente residem nas diferenças individuais. Contudo, a literatura não nos fornece resposta acurada em relação a isso.

O que se sabe é que existem diversas dificuldades de linguagem, de memória fonológica e de consciência fonológica em si, que podem ajudar a explicar algumas das dificuldades encontradas pelas pessoas com SD para que desenvolvam a aprendizagem da leitura e da escrita.\footnote{Para uma revisão de estudos, remetemos o/a leitor/a interessado/a ao texto de \citet{Lorandi2019}.} Alguns autores acreditam que os indivíduos com SD aprendem a ler e a escrever, em geral, por rotas que não são a fonológica \citep{Cossu1993, Hulme2012}.

Apesar de resultados diversos na literatura, alguns aspectos podem ser apontados em relação à consciência fonológica em pessoas com SD, e o mais importante deles é que elas apresentam habilidades de consciência fonológica, mesmo que em menor nível em relação a sujeitos sem a síndrome \citep{Lorandi2019}. Além disso, em \citet{Lorandi2019}, destacamos outros três aspectos gerais, encontrados em um amplo levantamento bibliográfico, que fazem menção: a) ao fato de que os indivíduos com SD apresentam, dentre as tarefas de consciência fonológica, mais dificuldade nas de rima; b) à ideia de que as dificuldades em consciência fonológica estão ligadas a déficits cognitivos e c) ao fato de que existem muitas diferenças individuais, tanto em termos de desempenho, quanto em relação ao resultado de treinamento e estimulação.

À medida que mais estudos se dedicam a desvendar a consciência fonológica e a escrita em pessoas com SD, mais promissores parecem os investimentos em estimulação e intervenção, embora muitos sejam os aspectos a serem considerados.

Em pesquisa recente, correlacionamos a consciência fonológica, a escrita e a leitura de pessoas com SD a aspectos como memória fonológica, consciência fonoarticulatória (CF), vocabulário receptivo, nomeação rápida, atenção seletiva e funções executivas. Os resultados preliminares aqui apresentados referem-se a 9 participantes com SD (idade: M=14,6, DP ±5,8)\footnote{O desvio padrão elevado deve-se à dificuldade em encontrar crianças jovens com SD e que soubessem ler e escrever. Assim, a variação de faixa etária ficou entre 11 e 28 anos de idade.} e a 7 participantes com DT (desenvolvimento típico) (idade: M=7,5, DP ±0,8). A Tabela \ref{tab:cap18tab1} resume os aspectos cuja correlação com escrita e leitura, tomadas aqui como variáveis dependentes, foram estatisticamente significativas (p=0,05).


\begin{Tabela}[H]
    \centering
    \caption{\emph{Correlações para escrita e leitura}}
     \label{tab:cap18tab1}

    \begin{tabularx}{\textwidth}{XlQlQl}
    \lsptoprule
& \textbf{CF} & \textbf{Memória fonológica} & \textbf{Vocabulário} &\textbf{Aspectos~não linguísticos}&\textbf{Escrita}\\
\midrule
\textbf{Escrita} & DT & SD & DT, SD &&\\
\textbf{Leitura} &&&& SD &DT, SD\\
\lspbottomrule
    \end{tabularx}
\end{Tabela}

Primeiramente, é importante destacar que há poucos participantes e que um número mais expressivo talvez pudesse revelar outros resultados. Além disso, há questões metodológicas importantes que sempre precisam ser revisadas. Contudo, os dados da Tabela \ref{tab:cap18tab1} parecem apontar para o fato de que a CF se correlaciona à escrita apenas para crianças com DT. A memória fonológica, também em nosso estudo, parece ser um aspecto importante na SD em relação à escrita, assim como o foi para a CF em nosso estudo anterior. O vocabulário também apresenta correlação significativa em relação à escrita, tanto para DT quanto para SD. Um aspecto cognitivo não linguístico correlacionou-se à leitura na SD: as funções executivas, avaliadas por meio do teste padronizado “Torre de Londres”, que avalia a capacidade de planejamento e raciocínio. Em suma, parece haver um comportamento diferenciado entre SD e DT em alguns aspectos, enquanto outros, como vocabulário, são comuns aos dois grupos. A CF, quando tomada como variável dependente, não apresentou correlação significativa com nenhum dos aspectos avaliados na SD.

Adicionalmente, todos os participantes da pesquisa fizeram atendimento fonoaudiológico precoce e frequentam escola regular. Suas mães, na grande maioria, tiveram gestações com mais de 35 anos e apresentam níveis altos de escolaridade. Além disso, a grande maioria das crianças e adolescentes participantes demonstram gosto pela leitura e acompanham contações de história desde muito cedo. Apesar de serem alfabetizados, têm uma fala com pouca inteligibilidade, com alguns casos (2 participantes) de fala ininteligível.

Tendo em vista as relatadas dificuldades em CF e na escrita em pessoas com SD, muitos estudos buscaram investigar formas de intervenção e estimulação dessas habilidades. Os ganhos são visíveis, tímidos para alguns, mais evidentes para outros. Uma abordagem sobre intervenção e estimulação é o tópico da próxima seção.

\section{Intervenção e estimulação}

\subsection{Estudos sobre desenvolvimento atípico}

Os estudos sobre a estimulação da consciência fonológica têm ganho espaço há um bom tempo nos meios acadêmicos, na busca de perceber o quanto ela pode influenciar positivamente o processo de aquisição das habilidades de leitura e de escrita. Tal assunto está continuamente em pauta nas discussões sobre métodos mais eficazes de alfabetização, na busca não apenas de facilitar o processo de aprendizagem, mas de construir bases sólidas no conhecimento do nosso sistema linguístico.

Ainda mais desafiadora tem sido a busca por intervenções que auxiliem o processo de aprendizagem de crianças com desenvolvimento atípico, cujas particularidades de seus comprometimentos cognitivos representam um desafio para os educadores. Em comparação a pesquisas realizadas com populações típicas, os estudos com crianças atípicas ainda são menos numerosos, revelando um campo de ação a ser mais investigado. Apesar disso, há bons indicativos de intervenções realizadas nessas populações com a estimulação da consciência fonológica, mostrando-se eficazes no processo de aprendizagem e desenvolvimento desses sujeitos.

\citet{Marques2018} realizou uma revisão crítica da literatura a fim de verificar os impactos da estimulação fonológica em populações atípicas, a saber: síndrome de Williams, autismo, SD e dislexia. Os estudos analisados tiveram diferentes metodologias e objetivos, alguns visando apenas à investigação do efeito da intervenção em consciência fonológica, positivo ou não, na população estudada, outros avaliando a eficácia de \emph{softwares} e outros instrumentos de intervenção em CF, realizando comparações entre métodos. Percebeu-se que a estimulação não se diferenciou significativamente, pois trabalharam as mesmas competências, apenas de maneiras distintas.

Há várias limitações nos estudos realizados, e as mais apontadas pelos pesquisadores foram a pouca duração das intervenções e também a quantidade reduzida de participantes, dada a escassez de pessoas com tais transtornos e deficiências (em contraste com os números massivos registados nas pesquisas com crianças com desenvolvimento típico). Apesar disso, os resultados encontrados convergiram para a eficácia da estimulação da consciência fonológica também em populações com desenvolvimento atípico, mostrando que, em todas as intervenções, foi possível perceber avanços nas crianças, mesmo que, em alguns casos, tenham sido modestos.

Outro fator importante foi que a maior parte das pesquisas observou resultados, não apenas nos indicadores de consciência fonológica, mas em outras áreas, como memória e leitura. Houve avanços também nas pesquisas que visaram à estimulação da forma da palavra, o que corroborou a indicação de que as intervenções possam sempre utilizar-se de diferentes métodos para obter melhores resultados.

Quando se trata da população com desenvolvimento atípico, é preciso ter consciência de que é necessário, ainda mais do que nas populações de desenvolvimento considerado típico, perceber as singularidades de cada criança e também de sua síndrome ou transtorno. É preciso conhecer as potencialidades e limitações que estão associadas a determinadas populações, para que as intervenções sejam direcionadas de forma correta, a fim de realmente favorecer o aprendizado. 

\subsection{Estudos sobre a Síndrome de Down}

Os estudos realizados sobre consciência fonológica na SD têm aberto espaço para diversas reflexões. Em primeira instância, questiona-se se tais intervenções são realmente eficazes. Além disso, existe a questão do método. O que beneficia mais os indivíduos com SD: análise fonêmica ou visualização de palavra inteira? Há uma abordagem universal para essa população? São muitas perguntas, e algumas ainda estão sem respostas, mas os estudos vigentes dão bons indicativos a respeito dos caminhos mais promissores.

No levantamento realizado por \citet{Marques2018}, foi possível perceber que as intervenções via consciência fonológica são eficazes para desenvolver as habilidades de decodificação leitora e de escrita, já que a maioria dos estudos verificou avanços estatisticamente significativos nas crianças, mesmo com intervenções realizadas em tempo relativamente curto.

Quanto ao método, os estudos que focaram esse quesito perceberam que os grupos estimulados com intervenção específica de CF beneficiaram-se mais do que aqueles que trabalharam com palavras inteiras. O trabalho de \citet{Cupples2002}, por exemplo, realizou uma intervenção comparativa entre um método de ensino analítico e um método de visualização de palavras inteiras, mostrando que o método analítico permitiu que as crianças fizessem generalizações de leitura para palavras que ainda não conheciam, concluindo que ele é mais eficaz para o processo de decodificação leitora. Alguns estudos, como o de \citet{Murray1996}, mostraram que pessoas com SD beneficiam mais das instruções que são dadas com apoio visual da letra, realizando a correspondência grafema-fonema. 

O que os estudos mostram, de uma forma geral, é que são necessárias mais pesquisas que apontem as características genéticas intrínsecas à síndrome e auxiliem outros pesquisadores a pensarem em alternativas de ensino que correspondam às necessidades dessa população. Sabendo-se que grande parte dos indivíduos com SD possuem dificuldades de memória fonológica, memória de trabalho e também deficiência intelectual, é preciso contemplar essas áreas com metodologias abrangentes. Os dados parecem mostrar que não há uma intervenção universal. Duas pessoas com SD, com a mesma idade e mesmo nível cognitivo, irão responder de maneira diversificada às mesmas intervenções, já que as diferenças individuais são peças-chave em todo processo de aprendizagem.

\section{Onde estamos e para onde vamos: Uma proposta holística, integrativa e multidisciplinar}

Embora os estudos que partem de metodologias que contam com muitos participantes e que analisam dados estatisticamente apontem para caminhos importantes a serem considerados, um número crescente de pesquisadores tem chamado a atenção para a importância de se abordarem as diferenças individuais a partir das trajetórias de desenvolvimento de cada pessoa, que são únicas. Em \citet{Lorandi2019}, inspirada pelo trabalho de \citet{KarmiloffSmith1992, KarmiloffSmith1998} e \citet{KarmiloffSmith2010, KarmiloffSmith2012, KarmiloffSmith2016}, propõe-se uma agenda de estudos na área do desenvolvimento de linguagem na SD que volte a investir em estudos longitudinais, para acompanhar essas trajetórias únicas.

\citet[6]{KarmiloffSmith2016} revelam importantes lições para abordarmos em relação ao desenvolvimento atípico:

\begin{quote}
 “Uma coisa é clara: cientistas não podem considerar aqueles com SD como um grupo homogêneo. A consideração da variação individual em múltiplos níveis abre uma série de novas questões levantadas nesta revisão que se mantêm escondidas nos estudos em nível de grupo na SD. Então os cientistas devem levar em consideração a importância crucial das diferenças individuais, se nós queremos entender interinamente as relações entre genótipo e fenótipo emergente.”   
\end{quote}

Além disso, Karmiloff-Smith e seus colaboradores \citep[6]{KarmiloffSmith2016} consideram que

\begin{quote}
    “[t]ambém é importante notar que ter um distúrbio neurodesenvolvimental como a SD efetivamente modifica o ambiente (ambos social e físico) no qual os bebês e as crianças se desenvolvem, em termos de expectativas dos pais e de suas interações com suas crianças. Uma visão mais complexa e dinâmica é, então, requerida, de como as diferenças individuais nos ambientes social, cultural e físico da criança interagem com diferenças individuais na genética e epigenética.”
\end{quote}

Tendo em mente as imensas contribuições de Karmiloff-Smith para o entendimento do desenvolvimento atípico, decidimos investir em uma abordagem complexa e dinâmica para nossa investigação, de modo a complementar os dados quantitativos de que já dispúnhamos. Além disso, acreditamos que as explicações que estávamos encontrando ainda não são suficientes para entendermos as diferenças individuais que marcadamente caracterizam as pessoas com SD. Em nosso estudo atual, investimos no casamento entre Teoria da Complexidade e Etnografia para um mergulho mais profundo no universo da SD. Após elucidarmos os conceitos básicos dessa abordagem, explicaremos nosso novo olhar para um entendimento das diferentes trajetórias de desenvolvimento da SD.

A Teoria da Complexidade (TC) é uma perspectiva relativamente recente, interessada “em como a desordem abre caminho à ordem, em como a complexidade surge na natureza” \citep[141]{LarsenFreeman1997}. A teoria baseia-se na premissa de que há na natureza sistemas “dinâmicos, complexos, não lineares, caóticos, imprevisíveis, sensíveis às condições iniciais, abertos, auto-organizáveis, sensíveis a \emph{feedback} e adaptativos.” \citep[142]{LarsenFreeman1997}. A TC busca explicar o comportamento de sistemas dinâmicos, ou seja, que mudam com o tempo.

Os sistemas vão-se complexificando à medida que seu comportamento não pode ser explicado unicamente a partir da soma de suas partes, mas pela interação de seus componentes. Os vários fatores em interação em um sistema também agem dinamicamente, sendo que podem ter importâncias diferentes em diferentes momentos da escala de tempo \citep{LarsenFreeman1997, LarsenFreeman2012, LarsenFreeman2008}. Os elementos e os agentes de um sistema complexo mudam com o tempo, assim como os caminhos pelos quais eles influenciam uns aos outros e as relações entre si: é o que caracteriza sua dinamicidade. Na interação mediada pela linguagem, os agentes reúnem dinamicamente seus recursos (\emph{soft-assembly}) para coadaptarem-se uns aos outros e ao contexto.

As mudanças nos sistemas podem ser graduais, ou seja, lineares, ou repentinas e dramáticas, não explicáveis unicamente por seus fatores desencadeantes, o que subjaz às mudanças de fase, em que uma nova ordem se auto-organiza, o que permite a emergência de comportamentos, fenômenos e propriedades. Assim, os sistemas são caracterizados como sendo não lineares. Nesse sentido, a não linearidade está intimamente relacionada ao efeito de desproporção em relação à causa e deve-se à sua sensibilidade a condições iniciais. Essa propriedade é usualmente explicada como o “efeito borboleta”.


Nesse sentido, “os menores detalhes em um sistema complexo podem ter efeitos profundos” \citep[52]{LarsenFreeman2011}. Pensando em desenvolvimento da linguagem, a TC sugere que apenas um olhar holístico, que integre diferentes fatores em interação, pode explicar fatores emergentes. Um mapeamento que integre aspectos biológicos, cognitivos, sociais e históricos ganha relevância sob este escopo \citep{LarsenFreeman1997, LarsenFreeman2008, Beckner2009}.

As mudanças no sistema podem ser desencadeadas por fatores externos, pelo que são considerados sistemas abertos. Além disso, sob essa perspectiva, toda a aprendizagem é resultado de emergência a partir de mecanismos construídos \emph{bottom-up}. A língua/gem, então, é um sistema em constante fluxo, constante mudança, aberto a forças externas; é uma construção cujas bordas são gradientes, não discretas. A estabilidade é apenas aparente.

Fugindo de explicações reducionistas, a TC propõe-se a explicações que não descontextualizem, segreguem, idealizem ou atemporalizem a língua/gem. Tra\-ta-se de uma teoria inerentemente multidisciplinar, que se desobriga da previsibilidade e da generalização, à medida que entende que as mudanças no sistema ocorrem localmente. Nesse sentido, todo resultado de pesquisa deve ser entendido apenas como tendência, ligada ao seu contexto de ocorrência, no seu tempo e com suas limitações próprias. Cada organismo está mudando e determinando, a partir de suas experiências de vida, o que é importante em seu mundo. Por isso, mesmo que generalizações possam existir em um nível de grupo, podem falhar quando se leva em conta o individual. Aprendizes individuais podem apresentar diferentes padrões, o que, mais uma vez, destaca a importância das trajetórias de desenvolvimento tomadas em suas particularidades. Essas características parecem-nos adequadamente abrangentes e com um potencial explicativo formidável. Vejamos como ficaria uma análise da estimulação sob esse olhar.

À medida que passamos a acompanhar o grupo de mães dos nossos participantes, por meio das redes sociais e de conversas tecidas ao longo de três anos de pesquisa, passamos a entender melhor, em uma perspectiva etnográfica, o papel importante da comunidade de prática que se construiu em torno da SD na cidade em que nossos participantes residem.

Realizamos uma entrevista narrativa para que as mães do grupo (cinco mães, para estes dados)\footnote{Para fins de manter o sigilo, as mães serão nomeadas como mãe 1, 2, 3, 4 5, assim como seus filhos serão P1, 2, 3, 4 e 5. Esta pesquisa foi aprovada em Comitê de Ética sob o número 059722/2017.} nos ensinassem sua perspectiva êmica\footnote{A etnografia trabalha com dois métodos: o ético, do pesquisador para a comunidade/dados (comumente usado em pesquisa clássica), e o êmico, da comunidade para o pesquisador, ou seja, a comunidade informa o pesquisador sobre sua realidade e/ou os fatos sob investigação.} de como se desenvolvem as crianças com a síndrome. Destacamos algumas frases ditas pelas mães, que ilustram o sistema complexo e dinâmico que caracteriza o desenvolvimento da linguagem da SD, além do enriquecedor olhar dos que acompanham esse desenvolvimento em todas as circunstâncias e contextos, uma vez que são bastante dedicadas aos filhos e à comunidade de prática que criaram e de que fazem parte.

Entre as mães, estão as duas fundadoras da associação, em que se reúnem mães e crianças. Essa associação promove momentos de encontros, conversas, piqueniques, além de estimulação com profissionais, como terapeutas ocupacionais e fonoaudiólogas. Apresentaremos uma breve análise de trechos destacados, para, então, explicá-los à luz da teoria.

O que a realidade apresentada pelas mães releva é algo muito importante, explicado pela Teoria da Complexidade: embora saibamos que, estruturalmente, o cérebro é o mesmo, apesar da alteração cromossômica, as expectativas derivadas dessas condições iniciais são capazes de estabelecer possibilidades de trajetórias de desenvolvimento radicalmente diferentes \citep{KarmiloffSmith1998, KarmiloffSmith2016}. A mãe 2 nos diz: no “P2 tem síndrome, não tem, né… escapatória. O que que ele precisa fazer? Tem que fazer fisio, tem que fazer fono, a gente viu tudo o que precisava fazer”. Nenhuma criança sem síndrome nasce com a expectativa paterna e materna de “ter que” fazer tantas coisas. Há uma pressão pela superação desde a saída da maternidade. Isso também tem um impacto na rotina das crianças, que, em vez de apenas brincar, assistir à televisão, engatinhar, descobrir o mundo, têm de ir a diversos terapeutas, sem escolher se querem ou não.

A pressão pela superação aliada às expectativas, positivas ou não, estabelece condições iniciais diferentes para essas crianças. De acordo com os pressupostos da Teoria da Complexidade, os sistemas complexos são sensíveis às condições iniciais. Em outras palavras, podemos dizer que as crianças com SD vão trilhar suas trajetórias de desenvolvimento por rotas diferentes, não apenas pelas condições cromossômicas, mas talvez - e principalmente - porque suas rotinas, envoltas na pressão pela superação, serão diferentes desde o início.

As interações sociais, tão importantes para o desenvolvimento da linguagem, também sofrerão influência em função da própria condição da síndrome, conforme apontado pela mãe 1, principalmente, mas também pelas mães 3 e 4, além da questão da linguagem, que é bastante variada entre as crianças. O preconceito estrutural de uma sociedade que se imagina homogênea e que exclui o que define como sendo diferente causa impactos interacionais importantes. Sabendo-se que trajetórias de desenvolvimento são sempre complexas e dinâmicas, a forma como a criança com SD poderá reunir seus recursos para coadaptar-se às suas interações será, é claro, limitada, mas também dificultada pelos outros com quem ela vai interagir, uma vez que não se enquadram nessa aparente “homogeneidade”, da qual ninguém parece querer escapar. Essas condições, impostas não pela síndrome, mas pela forma como a sociedade reage a ela, vão restringir as interações que o sistema “aprendizagem” poderá fazer com outros sistemas/elementos/agentes, uma vez que somos seres sociais. Nesse jogo complexo de viver, tudo tem seu preço.

Da sociedade que não aceita o que considera “fora do padrão”, emerge a escola, que não sabe lidar com as pessoas historicamente excluídas de seus espaços de convívio e aprendizagem. Não é incomum ouvirem-se relatos, como os da mãe 1, dizendo que a escola não sabe atender às necessidades das crianças com SD. O resultado disso, inevitavelmente, é o desinteresse das crianças pela escola, visto que a escola parece não fazer questão de sua inclusão.\footnote{Palavra que evitamos e que usamos com cuidado. Em uma sociedade igualitária e respeitosa, ninguém precisaria ser incluído.} A forma como as crianças aceitarão as forças externas representadas pela escola, em termos de ensino e consequente aprendizagem, serão, assim, modificadas: “Se não fazem questão que eu aprenda, eu não vou querer aprender”. Isso é natural para qualquer ser humano. “Pode ser que nenhuma dessas atividades tenha graça pra ele, ele não tá entendendo, não tá enxergando direito”, diz a mãe 3. As mães, que acompanham o desenvolvimento de seus filhos no dia a dia, mostram conhecimento de que, partindo de uma condição inicial como falta quase total e/ou dificuldade de visão, por exemplo, é difícil que estímulos visuais, nos quais as aulas, em geral, são baseadas, possam ser estimulantes. Para a escola, isso é problema de atenção e deve ser resolvido trancando-se a porta, diz a mãe 2.\footnote{Excerto 7, da mãe 1: “O P1 ama música, pode dizer assim que o P1 respira música, e com o musical ela conseguiu fazer o P1 levantar a cabeça, encarar o público, sentar quando tinha que sentar em apresentações dentro de teatros, entende, assim, ele sentava, ele levantava, ele dançava, ele voltava pro lugar e sentava, quando uma professora na sala de aula não conseguia fazer ele sentar sequer na cadeira, entende? A porta era trancada, porque ele fugia pela porta [..] o ambiente tá errado, o lugar tá errado, não ele”.}

A criança com SD lida com questões pesadas desde o início da vida: seja a indiferença, seja a superproteção, seja a quantidade de terapias (que obviamente são benéficas, mas também cansativas), à sensação de não pertencimento. Na complexidade do sistema “desenvolvimento linguístico”, que não é, na perspectiva da Teoria da Complexidade, descolado do social, do cultural, do biológico, do histórico, do cognitivo e do psicológico, as interações com outros sistemas serão impactadas. Isso vai desenhando tanto a rota de desenvolvimento, quanto o próprio desenvolvimento do cérebro, em uma análise neuroconstrutivista \citep{Mareschal2007a, Mareschal2007b}, com a qual a Teoria da Complexidade se alia. Em um efeito “cascata” \citep{KarmiloffSmith1998}, o universo das crianças com SD vai se circunscrevendo ao que conhecemos como o mundo de “uma criança com Síndrome de Down”, como diz a mãe 2.

Nessas situações, criam-se as comunidades de prática, como a das mães entrevistadas na pesquisa. Nelas são criadas uma sensação de pertencimento, a qual todo ser humano aprecia, como ilustra a mãe 2, quando fala sobre “[...] enxergarem da mesma forma como eu enxergo, entende, da mesma maneira como eu enxergo, vendo eles como um todo e acreditando, principalmente, neles”. Nessa comunidade, sentem-se à vontade não apenas as crianças, que se entendem entre si, sem julgamento, mas também as mães. Ali a pressão pelo “ser igual” diminui drasticamente. É o espaço do “o que vamos fazer com as nossas diferenças, juntos”. Um outro tipo de força externa é criado no sistema aberto, que é o próprio ambiente de estimulação do qual todos fazem parte: crianças, mães e terapeutas. Os/as terapeutas são escolhidos pela comunidade/associação e precisam, portanto, estar em alinhamento com o que estas mães querem para seus filhos e suas filhas.

Nesse ambiente de acolhimento e alinhamento, os sistemas reagem e se auto-organizam de outra forma, uma vez que esses sistemas em desenvolvimento encontram outras relações, mais saudáveis e de potencialização, não descrédito. Da própria comunidade de prática, emerge um novo olhar e uma nova forma de promover intervenção e estimulação. “A [nome da associação] é minha vida!”, diz a mãe 5. A comunidade de prática mudou completamente sua vida. A mãe 5, umas das fundadoras da associação, é uma grande influenciadora em prol da inclusão em sua cidade. Sua filha, com 8 anos, lê, escreve (embora não goste muito disso), ama matemática e música. Por meio de seu trabalho junto à associação, recebe pesquisas, organiza palestras e agrega novas mães à comunidade, que só cresce. Sua força no sistema da própria cidade, no acolhimento às famílias das pessoas com SD promoveu e promove outras oportunidades de desenvolvimento e estimulação, que, por sua vez, poderão resultar em pessoas com SD com outros tipos de participação/atuação social.

Todas as mães do grupo querem que seus filhos façam parte da sociedade de forma mais ampla e autônoma. Nenhuma delas pretende circunscrever seus filhos e suas filhas ao ambiente de conforto que a comunidade promove, tanto é que todos/as eles/elas frequentam escolas regulares da cidade. Mas esse ambiente, em que há um convívio saudável socialmente para seus filhos, promove uma força no sistema e possibilidades de interação entre elementos que podem estimular outras participações e mais ânimo para as inserções na sociedade, que nem sempre acolhe. Embora muito já tenha mudado, como diz a mãe 1 no excerto 2,\footnote{Excerto 2, mãe 1: “28 anos atrás, a dificuldade era aceitar, né. Aí veio a próxima geração, que a porta \emph{tava} aberta, e eles tinham que… tinham que ser trabalhados, né. Mas, na geração dele, tu tinha que abrir a porta.”} ainda há muito espaço para melhorias.

“Ela soube achar o canal dele”, diz a mãe 1. Nossa proposta é defender que uma intervenção/estimulação eficaz deve também “achar o canal” ou “os canais” das crianças e adolescentes com SD. Isso não é possível, em uma perspectiva da Teoria da Complexidade, sem que, primeiramente, se faça um mapeamento das condições iniciais, levando-se em consideração as diferenças individuais, tão marcadamente apontadas nos mais diversos estudos sobre o desenvolvimento linguístico das pessoas com trissomia. As condições iniciais, como tentamos ilustrar aqui, vão muito além da alteração cromossômica, que ainda responde muito pouco sobre as diferenças.

Após esse mapeamento, as estratégias devem ser elaboradas tendo-se em vista a complexidade do sistema, que deve envolver, preferencialmente, uma investigação etnográfica da pessoa, da família e das comunidades de prática nas quais participam, com especial atenção para os elementos em interação no sistema “aprendizagem” e no sistema “língua”, a partir de uma noção de que esses sistemas, sensíveis a \emph{feedback}, abertos e dinâmicos, nunca são estáticos. Acreditar nessa dinamicidade significa crer no papel crucial da estimulação. Contudo, a forma como essa estimulação será feita é quase tão importante quanto a estimulação em si, ou mais. Além disso, acreditamos que é fundamental aos terapeutas e aos professores ouvirem as famílias. Assim como nós aprendemos sobre o desenvolvimento nesta perspectiva êmica, mais do que na perspectiva ética quantitativa, pensamos que ainda mais têm a ganhar a escola, bem como os profissionais que investem na sua avaliação e intervenção. 

Seja na sala de aula, promovendo aprendizagem, seja na terapia, promovendo intervenções, é preciso que os profissionais compreendam que os sistemas em desenvolvimento seguem rotas nem sempre lineares. Como muitos excertos mostram, as rotas desenvolvimentais podem acontecer “em picos”,\footnote{Excerto 13, mãe 2: “Ele tem picos de desenvolvimento”.} podem parecer estáveis e estáticas (embora só aparentemente) e também podem não responder como se espera, em virtude de que, nos sistemas complexos, há desproporcionalidade entre causa e efeito. Nesse sentido, pequenas motivações/estimulações podem gerar grandes mudanças, que levem à emergência de novos comportamentos (como a relação do P1 com a música) e, ao mesmo tempo, o que achamos que é um grande incentivo, por vezes, pode promover, aos nossos olhos, pouco efeito. O que as noções de não linearidade e de desproporção causa-efeito nos revelam é que devemos investir continuamente em estimulação e acreditar no potencial de mudança dos sistemas, abertos às influências externas. Abraçar a ideia de que não podemos prever exatamente que resultados serão alcançados, outro pressuposto da Teoria da Complexidade, exige que sejamos criativos e insistentes nas estimulações, sempre partindo da única certeza: se os sistemas são abertos e dinâmicos, eles podem ser estimulados e, a seu tempo, permitir a emergência de novas aprendizagens.

Ao final da entrevista, solicitamos às mães que dessem um conselho a professores e terapeutas. Unanimamente, elas disseram: acreditem que eles são capazes de aprender. Também perguntamos o que elas diriam para os pais e mães que estão a receber crianças com SD em suas vidas. A mãe 3 nos responde: “seja feliz com teu filho”. Com essas falas, as mães deixam duas mensagens importantes, sustentadas pela Teoria da Complexidade: a) a de que todos têm potencial para aprendizagem, em ritmos, caminhos e condições diferentes mas possíveis, e b) a de que cada uma das famílias e, consequentemente, os professores e terapeutas, podem modificar algumas dessas condições iniciais às quais os sistemas são sensíveis: em vez de enxergarem as crianças com SD como “crianças a serem estimuladas, aprimoradas” para que “supram o que lhes falta”, que procurem viver uma realidade junto delas, com elas e para que elas sejam felizes e estimuladas para desenvolverem todo o SEU potencial (em uma perspectiva vygotskyana de zona de desenvolvimento proximal).

\section{Conclusão}

Ao finalizarmos este capítulo, a impressão que queremos deixar no leitor é a de que pesquisas quantitativas e qualitativas devem se complementar, não se excluindo, quando o assunto é avaliação e intervenção para as populações consideradas como sendo de desenvolvimento atípico, que apresentam trajetórias desenvolvimentais únicas, como todo ser humano (e não atípicas).

Além disso, é nossa intenção que todo o estudante aplicado, professor dedicado e profissional competente entenda que o desenvolvimento da linguagem é caracterizado como um sistema complexo e, como tal, não é sujeito a intervenções simplistas, em que se prevejam que os mesmos estímulos possam atingir pessoas diferentes da mesma forma. O desenvolvimento linguístico é não linear, perpassado pela interação de múltiplos fatores, em complexidade, está sempre em processo de mudança e é aberto aos estímulos. Nesse sentido, todos podem beneficiar de boa estimulação, apesar - e em virtude - de suas diferenças individuais. É preciso, como diz a mãe 2, “achar o canal”. Isso exige conhecimento, dedicação e, como diria Mary Poppins, “uma colherinha de açúcar”.

\section*{Notas de reconhecimento e agradecimento}

Este capítulo é uma homenagem à brilhante Annette Karmiloff-Smith, pesquisadora que nos deixou como legado a noção de desenvolvimento, bem como a importância do olhar para as diferenças individuais das trajetórias desenvolvimentais. 

Agradecemos imensamente às bolsistas de Iniciação Científica do “Annette-Karmiloff-Smith” Research Group for Language Development, da Unisinos, coordenado pela Professora Aline Lorandi, pela revisão final deste material: Nathaly Fernandes da Silva Bonato, Maria Eduarda Bohnenberger dos Santos, Laura Tavella Dapper e Giulia Danielle Soares Rodrigues.
 
Por último, agradecemos, também, ao CNPq pelo financiamento do projeto.



{\sloppy\printbibliography[heading=subbibliography,notkeyword=this]}
\end{document}
