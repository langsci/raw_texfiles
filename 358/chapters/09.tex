\documentclass[output=paper,colorlinks,citecolor=brown,booklanguage=portuguese]{langscibook} 
\ChapterDOI{10.5281/zenodo.7233229}
\title{A repetição de pseudopalavras na avaliação fonológica clínica}
\author{Inês Catarino\affiliation{Universidade de Lisboa,  Centro de Linguística} and  Letícia Almeida\affiliation{Universidade de Lisboa, Faculdade de Letras, Centro de Linguística}}

\abstract{Este capítulo é dedicado à avaliação fonológica clínica realizada através de testes de repetição de pseudopalavras. Começaremos por refletir sobre a relevância deste tipo de tarefas na avaliação clínica, antes de nos focarmos sobre a sua pertinência no caso específico das perturbações do desenvolvimento da linguagem. De seguida, centrar-nos-emos nas variáveis linguísticas que influenciam o desempenho infantil no processamento fonológico de pseudopalavras. Por fim, apresentaremos os instrumentos de repetição de pseudopalavras atualmente disponíveis para o Português Europeu (PE).

\textbf{Palavras-chave}: pseudopalavras; testes de repetição; alterações da linguagem; Perturbação do Desenvolvimento da Linguagem (PDL); avaliação fonológica.}

\IfFileExists{../localcommands.tex}{
   \addbibresource{../localbibliography.bib}
   % add all extra packages you need to load to this file

\usepackage{tabularx,multicol}
\usepackage{url}
\urlstyle{same}

\usepackage{listings}
\lstset{basicstyle=\ttfamily,tabsize=2,breaklines=true}

\usepackage{langsci-basic}
\usepackage{langsci-optional}
\usepackage{langsci-lgr}
\usepackage{langsci-osl}
% \usepackage{./langsci/styles/langsci-lgr}
% \usepackage{./langsci/styles/langsci-osl}
% \usepackage{langsci-gb4e}

\usepackage{tikz}
\usetikzlibrary{patterns,calc}
\pgfdeclarepatternformonly{south east lines}{\pgfqpoint{-0pt}{-0pt}}{\pgfqpoint{3pt}{3pt}}{\pgfqpoint{3pt}{3pt}}{
    \pgfsetlinewidth{0.6pt}
    \pgfpathmoveto{\pgfqpoint{0pt}{3pt}}
    \pgfpathlineto{\pgfqpoint{3pt}{0pt}}
    \pgfpathmoveto{\pgfqpoint{.2pt}{-.2pt}}
    \pgfpathlineto{\pgfqpoint{-.2pt}{.2pt}}
    \pgfpathmoveto{\pgfqpoint{3.2pt}{2.8pt}}
    \pgfpathlineto{\pgfqpoint{2.8pt}{3.2pt}}
    \pgfusepath{stroke}}
    
\usepackage{stmaryrd}
\usepackage{wasysym}
\usepackage{multirow}
\usepackage{caption}
\usepackage{subcaption}
\usepackage{mathrsfs}
\usepackage{qtree}

\usepackage{linguex}


   %pminos do not split footnotes
% \interfootnotelinepenalty=10000 %Footnote in Laporte chapters has to be split SN


%\DeclareIndexNameFormat{default}{%
%\nameparts{#1}%
%\usebibmacro{index:name}%
%{\index[names]}%
%{\namepartfamily}%
%{\namepartgiveni}%
% {}% L1
% {}% L2
%{\namepartprefix}% generates spurious space L3
%{\namepartsuffix}% generates spurious space L4
%}

%  {\DeclareIndexNameFormat{default}{%
%     \usebibmacro{index:name}{\index[names]}{#1}{#3}{#5}{#7}}}

%\DeclareIndexNameFormat{default}{%
%  \usebibmacro{index:name}{\sindex[nom]}{#1}{#3}{#5}{#7}}

%\DeclareIndexNameFormat{default}{%
%  \usebibmacro{index:name}{\sindex[person]}{#1}{#3}{#5}{#7}}
%\DeclareIndexNameFormat{default}{%
%\nameparts{#1} \usebibmacro{index:name}{\sindex[person]]}{\namepartfamily}{‌​\namepartgiven}{\nam‌​epartprefix}{\namepa‌​rtsuffix}}

%\newcommand{\smiley}{:)}

%\renewbibmacro*{index:name}[5]{%
%\usebibmacro{index:entry}{#1}%
%{\iffieldundef{usera}{}{\thefield{usera}\actualoperator}\mkbibindexname{#2}{#3}{#4}{#5}}}

% \newcommand{\noop}[1]{}

%remove for final
%\overfullrule=1mm

\newcommand{\tobi}[2]}}
\renewcommand{\S}[1]{\tobi{#1}{\textsc{*}}}

% this volume references
% puts: [this volume]
% already defined: \citetv
%\newcommand{\citepv}[1]{(\citeauthor{#1} \citeyear*{#1} [this volume])}
\newcommand{\citealtv}[1]{\citeauthor{#1} \citeyear*{#1} [this volume]}

%parentheses around example number
\newcommand{\pref}[1]{(\ref{#1})}

% in-text examples

\newcommand{\lnex}[1]{\textit{#1}} %target lang word
\newcommand{\lnlit}[1]{(lit.: `#1')} %literal reading
\newcommand{\lnlat}[1]{(#1)} % latinization
\newcommand{\lntrans}[1]{`#1'} %translation
\newcommand{\lnexl}[2]%
{\lnex{#1}{} \lnlat{#2}} % ex with latinization
\newcommand{\lnexlat}[3]{\lnex{#1}{} \lnlat{#2}{} \lntrans{#3}} % ex with latinization and tranl.

%ch01
\newcommand{\co}[1]{\mbox{\textbf{#1}}}

%ch09

\newcommand{\cyrbulg}[1]{\begin{otherlanguage*}{bulgarian}#1\end{otherlanguage*}}


%ch10
\newcommand{\nlp}{{\small NLP}}
\newcommand{\mwe}{{\small MWE}}
\newcommand{\rae}{{\small RAE}}
\newcommand{\lvc}{{\small LVC}}
\newcommand{\pos}{{\small P}o{\small S}}
%\newcommand{\todo}[1]{ \textcolor{red}{#1} }

%\renewcommand{\labelenumi}{\theenumi}
%\ainamefmt{{vv}{ll}{, ff}{, jj}} % fullname

\newcommand{\biberror}[1]{{\color{red}#1}}

\newcommand{\osenovaitem}{--~}
   %% hyphenation points for line breaks
%% Normally, automatic hyphenation in LaTeX is very good
%% If a word is mis-hyphenated, add it to this file
%%
%% add information to TeX file before \begin{document} with:
%% %% hyphenation points for line breaks
%% Normally, automatic hyphenation in LaTeX is very good
%% If a word is mis-hyphenated, add it to this file
%%
%% add information to TeX file before \begin{document} with:
%% %% hyphenation points for line breaks
%% Normally, automatic hyphenation in LaTeX is very good
%% If a word is mis-hyphenated, add it to this file
%%
%% add information to TeX file before \begin{document} with:
%% \include{localhyphenation}
\hyphenation{
    Beck-man
    Ngu-yen
    back-chan-nel
    back-chan-nels
    mo-not-o-nous
    ste-reo-typ-i-cal
}

\hyphenation{
    Beck-man
    Ngu-yen
    back-chan-nel
    back-chan-nels
    mo-not-o-nous
    ste-reo-typ-i-cal
}

\hyphenation{
    Beck-man
    Ngu-yen
    back-chan-nel
    back-chan-nels
    mo-not-o-nous
    ste-reo-typ-i-cal
}

   \boolfalse{bookcompile}
   \togglepaper[09]%%chapternumber
}{}

\begin{document}
\maketitle

\section{Porquê avaliar a fonologia com um instrumento de repetição de pseudopalavras?}\label{sec:cap9sec1}
Por norma, os testes estandardizados para a avaliação da linguagem recorrem a unidades lexicais reais da língua-alvo (isto é, \emph{palavras}) para a descrição do sistema fonológico infantil.\footnote{A norma adotada na escrita deste capítulo foi a do português europeu.}
% \footnote{Na sequência de restrições decorrentes do template usado para publicação na editora Language Science Press, os capítulos que integram o volume surgem com a referência [this volume], em inglês.}
Apesar de ser crucial para a avaliação fonológica nos percursos de desenvolvimento típico e atípico, o uso de palavras para este fim clínico pode apresentar algumas desvantagens.

O processamento de palavras requer a ativação das representações lexicais construídas, que se encontram armazenadas na memória de longo prazo. Isto significa que o sucesso infantil nos testes com palavras depende, em parte, do conhecimento prévio do falante. Assim, por exemplo, quando confrontada com um estímulo como ‘\emph{caderno}’ numa prova de repetição de palavras, a criança poderá recorrer ao conhecimento já existente no seu sistema lexical para apoiar a construção da forma fonológica a ser produzida. Tarefas que envolvem a produção ou a repetição de palavras avaliam, assim, capacidades linguísticas que vão para além do simples processamento fonológico (\citealp{Casalini2007, Dispaldro2013, Gathercole2006}; entre outros). Este facto é sobretudo problemático quando pensamos na avaliação de crianças com certos percursos de desenvolvimento patológico, uma vez que estas podem possuir vocabulários mais reduzidos do que os seus pares cronológicos, resultantes de dificuldades na aquisição do léxico da sua língua materna \citep{Edwards2004, Munson2005}.

O mesmo desafio é observado na avaliação de algumas crianças bilingues, que, devido a um menor contacto com a sua língua não materna, conhecerão menos palavras nesta língua e, por isso, poderão estar em desvantagem na repetição de unidades lexicais \citep{Chiat2015}.

Há, pois, uma correlação entre o conhecimento lexical previamente adquirido e o desempenho infantil aquando da produção ou da repetição de palavras. Um vocabulário mais reduzido ou um conhecimento lexical deficitário poderão justificar, em parte, o insucesso de crianças com perfis de desenvolvimento atípico nestas tarefas, pelo que o desempenho infantil nestes testes deve ser interpretado com cautela. Testes de avaliação fonológica que implicam o recrutamento de conhecimentos prévios correm o risco de confundir a diferença com o distúrbio (\citealp{Campbell1997}; \citealp[159]{CruzSantos2009}), isto é, de identificar \emph{incorretamente} uma criança com um perfil distinto do dos seus pares (como, por exemplo, uma criança bilingue) como tendo um desenvolvimento linguístico patológico.

De modo a minimizar a interferência das pistas lexicais, sublexicais ou semânticas no processamento fonológico, alguns autores propõem o recurso a unidades que tenham uma estrutura fonológica semelhante à das palavras da língua-alvo, mas sem um significado intrínseco conceptual. Qualquer falante do PE saberá que ‘caderno’ é uma palavra da sua língua e reconhecerá ‘noderca’ como uma unidade não lexical. Não obstante, este item é constituído pelos mesmos segmentos fonéticos e pelas mesmas sílabas (mas em posição diferente) que ‘caderno’. Combinações como ‘noderca’, que, apesar do seu estatuto não lexical, respeitam a estrutura fonológica da língua-alvo, são designadas por \emph{pseudopalavras}. As pseudopalavras são comummente objeto de análise em tarefas de repetição por assumirem um carácter especialmente relevante na avaliação fonológica de falantes com desenvolvimento fonológico típico, bem como de crianças com perturbações do desenvolvimento linguístico (cf. \sectref{sec:cap9sec2} deste capítulo).

Embora a repetição de pseudopalavras (RPP) pareça ser uma tarefa relativamente simples, múltiplos autores assinalam que a sua simplicidade é ilusória \citep{Archibald2008, Coady2008, Gathercole2006}. A capacidade de repetir padrões fonológicos novos (como ‘noderca’) implica o recrutamento de múltiplos processos de natureza percetiva, cognitiva e motora, alguns dos quais também ativados na aquisição da forma fonológica de novas palavras. Tal como a aprendizagem de palavras, a RPP implica a perceção e a discriminação adequada do material verbal, seguida da codificação da informação percecionada numa representação fonológica adequada e do seu armazenamento na memória de trabalho fonológica\footnote{A memória de trabalho fonológica é uma componente da memória a curto prazo que possibilita o armazenamento do material percecionado durante um espaço de tempo limitado. Possibilita o uso ou a manipulação (isto é, a realização de operações) da informação guardada, estando, por isso, associada ao processamento temporário do material retido pelo falante.} \citep{Archibald2008, Coady2008, Gathercole2006, Estes2007, Snowling1991}. A RPP assenta, portanto, numa componente da memória que permite o armazenamento fonológico temporário da informação verbal ouvida e que é fundamental para a aquisição do conhecimento de longo prazo.

O estudo de \citet{Gathercole1990} destaca-se por ser pioneiro na descrição do comportamento infantil no contexto de RPP. Nesta investigação e em trabalhos subsequentes, os autores defendem que a RPP constitui um índice da memória de trabalho fonológica. Segundo esta abordagem, como a forma fonológica das pseudopalavras não é familiar ao participante, a sua repetição não pode ser apoiada pela ativação do conhecimento lexical, armazenado na memória de longo prazo, como se verifica no processamento de palavras \citep{Gathercole1991, Gathercole1994}. Ao invés, a repetição de pseudopalavras depende sobretudo da capacidade de codificar e de armazenar temporariamente a representação fonológica do \emph{input} linguístico na memória de trabalho fonológica. Por outras palavras, quando confrontada com uma pseudopalavra, a criança não poderá recorrer ao seu léxico para apoiar a repetição do material ouvido. Este mecanismo, que funciona para a repetição de palavras (como ‘caderno’), torna-se ineficaz face a uma pseudopalavra (como ‘noderca’) porque esta é uma combinação fonológica totalmente nova para o participante, ou seja, com a qual não houve qualquer contacto prévio. A construção da forma fonológica de uma pseudopalavra e o seu armazenamento na memória temporária dependerá, portanto, das capacidades de processamento fonológico do participante e não do seu co\-nhecimento lexical.

Assim sendo, a RPP é considerada uma tarefa de processamento linguístico, e não de conhecimento linguístico cumulativo \citep{Campbell1997}. Isto significa que, em princípio, esta prova permite testar apenas a capacidade de processar novas informações, não dependendo do conhecimento de outros aspetos linguísticos, como, por exemplo, do vocabulário ou das estruturas da língua, nem do conhecimento do mundo \citep{Archibald2008}.

Sabemos, todavia, que a RPP está significativamente correlacionada com o conhecimento e as capacidades linguísticas individuais dos participantes: por norma, as crianças que obtêm bons resultados em testes estandardizados de avaliação da linguagem são bem-sucedidas quando repetem pseudopalavras (para o PE: \citealp{CruzSantos2009, Ribeiro2011}). Por contraste, as crianças que manifestam maiores dificuldades nas tarefas de RPP são geralmente aquelas que apresentam um quadro associado ao desenvolvimento linguístico patológico (\citealp{Bishop1996, ContiRamsdem2001, Weismer2000, Gathercole1994}; entre outros), razão pela qual a RPP é considerada, atualmente, como uma tarefa de especial utilidade na avaliação clínica.

Sublinhamos ainda que a RPP apresenta algumas características que a distinguem dos testes convencionalmente aplicados na avaliação fonológica, que tradicionalmente recorrem a unidades lexicais, conforme referido. Primeiro, a repetição é uma faculdade relativamente espontânea e um comportamento natural de qualquer criança, sendo realizada com bastante frequência \citep{Gathercole2006}. Por este motivo, as provas de RPP podem ser administradas a crianças de várias idades, encontrando-se estudadas junto da população pré-escolar \citep{Chiat2007}, de alunos do ensino básico (\citealp{Weismer2000, Gathercole1994}; entre outros; no PE: \citealp{CruzSantos2009, Ribeiro2011}) e de adolescentes \citep{ContiRamsdem2001}. Alguns autores defendem que a RPP pode constituir um instrumento de rastreio promissor para uma identificação precoce de percursos linguísticos atípicos (\citealp{ContiRamsden2003a}, citado por \citealp{Archibald2008, Chiat2007}). A RPP poderá ser igualmente útil na avaliação de crianças cujas dificuldades linguísticas estão ultrapassadas apenas aparentemente: estes sujeitos, apesar de exibirem um desempenho linguístico semelhante ao dos seus pares, mantêm um défice no que concerne o processamento de pseudopalavras \citep{Bishop1996, ContiRamsdem2001}.

\largerpage
Dependendo do objetivo científico ou clínico, os instrumentos de RPP podem ser testes de fácil e rápida aplicação. As respostas infantis obtidas podem ser pontuadas pelo avaliador num curto espaço de tempo, o que possibilita o acesso quase imediato aos resultados. A RPP é também menos exigente do que certos testes estandardizados de avaliação da linguagem porque não implica que o participante interprete ou tome decisões sobre os estímulos linguísticos \citep{Chiat2007}, nem tampouco envolve a formulação ou a compreensão linguística. Por esta razão, a RPP parece ser insensível ao número de línguas a que as crianças são expostas, embora esta conclusão ainda suscite algum debate junto da comunidade científica \citep{Chiat2015}.

\section{A repetição de pseudopalavras na avaliação da Perturbação do Desenvolvimento da Linguagem}\label{sec:cap9sec2}
Dada a relação próxima existente entre o desenvolvimento linguístico e a capacidade de repetir padrões fonológicos nunca antes ouvidos, múltiplos autores têm procurado descrever o contributo da RPP na avaliação do sistema atípico (\citealp{Bishop1996, Weismer2000, ContiRamsdem2001, Friedmann2008, Gallon2007, Marshall2009}; entre outros). Do vasto conjunto de patologias analisadas, até à data, pelas comunidades científicas internacional e nacional (no PE: \citealp{CruzSantos2009, Coutinho2014}), há uma que se destaca: a Perturbação do Desenvolvimento da Linguagem (PDL).

A PDL é uma patologia de etiologia desconhecida que compromete o normal desenvolvimento da linguagem, sendo as dificuldades observadas no sistema infantil muito heterogéneas: o défice observado pode afetar só uma das componentes linguísticas do sistema-alvo ou múltiplas componentes, simultaneamente \citep{Bishop2017a, Friedmann2008, Leonard2014a}. Apesar do défice linguístico que manifestam, as crianças diagnosticadas com PDL apresentam um quadro clínico globalmente favorável à aquisição normal da língua materna, não sendo tradicionalmente observados danos neurológicos, cognitivos ou auditivos, nem tampouco alterações oromotoras ou qualquer tipo de perturbações emocionais, sociais ou sensoriais que justifiquem o seu diagnóstico. Há, pois, uma dissociação significativa e inesperada entre o desenvolvimento cognitivo e global das crianças com PDL e as suas capacidades linguísticas no domínio da expressão e/ou no da compreensão.

Não existindo uma causa conhecida que justifique o desenvolvimento linguístico atípico destas crianças, o diagnóstico de PDL é realizado tradicionalmente com base em critérios de exclusão. A procura por marcadores clínicos\footnote{Considera-se como \emph{marcador clínico} qualquer tarefa que contribua para o diagnóstico diferencial, ou seja, que consiga identificar positivamente indivíduos que manifestam uma dada patologia \citep{ContiRamsdem2001}.} que permitam a identificação \emph{positiva} desta perturbação afigura-se como uma questão de especial interesse para as comunidades clínica e científica \citep{Ribeiro2011}. \citet{Leonard2014b} defende que os índices clínicos propostos nas últimas décadas não assumem um carácter universal, ou seja, não são partilhados por todas as línguas naturais. Uma tarefa de processamento fonológico que identifique com eficácia uma criança com PDL no inglês, por exemplo, poderá não ser particularmente útil na avaliação de falantes de línguas românicas, como é o caso do português. A pesquisa por um denominador comum (isto é, um método que permita identificar crianças com suspeita de PDL falantes de diferentes línguas) é um desafio do panorama científico atual.

Um índice que tem obtido resultados promissores na identificação de crianças com PDL num vasto conjunto de línguas naturais, entre as quais se inclui o PE \citep{CruzSantos2009, Catarino2019}, é o processamento de pseudopalavras \citep{Coady2008}. Múltiplas investigações demonstraram que, independentemente da sua língua materna, as crianças diagnosticadas com esta patologia tendem a manifestar dificuldades acentuadas na repeti\-ção de pseudopalavras, que são sobretudo evidentes no processamento de unidades polissilábicas (\citealp{Gathercole1990, Gathercole1991, Gathercole1994}; para o PE: \citealp{CruzSantos2009}) ou de maior complexidade silábica (\citealp{Almeida2019, Gallon2007, Marshall2002, Marshall2003}; entre outros; para o PE: \citealp{Catarino2019}). O défice observado junto de crianças com PDL na RPP contrasta tendencialmente com o comportamento verbal obtido por crianças com desenvolvimento típico, que registam taxas de acuidade elevadas no mesmo tipo de tarefas. Por exemplo, \citet[871]{Weismer2000} mostraram, para o inglês, que um fraco desempenho no instrumento aplicado era quatro vezes mais provável de corresponder a um dos participantes diagnosticados com PDL do que a uma das crianças do grupo de controlo, sem suspeita de défice linguístico.

Para muitos investigadores, a PDL caracteriza-se por um défice primário na memó\-ria de trabalho fonológica \citep{Gathercole1990, Gathercole1994, Montgomery1995}. Conforme tem sido referido, sendo as pseudopala\-vras itens nunca antes ouvidos pelos participantes, a sua repetição dependerá, em grande parte, da capacidade de guardar a informação na memória de trabalho fonológica, devido à ausência de um suporte lexical, morfológico ou semântico.\footnote{ Este suporte estará apenas disponível no processamento de estímulos com o estatuto de \emph{palavra}, conforme mencionado anteriormente, ou no de pseudopalavras de alta proximidade lexical (cf. \sectref{subsec:in_third_section} deste capítulo).} Os referidos autores predizem que, como, no sistema das crianças com PDL, esta componente da memória está altamente comprometida, qualquer armazenamento de material fonológico será, logo à partida, deficitário. O défice na memória de trabalho fonológica poderá, então, explicar o fraco desempenho dessas crianças na RPP.

Alguns autores atestam que a magnitude das dificuldades observadas na PDL no contexto de RPP é de tal ordem acentuada que certos instrumentos de RPP podem ser usados como marcadores clínicos sensíveis e específicos da PDL (ou seja, contribuem para a identificação positiva de indivíduos que manifestam esta patologia), embora sejam mais adequados a este fim quando aplicados juntamente com outras ferramentas clínicas (\citealp{Bishop1996, ContiRamsdem2001, Weismer2000, Estes2007}; no PE: \citealp{CruzSantos2009}).

Embora não se saiba ao certo por que motivo as crianças com PDL exibem problemas tão evidentes na reprodução de pseudopalavras, havendo diferentes hipóteses sob investigação (\citealp{Almeida2019, Coady2008, Ferre2015a, Gathercole1990, Gathercole2006, Marshall2003}; entre outros), há evidência de que o processamento deficitário destas unidades não lexicais tem uma forte componente genética, sendo altamente hereditário e distinguível de outros problemas característicos desta perturbação \citep{Bishop1996}.

O contributo da RPP na avaliação do sistema fonológico atípico tem sido atestado junto de indivíduos com outros diagnósticos linguísticos, para além da PDL. Estudos prévios revelaram que os padrões de comportamento das crianças com PDL são distintos dos exibidos por crianças diagnosticadas com outras patologias como, por exemplo, a dislexia \citep{Marshall2009} e outras dificuldades associadas à leitura, bem como o autismo \citep{Williams2013}. Se replicada em investigações futuras, esta tendência poderá indicar que a RPP é uma ferramenta útil para o diagnóstico diferencial, ou seja, que permite discriminar positivamente patologias de natureza distinta.

\section{Que variáveis influenciam o desempenho infantil nos testes de repetição de pseudopalavras?}\label{sec:cap9sec3}
Múltiplas variáveis podem influenciar o processamento fonológico de pseudopala\-vras nos percursos de desenvolvimento típico e atípico. Investigações prévias revelam um paralelismo entre o comportamento destes dois grupos de crianças: participantes diagnosticados com PDL tendem a manifestar dificuldades (embora estas sejam mais acentuadas) face às mesmas variáveis que influenciam negativamente o desempenho dos seus pares \citep{Coady2008, Estes2007}. Não obstante, para as comunidades científica e clínica, continua a ser importante perceber que variáveis melhor promovem a discriminação entre as produções de crianças com desenvolvimento típico e as dos participantes com PDL.

Sublinhamos que a magnitude do défice observado na RPP por crianças com PDL não é sistemática, estando diretamente associada ao tipo de instrumento aplicado, uma vez que o objetivo de cada investigação determina as características do teste e os critérios metodológicos adotados aquando da análise das produções verbais. A comparação entre os resultados obtidos com diferentes testes de RPP é, por isso, um exercício problemático e a interpretação do comportamento verbal observado na PDL deve ser feita com cautela, sendo importante não ignorar o efeito das variáveis manipuladas durante a elaboração do instrumento e o seu impacto no processamento infantil (\citealp{Archibald2006, Gallon2007}; \citealp[189]{Estes2007}; \citealp{Snowling1991}). 

De seguida, apresentamos algumas das principais variáveis que se encontram descritas na literatura internacional e nacional sobre a RPP. São elas:

\begin{enumerate}[align=left]
\item[(i)] a proximidade lexical, uma medida subjetiva que classifica as pseudopalavras em função da sua familiaridade ao léxico real da língua;

\item[(ii)] a probabilidade e frequência fonotática, medida pelas taxas de ocorrência ou de previsibilidade de uma dada combinatória de segmentos na língua-alvo;

\item[(iii)] a extensão de pseudopalavra, medida pelo número de sílabas que um dado estímulo apresenta;

\item[(iv)] a complexidade silábica, medida pela presença ou ausência de constituintes silábicos de domínio tardio no sistema infantil.
\end{enumerate}

Na Tabela \ref{tab:cap9tab1}, descrevemos as tendências de desempenho infantil face às variáveis apresentadas, sendo que o sinal ‘>’ indica um desempenho superior.



\begin{Tabela}
\caption{Variáveis que influenciam a repetição de pseudopalavras}
\label{tab:cap9tab1}


\begin{tabularx}{\textwidth}{QQQ}
\lsptoprule
 Variáveis&	Tendências de &	Exemplos      \\
& desempenho &\\
 \midrule
 Proximidade &	Alta proximidade & \emph{gasgo},\\
 lexical & lexical > Baixa & \emph{sapatilhador}  vs.\\
 & proximidade lexical& \emph{tupanfa}, \emph{azoitadida}\\ &&\citep{Ribeiro2011}\\
\tablevspace
Probabilidade ou & Alta probabilidade& \emph{[ɡɾɐʃu]}, \emph{[tɾɐʃɐ}] vs.\\
frequência fonotática	& ou frequência & \emph{[ɡlũfe}],\emph{ [tlũ]
}\\
& fonotática > Baixa & \citep{Coutinho2014}\\
&probabilidade ou&\\
&frequência fonotática&	\\
\tablevspace
Extensão de &  Itens de reduzida& \emph{naca}, \emph{prota},\\
pseudopalavra& extensão > Itens de & vs. \emph{melanifito},\\
& maior extensão& \emph{volturacidade}\\
&&\citep{CruzSantos2009}\\
\tablevspace
Constituência silábica&	Constituintes de& \emph{lafi}, \emph{pilu}\\
& domínio precoce > & vs. \emph{fiplu}, \emph{fal}\\
&Constituintes de & \citep{Catarino2019}\\
&domínio tardio&	\\
\lspbottomrule
\end{tabularx}
\end{Tabela}


Nas próximas secções, iremos descrever o impacto de cada uma das variáveis acima listadas.

\subsection{Proximidade lexical}\label{subsec:in_third_section}
A proximidade lexical é uma medida subjetiva tipicamente avaliada com base em juízos de falantes nativos adultos. A estes falantes é pedido que avaliem, numa escala, as unidades integradas num dado instrumento de RPP, classificando-as em função da sua (não) familiaridade com palavras reais da língua \citep{Ribeiro2011}. Por norma, as pseudopalavras que integram em si palavras, sílabas ou morfemas derivacionais da língua são consideradas como mais próximas do léxico-alvo. Por contraste, as pseudopalavras cuja estrutura não integra este tipo de informação linguística tendem a ser classificadas como mais afastadas do léxico-alvo. A Tabela \ref{tab:cap9tab2} ilustra as diferenças entre estes dois tipos de pseudopalavras, sendo as unidades formadas a partir de palavras reais (\emph{gago} → \emph{gasgo}) ou através da adição de informação morfológia (como o sufixo -ador em \emph{sapatilhador}).



\begin{Tabela}
\caption{Exemplos de contraste entre pseudopalavras com baixa e elevada proximidade lexical \citep[apêndice K]{Ribeiro2011}}
\label{tab:cap9tab2}


\begin{tabularx}{.8\textwidth}{Xl}
\lsptoprule
Baixa proximidade lexical & Alta proximidade lexical\\
\midrule
azoitadida & gasgo\\
tupanfa & sapatilhador\\
péu & viogem\\
inta & aracoleta\\
imarusma & amorismo\\
\lspbottomrule
\end{tabularx}
\end{Tabela}


A aferição da proximidade lexical dos estímulos de uma tarefa de RPP é uma medida de especial importância para a análise e interpretação dos resultados obtidos. Tradicionalmente, é predita e observada, junto de crianças com desenvolvimento típico, uma correlação positiva entre o aumento da proximidade lexical e o sucesso infantil: quanto maior for o índice de semelhança lexical da pseudopalavra, maior será o acerto obtido na sua repetição (\citealp{Archibald2006, Casalini2007, Dispaldro2013}; entre outros; para o PE: \citealp{Ribeiro2011}).

As unidades de alta proximidade lexical e as unidades de baixa proximidade lexical estão associadas a diferentes tipos de processamento. Quando confrontada com uma pseudopalavra de alta proximidade, uma criança que não apresente um défice linguístico recorrerá a conhecimentos prévios para construir, armazenar e recuperar a representação fonológica do estímulo. À medida que esta proximidade diminui, a criança não consegue usar eficazmente a mesma estratégia: torna-se necessário o recurso às suas capacidades de armazenamento na memória de trabalho fonológica. Por outras palavras, a repetição de uma pseudopalavra de alta proximidade é apoiada por mecanismos lexicais, sendo semelhante ao que se verifica no processamento de palavras. Face a estímulos não lexicais como ‘amorismo’ ou ‘sapatilhador’, a criança encontrará no seu léxico diferentes palavras que ajudarão a construir a forma fonológica destas pseudopalavras: em ambas, o conhecimento prévio dos sufixos ‘-ismo’ e ‘-ador’ auxiliará o processamento fonológico infantil. A repetição de uma pseudopalavra de baixa proximidade, não podendo beneficiar da ativação deste conhecimento prévio, dependerá sobretudo das capacidades de memória de trabalho fonológica do participante \citep{Casalini2007, Gathercole1995}. Portanto, o recurso aos conhecimentos lexicais já guardados na sua memória de longo prazo facilita a repetição de pseudopalavras de alta proximidade lexical, mas não se verifica na produção das unidades de baixa proximidade.

Quanto às crianças diagnosticadas com PDL, não se sabe ao certo se a ativação do conhecimento previamente adquirido funciona de modo eficaz, ou seja, se há vantagem para a repetição dos padrões de alta proximidade lexical. \citet{Estes2007} defendem que, apesar de terem um estatuto quase lexical, estas pseudopalavras poderão ser problemáticas para os participantes com PDL devido ao léxico reduzido dos mesmos, que dificulta ou, eventualmente, impossibilita o acesso às representações lexicais armazenadas na memória a longo prazo. Por outras palavras, como se assume que a PDL se caracteriza por um conhecimento lexical deficitário \citep{Edwards2004, Munson2005}, as crianças com PDL poderão não ter mesmo apoio lexical aquando da RPP. Se esta estratégia for, de facto, inacessível, não haverá qualquer vantagem na reprodução das pseudopalavras de elevada proximidade lexical \citep{Estes2007}.

Na PDL, a diminuição da proximidade lexical será também problemática: dado o papel fundamental que a memória de trabalho fonológica assume na repetição de pseudopalavras, e o facto de esta componente da memória estar especialmente comprometida no sistema das crianças diagnosticadas com PDL, é esperado observar que o desempenho infantil clínico piore nos estímulos de menor proximidade lexical \citep{Estes2007}.

Os dados atestados em estudos prévios sobre desempenho clínico, embora escassos, confirmam que as crianças com PDL observadas exibem melhores resultados aquando do aumento da proximidade lexical das pseudopalavras, tal como tem sido observado no comportamento verbal de crianças com desenvolvimento típico \citep{Jones2010, Munson2005}.

Em suma, atualmente, múltiplos investigadores defendem que a RPP não é uma tarefa totalmente \emph{content-free} (isto é, vazia de conteúdo lexical) na medida em que a ativação do conhecimento lexical e sublexical, que se encontra armazenado na memória de longo prazo, pode apoiar o processamento de algumas destas combinações fonológicas, caso as mesmas apresentem uma estrutura muito semelhante a palavras reais da língua-alvo. Isto revela que, apesar do estatuto não lexical das pseudopalavras, não é possível eliminar por completo a influência que o conhecimento a longo prazo exerce no processamento destas unidades.

\subsection{Probabilidade e frequência fonotática}
A par do impacto da proximidade lexical, a RPP é também influenciada por outras propriedades sublexicais dos estímulos apresentados, como a frequência e a probabilidade fonotática de certas estruturas na língua-alvo. 

O conhecimento dos padrões de frequência e de probabilidade fonotática de uma dada língua diz respeito à probabilidade com que os segmentos ocorrem ou coocorrem no léxico da mesma (\citealp{Edwards2004}, para o PE: \citealp{Coutinho2014}). Mesmo sem conhecer explicitamente estes padrões de frequência da sua língua materna, qualquer criança portuguesa tê-los-á presentes no seu sistema fonológico.

Múltiplos trabalhos têm confirmado que o conhecimento destes padrões de frequência ou de probabilidade fonotática facilitam a repetição das pseudopalavras. Unidades com índices elevados de frequência fonotática na língua-alvo registam maior sucesso neste tipo de tarefa do que estímulos constituídos por estruturas mais raras ou até inexistentes na língua-alvo (\citealp{Edwards2004, Jones2010}; para o PE: \citealp{Coutinho2014}).

Tradicionalmente, é assumido que, no contexto da RPP, a influência da probabilidade fonotática é mediada pelas capacidades de vocabulário individuais. Assim, é esperado que as crianças que apresentam um vocabulário de maior extensão sejam bem-sucedidas nos testes de RPP que envolvem a manipulação da probabilidade fonotática \citep{Munson2005}. Por contraste, as crianças que possuem vocabulários mais reduzidos (como, por exemplo, as crianças com PDL) terão um contacto distinto com as combinações fonotáticas da sua língua, o que faz com que o acesso às representações fonológicas seja mais frágil. Quando expostas a uma pseudopalavra, estas crianças terão um número menor de itens lexicais que possam ser usados como analogia durante o processo de criação da forma fonológica \citep[433]{Edwards2004}. Admitindo isto, podemos prever que uma criança com um conhecimento lexical deficitário terá menos contacto com combinações menos frequentes, pelo que o insucesso nestas é esperado. 

Em conclusão, o menor conhecimento da probabilidade ou da frequência fonotática das estruturas da sua língua materna poderá explicar o fraco desempenho das crianças com percurso de desenvolvimento atípico nos testes de RPP.

\subsection{Extensão de pseudopalavra}
Numa perspetiva histórica, a extensão de pseudopalavra, que é medida pelo número de sílabas que um dado estímulo apresenta, assume um especial destaque nas investigações dedicadas às tarefas de RPP. Sabemos que o desempenho infantil no contexto de RPP espelha as tendências observadas na aquisição fonológica infantil para a produção de palavras: independentemente do perfil linguístico da criança, as unidades lexicais e não lexicais de maior extensão (como, por exemplo, os polissílabos) são mais problemáticas do que as combinações de extensão reduzida (\citealp{Bishop1996, Gathercole1990, Gathercole1994, Montgomery1995}; para o PE: \citealp{CruzSantos2009, Ribeiro2011}).

A desvantagem na repetição dos estímulos de maior extensão é sobretudo evidente no desempenho das crianças com PDL, que revelam um défice substancial na repetição de polissílabos (\citealp{Gathercole1990, Gathercole1994, Montgomery1995}; entre outros; no PE: \citealp{CruzSantos2009}). Aquando do aumento do número de sílabas, as dificuldades observadas no desempenho clínico são de tal ordem acentuadas que alguns autores defendem que a variável \emph{extensão de pseudopalavra}, no contexto de RPP, pode contribuir para a identificação da PDL (\citealp{Gathercole2006}; para o PE: \citealp{CruzSantos2009}).

Nas últimas décadas, diferentes hipóteses têm sido avançadas para dar conta do fraco desempenho observado nos itens polissilábicos das provas de RPP por crianças com PDL. Tradicionalmente, é assumido que o processamento deficitário dos alvos polissilábicos decorre de uma limitação ou de uma sobrecarga das capacidades de armazenamento da memória de trabalho fonológica do participante, que leva à degradação da informação fonológica guardada \citep{Gathercole2006}. Isto significa que, quando confrontada com uma unidade polissilábica, a criança tentará guardar o material verbal percecionado, contudo, devido à sua extensão, nem toda a informação é armazenada com sucesso. Não estando o bom armazenamento da forma fonológica assegurado, a repetição da pseudopalavra será, por isso, desviante.

Apesar de a extensão de pseudopalavra continuar a ser uma variável de grande interesse na avaliação do processamento fonológico infantil clínico, é importante sublinhar que o baixo desempenho das crianças com PDL nas tarefas de RPP pode não ser uma consequência direta e exclusiva do aumento do número de sílabas \citep{Gallon2007, Estes2007, Marshall2002, Snowling1991}. Estudos recentes têm demonstrado que o défice clínico na RPP poderá estar associado a problemas na construção das representações fonológicas e não necessariamente no seu armazenamento (\citealp{Almeida2019, Ferre2015a, Gallon2007}; \citealp[45]{Marshall2002}).

\subsection{Constituência silábica}
Os dados de aquisição das línguas naturais atestam o contributo fundamental da constituência silábica para a descrição e a discussão sobre o conhecimento fonológico infantil (\citealp{Almeida2011, Fikkert1994}; entre outros; para o PE: \citealp{Amorim2014, Freitas2017, Ramalho2017}). Sabemos hoje que a emergência e a estabilização dos segmentos estão dependentes da posição silábica que os mesmos podem preencher: uma criança poderá, por exemplo, privilegiar a produção de /ɾ/ em contexto de Coda (isto é, em posição final de sílaba) muito antes de ter dominado o mesmo segmento em Ataque ramificado (isto é, em posição inicial de sílaba preenchida por duas consoantes).

Uma avaliação linguística puramente fonética não conseguirá explicar o contraste acima descrito no sistema infantil. É necessário olhar para além do segmento e analisar o seu papel no inventário silábico da criança para dar conta do real desenvolvimento fonológico.

Sabemos também que a complexidade silábica exerce um papel promotor na avaliação e na identificação de défices da linguagem (\citealp{Gallon2007, Ferre2015a, Marshall2009}; para o PE: \citealp{Lousada2012a, Ramalho2017}). Por norma, constituintes silábicos que são problemáticos no sistema fonológico típico (como é o caso da Coda e do Ataque Ramificado) sê-lo-ão ainda mais no sistema fonológico atípico e, por isso, serão de inclusão fundamental em provas de avaliação da linguagem.

Apesar do potencial da complexidade silábica na avaliação clínica, poucos são os instrumentos atualmente desenhados para avaliar o efeito desta variável no contexto de RPP, tanto no percurso de desenvolvimento típico, como no atípico. Resultados obtidos até à data permitem concluir que o desempenho infantil na RPP é consentâneo com as tendências observadas durante o desenvolvimento fonológico: em ambos, são observadas dificuldades acentuadas no processamento de formatos silábicos complexos ou de domínio tardio (\citealp{Almeida2019, Briscoe2001, Gallon2007, Marshall2009}; entre outros; no PE: \citealp{Catarino2019}). Ou seja, as crianças tendem a produzir as consoantes em Coda ou em Ataque ramificado de forma desviante ou até a apagá-las, conforme ilustrado na Tabela \ref{tab:cap9tab3}.



\begin{Tabela}
\caption{Exemplos de produções desviantes na Coda e no Ataque ramificado -- dados de desenvolvimento típico \citep{Catarino2019}.}
\label{tab:cap9tab3}


\begin{tabularx}{\textwidth}{Xl}
\lsptoprule
Ataque ramificado&	Coda lateral \\
\midrule
{[ˈ\emph{fl}i] → [ˈfɨli]} (IN, 6;02) & {[fiˈpu\emph{ɫ}] → [fiˈtu]}  (RD, 7;11) \\  \relax
{[ˈ\emph{pl}aklu] → [ˈplatu]} (EM, 6;00)&{[ˈpi\emph{ɫ}fu] → [ˈilɨfu]}  (MIG, 5;09)\\  \relax
{[fuˈ\emph{pl}i] → [fluˈpi]} (AC, 5;11)& {[ˈkli\emph{ɫ}] → [ˈpliw]} (GM, 6;03)\\  \relax
{[ˈ\emph{kl}u] → [ˈlu]} (MP, 7;06)& {[kufa\emph{ɫ}ˈpi] → [kuɫfaˈpi]} (AG, 8;03)\\  \relax
{[ˈ\emph{pl}iwʃ] → [ˈbɾiwʃ]} (CT, 7;07)& {[ˈfi\emph{ɫ}pɐ] → [ˈfilɨpɐ]} (INV, 7;07)\\
\lspbottomrule
\end{tabularx}
\end{Tabela}


Conforme referido, alguns autores defendem que o défice linguístico característico da PDL decorre, em parte, de dificuldades na construção das formas fonológicas e não (exclusivamente) de problemas na retenção temporária do material ouvido \citep{Almeida2019, Gallon2007, Marshall2002, Marshall2003, Marshall2009}. Esta conclusão decorre de estudos recentes que demonstraram que o aumento da extensão de pseudopalavra nem sempre é problemático para as crianças diagnosticadas com PDL \citep{Estes2007}. Unidades de extensão reduzida que se encontram associadas a uma maior complexidade silábica (por exemplo, os monossílabos como ‘fli’ ou ‘fipul’, que incluem, respetivamente, um Ataque ramificado e uma Coda final) podem também estar associadas a um processamento deficiente.

\citet{Gallon2007} compararam o desempenho de crianças inglesas com PDL num teste que integrava pseudopalavras que, sendo formadas pelo mesmo número de sílabas, encontravam-se associadas a distintos graus de complexidade. Nos resultados apurados para os itens dissilábicos os autores observaram um contraste decorrente da presença de constituintes prosódicos de maior complexidade (por exemplo, [\emph{k}ɛt  >> \emph{kl}ɛt >> \emph{kl}ɛst]). Os estímulos que apresentavam uma estrutura silábica mais simples ([\emph{k}ɛt]) foram mais bem produzidos do que os estímulos classificados como sendo de maior complexidade ([\emph{kl}ɛt] e [\emph{kl}ɛst]). Tal como os autores salientam, era esperado que as crianças testadas não manifestassem um comportamento distinto na repetição dos estímulos dissilábicos em função do seu grau de complexidade. Sendo itens com apenas duas sílabas, estas pseudopalavras são consideradas como itens de extensão reduzida, pelo que se previa uma relativa facilidade na sua repetição. Aliás, segundo dados reportados por Gathercole e colegas, a configuração dissilábica tende a não ter um especial impacto no comportamento verbal atípico \citep{Gathercole1990, Gathercole1994, Gathercole2006}, não sendo expetáveis problemas de maior ordem no armazenamento do material na memória de trabalho fonológica. Não obstante, o insucesso das crianças com PDL é observado em alguns dos itens dissilábicos porque as propriedades silábicas também influenciam a capacidade de repetir pseudopalavras. Há, pois, uma correlação entre o aumento da complexidade silábica das pseudopalavras e o declínio do desempenho infantil neste tipo de tarefas.

Na Tabela \ref{tab:cap9tab4}, apresentamos exemplos recolhidos junto de crianças portuguesas que atestam o padrão acima descrito na repetição de estímulos de curta extensão (com uma ou duas sílabas).



\begin{Tabela}
\caption{Exemplos de produções desviantes em monossílabos -- dados de crianças com PDL \citep{Catarino2019}.}
\label{tab:cap9tab4}


\begin{tabularx}{\textwidth}{Xl}
\lsptoprule
Ataque ramificado&	Coda lateral    \\
\midrule
{[ˈpliwʃ] \rightarrow [ˈiwʃ]} (S2, 5;11) & {[ˈfaɫ]  \rightarrow [ˈfaɾ̥]} (S4, 6;11)\\\relax
{[ˈfli] \rightarrow [ˈfi]} (S1, 6;02) & {[ˈplaɫ] \rightarrow [ˈplaw]} (S3, 10;01)\\
\lspbottomrule
\end{tabularx}
\end{Tabela}


Evidências como as acima descritas levam à hipótese de que a PDL se caracteriza por problemas na criação das formas fonológicas das palavras e pseudopalavras: para alguns autores, o défice linguístico que caracteriza o diagnóstico da PDL resulta de dificuldades na associação dos segmentos às posições silábicas de maior complexidade (como a Coda e o Ataque Ramificado). Por exemplo, \citet{Marshall2003} observaram um padrão não esperado na repetição de Ataques ramificados: para além da tendência para a redução do grupo consonântico, há muito descrita nos dados de aquisição, as respostas das crianças com PDL apresentavam também uma ocorrência significativa de Ataques ramificados em posições não-alvo. Por outras palavras, estes participantes produziram estruturas consonânticas complexas em sílabas que não apresentavam, originalmente, qualquer complexidade (por exemplo, [\emph{f}εklətə] → [\emph{fl}εklətə] ou [drε\emph{p}əkə] → [drε\emph{pl}əkə]).

Este comportamento revela que, na PDL, os erros que alteram a estrutura prosódica-alvo nem sempre se traduzem numa simplificação prosódica, como seria expectável. Com efeito, o desempenho clínico obtido sugere que as estruturas mais complexas são possíveis dentro do sistema fonológico atípico, havendo, contudo, opcionalidade dentro do mesmo, o que explica, assim, a possibilidade de estas estruturas serem omitidas por vezes, mas produzidas noutras. Dito de outro modo, apesar de o sistema fonológico atípico estar comprometido, os constituintes silábicos complexos estão disponíveis (isto é, as crianças são capazes de os processar e produzir), sendo, contudo, mais propícios a erros de produção, o que denota problemas na construção fonológica.

Apesar de haver um certo interesse, junto da comunidade científica internacional, na descrição do processamento dos Ataques ramificados, poucos são os estudos dedicados à análise da produção das Codas no contexto de RPP e na discussão das possíveis diferenças no desempenho infantil em função da posição silábica dos constituintes \citep{Almeida2019, Ferre2015a}.

Investigações recentes revelaram que o Ataque ramificado não constitui um candidato a marcador clínico da PDL no francês, ao contrário da Coda lateral em posição medial. Participantes francófonos diagnosticados com esta patologia manifestaram um défice acentuado na repetição das Codas laterais em posição medial, mas não nos Ataques ramificados com lateral \citep{Almeida2019, Ferre2015a}. Este comportamento vai ao encontro dos dados de aquisição do francês, que atestam a precedência do domínio do Ataque ramificado, comparativamente a uma estabilização mais tardia das Codas mediais \citep{Almeida2011, Rose2000}. As referidas investigações mostraram ainda que as produções desviantes nos dados clínicos tendem a ser mais frequentes nas Codas do que nos Ataques ramificados \citep{Almeida2019}.

No processamento das Codas mediais, um dos desvios predominantes no desempenho infantil é a metátese: através deste processo, que envolve uma alteração da posição-alvo, a consoante lateral, inicialmente integrada numa Coda medial, é produzida pela criança como um membro de um Ataque ramificado. Para os autores, este comportamento confirma as dificuldades associadas ao domínio das Codas laterais mediais: sendo estas ainda instáveis no sistema infantil, a criança opta por produzir a lateral na posição de Ataque ramificado, por esta ser, para si, menos problemática \citep[170]{Almeida2019}.

É de notar que a tendência observada nos estudos supracitados, que são focados no comportamento verbal dos falantes francófonos, não é consentânea com os dados de estudos prévios internacionais (nomeadamente, para o inglês): estes atestam, junto de crianças com o diagnóstico de PDL, dificuldades acentuadas no processamento dos Ataques ramificados \citep{Briscoe2001, Gallon2007, Marshall2009}. Face a estes padrões de desempenho distintos, \citet{Almeida2019} sugeriram que o défice no processamento da complexidade silábica poderá ser dependente da fonologia da língua-alvo. Isto significa que os marcadores clínicos de natureza silábica propostos para uma dada língua, como, por exemplo, o inglês, poderão não ser úteis noutras línguas.

Em suma, dados recentes têm confirmado, junto de crianças com PDL falantes nativas de diferentes línguas naturais, dificuldades substanciais no processamento de constituintes associados a uma maior complexidade silábica. Para certos autores, isto sugere que a PDL é caracterizada por um défice na criação das representações fonológicas: por integrar constituintes silábicos mais problemáticos, a construção da forma fonológica da pseudopalavra torna-se mais vulnerável a erros. É importante sublinhar que esta proposta não invalida o contributo da memória de trabalho fonológica para a RPP. O que admite, na verdade, é que o fraco desempenho dos participantes com PDL neste tipo de tarefas não pode ser uma consequência única e exclusiva de um potencial défice nesta componente da memória, sendo também influenciado pelas propriedades silábicas dos alvos. É, assim, fundamental não ignorar a natureza ou a qualidade das representações fonológicas que têm de ser criadas, armazenadas e recuperadas pelas crianças (\citealp[53]{Marshall2009}; \citealp{Snowling1991}).

\section{Instrumentos de RPP no Português Europeu}
Nas últimas décadas, múltiplos instrumentos de RPP têm sido desenhados ou adaptados com o propósito de avaliar crianças com percursos de desenvolvimento fonológico típico e atípico, falantes de diferentes línguas naturais \citep{Coady2008}. Na Tabela \ref{tab:cap9tab5}, listamos os testes de RPP atualmente disponíveis, no nosso conhecimento, para a população portuguesa.



\begin{Tabela}
\caption{Instrumentos de repetição de pseudopalavras no PE}
\label{tab:cap9tab5}


\begin{tabularx}{\textwidth}{Q>{\raggedright}p{2.7cm}Q}
\lsptoprule
Nome&	Fonte&	Variáveis
{manipuladas} \\
\midrule
EP-CNRep & \citet{CruzSantos2009} & Extensão de {pseudopalavra}\\
\tablevspace
Instrumento de Repetição
de Pseudopalavras &
	\citet{Ribeiro2011} &	Extensão de pseudopalavra;

{Proximidade} lexical;\\
&&Complexidade articulatória global.\\
\tablevspace
Repetição de

{Pseudopalavras}

{Linguística} e
Morfologicamente
Motivadas &	\citet{Coutinho2014}&	Probabilidade fonológica\\
\tablevspace
LITMUS-NWR-EP &	\citet{Almeida2015} 

\citet{Catarino2019} &	Complexidade silábica\\
\lspbottomrule
\end{tabularx}
\tabnote{EP-CNRep = {European} Portuguese-Children Nonword Repetition, LITMUS-NWR-EP = {Language} Impairment Testing in a Multilingual Society-Nonword Repetition-European Portuguese}
\end{Tabela}


O trabalho de \citet{CruzSantos2009} destaca-se como um contributo sobre o efeito da extensão de pseudopalavra no processamento infantil, tendo como objeto de estudo um instrumento proposto por \citet{Gathercole1994} e adaptado pela autora ao PE. Validado junto de crianças com desenvolvimento típico e crianças com diagnóstico de PDL, os dados recolhidos atestam o desempenho globalmente inferior da amostra clínica. Conforme predito, a autora verifica também, na precisão das respostas obtidas, um declínio inversamente proporcional ao aumento da extensão das pseudopalavras. \citet[191--192]{CruzSantos2009} conclui que o teste aplicado constitui um marcador clínico sensível e específico da PDL no PE, sendo, aliás, mais promissor para este fim do que as outras provas de avaliação linguística usadas pela autora na mesma investigação \citep[188]{CruzSantos2009}.

\citet{Ribeiro2011} criou e aplicou uma tarefa de repetição de pseudopalavras a um conjunto de crianças sem suspeita de alterações linguísticas. A autora desenhou as pseudopalavras do teste manipulando palavras reais do léxico do PE através de processos de substituição, adição, omissão ou transformação: por exemplo, a substituição da vogal medial de ‘viagem’ dá origem à pseudopalavra ‘viogem’, a adição segmental a ‘gago’ gera ‘gasgo’, a omissão da consoante inicial de ‘tambor’ cria a pseudopalavra ‘ambor’ e a alteração da ordem silábica de ‘pantufa’ origina ‘tupanfa’. Na construção do instrumento, a autora manipula múltiplas variáveis: as pseudopalavras variam em extensão de pseudopalavra, complexidade articulatória global e proximidade lexical, entre outras variáveis. Tendo em consideração o desempenho infantil, \citet{Ribeiro2011} conclui que todas as variáveis sob foco influenciam os dados de desenvolvimento típico, conforme esperado. Assim, por exemplo, verifica-se a existência de um declínio percentual aquando do aumento da extensão de pseudopalavra, da complexidade articulatória ou da redução da proximidade lexical.

O trabalho de \citet{Coutinho2014} é dedicado à influência da probabilidade fonotática e do conhe\-cimento morfológico na RPP junto de participantes diagnosticados com dislexia. A autora conclui que elevados índices de probabilidade fonotática, a par da presença de sufixos morfológicos da língua-alvo, conduzem a uma vantagem no processamento de pseudopalavras.

Por fim, o instrumento LITMUS-NWR-EP, da autoria de \citet{Almeida2015}, permite avaliar a constituência silábica no desempenho de crianças com desenvolvimento típico e com PDL fonológica. Resultados preliminares deste teste atestam a sua capacidade discriminativa entre os percursos de desenvolvimento referidos: conforme esperado, a amostra típica revela, no seu todo, maior facilidade na repetição dos estímulos do instrumento \citep{Catarino2019}.

\section{Conclusão}
Neste capítulo, expusemos uma revisão geral sobre os testes de repetição de pseudopalavras. Sublinhámos a pertinência destas tarefas na avaliação fonológica infantil, tendo destacado o seu potencial para a identificação da PDL, uma patologia cujo diagnóstico continua a ser um desafio para a comunidade clínica. Vimos que um vasto conjunto de estudos atestam junto de crianças com PDL dificuldades substanciais no processamento de pseudopalavras, e que o desempenho das crianças é negativamente influenciado pela extensão de pseudopalavra e pela complexidade silábica. Atualmente, ainda não se sabe ao certo qual será o constituinte silábico mais problemático na RPP, nem tampouco se existem diferenças a registar quer em função da língua-alvo, quer em função do perfil (típico e/ou atípico) das crianças testadas. Também ainda não foram investigadas todas as variáveis fonológicas com potencial importância na RPP, como por exemplo o acento de palavra. Sublinhamos, contudo, que uma reflexão neste sentido seria de especial relevância não só para a discussão sobre a pertinência da RPP na identificação da PDL, mas sobretudo para a pesquisa sobre os marcadores clínicos de natureza fonológica da PDL. Mais pesquisa será necessária para entender a relação existente entre o desenvolvimento linguístico atípico e a capacidade de repetir combinações fonológicas novas.

\section*{Agradecimentos}
Financiamento FCT pelo projeto estratégico UID/LIN/00214/2019.

{\sloppy\printbibliography[heading=subbibliography,notkeyword=this]}
\end{document}
