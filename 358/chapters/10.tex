\documentclass[output=paper,colorlinks,citecolor=brown,booklanguage=portuguese]{langscibook} 
\ChapterDOI{10.5281/zenodo.7233231}
\title{Modelos teóricos aplicados à avaliação e intervenção em consciência fonológica}
\author{Dina Caetano Alves\affiliation{Instituto Politécnico de Setúbal; Universidade de Lisboa, Centro de Linguística; Centro Interdisciplinar de Investigação Aplicada em Saúde do Instituto Politécnico de Setúbal}}


\abstract{Neste capítulo, é demonstrado o modo como a avaliação e a intervenção em consciência fonológica podem ser potenciadas, integrando recursos da teoria fonológica. 
A consciência fonológica é uma competência recrutada na fase da aprendizagem da leitura e da escrita, pelo que o seu papel é crucial no contexto educacional. No contexto clínico, a maioria das patologias de linguagem oral e/ou escrita também requerem avaliação e intervenção em consciência fonológica.
Tradicionalmente, a consciência fonológica é observada e estimulada através da operacionalização de tarefas cognitivas - algumas mais simples outras mais complexas (desde a segmentação à manipulação, por exemplo) -, subestimando a interferência da natureza do material linguístico com que essas operações são realizadas. Estudos em Fonologia Aplicada, Fonologia Clínica e Psicolinguística têm demonstrado o impacto das unidades prosódicas e segmentais, da relação entre essas unidades, bem como das propriedades fonológicas dos fonemas no desempenho de tarefas de consciência fonológica, pelo que qualquer atenção dirigida a essa competência deve atentar à natureza do material fonológico contemplado. 


Este capítulo encontra-se organizado em três secções. A primeira dedicada à consciência fonológica, a segunda, à avaliação da consciência fonológica e, a terceira, à intervenção em consciência fonológica, subdividindo-se esta em duas partes, a primeira relativa à implementação de programas de intervenção em contexto educacional, a segunda relativa à implementação de programas de intervenção em contexto clínico.


\textbf{Palavras-chave}: consciência fonológica; avaliação; intervenção; alfabetização; Perturbação (do Desenvolvimento) da Linguagem.}

\IfFileExists{../localcommands.tex}{
   \addbibresource{../localbibliography.bib}
   \usepackage{langsci-optional}
\usepackage{langsci-gb4e}
\usepackage{langsci-lgr}

\usepackage{listings}
\lstset{basicstyle=\ttfamily,tabsize=2,breaklines=true}

%added by author
% \usepackage{tipa}
\usepackage{multirow}
\graphicspath{{figures/}}
\usepackage{langsci-branding}

   
\newcommand{\sent}{\enumsentence}
\newcommand{\sents}{\eenumsentence}
\let\citeasnoun\citet

\renewcommand{\lsCoverTitleFont}[1]{\sffamily\addfontfeatures{Scale=MatchUppercase}\fontsize{44pt}{16mm}\selectfont #1}
  
   %% hyphenation points for line breaks
%% Normally, automatic hyphenation in LaTeX is very good
%% If a word is mis-hyphenated, add it to this file
%%
%% add information to TeX file before \begin{document} with:
%% %% hyphenation points for line breaks
%% Normally, automatic hyphenation in LaTeX is very good
%% If a word is mis-hyphenated, add it to this file
%%
%% add information to TeX file before \begin{document} with:
%% %% hyphenation points for line breaks
%% Normally, automatic hyphenation in LaTeX is very good
%% If a word is mis-hyphenated, add it to this file
%%
%% add information to TeX file before \begin{document} with:
%% \include{localhyphenation}
\hyphenation{
affri-ca-te
affri-ca-tes
an-no-tated
com-ple-ments
com-po-si-tio-na-li-ty
non-com-po-si-tio-na-li-ty
Gon-zá-lez
out-side
Ri-chárd
se-man-tics
STREU-SLE
Tie-de-mann
}
\hyphenation{
affri-ca-te
affri-ca-tes
an-no-tated
com-ple-ments
com-po-si-tio-na-li-ty
non-com-po-si-tio-na-li-ty
Gon-zá-lez
out-side
Ri-chárd
se-man-tics
STREU-SLE
Tie-de-mann
}
\hyphenation{
affri-ca-te
affri-ca-tes
an-no-tated
com-ple-ments
com-po-si-tio-na-li-ty
non-com-po-si-tio-na-li-ty
Gon-zá-lez
out-side
Ri-chárd
se-man-tics
STREU-SLE
Tie-de-mann
}
   \boolfalse{bookcompile}
   \togglepaper[10]%%chapternumber
}{}

\shorttitlerunninghead{Modelos aplicados à avaliação e intervenção em consciência fonológica}
\begin{document}
\maketitle

\section{Consciência fonológica}\label{sec:cap10sec1}
Esta secção é composta pela \sectref{sec:cap10sec1.1}, dedicada à definição de consciência fonológica e à sua relação com o conhecimento implícito explícito, e pela \sectref{sec:cap10sec1.2}, sobre a relação entre as perturbações da linguagem e a consciência fonológica.\footnote{A norma adotada na escrita deste capítulo foi a do português europeu.}
% \footnote{Na sequência de restrições decorrentes do template usado para publicação na editora Language Science Press, os capítulos que integram o volume surgem com a referência [this volume], em inglês.}

\subsection{Definição de consciência fonológica}\label{sec:cap10sec1.1}
Quando um indivíduo apresenta habilidades reflexivas e manipulativas do código verbal, significa que tem consciência linguística. A partir das décadas de 60/70, no âmbito da Linguística aplicada à Educação, assiste-se à classificação multidimensional de consciência linguística (lexical, fonológica, morfológica, sintática, semântica e pragmática) nos planos curriculares, tanto nos Estados Unidos da América \citep{Honda2010} como no Reino Unido \citep{Hudson1992}, levando progressivamente à consolidação e à difusão de expressões como \emph{phonological awareness} - consciência fonológica.

A consciência fonológica remete para a capacidade de, voluntariamente, prestar atenção aos sons da fala e à sua organização, e não ao significado da palavra ou do enunciado que estes veiculam. Supõe a mobilização da atenção do sujeito, tornando-o capaz de refletir, analisar, julgar, comparar e/ou manipular deliberadamente as unidades fonológicas como objeto dessa atenção, subentendendo que, para tal, o sujeito dispõe de determinadas capacidades atencionais e mnésicas \citep{Chard1999, Gillon2004}.

A consciência fonológica consiste numa manifestação do conhecimento metalinguístico, o qual pertence ao domínio da metacognição, tratando-se, portanto, de um conhecimento que o sujeito detém acerca dos seus próprios processos e produtos cognitivos \citep{Gombert1990}. Embora tratando-se de uma atividade implícita, as capacidades de percecionar ou discriminar fonemas constituem tarefas comuns na avaliação da consciência fonológica.
Sabendo que o acesso ao conhecimento fonológico da criança pode dar-se através da avaliação da consciência fonológica, \citet{Shaywitz2004} assume que uma criança incapaz de aceder e/ou analisar a estrutura interna de palavras, demonstrará dificuldades no uso da linguagem oral e/ou escrita. Segundo a mesma autora, as alterações do conhecimento fonológico podem manifestar-se num nível implícito -- através de dificuldades de produção e/ou de perceção -- e/ou num nível explícito -- através de dificuldades de representação e organização de sons e/ou traços, num sistema que estabelece contrastes de significado, espelhadas nas suas competências (meta)fonológicas, nomeadamente na oralidade, na leitura e/ou escrita (sobre a relação entre oralidade, escrita e consciência fonológica, leia-se o capítulo ‘Consciência linguística: aspetos fonológicos’ de \citealp{Miranda2017}).

Segundo \citet{Alves2019}, sempre que a linguagem é usada inconscientemente, para fins comunicativos, diz-se que o processamento é implícito; diz-se que é explícito quando os falantes ou ouvintes interrompem o fluxo de comunicação para explorar, demonstrar ou manifestar reflexões sobre o material linguístico, isto é, sobre a linguagem enquanto objeto de análise. A consciência fonológica recruta, assim, competências inerentes aos tipos de conhecimento implícito e explícito. 

A ausência ou alteração de uma ou mais competências (meta)linguísticas, observável em falantes e ouvintes, constitui um sinal de alarme no âmbito do desenvolvimento ou processamento da linguagem. 

\subsection{As perturbações da linguagem e a consciência fonológica}\label{sec:cap10sec1.2}
A perturbação da linguagem primária (vulgo Perturbação do Desenvolvimento da Linguagem)\footnote{Expressão formalmente introduzida em Portugal em \citet{Castro2019}.} é uma perturbação que se caracteriza por dificuldades no desenvolvimento da linguagem não atribuíveis a perturbações de natureza neurológica, motora, sensorial ou emocional, nem a questões de bilinguismo ou multilinguismo – nesses casos, as línguas implicadas também se encontram afetadas  (\citealp{Leonard1998}\footnote{Na época designada de \emph{Perturbação Específica da Linguagem}}; \citealp{Bishop2017}). As dificuldades de linguagem tendem a manifestar-se de forma seletiva, em áreas como o desenvolvimento lexical, fonológico, morfológico, sintático, semântico e pragmático, distinguindo-se vários subtipos em função do(s) módulo(s) linguístico(s) afetado(s) \citep{Rapin1996, Friedmann2008}. Os comportamentos linguísticos manifestados são diversificados e só uma avaliação ampla, rigorosa e seletiva, das competências da linguagem nas suas diferentes áreas, permite identificar com precisão as dificuldades, aspeto essencial ao planeamento da intervenção. O impacto destas dificuldades depende da severidade do problema, da sua identificação precoce e da rede de suporte da criança. Em consequência das alterações de linguagem, estas crianças apresentam dificuldades nas aprendizagens escolares, incluindo a leitura e a escrita. As dificuldades de leitura e escrita observáveis em operações de nível elementar (nível sublexical), comuns na fase emergente da aprendizagem da leitura e da escrita, implicam sempre uma intervenção ao nível da consciência fonológica \citep{Alves2014a}, exceto quando não reunidas as condições biológicas ou mentais mínimas para responder a este tipo de intervenção.

Num estudo sobre o perfil do terapeuta da fala com atuação a nível nacional, mais de 90\% afirma ter intervenção na área pediátrica \citep{Batista2011}. Num inquérito levado a cabo em Portugal, sobre a prestação de cuidados a crianças com perturbação da linguagem, constata-se que é a população de crianças entre os 4 e os 12 anos a mais típica da casuística dos terapeutas da fala -- dados coincidentes com a média europeia -- e que a grande maioria intervém em consciência fonológica \citep{Batista2011, Kunnari2017, Alves2018, Law2017}. De entre as patologias mais recorrentes, \citet{Batista2011} refere que 95\% são Perturbação dos Sons da Fala e 71\% Perturbação Específica da Aprendizagem com comprometimento da leitura e/ou escrita, patologias estas que comummente apresentam alterações do processamento (meta)fonológico, em que importa avaliar e, eventualmente, intervir em consciência fonológica.

Na atividade clínica em linguagem na criança, estes profissionais de saúde tendem a usar tanto instrumentos estandardizados como instrumentos desenvolvidos ou adaptados por si, sendo contudo reconhecida a necessidade de mais instrumentos, formação e orientações baseadas em evidência científica e clínica para suportar a sua atuação, nomeadamente na área da consciência fonológica \citep{Afonso2015, Batista2011, Alves2018, Law2017}. 
Até à data, os instrumentos de avaliação e intervenção têm sido sobretudo suportados por teorias de desenvolvimento linguístico ou comportamental, tendo como foco as tarefas em que as crianças manifestam mais dificuldades, não atentando ao problema de base fonológico e/ou psicolinguístico, responsável por essas dificuldades. Nas propostas apresentadas neste capítulo, tanto a avaliação como a intervenção em consciência fonológica são formatadas pela teoria fonológica não linear. Assim, veremos na \sectref{sec:cap10sec2} a importância que a avaliação prosódica (silábica, intrassilábica e acentual) e segmental\footnote{Neste capítulo, o termo “segmento(s)” e as expressões como “avaliação segmental” são usados de forma ambivalente, e neutra, por remeterem simultaneamente para a natureza cognitiva dos sons da fala (segmento fonológico = fonema) e para a natureza física dos mesmos (segmento fonético = fone).} (considerando o segmento em si, os traços distintivos, suas combinações e contrastes fonológicos) têm neste processo, dado estas dimensões serem responsáveis pela estabilização do conhecimento (meta)fonológico e, na \sectref{sec:cap10sec3}, a importância de dar continuidade a este \emph{rationale}, em termos de (re)habilitação, uma vez que as intervenções sistémicas tendem a desencadear o desbloqueio de dificuldades observáveis perifericamente.

\section{Avaliação da consciência fonológica}\label{sec:cap10sec2}
A consciência fonológica é uma competência complexa que emerge pelos 3 anos e se desenvolve ao longo da idade pré-escolar e escolar; é apontada como um forte preditor de sucesso da aprendizagem da leitura, devendo, portanto, ser alvo de avaliação nas idades referidas \citep{Bradley1978}. 

As unidades segmentais que representam o conhecimento fonológico (os fonemas), pertencentes ao domínio da Cognição, distinguem-se das unidades segmentais que constituem o \emph{output} fisicamente mensurável dessas unidades (os fones), do domínio da Física. Os fonemas são, assim, unidades abstratas que usamos para representar o conhecimento sobre as unidades sonoras da língua, conhecimento esse armazenado no cérebro, passível de ser processado a nível cognitivo central \citep[362]{Lousada2017}. 

Estas unidades fonémicas instanciam constituintes silábicos à medida que estes vão ficando disponíveis no conhecimento fonológico da criança, daí surgir o conceito de consciência silábica e intrassilábica, além do conceito de consciência fonémica ou segmental. O acento, tal como a sílaba, é tratado a nível prosódico e constitui um nível da consciência fonológica a observar, e eventualmente tratar, dadas as dificuldades frequentemente relatadas na literatura. Com efeito, apesar de os alunos terem adquirido grande parte da estrutura da sua língua no final do 1.º Ciclo, estes manifestam dificuldades em identificar a sílaba tónica de palavras \citep{Freitas2019} tal como alunos de ciclos seguintes, nomeadamente, do 3.º Ciclo e do Secundário \citep{Araujo2004}. Por esta razão, importa incluir essa dimensão no processo de avaliação da consciência fonológica.

Para o português, existem já alguns instrumentos de avaliação da consciência fonológica -- alguns já aferidos e/ou normalizados, outros não. Todos contemplam o inventário segmental e, de forma menos generalizada, as propriedades fonológicas do sistema-alvo. A avaliação do inventário segmental deve cobrir todas as propriedades fonológicas da língua, quer consonânticas quer vocálicas, mas mais importante do que garantir a avaliação das 19 consoantes do português, importa que o instrumento permita observar o impacto dos traços, das suas combinações e oposições (sistema contrastivo), na estabilização da consciência dos aspetos segmentais da língua (sobre este tema da avaliação e intervenção do/em conhecimento implícito, leiam-se, os capítulos de \citetv{chapters/04}, \citetv{chapters/05}, respetivamente).

Em \citeyear{Alves2012}, \citeauthor{Alves2012} verificou que, a nível metassegmental, a combinação de traços consonânticos que estabiliza mais precocemente no conhecimento fonológico das crianças é [+contínuo, $-$soante], seguida de [+contínuo, +soante], de [$-$contínuo; +soante] e de [-contínuo; -soante] (ver Tabela \ref{tab:cap10tab1}), diferindo assim do que se observa a nível do desenvolvimento segmental. 



\begin{Tabela}
\caption{Ordem de emergência do Modo de Articulação, a nível metassegmental (traços distintivos e respetivas classes naturais)}
\label{tab:cap10tab1}


\begin{tabularx}{\textwidth}{QQcQcQcQ}
\lsptoprule
\textbf{Traços}\newline \textbf{distintivos} & [+contínuo, $-$soante] & \multirow{2}{*}{{>}} & [+contínuo, +soante]  & \multirow{2}{*}{{>}} & [$-$contínuo, +soante]  & \multirow{2}{*}{{>}} & [$-$contínuo, $-$soante]  \\
\tablevspace
\textbf{Classes {naturais}} & fricativas & & líquidas && nasais &&  oclusivas\\
\lspbottomrule
\end{tabularx}
\end{Tabela}


Ainda a nível metassegmental, a especificação dos valores do traço [vozeamento] não apresenta uma tendência consistente. Já em termos da propriedade Ponto de Articulação, o Coronal [+anterior] surge em primeiro lugar, seguido dos traços Labial e Coronal [$-$anterior] e, por fim, do traço Dorsal, tal como o ilustra a Tabela \ref{tab:cap10tab2}.


\setlength{\tabcolsep}{4pt}
\begin{Tabela}
\caption{{Ordem de emergência do Ponto de Articulação, a nível metassegmental (traços distintivos e respetivas classes naturais)}}
\label{tab:cap10tab2}


\begin{tabularx}{\textwidth}{QQclcQcQ}
\lsptoprule
\textbf{Traços} \textbf{distintivos} & Coronal [+anterior] & {>} & Labial & e & Coronal [$-$anterior] & {>} & Dorsal\\
\tablevspace
\textbf{Classes} \textbf{naturais} & dentais e alveolares & {>} &  \multicolumn{3}{l}{bilabiais, labiodentais e palatais} & {>} & velares\\
\lspbottomrule
\end{tabularx}
\end{Tabela}


Os fonemas organizam-se em constituintes silábicos no domínio da sílaba, ficando gradualmente disponíveis no conhecimento prosódico das crianças (sobre avaliação do conhecimento implícito prosódico, vejam-se, os capítulos de \citetv{chapters/06}, \citetv{chapters/08}). É, assim, fundamental que essa dimensão também esteja representada na avaliação da consciência fonológica, tendo por base o modelo de Ataque e Rima descrito para o português europeu em \citet{Freitas2017}. De acordo com \citet{Andrade1993} e \citet{Vigario1993}, os formatos silábicos V e CV constituem os formatos mais frequentes no português europeu - por oposição aos formatos CCV, CVC ou, ainda, CCVC. A par da frequência de ocorrência, sabe-se também que palavras constituídas por formatos V e CV apresentam melhores desempenhos em tarefas de consciência fonológica do que palavras com formato CCV, CVC e CCVC \citep{Rios2009, Afonso2015}. Assim, ao nível da estrutura silábica, as autoras verificaram um efeito dessa variável, tornando a tarefa mais difícil quando os estímulos apresentam Ataque ramificado e/ou Rima ramificada, pelo que a avaliação da consciência fonológica também deve contemplar essas unidades e os factos relatados. As mesmas autoras também observaram melhores resultados em palavras dissilábicas do que em palavras monossilábicas ou polissilábicas, aspeto igualmente a ter em conta em contexto de avaliação. No estudo de \citet{Afonso2015}, apenas uma tarefa de consciência fonológica foi aplicada, a segmentação - metodologia relevante para testar o efeito das propriedades associadas a cada uma das unidades fonológicas testadas -, tendo-se verificado que é mais fácil executar tarefas com sílabas do que com palavras, e com palavras do que com constituintes silábicos ou fonemas (exemplo por ordem crescente de dificuldade: segmentação de palavras em sílabas > segmentação de frases em palavras > segmentação de palavras em constituintes ou fonemas).

A propósito da consciência de palavra, em \citeyear{Cardoso2011}, \citeauthor{Cardoso2011} desenvolveu um estudo com o objetivo de descrever o modo como crianças de idade pré-escolar, 4 e 5 anos, e escolar (1º e 2º anos do 1º ciclo do ensino básico) segmentam frases em palavras funcionais e lexicais. As palavras lexicais testadas obtiveram taxas de sucesso elevadas. Em termos das palavras funcionais, o estatuto prosódico das mesmas assumiu uma forte relevância neste estudo já que as crianças apresentaram maior consciência das palavras funcionais acentuadas comparativamente com as não acentuadas. Tais resultados reforçam a necessidade de a unidade palavra também constar nas tarefas de avaliação da consciência fonológica, considerando as diferentes categorias que esta assume a nível sintático, semântico e outros.
Diversos estudos desenvolvidos no domínio da avaliação da consciência fonológica, da leitura, da escrita e de outras competências metafonológicas recorrem a pseudopalavras, a fim de garantir a não interferência do conhecimento lexical no desempenho dos sujeitos (sobre o conceito e o uso de pseudopalavras, veja-se \citetv{chapters/09}). Ainda que implicando um maior esforço de processamento \citep{Moojen2009}, as pseudopalavras invalidam o acesso a informações mnésicas, visuais (gráficas), ortográficas (regras contextuais e aspetos etimológicos, por exemplo), semânticas e outras \citep{Grainger2003}. Por esta razão, é comum incluir pseudopalavras no processo de avaliação da consciência fonológica.

Segundo \citet{Chard1999}, as tarefas de consciência fonológica têm diferentes graus de dificuldade, qualquer que seja a unidade fonológica avaliada. Em 2015, \citeauthor{Afonso2015} verificou que, a nível lexical e silábico, a segmentação constitui uma tarefa menos complexa do que a identificação e esta, por sua vez, do que a omissão. Para o caso específico do fonema, \citet{Alves2012} apresenta uma escala de complexidade que aponta para a síntese/reconstrução como sendo a tarefa mais fácil, seguida, em conjunto, das tarefas de identificação/deteção e de segmentação, sendo já mais difícil a tarefa de exclusão/supressão, e, por último, a tarefa de manipulação.

Face às diferentes escalas descritas na literatura – diferenças decorrentes da disparidade categorial do material fonológico usado nas tarefas em apreço -, seria recomendável que a avaliação da consciência fonológica contemplasse os diferentes tipos de tarefa mencionadas, nomeadamente tarefas de segmentação, síntese, identificação, supressão e manipulação, em conformidade com as unidades-alvo a que devem estar associadas (palavra e pseudopalavra, sílaba, constituintes silábicos, acento e fonema). 

É portanto fulcral que a avaliação da consciência fonológica se proponha observar a representação fonológica das crianças ao nível da palavra, da sílaba, dos constituintes silábicos, do acento e dos fonemas (embaixadores dos traços fonológicos e do sistema contrastivo), através de palavras e pseudopalavras, e considerando o papel das diferentes tarefas de avaliação da consciência fonológica.

Seguem-se abaixo tarefas exemplificativas da avaliação desta competência, extraídas do instrumento eConF.IRA \citep{Castronoprelo}, constituído por 18 provas, algumas com três itens outras com oito.

\begin{quote}
\textit{Segmentação de frases em palavras} (consciência lexical)

\*[ao visualizar uma imagem representativa] Estás a ver o que estes meninos estão a fazer? Vais ouvir uma frase inteira e depois a mesma frase separada em palavras, com uma pausa entre cada palavra. “Esta bola é grande”. [pausa de 1 segundo entre cada palavra] “Esta – bola – é – grande”.

 Agora és tu. Ouves a frase, repetes e depois divides em palavras. Faz uma pausa entre cada palavra.


 \begin{Tabela}
 \begin{tabularx}{.8\textwidth}{Xl}
 \lsptoprule
 \textbf{Frase-estímulo} & \textbf{Resposta alvo}\\
 \midrule
 A Joana pinta o livro. & A | Joana | pinta | o | livro.\\
 Ela bebe-o & Ela | bebe | o.\\
 \lspbottomrule
 \end{tabularx}
 \end{Tabela}
 
 \end{quote}


 Logo nesta primeira tarefa, em ambas as frases-estímulo, usam-se palavras acentuadas (como o nome próprio \emph{Joana} e o comum \emph{livro}), mas também palavras átonas (artigos e pronomes átonos), que as crianças tendem a aglutinar com as tónicas \citep{Rios2009, Afonso2015} -- [a Joana] [pinta] [o livro] e [Ela] [bebe-o] -- exceto quanto ostentam um nível de consciência de palavra mais desenvolvido. 

\begin{quote}
\textit{Segmentação silábica de pseudopalavras} (consciência silábica)

Agora vamos separar em bocados palavras inventadas. A palavra é “tifeca”. Em bocados fica [pausa de 1 segundo entre cada sílaba] “ti – fe – ca”.

Agora és tu. Ouves uma palavra, repetes e depois divides em bocados, com uma pausa entre eles.



\begin{Tabela}
\begin{tabularx}{.8\textwidth}{Xl}
\lsptoprule
\textbf{Pseudopalavra-} & \textbf{Resposta-alvo e respetiva }\\
\textbf{estímulo} & \textbf{transcrição fonética}\\
\midrule
*lofa & lo | fa  >  [ˈlɔ.fɐ]\\
*ileçunte & i | le | çun | te  >  [i.lɨ.ˈsũ.tɨ]\\
\lspbottomrule
\end{tabularx}
\end{Tabela}

\end{quote}

Nesta tarefa, apresenta-se um primeiro estímulo constituído por duas sílabas e com formato silábico simples (*lofa: CV.CV), portanto mais fácil de segmentar \citep{Afonso2015}. Já o segundo estímulo, *ileçunte, é polissilábico e apresenta uma estrutura silábica passível de ser adulterada (de V.CV.CV.CV para VC.CV.CV), dada a natureza fonética de [ɨ] e a possibilidade de /l/ poder instanciar tanto um Ataque simples como uma Coda.

\begin{quote}
\textit{Supressão de sílaba final} (consciência silábica)

Neste jogo, vais tirar o último bocado de cada palavra, para fazer uma palavra diferente. Por exemplo, na palavra “asma” há dois bocados, [pausa de 1 segundo entre cada sílaba] {“as – ma”. Se tirares o último bocado (que é “ma”), fica: “ás”.}

Agora vais fazer com outras palavras.



\begin{Tabela}
\begin{tabularx}{.8\textwidth}{Xl}
\lsptoprule
\textbf{Palavra-estímulo} & \textbf{Resposta-alvo}\\
\midrule
nota & nó\\
irmã & ir\\
\lspbottomrule
\end{tabularx}
\end{Tabela}

\end{quote}

À semelhança do que se propôs na tarefa anterior, o estímulo \emph{nota} é dissilábico e apresenta uma estrutura silábica simples (CV) em ambas as sílabas. O estímulo \emph{irmã}, também dissilábico, apresenta uma estrutura silábica CVC.CV, sujeita a ser simplificada para CV.CV.CV, dada a possibilidade de /ɾ/ poder instanciar tanto uma Coda como um Ataque simples, resultando numa resposta como “i” (a letra, por exemplo) em vez de “ir” (resposta esperada).

Adicionalmente, constata-se que as respostas corretas recaem ora numa sílaba tónica (\emph{nó}), ora numa sílaba átona (\emph{ir}), permitindo assim testar o desempenho das crianças na execução de tarefas de consciência fonológica, quando confrontadas com a variável acento fonológico \citep{Afonso2015, Freitas2019}.

\begin{quote}
\textit{Inversão de sílabas} (consciência silábica)

Agora vamos trocar a ordem dos bocados nas palavras. Vais ver que vão aparecer palavras diferentes. Por exemplo, “ovo” tem dois bocados: [pausa de 1 segundo entre cada sílaba] o – vo. Se trocares a ordem dos bocados, fica “voou”.

Agora vais fazer com outras palavras.



\begin{Tabela}
\begin{tabularx}{.8\textwidth}{Xl}
\lsptoprule
\textbf{Palavra-estímulo} & \textbf{Resposta-alvo}\\
\midrule
avô & voa\\
barba & babar\\
\lspbottomrule
\end{tabularx}
\end{Tabela}

\end{quote}


Esta tarefa é psicolinguisticamente complexa -- é de tipo manipulativo -- e requer a capacidade prévia de segmentar palavras em sílabas. A primeira, \emph{avô}, apresenta estruturas silábicas simples (V.CV) comparativamente à segunda (CVC.CV), esperando-se por isso que as crianças tenham mais sucesso com \emph{avô} > \emph{voar} do que com \emph{barba} > \emph{babar}.

\begin{quote}
\textit{Identificação de rima} (consciência intrassilábica)

Agora vais ver quatro imagens. Uma em cima e 3 em baixo. Uma destas palavras [apontar para as de baixo] rima com a de cima [apontar para as de cima]. [nomear e apontar] “Lata, pata, laço, garrafa”. Qual é que rima com a de cima? [pausa de 3 segundos]

“pata”

Agora vais fazer com outras imagens. Dizes o nome das imagens que vês e dizes qual é que rima com a de cima.



\begin{Tabela}
\begin{tabularx}{.8\textwidth}{Xll}
\lsptoprule
\textbf{Estímulo-alvo} & \textbf{Rima e distratores} & \textbf{Resposta-alvo}\\
\midrule
papel  &  mel, pano, folha & mel\\
caracol  & caderno, sol, formiga & sol\\
\lspbottomrule
\end{tabularx}
\end{Tabela}

\end{quote}

A tarefa de identificação de rima é uma tarefa relativamente simples e recorrente nas provas de consciência fonológica. Em ambos os conjuntos apresentados, o estímulo-alvo tem mais sílabas do que a resposta-alvo, além de que cada um dos distratores começa pela mesma sílaba que a dos estímulos-alvo (\emph{pa}pel-\emph{pa}no e \emph{ca}racol-\emph{ca}derno), dificultando assim a tarefa do ponto de vista psicolinguístico.

\begin{quote}
\textit{Segmentação de rima} (consciência intrassilábica)

Vais ouvir agora uma palavra pequena e depois a mesma palavra, separada em dois bocados, com uma pausa entre cada bocado. Assim: “mão” [pausa de 1 segundo entre ataque e rima] “m – ão” “Agora és tu. Ouves a palavra, repetes e divides em dois bocados, com uma pausa entre eles.



\begin{Tabela}
\begin{tabularx}{.8\textwidth}{Xl}
\lsptoprule
\textbf{Palavra-estímulo} & \textbf{Resposta-alvo}\\
\midrule
flan & fl | an\\
pau & p | au\\
\lspbottomrule
\end{tabularx}
\end{Tabela}

\end{quote}

Esta consiste numa tarefa de rima fonológica, pouco comum nas provas de consciência fonológica. Nela pretende-se que a criança seja capaz de segmentar o Ataque da Rima, pelo que o primeiro estímulo apresenta um Ataque ramificado -- mais difícil de processar \citep{Afonso2015} -- instanciado por fonemas [+contínuos] -- mais fáceis de processar \citep{Alves2012} -- enquanto o segundo apresenta o inverso, isto é, um Ataque simples -- mais fácil de processar \citep{Afonso2015} -- instanciado por um fonema [$-$contínuo] -- mais difícil de processar \citep{Alves2012}.

\begin{quote}
\textit{Segmentação fonémica de pseudopalavras} (consciência fonémica) 

Vais ouvir agora uma palavra inventada e dividir em bocadinhos muito pequenos. Por exemplo, se dividirmos a palavra Séfa em bocadinhos pequenos fica [pausa de 1 segundo entre cada fonema] “s – é – f – a”.

Agora és tu. Ouves a palavra, repetes e divides em bocadinhos mais pequeninos. Não te esqueças de fazer uma pausa entre os bocados pequeninos.



\begin{Tabela}
\begin{tabularx}{\textwidth}{Xl}
\lsptoprule
\textbf{Pseudopalavra-}\textbf{estímulo} & \textbf{Resposta-alvo e respetiva }\textbf{transcrição fonética}\\
\midrule
*vula & v | u | l | a  >  [v.ˈu.l.ɐ]\\
*tabeca & t | a | b | e | c | a  >  [t.ɐ.b.ˈɛ.k.ɐ]\\
\lspbottomrule
\end{tabularx}
\end{Tabela}
\end{quote}


As tarefas de segmentação fonémica costumam emergir mais tardiamente no desenvolvimento da consciência fonológica e sabe-se que o uso de pseudopalavras interfere no desempenho das crianças, tornando a atividade mais fácil para uns, por não haver contaminação lexical, mas mais difícil para outros por razões eventualmente mnésicas ou até pela falta de suporte lexical \citep{Alves2012}.
Nos exemplos dados, apresenta-se, primeiro, uma pseudopalavra constituída apenas por fonemas consonânticos [+contínuos], mais fáceis de segmentar (* \emph{vula}) por oposição a outra constituída por consoantes [$-$contínuas], mais difíceis de segmentar (* \emph{tabeca}) \citep{Alves2012}.

\begin{quote}
\textit{Supressão de fonema inicial} (consciência fonémica)

Vais agora tirar o primeiro bocadinho pequenino das palavras para fazer uma palavra diferente. Por exemplo “porta”. O primeiro bocado pequenino é o ‘p’. Se tirarmos o ‘p’ à palavra “porta”, fica “horta”.

Agora vais fazer com outras palavras.



\begin{Tabela}
\begin{tabularx}{.8\textwidth}{Xl}
\lsptoprule
\textbf{Palavra-alvo} & \textbf{Resposta-alvo}\\
\midrule
chuva & uva\\
disco & isco\\
\lspbottomrule
\end{tabularx}
\end{Tabela}

\end{quote}

Nesta tarefa, espera-se que a supressão do /ʃ/ de \emph{chuva} tenha mais sucesso do que a supressão do /d/ de \emph{disco}, por o primeiro fonema ser [+contínuo] e o segundo [$-$contínuo]. Contudo, também será possível encontrar algum efeito da ortografia em crianças alfabetizadas, nomeadamente pelo confronto com o dígrafo \emph{ch} de \emph{chuva}.

\begin{quote}
\textit{Identificação da sílaba tónica} (consciência de acento)

Agora vais descobrir qual é o bocadinho da palavra que tem mais força. Por exemplo, na palavra “morada”, o bocadinho com mais força é “rá”.

Agora vais dizer-me qual o bocadinho com mais força nestas palavras.



\begin{Tabela}
\begin{tabularx}{.8\textwidth}{Xl}
\lsptoprule
\textbf{Palavra-alvo} & \textbf{Resposta-alvo}\\
\midrule
capa & cá\\
lápis & lá\\
\lspbottomrule
\end{tabularx}
\end{Tabela}

\end{quote}

A identificação da sílaba tónica também não é uma tarefa que integre sistematicamente as provas de consciência fonológica, apesar do seu papel na relação entre oralidade e escrita \citep{Miranda2017}. No exemplo dado, ambas as sílabas/respostas-alvo coincidem com a sílaba tónica da palavra-alvo (\emph{ca}pa e \emph{lá}pis), contudo, uma delas -- \emph{lápis} -- apresenta uma sílaba pesada (lá\emph{pis}), podendo, portanto, interferir no sucesso da tarefa. 

Com base nos exemplos apresentados, conclui-se que o controlo psicolinguístico das tarefas e o controlo fonológico dos estímulos (segmental e prosódico), suportado nos pressupostos da Fonologia não-linear, facilitam a identificação das variáveis que parecem favorecer ou desfavorecer o desempenho de crianças em tarefas de consciência fonológica. Tal constatação proporciona a caracterização de perfis (meta)fonológicos mais precisos e, por conseguinte, a definição de planos de intervenção mais adequados, quer do ponto de vista clínico, como educacional. 

\section{Intervenção em consciência fonológica}\label{sec:cap10sec3}
Os programas de intervenção em consciência fonológica têm como objetivo comum (re)habilitar as competências (meta)fonológicas de crianças com ou sem perturbação da linguagem. Ainda que tendo um objetivo comum, esses programas diferem em termos das tarefas aplicadas e/ou das unidades-alvo, já que, na literatura, não existe consenso quanto a esses aspetos \citep{Carvalho2012}.

Apesar das diferenças observadas, estas intervenções têm-se revelado eficazes, tanto no tratamento como na prevenção de determinadas dificuldades de leitura e de escrita.

A questão prévia à implementação de qualquer programa de intervenção em consciência fonológica prende-se não só com os critérios a considerar na base da escolha dos estímulos-alvo e das tarefas, como também do público a que se destina: crianças com ou sem patologia, e qual. Os grupos sem patologia juntam crianças em idade pré-escolar, a frequentar o jardim-de-infância, e podem integrar uma ou mais crianças com dificuldades diversas, de natureza linguística ou outras. Para estes grupos, devem ser desenhados programas a implementar em contexto educacional. Já os casos com patologia devem ser intervencionados de acordo com as dificuldades específicas observadas.

\subsection{Intervenção em consciência fonológica em contexto educacional}
No contexto educacional, é frequente verificarem-se dificuldades na execução de tarefas de consciência fonológica \citep{Alves2012, Carvalho2012}, pelo que a estimulação desta competência é imprescindível \citep{Freitas2007, Carvalho2012}. Estas dificuldades afetam sobretudo as unidades lexicais, silábicas, intrassilábicas, segmentais e acentuais, são observáveis em diferentes faixas etárias \citep{Araujo2004, Alves2012, Castelo2012, Afonso2015, Freitas2019} e evidenciam o efeito da tarefa aplicada \citep{Afonso2015}.

No contexto educacional, \citet{Ebbels2017} enquadram a atuação do profissional especializado em linguagem num modelo de intervenção multinível. De acordo com esse modelo, existem três níveis de intervenção. O nível 1, designado de ‘universal’, supõe que todas as crianças de um mesmo grupo educacional - da mesma turma - recebam a mesma estimulação, isto é, o mesmo programa de intervenção, para otimização das suas competências (meta)linguísticas, neste caso (meta)fonológicas. O nível 2, designado de ‘direcional’, dirige-se especialmente a criança em risco de desenvolver perturbações da linguagem. O programa implementado nesse nível tem como principal objetivo a prevenção da emergência de dificuldades nessa população. Tanto no nível 1 como no 2, a intervenção é realizada de forma indireta, no sentido em que é implementada pelo educador de infância, capacitado e supervisionado periodicamente pelo profissional especializado em linguagem. O nível 3, dirigido a crianças com alterações diagnosticadas, é explicitado na secção seguinte, ‘Intervenção em consciência fonológica no contexto clínico’.

Quando cruzado com o modelo de intervenção multinível, definido pelo Ministério da Educação ao abrigo do DL 54/2018, verifica-se que o modelo de \citeauthor{Ebbels2017} e de seus colaboradores se coaduna com a proposta educacional em vigor. As medidas universais, presentes em ambas as propostas, são transversais a todas as crianças do grupo (nível 1). As medidas seletivas são sobretudo dirigidas à população em risco de desenvolver problemas de linguagem (nível 2). As medidas adicionais são as mais recrutadas para a população com dificuldades específicas, nomeadamente com perturbações da linguagem e outras (nível 3).

Para atuar em conformidade com estas propostas, é necessário proceder-se a uma caracterização do perfil (meta)fonológico do grupo no início do ano letivo, a fim de se identificarem as tarefas e as unidades fonológicas alvo de intervenção. Esse levantamento consiste num diagnóstico e a sua execução deve obedecer aos procedimentos descritos na secção ‘Avaliação da consciência fonológica’. Traçado o perfil (meta)fonológico do grupo, deve passar-se à identificação das competências que não estão adquiridas, das que estão em desenvolvimento e das que estão adquiridas, por forma a desenhar-se o programa de intervenção a aplicar.\footnote{Esta\label{not:cap10not4} categorização tem por base a escala de desenvolvimento de \citet{LazzarottoVolcao2014}, adaptada de \citet{Yavas1991}, e prevê que: de 0\% a 50\% de correspondência ao alvo, a estrutura é não adquirida; de 51\% a 75\% de correspondência ao alvo, a estrutura está em aquisição; de 76\% a 100\% de correspondência ao alvo, a estrutura está adquirida.}

Supondo que a avaliação do perfil psicolinguístico e (meta)fonológico de um grupo em idade pré-escolar, constituído por 20 crianças a frequentar a sala dos 5 anos (10 meninos e 10 meninas), quatro delas bilingues (duas português-hindu e duas português-mandarim), revela dificuldades generalizadas em termos de consciência fonológica (com competências não adquiridas e outras em aquisição), exceto ao nível das segmentação de frases em palavras e das tarefas em torno de palavras constituídas por formatos silábicos não marcados (V e CV), deve proce\-der-se ao desenho de um programa de intervenção ajustado ao perfil sociolinguístico, psicolinguístico e (meta)fonológico do grupo, a implementar pelo educador de infância (intervenção direta), capacitado e supervisionado pelo profissional especializado em linguagem (intervenção indireta). \citet{Ebbels2017} recomendam que estes profissionais reúnam na fase inicial do processo, a fim de tratarem e interpretarem os resultados do diagnóstico em conjunto e, subsequentemente, desenharem o plano de intervenção para o grupo, e mais tarde, periodicamente (semanalmente, de preferência), à medida que o programa for sendo implementado, a fim de realizarem o ponto de situação semanal de cada sessão decorrida e prepararem a(s) sessão(ões) da semana seguinte. A grande maioria dos programas de nível 1 e 2 ocupa cerca de 3 meses do período letivo, com uma frequência geralmente semanal ou bissemanal, perfazendo um total que varia entre 12 e 24 sessões (4 a 8 sessões por mês, durante 3 meses). 

As atividades de estimulação da consciência fonológica devem ser adaptadas às orientações curriculares, ser independentes ou integradas noutras áreas curriculares, ser lúdicas, ter dinâmicas em grupo e/ou individuais, implementadas em sala ou na rua, e incluir várias tarefas de consciência fonológica (desde a perceção à manipulação) e as diferentes unidades prosódicas e segmentais visadas. A estimulação das unidades fonológicas, através das tarefas selecionadas, deve, no entanto, ser distinta do formato apresentado nos instrumentos de avaliação.

Segundo \citet{Wolfe2004}, os aspetos cognitivos envolvidos na consolidação de competências cognitivas, presentes no desenvolvimento e/ou na aprendizagem, culminam numa conceptualização. Para alcançar esse nível de representação abstrata, é necessário passar previamente por uma representação simbólica e esta, por sua vez, passar primeiro por uma experienciação sensorial, a um nível concreto. A perceção da fala é um fenómeno bimodal, no sentido em que integra e unifica duas modalidades de informação, a auditiva e a visual. Os \emph{inputs} visuais e auditivos influenciam-se mutuamente e complementam-se. Neste sentido, a condição que permite maior inteligibilidade da fala é a audiovisual \citep{Skipper2007}. “Os olhos contêm quase 70\% dos recetores sensoriais do corpo e enviam milhões de sinais a cada segundo ao longo dos nervos ópticos para os centros de processamento visual do cérebro \citep[145]{Wolfe2004}”. Para se estabilizar no córtex visual, a função da memória sensorial é conduzir a informação que entra no cérebro pelos recetores sensoriais e mantê-la durante uma fracção de segundos, até que se decida sobre a pertinência de a armazenar, ou não \citep[145]{Wolfe2004}. O mesmo se verifica relativamente ao armazenamento da informação auditiva no córtex auditivo. A audição e a visão são assim as principais vias de apreensão de informações fonoarticulatórias. Para \citet{VidorSouza2009}, a consciência articulatória -- capacidade de refletir sobre as características articulatórias gerais, não necessariamente associadas aos sons da fala -- constitui um pré-requisito da consciência fonoarticulatória. Esta, por sua vez, é definida como a capacidade de perceber que os sons mudam de acordo com o modo e/ou o lugar como e onde são produzidos pelos articuladores \citep{Santos2009}. A consciência fonoarticulatória é considerada uma habilidade prévia à consciência fonológica, competência fonológica explícita \citep{VidorSouza2009}.

Ao estimular a consciência fonológica, pretende-se estabilizar esse conhecimento em representações fonológicas. Forçosamente, esse trabalho passa por uma concretização das unidades a trabalhar\footnote{As representações e codificações das unidades e propriedades fonológicas apresentadas nesta secção são meramente exemplificativas. As que aqui se disponibilizam são usadas em diferentes propostas de intervenção -- algumas publicadas \citep{Freitas2007, Carvalho2012, Carvalho2017}, outras não -- e provêm do instrumento de intervenção Os Sons d’A Relicário \citep{Alves2011}, cujo desenvolvimento é explicitado em \citet{Alves2014a}. Encontra-se atualmente em processo de edição uma nova versão deste instrumento, cuja designação se altera para MIF - Metodologia de Intervenção Fonológica não linear \citep{Alvesemprep.}. Esta versão propõe uma gama revista e ampliada de símbolos, ícones e cores, que garante a cobertura de todas unidades e propriedades fonológicas relevantes do português europeu.}, nomeadamente, a palavra, representável através de retângulos, a sílaba, representável através de círculos -- que devem caber nos retângulos --, as diferentes estruturas silábicas, representáveis através de triângulos devidamente organizados -- que devem caber nos círculos -- e os fonemas, representáveis através de onomatopeias -- que devem caber nos triângulos. Veja-se abaixo, na Figura \ref{fig:cap10fig1}, a representação simbólica da oração ‘a chuva molha’.


\begin{Figura}
    \centering
    \includegraphics[width=\textwidth]{figures/Imagem1.jpg}


    \caption{{ Representação simbólica da oração ‘a chuva molha’}}
    \label{fig:cap10fig1}
\end{Figura}


Este tipo de representação proporciona a experienciação sensorial, nomeadamente visual, da cadeia sonora ‘a chuva molha’, como também a sua representação simbólica em termos de sua constituição prosódica e segmental, facilitando assim a execução das diferentes tarefas de consciência fonológica habitualmente exercitadas, como sejam, a perceção, a produção/evocação, a identificação, a segmentação, a supressão, a síntese e a manipulação. 

Na proposta de intervenção que abaixo se ilustra, estão contempladas as unidades palavra (\emph{retângulo}), sílaba (\emph{círculo}) e acento (\emph{círculo preenchido}), na dimensão prosódica; as unidades segmento (\emph{triângulo}) e traço distintivo (\emph{símbolos e códigos de cor}), na dimensão segmental (ver Figura \ref{fig:cap10fig2} e Figura \ref{fig:cap10fig3}).


\begin{Figura}
    \centering
    \includegraphics[width=4cm, height=2cm]{figures/Imagem2.png}


    \caption{{Representação das unidades ‘palavra’, ‘sílaba’, ‘acento e ‘segmento’ na palavra ‘alho’}}
    \label{fig:cap10fig2}
\end{Figura}


Os valores do traço [voz] são representados pelas cores azul (não vozeado) e vermelho (vozeado), e o traço [+nas] pela cor verde (nasal) (ver Figura \ref{fig:cap10fig3}); os triângulos pretos remetem para o segmento de uma forma neutra.


\begin{Figura}
    \centering
    \includegraphics[width=7.5cm, height=2cm]{figures/Imagem3.png}


    \caption{{Representação de segmentos vozeados (triângulo vermelho), não vozeados (triângulo azul) e nasais (triângulo verde)}}
    \label{fig:cap10fig3}
\end{Figura}


Ainda neste proposta, os segmentos do português europeu são representados por onomatopeias - uma por cada fonema (ver Figura \ref{fig:cap10fig4}). 


\begin{Figura}
    \includegraphics[width=4cm, height=3.5cm]{figures/Imagem4.jpg}
    \caption{Representação onomatopeica do fonema /ʃ/\\
    \centering Imagem cedida pela Relicário de Sons©}
    \label{fig:cap10fig4}
\end{Figura}


As quatro classes de modo de articulação dos fonemas consonânticos do português europeu são representadas por símbolos -- fricativas, líquidas, nasais e oclusivas. Esses símbolos podem ser vermelhos -- fonemas vozeados das classes de modo que esses símbolos representam --, azuis -- fonemas não vozeados das classes de modo que esses símbolos representam -- ou verdes e vermelhos – fonemas nasais. 

Os símbolos em tons de preto e cinza representam indiferenciadamente os fonemas das quatro classes de modo e os vermelhos e azuis, o conjunto dos fonemas vozeados e não vozeados da classe das obstruintes (ver Tabela \ref{tab:cap10tab3}).


\begin{Tabela}
\caption{{Codificação dos modos de articulação dos fonemas consonânticos do português europeu\\
\emph{Imagens cedidas pela Relicário de Sons©}}
}
\label{tab:cap10tab3}

\fittable{
\begin{tabular}{ >{\columncolor{lightgray}}p{2,6cm} c c c c l}
\hline
\textbf{Traços da Raiz e do Nó}
& ~
&  \multicolumn{3}{c} {\cellcolor{gray}\textbf{Traços da Raiz e do Nó Laríngeo}}\\

\textbf{Cavidade Oral}
& \hphantom{sal} neutro \hphantom{sal}
& \cellcolor{gray}[+vozeamento]
& \cellcolor{gray}[-vozeamento]
& \cellcolor{gray}[±vozeamento]\\
\emph{Classes naturais}
&
& \cellcolor{gray}([±soante])
& \cellcolor{gray} ~
& \cellcolor{gray} ~\\
\*[-soante; {+contínuo]} \hphantom{sal}\hphantom{sal}\hphantom{sal}
{\textbf{\emph{Fricativas}}}
& \multirow{7}{*}{\RaiseImage\includegraphics[width=1.5cm, height=7cm]{figures/A.jpg}}& \multirow{7}{*}{\RaiseImage\includegraphics[width=1.5cm, height=7cm]{figures/B.jpg}}&\multirow{7}{*}{\RaiseImage\includegraphics[width=1.5cm, height=7cm]{figures/C.jpg}} & \multirow{7}{*}{\RaiseImage\includegraphics[width=1.5cm, height=7cm]{figures/D.jpg}}
\\
&&&&\\
\*[+soante; {+contínuo]}
\hphantom{sal}{\textbf{\emph{Líquidas}}}
&&&&
&\\
&&&&\\
\*[+soante; {-contínuo]}
\hphantom{sal}{\textbf{\emph{Nasais}}}
&&&&\\
&&&&\\
\*[-soante;  { -contínuo]}
\hphantom{sal}{\textbf{\emph{Oclusivas}}}
&&&&\\
\hline
\end{tabular}
}
\end{Tabela}


Os pontos de articulação dos fonemas consonânticos do português europeu também são representadas por símbolos – Labial, Coronal anterior e não anterior, e Dorsal. Esses símbolos podem ser vermelhos - fonemas vozeados das classes de ponto que esses símbolos representam -, azuis – fonemas não vozeados das classes de ponto que esses símbolos representam - ou verdes – fonemas nasais das classes de ponto que representam. Os símbolos em tons de preto e cinza representam indiferenciadamente os fonemas das quatro classes de ponto (ver Tabela \ref{tab:cap10tab4}).




\begin{Tabela}
\caption{{Codificação dos pontos de articulação dos fonemas consonânticos do português europeu}\\
\small{\emph{Imagens cedidas pela Relicário de Sons©}}}
\label{tab:cap10tab4}

\fittable{
\begin{tabular}{ >{\columncolor{lightgray}}p{2,6cm} c c c c l l  }
\hline
\textbf{Traços do Nó Ponto de}
&
&  \multicolumn{3}{c} {\cellcolor{gray}\textbf{Traços da Raiz e do Nó Laríngeo}}\\

\textbf{Articulação}
& \hphantom{sal} neutro \hphantom{sal}
& \cellcolor{gray}[+nasalidade]
& \cellcolor{gray} [+vozeamento]
& \cellcolor{gray} [-vozeamento]\\
\emph{Classes naturais}
&
& \cellcolor{gray} ~
& \cellcolor{gray} ~
& \cellcolor{gray} \\
{Labial} 
&{\multirow{4}{*}{\RaiseImage\includegraphics[width=1.5cm, height=6.1cm]{figures/E.png}}}&{\multirow{4}{*}{\RaiseImage\includegraphics[width=2cm, height=6.1cm]{figures/F.png}}}&{\multirow{4}{*}{\RaiseImage\includegraphics[width=1.9cm, height=6.1cm]{figures/G.png}}}&{\multirow{4}{*}{\RaiseImage\includegraphics[width=1.5cm, height=6.1cm]{figures/H.png}}}\\
{\textbf{\emph{labiais}}}&&&&\\
Coronal {[+anterior]} 
& &&&\\
{\textbf{\emph{coronais anteriores}}}&&&&\\
Coronal ~~{[-anterior]} 
& &&&\\
{\textbf{\emph{coronais não anteriores}}}&&&&\\
Dorsal 
& &&&\\
{\textbf{\emph{dorsais}}} &&&&\\
\hline
\end{tabular}
}
\end{Tabela}

As propostas de codificação apresentadas nas figuras \ref{fig:cap10fig6} e \ref{fig:cap10fig7} promovem a conceptualização das representações fonológicas bem como o resultado de diferentes combinações de traços, facilitando por um lado o processo de intervenção e espelhando, por outro, o que acontece naturalmente ao longo do desenvolvimento (meta)fonológico. 
Retomando o exemplo do grupo a intervencionar, descrito anteriormente, será possível começar a intervenção com tarefas e unidades já dominadas pelo grupo – a segmentação de palavras em frases e tarefas em torno de palavras constituídas por formatos V e CV – para depois serem introduzidos constituintes ainda ausentes nas representações metafonológicas desse grupo (formatos marcados e de emergência mais tardia, nomeadamente formatos CCV, CVC e CCVC), complexificando assim as estruturas alvo de intervenção, tal como representadas na Figura \ref{fig:cap10fig5}. 



\begin{Figura}
    \centering
    \includegraphics[width=\linewidth]{figures/Imagem17.jpg}



    \caption{{Codificação das palavras <cara>, <prato>, <parque> e <flor>, respetivamente}\\
    \emph{\small{Imagens cedidas pela Relicário de Sons©}}}
    \label{fig:cap10fig5}
\end{Figura}


Na Figura \ref{fig:cap10fig5}, é possível observar a instanciação de /ɾ/ - representado pela onomatopeia de um ‘grilo’ - em três constituintes silábicos distintos, o Ataque simples (<\emph{cara}>), o Ataque ramificado (<\emph{prato}>) e a Coda medial (<\emph{parque}>) e final (<\emph{flor}>).

Ainda a nível da intervenção prosódica, sugere-se a estimulação da consciência do acento, identificando, por exemplo, a sílaba tónica de palavras oxítonas, paroxítonas e proparoxítonas, como proposto nas atividades do projeto a Força das Sílabas\footnote{O recurso ‘Laboratório Gramatical Digital: A Força das Sílaba’, inserido no projeto major ‘RED.PT Laboratório Gramatical Digital’, visa promover o desenvolvimento de competências metafonológicas dos alunos em torno do acento fonológico em português europeu (\citealp{Cardoso2020}, disponível em \url{https://craftsman.pt/redpt/ega/}).} - ver exemplo da Figura \ref{fig:cap10fig6}, em que o círculo preenchido (à esquerda) representa a sílaba tónica da palavra \emph{óculos}.



\begin{Figura}
    \centering
    \includegraphics[width=\linewidth]{figures/Imagem18.png}



    \caption{{ Identificação da sílaba tónica}\\ \emph{\small{Fonte original: \url{https://craftsman.pt/redpt/ega}}}}
    \label{fig:cap10fig6}
\end{Figura}


Nesta tarefa, espera-se que o aluno oiça a palavra-alvo e selecione a sílaba (representada por círculos) que corresponde à sílaba tónica (sílaba “forte”), deixan\-do-a preenchida. 

A referida aplicação dispõe ainda de mais duas atividades, adequadas à estimulação da consciência do acento: a Experiência 1 e a Experiência 2 (ver Figura \ref{fig:cap10fig7}).



\begin{Figura}
    \centering
    \includegraphics[width=\linewidth]{figures/Imagem19.png}



    \caption{{Atividade da Experiência 1}\\ \emph{\small{Fonte original: \url{https://craftsman.pt/redpt/ega/3}}}}
    \label{fig:cap10fig7}
\end{Figura}


Na Experiência 1, por exemplo, espera-se que o aluno preste atenção à última palavra do excerto da canção ‘Pica do 7’, do cantor António Azambujo, e que repita o verso, com autogravação, para conseguir comparar a sua produção com a do áudio original. Nesta atividade, os alunos são incentivados a usar o truque de ‘chamar a palavra de forma mais lenta na sílaba tónica’, a fim de identificar mais facilmente a sílaba tónica. Na experiência 2, os alunos são convidados a identificar a sílaba tónica através de outro truque, nomeadamente o de “chamar com mais força a sílaba tónica”.

\subsection{Intervenção em consciência fonológica no contexto clínico}
O nível de intervenção 3 definido por \citet{Ebbels2017} é individualizado e diri\-ge-se a crianças com alterações diagnosticadas. Neste nível, a intervenção do profissional especializado em linguagem é indireta (nível 3A) e direta (nível 3B), em termos de planeamento e implementação do programa. Esta atuação carece do estabelecimento de um diagnóstico clínico e do desenho de um instrumento de auxílio à tomada de decisão terapêutica. O referido instrumento deve permitir traçar o perfil (meta)fonológico do utente, identificar as unidades e as tarefas mais e menos estáveis nesse perfil (perfil psicolinguístico), identificar a teoria de intervenção que melhor se aplica (neste caso, a não linear), e monitorizar a evolução do para, assim, ir ajustando a intervenção. 

A título exemplificativo, se uma criança diagnosticada com disortografia apresentar dificuldades na escrita de palavras como <*susa> (em vez de <chucha>), <*manã> em vez de <manhã>, <*riza> (em vez de <rija>, pode concluir-se que as suas dificuldades se prendem com a não especificação dos valores do traço Coronal [±anterior], passível de ser estabilizado através de tarefas (meta)fonológicas do conhecimento implícito e explícito, tal como o ilustra a Figura \ref{fig:cap10fig8}.



\begin{Figura}
    \includegraphics[width=\linewidth]{figures/Imagem20.png}
    \caption{Representação parcial (ramificação) do instrumento de apoio à tomada de decisão de intervenção fonológica não linear\\
    {\small Fonte original: Phonovia – \citet{Alvesemprep.}, adaptado de \citet{Julio2019}}}
    \parbox{\textwidth}{\scriptsize \citet{Julio2019}. {+Consciência Fonológica [contexto clínico] – PhonoVia: um instrumento de apoio à utilização do +Consciência Fonológica.} Dissertação de projeto desenvolvido no âmbito da unidade curricular de Educação para a Prática V. Escola Superior de Saúde do Instituto Politécnico de Setúbal, Portugal.}
    \label{fig:cap10fig8}
\end{Figura}

\noindent{Se as dificuldades observadas incidirem em tarefas do conhecimento implícito, mais elementares, sugere-se começar pelo bombardeamento auditivo do contraste Coronal [±anterior] até que este atinja um sucesso superior ou igual a 50\%.\footnote{\emph{Cf}. nota \ref{not:cap10not4}} Se a percentagem de sucesso se situar entre os 50\% e os 79\%, as atividades de bombardeamento deverão ser reforçadas e as de discriminação iniciadas (com estímulos mais elementares). Se o sucesso na tarefa de bombardeamento igualar ou ultrapassar os 80\%, deve passar-se para a tarefa seguinte, nomeadamente a tarefa de discriminação. Ainda ao nível do conhecimento implícito, o mesmo circuito deve ser seguido até chegar à tarefa de produção/evocação oral. Depois disso, prossegue-se para o nível do conhecimento explícito, constituído por tarefas de localização, identificação, síntese, manipulação, segmentação e produção escrita. A passagem de umas tarefas para as outras deve obedecer ao mesmo circuito que o explicitado anteriormente e considerar as mesmas percentagens.}

As formas geométricas, os símbolos e os códigos de cor apresentados na secção anterior também podem integrar os planos de intervenção de casos clínicos, desde que ajustados ao diagnóstico e perfil (meta)fonológico identificado. O mesmo se aplica à ordem das tarefas que constituem os conhecimentos implícitos e explícitos na medida em que estas podem e devem ser alteradas em função do perfil do (meta)fonológico sujeito a intervenção.

As intervenções junto da população com patologia requerem a aplicação de um conjunto de atividades desenhadas especialmente para essa população, a fim de promover o desenvolvimento da linguagem oral e/ou escrita, e demover as barreiras que impedem a sua participação ativa na sociedade. 

\section{Conclusão}
A definição de consciência fonológica tem sido relativamente consensual ao longo dos anos, contrariamente às abordagens de avaliação e intervenção nesta área. Nestes níveis de atuação, ainda se verificam práticas incidentes na observação do desempenho (meta)fonológico, independentemente do que o elicita. Os estudos no domínio da Psicolinguística, da Fonologia Clínica e da Fonologia Aplicada – em particular da Fonologia não linear - têm-se revelado muito úteis na identificação das variáveis fonológicas e psicolinguísticas responsáveis pela variação dos desempenhos de consciência fonológica observados em diferentes etapas do desenvolvimento linguístico, pelo que estas devem ser controladas quer no processo de avaliação, quer no processo de intervenção. Do ponto de vista fonológico, isto deve traduzir-se numa seleção cuidada das palavras, em termos da sua tipologia gramatical e prosódica - extensão, formato silábico e acento (palavras oxítonas, paroxítonas e proparoxítonas) - e em termos fonémicos (segmento, traços, combinações de traços e contrastes). Do ponto de vista psicolinguístico, traduz-se numa aplicação diversificada de tarefas de consciência fonológica visto que as mesmas interferem diferentemente no desempenho (meta)cognitivo, (meta)linguístico e neste caso, em particular, (meta)fonológico. 

A adoção de tais procedimentos proporciona uma caracterização mais fidedigna do perfil (meta)fonológico dos sujeitos sob avaliação - ou de grupos de sujeitos -, oferecendo melhores condições ao desenho do plano da intervenção e, por conseguinte, aos resultados decorrentes da sua implementação (eficácia), quer no contexto educacional como clínico.

Os recursos de intervenção apresentados neste capítulo visam facilitar a explicitação dos gestos articulatórios presentes na fala. Esta etapa constitui o ponto de partida de acesso às informações fonológicas, dada a mediação que a consciência fonoarticulatória estabelece entre os aspetos físicos e cognitivos da fala, servindo de alavanca ao desenvolvimento da consciência fonológica \citep{VidorSouza2009}. Assim, o processo começa na consciência fonoarticulatória – suportada em dados periféricos (anatomofisiológicos e neuromotores) – e prossegue até à consciência fonológica - suportada em dados centrais (cognivo-linguísticos, em particular, fonológicos). Em contexto de intervenção, (i) o tipo e a natureza da informação a recrutar (unidades fonológicas, propriedades fonológicas e tarefas psicolinguísticas), (ii) a seleção da dimensão em que essa informação deve ser realizada/recuperada (a dimensão fonética ou fonológica), (iii) o recurso a estímulos e estratégias e (iv) a frequência de exposição a esses estímulos, devem ser ajustados às necessidades e capacidades da criança, como também aos fatores ambientais que a circundam (como o método de ensino, a(s) língua(s) falada(s), entre outros fatores). Independentemente da dosagem e do tempo de intervenção prescritos, os ingredientes deverão diminuir gradualmente, até que a criança se torne autónoma na gestão das necessidades da sua consciência fonológica.


{\sloppy\printbibliography[heading=subbibliography,notkeyword=this]}
\end{document}
