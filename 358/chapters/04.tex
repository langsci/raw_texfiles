\documentclass[output=paper,colorlinks,citecolor=brown,booklanguage=portuguese]{langscibook} 
\ChapterDOI{10.5281/zenodo.7233219}
\title{O segmento: Teoria fonológica e avaliação}
\author{Cristiane Lazzarotto-Volcão\affiliation{Universidade Federal de Santa Catarina (UFSC)} and Aline Mara de Oliveira\affiliation{Universidade Federal de Santa Catarina (UFSC)} and Carmen Lúcia Barreto Matzenauer\affiliation{Universidade Federal de Pelotas}}



\abstract{Considerando a relevância de a avaliação clínica da fala estar alicerçada em princípios advindos de teorias fonológicas, este capítulo apresenta três abordagens teóricas de modo a subsidiar essa avaliação. A primeira delas é a Fonologia Natural, na qual está subjacente a visão de que a diferença entre a gramática fonológica atípica e a gramática fonológica alvo é motivada por processos fonológicos simplificadores que são inatos, sendo que a identificação desses processos aplicados pela criança é capaz de revelar a atipicidade de sua fonologia. A segunda é a Fonologia Autossegmental, segundo a qual um inventário fonológico atípico decorre de problemas na aquisição de traços que compõem a estrutura interna dos segmentos e que respondem pelos contrastes fonológicos da língua alvo da aquisição, atribuindo a essas unidades da fonologia e ao seu funcionamento coocorrente a motivação para a gramática fonológica atípica. O terceiro modelo aqui abordado é a Fonologia Gestual, de acordo com a qual a atipicidade de uma gramática fonológica decorre de problemas gestuais na produção e na representação da fala, ou seja, ao considerar que o ‘gesto articulatório’ tem natureza motora, simbólica e dinâmica, interpretam-se gestualmente os fatos de uma fonologia atípica, analisando-os por meio de dados articulatórios.


\textbf{Palavras-chave}: Teoria Fonológica; Avaliação da Fala, Fonologia Autossegmental, Fonologia Gestual.}

\IfFileExists{../localcommands.tex}{
   \addbibresource{../localbibliography.bib}
   \usepackage{langsci-optional}
\usepackage{langsci-gb4e}
\usepackage{langsci-lgr}

\usepackage{listings}
\lstset{basicstyle=\ttfamily,tabsize=2,breaklines=true}

%added by author
% \usepackage{tipa}
\usepackage{multirow}
\graphicspath{{figures/}}
\usepackage{langsci-branding}

   
\newcommand{\sent}{\enumsentence}
\newcommand{\sents}{\eenumsentence}
\let\citeasnoun\citet

\renewcommand{\lsCoverTitleFont}[1]{\sffamily\addfontfeatures{Scale=MatchUppercase}\fontsize{44pt}{16mm}\selectfont #1}
  
   %% hyphenation points for line breaks
%% Normally, automatic hyphenation in LaTeX is very good
%% If a word is mis-hyphenated, add it to this file
%%
%% add information to TeX file before \begin{document} with:
%% %% hyphenation points for line breaks
%% Normally, automatic hyphenation in LaTeX is very good
%% If a word is mis-hyphenated, add it to this file
%%
%% add information to TeX file before \begin{document} with:
%% %% hyphenation points for line breaks
%% Normally, automatic hyphenation in LaTeX is very good
%% If a word is mis-hyphenated, add it to this file
%%
%% add information to TeX file before \begin{document} with:
%% \include{localhyphenation}
\hyphenation{
affri-ca-te
affri-ca-tes
an-no-tated
com-ple-ments
com-po-si-tio-na-li-ty
non-com-po-si-tio-na-li-ty
Gon-zá-lez
out-side
Ri-chárd
se-man-tics
STREU-SLE
Tie-de-mann
}
\hyphenation{
affri-ca-te
affri-ca-tes
an-no-tated
com-ple-ments
com-po-si-tio-na-li-ty
non-com-po-si-tio-na-li-ty
Gon-zá-lez
out-side
Ri-chárd
se-man-tics
STREU-SLE
Tie-de-mann
}
\hyphenation{
affri-ca-te
affri-ca-tes
an-no-tated
com-ple-ments
com-po-si-tio-na-li-ty
non-com-po-si-tio-na-li-ty
Gon-zá-lez
out-side
Ri-chárd
se-man-tics
STREU-SLE
Tie-de-mann
}
   \boolfalse{bookcompile}
   \togglepaper[04]%%chapternumber
}{}

\begin{document}
\maketitle
\lehead{Lazzarotto-Volcão, de Oliveira \& Barreto Matzenauer}

\section{Introdução}
\subsection{Palavras iniciais}
O foco deste capítulo atende ao objetivo de apresentar os pressupostos que integram uma avaliação clínica da fala cujas bases estejam em teorias fonológicas.\footnote{A norma adotada na escrita deste capítulo foi a do português brasileiro.}
% \footnote{Na sequência de restrições decorrentes do template usado para publicação na editora Language Science Press, os capítulos que integram o volume surgem com a referência [this volume], em inglês.}
Defende-se não apenas a pertinência de a avaliação clínica apresentar esse alicerce teórico, mas também se argumenta no sentido de considerá-la recomendável para a formulação segura de diagnóstico de trans\-tornos fonológicos e para a proposição de uma intervenção terapêutica adequada e eficaz. A discussão está centrada na aquisição do “segmento” como unidade da fonologia das línguas.

Tem-se o ponto de partida em uma questão que se faz preliminar: qual é a relevância de se buscarem teorias fonológicas ao abordar-se a avaliação clínica da fala? A resposta a essa questão pode ser encaminhada no sentido de explicitar-se que a “fala” não consiste na simples articulação de sons quaisquer, aleatoriamente combinados. A “fala”, no uso que aqui lhe é conferido, consiste na manifestação de um sistema de unidades linguísticas, ou seja, de uma organização em que unidades linguísticas menores se combinam para formar unidades maiores, segundo regras e restrições determinadas, a fim de poder veicular significado. Esse sistema de relações entre as unidades, essa organização perfeitamente estruturada é a “língua”, é a gramática da língua. Explica \citet[9]{MouraNeves2010} que a gramática não está na língua: a gramática é a língua; é o que constrói significado; “é a responsável pelo entrelaçamento discursivo-textual das relações que se estabelecem na sociocomunicação”.

Em sendo assim, na aquisição da linguagem, o alvo do processo de desenvolvimento, para as crianças, é a gramática, ou seja, é o conhecimento das unidades linguísticas e das regras de funcionamento e estruturação da sua língua; é o conhecimento do mecanismo que relaciona as unidades linguísticas, das menores às maiores, e que permite, ao falante, o uso normal da língua; é essa abstração ou representação na mente/cérebro composta pelas unidades linguísticas e pelas relações que podem entre elas ser estabelecidas. A gramática, ao organizar todos os componentes da língua, institui-se como um sistema computacional na mente/cérebro dos falantes. É exatamente a gramática o que permite que uma língua seja adquirida.

A gramática fonológica, ao lado dos outros subsistemas que compõem uma língua (como o subsistema morfológico, o sintático, o semântico), é um dos alvos da aquisição da linguagem pela criança: precisam ser dominados os segmentos fonológicos (vogais e consoantes) que integram essa gramática, como também as regras e restrições que determinam as relações entre eles no funcionamento na língua. 

No processo de desenvolvimento fonológico considerado típico, há a construção gradual da gramática fonológica: os segmentos e as regras/restrições de combinações entre as unidades vão sendo gradativamente incorporados ao sistema linguístico da criança. O desenvolvimento fonológico é delineado por uma sucessão de estágios, com a gradual anexação de segmentos e de possibilidades de distribuição e de combinação entre eles na estruturação de unidades maiores, como sílabas, pés métricos e palavras. Em cada estágio do desenvolvimento, tem-se o estabelecimento de uma gramática, constituída por segmentos e por regras de relações entre essas unidades. Os segmentos e as regras que se mostram em funcionamento em cada estágio da aquisição fonológica evidenciam que, naquele momento, é um sistema linguístico que está em operação, embora ainda seja diferente do sistema alvo. É reconhecível que a criança sempre usa uma língua, sempre tem uma gramática em quaisquer estágios do processo de aquisição, mesmo naqueles mais precoces.

Quando é diagnosticado como atípico o processo de desenvolvimento fonológico, também está em operação um sistema linguístico, uma gramática fonológica: mesmo com omissões, substituições e distorções de alguns sons, a fala com transtornos apresenta padrões, mostra uma organização das unidades linguísticas, embora seja diferente da organização do sistema alvo da aquisição. É reconhecível que é a “língua” que a criança com transtornos de fala utiliza e, portanto, há uma gramática, mesmo que o transtorno seja muito severo e o grau de inteligibilidade seja pequeno.

É relevante explicitar que se considera atípica a aquisição fonológica quando a criança, em comparação com os seus pares e com a comunidade em que está inserida, apresenta diferente organização linguística, seja no inventário de segmentos (consoantes e vogais), nas sequências de segmentos (sílabas) e/ou no emprego de processos fonológicos.

Assume-se, portanto, que a criança, durante o processo de desenvolvimento linguístico, seja típico ou atípico, sempre detém uma gramática, pois toda sua produção linguística é reconhecida como “língua”, já que representa uma organização das unidades linguísticas. Então, como decorrência, pode-se afirmar ser plenamente relevante buscarem-se teorias fonológicas ao tratar-se a avaliação clínica da fala, oferecendo-se uma resposta à questão proposta no início do capítulo. Evocar essa base para a avaliação clínica, ou seja, para a avaliação da gramática em operação em crianças com potenciais transtornos de fala, é pertinente porque toda teoria fonológica tem de cumprir a tarefa exatamente de explicitar as gramáticas fonológicas, ou seja, de elucidar o conhecimento fonológico que o falante detém, a natureza das representações fonológicas, chegando a oferecer, como afirmam \citet[245]{Clements1995}, explicações para generalizações em diferentes domínios, incluindo, entre outros, a aquisição da linguagem, os transtornos linguísticos, a mudança histórica. 

As teorias fonológicas, portanto, legitimamente têm de dispor do aparato necessário à descrição, à explicação e à formalização das unidades e dos fenômenos que compõem toda gramática fonológica, incluindo-se as gramáticas que integram o desenvolvimento da fonologia de uma língua em processo de aquisição, seja ele típico ou atípico.  

\subsection{A busca da base da avaliação da fala em teorias fonológicas}



Diante dos diferentes modelos teóricos propostos para desvelar e explicar o funcionamento das fonologias das línguas, o campo de estudos da aquisição do componente fonológico dos sistemas linguísticos elegeu alguns, considerando o alto poder explicativo relativamente a determinadas unidades da fonologia.

Com o foco na unidade “segmento”, numerosas investigações sobre a aquisição da fonologia, centradas preponderantemente na produção linguística, tanto típica, quanto atípica, fundamentaram suas análises precipuamente nos quadros teóricos da Fonologia Natural \citep{Stampe1973} e da Fonologia Autossegmental \citep{Clements1985, Clements1991, Clements1995} , embora também tenha havido estudos embasados na Fonologia Gerativa \citep{Chomsky1968}, e, mais recentemente, nos Princípios Fonológicos Baseados em Traços \citep{Clements2009}, na Teoria da Otimidade \citep{Prince1993, Boersma2011, Boersma2009}, e na Fonologia Gestual \citep{Browman1989, Browman1992, Albano2009}. Nesses cenários teóricos, encontram-se pesquisas sobre o processo de aquisição da fonologia de diferentes línguas, como, por exemplo, o Inglês, o Holandês, o Francês e o Português, tanto o Português Brasileiro (PB), como o Português Europeu (PE).

Ao tratar-se particularmente da avaliação clínica da fala, alcançaram maior frequência e relevância os estudos situados no contexto teórico da Fonologia Natural e da Fonologia Autossegmental, razão por que receberam atenção especial neste capítulo. Também aqui ganha destaque a avaliação clínica que encontra suporte teórico na Fonologia Gestual, considerando sua atualidade nas investigações da área de aquisição fonológica atípica.

Destaca-se também a pertinência de se discutirem essas três abordagens teóricas na avaliação clínica, em virtude de elegerem diferentes unidades linguísticas de análise fonológica, o que implica se voltarem com diferentes perspectivas para a aquisição dos segmentos no processo de desenvolvimento fonológico: enquanto para a Fonologia Natural a unidade de análise é o segmento, para a Fonologia Autossegmental a unidade de análise é o traço fonológico e, para a Fonologia Gestual, a unidade de análise é o gesto.

A Fonologia Natural tem o foco no segmento como unidade que integra processos fonológicos, interpretados estes como operações mentais inatas, portanto, universais, que, com força simplificadora, precisam ser suprimidas no curso do desenvolvimento fonológico em direção ao sistema linguístico alvo da aquisição. Por operar com segmentos, a Fonologia Natural os categoriza também como constituintes das sílabas, reconhecendo a interface segmento/prosódia ao entender que, na aquisição, o segmento pode receber tratamento diferenciado em razão da posição silábica que ocupa. A avaliação clínica incorpora, sob esse enfoque, os segmentos como unidade fonológica e como constituinte de uma unidade prosódica: a sílaba.

Diferentemente, a Fonologia Autossegmental e a Fonologia Gestual veem o segmento como o resultado de unidades menores: os traços para a primeira, os gestos para a segunda. Essas unidades mínimas respondem pela ordem de emergência dos segmentos e também pelos processos, sejam de omissão ou de substituição, de que são alvo nos vários estágios que integram o desenvolvimento fonológico típico, bem como daqueles processos observados em casos de fonologia diagnosticada como atípica. Nessas abordagens teóricas, a avaliação clínica examina acuradamente as propriedades que compõem a estrutura interna dos segmentos, buscando com detalhe as motivações do ordenamento de sua emergência na construção gradual do inventário fonológico da língua alvo, assim como de seu comportamento como parte do sistema nos diferentes estágios desenvolvimentais. Os movimentos fonológicos que mostram os segmentos no processo de aquisição da fonologia são aqui explicados, portanto, com base nas unidades menores que compõem a sua estrutura interna.

\section{Identificação dos inventários fonético e fonológico}



Toda avaliação clínica dos sons da fala, independentemente da perspectiva teórica adotada pelo avaliador, precisa começar pelo levantamento dos inventários fonéti\-co e fonológico da criança. Sendo assim, após a obtenção dos dados de fala da criança avaliada e posterior análise contrastiva – que consiste em uma comparação entre a fala da criança e aquela esperada de acordo com a sua comunidade de fala – procede-se à determinação desses inventários.

\subsection{Inventário fonético}



O inventário fonético é o primeiro passo da avaliação, pois é a partir dos dados de produção efetivamente realizados pela criança que todas as outras análises se tornarão possíveis. Nesta etapa da avaliação, é possível recorrer, para além da tradicional análise auditiva, a análises mais objetivas, como a análise acústica, através do Software Praat, por exemplo, muito útil em contexto clínico e acadêmico; ou análises que dependem de instrumentos mais específicos, como a ultrassonografia de língua. Tais instrumentos, além de auxiliarem o avaliador na tarefa de transcrição da fala da criança (conforme \citetv{chapters/06}), podem re\-velar contrastes encobertos que não são recuperáveis pelo ouvido. Na seção \ref{sec:cap4sec5} deste capítulo, abordar-se-á esse aspecto.

No inventário fonético devem ser anotados todos os sons de fala produzidos pela criança, inclusive sons que não pertençam ao conjunto de sons da sua língua materna/dialeto. Mesmo que a criança tenha produzido apenas uma vez determinado som, ou só o tenha feito mediante repetição imediata, este também deverá ser incluído no inventário fonético. Esse cuidado é importante, pois o inventário fonético tem como principal objetivo revelar a capacidade articulatória da criança.

\subsection{Inventário fonológico}



O inventário fonológico da criança é construído a partir da análise das produções infantis, levando-se em conta o percentual de erros e acertos referentemente ao modelo de fala do adulto. Há o entendimento de que, dependendo do índice de emprego de cada segmento em consonância com o alvo da língua, é possível interpretar-se se determinado segmento já integra o inventário da criança com efetivo valor contrastivo, ou se ainda não assumiu o \emph{status} de fonema. Nesta seção, será apresentada uma breve descrição de como determinar o inventário fonológico, segundo o instrumento Avaliação Fonológica da Criança (AFC), proposto por \citet{Yavas1991}. A AFC foi a primeira proposta de avaliação clínica da fala com base em teorias fonológicas no Brasil, amplamente conhecida por fonoaudiólogas e terapeutas da fala, tanto no Brasil, quanto em Portugal.

É importante destacar que o instrumento incorpora a necessidade de avaliação dos segmentos da língua, levando-se em consideração a estrutura silábica, já que a aquisição do segmento pode ser condicionada pela posição que ocupa na sílaba, conforme vários estudos já demonstraram para o Holandês \citep{Fikkert1994}, para o PB \citep{Ribas2002, Mezzomo2004} e para o PE \citep{Freitas1997}. É através da análise da variabilidade de produção da criança e do cálculo do percentual de ocorrência de cada variante, em cada posição silábica, que se obtêm os dados para a definição do inventário fonológico. 

A partir dos resultados obtidos, pode construir-se o inventário fonológico da criança. Nesse inventário aparecerão os segmentos adquiridos, em aquisição e não adquiridos. Na literatura da área encontram-se diferentes propostas para determinar o \emph{status} de aquisição dos fonemas, em função de diferentes metodologias de recolha e de análise dos dados. Na AFC, os autores recomendam adotar os seguintes critérios:


    \begin{enumerate}
        \item [a)] acerto inferior a 50\%: a criança não possui o fonema;
        \item[b)]	acerto de 51\% a 75\%: a criança possui o segmento em concorrência com o que o substitui;
        \item[c)] acerto de 76\% a 85\%: a criança já adquiriu o fonema, mas deve-se registrar o som ainda empregado em seu lugar; 
        \item[d)] acerto de 86\% a 100\%: o segmento-alvo foi efetivamente adquirido pela criança.
    \end{enumerate}

\citet{LazzarottoVolcao2014} propôs a utilização de três faixas percentuais, de modo a simplificar a análise, especialmente em contexto clínico. Assim, (a) produções corretas abaixo de 50\% indicam a não aquisição do segmento; (b) produções corretas entre 51\% e 85\% indicam uma aquisição em curso; (c) produções corretas a partir de 86\% indicam que o segmento já está adquirido. 

Nas próximas seções, apresentam-se os três modelos fonológicos aqui eleitos que com muita pertinência vêm sendo utilizados na prática clínica de terapeutas.

\section{Fonologia Natural e avaliação clínica da fala }



Para a Fonologia Natural, proposta por \citet{Stampe1973} e \citet{Stampe1979}, a aquisição fonológica é explicitada a partir da noção de processo fonológico, definido como uma operação mental que se aplica à fala para substituir, em lugar de uma classe de sons ou sequência de sons que apresentam uma dificuldade específica comum para a capacidade de fala do indivíduo, uma classe alternativa idêntica em todos os outros sentidos, porém desprovida da propriedade difícil \citep[1]{Stampe1973}. Assim, de acordo com essa pressuposição, a criança tem em sua mente algumas operações inatas, as quais são aplicadas sempre que se depara com algum alvo que ainda não é capaz de produzir. 

Há diversos estudos importantes da área que se baseiam nessa análise, tanto no Brasil \citep{Lamprecht1990}, quanto em Portugal \citep{Lousada2012}, apenas para citar alguns. Esse modelo, embora já contestado por outras propostas teóricas mais atuais, teve uma grande relevância, pois foi capaz de demonstrar generalizações importantes para a descrição da aquisição fonológica típica e atípica em diferentes línguas, sendo ainda amplamente utilizado como embasamento teórico para a avaliação e terapia de fala, tanto no Brasil, quanto em Portugal.





Do ponto de vista descritivo, as análises por processos são muito adequadas ao estudo da fonologia de crianças com suspeita de transtorno fonológico, porém, como se verá adiante, há certas fragilidades em uma análise exclusiva por meio de processos. 

De acordo com o instrumento AFC \citep{Yavas1991}, após estabelecer-se o inventário fonológico de uma criança, por meio da análise contrastiva entre sua fala e a da sua comunidade linguística, é possível descrever quais os processos operantes na gramática da criança. Há duas grandes categorias de processos: aqueles relacionados às estruturas silábicas, chamados de “processos de reestruturação silábica”, e os que se relacionam aos segmentos, chamados de “processos de apagamento e de substituição”. 

Abaixo, citam-se alguns desses processos, a título ilustrativo, e as produções fonéticas decorrentes da aplicação do processo. Utilizam-se como alvo, também como ilustração, possíveis \emph{outputs} fonéticos de variedades brasileiras do português.

\subsection{Processos de estruturação silábica}


Apagamento de coda medial: a consoante nessa posição silábica não é realizada foneticamente. 
\ea {[ˈsestɐ]} → [ˈsetɐ] (cesta)
\z
\ea {[ˈpɔɾtɐ]} → [ˈpɔtɐ] (porta)
\z
Apagamento de coda final: a consoante nessa posição não é realizada foneticamente.
\ea {[a'sukɐɾ]} → [a'sukɐ] (açúcar)
\z
Redução de encontro consonantal (grupo consonântico): essa estrutura silábica é produzida como ataque simples, sendo que a C2 costuma não ser realizada foneticamente.
\ea {[bɾĩ'kedʊ}] → [bĩ'kedʊ] (brinquedo)
\z

\subsection{Processos de apagamento}

Apagamento de líquida em Ataque Inicial (lateral e/ou não lateral)
\ea {[ˈxatʊ]} → [ˈatʊ] (rato)
\z
\ea {[ˈlapɪs]} → [ˈapɪs] (lapis)
\z
Apagamento de líquida em Ataque Medial (lateral e/ou não lateral)
\ea {[ˈmiʎʊ]} → [ˈmiʊ] (milho)
\z
\ea {[pi‘ɾatɐ]} → [piˈatɐ] (pirata)
\z
Apagamento de nasal
\ea {[ˈniɲʊ]} → [ˈniʊ] (ninho)
\z



\subsection{Processos de substituição}

Posteriorização de fricativas: quando o segmento-alvo é uma fricativa anterior e em seu lugar é empregada uma fricativa posterior.
\ea {[ˈmasɐ]} → [ˈmaʃɐ] (massa)
\z
Substituição de líquida: quando o segmento-alvo é uma líquida e em seu lugar é empregada outra líquida ou outro fone.
\ea  {[vaˈsoɾɐ]} → [vaˈsolɐ] (vassoura)
\z
\ea {[muˈʎɛɾ]} → [muˈjɛɾ] (mulher)
\z
Plosivização: quando o segmento-alvo é, em geral, uma fricativa que acaba sendo realizada como plosiva.
\ea {[ˈsapʊ]} → [ˈtapʊ] (sapo)
\z



Há outra categoria de processo que envolve a substituição de um segmento por outro, mas motivado por semelhança fonética entre segmentos presentes na palavra. A esses processos dá-se o nome de “processos de assimilação”, conforme os exemplos a seguir:
\ea {[baˈnɐ̴na}]  → [maˈnɐ̴na] (banana)
\z
\ea {[ˈʃavɪ]} → [ˈfavɪ] (chave)
\z
\ea {[amaˈɾεlʊ]} → [amaˈlεlʊ] (amarelu)
\z
A análise por meio de processos também pode ser feita de forma quantitativa, através da análise do percentual de ocorrência de cada processo. Tome-se como exemplo o caso de uma criança que, durante a sessão de avaliação de sua fala, fez dez tentativas de produção de palavras com o segmento /ʎ/. Em cinco delas, a criança empregou o fone [l] em seu lugar e nas outras cinco, o fone [j]. Com isso, tem-se 50\% de aplicação do “processo de substituição de líquida” e 50\% de aplicação do “processo de semivocalização”. A análise por meio dos percentuais de ocorrência pode ser útil em algumas abordagens terapêuticas, conforme será abordado em \textcitetv{chapters/05}.

Outro aspecto importante de qualquer avaliação da fala é a determinação da gravidade do transtorno, tanto em termos diagnósticos, quanto em termos de escolha do modelo terapêutico a ser adotado, já que a literatura aponta que há uma relação favorável entre determinados modelos e os diferentes níveis de gravidade do transtorno \citep{Bagetti2005, Pagliarin2009}. A literatura não aponta nenhuma metodologia específica para se chegar ao nível de gravidade do transtorno de uma criança, através de uma análise por meio de processos. Contudo, é possível tomar como parâmetro análises qualitativas, tais quais as apresentadas por \citet{Grunwell1997}. A autora caracterizou os DF em três categorias de desenvolvimento, a saber:



\begin{itemize}
    \item [a)] desenvolvimento atrasado -- quando a criança apresenta padrões de pronúncia normais, mas a aquisição se dá de forma mais lenta do que o esperado, ou seja, há processos fonológicos comuns na aquisição típica em idades mais precoces;
    \item[b)] desenvolvimento variável -- a criança utiliza padrões de estágios diferentes, ou seja, apresenta processos típicos de fases iniciais da aquisição e processos de fases mais finais; 
    \item[c)] desenvolvimento diferente --  a criança usa padrões incomuns  no desenvolvimento  fonológico típico, ou seja, processos que são pouco frequentes na aquisição típica.
\end{itemize}

A abordagem por meio de processos, conforme aponta a literatura, apresenta uma fragilidade que é o fato de que essa perspectiva teórica apresenta generalizações muito amplas, pois pode englobar como um mesmo processo fenômenos que decorrem do comportamento de diferentes traços distintivos. Isso ocorre, por exemplo, ao identificar-se como “processo de posteriorização” tanto a realização [ˈʃinʊ] para [ˈsinʊ] (\emph{sino}), como [ka'pekɪ] para [ta'petʃɪ] (\emph{tapete}).

Além disso, o modelo prevê divisões excessivas, em que dois processos (ou mais) podem referir-se ao mesmo fenômeno \citep{Matzenauer2004}. A autora dá o exemplo de saídas fonéticas para [ˈpɾatʊ] produzidas pela mesma criança, na mesma sessão, como [ˈpatʊ], [ˈplatʊ] e [ˈpaɾatʊ]. Neste caso, registrar-se-iam três processos: “redução de encontro consonantal”, “substituição de líquida” e “epêntese”. Porém, os três processos referem-se ao fato de a estrutura silábica do ataque ramificado com /ɾ/ não estar disponível na gramática da criança, conforme \textcitetv{chapters/06}, neste volume.

Outro aspecto problemático de uma análise exclusiva por meio de processos diz respeito ao acompanhamento da evolução terapêutica, isto é, à quantificação e à qualificação das mudanças no sistema fonológico da criança. Por meio de processos, é possível verificar se houve redução ou aumento no percentual de uso de determinado processo, ou se novos processos passaram a surgir no sistema da criança. Nesse último caso, isso poderia ser interpretado como problemático, já que, de acordo com a proposta teórica subjacente, a criança precisa eliminar os processos de modo a adquirir o sistema-alvo. Contudo, o surgimento de novos processos no decurso terapêutico pode indicar a presença de novas estratégias de reparo que evidenciam a aquisição de estruturas (traços, gestos, regras) antes ausentes em sua gramática.

Por fim, apesar do seu poder descritivo e do seu uso generalizado, uma análise por meio de processos tem como pressuposto que os processos são operações mentais inatas. Dessa forma, conforme referido anteriormente, o processo de aquisição é entendido como um processo de perda, uma vez que a criança precisa eliminar os processos mentais inatos que não estão presentes na fala do adulto. Contudo, essa teorização dificilmente encontraria respaldo empírico, se se considerarem estudos do funcionamento cerebral relativamente a novas aprendizagens, por exemplo. Nesse caso, não há perda, mas ganho, tanto em termos de ativação de sinapses, como em termos de reforço das já existentes.

\section{Fonologia Autossegmental e avaliação clínica da fala}



A Fonologia Autossegmental, proposta por \citet{Goldsmith1976}, surge como uma proposta alternativa para análise das fonologias das línguas. Pela abordagem de análise dos fonemas em suas unidades menores, nomeadamente os traços distintivos, vigente à época era o modelo linear de \citet{Chomsky1968}, os fonemas eram formados por matrizes de traços, sem um ordenamento ou hierarquia entre eles. Contudo, esse modelo apresentava dificuldades para explicar e formalizar certos fenômenos fonológicos das línguas, como casos em que determinado segmento desaparecia, mas certas características suas permaneciam na palavra, como o tom ou a nasalidade, por exemplo. 

Com uma visão diferenciada, a Fonologia Autossegmental traz como pressuposto que os traços fonológicos funcionam como autossegmentos e são relativamente autônomos em relação ao segmento. Também postula que não há uma relação de um-para-um entre segmento e o conjunto de traços que o compõem. Daí decorrem duas implicações: os traços podem estender-se para aquém ou para além do segmento e o apagamento de um segmento não implica necessariamente o apagamento de todos os traços da sua estrutura interna. Além disso, outra importante contribuição desse modelo fonológico é a noção de que essa estrutura interna é formada por traços organizados hierarquicamente, permitindo o reconhecimento de que há traços que funcionam solidariamente com outros.

\citet{Clements1985} e \citet{Clements1995}, a partir de tais pressupostos, propuseram a Geometria de Traços, segundo a qual o segmento é formado por uma estrutura interna que possui nós hierarquicamente organizados, em que os nós terminais são os traços e os nós intermediários são classes de traços \citep{MatzenauerHernandorena2001}. 

Para a representação das consoantes, a Geometria de Traços prevê a estrutura arbórea representada pela Figura \ref{fig:cap4fig1}.



\begin{Figura}
%     \includegraphics[width=8cm, height=6cm]{figures/Imagem37.png}
    \begin{forest}
      [Nó de raiz, name=raiz
        [Laríngeo
            [{[voz]}]
        ]
        [{[nasal]}]
        [{Nó cavidade oral}
            [Ponto de C
                [{[labial]}]
                [{[coronal]}
                    [{[$\pm$anterior]}]
                ]
                [{[dorsal]}]
            ]
        [{[contínuo]}]
      ]
      ]
    \node[right=-1mm of raiz]{\scriptsize\featurebox{$\pm$soante\\$\pm$aproximante\\$-$vocoide}};
    \end{forest}
    \caption{Geometria de Traços para as consoantes do PB.\\
    Fonte: as autoras, a partir de \citet{Clements1985}}
    \label{fig:cap4fig1}
\end{Figura}


Com base nessa estrutura, os autores postulam que a organização dos traços é universalmente determinada e que cada regra fonológica realiza apenas uma única operação na geometria. Esse último princípio é capaz de representar a naturalidade de fenômenos fonológicos. Para mais detalhes sobre a Fonologia Autossegmental e a Geometria de Traços, sugere-se a obra de \citet{Bisol2014}.

Nessa concepção, a aquisição fonológica passou a ser vista como uma montagem gradual da estrutura que caracteriza esses segmentos, através da ligação sucessiva dos diferentes tiers, sem considerar que a criança possua na subjacência, desde o início, um sistema fonológico semelhante ao do adulto \citep{MatzenauerHernandorena1996, MatzenauerHernandorena2001}.

Partindo de tais pressupostos, surgem propostas de análise dos dados de fala de crianças que consideram as relações implicacionais entre esses traços, a hierarquia interna do segmento e os contrastes fonológicos produzidos pelas diferentes configurações de traços. 

Um desses modelos é o MICT -- Modelo Implicacional de Complexidade de Traços de \citet{Mota1996}, proposto para dar conta do processo de aquisição fonológica em crianças falantes do PB e que está mais bem detalhado em \textcitetv{chapters/05}. Com base nesse modelo, diz-se que o percurso da aquisição fonológica é conduzido a partir das rotas que a criança precisa trilhar, em direção ao sistema-al\-vo do adulto. Para utilizar tal modelo na avaliação clínica da fala, a partir do inventário fonológico da criança, é possível identificar quais rotas já foram trilhadas e qual o grau de complexidade de seu sistema. Quanto mais distante do nível 0 indicado pelo modelo, mais complexo é o sistema da criança. Dessa forma, a gravidade do transtorno também pode ser analisada de forma qualitativa: quanto menos rotas trilhadas, mais grave será o transtorno.

Outro aspecto positivo dessa abordagem é a possibilidade de identificar dificuldades específicas com determinadas coocorrências de traços que fazem com que a criança atinja níveis de complexidade distintos, em se comparando uma rota com outra.

O MICT também é um importante auxiliar para o processo terapêutico, conforme pode ser visto em \textcitetv{chapters/05}, pois permite identificar quais os alvos a serem tratados nas sessões terapêuticas, a partir das relações implicacionais que há entre eles. 

Apesar desses benefícios, o MICT apresenta a desvantagem de não mostrar, a partir do seu desenho, pequenas evoluções que a criança possa fazer durante a intervenção terapêutica, pois não capta mudanças de estratégias de reparo, por exemplo, na trilha das rotas. Se há aquisição do segmento, a rota é marcada como trilhada, mas se há mudança apenas em parte de sua estrutura interna (evidenciada pela presença de nova estratégia de reparo), não há como indicá-la no modelo.

Outro modelo baseado na Fonologia Autossegmental é o Modelo Padrão de Aquisição de Contrastes (PAC), proposto por \citet{LazzarottoVolcao2009}, que tem por base a Geometria de Traços e os Princípios Fonológicos Baseados em Traços, propostos por \citeauthor{Clements2009}, em \citeyear{Clements2009}, em que são feitas considerações cruciais acerca do papel dos traços e de princípios fonológicos baseados em traços no estudo e descrição dos inventários das línguas. O argumento é que princípios universais com base em traços estão subjacentes à constituição de inventários fonológicos.

Os cinco princípios fonológicos baseados em traços propostos por \citeauthor{Clements2009}, de forma resumida, são:

\begin{itemize}
    \item [a)] \emph{Feature Bounding} (Limitação de Traços) – este princípio refere-se ao poder que os traços possuem de aumentar o número de categorias potencialmente contrastivas em um sistema.
    \item[b)] \emph{Feature Economy} (Economia de Traços) – este princípio estabelece que os traços tendem a ser combinados maximamente.
    \item[c)] \emph{Marked Feature Avoidance} (Evitação de Traços Marcados) – este princípio afirma que certos valores de traços tendem a ser evitados pelas línguas.
    \item[d)] \emph{Robustness} (Robustez) – este princípio diz respeito à diferente força que os traços têm de estabelecer contrastes, o que implica que há contrastes mais robustos ou menos robustos, dependendo dos traços que os definem.
    \item[e)] \emph{Phonological Enhancement} (Reforço Fonológico) – por fim, este princípio refere-se ao fato de valores marcados de traços poderem ser introduzidos em um sistema para reforçar contrastes perceptuais fracos.
\end{itemize}

O modelo PAC, construído a partir desses princípios, teve como objetivo observar se a aquisição fonológica típica do PB também é guiada pelos princípios fonológicos descritos, o que foi confirmado em \citet{LazzarottoVolcao2009}, com pequenas reinterpretações para dar conta de especificidades da língua em questão. Da mesma forma, a autora teve como objetivo verificar se a aquisição atípica também seria governada pelos mesmos princípios. A análise do comportamento dos sistemas de crianças com transtornos em relação a três deles – os Princípios da Economia de Traços, da Evitação de Traços Marcados e da Robustez – mostrou-se relevante para uma melhor compreensão do funcionamento de sistemas fonológicos desviantes.

De acordo com esse modelo, a base de compreensão do \emph{output} da avaliação são os contrastes fonológicos presentes, instáveis e ausentes no sistema da criança, razão por que opera com traços como unidades mínimas de análise. Para que se possa fazer essa avaliação, é necessária uma análise das coocorrências de traços presentes, capazes de criar as diferentes classes naturais previstas na língua-alvo.

Segundo o modelo PAC, os contrastes presentes na gramática do PB são adquiridos em quatro etapas distintas e de complexidade crescente, as quais seguem o padrão de aquisição de crianças típicas (ver \citealt{LazzarottoVolcao2009}).

Tomando por base esse padrão, a fonologia da criança pode ser avaliada em termos de ausência/presença de contrastes e mapeada segundo o esquema proposto pelo modelo, que está demonstrado na Figuras~\ref{fig:cap4fig2-1}--\ref{fig:cap4fig2-2}.



\begin{figure}
%     \includegraphics[width=\linewidth]{figures/Imagem38.png}
%     \caption{Modelo padrão de aquisição de contraste. Fonte: as autoras}
    \begin{forest} for tree={forked edge,rectangle,draw, grow'=east}
        [Soantes,fill=red!50
            [Nasais,fill=yellow
                [Cor,fill=red!50
                    [Ant /n/,fill=red!50]
                    [Não-anterior,fill=red!50]
                ]
                [Labial /m/,fill=red!50]
            ]
            [Líq,fill=yellow
                    [Líq lat,fill=green!50
                    [Cor
                        [Ant /l/,fill=green!50]
                        [Não-ant /ʎ/,fill=green!50]
                    ]
                    ]
                    [Líq não-lat,fill=green!50
                        [Dorsal /ʀ/,fill=green!50]
                        [Coronal /ɾ/,fill=green!50]
                    ]
            ]
        ]
    \end{forest}
    \caption{Modelo padrão de aquisição de contraste para consoantes soantes.
             \protect\tikz \protect\node [anchor=base, draw=black!75, fill=red!50, minimum width=1.5em] (red) {}; 1ª etapa da aquisição – até 2:00
             \protect\tikz \protect\node [anchor=base, draw=black!75, fill=yellow, minimum width=1.5em] (yellow) {}; 3ª etapa da aquisição – 2:8 a 3:00
             \protect\tikz \protect\node [anchor=base, draw=black!75, fill=green!50, minimum width=1.5em] (green) {}; 4ª etapa da aquisição – 3:4 a 4:2
             \protect\tikz \protect\node [anchor=base, draw=black!75, fill=white, minimum width=1.5em] (white) {}; Contrastes ausentes
             \label{fig:cap4fig2-1}}
\end{figure}

\begin{figure}
    \begin{forest} for tree={forked edge,rectangle,draw,grow'=east}
        [Obstr,fill=red!50
            [Plos,fill=blue!50
                    [Cor,fill=red!50
                        [Surda /t/,fill=red!50]
                        [Sonora /d/,fill=red!50]
                    ]
                    [Labiais,fill=red!50
                        [Surda /p/,fill=red!50]
                        [Sonora /b/,fill=red!50]
                    ]
                    [Dorsais,fill=red!50
                        [Surda /k/,fill=red!50]
                        [Sonora /g/,fill=red!50]
                    ]
            ]
            [Fricat,fill=blue!50
                [Labiais,fill=blue!50
                    [Surda /f/,fill=blue!50]
                    [Sonora /v/,fill=blue!50]
                ]
                [Cor,fill=blue!50
                    [Ant,fill=yellow
                        [Surda /s/,fill=blue!50]
                        [Sonora /z/,fill=blue!50]
                    ]
                    [Não-ant,fill=yellow
                        [Surda /ʃ/,fill=yellow]
                        [Sonora /ʒ/,fill=yellow]
                    ]
                ]
            ]
        ]
    \end{forest}
    \caption{Modelo padrão de aquisição de contraste para consoantes obstruintes.
             \protect\tikz \protect\node [anchor=base, draw=black!75, fill=red!50, minimum width=1.5em] (red) {}; 1ª etapa da aquisição – até 2:00
             \protect\tikz \protect\node [anchor=base, draw=black!75, fill=blue!50, minimum width=1.5em] (blue) {}; 2ª etapa da aquisição – 1:8 a 2:6
             \protect\tikz \protect\node [anchor=base, draw=black!75, fill=yellow, minimum width=1.5em] (yellow) {}; 3ª etapa da aquisição – 2:8 a 3:00
             \label{fig:cap4fig2-2}}
\end{figure}


A proposta de identificação dos contrastes, relacionados com as diferentes etapas do percurso da aquisição, tem como objetivo permitir ao avaliador identificar se a criança apresenta atraso ou transtorno nesse percurso. O modelo propõe a identificação do atraso quando a criança segue as etapas previstas pelo modelo, mas em idades diferentes do esperado. Já o transtorno ou a perturbação fonológica são identificados quando a criança apresenta um padrão de aquisição diferente do proposto, em que contrastes iniciais podem estar ausentes, enquanto outros, mais tardios, se encontram presentes.

Para a identificação de quais contrastes estão presentes, a autora propõe uma análise do percentual de acerto do contraste. Identifica-se o número total de produção (correta e incorreta) dos segmentos pertencentes às duas classes naturais envolvidas no estabelecimento do contraste (por exemplo, soantes e não soantes), na posição de Ataque Simples, uma vez que o modelo não avalia a aquisição das diferentes estruturas silábicas, já que assume ser a posição de Ataque Simples aquela em que todos os segmentos do PB emergem inicialmente. 

Contabiliza-se o número de produções corretas e o número de produções incorretas desses segmentos e, a seguir, analisam-se os padrões dos erros cometidos, para verificar se envolvem uma troca de uma classe para outra, dentre as envolvidas na produção do contraste.

A identificação dos contrastes ausentes ou instáveis serve como um guia para a escolha dos segmentos-alvo a serem trabalhados nas sessões terapêuticas, utilizando também o princípio da generalização e das relações estabelecidas entre os contrastes em termos de robustez: ao adquirir um contraste mais robusto, o modelo prevê que os menos robustos surgirão por generalização. \citet{Reis2018} utilizou o modelo nessa perspectiva e encontrou evidências positivas nesse sentido.

Outro aspecto importante do PAC é a possibilidade de representação do sistema fonológico da criança por meio de um desenho, o que facilita para que pessoas leigas, por exemplo, consigam entender o sistema da criança e possam ver as evoluções que ela faz ao longo do percurso, mesmo que ainda não tenham atingido os alvos necessários da gramática do adulto. O PAC consegue capturar, através da análise dos contrastes, as mudanças que a criança vai fazendo na estrutura interna dos segmentos, mesmo quando ainda não os adquiriu.

Apesar das vantagens apontadas, ainda são necessários mais estudos com base nessa análise, especialmente aqueles realizados com crianças em processo terapêutico, de modo a verificar a pertinência do modelo para a terapia fonoaudiológica.


\section{Fonologia Gestual e avaliação clínica da fala}\label{sec:cap4sec5}

No interior da perspectiva teórica da Fonologia Gestual - FG (Fonologia Articulatória), proposta por \citet{Browman1989, Browman1990, Browman1992}, o primitivo de análise fonológica passa ser o “gesto articulatório” (diferindo tanto do traço distintivo, quanto do segmento). Dessa forma, o “gesto articulatório”, de natureza motora, simbólica e dinâmica, é que permite, concomitantemente, exercer o contraste fonológico e a caracterização dos movimentos articulatórios. Pautado na Dinâmica de Tarefas (TD) (ver em \citealp{Browman1992}), o gesto articulatório é capaz de realizar diferentes ações, envolvendo a coordenação de vários articuladores que, por sua vez, buscam realizar uma tarefa.

A partir do modelo da FG, a tarefa de cada gesto é definida por variáveis do trato (protrusão labial; abertura labial; ponta da língua; corpo da língua; abertura vélica; abertura glotal) correlacionadas tanto com a localização da constrição do trato vocal (labial, dentária, alveolar, pós alveolar, palatal, velar, uvular e faríngea), quanto com o grau de constrição (fechada, crítica, estreita, média e larga) \citep{Browman1992, Albano2001}, representado por pautas gestuais. 

Nas pautas gestuais, os gestos seriam como um retângulo, sendo que a sua dimensão horizontal traria informações sobre o tempo. Isso significa dizer que o gesto detém, ao mesmo tempo, um aspecto espacial (que corresponde a um alvo que deve ser atingido, envolvendo uma variável do trato), dependente de um aspecto temporal intrínseco. Os gestos podem se sobrepor, tanto em termos espaciais (magnitude), quanto temporais (temporal), acarretando a sobreposição de gestos articulatórios e/ou a ativação parciais de gestos envolvidos durante a fala \citep{Pouplier2005}.

Outro fator importante considerado neste modelo é a coordenação intergestual envolvida em cada padrão silábico. A versão atual da FG propõe um padrão de coordenação intergestual específico para cada padrão silábico, modelado a partir do modelo de osciladores acoplados \citep{Goldstein2007}. Na coordenação intergestual de uma sílaba CV, as constrições que configuram o trato vocal na produção da consoante e na produção da vogal iniciam-se de modo sincrônico ou em fase. Isso significa dizer que, durante a produção da sílaba [´pa], os gestos envolvidos na produção da consoante e da vogal iniciam-se concomitantemente, ou seja, o fechamento labial para o [p] e o estreitamento faríngeo para a produção de [a] iniciam-se ao mesmo tempo, estabelecendo uma relação de fase entre esses dois gestos envolvidos. Já na sílaba do tipo VC, a ativação do gesto da consoante é acionada depois da ativação do gesto de vogal, estabelecendo um acoplamento antifásico entre os gestos. Por fim, em uma sílaba do tipo CCV, a coordenação intergestual é composta por uma relação em fase e antifase.  Com o objetivo de garantir o contraste no interior da sílaba, enquanto a relação de faseamento estabelecida entre os gestos das consoantes é de antifase, a relação entre cada consoante e a vogal é em fase \citep{Browman2000, Marin2010}. 

Durante a aquisição fonológica, algumas crianças apresentam mais dificuldades na coordenação inter e intragestuais imbricadas nos padrões silábicos (CV, VC e CCV). Os “erros” classificados por outras teorias como omissão, inserção, substituição ou assimilação, serão analisados pela FG como decorrentes de relações errôneas na coordenação dos gestos envolvidos, com variações quanto à magnitude dos gestos e/ou à sobreposição entre os gestos \citep{Pouplier2005}.

Dessa maneira, a implementação de uma metodologia instrumental de análise (análise acústica e/ou articulatória) permite o detalhamento fonético na análise da produção de fala \citep{Albano2001}. As análises articulatórias têm ganhado destaque na análise da produção da fala, especialmente as análises ultrassonográficas do movimento da língua.  Por meio das imagens de ultrassom, é possível investigar os “erros” de fala, uma vez que a imagem do contorno da língua propicia informações para a visualização direta dos articuladores envolvidos na produção de fala. A ultrassonografia de língua é uma técnica não invasiva, segura, rápida e de baixo custo \citep{Wiethan2015, Cleland2015}. A Figura \ref{fig:cap4fig3} ilustra um exemplo da imagem da ultrassonografia de língua.





\begin{Figura}
    \includegraphics[width=.8\textwidth]{figures/Imagem39.png}
    \caption{Imagem do ultrassom de língua de uma criança durante a produção de fala. Da esquerda para a direita, as setas indicam a ponta de língua, dorso e raiz, respectivamente.\\
    Fonte: as autoras}
    \label{fig:cap4fig3}
\end{Figura}



Frente ao posto, a análise articulatória permite caracterizar a produção de sons que envolvem a língua em termos gestuais e investigar a magnitude do gesto ao longo tempo. Aliada à análise ultrassonográfica, é possível a identificação dos gestos articulatórios de crianças com Transtornos dos Sons da Fala (TSF).

\citet{Shriberg2019a} propuseram o Sistema de Classificação dos Transtornos dos Sons da Fala (SCTSF), sendo este um termo genérico relacionado a qualquer combinação de dificuldades com a percepção, a produção motora e/ou a representação fonológica dos segmentos (incluindo regras fonotáticas que governam a forma e a estrutura segmental, bem como aspectos prosódicos) que afetam a inteligibilidade da fala \citep{ASHA2007, ASHA2017, Shriberg2019a}. Dentre os SCTSFs, têm-se as crianças classificadas com atraso de fala (AF), com apraxia de fala na infância (AFI) e com atraso motor de fala (AMF).

O atraso de fala pode ser decorrente de fatores genéticos, de otite média de repetição e/ou de alterações no desenvolvimento psicossocial. As crianças apresentam omissões e/ou substituições inadequadas dos sons da fala, com possíveis classificações dos tipos de erros de sons da fala (processo fonológicos alterados) \citep{Shriberg2019a}.

A apraxia de fala na infância (AFI) é caracterizada por transtorno de origem neurológica, que acomete a consistência e a precisão dos movimentos da fala na ausência de défices neuromusculares \citep{ASHA2007, ASHA2017}. E crianças com atraso motor de fala (AMF) apresentam dificuldades na execução neuromotora em decorrência de um atraso na maturação do sistema motor da fala, o que causa dificuldades na precisão articulatória, na estabilidade da fala, na voz e na prosódia \citep{Shriberg2019a}.

Crianças com AF, AFI e AMF comumente são confundidas, durante o diagnóstico e o tratamento fonoaudiológico, já que apresentam “erros” fonológicos que contribuem para diagnósticos enganosos \citep{Nijland2003, Aziz2010}. 

A partir de dados articulatórios, a análise dos “erros” na produção de fala de crianças com TSF (e também com desenvolvimento típico de linguagem) pode identificar a complexa relação entre a dinâmica dos gestos articulatórios e o sistema motor de fala imaturo com as limitações das habilidades motoras da fala e das restrições físicas, fisiológicas e funcionais. Tanto nas produções fônicas de crianças com desenvolvimento típico de linguagem, quanto nas crianças com SCTSF (e até mesmo na fala cotidiana de adultos), os “erros” no interior do modelo da FG serão analisados como estratégias compensatórias em que o falante busca estabilizar a coordenação gestual, ou seja, a presença de “erros” pode ser interpretada gestualmente (e analisada por meio de dados articulatórios) como gestos intrusos, diminuição da taxa de elocução, aumento da 
amplitude dos gestos, omissão de segmentos/gestos/sílabas e do atraso entre os articuladores \citep{Namasivayam2020}. 

Apesar de a FG aliada à análise articulatória ter contribuído com novas interpretações acerca dos “erros” identificados na produção de fala atípica, o modelo não prevê o \emph{feedback} auditivo e os mecanismos neurais subjacentes aos componentes das produções de fala \citep{Namasivayam2020}, o que pode ser considerado uma limitação do modelo. 

Além disso, apesar do número crescente de estudos utilizando a metodologias articulatórias no interior do arcabouço da Fonologia Gestual, a qual se aprimorou na identificação de contrastes encobertos ou gestos indiferenciados, por exemplo, faz-se necessário ampliar os estudos com diferentes grupos clínicos e compará-los entre si, a fim de promover diagnóstico diferencial por meio da análise detalhada dos gestos articulatórios.

\section{Conclusão}



Independentemente da visão teórica que ofereça os subsídios para a análise de casos de fonologia atípica, importa entender-se que a avaliação clínica tem relevância crucial, pois é o diagnóstico correto que conduzirá à escolha do tratamento adequado. E a avaliação consiste no desvelamento da gramática fonológica da criança, ou seja, na essencial identificação: (a) do inventário fonético, (b) do sistema de segmentos que têm \emph{status} fonológico e (c) do funcionamento desses segmentos como constituintes silábicos.

Explicando-se com maior detalhe, tem-se, quanto ao primeiro ponto, que a avaliação do inventário fonético revela a capacidade articulatória da criança; quanto ao segundo, vê-se que a avaliação de segmentos fonológicos mostra a capacidade fonológica relativa ao emprego das unidades menores (traços ou gestos) que os compõem de modo a implicar contrastes de significado e, ainda, quanto ao terceiro fato, verifica-se que a avaliação dos segmentos como constituintes de uma sílaba (ataque, núcleo ou coda) explicita a capacidade fonológica de combinar segmentos, formando estruturas sintagmáticas presentes no sistema alvo. E o pressuposto que aqui se apresenta é que a identificação de uma gramática fonológica e o entendimento de como ela funciona exigem o suporte de uma teoria fonológica, com o discernimento das unidades que compõem a fonologia de uma língua, como traços, gestos, segmentos e sílabas.

É a avaliação clínica que determinará o lócus do transtorno, ou seja, que apontará a(s) unidade(s) cuja realização ou emprego não correspondem ao alvo da língua e que, portanto, revelará a natureza do transtorno de fala. É, pois, a avaliação clínica que dará a conhecer o funcionamento de um sistema fonológico que está diferente do esperado e que precisa não apenas chegar ao sistema alvo, mas também com ele identificar-se no menor período de tempo possível, em um procedimento terapêutico adequado e efetivo.

Os três caminhos teóricos aqui destacados conduzem a interpretações diferentes quanto ao entendimento da natureza dos transtornos fonológicos e também apontam para encaminhamentos terapêuticos que podem ser diversos: (a) a uma avaliação clínica com o suporte da Fonologia Natural está subjacente a visão de que a diferença entre a gramática fonológica atípica e a gramática fonológica alvo é motivada por processos fonológicos simplificadores, que são inatos, e que a identificação desses processos aplicados pela criança é capaz de revelar a atipicidade de sua fonologia; (b) a uma avaliação clínica com base na Fonologia Autossegmental subjaz a compreensão de que um inventário fonológico atípico decorre de problemas na aquisição de traços que compõem a estrutura interna dos segmentos e que respondem pelos contrastes fonológicos, atribuindo a essas unidades da fonologia e ao seu funcionamento coocorrente a motivação para a  gramática fonológica atípica; (c) a uma avaliação clínica com fundamento na Fonologia Gestual encontra-se, como substrato, o entendimento de que a atipicidade de uma gramática fonológica decorre de problemas gestuais na produção e na representação da fala, ou seja, ao considerar que o  ‘gesto articulatório’ tem natureza motora, simbólica e dinâmica, interpretam-se gestualmente os fatos de uma fonologia atípica, analisando-os por meio de dados articulatórios.

São visões diferentes, mas todas capazes de oferecer explicações reveladoras a uma avaliação clínica. Cabe ao terapeuta a escolha do substrato teórico que lhe proporcionar o aparato mais elucidativo para o estabelecimento de diagnósticos apropriados e para o esclarecimento da natureza de cada gramática fonológica atípica e da gravidade do transtorno que ela possa apresentar. O importante é apropriar-se de um modelo teórico que evidencie atipicidades fonológicas no processo de aquisição de um sistema linguístico e que subsidie a determinação de caminhos para tratamentos eficazes.


{\sloppy\printbibliography[heading=subbibliography,notkeyword=this]}
\clearpage
\end{document}
