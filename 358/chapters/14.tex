\documentclass[output=paper,colorlinks,citecolor=brown,booklanguage=portuguese]{langscibook} 
\ChapterDOI{10.5281/zenodo.7233239}
\title{Um procedimento de intervenção no processamento de estruturas de alto custo}
\author{Letícia Maria Sicuro Corrêa  \affiliation{Pontifícia Universidade Católica do Rio de Janeiro (PUC-Rio); LAPAL (Laboratório de Psicolinguística e Aquisição da Linguagem - PUC-Rio)} and  Marina R. A. Augusto \affiliation{LAPAL (Laboratório de Psicolinguística e Aquisição da Linguagem - PUC-Rio); Universidade do Estado do Rio de Janeiro (UERJ)} and  Tatiana Bagetti \affiliation{LAPAL (Laboratório de Psicolinguística e Aquisição da Linguagem - PUC-Rio); Universidade Federal Fluminense (UFF); Instituto de Saúde de Nova Friburgo}}

\abstract{Este capítulo apresenta uma proposta de atividade linguística, centrada em estruturas de alto custo de processamento, que pode se mostrar promissora para o quadro do Transtorno do Desenvolvimento da Linguagem (TDL ). O procedimento é teoricamente embasado, sendo fundamentado em hipóteses e evidências da pesquisa linguística/psicolinguística acerca da possível natureza das dificuldades encontradas no domínio da sintaxe por esses indivíduos. Acredita-se assim que possa ser relevante para a prática terapêutica e escolar, embora ainda não se tenha evidências conclusivas da eficácia do procedimento aplicado.

\textbf{Palavras-chave}: Custo de processamento; Transtorno do desenvolvimento da Linguagem; Intervenção metalinguística}

\IfFileExists{../localcommands.tex}{
   \addbibresource{../localbibliography.bib}
   % add all extra packages you need to load to this file

\usepackage{tabularx,multicol}
\usepackage{url}
\urlstyle{same}

\usepackage{listings}
\lstset{basicstyle=\ttfamily,tabsize=2,breaklines=true}

\usepackage{langsci-basic}
\usepackage{langsci-optional}
\usepackage{langsci-lgr}
\usepackage{langsci-osl}
% \usepackage{./langsci/styles/langsci-lgr}
% \usepackage{./langsci/styles/langsci-osl}
% \usepackage{langsci-gb4e}

\usepackage{tikz}
\usetikzlibrary{patterns,calc}
\pgfdeclarepatternformonly{south east lines}{\pgfqpoint{-0pt}{-0pt}}{\pgfqpoint{3pt}{3pt}}{\pgfqpoint{3pt}{3pt}}{
    \pgfsetlinewidth{0.6pt}
    \pgfpathmoveto{\pgfqpoint{0pt}{3pt}}
    \pgfpathlineto{\pgfqpoint{3pt}{0pt}}
    \pgfpathmoveto{\pgfqpoint{.2pt}{-.2pt}}
    \pgfpathlineto{\pgfqpoint{-.2pt}{.2pt}}
    \pgfpathmoveto{\pgfqpoint{3.2pt}{2.8pt}}
    \pgfpathlineto{\pgfqpoint{2.8pt}{3.2pt}}
    \pgfusepath{stroke}}
    
\usepackage{stmaryrd}
\usepackage{wasysym}
\usepackage{multirow}
\usepackage{caption}
\usepackage{subcaption}
\usepackage{mathrsfs}
\usepackage{qtree}

\usepackage{linguex}


   %pminos do not split footnotes
% \interfootnotelinepenalty=10000 %Footnote in Laporte chapters has to be split SN


%\DeclareIndexNameFormat{default}{%
%\nameparts{#1}%
%\usebibmacro{index:name}%
%{\index[names]}%
%{\namepartfamily}%
%{\namepartgiveni}%
% {}% L1
% {}% L2
%{\namepartprefix}% generates spurious space L3
%{\namepartsuffix}% generates spurious space L4
%}

%  {\DeclareIndexNameFormat{default}{%
%     \usebibmacro{index:name}{\index[names]}{#1}{#3}{#5}{#7}}}

%\DeclareIndexNameFormat{default}{%
%  \usebibmacro{index:name}{\sindex[nom]}{#1}{#3}{#5}{#7}}

%\DeclareIndexNameFormat{default}{%
%  \usebibmacro{index:name}{\sindex[person]}{#1}{#3}{#5}{#7}}
%\DeclareIndexNameFormat{default}{%
%\nameparts{#1} \usebibmacro{index:name}{\sindex[person]]}{\namepartfamily}{‌​\namepartgiven}{\nam‌​epartprefix}{\namepa‌​rtsuffix}}

%\newcommand{\smiley}{:)}

%\renewbibmacro*{index:name}[5]{%
%\usebibmacro{index:entry}{#1}%
%{\iffieldundef{usera}{}{\thefield{usera}\actualoperator}\mkbibindexname{#2}{#3}{#4}{#5}}}

% \newcommand{\noop}[1]{}

%remove for final
%\overfullrule=1mm

\newcommand{\tobi}[2]}}
\renewcommand{\S}[1]{\tobi{#1}{\textsc{*}}}

% this volume references
% puts: [this volume]
% already defined: \citetv
%\newcommand{\citepv}[1]{(\citeauthor{#1} \citeyear*{#1} [this volume])}
\newcommand{\citealtv}[1]{\citeauthor{#1} \citeyear*{#1} [this volume]}

%parentheses around example number
\newcommand{\pref}[1]{(\ref{#1})}

% in-text examples

\newcommand{\lnex}[1]{\textit{#1}} %target lang word
\newcommand{\lnlit}[1]{(lit.: `#1')} %literal reading
\newcommand{\lnlat}[1]{(#1)} % latinization
\newcommand{\lntrans}[1]{`#1'} %translation
\newcommand{\lnexl}[2]%
{\lnex{#1}{} \lnlat{#2}} % ex with latinization
\newcommand{\lnexlat}[3]{\lnex{#1}{} \lnlat{#2}{} \lntrans{#3}} % ex with latinization and tranl.

%ch01
\newcommand{\co}[1]{\mbox{\textbf{#1}}}

%ch09

\newcommand{\cyrbulg}[1]{\begin{otherlanguage*}{bulgarian}#1\end{otherlanguage*}}


%ch10
\newcommand{\nlp}{{\small NLP}}
\newcommand{\mwe}{{\small MWE}}
\newcommand{\rae}{{\small RAE}}
\newcommand{\lvc}{{\small LVC}}
\newcommand{\pos}{{\small P}o{\small S}}
%\newcommand{\todo}[1]{ \textcolor{red}{#1} }

%\renewcommand{\labelenumi}{\theenumi}
%\ainamefmt{{vv}{ll}{, ff}{, jj}} % fullname

\newcommand{\biberror}[1]{{\color{red}#1}}

\newcommand{\osenovaitem}{--~}
   %% hyphenation points for line breaks
%% Normally, automatic hyphenation in LaTeX is very good
%% If a word is mis-hyphenated, add it to this file
%%
%% add information to TeX file before \begin{document} with:
%% %% hyphenation points for line breaks
%% Normally, automatic hyphenation in LaTeX is very good
%% If a word is mis-hyphenated, add it to this file
%%
%% add information to TeX file before \begin{document} with:
%% %% hyphenation points for line breaks
%% Normally, automatic hyphenation in LaTeX is very good
%% If a word is mis-hyphenated, add it to this file
%%
%% add information to TeX file before \begin{document} with:
%% \include{localhyphenation}
\hyphenation{
    Beck-man
    Ngu-yen
    back-chan-nel
    back-chan-nels
    mo-not-o-nous
    ste-reo-typ-i-cal
}

\hyphenation{
    Beck-man
    Ngu-yen
    back-chan-nel
    back-chan-nels
    mo-not-o-nous
    ste-reo-typ-i-cal
}

\hyphenation{
    Beck-man
    Ngu-yen
    back-chan-nel
    back-chan-nels
    mo-not-o-nous
    ste-reo-typ-i-cal
}

   \boolfalse{bookcompile}
   \togglepaper[14]%%chapternumber
}{}

\shorttitlerunninghead{Intervenção no processamento de estruturas de alto custo}
\begin{document}
\maketitle

\section{Introdução}
Uma das dificuldades que tem sido apontada como característica do quadro do Transtorno do Desenvolvimento da Linguagem (TDL), no que concerne à sintaxe, reside na compreensão ou na produção das chamadas \emph{estruturas de alto custo de processamento}, tais como orações na voz passiva e orações relativas (adjetivas), por exemplo (ver também \textcitetv{chapters/13}).\footnote{A norma adotada na escrita deste capítulo foi a do português brasileiro.}
% \footnote{Na sequência de restrições decorrentes do template usado para publicação na editora Language Science Press, os capítulos que integram o volume surgem com a referência [this volume], em inglês.}
Não há até então clareza quanto à natureza dessas dificuldades: em que medida decorrem de problemas inerentes ao domínio da linguagem ou de comprometimentos no uso efetivo da memória de trabalho no processamento linguístico \citep{Archibald2006, Correa2020, Friedmann2009, Marinis2011, Ullman2005}. Seria possível para fonoaudiólogos/terapeutas da fala ou professores intervirem em dificuldades no desenvolvimento da linguagem, de modo a propiciar à criança plena proficiência em sua língua materna, particularmente no que concerne a habilidades no domínio da sintaxe? Acreditamos que, em princípio, sim, ainda que esta seja uma questão empírica cuja resolução envolve muitas etapas, desde a criação de procedimentos fundamentados em teorias linguísticas/psicolinguísticas até o acompanhamento de longo prazo do desempenho das crianças submetidas a tais procedimentos, em contraste com um grupo pareado sem qualquer intervenção – tarefa difícil de ser levada a cabo de forma satisfatória. Neste capítulo, buscamos dar um passo na direção de satisfazer o primeiro requisito, apresentando um procedimento fundamentado em hipóteses e evidências da pesquisa linguística/psicolinguística, trazendo ainda, contudo, poucas evidências relativas ao desempenho pós-procedimento. Consideramos que, ainda que sem evidências conclusivas da eficácia do procedimento aplicado, a disseminação das ideias que nos levaram a conduzi-lo pode contribuir, de algum modo, para a prática terapêutica e escolar e incentivar a condução de estudos controlados em contextos terapêuticos.

Este capítulo se organiza da seguinte forma. Com o intuito de responder à questão inicial, começamos por considerar o que o processamento linguístico envolve, em geral, e o que torna o processamento de determinadas estruturas uma tarefa de alto custo. Em seguida, trazemos uma breve revisão de práticas de intervenção em uso na área clínica, com vista a demonstrar a carência de procedimentos especificamente voltados para o tipo de habilidades aqui focalizadas. Apresentamos então, o procedimento por nós aplicado, de forma ainda exploratória, o qual nomeamos, à guisa de facilitar a referência a ele no texto, de ProMetaS (Procedimento Metalinguístico-lúdico de intervenção no domínio da Sintaxe), e trazemos resultados preliminares sugestivos de um efeito promissor. Por fim, as considerações finais sinalizam possíveis desdobramentos.  

\section{Processamento de estruturas de alto custo}
\largerpage[-2]
O processamento da linguagem é um tipo particular de processamento de informação, ou seja, transformação do que chamamos de dados de entrada em dados de saída, por meio de uma série de operações conduzidas ao longo do tempo. 

À medida que ouvimos, lemos ou produzimos enunciados linguísticos, nossa mente conduz uma série de operações. Essas operações, implementadas no cérebro, são automáticas, não passam pela nossa consciência e transcorrem de forma muito rápida (em milésimos de segundos). Sua condução depende do conhecimento da língua em questão e dos recursos necessários ao desempenho linguístico. 

Na produção de enunciados linguísticos, partimos de uma ideia e de uma intenção, a serem gramaticalmente organizadas. Na compreensão da linguagem, partimos da percepção do sinal da fala/escrita e chegamos a uma representação semântica do enunciado. Essa representação deve ser mapeada em eventos/situações do mundo real ou imaginário, enriquecida por meio de inferências de ordem pragmática, integrada a outras representações de natureza semântica, ao longo do discurso, e então incorporada à nossa base de conhecimento. 

A condução desses processos pode ser mais ou menos custosa para qualquer falante da língua, em função de uma série de fatores, dentre os quais, o tipo de estrutura sintática subjacente ao enunciado a ser produzido ou compreendido. Alguns indivíduos, contudo, podem ter esse tipo de dificuldade acentuada, comprometendo sua produção e/ou compreensão. A produção pode resultar em enunciados truncados, ou que não correspondem precisamente à ideia e intenção de fala inicial. Na compreensão, a criação de uma representação semântica e seu mapeamento em eventos no contexto de fala podem ser afetados por dificuldades na condução da análise sintática do enunciado. Esse tipo de dificuldade é sugestivo de um transtorno (o Transtorno do Desenvolvimento da Linguagem – TDL), que afeta a aquisição e/ou o uso efetivo da língua no desempenho linguístico, na ausência de outros problemas que possam explicar um desenvolvimento linguístico atípico. As manifestações desse tipo de transtorno podem, ainda, coexistir com outras alterações, como o TDAH (Transtorno do Déficit de Atenção e Hiperatividade), por exemplo. 

A pesquisa linguística e psicolinguística nos permite apontar possíveis causas para dificuldades que indivíduos com sintomas de TDL podem apresentar – seja no que concerne ao conhecimento da língua requerido, seja no que diz respeito aos recursos e habilidades necessários para que a produção e a compreensão da linguagem se realizem efetivamente. Uma das dificuldades que parece ser comum a diferentes línguas, independentemente de sua família linguística ou riqueza morfológica, reside nas chamadas estruturas de alto custo.

Na teoria gerativa chomskyana, essas estruturas têm sido caracterizadas a partir da noção de movimento sintático: o elemento é gerado na posição em que é semanticamente interpretado (marcado com \longrule ~nos exemplos abaixo) e se move até a posição onde aparece fonologicamente (em itálico, nos exemplos abaixo).

\ea	O \emph{menino} foi perseguido \longrule ~pelo cachorro.
\z
\ea Eu encontrei a \emph{menina} que o diretor procurava \longrule.
\z


Para caracterizar o tipo de demandas que essas estruturas impõem ao processamento da linguagem, comecemos por considerar o processo de compreensão. 

A compreensão de enunciados linguísticos se faz de forma incremental, da “esquerda para a direita”, ao longo do tempo, à medida que “pacotes” informacionais (delimitados com base na prosódia, por exemplo) são recortados e os itens lexicais neles contidos são reconhecidos na sua forma fônica (ou gráfica, com seu correspondente fônico, se na leitura). As propriedades gramaticais e semânticas desses itens são então recuperadas do léxico mental (entendido, grosso modo, como um componente da memória de longo prazo que retém nosso conhecimento relativo às propriedades - fonológicas, semânticas e gramaticais, ou formais - das palavras ou morfemas da língua). As propriedades gramaticais dos elementos do léxico fornecem a informação necessária para que uma análise sintática (processamento sintático) seja conduzida. Essa análise consiste em construir uma estrutura em que esses elementos se relacionem de forma hierárquica. É com base nessa estrutura hierárquica e nas propriedades semânticas dos elementos do léxico que a compõem, que o significado do enunciado pode ser obtido. Esse significado pode ser pensado em termos de uma estrutura em que há argumentos e predicados, dando origem a uma proposição – correspondente à ideia do falante/escritor, com pistas para sua intenção de fala (naquilo que o conhecimento compartilhado da língua permite) (\figref{fig:14:menino}, ver também \textcitetv{chapters/12}).

% \begin{enumerate}
%     \begin{enumerate}
%         \item []
%         \begin{tikzpicture}[sibling distance=2.3cm,
%                             level distance=1.2cm,
%                             edge from parent path={(\tikzparentnode.south) --(\tikzchildnode.north)},
%                             inner sep=1.5pt,
%                             level1/.style={sibling distance=3.5em},
%                             level 2/.style={sibling distance=3em}
%         ]´
%         \node {}
%         child {node{} child{node{O}} child{node{menino}}} child{ node{} child{node {leu}} child{node{} child {node{um} } child{node{livro}}}};
%         \end{tikzpicture}
%             \z
% \z



\begin{figure}
\caption{O menino leu um livro ... (elementos do léxico em sequência)\\
    Estrutura hierárquica}
    \label{fig:14:menino}
{
\begin{forest} nice empty nodes
  [
    [
        [O]
        [menino]
    ]
    [
        [leu]
        [
            [um]
            [livro]
        ]
    ]
  ]
\end{forest}
}
\end{figure}

O processo de construção de estruturas hierárquicas na compreensão de enunciados linguísticos é bem robusto, ou seja, muito resistente a condições adversas, uma vez que a capacidade de construir esse tipo de estrutura parece ser o mais fundamental da linguagem humana. Contudo, pode haver dificuldades no reconhecimento da informação gramatical dos elementos do léxico, quando da construção de uma estrutura hierárquica – seja por não ter sido ainda devidamente adquirida, ou por não estar imediatamente acessível, comprometendo a construção da estrutura hierárquica e, consequentemente, a interpretação semântica do enunciado.

Dificuldades de ordem sintática podem também decorrer do tipo de demanda que o estabelecimento de relações entre elementos em diferentes posições na estrutura hierárquica pode apresentar. Notem que a análise de uma sequência de elementos do léxico em uma estrutura hierárquica requer que essa sequência fique temporariamente mantida em uma memória. O sistema de memória que mantém representações mentais temporariamente ativas, no desempenho de uma tarefa cognitiva, chama-se \emph{memória de trabalho}. No processamento linguístico, a representação da sequência de elementos do léxico analisada em termos de relações hierárquicas é mantida na memória de trabalho por curto espaço de tempo -  em princípio, o suficiente para que uma interpretação semântica seja obtida. Uma vez que um dado segmento seja semanticamente interpretado, sua representação, digamos, mais literal (na qual a sequência linear dos itens se mantém), pode se esvair da memória de trabalho, cedendo espaço para que novos itens do léxico sejam reconhecidos e analisados.

As estruturas geradas por meio de movimento sintático irão requerer que relações sintáticas se estabeleçam a longa distância, considerando-se a sequência linear em que os elementos do léxico se apresentam, o que pode criar alto custo de processamento. Vejamos alguns exemplos.

\ea\label{ex:14:4}	\emph{O leão} foi puxado \longrule ~pelo tigre.
\z
\ea\label{ex:14:5}	\emph{Que leão} o tigre puxou \longrule?
\z
\ea\label{ex:14:6} 	Mostra \emph{o leão} que o tigre puxou \longrule.
\z
\ea\label{ex:14:7}	\emph{O leão} que o tigre puxou \longrule ~rugiu.
\z

Em (4), uma vez que o primeiro sintagma é processado, este é mantido na memória (sem necessariamente uma função sintática e um papel temático) até o reconhecimento do verbo (que permite que o sintagma mantido na memória seja analisado como sujeito). Seu papel temático não pode, contudo, ser atribuído prontamente, pois isso pode levar a erro. É necessário que o verbo auxiliar e o morfema de particípio do verbo principal (por exemplo, SER  \textsc{Verbo}-ADO) sejam reconhecidos como informação gramatical relativa à propriedade formal \emph{voz}, o que acarreta a atribuição do papel de paciente ou tema ao sujeito, associado à posição de base como argumento interno do verbo principal. Nesse caso, se o ouvinte não reconhece essa informação (descontínua), se não mantém o primeiro sintagma ativo na memória até que essa informação seja reconhecida e a relação temática possa ser estabelecida, a compreensão pode ficar prejudicada. 

Nesse exemplo, o sintagma sujeito e o complemento da preposição têm papéis temáticos que podem ser intercambiáveis. Dizemos então que estamos diante de \emph{passivas reversíveis}. A reversibilidade de papéis temáticos traz uma dificuldade que independe da estrutura sintática, mas que é acentuada quando essa estrutura impõe demandas específicas, como no caso das passivas. Como os papéis de \emph{agente} ou de \emph{experenciador} são os mais usuais para sujeitos, particularmente animados, se o ouvinte/leitor usa a estratégia de atribuir imediatamente esses papéis ao primeiro sintagma, tomado como sujeito, sua análise sintática será prejudicada. Assim que essa interpretação se mostrar incompatível com a informação gramatical subsequente, será necessária uma reanálise, impondo custo de processamento. 

Em (5), o ouvinte/leitor deve reconhecer o \emph{que}+\emph{N} como indicativo de que um constituinte foi deslocado para o início da sentença (a posição mais à esquerda), em uma estrutura interrogativa, mas sua posição de origem – como objeto do verbo – só poderá ser identificada mediante o reconhecimento do verbo como transitivo direto. Até lá, o constituinte \emph{que leão } terá de ficar mantido, não analisado (sem função sintática e sem papel temático) na memória de trabalho. Logo, se a informação gramatical relevante (\emph{que}+\emph{N}) não for prontamente reconhecida, a análise estará comprometida, e consequentemente sua interpretação semântica também estará. De forma análoga, se o tempo necessário para a identificação da ausência de um elemento lexical como complemento do verbo exceder o tempo em que \emph{que}+\emph{N} puder ficar ativo na memória, a análise será prejudicada. 

Em (6), dificuldade semelhante se apresenta – o reconhecimento de \emph{o leão que} sinaliza que este constituinte tem de ser mantido na memória de trabalho até que sua posição sintática na oração relativa possa ser reconhecida. Note que além da distância entre a posição mais à esquerda desta oração e a posição de objeto do verbo, há um elemento interveniente (o sujeito da interrogativa em (5) e da relativa em (6)) que tem propriedades muito semelhantes às do elemento movido. Tal presença pode afetar o preenchimento da posição do objeto ou simplesmente causar uma sobrecarga que impeça a finalização da análise. Nesse caso, o elemento interveniente (\emph{o tigre}) pode ser tomado como objeto do verbo, de forma equivocada, comprometendo a correta interpretação semântica de (5) e de (6). O efeito desse elemento será maior quanto mais semelhante ao elemento movido este for \citep{Grillo2008, Friedmann2009}. 

No caso de (7), as demandas de (6) são ampliadas, pois a oração relativa está encaixada no sujeito da oração principal. Mesmo que a análise da oração relativa tenha sido concluída de forma adequada, há ainda a demanda de manter o sujeito (um sintagma complexo) na memória, até que sua relação com o verbo da oração principal se estabeleça. Essa demanda adicional pode comprometer a análise e, em consequência, o sintagma mais recentemente analisado pode ser tomado como sujeito da oração principal, acarretando erro na interpretação semântica (como \emph{o tigre rugiu} em (7)). 

Nesse caso, o custo não decorre do processamento da relativa de objeto em si, mas do fato de termos um primeiro sintagma “pesado”, isto é, com uma carga de informação grande pela presença da oração relativa – a ser analisado como sujeito, mediante o reconhecimento do verbo da oração principal. 

Interpretações semânticas equivocadas acarretam mapeamento inadequado em eventos do mundo externo à língua. E esse mapeamento, por si só, pode ser dificultado, dependendo da tarefa em questão. Assim sendo, quanto menor a demanda imposta pela tarefa ao mapeamento enunciado-evento, mais chances teremos de avaliar em que medida o processamento sintático está comprometido.\footnote{Para dificuldades no mapeamento sentença-evento em tarefas de reconhecimento de imagens a partir da compreensão de orações relativas, ver \citep{Correa2019}.} 

Vimos que, em todas as sentenças exemplificadas, há uma relação de dependência descontínua entre os elementos apresentados linearmente (auxiliar e particípio) e/ou de longa distância, entre o elemento deslocado e sua posição de origem. 

Para superar as demandas que essas estruturas impõem é necessário o pronto reconhecimento da informação gramatical que as sinaliza e a otimização dos recursos da memória de trabalho. Essa otimização irá envolver as chamadas \emph{estratégias de ensaio} \citep{Baddeley1990} que possibilitam a manutenção de uma sequência analisada hierarquicamente, na memória de trabalho, pelo tempo requerido para que as relações de longa distância se estabeleçam. 

Distinguir em que medida é o pronto reconhecimento da informação gramaticalmente relevante que pode estar comprometido ou dificuldades na manutenção de sequências na memória de trabalho, ou ainda, possíveis dificuldades específicas diante de um elemento interveniente semelhante ao movido, é um dos desafios na caracterização do tipo de problema que se manifesta no TDL. 

Em suma, as dificuldades que as demandas das estruturas de alto custo apresentam na compreensão são: (i) o ouvinte/leitor não recupera a informação gramaticalmente relevante dos elementos do léxico. Isso pode decorrer de um problema na aquisição da língua – aquela informação não foi devidamente incorporada ou registrada no léxico mental. Pode também decorrer de um problema de acesso pouco eficiente, na implementação dessa operação no cérebro – como quando tentamos ligar um aparelho elétrico e este ora liga, ora não liga, ou é necessário um tempo além do normalmente requerido para que o acesso aconteça; (ii) o ouvinte/leitor tem dificuldade em manter estruturas parcialmente analisadas na memória de trabalho, de modo que, se essas estruturas precisam ser recuperadas posteriormente, em ponto mais avançado da análise do enunciado, elas não se encontram mais acessíveis, o que impede que a análise seja finalizada de forma satisfatória. Essa dificuldade pode residir na capacidade ou na habilidade de a criança lidar com a presença de um elemento interveniente que só se distingue do movido por uma propriedade gramatical ou traço formal (relativo à possibilidade de movimento sintático desse último) - dificuldade essa que tem sido tomada como inerente a processos de natureza maturacional e que pode se estender até uma infância tardia, no caso do TDL \citep{Friedmann2009}; (iii) o ouvinte/leitor tem dificuldade em mapear uma estrutura semanticamente interpretada em eventos do mundo, ao longo do processo, o que pode dificultar a condução da análise ainda por fazer. 

No que diz respeito à produção, passivas requerem a codificação de um elemento com papel temático (tema) atribuído pelo verbo como sujeito da sentença. Logo, essa relação tem de estar plenamente estabelecida na mente do falante antes da codificação do sujeito, e a informação gramatical relativa à \emph{voz} prontamente acessível. No caso de relativas, a emissão de um objeto relativizado pressupõe que o complemento do verbo (no caso de relativas de objeto direto) ou de uma preposição (no caso de objeto indireto) tenha sido planejado, de modo que seu deslocamento para a posição mais à esquerda da sentença aconteça. Tal planejamento prévio à emissão do sintagma mais à esquerda não só traz alta demanda linguística e cognitiva, em si mesmo, como requer que a estrutura planejada se mantenha na memória até a produção do verbo. Caso o falante tenha dificuldade de lidar com essas demandas ou introduza um sintagma cuja relação com o verbo ainda não foi plenamente estabelecida, na fala fluente, estratégias simplificadoras da estrutura ou de último recurso podem ser requeridas. Uma estratégia eficiente seria, por exemplo, formular uma oração relativa de sujeito em sentença na voz passiva, como em (8), ao invés de uma relativa de objeto como (6). Relativas cortadoras (9) assim como estruturas com um pronome resumptivo (10) seriam alternativas de menor custo na produção de relativas padrão de objeto indireto que, em línguas como o português, requerem que a preposição  também seja movida para a posição mais à esquerda (11).\footnote{Ver \citet{Correa2018} para estratégias na produção de relativas de objeto direto, indireto e genitivas diante de situações que favorecem e que não favorecem o planejamento prévio do sintagma relativizado.}


\ea\label{ex:14:8}	 ... o leão que foi puxado pelo tigre ...
\z
\ea\label{ex:14:9} 	... o menino que a professora falou...
\z
\ea\label{ex:14:10} 	... o menino que a professora falou dele...
\z
\ea\label{ex:14:11} 	... o menino de quem a professora falou...
\z

Vê-se que, na produção, estratégias de menor custo fornecidas pela língua são usadas por qualquer falante, não necessariamente evidenciando dificuldades de processamento. Quando há truncamentos ou alteração de papeis temáticos é que comprometimentos no domínio da sintaxe podem ficar evidentes.\footnote{Ver \citet{Correa2021} para esse tipo de produção.} Já na compreensão, o uso de estratégias cognitivas não levaria a um bom resultado, diante da situação-problema trazida pelo alto custo de processamento decorrente de o ouvinte/leitor não recuperar a informação gramaticalmente relevante dos elementos do léxico (em (i) acima), ou de o ouvinte/leitor ter dificuldade em manter estruturas parcialmente analisadas na memória de trabalho (em (ii) acima), por exemplo. Em (4), tomar o primeiro sintagma como sujeito e atribuir a este o papel de agente acarretaria erro. Assim sendo, a avaliação de habilidades linguísticas na compreensão torna-se crucial para a identificação de um transtorno de linguagem no domínio da sintaxe. Nesse tipo de avaliação, a dificuldade do ouvinte/leitor em mapear uma estrutura semanticamente interpretada em eventos do mundo (em (iii) acima) pode ser minimizada em função do tipo de tarefa utilizada, como será demonstrado a seguir. 

Em suma, no que concerne à compreensão, as dificuldades (i-iii) precisam ser levadas em conta quando se concebem procedimentos de intervenção no TDL que sejam linguística e psicolinguisticamente fundamentados. No que concerne à produção, o uso de estratégias de baixo custo pode ser reconhecido como uma solução bem-sucedida diante de dificuldades de linguagem.


\section{Intervenção clínica em dificuldades de linguagem no domínio da sintaxe }
Os procedimentos clínicos disponíveis para avaliação do processamento sintático são raros, o que faz com que seja difícil identificar crianças com TDL com alterações nesse domínio, principalmente aquelas que possuam alterações mais sutis. Em um estudo realizado por \citet{Selin2019}, em que foram entrevistados fonoaudiólogos/terapeutas da fala norte-americanos a fim de investigar sua avaliação e tomada de decisão clínica quanto à intervenção em crianças com TDL, foi verificado que são usadas avaliações abrangentes de linguagem e do vocabulário, mais frequentemente do que de aspectos morfossintáticos e sintáticos. Alguns estudos voltados para a possibilidade de intervenção clínica no TDL focalizam questões de ordem sintática, como \citet{Levy2009}, no hebraico, e \citet{Balthazar2018} no inglês, os quais partem de uma teorização linguística acerca das estruturas trabalhadas. Contudo, estudos dessa natureza no português brasileiro (PB) e europeu (PE) são ainda incipientes. 

\citet{Levy2009} descreveram uma intervenção sintática em um adolescente de 12:2 anos de idade (G.), falante do hebraico, que apresentava dificuldades na compreensão e na produção de estruturas sintáticas que envolvem movimento sintático, como orações relativas, sentenças topicalizadas e sentenças com movimento de verbo. As autoras partiram do pressuposto de que programas de intervenção metalinguística são eficientes no tratamento de crianças com TDL e adultos com agramatismo, e propuseram uma intervenção metalinguística nas estruturas sintáticas alteradas. A intervenção foi composta por 16 sessões, ao longo de seis meses, que incluíram o ensino explícito do movimento sintático. A intervenção foi dividida em três partes: explicação, treinamento e teste. As estruturas foram apresentadas por escrito e depois oralmente. Primeiramente, quem aplicou o procedimento conversava com o adolescente sobre a estrutura argumental das sentenças (incluindo-se tipos de verbos e atribuição de papéis temáticos), com o objetivo de transformar o conhecimento implícito em explícito, aproveitando a sensibilidade intacta de G. a essas estruturas, e dessa forma, preparando-o para o reconhecimento das propriedades das estruturas que envolvem movimento. O adolescente era convidado a escrever frases com vários tipos de verbos e a relação entre verbos e argumentos era explicitada. Estímulos visuais foram inseridos, de modo a solicitar que G. grifasse com cores diferentes agente e tema, tornando mais explícito esse conhecimento. Também foram usadas analogias para explicitar a relação entre verbos e argumentos (como o verbo seria um oficial e os argumentos simples soldados). Em seguida, a tarefa era realizada de forma oral, sem a possibilidade de apoio visual (cores ou indicações escritas). Nas sessões seguintes, o movimento sintático era explorado por meio de cartões e histórias para tornar explícitas as estruturas enfocadas. A intervenção realizada resultou em melhoras na compreensão das estruturas trabalhadas e foi possível observar generalização para as que não foram diretamente praticadas. Ou seja, embora o movimento sintático só tenha sido abordado em orações relativas e estruturas de topicalização, outras estruturas que envolvem esse movimento, como as interrogativas QU, também apresentaram melhoras, mesmo após 10 meses da aplicação do procedimento. 

No estudo realizado em língua inglesa por \citet{Balthazar2018}, em crianças com TDL, também foram enfocadas questões relativas ao processamento sintático. Nesse trabalho, 30 crianças com TDL de 10 a 14 anos receberam intervenção por meio de atividades em que sentenças complexas do inglês (relativas, adverbiais e completivas) foram apresentadas oralmente e por escrito, utilizan\-do-se \emph{softwares} (\emph{Power point} e \emph{Word}), assim como tarefas envolvendo papel e lápis. Inicialmente eram fornecidas explicações sobre as características estruturais das sentenças trabalhadas, com exemplos, e as crianças deveriam repeti-las. Em seguida, foram realizadas atividades que envolviam tarefas para treinar o conhecimento explícito das estruturas em questão. Foi verificado que, após um período de 9 semanas, as crianças apresentaram um melhor desempenho na reavaliação, com mudanças mais expressivas nas orações adverbiais e relativas. 

No português brasileiro (PB), as intervenções clínicas em crianças com TDL geralmente focalizam as manifestações alteradas na comunicação, mais especificamente comprometimentos na linguagem verbal, sem necessariamente pressupor um particular modelo da gramática da língua. Um acompanhamento longitudinal fonoaudiológico, em estudo de caso com uma criança diagnosticada com TDL, dos 5 aos 9 anos de idade, é relatado por \citet{Nicolielo2014}. O processo de intervenção concentrou-se no domínio fonológico, por meio do Ciclos modificado, comumente utilizado em crianças com transtornos fonológicos \citep{Tyler1987}, já adaptado para o PB \citep{Mota1990}. Habilidades pragmáticas, sintáticas, semânticas e relacionadas à memória e à consciência fonológica também foram trabalhadas, mas não como componentes principais da intervenção, e, aos 7 anos, foram também conduzidas atividades que envolviam habilidades importantes para o desenvolvimento da leitura e escrita. A criança apresentou melhoras quanto à inteligibilidade da fala. Em relação à sintaxe, aos 8 anos, foram observadas produções de orações interrogativas QU, sentenças complexas com subordinação, negativas com posicionamento adequado do advérbio de negação, assim como a utilização de flexão verbal e nominal, antes ausentes da fala da criança.  

\citet{BefiLopes2013} sugerem procedimentos de intervenção em crianças com TDL com dificuldades em habilidades relativas à produção de narrativas, os quais incluem aspectos gramaticais. Nesse contexto, foram indicadas atividades que envolvem a sequencialização cronológica de fatos ou cenas em episódios e a estimulação da produção de estruturas que indicam a ligação entre estes, como sentenças complexas com conjunções coordenativas aditivas (e), adversativas (mas), explicativas (porque), assim como subordinativas integrantes (que, se), causais (porque, por isto), comparativas (do que), temporais (quando), dentre outras. 

As atividades propostas incluíram tarefa com orações coordenadas e subordinadas para a criança completar; produção de frases induzida por desenhos; e contação de histórias com fantoches, dando ênfase aos elementos funcionais e às conjunções enfocadas. 

Para a reabilitação da compreensão, \citet{BefiLopes2013b} indicam atividades que tenham como objetivos a ampliação do vocabulário, em categorias semânticas bem delimitadas. Sugerem ainda brincadeiras e atividades que estimulem a compreensão e a produção de sentenças simples em ordem canônica (SVO), incluindo sentenças com pronomes pessoais, possessivos e interrogativos. Também é indicada intervenção na compreensão da flexão verbal e nominal, por meio de atividades lúdicas em que enunciados são produzidos pelo terapeuta, chamando-se a atenção da criança para morfemas de tempo, pessoa, número e gênero, por exemplo.  São sugeridos jogos e brincadeiras de faz-de-conta que estimulem a percepção e a produção desses elementos pelas crianças. 

Para a reabilitação da morfossintaxe em crianças com TDL, \citet{BefiLopes2013a} propõem atividades que têm como objetivo aumentar a extensão e a complexidade das sentenças, promover o uso da concordância verbal de gênero e número e a elaboração discursiva. Para o trabalho com palavras de classes fechadas (como artigos, conectivos, preposições), foram concebidas brincadeiras livres e a utilização de miniaturas e brinquedos que possibilitasse a representação de ações, cujo enunciado as contivesse. Para lidar com a concordância, foram propostas, por exemplo, atividades com figuras que representam cenas, e a tarefa da criança era descrever as figuras, produzindo frases com a concordância sujei\-to-verbo adequada.  

No PE, um programa que se destina às habilidades morfossintáticas, semânticas e metalinguísticas em crianças com Perturbação do Desenvolvimento da Linguagem (PDL) em idade pré-escolar ou no início da idade escolar é o Programa de Intervenção em Competências Linguísticas (PICL) \citep{Lousada2016a}. No nível da morfossintaxe, são apresentadas atividades relativas à compreensão e à produção de frases simples e complexas por coordenação e subordinação, frases declarativas e interrogativas, ativas e passivas e atividades que fazem uso da flexão nominal de gênero e número e de derivação de palavras. \citet{Lousada2016b} investigaram e eficácia do PICL em um grupo de 7 crianças em idade pré-escolar com PDL, comparadas com um grupo controle que também foi constituído por 7 crianças com as mesmas alterações de linguagem. As atividades foram apresentadas no formato de jogos e diferentes técnicas foram usadas para desenvolver a linguagem, como modelagem, expansão, reformulação e imitação. O PICL foi aplicado no primeiro grupo em 8 sessões individuais (2 vezes por semana). e o grupo controle foi mantido sem intervenção. O desempenho dos dois grupos foi comparado, então, após 4 semanas de intervenção. O grupo experimental apresentou melhoras significativas nas habilidades de linguagem receptiva, expressiva e metalinguísticas, sendo que o mesmo resultado não foi encontrado no grupo controle. Porém, após 4 semanas de intervenção neste grupo, também foi observada melhora significativa em suas habilidades linguísticas. 

No PE, também há um instrumento que enfoca habilidades sintáticas e morfológicas, além dos domínios fonológico, semântico e pragmático, para a intervenção em crianças de 5 a 8 anos, com perturbação de linguagem -- “a Cartola Mágica” \citep{Dias2015}.  Esse instrumento faz uso da figura de um gato que possui uma cartola com gavetas que simbolizam os domínios da linguagem a serem trabalhados. Na gaveta destinada à sintaxe, são apresentados cartões associados a frases escritas de forma sintaticamente desordenada e a tarefa da criança é reestruturar a frase de forma gramaticalmente adequada, com sentido lógico. 

Como é possível constatar, tanto no PB quanto no PE, ainda são escassas as pesquisas em terapia de base linguística, para crianças com TDL, voltada especificamente para o domínio da sintaxe. Nos procedimentos de intervenção conduzidos no PB e no PE, aqui relatados, observa-se o uso de tarefas em que as estruturas em questão são trabalhadas de forma implícita, diferentemente dos estudos conduzidos em hebraico e em inglês, anteriormente citados, que exploram o conhecimento intuitivo da gramática de forma explícita, como base para as atividades lúdicas realizadas. 

No estudo a ser aqui relatado, fez-se uso de atividades de natureza metalinguística, ainda que de forma menos explícita do que nos estudos conduzidos em inglês e hebraico, fazendo uso de recursos gráficos que chamam a atenção para os constituintes, relações de dependência e movimento sintático. Tal como nos estudos conduzidos em PB e PE, predominam atividades lúdicas em que a criança é solicitada a fazer uso das estruturas em questão.

\section{ProMetaS}
\subsection{Identificação de crianças em risco de TDL no domínio da sintaxe}\label{subsec:cap14sec41}
Esse estudo foi conduzido com crianças em idade escolar, de 8 a 12 anos, estudantes de quatro escolas da rede pública municipal de ensino, três situadas na zona sul e uma na zona norte do Rio de Janeiro. Crianças com suspeita de TDL foram identificadas, mediante a aplicação de uma bateria de testes centrada na compreensão de estruturas sintáticas de alto custo - o módulo sintático do MABILIN (Módulos de Avaliação de Habilidades Linguísticas).\footnote{MABILIN - Módulos de Avaliação de Habilidades Linguísticas – é uma bateria de testes desenvolvida no LAPAL (Laboratório de Psicolinguística e Aquisição da Linguagem, da PUC-Rio (Pontifícia Universidade Católica do Rio de Janeiro \citep{Correa2000}. O módulo sintático foi adaptado para o português europeu no CLUNL (Centro de Linguística da Universidade Nova de Lisboa). As duas versões (PB e PE) estão disponíveis na plataforma, [\url{http://www.fcsh.unl.pt/mabilin/}].} Uma vez identificadas com base em testes de compreensão, tarefas de produção eliciada de estruturas de alto custo complementaram a avaliação. 34 crianças cujo desempenho foi considerado sugestivo de TDL constituíram dois grupos. Destas, um grupo de 21 crianças na faixa de 8 a 12 anos de idade (12 meninos e 9 meninas) foi submetido a um conjunto de atividades voltadas para a apresentação de estruturas de alto custo, sua compreensão e produção, aqui denominado Procedimento Metalinguístico-lúdico de intervenção no domínio da sintaxe (ProMetaS). As demais 13 crianças (10 meninos; 3 meninas) não foram submetidas a qualquer procedimento de intervenção linguística durante o período em que o ProMetaS foi conduzido. Uma reavaliação da compreensão dessas estruturas, por meio do MABILIN, após o ProMetaS, foi tomada como possível indicador (provisório) da efetividade do procedimento utilizado.

O módulo sintático do MABILIN acessa as sentenças de alto custo por meio de uma tarefa de seleção de imagem, estruturando-se em três blocos, contendo 8 instâncias dos 13 tipos de estrutura apresentados (condições de teste), que incluem as sentenças de alto custo acima apresentadas: sentenças passivas (com papéis temáticos reversíveis e irreversíveis), interrogativas QU (\emph{com quem}?\footnote{As interrogativas no PB admitem a estrutura \emph{Quem puxou o tigre/o tigre puxou}?, utilizada no teste, além da estrutura \emph{Quem é que puxou o tigre/o tigre puxou}? preferida no PE.}) e orações relativas (de sujeito e de objeto) ramificadas à direita, ou seja, que modificam o objeto da oração principal (12) e encaixadas no sujeito (13).


\ea\label{ex:14:12} 	Mostra o cachorro que o porco chamou.
\z
\ea\label{ex:14:13} 	O elefante que o tigre lavou dormiu.
\z

O teste é realizado, de maneira lúdica, como um jogo, com imagens na tela de um computador. As crianças devem identificar a imagem que combina com o enunciado proferido oralmente pelo experimentador, dentre três figuras: uma que corresponde ao alvo e duas distratoras. O uso de imagens simples (Figura \ref{fig:cap14fig1}) e complexas (Figura \ref{fig:cap14fig2}), relativas ao evento apresentado (Fig.\ref{fig:cap14fig1}: \emph{Mostra a borboleta que o coelho pintou}; Fig.\ref{fig:cap14fig2}: \emph{Mostra a borboleta que a abelha molhou}) e a apresentação de três possíveis respostas visaram a minimizar o peso de dificuldades provenientes do mapeamento sentença-imagem na avaliação. As respostas das crianças são automaticamente registradas por meio do programa MABILIN, que gera um resultado que indica o nível de dificuldade da criança na realização do teste – leve, moderado ou severo.\footnote{O nível da dificuldade é determinado com base no número de condições em que a média dos acertos ficou 2 desvios-padrão abaixo da média obtida com crianças da mesma faixa etária sem dificuldades de linguagem, tomada como referência.}



\begin{figure}
\subfigure[Imagem simples]{
  \includegraphics[width=.49\linewidth]{figures/44.png}
  \label{fig:cap14fig1}
  }%
  \subfigure[Imagem complexa]{
 \includegraphics[width=.49\linewidth]{figures/45.png}
  \label{fig:cap14fig2}
  }
  \caption{~}
\end{figure}



As crianças assim identificadas foram adicionalmente submetidas a um teste de produção eliciada de passivas, interrogativas QU/+N e relativas, reiterando-se, ou não, a suspeita de comprometimento sintático.\footnote{Também foi aplicado um teste cognitivo não-verbal \citep{Oliveira2000} no qual as crianças apresentaram desempenho compatível com a idade.} Crianças que não demonstraram dificuldades na produção das estruturas em questão não foram indicadas para o ProMetaS.

\subsection{Atividades}
As atividades do ProMetaS constituem um conjunto de três blocos, cada um com três partes, conduzido em nove sessões semanais,\footnote{No caso de ausência da criança na escola ou de evento do calendário escolar que impedisse a atividade, as sessões foram retomadas na sequência prevista após a interrupção (de não mais do que uma semana).} com duração de cerca de 40-50 minutos, aplicadas individualmente em uma sala isolada da escola da criança. 

As atividades concentraram-se em dois tipos de estruturas de alto custo computacional, comumente comprometidas no TDL: as orações passivas e as orações relativas, particularmente as de objeto, como nos exemplos (1) e (2), aqui retomados como (14) e (15). 

Vimos que, de um ponto de vista formal, ambas as estruturas compartilham a característica de apresentarem um elemento que aparece (é fonologicamente produzido/percebido) em uma posição (em itálico nos exemplos abaixo), mas que precisa ser associado a outra posição na sentença a fim de ser semanticamente interpretado (marcado com um \longrule ~nos exemplos abaixo).


\ea\label{ex:14:14}	O \emph{menino} foi perseguido \longrule ~pelo cachorro.
\z
\ea\label{ex:14:15}	Eu encontrei a \emph{menina} que o diretor procurava \longrule.
\z

As dificuldades no processamento desse tipo de estrutura, como vimos na Seção 2, afetam o estabelecimento da relação entre essas duas posições.

Neste estudo, não buscamos diferenciar as possíveis causas (i-iii) para as dificuldades apresentadas (ver (i-iii) na secção 2). Consideramos que o tipo de atividade proposto poderia beneficiar quaisquer dos casos previstos, a menos que restrições de ordem maturacional estendidas por toda a infância impedissem a representação ou a recuperação da informação gramaticalmente relevante durante a análise das sentenças. 

As sessões foram conduzidas com suporte de computador portátil e o programa \emph{Power-point}. O material de base para as atividades foi criado em pranchas (slides) com o uso de recursos visuais como animações, destaques em pisca-pisca, dentre outros. O procedimento segue um protocolo para apresentação das pranchas, interação com a criança e leitura de histórias, com suporte de imagens.

Incorporamos o pressuposto de que há vantagens em se explorar uma apreensão de natureza metalinguística das estruturas em questão, como em \citet{Levy2009}. Contudo, optamos por não fazer referência verbal explícita a movimento sintático, e sim fazer uso de recursos visuais (como deslocamento dos constituintes das orações e/ou de imagem correspondente, por meio do recurso de animação do \emph{Power Point}) para que esse movimento fosse percebido. 

O Bloco 1 focaliza orações passivas, o Bloco 2, orações relativas e o Bloco 3, ambas as estruturas. A apresentação é sempre iniciada pelas estruturas mais simples, as quais desempenham, no discurso, uma função semelhante àquela das estruturas a serem especificamente trabalhadas. Assim, para passivas, inicia-se com estruturas de topicalização (deslocamento à esquerda) (16), muito comuns em PB como alternativa ao uso de passivas em certos contextos \citep{Gabriel2001}. Nessas estruturas, um elemento mais à esquerda precisa ser relacionado a outro ponto da sentença, mas não há morfologia de passiva e mantém-se o agente como sujeito da sentença. Introduzem-se, então, passivas curtas (17), menos custosas do que passivas longas, com o agente da passiva (18). Passivas irreversíveis (18) também são trabalhadas antes de passivas reversíveis (19), visto que, nas primeiras, não há possibilidade de outro elemento diferente do sujeito ter o papel de tema/afetado. Verbos de ação também são apresentados antes de verbos psicológicos (20), diante da dificuldade já evidenciada por estes últimos (para aquisição de passivas, ver \citet{Correa2017}).    


\ea\label{ex:14:16}	O casaco, o Pedro largou \longrule ~no sofá.
\z
\ea\label{ex:14:17} 	A televisão foi consertada \longrule.
\z
\ea\label{ex:14:18} 	A televisão foi consertada \longrule ~pelo pai do Rafael.
\z
\ea\label{ex:14:19} 	O elefante foi molhado \longrule ~pelo macaco.
\z
\ea\label{ex:14:20}	Sansão é respeitado \longrule ~por Raul.
\z

O Bloco 2, para a apresentação de orações relativas, inicia-se por modificadores nominais mais simples, como adjetivos ou PPs (21), a fim de se enfatizar a função de modificador restritivo, que será também desempenhada por orações relativas. A apresentação de relativas de sujeito (22) precede a de relativas de objeto direto (23) e todas as demais. Relativas de objeto indireto/oblíquas são apresentadas na forma padrão (24a). No entanto, relativas cortadoras (24b) ou com pronome resumptivo (24c) são plenamente aceitas na produção das crianças falantes de PB. 


\ea\label{ex:14:21}	O menino de camisa listrada chegou.
\z
\ea\label{ex:14:22} 	Este é o rato que comeu o trigo.
\z
\ea\label{ex:14:23} 	Esta é a casa que o Pedro fez.
\z
\ea\label{ex:14:24}
    \ea Esta é a cliente para quem a funcionária ligou.
    \ex Esta é a cliente que a funcionária ligou.
    \ex Esta é a cliente que a funcionária ligou pra ela.
    \z
\z

A maioria das sessões começa com a narração oral de uma história, na qual há várias instâncias das estruturas a serem trabalhadas. Essas orações são então a\-presentadas em pranchas subsequentes à narração da história, em que certos aspectos são salientados graficamente, em atividades de sensibilização para uma maior consciência metalinguística (\emph{cf}. Fig. \ref{fig:cap14fig3} e \ref{fig:cap14fig4}).


\begin{Figura}
  \subfigure[Atividade metalinguística (passiva)]{
  \includegraphics[width=.49\linewidth]{figures/46.png}
  \label{fig:cap14fig3}
  }%
   \subfigure[Atividade metalinguística   (relativa)]{
  \includegraphics[width=.49\linewidth]{figures/47.png}
  \label{fig:cap14fig4}
  }

{Obs: A linha pontilhada indica movimento feito com recurso de animação }
\caption{~}
\end{Figura}



\hspace*{-1.4pt}As atividades subsequentes são sempre contextualizadas ou apresentadas como um jogo. Estas incluem tarefas de formação de conjuntos e subconjuntos, com vista a sinalizar a necessidade de se explicitar, de forma precisa, informação referente ao agente da passiva ou a um particular elemento caracterizado por uma oração relativa.  Na atividade ilustrada na Figura \ref{fig:cap14fig5}, faz-se uso da formação de conjunto como forma de enfatizar a necessidade de explicitar o agente da passiva nesse contexto. Clica-se em cada objeto, pergunta-se a criança se este pode ser consertado por um eletricista ou por uma costureira e desloca-se o objeto para o diagrama em questão. Ao longo da atividade a criança passa a produzir o agente da passiva diante do objeto a ser movido para um dos conjuntos. Tarefa semelhante é utilizada para a compreensão de estruturas relativas. Na Figura \ref{fig:cap14fig6}, as instruções eram: Vamos montar conjuntos: (i) o conjunto dos bichos que bebem leite; (ii) o conjunto dos bichos que voam; (iii) e vamos dar uma estrela para o bicho que pertence aos dois conjuntos:


\begin{Figura}
\includegraphics[width=1\linewidth]{figures/48.png}
 \caption{Formação de conjuntos com passivas}
 \label{fig:cap14fig5}
 \end{Figura}
 
 \begin{Figura}
   \includegraphics[width=.5\linewidth]{figures/49.png}
\caption{Subconjuntos e relativas}
\label{fig:cap14fig6}  
 \end{Figura}



Outras tarefas requerem da criança atenção (i) à informação a ser codificada na estrutura em questão; (ii) à informação a ser mapeada na imagem apresentada; (iii) ao valor verdade do enunciado apresentado.

Na Figura \ref{fig:cap14fig7}, por exemplo, diz-se para a criança que algo acontece (\emph{Olha, um copo foi quebrado!}; \emph{Olha, um balão vai ser estourado!}) e ela tem de descobrir quem foi o culpado. A criança é então instruída a procurar pistas (um caco de vidro, perto do gato; uma agulha, na mão da cozinheira). A ideia é que, ao perceber a pista, a criança produza ou complete a sentença com o agente da passiva: \emph{O copo foi quebrado pelo gato}; \emph{O balão vai ser estourado pela cozinheira}. 



\begin{Figura}
    \includegraphics[width=1\linewidth]{figures/50.png}
    \caption{{Atividade de elicitação de estrutura passiva com identificação do agente da passiva}}
    \label{fig:cap14fig7}
\end{Figura}



Diante da Figura \ref{fig:cap14fig8}, a criança tem de estar atenta à sequência dos elementos apresentados por meio de uma oração relativa de objeto. 

\begin{Figura}
    \includegraphics[width=1\linewidth]{figures/51.png}
    \caption{{Atividade de compreensão de estrutura relativa}}
    \label{fig:cap14fig8}
\end{Figura}



\newpage
A Figura \ref{fig:cap14fig9} ilustra o \emph{Jogo do certo ou errado}.\footnote{Frases utilizadas na imagem: Sentenças verdadeiras: A menina de blusa amarela pode ser derrubada pelo balanço; A bola vai ser chutada pelo menino de camiseta vermelha. Sentenças falsas: Um menino está sendo carregado pelo palhaço; O chinelo está sendo calçado pela menina de rosa; O carrinho do bebê está sendo empurrado por um menino.} Apresenta-se a criança uma sentença afirmativa e ela deve julgá-la como verdadeira ou falsa diante da cena do parque. Ao reconhecer uma afirmativa falsa, a criança deve apresentar a verdadeira, fazendo uso da voz passiva.



\begin{Figura}
    \includegraphics[width=1\linewidth]{figures/52.png}
    \caption{Jogo do certo ou errado (sentenças verdadeiras \Checkmark, sentenças falsas *) .}
    \label{fig:cap14fig9}
\end{Figura}



As atividades foram bem recebidas pelas crianças e há relatos das professoras indicando que, depois do ProMetaS, as crianças se mostraram mais receptivas e criativas em atividades da escola. 


\section{Possível efeito do ProMetaS}
As 21 crianças submetidas ao ProMetaS e as 13 crianças do grupo controle foram retestadas por meio do MABILIN cerca de três meses após a primeira avaliação, o que corresponde aproximadamente ao período de intervenção, para as primeiras, seguido de um mês de intervalo. 

Apresentamos, a seguir, os resultados obtidos na primeira e na segunda testagem, em cada grupo, a partir da aplicação de um teste t-student, em que a soma dos escores obtidos nas 13 condições de teste do MABILIN foi comparada em função do momento da avaliação, o que inclui interrogativas com \emph{quem} e com \emph{que} +N, não trabalhadas explicitamente no procedimento utilizado. As Tabelas \ref{tab:cap14tab1} e \ref{tab:cap14tab2} apresentam os resultados individuais do grupo submetido ao ProMetaS e do grupo controle.



\begin{Tabela}[H]
\caption{{Total de acertos no MABILIN em duas testagens (pré e pós ProMetaS)}}
\label{tab:cap14tab1}


\begin{tabular}{lcrrS}
\lsptoprule
&& \textbf{Total de acertos -} & \textbf{Total de acertos -} & \textbf{Valor de p*}\\
\textbf{Crianças} & \textbf{sexo} & \textbf{primeira testagem} & \textbf{segunda testagem}& \textbf{significativo}\\
\lsptoprule
AD & F&	34&	46&0,02*\\
\addlinespace
AR&M&	34&	49&0,03*\\
\addlinespace
BR & M&	50	&55	&0,02*\\
\addlinespace
CA & M&	44&	55&0,02*\\
\addlinespace
CE & M&	46&	54&0,02*\\
\addlinespace
DA & M&	48&	52&0,14\\
\addlinespace
ED & F& 	44&	53&0,01*\\
\addlinespace
EM & F&	26&	38&0,001*\\
\addlinespace
FRA& M&	27&	51&0,0001*\\
\addlinespace
GAB & F& 	24&	48&0,002*\\
\addlinespace
JOP & M&	35&	32&0,277\\
\addlinespace
KEV&M&	33	&47&0,005*\\
\addlinespace
LI & F&	34&	50&0,001*\\
\addlinespace
LOR & F&	37&	51& 0,05*\\
\addlinespace
MA & M&	47&	55&0,01*\\
\addlinespace
RA & M&	41&	56&0,0001*\\
\addlinespace
RAY&M&	41&	49&0,01*\\
\addlinespace
RY & M&	41&	46&0,2\\
\addlinespace
TAI & F&	39&	40&0,4\\
\addlinespace
TAT & F&	50&	55&0,7\\
\addlinespace
YAS & F&	27&	36&0,05*\\
\addlinespace
\lspbottomrule
\end{tabular}
\end{Tabela}


\begin{Tabela}[H]
\caption{{ Total de acertos no MABILIN em duas testagens (grupo controle).}}
\label{tab:cap14tab2}


\begin{tabular}{lcrrS}
\lsptoprule
  & & \textbf{Total de acertos -} & \textbf{Total de acertos -} & \textbf{Valor de p*} \\
\textbf{Crianças} &\textbf{sexo} & \textbf{primeira testagem} & \textbf{segunda testagem} & \textbf{significativo}\\
\addlinespace
\midrule
CAI & M&	48&	53&	0.11\\
\addlinespace
CAM & F&	53&	50&	0.25\\
\addlinespace
CAU & M&	53&	52&	0,37\\
\addlinespace
CIC & M&	42&	48&	 0,06\\
\addlinespace
ESP & F&	44&	50&	0,1\\
\addlinespace
GUS & M&	42&	51&	  0,03*\\
\addlinespace
IAG & M&	46	&49	&0.21\\
\addlinespace
JV & M&	47&	49	&0,23\\
\addlinespace
MAU & M&	48&	47&	0,44\\
\addlinespace
NIC & M&	41&	40&	0,42\\
\addlinespace
PAB & M&	49	&47&	0,18\\
\addlinespace
RAY & M&	53&	54&	0,31\\
\addlinespace
SAB & F&	54&	52&	0,18\\
\addlinespace
\lspbottomrule
\end{tabular}
\end{Tabela}


Como pode ser observado, apenas 5 crianças do grupo ProMetaS não apresentaram diferença significativa entre os dois momentos da aplicação do MABILIN, ou seja, as demais 16 crianças, 76,2\% do total de participantes, apresentaram melhor desempenho na segunda testagem. Por outro lado, no grupo controle, apenas uma apresentou diferença significativa entre o desempenho na primeira e na segunda aplicação do MABILIN. A comparação entre grupos (Médias: 38,19 (ProMetaS); 48,38 (controle) t(19) = p<.0001) também demonstra que,  em termos gerais, os resultados dos grupos diferem significativamente em função do intervalo entre a primeira e a segunda avaliação. 

Nova retestagem seria requerida com intervalo de tempo maior, após a intervenção. No entanto, não foi possível manter o contato com a maior parte das crianças. De todo modo, os resultados na reavaliação conduzida mostraram-se promissores.

\section{Conclusão}
Consideramos que o tipo de conscientização metalinguística despertado e o tipo de atividades sugeridas podem servir de estímulo para professores e terapeutas da fala/fonoaudiólogos replicarem esse tipo de procedimento e criarem materiais semelhantes. 

Como dissemos anteriormente, nova retestagem seria necessária para uma avaliação da permanência do efeito do procedimento a mais longo prazo. Uma investigação voltada para o efeito de cada recurso utilizado (imagens em destaque, ou em movimento; uso de formação de conjuntos, por exemplo) seria bem-vinda, de modo a distinguirmos quais destes seriam mais efetivos. Um treinamento de profissionais de saúde ou professores diante dos protocolos criados também seria ideal, para garantir que a dinâmica da interação aplicador(a)/criança corresponda aos objetivos de trazer consciência metalinguística de forma indireta, lúdica e funcional. 

Ainda que tais aprofundamentos ou desdobramentos sejam requeridos, antes de podermos constatar a eficácia do procedimento criado, acreditamos que a criança poderá se beneficiar de atividades como as sugeridas, independentemente de suas dificuldades estarem no reconhecimento ou no acesso aos traços formais dos elementos do léxico que possibilitam a computação de relações de dependência de longa distância, ou na manutenção da informação linguisticamente relevante na memória de trabalho para o estabelecimento dessas relações.  


{\sloppy\printbibliography[heading=subbibliography,notkeyword=this]}
\end{document}
