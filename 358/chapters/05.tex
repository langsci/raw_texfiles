\documentclass[output=paper,colorlinks,citecolor=brown,booklanguage=portuguese]{langscibook} 
\ChapterDOI{10.5281/zenodo.7233221}
\title{O segmento: Teoria fonológica e intervenção clínica}
\author{Helena Bolli Mota\affiliation{Universidade Federal de Santa Maria}  and  Fernanda Marafiga Wiethan\affiliation{Universidade Federal de Santa Maria}   and   Vanessa Giacchini \affiliation{Universidade Federal do Rio Grande do Norte}}

\abstract{Neste capítulo apresentaremos os principais aspectos relacionados com a intervenção dos transtornos fonológicos uma vez que estes são beneficiados por abordagens terapêuticas que têm por base princípios linguísticos relacionando algumas das principais teorias fonológicas com os modelos de terapia que delas se originaram. Mostraremos os diferentes enfoques de tratamento com base em diferentes teorias fonológicas que foram propostas para se chegar à reorganização fonológica de uma forma mais rápida e eficiente. Essas teorias serviram de base para a proposição de diferentes abordagens terapêuticas para o tratamento dos transtornos dos sons da fala de origem fonológica, as quais iremos apresentar no capítulo como o Modelo de Ciclos e de Ciclos Modificado, o Modelo de Pares Mínimos/Oposições Máximas/\emph{Empty Set}, o Modelo de Oposições Múltiplas, o Modelo ABAB-Retirada e Provas Múltiplas e o Modelo Metaphon.


\textbf{Palavras-chave:} fonologia; segmentos; transtorno fonológico.}

\IfFileExists{../localcommands.tex}{
   \addbibresource{../localbibliography.bib}
   \usepackage{langsci-optional}
\usepackage{langsci-gb4e}
\usepackage{langsci-lgr}

\usepackage{listings}
\lstset{basicstyle=\ttfamily,tabsize=2,breaklines=true}

%added by author
% \usepackage{tipa}
\usepackage{multirow}
\graphicspath{{figures/}}
\usepackage{langsci-branding}

   
\newcommand{\sent}{\enumsentence}
\newcommand{\sents}{\eenumsentence}
\let\citeasnoun\citet

\renewcommand{\lsCoverTitleFont}[1]{\sffamily\addfontfeatures{Scale=MatchUppercase}\fontsize{44pt}{16mm}\selectfont #1}
  
   %% hyphenation points for line breaks
%% Normally, automatic hyphenation in LaTeX is very good
%% If a word is mis-hyphenated, add it to this file
%%
%% add information to TeX file before \begin{document} with:
%% %% hyphenation points for line breaks
%% Normally, automatic hyphenation in LaTeX is very good
%% If a word is mis-hyphenated, add it to this file
%%
%% add information to TeX file before \begin{document} with:
%% %% hyphenation points for line breaks
%% Normally, automatic hyphenation in LaTeX is very good
%% If a word is mis-hyphenated, add it to this file
%%
%% add information to TeX file before \begin{document} with:
%% \include{localhyphenation}
\hyphenation{
affri-ca-te
affri-ca-tes
an-no-tated
com-ple-ments
com-po-si-tio-na-li-ty
non-com-po-si-tio-na-li-ty
Gon-zá-lez
out-side
Ri-chárd
se-man-tics
STREU-SLE
Tie-de-mann
}
\hyphenation{
affri-ca-te
affri-ca-tes
an-no-tated
com-ple-ments
com-po-si-tio-na-li-ty
non-com-po-si-tio-na-li-ty
Gon-zá-lez
out-side
Ri-chárd
se-man-tics
STREU-SLE
Tie-de-mann
}
\hyphenation{
affri-ca-te
affri-ca-tes
an-no-tated
com-ple-ments
com-po-si-tio-na-li-ty
non-com-po-si-tio-na-li-ty
Gon-zá-lez
out-side
Ri-chárd
se-man-tics
STREU-SLE
Tie-de-mann
}
   \boolfalse{bookcompile}
   \togglepaper[05]%%chapternumber
}{}

\begin{document}
\maketitle

\section{Introdução}\label{sec:cap5sec1}

Os transtornos dos sons da fala (TSF) são as alterações ou distúrbios de comunicação mais comuns na infância, com prevalência de 2,3\% a 24,6\% \citep{Wren2016}.\footnote{A norma adotada na escrita deste capítulo foi a do português brasileiro.}
% \footnote{Na sequência de restrições decorrentes do template usado para publicação na editora Language Science Press, os capítulos que integram o volume surgem com a referência [this volume], em inglês.}

Enquanto a maioria das crianças adquire a fala típica por volta dos 5 anos, outras com essa idade, continuam apresentando erros em sua fala, tais como omissões, substituições e distorções de alguns sons. Essas desordens em geral persistem e necessitam de tratamento para que sejam resolvidas.

Dada a alta prevalência dessas alterações, não é de se surpreender que as crianças com TSF compreendam uma grande proporção de pacientes para os fonoaudiólogos no mundo inteiro.

Embora não haja um sistema universalmente aceito para se classificar as diferentes apresentações dos TSF, existe uma concordância de que as crianças que apresentam essas desordens formam um grupo heterogêneo \citep{Waring2013}.

Para algumas crianças as causas das alterações de fala são conhecidas, por exemplo, quando estão associadas a uma perda auditiva, a uma fissura palatina ou a uma síndrome genética. No entanto, para a maioria das crianças essas causas são desconhecidas \citep{Shriberg2010}. Nesses casos, considera-se a origem da desordem ou uma dificuldade cognitivo-linguística para a aquisição do sistema fonológico da língua ambiente ou uma dificuldade para adquirir o correto plano motor para um som ou sequência de sons particular \citep{Waring2013}.

As diferentes causas dão origem a diferentes subtipos ou combinações de diferentes subtipos de TSF, tais como atraso ou transtorno fonológico, distúrbio da articulação ou apraxia de fala na infância (AFI).

Para que se possa chegar a um diagnóstico correto e determinar de forma precisa o tipo de alteração de fala que se apresenta, uma boa avaliação é fundamental, o que, por sua vez, conduzirá a uma intervenção mais adequada para cada uma das desordens.

Neste capítulo abordaremos o tratamento dos TSF de origem cognitivo-linguís\-ti\-ca, também chamados de transtornos fonológicos.

O transtorno fonológico é caracterizado por alterações que ocorrem na fala da criança, em que esta realizará uma produção inadequada dos fonemas, bem como o uso inadequado das regras fonológicas da língua \citep{Wertzner2002}. Esse transtorno linguístico é portado por crianças que apresentam alterações na produção da fala, na ausência de determinados fatores etiológicos como: dificuldade geral de aprendizagem, déficit intelectual, desordem neuromotora, transtornos psiquiátricos, problemas otológicos ou fatores ambientais \citep{Mota1996}. 

O objetivo do capítulo é apresentar os princípios gerais da terapia com base fonológica e as principais teorias fonológicas que deram origem a diferentes abordagens terapêuticas citadas na literatura, usadas no tratamento dos transtornos fonológicos.  

Intervenção fonológica é um termo geral para expressar as diferentes maneiras de se tratar as alterações do nível fonológico da linguagem. Há vários enfoques possíveis para se fazer uma intervenção fonológica, no entanto, existem alguns princípios gerais que são compartilhados por esses diferentes enfoques de tratamento. O principal deles é que existem regularidades na linguagem falada, isto é, os padrões de pronúncia são regidos por regras e são previsíveis. A partir desse princípio básico é que derivam todos os outros que irão nortear os procedimentos de análise (cf. \citetv{chapters/04}, \citetv{chapters/06}) e de terapia fonológica. 

Um comportamento orientado por regras implica organização, portanto uma avaliação e análise fonológicas revelam a organização subjacente aos padrões fonológicos, bem como transtornos e inadequações nesta organização. Assim, qualquer intervenção fonológica deve basear-se sempre em uma avaliação e análise fonológicas, que definirão os objetivos terapêuticos (cf. \citetv{chapters/04}, \citetv{chapters/06}).

Um tratamento fonológico deve ser planejado para mudar os aspectos deficientes da organização dos padrões de sons da criança e construir um sistema de contrastes mais adequado, da maneira mais efetiva possível. 

Os princípios descritos anteriormente evidenciam não só a necessidade de se fazer uma avaliação e uma análise fonológica, mas também a necessidade de se conhecer bem a natureza do desenvolvimento fonológico tanto típico quanto com transtornos, a fim de que se possa ter uma maior compreensão do mecanismo de aquisição fonológica.

Para \citet{StoelGammon1985}, além dos princípios acima citados, uma intervenção fonológica também se caracteriza pelos seguintes aspectos:

\begin{itemize}
\item[(i)] É marcada por atividades conceituais e não apenas motoras: a natureza conceitual da terapia fonológica está relacionada com as unidades linguísticas que são a base do tratamento – contrastes, traços distintivos, processos fonológicos. Embora essas unidades sejam conceitos abstratos não observados diretamente, os padrões superficiais levam a supor que a criança não tem conhecimento subjacente de que os sons introduzem contrastes de significado.

\item[(ii)] Tem como meta a generalização: em uma terapia fonológica bem sucedida deve ocorrer generalização, que é a ampliação da produção e uso correto de alvos treinados em outros contextos ou ambientes não treinados, sendo, portanto, o critério mais importante para se medir a eficácia terapêutica \citep{Elbert1986}.
\end{itemize}

Existem dois enfoques diferentes para o estudo da generalização em crianças com transtornos fonológicos. Um deles procura identificar as propriedades estruturais da generalização ou as circunstâncias sob as quais ela ocorre. O outro enfoque examina as propriedades funcionais da generalização ou como esta é usada por uma criança para modificar seu sistema fonológico.

Uma intervenção fonológica bem sucedida deve promover diferentes tipos de generalização, sendo os mais importantes:

\sloppy
\begin{description}
\item[Generalização a itens (palavras) não utilizados no tratamento:] Nesse tipo de generalização, as crianças produzem os sons-alvo não apenas nas palavras em que foram treinados, mas também em outras palavras não analisadas. Esse tipo de generalização indica que são necessárias poucas palavras para se promover a aquisição de um determinado som.

\item[Generalização para outra posição na palavra:] quando, ao adquirir um fonema numa determinada posição da palavra, a criança o realiza corretamente em outras posições não trabalhadas em terapia. Por exemplo, ao adquirir o [s] em \emph{onset} medial, a criança o realiza em posição de coda.

\item[Generalização para outras unidades linguísticas:] é o uso dos sons tratados em unidades linguísticas mais complexas. Geralmente, o tratamento inicia com a estimulação do som-alvo em palavras e, após, a produção é ampliada para outros níveis linguísticos como a sentença e a conversação.

\item[Generalização dentro de uma classe de sons:] é um tipo de generalização que ocorre quando a criança, ao adquirir um fonema, estende essa produção correta a outros fonemas pertencentes à mesma classe do som trabalhado. Por exemplo, a criança adquire a fricativa [f] e passa a produzir as demais fricativas corretamente sem tratamento específico para cada uma delas. Esse tipo de generalização indica que sons que se relacionam entre si podem ser adquiridos sem intervenção direta em todos eles. Assim, tratando-se apenas um som incorreto ou um aspecto de um padrão de erro, vários outros sons incorretos podem ser corrigidos, trazendo, como resultado, um tratamento mais eficiente.

\item[Generalização para outras classes de sons:] esse é um tipo de generalização mais complexo, que ocorre quando a criança, ao adquirir um som, estende essa produção correta a outros fonemas que não pertencem à mesma classe do som trabalhado. Por exemplo, a estimulação da fricativa [z] pode resultar na produção correta da líquida [l]. É um tipo de generalização bastante desejado na intervenção fonológica, por reduzir o tempo de terapia.

\item[Generalização para outras situações ou contextos:] é o uso de sons tratados em novas situações, novos contextos ou com novos ouvintes. O ensino da produção correta dos sons geralmente ocorre em um ambiente de clínica, guiado por um terapeuta, mas a criança deve ser capaz de usar corretamente esses sons em outras situações fora da clínica, por exemplo, em sua casa ou na escola. As produções corretas também devem acontecer com outros ouvintes que não apenas o terapeuta.

\item[Generalização baseada em relações implicacionais:] um outro tipo de generalização observada dentro de uma mesma classe de sons e também para outras baseia-se nas relações implicacionais existentes entre os sons. Relações implicacionais referem-se a um fenômeno linguístico pelo qual a ocorrência de um determinado som ou classe de sons, em uma língua, implica a ocorrência de outro som ou classe de sons naquela mesma língua. Por exemplo, se uma língua tiver fricativas, então ela terá também plosivas; se uma língua tiver obstruintes sonoras, então ela terá também obstruintes surdas.
\end{description}
\fussy

\citet{Gierut1992} refere que, sendo a generalização o objetivo de um tratamento efetivo para desordens fonológicas, a chave para que ela ocorra pode estar na seleção inicial dos sons-alvo para o tratamento.

\citet{Mota2001} salienta que existem fatores como o nível de inteligência, aspectos motivacionais e variáveis intra-sujeito que podem interferir na generalização e que são relevantes em qualquer procedimento terapêutico. Esse é o enfoque funcional da generalização e busca-se, na sua análise, a identificação das variáveis que são próprias de cada sujeito. 

\section{Abordagens terapêuticas}\label{sec:sec2cap5}

A década de 1980 representou um marco para o tratamento dos transtornos fonológicos a partir da introdução das teorias linguísticas para o embasamento das diferentes abordagens terapêuticas. Até então, o enfoque terapêutico para quaisquer alterações de fala era baseado em aspectos fonéticos, ou seja, a ênfase era dada ao treino articulatório.

Na abordagem articulatória, os segmentos eram tratados isoladamente, depois em sílabas, palavras e frases, finalizando com a fala espontânea. Esse tipo de abordagem não é adequado ao tratamento dos transtornos fonológicos por não considerar os aspectos organizacionais da língua.

Assim, diferentes enfoques de tratamento com base em diferentes teorias fonológicas foram propostos para se chegar à reorganização fonológica de uma forma mais rápida e eficiente. 

Dentre as teorias fonológicas que inspiraram a criação de modelos de tratamento para os transtornos fonológicos, pode-se citar como as mais importantes: a fonologia natural, a fonologia gerativa, a teoria dos traços distintivos, a teoria autossegmental, a geometria de traços, a teoria da marcação e das leis implicacionais, a teoria da sílaba (cf. \citetv{chapters/04}, \citetv{chapters/06}).


Essas teorias serviram de base para a proposição de diferentes abordagens terapêuticas para o tratamento dos TSF de origem fonológica, dentre as quais podemos salientar o Modelo de Ciclos e de Ciclos Modificado, o Modelo de Pares Mínimos/Oposições Máximas/\emph{Empty Set}, o Modelo de Oposições Múltiplas, o Modelo ABAB-Retirada e Provas Múltiplas e o Modelo Metaphon.

\subsection{Os processos fonológicos na terapia de transtornos fonológicos -- a Teoria da Fonologia Natural}\label{sec:cap5sec2ponto1}
\largerpage

A fonologia natural é uma proposta teórica apresentada por \citeauthor{Stampe1973}, em sua tese de Doutorado defendida em \citeyear{Stampe1973}. A teoria ficou mais conhecida com as publicações seguintes: \citet{Stampe1979a} e \citet{Stampe1979}. A tese básica é que os padrões sonoros das línguas, durante o desenvolvimento em cada pessoa bem como em sua evolução ao longo dos séculos, são governados por forças implícitas na produção e percepção da fala humanas. Essas forças fonéticas implícitas são manifestadas por meio de processos, ou seja, substituições sistemáticas, porém não conscientes, que adaptam nossas intenções fonológicas às nossas capacidades fonéticas, e as quais nos possibilitam perceber nas falas alheias a intenção de fala subjacente a essas adaptações fonéticas superficiais. O sistema fonológico específico de nossa língua nativa é o resíduo de um sistema universal de processos que reflete as limitações fonéticas da criança. Na infância, esses processos oferecem pronúncias provisórias que permitem a comunicação com pais, irmãos e outros destinatários próximos, enquanto não obtém a maturação de pronúncia da língua \citep{Stampe1979}.

O ponto chave para compreender o modelo terapêutico proposto a seguir, é a noção dos processos fonológicos presentes na fala da criança. Para mais informações a respeito da fonologia natural, consultar \textcitetv{chapters/04}.

É chamada de teoria natural porque apresenta a fonologia da língua como um reflexo natural das necessidades, capacidades e ambiente de seus usuários, ao invés do que os autores chamam de “uma mera instituição convencional”, como uma crítica ao modelo de Chomsky e Halle. Também é uma teoria natural no sentido de que almeja explicar sua matéria prima, a fim de mostrar que ela segue naturalmente a partir da natureza das coisas \citep{Stampe1979}. Os processos fonológicos são naturais porque derivam das necessidades e dificuldades articulatórias e perceptuais do ser humano e são inatos porque são limitações com as quais a criança nasce e precisa superar por não fazerem parte de sua língua materna \citep{Yavas2001}.

Como já mencionado, a ideia central da teoria é a presença de processos fonológicos na fala da criança, definidos por \citet{Stampe1979a} como uma operação mental para substituir uma classe ou sequência de sons que apresentam uma dificuldade específica comum para a capacidade de fala do indivíduo, uma classe alternativa idêntica em todos os outros sentidos, porém desprovida da propriedade difícil. Os processos fonológicos aparecem na fala infantil visando facilitar aspectos que sejam difíceis para a criança em termos articulatórios, motores ou de planejamento. Eles são universais, isto é, encontrados em todas as crianças, que iniciam sua fala com todos os processos em operação. À medida que a criança entra em contato com a língua, os processos precisam ser superados, revisados ou limitados \citep{Yavas2001}.

Na área clínica, a aplicação da fonologia natural facilita a comparação entre padrões típicos e atípicos, possibilitando que o terapeuta estabeleça com segurança as prioridades e estratégias para o tratamento. Por isso é importante descrever quais são os processos mais comuns encontrados no desenvolvimento \citep{Yavas2001}. 

Os processos fonológicos podem ser classificados em processos de estrutura silábica, processos de substituição e processos de assimilação.

Os processos de estrutura silábica envolvem a simplificação de sílabas mais complexas para outras mais simples, por exemplo, o processo de redução de encontro consonantal, em que ocorre o apagamento de um dos elementos do \emph{onset} complexo, geralmente a líquida. 

\begin{enumerate}
    \item [(1)]	
    \begin{enumerate}
    \item[a.]   placa $\to$ [ˈpaka]
\item[b.]	abraço $\to$ [aˈbasu]
\end{enumerate}
\end{enumerate}

\largerpage
Nos processos de substituição, um fonema mais complexo é substituído por outro, em geral, mais fácil de ser realizado, como por exemplo, o processo de dessonorização de obstruinte em que plosivas, fricativas ou africadas sonoras são produzidas como surdas.

\begin{enumerate}
    \item [(2)]
    \begin{enumerate}
    \item[a.] dedo $\to$ [ˈtetu]
 \item[b.] vovô $\to$ [foˈfo]
 \item[c.] diz $\to$ [ˈtʃis]
\end{enumerate}
\end{enumerate}

\newpage
Os processos de assimilação fazem com que um fonema assimile características de outro fonema da palavra, de modo que se tornem parecidos, facilitando a produção da mesma. Um exemplo, seria o processo de assimilação da coda nasal.

\begin{enumerate}
    \item [(3)]	brinca $\to$ [ˈmĩⁿkɑ]
\end{enumerate}

É importante destacar que há uma grande variedade de processos fonológicos observados na prática clínica e descritos na literatura. Além disso, é bastante comum a coocorrência de processos, ou seja, mais de um processo fonológico se apresentando em uma única palavra, conforme pôde ser observado em alguns dos exemplos oferecidos. Outro fator relevante é que a maioria deles está presente tanto no desenvolvimento típico quanto atípico. Assim, é fundamental conhecer a ordem de aquisição dos fonemas na região em que a criança reside para assim determinar se o processo fonológico encontrado é parte do desenvolvimento típico ou já se configura como transtorno fonológico, sendo necessária terapia fonológica.

Dessa forma, apresentam-se a seguir os modelos de terapia fonológica baseados em processos fonológicos.	

\subsubsection{Modelos de Ciclos e Ciclos Modificado}\label{sec:cap5secdoisumum}

O Modelo de Ciclos foi proposto por \citet{Hodson1983} e tem como procedimentos básicos a estimulação e a produção. O processo de estimulação engloba a audição, o tato e a visão. A estimulação auditiva é realizada por meio do bombardeio auditivo, que consiste em fazer a criança ouvir uma lista de palavras que contém o som-alvo. A recomendação dos autores é que haja amplificação dos sons ouvidos, já que crianças com transtornos fonológicos podem ter dificuldades em processar os sons-alvo e estabelecer os contrastes.

A estimulação tátil e visual também faz parte do modelo, como auxílio ao desenvolvimento da consciência do fonema trabalhado. Pistas táteis são utilizadas quando novos sons são apresentados por intermédio da sensação do tato, sobretudo vibração das pregas vocais e gestos, e vão sendo retiradas à medida que a produção vá se tornando mais fácil. As pistas visuais podem ser trabalhadas com o uso do espelho, o modelo da boca do terapeuta, imagens da posição de lábios e língua na produção de cada fonema e, mais atualmente, vídeos e programas que mostrem os pontos articulatórios na produção de cada fonema, por meio do celular, tablet ou computador.

O processo de produção é também enfatizado, pois, à medida que a criança desenvolve a consciência de um novo padrão de som, necessita exercitar a sua produção, a fim de internalizar as imagens sinestésicas e auditivas e criar um sistema de autocontrole. O Modelo de Ciclos concentra-se, principalmente, em maneiras de facilitar as produções corretas da criança. 

O Modelo de Ciclos Modificado \citep{Tyler1987} baseia-se na abordagem de \citet{Hodson1983}, porém, apresenta-se de modo mais estruturado do que a proposta original, uma vez que cada ciclo tem a duração de três semanas e, em cada uma delas, estimula-se um processo fonológico. As sessões ocorrem duas vezes por semana e têm duração de 50 minutos.

Trabalha-se um som por sessão, passando-se para outro som na sessão seguinte, desde que a criança obtenha mais de 20\% de correção na produção do fonema nas palavras estímulo daquela sessão. Se não houver correção superior a 20\%, o fonema em questão deve ser trabalhado por mais uma sessão e complementado com posterior prática em casa.

Ao final do ciclo, realiza-se uma sondagem da generalização para verificar o aproveitamento da criança. Na sondagem, cada som que foi estimulado é testado em seis palavras diferentes das utilizadas em terapia para a prática de produção. As palavras da sondagem devem ser apresentadas à criança sob a forma de figuras e esta deve nomeá-las espontaneamente, sem o modelo do terapeuta.

Se para cada um dos sons-alvo testado na sondagem a criança obtiver 50\% de respostas corretas ou mais, passa-se para o segundo ciclo, com os mesmos processos, sons e palavras-alvo, porém com a produção em sentenças. Se, no entanto, a sondagem de um som indicar menos de 50\% de produções corretas, repete-se o som em palavras isoladas. No final do segundo ciclo, havendo novamente menos de 50\% de produções corretas para cada som-alvo, este poderá ser substituído por outros sons que necessitem tratamento ou continuar sendo estimulado no âmbito da palavra até que atinja o percentual mínimo de 50\% de correção na sondagem.

As sessões também incluem uma forma modificada do bombardeio auditivo de \citet{Hodson1983}. A lista de aproximadamente 15 palavras contendo o som-alvo é lida para a criança sem amplificação no início e no final da sessão. A ênfase principal também é a de que a criança enuncie o maior número possível de produções corretas dos sons estimulados em cinco a dez palavras-estímulo. Recomenda-se a realização de jogos e atividades lúdicas contendo as palavras-al\-vo, a fim de motivar o engajamento da criança.

Os pais são orientados a continuar a estimulação das palavras-alvo em casa. Entrega-se ao responsável pela criança a lista de palavras do bombardeio auditivo e as figuras representativas das palavras-estímulo, instruindo-o a praticá-las diariamente até à sessão seguinte.

No segundo ciclo, mantém-se a estrutura das sessões de terapia com os mesmos sons-alvo. Porém, o nível de complexidade da prática de produção passa de palavras isoladas para sentenças (desde que se tenha obtido os 50\% ou mais de correção na sondagem). As palavras-estímulo são trabalhadas em frases, pequenas histórias, descrições ou na fala espontânea, por meio de diversas atividades lúdicas que possibilitem esses tipos de produções.

\subsubsection{Escolha dos processos fonológicos e alvos da terapia}\label{sec:cap5sec2umdois}

Antes de iniciar a terapia, é necessário planejar quais serão os processos fonológicos, sons e palavras-alvo estimulados. \citet{Hodson1983} sugerem um critério objetivo para a escolha dos processos tratados: devem ocorrer em, no mínimo, 40\% das possibilidades dos contextos testados. Para a escolha dos sons-alvo, \citet{Edwards1992} sugere os seguintes critérios:

\begin{itemize}
    \item Escolher sons que façam parte do inventário fonético da criança.
\item	Escolher sons para os quais a criança seja mais estimulável (tenha mais facilidade para adquirir).
\item	Escolher sons que contribuirão para melhorar a inteligibilidade de fala.
\item	Escolher sons que ocorrem com mais frequência na língua.
\item	Escolher sons que são adquiridos mais cedo.
\item	Escolher sons importantes para a criança, como por exemplo sons que contenham em seu nome.
\item	Escolher sons que sejam fáceis de produzir na posição em que estão sendo trabalhados.
\end{itemize}
\subsection{Os traços distintivos na terapia de TSF}\label{sec:cap5sec2dois}

\subsubsection{Fonologia Autossegmental}\label{sec:cap5sec2doisum}
A fonologia autossegmental é uma abordagem não linear que possibilita entender os processos fonológicos que envolve vogais e consoantes a partir de representações fonológicas multidimensionais com arranjos em diversas camadas que são ligadas entre si por linhas de associação. Essa proposta de não linearidade e à organização dos traços dos segmentos surgiu a partir de alterações à Fonologia Gerativa, proposta por \citeauthor{Chomsky1968} em \citeyear{Chomsky1968} \citep{Chomsky1968}. A Fonologia Autossegmental investiga sobre o lado fonético da representação linguística. É uma teoria que busca apresentar como os diferentes componentes do aparato articulatório estão coordenados para a produção dos sons da fala. A Fonologia Autossegmental, serviu de base para a proposta da Geometria de Traços. Para informações mais detalhadas a respeito da fonologia autossegmental, consultar o capítulo de \textcitetv{chapters/04}.

A partir da proposta advinda pela Fonologia Autossegmental - em propor uma hierarquia aos traços e analisar os segmentos em camadas - a descrição da aquisição fonológica e a terapia das alterações de fala foram beneficiadas de maneira significativa. Os modelos de terapia que enfocam a seleção dos alvos em traços distintivos e na contrastividade dos sons ganharam destaque. Propostas como o Modelo de Pares Mínimos (oposição mínima, oposição máxima, oposição múltipla) e \emph{Empty Set} são empregados de maneira satisfatória nos casos de TSF de origem fonológica. A análise da aquisição dos contrastes vem sendo empregada na descrição da aquisição fonológica \citep{LazzarottoVolcao2009}, mas não há estudos da aplicação dessa proposta de análise na terapia.

\subsubsection{Terapia com Base em Pares Mínimos}\label{sec:cap5sec2doisdois}

O Modelo terapêutico de Pares Mínimos busca confrontar palavras que diferem em apenas um fonema -- sendo que esse pode diferir em poucos traços ([ˈ\textbf{ɡ}atu] X [ˈ\textbf{b}atu]) ou em vários traços ([ˈ\textbf{p}ato] X [ˈ\textbf{X}ato]) -- enfocando a função dos fonemas na língua. Quando os segmentos contrastantes de um par mínimo diferem somente em um traço distintivo, há um contraste de oposições mínimas. Por exemplo, as palavras “gato” e “bato” formam um par mínimo com oposições mínimas, pois diferem uma da outra por apenas um contraste de fonemas e estes diferem entre si somente quanto ao traço de ponto de articulação, [g] (oclusiva velar sonora) X [b] (oclusiva bilabial sonora).

Quando os segmentos contrastantes de um par mínimo diferem em vários traços distintivos, há um contraste de oposições máximas. Por exemplo, as palavras “pato” e “rato” formam um par mínimo com oposições máximas. Essas palavras diferem em apenas um som, mas esses sons diferem entre si em quatro traços distintivos – [soante], [vozeado], [contínuo] e quanto ao traço de ponto de articulação [bilabial] [velar]. 

O modelo de pares mínimos é baseado nos traços distintivos, desse modo, defende-se a ideia que, se a criança adquire o traço distintivo ou a combinação de traços que apresentava dificuldade, ela irá estender o conhecimento aos demais fonemas que compartilham desse traço.  

A premissa da terapia baseada nesse modelo é colocar a criança em uma situação de quebra de comunicação, em que a troca de um fonema por outro prejudica o entendimento pelo interlocutor. Nesse modelo busca-se fazer a oposição entre um som conhecido pela criança - que ela possui no seu sistema fonológico -- e um som desconhecido - ausente no seu sistema fonológico -- criando momentos de confusão na comunicação ao adotar o fonema incorreto \citep{Weiner1981}. 

Para a organização da terapia com enfoque em Pares Mínimos, sugere-se o procedimento de Percepção-Produção/Pares Mínimos apresentado por \citet{Tyler1987}. Os autores propõem 5 níveis de treinamento, sendo um direcionado para a percepção e identificação do fonema e os restantes para a produção do segmento ausente (1 nível de Perceção e 4 níveis de Produção). 

O modelo de Pares Mínimos -- Oposições Mínimas ou Máximas é indicado para os casos em que a criança possui poucos processos atuantes na fala.

\subsubsection{Modelo de Oposições Máximas/ Empty Set}\label{sec:cap5sec2doistres}
Para \citet{Gierut1992}, a maneira como as palavras diferem em um par mínimo é fundamental para as mudanças fonológicas que ocorrem durante o tratamento. Além dos traços distintivos, há mais duas dimensões nas quais os fonemas podem diferir: natureza das distinções dos traços e relação dos fonemas contrastantes com a gramática da criança antes do tratamento.

Quanto à natureza das distinções de traços, os fonemas podem diferir por traços de classes principais ou somente por traços de classes não-principais. Os traços de classes principais distinguem as grandes classes de sons. Eles envolvem o nó de raiz dos fonemas (traços [soante], [vocóide] e [aproximante]). As distinções de classes não-principais referem-se a todas as outras diferenças de traços de ponto, modo e sonoridade entre os fonemas (para informações mais detalhadas consulte-se o capítulo de \citetv{chapters/04}).

Quanto à relação dos fonemas contrastantes com a gramática da criança antes do tratamento, os fonemas podem ser conhecidos ou desconhecidos. Ou seja, pode-se trabalhar dois fonemas que não estão no inventário fonológico da criança ou contrastar um fonema novo com um fonema conhecido.

Para \citet{Gierut1992}, os fonemas em um par mínimo podem diferir no mínimo em três dimensões: a) quanto ao número de oposições distintivas; b) quanto a natureza das distinções de traços; c) quanto à relação com a gramática da criança antes do tratamento. Para a autora essas três dimensões são essenciais para a mudança fonológica no sistema das crianças. Com base em uma série de estudos \citep{Gierut1989, Gierut1990, Gierut1991, Gierut2001, Gierut1992a}, a autora propõe uma hierarquia de combinação de variáveis (Figura \ref{fig:cap5fig1}) que pode provocar uma maior ou uma menor mudança fonológica a partir da combinação e utilização dos diferentes fonemas na terapia.



\begin{Figura}
%     \includegraphics{figures/72.png}
    \fittable{
    \fbox{
    \begin{tikzpicture}
    \node(center){=};
    \node(2b)[left=0mm of center,text width=5.7cm]{{2 fonemas novos\\distinções de classes não principais\\distinções máximas de traços}};
    \node(2a)[above=1mm of 2b,text width=5.7cm]{{2 fonemas novos\\distinções de classes principais\\distinções máximas de traços}};
    \node(1a)[right=0mm of center,text width=6cm]{{1 fonema novo\\distinções de classes principais\\distinções máximas de traços}};
    \node(1b)[below=5mm of 1a,text width=6cm]{{1 fonema novo\\distinções de classes não principais\\distinções máximas de traços}};
    \node(mais)[above of=2a,xshift=-35mm]{MAIS};
    \node(menos)[below=6cm of mais]{MENOS};
    \draw[<->](mais) -- (menos);
    \draw(2a.east) -- (1a.north west);
    \draw(1a.south west) -- (1b);
    \end{tikzpicture}
    }
    }
    \caption{Hierarquia de formatos que pode provocar maior ou menor mudança fonológica \citep{Mota1996}}
    \label{fig:cap5fig1}
\end{Figura}


O modelo \emph{Empty Set} \citep{Gierut1992} segue o mesmo padrão de oposição entre pares de palavras que se diferenciam em apenas um fonema. Pela hierarquia apresentada é possível verificar que a maior mudança fonológica será alcançada quando for possível empregar na terapia de fala dois fonemas ainda ausentes no sistema da criança, sendo que esses fonemas possuam distinção de classe principal e que possuam distinção máxima de traços. Uma menor mudança no sistema irá ocorrer quando for empregado apenas um fonema novo (que ainda não está presente no inventário), com distinção de classe não-principal e distinção máxima de traços.

No Brasil, o Modelo de Pares Mínimos Oposição Mínima/Oposições Máximas{\slash}\emph{Emp\-ty Set} foi aplicado por \citet{Bagetti2005}, sendo eficaz na reestruturação dos sistemas fonológicos das crianças tratadas. Assim, os procedimentos iniciais para a escolha dos alvos de tratamento são:
\begin{itemize}
\item [(i)] Iniciar por uma análise fonológica aprofundada (análise contrastiva e de traços distintivos).
\item [(ii)] Determinar quais os traços em que a criança apresenta dificuldades.
\item [(iii)] Determinar quais são os fonemas novos no sistema fonológico da criança. Pode-se utilizar o critério de análise fonológica de \citet{Bernhardt1992}: fonema com ocorrência de 0 a 39\% das vezes -- não adquirido (fonema novo); ocorrência de 40 a 79\% das vezes - parcialmente adquirido; ocorrência de 80\% a 100\% -- adquirido.
\item [(iv)] Escolher os sons-alvo do tratamento: os critérios para a escolha dos alvos devem basear-se na hierarquia de \citet{Gierut1992}, já apresentada.
\item [(v)] Selecionar os pares mínimos que irão contrastar os sons-alvo escolhidos: pode-se selecionar de dois a oito pares mínimos, sendo que os fonemas-al\-vo devem estar sempre na mesma posição da palavra.
\end{itemize}

Quanto aos procedimentos da terapia, as autoras recomendam:

\begin{itemize}
\item [(i)] Iniciar com a determinação de uma linha de base para cada som-problema: essa linha é uma sondagem realizada antes de se iniciar o tratamento fonológico. Devem-se verificar quais são os sons ausentes e parcialmente adquiridos no sistema fonológico da criança, nas diferentes posições silábicas. Cada som não adquirido em cada posição silábica deve ser testado, selecionando-se seis palavras que contenham o som e seja possível representar por figuras. A criança deverá nomear essas figuras espontaneamente sendo gravada pelo terapeuta para posterior transcrição fonética. Preferencialmente, o único som ausente ou parcialmente adquirido deverá ser o testado. A linha de base é realizada para se ter um parâmetro inicial para comparar os progressos do tratamento e as generalizações ocorridas no seu andamento.
\item [(ii)] Realizar cinco sessões de estimulação com os pares mínimos.
\item [(iii)] Na sexta sessão, realizar a sondagem da generalização -- realizada com as mesmas figuras da linha de base. Se a criança atingir 50\% de produções corretas, realizar mais cinco sessões com os mesmos pares inseridos em sentença, porém, se as produções corretas ocorrerem em percentual inferior, repetir novamente o tratamento usando apenas palavras.
\item [(iv)] Novamente na sexta sessão, realizar outra sondagem. Se na sondagem, após o ciclo com sentenças, a criança atingir 80\% de produções corretas, já se podem determinar novos sons-alvo. Caso contrário, deve ser repetido o tratamento usando sentença. 
\item [(v)] Após 20 sessões de terapia (sem contar as da sondagem), realizar nova avaliação geral do sistema fonológico da criança, coletando os dados de fala por meio do mesmo instrumento usado na avaliação inicial e realizando a análise contrastiva e de traços distintivos.
\end{itemize}

A organização da sessão terapêutica deve seguir a seguinte estrutura: 

\begin{itemize}
\item [(i)] Cada sessão deve ser iniciada pela leitura do bombardeio auditivo dirigido à criança. Este deve conter 16 palavras -- oito para cada som-alvo do par.
\item [(ii)] A estimulação da produção correta dos pares mínimos é realizada em duas etapas: imitação e produção espontânea. Na etapa de imitação, a criança deve produzir os alvos após o modelo do terapeuta; na etapa de produção espontânea, os alvos devem ser produzidos pela criança sem o modelo. No final da sessão deve-se calcular a porcentagem de produções corretas realizadas pela criança. Na fase de imitação, a criança deve atingir um percentual de produções corretas de 80\% ou mais. Nesse caso, na sessão seguinte, deve-se passar para a fase de produção espontânea.
\item [(iii)] No final da sessão, deve realizar-se novamente o bombardeio auditivo.
\item [(iv)] Fornecer orientações aos pais ou responsáveis, do mesmo modo que é realizado nos modelos de ciclos.
\end{itemize}

\subsubsection{Modelo de Oposições Múltiplas}\label{sec:cap5sec2doisquatro}

As crianças com alterações fonológicas mais severas, em que há substituições múltiplas de vários sons por determinado segmento, têm beneficiado do Modelo de Oposições Múltiplas \citep{Williams2000}. Por esse modelo a criança é exposta a vários sons ao mesmo tempo. Por exemplo, uma criança que substitui os segmentos /k, d, s, d/ por /t/ apresentaria um menor número de traços distintivos e isso geraria uma grande quantidade de homônimos na fala, consequentemente, uma acentuada ininteligibilidade. Adotando-se o modelo de Oposições Múltiplas, todos os sons substituídos (/k, d, s, p/) seriam estimulados, fazendo oposição ao fonema produzido (/t/) \citep{Williams2000, Pagliarin2009a}.

O Modelo de Oposições Múltiplas é uma variação do Modelo de Pares Mínimos -- Oposição Mínima/Máxima e é indicado para o tratamento de múltiplos fonemas substituídos pela criança. Busca-se com o tratamento a organização de um grande número de contrastes da língua e não apenas o arranjo dos pares mínimos \citep{Williams2003}.

No Brasil, este modelo foi testado e adaptado por \citet{Ceron2009}. Quanto à estrutura terapêutica, a autora baseou-se no Modelo de Oposições Múltiplas, já que não há indicação sobre os procedimentos na proposta de \citet{Williams2000}. Para o bombardeio auditivo, devem-se utilizar cinco palavras de cada som-alvo na mesma posição trabalhada.

Esclarecendo a escolha dos alvos para terapia, uma das crianças tratadas na pesquisa de \citeauthor{Ceron2009} (op. cit) substituía os alvos /ʒ/, /l/, /ʎ/ e /ɾ/ por [z], assim a autora propôs as seguintes palavras-alvo: [ˈkaza], [ˈkaʒa], [ˈkala], [ˈkaʎa] e [ˈkaɾa]. Ressalta-se que, em Português, o terapeuta, muitas vezes, necessita utilizar palavras sem significado, assim o sentido das mesmas deve ser atribuído por meio de acordo entre criança e terapeuta e representada por figuras. Por exemplo, a palavra “Kája” poderia ser um nome de menina ou mesmo um objeto.

Não há um limite quanto ao número de alvos trabalhados. Esse fator é determinado pela preferência sistemática por um som de cada criança. Ou seja, uma criança poderia substituir cinco sons por determinado fonema e outra poderia substituir apenas dois. Assim o tratamento traria seis ou três palavras-alvo, por exemplo.

Os modelos baseados em traços distintivos defendem a ideia de que, ao adquirir um traço ou combinação de traços, esse conhecimento será compartilhado com os demais segmentos que possuem essa característica. Desse modo, diferentes sons podem beneficiar com a estimulação de um traço sem a necessidade de trabalhar de maneira independente cada um dos fonemas. A premissa central dos modelos terapêuticos é enfatizar a comunicação, mostrando para a criança que um erro ou a substituição de um som por outro pode provocar a alteração na palavra falada, dificultando a inteligibilidade da fala.

\subsection{As relações implicacionais na terapia dos transtornos fonológicos}\label{sec:cap5sec2tres}
Com base em uma hierarquia implicacional é possível estabelecer as projeções de aquisição dos traços na ampliação dos inventários fonológicos de crianças com restrições severas em seus sistemas de contrastes. A ideia principal de um programa de terapia que considera as relações implicacionais entre os traços distintivos é a de que a seleção dos sons-alvo de tratamento deve basear-se no que é linguisticamente mais complexo. O trabalho para a aquisição de propriedades difíceis ou complexas acarreta a aquisição de todas as propriedades mais simples ou menos complexas \citep{Jakobson1972, Dinnsen1990, Mota1996}.

Fundamentada na Teoria Autossegmental, especialmente na Geometria de Traços de \citet{Clements1995} e na teoria de restrições de \citet{Calabrese1995}, \citet{Mota1996} propôs o Modelo Implicacional de Complexidade Traços -- MICT. Esse Modelo foi construído a partir dos dados de 25 crianças com transtorno fonológico na faixa etária de 4 a 10 anos que não receberam tratamento fonoaudiológico prévio. O MICT é capaz de prever as possibilidades da aquisição segmental do Português Brasileiro sob a forma de caminhos a serem percorridos durante o processo de aquisição. Além disso, o Modelo pode explicar os atrasos no processo de aquisição e auxiliar na determinação da gravidade do transtorno fonológico, a partir dos níveis de complexidades e rotas percorridas \citep{Mota1996, Mota2001}. O MICT representa as diversas relações implicacionais entre traços, estando estruturado sob forma arbórea, conforme ilustrado na Figura \ref{fig:cap5fig2}.

De acordo com o MICT, há uma raiz correspondente ao \emph{Estado 0}, em que apenas estruturas e traços não-marcados estão presentes. Esse \emph{Estado 0} equivale ao que é dado na Gramática Universal (GU), sistema básico composto pelos fonemas /p, t, m, n/, que correspondem aos sons menos complexos e os primeiros sons produzidos pela criança \citep{Mota1996}.

A partir do \emph{Estado 0}, a criança progride em direção a um aumento de complexidade em seu sistema. Assim, quanto mais distante do \emph{Estado 0} os traços estiverem, mais complexos eles são. Isso ocorre através das evidências do \emph{input} e das próprias capacidades cognitivas e articulatórias. Dessa forma, a criança determina especificações de outros traços mais complexos, que não estavam presentes na representação inicial. Essas especificações ocorrem de forma gradual, tanto em termos de quais traços são especificados primeiros, como em termos de expansão desses no sistema (\citeauthor[op. cit.]{Mota1996}).



%     \includegraphics{figures/73.png}
\begin{Figura}
\small
\begin{tikzpicture}
% 		\matrix(m)[draw,rectangle,matrix of nodes]
% 		{
% 			0&&&&&&&\\
% 			N=1&&&&&&&x\\
% 			N=2&&&&&x&&\\
% 			N=3&x&&&&&&\\
% 			N=4&x&&&&&&\\
% 			N=5&&x&&&&&\\
% 			N=6&&&&&&x&\\
% 			N=7&&&&x&&&\\
% 			N=8&&&x&&&&\\
% 			N=9&x&&&&&&x\\
% 		};
		   \matrix [matrix of nodes, text width=1cm](A){
%     a_{1,1} & a_{1,2} & \dots  & a_{1,j} & \dots & a_{1,n}\\
%     a_{2,1} & a_{2,2} & \dots  & a_{2,j} & \dots & a_{2,n}\\
%     \vdots  & \vdots  &  & \vdots  &  & \vdots\\
%     a_{i,1} & a_{i,2} & \dots  & a_{i,j} & \dots & a_{i,n}\\
%     \vdots  & \vdots  &  & \vdots  &  & \vdots\\
%     a_{n,1} & a_{n,2} & \dots  & a_{n,j} & \dots & a_{n,n}\\
			(N = Nível de complexidade)&~&~&~&~&~&~&\\
			N=1&~&~&~&~&~&~&{[-ant]\\ (ñ)}\\
			N=2&~&~&~&~&{[+voz]\\ (b,d)}&~&\\
			N=3&{\mbox{´[dors]/}\\\mbox{(-voz)} (k)}&~&~&~&~&~&\\
			N=4&{[dors, +voz]\\ (g)}&~&~&~&~&~&\\
			N=5&~&{[+cont] (±voz)\\ (f,v,s,z)}&~&~&~&~&\\
			N=6&~&~&~&~&~&{[+aprox]\\ (l)}&\\
			N=7&~&~&~&{\mbox{[cor, -ant]/} (+cont)\\ (S,Z)}&~&~&\\
			N=8&~&~&{[+aprox, +cont]\\ (r)}&~&~&~&\\
			N=9&{[+aprox, +cont, dors]\\ (R)}&~&~&~&~&~&{[+aprox, -ant]\\ (ʎ)}\\
   };
%     \node[
%      fit=(A-4-6)(A-4-6),
%      inner xsep=20pt,inner ysep=0,
%      label=right: $i$-ième ligne
%     ](L) {};
%
%    \node[
%      fit=(A-6-4)(A-6-4),
%      inner xsep=20pt,inner ysep=20pt,
%      label=below: $j$-ième colonne
%      ](C) {};

%     \draw[->](L.east)-- ([xshift=12pt]A-4-6.east);
%     \draw[->](C.south)-- (A-6-4);
                \node(legend)[above=0mm of A-1-6,rectangle,draw,text width=3cm]{%
                    [-vocóide]\\{}
                    [-aprox]\\{}
                    [±soante]\\{}
                    [-voz]\\{}
                    [+voz]/([+soante])\\{}
                    [-contínuo]\\{}
                    [cor, +ant]\\{}
                    [lab]\\{}
                    };
                \draw(A-10-2)--(A-6-3) node[midway,right] {B7};
                \draw(A-9-4)--(A-6-3) node[midway,left] {B5};
                \draw(A-8-5)--(A-6-3) node[midway,above] {B4};
                \draw(A-3-6)--(A-6-3) node[midway,below] {B2};
                \draw[dashed](A-7-7)--(A-10-2);
                \draw[dashed](A-7-7)--(A-9-4);
                \draw(A-2-8)--(A-10-8) node[midway,left] {C3};
                \draw(A-2-8)--(A-8-5) node[midway,yshift=11mm,xshift=3mm] {C2};
                \draw(A-3-6)--(A-7-7) node[midway,yshift=3mm,xshift=-3mm] {B3};
                \draw(A-7-7)--(A-10-8) node[midway,left] {B6};
                \draw[dashed](A-3-6)--(A-5-2);
                \draw(A-4-2)--(A-5-2) node[midway,left] {A2};
                \draw(A-5-2)--(A-10-2) node[midway,left] {A3};
                \draw(legend.south)--(A-4-2) node[midway,above] {A1};
                \draw(legend.south)--(A-3-6) node[midway,left] {B1};
                \draw(legend.south)--(A-2-8) node[midway,below] {C1};
                \node(estado)[left=0mm of legend.north west,yshift=.1em]{Estado 0};
\end{tikzpicture}


    \caption{Modelo Implicacional de Complexidade de Traços (MICT) proposto por \citet[154]{Mota1996}.}
    \label{fig:cap5fig2}
\end{Figura}


Entretanto, o aumento de complexidade não ocorre da mesma forma para todas as crianças, uma vez que existem relações implicacionais, onde a presença de certos traços marcados no sistema implica a presença de outros traços marcados. Assim, as crianças não seguem a mesma rota de aquisição, mas os caminhos percorridos para o desenvolvimento da complexidade nos sistemas seguem leis implicacionais existentes entre os traços marcados (\citeauthor[op. cit]{Mota1996}). De acordo com o MICT, os primeiros traços marcados a serem especificados são o [-ant], que faz com que se estabeleça a representação de /ɲ/, o [+voz], levando às representações de /b/ e/ou /d/ e o [dors], levando à representação de /k/. A especificação desses traços não ocorre simultaneamente. Assim, a criança pode especificar primeiramente o traço [-ant], ou o traço [+voz] ou o traço [dors]. Logo o sistema é expandido através de uma distinção de ponto ou de uma distinção de vozeamento. A combinação de dois traços marcados só irá ocorrer após esses dois traços terem sido especificados individualmente. Por exemplo, os traços marcados [+aprox,-ant] apenas serão estabelecidos quando ambos os traços já tiverem sido especificados individualmente nas estruturas menos complexas, isto é, o /ʎ/ vai surgir no sistema se neste já houver /l/ e /ɲ/, representando um grau a mais de complexidade no sistema. Contudo, a relação de marcação de alguns traços pode mostrar-se mais fraca sendo representada no modelo por uma linha pontilhada. Isso ocorre na especificação dos traços [dors, +voz], [+aprox, +cont, dors], [+aprox, +cont] em que não há obrigatoriedade do estabelecimento individual dos traços [+voz], [+aprox], [+cont], respectivamente. Uma vez especificado o traço [+voz], a criança pode especificar o traço [+cont], levando à representação das fricativas labiais e coronais [+anterior] /f/ e/ou /v/ e/ou /s/ e/ou /z/ e/ou o traço [+aprox], levando à representação de /l/. Seguindo o processo de aquisição, depois de especificar o traço [+cont] e já tendo especificado o traço [-ant] para as nasais, a criança pode combinar esses dois traços, levando à representação de /ʃ/ e/ou /ʒ/. A combinação de traços [+aprox, +cont] para a representação de /ɾ/ depende da especificação de [+cont] no sistema e, geralmente, da presença de [+aprox], embora essa última relação implicacional também seja mais fraca. A combinação de traços [+aprox, -ant], necessária para a representação de /ʎ/, mantém relação implicacional com o traço [+aprox] e com o traço [-ant]. Isso significa que um sistema só terá /ʎ/ se os fonemas /l/ e /ɲ/ estiverem presentes.  Finalizando o processo de aquisição, tem-se a combinação de traços [+aprox,+cont,dors] para a representação de /R/, que mantém relação implicacional com a combinação de traços [dors, +voz], com o traço [+cont] e, na maior parte das vezes, com o traço [+aprox]. Assim, para o sistema apresentar /R/, é necessária a presença do fonema /g/, pelo menos uma fricativa labial ou coronal e, geralmente, a líquida /l/. 

A disposição dos traços em diferentes níveis na representação do modelo indica os diferentes graus de complexidade entre os traços marcados. Os segmentos se tornam mais complexos de acordo com a sua distância do nível 0. Neste modelo, os segmentos que constituem o nível 9 (/R/ e /ʎ/) são, portanto, os mais marcados e mais complexos.

Segundo \citet{Mota1996}, embora o MICT tenha sido construído a partir dos dados de crianças com transtorno fonológico, o modelo pode ser válido para representar a aquisição fonológica típica. 

O uso do MICT na terapia serve de base para a escolha dos alvos a serem tratados, uma vez que se partindo das distinções mais complexas dos traços, as distinções menos complexas podem ser obtidas automaticamente, sem a necessidade de um tratamento direto.

O MICT pode ser usado como princípio teórico nas decisões e previsões para a terapia fonoaudiológica, indicando as rotas que a criança pode escolher, e os níveis que atingirão pela intervenção.

\citet{Brancalioni2015} apresentou as possíveis generalizações com base nas relações implicacionais apresentadas pelo MICT, considerando também o estudo de \citet{Rangel1998}, que propôs algumas modificações nos níveis mais complexos do MICT: por exemplo, a partir do tratamento do som-alvo /ɲ/, [-ant], N1, seriam previstas generalizações para os fonemas /p, t, m, n/ [-voz], [-aprox], [±soan], [-voz], [+voz]/([-soan]), [-cont], [cor, ant], [lab], \emph{Estado 0} (Rota C1). O tratamento com o som-alvo /g/ [dors]/(+voz), N4, poderia levar a generalizações para os fonemas /k/ [dors]/(-voz), N3 (Rota A2); e /p, t, m, n/ [-voz],[-aprox], [±soan], [-voz], [+voz]/([-soan]), [-cont], [cor, -ant], [lab], \emph{Estado 0} (Rota A1). Se um dos sons-alvo /ʃ, ʒ/ [cor,+cont]/(-ant), N6, são escolhidos para o tratamento, poderiam acontecer generalizações para os fonemas: /f, v, s, z/ [+cont] (±voz), N5 (Rota B4); /b,d/ [+voz], N2 (Rota B2); /ɲ/ [-ant], N1 (Rota C2); e /p, t, m, n/ [-voz],[-aprox], [±soan], [-voz], [+voz]/([-soan]), [-cont], [cor, -ant], [lab], \emph{Estado 0} (Rotas B1 ou C1).  A idéia é de que, ao se tratar fonemas que estejam em níveis mais complexos na hierarquia, os fonemas menos complexos, que mantêm relações de implicação com o som-alvo, sejam adquiridos sem a necessidade de um tratamento direto.

Vários estudos foram realizados tendo com objetivo verificar as generalizações implicacionas pelo MICT ocorridas no tratamento de crianças com transtorno fonológico \citep{KeskeSoares1998, KeskeSoares2001, Schaefer1999, Mota2005, AlmeidaSpindola2007, SilvaBarberena2008}. Esses estudos revelam que o Modelo Implicacional de Complexidade de Traços -- MICT pode auxiliar na tomada de decisões quanto ao tratamento, uma vez que exerce implicações importantes na prática clínica, para selecionar alvos e prever generalizações. Entretanto, nem todas as generalizações previstas pelo MICT foram verificadas em tais estudos.

\subsubsection{Modelo ABAB-Retirada e Provas Múltiplas}\label{sec:cap5sec2tresum}

O Modelo ABAB-Retirada e Provas Múltiplas \citep{Tyler1994} baseia-se na hierarquia implicacional de traços para a escolha dos sons-alvo de tratamento. A aplicação do modelo dá-se da seguinte maneira:

\begin{itemize}
\item [A1:] coleta inicial dos dados por meio de nomeações mas preferencialmente por fala espontâneas. Esses dados passam por análise contrastiva e análise de traços distintivos para determinar o nível da hierarquia de traços em que se encontra o sistema da criança e assim determinar os sons-alvo do tratamento.
\item [B1:] Início da intervenção terapêutica por meio da estimulação de seis palavras contendo o som-alvo. Tem duração de aproximadamente cinco semanas (nove sessões). São realizadas duas sessões semanais com duração de 45 minutos cada.
\item[A2:] segundo período de retirada com duração aproximada de três semanas (cinco sessões). Não há intervenção direta sobre os sons-alvo. São aplicadas provas de generalização (PGs) e coletadas amostras de fala espontânea da criança.
\item[B2:] segundo ciclo de tratamento seguido novamente de outro período de retirada (A3). 
\end{itemize}

É possível ainda realizar um terceiro ciclo de tratamento (B3), com mais cinco semanas de intervenção e mais um período de retirada (A4). No final do tratamento, é feita uma reavaliação dos dados de fala, com nova coleta de dados de fala e nomeação espontânea, e a aplicação de PG, objetivando verificar as mudanças ocorridas no sistema fonológico da criança.

O modelo inclui as provas múltiplas, que são duas medidas separadas de desempenho: PG, realizada durante os períodos de retirada, e prova-alvo básica (PAB), realizada durante os ciclos de tratamento.

A PG é realizada mediante a aplicação do instrumento de coleta de dados (nomeação espontânea) e tem por objetivo verificar as evidências de generalização dos traços trabalhados nas sessões, a partir de um som-alvo, aos sons não tratados, portanto, aos traços da hierarquia não treinados.

A PAB é aplicada com o objetivo de avaliar o progresso do som-alvo durante a intervenção terapêutica. Essa avaliação é realizada no início, no meio e no final de cada ciclo de tratamento e constitui-se de seis palavras representadas por figuras que contêm o som-alvo selecionado em diferentes posições na estrutura da sílaba e da palavra. No início do ciclo de tratamento, as palavras da PAB podem ser apresentadas à criança sob a forma de “imitação retardada”; no final do ciclo devem prevalecer as produções espontâneas. Considera-se que, havendo uma produção correta de mais de 50\% das palavras-alvo trabalhadas no ciclo, um novo som-al\-vo pode ser introduzido no próximo ciclo. Caso ocorra um percentual inferior a 50\%, o ciclo seguinte deve enfocar o mesmo som-alvo com palavras-estímulo diferentes.

A estrutura da sessão de terapia utiliza os mesmos procedimentos do Modelo de Ciclos e o bombardeio auditivo deve conter 15 palavras.

\subsection{A consciência fonológica na terapia dos transtornos fonológicos -- Metaphon}\label{sec:cap5sec2quatro}
O Metaphon \citep{Dean1986} é um procedimento terapêutico que tem por objetivo promover mudanças no sistema fonológico pelo desenvolvimento e utilização da consciência metafonológica. São princípios centrais do Metaphon:

\begin{itemize}
\item [(i)] A terapia baseia-se em uma avaliação linguística precisa e detalhada.
\item [(ii)] A terapia deve focalizar as mudanças fonológicas por meio da consciência metalinguística.
\item [(iii)] A terapia deve maximizar as oportunidades de aprendizagem oferecidas à criança.
\end{itemize}

O que distingue o Metaphon de outros programas de terapia para transtornos fonológicos é a premissa de que, a fim de conduzir uma mudança consciente e inconsciente em seu sistema fonológico, a criança deve estar ciente do modo pelo qual os sons contrastam. Assim, a finalidade é desenvolver consciência do contraste natural dos sons da fala, os quais transmitem significado, e de que a ordem dos sons pode variar nas palavras.

A terapia Metaphon é dividida em duas fases, que focalizam, cada uma, um aspecto específico da consciência metalinguística. O objetivo principal da Fase I é facilitar o conhecimento da natureza contrastiva dos fonemas (consciência fonológica), e o objetivo da Fase II é demonstrar a importância desses contrastes para uma comunicação efetiva (consciência comunicativa).

\paragraph*{Fase I.} Compreende quatro níveis: do conceito, do som, do fonema e da palavra. A Fase I é a mais importante, pois objetiva fazer com que a criança se interesse pela fonologia da língua-alvo do adulto, chamando sua atenção para as propriedades do som e seus contrastes naturais, mostrando que esses contrastes entre os sons transmitem significado e que a relação desses traços pode ser manipulada para aumentar as probabilidades de ser entendido.

No nível do conceito, o vocabulário compartilhado entre criança e terapeuta, sobre o contraste em questão, é desenvolvido. Por exemplo, os termos “comprido/curto” podem ser usados para designar fricativa/plosiva.

Os níveis do conceito e do som fornecem uma base comum para a extensão dos conceitos para o fonema.  Em cada nível, a criança e o terapeuta se revezam nos papeis de falante e ouvinte. Esse estágio inicial da Fase I proporciona amplas oportunidades para as crianças obterem sucesso, o que é importante, especialmente para aquelas que já tiveram fracasso comunicativo.

O próximo nível da Fase I é o do fonema, quando a compreensão do contraste em questão é transferida para os sons da fala. Inicialmente a criança e o terapeuta produzem, espontaneamente, uma gama de sons contrastivos da fala. Cartões ilustrativos, com gravuras referentes ao contraste em questão, são introduzidos para ajudar a criança a produzir um som particular. Por exemplo, se um cartão mostra a frente de uma casa, um som “anterior” (bilabial ou alveolar) deverá ser produzido ao invés de um som “posterior” (velar).

Por intermédio do nível fonêmico, o terapeuta pode acessar o conhecimento adquirido nos níveis do conceito e do som. O vocabulário compartilhado fornece um meio para a discussão dos erros e dos acertos.

No período final da Fase I -- nível da palavra, a criança é apenas ouvinte. Nesse nível propõe-se que ela ouça palavras que contrastem minimamente e que julgue, por exemplo, se a palavra dita pelo terapeuta inicia com um som “barulhento” (sonoro) ou “silencioso” (surdo). Essas atividades conduzem-na à Fase II, em que a ênfase é a estimulação de uso de alvos contrastantes em situações comunicativas.

\paragraph*{Fase II.} Essa fase desenvolve a consciência comunicativa utilizando o aprendizado da fase I -- no nível de pares mínimos, a criança é colocada em uma situação em que deve indicar a distinção entre as figuras que representam um par mínimo. Essa atividade reflete e consolida a aprendizagem da Fase I. O terapeuta e a criança se revezam para selecionar um cartão que contenha um membro do par mínimo. O falante, então, nomeia a figura do cartão. O ouvinte indica qual das palavras do par ele ouviu, apontando para um cartão resposta. Cada troca termina com o cartão do falante sendo incorporado em uma atividade do tipo da que foi desenvolvida na Fase I. Por exemplo, tendo feito a distinção entre “bola” e “gola”, o cartão resposta poderia ser comparado à frente ou aos fundos de uma casa (pontos articulatórios /b/ e /g/).

A Fase I é reforçada continuamente e o conhecimento nela adquirido fornece uma base para a discussão dos erros e dos acertos na Fase II.

Para limitar as demandas de processamento cognitivo, apenas um par de palavras é utilizado. Porém, para trabalhar diferentes classes de sons, podem-se utilizar dois ou três pares diferentes ilustrando o mesmo contraste.

O estágio final da Fase II, isto é, o nível da sentença, fornece a oportunidade para o contraste alvo ser mantido em um contexto. Para isso, os pares mínimos são colocados em uma sentença (os mesmos do nível anterior).

O objetivo da Fase II é transferir o conhecimento metafonológico, adquirido na Fase I, para situações reais de comunicação; construir a consciência de que o \emph{output} pode não estar refletindo a intenção comunicativa; e desenvolver a consciência de que é possível alterar ou reparar o \emph{output} para aumentar as chances de comunicação efetiva.

\section{Conclusão}
Dada a prevalência dos TSF na população infantil, o tema intervenção é muito importante. Os TSF de origem fonológica são beneficiados por abordagens terapêuticas que têm por base princípios linguísticos. Neste capítulo, buscamos apresentar modelos de terapia que têm na base teorias fonológicas.

Dos modelos apresentados, podemos observar que alguns preconizam um tipo de estimulação que parte do que é menos complexo e vai agregando dificuldade ao longo do tempo. Como exemplo deste tipo de abordagem, temos o Modelo de Ciclos/Ciclos Modificado, o Modelo de Pares Mínimos (oposições mínimas) e o Modelo Metaphon. Em geral, esses modelos utilizam a ordem de aquisição típica dos sons da língua para escolher por onde começar a terapia.

Outros modelos, por sua vez, partem de aspectos mais complexos da língua e esperam que os menos complexos sejam adquiridos sem a necessidade de um tratamento direto por meio da generalização. Modelos baseados em relações implicacionais, como o ABAB-Retirada ou o Modelo de Oposições Máximas/\emph{Empty Set} são exemplos deste tipo de abordagem. Cabe ao fonoaudiólogo a decisão sobre qual tipo de abordagem escolher. Embora os modelos que partem de uma complexidade maior sejam mais generalizantes, muitas crianças não se adaptam a esse tipo de terapia. Conhecer muito bem o sistema fonológico de cada paciente antes do tratamento assim como suas características individuais é fundamental para que o fonoaudiólogo escolha o tratamento mais adequado. 

Também é possível observar que não há abordagem terapêutica superior ou inferior, e sim a que melhor se adapta a determinado paciente e que o terapeuta domina com mais maestria.

Assim, considera-se que este tema é inesgotável e quanto mais estudos forem realizados sobre o assunto, mais subsídios os fonoaudiólogos terão para executarem planos terapêuticos eficazes.

\largerpage
{\sloppy\printbibliography[heading=subbibliography,notkeyword=this]}
\end{document}
