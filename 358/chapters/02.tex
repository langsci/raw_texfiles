\documentclass[output=paper,colorlinks,citecolor=brown,booklanguage=portuguese]{langscibook} 
\ChapterDOI{10.5281/zenodo.7233215}
\title{Sintaxe clínica}
\author{João Costa\affiliation{Universidade Nova de Lisboa, Faculdade de Ciências Sociais e Humanas}}

\abstract{Os défices no conhecimento sintático nem sempre são evidentes, por se manifestarem sobretudo por défices de compreensão ou por estruturas que não são produzidas, mais do que pela produção de enunciados percetíveis como agramaticais.

Depois de uma breve caracterização da natureza do conhecimento sintático ao dispor dos falantes, apresentam-se alguns exemplos de perturbação específica desta componente da gramática em contexto clínico, mostrando-se o contributo dos instrumentos da linguística teórico para melhores resultados ao nível da descrição, da clarificação, do diagnóstico, da avaliação e da intervenção.\\

\textbf{Palavras-chave}: sintaxe; Perturbação do Desenvolvimento da Linguagem; agramatismo; dependências}

\IfFileExists{../localcommands.tex}{
   \addbibresource{../localbibliography.bib}
   % add all extra packages you need to load to this file

\usepackage{tabularx,multicol}
\usepackage{url}
\urlstyle{same}

\usepackage{listings}
\lstset{basicstyle=\ttfamily,tabsize=2,breaklines=true}

\usepackage{langsci-basic}
\usepackage{langsci-optional}
\usepackage{langsci-lgr}
\usepackage{langsci-osl}
% \usepackage{./langsci/styles/langsci-lgr}
% \usepackage{./langsci/styles/langsci-osl}
% \usepackage{langsci-gb4e}

\usepackage{tikz}
\usetikzlibrary{patterns,calc}
\pgfdeclarepatternformonly{south east lines}{\pgfqpoint{-0pt}{-0pt}}{\pgfqpoint{3pt}{3pt}}{\pgfqpoint{3pt}{3pt}}{
    \pgfsetlinewidth{0.6pt}
    \pgfpathmoveto{\pgfqpoint{0pt}{3pt}}
    \pgfpathlineto{\pgfqpoint{3pt}{0pt}}
    \pgfpathmoveto{\pgfqpoint{.2pt}{-.2pt}}
    \pgfpathlineto{\pgfqpoint{-.2pt}{.2pt}}
    \pgfpathmoveto{\pgfqpoint{3.2pt}{2.8pt}}
    \pgfpathlineto{\pgfqpoint{2.8pt}{3.2pt}}
    \pgfusepath{stroke}}
    
\usepackage{stmaryrd}
\usepackage{wasysym}
\usepackage{multirow}
\usepackage{caption}
\usepackage{subcaption}
\usepackage{mathrsfs}
\usepackage{qtree}

\usepackage{linguex}


   %pminos do not split footnotes
% \interfootnotelinepenalty=10000 %Footnote in Laporte chapters has to be split SN


%\DeclareIndexNameFormat{default}{%
%\nameparts{#1}%
%\usebibmacro{index:name}%
%{\index[names]}%
%{\namepartfamily}%
%{\namepartgiveni}%
% {}% L1
% {}% L2
%{\namepartprefix}% generates spurious space L3
%{\namepartsuffix}% generates spurious space L4
%}

%  {\DeclareIndexNameFormat{default}{%
%     \usebibmacro{index:name}{\index[names]}{#1}{#3}{#5}{#7}}}

%\DeclareIndexNameFormat{default}{%
%  \usebibmacro{index:name}{\sindex[nom]}{#1}{#3}{#5}{#7}}

%\DeclareIndexNameFormat{default}{%
%  \usebibmacro{index:name}{\sindex[person]}{#1}{#3}{#5}{#7}}
%\DeclareIndexNameFormat{default}{%
%\nameparts{#1} \usebibmacro{index:name}{\sindex[person]]}{\namepartfamily}{‌​\namepartgiven}{\nam‌​epartprefix}{\namepa‌​rtsuffix}}

%\newcommand{\smiley}{:)}

%\renewbibmacro*{index:name}[5]{%
%\usebibmacro{index:entry}{#1}%
%{\iffieldundef{usera}{}{\thefield{usera}\actualoperator}\mkbibindexname{#2}{#3}{#4}{#5}}}

% \newcommand{\noop}[1]{}

%remove for final
%\overfullrule=1mm

\newcommand{\tobi}[2]}}
\renewcommand{\S}[1]{\tobi{#1}{\textsc{*}}}

% this volume references
% puts: [this volume]
% already defined: \citetv
%\newcommand{\citepv}[1]{(\citeauthor{#1} \citeyear*{#1} [this volume])}
\newcommand{\citealtv}[1]{\citeauthor{#1} \citeyear*{#1} [this volume]}

%parentheses around example number
\newcommand{\pref}[1]{(\ref{#1})}

% in-text examples

\newcommand{\lnex}[1]{\textit{#1}} %target lang word
\newcommand{\lnlit}[1]{(lit.: `#1')} %literal reading
\newcommand{\lnlat}[1]{(#1)} % latinization
\newcommand{\lntrans}[1]{`#1'} %translation
\newcommand{\lnexl}[2]%
{\lnex{#1}{} \lnlat{#2}} % ex with latinization
\newcommand{\lnexlat}[3]{\lnex{#1}{} \lnlat{#2}{} \lntrans{#3}} % ex with latinization and tranl.

%ch01
\newcommand{\co}[1]{\mbox{\textbf{#1}}}

%ch09

\newcommand{\cyrbulg}[1]{\begin{otherlanguage*}{bulgarian}#1\end{otherlanguage*}}


%ch10
\newcommand{\nlp}{{\small NLP}}
\newcommand{\mwe}{{\small MWE}}
\newcommand{\rae}{{\small RAE}}
\newcommand{\lvc}{{\small LVC}}
\newcommand{\pos}{{\small P}o{\small S}}
%\newcommand{\todo}[1]{ \textcolor{red}{#1} }

%\renewcommand{\labelenumi}{\theenumi}
%\ainamefmt{{vv}{ll}{, ff}{, jj}} % fullname

\newcommand{\biberror}[1]{{\color{red}#1}}

\newcommand{\osenovaitem}{--~}
   %% hyphenation points for line breaks
%% Normally, automatic hyphenation in LaTeX is very good
%% If a word is mis-hyphenated, add it to this file
%%
%% add information to TeX file before \begin{document} with:
%% %% hyphenation points for line breaks
%% Normally, automatic hyphenation in LaTeX is very good
%% If a word is mis-hyphenated, add it to this file
%%
%% add information to TeX file before \begin{document} with:
%% %% hyphenation points for line breaks
%% Normally, automatic hyphenation in LaTeX is very good
%% If a word is mis-hyphenated, add it to this file
%%
%% add information to TeX file before \begin{document} with:
%% \include{localhyphenation}
\hyphenation{
    Beck-man
    Ngu-yen
    back-chan-nel
    back-chan-nels
    mo-not-o-nous
    ste-reo-typ-i-cal
}

\hyphenation{
    Beck-man
    Ngu-yen
    back-chan-nel
    back-chan-nels
    mo-not-o-nous
    ste-reo-typ-i-cal
}

\hyphenation{
    Beck-man
    Ngu-yen
    back-chan-nel
    back-chan-nels
    mo-not-o-nous
    ste-reo-typ-i-cal
}

   \boolfalse{bookcompile}
   \togglepaper[02]%%chapternumber
}{}

\begin{document}
\maketitle

\section{A sintaxe -- o conhecimento linguístico menos consciente}
O conhecimento sintático é, provavelmente, aquele de que os falantes têm menos consciência, o que não significa que tenham menos intuições ou capacidade.\footnote{A norma adotada na escrita deste capítulo foi a do português europeu.}
% \footnote{Na sequência de restrições decorrentes do template usado para publicação na editora Language Science Press, os capítulos que integram o volume surgem com a referência [this volume], em inglês.}
É o nosso conhecimento sintático e morfossintático que nos permite operações tão elementares quanto juntar palavras para formar frases ou estabelecer concordância verbal, como em \REF{ex:cap2ex1} e \REF{ex:cap2ex2}:

\ea\label{ex:cap2ex1}
    \ea O cão mordeu o gato.
    \ex O gato mordeu o cão.
    \z
\z
\newpage
\ea\label{ex:cap2ex2}
    \ea {Tu és feliz.}
    \ex[*\footnotemark]{Tu é feliz.}
    \z
    \footnotetext{* Representa estrutura agramatical.}
\z

Este conhecimento não nos é explicitamente ensinado e, no entanto, é adquirido desde muito cedo \citep{Guasti2002}. Que a aquisição é precoce é evidenciado pelos comportamentos linguísticos das crianças. A título de exemplo, considerem-se as seguintes propriedades da língua:
\begin{enumerate}
  \item[] Direcionalidade
  \item[] Sujeito nulo e objeto nulo
  \item[] Recursividade
\end{enumerate}


\subsection{Direcionalidade}

Em diferentes línguas do mundo, a ordem dos constituintes da frase pode ser diversa. Por exemplo, línguas como o português ou o inglês têm ordem V(verbo) O(objeto), enquanto línguas como o turco ou o japonês têm ordem OV:

\ea\label{ex:cap2ex3}
    \ea {Português}:\\
        Eu vi o sapo
    \ex \emph{Inglês}:\\
    I saw the frog.
    \ex \emph{Turco}:\\
    \gll Kurbagăyi gördüm.\\
    sapo     vi\\
    \ex \emph{Japonês}:\\
    \gll Kaeru o mimashita.\\
    sapo  {} vi\\
    \z
\z

Nenhum de nós se recorda de nos terem ensinado que o português é uma língua VO, nem em casa, nem na escola. Os dados da aquisição da linguagem mostram que as crianças, quando começam a fazer pequenas frases, apenas com verbo e objeto, não se enganam (\citealp{Guasti2002}, entre outros). Isto significa que não se encontram, nas produções das crianças, alternâncias entre VO e OV. Por outras palavras, uma criança portuguesa dirá sempre “quero água” e não “água quero”, nem passa por momentos de hesitação, como se ainda estivesse a descobrir qual a ordem entre verbo e complemento na sua língua.  Alguma evidência, sistematizada em \citet{Nespor2008}, parece sugerir que a distinção entre os dois tipos de língua pode ser descoberta pelas crianças muito cedo a partir do padrão prosódico, em particular pela estrutura rítmica de cada língua. Esta é uma ideia muito interessante, já que pode significar que estamos a desenvolver o nosso conhecimento sintático desde muito mais cedo do poderíamos imaginar, tendo em consideração que a sensibilidade dos bebés aos padrões rítmicos se manifesta poucos dias após o nascimento \citep{Mehler2000}. 

\subsection{Sujeito nulo e objeto nulo}

Vejamos outro exemplo do conhecimento sintático de que não temos consciência. Muitos sabem, até porque o aprendem explicitamente na escola, que o português europeu é uma língua de sujeito nulo, conforme ilustrado aqui:

\ea\label{ex:cap2ex4} 
    \ea Tu és simpática
    \ex És simpática
\z
\z

Desta propriedade da língua temos muita consciência, porque convivemos com línguas que não partilham esta característica. Em inglês, por exemplo, o sujeito não pode ser nulo:

\ea\label{ex:cap2ex5} 
    \ea[]{You are nice.}
    \ex[*]{Are nice.}
\z
\z

Se esta característica do português europeu é bem conhecida, do facto de a língua poder também não realizar o complemento dos verbos \citep{Raposo1986}, há menor consciência:

\ea\label{ex:cap2:ex6} 
    \ea Comprei um jornal e troquei-o por outro.
    \ex Comprei um jornal e troquei por outros.
    \z
    \z
    
Na frase (6b), o pronome está omisso, o que não acontece na generalidade das línguas românicas, como o francês, o italiano ou o espanhol. Esta característica do português mostra bem que temos conhecimento sintático, mas não temos consciência de todo o conhecimento que temos.

\subsection{Recursividade}

Uma das propriedades mais fascinantes da sintaxe das línguas naturais é a recursividade. Esta é uma característica comum a todas as línguas, que possibilita a capacidade de repetir estruturas, através de coordenação ou subordinação, como nos exemplos seguintes:

\ea\label{ex:cap2ex7}
    \ea Comprei batatas.
    \ex Comprei cenouras e batatas.
    \ex Comprei bananas, cenouras e batatas.
    \ex Comprei maçãs, bananas, cenouras e batatas.
    \z
\z
\ea\label{ex:cap2ex8}
    \ea Está a chover.
    \ex A Maria disse que está a chover.
    \ex O Pedro acha que a Maria disse que está a chover.
    \ex O Zé acredita que o Pedro acha que a Maria disse que está a chover.
\z
\z

Facilmente, percebemos, sem nunca disso nos termos dado conta, que temos um potencial inato para construir frases que, no limite, podiam não ter fim. Poderíamos continuar a coordenar complementos a seguir ao verbo “comprei” ou a criar novos sujeitos e verbos subordinantes em \REF{ex:cap2ex8}. A recursividade é tão mais fascinante quanto nos apercebemos que temos conhecimento que nos permite gerar estas frases potencialmente infinitas e sabemos que nunca ninguém nos ensinou que podíamos fazê-lo.

Genericamente, enquanto falantes somos menos sensíveis a questões relacionadas com o nosso conhecimento sintático do que àquele que envolve outras componentes da gramática. Notamos variações nos sons. Notamos se alguém constrói uma frase sem sentido. Identificamos palavras que não conhecemos, mas somos muito menos sensíveis à sintaxe. Por exemplo, quando comparamos as variedades europeia e brasileira do português, notamos a diferença nos sons, notamos a variação nas palavras, mas poucos reconhecerão que uma das diferenças entre as duas variedades está nas propriedades do sujeito nulo:

\ea\label{ex:cap2ex9} A Maria disse que o Pedro desconfiava que estava grávida.
\z

A frase em \REF{ex:cap2ex9} é aceitável em português europeu, mas agramatical em português brasileiro, dadas as especificidades do sujeito nulo nesta variedade. Perante dois antecedentes possíveis, o sujeito nulo em português brasileiro só pode ter como antecedente o mais próximo. Assim, em \REF{ex:cap2ex9} força-se a leitura anómala e com problema de concordância de género, segundo a qual o “grávido” seria o Pedro e não a Maria. Em português europeu, essa característica não existe. Estas diferenças têm sido estudadas amplamente por inúmeros linguistas (veja-se \citealt{Rodrigues2004}) para uma síntese das diferentes análises. Sendo, para este texto, irrelevante a explicação por trás destas diferenças, importa-me chamar a atenção para o facto de, enquanto falantes, não nos apercebermos tão claramente do papel que a sintaxe desempenha no nosso conhecimento e performance.

Esta aparente “invisibilidade” da sintaxe não é surpreendente, quando refletimos sobre o seu papel na arquitetura da nossa competência linguística. Quando observamos uma língua, pensamos imediatamente no conjunto de palavras e sons, bem como no seu significado. Há, pois, uma relação entre som e significado que se estabelece e que faz com que consigamos interpretar de forma automática aquilo que ouvimos ou juntar sons que compõem morfemas e palavras, que por sua vez se combinam com outras para criar frases com sentido. Retomemos o exemplo em \REF{ex:cap2ex1}:

\begin{exe}
\exr{ex:cap2ex1}
    \ea O cão mordeu o gato.
    \ex O gato mordeu o cão.
    \z
\end{exe}
    
Estas duas frases têm exatamente os mesmos sons, as mesmas palavras, mas têm significados distintos. A forma como organizamos os constituintes da frase tem implicações na interpretação que lhe atribuímos.

Podemos, desta forma, entender qual o papel da sintaxe na linguagem humana. Poderíamos descrevê-la como o conjunto de princípios e operações abstratas que nos permitem organizar sons em palavras, e estas em frases que têm significado. A sintaxe é, assim, o mediador entre o som e o significado. Não se ouve, não se compreende, apenas contribui para que a relação se estabeleça. Esta sua “invisibilidade” explica que haja uma menor consciência do conhecimento que dela temos.

\section{A sintaxe em contexto clínico}
A longa introdução que antecede esta secção teve como propósito tornar um pouco mais clara a natureza do conhecimento sintático. Por ser menos evidente, é mais fácil passar despercebida a inabilidade sintática em situações de vulnerabilidade linguística. Contudo, sendo a componente onde mais facilmente se encontram dimensões puramente linguísticas e não identificáveis noutras habilidades cognitivas, talvez constitua uma porta de entrada para um mais profundo conhecimento da nossa faculdade da linguagem.

Quando nos deparamos com uma dificuldade linguística adquirida, na sequência de outra situação clínica, ou precoce, que se evidencia num desenvolvimento atípico, há evidências que nos despertam para essa dificuldade. Se os sons não forem pronunciados da mesma forma, se as palavras ou as frases forem produzidas com um sentido alterado, há uma evidência muito clara de que algo não está bem. Em contexto de perturbações sintáticas, essa evidência é muito mais difícil de obter, dado que se pode manifestar pela ausência de produções ou por falhas no processamento que não se detetam.

Estudar a aquisição, desenvolvimento e o conhecimento da linguagem implica integrar fatores que são estritamente linguísticos e fatores cognitivos gerais. \citet{Hauser2002} estabelecem esta diferença designando as duas áreas como Faculdade da Linguagem em Sentido Amplo (Broad Faculty of Language) e Faculdade da Linguagem em Sentido Estrito (Narrow Faculty of Language). A primeira inclui não apenas as propriedades intrínsecas ao funcionamento das línguas, mas também todos os subsistemas da cognição que com eles interagem. Por exemplo, a memória, a atenção, os sistemas de organização neuronal são componentes da cognição fundamentais para bons desempenhos linguísticos, mas não estão especificamente alocados ao funcionamento da linguagem. Sem eles não há linguagem, mas eles não explicam o funcionamento das línguas. Já quando estudamos a Faculdade da Linguagem em Sentido Estrito, descobrimos aquelas propriedades que correspondem a conhecimento estritamente linguístico. Vejamos dois exemplos:

A possibilidade que temos de pronunciar um constituinte de uma frase longe da posição em que é interpretado decorre apenas de mecanismos linguísticos. Consideremos o seguinte exemplo. Em \REF{ex:cap2ex10}, interpretamos \emph{o gato} como complemento da forma verbal \emph{mordeu}.

\ea\label{ex:cap2ex10} Todos disseram que o cão mordeu o gato.
\z

Fazemo-lo porque, como já se evidenciou, temos conhecimento implícito sobre a ordem em que aparece sujeito, verbo e complemento.  Em \REF{ex:cap2ex11}, apesar de estar numa posição distinta, continuamos a interpretar \emph{o gato} como complemento do mesmo verbo:

\ea\label{ex:cap2ex11} O gato, todos disseram que o cão o mordeu.
\z

A comparação entre \REF{ex:cap2ex10} e \REF{ex:cap2ex11} atesta da nossa capacidade de estabelecer relações de significado entre expressões que não estão adjacentes, próximas uma da outra. Um grande avanço no conhecimento linguístico, desde o final dos anos 60 do século passado, consistiu na identificação de propriedades e restrições desta capacidade, em múltiplos trabalhos, de que \citet{Ross1967} é pioneiro e um dos grandes impulsionadores. Um dos aspetos mais interessantes é a constatação de que as restrições no estabelecimento de relações de dependência, como em \REF{ex:cap2ex11}, são estritamente do foro do conhecimento linguístico. São, pois, propriedades linguísticas em sentido estrito. 

A relação entre os fatores internos e externos que regulam a competência linguística só nas últimas duas décadas tem recebido mais atenção. Tal deveu-se a uma necessidade de perceber melhor propriedades linguísticas, até aos anos 50 completamente desconhecidas, a partir da rutura paradigmática iniciada por \citeauthor{Chomsky1957} em \citeyear{Chomsky1957}.

Conforme enunciado em \citet{Sprouse2013}, devem distinguir-se três níveis fundamentais de interação na abordagem aos fenómenos e processos linguísticos. O nível computacional é o nível da teoria gramatical, constituindo-se no que nos permite descrever como é o sistema de geração de enunciados linguísticos e por que motivo esse sistema tem o formato que tem. O nível algorítmico, estudado na psicolinguística, estuda a forma como o sistema é usado e quais os mecanismos e processos ativados para a produção e compreensão dos enunciados que o sistema computacional tem o potencial de gerar. Finalmente, o nível implementacional relaciona-se com a representação física da língua no cérebro, isto é, com a atividade eletroquímica e com a representação neuronal concreta da linguagem. Este é o nível tradicionalmente estudado pela neurolinguística.

Tal como referido, a investigação sobre as relações de interface entre estes três níveis tem vindo a ganhar crescente atenção nos últimos anos. \citet{Chomsky2001}, num texto cujo título é autoexplicativo, explicita a necessidade de construir modelos de análise linguística que não se limitem a simplesmente explicar o funcionamento das línguas de forma adequada, mas também a compreendê-lo na relação com os níveis implementacional e algorítimico. Em termos mais simples, perceber, por exemplo, como compreendo a frase \emph{o cão mordeu o gato} implica conhecer os fatores estritamente linguísticos que permitem a produção do enunciado (desde a estrutura dos sons, ao estabelecimento de processos de concordância e flexão verbal, até à construção do significado da frase), mas também entender quais os processos que explicam como percebo, compreendo e interpreto e os sistemas físicos ativados para que esta atividade se concretize.

Em contexto clínico, percebemos bem as relações de interdependência. Consideremos o caso de algumas anomias. Posso, após uma lesão, ter dificuldades de acesso lexical, mas isso não significa que não conheça as palavras e o seu significado. Apenas, em alguns contextos de uso tenho dificuldade de “ir buscar” essas palavras. É preciso conhecer relações entre organização das redes de palavras e os mecanismos de recuperação de informação.

De igual modo, posso, por motivos meramente articulatórios, ter dificuldade na produção de alguns sons. Por exemplo, a fenda palatina condiciona a nasalidade das vogais e de algumas consoantes. Este é um exemplo da interação entre o funcionamento dos sons da fala e as dimensões físicas, implementacionais, que podem explicar este comportamento desviante face à produção típica.

A linguística clínica é geralmente referida como a aplicação da linguística ao estudo da incapacidade linguística em qualquer uma das suas formas \citep{Crystal1981}. Tradicionalmente, incluem-se, nos tipos de incapacidade linguística, três tipos maiores: as dificuldades que advêm do sistema auditivo, do sistema articulatório e do sistema central. O reconhecimento das vantagens inequívocas do conhecimento do funcionamento das línguas e da linguagem para intervenções mais precisas em contexto clínico é, por um lado, o resultado da investigação clínica e da linguística, mas tem constituído também uma alavanca para o desenvolvimento do estudo das relações entre a Faculdade da Linguagem em sentido estrito e a Faculdade da Linguagem em sentido amplo. Por outras palavras, entender uma perturbação ou incapacidade linguística envolve a capacidade de identificar e discernir entre o que são perturbações nas propriedades estritamente linguísticas, nos mecanismos cognitivos gerais envolvidos ou nas condições físicas e neurológicas para a produção e compreensão de enunciados.

Centremo-nos nas perturbações no sistema central da linguagem. Considera-se que estas têm a sua base no cérebro enquanto mediador entre a fala e a perceção. Comummente, destas perturbações resultam as afasias.  A principal evidência de que a linguagem tem representação cerebral e uma base biológica forte advém precisamente da constatação de que lesões em diferentes áreas do cérebro induzem perturbações linguísticas específicas. As lesões mais amplamente estudadas e mais utilizadas para atestar da autonomia da Faculdade da Linguagem são as afasias de Broca (induzida por lesão na parte anterior do cérebro) e de Wernicke (induzida por lesão na parte posterior do cérebro), designadas segundo o nome dos neurologistas que as descobriram. A afasia de Broca caracteriza-se por dificuldades na produção de algumas estruturas gramaticais, enquanto a de Wernicke se identifica pela dificuldade ou incapacidade em compreender enunciados que se ouvem ou se leem.

Estas duas condições receberam grande interesse por parte da investigação desenvolvida em linguística. Por um lado, servem de argumento para a relativa autonomia da Faculdade da Linguagem, já que estamos perante casos de perturbação linguística sem afetação de outras capacidades cognitivas. Por outro lado, mesmo internamente ao conhecimento linguístico, estes dois tipos de perturbação permitem confirmar a autonomia das várias componentes da gramática. Lesões em zonas diferentes do cérebro são preditoras de problemas apenas na produção ou apenas na compreensão, apenas na performance sintática ou apenas na performance semântica. Em particular, o agramatismo é, no conjunto dos diferentes quadros afásicos, uma evidência forte de que a sintaxe, que como vimos integra propriedades inerentemente linguísticas e princípios que apenas são relevantes para o funcionamento da linguagem, é uma componente autónoma de outras, dado que existe a possibilidade de apenas a sintaxe se encontrar prejudicada.

Também o caso dos \emph{savants}, em particular o que foi estudado em \citet{Smith1995}, constitui uma evidência clínica para a autonomia do conhecimento linguístico. No caso em apreço, estamos perante um jovem com vários défices cognitivos, mas não só sem dificuldades linguísticas, como revelador de uma sobredotação na capacidade de aprender línguas naturais. Embora haja algumas áreas de interdependência expectáveis (como, por exemplo, na interpretação da contrafactualidade ou da ironia), atesta-se uma área da cognição, independente das outras e especificamente linguística.

No desenvolvimento linguístico das crianças, são várias as perturbações que emergem que nos ajudam a perceber a autonomia da linguagem não apenas em relação a outras dimensões da cognição, mas também no que concerne à função comunicativa que assume. Três patologias são, geralmente, comparadas. As crianças com síndrome de Down têm, tipicamente, desvios significativos em tarefas cognitivas de âmbito geral que são acompanhados de perturbações ao nível das capacidades linguísticas. Esta síndrome poderia ser, pois, um argumento claro para a indissociabilidade entre cognição e linguagem. Há, no entanto, outros contextos que permitem refutar essa hipótese. A Perturbação do Desenvolvimento da Linguagem (PDL)\footnote{Termo atualmente utilizado para referir as perturbações de linguagem, que ocorrem sem causa aparente \citep{Bishop2017}. Os termos anteriormente utilizados para referir estas perturbações de linguagem eram Perturbação Específica da Linguagem (PEL) ou Perturbação Específica do Desenvolvimento Linguístico (PEDL).} consiste na constatação de que há dificuldades no desenvolvimento da linguagem em crianças que, em tarefas não verbais de testes de avaliação da capacidade cognitiva, não manifestam dificuldades \citep{Leonard1981}. Será, pois, um caso de desenvolvimento típico da cognição em paralelo com desenvolvimento atípico da linguagem. Por outro lado, crianças com Síndrome de Williams manifestam uma dificuldade geral em tarefas cognitivas, por exemplo na orientação espacial ou na capacidade de representação simbólica, sem qualquer desvio no desenvolvimento da linguagem. Estes exemplos de autonomia são muito importantes para compreensão plena dos comportamentos linguísticos e para a análise intrinsecamente linguística das várias construções e operações (ver \citealt{TagerFlusberg2006} para uma sistematização das características das diferentes patologias).

A capacidade de pensar a linguística clínica como uma área advém também dos progressos feitos pela própria teoria linguística. Hoje, temos muito mais conhecimento do que tínhamos apenas há menos de um século sobre as propriedades inerentes aos comportamentos linguísticos – estrutura interna dos sons, processos de formação de palavras, propriedades sintáticas, interpretações possíveis. Sabemos mais sobre os limites da variação e sobre o processo típico de aquisição e desenvolvimento das línguas. No âmbito da sintaxe, cito apenas dois exemplos. 

Temos hoje instrumentos de análise precisos que nos mostram a diferença entre verbos intransitivos \REF{ex:cap2ex12a} e inacusativos \REF{ex:cap2ex12b}, apesar de parecerem pertencer à mesma classe \citep{Perlmutter1978, Burzio1981}:

\ea
    \ea\label{ex:cap2ex12a} O João tossiu.
    \ex\label{ex:cap2ex12b} O João desmaiou.
    \z
\z

Este conhecimento permite-nos entender por que motivo uma categoria de verbos é mais dificilmente produzida do que a outra no agramatismo (\citet{Kegl1985}, \citet{Grodzinsky1995}, \citet{Beretta2001}  e \citet{Thompson2003} reportam maiores dificuldades na produção de verbos inacusativos do que na de verbos intransitivos).

Temos também um conhecimento mais aprofundado das diferenças estruturais subjacentes a interrogativas globais \REF{ex:cap2ex13a}, parciais \REF{ex:cap2ex13b} ou \emph{in-situ} \REF{ex:cap2ex13c}:

\ea
        \ea\label{ex:cap2ex13a} Leste o livro?
        \ex\label{ex:cap2ex13b} O que leste?
        \ex\label{ex:cap2ex13c} Leste o quê?
    \z
\z

As propriedades diferentes destas três formas de fazer perguntas per\-mi\-tem-nos hoje usar as interrogativas como forma de prever comportamentos diferenciados em contexto de perturbação da linguagem. Veremos, na próxima secção, como esta distinção é importante para a descrição do que acontece em diferentes quadros clínicos. 

Podemos ter, assim, avanços no contributo da linguística para a prática clínica, dado que sabemos muito mais hoje sobre as propriedades da língua, podendo afunilar a compreensão das diferentes perturbações, em vez de traçar diagnósticos gerais e globalizantes.

Conforme explicitado em \citet{Costa2016}, a Linguística Clínica tem cinco objetivos claros:


\begin{quote}
\begin{enumerate}
    \item[(a)] Descrição -- através dos instrumentos da linguística, é possível descrever os comportamentos linguísticos associados a diferentes patologias, tendo em vista diagnósticos diferenciais, isto é, uma caracterização das propriedades específicas associadas a cada subtipo de perturbação.
    \item[(b)] 	Clarificação -- decorre de uma descrição mais fina e detalhada uma clarificação dos aspetos envolvidos em cada tipo de perturbação.
    \item[(c)] 	Diagnóstico -- como foi referido, pretende-se que da interação entre a teoria linguística e o conhecimento clínico se obtenham diagnósticos para definir caminhos para a intervenção.
    \item[(d)] 	Avaliação -- para que o diagnóstico seja possível, é necessária a criação de instrumentos que o viabilizem. Esta é uma das áreas de maior carência para os profissionais, que pedem instrumentos de avaliação adequados à língua com que trabalham. Só com um bom conhecimento da língua, é possível a construção de instrumentos de avaliação precisos. Por outro lado, só com um conhecimento preciso do que está afetado em cada contexto de perturbação, é possível construir instrumentos de avaliação finos e apropriados. Há uma óbvia interdependência entre o diagnóstico e a avaliação, que requer muitos estudos exploratórios até ao apuramento dos instrumentos mais fiáveis.
    \item[(e)] 	Intervenção -- como será evidente, o objetivo último desta disciplina será contribuir para a reabilitação das perturbações linguísticas dos pacientes. Pretende-se, portanto, entender quais os contributos que a Linguística pode dar para a construção dos melhores instrumentos de intervenção terapêutica e para a monitorização dos resultados da intervenção.
\end{enumerate}
\end{quote}

Se, nesta relação em que a linguística informa a clínica e vice-versa, conseguimos encontrar contextos muito claros de perturbações estritamente sintáticas, torna-se possível perceber do que se fala quando se refere a sintaxe clínica. Tra\-ta-se do domínio da linguística clínica que se dedica à descrição, à clarificação, ao diagnóstico, à avaliação e à intervenção de perturbações sintáticas.

Para ilustrar instâncias do potencial da sintaxe clínica, apresentarei, de seguida, alguns exemplos bem conhecidos na literatura de referência que nos permitem singularizar e evidenciar cada um dos aspetos referidos, desde a descrição à intervenção.

Antes de o fazer, não quero deixar de referir aquela que me parece ser uma implicação muito prática para a formação daqueles que têm a competência de acompanhar crianças e adultos com perturbações da linguagem. Tradicionalmente, a formação de terapeutas da fala centra-se mais em aspetos globais das relações entre linguagem e comunicação, com uma atenção muito especial aos aspetos da fonologia e da fonética articulatória. Tudo o que é referido neste texto torna claro que é importante que a formação inclua domínios mais específicos, para que a decisão na prática de intervenção terapêutica seja suportada por um conhecimento mais fino do que a mera avaliação global de uma performance linguística que pode esconder características muito diferentes na sua especificidade.


\section{Da descrição à intervenção: Exemplos da sintaxe clínica}
\subsection{Descrição: Exemplo da PDL}
Para além de uma concentração inicial da terapia da fala em aspetos mais superficiais -- e por isso mais evidentes -- do comportamento linguístico nas questões ligadas com a produção/expressão e com a audição/perceção, assistimos, por vezes a descrições demasiado abrangentes das dificuldades linguísticas das crianças e dos adultos.

Tomemos como exemplo a classificação de dislexia. Uma dislexia é uma perturbação da leitura, podendo ter manifestações completamente distintas, desde a troca de letras, a migração para uma palavra seguinte, a incapacidade de ler se não acompanhar com leitura em voz alta, apenas para citar alguns exemplos. Há escolas, tanto em Portugal como noutros países, em que a taxa de alunos disléxicos é anormalmente superior àquilo que é a média normal de incidência da dislexia a nível global. Isto pode ter várias interpretações, podendo tratar-se de sobrediagnóstico ou de um diagnóstico demasiado global, em que várias dificuldades só se tornam difíceis apenas aquando da aprendizagem da leitura. Nos vários casos, estamos perante falhas na descrição dos comportamentos linguísticos.

Retomando as diferenças entre o comportamento de crianças com Síndrome de Down ou com PDL, referidas na secção anterior a propósito da correlação com outros problemas cognitivos, é importante entender como, ao nível sintático, estas duas perturbações podem distinguir-se \citep{Lely1996, Ring2005}. \citeauthor{Lely1996} estudou o comportamento de crianças e adolescentes com PDL relativamente à sua capacidade de interpretar passivas, mostrando que adolescentes com PDL interpretam passivas curtas \REF{ex:cap2ex14b} melhor do que passivas longas \REF{ex:cap2ex14a}. Nas passivas longas os adolescentes com PDL atribuem mais interpretações inversas do que as crianças com desenvolvimento típico \REF{ex:cap2ex15} e tendem privilegiar as passivas ambíguas na sua leitura estativa (adjetival) (\citealt{Guasti2002}). De acordo com \citet{Ring2005}, a diferença de relação entre idade e compreensão nos dois grupos de crianças e adolescentes parece evidenciar que, em PDL, há um atraso significativo (para referências adicionais sobre PDL sintática ver \textcitetv{chapters/12}. 

\ea
        \ea\label{ex:cap2ex14a} O João foi assaltado por um idoso.
        \ex\label{ex:cap2ex14b} O João foi assaltado.
    \z
\z
\ea\label{ex:cap2ex15} O João foi assaltado por um idoso.
    \ea {Interpretação:} Quem assaltou o João foi um idoso.
    \ex \emph{Interpretação inversa:} O João assaltou o idoso.
\z
\z

O mesmo não acontece nos dados de \citeauthor{Ring2005} para a Síndrome de Down. Com estes participantes, não se registam diferenças tão grandes entre passivas longas e curtas como nas crianças com PDL e a compreensão invertida de passivas foi significativamente maior. Os comportamentos diferenciados na sua incidência e prevalência indiciam que, para uma melhor descrição das diferentes etiologias, são necessárias descrições mais finas -- impossíveis sem um conhecimento sintático mais aprofundado.

Avançando um pouco para a dimensão explicativa, a diferença encontrada em \citet{Perovic2013} baseia-se na ideia de que, em síndrome de Down, há dificuldade na produção de cadeias argumentais, que regulam a criação de passivas e de ligação entre um grupo nominal e um pronome reflexo, como em \REF{ex:cap2ex16}:

\ea\label{ex:cap2ex16} O João\textsubscript{i} magoou-se\textsubscript{i}.
\z

Já em PDL, não se encontram dificuldades na compreensão de frases como \REF{ex:cap2ex16}, mas preserva-se a dificuldade na interpretação de passivas, o que significa que a natureza do problema é distinta.

Em \citet{Friedmann2006}, ainda a propósito de PDL, pode refor\-çar-se a importância de recorrer a conhecimento sintático de algum pormenor para se chegar a uma boa descrição do comportamento linguístico associado a cada perturbação. As autoras mostram que há uma semelhança clara, tanto ao nível das taxas de dificuldade na interpretação quanto na duração das dificuldades ao longo do desenvolvimento, entre orações relativas \REF{ex:cap2ex17a}, frases interrogativas \REF{ex:cap2ex17b} e topicalizações com ordem de palavras OSV \REF{ex:cap2ex17c}:

\ea\label{ex:cap2ex17}
        \ea\label{ex:cap2ex17a} Conheço a menina que a mãe viu.
        \ex\label{ex:cap2ex17b} Que menina é que a mãe viu?
        \ex\label{ex:cap2ex17c} Essa menina, a mãe viu.
    \z
\z

Numa descrição apressada, poderia concluir-se que as crianças com PDL têm dificuldade em interpretar frases relativas, interrogativas e topicalizações. A partir destas conclusões, bastaria incluir algumas destas estruturas em testes de avaliação das capacidades linguísticas e procurar diagnosticar.

Contudo, a realidade é menos simples. Como \citeauthor{Friedmann2006} mostram, em estruturas bastante semelhantes, as crianças com PDL não manifestam dificuldades:

\ea\label{ex:cap2ex18}
        \ea Conheço a menina que viu a mãe.
        \ex Que menina é que viu a mãe?
        \ex A menina viu a mãe.
    \z
\z

A grande diferença entre \REF{ex:cap2ex17} e \REF{ex:cap2ex18} reside na função sintática desempenhada pelo grupo nominal “a menina”. Nas frases em \REF{ex:cap2ex17}, esta expressão é o complemento direto. Nas frases em \REF{ex:cap2ex18}, é o sujeito. Isto significa que não estamos perante dificuldades com frases relativas, interrogativas e topicalizações de um modo geral, mas apenas com casos em que o complemento direto surge numa posição linearmente anterior e estruturalmente superior ao sujeito. Há, portanto, nesta perturbação uma sensibilidade à estrutura e à função sintática.\footnote{ Veja-se também menção a estas questões em \textcitetv{chapters/13}.}  

Este poderia ser o fim da história, em termos de descrição deste comportamento sintático. Contudo, as autoras mostram que, noutras estruturas semelhantes, em que o pronome relativo e o interrogativo têm a função de complemento direto, como em \REF{ex:cap2ex19}, as crianças com PDL não têm dificuldades:

\ea\label{ex:cap2ex19}
    \ea\label{ex:cap2ex19a} Conheço quem a mãe viu.
    \ex\label{ex:cap2ex19b} Quem é que a mãe viu.
    \z
\z

A principal diferença entre \xxref{ex:cap2ex19a}{ex:cap2ex19b} e \xxref{ex:cap2ex17a}{ex:cap2ex17b} está na presença de dois nomes (\emph{menina} e \emph{mãe}) ou de apenas um. A descrição torna-se ainda mais fina: as dificuldades de interpretação ocorrem quando há deslocação do complemento direto, mas apenas se este integra um nome.

Tendo aqui como objetivo dar um exemplo da importância da dscrição fina das construções para que os restantes objetivos da linguística clínica, não posso deixar de salientar o risco de descrições superficiais. Conforme se demonstrou, podem não ser suficientes para fazer a distinção entre diferentes patologias ou para construir instrumentos de avaliação que afiram as características específicas de cada população ou de cada patologia. Para chegarmos a este nível de descrição, são imprescindíveis os instrumentos de análise linguística.


\subsection{Clarificação: Um exemplo do agramatismo}
O agramatismo foi referido anteriormente como uma perturbação, tipicamente decorrente de uma lesão cerebral na área de Broca, que afeta predominantemente a componente sintática. Conforme no exemplo anterior, uma boa descrição do comportamento linguístico é essencial. 

Os doentes com comportamento que indicia agramatismo revelam dificuldades na construção de alguns contextos sintáticos específicos. Como explicitado em \citet{Friedmann1994}, evidenciam-se dificuldades na produção de frases subordinadas \REF{ex:cap2ex20a}, de interrogativas parciais (i.e., com palavras Qu-) \REF{ex:cap2ex20b}, mas não na produção de interrogativas totais (as de resposta sim/não) \REF{ex:cap2ex20c}:

\ea
        \ea\label{ex:cap2ex20a} O Pedro disse \textbf{que} está a chover.
        \ex\label{ex:cap2ex20b} Que menina é que a mãe viu?
        \ex\label{ex:cap2ex20c} Está a chover?
    \z
\z

Continuando a mera descrição de diferentes incapacidades, encontramos aqui alguma semelhança com PDL, sobretudo nas interrogativas. Conforme descrito, crianças com PDL e agramáticos têm ambos dificuldades na construção e compreensão de frases interrogativas do tipo \REF{ex:cap2ex20b}. Contudo, \citeauthor{Friedmann1994} mostram que esta similitude é apenas superficial. Se, como vimos para PDL, não se pode dizer que haja uma dificuldade global com interrogativas, também no caso do agramatismo uma descrição nesse sentido estaria errada. \REF{ex:cap2ex20c} é uma frase interrogativa, mas não uma interrogativa parcial. Por outro lado, os autores mostram que os agramáticos não fazem distinção entre interrogativas de sujeito \REF{ex:cap2ex21a} ou de objeto \REF{ex:cap2ex21b}, o mesmo acontecendo com relativas \xxref{ex:cap2ex22a}{ex:cap2ex22b}:

\ea
        \ea\label{ex:cap2ex21a} Que idoso é que assaltou o Pedro?
        \ex\label{ex:cap2ex21b}Que idoso é que o Pedro assaltou?
    \z
\z
\ea
        \ea\label{ex:cap2ex22a} Conheço o idoso que assaltou o Pedro.
        \ex\label{ex:cap2ex22b} Conheço o idoso que o Pedro assaltou.
    \z
\z

De igual modo, \citeauthor{Friedmann1994} mostram que, nos agramáticos, a incapacidade não diferencia interrogativas e relativas com ou sem nomes expressos (cf. \REF{ex:cap2ex17} vs. \REF{ex:cap2ex19}), como acontece na PDL.

Para os agramáticos, todas estas frases estão prejudicadas, sobretudo na produção. Se a descrição das diferenças é, por si só, uma conclusão interessante, importa saber mais. E aqui, para além dos instrumentos de descrição e análise linguística, é crucial recorrer a modelos teóricos explicativos. Por outras palavras, responder à pergunta \emph{o que explica estas diferenças}? pode ser particularmente interessante. A explicação apresentada por \citet{Friedmann1994} recorre a instrumentos de análise sintática, em particular às hipóteses sobre configuração estrutural das frases e sobre estabelecimento de cadeias referenciais. Para estabelecermos essa diferença, consideremos alguns exemplos simples. Quando falamos de estrutura sintática, referimo-nos à constatação de que uma frase não se constrói por mera concatenação de palavras de forma linear. Pelo contrário, a relação entre os constituintes de uma frase é hierárquica. Um dos mais claros exemplos vem da ambiguidade estrutural (ver capítulo de \citetv{chapters/12}), em frases como \REF{ex:cap2ex23}:

\ea\label{ex:cap2ex23} O Pedro viu o filho do ator que estava a brincar.
\z

A frase \REF{ex:cap2ex23}, analisada para o português europeu em \citet{Brito1995}, \citet{Grillo2014} e \citet{Fernandes2020}, é ambígua. Embora haja interpretações preferenciais, o significado pode ser que \emph{o ator estava a brincar} ou que \emph{o filho estava a brincar}. Não há qualquer ambiguidade em nenhuma das palavras da oração relativa, apenas no seu antecedente, isto é, na relação que se estabelece entre antecedentes, captada pelas representações abaixo:

\ea
        \ea\label{ex:cap2ex24a} \emph{interpretação em que o ator brinca:}\\
         Eu vi [o filho d[o ator [que estava a brincar]]]
        \ex\label{ex:cap2ex24b} \emph{interpretação em que o filho brinca:}\\
         Eu vi [o filho [do ator] que estava a brincar]]
    \z
\z

A diferença entre as reações hierárquicas capta a diferença de interpretação, na medida em que em \REF{ex:cap2ex24a} a oração relativa forma constituinte dependente de \emph{ator}, enquanto em \REF{ex:cap2ex24b} a relativa está dependente do nome \emph{filho}.

Ao reconhecer-se a existência de relações hierárquicas, a teoria sintática tem-se dedicado, desde o início dos anos 80 (ver \citealt{Rizzi2016} para uma síntese da evolução do conhecimento nesta área), a pesquisar a identificação e categorização de cada um dos constituintes que compõem um enunciado.

A título de exemplo, percebe-se que uma frase introduzida por uma conjunção subordinativa não é intersubstituível com uma frase raiz:

\ea
    \ea[]{[Está a chover].}\label{ex:cap2ex25a}
    \ex[]{Eu disse [que está a chover].}\label{ex:cap2ex25b}
    \ex[*]{[Que está a chover].}\label{ex:cap2ex25c}
    \ex[*]{Eu disse [está a chover].}\label{ex:cap2ex25d}
    \z
\z

Isto significa que, apesar de estarmos perante dois tipos de frase com enormes semelhanças entre si, elas não podem ocupar a mesma posição numa frase. Note-se que, a substituição da subordinada completiva em \REF{ex:cap2ex25b} opera com naturalidade com um grupo nominal:

\ea\label{ex:cap2ex26} Eu disse [essas mentiras].
\z

Há, contudo, muitos contextos em que se pode mostrar que um constituinte como a subordinada completiva não é também substituível por um grupo nominal:

\ea\label{ex:cap2ex27}
        \ea[]{Eu não acredito [que está a chover].}
        \ex[*]{Eu não acredito [essas mentiras].}
    \z
\z

Como em qualquer processo taxonómico, também na teoria sintática se identifica como igual o que tem comportamento igual e de forma diferente o que não tem o mesmo argumento. Nestes exemplos vimos que a subordinada não se comporta como uma frase raiz. É sempre dependente, por isso estes constituintes têm designações diferentes:

\ea\label{ex:cap2ex28}
        \ea {[está a chover]} = TP (grupo temporal)
        \ex {[que está a chover]} = CP (grupo (introduzido por) complementador)
        \ex {[essas mentiras]} = NP (grupo nominal)
    \z
\z

Esta categorização tem permitido uma descrição cada vez mais fina das construções sintáticas e, não cabendo aqui uma resenha das propriedades dos constituintes, desde \citet{Chomsky1986} que se considera que o nível CP é ativado para a construção de frases subordinadas e de frases raiz com constituintes deslocados. De acordo com esta proposta, podemos resumir a análise das diferentes estruturas, conforme explicitado em \REF{ex:cap2ex29} e na \figref{fig:cap2tree}:
%\REF{ex:cap2ex30}:

\ea\label{ex:cap2ex29}
   \ea {Interrogativas parciais de objeto (i) e de sujeito (ii):}
    \ea {[\textsubscript{CP} Quem é que [\textsubscript{TP} tu viste?]]}
    \ex {[\textsubscript{CP} Quem é que [\textsubscript{TP} viu a menina?]]}
    \z
    \ex  {Relativas de objeto (i) e de sujeito (ii):}
        \ea Eu conheço a menina [\textsubscript{CP} que [\textsubscript{TP} a Maria viu.]]. 
        \ex Eu conheço a menina [\textsubscript{CP} que [\textsubscript{TP} viu a Maria.]].
    \z
    \ex  {Interrogativas totais:}\\
     {[\textsubscript{TP} Tu viste a menina?]}
    \ex  {Oração subordinada:}\\
     Eu disse [\textsubscript{CP} que [\textsubscript{TP} está a chover]]
    \z
\z

\citet{Friedmann1997} oferecem uma explicação simples para o comportamento dos agramáticos, através da \emph{Tree Pruning Hypothesis} (poderia cha\-mar-se-lhe em português algo como a hipótese de poda das árvores). De acordo com esta análise, o défice do agramatismo é estrutural, explicando-se pela dificuldade de construção de estrutura sintática, de acordo com a hierarquia de categorias que explica o funcionamento da estrutura.

De forma muito interessante, os autores mostram relações de dependência entre comportamentos sintáticos, que se correlacionam com graus de severidade do agramatismo. Assim, no agramatismo severo, deteta-se não apenas dificuldade em produzir estruturas que envolvem CP, mas também dificuldades na flexão verbal (tempo e/ou concordância), expressas pela categoria TP.\footnote{Note-se que, desde \citet{Chomsky1986}, se assume que uma frase raiz é uma projeção da categoria tempo, já que existe um contraste de gramaticalidade entre frases raiz com ou sem morfologia flexional:

\ea
    \ea[]{O lobo comeu o cordeiro.}
    \ex[*]{O lobo ter comido o cordeiro.}
    \z
\z
} Nos pacientes com um grau de menor severidade, podem encontrar-se casos de dificuldade de produção de estruturas que envolvem CP (relativas, interrogativas, topicalizações), mas não sem desvios na flexão verbal. Uma descoberta fundamental para entender as explicações por trás destes comportamentos está na ausência de pacientes que manifestem apenas problemas nas estruturas em que CP está envolvido sem que haja problemas na flexão verbal.

Este comportamento e esta correlação entre CP e TP explicam-se facilmente através da \emph{Tree Pruning Hypothesis}. De acordo com a proposta, os agramáticos têm dificuldades no acesso a níveis de estrutura, que podem ser caracterizados como cortes na estrutura sintática. Na figura seguinte, extraída de \citet{Friedmann1997}, pode visualizar-se de forma mais clara a representação da hipótese.

%\ea\label{ex:cap2ex30} Tree Pruning Hypothesis:

%    \begin{figure}[H]
%    \includegraphics[width=6cm, height=5cm]{figures/Imagem36.png}
%    \label{fig:my_label}
%\end{figure}
\begin{figure}
\begin{forest}
[CP []
    [C$\prime$
        [C]
        [TP, name=tp
            []
            [T$\prime$
                [T$^0$]
                [NegP, name=negp
                    []
                    [AgrP
                        []
                        [Agr$\prime$
                            [Agr$^0$]
                            [VP, name=vp
                                [NP]
                                [V$\prime$
                                    [V]
                                    [NP]
                                ]
                            ]
                        ]
                    ]
                ]
            ]
        ]
    ]
]
\draw([xshift=-.8cm, yshift=-.8cm]tp) arc[start angle=180,end angle=60,radius=2cm] node [pos=.8, above] (c) {only C impaired};
\draw([xshift=-.8cm, yshift=-.8cm]negp) arc[start angle=180,end angle=60,radius=2cm] node [pos=.8, above] (ct) {C and T impaired};
\draw([xshift=-.8cm, yshift=-.8cm]vp) arc[start angle=180,end angle=60,radius=2cm] node [pos=.9, above] (ctagr) {C, T and Agr impaired};
\end{forest}
\caption{Tree Pruning Hypothesis}
\label{fig:cap2tree}
\end{figure}
%\z

Ignorando, para a presente discussão, a diferenciação de maior pormenor entre a categoria tempo, negação e concordância, percebe-se através desta representação as consequências da diferença entre os pontos de “corte” na estrutura. Com efeito, uma consequência da existência de diferentes camadas de estrutura que se relacionam hierarquicamente, ao não ter acesso a CP (no corte superior), posso ainda ter acesso a TP. Mas se o corte se der abaixo de TP, toda a estrutura acima é afetada. Não há, pois, nesta hipótese como prever o comportamento inexistente: dificuldades ao nível de CP sem dificuldades ao nível de T.

Podemos sintetizar, assim, as predições da hipótese (\tabref{tab:cap2tab1}).



\begin{Tabela}
\caption {Predições da hipótese}
\label{tab:cap2tab1}
\begin{tabularx}{\textwidth}{Qll}
\lsptoprule
 \textbf{Nível da estrutura afetado} & \textbf{Estrutura(s) {afetada(s)}} & \textbf{Estrutura não {afetada}}\\
\midrule
 C & Interrogativas {parciais} & Flexão verbal\\
 & Subordinadas &\\
\tablevspace
 C+T & Interrogativas {parciais}&\\
 & Subordinadas &\\
 & Flexão verbal &\\
\tablevspace
 *T – C & Flexão verbal & Interrogativas {parciais}\\
 \emph{(comportamento não  detetado e linguisticamente não predito} && Subordinadas\\
% \addlinespace
 \lspbottomrule
\end{tabularx}

\end{Tabela}

Esta importância da análise linguística na justificação da descrição encontrada é um instrumento precioso para se estabelecerem diagnósticos diferenciais. Quan\-do comparamos o comportamento descrito nos casos de PDL e nos casos de agramatismo, podemos entender o quanto os pormenores fazem a diferença (\tabref{tab:cap2tab2}).



\begin{Tabela}
\caption{Estruturas linguísticas afetadas em casos de PDL.}
\label{tab:cap2tab2}
\begin{tabularx}{\textwidth}{Qll}
\lsptoprule
\textbf{Estrutura linguística afetada} & \textbf{PDL} & \textbf{Agramatismo}\\
\midrule
Subordinação& Não & Sim\\ 
\tablevspace
Interrogativas parciais&&\\
\tablevspace
Interrogativas parciais/Relativas de sujeito& Não & Sim\\ 
\tablevspace
Interrogativas parciais/Relativas de objeto com nome expresso & Sim & Sim\\ 
\tablevspace
Interrogativas parciais de objeto sem nome expresso & Não & Sim\\
\tablevspace
Interrogativas totais& Não & Não\\
\lspbottomrule
\end{tabularx}

\end{Tabela}

A diferença reside, pois, nas operações sintáticas envolvidas. Na PDL, as perturbações manifestadas resultam de um défice em estruturas em que a relação de dependência entre um constituinte deslocado e o seu lugar de origem não se estabelece. No agramatismo, deparamo-nos com um défice na capacidade de aceder à própria estrutura sintática. 

\largerpage
Estes casos foram escolhidos como exemplo, dado que ilustram como uma análise superficial e sem os instrumentos da teoria linguística tornaria impossível a distinção estabelecida entre os dois tipos de perturbação.


\subsection{Diagnóstico: A importância dos diagnósticos diferenciais}

Nos dois exemplos anteriores, vimos a importância da teoria linguística e da análise linguística para uma descrição e interpretação precisa de cada uma das perturbações sintáticas.

Como é evidente, quanto melhores forem a descrição e análise, melhor será o diagnóstico, que não será feito numa perspetiva globalizante, mas podendo procurar características muito específicas. Poderia continuar a explorar a diferença entre perturbações, para mostrar a relevância da boa análise sintática, mas parece-me mais relevante, para explicar mais precisamente a importância dos diagnósticos diferenciais e o contributo do conhecimento sintático, referir-me a dois projetos europeus recentes que colocaram a tónica na caracterização específica das perturbações em diferentes populações.

O projeto \emph{COST A33: Cross-linguisticaly Robust Stages of Children’s Linguistic Performance}\footnote{Para mais informação sobre o projeto COST A33, ver \url{https://www.cost.eu/actions/A33/\#tabs|Name:overview} (acedido a 30 de novembro de 2020).} visa identificar um conjunto de estruturas linguísticas que, no desenvolvimento da competência das crianças, sejam imunes à variação interlinguística. Este objetivo parte de uma constatação. Tendo em conta a enorme mobilidade entre cidadãos, que é a normalidade em territórios marcadamente multilingues, como a Europa, há muitas crianças que crescem bilingues, diferindo a língua falada em casa da língua oficial e falada no resto do território. Conforme identificado na descrição deste projeto, muitas destas crianças são subdiagnosticadas para perturbações do desenvolvimento linguístico. Isto deriva de, em contexto de avaliação e diagnóstico, poder atribuir-se as dificuldades manifestadas não a uma perturbação, mas sim à dominância da outra língua que está a ser adquirida.

Com esta preocupação em mente, o projeto visou encontrar características ou comportamentos linguísticos cujo desenvolvimento esteja estabilizado aos 5 anos em todas as línguas estudadas. O projeto fez uma análise sistemática do comportamento linguístico de algumas estruturas em 27 línguas. Procurou-se, portanto, encontrar marcadores de desenvolvimento linguístico universais. Mal comparado, tal como o conseguir sentar-se a uma determinada idade é um indicador de desenvolvimento saudável das crianças em qualquer cultura ou país do mundo, também algumas estruturas linguísticas têm esse potencial de ser usadas como instrumento de diagnóstico. Por ser a sintaxe a componente da gramática menos dependente de fatores externos, como a sonoridade, os sons específicos da fala ou a interação com dimensões que envolvem o conhecimento do mundo, estudaram-se no projeto construções sintáticas, tendo sido possível identificar-se várias áreas de robustez, das quais exemplifico três:

\ea\label{ex:cap2ex31} Passivas longas:\\
    A menina foi assaltada por uma idosa.
\z

Em todas as línguas testadas, aos 5 anos, as crianças já não revelavam dificuldades na interpretação de passivas longas. Assim, uma eventual dificuldade poderá ser interpretada como um marcador de perturbação.

\ea\label{ex:cap2ex32} Produção de pronomes clíticos:\\
    Ele feriu-se.
\z

Em todas as línguas testadas, aos 5 anos, as crianças já não manifestam dificuldade na produção de pronomes clíticos, nem quanto à sua realização, nem quanto à sua posição pré ou pós-verbal (proclítica ou enclítica). Percebeu-se que, para encontrar a robustez deste marcador, é necessário controlar o tipo de pronome usado (reflexo ou não reflexo) e a coexistência com construções de objeto nulo, como as existentes em português \citep{Varlokosta2015}. Assim, se uma criança de 5 anos não produzir um pronome clítico, independentemente da(s) língua(s) que fala, estamos perante um indicador de perturbação da linguagem.

% \subsection{Interrogativas e relativas de sujeito}

Conforme descrito, com exceção de contexto de agramatismo, os vários trabalhos desenvolvidos mostram que as relativas e interrogativas de objeto com nome expresso \REF{ex:cap2ex33} são adquiridas mais tardiamente e com uma dificuldade persistente em quadros de perturbação:

\ea\label{ex:cap2ex33} 
    \ea Esta é a \textbf{menina} que a mãe viu.
    \ex Que \textbf{menina} é que a mãe viu?
    \z
\z

Nas 27 línguas testadas, percebeu-se que a dificuldade com as construções de objeto existia, mas um resultado também importante foi a constatação de que, em nenhuma das línguas, as crianças de 5 anos exibiam dificuldades na interpretação de relativas e interrogativas de sujeito, como as que se ilustram em \REF{ex:cap2ex34}:

\ea\label{ex:cap2ex34}
        \ea Esta é a menina que viu a mãe.
        \ex Que menina viu a mãe?
    \z
\z

Pode, assim, considerar-se que, se uma criança avaliada aos 5 anos tiver dificuldades na compreensão de enunciados como \REF{ex:cap2ex34}, podemos estar perante um caso de perturbação, independentemente da sua língua materna.

Percebe-se, assim, a relevância da sintaxe clínica para o diagnóstico. O estudo da produção e interpretação de estruturas sintáticas permite não apenas diagnósticos diferenciados, mas também a identificação de marcadores robustos de desenvolvimento, que possam ser utilizados de forma universal, podendo-se perspetivar a capacidade de acompanhamento de todas as crianças, independentemente da língua que falam.

Um outro projeto de enorme relevância resulta das conclusões deste que acabo de descrever. No projeto \emph{COST IS0804 Language Impairment in a Multilingual Society: Linguistic Patterns and the  Road to Assessment}\footnote{Para mais informação sobre este projeto e os seus resultados, consultar \url{https://www.bi-sli.org/} (acedido a 29 de novembro de 2020).} deu-se um passo adiante: entender marcas específicas do desenvolvimento linguístico cruzando o fator perturbação com o monolinguismo.

O foco foi, pois, fazer comparações sistemáticas entre os seguintes grupos:

\begin{enumerate}
    \item [a)] Crianças monolingues com desenvolvimento típico;
    \item[b)] Crianças monolingues com PDL;
    \item[c)] Crianças bilingues com desenvolvimento típico;
    \item[d)] Crianças bilingues com PDL.
\end{enumerate}

Alguns dos pressupostos deste projeto estão descritos em \citet{ArmonLotem2010}. De novo, foram exploradas as habilidades sintáticas das crianças. O que se torna, novamente mais relevante é a capacidade que temos de, através da análise minuciosa de estruturas linguísticas, detetar comportamentos que diferenciam a perturbação do desenvolvimento normal em contexto monolingue e bilingue. O impacto social deste tipo de investigação é extraordinário, já que, como podemos imaginar, se na descrição do projeto COST A33, se parte da constatação de que há crianças subdiagnosticadas por serem bilingues, com este projeto percebe-se a importância de constituir grupos de análise diferentes para juntarmos, aos marcadores de robustez, conjuntos de marcadores de dificuldades específicas em diferentes contextos. Como o próprio nome do projeto indica, só após uma boa descrição de estruturas, análise e diagnóstico, é possível uma melhor avaliação de cada caso.


\subsection{Avaliação: A necessidade do rastreio}
Como é evidente, não é possível diagnosticar sem que se desenvolvam instrumentos precisos de avaliação. Se nos centrarmos na PDL, num primeiro momento foi importante sinalizar a sua existência, isolando-a de outros défices cognitivos. Para tal, foi necessário fazerem-se observações dos dados em crianças avaliadas que mostrassem uma sólida assimetria entre tarefas cognitivas verbais e não-verbais \citep{Bishop1979}.

\citet{Friedmann2008} apresentam evidências claras para a existência de subtipos de PDL:

\begin{enumerate}
    \item [a)] \emph{PDL fonológica:}
    \item[] Neste subtipo são apenas identificadas dificuldades a nível da produção de sons e de estruturas fonológicas, sem que outras componentes se manifestem comprometidas.
    \item[b)] \emph{PDL pragmática:}
    \item[] Neste caso, não há compromisso de outras competências para além das discursivas, no que diz respeito ao cumprimento de regras conversacionais de quantidade, relevância e informação \citep{Grice1975}. Nestes casos, a fronteira entre esta caracterização e vários quadros possíveis no espectro do autismo é controversa. Concorre, sobretudo, para confirmar aqui a seletividade das perturbações de desenvolvimento linguístico.
    \item[c)] \emph{PDL lexical:}
    \item[] Este é o caso em que se encontra apenas perturbado o acesso a palavras ou a capacidade de compreender palavras específicas.
    \item[d)] \emph{PDL sintática:}
    \item[] Neste contexto, apenas a capacidade de produção e compreensão de algumas estruturas sintáticas é afetada, sem que haja outras evidências de perturbação nem no acesso a palavras, nem nos comportamentos fonético-fonológicos, nem na interação discursiva.
\end{enumerate}

Muita investigação dedica-se a encontrar preditores de PDL, como por exemplo a memória de curto prazo ou a resposta a atividades de função executiva não verbal \citep{Lum2010, Lum2012, Henry2012}. Estas correlações que manifestam com potencial preditor são importantes em duas dimensões: por um lado, permitem identificar crianças em risco de PDL e antecipar os cuidados, vigilância e eventual intervenção ao longo do seu desenvolvimento; por outro lado, abrem janelas para um entendimento sobre capacidades cognitivas gerais e PDL. Estas correlações não permitem, contudo, uma avaliação de subtipos de PDL. Ora, sendo reconhecido que a PDL sintática existe, é fundamental que os instrumentos de avaliação sejam suficientemente capazes de a detetar. Acresce que a PDL sintática, dada a natureza desta componente na organização do conhecimento linguístico, não tem uma manifestação clara, dado que as crianças ou não produzem determinadas estruturas – algo que nunca é saliente -- ou não as interpretam bem. Por exemplo, numa frase como \REF{ex:cap2ex35}, uma criança com PDL sintática não sabe se está a ser perguntado quem comeu quem:

\ea\label{ex:cap2ex35} Que porquinhos é que o lobo comeu?
\z

Se nos recordarmos que muitas instruções, enunciados e tarefas em contexto escolar têm exatamente o formato de \REF{ex:cap2ex35}, é fácil que a PDL seja entendida apenas como um problema de atenção, o que pode levar a um diagnóstico errado.

A solução passará, pois, por criar testes de rastreio precisos e bem informados pelas especificidades das diferentes construções. Esta é uma tarefa em que ainda não há muitos materiais (note-se a importância da adaptação para o português do teste Schlichting para a avaliação de componente sintática, Teste de Avaliação da Competência Sintática (Sin:TACS, \citealp{Vieira2018}). Esta não é uma tarefa simples, já que a especificidade e o detalhe necessários para se construírem testes que permitam avaliações de apenas algumas componentes conflituam com o número de itens por teste. No caso da PDL sintática, só com um conjunto significativo de tokens da mesma construção se podem extrair conclusões, tendo em conta que a dificuldade deteta-se em comportamentos opcionais e nem sempre categóricos.\footnote{Veja-se o capítulo 4.1. de \textcitetv{chapters/14} onde é mencionada a plataforma MABILIN (Módulos de Avaliação de Habilidades Linguísticas) com uma avaliação sintática concebida originalmente em PB e adaptada  para o PE no Centro de Linguística da Universidade Nova de Lisboa. O módulo 1 do MABILIN é especificamente dirigido à avaliação da compreensão de estruturas consideradas de alto custo que, caracteristicamente, se encontram comprometidas em crianças com PDL.} Por exemplo, uma criança com PDL interpreta de forma aleatória uma frase como \REF{ex:cap2ex35}; se um instrumento de avaliação contiver apenas dois itens deste tipo, pode acontecer que os resultados não revelem a arbitrariedade. Num teste com 20 itens do mesmo tipo, já seria possível a distinção, mas a fadiga de aplicação torna-o inviável. Esta compatibilização entre os resultados de trabalho experimental e a sua transposição para instrumentos de rastreio e avaliação é tudo menos óbvia. Por isso, é necessária mais investigação e um aprofundamento cada vez maior da relação entre preditores e desempenhos.


\subsection{Intervenção: Um caminho a trilhar}
É sobretudo na área da intervenção que ainda há muito por fazer, nestas pontes que se estabelecem entre a sintaxe teórica e a prática clínica. \citet{Law2004} mostram que a terapia sistemática revela efeitos nas dificuldades fonológicas e lexicais, mas os resultados não são tão evidentes nas dificuldades sintáticas.

A evidência que existe aponta, contudo, para alguns resultados interessantes, dos quais saliento apenas um. \citet{Roth1984} mostra que é possível melhorar a produção e compreensão de orações relativas através de tarefas que mobilizam o conhecimento implícito, como por exemplo a conversão da relativa em duas frases simples coordenadas, alternando com a produção de relativas para a mesma situação, como em \REF{ex:cap2ex36}:

\ea\label{ex:cap2ex36}
        \ea A menina que a mãe viu está ali.
        \ex A mãe viu a menina e a menina está ali.
    \z
\z

As duas frases em \REF{ex:cap2ex36} podem ser usadas em contextos semelhantes. A sua produção em conjunto com a apresentação de imagens permite a melhoria dos desempenhos.

Sob ponto de vista da articulação entre práticas de intervenção e teoria linguística, são particularmente interessantes os estudos que mostram que a explicitação do conhecimento envolvido nas construções mais afetadas permite que se encontrem melhorias significativas no desempenho (\citet{Nickels1991, Schwartz1994, Shapiro2006, Levy2009}, entre outros). Estes autores mostram que a intervenção assente em atividades metalinguísticas revela grandes benefícios. Isto é importante, dado que acrescenta evidência ao papel que a consciência linguística (enquanto estado intermédio entre o conhecimento implícito e conhecimento explícito) pode desempenhar na performance linguística, seja em contexto de uso da língua em contextos diferenciados \citep{Duarte2008}, seja, como aqui se mostra em contexto terapêutico para melhorar a eficácia da intervenção.


\section{Conclusão}
Constituindo-se como uma apresentação sem pretensões de cobertura exaustiva do muito trabalho existente na área da sintaxe clínica, procurou-se mostrar o quanto a teoria linguística traz ganhos para um trabalho mais especializado e com maior potencialidade em todas as dimensões envolvidas na prática (avaliação, clarificação, diagnóstico e intervenção). Por este motivo, uma boa base científica é fundamental na formação de terapeutas da fala, que terão de ser capazes de olhar para a produção e compreensão da língua nas suas várias componentes.

Esta é uma área em que podemos antecipar desenvolvimentos importantes, dado que há cada vez mais dados sobre a atividade neurológica envolvida nas tarefas, sendo possível que se abram novas perspetivas sobre a relação entre linguagem e cognição.


{\sloppy\printbibliography[heading=subbibliography,notkeyword=this]}
\end{document}
