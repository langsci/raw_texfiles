\documentclass[output=paper,colorlinks,citecolor=brown,booklanguage=portuguese]{langscibook} 
\ChapterDOI{10.5281/zenodo.7233237}
\title{Subordinação e complexidade sintática no desenvolvimento atípico: uma área central na sintaxe clínica}
\author{Ana Lúcia Santos \affiliation{Universidade de Lisboa, Faculdade de Letras, Centro de Linguística da Universidade de Lisboa} and  Alexandrina Martins \affiliation{Universidade de Lisboa, Faculdade de Letras, Centro de Linguística da Universidade de Lisboa; Universidade de Aveiro}}

\abstract{Neste capítulo, mostramos como a subordinação, entendida também como medida de complexidade sintática, é relevante na caracterização do desenvolvimento linguístico atípico, particularmente em dois grupos de indivíduos, os que apresentam diagnóstico de Perturbação do Desenvolvimento da Linguagem (PDL) e os que apresentam diagnóstico de Perturbação do Espetro do Autismo (PEA). Assim, começaremos por sintetizar alguns factos e marcos relevantes no desenvolvimento linguístico típico, passando depois a uma síntese de trabalhos que permitem o reconhecimento, em crianças com PDL e PEA, de alguns aspetos do desenvolvimento de estruturas de subordinação, registando-se características da resposta a tarefas de produção e de compreensão. A comparação com o percurso de desenvolvimento típico centrar-se-á, por um lado, nas faixas etárias em que se observam os diferentes comportamentos e, por outro, em diferenças qualitativas nas respostas a diferentes tarefas.

\textbf{Palavras-chave}: subordinação; completivas; relativas; Perturbação do Desenvolvimento da Linguagem (PDL); Perturbação do Espetro do Autismo (PEA)}

\IfFileExists{../localcommands.tex}{
   \addbibresource{../localbibliography.bib}
   \usepackage{langsci-optional}
\usepackage{langsci-gb4e}
\usepackage{langsci-lgr}

\usepackage{listings}
\lstset{basicstyle=\ttfamily,tabsize=2,breaklines=true}

%added by author
% \usepackage{tipa}
\usepackage{multirow}
\graphicspath{{figures/}}
\usepackage{langsci-branding}

   
\newcommand{\sent}{\enumsentence}
\newcommand{\sents}{\eenumsentence}
\let\citeasnoun\citet

\renewcommand{\lsCoverTitleFont}[1]{\sffamily\addfontfeatures{Scale=MatchUppercase}\fontsize{44pt}{16mm}\selectfont #1}
  
   %% hyphenation points for line breaks
%% Normally, automatic hyphenation in LaTeX is very good
%% If a word is mis-hyphenated, add it to this file
%%
%% add information to TeX file before \begin{document} with:
%% %% hyphenation points for line breaks
%% Normally, automatic hyphenation in LaTeX is very good
%% If a word is mis-hyphenated, add it to this file
%%
%% add information to TeX file before \begin{document} with:
%% %% hyphenation points for line breaks
%% Normally, automatic hyphenation in LaTeX is very good
%% If a word is mis-hyphenated, add it to this file
%%
%% add information to TeX file before \begin{document} with:
%% \include{localhyphenation}
\hyphenation{
affri-ca-te
affri-ca-tes
an-no-tated
com-ple-ments
com-po-si-tio-na-li-ty
non-com-po-si-tio-na-li-ty
Gon-zá-lez
out-side
Ri-chárd
se-man-tics
STREU-SLE
Tie-de-mann
}
\hyphenation{
affri-ca-te
affri-ca-tes
an-no-tated
com-ple-ments
com-po-si-tio-na-li-ty
non-com-po-si-tio-na-li-ty
Gon-zá-lez
out-side
Ri-chárd
se-man-tics
STREU-SLE
Tie-de-mann
}
\hyphenation{
affri-ca-te
affri-ca-tes
an-no-tated
com-ple-ments
com-po-si-tio-na-li-ty
non-com-po-si-tio-na-li-ty
Gon-zá-lez
out-side
Ri-chárd
se-man-tics
STREU-SLE
Tie-de-mann
}
   \boolfalse{bookcompile}
   \togglepaper[13]%%chapternumber
}{}

\shorttitlerunninghead{Subordinação e complexidade sintática no desenvolvimento atípico}
\begin{document}
\shorttitlerunninghead{Subordinação e complexidade sintática no desenvolvimento atípico}
\maketitle

\section{A relevância da subordinação na avaliação do desenvolvimento linguístico}\label{sec:parteumcap13}
A observação do discurso espontâneo das crianças mostra-nos que, depois de um período inicial em que a criança passa da produção de enunciados com apenas uma palavra à combinação de mais de uma palavra, maioritariamente casos de combinação de duas palavras, observamos uma grande expansão dos enunciados que as crianças conseguem produzir entre os 2;0 e os 3;0 anos.\footnote{A norma adotada na escrita deste capítulo foi a do português europeu.}
% \footnote{Na sequência de restrições decorrentes do template usado para publicação na editora Language Science Press, os capítulos que integram o volume surgem com a referência [this volume], em inglês.}
Essa expansão afeta a extensão e, ao mesmo tempo, a complexidade dos enunciados produzidos. Uma marca clássica de complexidade em termos sintáticos é a presença de subordinação. A subordinação, que é possível graças à propriedade da recursividade que caracteriza a linguagem humana, permite o encaixe de um domínio oracional num outro domínio oracional.\footnote{Para uma discussão tecnicamente mais profunda da forma como o encaixe de orações determina o nível de complexidade e sobre a forma como isso se poderá refletir na ordem de emergência de diferentes estruturas em português, veja-se \citet{Soares2006}.}

Os exemplos que apresentamos de seguida correspondem a enunciados produzidos por crianças nesta faixa etária e que são exemplos da emergência de subordinação que tão bem caracteriza o desenvolvimento linguístico neste período (sobre diferentes subtipos de subordinação normalmente considerados em português, veja-se \citetv{chapters/12}). Estes exemplos são extraídos de um \emph{corpus} que reúne amostras longitudinais da produção espontânea de três crianças, \emph{corpus} SANTOS, publicamente disponível na base de dados CHILDES.\footnote{\url{https://sla.talkbank.org/TBB/childes/Romance/Portuguese/Santos} Quando os dados citados não têm outra indicação de fonte, são citados a partir deste \emph{corpus}. Os grafemas entre parênteses correspondem a segmentos que foram omitidos na produção das crianças.} Em (1) a (4), encontramos exemplos da produção de orações completivas. Os enunciados em (1) e (2) contêm orações completivas infinitivas (assinaladas entre parênteses retos), selecionadas como complemento do verbo \emph{querer}, e são o tipo de oração completiva que mais precocemente encontramos nestes dados. Já os casos em (3) e (4) são casos de orações completivas finitas, que emergem um pouco depois. Nestes casos, temos já um complementador lexicalmente realizado (\emph{que}), que assinala de forma clara a subordinação, questão a que voltaremos. Além disso, estes dois casos exemplificam também casos classicamente observáveis nestas primeiras produções: completivas finitas (de indicativo) selecionadas como complemento do verbo \emph{achar} e, só mais tardiamente e aparecendo muito raramente na produção espontânea, completivas finitas exibindo conjuntivo que são selecionadas como complemento do verbo \emph{querer}. Mais raras na produção espontânea em geral e mais tardias também em termos de aquisição (relativamente às completivas introduzidas por \emph{que}) são as completivas finitas introduzidas pelo complementador \emph{se} - para informação mais completa sobre o conjunto destas estruturas no percurso de desenvolvimento típico em português europeu (PE), veja-se \citet{Soares2006} e a síntese em \citet{Santos2017a}. 

\ea
    \label{ex:13:1} a (I)nê(s) que(r) [ i(r) pa(ra) casa ]. \hfill (INI, 1;10.29)
    \z
    \ea
    \label{ex:13:2} a foca qu(er)ia [ sai(r) ].	\hfill   			(TOM, 2;1.7)
    \z
    \ea
    \label{ex:13:3} acho [ que é difí(cil) ]. 		\hfill 		(TOM, 2;5.3)
    \z
    \ea
    \label{ex:13:4} queres [ que eu jogue ]?	\hfill 	(TOM, 2;11.12)
\z
Na mesma faixa etária, encontram-se exemplos de produção espontânea de outro tipo de subordinadas, que têm atraído a atenção dos investigadores: orações relativas, em particular orações relativas com antecedente (sobre o percurso de aquisição deste tipo de estruturas e outras relacionadas, veja-se a síntese em \citealp{Lobo2017}). São exemplos de produção de relativas os casos em (5) e em (6), em que assinalamos com parênteses retos a relativa e o sintagma nominal (SN) em que se encontra encaixada (para simplificar a apresentação, omitimos a transcrição de pausas na produção dos enunciados).

\ea
    \label{ex:13:5} tenho  [ este  [que é do pai ]]. \hfill(TOM, 2;9.7)
    \z
\ea
    \label{ex:13:6} tive [um popó  [qu(e) o João deu ]]. \hfill	(TOM, 2;9.7)
\z
Estes dois enunciados, produzidos pela mesma criança, com 2;9, correspondem a casos de orações relativas em que também observamos o complementador \emph{que}, sendo este o tipo de orações relativas que emerge mais cedo na produção das crianças.\footnote{O complementador \emph{que} em orações relativas é analisado como um pronome relativo nas gramáticas tradicionais. Assumiremos aqui que se trata de facto de um complementador, já que é essa a visão que sustenta grande parte da investigação recente e relevante para as questões que nos ocupam. As estruturas que apresentamos são estruturas simplificadas.} Estes enunciados exemplificam ainda diferentes tipos de orações relativas, a saber, uma relativa de sujeito em (5) e uma relativa de objeto em (6). Esta classificação refere-se à interpretação da posição vazia na oração relativa, como se observa em (5’) e (6’): em (5’) percebemos que, quando interpretamos a frase, interpretamos “este” como sujeito da oração relativa e, no caso de (6’), interpretamos a frase como tendo a posição de complemento do verbo “dar” na oração relativa preenchida por um SN cujo núcleo é “popó” (representamos essas cadeias com o uso de índices).

\begin{exe}
\exp{ex:13:5}  tenho  [ este\textsubscript{i}  [que [-]\textsubscript{i}  é do pai ]].
\exp{ex:13:6} tive [um popó\textsubscript{i}  [qu(e) o João deu  [-]\textsubscript{i} ]].
\end{exe}

Neste período de desenvolvimento, é ainda observável a produção espontânea de algumas orações adverbiais, pelo menos, temporais introduzidas por \emph{quando} (7), condicionais introduzidas por \emph{se}, como (8), com o verbo da subordinada no conjuntivo, e orações finais introduzidas por \emph{para}, como em (9). Mais uma vez, identificamos com parênteses retos as subordinadas relevantes. Para alguma reflexão sobre o uso precoce de conectores, entre os quais \emph{se}, veja-se \citet{Costa2008}. Para mais informação sobre o tipo de orações finais em (9), veja-se \citet{Santos2013}.

\ea
\label{ex:13:7} depois eu vou fazer depois [ quand(o) arruma(r) isso ].\hfill 	(TOM, 2;6.6)
    \z
\ea
 \label{ex:13:8} [ se puseres assim ] fica o seis .			\hfill	(TOM, 2;9.7)
    \z
\ea
 \label{ex:13:9} ponh(o) aí [ pa(ra) faze(re)s ] (.) (es)tá?	\hfill 	(TOM, 2;8.9)
\z
Neste capítulo, centrar-nos-emos apenas nos casos das orações completivas e das orações relativas. No entanto, os exemplos de produção precoce de subordinação adverbial completam a imagem que podemos construir da explosão de desenvolvimento sintático que caracteriza tipicamente este período entre os 2;0 e os 3;0. Sendo a emergência da subordinação uma característica importante do desenvolvimento linguístico neste período, ela é naturalmente muito relevante na construção de instrumentos que pretendam aferir o desenvolvimento nesta faixa etária. Por isso mesmo, a adaptação para o português, e para uma faixa etária entre os 30 e os 48 meses, de um instrumento de avaliação conhecido, o \emph{MacArthur-Bates Communicative Development Inventory - III}, integra uma subescala de complexidade sintática maioritariamente centrada em estruturas de subordinação. O estudo de validação deste questionário \citep{Cadime2021} mostra a capacidade dessa mesma escala para captar o desenvolvimento linguístico nesta faixa etária.

Tendo por base esta primeira descrição dos dados de produção no desenvolvimento típico, é necessário perceber o que nos mostram, do ponto de vista da estrutura sintática. Essa questão passa por perceber o que têm estas estruturas em comum. Na verdade, estruturas como subordinadas completivas, relativas ou adverbiais são casos de estruturas em que temos evidência para a projeção de um CP (\emph{Complementizer Phrase}, i.e. um grupo cujo núcleo é um complementador, C), como indicado em (10) a (14). Essa evidência advém da presença de um elemento no domínio CP, nomeadamente o núcleo do CP preenchido por um complementador lexicalmente realizado (\emph{que} em 10 e 11; \emph{se} em 12). No caso de completivas infinitivas do português como em (13), que repete (2), considera-se também que está projetado um CP, mas que, neste caso, o núcleo do CP (i.e. C) não é lexicalmente realizado (aqui representado por Ø). No entanto, há completivas infinitivas em português que correspondem a um CP cujo núcleo é preenchido pelo complementador \emph{para} (homófono da preposição \emph{para}), como em (14).

\ea
    \label{ex:13:10} acho [\textsubscript{CP} que é difí(cil) ].\hfill 		(TOM, 2;5.3)
    \z
\ea
\label{ex:13:11} tenho  [ este\textsubscript{i}  [\textsubscript{CP} que [-]\textsubscript{i}  é do pai ]].	\hfill	(TOM, 2;9.7)
\z

\ea
\label{ex:13:12} [\textsubscript{CP} se puseres assim ] fica o seis .\hfill			(TOM, 2;9.7)
    \z
\ea
\label{ex:13:13} a foca qu(er)ia [\textsubscript{CP} Ø   sai(r) ].	\hfill		(TOM, 2;1.7)
    \z
\ea
\label{ex:13:14} a foca disse ao menino [\textsubscript{CP} para sair ].
\z
Na verdade, alguns trabalhos na tradição de investigação em gramática generativa sobre desenvolvimento linguístico sugeriram que, em fases iniciais do desenvolvimento, as crianças não projetariam o domínio CP (veja-se, em particular, \citealp{Radford1988, Radford1996, Clahsen1990/1991, Meisel1992} e, muito recentemente, \citealp{Friedmann2021}). Embora essa posição tenha sido frequentemente discutida e contestada, nomeadamente porque, como lembram \citet{Poeppel1993}, a ausência de evidência para uma dada categoria não é evidência para a sua ausência, a questão é que a evidência para a projeção do domínio CP aparece precisamente, no desenvolvimento típico, nesta faixa etária e com este tipo de produções.  

É importante também sublinhar que é nesta faixa etária que, no desenvolvimento típico, se observa o fenómeno de omissão de complementador, observável em casos de orações completivas como em (15) e (16), este último um caso apresentado em \citet{Santos2017a}, em que o complementador esperado seria \emph{se}.

\ea
    \label{ex:13:15} Passou Ø a Ma(r)ta (es)tava a chorar. (SAN 2;6.3) \citep[365]{Soares2006}
    \z
    \ea
\label{ex:13:16} e(u) [//] \# <e(u) vo(u)> [/] e(u) vo(u) <ver Ø e(n)co(n)t(r)o> [?].\hfill  (INI, 2:4.19)
\z
O fenómeno de omissão de complementador nesta faixa etária é conhecido e tem sido documentado noutras línguas (veja-se \citealp{Meisel1992, Mastropavlou2011}). Esta omissão, contudo, não deve ser entendida como manifestação de incapacidade para projetar o domínio CP: não identificamos uma faixa etária em que a omissão ocorra sistematicamente; pelo contrário, este é sempre um fenómeno que se observa ocasionalmente, num período do desenvolvimento em que a criança também produz estruturas com complementadores lexicalmente realizados, como observa, aliás, \citet{Soares2006}. Assim, será mais adequado entender este fenómeno como manifestação de instabilidade na definição das propriedades que podem obrigar à realização do complementador num determinado contexto.

Tendo em conta estes dados, poder-se-ia pensar que, no desenvolvimento típico, este tipo de estruturas atinge um desenvolvimento pleno na faixa etária dos 2-3 anos. Não é o caso, contudo. Os trabalhos desenvolvidos com crianças com desenvolvimento típico em idade pré-escolar (3-5 anos), maioritariamente trabalhos experimentais centrados quer na produção quer na compreensão, mostraram que as orações subordinadas relativas e as completivas apresentam vários tipos de dificuldade. 

Vejamos em primeiro lugar o caso das orações relativas. Considere-se o contraste entre relativas de sujeito e relativas de objeto mencionado anteriormente nesta secção. Um número considerável de trabalhos mostrou que crianças em idade pré-escolar apresentam resultados diferentes na produção e na compreensão de relativas de sujeito e relativas de objeto como as que apresentamos em (17) e (18), retiradas do trabalho sobre o tema apresentado em \citet{Costa2011}. Como fizemos antes, usamos parênteses retos para identificar a oração relativa e o SN em que se encontra encaixada e identificamos a posição vazia na oração relativa, bem como a sua interpretação.

\ea
    \label{ex:13:17} Mostra-me [o hipopótamo\textsubscript{i} [que [-]\textsubscript{i} seca o menino]]. \emph{Relativa de sujeito}
    \z
    \ea
\label{ex:13:18} Mostra-me [o hipopótamo\textsubscript{i} [que o \emph{menino} seca [-]\textsubscript{i} ]]. \emph{Relativa de objeto}
\z
Os vários trabalhos que têm sido desenvolvidos sobre estas estruturas, como é o caso de \citet{Arosio2006}, \citet{Belletti2010}, \citet{Friedmann2009}, mostram que as crianças em idade pré-escolar e pelo menos no início da idade escolar apresentam maiores dificuldades na produção e compreensão de relativas de objeto. O mesmo é confirmado por \citet{Costa2011} para o português (veja-se também a menção a estas questões em \citetv{chapters/02}). Neste caso, foram testadas 60 crianças entre os 3;9 e 6;2, falantes monolingues do PE, quer numa tarefa de compreensão, quer numa tarefa de produção dos dois tipos de relativas. No que diz respeito à tarefa de compreensão, o acerto, no caso das relativas de objeto, atingiu apenas 68\%, contrastando com 96\% de acerto em relativas de sujeito (os adultos testados como grupo de controlo apresentaram respostas esperadas em todos os casos). Já no caso da tarefa de produção, a evitação da produção de relativas de objeto foi expressiva: apenas 31\% dos contextos resultaram na produção de relativas de objeto, sendo que 78\% dos contextos relevantes apresentados resultaram na produção de relativas de sujeito. Na tarefa de produção, contudo, os resultados mais relevantes são os que consideram a análise qualitativa das respostas, já que também os adultos testados produziram menos relativas de objeto do que relativas de sujeito: enquanto os adultos produzem estruturas gramaticais, como relativas de sujeito com uma estrutura passiva, as crianças apresentam vários tipos de resposta que não seriam possíveis na gramática adulta. Entre estas respostas, destacam-se a omissão do complementador (como em 19) ou a produção de uma relativa de sujeito em vez de uma relativa de objeto.

\ea
    \label{ex:13:19} Gostava de ser o menino Ø o avô visita.
\z

Na perspetiva de \citet{Friedmann2009}, as dificuldades com relativas de objeto, como em (18), resultam do facto de, entre a posição em que o SN que tem por núcleo “hipopótamo” é realizado e a posição em que é interpretado (a posição de objeto na oração relativa), intervir um outro SN com propriedades semelhantes (em particular, trata-se também de um nome e não de um pronome). Por isto, estas dificuldades com relativas de objeto têm sido entendidas como um efeito de intervenção – o SN sujeito intervém na interpretação da cadeia formada pela posição vazia na posição de objeto da relativa e pela posição do antecedente da relativa em que o SN é efetivamente realizado. Para uma apresentação mais completa desta questão, estendendo-a a estruturas aparentadas, sugere-se a leitura de \citet{Lobo2017}; veja-se ainda \citetv{chapters/14}). Na verdade, a dificuldade observada no caso das relativas de objeto parece prolon\-gar-se, manifestando-se, em idade adulta, como um maior custo de processamento deste tipo de relativas: embora os adultos possam compreender relativas de objeto adequadamente, vários estudos mostram que o processamento de frases com este tipo de configuração pode ser custoso e demorado \citep{King1991, Gordon2001, Costa2012, Delgado2021}.

Já as orações completivas ocorrem como argumentos de um predicador, como um verbo, sendo algumas das suas propriedades determinadas por esse predicador. Por exemplo, a escolha entre modo indicativo ou conjuntivo em completivas selecionadas por verbos, que Jesus (2014) mostrou ser uma dificuldade que se prolonga ainda em idades que correspondem ao 1º ciclo de escolaridade, é algo que depende do verbo que seleciona a completiva (veja-se a síntese em Santos 2017). Também a interpretação de um sujeito nulo em completivas finitas de conjuntivo vs. indicativo poderá apresentar uma dificuldade prolongada no desenvolvimento típico (veja-se \citealp{Silva2015} e \citealp{Gamas2016}). Contudo, pela sua relevância para este capítulo, centrar-nos-emos aqui nas completivas infinitivas e na forma como se determina a interpretação do seu sujeito. Considerem-se os exemplos em (20), (21) e (22):

\ea
    \label{ex:13:20} Os porcos\textsubscript{i} querem [ [-]\textsubscript{i} saltar].
    \z
    \ea
\label{ex:13:21} \emph{Os porcos}\textsubscript{i} prometeram a\emph{os cavalos} [ [-]\textsubscript{i}  saltar].
\z
\ea
\label{ex:13:22} \emph{Os porcos} obrigaram \emph{os cavalos}\textsubscript{i} [a [-]\textsubscript{i} saltar].
\z

No caso do tipo de orações completivas infinitivas aqui exemplificado, com infinitivo não flexionado, a referência do sujeito da infinitiva é obrigatoriamente estabelecida por um SN na subordinante. Esse SN é chamado controlador e diz-se que o sujeito da oração subordinada é controlado. No caso de completivas que são complemento de verbos como \emph{querer}, que apenas têm um complemento, o sujeito da matriz funciona como controlador e temos uma estrutura de controlo de sujeito. No entanto, os exemplos em (21) e (22) são casos em que o verbo da matriz é ditransitivo e, portanto, há dois SNs que poderiam funcionar como controladores (em itálico). No entanto, o conhecimento que adquirimos sobre o verbo \emph{prometer} permite-nos saber que em (21) o sujeito da matriz é o controlador, resultando numa estrutura de controlo de sujeito; o nosso conhecimento sobre o verbo \emph{obrigar} permite-nos saber que em (22) o objeto direto é o controlador, resultando numa estrutura de controlo de objeto. \citet{Agostinho2014}, através de uma tarefa experimental aplicada a crianças de 3, 4 e 5 anos, mostrou, para o português, que estruturas como (20) não levantam problemas de compreensão. Os efeitos observam-se com estruturas de controlo com verbos ditransitivos, como (21) e (22).  As dificuldades na compreensão de estruturas de controlo de sujeito com um verbo ditransitivo, como \emph{prometer} (em (21)) são não só mais acentuadas inicialmente (aos 3 anos mais de 50\% das respostas correspondem à interpretação de estruturas como (21) como casos de controlo de objeto), como também mais prolongadas no tempo (aos 5 anos atinge-se pouco mais de 50\% de acerto). As dificuldades prolongadas na interpretação de estruturas com \emph{promise} ‘prometer’ tinham já sido documentadas no trabalho clássico de \citet{Chomsky1969} (veja-se também, mais recentemente, \citealp{Mateu2016}). Para uma apresentação mais completa desta e de outras questões relativas à aquisição de subordinadas completivas, veja-se a síntese em \citet{Santos2017a, Santos2021}. Nas secções seguintes, veremos de que forma o comportamento de crianças com diferentes tipos de desenvolvimento linguístico atípico, associados, fundamentalmente, a PDL e PEA, pode ser caracterizado através da produção e compreensão de estruturas de subordinação como as que destacámos nesta secção.

\section{A relevância da subordinação na caracterização da Perturbação do Desenvolvimento da Linguagem}
\subsection{Nota prévia: PDL e PEL /PEDL – a evolução da classificação}
Apesar de ser consensual que a aquisição das línguas ocorre de forma espontânea, i.e., sem necessidade de instrução formal, existe um grupo de crianças cujo percurso de aquisição segue um trajeto atípico, sem que nada possa justificar as dificuldades de linguagem que apresentam. Atualmente, o nome utilizado para referir esta condição na prática clínica é Perturbação do Desenvolvimento da Linguagem (PDL) (correspondente ao termo \emph{Developmental Language Disorder} (DLD), em inglês, veja-se \citealp{Bishop2017}).\footnote{Termo equivalente a Transtorno do Desenvolvimento da Linguagem (TDL) em português brasileiro.} Não obstante o consenso atualmente obtido quanto ao rótulo a atribuir a esta perturbação do desenvolvimento, a classificação sofreu bastantes alterações ao longo dos tempos, com vários autores a chamarem a atenção para o custo que as constantes modificações da classificação podem acarretar, sob pena de serem perdidos dados relevantes da investigação \citep{Ebbels2014a}. Os termos anteriormente utilizados para referir este tipo de perturbações de linguagem, que ocorrem sem causa aparente, eram Perturbação Específica da Linguagem (PEL) ou Perturbação Específica do Desenvolvimento Linguístico (PEDL),\footnote{Em inglês \emph{Specific Language Impairment} (SLI).} termos associados a conceitos que se baseavam essencialmente em critérios de exclusão. Porém, por se considerar que alguns dos critérios de exclusão reuniam pouca utilidade etiológica, tornou-se necessária uma revisão da classificação, associada à revisão do termo e, sobretudo, do conceito. Um dos critérios considerado como mais problemático dizia respeito à necessidade de uma discrepância entre os valores de QI não-verbal e o desempenho em tarefas de linguagem para classificação como PEL / PEDL (sobre o tema veja-se \citealp{Bishop1994, Jakubowicz2008}). Já no caso do diagnóstico de PDL, o valor de QI não-verbal abaixo de 85 não é critério de exclusão, passando a ser possível atribuir este diagnóstico a crianças que tenham um défice cognitivo ligeiro, desde que não cumpram critérios para uma Perturbação Intelectual (i.e., QI abaixo de 70 - ver DSM – V; \citealp{Association2013}).

Paralelamente, deixou também de se considerar como critério de exclusão a presença de condições socialmente adversas que possam ter impacto no desenvolvimento linguístico da criança. Uma outra alteração à recente classificação diz respeito à possibilidade de comorbilidade com determinadas condições neurodesenvolvimentais que, atualmente, se consideram poder coocorrer com o diagnóstico de PDL, como é o caso da Perturbação de Hiperatividade e Défice de Atenção. 

Obviamente, o alargamento dos critérios de inclusão possibilita a atribuição do diagnóstico de PDL a indivíduos que eram excluídos do diagnóstico de PEL/PEDL e que, desta forma, poderão ter acesso a serviços de saúde e educacionais, o que, sem um diagnóstico claro, poderia ficar comprometido. No entanto, é também inegável o risco de um grupo já caracterizado por grande heterogeneidade se tornar ainda menos uniforme, podendo haver um impacto nas amostras estudadas na investigação, que naturalmente contarão com maior grau de variabilidade interindividual. 

Apesar de atualmente a classificação estabelecida nas áreas clínica e científica ser PDL, é importante referir que os estudos mencionados ao longo deste capítulo, que associamos à caracterização de PDL, mas que podem ser anteriores a \citet{Bishop2017}, foram elaborados no quadro teórico e conceptual da classificação de PEL/PEDL, tendo como base os critérios então vigentes. Assim sendo, os resultados devem ser interpretados como referentes a indivíduos que se enquadram nos critérios mais restritivos que serviam de base ao diagnóstico de PEL/PEDL. 

\subsection{A subordinação na produção linguística espontânea e provocada em PDL}\label{sec:partedoispontodoiscap13}
A análise de produção espontânea e provocada de indivíduos diagnosticados com PDL (que têm afetada a componente sintática da língua, muitos classificáveis como PEL sintática, de acordo com os critérios de \citealp{Friedmann2008}) tem permitido explorar a presença da subordinação como medida de avaliação do desenvolvimento linguístico.

Assim, alguns estudos de produção espontânea, que essencialmente têm tentado apurar quais as estratégias utilizadas pelas crianças para evitar a complexidade, descrevem baixas taxas de produção de relativas e completivas em indivíduos com PDL. Adicionalmente, verifica-se que a maioria das (poucas) relativas produzidas diz respeito a enunciados com menor nível de encaixe, perfil que se assemelha ao das crianças mais novas \citep{Delage2008}. Olhando de forma mais global para a produção de estruturas subordinadas e recorrendo a uma tarefa de produção semiestruturada, \citet{Martinsempreparacao} mostra diferenças marcadas entre a frequência e a densidade de subordinadas em crianças com PDL e desenvolvimento típico (DT) falantes de PE: as crianças com PDL de 8-11 anos testadas produzem subordinadas em apenas 20\% dos enunciados, contrastando com a produção em 67\% dos enunciados por parte de crianças classificadas como DT na mesma faixa etária e aproximando-se antes dos 28\% de enunciados contendo subordinadas produzidos por crianças DT com 4 anos; por outro lado, o mesmo grupo PDL de 8-11 anos produz apenas 12\% de enunciados com pelo menos duas subordinadas, contrastando com 41\% de enunciados do mesmo tipo produzidos pelas crianças DT da mesma idade e aproximando-se, mais uma vez, dos valores obtidos pelo grupo de crianças DT de 4 anos (11\%). O recurso à subordinação como medida de complexidade sintática marca, assim, uma diferença expressiva entre crianças diagnosticadas com PDL e controlos da mesma faixa etária, verificando-se que os indivíduos com PDL recorrem menos à subordinação do que crianças com DT. 

Além disso, a investigação sobre produção de subordinação em indivíduos diagnosticados com PDL revela ainda que, nas tentativas de produção de frases subordinadas, são observadas taxas mais elevadas de agramaticalidade. Por este motivo, considera-se que a produção de enunciados com maior nível de complexidade está associada ao aumento dos níveis de agramaticalidade, i.e., quanto maior a complexidade de determinado enunciado, mais provável será que resulte numa produção agramatical \citep{Hamann2007}.

Um dos tipos de desvio à gramática-alvo que é possível observar é a omissão de complementador, como podemos observar nos dados do português apresentados por \citet[79]{SuaKay19971998} e de que listamos exemplos de seguida (os complementadores \emph{que} e \emph{se} são apresentados entre parênteses curvos por não terem sido produzidos). 

\ea
    \label{ex:13:23}  e vê [//] e viu (que) o cão estava aqui preso (6;9)
    \z
    \ea
\label{ex:13:24} pensava (que) estava aqui	(6;9)
\z
\ea
\label{ex:13:25} depois o menino disse ao cão assim shh@i para saber (se) estava lá    (4;9)
\z

É muito interessante observar este tipo de comportamento em PDL, atestado também para outras línguas (veja-se, por exemplo, \citealp{Mastropavlou2011} para o grego), já que também o observámos em crianças com desenvolvimento típico. A questão fundamental, neste caso, será que este é um comportamento que, no desenvolvimento típico, se observa ocasionalmente na faixa etária em que as crianças produzem as primeiras subordinadas, nomeadamente na faixa dos 2-3 anos e, eventualmente, como atestado em estudos de produção provocada de relativas, em casos em que a criança é forçada a produzir uma estrutura especialmente custosa, como as relativas de objeto (ver secção \ref{sec:parteumcap13}). No caso dos dados de \citeauthor{SuaKay19971998}, estas estruturas são produzidas, com frequência, por crianças mais velhas, mesmo em idade escolar (as crianças testadas têm entre quatro e nove anos). No caso dos dados apresentados por \citet{Mastropavlou2011}, é interessante que a omissão de complementador no grupo PDL seja não só mais frequente do que em crianças com idades semelhantes, mas também mais frequente do que num grupo de controlo de crianças mais novas, mas emparelhadas por critérios de desempenho linguístico. Contudo, tal como no caso da reflexão que fizemos sobre a omissão de complementador em crianças com DT, seria porventura simplista concluir imediatamente que esta omissão significa que estas crianças não projetam o domínio de CP, até porque esta omissão não será forçosamente sistemática; poderá ser que, até certo ponto e como na gramática das crianças com DT numa faixa etária mais baixa, se observe instabilidade na definição de propriedades que determinam a realização lexical dos complementadores em alguns contextos (veja-se também, a este propósito, \citealp{Mastropavlou2011}).

O trabalho de \citet{Martinsempreparacao} vem trazer-nos algumas notas relevantes no que diz respeito à análise específica do PE e ao tipo de estruturas produzidas por crianças com desenvolvimento atípico. Em PE, temos disponíveis estruturas de infinitivo preposicionado como a que apresentamos em (26), produzida por uma criança com PDL, num contexto discursivamente semelhante a um contexto que poderia também justificar a produção de uma oração relativa e em que esta estrutura de infinitivo preposicionado surge isolada e não como complemento de um verbo. 

\ea
    \label{ex:13:26} [O macaco [a lamber o porco] ] e [o porco [a lamber o macaco]].
\z
    
\citet{Martinsempreparacao}, de que se extrai o exemplo em (26), regista uma elevada proporção de produções que apresentam infinitivos preposicionados quer nos grupos DT, quer nos grupos PDL (e PEA, também).  O que é interessante é que (i) nas crianças DT mais velhas (8-11), de idades semelhantes às do grupo PDL, regista-se também a produção de infinitivos preposicionados como complemento do verbo \emph{ver} (para a discussão destas estruturas no desenvolvimento típico, veja-se \citealp{Santos2016}); (ii) a proporção de enunciados com infinitivos preposicionados é mais alta nos grupos com desenvolvimento atípico do que nos restantes grupos, mesmo de crianças mais novas (os grupos com desenvolvimento atípico com 8-11 anos aproximam-se nesta medida do grupo de 4 anos), sendo que o recurso a este tipo de enunciados parece permitir evitar a produção de outro tipo de subordinadas que ocorrem com valores muito mais baixos no caso de PDL (4\% de relativas no grupo PDL vs. 39\% no grupo DT da mesma faixa etária).

No que diz respeito especificamente à produção de orações relativas, vários estudos de produção espontânea ou semiestruturada têm tido dificuldade em identificar relativas no discurso de crianças com PDL. É o caso de \citet[80]{SuaKay19971998}, estudo baseado em narrativas, que não regista a produção de relativas. É também o caso de estudos longitudinais com crianças já em idade escolar, como o de \citet{Cipriani1998}, com uma criança italiana de 6;2-13;5 anos, ou o de \citet{Lely1997}, que segue durante dois anos uma criança inglesa a partir dos 10;3 (neste último caso, também não se regista a produção de completivas finitas). Já \citet{Marinellie2004}, que analisa 100 enunciados de produção espontânea de 15 crianças com PDL, em idade escolar, e que analisa de forma mais global a complexidade sintática dos enunciados produzidos, mostra que crianças com PDL produzem relativas, embora menos do que o grupo de controlo, constituído por crianças com DT. A heterogeneidade é, como se sabe, característica deste grupo de indivíduos. Contudo, e como inicialmente indicámos, estes estudos apontam para uma menor frequência deste tipo de estruturas, aparentemente evitadas por este grupo com desenvolvimento atípico. Trata-se de estruturas que, em contextos em que não se tornam indispensáveis à comunicação, como por exemplo o contexto das narrativas de \citeauthor{SuaKay19971998}, podem nem sequer chegar a ser observadas. Estes estudos mostram, pois, a complementaridade da produção espontânea e provocada na avaliação do desenvolvimento.

De facto, vários estudos que testam crianças em idade escolar ou mesmo já na adolescência confirmam que os indivíduos com PDL têm dificuldade em produzir orações relativas, embora as produzam. Contudo, essa dificuldade é mais expressiva na produção de relativas de objeto, como mostraram \citet{Novogrodsky2006}, apresentando resultados de crianças (PDL) entre os 9;3 e os 14;6 e, no caso do PE, \citet{Costa2009}, que testam crianças (também com PDL) entre os 5;9 e os 11;3. Os dados de produção provocada de \citet{Martinsempreparacao}, obtidos com crianças entre os 8 e os 11 anos, falantes de português, confirmam estes resultados. Neste trabalho, na aplicação de uma tarefa de imitação provocada, verificou-se que o erro mais produzido pelas crianças do grupo PDL foi a produção de uma relativa de sujeito quando era pretendida a produção de uma relativa de objeto. As relativas de objeto são precisamente o tipo de relativa que, como observámos na secção 1, levanta maiores dificuldades a crianças com DT; no caso das crianças diagnosticadas com PDL e que têm afetada a componente sintática da gramática, estas dificuldades parecem ser mais expressivas e prolongadas. Das dificuldades dos indivíduos com PDL na compreensão dessa estrutura, ocupar-nos-emos na secção que se segue.

\subsection{A compreensão de orações relativas de objeto e a persistência de dificuldades de compreensão em crianças com PDL }
Uma das principais questões às quais os estudos em aquisição se têm dedicado diz respeito à identificação de marcadores clínicos que possam, de forma inequívoca, reconhecer quais as crianças que seguem percursos de aquisição linguística atípica. Cada vez se torna mais evidente que o conhecimento dos processos de aquisição em populações com desenvolvimento típico fornece dados cruciais que orientam a investigação em grupos clínicos. Com efeito, os estudos em aquisição típica permitiram identificar diversas estruturas sintáticas que parecem ser mais problemáticas para as crianças com desenvolvimento típico, sendo a sua estabilização (ou aquisição) mais tardia, e, paralelamente, foi-se tornando evidente que muitas dessas estruturas são ainda mais problemáticas (e de forma mais prolongada) para populações com défices sintáticos, como é o caso de indivíduos com PDL. Ou seja, a persistência de dificuldades com essas estruturas é reconhecida como marca de desenvolvimento linguístico atípico.

Como exemplo da persistência de dificuldades em crianças com PDL, podemos apontar a assimetria verificada entre a compreensão e a produção de relativas e interrogativas de sujeito (27a e 28a, respetivamente) e relativas e interrogativas de objeto (27b e 28b, respetivamente), com claro privilégio das primeiras.  

\ea
    \label{ex:13:27}
\ea \begin{tabbing}Relativ\=a de sujeito\\
               \>Este é o menino que abraçou o pai.\end{tabbing}
\ex \begin{tabbing} Relativ\=a de objeto\\
                \>Este é o menino que o pai abraçou. \end{tabbing}
                \z
                \z
\ea
    \label{ex:13:28}
    \ea \begin{tabbing}Interro\=gativa de sujeito\\
               \>Que menino é que abraçou o pai?\end{tabbing}
\ex \begin{tabbing}Interro\=gativa de objeto\\
\>Que menino é que o pai abraçou?  \end{tabbing}  
\z
\z

Esta assimetria, que tem sido amplamente descrita para populações com DT até sensivelmente à idade escolar (veja-se a secção \ref{sec:parteumcap13}, que descreve o problema da aquisição de relativas; veja-se \citealp{Cerejeira2009} para a aquisição de interrogativas no desenvolvimento típico em PE), tem sido também comprovada de forma mais prolongada no tempo em indivíduos com PDL, considerando-se a dificuldade prolongada uma marca da condição, com a persistência de dificuldades com relativas e interrogativas de objeto, quando comparadas com relativas e interrogativas de sujeito, mesmo até à adolescência \citep{Friedmann2004, Friedmann2011}. A dificuldade marcada na compreensão de relativas de objeto mantém-se mesmo em comparação com o desempenho na compreensão de outras estruturas que sabemos poderem ser difíceis, como relativas que envolvem passivas (veja-se o estudo de \citealp{Arosio2017}). A mesma assimetria na compreensão e produção de relativas de sujeito e de objeto foi também encontrada em crianças com PDL falantes do PE, com idades entre os 5;9 e os 11;3 anos de idade \citep{Costa2009, Ferreira2008}.  

Também para o PE, \citet{Martins2017a} descrevem dificuldades com relativas de objeto, que afetam tanto a produção como a compreensão de crianças com PDL, com idades entre os 8 e os 11 anos. Os resultados apontam para dificuldades mais marcadas em crianças com PDL, quando comparadas com crianças DT da mesma idade. 

Esta dificuldade prolongada com relativas de objeto, e que parece ser uma característica distintiva dos indivíduos com PDL, tem sido alvo de aprofundada investigação, com a contribuição de vários autores, que têm defendido que as dificuldades apresentadas com as relativas e interrogativas de objeto não podem ser explicadas simplesmente por dificuldades globais com as operações sintáticas que permitem derivar uma relativa ou uma interrogativa, já que essas dificuldades afetariam tanto as relativas de sujeito como as de objeto. Assim, as dificuldades na interpretação e produção deste tipo de estruturas, como se explicou sumariamente na secção 1, poderão ser explicadas como resultado de efeitos de intervenção de um interveniente que potencialmente concorre para a interpretação da cadeia (ver \citealp{Friedmann2011}). Consequentemente, foi crescendo uma linha de investigação que explica dificuldades acrescidas, no caso de indivíduos com PDL, tanto com relativas de objeto como com interrogativas de objeto, como um resultado de efeitos de intervenção, que afetariam de forma prolongada os indivíduos com PDL. De acordo com esta hipótese, as crianças, ao contrário dos adultos, têm dificuldades em computar uma estrutura como (29), em que o SN sujeito intervém entre a posição em que o objeto ([que menino]) é produzido e a posição de argumento interno em que é interpretado. Esta explicação é semelhante à que se apresenta para o caso das dificuldades com relativas de objeto (veja-se a secção \ref{sec:parteumcap13}). 

\ea
    \label{ex:13:29} 	[Que menino]\textsubscript{i} é que [o pai] abraçou [-]\textsubscript{i}?
\z

Na secção \ref{sec:parteumcap13}, em que nos dedicámos apenas a estruturas de subordinação, não falámos deste paralelismo na dificuldade de interpretação de relativas de objeto e interrogativas de objeto. No entanto, neste ponto do nosso texto, ele tem um papel acrescido: existindo dificuldades nas duas estruturas e observando-se, no caso de PDL, dificuldades prolongadas e paralelas nas duas estruturas e não apenas nas relativas, percebemos que os problemas que as relativas, nomeadamente de objeto, levantam no caso de PDL não se ficam a dever apenas ao facto de se tratar de uma estrutura de subordinação, já que no caso das interrogativas do tipo de (29) não temos subordinação (embora tenhamos um domínio CP projetado).

\subsection{Alguns dados sobre compreensão de completivas infinitivas em PDL}\label{sec:partedoispontoquatrocap13}
Um aspeto muito menos explorado no desenvolvimento atípico, nomeadamente na população diagnosticada com PDL, tem sido a compreensão de estruturas de controlo, como as que descrevemos na secção \ref{sec:parteumcap13}.  \citet{Martins2018} e \citet{Martinsempreparacao} apresentam alguns resultados que aqui descreveremos de forma breve. Neste trabalho, testou-se o tipo de estruturas em (30) e (31), semelhante ao tipo de estrutura que observámos em (21) e (22), na secção \ref{sec:parteumcap13}. 

\ea
    \label{ex:13:30} O \emph{porco}\textsubscript{i} prometeu a\emph{o cavalo} [ [-]\textsubscript{i}  saltar]. \hfill Controlo de sujeito
    \z
    \ea
\label{ex:13:31} O \emph{porco} disse a\emph{o cavalo}\textsubscript{i} [para [-]\textsubscript{i} correr]. \hfill Controlo de objeto
\z

Foram testadas crianças com diagnóstico de PDL entre os 8 e os 11 anos e quatro grupos de crianças com desenvolvimento típico (de 3, de 4, de 5-7 anos e de 8-11 anos). Foi ainda testado um grupo de crianças com diagnóstico de PEA de que se falará na próxima secção. Como observámos na secção \ref{sec:parteumcap13}, trabalhos anteriores identificaram dificuldades, por parte de crianças com desenvolvimento típico em idade pré-escolar, na compreensão de estruturas de controlo de sujeito com um verbo ditransitivo (\emph{prometer}), como é o caso de (30). Seria essa a estrutura em que se esperavam dificuldades prolongadas no caso de PDL. Os resultados das diferentes tarefas usadas para a avaliação da compreensão da estrutura são claros: esta é, de facto, uma estrutura em que o desempenho das crianças PDL é significativamente inferior ao desempenho das crianças DT com a mesma idade. Na verdade, os dados mostram que, no caso das crianças DT, até aos 5-7 anos a dificuldade é acentuada na compreensão de controlo de sujeito com \emph{prometer}, o que contrasta com um desempenho de praticamente 100\% de acerto aos 8-11 anos. Contudo, no caso das crianças PDL de 8-11 anos, o desempenho na compreensão desta estrutura é significativamente mais baixo, pouco acima dos níveis atingidos pelas crianças DT de 5-7 anos. 

Na verdade, e porque o mesmo trabalho testava quer controlo com \emph{prometer}, quer relativas de objeto, foi possível também observar que a diferença entre o grupo PDL e o grupo DT da mesma idade é maior no caso da estrutura de controlo de sujeito do que no caso das relativas de objeto: embora o grupo PDL tenha sempre um desempenho abaixo do grupo DT nas duas estruturas, visto que o grupo DT resolveu aos 8-11 anos o problema com a estrutura de controlo de sujeito, a diferença é neste caso mais marcada do que no caso das relativas de objeto, em que o desempenho nunca atinge níveis muito elevados, mesmo no grupo DT. Conclui-se, assim, que as dificuldades tardias e marcadas em controlo de sujeito com \emph{prometer} podem assinalar de forma clara um percurso de desenvolvimento atípico. 

\section{A relevância das estruturas de subordinação na comparação entre duas populações: PDL e PEA}
\subsection{A discussão sobre semelhanças e diferenças entre PDL e PEA}
PDL e PEA eram, tradicionalmente, consideradas como duas condições distintas sem qualquer partilha de etiologia. No entanto, os resultados de alguns estudos foram apontando para a possibilidade de existir uma relação entre estas duas condições desenvolvimentais, chegando mesmo a ter sido equacionada uma possível etiologia comum ou até mesmo uma interseção a nível genético \citep{Bishop2010, Leyfer2008, Ruser2007}. 

Os pressupostos que levaram à consideração de uma base comum partilhada por PDL e PEA assentavam em descrições de um padrão de desenvolvimento linguístico atípico em algumas crianças com PEA que se considerou ser semelhante a outras condições desenvolvimentais que afetam a aquisição da língua materna (veja-se \citealp{TagerFlusberg2006a}). Consequentemente, gerou-se grande interesse em descrever o desenvolvimento linguístico de indivíduos com PDL e PEA, de forma comparativa. 

O objetivo da investigação na área consistia em verificar se indivíduos com PDL e indivíduos com PEA com algum grau de comprometimento da linguagem teriam resultados semelhantes em testes de linguagem, o que poderia eventualmente apontar para a possibilidade de uma etiologia compartilhada pelas duas condições. A necessidade de aumentar este conhecimento alterou o foco dos estudos em desenvolvimento linguístico em indivíduos com PEA, que, até então, eram dedicados, quase exclusivamente, a questões de pragmática. 

Um dos primeiros estudos centrados precisamente nas capacidades linguísticas formais nesta população corresponde ao trabalho de \citet{Kjelgaard2001}. Os autores defendem que, embora o comprometimento da linguagem não possa ser generalizado a todos os casos de PEA, as dificuldades a este nível são encontradas numa larga proporção dos indivíduos e o seu desempenho linguístico parece de alguma forma assemelhar-se ao perfil descrito na literatura para crianças diagnosticadas com PDL. 

\subsection{Semelhanças}
\largerpage

Parece ser evidente que existe um grupo de crianças com PEA que revela grave comprometimento da linguagem, podendo as manifestações clínicas desse comprometimento ser comuns a PDL e PEA. 

Considerando a necessidade de um maior conhecimento das dificuldades de linguagem em PEA, a literatura tem dado cada vez mais destaque à aquisição do conhecimento sintático por parte deste grupo, sendo, apesar disso, ainda evidente uma grande lacuna nesta área, especialmente quando consideramos o grau de desenvolvimento da investigação que tem sido dedicada a casos de PDL. No entanto, os mesmos tópicos explorados para a população com PDL têm, gradualmente, sido estudados também para casos de PEA, precisamente com o objetivo de apurar quais são as semelhanças e as diferenças entre estas duas condições, no que diz respeito ao desenvolvimento linguístico. 

Os estudos em aquisição sintática atípica, nomeadamente em populações com PDL e, mais recentemente, em alguns casos de PEA, têm-se debruçado sobre a natureza das dificuldades linguísticas exibidas por indivíduos diagnosticados com as referidas condições, surgindo grande interesse pelas dificuldades reveladas com os processos de subordinação. 

Como vimos na secção \ref{sec:partedoispontodoiscap13}, vários estudos de produção com crianças com PDL sugerem uma menor frequência da subordinação na produção desses indivíduos, medida quer em número de enunciados que contêm subordinadas, quer em níveis de encaixe observados em estruturas de subordinação. Estes efeitos da complexidade, embora menos estudados, têm sido alvo de crescente interesse em relação a populações com PEA, começando a ser descritos efeitos de alguma forma semelhantes. De forma paralela àquilo que acontece com PDL, alguns autores descrevem, também em casos de PEA, a ausência quase total de relativas e de completivas nas produções dos indivíduos, com piores resultados do que indivíduos com DT da mesma idade \citep{Durrleman2009}. 

Para além da diminuta produção de enunciados contendo subordinação, também a maior facilidade na compreensão de relativas de sujeito, quando comparadas com relativas de objeto (e também interrogativas de sujeito, quando comparadas com interrogativas de objeto), parece caracterizar a população com PEA, sendo possível verificar que o comportamento dos indivíduos em estudo se assemelha, como acontece no caso de PDL, a crianças mais novas (\citealp{Riches2010}; para o português, \citealp{Martins2018} e \citealp{Martinsempreparacao}).

No entanto, apesar de, em vários estudos, serem destacadas as semelhanças entre crianças diagnosticadas com PEA e crianças com PDL em termos de resultados globais, verifica-se que paralelamente são descritas algumas diferenças ao nível do desempenho quando observado de forma mais detalhada. 

\newpage
\subsection{Quando a observação fina de algumas estruturas complexas nos aponta diferenças}
A verificação de um certo grau de semelhança entre os perfis linguísticos de crianças com PDL e PEA levou a que fosse atribuída maior importância ao desempenho na compreensão e produção de estruturas sintáticas específicas, tentando os investigadores verificar se o comportamento dos dois grupos é de facto equivalente ou se, pelo contrário, existem diferenças que sugerem um distanciamento entre as duas condições. 

Na verdade, verifica-se que, em alguns dos casos, apesar de valores globais similares na compreensão e produção de determinadas estruturas, são evidentes diferenças de \emph{performance}, com comportamentos que, após uma análise mais fina, afinal se evidenciam como diferentes, deitando por terra a hipótese da existência de uma base comum às duas condições \citep{Whitehouse2008}. 

No estudo de \citet{Martinsempreparacao}, que compara grupos de crianças com PDL e PEA (em ambos os grupos, crianças sem défice cognitivo e com perturbação sintática), com idades entre os 8 e os 11 anos, mostra-se que, de uma forma geral, os dois grupos clínicos parecem afetados por fatores de complexidade sintática. Em relação à produção de subordinação, numa tarefa de produção semiestruturada, as crianças dos dois grupos clínicos manifestaram dificuldades, pois não só apresentam taxas de produção de subordinação mais baixas do que o grupo DT da mesma idade, como também exibem níveis de agramaticalidade superiores, sendo as suas taxas de produções marcadas por algum tipo de erro gramatical equiparáveis às registadas nos grupos de crianças mais novas.

Todavia, para além das dificuldades evidentes com a subordinação que aproximam os dois grupos clínicos, são observáveis diferenças de comportamento entre as crianças do grupo PDL e PEA. As diferenças são visíveis tanto no que respeita ao tipo de orações produzidas, pois apenas as crianças com PEA produzem completivas na tarefa em causa, como também em relação aos contextos de ocorrência dessas mesmas estruturas, evidenciando diferenças de desempenho, que apontam para défices de natureza diferente. Para além de as crianças do grupo PEA se revelarem capazes de produzir maior número de orações subordinadas, relativamente a crianças com PDL, ainda que a diferença seja discreta, produzem mais frequentemente enunciados mais complexos (com mais do que uma oração subordinada por enunciado e com maior nível de encaixe). Relativamente às estruturas de infinitivo preposicionado mencionadas na secção \ref{sec:partedoispontodoiscap13}, e embora tanto as crianças PEA como PDL da mesma idade (8-11) tenham recorrido frequentemente a essas estruturas, apenas as crianças PEA as produziram como complemento de um verbo, o percetivo \emph{ver}, como em (32):

\ea
    \label{ex:13:32} Aqui eu [vejo [um porco [a lamber]]]
\z

Mas se, por um lado, numa tarefa de produção semiestruturada, as crianças com PEA parecem menos afetadas do que as crianças com PDL no que diz respeito à capacidade de produção de subordinação, por outro, numa tarefa de repetição centrada em diferentes tipos de orações relativas cometem erros que indicam dificuldade na especificação de núcleos associados a domínios superiores da derivação.  Enquanto as crianças com PDL testadas neste grupo não omitiram o complementador na repetição de relativas mais simples, pelo contrário, as crianças com PEA revelaram omissão do complementador ‘\emph{que}’, mesmo no caso das relativas mais simples (ver exemplo (33)). 

\ea
    \label{ex:13:33}	Item: Este é o porco que lambeu o macaco.\\
         Resposta: Este é o porco lambeu o macaco.
\z

No estudo referido, verifica-se assim que, para além do tipo de orações subordinadas produzidas pelas crianças, também os tipos de erros cometidos pelas crianças dos dois grupos na repetição de diferentes tipos de relativas podem indiciar, se confirmados em estudos posteriores, dificuldades de natureza diferente. 

Finalmente, voltemos à questão da compreensão de duas estruturas que já vimos justificarem dificuldades prolongadas de compreensão em populações com desenvolvimento atípico: as orações relativas de objeto e o controlo de sujeito com um verbo ditransitivo como \emph{prometer}. Na verdade, e no que diz respeito às estruturas de controlo, o trabalho de \citet{Janke2015} sugere já dificuldades na compreensão de estruturas de controlo com \emph{promise} ‘prometer’ em indivíduos com diagnóstico de PEA. \citet{Martinsempreparacao}, com resultados já publicados em \citet{Martins2018}, testa a compreensão de estruturas de controlo a par de relativas de objeto, quer em PDL, quer em PEA, comparando dois grupos com a mesma idade (8-11 anos), ambos com valores normais em QI não verbal. O estudo confirma que o grupo PEA, tal como o grupo PDL (veja-se a secção \ref{sec:partedoispontoquatrocap13}), apresenta dificuldades quer em relativas de objeto quer em controlo de sujeito que explicam um desempenho significativamente inferior ao desempenho do grupo de controlo com a mesma idade. Esta é a dimensão em que encontramos semelhança entre os grupos. Contudo, os resultados também mostram que, nas várias tarefas, nas estruturas de controlo de sujeito com \emph{prometer}, o desempenho do grupo PEA é sempre inferior ao desempenho do grupo PDL; essa diferença clara não se observa no caso das relativas de objeto. Esta é, certamente, uma área de trabalho ainda a explorar. 

No entanto, apesar de os resultados de \citet{Martinsempreparacao} poderem indicar dificuldades de natureza diferente, é inegável que ambos os grupos clínicos revelam dificuldades com a subordinação em geral, dificuldades estas que são visíveis não apenas nas baixas taxas de subordinadas produzidas pelos dois grupos clínicos, como também nos níveis de agramaticalidade observados na produção dos dois grupos. 

No estudo, apesar de serem apresentadas evidências de acesso ao domínio do complementador pelas crianças dos dois grupos, é colocada a hipótese da persistência de algumas dificuldades a este nível, o que poderá querer dizer que nem todas as propriedades do domínio CP possam estar totalmente adquiridas, mesmo em idade escolar, no caso de crianças PDL e PEA. 

\section{Estratégias de estimulação do desenvolvimento linguístico }
Como vimos, parece ser evidente que as estruturas que são descritas como sendo de aquisição (ou estabilização) mais tardia em DT também se assumem como mais problemáticas em grupos clínicos. A literatura tem vindo a indicar que as crianças mantêm, mesmo em idade escolar, dificuldades com várias estruturas, que se traduzem em dificuldades ao nível da compreensão e da produção. 

Um dos grandes desafios dos terapeutas da fala que trabalham com crianças com dificuldades de linguagem, e de forma particular em crianças com défices sintáticos, diz respeito à planificação da intervenção linguística. 

A maior parte dos programas de intervenção em linguagem, e a área da sintaxe não é exceção, consiste no treino de produção e compreensão de estruturas que devem incluir as estruturas sintáticas nas quais a criança revela maiores dificuldades. Podemos pensar em duas linhas de trabalho em intervenção, que não se excluem: uma linha será a que explora sobretudo a exposição das crianças aos dados; a outra explora a possibilidade de usar a explicitação do conhecimento linguístico a favor da promoção do desempenho linguístico. 

O primeiro tipo de intervenção partirá da maximização das oportunidades que a criança tem de exposição à estrutura, i.e., a multiplicação da evidência relevante para a aquisição. No caso das estruturas que nos ocupam neste capítulo, como as relativas de objeto, é particularmente importante assegurar que a interpretação atribuída à estrutura é a relevante. Nessa linha, há trabalhos que exploram a conversão em jogo, nomeadamente em jogo numa \emph{app}, de atividades de escolha de imagem que normalmente associamos a testes experimentais. A criança treina, assim, a interpretação das estruturas, recebendo (ao contrário do que acontece numa experiência em que se avalia o seu desempenho) \emph{feedback} relativamente à correção da resposta, sempre sem explicitação, mas com possibilidade de nova tentativa de resposta. De alguma forma, junta-se à evidência positiva a evidência negativa. Este tipo de solução está a ser explorado num trabalho de S. Silleresi e E. Beccaluva, em colaboração com outros investigadores da Universidade de Milão \citep{Silleresi2020}. Veja-se ainda em \citet{Ebbels2008} a descrição de uma série de métodos que podem caber neste tipo de linha de trabalho em intervenção.

Contudo, muitos estudos recentes indicam que a intervenção ao nível das competências linguísticas pode envolver técnicas de treino explícito, através dos quais a linguagem pode ser “ensinada” \citep{Law2017}. O principal objetivo deste tipo de intervenção consiste em melhorar a compreensão e produção de estruturas sintáticas previamente selecionadas, através da utilização de tarefas que envolvem competências metalinguísticas, competências estas que estarão asseguradas em população em idade escolar (a maioria das crianças já diagnosticadas e em programas de intervenção) e com um QI não verbal não afetado ou pelo menos não afetado de forma relevante. A base da intervenção consiste na explicitação das operações sintáticas envolvidas na derivação dessas mesmas estruturas, na maior parte dos casos usando pistas visuais que ajudam a codificar partes do discurso \citep{Ebbels2008, Ebbels2014b}. Algumas das técnicas metalinguísticas que podem ser utilizadas correspondem a tarefas de manipulação de frases, incluindo alargamento, substituição, reduções, segmentação, deslocação, tentando sensibilizar as crianças para a complexidade sintática das estruturas, como se espera que aconteça em contexto escolar na promoção do desenvolvimento metalinguístico (ver \citealp{Duarte2008a}). Por exemplo, a exploração da interpretação de uma interrogativa de objeto poderá envolver a identificação de sujeito e objeto, manipulando depois a frase (que pode mesmo ser recortada em papel) para reconstituir a ordem de palavras correspondente a uma frase declarativa. Com adaptações, será possível uma abordagem semelhante no caso de manipulação de orações relativas de objeto.

Os pressupostos das abordagens metalinguísticas baseiam-se na hipótese de que crianças com perturbações de linguagem, nomeadamente crianças com PDL, têm dificuldades em adquirir gramática implicitamente e beneficiam do treino explícito das estruturas (veja-se \citealp{Ebbels2014}). Os resultados obtidos com abordagens metalinguísticas têm sido promissores e indicadores de evolução nos grupos estudados. Essencialmente, este tipo de abordagens tem sido utilizado em crianças com perturbações de linguagem em idade escolar, focando estruturas sintáticas específicas, como relativas, interrogativas e passivas (ver \citealp{Levy2009, Ebbels2010}). Em \citet{Riches2013}, encontramos uma abordagem alternativa, de base \emph{usage-based}, focada em passivas.

As abordagens de intervenção existentes tentam contornar as dificuldades observadas, através da apresentação das frases, visualmente/ortograficamente (qu\-ando aplicável), fornecendo assim uma representação estável, com recurso à codificação com cores e/ou formas, usadas para destacar propriedades específicas das frases, como a função de determinados constituintes, ou mesmo demonstrar explicitamente processos sintáticos mais complexos, como operações que alteram a ordem básica de palavras nas frases. Uma destas abordagens, que merece maior destaque e que tem sido aplicada a diferentes tipos de estruturas, corresponde ao \emph{Shape Coding} \citep{Ebbels2017}. Esta abordagem, que consiste na codificação dos vários constituintes da frase através de cores, formas e setas, tem sido aplicada a crianças com dificuldades sintáticas, sendo evidentes resultados com a morfologia verbal, a estrutura argumental do verbo e estruturas complexas, como passivas, interrogativas e coordenadas \citep{Ebbels2008, Ebbels2014}. 

Também \citet{Levy2009} descrevem algumas estratégias para a intervenção em casos de PDL. Destaque-se o facto de este trabalho mostrar que as evoluções são registadas mesmo quando as estruturas sintáticas não são trabalhadas diretamente, desde que trabalhadas as operações sintáticas relevantes. No caso estudado, foram registadas melhorias nas interrogativas parciais (ou seja, as que envolvem um pronome ou advérbio interrogativo), mesmo quando apenas foram trabalhadas diretamente as relativas (que implicam operações sintáticas semelhantes), revelando assim a possibilidade de transferência do efeito da intervenção para outra estrutura e a eficácia e importância das abordagens metalinguísticas. Encontramos a descrição de um procedimento de intervenção centrado em competência metalinguística em \textcitetv{chapters/14}.

\section{Conclusão}
O conhecimento pormenorizado do desenvolvimento sintático oferece-nos pistas para a avaliação e a intervenção em casos de desenvolvimento atípico. Nesses casos, é necessário, em primeiro lugar, conhecer os resultados da investigação, quer em desenvolvimento típico, quer nos diferentes casos de desenvolvimento atípico. O trabalho até agora realizado já mostrou que, entre outras, as estruturas de subordinação, exemplo de complexidade sintática, oferecem um campo de observação particularmente relevante, observando-se, na população com desenvolvimento atípico, efeitos de dificuldade prolongada em estruturas que também oferecem dificuldades no desenvolvimento típico – é o caso, explorado neste capítulo, das relativas de objeto e do controlo de sujeito com ditransitivos. Tentámos ainda mostrar que a comparação sistemática de diferentes populações, a saber PDL e PEA, na compreensão e produção das diferentes estruturas de interesse fornece pistas para a melhor compreensão de cada condição, sugerindo algumas semelhanças, mas permitindo também a distinção destes dois grupos.

\section*{Agradecimentos}
Este trabalho é financiado por fundos nacionais através da FCT – Fundação para a Ciência e a Tecnologia, I.P., no âmbito do projeto UIDB/00214/2020 (Centro de Linguística da Universidade de Lisboa).


{\sloppy\printbibliography[heading=subbibliography,notkeyword=this]}
\end{document}
