\documentclass[output=paper,colorlinks,citecolor=brown,booklanguage=portuguese]{langscibook} 
\ChapterDOI{10.5281/zenodo.7233223}
\title[Para além do segmento]{Para além do segmento: Recolha de dados e avaliação à luz da fonologia não linear}
\author{Ana Margarida Ramalho\affiliation{Universidade de Lisboa, Faculdade de Letras, Centro de Linguística; Hospital do Espírito Santo de Évora} and  Susana Rodrigues\affiliation{Universidade de Lisboa, Faculdade de Letras, Centro de Linguística; Universidade do Algarve} and  Maria João Freitas\affiliation{Universidade de Lisboa, Faculdade de Letras, Centro de Linguística}}

\abstract{Neste capítulo, são apresentados os constituintes prosódicos da fonologia não linear que têm sido amplamente testados na avaliação fonológica de crianças com desenvolvimento típico e com desenvolvimento atípico, em diferentes comunidades científicas e profissionais desde os anos oitenta. Focar-nos-emos nos constituintes silábicos, no acento de palavra e na palavra prosódica, ilustrando, com dados empíricos da produção por crianças portuguesas, a sua eficácia na avaliação fonológica infantil. Faremos uma revisão dos instrumentos de avaliação fonológica disponíveis para o português europeu, mostrando como e se espelham a organização prosódica proposta na fonologia não linear. Por fim, tecemos considerações sobre os cuidados a ter na recolha e tratamento de dados, para que o processo de avaliação fonológica infantil seja empiricamente robusto e permita diagnósticos o mais rigorosos possível.


\textbf{Palavras-chave}: fonologia não linear; segmentos; prosódia; avaliação; diagnóstico.} 

\IfFileExists{../localcommands.tex}{
   \addbibresource{../localbibliography.bib}
   \usepackage{langsci-optional}
\usepackage{langsci-gb4e}
\usepackage{langsci-lgr}

\usepackage{listings}
\lstset{basicstyle=\ttfamily,tabsize=2,breaklines=true}

%added by author
% \usepackage{tipa}
\usepackage{multirow}
\graphicspath{{figures/}}
\usepackage{langsci-branding}

   
\newcommand{\sent}{\enumsentence}
\newcommand{\sents}{\eenumsentence}
\let\citeasnoun\citet

\renewcommand{\lsCoverTitleFont}[1]{\sffamily\addfontfeatures{Scale=MatchUppercase}\fontsize{44pt}{16mm}\selectfont #1}
  
   %% hyphenation points for line breaks
%% Normally, automatic hyphenation in LaTeX is very good
%% If a word is mis-hyphenated, add it to this file
%%
%% add information to TeX file before \begin{document} with:
%% %% hyphenation points for line breaks
%% Normally, automatic hyphenation in LaTeX is very good
%% If a word is mis-hyphenated, add it to this file
%%
%% add information to TeX file before \begin{document} with:
%% %% hyphenation points for line breaks
%% Normally, automatic hyphenation in LaTeX is very good
%% If a word is mis-hyphenated, add it to this file
%%
%% add information to TeX file before \begin{document} with:
%% \include{localhyphenation}
\hyphenation{
affri-ca-te
affri-ca-tes
an-no-tated
com-ple-ments
com-po-si-tio-na-li-ty
non-com-po-si-tio-na-li-ty
Gon-zá-lez
out-side
Ri-chárd
se-man-tics
STREU-SLE
Tie-de-mann
}
\hyphenation{
affri-ca-te
affri-ca-tes
an-no-tated
com-ple-ments
com-po-si-tio-na-li-ty
non-com-po-si-tio-na-li-ty
Gon-zá-lez
out-side
Ri-chárd
se-man-tics
STREU-SLE
Tie-de-mann
}
\hyphenation{
affri-ca-te
affri-ca-tes
an-no-tated
com-ple-ments
com-po-si-tio-na-li-ty
non-com-po-si-tio-na-li-ty
Gon-zá-lez
out-side
Ri-chárd
se-man-tics
STREU-SLE
Tie-de-mann
}
   \boolfalse{bookcompile}
   \togglepaper[06]%%chapternumber
}{}

\begin{document}
\maketitle

\section{Contributos da teoria fonológica para a avaliação clínica }\label{sec:cap6sec1}
Até aos anos 70 do século XX, os sons da fala, entendidos como unidades discretas dos enunciados que usamos para comunicar (e que também designamos como segmentos), sempre foram as unidades fonológicas mais utilizadas na descrição da estrutura sonora das línguas.\footnote{A norma adotada na escrita deste capítulo foi a do português europeu.}
% \footnote{Na sequência de restrições decorrentes do template usado para publicação na editora Language Science Press, os capítulos que integram o volume surgem com a referência [this volume], em inglês.}
Tal subjaz ao facto de a avaliação do desempenho fonológico infantil se basear, quase exclusivamente e em muitas comunidades de terapeutas da fala / fonoaudiólogos por todo o mundo, na observação dos sons da fala. Maioritariamente, o desenvolvimento fonológico infantil é avaliado em função de cálculos associados à taxa de sucesso na produção de consoantes, não sendo tidos em consideração outros aspetos fonológicos da(s) língua(s) em aquisição, de que falaremos ao longo deste capítulo. A quase exclusividade do uso desta unidade no processo de avaliação fonológica infantil terá, ainda, na base o predomínio de sistemas de escrita alfabética no mundo, que associam sons da fala a grafemas específicos. A proficiência no uso destas unidades é, assim, essencial tanto na produção de enunciados orais inteligíveis como na aprendizagem da leitura e da escrita, sobretudo nas comunidades humanas com sistemas de escrita alfabética.

A produtividade dos sons da fala na comunicação é visível na construção de contrastes lexicais e é facilmente observável na presença de pares mínimos nas línguas do mundo. Vejam-se, a título ilustrativo, os pares mínimos [pέ] / [pá] ou [mέtɐ] / [sέtɐ] em português: a comutação de um som da fala por outro ([έ] / [á] e [m] / [s], respetivamente) implica um contraste de significado, o que ilustra o papel crucial destas unidades na comunicação verbal. 

A partir dos anos 70/80, na história da fonologia, os segmentos deixaram de ser os únicos constituintes a integrar a descrição das propriedades dos enunciados de fala: para além de decomponíveis em traços distintivos, os segmentos orga\-nizam-se em unidades prosódicas, ou suprassegmentais, que conferem ritmo e melodia à fala. Sem ritmo e melodia, os enunciados orais seriam semelhantes aos das vozes de robôs nos primeiros filmes de ficção científica. Sabemos hoje que estes aspetos prosódicos são de desenvolvimento precoce na infância e cruciais para a aquisição de várias estruturas linguísticas, condicionando a boa formação dos enunciados de fala. 

Assim, numa visão multidimensional da estrutura sonora das línguas, consideramos atualmente que os segmentos, constituídos por traços distintivos, se organizam em constituintes silábicos dentro das sílabas, as quais se agrupam em palavras; numa mesma palavra, o acento (descrito através da unidade \emph{pé}, na Figura \ref{fig:cap6fig1}) torna uma sílaba proeminente em relação à(s) restante(s) sílaba(s) da palavra, não acentuadas; por sua vez, as palavras, enquanto unidades prosódicas, organizam-se em grupos ou sintagmas para formar enunciados de fala bem formados. Esta visão fonológica está representada na Figura \ref{fig:cap6fig1} e foi proposta no âmbito de modelos teóricos que floresceram nos anos 80 do século XX, em função dos quais se assume, atualmente, uma representação hierárquica dos constituintes fonológicos (para apresentação destes conceitos, cf. \citealp{Bernhardt2000, Mateus2016}):




\begin{Figura}
    %\includegraphics[draft=false, width=4.5cm, height=7cm]{figures/Imagem40.png}
    \begin{forest}
    [Enunciado
        [Sintagma Entoacional
            [Sintagma Fonológico
                [Palavra prosódica
                    [Pé
                        [Sílaba
                            [Segmento
                                [Traços distintivos]
                            ]
                        ]
                    ]
                ]
            ]
        ]
    ]
    \end{forest}
    \caption{Organização hierárquica dos constituintes fonológicos na perspetiva da fonologia não linear.}
    \label{fig:cap6fig1}
\end{Figura}



Esta conceção da estrutura sonora das línguas como um conjunto de propriedades segmentais e prosódicas representadas por diferentes constituintes hierarquicamente organizados levou linguistas e terapeutas da fala / fonoaudiólogos a avaliar o desenvolvimento fonológico infantil com base em muito mais do que apenas as unidades discretas a que chamamos sons da fala. Esta visão não linear (ou multilinear) da estrutura sonora das línguas, que integra vários níveis representacionais, opõe-se à visão linear em vigor até aos anos 70 do século XX, que considerava o segmento como a unidade de análise fonológica por excelência.

Vários dos constituintes fonológicos listados na Figura \ref{fig:cap6fig1} têm sido adotados na descrição do desenvolvimento fonológico de crianças com perfis típico e atípico, em contexto monolingue e bilingue. Estes estudos testam a adequação do modelo teórico da fonologia não linear à descrição tanto do conhecimento implícito (cf. \citealp{Freitas1997, Correia2009, Costa2010, Lousada2012a, Amorim2014, Baptista2015, Ramalho2017} para o português europeu (PE)) como da consciência fonológica nas crianças (cf. \citetv{chapters/10}). 

Neste capítulo, centrar-nos-emos nas relações entre os segmentos e os constituintes prosódicos \emph{sílaba}, \emph{pé} e \emph{palavra prosódica} (os restantes constituintes prosódicos são abordados em \citetv{chapters/08}). As variáveis abaixo são instrumentos de trabalho que nos têm permitido usar estes constituintes na avaliação do desenvolvimento fonológico infantil (cf. \citealp{Freitas2017, Matzenauer2017, Santos2017} para mais detalhe sobre cada um dos constituintes na descrição do desenvolvimento fonológico infantil em contexto típico):


\begin{itemize}[align=left]
    \item [(i)] \emph{inventário segmental e sua organização em classes naturais;}
    \item[(ii)] \emph{constituintes silábicos} (Ataque, Rima,\footnote{Usamos Rima como constituinte silábico que domina o Núcleo e a Coda na estrutura interna da sílaba, e não como conceito da versificação, que remete para paralelismo estrutural entre duas palavras, podendo abranger mais do que uma sílaba (\emph{ga\textbf{linha}} rima com \emph{co\textbf{zinha}}; \emph{Ta\textbf{reco}} rima com \emph{bo\textbf{neco}}).} Núcleo, Coda);
    \item[(iii)] \emph{acento de palavra} (posições tónica e átona; padrões acentuais (proparoxítono, paroxítono, oxítono));
    \item[(iv)] \emph{posição na palavra} (inicial, medial, final);
    \item[(v)] \emph{extensão de palavra} (monossílabos, dissílabos, trissílabos, polissílabos).
\end{itemize}

Estas variáveis têm sido usadas tanto na descrição do conhecimento implícito como no da consciência fonológica,\footnote{O conhecimento implícito é intuitivo e decorre da capacidade universal, inerente aos seres humanos, de adquirir naturalmente um sistema linguístico e de o usar automaticamente. A consciência linguística implica a capacidade de identificar e manipular unidades do oral, em tarefas formais desenhadas para o efeito, recrutando processamento não automático.} tendo sido demonstrada a sua produtividade na avaliação de competências fonológicas e na programação da intervenção clínica (cf. \citetv{chapters/07}, \citetv{chapters/08}) e confirmada, assim, a adequação da fonologia não linear à descrição do conhecimento implícito infantil quanto à estrutura sonora da sua língua \citep{Bernhardt2000}. 


\subsection{Inventário segmental}
Os segmentos (consoantes, vogais e semivogais) organizam-se em classes naturais, definidas como conjuntos de segmentos que partilham determinadas propriedades fonéticas e fonológicas. No caso das consoantes, usamos as seguintes classes naturais, designadas com base na fonética articulatória e em traços distintivos:


\begin{enumerate}[label=(\roman*)]
 \item Modo de Articulação\\
    {[$–$soante; $–$contínuo]} (oclusivas)\\
    {[$–$soante; +contínuo]} (fricativas)\\
    {[+soante; $–$contínuo]} (consoantes nasais)\\ 
    {[+soante; +contínuo]} (líquidas)
\item  Ponto de Articulação\\
    labial (bilabiais e labiodentais)\\ 
    coronal [+anterior] (dentais e alveolares)\\
    coronal [$-$anterior] (palatais)\\
    dorsal (velares, uvulares)
\item Vozeamento\\
 {[$–$vozeado]} (surda)\\
 {[+vozeado]} (sonora)\\
\end{enumerate}

A Tabela \ref{tab:cap6tab1} contém as várias classes naturais referidas e os segmentos que as constituem.



\begin{sidewaystable}
\caption{Classes naturais de consoantes e sua caracterização em traços distintivos (cf. \citealp{Mateus2016})}
\label{tab:cap6tab1}
\footnotesize
\begin{tabularx}{\textwidth}{cQcccccccccccc}
%\RaiseImage \includegraphics[draft=false, width=\linewidth]{figures/Imagem41.png}\\
\lsptoprule
& & \multicolumn{12}{c}{\textbf{Ponto de Articulação}}\\
\cmidrule(lr){3-14}
 \multicolumn{2}{l}{\textbf{Classe Natural}} & \multicolumn{4}{c}{\textbf{Labial}} & \multicolumn{3}{c}{\textbf{Coronal} [+anterior]} & \multicolumn{2}{c}{\makecell{\textbf{Coronal}\\{[$-$anterior]}}} & \multicolumn{3}{c}{\textbf{Dorsal}}\\
\cmidrule(lr){3-6}\cmidrule(lr){7-9}\cmidrule(lr){10-11}\cmidrule(lr){12-14}
\multicolumn{2}{l}{[traços]} & \multicolumn{2}{c}{\textbf{Bilabial}} & \multicolumn{2}{c}{\textbf{Labio-dental}} & \multicolumn{2}{c}{\textbf{Dental}} & \textbf{Alveolar} & \multicolumn{2}{c}{\textbf{Palatal}} & \multicolumn{2}{c}{\textbf{Velar}} & \textbf{Uvular}\\
& & [$-$voz] & [+voz] & [$-$voz] & [+voz] & [$-$voz] & [+voz] & & [$-$voz] & [+voz] & [$-$voz] & [+voz] & \\
\midrule
\multirow{10}{*}{\rotatebox[origin=c]{90}{Modo de Articulação}} & \textbf{Oclusivas} & p &  b & & & t & d & & & & k & g & \\
& [$–$soante; $–$contínuo] & & & & & & & & & & & & \\
& \textbf{Nasais} & \multicolumn{2}{c}{m} & & & & & n & \multicolumn{2}{c}{ɲ} & & & \\
& [+soante; $–$contínuo] & & & & & & & & & & & & \\
& \textbf{Fricativas} & & & f & v & s & z & & ʃ & ʒ & & & \\
& [$–$soante; +contínuo] & & & & & & & & & & & & \\
& \textbf{Laterais} & & & & & & & l & \multicolumn{2}{c}{ʎ} & & & \\
& [+soante; +contínuo] [lateral] & & & & & & & & & & & & \\
& \textbf{Vibrantes} & & & & & & & ɾ & & & & & ʀ \\
& [+soante; +contínuo] & & & & & & & & & & & & \\
\lspbottomrule
\end{tabularx}
\end{sidewaystable}


A aquisição das consoantes no desenvolvimento infantil em português segue, genericamente, as seguintes ordens de complexidade crescente, consentâneas com tendências universais: 

\begin{enumerate}[label=(\roman*)]
\item Modo de Articulação:\\
    oclusivas; consoantes nasais $\gg$ fricativas $\gg$ líquidas;
\item  Ponto de Articulação: \\
    anteriores (labiais, dentais, alveolares) $\gg$ posteriores (palatais, velares, uvular);
\item Vozeamento:\\
    surdas $\gg$ sonoras.
\end{enumerate}

Porém, estas tendências não são invioláveis e desde há muito sabemos que a aquisição de uma dada consoante depende da coocorrência de traços de ponto de articulação, modo de articulação e vozeamento (cf. \citetv{chapters/04}, \citetv{chapters/05}), bem como do seu estatuto prosódico, como veremos a seguir.

Embora menos estudadas, as vogais e as semivogais integram igualmente o inventário segmental de uma língua. Caracterizamo-las em função do grau de altura e do ponto de articulação:


\begin{enumerate}[label=(\roman*)]
\item Altura\\
    {[+alto]} (altas)\\
    {[$–$alto; $–$baixo]} (médias)\\
    {[+baixo]} (baixas)
\item Ponto de Articulação\\
    coronal (anteriores)\\
    dorsal (centrais)\\
    labial (posteriores ou arredondadas)
\end{enumerate}

A Tabela \ref{tab:cap6tab2} contém as várias classes naturais acima referidas.



\begin{Tabela}
\caption{Classes naturais de vogais e semivogais e sua caracterização em traços distintivos (PE)}
\label{tab:cap6tab2}


\begin{tabularx}{.8\textwidth}{Xllll}
\lsptoprule
  & &  \multicolumn{3}{c}{\textbf{Ponto de Articulação}}\\
  \cmidrule(lr){3-5}
  & \textbf{Classe Natural} & \textbf{Anterior}  &\textbf{Central} & \textbf{Posterior} \\
  & [traços] & Coronal & Dorsal & Labial\\
  \midrule
  \parbox[t]{2mm}{\multirow{3}{*}{\rotatebox[origin=c]{90}{\textbf{Altura}}}} & \textbf{alto} [+alto] & i, ĩ, j, j &  ɨ & u, ũ, w, w̃ \\
  & \textbf{médio} [-alto; -baixo] & e, ӗ & ɐ, ɐ̃ & o, õ\\
  & \textbf{baixo} [+baixo] & ɛ & a & ɔ\\
\lspbottomrule
\end{tabularx}
\end{Tabela}



\subsection{Constituintes silábicos}

Os constituintes silábicos (Ataque, Rima, Núcleo, Coda), que integram a \emph{sílaba} (cf. Figura \ref{fig:cap6fig1}), exercem restrições sobre os segmentos que se lhes associam: em Ataque não ramificado simples, todas as consoantes do português são possíveis. Já em Ataque ramificado, são legítimas apenas sequências consonânticas de \emph{obstruinte} + \emph{líquida}. Em Coda, apenas líquidas alveolares e fricativas palatais são possíveis. O Núcleo pode ser ocupado por uma vogal ou por um ditongo. Na Tabela \ref{tab:cap6tab3} são listados todos os constituintes silábicos e segmentos por estes licenciados em PE.




\begin{Tabela}
\caption{Constituintes silábicos em PE}
\label{tab:cap6tab3}


\begin{tabularx}{\textwidth}{lQl}
\lsptoprule
\textbf{Constituinte} & \textbf{Segmentos licenciados} & \textbf{Exemplo}  \\
\midrule
\textbf{Ataque}&&\\
\tablevspace
Ataque não & Não ocorrência de consoante & [ˈ\_azɐ]\\
ramificado vazio &&\\
\tablevspace
Ataque não & Qualquer consoante & [ˈ\textbf{p}ato]\\
ramificado simples &&\\
\tablevspace
Ataque ramificado & Sequência de 2 consoantes: & [ˈ\textbf{pl}aʃtiku] \\
& C1 -- obstruintes (fricativas ou oclusivas) & [ˈ\textbf{fɾ}utɐ]\\
& C2 -- líquidas alveolares ([l] ou [ɾ]) &\\
\midrule
\multicolumn{2}{l}{\textbf{Rima} (só Núcleo ou Núcleo+Coda)}&\\
\tablevspace
Núcleo não & Todas as vogais & [ˈk\textbf{a}z\textbf{ɐ}]\\
ramificado &&\\
\tablevspace
Núcleo ramificado & Sequências de: & \\
& Vogal+Glide (ditongo decrescente) & [ˈp\textbf{aj}]\\
& Glide+Vogal (ditongo crescente) & [ˈp\textbf{ja}ɾ]\\
\midrule
\textbf{Coda} & Consoante em final de sílaba: &\\ 
& - líquida alveolar & [ˈka\textbf{ɫ}dɐ]\\
&&[ˈka\textbf{ɾ}tɐ]\\
\tablevspace
& - fricativa palatal (surda ou sonora) & [ˈpa\textbf{ʃ}tɐ]\\
&& [ˈsi\textbf{ʒ}nɨ]\\
\lspbottomrule
\end{tabularx}
\end{Tabela}


O facto de uma consoante fazer parte do inventário fonético da criança, e ser, portanto, produzida nos seus enunciados, não significa que esteja adquirida em todas as posições silábicas. Os dados abaixo são de uma criança portuguesa com 7;06 anos e diagnosticada com Perturbação de Desenvolvimento da Linguagem (PDL; termo recentemente proposto para designar as até aqui conhecidas como Perturbações Específicas de Linguagem – PEL). Mostram produção de [l] em Ataque simples mas não em Ataque ramificado ou em Coda: 


\ea	Produção de [l] em criança com PDL (\emph{corpus} PHONODIS -- \emph{criança} L -- 7;06)\\
\gllllll {} \textbf{Alvo}  {} \textbf{Produção da criança}  \\
    \emph{crocodilo} [kɾukuˈdilu] → [kukuˈdilu]\\
    \emph{livros}  [ˈlivɾuʃ]  → [ˈlifuʃ̪͆]\\
    \emph{golfinhos}  [goɫˈfiɲuʃ]  → [guˈfiɲuʃ]\\
    \emph{flor}  [ˈfloɾ]  → [ˈfolɨ]\\
    \emph{planta} [ˈplɐ̃tɐ] → [ˈpɐ̃tɐ]\\

\z

Estes dados ilustram o facto amplamente demonstrado para várias línguas de que um segmento pode estar adquirido num dado contexto silábico mas não noutro(s). Assim, nos dados acima, observamos um défice, não na aquisição do segmento [l], mas na aquisição da estrutura silábica (neste caso, os constituintes Coda e Ataque ramificado), generalização esta crucial para a planificação da intervenção, que deverá incluir vários tipos de Ataques Ramificados e de Codas, de forma a promover a aquisição destas estruturas silábicas.

As taxas de sucesso na Figura \ref{fig:cap6fig2}, de crianças portuguesas com desenvolvimento típico, ilustram a aquisição gradual dos segmentos (no caso, [ɾ]) em função do seu estatuto silábico (em Ataque Simples, como em \emph{pu\uline{r}é}; em Ataque Ramificado, como em \emph{\uline{br}uxa}; em Coda, como em \emph{po\uline{r}ta}):




\begin{Figura}
    %\includegraphics[draft=false, width=\linewidth]{figures/Imagem42.png}
    \begin{tikzpicture}
    \begin{axis}[ybar=5pt,
                ylabel = {\%},
                enlarge x limits={.4},
                %width  = \textwidth,
                height = .5\textheight,
                axis lines*=left,
                ymin = 0,
                ymax = 100,
                /pgf/number format/use comma,
                xtick = {1,2,3},
                xticklabels={[ɾ] AS,
                [ɾ] AR,
                [ɾ] Cd},
            nodes near coords,
            legend style={at={(0.5,-0.1)},anchor=north},
            %legend pos = north east,
            legend columns={-1},
            ]
        \addplot+[lsYellow]
	coordinates{
	(1,80)
	(2,21.5)
	(3,41.7)
	};
	\addplot+[lsLightOrange]
	coordinates{
	(1,90.3)
	(2,48.2)
	(3,64.7)
	};
	\addplot+[lsMidOrange]
	coordinates{
	(1,96.3)
	(2,66.6)
	(3,78.9)
	};
        \legend{G1 [3;00--4;00],G2 [4;00--5;00],G3 [5;00--6;00]}
    \end{axis}
    \end{tikzpicture}
    \caption{Taxas de sucesso para [ɾ] em contexto típico, em função do constituinte silábico \citep{Ramalho2017}.}
    \label{fig:cap6fig2}
\end{Figura}


Os dados acima mostram que a aquisição de [ɾ] varia em função da posição silábica em que os segmentos ocorrem, observando-se a seguinte tendência de estabilização: ataque simples$\gg$coda$\gg$ ataque ramificado.

Resultados relatados em \citet{Lousada2012a} e \citet{Ramalho2017a} mostram a pertinência do uso da relação entre segmentos e constituintes silábicos na avaliação de produções de crianças diagnosticadas com Perturbação dos Sons da Fala ou Perturbação do Desenvolvimento da Linguagem, com comprometimento do módulo fonológico.

\subsection{Acento de palavra}
O acento de palavra (associado ao constituinte \emph{pé} (cf. Figura \ref{fig:cap6fig1})) torna uma sílaba percetivamente mais proeminente, por oposição à(s) restante(s) sílaba(s) da palavra: a sílaba tónica (forte, a negrito) contrasta, assim, com a(s) sílaba(s) átona(s) (fraca(s)) numa palavra, como em \emph{gol\textbf{fi}nhos} e \emph{croco\textbf{di}lo}.

Identificamos 3 padrões acentuais no português: proparoxítono (esdrúxulas: \emph{música}); paroxítono (graves: \emph{palhaço}); oxítono (agudas: \emph{jacaré}). De acordo com \citet{Vigario2006}, obser\-vam-se as seguintes frequências de ocorrência no PE: paroxítonas (76,44\%) > oxítonas (21,6\%) > proparoxítonas (1,99\%). O padrão paroxítono é o mais frequente em português, logo, o de grau de complexidade menor, a ter em consideração na construção de tarefas de avaliação e de intervenção em contexto clínico.

O acento de palavra é um fator com impacto no processamento fonológico durante a aquisição \citep{Bernhardt2000}. As primeiras produções das crianças são habitualmente produções de sílabas tónicas das palavras ([ˈgɔ] \emph{gosta} (Inês: 1;05)), considerando-se que este é um contexto promotor da produção das estruturas em aquisição. Dados de várias línguas mostram que uma consoante tende a ser adquirida em sílaba tónica antes de o ser em sílaba átona. Na \figref{fig:cap6fig3}, são fornecidas taxas de sucesso para a Coda [ɾ] em posições tónica e átona, em crianças portuguesas com desenvolvimento típico:



\begin{Figura}
 %   \includegraphics[draft=false, width=\linewidth]{figures/Imagem43.png}
    \begin{tikzpicture}
    \begin{axis}[ybar=7pt,
                ylabel = {\%},
                enlarge x limits={.4},
                %width  = \textwidth,
                height = .5\textheight,
                axis lines*=left,
                ymin = 0,
                ymax = 100,
                /pgf/number format/use comma,
                xtick = {1,2,3},
                xticklabels={G1 [3;00--4;00],G2 [4;00--5;00],G3 [5;00--6;00]},
            nodes near coords,
            legend style={at={(0.5,-0.1)},anchor=north},
            %legend pos = north east,
            legend columns={-1},
            ]
        \addplot+[lsYellow]
	coordinates{
	(1,21.09)
	(2,50)
	(3,60.9)
	};
	\addplot+[lsMidOrange]
	coordinates{
	(1,50.9)
	(2,73.4)
	(3,89.5)
	};
        \legend{{$[$ɾ$]$ átono},{$[$ɾ$]$ tónico}}
    \end{axis}
    \end{tikzpicture}
    \caption{Taxas de sucesso para a Coda [ɾ] em contexto típico, em função do acento de palavra \citep{Ramalho2017}.}
    \label{fig:cap6fig3}
\end{Figura}


Os dados na Figura \ref{fig:cap6fig3} mostram, globalmente, um efeito promotor da posição tónica na aquisição de [ɾ], mais expressiva em idades iniciais, tendência que decorre da aquisição precoce do acento de palavra \citep{Correia2009, Santos2017}. 

Estudos como o de \citet{Marshall2009}, entre outros, mostram o efeito do acento de palavra na aquisição fonológica em crianças diagnosticadas com Perturbação de Desenvolvimento da Linguagem e com Dislexia.


\subsection{Posição na palavra}
No que diz respeito à variável \emph{posição na palavra}, o comportamento dos segmentos pode ser descrito em função da sua distribuição na palavra (inicial ou medial para o Ataque; medial ou final para a Coda). Trata-se de um aspeto associado ao constituinte \emph{palavra prosódica} (cf. Figura \ref{fig:cap6fig1}). Sabemos, da investigação em psicolinguística, que a posição inicial de palavra promove o acesso lexical (por exemplo, é mais fácil ilustrar a presença de um som numa palavra selecionando itens com esse som em início de palavra). Tendencialmente, no desenvolvimento infantil, uma consoante em Ataque é adquirida primeiramente em início de palavra, antes de o ser em posição medial (\citealp{Costa2010}, embora a autora registe a aquisição de /l/ primeiramente em posição medial de palavra no PE, tal como observado em crianças brasileiras \citep{MatzenauerHernandorena1997}.

No caso da Coda fricativa, sabemos que esta pode ocorrer em português nas posições medial e final (\emph{pa\textbf{s}ta}, \emph{palhaço\textbf{s}}). Está amplamente descrito o facto de as crianças portuguesas adquirirem a Coda fricativa final em primeiro lugar, por volta dos 2;0, muito antes da sua aquisição em posição medial, na faixa etária dos 3;6--3;11 \citep{Freitas1997, Mendes2013a, Amorim2014, Baptista2015, Ramalho2017}, facto que pode estar associado à aquisição precoce da marcação do plural e ilustrar a interface entre módulos gramaticais (neste caso, fonologia -- morfos\-sintaxe) na aquisição.

Resultados relatados em \citet{Lousada2012a} mostram a pertinência do uso da relação entre os segmentos e a sua distribuição na palavra na avaliação de produções de crianças diagnosticadas com Perturbação de Desenvolvimento da Linguagem.


\subsection{Extensão de palavra}
\largerpage
A variável \emph{extensão de palavra}, também associada à palavra prosódica (cf. \figref{fig:cap6fig1}), é medida em número de sílabas numa palavra. Esta variável tem vindo a ser descrita na literatura como relevante na avaliação fonológica (\citealp{Stackhouse1993, Bradford1996, Bernhardt2000}; entre outros), pese embora só mais recentemente venha a ser documentada para o PE \citep{Ramalho2017, Ramalho2017a}. Vários trabalhos internacionais têm demonstrado a sua produtividade na avaliação, associada ao desempenho de crianças em tarefas de soletração, de nomeação e de repetição de palavras e de pseudopalalavras (e.g. \citealp{James2006, Mason2015, Masso2017}; cf. \citetv{chapters/09}). Os resultados obtidos têm demonstrado um aumento da complexidade da tarefa associado ao aumento da extensão de palavra. Na Tabela \ref{tab:cap6tab4},  re\-gistam-se dados de frequência das diferentes extensões de palavra em PE, que mostram ser o dissílabo o formato de palavra mais frequente, logo, o de grau de complexidade menor na construção de tarefas de avaliação e de intervenção em contexto clínico.




\begin{Tabela}
\caption{Extensão de palavra: distribuição das frequências relativas no PE.}
\label{tab:cap6tab4}


\begin{tabularx}{\textwidth}{Qrrrr}
\lsptoprule
& \textbf{\small{Monossílabos}} & \textbf{\small{Dissílabos}} & \textbf{\small{Trissílabos}} & \textbf{\small{Polissílabos}}\\
\midrule
Corpus TA90PE -- & 31,46\% &	\textbf{42,55\%} &	18,35\% &	7,6\%\\
 Produções de adultos & &&&  \\
 \citep{Vigario2005}&&&&\\
 \tablevspace
 Fala adulta não dirigida à crianç & 29,5\% &	\textbf{43,6\%} &	26,9\% &	< 5\%\\
 \citep{Vigario2006} &&&&\\
 \tablevspace
 Fala adulta dirigida à criança & 43,9\%	& \textbf{46,6\%} &	8\% &	7,4\%\\
 \citep{Vigario2006}\\
 \tablevspace
 Fala infantil & 28,6\% &	\textbf{46,6\%} &	19\% &	--\\
 \citep{Vigario2006} &&&&\\
\lspbottomrule
\end{tabularx}
\end{Tabela}


A acumulação de complexidade associada às variáveis acima descritas implica o aumento de complexidade fonológica. Muitos polissílabos, além de complexos devido ao número elevado de sílabas, são-no também por acumularem estruturas complexas relativas à estrutura silábica, ao acento e à posição na palavra (veja-se um efeito semelhante na avaliação sintática, em \citetv{chapters/13}). Observem-se na Tabela \ref{tab:cap6tab5} as seguintes palavras com complexidade cumulativa, em que habitualmente as crianças demonstram comprometimento.



\begin{Tabela}
\caption{Palavras com diferentes tipos de complexidade fonológica (AR = Ataque ramificado; Cd = Coda)}
\label{tab:cap6tab5}


\begin{tabularx}{.95\textwidth}{lllll}
\lsptoprule
  & \textbf{Extensão de} & \textbf{Estrutura} & \textbf{Acento} & \textbf{Posição na}\\
  & \textbf{palavra} & \textbf{silábica} & &\textbf{palavra}\\
\midrule
  \emph{Crocodilo} & X & X & X & \uline{\phantom{xxxxxx}}\\
 & (polissílabo) & (AR) & (AR átono) &\\
 \tablevspace
 \emph{Frigorífico} & X & X & \uline{\phantom{xxxxxx}} & \uline{\phantom{xxxxxx}}\\
 & (polissílabo) &  (AR) &  &  \\
 \tablevspace
 \emph{Fotografia} & X & X & X & X\\
 & (polissílabo) &  (AR) & (AR átono) & (AR medial)\\
 \tablevspace
 \emph{Retroescavadora} & X & X & X & X\\
 & (polissílabo) &  (AR) & (AR átono;& (AR medial;\\
 &&& Cd átona) & Cd medial)\\
\lspbottomrule
 \end{tabularx}
 \end{Tabela}


 Por fim, registamos na Tabela \ref{tab:cap6tab6} estudos sobre desenvolvimento atípico que têm mostrado a adequação da fonologia não linear e a produtividade das variáveis aqui focadas na avaliação em contexto clínico, sendo apontadas como potenciais marcadores clínicos.



\begin{Tabela}
\caption{Estudos sobre a produtividade de variáveis fonológicas na discriminação de diferentes patologias.}
\label{tab:cap6tab6}


\begin{tabularx}{\textwidth}{p{2.1cm}p{3.6cm}Q}
\lsptoprule
\textbf{Estudo} & \textbf{Tipo de perturbação} & \textbf{Variáveis fonológicas}\\
\midrule
\citet{Marshall2009} & Crianças com PDL e {crianças} com Dislexia &  Efeito da posição na palavra e do acento de palavra.\\
\tablevspace
\citet{BlancoDutra2009} & Crianças com {Perturbação} Fonológica (em terapia) & Efeito de acento de palavra, {extensão} de palavra, posição na palavra e estrutura silábica.\\
\tablevspace
\citet{Tamburelli2013} & Crianças com PDL & Efeito da estrutura silábica.\\
\tablevspace
\citet{Demuth2014} & Crianças com PDL & Efeito do licenciamento prosódico.\\
\tablevspace
\citet{Ferre2015} & Crianças com PDL & Efeito da estrutura silábica.\\
\tablevspace
\citet{Masso2017} & Crianças com {Perturbação} dos sons da fala de base fonológica & Efeito da extensão de palavra.\\
\tablevspace
\citet{Lousada2012a} & Alterações fonológicas em crianças com PDL  & Efeito de propriedades {segmentais} (vozeamento), acento de palavra, estrutura silábica e posição na palavra.\\
\tablevspace
\citet{Baptista2015} & Crianças com otite {média} com derrame & Efeito de propriedades {segmentais} (modo de articulação e vozeamento), de estrutura silábica e de posição na palavra.\\
\tablevspace
\citet{Ramalho2017a} & Crianças com PDL e PSF (atraso fonológico) ({estudo} preliminar com 3 crianças) & Efeito de estrutura silábica, de acento de palavra, de posição na palavra e de extensão de palavra.\\
\lspbottomrule
\end{tabularx}
\end{Tabela}


Na próxima secção, daremos exemplos de instrumentos e medidas de avaliação que têm na base o modelo teórico da fonologia não linear ou que podem ser usados nesta perspetiva.

\section{Instrumentos de avaliação}\label{sec:cap6sec2}
A construção de instrumentos que permitam testar as variáveis fonológicas relevantes na língua e que sejam, simultaneamente, práticos e fáceis de usar em contexto clínico tem constituído um desafio no que ao desenvolvimento de instrumentos de avaliação diz respeito, por parte de terapeutas da fala ou fonoaudiólogos, linguistas e investigadores. 

Existem inúmeros instrumentos desenvolvidos para o português (cf. \citealp{Viana2017} e \citealp{Ramalho2017} para descrições e breves análises dos diversos instrumentos de avaliação fonológica disponíveis em Portugal; para instrumentos disponíveis no Brasil, consultem-se, \citetv{chapters/04}, \citetv{chapters/03}). Os testes de avaliação de produção mais usados no PE são o TFF-ALPE \citealp{Mendes2013a} e o TAV \citep{Guimaraes2014}. O TFF-ALPE sugere o uso de processos fonológicos como modelo de análise fonológica. O TAV sugere uma análise segmental, baseada na análise SODA (\emph{Substituições, Omissões, Distorções} e \emph{Adições}). Apesar de alguns segmentos serem testados em diferentes estruturas silábicas e posições na palavra, não há, em ambos os instrumentos, uma proposta de análise que contemple explicitamente aspetos como \emph{constituintes silábicos}, \emph{acento} ou \emph{extensão de palavra}.

Face à necessidade de avaliar a fonologia numa perspetiva multidimensional, foi desenvolvido para o PE um instrumento de avaliação da fonologia de base não linear, o CLCP-PE \citep{Ramalho2014, Ramalho2017}, no âmbito do projeto \emph{Crosslinguistic Child Phonology Project} (CLCP), coordenado por \citet{Bernhardt2000}. O CLCP-PE 1.2 está validado para o PE e integra 150 palavras e 42 cenários de imagens, a partir dos quais a criança pode nomear as palavras-alvo presentes no teste. Os alvos foram selecionados em função de variáveis fonológicas que se mostraram, globalmente, relevantes e produtivas para o PE, a saber: \emph{inventário segmental} (e sua organização em classes naturais), \emph{constituência silábica}, \emph{acento de palavra}, \emph{posição na palavra} e \emph{extensão de palavra}. O teste permite, além da nomeação dos itens, o reconto da história e a realização de tarefas de repetição, tarefas que devem ser consideradas no contexto da avaliação da produção \citep{Lousada2017a, Bates2017}.

A necessidade de quantificar a severidade da perturbação/desvio, bem como a de avaliar a eficácia da intervenção, tem sido uma preocupação em contexto clínico. Com efeito, têm vindo a ser propostas medidas quantitativas que permitem aos profissionais quantificar a alteração fonológica e traduzir, de forma mais qualitativa, a complexidade fonológica associada à alteração (cf. \citetv{chapters/03}, \citetv{chapters/04}, \citetv{chapters/05}, \citetv{chapters/07}).

Em contexto português, o TFF-ALPE apela ao uso da percentagem de consoantes corretas (PCC), percentagem de vogais corretas (PVC) e percentagem de ocorrência dos processos fonológicos (POPF). As medidas PCC e PCV têm sido amplamente descritas e revistas desde a sua proposta inicial \citep{Shriberg1982a} mas centram-se sobretudo em aspetos segmentais. De forma a facilitar o tratamento dos dados e a obtenção das medidas fonológicas, \citet{Jesus2013} disponibilizaram o instrumento Ferramentas para Análise Fonológica Automática (FAFA), disponibilizadas \emph{online} em formato de folha de cálculo, que permitem analisar os seguintes parâmetros fonéticos e fonológicos: \emph{inventário fonético, percentagem de consoantes corretas, percentagem de vogais corretas, processos fonológicos, percentagem de fonemas corretos, inventário de estruturas silábicas} e \emph{percentagem de estruturas silábicas corretas}.

Embora as medidas acima sejam válidas e revelantes em contexto clínico e de investigação \citep{Saraiva2017}, de há uns anos a esta parte, têm sido propostas outras medidas, centradas na palavra, as designadas \emph{whole-word measures} (WWM), que possibilitam a exploração do sistema fonológico para além do segmento, através da discriminação de aspetos como a complexidade segmental, a complexidade silábica, a complexidade fonotática e a sua combinação em palavras (\citealp{Ingram2002, StoelGammon2010, Arias2013, Mason2015}; entre outros). Estas medidas, que traduzem a complexidade prosódica na palavra, têm vindo a demonstrar capacidade discriminativa na análise, sendo candidatos robustos para a diferenciação de quadros clínicos (\citealp{Ramalho2017a, Bernhardt2020}; entre outros). Alguns exemplos de medidas centradas na palavra são: 

\begin{enumerate}
    \item [a)] \emph{Whole Word Accuracy} (WWA) \citep{Schmitt1983} ou \emph{Whole Word Match} \citep{Bernhardt2000, Ramalho2017} -- A taxa (ou percentagem) de acerto em palavra pode ser obtida a partir da aplicação de qualquer teste de avaliação dos sons da fala ou de uma amostra de discurso espontâneo, possibilitando estimar a competência da criança através da contabilização da taxa de acerto de produção em palavra. É calculada da seguinte forma: \emph{palavra produzida de forma incorreta/total de palavras produzidasx100};
    \item[b)] PMLU (\emph{Phonological Mean Length of Utterance}, \citealp{Ingram2002}) ou Extensão Média do Enunciado Fonológico (EMEF) -- Avalia a extensão média do enunciado fonológico através da medição do número de segmentos na palavra, por um lado, e da contabilização do número correto de consoantes. Integra medidas complementares de base não linear (e.g. \emph{word shape analysis} (WSA); \emph{proportion of whole word proximity} (PWP); \emph{proportion of whole word correctness} (PWC) (baseado na medida WWA \citep{Schmitt1983}; \emph{proportion of whole word variation} (PWV));
    \item[c)] MULTI-PCC \citep{Larrivee1999} -- Consiste na aplicação da PCC a palavras polissilábicas, através de prova de repetição de palavras e de pseudopalavras;
    \item[d)] PWP (\emph{phonological word proximity}) \citep{StoelGammon2010} -- O índice é contabilizado a partir do “tipo” de palavras usadas pela criança: palavras mais complexas têm um índice de complexidade fonológica superior (e.g. a palavra \emph{prateleira} tem um índice superior quando comparada com a palavra \emph{pato}). 
\end{enumerate}

O CLCP-PE propõe que, além da análise por classe natural, PCC/PVC por segmento e estrutura silábica, seja usada a taxa de acerto em palavra \citep{Bernhardt2000, Ramalho2017}, uma \emph{WWM}.

De forma a agilizar o processo de transcrição, contabilização e análise dos resultados, é sugerida a utilização do CLCP-PE em complementaridade com o \emph{software} de análise fonológica PHON\footnote{Para acesso ao \emph{download} bem como a outras informações sobre utilização do \emph{software}, consulte-se \url{https://www.phon.ca/phon-manual/getting_started.html}}  \citep{Hedlund2019}. O PHON possibilita aos utilizadores a introdução de informações diversas (informações pessoais, transcrição ortográfica, transcrição fonética alvo e produção da criança (cf. \sectref{sec:cap6sec3} deste capítulo)), sendo possível associar ficheiros de som às transcrições fonéticas e realizar análise acústica através da ligação ao programa Praat \citep{Boersma2020}. O PHON permite obter análises qualitativas e quantitativas, espelhadas em medidas como: PCC, PCV, análise com base em processos fonológicos, PMLU, WWM, \emph{syllable shape match}, estratégias de reconstrução (por omissão, substituição, adição), entre outras.

Nesta secção, pretendemos evidenciar a importância dos diferentes níveis de análise fonológica não linear e o impacto deste tipo de análise no trabalho em fonologia clínica, particularmente na avaliação (e intervenção, tratada em \citetv{chapters/07}) junto de crianças com alterações fonológicas. Independentemente do teste utilizado pelo/a terapeuta ou fonoaudiólogo/a, será sempre relevante que o profissional, ao selecionar e analisar as palavras (alvos) produzidas, tenha em consideração os diferentes níveis de complexidade fonológica subjacente a esses alvos. 

Na secção que se segue, faremos referência a metodologias usadas para a recolha dos dados de fala, aspeto crucial para o rigor a conferir a qualquer tarefa de avaliação.

\section{Recolha de dados de fala}\label{sec:cap6sec3}
Depois de apresentados aspetos relevantes no que diz respeito à avaliação fonológica não linear, importa colocar em destaque alguns aspetos essenciais relativos à forma como devem ser recolhidos e transcritos os dados de fala que servirão de base à análise fonológica apresentada ao longo deste capítulo.

A realização de uma transcrição fonética cuidada deve ser encarada como uma componente importante no processo de recolha e apresentação dos dados, sendo que os procedimentos adotados durante esta tarefa poderão influenciar o processo de análise e, consequentemente, a interpretação das produções de fala recolhidas.

De modo a que a transcrição fonética possa ser considerada como uma ferramenta útil e efetiva, é essencial que sejam explicitados aspetos relacionados com a sua aplicação em contexto clínico, pelo que se impõe apresentar algumas considerações sobre a recolha de dados para a avaliação. A este propósito, cita-se um documento emanado do \emph{Royal College of Speech \& Language Therapists}, no qual são apresentadas boas práticas para a transcrição fonética de amostras de fala de crianças nos contextos de prática clínica e de investigação \citep{Bates2017}.

A variabilidade da fala humana é uma realidade e a transcrição das produções de uma criança e/ou de produções alteradas continua a ser uma tarefa desafiante \citep{Stemberger2019}, mas essencial para o processo de avaliação. Sendo a transcrição fonética a primeira etapa no processo de avaliação, é importante que exista um elevado grau de precisão e de confiabilidade nesta tarefa, pois será a base para a formulação de hipóteses acerca do comportamento do falante \citep{Howard2002}. Transcrever com precisão deve ser condição aplicada não só ao registo de segmentos consonânticos, mas também à transcrição dos segmentos vocálicos, sendo importante ter em consideração a influência que o contexto vocálico tem na produção das consoantes, mas também as realizações das próprias vogais. A amostra de fala recolhida deve, assim, permitir que seja também considerada a forma como a criança produz as vogais num conjunto alargado de contextos \citep{Bates2017}, uma vez que problemas na produção das vogais podem refletir-se num maior comprometimento ao nível da inteligibilidade do discurso, sendo estas, muitas vezes, um indicador de alterações inerentes a um maior grau de gravidade \citep{Bowen2015, Shriberg2012}. Situações em que a precisão da transcrição fonética não esteja garantida podem colocar em risco uma adequada definição do diagnóstico terapêutico e todo o planeamento da intervenção \citep{Bates2017}.

O treino e a experiência de quem recolhe e regista as produções de fala são indispensáveis para a caracterização da amostra de fala, pelo que é relevante que o/a terapeuta da fala ou fonoaudiólogo/a, o/a linguista ou o/a investigador/a esteja familiarizado não só com os símbolos e diacríticos do Alfabeto Fonético Internacional (AFI) (\emph{International Phonetic Alphabet} (IPA)), em geral, como com os da sua comunidade linguística, em particular. Os símbolos do IPA permitem a transcrição fonética dos sons de todas as línguas do mundo, na ausência de alterações/ perturbações, o que, em algumas situações, pode representar um obstáculo à transcrição de produções de fala com características atípicas. Existe, no entanto, disponível, no \emph{site} da \emph{International Phonetic Association}, um conjunto de símbolos que podem ser utilizados nas transcrições de produção de fala patológica (\emph{ExtIPA Symbols for Disordered Speech}). Neste sentido, importa referir que o treino e prática da transcrição fonética com objetivos clínicos deve acontecer, não só com recurso a exemplos de fala de sujeitos adultos sem patologia, mas também, e numa fase posterior, recorrendo a exemplos de produções de crianças com e sem perturbação.

No seguimento da ideia de que é necessário o treino de competências de transcrição fonética com fins clínicos, é sabido que, numa primeira etapa, é extremamente importante desenvolver a capacidade de observação (auditiva e visual) para que seja possível realizar um registo válido e confiável das produções. Simultaneamente, é esperado que este registo possa ser claro e inequívoco também numa fase posterior, não só para quem fez o registo, mas também para outros profissionais treinados para o efeito que consultem os registos.

É, assim, essencial estar atento a comportamentos fonéticos subtis e, eventualmente, atípicos, que possam ter implicações fonológicas significativas \citep{Howard2002}.

Ainda que a transcrição fonética possa não representar em pleno aquilo que o ouvinte regista com base na sua perceção, esta é uma tarefa que pode ser mais ou menos especificada, recorrendo, para isso, a procedimentos de transcrição fonética estreita ou larga, respetivamente. A transcrição fonética estreita caracteri\-za-se por apresentar um maior detalhe fonético, assinalando variações que ocorrem nos sons da fala em função do contexto, ou seja, procura ser mais próxima do contínuo sonoro. A transcrição fonética larga cinge-se a uma representação mais geral e distante do contínuo sonoro, fornecendo menor detalhe fonético \citep{Mateus2016}. Vejam-se os exemplos da Tabela \ref{tab:cap6tab7}.



\begin{Tabela}
\caption{Exemplos de transcrição fonética larga e estreita. }
\label{tab:cap6tab7}


\begin{tabularx}{\textwidth}{Ql}
\lsptoprule
     \textbf{Tipo de representação} & \textbf{Exemplo}\\
     \midrule
     Ortográfica &	< Os sapos apanharam as moscas. >\\
     \tablevspace
     Transcrição fonética larga &	[uʃ ˈsapuʃ ɐpɐˈɲaɾɐ̃w̃ ɐʃ ˈmoʃkɐʃ]\\
     \tablevspace
     Transcrição fonética estreita (produção típica) &	[uˈʃapuzɐpɐˈɲaɾɐ̃w̃ɐʒˈmoʃkɐʃ]\\
     \tablevspace
   Transcrição fonética estreita (produção atípica) &	[uˈʃ̪͆apuz̪͆ɐpɐˈɲaɾɐ̃w̃ɐʒ̪͆ˈmoʃ̪͆kɐʃ̪͆]
  \\
  \lspbottomrule
\end{tabularx}
\end{Tabela}


O detalhe das transcrições fonéticas em contexto clínico é muitas vezes questionado, dado o seu caráter moroso e trabalhoso. No entanto, não é possível prever, num primeiro contacto com produções atípicas, quais serão os aspetos que terão relevância clínica. Em casos de desenvolvimento fonológico atípico, nos quais frequentemente se encontram padrões que reduzem a inteligibilidade do discurso e que podem indicar comprometimentos mais ou menos complexos no sistema dos sons da fala, é essencial utilizar procedimentos de transcrição fonética que permitam uma compreensão ampla dos dados de fala recolhidos (em palavra isolada e em fala encadeada). Despender de algum tempo na fase inicial de recolha e de análise dos dados, que irá alimentar diretamente todo o processo de intervenção terapêutica, poderá ter um impacto importante na redução do tempo total de intervenção \citep{Howard2002}.

No que diz respeito às amostras de fala a transcrever, existem várias considerações importantes a fazer, visto que, embora a transcrição possa ocorrer ao mesmo tempo em que a criança está a falar, este é um procedimento que acarreta algumas limitações.

A transcrição fonética em tempo real, no momento da avaliação, é necessária e importante, sempre que possível, visto que esta é a forma mais eficiente de recolha dos dados recorrendo à informação auditiva e visual plena \citep{Bates2017}. Contudo, dados de fala confiáveis requerem gravação para posterior análise do conjunto completo de palavras, mesmo que ocorra transcrição no momento da avaliação \citep{Stemberger2019}.

Para além das pistas auditivas, tem sido demonstrado que a informação visual representa uma pista especialmente informativa em relação aos movimentos labiais e mandibulares \citep{Munhall2004}. Já em \citeyear{McGurk1976}, \citeauthor{McGurk1976}, apresentaram um fenómeno comumente designado por efeito de McGurk, que mostra que aspetos visuais associados à produção de fala podem ser tão relevantes como aspetos auditivos na perceção e categorização dos sons da fala. Veja-se o seguinte exemplo: ao observar um falante a produzir repetida e silenciosamente a sílaba “ga” enquanto se ouve uma gravação da mesma pessoa a dizer a sílaba “ba” repetidamente, o ouvinte pode percecionar a sílaba “da”, uma vez que o cérebro integra, em simultâneo, informação auditiva e visual. Assim, assume-se que a gravação audiovisual, sempre que possível, é preferível quando comparada com uma simples gravação áudio, garantindo que inclui informação auditiva e visual de forma sincronizada \citep{Stemberger2019}.

Como já referido anteriormente, é necessário dedicar tempo ao treino e prática das competências de transcrição fonética, visto que, em contexto de prática clínica, a avaliação da produção dos sons da fala depende, em grande medida, da avaliação percetiva/auditiva do clínico, não só por questões de conveniência, mas também pelo facto de ser um procedimento económico e relativamente robusto. Porém, por se tratar de uma tarefa que se baseia na perceção do ouvinte, e embora o sistema auditivo humano seja notável na sua capacidade de separar o sinal de fala do ruído e conseguir uma interpretação fonética desse sinal, existem limites. A perceção auditiva pode ser, por si só, insuficiente para detetar certos detalhes fonéticos das produções dos falantes, como por exemplo características específicas das consoantes oclusivas, a duração dos segmentos, o ruído de fricção, as frequências dos formantes de vogais e líquidas, a presença de vozeamento, entre outros (veja-se \citealp{Mateus2016} para propriedades acústicas segmentais dos sons da fala), pelo que é crucial que sejam feitos registos áudio (e vídeo, sempre que possível) das produções das crianças, para confrontar as transcrições feitas em tempo real com as transcrições em diferido.

Existem várias técnicas que, embora não substituam o papel da perceção humana, são complementares e informativas quanto aos aspetos “menos visíveis” da produção de fala. No sentido de poderem ser colmatadas algumas das limitações já enunciadas, a análise acústica pode ser um recurso complementar à avaliação, por permitir a identificação de propriedades acústicas que podem não ser detetadas pelo ouvido humano \citep{Kent2010, Wertzner2017, Stemberger2019}. A referência à análise acústica como complemento à avaliação em contexto de prática clínica surge também pelo facto de ser uma técnica acessível, a partir de \emph{software} gratuito, como é o caso do Praat \citep{Boersma2020}.

O papel da análise acústica dos sons da fala deve também ser considerado, na medida em que existe relação entre alguns parâmetros acústicos e articulatórios que contribuem para a caracterização e análise dos segmentos (e.g. valores de frequência do segundo formante (F2) estão relacionados com o avanço/recuo da língua, em que os valores mais elevados são característicos de sons produzidos com o avanço da língua; valores de frequência do primeiro formante (F1) relacionam-se com a altura da língua de forma inversa, ou seja, sons produzidos com a língua numa posição mais baixa apresentam valores de F1 mais altos) (\citealp{Kent2002, Mateus2016} veja-se \citetv{chapters/03}). A determinação destes detalhes pode ser relevante quer para a avaliação, quer para a reavaliação das evoluções terapêuticas, nas quais muitas vezes pequenas evoluções podem ser percebidas a partir da análise espectrográfica, ainda que não sejam audíveis. Veja-se o seguinte exemplo:  no estudo de \citet{PaganNeves2010}, com o objetivo de descrever as características acústicas das líquidas /l/ e /ɾ/, foram analisadas produções de 20 crianças com e sem perturbação fonológica. Os resultados mostraram que, apesar de auditivamente ser possível identificar a substituição do /ɾ/ por [l], acusticamente observaram-se diferenças entre a produção do [l] conforme ao alvo e a produção do [l] em substituição do /ɾ/. 

Deve ainda ser feita referência à utilização da análise acústica como uma técnica também importante na análise de aspetos prosódicos que correspondem à variação de propriedades acústicas como a duração, a intensidade e a frequência fundamental, presentes no sinal de fala, dado que a utilização do IPA permite apenas uma transcrição segmental do contínuo de fala, deixando uma lacuna no que à transcrição de aspetos suprassegmentais diz respeito (com exceção da representação do acento de palavra) \citep{Mateus2016}.

\subsection{Recursos}
Os processos de transcrição e de interpretação dos dados são facilitados se as ferramentas adequadas estiverem disponíveis. Assim, e relativamente aos recursos que podem ser utilizados para auxiliar no processo de treino e prática de transcrição fonética, optou-se, neste capítulo e em jeito de síntese, por identificar apenas alguns tópicos úteis que podem ser identificados através de pesquisas \emph{online}, em vez de listar \emph{links} e/ou exemplos de recursos que, pela necessidade de atualização e evolução, podem rapidamente ficar desatualizados. Deste modo, sugerem-se como relevantes alguns recursos:

\begin{itemize}[align=left]
\item[(i)] Quadro dos símbolos do IPA;
\item[(ii)] Extensões do IPA (extIPA) para amostras de fala com alterações;
\item[(iii)] Quadros interativos dos símbolos do IPA, com gravações áudio dos diferentes sons, com a função de transcrição fonética e outras informações sobre os símbolos fonéticos;
\item[(iv)] Quadros interativos dos símbolos do IPA que combinem informações auditivas e visuais dos diferentes sons (e.g. \emph{Seeing Speech} \citep{Lawson2018}); 
\item[(v)] Amostras de fala (típica, atípica, infantil, adulta) para treinar transcrição fonética (pesquisar informação sobre práticas de transcrição fonética, e.g. \emph{webfon} \citep{Bates2019});
\item[(vi)] \emph{Software} de gravação e análise de som, de acesso livre (e.g. \emph{Praat} \citep{Boersma2020}, \emph{Audacity} \citep{Team2020}, entre outros.
\end{itemize}
A utilização rotineira e adequada de recursos e instrumentos de avaliação de dados de fala requer treino e prática, como qualquer outra ferramenta de laboratório, e deve ser integrada desde a formação inicial dos profissionais que irão lidar com estes aspetos.

\section{Conclusão}
Defendemos, na \sectref{sec:cap6sec1} deste capítulo, uma metodologia de avaliação do desenvolvimento fonológico infantil baseada no modelo teórico da fonologia não linear, listando variáveis amplamente testadas e com eficácia sobejamente demonstrada, defendendo o seu uso generalizado na análise de amostras de fala produzidas por crianças com desenvolvimento atípico. Na \sectref{sec:cap6sec2}, ilustrámos o uso dessas mesmas variáveis em contexto de avaliação do desenvolvimento fonológico em crianças com perfis linguísticos típico e atípico. Na \sectref{sec:cap6sec3}, chamámos a atenção para os cuidados a ter na recolha e tratamento dos dados de fala que servirão de base à análise fonológica, no sentido de proceder a diagnósticos rigorosos, que certamente contribuirão para a eficácia das intervenções terapêuticas.

Nas últimas décadas, o diálogo entre terapeutas da fala ou fonoaudiólogos/as e linguistas tem-nos permitido testar a adequação da fonologia não linear à representação do conhecimento fonológico dos falantes, definir padrões para o desenvolvimento típico e contribuir para a construção de instrumentos de avaliação e de intervenção cada vez mais rigorosos e eficazes, para uso em contexto clínico. O diálogo entre estes profissionais tem sido muito produtivo no sentido do desenvolvimento da Fonologia Clínica tanto no Brasil como em Portugal. Sublinhe-se, no entanto, que os estudos na perspetiva da Fonologia Clínica focados no português tiveram início no Brasil, com os trabalhos seminais de \citet{Lamprecht1986}, \citet{Yavas1988} e \citet{Matzenauer‐Hernandorena1990}, sendo os autores portugueses aqui citados herdeiros de uma ponte frutuosa entre terapeutas da fala  ou fonoaudiólogos/as e linguistas, observável noutras comunidades de profissionais e investigadores na área da Fonologia Clínica (cf. \citealp{Bernhardt2000} e \citealp{Dinnsen2016} para estados da arte). 

\section*{Agradecimentos}
Financiamento FCT pelo projeto estratégico UID/LIN/00214/2019.

{\sloppy\printbibliography[heading=subbibliography,notkeyword=this]}
\end{document}
