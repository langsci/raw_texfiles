\documentclass[output=paper,colorlinks,citecolor=brown,booklanguage=portuguese]{langscibook} 
\ChapterDOI{10.5281/zenodo.7233233}
\title{Processamento auditivo: da avaliação à intervenção}
\author{Ana Catarina Baptista \affiliation{Universidade do Algarve; Centro de Linguística da Universidade de Lisboa} and  Catarina Oliveira\affiliation{Universidade de Aveiro} and  Jorge Humberto Martins\affiliation{Centro Hospitalar Universitário de Coimbra, E.P.E}}

\abstract{É inegável e amplamente estudado o contributo da audição nos
processos comunicativos, especialmente em idade pediátrica e
particularmente no que diz respeito ao desenvolvimento linguístico,
nas modalidades oral e escrita. Contudo, a referência aos aspetos
auditivos que se ocupam da vertente periférica da audição são mais
frequentemente encontradas na literatura do que as questões auditivas
de processamento (central). Tão importante quanto perceber se a
pessoa é capaz de detetar o som, é fundamental compreender a sua
capacidade para o interpretar. Neste capítulo será abordado o
processamento auditivo em idade pediátrica, fazendo-se um breve
enquadramento conceptual, apresentando-se de forma sumária os
principais aspetos da fisiologia da audição. Serão apresentados os
principais fatores de risco e etiologias das perturbações do processamento auditivo, sendo caracterizado o perfil destas crianças.
Por fim, são apresentados argumentos que validem a relevância de
uma avaliação criteriosa que permita uma intervenção eficiente nas
crianças com perturbações do processamento auditivo.

\textbf{Palavras-chave}: audição; processamento auditivo; avaliação;
intervenção.}

\IfFileExists{../localcommands.tex}{
   \addbibresource{../localbibliography.bib}
   % add all extra packages you need to load to this file

\usepackage{tabularx,multicol}
\usepackage{url}
\urlstyle{same}

\usepackage{listings}
\lstset{basicstyle=\ttfamily,tabsize=2,breaklines=true}

\usepackage{langsci-basic}
\usepackage{langsci-optional}
\usepackage{langsci-lgr}
\usepackage{langsci-osl}
% \usepackage{./langsci/styles/langsci-lgr}
% \usepackage{./langsci/styles/langsci-osl}
% \usepackage{langsci-gb4e}

\usepackage{tikz}
\usetikzlibrary{patterns,calc}
\pgfdeclarepatternformonly{south east lines}{\pgfqpoint{-0pt}{-0pt}}{\pgfqpoint{3pt}{3pt}}{\pgfqpoint{3pt}{3pt}}{
    \pgfsetlinewidth{0.6pt}
    \pgfpathmoveto{\pgfqpoint{0pt}{3pt}}
    \pgfpathlineto{\pgfqpoint{3pt}{0pt}}
    \pgfpathmoveto{\pgfqpoint{.2pt}{-.2pt}}
    \pgfpathlineto{\pgfqpoint{-.2pt}{.2pt}}
    \pgfpathmoveto{\pgfqpoint{3.2pt}{2.8pt}}
    \pgfpathlineto{\pgfqpoint{2.8pt}{3.2pt}}
    \pgfusepath{stroke}}
    
\usepackage{stmaryrd}
\usepackage{wasysym}
\usepackage{multirow}
\usepackage{caption}
\usepackage{subcaption}
\usepackage{mathrsfs}
\usepackage{qtree}

\usepackage{linguex}


   %pminos do not split footnotes
% \interfootnotelinepenalty=10000 %Footnote in Laporte chapters has to be split SN


%\DeclareIndexNameFormat{default}{%
%\nameparts{#1}%
%\usebibmacro{index:name}%
%{\index[names]}%
%{\namepartfamily}%
%{\namepartgiveni}%
% {}% L1
% {}% L2
%{\namepartprefix}% generates spurious space L3
%{\namepartsuffix}% generates spurious space L4
%}

%  {\DeclareIndexNameFormat{default}{%
%     \usebibmacro{index:name}{\index[names]}{#1}{#3}{#5}{#7}}}

%\DeclareIndexNameFormat{default}{%
%  \usebibmacro{index:name}{\sindex[nom]}{#1}{#3}{#5}{#7}}

%\DeclareIndexNameFormat{default}{%
%  \usebibmacro{index:name}{\sindex[person]}{#1}{#3}{#5}{#7}}
%\DeclareIndexNameFormat{default}{%
%\nameparts{#1} \usebibmacro{index:name}{\sindex[person]]}{\namepartfamily}{‌​\namepartgiven}{\nam‌​epartprefix}{\namepa‌​rtsuffix}}

%\newcommand{\smiley}{:)}

%\renewbibmacro*{index:name}[5]{%
%\usebibmacro{index:entry}{#1}%
%{\iffieldundef{usera}{}{\thefield{usera}\actualoperator}\mkbibindexname{#2}{#3}{#4}{#5}}}

% \newcommand{\noop}[1]{}

%remove for final
%\overfullrule=1mm

\newcommand{\tobi}[2]}}
\renewcommand{\S}[1]{\tobi{#1}{\textsc{*}}}

% this volume references
% puts: [this volume]
% already defined: \citetv
%\newcommand{\citepv}[1]{(\citeauthor{#1} \citeyear*{#1} [this volume])}
\newcommand{\citealtv}[1]{\citeauthor{#1} \citeyear*{#1} [this volume]}

%parentheses around example number
\newcommand{\pref}[1]{(\ref{#1})}

% in-text examples

\newcommand{\lnex}[1]{\textit{#1}} %target lang word
\newcommand{\lnlit}[1]{(lit.: `#1')} %literal reading
\newcommand{\lnlat}[1]{(#1)} % latinization
\newcommand{\lntrans}[1]{`#1'} %translation
\newcommand{\lnexl}[2]%
{\lnex{#1}{} \lnlat{#2}} % ex with latinization
\newcommand{\lnexlat}[3]{\lnex{#1}{} \lnlat{#2}{} \lntrans{#3}} % ex with latinization and tranl.

%ch01
\newcommand{\co}[1]{\mbox{\textbf{#1}}}

%ch09

\newcommand{\cyrbulg}[1]{\begin{otherlanguage*}{bulgarian}#1\end{otherlanguage*}}


%ch10
\newcommand{\nlp}{{\small NLP}}
\newcommand{\mwe}{{\small MWE}}
\newcommand{\rae}{{\small RAE}}
\newcommand{\lvc}{{\small LVC}}
\newcommand{\pos}{{\small P}o{\small S}}
%\newcommand{\todo}[1]{ \textcolor{red}{#1} }

%\renewcommand{\labelenumi}{\theenumi}
%\ainamefmt{{vv}{ll}{, ff}{, jj}} % fullname

\newcommand{\biberror}[1]{{\color{red}#1}}

\newcommand{\osenovaitem}{--~}
   %% hyphenation points for line breaks
%% Normally, automatic hyphenation in LaTeX is very good
%% If a word is mis-hyphenated, add it to this file
%%
%% add information to TeX file before \begin{document} with:
%% %% hyphenation points for line breaks
%% Normally, automatic hyphenation in LaTeX is very good
%% If a word is mis-hyphenated, add it to this file
%%
%% add information to TeX file before \begin{document} with:
%% %% hyphenation points for line breaks
%% Normally, automatic hyphenation in LaTeX is very good
%% If a word is mis-hyphenated, add it to this file
%%
%% add information to TeX file before \begin{document} with:
%% \include{localhyphenation}
\hyphenation{
    Beck-man
    Ngu-yen
    back-chan-nel
    back-chan-nels
    mo-not-o-nous
    ste-reo-typ-i-cal
}

\hyphenation{
    Beck-man
    Ngu-yen
    back-chan-nel
    back-chan-nels
    mo-not-o-nous
    ste-reo-typ-i-cal
}

\hyphenation{
    Beck-man
    Ngu-yen
    back-chan-nel
    back-chan-nels
    mo-not-o-nous
    ste-reo-typ-i-cal
}

   \boolfalse{bookcompile}
   \togglepaper[11]%%chapternumber
}{}

\begin{document}
\maketitle

\section{Definição conceptual de processamento auditivo}
A audição é uma função sensorial importante para o desenvolvimento da linguagem e cumpre um papel fundamental na comunicação oral e escrita \citep{Terto2011}.\footnote{A norma adotada na escrita deste capítulo foi a do português europeu.}
% \footnote{Na sequência de restrições decorrentes do template usado para publicação na editora Language Science Press, os capítulos que integram o volume surgem com a referência [this volume], em inglês.}
A tarefa de ouvir não deve ser encarada como uma simples deteção de um sinal acústico, uma vez que envolve circuitos de funcionamento neuronais e sistemas cognitivos cruciais para descodificar, compreender, apurar e analisar o som. Deste modo, compreende-se que as experiências sensoriais são fundamentais para a maturação da via auditiva e para o processamento auditivo, desempenhando um papel fulcral no desenvolvimento linguístico.\footnote{Financiamento FCT pelo projeto estratégico UID/LIN/00214/2019.}

É inegável que a maioria das crianças adquire as suas capacidades linguísticas de forma rápida e eficiente nos primeiros anos de vida, sem que seja necessário algum tipo de ensino formal explícito. Ao longo dos anos, a ciência tem vindo a demonstrar que o desenvolvimento linguístico começa muito cedo, verifican\-do-se que, antes de completar o seu primeiro ano de vida, o bebé já demonstra impressionantes capacidades \citep{Frota2017}, tais como:

\begin{enumerate}
    \item [i.]	ter sensibilidade à entoação do discurso, diferenciando um tom de zanga de um tom de alegria;  
\item[ii.]	reconhecer a importância da melodia, distinguindo uma afirmação de uma interrogação;
\item [iii.]	distinguir os sons da sua língua materna de sons de outras línguas;
\item[iv.]	detetar o início e o final de cada palavra, uma vez que aquilo que o bebé ouve é um continuum sonoro. 
\end{enumerate}
A manifestação destas competências durante os primeiros meses de vida do bebé mostra-nos que o 1º ano de vida constitui um “período crítico” para o desenvolvimento da linguagem, período este que pode definir-se como o momento ótimo em que o cérebro é capaz de melhor reestruturar os seus caminhos neuronais em função das experiências vividas \citep{Zhao2016}. Os “períodos críticos” também chamados de “períodos cruciais”, vêm realçar a importância da qualidade das experiências linguísticas do contexto em que o bebé/criança se insere, demonstrando a relação de interdependência existente entre os fatores hereditários para a linguagem e os fatores ambientais de comunicação verbal onde a criança se desenvolve \citep{SimSim2008}. 

O processamento auditivo (PA) desenvolve-se sobretudo nestes primeiros anos de vida e depende da integridade fisiológica e da estimulação ambiental que a criança recebe, na relação de interdependência já referida anteriormente. Contudo, apesar de a grande maioria das crianças conseguir alcançar tamanhos feitos linguísticos nos primeiros anos de vida, é também verdade que, para algumas crianças, a compreensão do \emph{continuum} sonoro de fala pode constituir um desafio enorme, sobretudo em situações desfavoráveis de escuta.  Assim, uma privação sensorial nesta fase da vida pode implicar a presença de riscos para o seu desenvolvimento, nomeadamente no que diz respeito ao processo de aquisição e desenvolvimento da linguagem e da fala, bem como para a sua aprendizagem, uma vez que é nesta fase que a criança vai aprender como ouvir.

\subsection{Breve enquadramento da fisiologia da audição}
O mecanismo de ouvir é complexo e permite a compreensão da mensagem recebida, pelo que o ato de ouvir não termina com a deteção do estímulo acústico. Neste processo, são empregues inúmeros mecanismos neurofisiológicos e cognitivos para a descodificação, perceção, reconhecimento e interpretação do estímulo auditivo recebido \citep{Bellis2003}.

O ouvido humano consegue detetar uma gama frequencial e de intensidade bastante alargada, o que nos permite efetuar a discriminação de sons com pequena diferença frequencial e com cerca de 1 dB de intensidade. O sistema auditivo normal permite-nos compreender a fala em situações de ruído ambiente. Estas competências auditivas são o resultado do trabalho conjunto, efetuado com grande precisão, entre o sistema auditivo externo e o sistema auditivo central \citep{Musiek2012}.

O ouvido externo e o ouvido médio recolhem, amplificam e conduzem as ondas sonoras ao ouvido interno, onde se encontram os recetores auditivos, que irão ser estimulados. O ouvido interno envia, através do nervo auditivo, a informação para o sistema nervoso auditivo central (Figura \ref{fig:cap11-im1}).



\begin{Figura}
    \includegraphics{figures/74por.pdf}
\caption{Principais núcleos envolvidos na audição binaural \citep{Porter2012}.}
\label{fig:cap11-im1}
\end{Figura}


A informação proveniente do nervo auditivo passa para os núcleos cocleares, que são o primeiro centro integrador localizado na via auditiva primária, na parte inferior do tronco cerebral, nos pedúnculos cerebelosos inferiores \citep{Aquino2002, Pujol2003, Martinez2003, Moller2006, Seikel2010}. As fibras nervosas dos núcleos cocleares dirigem-se então, para o segundo centro integrador, o complexo olivar superior (COS). O COS é o primeiro centro da via contralateral e está dividido em três núcleos principais: o olivar superior lateral (COSL), o olivar superior médio (COSM) e o núcleo medial do corpo trapezóide \citep{Musiek1986a, Martinez2003, Neijenhuis2003}.

O lemnisco lateral é o maior feixe de fibras do sistema auditivo ascendente \citep{Moller2006}.

O colículo inferior (3º centro integrador) recebe um afluxo muito grande de fibras nervosas de ambas as olivas superiores laterais e, indiretamente, dos núcleos cocleares, através do lemnisco lateral, permitindo o seu envolvimento no mecanismo de localização sonora \citep{Pickles1982, Moller1983, Bonaldi2004, Musiek2006, Seikel2010}. O colículo inferior é responsável pelo processamento de sons com padrões temporais complexos, possuindo neurónios que só respondem a sons modulados em frequência, enquanto outros respondem a sons de durações específicas \citep{Gulick1989, Purves2005}. O colículo inferior possui neurónios sensíveis à diferença interaural de tempo, da mesma forma que o complexo olivar superior \citep{Seikel2010}.

O corpo geniculado médio (CGM) localiza-se na face dorso-lateral inferior do tálamo, quarto e último centro integrador antes do córtex, e recebe somente fibras ipsilaterais do colículo inferior. Todavia, tanto os neurónios do tálamo, como os do córtex, respondem à estimulação de ambos os ouvidos em 90\% das vezes, mostrando que a audição a este nível é predominantemente bilateral. Este quarto centro integrador realiza ainda um importante trabalho de integração/ preparação da resposta motora (por exemplo vocal) \citep{Aquino2002, Pujol2003}.

O corpo caloso (CC) é a maior comissura do encéfalo, está localizado na base da fissura longitudinal, e liga os dois hemisférios cerebrais, sendo responsável pela comunicação e integração da informação pelos dois hemisférios cerebrais \citep{Snell2001, Bellis2003}.

As áreas corticais de um hemisfério comunicam com a área homóloga do outro hemisfério através do corpo caloso. As áreas corticais auditivas comunicam entre si por meio de fibras que atravessam principalmente os dois terços posteriores do corpo caloso (a região posterior do corpo e anterior do esplénio do corpo caloso), recebem informação binaural e têm uma enorme capacidade de extrair um sinal sonoro de um ruído de fundo \citep{Musiek1986, Bellis2003, Moller2006}.

O córtex auditivo marca o fim da via auditiva ascendente. Aqui, a mensagem é reconhecida, memorizada e possivelmente integrada numa resposta motora, uma vez que foi sendo descodificada pelos núcleos inferiores \citep{Pujol2003}.

Na formação reticular do tronco cerebral, onde ocorrem inúmeras sinapses, as informações auditivas são confrontadas com outras informações sensoriais, cabendo aos centros de vigília e motivação, a seleção da modalidade sensorial prioritária, a cada momento \citep{Pujol2003}.

O processamento dos sinais auditivos implica um conjunto competências auditivas, designadamente, localização e lateralização sonora, discriminação auditiva, reconhecimento de padrões auditivos, processamento auditivo temporal (resolução, mascaramento, integração e ordenação temporais), desempenho auditivo com sinais acústicos competitivos (figura-fundo) e desempenho auditivo na presença de sinais acústicos degradados (fechamento) \citep{ASHA1996, Roggia2008, Yathiraj2018}. Para além destas, o PA inclui também as competências relacionadas com o processamento binaural, designadamente a escuta dicótica (separação binaural e integração binaural), a interação binaural e a integração interhemisférica \citep{Ferre2015b}.

É necessário compreender que a maturação de diferentes processos auditivos está associada a uma diferente maturação neuronal e que as competências auditivas que dependem de áreas cognitivas superiores possuem uma maturação mais tardia \citep{Yathiraj2015, Jain2015}.

\subsection{A Perturbação do Processamento Auditivo}
Descrita pela primeira vez em 1954 por Mykelbust (citado por \citealp{BSA2018}), a Perturbação do Processamento Auditivo (PPA) foi apresentada como a dificuldade para “estruturar o mundo auditivo”. Esta perturbação resulta de um défice do sistema nervoso central (SNC) para usar as informações enviadas pelo sistema auditivo periférico e tem como consequência dificuldades numa ou mais competências auditivas, sem comprometimento da audição periférica \citep{ASHA2005a, Bellis2008, Chermak2017, Ferre2015b, Iliadou2017, Koravand2017, Pereira2014}.

O diagnóstico de PPA é clínico, baseado numa anamnese cuidada e confirmado por testes comportamentais específicos, que devem ser cuidadosamente selecionados após a avaliação audiológica básica, composta por audiograma tonal, audiograma vocal, impedanciometria, reflexos acústicos ipsi e contralaterais, otoemissões acústicas e procedimentos eletrofisiológicos.

Relativamente aos vários estudos de prevalência da PPA, os valores apresentam alguma variação entre si. Assim, alguns estudos de prevalência da PPA realizados estimam que a sua prevalência esteja compreendida entre 2-3\% \citep{Chermak1997}, outros apontam para uma prevalência ligeiramente superior, com percentagens entre 3-5\% \citep{OBeirne2012}, com uma relação 2:1 entre indivíduos do sexo masculino e feminino. Outros autores apontam para uma prevalência de 7\% nas crianças \citep{Bamiou2001}, enquanto \citet{Saunders1992} indicam uma prevalência de 10\%.

\subsection{Etiologia e fatores de risco}
A etiologia da PPA pode constituir um enorme desafio para todos os profissionais que com ela lidam. Por este motivo, é muitas vezes descrita uma etiologia desconhecida, idiopática \citep{Bellis2015}.  Para além disso, a PPA existe frequentemente em comorbilidade com outras perturbações como as Perturbações da Linguagem, as Perturbações dos Sons da Fala, as Perturbações da Aprendizagem, o Défice de Atenção, entre outras, o que pode dificultar ainda mais o processo de definição da etiologia da PPA \citep{ASHA2005, BSA2018, Yathiraj2018, Zalcman2007}. 

Importa ainda clarificar que as dificuldades de processamento auditivo não devem ser encaradas como uma consequência das perturbações em outras áreas como as que foram apresentadas anteriormente \citep{ASHA2005, BSA2011, Nunes2015}, devendo recorrer-se ao termo de comorbilidade, sempre que se está perante a coexistência de dificuldades de linguagem e de audição \citep{Jerger2000}.

Diversas investigações têm procurado encontrar as causas da PPA, sendo referidas mais frequentemente as seguintes:

\begin{enumerate}
    \item [(i)] Prematuridade; 
\item[(ii)] Complicações durante os períodos pré, peri e pós-natais; 
\item[(iii)] Comprometimento na neuromaturação do sistema auditivo; 
\item[(iv)] Privação auditiva decorrente sobretudo de otites médias crónicas.
\end{enumerate}
No sentido de tentar simplificar e harmonizar a categorização da PPA, em 2007, a Sociedade Britânica de Audiologia categoriza as PPA em três grupos principais: 

\begin{enumerate}[align=left]
    \item [1.]  Perturbação de desenvolvimento do PA primária - manifesta-se na infância, com limiares auditivos dentro da normalidade, sem etiologia conhecida ou fatores de risco associados, podendo prolongar-se até à idade adulta em alguns indivíduos; pode ocorrer recuperação espontânea com a progressão da idade; 
\item[2.]	Perturbação adquirida do PA - normalmente associada a um evento patológico identificado ou conhecido, como lesões ou perturbações do SNC e/ou alterações degenerativas do sistema auditivo relacionadas com o envelhecimento;
\item[3.]	Perturbação secundária do PA - atraso de maturação do sistema auditivo central resultado de: hipoacusia periférica instalada; privação auditiva por falta de estimulação ou otites de repetição na infância; infeções congénitas (citomegalovírus, rubéola, sífilis, toxoplasmose, herpes); prematuridade \citep{BSA2007}.
\end{enumerate}

\subsection{As otites médias e a PPA}
Dentro dos três principais grupos etiológicos da PPA acima listados, destacam-se as otites médias, pela sua elevada prevalência, especialmente na primeira infância. A Otite Média Aguda (OMA) e a otite média serosa (OMS) (crónica) são as duas formas mais relevantes de OM, verificando-se um \emph{continuum} entre estas duas entidades clínicas.

Existem estudos que apontam para o facto de cerca de 80\% das crianças terem pelo menos um episódio de OMS até aos oito anos de idade \citep{Bluestone2004}. Esta condição é frequentemente acompanhada por perdas auditivas condutivas de grau leve a moderado, com um caráter flutuante e sem sintomas agudos como a febre e a dor. A perda auditiva resultante da OMS pode comprometer o desenvolvimento linguístico, especialmente em determinados períodos críticos do desenvolvimento infantil. 

Alguns investigadores procuraram encontrar marcadores específicos em crianças com histórico de otites médias, tendo sido demonstrado que estas crianças apresentavam desempenhos mais baixos em testes de perceção e processamento auditivo e uma elevada incidência de dificuldades académicas, especialmente ao nível das competências de leitura e escrita \citep{Shriberg1983, Roberts1991}. Por seu lado, não foram encontrados marcadores fonológicos específicos em crianças com historial clínico de otite média \citep{Wertzner2007a}. Num estudo longitudinal, onde foram avaliadas crianças em idade pré-escolar com OMS até um ano após a cirurgia para colocação de tubos de ventilação trâns-timpânicos (TVTT), ficou demonstrado que, mesmo quando as crianças recuperam a audição periférica, as repercussões linguísticas mantêm-se, nomeadamente no que diz respeito à memória auditiva, à reflexão sobre a língua e à produção de fricativas vozeadas e de líquidas laterais \citep{Baptista2015}, colocando estas crianças em risco de apresentarem uma PPA, com consequentes compromissos do seu percurso académico. Assim, é possível afirmar que a perda auditiva decorrente dos episódios de OMS também prejudica a maturação das vias auditivas centrais, comprometendo o processo de organização de toda a informação auditiva \citep{Martins2010}. 

\subsection{Caracterização das crianças com Perturbação de Processamento Auditivo}
Ainda que as crianças com PPA constituam um grupo relativamente heterogéneo, que não deve ser descrito com um perfil comportamental único dada a variação no grau e natureza da sua perturbação \citep{Martins2010}, é possível listar um conjunto de características mais frequentemente encontradas nestas crianças \citep{Keith2000}:

\begin{itemize}
    \item [-] Dificuldades específicas em ambientes com ruído, em situações de competição sonora, com sons concorrentes;
\item[-] Dificuldade em manter a atenção a estímulos sonoros;
\item[-] Dificuldades na compreensão de mensagens orais e no seguimento de ordens complexas, solicitando a repetição das informações e demonstrando um atraso nas respostas;
\item[-] Uso de expressões como “hã?” ou “o quê?” e pedir para repetir frequentemente;
\item[-] Uso de respostas inconsistentes para estímulos auditivos;
\item[-] Redução da atenção auditiva;
\item[-] Dificuldades de discriminação auditiva de sons da fala;
\item[-] Constrangimentos em diversas tarefas de consciência fonológica;
\item[-] Baixo desempenho escolar, especialmente quanto à leitura e à escrita;
\item[-] Dificuldades na produção dos sons da fala;
\item[-] Problemas de linguagem expressiva quanto à estrutura gramatical da língua;
\item[-] Dificuldades pragmáticas, nomeadamente na compreensão de palavras de duplo sentido, piadas e sarcasmo;
\item[-] Dificuldades de memória auditiva (instruções, nomes, histórias);
\item[-] Dificuldade em aprender músicas e em reconhecer os padrões rítmicos e melódicos;
\item[-] Dificuldade na localização espacial do som;
\item[-] Dificuldades na aprendizagem de uma segunda língua.
\end{itemize}
Para além das características referidas, estas crianças poderão apresentar tendência para ser mais distraídas, agitadas, ou muito quietas, demonstrando, por vezes, algum desajustamento na brincadeira com as outras crianças.  Poderão também sentir-se frustradas quando adquirem consciência das suas dificuldades comunicativas, sociais e de aprendizagem.

\section{Avaliação do Processamento Auditivo}
De acordo com as \emph{guidelines} definidas por uma organização profissional canadiana dos Terapeutas da Fala e dos Audiologistas \citep{CISG2012}, existe um conjunto de pressupostos que devem ser respeitados antes de se iniciar o processo de avaliação do PA, que permitam assegurar a pertinência e adequação deste processo:

\begin{enumerate}
    \item [1.]	A decisão de encaminhamento para uma avaliação do PA deve ter uma base sólida no impacto que as dificuldades têm numa perspetiva ecológica da criança, ou seja, no desempenho e na participação na escola, em casa e na comunidade;
\item[2.]	No que diz respeito à idade, deve-se considerar e respeitar a curva maturacional do desenvolvimento das competências de processamento auditivo;
\item[3.]	 Deve ter-se presente que avaliação do PA através de testes comportamentais exige que a criança compreenda as tarefas solicitadas e apresente um nível de linguagem que lhe permita compreender os estímulos verbais, demonstrando atenção e memória auditivas suficientes para desempenhar as tarefas propostas.
\end{enumerate}

\subsection{Rastreio}
A intervenção precoce na infância refere-se a um conjunto de ações de apoio integrado dirigido à criança e à sua família, que inclui medidas de origem preventiva e de reabilitação, nomeadamente em contexto académico, de saúde e em contexto social \citep{Goncalves2014}.

O rastreio do PA permite a identificação de possível PPA, levando a uma intervenção precoce e uma minimização das consequências desta no desenvolvimento da criança e na sua aprendizagem \citep{Tillery2009}. Este rastreio, numa fase inicial, pode ser feito através de questionários comportamentais, quer aos pais/cuidadores, quer aos educadores de infância/professores, de forma a, por um lado, identificar crianças em risco de apresentar uma PPA e, por outro lado, reunir as informações mais relevantes  sobre o potencial impacto da PPA na qualidade de vida destas crianças \citep{AAA2010, ASHA2005, Volpatto2019}.

É, pois, de extrema importância que todos os que lidam diretamente com estas crianças estejam alerta para os principais sinais de uma eventual PPA e que, perante uma situação de suspeita, façam o encaminhamento necessário para uma avaliação aprofundada do PA por parte do audiologista. Desta forma torna-se possível investir de forma mais determinante na estimulação precoce, a partir de abordagens seguras e adequadas para desenvolvimento de todas as funções auditivas \citep{ASHA2005}.

Na literatura internacional, estão disponíveis vários questionários comportamentais que podem ajudar na identificação de eventuais alterações do PA. São exemplos destes questionários: 

\begin{enumerate}
    \item [(a)] ECLiPS – Evaluation of Children’s Listening and Processing Skills \citep{Barry2015};
\item[(b)] CHAPPS – Children’s Auditory Processing Performance Scale \citep{Smoski1998};
\item[(c)] SIFTER – Screening Instrument for Targeting Educational Risk \citep{Keith2019, Volpatto2019};
\item[(d)] LIFE-7 – Listening Inventories for Education \citep{Anderson1999, Barry2015}; 
\item[(e)] TEAP- Teacher Evaluation of Auditory Performance \citep{Keith2019, Barry2015}.
\end{enumerate}
No caso do português europeu, existem quatro questionários complementares à avaliação do PA: 

\begin{enumerate}
    \item [–] Scale of Auditory Behaviours (SAB) \citep{Nunes2015, Nunes2013};
\item [–] LIFE-R – Listening Inventory For Education – Revised \citep{Quadros2017};
\item [–] Escala de Desempenho do Processamento Auditivo na Criança (P-CHAPPS) \citep{Oliveira2013};
\item [–] Checklist de Fisher \citep{Martins2010}.
\end{enumerate}
\subsection{Avaliação}
Para a avaliação do conjunto de comportamentos que estão frequentemente associados à PPA, é fundamental recorrer a uma equipa multidisciplinar, envolvendo não só os profissionais de saúde, como o médico ORL, o audiologista, o terapeuta da fala, o psicólogo, mas também os profissionais de educação, como os educadores de infância e os professores, não esquecendo o importante papel da família/ cuidadores \citep{BSA2007}.

O principal objetivo da avaliação do PA em crianças é verificar a integridade e o estado de maturação da via auditiva, tendo em vista um correto diagnóstico e planeamento da intervenção terapêutica \citep{Bellis2015, ASHA1996}. Uma avaliação adequada deve contemplar a avaliação da via auditiva periférica e da via auditiva central \citep{Musiek2007}. A avaliação formal do PA deverá ser efetuada após os 5 anos de idade \citep{Lucker2015, Martins2017b, Shapiro2016}.

Assim, a avaliação do PA poderá ser solicitada em crianças em idade escolar que apresentem alterações do comportamento, da atenção, dificuldades auditivas não orgânicas, suspeita de perturbações de linguagem e/ou de fala, dificuldades de leitura e de escrita.

A avaliação do PA deve ser iniciada através de uma anamnese detalhada, de forma a recolher um conjunto de informações relevantes referentes aos períodos pré, peri e pós-natais, estado de saúde geral e história clínica, antecedentes familiares de queixas semelhantes, desenvolvimento linguístico e de fala, desenvolvimento emocional e social, percurso académico e, ainda, questões especificamente relacionadas com os sintomas e comportamentos frequentemente encontrados em crianças com PPA. O acesso a este conjunto de informações é decisivo na determinação da natureza e do tipo de PPA \citep{ASHA2005}. 

É fundamental que o profissional responsável pela avaliação do PPA esteja ciente de que o desempenho da criança pode ser afetado por outras competências não-auditivas, tais como a atenção, a motivação, a memória e a sua capacidade cognitiva \citep{Koravand2017}, nomeadamente a sua capacidade de compreensão e expressão linguísticas. Estas competências podem afetar negativamente o desempenho das crianças nas provas de avaliação do PA, já que podem não ser capazes de realizar determinadas tarefas, porque simplesmente não compreenderam a instrução dada.  

As baterias de testes de avaliação do PA devem incluir estímulos verbais e não verbais. O recurso a diferentes testes com estímulos variados permite recolher informação menos influenciada pelas capacidades linguísticas, avaliar os diversos mecanismos e processos do processamento auditivo, bem como diversos níveis e regiões da via auditiva central \citep{Moore2011, ASHA2005}.

Uma avaliação global do PA deverá ser composta por testes que avaliem as diferentes competências auditivas: localização e lateralização do som, discriminação auditiva, processamento auditivo temporal, processamento auditivo de padrões, escuta dicótica, performance auditiva com estímulos em competição e performance auditiva com estímulos acústicos degradados \citep{AAA2010, CISG2012}.

Os testes de avaliação do PA são agrupados em \citep{Bellis1999, AAA2010, ASHA2005, Keith2019}:

\emph{Testes monoaurais de baixa redundância} – Os testes monoaurais de baixa redundância estão entre os testes mais antigos usados na avaliação do PA \citep{Bocca1954}. São realizados mediante a apresentação unilateral de sinais acústicos em que a redundância extrínseca do sinal da fala é diminuída, pela alteração das características espetrais, temporais ou de intensidade. A redundância extrínseca é fornecida pelas pistas fonémicas, prosódicas, morfológicas, sintáticas e semânticas. Estes testes avaliam a função central do processamento e encerramento auditivo da informação, que engloba a atenção e a representação fonológica. Permitem avaliar a capacidade do indivíduo testado em realizar o “fechamento” auditivo, a figura-fundo e a discriminação, quando uma parte do sinal está distorcida ou ausente. Os testes monoaurais de baixa redundância para uso clínico classificam-se em: testes de fala filtrada com filtros passa-baixo; testes de fala no ruído e testes de fala comprimida no tempo \citep{Krishnamurti2007}. Após os trabalhos realizados por \citep{Martins2017b}, existem dados normativos para o português europeu, nos testes de fala filtrada e no teste de fala no ruído, para crianças dos 5 aos 11 anos, e para a faixa etária dos 12 aos 59 anos \citep{Martins2017b}.

\emph{Testes dicóticos} – Os testes dicóticos são caracterizados pela apresentação simultânea, nos dois ouvidos, de estímulos diferentes \citep{Bellis2003}. Os testes dicóticos permitem avaliar a integração e a separação binaural, ou seja, a capacidade do indivíduo testado para repetir tudo o que ouve ou para dirigir a atenção para um só ouvido. Os testes mais utilizados são: teste dicótico de dígitos, teste dicótico de consoante-vogal, teste dicótico não verbal, teste de identificação de frases com mensagem competitiva contralateral (SSI-CCM) e teste de dissílabos alternados (SSW).

O SSW foi um dos primeiros testes de avaliação do PA e tem sido usado pelos audiologistas nos Estados Unidos da América durante os últimos 30 anos. Atualmente, continua a ser um dos testes de avaliação do PA com maior aplicação clínica \citep{Musiek1987, Musiek2001}. O teste SSW é composto por 160 espondeus divididos em 40 itens de 4 dissílabos cada. Normalmente, é efetuado a 50 dB SL (nas situações em que este valor produz desconforto ao indivíduo testado, a intensidade de apresentação do estímulo reduz-se para 25 dB SL). O SSW permite avaliar ambos os ouvidos em situação de estímulo normal e em condição de competição \citep{Katz1962, Katz1963, Katz1968}. O teste SSW está disponível para o português europeu na Bateria de Avaliação de Processamento Auditivo (BAPA-PE), desenvolvida por \citeauthor{Martins2018a} \citep{Martins2018a}. Inclui dados normativos para crianças dos 5 aos 11 anos de idade e para a faixa etária dos 12 aos 59 anos \citep{Martins2017b, Martins2008, Martins2014}.

\emph{Testes de processamento temporal} – O processamento auditivo temporal pode ser definido como a perceção das características temporais ou da alteração de características de duração de um som dentro de um intervalo de tempo limitado ou definido. Musiek e colaboradores referem que o processamento temporal pode ser o elemento subjacente a muitas das capacidades de processamento auditivo, incluindo o processamento de sinais acústicos verbais e não verbais. O processamento temporal pode ser dividido em 4 subtipos: mascaramento temporal, ordenação ou sequenciação temporal, integração ou somação temporal, resolução ou discriminação temporal \citep{Musiek2005}. Os testes que avaliam as capacidades de ordenação temporal, de frequência e de duração são utilizados principalmente para a análise dos aspetos rítmicos,  como a acentuação e a prosódia da fala \citep{Houston2002, Shinn2003}. Segundo diversos investigadores \citep{Pinheiro1985, Musiek1990, Olsen1991}, os testes de processamento temporal são especialmente indicados para avaliação de patologia cortical. Estes testes permitem efetuar a avaliação das capacidades auditivas de ordenação, discriminação, resolução e integração temporal. Os mais frequentemente utilizados são: sequência de padrão de frequência (Pitch Pattern Sequence), sequência de padrão de duração (Duration Pattern Sequence) e teste de deteção de intervalos no ruído (Gap In Noise GIN).

\emph{Testes de interação binaural} – Nos  testes de interação binaural, é necessária a interação dos dois ouvidos para conseguir compreender o sinal dicótico ouvido, separado por diversos fatores: tempo, frequência e intensidade entre os dois ouvidos. Os estímulos são apresentados nos dois ouvidos (mas aspetos diferentes do estímulo são apresentados em cada ouvido) e é necessário efetuar a sua integração, que se julga ter lugar no tronco cerebral, pelo que estes testes estão indicados para pesquisa de lesões nesta estrutura nervosa. Os testes de interação binaural são: teste de perceção da fala rapidamente alternada \citep{Gelfand2001}, teste de fusão binaural \citep{Matzker1959, Stach2000, Gelfand2001} e teste de identificação de frases com mensagem competitiva contralateral (SSI-CCM) \citep{Willeford1999}.

\section{Intervenção}
Após a avaliação comportamental do PA, deverá ser elaborado um relatório detalhado que dê conta do desempenho da pessoa avaliada. Este irá servir de base para o planeamento da intervenção a nível do PA. A intervenção deverá ser planificada e levada a cabo por profissionais experientes que conheçam bem as competências auditivas avaliadas por cada um dos testes.

No planeamento da intervenção, os défices de processamento auditivo identificados na avaliação devem guiar os objetivos do plano terapêutico. Deve ser dada prioridade aos défices auditivos com maior impacto funcional no desempenho académico, profissional, bem como a nível social \citep{AAA2010}.

O plano de intervenção deverá ter em conta: 

\begin{enumerate}
    \item [(1)] modificações ambientais – visando melhorar o acesso à informação auditiva, através da melhoria do sinal acústico e da facilitação da escuta na escola, em casa, no trabalho ou noutros locais relevantes; 
\item[(2)] estratégias compensatórias ou treino dos recursos centrais – visando ajudar o indivíduo a superar a disfunção residual e a lidar com questões motivacionais ou défices associados, através do reforço competências de tipo \emph{top-down}; 
\item [(3)] reabilitação direta das competências – visando melhorar o desempenho auditivo através da alteração da forma como o cérebro processa o estímulo auditivo, com atividades de tipo \emph{bottom-up}, como o treino auditivo \citep{Bellis2008}.
\end{enumerate}
\subsection{Programa de Intervenção no Processamento Auditivo (PIPA)}
Para a realização do treino auditivo, estão disponíveis diversos programas de intervenção \citep{Miller2005}, muitos deles em formato de \emph{software}, atualmente também disponíveis para \emph{tablets} e \emph{smartphones} \citep{BSA2018}. Os chamados Computer-Based Auditory Training (CBAT) permitem o trei\-no de diferentes competências auditivas \citep{Bellis2015, Keith2019}, incluem estímulos acústicos verbais e não-verbais, muitas vezes combinados com algumas tarefas de linguagem e memória \citep{Aquino2002a, Loo2010}, \emph{feedback}/reforço positivo e possibilidade de treino intensivo e adaptativo. Especialmente atrativos para a população pediátrica com alterações no PA concomitantes com dificuldades de linguagem, de aprendizagem e de leitura \citep{AAA2010, Loo2010}, estes programas são ferramentas eficazes na promoção da motivação das crianças.

Tendo em conta que o treino auditivo deverá incluir estímulos verbais \citep{AAA2010, Bellis2012} e muitos destes programas foram desenhados, testados e validados para outras línguas, não são passíveis de ser usados na intervenção com a população cuja língua materna é o português europeu.

Especificamente para o português europeu, destaca-se o Programa de Intervenção em Processamento Auditivo (PIPA), destinado a crianças em idade escolar (6 e os 10 anos) \citep{Luis2019, Abrantes2019}. O PIPA engloba tarefas de: discriminação auditiva, atenção auditiva, memória auditiva, integração binaural, separação binaural, fusão binaural, figura-fundo e fechamento. Foram consideradas competências auditivas que implicam estímulos verbais (em detrimento das que envolvem estímulos não-verbais), selecionados e balanceados de acordo com critérios linguísticos rigorosos, nomeadamente extensão de palavra e estrutura silábica do estímulo, tendo como referência os dados de frequência disponíveis para o português europeu \citep{Guerreiro2007, Vigario2006}.

O programa tem como enquadramento lúdico um jardim zoológico, com vários espaços distintos, sendo que a cada um deles se encontra associada uma competência auditiva: (baía dos golfinhos – discriminação auditiva; alimentação dos pelicanos – atenção auditiva; selva encantada – memória auditiva; céu colorido – separação binaural; esconderijo dos rastejantes – integração binaural;  quinta do tio Manel – fusão binaural; parque pré-histórico – fechamento; floresta mágica – figura-fundo). Embora, para efeitos de estruturação do programa, se assuma que em cada uma das secções é estimulada uma competência auditiva principal, concomitantemente são treinadas outras competências linguísticas e auditivas \citep{Magimairaj2018}.

Cada secção do programa inclui diferentes atividades, hierarquicamente organizadas em função do grau de dificuldade. Em cada uma das atividades são apresentados cerca de 10 a 15 estímulos consecutivos e, sempre que a criança atinge 75\% de acertos \citep{Weihing2015}, conquista um animal e pode avançar para o nível seguinte. 

Em alguns jogos, o terapeuta da fala pode manipular algumas condições, tais como desníveis de intensidade dos estímulos, relação sinal-ruído, seleção do ouvido para apresentação dos estímulos (ouvido direito vs. ouvido esquerdo) e desnível temporal de apresentação dos estímulos em escuta dicótica. Poderá ainda monitorizar o desempenho/progresso da criança e realizar itens de treino no início de cada atividade, de modo a familiarizar a criança com a tarefa. O PIPA inclui também um manual que contempla, para cada atividade: os objetivos, as instruções dadas à criança, os procedimentos a seguir, os materiais e os estímulos e o reforço/ recompensa. Esta última é fundamental para promover a motivação e o interesse da criança, fundamentais para o sucesso da intervenção \citep{Stroiek2015}.

Apesar de o programa englobar atividades de estimulação de várias competências auditivas, os espaços a visitar por cada criança dependem, quer do diagnóstico prévio realizado pelo audiologista, quer do plano de intervenção traçado pelo terapeuta da fala.

O PIPA foi sujeito a uma validação de conteúdo, fator crucial no desenvolvimento de um novo instrumento \citep{Alexandre2011, Coutinho2014a, Grant1997}. Dois painéis de peritos, cuidadosamente selecionados de acordo com critérios de experiência clínica e de conhecimento aprofundado na área do PA, analisaram o programa quanto à abrangência, clareza, adequação e pertinência dos conteúdos. Os resultados obtidos demonstraram a validade de conteúdo do PIPA, com um IVC global 0,95. Dos vinte e oito itens analisados, vinte e dois apresentam um IVC de 1. 

O desenvolvimento do PIPA em formato digital está ainda em curso. Do mesmo modo, encontra-se em planeamento um estudo piloto de aceitabilidade, com crianças em idade escolar sem patologia do PA. Futuramente, seria importante a realização de estudos de eficácia do PIPA, com crianças com PPA, com afetação de uma ou mais competências consideradas no programa.

\section{Considerações finais}
A PPA, que resulta de um défice no processamento da informação auditiva, afetando uma ou várias competências auditivas, está comumente associada a perturbações da linguagem, da aprendizagem e da leitura e da escrita, com implicações nas atividades de vida diária da criança. 

O rastreio é determinante para identificar crianças em risco de desenvolver uma PPA, uma vez que possibilita uma intervenção precoce, minimizando assim o impacto da perturbação no desenvolvimento da criança.

Já o correto diagnóstico requer a intervenção de uma equipa multidisciplinar. A avaliação inclui testes comportamentais, que avaliam as diferentes competências auditivas, através de sons verbais e não-verbais.

A intervenção deve ser implementada o mais precocemente possível após o diagnóstico, explorando diferentes estratégias e abordagens, tendo em conta as competências auditivas comprometidas e com maior impacto funcional na vida da criança. Ao longo dos anos têm sido desenvolvidos vários programas de treino auditivo, destacando-se o PIPA, especificamente desenhado para a intervenção com a população cuja língua materna é o português europeu.


{\sloppy\printbibliography[heading=subbibliography,notkeyword=this]}
\end{document}
