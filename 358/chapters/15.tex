\documentclass[output=paper,colorlinks,citecolor=brown,booklanguage=portuguese]{langscibook}
\ChapterDOI{10.5281/zenodo.7233241}
\title{Interface gramática-pragmática e comprometimentos no desempenho linguístico}
\author{Letícia Maria Sicuro Corrêa\affiliation{Pontifícia Universidade Católica do Rio de Janeiro (PUC-Rio); LAPAL (Laboratório de Psicolinguística e Aquisição da Linguagem - PUC-Rio))}  and Vanessa G. Ribeiro\affiliation{Pontifícia Universidade Católica do Rio de Janeiro (PUC-Rio); LAPAL (Laboratório de Psicolinguística e Aquisição da Linguagem - PUC-Rio))}  and  Jacqueline R. Longchamps \affiliation{Pontifícia Universidade Católica do Rio de Janeiro (PUC-Rio); LAPAL (Laboratório de Psicolinguística e Aquisição da Linguagem - PUC-Rio))}}


\abstract{Neste capítulo, trazemos a caracterização de uma interface gramática-pragmática nas distinções de ordem intencional em elementos de categorias funcionais no léxico, a qual pode permitir prever comprometimentos no uso da língua por crianças em idade escolar com dificuldades de linguagem e no espectro autista.  Focali\-zam-se distinções gramaticais pertinentes a definitude, aspecto e modo em tarefas de produção e  compreensão narrativa e de identificação de imagens. Os resultados sugerem diferenças sutis entre esses grupos e o grupo controle, de desenvolvimento típico, as quais requerem atenção da parte de professores e terapeutas.

\textbf{Palavras-chave:} Interface gramática-pragmática; Dificuldades de aprendizagem; Transtorno do Espectro Autista.} 


\IfFileExists{../localcommands.tex}{
   \addbibresource{../localbibliography.bib}
   \usepackage{langsci-optional}
\usepackage{langsci-gb4e}
\usepackage{langsci-lgr}

\usepackage{listings}
\lstset{basicstyle=\ttfamily,tabsize=2,breaklines=true}

%added by author
% \usepackage{tipa}
\usepackage{multirow}
\graphicspath{{figures/}}
\usepackage{langsci-branding}

   
\newcommand{\sent}{\enumsentence}
\newcommand{\sents}{\eenumsentence}
\let\citeasnoun\citet

\renewcommand{\lsCoverTitleFont}[1]{\sffamily\addfontfeatures{Scale=MatchUppercase}\fontsize{44pt}{16mm}\selectfont #1}
  
   %% hyphenation points for line breaks
%% Normally, automatic hyphenation in LaTeX is very good
%% If a word is mis-hyphenated, add it to this file
%%
%% add information to TeX file before \begin{document} with:
%% %% hyphenation points for line breaks
%% Normally, automatic hyphenation in LaTeX is very good
%% If a word is mis-hyphenated, add it to this file
%%
%% add information to TeX file before \begin{document} with:
%% %% hyphenation points for line breaks
%% Normally, automatic hyphenation in LaTeX is very good
%% If a word is mis-hyphenated, add it to this file
%%
%% add information to TeX file before \begin{document} with:
%% \include{localhyphenation}
\hyphenation{
affri-ca-te
affri-ca-tes
an-no-tated
com-ple-ments
com-po-si-tio-na-li-ty
non-com-po-si-tio-na-li-ty
Gon-zá-lez
out-side
Ri-chárd
se-man-tics
STREU-SLE
Tie-de-mann
}
\hyphenation{
affri-ca-te
affri-ca-tes
an-no-tated
com-ple-ments
com-po-si-tio-na-li-ty
non-com-po-si-tio-na-li-ty
Gon-zá-lez
out-side
Ri-chárd
se-man-tics
STREU-SLE
Tie-de-mann
}
\hyphenation{
affri-ca-te
affri-ca-tes
an-no-tated
com-ple-ments
com-po-si-tio-na-li-ty
non-com-po-si-tio-na-li-ty
Gon-zá-lez
out-side
Ri-chárd
se-man-tics
STREU-SLE
Tie-de-mann
}
   \boolfalse{bookcompile}
   \togglepaper[15]%%chapternumber
}{}

\shorttitlerunninghead{Interface gramática-pragmática e desempenho linguístico}
\begin{document}
\maketitle

\section{Introdução}
Um dos desafios com que professores necessariamente deparam são crianças/adolescentes com desempenho escolar insatisfatório, ainda que este pareça incompatível com seu potencial \citep{Krull1996}.\footnote{A norma adotada na escrita deste capítulo foi a do português brasileiro.}
% \footnote{Na sequência de restrições decorrentes do template usado para publicação na editora Language Science Press, os capítulos que integram o volume surgem com a referência [this volume], em inglês.}
Uma série de razões pode ser encontrada para um baixo rendimento escolar, tais como déficit de atenção e hiperatividade, deficiência visual ou auditiva, má alimentação, problemas de ordem emocional e/ou social, dentre muitas outras, o que requer um amplo escrutínio \citep{Banerjee2016, Karande2005}. Nesse escrutínio, contudo, não é comum que problemas vinculados à linguagem sejam apontados como uma das possíveis fontes de dificuldade no desempenho escolar. Há, no entanto, sintomas compartilhados por crianças com dificuldades de aprendizagem (de agora em diante DAp) e aquelas com comprometimentos no domínio da pragmática \citep{Bishop1989, Flanagan2005, Gerber1996, Norbury2002}.\footnote{Comprometimentos identificados como \emph{Pragmatic Language Impairment} (PLI) (termo introduzido por \citet{Bishop1998}, em substituição a \emph{Semantic-Pragmatic Disorder}, na taxionomia de \citet{Bishop1987}), para designar manifestações também identificadas como \emph{Semantic Pragmatic Syndrome} \citep{Rapin1983}. Em 2013, o PLI foi inserido no DSM-5 (\emph{American Psychiatric Association} 2013 como \emph{Social Communication Disorder} (SCD), incluindo dificuldades de comunicação não necessariamente de natureza verbal.} Uma vez que o uso da língua envolve habilidades de ordem pragmática, é possível, portanto, que dificuldades de linguagem em uma interface com a pragmática venham a afetar o desempenho escolar.

Outra questão que desafia professores e terapeutas da fala diz respeito à linguagem no espectro autista. É estimado que uma em cada 54 crianças em idade escolar se situem nesse espectro, com comprometimentos que afetam o uso adequado da língua \citep{Maenner2016}. Diante da proposta de inclusão dessas crianças em escolas regulares, é importante sinalizar que aspectos da linguagem podem estar afetados, com vista a uma atenção cuidadosa, da parte de professores, além de uma intervenção direcionada, da parte de terapeutas. 

Comprometimentos de ordem pragmática são muito amplos e não necessariamente envolvem conhecimento de natureza gramatical. Capacidade de compreender linguagem figurada e ironia, assim como atenção ou uso de sinais não verbais na regulação da interação interpessoal são apontados nos critérios diagnósticos de uma perturbação de natureza pragmática \citep{Ketelaars2017}. Os recursos verbais afetados são, no entanto, menos explícitos na caracterização desses critérios.

Comprometimentos no domínio da pragmática são usualmente vistos em distribuição complementar àqueles que afetam a gramática. \citet{Friedmann2008} introduziram uma tipologia para distinguir diferentes tipos do então denominado SLI (\emph{Specific Language Impairment}), hoje renomeado como TDL/PDL (Transtorno ou Perturbação do Desenvolvimento da Linguagem), a qual inclui o PraSLI, equivalente ao PLI (\emph{Pragmatic Language Impairment}) \citep{Bishop1989, Norbury2002}.\footnote{O termo SLI (\emph{Specific Language Impairment}), com uso estável na literatura clínica e científica desde \citet{Fey1983}, foi recentemente abandonado em função do consórcio CATALISE \citep{Bishop2016}.} Em \citet{Friedmann2008}, foram identificadas crianças/adolescentes na faixa de 9-14 anos, com comprometimentos no discurso, tanto em termos de relevância quanto de quantidade, com base em análise de narrativas e de testes específicos. Essas crianças, ainda que pudessem apresentar dificuldades em tarefas de nomeação e problemas de ordem fonológica, tiveram bom desempenho na compreensão de estruturas de alto custo (ver \citetv{chapters/14}), o que pode ser tomado como indicativo de ausência de comprometimentos no domínio mais estrito da língua. Existe, entretanto, uma relação entre gramática e pragmática que não tem sido especificamente explorada. Trata-se do modo como a informação pertinente a intencionalidade pode ser codificada verbalmente, levando em conta o momento de fala e a perspectiva do falante. Essa informação diz respeito à força ilocucionária do enunciado (se assertivo ou interrogativo, por exemplo), ao modo como o falante situa o evento apresentado verbalmente no tempo, como relaciona eventos no tempo e os apresenta como algo factual (\emph{realis}) ou como possibilidade (\emph{irrealis}). Uma vez que tais distinções estejam codificadas na gramática da língua, sua expressão torna-se compulsória (e automática) para o falante \citep{Slobin1996}, estabelecendo-se, desse modo, uma interface gramática-pragmática em traços formais (gramaticais) do léxico.

O falante também leva em conta a perspectiva do ouvinte ao fazer referência a um indivíduo ou objeto introduzido anteriormente no discurso ou não, e cabe ao ouvinte inferir as intenções do falante, com base nessas distinções gramaticais, e outras informações que possam enriquecer a compreensão. 

A produção e a compreensão de enunciados linguísticos são pois, em grande medida, dependentes de Teoria da Mente (capacidade de reconhecer um estado mental em si próprio e de inferir o estado mental de outrem, que pode diferir do seu) \citep{BaronCohen2000, Premack1978}, comprometida no espectro autista e em crianças com dificuldades de ordem pragmática, assim como de critérios de relevância, que orientam o uso pragmaticamente adequado da língua. 

Neste capítulo, assumimos uma perspectiva cognitiva para a pragmática, tal como veiculada na Teoria da Relevância, segundo a qual a comunicação humana (seja verbal ou não verbal) se caracteriza fundamentalmente pela expressão e o reconhecimento de intenções \citep{Sperber2002}. A interface gramática-pragmática estará aqui circunscrita à informação de natureza intencional codificada nos traços semânticos/formais de elementos de categorias funcionais do léxico.  Outros aspectos relevantes no desenvolvimento de habilidades pragmáticas na comunicação verbal, não necessariamente dependentes da codificação de traços formais do léxico, como a compreensão de implicaturas e metáforas, por exemplo, não serão aqui enfocados.

Nesse arcabouço teórico, apresentamos o resultado de avaliações de habilidades linguísticas de crianças com dificuldade de aprendizagem e no espectro autista. O capítulo segue da seguinte forma. A seção \ref{sec:doiscap15} apresenta o modo como concebemos a interação entre domínios da cognição humana na produção/compreensão da linguagem. A seção \ref{sec:trêscap15} traz resultados de dois experimentos conduzidos com crianças DAp e um grupo controle: (i) de produção eliciada de narrativas, com vista a detectar o modo como referentes com diferentes graus de relevância são introduzidos e recuperados, e como contrastes aspectuais são estabelecidos; (ii) de compreensão de narrativas, em que se destacam inferências com base em contraste aspectual. Considera-se a possibilidade de um comprometimento pragmático amplo e de um seletivo na interface gramática-pragmática, em crianças com DAp à luz de resultados em \citet{Longchamps2014}. A seção \ref{sec:quatrocap15} traz uma breve apresentação dos comprometimentos da linguagem no espectro autista e, na seção \ref{sec:cincocap15}, são apresentados os resultados de um experimento em que se explora a compreensão do contraste entre modo \emph{realis/irrealis} por crianças nesse espectro. As considerações finais chamam atenção para a necessidade de habilidades sutis pertinentes à interface gramática-pragmática serem exploradas, o que é fundamental para uma intervenção direcionada às dificuldades específicas de crianças DAp e no espectro autista. 


\section{Interface gramática-pragmática }\label{sec:doiscap15}
A produção da fala e a compreensão de enunciados verbais dependem não somente do conhecimento da língua em questão como de outras formas de conhecimento que com este interagem na mente. A interação entre a língua e as outras formas de conhecimento se estabelece no léxico. Essa interação possibilita ao falante transformar uma intenção em um ato de fala, e ao ouvinte percebê-lo como tal e chegar a uma compreensão da mensagem. Podemos dizer, então, que é no léxico que se criam as possíveis interfaces da língua com outros recursos cognitivos que possibilitam a interação social/cognitiva humana por um meio linguístico. 

Os elementos do léxico podem ser concebidos como constituídos de traços (ou propriedades) semânticos, fonológicos e formais. Esses últimos, ainda que possam ser de natureza semântica, servem à computação sintática (ou seja, informam como os elementos do léxico podem ser relacionados sintaticamente), o que possibilita ao falante/ouvinte produzir e compreender enunciados verbais, com base em seu conhecimento da gramática da língua. Vejamos como o conhecimento da língua, representado no léxico, mantém relação com outros domínios da cognição. Para isso, é necessário apresentar a concepção de língua aqui assumida.

No contexto do programa Minimalista da Linguística gerativista, a língua(gem) é concebida em termos de um sistema computacional universal (comum a todos os seres humanos) e de um léxico. O primeiro consiste de um pequeno número de operações que atuam sobre os traços formais de elementos do léxico, na construção da estrutura sintática subjacente às expressões linguísticas geradas \citep{Chomsky1995}. O léxico, que se constitui no processo de aquisição da linguagem, é específico de cada língua, ainda que com propriedades em comum entre línguas. Em qualquer língua humana, os elementos do léxico se distribuem em categorias ditas lexicais, que incluem nomes, adjetivos, verbos, as quais são ricas em traços semânticos; e categorias funcionais (também chamadas gramaticais), que são relevantes para a estrutura sintática e cujo conteúdo semântico é mínimo. Assim, ainda que os elementos do léxico das diferentes línguas tenham propriedades específicas, que irão distinguir suas gramáticas, as categorias lexicais e funcionais, presentes em todas as línguas, permitem que se estabeleçam relações entre a língua interna – desenvolvida no curso da aquisição da linguagem – e a cognição mais ampla, que inclui os chamados sistemas conceituais-intencionais \citep{Chomsky1995}. 

Sistemas conceituais dizem respeito a categorizações possíveis, e a propriedades que definem tais categorias, tendo em conta a constituição biológica do aparato perceptual/cognitivo do ser humano e o tipo de experiência que possa resultar de sua interação com o mundo. As categorias chamadas lexicais, com seus traços semânticos, estão predominantemente vinculadas aos sistemas conceituais da cognição mais ampla. Na formulação de enunciados linguísticos, os elementos de categorias lexicais (as chamadas palavras de conteúdo) são essenciais para a expressão das relações semânticas entre predicados e argumentos, sendo, portanto, cruciais para a codificação gramatical de uma mensagem. Contudo, no acesso ao léxico mental, por parte do falante, não só os traços semânticos dos itens lexicais determinam o que é selecionado. O grau de formalidade, a positividade/ negatividade, a polidez ou a agressividade associada a determinadas palavras é uma forma de conhecimento que vincula seu possível significado a possíveis situações de uso (dizer que alguém está gordo ou acima do peso pode trazer diferentes efeitos no ouvinte, por exemplo). As implicações trazidas pela escolha de determinadas palavras também podem carregar um viés avaliativo, em função do propósito na elocução, da parte do falante (consciente ou não). Assim, diante de uma ideia, a recuperação de determinadas palavras no léxico não diz respeito apenas à proposição (relação de natureza lógica) a ser verbalmente apresentada, diz respeito também à intenção do falante diante de um dado interlocutor (ou de interlocutores virtuais) e do contexto de fala (ou do contexto criado e presumido na escrita), ao transmitir uma dada informação, o que envolve conhecimento de ordem pragmática. Podemos dizer, portanto, que conhecimento de ordem pragmática interage, de forma ampla, com o conhecimento lexical de ordem conceptual, representado no léxico mental, que é acionado na produção e na compreensão de enunciados. 

Sistemas intencionais são de mais difícil caracterização.\footnote{Intencionalidade: conceito que remonta à Escolástica, trazido por Franz Brentano (1838-1917) para a Psicologia, diz respeito, grosso modo, a estados mentais.} Dizem respeito a estados mentais pertinentes a volição, crenças, expectativas, decisão e à nossa teoria ou entendimento intuitivo sobre estes, a qual nos permite interpretar, prever ou explicar o comportamento de outros seres humanos \citep{Dennett1971}. Sistemas intencionais interagem, portanto, com o conhecimento de natureza pragmática e com a Teoria da Mente. Informação pertinente a intencionalidade é representada no léxico em traços semânticos de elementos de categorias lexicais que permitem expressar o estado mental do falante (querer, desejar, ansiar, por exemplo), seu viés avaliativo (em adjetivos, advérbios) ou o lugar e tempo em que se situa (hoje, aqui). Entretanto, são os traços semânticos/formais dos elementos de categorias funcionais que são essencialmente de natureza intencional. Esses traços codificam gramaticalmente a força ilocucionária do enunciado (asserção, interrogação), no elemento funcional que ocupa a posição mais alta (o nó mais alto) da estrutura sintática; codificam o tempo do evento, no nó sintático correspondente a elemento funcional tempo (T), nessa estrutura; assim como o aspecto verbal (Asp), que relaciona eventos no tempo, e o modo (M) (\emph{realis} ou \emph{irrealis}) como o evento é apresentado. Em línguas como o português, essas distinções, de ordem intencional, são expressas na morfologia, ainda que possam também ser expressas por meio de combinações de itens lexicais e da interação com expressões adverbiais, tipos de complementos, entre outros. A expressão dessas distinçõees na flexão é, não obstante, em grande medida compulsória na língua. Por exemplo, ao codificar um evento no passado, o falante de português (diferentemente de um falante de uma língua como o chinês) necessariamente usará a expressão morfológica de tempo. 

A categoria funcional determinante (D) (nó mais alto dos constituintes nominais da sentença), por sua vez, apresenta o traço de definitude (+ /- definido), que distingue referentes que são introduzidos no discurso daqueles já introduzidos no discurso ou passíveis de serem recuperados pelo ouvinte, a partir de um esquema de conhecimento compartilhado culturalmente (como, por exemplo, o esquema de escola, que nos permite nos referirmos aos alunos, sem introdução prévia explícita no discurso). Vemos então que as categorias funcionais servem de base não só para a estruturação de enunciados como são fundamentais para que uma interface entre gramática e pragmática se estabeleça no léxico \citep{Correa2005, Correa2008}.

A Figura \ref{fig:cap15fig1} abaixo apresenta, de forma esquemática, o modo como concebemos a interação entre o léxico e outros domínios da cognição, para que o uso efetivo da língua aconteça. Partimos de uma simplificação do modelo clássico de produção da fala de \citet{Levelt1989}, que prevê uma sequência de etapas começando da conceptualização da mensagem, passando pela codificação gramatical e codificação fonológica, chegando ao enunciado articulado, que é aqui também apresentado como \emph{ato de fala} \citep{Searle1969}. À esquerda dessa sequência, procuramos explicitar o que estaria envolvido nessas etapas, levando em conta o falante em uma situação de fala. Os três balões de pensamento buscam indicar os componentes mentais que irão alimentar o que podemos chamar de \emph{pensamento para a fala}, no acesso ao léxico mental, na formulação de enunciados linguísticos \citep{Slobin1996}. À luz desse conceito, distinções conceituais ou intencionais representadas na gramática da língua serão necessariamente expressas na morfologia, enquanto opcionalmente expressas por outros meios, nas línguas em que tais distinções não são assim codificadas. Os componentes em figuras quase retangulares representam domínios da cognição que, tal como o conhecimento linguístico, ou gramática, são recrutados na produção da fala. Observe-se que o conhecimento pragmático inclui princípios de relevância. Assim, o falante usará dos recursos disponíveis na língua, ou de outros meios de forma a tornar ostensivo (explícito ou suficientemente claro), o que toma como relevante e, dessa forma, minimiza o custo cognitivo de seu reconhecimento por parte do ouvinte. Por exemplo, o falante pode dar ênfase à informação relativa ao tempo do evento por meio da ênfase no acento tônico na sílaba que contém desinência verbal, como em \emph{João cheGOU!}. O ouvinte, por sua vez, ao pressupor a expressão de intenções de fala por parte do falante, deverá buscar (na informação verbal ou nos recursos que a acompanham) as pistas que lhe permitirão inferir o que seria tomado como relevante por aquele \citep{Sperber2001}.\footnote{Para um detalhamento dos princípios de relevância e da importância a estes atribuída na avaliação de habilidades pragmáticas, ver \citet{Longchamps2014}.} As linhas pontilhadas com setas nas duas pontas representam inter-relações entre domínios, ou possíveis interfaces entre domínios da cognição entre si e com o léxico. Dentre estas, vamos focalizar a relação entre sistemas intencionais e o léxico, no que concerne à informação de ordem intencional veiculada pelos traços semânticos/formais de categorias funcionais, o que aqui denominamos \emph{interface gramática-pragmática} (IGP). Na próxima seção, apresentaremos o resultado de experimentos de produção e de compreensão que visaram a avaliar em que medida dificuldades de aprendizagem podem ser associadas a comprometimentos nessa interface e recuperamos resultados de uma avaliação de habilidades pragmáticas não dependentes dessa interface, originalmente apresentados em \citet{Longchamps2014}. 




\begin{Figura}
    \includegraphics[width=\linewidth]{figures/53.png}
    \caption{{A produção de enunciados/atos de fala}}
    \label{fig:cap15fig1}
\end{Figura}


\section{Interface gramática-pragmática e problemas de aprendizagem}\label{sec:trêscap15}

Como dito anteriormente, tem sido constatado que crianças com problemas de aprendizagem apresentam dificuldades de ordem pragmática semelhantes àquelas encontradas em crianças com um transtorno de linguagem de natureza pragmática \citep{Bishop1989, Flanagan2005, Gerber1996, Norbury2002}. Diante disso, em um projeto conduzido em nosso laboratório, buscamos rastrear possíveis casos de TDL de natureza pragmática em crianças cujo desempenho escolar era considerado abaixo da média por parte dos professores.\footnote{Projeto DEL (Déficit Específico da Linguagem) e DAp (Dificuldades de Aprendizagem): pontos em comum, especificidade no DEL e bases de possíveis intervenções (com suporte FAPERJ-CNE 2011 para a primeira autora). conduzido no LAPAL (Laboratório de Psicolinguística e Aquisição da Linguagem – PUC-Rio).}  Nesse projeto, partimos de uma interpretação do termo \emph{específico}, da então utilizada denominação \emph{Déficit} ou \emph{Distúrbio ou Perturbação Específico/a da Linguagem} (como possíveis traduções para o termo SLI – \emph{Specific Language Impairment}), segundo a qual este termo não apenas sinalizaria ausência de etiologia conhecida para o transtorno (tal como é seu entendimento na área clínica), como poderia sinalizar a especificidade do transtorno ao domínio da linguagem, atribuindo-se uma estrutura modular à mente. Assim, tomamos como hipótese de trabalho que um comprometimento específico do domínio da linguagem, no que concerne à pragmática, deveria afetar fundamentalmente o que foi acima caracterizado como IGP. Diante disso seria, em princípio, possível supor um subconjunto de crianças com um comprometimento seletivo ou mais restrito a essa interface (a serem identificadas com o TDL-pragmático). Uma série de experimentos foi então criada com vista a explorar em que medida os contrastes gramaticais pertinentes à intencionalidade seriam expressos ou levados em conta por crianças com DAp. Destes, selecionamos dois a serem aqui apresentados.\footnote{Para a descrição detalhada da série de experimentos criados, ver \citep{Longchamps2014, Longchamps2011, Longchamps2013, Longchamps2014a}.}

\subsection{Referência e contraste aspectual na produção de narrativas }

Uma das dificuldades apontadas em crianças com comprometimentos da lingua\-gem de natureza pragmática reside na distinção entre referência definida e indefinida \citep{Bishop1989, Polite2011}. Definitude pode ser caracterizada como traço semântico/formal da categoria funcional D (determinante) que, em português, tal como em várias línguas, apresenta a oposição definido/indefinido por meio de distinções morfológicas/lexicais, como nos artigos (o/a; um/uma). Sintagmas determinantes (DPs)\footnote{ DP (\emph{Determiner phrase}) é a sigla convencionalmente utilizada para caracterizar a projeção máxima da categoria D (sintagmas nucleados por um determinante com um sintagma nominal (NP – \emph{Noun phrase}) como complemento, assim como pronomes pessoais e indefinidos, não ramificados).} indefinidos introduzem referentes no discurso, DPs definidos, plenos ou pronominais, pressupõem conhecimento compartilhado do referente em questão por falante/ouvinte. Seu uso é função de condições de acessibilidade relacionadas a posições de proeminência/relevância no discurso, dentre outras \citep{Ariel1991, Arnold2010}. Independentemente de acessibilidade, DPs definidos plenos delimitam passos da narrativa. Podem reintroduzir um referente por meio de uma relação de hiper/hiponímia, possibilitando que a relação conceitual entre os termos (por exemplo, fruta/maçã) seja recuperada ou inferida, o que facilita a coesão.  Podem também implicar unicidade/totalidade do conjunto a que se refere (se eu digo \emph{vou pegar a maçã}, isso implica que há apenas uma maçã; \emph{vou pegar as maçãs} indica que são todas as maçãs) (\emph{cf}. \citealp{Longchamps2014, Lyons1999}). Constata-se que, embora crianças façam uso de formas definidas e indefinidas já aos três anos de idade, a oposição novo/pressuposto na produção de narrativas não está estabelecida antes dos 7 anos de idade. Formas definidas tendem a ser usadas na introdução de referentes, o que indica que, mesmo no desenvolvimento típico, levar em conta a perspectiva do outro, o contexto discursivo e da enunciação, no relato de eventos, é uma aquisição tardia \citep{Hickmann1999, Hickmann1996, KarmiloffSmith1979}. Quanto à unicidade/totalidade da referência definida, mesmo adultos têm dificuldade em perceber essa informação, quando expressa exclusivamente pelo artigo, isto é, sem marcas ostensivas (como advérbios, como \emph{só}, \emph{apenas}; quantificadores, como \emph{todos}; ou ênfase prosódica no determinante) \citep{KarmiloffSmith1979, Longchamps2014, Longchamps2013}. 

A produção e a compreensão de narrativas também é crucialmente dependente de distinções aspectuais. \emph{Aspecto} gramatical codifica na língua o tempo interno de uma situação ou o relacionamento de eventos no tempo, situado na perspectiva do falante \citep{Comrie1976, Kazanina2007}. Na produção narrativa, distinções aspectuais (perfeito/imperfeito/contínuo ou progressivo) são fundamentais para o contraste entre o que é posto como fundo (\emph{brackground}) e o que é posto em evidência na sequência de passos narrativos, contribuindo, desse modo, para a coesão discursiva. Tal como no que diz respeito a definitude, a morfologia correspondente a aspecto é identificada antes dos quatro anos de idade, no desenvolvimento típico, mas distinções contrastivas entre impercfectivo e perfectivo são aquisições tardias \citep{Hickmann1999, McShane1988, Rodrigues2007}.
Com vista a avaliar a habilidade de crianças com DAp no estabelecimento de contrastes de definitude e aspecto, na distinção entre informação de fundo (imperfeito) e passos na narrativa (perfeito), apresentamos tarefas de produção de narrativas induzida por imagens. Os participantes foram 40 crianças: 20 com DAp (7 meninas); 20 com bom desempenho escolar (6 meninas), de duas escolas municipais do Rio de Janeiro (idade média 10:3 e 9:9 respectivamente), compondo os grupos (DAp) e controle (CTL), respectivamente.\footnote{As escolas municipais recebem predominantemente crianças de famílias de baixo-médio nível socioeconômico e de escolaridade.} Em cada grupo, 10 crianças foram submetidas a uma tarefa de reconto e 10 a uma tarefa de narração livre de três breves histórias, de estrutura semelhante.\footnote{Os dados de \citet{Longchamps2014} foram realizados para os presentes propósitos. Duas crianças originalmente alocadas para o grupo controle das narrativas livres foram eliminadas na presente análise, uma vez que suas idades (7.10 e 8.3 anos) haviam ficado abaixo da menor idade de crianças do grupo DAp alocadas para esta tarefa (9;6).} Cada história envolvia três personagens (referentes) do gênero masculino: um protagonista humano (personagem principal) e dois personagens não humanos do mesmo tipo (dois cachorros; dois cavalos; dois gatos), sendo um coadjuvante (com possível relação de posse pelo personagem humano) e um secundário, com relevância na complicação da narrativa. Para cada história, um conjunto de três figuras em sequência apresentava o enredo. 

Na tarefa de reconto, a experimentadora contava cada uma das histórias à criança, diante da sequência de imagens correspondentes à narrativa. A criança deveria recontar a história para outra criança que ouviria a gravação de sua voz (de modo a motivar a consciência do que seria conhecimento compartilhado, evitando-se o uso de dêiticos). Na tarefa de narração livre, a criança observava a sequência de imagens e era solicitada a contar a história para outra criança, ausente, a qual ouviria a gravação. Apresentamos abaixo um exemplo das histórias utilizadas (1). A Figura \ref{fig:cap15fig2} ilustra o material visual que a acompanha. 
%
% \begin{enumerate}
% \item[\emph{(1)}]
%
% \item []

\begin{Figura}[t]
 \includegraphics[width=1\linewidth]{figures/54.png}


% \subcaption{
\parbox{\textwidth}{\noindent\raggedright
\textbf{\emph{Um menino}} (personagem principal) \emph{estava brincando com \textbf{seu cachorrinho}} (personagem coadjuvante) \emph{numa pracinha. Ele estava atirando um pauzinho para o cãozinho pegar, quando \textbf{um outro cachorro}, grande e bravo} (personagem secundário), \emph{surgiu. O menino ficou desesperado pensando que o cachorro grande e bravo atacaria seu cachorrinho. Mas, para sua enorme surpresa, os dois cachorros ficaram amigos!}
}
% }


    \caption{Sequência de imagens de suporte à produção narrativa}
    \label{fig:cap15fig2}
\end{Figura}

% \end{enumerate}


Para cada tarefa, foram computados, em cada grupo: (i) o número de DPs pragmaticamente adequados na introdução de cada personagem; (ii) o número de formas não ambíguas com traço [+ definido] (pronominais ou DPs plenos) na manutenção da referência; (iii) a presença de contraste aspectual (perfeito/imper\-feito-contínuo) indicativo de criação do fundo para os passos narrativos.\footnote{Na introdução dos personagens, o escore máximo (3) corresponde ao número de histórias apresentadas. Na manutenção da referência, o número de ocorrências de DP definidos plenos e formas pronominais depende do quanto cada criança desenvolve a narrativa. Os percentuais de formas não ambíguas foram calculados em função do total de formas definidas produzidas nas três narrativas. Para o contraste aspectual, o escore máximo foi (3), relativo à cena inicial (\emph{background}) e à introdução da ação narrativa.} Para o personagem principal (P1), a introdução esperada é feita por meio do artigo indefinido. Para o personagem coadjuvante (P2), tanto o artigo indefinido quanto um possessivo (vinculando-o ao personagem principal) seriam pragmaticamente adequados. Para o personagem secundário (P3), o artigo indefinido e/ou o prono\-me indefinido “outro” seriam esperados. A Tabela \ref{tab:cap15tab1} apresenta os percentuais obtidos nas tarefas de reconto e de narração livre nos dois grupos. 



\begin{Tabela}
\caption{\% de ocorrências pragmaticamente adequadas de formas referenciais para cada personagem e de contraste aspectual por tarefa e grupo (Totais entre parênteses) }
\label{tab:cap15tab1}


\begin{tabularx}{\textwidth}{Xlrrrrrrr}
\lsptoprule
        &    & \multicolumn{3}{l}{Introdução do}    & \multicolumn{3}{l}{Manutenção da}    & Contraste \\
 Tarefa & Grupo    & \multicolumn{3}{l}{personagem}     & \multicolumn{3}{l}{referência}     & Aspectual\\
%  \cmidrule{3-8}
\midrule
 \addlinespace
        &    & P1  & P2    &	P3  &	P1  &	P2   &	P3\\
Reconto &DAp &	50 & 70    & 70     & 75.5  & 58.2   &33.3  &50\\
        &    & (30)& (30)  & (30)   &(49)   &(55)    &(15)  &(30)\\
        & CTL& 70  & 96.7  & 96.7   &81.5   &74.6    &40    &73.3\\
        &    & (30)& (30)  &(30)    & (65)  & (59)   & (20) &(30)\\
\hline
Narr.   & DAp& 56.7&63.3   &80      &75.6   &61.5    &57.5  &50.0\\
Livre   &    & (30)&(30)   &(30)    & (78)  & (78)   & (40) & (30)\\
        & CTL& 58.3&79.2   &83.3    &83.1   &75      &75.0  &76.7\\
        &    & (24)&(24)   &(24)    &(59)   &(80)    &(25)  &(24)\\
\midrule
Média Total& &58.7&77.3    &	82.5& 78.9  &	67.3&	51.5&	64.2\\
\lspbottomrule
\end{tabularx}


\parbox{\textwidth}{\footnotesize Obs. Os números entre parênteses representam o total sobre o qual os percentuais foram calculados.}
\end{Tabela}


Observa-se na Tabela \ref{tab:cap15tab1} que os percentuais do grupo DAp são, em geral, mais baixos do que os do grupo CTL particularmente na tarefa de reconto. No que concerne à introdução dos personagens, nessa tarefa, a diferença dos escores dos dois grupos é  estatisticamente significativa\footnote{Os dados relativos à introdução dos personagens e ao estabelecimento de contraste aspectual (ambos com máximo score = 3) foram analisados por meio do teste não paramétrico Mann-Whitney.} (Soma dos ranques: 764,5 (DAp ), 1065,5 (CTL); média dos ranques: 25,48 (DAp); 35,52 (CTL); U-valor 600,5 (DAp); 299,5 (CTL); desvio padrão (DP) 67,64), Z-Score = -2.21766, p =  0,03. Na manutenção da referência, observa-se um maior número de referências não ambíguas ao personagem principal, o que não trouxe diferença entre os grupos. Quanto à distinção aspectual, esta foi feita de forma mais consistente no grupo controle (Soma dos ranques: 309,5 (DAp ), 431,5 (CTL); média dos ranques: 23,97 (DAp); 23,97 (CTL); U-valor 260 (DAp); 99,5 (CTL); DP 34,21), Z-Score =- 2.33882, p = 0.02. Parece que a criança adota um estilo mais descritivo quando a distinção fundo/ação narrativa não é explicitamente marcada. Temos como exemplo a narrativa de um menino (11;7 anos) do grupo DAp: \emph{O primeiro quadradinho diz que o menininho jogou o... um graveto para o cãozinho pegar. Aí, um graveto caiu em cima do outro cachorro.} Em suma, no que concerne à produção de narrativas simples, com suporte de imagem, é a introdução do referente dependente do artigo indefinido e a adoção de um modo caracteristicamente narrativo, com distinção aspectual marcada, o que mais distingue os grupos com DAp e CTL. 

\subsection{Referência e contraste aspectual na compreensão de narrativas}
No que concerne à compreensão, as mesmas crianças do experimento de produção (DAp e CTL) foram apresentadas a quatro narrativas, com e sem apoio visual. Em (2), apresentamos um exemplo de narrativa e na Figura \ref{fig:cap15fig3}, a ilustração que a acompanha na condição com apoio visual. 


\begin{enumerate}
\item[(2)] \emph{Um dia, João estava tirando uma soneca no sofá da sala quando o telefone tocou. Ele ficou muito aborrecido! Depois de atender ao telefone, e ainda sonolento, João sentiu sede e resolveu pegar uma bebida na geladeira. O suco de uva estava uma delícia! Quando terminou, João voltou rapidamente para o sofá, mas quando ele estava pensando que ia voltar a tirar sua soneca, sua mãe se aproximou e lhe disse: “João, seu preguiçoso, vai fazer o seu dever de casa!”. Novamente, ele ficou bastante aborrecido, mas obedeceu e fez o que sua mãe disse. Naquele mesmo dia, mais tarde, João foi jogar uma partida de futebol na praça. Seus amigos esperavam que ele fizesse muitos gols. Mas João, infelizmente, não jogou nada bem. Quando um dos meninos perguntou pelos famosos gols, ele calmamente respondeu: “Ué, eu estava cansado! Também, não me deixaram dormir!”.}

\item [] \begin{Figura}
    \includegraphics[width=1\linewidth]{figures/55.png}


    \caption{{Exemplo de apoio visual para narrativa}}
    \label{fig:cap15fig3}
\end{Figura}
\end{enumerate}


A apresentação das histórias foi feita de forma interativa, entremeada de perguntas literais (cuja resposta pode ser recuperada diretamente do texto apresentado (por exemplo, \emph{Onde o João estava dormindo?}) e de perguntas inferenciais, dependentes da compreensão do contraste indefinido/definido (nesse caso envolvendo hiperonímia, por exemplo, \emph{Que bebida o João tomou?}), aspectual (\emph{Por que o João ficou aborrecido?}) e de modo (\emph{realis} (indicativo)/\emph{irrealis} (imperativo))\footnote{O subjuntivo não foi utilizado por não necessariamente estar presente na variedade do português brasileiro adquirida antes da escolarização.} Para cada história, 6 perguntas inferenciais (2 para cada tipo de contraste acima mencionado) foram formuladas. No caso das perguntas inferenciais de aspecto, por exemplo, a criança precisava inferir a razão de um comportamento, sentimento ou acontecimento, com base em um evento que ocorreu quando algo mais se encontrava em progresso. Assim, diante do trecho “\emph{Um dia, o João estava tirando uma soneca no sofá da sala quando o telefone tocou. Ele ficou muito aborrecido!}, e da pergunta \emph{Por que o João ficou aborrecido?} Em (2), a resposta envolveria inferir que a causa do aborrecimento foi o telefone ter tocado quando João estava dormindo (relacionando, dessa forma, o processo em andamento com um evento pontual). Quanto a modo, na narrativa acima, uma pergunta inferencial seria: \emph{A mãe do João achava que ele tinha feito o dever de casa?} 

Uma série de quatro perguntas indutoras foi elaborada de modo a conduzir a criança à resposta esperada, em caso de dificuldades, a serem apresentadas na seguinte sequência: (i) o trecho relevante era relido e era dito à criança que aquela parte continha uma pista; (ii) o contraste linguístico era parafraseado e a pergunta repetida; (iii) e (iv) explicitavam o conhecimento de ordem conceitual ou de mundo necessário para a inferência.\footnote{O uso de perguntas indutoras (\emph{prompts}) foi inspirado em \citep{Norbury2002}} Note-se que as perguntas indutoras tornam cada vez mais ostensivo o contraste de ordem intencional em questão. As respostas de cada criança foram pontuadas de 0 a 5 (0 para resposta-alvo não atingida após a 4ª pergunta-indutora e 5 para resposta-alvo atingida sem pergunta-indutora). Ver a figura \ref{qua:cap15qua1} para exemplos das perguntas indutoras relativas ao contraste aspectual, a partir da pergunta inferencial: \emph{Por que o João ficou aborrecido}, em (2).

\begin{Quadro}
\fbox{\parbox{.9\textwidth}{
Perguntas indutoras:\\
1ª) Ouça esta parte – ela contém uma pista: \emph{Um dia, o João estava tirando uma}
\emph{soneca no sofá da sala quando o telefone tocou. Ele ficou muito aborrecido!} E aí,
por que João ficou aborrecido?\\
2ª) O que João estava fazendo na hora em que telefone tocou? E então, por que o
João ficou aborrecido?\\
3ª) Telefone faz barulho? Então, por que João ficou aborrecido?\\
4ª) As pessoas ficam aborrecidas quando acordam com barulho? Então, por que
João ficou aborrecido?
}
}
\caption{{Exemplo de perguntas indutoras para perguntas inferenciais}}
\label{qua:cap15qua1}
\end{Quadro}

A presença de apoio visual não acarretou efeito facilitador. Crianças do grupo DAp tiveram pontuação significativamente mais baixa do que o grupo CTL no que diz respeito a inferências com base em contraste aspectual (Soma dos ranques: 285 (DAp ); 535 (CTL); média dos ranques: 14,25 (DAp); 26,75 (CTL); U-valor 325 (DAp); 75 (CTL); DP 36,96), Z-Score =-3,3; p =  0,0008. Também houve diferença entre grupos, no que diz respeito a inferências com base em definitude/hiperonímia, ainda que com escores relativamente altos (Soma dos ranques: 336,4 (DAp ); 483,5 (CTL); média dos ranques: 16,82 (DAp); 24,18 (CTL); U-valor 273,5 (DAp); 126,5 (CTL); DP 36,96), Z-Score = 1.97466; p =  0,02. Já no que concerne a modo, não houve diferença significativa entre os grupos e a pontuação foi relativamente baixa em ambos (Médias: 11,1 DAp; 13 CTL). Em suma, o contraste aspectual não parece ser suficientemente informativo para que grande parte das crianças, particularmente do grupo DAp, construa, por inferência, a intenção do falante de estabelecer uma relação de causalidade, no caso, entre dois eventos. Ainda que crianças de ambos os grupos recuperem um antecedente reintroduzido por hiperonímia, mais crianças DAp precisaram de perguntas indutoras (67,8\% DAp; 44,5\% CTL).\footnote{Os dados foram analisados por meio do teste não paramétrico Mann-Whitney.} 

Os resultados aqui relatados são compatíveis com a ideia de que um comprometimento na IGP pode afetar crianças DAp. No estudo mais amplo do qual esses experimentos fazem parte (Longchamps, 2014), verificou-se que, do total de 69 crianças DAp submetidas a um ou mais dos 6 experimentos conduzidos, 82,6\% tiveram escores dois desvios padrão abaixo da média do grupo CTL, em pelo menos uma das condições testadas. Todas as crianças DAp foram submetidas a um teste de habilidades sintáticas\footnote{MABILIN (Módulos de Avaliação de Habilidades Linguísticas) (\emph{cf}. \citetv{chapters/14})} e 36,2\% revelaram dificuldades apenas nas tarefas voltadas para a interface gramática-pragmática. 25 crianças DAp foram também submetidas a um teste que avaliou uma habilidade característica do comprometimento pragmático da linguagem em sentido mais amplo – a compreensão de frases idiomáticas (\emph{dar a volta por cima}; \emph{dar com o nariz na porta}, dentre outras) em pequenas narrativas. 60\% tiveram dificuldade nesta tarefa e nas tarefas voltadas à IGP; 12\% apenas na tarefa com expressões idiomáticas e 16\% apenas em contrastes da IGP. É possível, portanto, que dificuldades pertinentes à IGP sejam, em sua maioria, manifestações secundárias de um comprometimento do componente pragmático da cognição que interage com a língua e um pequeno subconjunto apresente um comprometimento seletivo primário nessa interface, sugestivo de um tipo particular de TDL. Em ambos os casos, atenção à IGP, por parte de professores e terapeutas, é necessária.

\section{Comprometimentos de ordem pragmática no TEA }\label{sec:quatrocap15}
O Manual Diagnóstico e Estatístico de Transtornos Mentais (DSM-5, 2014)\footnote{\citet{APA2014}. Porto Alegre: Artmed.} caracteriza o Transtorno do Espectro Autista (TEA) como um transtorno do neurodesenvolvimento manifesto em alterações comportamentais e comunicativas. Dificuldades na comunicação e no uso da linguagem são fatores preponderantes na composição do diagnóstico de um indivíduo no espectro autista. Os níveis de gravidade dentro do espectro são, contudo, bastante variados. No que se refere à linguagem, observam-se desde indivíduos com linguagem aparentemente normal (embora seja possível notar inadequações em seu uso) até aqueles em que a comunicação por meio da linguagem verbal é inexistente. Nota-se, portanto, que as dificuldades de linguagem no TEA são majoritariamente constatadas na comunicação. Discurso comprometido em termos de relevância, dificuldade na troca de turnos, na conversação, uso de pronomes sem estabelecimento prévio de seus antecedentes, preferência por DPs plenos em contextos em que crianças com desenvolvimento típico empregam pronomes são algumas das manifestações sugestivas de comprometimento de ordem pragmática no desempenho linguístico de crianças com TEA \citep{APA2014, Longchamps2014, Ribeiro2019}. No que diz respeito à sintaxe/fonologia, é possível diferenciar, no espectro, um grupo com a linguagem comprometida (\emph{Language Impaired} – ALI) – com características semelhantes ao TDL/PDL –, em contraste com outro sem o mesmo tipo de comprometimento (\emph{Normal Language} – ALN) \citep{Boucher2012}. Entretanto, habilidades dependentes de contrastes de natureza intencional representados em elementos de categorias funcionais, ou seja, habilidades pertinentes à IGP, tal como aqui caracterizadas, não têm sido diretamente exploradas. A fim de verificar o desempenho de crianças no espectro autista (crianças TEA), no que se refere a habilidades relacionadas a essa interface, foi conduzido um conjunto de tarefas experimentais, que incluíram a compreensão de contrastes de \emph{Aspecto, Modo, Definitude}, assim como a interpretação de reflexivos e pronominais acusativos de 1ª e 3ª pessoa \citep{Ribeiro2019}. Ainda que com desempenho inferior ao do grupo controle no conjunto das tarefas, as dificuldades de crianças no espectro autista foram particularmente detectadas em reflexivos e pronominais, e na compreensão do modo \emph{irrealis}. A seção abaixo apresenta o experimento voltado ao modo verbal.\footnote{Para reflexivos e pronominais, com referência a peculiaridades da aquisição do PB ver \citet{Correa2020a}.}  


\section{Desempenho na interface gramática-pragmática de crianças no espectro autista}\label{sec:cincocap15}

As crianças no espectro autista participantes do estudo reportado em \citet{Ribeiro2019} eram estudantes de uma escola localizada na zona norte do Rio de Janeiro, voltada especificamente para o ensino de alunos com diagnóstico de TEA. Para todos os testes, foram recrutadas 10 crianças, todos meninos, com faixa etária entre sete e treze anos (idade média 10;5 anos), considerados sem dificuldades de linguagem ou de aprendizagem, por meio de critérios estabelecidos pela escola. Para fins de controle, outras 15 crianças sem queixas de linguagem ou de aprendizagem, da mesma faixa etária, foram recrutadas de uma escola municipal, também na zona norte da cidade.

No experimento referente a \emph{modo}, avaliou-se a compreensão de sentenças no modo indicativo, quando o evento se configura na realidade (\emph{realis}); e subjuntivo/imperativo, quando o que se quer dizer se expressa como uma dúvida, uma suposição (subjuntivo) ou, no caso do imperativo, uma ordem (ambos \emph{irrealis}). A distinção entre \emph{realis} e \emph{irrealis} é expressa na morfologia do verbo principal ou do verbo auxiliar em uma locução verbal ou tempo composto. Sendo \emph{irrealis} algo não factual, no campo das possibilidades, é necessário que o ouvinte seja capaz de inferir que algo ainda não aconteceu, ainda que seja esperado ou desejado. Portanto, espera-se que o modo \emph{irrealis} seja mais custoso para a compreensão do que o \emph{realis}, que corresponde a algo factual. Verificou-se também se o tipo de discurso (direto \emph{vs} indireto) influenciaria a interpretação do modo verbal. Foram apresentadas três sentenças para cada condição experimental. As sentenças (3-6) exemplificam assas condições. 

\begin{enumerate}
    \item [(3)]	\emph{Vovó/vovô disse: Você roeu o osso, cachorro!} (Discurso direto no indicativo, \emph{realis})	
\item[(4)]	\emph{Vovó/vovô disse: Vai beber o suco, menino!} (Discurso direto imperativo, \emph{irrealis}) 
\item[(5)]	 \emph{A vovó disse que o menino tinha escorregado.} (Discurso indireto no indicativo, \emph{realis})
\item[(6)]	\emph{A vovó disse que a menina fosse comer a maçã.} (Discurso indireto no subjuntivo, \emph{irrealis})
\end{enumerate}
Dois personagens foram introduzidos no teste para apresentar as frases experimentais, uma vovó e um vovô, que ora davam ordens ora contavam os últimos acontecimentos. As falas dos personagens foram gravadas e, a partir de um clique no balão de fala da vovó ou do vovô, o enunciado era emitido e o participante solicitado a escolher o quadro correspondente à fala do personagem (\emph{cf}. Figuras \ref{fig:cap15fig4}-\ref{fig:cap15fig7}).\footnote{Os desenhos dos personagens e das cenas foram criados por meio de um \emph{software} disponibilizado no site ToonDoo, voltado para criação de histórias em quadrinhos. Os slides elaborados por meio dessas imagens foram apresentados por meio do programa Power Point em um laptop Macbook Air. As respostas dos participantes foram colhidas manualmente.}  



\begin{Figura}
\subfigure[Você roeu o osso, cachorro! (Discurso direto, Modo Indicativo {[Realis]})]{
  \includegraphics[width=.45\linewidth]{figures/56.png}
  \label{fig:cap15fig4}
  }
\subfigure[O vovô disse que a menina fosse comer a maçã. (Discurso indireto, Modo Subjuntivo (Irrealis).]{
  \includegraphics[width=.45\linewidth]{figures/57.png}
  \label{fig:cap15fig5}
}
\caption{~}
\end{Figura}


\begin{Figura}
\subfigure[Vai dormir, menina! (Discurso direto, Modo Imperativo (Irrealis).]{
  \includegraphics[width=.45\linewidth]{figures/58.png}
  \label{fig:cap15fig6}
  }
  \subfigure[O vovô disse que a menina tinha atendido ao telefone (Discurso indireto, Modo Indicativo {[Realis]})]{
  \includegraphics[width=.45\linewidth]{figures/59.png}
  \label{fig:cap15fig7}
  }
  \caption{~}
\end{Figura}


O número de respostas corretas (independentemente de \emph{modo} e \emph{tipo de discurso}) foi significativamente maior no grupo controle do que no grupo TEA (TEA: soma dos ranques 96; média dos ranques 9,6, U = 109; CTL: soma dos ranques 229; média dos ranques 15,27; U= 42; desvio padrão 18,03), Z-score = 1,85825;  \emph{p} = 0.03.\footnote{A comparação entre grupos foi feita por meio do teste Mann-Withney e inter-sujeitos por meio do teste Friedmann, ambos não paramétricos.  } 

A Figura \ref{fig:cap15fig8} ilustra o efeito de \emph{Modo} em cada grupo.


\begin{figure}[t]
\pgfplotstableread{
1 2.8      1.5
2 2.9      2.3
}\dataset
\begin{tikzpicture}
\begin{axis}[ybar,
        width=.6\textwidth,
        height=.3\textheight,
        enlarge x limits=.5,
        ymin=0,
        xtick=data,
%         bar width=15mm,
        nodes near coords,
        nodes near coords style={font=\footnotesize},
        legend cell align=left,
        legend style={font=\footnotesize,at={(1.05,.05)},anchor=south west},
        xticklabels = {
            TEA,
            Controle
        },
      axis y line*=left,
      axis x line*=bottom,
        x tick label style={align=center,text width=2cm},
        ticklabel style = {font=\footnotesize},
        ]
\addplot[draw=black,fill=tmnlpone] table[x index=0,y index=1] \dataset;
\addplot[draw=black,fill=tmnlptwo] table[x index=0,y index=2] \dataset;
\legend{Realis,Irrealis}
\end{axis}
\end{tikzpicture}
    \caption{{Médias de acertos, em função de Grupo e Modo verbal (máx. score=3)}}
    \label{fig:cap15fig8}
\end{figure}




Como se pode observar na Figura \ref{fig:cap15fig8}, não houve dificuldade na compreensão do modo \emph{realis} em ambos os grupos. No modo \emph{irrealis}, por sua vez, as médias de acertos são mais baixas nos dois grupos, ainda que o desempenho no grupo TEA tenha sido significativamente inferior ao do grupo CTL (Soma dos ranques: 96 (TEA), 229 (CTL); média dos ranques: 8,25 (TEA); 16,17 (CTL); U-valor 109 (TEA); 75 (CTL); DP = 18,03), Z-Score = 1.85825 \emph{p} = 0,03. 

Na Figura \ref{fig:cap15fig9}, constata-se que essa diferença entre grupos se manifesta particularmente no discurso direto (Soma dos ranques: 82,5 (TEA), 242,5 (CTL); média dos ranques: 9,6 (TEA); 15,27 (CTL); U-valor 122,5 (TEA); 27,5 (CTL); DP = 18,03), Z-Score = 1.85825 \emph{p} = 0,03. 



\begin{Figura}
%     \includegraphics[width=\linewidth]{figures/61.png}
    \pgfplotstableread{
1 2.8  2.7   1.4   1.7
2 2.9  2.9   2.5   2.1
}\dataset
    \begin{tikzpicture}
\begin{axis}[ybar,
        width=.7\textwidth,
        height=.3\textheight,
        ymin=0,
        xtick=data,
        enlarge x limits=.5,
%         bar width=15mm,
        nodes near coords,
        nodes near coords style={font=\footnotesize},
        legend cell align=left,
        legend style={font=\footnotesize,at={(1.05,.05)},anchor=south west},
        xticklabels = {
            TEA,
            Controle
        },
      axis y line*=left,
      axis x line*=bottom,
        x tick label style={align=center,text width=2cm},
        ticklabel style = {font=\footnotesize},
        ]
\addplot[draw=black,fill=tmnlpone] table[x index=0,y index=1] \dataset;
\addplot[draw=black,fill=tmnlptwo] table[x index=0,y index=2] \dataset;
\addplot[draw=black,fill=tmnlpthree] table[x index=0,y index=3] \dataset;
\addplot[draw=black,fill=tmnlpfour] table[x index=0,y index=4] \dataset;
\legend{Realis/Direto, Realis/Indireto,Irrealis/Direto,Irrealis/Indireto}
\end{axis}
\end{tikzpicture}

    \caption{{Médias de acertos nos grupos TEA e CTL, em função da variável Modo verbal e do Tipo de discurso (máx. escore=3)}}
    \label{fig:cap15fig9}
\end{Figura}


O efeito do tipo de discurso independentemente de modo e de grupo não foi significativo. Porém, no grupo TEA, um efeito do tipo de discurso pode ser constatado, com pior desempenho no discurso direto irrealis  X\textsuperscript{2}\textsubscript{r} = 6,4 (1, \emph{N} = 10), \emph{p} = 0,01. Em suma, os resultados sugerem que ainda que não haja queixas de linguagem no grupo TEA testado, diferenças sutis, pertinentes à IGP podem ser detectadas, em situação mais caracteristicamente dialógica (discurso direto), talvez como índice de dificuldades de ordem interativa no espectro. 


\section{Conclusão}

Há dificuldades de ordem pragmática observadas em crianças DAp e TEA que podem ser situadas na chamada interface gramática-pragmática (IGP) como aqui caracterizada, ou seja, que comprometem a maneira como intenções de fala se articulam com o modo como enunciados são estruturados e interpretados a partir de elementos de categorias funcionais. Ainda que dificuldades de ordem pragmática possam, em geral, comprometer a linguagem de forma secundária, um comprometimento seletivo, sugestivo de um transtorno linguístico primário na IGP, é uma possibilidade a ser investigada em crianças DAp. A identificação de comprometimentos sutis, como os aqui investigados, é um caminho para que intervenções venham a atuar nos problemas específicos que as crianças apresentam. O uso de perguntas indutoras, tornando ostensivos os contrastes gramaticais expressos por elementos funcionais é um possível caminho de intervenção. 



{\sloppy\printbibliography[heading=subbibliography,notkeyword=this]}
\end{document}
