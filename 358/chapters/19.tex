\documentclass[output=paper,colorlinks,citecolor=brown,booklanguage=portuguese]{langscibook} 
\ChapterDOI{10.5281/zenodo.7233249}
\title{Aquisição e desenvolvimento atípico da linguagem em crianças surdas}
\author{Ana Mineiro \affiliation{Instituto de Ciências da Saúde, Universidade Católica Portuguesa, CIIS} and  Sofia Lynce de Faria\affiliation{Instituto de Ciências da Saúde, Universidade Católica Portuguesa, CIIS} and  Mara Moita\affiliation{Instituto de Ciências da Saúde, Universidade Católica Portuguesa, CIIS; Universidade Nova de Lisboa, CLUNL} }

\abstract{O presente capítulo aborda os efeitos da surdez na aquisição e desenvolvimento da linguagem na modalidade linguística gestual e oral. Começando por uma breve descrição dos diferentes graus, tipos e causas que caracterizam a surdez, e que devem ser considerados aquando da observação do desempenho linguístico de uma criança surda, o capítulo explora vários indicadores de um desenvolvimento atípico na aquisição de uma língua gestual e, particularmente, da Língua Gestual Portuguesa, em diferentes áreas do conhecimento linguístico (fonologia, morfologia, sintaxe e semântica). Adicionalmente, com base nos primeiros estudos de linguagem realizados a crianças portuguesas surdas com implante(s) cocleare(s), descreve-se o desenvolvimento atípico na aquisição de uma língua oral (particularmente, do Português Europeu), revelando, em específico, as dificuldades identificadas em tarefas de produção de conhecimento fonológico e em tarefas de compreensão e produção de conhecimento sintático. Em jeito de conclusão, são colocadas questões associadas aos instrumentos de avaliação e de diagnóstico de conhecimento linguístico utilizados em crianças surdas. 

\textbf{Palavras-chave:} Surdez; Desenvolvimento atípico; Aquisição de língua gestual; Aquisição linguística oral. }


\IfFileExists{../localcommands.tex}{
   \addbibresource{../localbibliography.bib}
   % add all extra packages you need to load to this file

\usepackage{tabularx,multicol}
\usepackage{url}
\urlstyle{same}

\usepackage{listings}
\lstset{basicstyle=\ttfamily,tabsize=2,breaklines=true}

\usepackage{langsci-basic}
\usepackage{langsci-optional}
\usepackage{langsci-lgr}
\usepackage{langsci-osl}
% \usepackage{./langsci/styles/langsci-lgr}
% \usepackage{./langsci/styles/langsci-osl}
% \usepackage{langsci-gb4e}

\usepackage{tikz}
\usetikzlibrary{patterns,calc}
\pgfdeclarepatternformonly{south east lines}{\pgfqpoint{-0pt}{-0pt}}{\pgfqpoint{3pt}{3pt}}{\pgfqpoint{3pt}{3pt}}{
    \pgfsetlinewidth{0.6pt}
    \pgfpathmoveto{\pgfqpoint{0pt}{3pt}}
    \pgfpathlineto{\pgfqpoint{3pt}{0pt}}
    \pgfpathmoveto{\pgfqpoint{.2pt}{-.2pt}}
    \pgfpathlineto{\pgfqpoint{-.2pt}{.2pt}}
    \pgfpathmoveto{\pgfqpoint{3.2pt}{2.8pt}}
    \pgfpathlineto{\pgfqpoint{2.8pt}{3.2pt}}
    \pgfusepath{stroke}}
    
\usepackage{stmaryrd}
\usepackage{wasysym}
\usepackage{multirow}
\usepackage{caption}
\usepackage{subcaption}
\usepackage{mathrsfs}
\usepackage{qtree}

\usepackage{linguex}


   %pminos do not split footnotes
% \interfootnotelinepenalty=10000 %Footnote in Laporte chapters has to be split SN


%\DeclareIndexNameFormat{default}{%
%\nameparts{#1}%
%\usebibmacro{index:name}%
%{\index[names]}%
%{\namepartfamily}%
%{\namepartgiveni}%
% {}% L1
% {}% L2
%{\namepartprefix}% generates spurious space L3
%{\namepartsuffix}% generates spurious space L4
%}

%  {\DeclareIndexNameFormat{default}{%
%     \usebibmacro{index:name}{\index[names]}{#1}{#3}{#5}{#7}}}

%\DeclareIndexNameFormat{default}{%
%  \usebibmacro{index:name}{\sindex[nom]}{#1}{#3}{#5}{#7}}

%\DeclareIndexNameFormat{default}{%
%  \usebibmacro{index:name}{\sindex[person]}{#1}{#3}{#5}{#7}}
%\DeclareIndexNameFormat{default}{%
%\nameparts{#1} \usebibmacro{index:name}{\sindex[person]]}{\namepartfamily}{‌​\namepartgiven}{\nam‌​epartprefix}{\namepa‌​rtsuffix}}

%\newcommand{\smiley}{:)}

%\renewbibmacro*{index:name}[5]{%
%\usebibmacro{index:entry}{#1}%
%{\iffieldundef{usera}{}{\thefield{usera}\actualoperator}\mkbibindexname{#2}{#3}{#4}{#5}}}

% \newcommand{\noop}[1]{}

%remove for final
%\overfullrule=1mm

\newcommand{\tobi}[2]}}
\renewcommand{\S}[1]{\tobi{#1}{\textsc{*}}}

% this volume references
% puts: [this volume]
% already defined: \citetv
%\newcommand{\citepv}[1]{(\citeauthor{#1} \citeyear*{#1} [this volume])}
\newcommand{\citealtv}[1]{\citeauthor{#1} \citeyear*{#1} [this volume]}

%parentheses around example number
\newcommand{\pref}[1]{(\ref{#1})}

% in-text examples

\newcommand{\lnex}[1]{\textit{#1}} %target lang word
\newcommand{\lnlit}[1]{(lit.: `#1')} %literal reading
\newcommand{\lnlat}[1]{(#1)} % latinization
\newcommand{\lntrans}[1]{`#1'} %translation
\newcommand{\lnexl}[2]%
{\lnex{#1}{} \lnlat{#2}} % ex with latinization
\newcommand{\lnexlat}[3]{\lnex{#1}{} \lnlat{#2}{} \lntrans{#3}} % ex with latinization and tranl.

%ch01
\newcommand{\co}[1]{\mbox{\textbf{#1}}}

%ch09

\newcommand{\cyrbulg}[1]{\begin{otherlanguage*}{bulgarian}#1\end{otherlanguage*}}


%ch10
\newcommand{\nlp}{{\small NLP}}
\newcommand{\mwe}{{\small MWE}}
\newcommand{\rae}{{\small RAE}}
\newcommand{\lvc}{{\small LVC}}
\newcommand{\pos}{{\small P}o{\small S}}
%\newcommand{\todo}[1]{ \textcolor{red}{#1} }

%\renewcommand{\labelenumi}{\theenumi}
%\ainamefmt{{vv}{ll}{, ff}{, jj}} % fullname

\newcommand{\biberror}[1]{{\color{red}#1}}

\newcommand{\osenovaitem}{--~}
   %% hyphenation points for line breaks
%% Normally, automatic hyphenation in LaTeX is very good
%% If a word is mis-hyphenated, add it to this file
%%
%% add information to TeX file before \begin{document} with:
%% %% hyphenation points for line breaks
%% Normally, automatic hyphenation in LaTeX is very good
%% If a word is mis-hyphenated, add it to this file
%%
%% add information to TeX file before \begin{document} with:
%% %% hyphenation points for line breaks
%% Normally, automatic hyphenation in LaTeX is very good
%% If a word is mis-hyphenated, add it to this file
%%
%% add information to TeX file before \begin{document} with:
%% \include{localhyphenation}
\hyphenation{
    Beck-man
    Ngu-yen
    back-chan-nel
    back-chan-nels
    mo-not-o-nous
    ste-reo-typ-i-cal
}

\hyphenation{
    Beck-man
    Ngu-yen
    back-chan-nel
    back-chan-nels
    mo-not-o-nous
    ste-reo-typ-i-cal
}

\hyphenation{
    Beck-man
    Ngu-yen
    back-chan-nel
    back-chan-nels
    mo-not-o-nous
    ste-reo-typ-i-cal
}

   \boolfalse{bookcompile}
   \togglepaper[19]%%chapternumber
}{}

\begin{document}
\maketitle

\section{Introdução}

Surdez é um termo genérico utilizado para definir uma perda de audição, correntemente, também denominada pela classe médica como hipoacusia \citep{Paco2010}.\footnote{A norma adotada na escrita deste capítulo foi a do português europeu.}
% \footnote{Na sequência de restrições decorrentes do template usado para publicação na editora Language Science Press, os capítulos que integram o volume surgem com a referência [this volume], em inglês.}
A surdez é um dos défices sensoriais mais frequentes em todo o mundo. A Organização Mundial de Saúde, em 2018, divulgou a existência de 466 milhões de pessoas surdas em todo o mundo, das quais 432 serão adultos e 34 milhões serão crianças. Estima-se que 90\%-95\% das crianças surdas são filhas de pais ouvintes \citep{Mitchell2004}, pelo que não estarão expostas, desde logo, a um \emph{input} linguístico que assente numa modalidade apropriada às suas capacidades sensoriais. Deste modo, ficará condicionada a aquisição e desenvolvimento de uma língua no(s) seu(s) primeiro(s) ano(s) de vida. Algumas pessoas surdas\footnote{Neste capítulo, adota-se o termo surdo para designar sujeitos, cuja a perda auditiva é congénita ou adquirida, unilateral ou bilateral, e de qualquer grau ou tipo. Não utilizamos o termo deficiente auditivo por respeito à comunidade surda que não se revê nessa perspetiva cultural.} adquirem uma língua gestual, outras adquirem uma língua oral, e outras, ainda, adquirem duas línguas de duas modalidades distintas, gestual e oral, sendo estas designadas bilingues bimodais.

Para compreender melhor o impacto da surdez na aquisição e desenvolvimento de uma língua, importa ter em consideração que o processo de aquisição de linguagem decorre num curto espaço de tempo e envolve operações de elevada complexidade \citep{SimSim2017}. Sabe-se também que este processo começa logo durante o período fetal, uma vez que logo nos primeiros dias/horas de vida os bebés ouvintes percecionam diversas propriedades sonoras da fala sendo capazes de percecionar efetivamente estímulos linguísticos \citep{Moon2013}. 

Neste contexto, a exposição tardia a uma língua tem sido indicada como causa para as dificuldades na aquisição e desenvolvimento da linguagem na criança surda. A distinção entre um desenvolvimento atípico consequente de uma exposição tardia à língua ou resultante de uma perturbação de linguagem tem sido difícil de determinar na população infantil surda perante a dificuldade de se encontrar e avaliar crianças surdas expostas a uma língua desde o nascimento. Neste contexto, as dificuldades linguísticas resultantes de uma exposição tardia poderiam ser concomitantes às perturbações de linguagem diagnosticadas na infância. Seguindo estes pressupostos, abordaremos o desenvolvimento atípico de uma língua gestual (a Língua Gestual Portuguesa, doravante LGP) assim como o desenvolvimento atípico de uma língua oral (o Português Europeu, doravante PE), em crianças com surdez severa a profunda bilateral.

Importa, ainda, ter em consideração que, quando se observa, se analisa, se avalia ou se descreve o desempenho linguístico de uma criança com surdez severa a profunda bilateral, existe um conjunto de variáveis extralinguísticas que deverão ser consideradas (tais como, tipo e grau de surdez, idade de diagnóstico da surdez; tempo de privação ao \emph{input} linguístico; uso de dispositivos auditivos; ensino monolingue \emph{versus} bilingue; intervenção precoce em terapia da fala; envolvimento parental, entre outras). Na literatura, estas variáveis extralinguísticas revelam-se determinantes para a aquisição e desenvolvimento da linguagem.

Acresce ainda dizer que, na literatura, não é consensual a existência de implicações negativas no desenvolvimento cognitivo associadas à privação de \emph{input} linguístico nos primeiros anos de vida resultante da surdez. Atualmente, considera-se que o impacto da surdez no desenvolvimento cognitivo é bastante variável e, eventualmente, influenciado pelo meio ambiente em que a criança cresce \citep{Cardoso2009, Hall2019}. Neste contexto, importa, anteriormente à observação e/ou avaliação linguística de uma criança surda, rastrear possíveis défices cognitivos.

\section{Classificação de surdez}
Antes de abordarmos a classificação da surdez, importa compreender que a audição é a capacidade sensorial que permite ao ser humano captar o estímulo sonoro através do canal auditivo externo (ouvido externo) sob a forma de ondas de pressão, transmitindo-o pela cadeia de ossículos (ouvido médio) à janela oval (ouvido interno) fazendo vibrar a membrana basilar, onde estão alojadas as células ciliadas externas e as células ciliadas internas (e.g. \citealp{Rego2017}). Estas últimas são responsáveis por gerar potenciais de ação no nervo auditivo que serão transmitidos ao córtex auditivo onde serão processadas como som (e.g. \citealp{Oliveira1993}).

A surdez pode ser caracterizada quanto ao grau, tipo e etiologia. Na classificação de acordo com o grau, vários países europeus adotaram uma terminologia similar, que teve por base a Recomendação 02/1 do \emph{Bureau International d’Audio-phonologie} \citep{BIAP1996}. Os graus são definidos pelos intervalos onde a média dos limiares de audição, em decibéis (dB), por via aérea, falham. O limiar de audição corresponde ao nível sonoro abaixo do qual o ouvido do ser humano é incapaz de detetar qualquer som. Deste modo, avalia-se a intensidade sonora (em dB) de que o sujeito necessita para perceber um som nas diversas frequências \citep{Cardona2013}. A classificação do \citet{BIAP1996} descreve cinco categorias:

\begin{enumerate}
\item 	perda auditiva ligeira (entre os 21 - 40 dB) - a fala é percebida com alguma dificuldade, em particular, quando a frequência subjetiva utilizada é reduzida;
\item 	perda auditiva moderada (entre os 41 dB - 70 dB) - a fala é percebida quando a mensagem é produzida com uma intensidade vocal subjetiva elevada. Neste contexto, o sujeito entende melhor o que lhe é dito, se estiver mais próximo do interlocutor;
\item 	perda auditiva severa (entre os 71 - 90 dB) - a mensagem falada é percebida se a intensidade vocal subjetiva for elevada e proferida próxima ao ouvido do sujeito;
\item 	perda auditiva profunda (entre os 91 - 119 dB) - a mensagem falada não é percebida. Porém, os ruídos com uma intensidade subjetiva elevada são percebidos pelo sujeito;
\item 	cofose (superior a 120 dB), o sujeito não percebe qualquer som.
\end{enumerate}
A perda auditiva poderá também ser classificada quanto à sua localização topográfica, em três tipos \citep{Monteiro2018}: 

\begin{enumerate}
\item  	perda auditiva de condução (transmissão) - as ondas sonoras não alcançam o ouvido interno adequadamente por alterações no ouvido externo e/ou no ouvido médio;
\item 	perda auditiva sensorioneural (perceção) - existência de lesões no ouvido interno e/ou nervo auditivo;
\item 	perda auditiva mista - as vias aérea e óssea encontram-se afetadas, resultando em lesões do tipo condutiva e sensorioneural num mesmo ouvido.
\end{enumerate}
A classificação da surdez poderá ainda realizar-se através da sua etiologia, identifi\-cando-se dois grandes grupos de fatores, os hereditários e os não hereditários. 

No contexto de surdez sensorioneural de grau severo a profundo, o implante coclear (IC) é o dispositivo biomédico indicado em casos em que os sujeitos não beneficiem do uso de próteses auditivas convencionais bem-adaptadas ou o fazem minimamente \citep{Monteiro2018, Paco2010}. O IC é composto por componentes internos e externos que comunicam transcutaneamente via sinais de radiofrequência. Dos componentes externos fazem parte o microfone, o processador de fala e o transmissor. Estes três componentes têm a função de captar os sons ambientais tal como a fala e a música, e de transmiti-los aos componentes internos, o recetor-estimulador e os elétrodos. O recetor-estimulador é responsável por passar o estímulo elétrico ao longo da matriz de elétrodos, que, por sua vez, estimulam o nervo auditivo de forma a que este envie a informação até ao córtex cerebral para que os estímulos elétricos sejam percecionados como sons \citep{Cardona2013}. A ativação e programação do dispositivo terá lugar, 2 a 4 semanas, após a realização da cirurgia \citep{Yawn2015}.

Neste capítulo, abordaremos a aquisição e desenvolvimento atípico da língua gestual em crianças surdas e da língua oral em crianças surdas com IC. 

\section{Aquisição e desenvolvimento atípico da língua gestual}

As línguas gestuais são as línguas naturais das comunidades surdas que foram evoluindo a partir da organização social e linguística dessas mesmas comunidades e da sua história. Existem inúmeras línguas gestuais no mundo inteiro, com léxicos e gramáticas diferentes.

Assim, todas as línguas gestuais do mundo apresentam informação fonológica, morfológica, sintática, semântica, pragmática e discursiva a partir de articuladores manuais (mãos e dedos) e não manuais (tronco, cabeça, braços) representados no espaço em frente ao gestuante.

Muitos investigadores têm-se dedicado ao estudo da base neural das línguas gestuais através de estudos de neuroimagem, observando que estas línguas de modalidade visuo-gestual se sedimentam nas mesmas áreas corticais das línguas orais \citep{Cheng2019, MacSweeney2008}. As diferenças encontradas estão associadas, sobretudo, ao processamento da modalidade linguística visuo-gestual \emph{versus} modalidade auditiva-oral \citep{Emmorey2007}.

Também a investigação na aquisição e desenvolvimento de língua gestual tem procurado estabelecer um paralelismo com a aquisição e desenvolvimento das línguas orais, observando-se que, quando a exposição à língua gestual é precoce, as etapas de desenvolvimento linguístico desta modalidade linguística são, em geral, paralelas às etapas observadas nas crianças ouvintes a adquirir uma língua oral desde o seu nascimento \citep{Newport1985, Schick2005}. Tal como acontece com a aquisição de uma língua oral nas crianças ouvintes, o processo de aquisição e desenvolvimento de uma língua gestual em crianças surdas pode refletir um desenvolvimento típico (doravante DT) ou um desenvolvimento atípico.

O conhecimento linguístico gestual de crianças surdas tem vindo a ser analisado, sobretudo, com recurso à adaptação de instrumentos de avaliação da linguagem oral/verbal, sendo, ainda, escassa a existência de instrumentos de avaliação direcionados ao conhecimento linguístico gestual \citep{Haug2020, Mann2015}.

Nas próximas secções, exploraremos como é que o desenvolvimento atípico de uma língua gestual tem vindo a ser observado nas várias áreas da linguagem.

\subsection{Considerações gerais sobre o desenvolvimento atípico da língua gestual}
Uma criança com desenvolvimento atípico de linguagem pode apresentar dificuldades em diferentes aspetos e áreas da linguagem, ou seja, pode apresentar dificuldades com o conteúdo, com a forma ou inclusivamente com o uso em diferentes áreas do conhecimento linguístico. Estas crianças podem até ter dificuldades numa ou mais do que uma área da linguagem em contexto de produção e/ou de compreensão.

Nos últimos anos, têm surgido um conjunto de estudos sobre crianças surdas gestuantes de British Sign Language (BSL) que apresentam dificuldades linguísticas específicas revelando a existência de alguns padrões de desenvolvimento linguístico atípico, eventualmente, transversais às línguas independentemente da sua modalidade. Estes estudos foram realizados, sobretudo, com base em tarefas de compreensão e de produção paralelas às tarefas utilizadas em estudos sobre o desenvolvimento da linguagem oral.

Na observação do conhecimento fonológico, assume-se que a identificação de dificuldades no desempenho de uma tarefa de repetição de pseudopalavras poderá refletir num desenvolvimento atípico da linguagem oral \citep{Archibald2007}. Através da manipulação dos parâmetros fonológicos como a configuração e o movimento, este tipo de tarefa foi utilizado na avaliação de crianças surdas gestuantes previamente identificadas com desenvolvimento atípico \citep{Mason2010, Morgan2007, Sehyr2018}. Contudo, nem todas as crianças identificadas apresentaram dificuldades na repetição de pseudopalavras.

Ainda no contexto de produção, o planeamento, a articulação e a coordenação dos gestos é por si só uma tarefa difícil que exige a execução de movimentos no tempo e na sequência certa. As capacidades musculoesqueléticas das crianças para movimentar as mãos, os braços, o corpo e a cabeça podem estar comprometidas e, nesse caso, provocar uma dificuldade acrescida na produção da linguagem resultante de uma disartria \citep{Tyrone2014}. 

Importa, ainda, referir que dificuldades no acesso visual, a porta de entrada das línguas gestuais, podem impedir uma assimilação correta dos gestos que, se não são bem visualizados, serão incorretamente produzidos. É revelante salientar que, no que concerne a compreensão da linguagem, as crianças poderão, por vezes, não processar o que estão a ver e, por isso, não compreender o que os gestos significam, como se interligam uns com os outros através de uma estrutura gramatical ou ainda como são ou podem ser usados em contextos diferentes.

O conhecimento semântico-lexical tem sido avaliado nas línguas orais com base em medições do vocabulário expressivo e/ou recetivo \citep{SuaKay2011}. Em crianças surdas gestuantes, este conhecimento tem sido estudado através de adaptação de testes de vocabulário específicos para as línguas orais, como o Peabody Picture Vocabulary \citep{Dunn1981, Morgan2007} ou através de tarefas de fluência semântica construídas para o propósito \citep{Marshall2013}. Nestes estudos realizados com crianças surdas gestuantes identificadas por apresentarem dificuldades linguísticas, observou-se que estas não apresentaram diferenças significativas tanto ao nível expressivo como recetivo do vocabulário em comparação com os seus pares com DT. Contudo, verificou-se que as crianças que, inicialmente, foram identificadas com um desenvolvimento da linguagem atípico apresentaram um tempo de reação superior no que respeita ao acesso lexical quer este fosse mapeado por características fonológicas ou semânticas. Marshall e colegas \citep{Marshall2013} afirmam, assim, que as crianças com desenvolvimento atípico da linguagem poderão apresentar um acesso lexical menos eficiente do que as crianças com DT. 


\largerpage
A observação do conhecimento morfossintático e sintático tem sido realizada da análise à compreensão e à produção de orações relativas e de interrogativas-q e de fenómenos de concordância morfossintática, uma vez que é nestas estruturas e fenómenos que as crianças ouvintes com desenvolvimento atípico apresentam maiores dificuldades \citep{Martins2017}. Na observação de padrões atípicos no desenvolvimento da língua gestual em crianças surdas, o conhecimento morfossintático foi analisado através da recolha de discurso espontâneo ou semicontrolado \citep{Herman2014, QuintoPozos2013, QuintoPozos2011} ou da recolha de produções sintáticas por repetição  \citep{Marshall2015, Morgan2007}. Nestes estudos, em geral, foram verificadas dificuldades na produção de flexão verbal (em específico, na concordância marcada no espaço sintático)\footnote{O espaço sintático é o espaço que envolve o gestuante e no qual são identificados os referentes da estrutura sintática. A flexão verbal é marcada através das relações de concordância entre os referentes.} e na derivação morfológica (com dificuldades na distinção entre nome-verbo). Com o objetivo de observar padrões diferenciadores entre o desenvolvimento morfossintático e sintático típico e atípico em crianças surdas, \citet{Mason2010} avaliaram um conjunto de crianças surdas com desenvolvimento linguístico típico e outras com desenvolvimento linguístico atípico através de testes de linguagem validados para a BSL. Neste estudo identificaram que a compreensão e a produção de estruturas sintáticas complexas bem como estruturas com morfologia complexa (que envolvem concordância e classificadores)\footnote{Um classificador é um gesto que representa iconicamente um único referente (nominal) ou uma ação com o seu argumento interno (verbal). Os classificadores são frequentes nas línguas gestuais e são descritos como fenómenos linguísticos de natureza fonomorfossintática.} são estruturas frágeis e problemáticas aquando da aquisição de uma língua gestual. Ainda no contexto sintático, verificam-se dificuldades na ordem de palavras/gestos assim como nas estruturas com elementos de negação em contexto de repetição de frases. No contexto discursivo, verifica-se ainda que as estruturas sintáticas produzidas em narrativas foram mais curtas nas crianças surdas com desenvolvimento atípico do que nos seus pares com DT \citep{Herman2014}.

A investigação realizada a crianças surdas gestuantes mostra-nos, em geral, que a atipicidade do desenvolvimento da linguagem na modalidade gestual aparenta exibir características semelhantes às dificuldades linguísticas de crianças com desenvolvimento atípico de modalidade oral, tanto na componente de produção como na componente de compreensão da linguagem. 

O grande desafio para educadores, médicos, terapeutas da fala e investigadores é a seleção e a utilização de instrumentos e de critérios de avaliação adequados para avaliar o conhecimento linguístico, identificar dificuldades linguísticas ou diagnosticar um desenvolvimento atípico de uma língua gestual \citet{Henner2018}. Tradicionalmente, são utilizadas e adaptadas as mesmas escalas, testes e ferramentas que se aplicam às crianças ouvintes com desenvolvimento (a)típico de linguagem oral. Tal recurso pode não ser considerado totalmente correto por se tratar de um sistema linguístico de modalidade diferente (visuo-gestual e não auditiva-oral) e com especificidades linguísticas e gramaticais particulares.

No entanto, este tipo de avaliação ainda que com instrumentos eventualmente não adequados oferece possibilidades não menosprezáveis de observar e analisar dificuldades linguísticas independentemente da modalidade assim como compre\-endê-las também na especificidade da sua modalidade.

\subsection{Principais dificuldades identificadas no desenvolvimento linguístico atípico em crianças surdas gestuantes de Língua Gestual Portuguesa}

A Língua Gestual Portuguesa (doravante LGP) é uma língua de escolarização recente e reconhecida na constituição portuguesa, desde 1997, como língua na qual se pode desenrolar o ensino-aprendizagem das crianças surdas portuguesas. Neste momento e desde 2008, existe a possibilidade de o ensino dos surdos assentar num modelo bilingue, através das 17 escolas de referência espalhadas pelo país. Estima-se que existam cerca de 600 alunos surdos em idade escolar. 

A investigação na LGP é muito recente e tem evoluído de forma vagarosa, em parte pelas relações nem sempre fluídas entre a comunidade surda e a sociedade portuguesa ouvinte. Nesse sentido, a investigação sobre aquisição e desenvolvimento da LGP enquanto língua materna também tem vindo a demorar, ainda que projetos como o AQUI\_LGP (refª PTDC/CLE-LIN 111889/2010) tenham retratado o DT, longitudinal, da LGP em crianças surdas \citep{Carmo2013}. 

Em Portugal, para a LGP e até à data, não foi realizado nenhum estudo de caso ou um projeto mais extensivo que relate ou investigue o desenvolvimento atípico da linguagem em crianças surdas gestuantes, com LGP como língua materna, e não foi, ainda, recolhida uma amostra que nos permita, com segurança, afirmar quais são as principais áreas e estruturas linguísticas afetadas pelo desenvolvimento atípico de uma criança surda, com LGP como língua materna.

Encontra-se, em curso, um projeto europeu pioneiro (DOT\_DEAF, KA - UK01-KA202-061953) que visa despertar os profissionais de educação de surdos (professores, terapeutas e médicos) para a necessidade de se observar o desenvolvimento das crianças surdas com base em padrões já reconhecidos para o DT e identificar eventuais desvios de desenvolvimento linguístico. No âmbito deste projeto, no qual Portugal participa, foram identificados pelos professores de surdos, num inquérito preliminar que se encontra em expansão e melhoria \citep{Mineiroempreparacao}, as seguintes questões de desenvolvimento da linguagem que se enquadram no desenvolvimento atípico das línguas gestuais e, em particular, da LGP. 

No que respeita à fonologia da LGP, os professores consultados referem nos seus alunos a apresentação dos seguintes desvios linguísticos:

\begin{itemize}
    \item gaguez gestual identificada através de hesitação nos parâmetros movimento e configuração manual;
\item	erros de produção no parâmetro expressão facial;
\item	utilização atípica da mão dominante que é substituída de forma aparentemente aleatória pela mão não dominante. 
\end{itemize}
Também foi referido pelos inquiridos dificuldades de compreensão dos gestos produzidos por colegas e pelos professores, o que exibirá dificuldades na área semântico-lexical.

No que respeita ao espaço sintático, foram referidos erros de uso, nomeadamente falhas de informação de referência relativas ao tempo, ao lugar e aos referentes envolvidos na narrativa.

\section{Considerações gerais sobre a aquisição e desenvolvimento de uma língua oral em crianças surdas utilizadoras de implante coclear}
Uma surdez congénita ou que surja durante o primeiro ano de vida priva a criança de uma exposição à língua oral limitando a sua aquisição e desenvolvimento \citep{Hall2019}. Como já referido, a reabilitação cirúrgica da surdez sensorioneural, de grau severo a profundo, com recurso ao IC possibilita uma exposição ao \emph{input} linguístico por via auditiva conduzindo a melhorias na aquisição da linguagem oral (e.g. \citealp{Tavakoli2015}), pese embora se reconheça que o IC não permite ouvir com a mesma qualidade sonora que é experienciada pelas pessoas ouvintes. Contudo, é indubitável que o IC revolucionou o processo de aquisição e de desenvolvimento da linguagem oral de crianças surdas, que de outro modo, enfrentariam dificuldades na comunicação efetiva diária com o “mundo ouvinte”. 

Um dos aspetos mais problemáticos no estudo desta população infantil é a variabilidade no seu desempenho linguístico oral após a implantação coclear. Neste contexto, a literatura tem identificado variáveis extralinguísticas que parecem explicar parte desta variabilidade, nomeadamente: idade de diagnóstico da surdez; duração da surdez/tempo de privação ao \emph{input} linguístico oral; idade de implantação; idade auditiva/tempo de experiência auditiva robusta (com o IC); modo de comunicação preferencial;\footnote{O modo de comunicação preferencial tem sido dicotomizado em: (i) comunicação oral, construída numa dependência de audição e de linguagem oral para o máximo benefício do uso de qualquer dispositivo auditivo; e (ii) comunicação total, construída numa combinação flexível de modalidades, particularmente, oral-auditiva e visuo-gestual para maiores benefícios linguísticos.} envolvimento parental, entre outras. Adicionalmente, fatores intrínsecos como inteligência não verbal e funcionamento executivo têm sido reconhecidos como preditores dos resultados linguísticos de crianças surdas, após a colocação do IC \citep{Beer2014, Dawson2002}.

Importa, ainda, referir que a ativação do IC marca o início da idade auditiva e a consequente exposição à língua oral quando acompanhada por uso regular e consistente do dispositivo, pelo que o desempenho linguístico das crianças surdas com IC deverá ser analisado tendo como referência a idade auditiva, além da cronológica.

Nesta secção, abordaremos a aquisição e desenvolvimento fonológico e sintático em crianças surdas utilizadoras de IC (com um uso regular e consistente) falantes de uma língua oral, em específico, do PE. Em Portugal, os instrumentos de avaliação utilizados nesta população infantil são, até ao momento, divergentes entre os especialistas da área, pelo que não serão o foco desta secção. Espera-se que, em breve, possamos divulgar um protocolo de avaliação para estas crianças que reúna consenso entre os especialistas da área em Portugal.

\subsection{Aquisição e desenvolvimento fonológico da língua oral em crianças surdas utilizadoras de implante coclear}

Nas últimas décadas, a literatura tem vindo a demonstrar que o primeiro ano de vida é fundamental para o processo de aquisição e desenvolvimento da linguagem. Paralelamente, as normas clínicas para implantação coclear em idade pediátrica, um pouco por todo o mundo, passaram a abranger crianças mais pequenas. 

De um modo global, as tendências gerais de aquisição descritas para o desenvolvimento fonológico de crianças normo-ouvintes, em diferentes línguas, também se verificam para crianças surdas utilizadoras de IC, pese embora o índice de aquisição seja, usualmente, mais lento \citep{Serry1999, Lynce2019}. 

As fricativas e as líquidas são, inúmeras vezes, identificadas como as classes naturais mais problemáticas nas crianças ouvintes com um desenvolvimento típico nas diferentes línguas (e.g. \citealp{Bernhardt1998, Demuth2014, Johnson2010}) correspondendo àquilo que acontece em crianças surdas utilizadoras de IC durante a aquisição de diferentes línguas (e.g. \citealp{Chin2000, Spencer2013, WarnerCzyz2008}).

Quanto ao ponto de articulação, a produção de consoantes alveolares e palatais tende a estar associada a dificuldades, em crianças surdas utilizadoras de IC, pelas concentrações de energia em níveis, relativamente, elevados de frequência e baixos de intensidade \citep{Blamey2001}.

No que respeita ao contraste do vozeamento, a literatura internacional relata resultados contraditórios. \citet{Flipsen2008} reportaram dificuldades no processamento do contraste de vozeamento nas fricativas. Porém, \citet{Grandon2017}, por meio de uma análise acústica, verificaram que crianças surdas, utilizadoras de IC, apresentam ao longo do tempo menos dificuldades nas pistas de duração (vozeamento) do que nas caraterísticas de frequência (segundo formante das vogais e momentos espetrais das fricativas).

Relativamente aos processos fonológicos (i.e., estratégias de reparação) envolvendo a estrutura silábica, os padrões de omissão de consoante final e de redução do grupo consonântico parecem uma constante no comportamento de crianças surdas implantadas a adquirir diferentes línguas, após avaliada a produção de palavras isoladas por meio de testes normativos \citep{Doble2006, Flipsen2008, Grogan1995}. Contudo, \citet{Faes2017} reportaram que crianças surdas com IC, a adquirir o holandês, conseguem produzir grupos consonânticos aos cinco anos de idade, ao analisarem o seu discurso espontâneo.

Para o PE, o primeiro estudo totalmente dedicado ao desenvolvimento fonológico de crianças surdas utilizadoras de IC relata que estas necessitam de mais tempo para a produção de determinados segmentos-alvo e identifica semelhanças, mas também diferenças na ordem de aquisição dos segmentos comparativamente aos seus pares normo-ouvintes \citep{Lynce2019}. 

Particularizando, os resultados destas crianças derivados da aplicação do Teste Fonético-Fonológico ALPE (TFF-ALPE) de \citet{Mendes2013}, parecem ser, em geral, parcialmente justificados pela produção visualmente mais saliente de determinadas consoantes (i.e. com pontos de articulação mais visíveis) colocando-as em vantagem para uma produção precoce relativamente a outras. Porém, à medida que a experiência auditiva robusta (via IC) se prolonga no tempo, os padrões de desenvolvimento destas crianças tornam-se mais próximos dos padrões dos seus pares normo-ouvintes (Tabela \ref{tab:tabela1}).


\begin{Tabela}
    \begin{tabularx}{\textwidth}{lQQ}
\lsptoprule
\textbf{Idade}& 	\textbf{Crianças ouvintes}&	\textbf{Crianças surdas}   \textbf{utilizadoras de IC}\\
\midrule

\textbf{[2;10 - 4;4[}   &	/p, t, k, b, d, g, f, s, ʃ, v, m, n, ɲ, ʀ , l, ʎ, ʃ em coda/  &	/p, k, b, d, f, ʃ, m, ɲ, ʀ l, ʃ em coda/\\
\tablevspace
 \textbf{[5;04 -7;03[}  &	/p, t, k, b, d, g, f, s, ʃ, v, m, n, ɲ, ʀ , l, ʎ, ʃ em coda, z, ʒ, ɾ, ɾ em coda, ɫ em coda, pl, kl, fl, fɾ, vɾ, bɾ, kɾ, pɾ, tɾ, dɾ, gɾ/  &	/p, t, k, b, g, f, ʃ, v, m, ɲ, ʀ, l , ʃ em coda, z, ʒ/ \\
\lspbottomrule
    \end{tabularx}
    \caption{{Inventário consonântico das crianças surdas utilizadoras de IC do PE tendo em conta a sua idade auditiva, e respetiva comparação com os dados normativos do TFF-ALPE (tendo como referência a idade cronológica das crianças     normo-ouvintes)}}
    \label{tab:tabela1}


\end{Tabela}

No que respeita ao modo de articulação, crianças surdas com IC a adquirir o PE apresentaram uma emergência precoce de oclusivas e nasais contrastando com a aquisição de fricativas e líquidas \citep{Lynce2019}. 

Quanto ao ponto de articulação, crianças surdas com IC falantes do PE revelaram dificuldades na produção de consoantes coronais (/d, n, s, ɾ, ʎ/). Neste contexto, verifica-se uma preferência destas crianças pelo uso de consoantes labiais e dorsais em detrimento das coronais. Especula-se, a este nível, que os movimentos indiferenciados da língua na produção de consoantes coronais e uma maior consciencialização na produção das consoantes labiais e dorsais comparativamente às coronais, pelo processamento visual e motor envolvidos, promova a promoção das primeiras (pelos movimentos labiais visivelmente capturados) e das segundas (pelo esforço muscular posterior envolvido) e dificulte a produção das terceiras \citep{Lynce2019}.

Relativamente ao contraste de vozeamento, no PE, não se identificaram dificuldades neste âmbito, já que 4 dos 6 contrastes de vozeamento possíveis foram produzidos. Ficaram, apenas, em falta os contrastes (/t, d/ e /s, z), mas a sua ausência dever-se-á, muito provavelmente, aos problemas previamente relatados com os segmentos coronais.

A relevância da interface entre os segmentos e a estrutura prosódica no desenvolvimento fonológico tem sido reportada, quer em crianças com um desenvolvimento típico, quer em crianças com um desenvolvimento atípico (e.g. \citealp{Demuth2014, Fikkert1994, Freitas1997, Marshall2009, Marshall2015}). Neste contexto, a posição que uma determinada consoante ocupa na sílaba terá um impacto significativo na sua aquisição (e.g. para o PE, \citealp{Amorim2014, Freitas1997, Mendes2013, Ramalho2017}). Com foco na aquisição da estrutura da sílaba do PE em crianças surdas com IC, \citet{Lynce2019} observaram que a produção de segmentos em coda foi, apenas, atestada para a fricativa /ʃ/, encontrando-se as codas líquidas ausentes. Estes resultados são distintos dos observados para os seus pares normo-ouvintes que adquirem a coda /ɾ/ entre [4.6 - 4.12[ anos de idade e a coda /ɫ/ entre [5.0 - 5.6[ \citep{Mendes2013}. Na aquisição de ataques ramificados no PE, o esperado é que os grupos consonânticos com Consoante+/l/ (C/l/) sejam adquiridos entre os [4.0 - 4.6[ e os grupos consonânticos com Consoante+/ɾ/ (C/ɾ/) entre os [5.0 - 5.6[ \citep{Mendes2013}.  Deste modo, a aquisição do segmento /l/ apenas em posição de ataque simples e não em coda ou em ataque ramificado por crianças surdas com IC, evidencia que as dificuldades com o /l/ não serão de foro segmental, mas prosódico, particularmente silábico, o que corresponde ao comportamento observado para crianças com um desenvolvimento típico, a adquirir o PE. 

Quanto aos processos fonológicos usados pelas crianças surdas utilizadoras de IC para lidar com estruturas-alvo problemáticas (Tabela \ref{tab:tabela2}), estes não corresponderam integralmente aos observados para crianças com um desenvolvimento típico, a adquirir o PE \citep{Mendes2013}.


\begin{Tabela}
\small
\begin{tabularx}{\textwidth}{p{5cm}QQ}
\lsptoprule


\textbf{Crianças normo-ouvintes: Faixa etária provável de supressão dos processos fonológicos listados pelo TFF-ALPE}            & \textbf{Crianças surdas com implante coclear: Processos fonológicos produzidos entre [2;10 - 4;4[}          & \textbf{Crianças surdas com implante coclear: Processos fonológicos produzidos entre [5;4--7;03[}\\

\midrule

{Omissão de consoante final \newline- [6;06-6;12[} &
Omissão de consoante final &
Omissão de consoante final\\


{Omissão de sílaba átona pré-tónica - >[6;06-6;12[}&
--	&
--\\


{Redução de grupo consonântico \newline-[6;06-6;12[} &	Redução de grupo consonântico&	Redução de grupo {consonântico}\\


{Semivocalização de líquida \newline- [6;06-6;12[}&
--	&
--\\


{Oclusão \newline- [3;0-3;06[}&
--	&
--\\


{Anteriorização \newline- [3;0-3;06[}&
--	&
--\\


{Posteriorização \newline- [3;0-3;06[}&
--	&
--\\


{Palatalização \newline- [4;0-4;06[}&
--	&
--\\


{Despalatalização \newline- [4;0-4;06[}&
--	&
--\\


{Desvozeamento \newline- [5;0-5;06[}&
--	&
--\\

\cline{2-3}
& \multicolumn{2}{|p{5.5cm}@{}|}{\strut\textbf{Processos fonológicos adicionais\newline (não listados pelo TFF-ALPE)}}\\
& \multicolumn{2}{|p{5.5cm}@{}|}{{Omissão de ataques simples\newline ([‘bɔlɐ] > [´bɔɐ ])}}\\
& \multicolumn{2}{|p{5.5cm}@{}|}{{Preenchimento de ataque vazio\newline ([aɫtu]> [´laɫtu])}}\\
& \multicolumn{2}{|p{5.5cm}@{}|}{{Fricatização de oclusivas\newline ([ˈplɐ̃tɐ]>[ˈflɐ̃tɐ])}}\\
& \multicolumn{2}{|p{5.5cm}@{}|}{{Vozeamento de oclusivas e fricativas\newline ([suˈpɾaɾ]>[suˈbaɾ]/[ˈɡaɾfu]>[ˈɡaɾvɨ])])}}\\
\cline{2-3}
\bottomrule
\end{tabularx}
\caption{{Produção de processos fonológicos de crianças surdas utilizadoras de IC do PE tendo em conta a sua idade auditiva, e respetiva comparação com os dados normativos do TFF-ALPE (tendo como referência a idade cronológica das crianças normo-ouvintes)}}
\label{tab:tabela2}
\end{Tabela}

Nos processos fonológicos envolvendo a estrutura silábica, os padrões de omissão de consoante final e de redução do grupo consonântico, estiveram presentes nas crianças surdas, mesmo naquelas com mais experiência auditiva via IC \citep{Lynce2019}. Porém, crianças  com um desenvolvimento típico, a adquirir o PE, parecem eliminar este tipo de estratégias de reparação até aos 6;12 \citep{Mendes2013}. Também, o uso de ataques vazios para substituir ataques-alvo problemáticos foi atestado com regularidade pelas crianças surdas com IC, falantes do PE \citep{Lynce2019}, o que representa um marcador de imaturidade do sistema fonológico aliado à ausência de codas líquidas e de grupos consonânticos \citep{Freitas1997}. Além disso, esta população também realiza o preenchimento de ataques vazios. Ainda neste estudo, os processos a nível do segmento identificados pelo TFF-ALPE parecem estar ultrapassados, mesmo nas crianças mais novas da amostra, com dois a quatro anos de idade auditiva. Este comportamento não coincide com o observado para a população infantil normo-ouvinte, já que, por exemplo, o processo segmental de desvozeamento, surge como omisso, apenas entre os [5;0-5;06[. Porém, observaram-se processos a nível do segmento que não foram listados pelo TFF-ALPE (ver Tabela \ref{tab:tabela2}).

Um outro estudo realizado junto de crianças surdas com IC, falantes do PE, onde se aplicou a Grelha de Observação Nível Escolar (GOL-E)  de \citet{SuaKay2003}, analisou de entre os vários domínios linguísticos, o fonológico, por meio da discriminação de palavras e de pseudopalavras, da identificação de rimas e da segmentação silábica \citep{Ramos2015}. Os resultados mostraram que as crianças surdas com IC com uma idade auditiva média de 6;5 exibiram valores semelhantes aos normativos nas tarefas de discriminação auditiva de palavras e na identificação de rimas. Contudo, na discriminação auditiva de pseudopalavras as crianças surdas com IC ficaram aquém do esperado para a idade auditiva, tendo em conta os dados normativos disponíveis pelo teste. Estes resultados díspares na discriminação de palavras e pseudopalvras poderão dever-se ao facto destas crianças utilizarem como estratégia para compensar o défice no processamento fonológico as representações lexicais. Já na segmentação silábica, as crianças surdas utilizadoras de IC apresentaram resultados estatisticamente superiores quando comparados com os dados normativos de crianças com 6;5 de idade com um desenvolvimento típico.  A segmentação silábica tende a ser um dos aspetos mais trabalhados em contexto terapêutico o que poderá ajudar a explicar, pelo menos, em parte os valores alcançados pelos participantes do presente estudo.

\subsection{Aquisição e desenvolvimento sintático da língua oral em crianças surdas utilizadoras de implante coclear}

Na aquisição e desenvolvimento sintático, a literatura, em geral, indica que as crianças surdas com IC apresentam dificuldades na compreensão e na produção de estruturas sintáticas, tais como interrogativas-q, estruturas relativas e estruturas passivas. A total privação auditiva durante o(s) primeiro(s) ano(s) de vida, ou seja, durante o período crítico para a aquisição da linguagem, tem sido indicada como o fator responsável pelas dificuldades verificadas na aquisição destas estruturas (e.g. \citealp{Friedmann2006, Friedmann2011a, Ruigendijk2017}), que têm em comum o envolvimento de um fenómeno na sua derivação, denominado movimento sintático.\footnote{No âmbito do Modelo Teoria de Princípios e Parâmetros, estruturas sintáticas que envolvem movimento sintático são estruturas resultantes do movimento de constituintes de uma posição de base para uma posição não argumental ou para uma posição argumental, mantendo-se sempre, na posição de base, a relação temática do constituinte com o verbo. Para uma interpretação adequada da estrutura, terá de ser estabelecida uma relação de dependência entre o elemento movido e a posição de base. Por exemplo, na derivação de estruturas interrogativas-Q de objeto direto, ocorre o movimento do elemento-q \emph{quem} da posição base de objeto direto para a posição não argumental de especificador da categoria funcional do sintagma: 
\begin{enumerate}
\item  	A mãe está a abraça [quem]? (interrogativa-q \emph{in situ})
\item 	Quem\textsubscript{i} é que a mãe está a abraçar \_\_\_\textsubscript{ti}? (interrogativa-q com movimento-q)
\end{enumerate}} Estas dificuldades têm sido justificadas através da incompreensão de estruturas com movimento sintático ou através do uso de estruturas que não envolvem esse tipo de movimento como estruturas alternativas.

No contexto de aquisição e desenvolvimento de estruturas interrogativas-q e de estruturas relativas (e.g. \citealp{Friedmann2006, Friedmann2011a, Ruigendijk2017, Volpato2014}), as crianças surdas com IC e com idades cronológicas entre os 8 e os 13 anos apresentam desempenhos desiguais entre estruturas de sujeito (ex. Quem é que pisa o macaco? / O menino que pisa o macaco.) e de estruturas de objeto direto (OD) (ex. \emph{Quem é que o macaco pisa?} / \emph{O menino que o macaco pisa.}) e de objeto preposicionado (OP) (ex. \emph{De quem é que o menino foge?} / \emph{O menino de que o cão foge.}),\footnote{Assume-se que estas dificuldades poderão ser causadas pela intervenção do constituinte nominal sujeito durante o movimento do elemento-q objeto para a periferia esquerda da estrutura sintática \citep{Friedmann2009}.} tal como observado nas crianças ouvintes com DT. Contudo, as crianças surdas com IC apresentam, até faixas etárias mais tardias, um desempenho inferior em estruturas que envolvem o movimento do elemento-q com função sintática de OD e de OP, revelando, em contexto de produção, um maior uso de estratégias desviantes em substituição das estruturas-alvo.

As variáveis extralinguísticas identificadas como determinantes na compreensão e na produção de estruturas interrogativas-q e nas estruturas relativas em crianças surdas com IC são: a idade de ativação do IC, o diagnóstico precoce da surdez e a intervenção precoce oral \citep{Friedmann2006, Szterman2015}. Sugerindo-se, na literatura, que o primeiro ano de vida corresponderá ao período crítico para a aquisição do movimento sintático \citep{Friedmann2015}. Recentemente, identificou-se, ainda, que a privação total de \emph{input} linguístico no(s)s primeiro(s) ano(s) de vida poderá limitar a memória de trabalho, associando-se esta limitação às dificuldades na compreensão de estruturas relativas de objeto \citep{Volpato2020}.

Na aquisição e desenvolvimento do PE em crianças surdas implantadas, obser\-vam-se as mesmas dificuldades na compreensão e na produção de estruturas interrogativas-q e de estruturas relativas, tanto em contexto de avaliação da linguagem como em contexto de tarefas de compreensão e de produção construídas para o propósito.

Em contexto de avaliação da linguagem, através do TALC \citep{SuaKay2011}, no âmbito do projeto CLIC (Refª EXPL/MHC-LIN 0449/2013) e de um estudo longitudinal posterior de \citep{Lynce2019a}, as crianças surdas utilizadoras de IC em idade auditiva pré-escolar apresentaram dificuldades na compreensão de estruturas relativas de OD e de OP. Observou-se que estas dificuldades foram-se acentuando à medida que a idade auditiva foi avançando, revelando que o desenvolvimento sintático da população infantil implantada não acompanha o desenvolvimento dos seus pares ouvintes.

Em contexto de tarefas de compreensão de estruturas interrogativas-q e de estruturas relativas em crianças surdas com IC com idades cronológicas entre os 4;00 e os 16;11 anos e com idades auditivas entre os 2;00 e os 14;11 anos, os dados espelham resultados distintos conforme o tipo de idade em análise comparativamente à população ouvinte \citep{Moita2018, Moita2019}. Quando a análise aos dados de compreensão foi realizada com base na idade cronológica da população infantil surda implantada, esta apresentou um desempenho significativamente inferior na compreensão de todos os tipos de estruturas-alvo em todas as faixas etárias, em comparação com os seus pares ouvintes. Por sua vez, quando a análise aos dados de compreensão foi realizada com base na idade auditiva da mesma população, observou-se que, em geral, as crianças surdas com colocação do IC até aos 2;11 anos de idade apresentam, a partir dos 8/ 9 de idade auditiva, uma compreensão de estruturas interrogativas-Q e de estruturas relativas semelhante à compreensão das crianças ouvintes com idades cronológicas paralelas.


\begin{Tabela}
\begin{tabularx}{.8\textwidth}{Q}
\lsptoprule
\textbf{Compreensão de estruturas interrogativas-q e de estruturas relativas} \\
\midrule

Dificuldades com as estruturas com movimento do elemento-q
com função sintática de OD e de OP até idades cronológicas
mais tardias do que as crianças ouvintes.\\
\tablevspace


Desempenho semelhante ao das crianças ouvintes com DT a
partir dos 6/ 7 anos de idade auditiva, quando a colocação do IC
foi realizada até aos 2;11 anos de idade cronológica.\\
\tablevspace


Melhor desempenho com estruturas de OD e de OP em crianças
surdas com a colocação do IC realizada até aos 1;11 anos de
idade, em comparação com o desempenho de crianças com a
colocação do IC realizada a partir dos 3;00 anos de idade.\\
\lspbottomrule
\end{tabularx}

\caption{{Principais resultados identificados na compreensão de estruturas interrogativas-q e de estruturas relativas em crianças surdas utilizadoras de IC falantes do PE em comparação com os seus pares ouvintes.}}
\label{tab:Tabela3}
\end{Tabela}

Na análise à produção de estruturas interrogativas-q em crianças surdas utilizadoras de IC falantes do PE, os resultados apresentam-se distintos com base na metodologia utilizada. Se por um lado, no discurso espontâneo da população infantil surda com IC em idade auditiva até aos 5;11 pré-escolares predominantemente produziu interrogativas-q de OD e de OP com movimento sintático em detrimento de interrogativas \emph{in situ} (sem movimento) \citep{Costa2016a}. Por outro lado, em tarefas de elicitação \citep{Moita2018}, as crianças surdas utilizadoras de IC com idades auditivas entre os 2;00 e os 14;11 anos apresentaram preferência na produção das mesmas estruturas com o elemento interrogativo \emph{in situ}, tal como é observado na mesma população falante de outras línguas orais. As primeiras produções de estruturas interrogativas-q com movimento do elemento-q de OD são registadas a partir da faixa etária dos 4 anos de idade auditiva e do elemento-q de OP a partir da faixa etária dos 6 anos de idade auditiva. Neste estudo, observa-se, também, uma maior variabilidade no tipo de estruturas alternativas à estrutura-alvo, em comparação com as crianças ouvintes. As diferenças de resultados entre os dois tipos de contexto de produção poderão estar associadas a uma influência do contexto de elicitação.

No contexto de aquisição e desenvolvimento de estruturas passivas, o único estudo sobre a aquisição e desenvolvimento destas estruturas na população infantil surda com IC falantes de outras línguas orais que não o PE foi de \citet{Ruigendijk2017} com base em tarefas de compreensão e de produção construídas para o propósito. Embora, neste estudo, se tenham observado dificuldades na compreensão de estruturas passivas com verbos transitivos semanticamente reversíveis,\footnote{Verbos transitivos semanticamente reversíveis são verbos cujos argumentos podem desempenhar a ação com papéis semânticos reversíveis.} não foram identificadas diferenças significativas no desempenho geral entre a população infantil surda utilizadora de IC com idades cronológicas entre os 9;00 e os 12;11 anos e os seus pares ouvintes.

Na análise à compreensão de estruturas passivas com verbos semanticamente reversíveis, com base em tarefas construídas para o propósito, de crianças surdas utilizadoras de IC falantes do PE com idades cronológicas entre os 4;00 e os 16;11 anos e com idades auditivas entre os 2;00 e os 14;11 anos \citep{Moita2019}, esta população infantil apresentou dificuldades significativas na compreensão de estruturas passivas longas e curtas agentivas (ex. \emph{A avó é abraçada pela menina.} / \emph{A avó é abraçada.}) em todas as faixas etárias auditivas e cronológicas, em comparação com os seus pares ouvintes (ver Tabela \ref{tab:tabela4}). Na aquisição de estruturas passivas longas e curtas não agentivas (ex. \emph{A avó é amada pelo pai.}; \emph{A avó é amada.}), as crianças surdas utilizadoras de IC com idades auditivas até aos 6/ 7 anos apresentaram um desempenho semelhante ao dos seus pares ouvintes. As diferenças significativas a partir dos 6/ 7 anos poderão estar associadas ao facto das crianças ouvintes poderem atingir um domínio linguístico sobre as estruturas passivas não alcançável nas crianças surdas com IC.


\begin{Tabela}
\begin{tabularx}{.8\textwidth}{Q}
  \lsptoprule
  \textbf{Compreensão de estruturas passivas}   \\
\midrule
Desempenho inferior significativo em estruturas passivas longas
e curtas agentivas com verbos transitivos semanticamente reversíveis até aos 14 anos de
idade auditiva e 16 anos de idade cronológica.\\
\tablevspace

Desempenho inferior significativo em estruturas passivas longas
e curtas não agentivas com verbos transitivos semanticamente reversíveis a partir dos 6/7
anos de idade auditiva. \\
\tablevspace


Melhor desempenho em estruturas passivas longas e curtas
agentivas em crianças surdas com colocação do IC realizada até
aos 2;11.\\
\lspbottomrule
\end{tabularx}

\caption{{Principais resultados identificados na compreensão de estruturas passivas em crianças surdas utilizadoras de IC falantes do PE em comparação com os seus pares ouvintes}}
\label{tab:tabela4}
\end{Tabela}

Ainda na análise ao discurso espontâneo \citep{Costa2016a}, observaram-se as respostas afirmativas a perguntas sim-não (ex. \emph{Foste à praia?}) que podem ser: (i) respostas verbais com elipse de VP\footnote{As estruturas elipses de VP são caracterizadas como estruturas em que os argumentos e os adjuntos do sintagma nominal podem ser elididos, mas o verbo é produzido por ter sido movido para a projeção funcional, ou seja, uma vez que são estruturas sem ambiguidade do movimento do verbo.} (ex. \emph{Fui.}); ou (ii) respostas sim (ex. \emph{Sim.}). Nesta análise, observou-se que as crianças surdas utilizadoras de IC com idades auditivas entre os 2;00 aos 5;11 anos apresentam prevalência por respostas sim, ao contrário do que é observado nas crianças ouvintes, em que respostas verbais com elipse de VP são registadas logo nas primeiras produções da criança \citep{Santos2009a}.

\section{Conclusão}

A surdez congénita ou adquirida no primeiro ano de vida resulta na privação à exposição da língua oral no período crítico em que esta deveria ser adquirida. A exposição tardia a uma língua (gestual ou oral) manifestar-se-á em dificuldades na aquisição e desenvolvimento de determinados elementos e estruturas linguísticas, além das possíveis dificuldades derivadas de perturbações da linguagem frequentes na infância.

No que diz respeito à aquisição e desenvolvimento da LGP, têm sido reportados alguns problemas no que respeita às várias áreas da linguagem, nomeadamente na componente fonológica (e.g. gaguez gestual, erros de produção em diversos parâmetros fonológicos), na componente morfossintática e sintática (e.g. erros de uso e falhas de informação relativamente ao tempo, ao lugar e aos referentes envolvidos na narrativa gestual), e na componente semântica (e.g. dificuldade das crianças na compreensão de gestos produzidos pelos pares ou por adultos).  

No contexto da aquisição e desenvolvimento fonológico do PE, os problemas identificados em crianças surdas utilizadoras de IC podem ser atribuídos a dificuldades na construção das representações fonológicas.

No contexto de aquisição e desenvolvimento sintático oral, a população infantil surda portuguesa com IC tem apresentado dificuldades na compreensão de estruturas interrogativas-q e de estruturas relativas, especificamente quando o elemento-q tem função sintática de OD e de OP. Estas dificuldades parecem estar presentes até idades mais tardias das identificadas em crianças ouvintes com DT. Também as estruturas passivas com verbos semanticamente reversíveis se apresentam problemáticas para esta população infantil. A variável idade de colocação de IC é determinante para que estas dificuldades sintáticas se atenuem à medida que o tempo de exposição à língua aumenta. 

Em Portugal, os instrumentos de avaliação da linguagem (gestual ou oral) utilizados na população infantil surda com e sem IC são, por vezes, inadequados para avaliar o seu conhecimento linguístico. Acreditamos que um protocolo de avaliação, a nível nacional, para esta população permitirá uma avaliação minuciosa do seu desenvolvimento linguístico independentemente da(s) modalidade(s) linguística(s) utilizadas pela criança (gestual e/ ou oral).


{\sloppy\printbibliography[heading=subbibliography,notkeyword=this]}
\end{document}
