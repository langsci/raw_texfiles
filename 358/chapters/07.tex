\documentclass[output=paper,colorlinks,citecolor=brown,booklanguage=portuguese]{langscibook} 
\ChapterDOI{10.5281/zenodo.7233225}
\title{Para além do segmento: Teoria fonológica e intervenção clínica}
\author{Marisa Lousada\affiliation{CINTESIS.UA, Escola Superior de Saúde, Universidade de Aveiro} and  Ana Margarida Ramalho\affiliation{Universidade de Lisboa, Faculdade de Letras, Centro de Linguística, Hospital do Espírito Santo - Évora (EPE)} and  Tânia Reis\affiliation{Relicário de Sons}}



\abstract{Na \sectref{sec:cap7sec1} deste capítulo é explicada a importância da evidência científica para a prática clínica com crianças com alterações fonológicas e são descritas as principais categorias de intervenção em função do foco da intervenção com estas crianças. Na  \sectref{sec:cap7sec2}, descrevem-se as principais medidas de resultados utilizadas para medir a eficácia e eficiência da intervenção com crianças com alterações fonológicas, dando alguns exemplos de ferramentas/ instrumentos que podem ser utilizadas/utilizados com crianças falantes do português europeu (PE). Na  \sectref{sec:cap7sec3} apresentam-se os critérios a ter em conta para o planeamento da intervenção terapêutica. Estes critérios envolvem os critérios fonológicos para seleção dos alvos, a seleção das tarefas psicolinguísticas, terminando com exemplos para elaboração de objetivos terapêuticos. Na última secção citam-se alguns estudos de intervenção com crianças falantes do PE com perturbação fonológica.


\textbf{Palavras-chave}: fonologia; medidas de resultados; intervenção; evidência científica; crianças}

\IfFileExists{../localcommands.tex}{
   \addbibresource{../localbibliography.bib}
   % add all extra packages you need to load to this file

\usepackage{tabularx,multicol}
\usepackage{url}
\urlstyle{same}

\usepackage{listings}
\lstset{basicstyle=\ttfamily,tabsize=2,breaklines=true}

\usepackage{langsci-basic}
\usepackage{langsci-optional}
\usepackage{langsci-lgr}
\usepackage{langsci-osl}
% \usepackage{./langsci/styles/langsci-lgr}
% \usepackage{./langsci/styles/langsci-osl}
% \usepackage{langsci-gb4e}

\usepackage{tikz}
\usetikzlibrary{patterns,calc}
\pgfdeclarepatternformonly{south east lines}{\pgfqpoint{-0pt}{-0pt}}{\pgfqpoint{3pt}{3pt}}{\pgfqpoint{3pt}{3pt}}{
    \pgfsetlinewidth{0.6pt}
    \pgfpathmoveto{\pgfqpoint{0pt}{3pt}}
    \pgfpathlineto{\pgfqpoint{3pt}{0pt}}
    \pgfpathmoveto{\pgfqpoint{.2pt}{-.2pt}}
    \pgfpathlineto{\pgfqpoint{-.2pt}{.2pt}}
    \pgfpathmoveto{\pgfqpoint{3.2pt}{2.8pt}}
    \pgfpathlineto{\pgfqpoint{2.8pt}{3.2pt}}
    \pgfusepath{stroke}}
    
\usepackage{stmaryrd}
\usepackage{wasysym}
\usepackage{multirow}
\usepackage{caption}
\usepackage{subcaption}
\usepackage{mathrsfs}
\usepackage{qtree}

\usepackage{linguex}


   %pminos do not split footnotes
% \interfootnotelinepenalty=10000 %Footnote in Laporte chapters has to be split SN


%\DeclareIndexNameFormat{default}{%
%\nameparts{#1}%
%\usebibmacro{index:name}%
%{\index[names]}%
%{\namepartfamily}%
%{\namepartgiveni}%
% {}% L1
% {}% L2
%{\namepartprefix}% generates spurious space L3
%{\namepartsuffix}% generates spurious space L4
%}

%  {\DeclareIndexNameFormat{default}{%
%     \usebibmacro{index:name}{\index[names]}{#1}{#3}{#5}{#7}}}

%\DeclareIndexNameFormat{default}{%
%  \usebibmacro{index:name}{\sindex[nom]}{#1}{#3}{#5}{#7}}

%\DeclareIndexNameFormat{default}{%
%  \usebibmacro{index:name}{\sindex[person]}{#1}{#3}{#5}{#7}}
%\DeclareIndexNameFormat{default}{%
%\nameparts{#1} \usebibmacro{index:name}{\sindex[person]]}{\namepartfamily}{‌​\namepartgiven}{\nam‌​epartprefix}{\namepa‌​rtsuffix}}

%\newcommand{\smiley}{:)}

%\renewbibmacro*{index:name}[5]{%
%\usebibmacro{index:entry}{#1}%
%{\iffieldundef{usera}{}{\thefield{usera}\actualoperator}\mkbibindexname{#2}{#3}{#4}{#5}}}

% \newcommand{\noop}[1]{}

%remove for final
%\overfullrule=1mm

\newcommand{\tobi}[2]}}
\renewcommand{\S}[1]{\tobi{#1}{\textsc{*}}}

% this volume references
% puts: [this volume]
% already defined: \citetv
%\newcommand{\citepv}[1]{(\citeauthor{#1} \citeyear*{#1} [this volume])}
\newcommand{\citealtv}[1]{\citeauthor{#1} \citeyear*{#1} [this volume]}

%parentheses around example number
\newcommand{\pref}[1]{(\ref{#1})}

% in-text examples

\newcommand{\lnex}[1]{\textit{#1}} %target lang word
\newcommand{\lnlit}[1]{(lit.: `#1')} %literal reading
\newcommand{\lnlat}[1]{(#1)} % latinization
\newcommand{\lntrans}[1]{`#1'} %translation
\newcommand{\lnexl}[2]%
{\lnex{#1}{} \lnlat{#2}} % ex with latinization
\newcommand{\lnexlat}[3]{\lnex{#1}{} \lnlat{#2}{} \lntrans{#3}} % ex with latinization and tranl.

%ch01
\newcommand{\co}[1]{\mbox{\textbf{#1}}}

%ch09

\newcommand{\cyrbulg}[1]{\begin{otherlanguage*}{bulgarian}#1\end{otherlanguage*}}


%ch10
\newcommand{\nlp}{{\small NLP}}
\newcommand{\mwe}{{\small MWE}}
\newcommand{\rae}{{\small RAE}}
\newcommand{\lvc}{{\small LVC}}
\newcommand{\pos}{{\small P}o{\small S}}
%\newcommand{\todo}[1]{ \textcolor{red}{#1} }

%\renewcommand{\labelenumi}{\theenumi}
%\ainamefmt{{vv}{ll}{, ff}{, jj}} % fullname

\newcommand{\biberror}[1]{{\color{red}#1}}

\newcommand{\osenovaitem}{--~}
   %% hyphenation points for line breaks
%% Normally, automatic hyphenation in LaTeX is very good
%% If a word is mis-hyphenated, add it to this file
%%
%% add information to TeX file before \begin{document} with:
%% %% hyphenation points for line breaks
%% Normally, automatic hyphenation in LaTeX is very good
%% If a word is mis-hyphenated, add it to this file
%%
%% add information to TeX file before \begin{document} with:
%% %% hyphenation points for line breaks
%% Normally, automatic hyphenation in LaTeX is very good
%% If a word is mis-hyphenated, add it to this file
%%
%% add information to TeX file before \begin{document} with:
%% \include{localhyphenation}
\hyphenation{
    Beck-man
    Ngu-yen
    back-chan-nel
    back-chan-nels
    mo-not-o-nous
    ste-reo-typ-i-cal
}

\hyphenation{
    Beck-man
    Ngu-yen
    back-chan-nel
    back-chan-nels
    mo-not-o-nous
    ste-reo-typ-i-cal
}

\hyphenation{
    Beck-man
    Ngu-yen
    back-chan-nel
    back-chan-nels
    mo-not-o-nous
    ste-reo-typ-i-cal
}

   \boolfalse{bookcompile}
   \togglepaper[07]%%chapternumber
}{}

\begin{document}
\maketitle

\section{A importância da evidência para a prática}\label{sec:cap7sec1}
As crianças com alterações\footnote{O termo “alterações” será utilizado ao longo do capítulo para se referir a um comprometimento de natureza fonológica associado a diferentes diagnósticos (e.g. Perturbações dos Sons da Fala, Perturbação do Desenvolvimento da Linguagem, Perturbação da Aprendizagem Específica).} fonológicas apresentam problemas na representação mental dos sons, com comprometimento ao nível das unidades segmentais e das unidades prosódicas (por exemplo, constituintes silábicos, acento de palavra, extensão de palavra e posição na palavra).\footnote{A norma adotada na escrita deste capítulo foi a do português europeu.}
% \footnote{Na sequência de restrições decorrentes do template usado para publicação na editora Language Science Press, os capítulos que integram o volume surgem com a referência [this volume], em inglês.}


Assim, estas crianças podem apresentar alterações na produção (e.g., \citealp{Lousada2013, Ramalho2017a}), percetivo-auditivas (e.g., \citealp{Hearnshaw2018}) e cogniti\-vo-linguísticas, associadas a alterações de processamento fonológico e de consciência fonológica  (e.g., \citealp{Gillon2005, Burgoyne2019, Torgensen1994, Castro2007}) e, por vezes, na aprendizagem da leitura e da escrita (e.g., \citealp{Burgoyne2019}). Apresentam, como consequência, a inteligibilidade reduzida \citep{Lousada2014}, com um impacto negativo na participação em diversos contextos (casa, escola e comunidade) \citep{Ertmer2010} e com diferentes parceiros comunicativos \citep{Lousada2014}. 

A avaliação destas crianças deve ser pormenorizada e incluir provas de avaliação do processamento fonológico: da produção, da perceção, de acesso lexical, de memória fonológica e de consciência fonológica (através da nomeação, da repetição de palavras, de pseudopalavras e/ou frases, da fala encadeada, da discriminação, da perceção, da memória de dígitos e da consciência fonológica), de forma a facilitar o planeamento da intervenção. As provas de avaliação devem estar validadas para a língua alvo e considerar as diferentes variáveis fonológicas, anteriormente referidas.  

A intervenção (de que falaremos ao longo deste capítulo) deverá centrar-se nas vias fonológicas identificadas como alteradas, independentemente do diagnóstico específico: de perturbação dos sons da fala de base fonológica, de perturbação do desenvolvimento da linguagem com alterações na fonologia ou até de perturbação da aprendizagem específica / dislexia.

Nos últimos anos têm sido desenvolvidas diferentes intervenções para crianças com alterações fonológicas, que variam nos métodos utilizados \citep{Baker2011, Law2012}, tendo sido publicados diversos estudos de eficácia da interven\-ção em crianças com alterações fonológicas. Os terapeutas da fala devem considerar a prática baseada na evidência na sua intervenção em crianças com alterações fonológicas. As revisões sistemáticas e meta-análises são uma boa solução para a consolidação e síntese da evidência científica \citep{Garrett2006}. \citet{Wren2018} apresentam um modelo com 5 categorias de intervenção em função da área em que é esperada a mudança e que constitui o foco da intervenção: ambiental “environmental”, percetivo-auditiva, cognitivo-linguística, produção, e integrada. A abordagem ambiental é distinta das restantes na medida em que pressupõe uma intervenção que se foca nas interações do dia-a-dia, em vez de atividades específicas. Inclui procedimentos, por vezes, descritos como \emph{intervenção naturalista}, assim como a modelagem e a reformulação das produções espontâneas \citep{Camarata2010}. As intervenções percetivo-auditivas têm como objetivo promover as capacidades percetivas da criança para a melhoria da produção. Inclui atividades que visam o aumento da exposição aos segmentos alvo, como a estimulação auditiva focada e as tarefas de discriminação implementadas para aumentar as capacidades de perceção dos segmentos \citep{Hodson1991, Rvachew2010}. As intervenções cognitivo-linguísticas envol\-vem a criança em atividades de níveis elevados de processamento em que se trabalha a consciência fonológica de forma a promover melhorias, através do confronto com um reduzido conjunto de contrastes e de atividades de consciência fonológica. As intervenções que se focam na produção visam promover alterações através de tarefas oromotoras, utilizando a colocação fonética, a imitação e o treino repetitivo/sistemático (\emph{drills}). As intervenções integradas combinam dois ou mais tipos de intervenção, nomeadamente a abordagem psicolinguística \citep{Stackhouse1997}. 

A intensidade da intervenção (frequência, duração das sessões, duração total) varia muito nos diferentes programas disponíveis \citep{Wren2018}, o que dificulta o estabelecimento de recomendações relativamente à intensidade mais indicada. Os objetivos do tratamento também variam em função das alterações específicas de cada criança \citep{Law2017}. 

\section{Como medir os resultados da intervenção?}\label{sec:cap7sec2}
É essencial a recolha de medidas de resultados objetivas antes e depois de um período de tratamento, de forma a medir a eficácia e eficiência da intervenção. Estas medidas são muito variáveis nos estudos publicados \citep{Baker2011, Wren2018}. A maioria centra-se em aspetos específicos da produção, como a percentagem de ocorrência de processos fonológicos (e.g., \citealp{Weiner1981, Lousada2013}), a taxa de acerto em formato silábico (e.g., \citealp{Major1998}), a taxa de acerto em palavra (e.g., \citealp{StoelGammon2010}), a taxa de acerto em função da extensão de palavra (e.g., \citealp{Hammarstroem2019}), a percentagem de inconsistência (e.g., \citealp{McIntosh2008}), a percentagem de consoantes corretas (PCC) (e.g., \citealp{Tyler2003, Lousada2013}), a percentagem de acerto em padrão acentual (e.g., \citealp{Hammarstroem2019}), o inventário fonético e o inventário fonológico (e.g., \citealp{Dinnsen1992, Yavas2001, Lousada2013}), a cotação em teste estandardizado, a diversidade de consoantes e estruturas silábicas, a inteligibilidade \citep{Lousada2014}, entre outras. Para além destas medidas, deverá ser realizada uma análise fonológica dos padrões de erro, podendo para isso recorrer a diferentes propostas, tais como os processos fonológicos (ver \citetv{chapters/03}), os traços distintivos (ver \citetv{chapters/04}) ou o modelo de análise de fonologia não linear \citep{Bernhardt2000}. Outro tipo de medidas inclui a percentagem de consciência fonológica correta (ver \citetv{chapters/10}), a cotação em tarefas de leitura e de soletração, a cotação em provas de perceção fonémica, a classificação da própria criança relativamente à intervenção (e.g., se gostou da intervenção) e a satisfação dos pais relativamente ao programa de intervenção implementado.  

Estão disponíveis atualmente para o português europeu (PE) vários instrumentos e ferramentas que podem auxiliar o terapeuta na recolha destas medidas (para informação mais detalhada sobre estes recursos, consulte-se o capítulo de Ramalho, Rodrigues \& Freitas, neste volume), nomeadamente, o Teste Fonético-Fonológico (ALPE) \citep{Mendes2013, Lousada2012}, o Crosslinguistic Child Phonology Project (CLCP-EP) \citep{Ramalho2014}, as Ferramentas de Análise Fonológica Automática \citep{Saraiva2017}, disponíveis em acesso livre \citep{Jesus2013}, o Phon \citep{Rose2014, Rose2006, Hedlund2019}, a The Locke Speech Perception-Speech Production Task - versão portuguesa \citep{Lousada2016}, disponível em acesso livre, instrumentos de avaliação da consciência fonológica (e.g., \citetv{chapters/10}; Tarefas de Consciência Fonológica para Crianças do 1.º Ciclo do Ensino Básico \citep{Afonso2015}, disponíveis em acesso livre, a prova de repetição de pseudo-palavras LITMUS-NWR-EP \citep{Almeida2015}, a versão portuguesa do Speech Participation and Activity Assessment of Children (SPAA-C) \citep{McLeod2004}, traduzida e adaptada por \citet{Ramos2017}, que permite analisar o impacto destas perturbações na vida quotidiana da criança e a versão portuguesa da Intelligibility in Context Scale \citep{McLeod2012}, disponível em acesso livre e validada para o PE com crianças com idades compreendidas entre os 3 anos e 11 meses e os 6 anos e 2 meses \citep{Lousada2019}.

\section{Planeamento da Intervenção}\label{sec:cap7sec3}
\subsection{Identificação/Definição de objetivos de intervenção}
Para o processo de intervenção devem seguir-se alguns passos após a avaliação e análise do sistema fonológico da criança e identificação das estruturas alvo (e.g., segmento, classe natural, sílaba, acento, palavra prosódica) presentes e ausentes na fonologia da criança, estabelecendo-se critérios para a priorização de alvos: elaboração de listas de estímulos (e.g., palavras) com base nos critérios fonológicos e elaboração de atividades que utilizem os segmentos-alvo e favoreçam a reabilitação fonológica e a generalização \citep{Lowe1996, Bernhardt2000}.

\subsection{Seleção dos Alvos -- critérios fonológicos para seleção dos estímulos}
Ao longo das últimas décadas têm vindo a ser propostas várias abordagens de intervenção nas alterações fonológicas. Tradicionalmente, a intervenção basea\-va-se na produção fonética -- treino articulatório -- dos sons, assumindo-se que as alterações de fala teriam sempre uma natureza oromotora. Cada som era trabalhado isoladamente, partindo do som isolado até à sua produção em frases. A escolha e ordem dos segmentos a trabalhar tinham como critério a ordem de aquisição, ou seja, os primeiros sons a serem adquiridos no processo de desenvolvimento da fala seriam os primeiros a serem alvo de intervenção \citep{Riper1984, Mota2001, Alves2018, Yavas2001}. Em \citeyear{Ingram1976}, \citeauthor{Ingram1976} propõe que a natureza das alterações de produção poderá ser fonológica. Este pressuposto teve repercussões muito importantes tanto na avaliação como na intervenção das alterações fonológicas. 

Independentemente das diferentes propostas/modelos de intervenção que podem ser encontrados, os princípios terapêuticos são idênticos para todos, já que assumem os princípios fonológicos que estão na base das alterações encontradas. Assim, com o processo de intervenção terapêutica pretende-se uma reorganização do sistema fonológico de acordo com os padrões identificados na avaliação. É objetivo mudar padrões fonológicos -- considerando as unidades fonológicas (e não motoras) e a sua natureza sistemática, a forma como os segmentos são utilizados contrastivamente e como se combinam na língua alvo -- com o intuito de construir um sistema de contrastes da forma mais eficaz possível. A finalidade da intervenção é a generalização da informação fonológica, identificada na avaliação e alvo de intervenção, tanto a outros segmentos, que também apresentam essa mesma propriedade, como a outros contextos. Um mo\-delo de intervenção fonológico, ao contrário de um modelo fonético (mais tradicional, cujo objetivo é aperfeiçoar os segmentos de forma isolada) pretende reconhecer e selecionar os processos e os sons alvos que possibilitam uma reorganização, que contenham a informação fonológica necessária para a reestruturação das redes fonológicas e generalização a outros segmentos e a outros contextos.  

Quando a intervenção visa a diminuição da ocorrência de processos fonológicos, a escolha dos processos e estímulos alvo pode considerar vários critérios como a estimulabilidade,\footnote{Capacidade para produzir segmentos por repetição do modelo do examinador \citep{Rvachew2018}.} a percentagem de ocorrência dos processo fonológicos (por exemplo, superior a 40\%), o impacto na inteligibilidade e a sequência de aquisição dos segmentos \citep{Dodd2000}.

Esta escolha deve ser baseada na estrutura do sistema fonológico de cada criança descrito e estudado na avaliação \citep{Mota2001, Lowe1996, StoelGammon2010, Alves2018, Elbert1986, Matzenauer2008}.
No trabalho de \citet{Elbert1986}, com o objetivo de encontrar o potencial máximo de generalização, sugerem-se critérios de seleção dos sons-alvo de intervenção, com base em leis implicacionais, através das quais poderíamos prever aquisições que resultam da estimulação a partir de um segmento (ou um par). Por exemplo, se uma criança não apresenta no seu sistema /s/ e /z/, e substitui estes segmentos por [ʃ] e [ʒ], respetivamente, deverá ser selecionado o segmento /z/ (mais complexo e com maior probabilidade de generalização) na intervenção. Prevê-se com este trabalho a aquisição de /s/ (ver capítulo de \citetv{chapters/05}). Espera-se que a aquisição do traço de ponto de articulação [+ anterior] generalize a outros segmentos (neste caso /s/).

Baseado nas teorias linguísticas, o nível da palavra tornou-se um dos pontos de partida para a intervenção fonológica. Com o surgimento da fonologia não linear, para além de se considerar os segmentos e as suas propriedades intrínsecas, num contexto de palavra, deverão ser tidos em conta aspetos prosódicos, tais como a estrutura silábica (por exemplo: um segmento poderá estar adquirido numa estrutura silábica e não em outra), o contexto de palavra ou a frase (ver \citetv{chapters/06}, \citetv{chapters/08}). 

De acordo com \citet{Gierut2001}, os resultados da intervenção terapêutica nas Perturbações Fono\-lógicas podem ser definidos em três dimensões: a eficácia, os efeitos e a eficiência. De acordo com a autora, a eficiência da intervenção poderá constituir a dimensão com maior relevância para a clínica de crianças com perturbações fonológicas. Esta eficiência do tratamento pode estar relacionada com a seleção inicial dos segmentos-alvo para a intervenção terapêutica, já que uma escolha poderá conduzir a uma generalização mais rápida e eficiente, ou seja a uma expansão e transferência de propriedades fonológicas. As diferenças observadas nos processos de intervenção terapêutica e graus de generalização podem ser explicadas pela escolha inicial do segmento alvo (\citetv{chapters/05}, \citealt{KeskeSoares2001, Duarte2006}).

\subsection{Aspetos segmentais}

Para a seleção do segmento alvo a considerar na intervenção, \citet{Elbert1986} sugerem previsões que possibilitam a aquisição de outros segmentos não trabalhados diretamente. Neste sentido, existem modelos fonológicos que propõem critérios para a seleção dos segmentos a utilizar na intervenção, entre os quais: o Modelo Implicacional de Complexidade de Traços (MICT), o Modelo de Relações de Distância entre Traços Distintivos (MOTIDT) ou o Modelo Padrão de Aquisição de Contrastes (respetivamente, \citealp{Duarte2006, Mota1996, LazzarottoVolcao2009}). Estes modelos pretendem promover a identificação das combinações em falta no sistema, possibilitando o entendimento dos segmentos que terão implicações na emergência de outros, tendo em conta os princípios fonológicos, recorrendo também ao conceito de aquisição implicacional \citep{KeskeSoares2001, KeskeSoares1996}. Sobre estes e outros modelos consultem-se os capítulos de \textcitetv{chapters/04} and \textcitetv{chapters/05}.

Na seleção das palavras ou outras unidades a trabalhar, torna-se importante ter em conta o restante contexto fonológico. Preferencialmente, o material selecionado não deverá conter outras unidades que sejam uma dificuldade adicional para a criança. Inicialmente as palavras deverão ser de acesso lexical fácil (e.g., o nome da irmã) para a criança e deverão promover a capacidade comunicativa da criança. A posição em que o som alvo surge é um critério importante, já que, num dado sistema linguístico, um segmento alvo poderá ser facilitado com um contexto inicial de palavra e, noutro sistema, um segmento alvo terá melhores resultados num contexto medial (e.g. a produção da líquida lateral /l/ em ataque simples poderá ser favorecida pelas vogais [a], [i] e [u] em adjacência) \citealp{MatzenauerHernandorena1997}). Deverão ainda ser considerados aspetos prosódicos (estrutura silábica, acento, palavra prosódica), que serão detalhados na secção seguinte \citep{Mota2001, Grunwell1990}.

\subsection{Aspetos prosódicos}

A integração de aspetos de natureza prosódica na intervenção nem sempre é claramente explicitada. No entanto, esta raramente é excluída da intervenção te\-rapêutica. 

Quando se trata de selecionar alvos e definir objetivos de intervenção, é comum incluir-se o segmento em diferentes tipos de sílabas ou formatos silábicos/de palavra, bem como em tarefas de fala encadeada. Ao fazerem isto, os terapeutas apelam a competências suprassegmentais, ou prosódicas, que permitem introduzir ritmo e melodia na fala. De facto, e de acordo com a perspetiva fonológica não-linear (ou multilinear), deve considerar-se o segmento tendo em conta não apenas a sua estrutura interna (como referido na secção anterior), mas também tendo em consideração a estrutura prosódica em que este se insere (veja-se \citetv{chapters/06}, \citetv{chapters/08}).

Na intervenção devem, por isso, ser consideradas como variáveis de intervenção aquelas que têm sido relevantes para a descrição da língua. Vários trabalhos sobre o PE (e.g., \citealp{Freitas1997, Amorim2014, Ramalho2017}) e internacionais \citep{Bernhardt2020}, de aquisição em contexto típico e atípico \citep{Lousada2012, Baptista2015, Ramalho2017a}, têm vindo a demonstrar a produtividade destas variáveis.

Nesta secção, centrar-nos-emos apenas nas unidades associadas à sílaba, ao acento e à palavra prosódica, podendo considerar-se os seguintes níveis para a intervenção \citep{Bernhardt2000}: estrutura silábica, acento, posição na palavra e extensão de palavra.

\emph{Estrutura silábica}: A integração do segmento na estrutura silábica envolve dife\-rentes níveis, considerando-se aqui a adaptação do modelo Ataque-Rima para o PE \citep{Freitas2017}. As consoantes podem estar associadas a diferentes constituintes silábicos, enquanto as vogais estão associadas ao constituinte Núcleo (simples ou ramificado). Contemplá-los na intervenção é fundamental para uma intervenção mais eficiente. Após a realização da avaliação, se se verificarem alterações em determinado segmento, importa verificar qual a relação com a estrutura silábica em que este se insere para, no planeamento da intervenção, contemplar apenas os contextos em que o segmento não está estabilizado (para o PE deve ter-se em especial atenção os segmentos que podem ocorrer em coda [ɾ, l, ʃ] e em ataque ramificado [ɾ, l]. Aumentar a complexidade silábica pode implicar aumentar a complexidade das exigências realizadas à criança, pelo que, ao fazê-lo, o terapeuta deve estar consciente deste processo.

Importa que quem orienta a intervenção contemple não só o tipo de segmento mas também o seu estatuto silábico. À luz dos modelos de intervenção que apelam à generalização, não é incomum selecionarem-se alvos mais complexos para promover a aquisição dos menos complexos. Por exemplo, pode selecionar-se um Ataque ramificado para iniciar a intervenção com uma criança que não produz [ɾ] (note-se que isolar a sílaba com o segmento em posição inicial absoluta não é natural para a criança, por não ser esta uma estrutura possível no PE).

\emph{Acento de palavra}: Uma vez que as primeiras produções das crianças são habitualmente produções de sílabas tónicas, pode considerar-se que este é um aspeto promotor da aquisição, devendo ser contemplado aquando do planeamento da intervenção. Por outro lado, o padrão paroxítono é considerado o padrão não marcado e mais frequente no PE, ocorrendo em cerca de 80\% de nomes e adjetivos, sendo importante na intervenção iniciar o trabalho com palavras com padrão acentual mais frequente (paroxítono). Por exemplo, quando o alvo é o segmento [k], deve iniciar-se a intervenção com palavras paroxítonas como \emph{cola} e só depois selecionar alvos como \emph{café}). Assim, associar produções de determinado segmento consonântico a sílaba com vogal em posição tónica pode promover e facilitar a produção na criança. Este é um fator relevante em intervenção, quer o foco do trabalho incida sobre a própria vogal ou sobre consoantes associadas a esta variável. Para o Inglês, \citet{Bernhardt2000} sugerem que se comece pela sequência de sílabas num padrão Sw\footnote{S para \emph{stressed} (sílaba forte/tónica); w para \emph{weak} (sílaba fraca/átona).} (\emph{\textbf{co}la}), seguida da sequência wS (\emph{ca\textbf{fé}}), e só depois wSw (\emph{ma\textbf{ca}co}). No caso do PE, pode fazer sentido usar esta sequência no estabelecimento de objetivos, uma vez que o formato Sw engloba 2 fatores promotores de produção (padrão não marcado/baixa complexidade e e\-levada frequência na língua). Esta variável deve ser contemplada em conjunto com outras, por exemplo com a extensão de palavra, iniciando por \emph{\textbf{pa}to} e só depois \emph{sa\textbf{pa}to}. Finalmente, deve inserir-se o padrão proparoxítono, sendo para tal necessária a existência de 3 sílabas Sww \emph{\textbf{ár}vore} (muitas vezes, produzida como ['avɨ], com preservação da sílaba tónica e recurso ao padrão dissilábico, paro\-xítono, o mais frequente na língua) para o segmento [ɾ] em coda, depois de o mesmo já ter sido trabalhado em dissílabos paroxítonos como \emph{arco}, \emph{barco} ou \emph{cantar}.

\emph{Posição na palavra}: A posição na palavra, variável fonológica associada ao domínio da palavra prosódica, é uma variável cuja descrição se tem mostrado relevante para a descrição das línguas, bem como nos processos de aquisição em contexto típico \citep{Freitas1997} e atípico \citep{Lousada2012, Ramalho2017, Ramalho2018} devendo, por isso, ser considerada aquando da planificação da intervenção. Quer para o português, quer para outras línguas, a posição inicial e a posição final são posições de proeminência prosódica \citep{Vigario2003}, aspeto que também deve ser considerado na intervenção terapêutica. No que diz respeito à posição na palavra, os segmentos podem ocorrer em posição inicial, medial ou final. Se se considerar uma abordagem multilinear, faz sentido abordar a posição na palavra, relacionando-a com a estrutura silábica. Nesta perspetiva, podem considerar-se as posições inicial e medial para os constituintes silábicos \emph{Ataque simples} e \emph{Ataque ramificado} e as posições \emph{medial} e \emph{final} para o constituinte silábico \emph{Coda}. A variável \emph{posição na palavra} tem demonstrado particular relevância na interação com a constituência silábica, particularmente com o constituinte \emph{Coda}. Vários trabalhos têm demonstrado o efeito de posição de palavra na \emph{Coda}, verifican\-do-se que, no caso do PE, uma emergência e estabilização anteriores em posição final \citep{Freitas1997, Amorim2014}.

\emph{Extensão de palavra}: Vários trabalhos internacionais têm demonstrado a produtividade da variável \emph{extensão de palavra}, associada ao desempenho de crianças em tarefas de soletração, de nomeação e de repetição de palavras e pseudopalavras (e.g. \citealp{James2006, Lee2007, Mason2015, Masso2017}, entre outros). Os resultados obtidos têm demonstrado um aumento da complexidade da tarefa associado ao aumento da extensão de palavra. À comple\-xidade evidenciada, associa-se a elevada frequência de palavras dissilábicas e baixa frequência de palavras polissilábicas.  Neste sentido, importa também considerar esta variável no contexto da intervenção, sendo muito relevante observar fenómenos de apagamento de sílabas átonas pré-tónicas e também pós-tónicas, no contexto de palavras polissilábicas. Retomando o exemplo da palavra \emph{árvore}, esta é muitas vezes produzida como ['avɨ], com apagamento de sílaba pós-tónica. Importa contemplar estes aspetos na delineação de objetivos, bem como na seleção de alvos a incluir na intervenção terapêutica.

Para finalizar, importa referir que a acumulação de variáveis implica o aumento de complexidade fonológica. Muitos polissílabos, além de complexos devido ao número de sílabas, são-no também por acumularem complexidade relativa à estrutura silábica, acento e posição na palavra. 

\subsection{Seleção de tarefas (psico)linguísticas}

A seleção das tarefas a realizar deve ser, sempre, orientada pela análise dos resultados da avaliação. Em função das alterações observadas poderá ser selecionada a abordagem que melhor se adequar aos objetivos traçados e ao perfil linguístico da criança. Têm sido propostos vários modelos ou abordagens que pretendem orientar a organização das sessões terapêuticas, uns pela escolha das tarefas a executar (bombardeamento auditivo, discriminação auditiva, repetição, tarefas de consciência fonológica, tarefas de produção, entre outras), outros pelo estímulo ou processo alvo a considerar, outros ainda pela escolha dos momentos de introdução destes estímulos. Alguns destes modelos são baseados em aspetos lineares (articulatórios ou fonológicos) e outros em aspetos não lineares: Modelo de Ciclos \citep{Hodson2006}, Modelo de Ciclos Modificado \citep{Bagetti2005} , Modelo de Pares Mínimos \citep{Weiner1981}, Modelo de Oposição Máximas \citep{Gierut1989}, Modelo ABAB-Retirada  \citep{Tyler1994}, Methaphon \citep{Dean1986}, entre outros \citep{Mota2001}. Sobre estes e outros modelos, consulte-se o capítulo de \textcitetv{chapters/05}.  A Tabela \ref{tab:cap7tab1} ilustra as tarefas que poderão ser implementadas na intervenção.



\begin{Tabela}[t]
\caption{Tarefas que podem ser utilizadas na intervenção}
\label{tab:cap7tab1}
\begin{tabularx}{\textwidth}{p{3cm}Q}
\lsptoprule
\textbf{Tarefa} &  \textbf{Exemplos de Procedimento}\\
\midrule
Bombardeamento auditivo & Ouvir palavras que contêm o alvo (e.g. lista de 15 palavras) \\
\tablevspace
Discriminação & Dizer se dois estímulos-alvo que ouviu são iguais ou diferentes \\
\tablevspace
Perceção & Dizer se ouviu o alvo\\
 & Identificar o alvo produzido (verbalmente ou em áudio gravado) num conjunto de imagens (p.e. apresentadas em escolha múltipla), e.g. identificar ‘caça’ no conjunto de imagens de ‘caça’, ‘casa’, capa. \\
 \tablevspace
Repetição & Repetir o alvo \\
\tablevspace
Produção & Nomear o alvo (com ou sem suporte visual) \\
\tablevspace
Memória & A partir de um estímulo verbal produzido (oralmente ou gravado em áudio), identificar (em sequência de imagens ou através de repetição) a ordem em que os estímulos-alvo ocorreram (pelo menos, dois estímulos). \\
\tablevspace
Nomeação\newline rápida & Nomear rapidamente um conjunto de imagens, que se vão repetindo em diferentes linhas. \\
\tablevspace
Consciência\newline fonológica & Segmentação, síntese, identificação, exclusão, supressão, manipulação \\
\tablevspace
Evocação\newline fonoló\-gica & Dizer palavras com o estímulo-alvo (e.g. sílaba inicial, som final, etc.) \\
\lspbottomrule
\end{tabularx}
\end{Tabela}


Seguidamente, são apresentados alguns exemplos de formulação de objetivos terapêuticos, tendo em consideração os critérios fonológicos propostos nesta publicação. Não serão apresentadas as taxas de sucesso, uma vez que estas dependerão sempre da fase de terapia. Não obstante este aspeto, os terapeutas devem seguir algum critério para considerar que o alvo está adquirido. Sugere-se que considerem uma estrutura como adquirida quando a taxa de acerto for superior a 80\% (o limiar varia entre 75\% e 90\%, em função de diferentes estudos e testes, ver \citetv{chapters/04}), um critério considerado em vários testes de avaliação e estudos de aquisição da fonologia (e.g. \citealp{Yavas2001, Mendes2013, Ramalho2017, Reis2018}; entre outros).

Listamos abaixo exemplos de objetivos que incluem os critérios fonológicos apresentados:

\begin{enumerate}[label=Exemplo \arabic*., align =left]
\item A criança deverá ser capaz de discriminar o contraste [+/-vozeado] nas oclusivas dorsais ([k, g]), em ataque simples inicial de palavras dissilábicas (sugere-se que, para definição da percentagem de aquisição do alvo, se considerem critérios de aquisição segmental), e.g. cola/gola.

\item A criança deverá ser capaz de repetir palavras dissilábicas, com uma estrutura silábica CV.CVC, com a Coda contendo o alvo [ɫ],em posição final tónica (e.g. papel).

\item A criança deverá ser capaz de produzir palavras dissilábicas, com estrutura silábica CV.CV, em que C na primeira sílaba é uma oclusiva labial e C na segunda sílaba é uma oclusiva dorsal (e.g. boca).

\sloppy
\item A criança deverá ser capaz de produzir palavras polissilábicas com Ataque ramificado átono em posição medial de palavra, e.g.\newline (C)V.CCV.CV.CV) \emph{obrigado}.
\fussy

\item A criança deverá ser capaz de repetir palavras dissilábicas com fricativas coronais, cuja sequência silábica alterne entre coronal [+ anterior] na primeira sílaba e coronal [-anterior] na segunda sílaba (e.g., \emph{sushi}).

\item A criança deverá ser capaz de produzir V em palavras dissilábicas com Ataque inicial vazio átono (V.CV) (e.g. \emph{olá}).
\end{enumerate}

\section{Estudos de intervenção com crianças falan\-tes do PE, com desenvolvimento atípico, em idade pré-escolar}\label{sec:cap7sec4}

Foram encontrados alguns estudos de intervenção realizados com crianças falantes do PE com alterações fonológicas \citep{Lousada2013, Lousada2014, Pedro2014, Pedro2018}. \citet{Lousada2013} realizou um estudo randomizado controlado, no qual comparou a eficácia da terapia articulatória tradicional com uma abordagem fonológica, que incluiu a terapia de consciência fonológica de \citet{Gillon2007} e atividades de audição e discriminação auditiva. A terapia de consciência fonoló\-gica de \citet{Gillon2007} tem como objetivo, não só a melhoria da produção da criança como a melhoria de competências de consciência segmental e intrassilábica e de relação letra-som. No estudo de \citet{Lousada2013}, a intervenção foi planeada em três blocos (nove semanas + oito semanas + oito semanas), sendo que em cada bloco se trabalhou um padrão de erro. Foram assim realizadas 25 sessões semanais, com uma duração de 45 minutos e todas as crianças foram tratadas pela mesma terapeuta da fala. Os alvos de intervenção foram identificados de acordo com a percentagem de ocorrência dos processos fonológicos (processos com ocorrência >40\% foram prioritários), a ordem de aquisição das consoantes para o PE, a estimulabilidade (sons estimuláveis foram prioritários) e o impacto na inteligibilidade. Foi selecionada uma lista de 12 palavras para as sessões de intervenção, que contêm o alvo definido, e selecionada uma lista de 5 palavras com características idênticas, que não foi utilizada durante os blocos de intervenção e serviu para uma prova de generalização. No final de cada bloco, foi solicitada a produção espontânea das 5 palavras (prova de generalização), de forma a verificar se a criança generalizou o alvo a palavras não utilizadas durante o tratamento. Esta prova permite que o terapeuta monitorize a evolução da criança e planeie as sessões seguintes. Os resultados mostraram diferenças estatisticamente significativas entre os grupos com uma melhoria mais evidente no grupo tratado com a abordagem fonológica ao nível da produção (aumento da PCC, diminuição da percentagem de padrões de erro e aumento do inventário fonético) e da inteligibilidade \citep{Lousada2013, Lousada2014}. 

O estudo de \citet{Pedro2014} comparou a eficácia entre o programa Parents and Children Together (PACT), de \citet{Bowen1999}, e a terapia de consciência fonológica de \citet{Gillon2007}, com idades compreendidas entre os 5;0 e os 6;1. A intervenção foi organizada em dois blocos, sendo que em cada bloco se trabalhou um padrão de erro. Foram realizadas 10 sessões de intervenção semanais, com uma duração de 45 minutos. Os resultados mostraram que ambos os programas promoveram melhorias significativas na produção pelo  aumento da PCC, diminuição da percentagem de padrões de erro e aumento do inventário fonético e melhorias na consciência silábica das crianças. Adicionalmente, verificou-se que o grupo tratado com o programa de \citet{Gillon2007} melhorou também ao nível de consciência fonémica. 

\citet{Pedro2018} analisaram os efeitos da intervenção com recurso aos Cartões L\&S em 7 crianças com atraso fonológico, com idades compreendidas entre os 5;3 e os 6;5. Estes cartões L\&S (estímulos visuais) visam trabalhar a correspondência som-letra e a melhoria da produção dos sons. Foram implementadas 6 sessões de intervenção, semanais e individuais. Os alvos de intervenção foram estabelecidos de acordo com a percentagem de ocorrência dos processos fonológicos, a ordem de aquisição das consoantes para o PE, e a estimulabilidade. Os resultados mostraram uma melhoria significativa ao nível da correspondência letra-som e da produção oral (PCC, padrões de erro e inventário fonético).

\citet{Alves2018}, através de um estudo de caso, observaram as aquisições obtidas através da intervenção baseada num modelo de intervenção assente em pressupostos fonológicos não lineares (ver \citetv{chapters/06}), assim como a eficiência de princípios implicacionais e a sua generalização ao sistema fonológico. A metodologia utilizada ao longo das sessões de intervenção realizadas seguiu a de Os Sons d’ A Relicário. Treino do Conhecimento Fonológico: Associação de Onomatopeias aos Sons da Fala, que visa a estimulação/reabilitação do conhecimento fonológico. As estratégias propostas na metodologia foram escolhidas em função dos objetivos terapêuticos e das unidades fonológicas pertinentes para a intervenção, tendo em conta pressupostos teóricos da fonologia não linear. São considerados dois níveis nesta metodologia: o prosódico e o segmental, representados por formas geométricas distintas que assumem papéis (unidades) e posições distintos (domínio da organização), e que permitem a acomodação física de unidades que sejam dominadas por outras que estão representadas acima (domínio da hierarquização). Dentro do nível segmental, o instrumento prevê representações distintas para os diferentes grupos de propriedades fonémicas através de códigos de cor, de representações icónicas e de pistas multissensoriais. Nesta abordagem, todos os segmentos consonânticos e vocálicos do português estão representados por uma onomatopeia, não enquanto modelo para o treino articulatório, mas enquanto estratégia fonémica mediadora da ou das propriedades fonológicas a reabilitar (cf. \citetv{chapters/10}). \citet{Costa2013}, utilizando a mesma metodologia, encontrou melhorias na execução de tarefas de consciência fonológica após intervenção de uma criança com diagnóstico de Dislexia. Vários trabalhos têm recorrido a esta metodologia como estratégias para complementar e potenciar os resultados de programas de intervenção ou profiláticos de consciência fonológica para promoção de competências da leitura e da escrita \citep{Freitas2007, Cysne2012}. 

A intervenção implicacional já foi aplicada a um estudo de caso de uma criança com perturbação fonológica, verificando-se a eficiência da intervenção baseada neste pressuposto. Observou-se a generalização de propriedades fonológicas ao sistema, a partir da estimulação de um segmento alvo com informações fonológicas mais complexas. Desta forma, e como exemplo, num sistema fonológico em que se observa ausência de todos os segmentos da classe das fricativas (/f/, /v/, /s/, /z/, /ʃ/, /ʒ/)  e do rótico dorsal /ʀ/, a escolha deste último como alvo de estimulação, com o objetivo de promover a aquisição do traço [+cont], promove a emergência não só do próprio segmento /ʀ/, como de todos os segmentos da classe das fricativas, por combinação do traço [+cont] a outros já existentes no sistema \citep{Mota2001, Reis2018, Alves2018, Reis2020}. Na Tabela \ref{tab:cap7tab2}, apresentam-se alguns dos estudos encontrados de intervenção em crianças com alterações fonológicas para o PE.



\begin{Tabela}
\caption{Estudos de intervenção para o português europeu.}
\label{tab:cap7tab2}


\begin{tabularx}{\textwidth}{p{4cm}Q}
\lsptoprule
\textbf{Abordagem} & \textbf{Foco da intervenção} \\
\midrule
Gillon -- PE \citep{Gillon2007, Lousada2012}  &  Treino fonológico baseado em consciência fonoló\-gica, integrando aspetos referentes ao segmento, à sílaba e à palavra (e.g., extensão de palavra), cons\-tituindo uma intervenção cognitivo-linguística. A seleção dos alvos tem em conta a análise por processos fonológicos.\\
\tablevspace
PACT -- adaptação ao PE \citep{Pedro2014} & Adaptação ao PE da versão original \citep{Bowen1999}. Abordagem em 5 etapas: educação familiar, atividades de produção fonética, tarefas metalinguísticas, treino de exemplos múltiplos e trabalho de casa. Deve existir uma grande colaboração dos pais/cuidadores e a terapia é planeada em blocos, intercalada com pausas. Baseia-se em diferentes níveis (fonológico, fonético e percetivo (traços distintivos e aquisição de contrastes; treino fonético, consciência fonológica)) constituindo uma intervenção integrada. \\
\tablevspace
Cartões L\&S (\citealt{Pedro2014}, disponível em \url{http://acsa.web.ua.pt/LS.htm}) & Correspondência som-letra e melhoria da produção dos sons da fala \\
\tablevspace
Os Sons d’Relicário \citep{Alves2014, Alves2018, Alves2014a} & Treino fonológico com base na fonologia não li\-near, considerando aspetos prosódicos e segmentais, através da codificação dos mesmos com cores, formas, ícones e onomatopeias.\\
\lspbottomrule
\end{tabularx}
\end{Tabela}


\section{Conclusão}
Conclui-se que um conhecimento aprofundado da linguística permite  aos te\-rapeutas da fala planear a intervenção em fonologia de forma adequada, com vista a uma maior eficácia e eficiência do tratamento de crianças com perturbação fonológica. 

\section*{Agradecimentos}
This work was supported by the National Funds through FCT – Fundação para a Ciência e a Tecnologia, I.P., within CINTESIS, R\&D Unit (reference UIDB/4255{\slash}2020 and UIDP/4255/2020) and CLUL (UIDB/00214/2020).

{\sloppy\printbibliography[heading=subbibliography,notkeyword=this]}
\end{document}
