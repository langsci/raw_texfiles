\documentclass[output=paper,colorlinks,citecolor=brown,booklanguage=portuguese]{langscibook}
\ChapterDOI{10.5281/zenodo.7233217}
\title{Avaliação para o diagnóstico do transtorno dos sons da fala}
\author{Haydée Fiszbein Wertzner\affiliation{Universidade de São Paulo, Faculdade de Medicina, Laboratório de Investigação Fonoaudiológica em Fonologia (LIFFON)}}

\abstract{Na Secção \ref{sec1cap3} deste capítulo é mostrada a importância da avaliação fonoaudiológica para o diagnóstico do transtorno dos sons da fala. Em seguida, é detalhada a avaliação fonológica e fonética, bem como a análise dos processos fonológicos e do inventário fonético. Na secção \ref{sec:sec3cap3} são abordadas as provas complementares tanto cognitivo-linguísticas como de produção motora da fala. Os resultados fornecidos pelas provas complementares auxiliam o fonoaudiólogo na identificação do subtipo de transtorno dos sons da fala.


\textbf{Palavras-chave}: avaliação; diagnóstico; criança; fonoaudiologia; transtorno dos sons da fala.}

\IfFileExists{../localcommands.tex}{
   \addbibresource{../localbibliography.bib}
   \usepackage{langsci-optional}
\usepackage{langsci-gb4e}
\usepackage{langsci-lgr}

\usepackage{listings}
\lstset{basicstyle=\ttfamily,tabsize=2,breaklines=true}

%added by author
% \usepackage{tipa}
\usepackage{multirow}
\graphicspath{{figures/}}
\usepackage{langsci-branding}

   
\newcommand{\sent}{\enumsentence}
\newcommand{\sents}{\eenumsentence}
\let\citeasnoun\citet

\renewcommand{\lsCoverTitleFont}[1]{\sffamily\addfontfeatures{Scale=MatchUppercase}\fontsize{44pt}{16mm}\selectfont #1}
  
   %% hyphenation points for line breaks
%% Normally, automatic hyphenation in LaTeX is very good
%% If a word is mis-hyphenated, add it to this file
%%
%% add information to TeX file before \begin{document} with:
%% %% hyphenation points for line breaks
%% Normally, automatic hyphenation in LaTeX is very good
%% If a word is mis-hyphenated, add it to this file
%%
%% add information to TeX file before \begin{document} with:
%% %% hyphenation points for line breaks
%% Normally, automatic hyphenation in LaTeX is very good
%% If a word is mis-hyphenated, add it to this file
%%
%% add information to TeX file before \begin{document} with:
%% \include{localhyphenation}
\hyphenation{
affri-ca-te
affri-ca-tes
an-no-tated
com-ple-ments
com-po-si-tio-na-li-ty
non-com-po-si-tio-na-li-ty
Gon-zá-lez
out-side
Ri-chárd
se-man-tics
STREU-SLE
Tie-de-mann
}
\hyphenation{
affri-ca-te
affri-ca-tes
an-no-tated
com-ple-ments
com-po-si-tio-na-li-ty
non-com-po-si-tio-na-li-ty
Gon-zá-lez
out-side
Ri-chárd
se-man-tics
STREU-SLE
Tie-de-mann
}
\hyphenation{
affri-ca-te
affri-ca-tes
an-no-tated
com-ple-ments
com-po-si-tio-na-li-ty
non-com-po-si-tio-na-li-ty
Gon-zá-lez
out-side
Ri-chárd
se-man-tics
STREU-SLE
Tie-de-mann
}
   \boolfalse{bookcompile}
   \togglepaper[03]%%chapternumber
}{}

\begin{document}
\maketitle

\section{A avaliação como principal instrumento de diagnóstico}\label{sec1cap3}

Todo fonoaudiólogo deve apoiar a sua atividade profissional na prática baseada em evidências.\footnote{A norma adotada na escrita deste capítulo foi a do português brasileiro.}
% \footnote{Na sequência de restrições decorrentes do template usado para publicação na editora Language Science Press, os capítulos que integram o volume surgem com a referência [this volume], em inglês.}
Essa prática se inicia pelo acolhimento da queixa trazida pelos familiares e pela própria criança em relação às suas dificuldades de comunicação. A partir da análise inicial da queixa, o fonoaudiólogo deve preparar-se para iniciar a avaliação e confirmar ou não a hipótese diagnóstica.

A avaliação do transtorno dos sons da fala (TSF), em geral, é feita em crianças, predominantemente, entre cinco e oito anos de idade. Nessa avaliação constam procedimentos que verificam como está a fala da criança e os recursos usados para se comunicar. Assim, o fonoaudiólogo precisa obter dados da interação entre os aspectos cognitivo-linguístico (fonológico), motor da fala e perceptivo sendo fundamental que se trace esse perfil da criança. 

Especificamente, para o diagnóstico do TSF deve selecionar-se o tipo de avaliação a ser aplicada na criança que, por sua vez, pode ser padronizada com norma referenciada, critério referenciado ou informal \citep{McLeodEtAl2017}. O tipo de prova aplicada não propicia o diagnóstico, pois para obtê-lo o fonoaudiólogo precisa conhecer as provas da avaliação, saber aplicar cada uma delas e, principalmente, saber como analisar os dados coletados. Essa análise inclui a transcrição cuidadosa do material, assim como a análise proposta no manual do instrumento da avaliação, que permite classificar os resultados em adequados ou não para a criança avaliada.

A partir dessa análise, o terapeuta pode estabelecer um diagnóstico inicial de TSF, mas precisa selecionar quais avaliações complementares devem ser aplicadas para identificar os aspectos mais comprometidos, ou seja, a organização e a representação fonológica, a produção motora de fala e/ou a percepção auditiva \citep{Wertzner2002}.

Dois instrumentos para avaliação do TSF para o português brasileiro foram desenvolvidos no mesmo período, com o objetivo de verificarem a Fonologia e o inventário fonético de crianças. O primeiro deles é a Avaliação Fonológica da Criança (AFC) de \citet{Yavas1991} e o segundo as provas de Fonologia do Teste de Linguagem Infantil ABFW nas áreas de Fonologia, Vocabulário, Pragmática e Fluência (Fonologia ABFW) \citep{Wertzner2004}, elaboradas por \citeauthor{Wertzner1992} em \citeyear{Wertzner1992}, em sua pesquisa de doutoramento.

Especificamente, as provas de Fonologia ABFW \citet{Wertzner1992,Wertzner2004} vêm sendo muito estudadas, principalmente, nos estudos desenvolvidos no laboratório de investigação fonoaudiológica em Fonologia (LIFFON) do Departamento de Fisioterapia, Fonoaudiologia e Terapia Ocupacional da Faculdade de Medicina da Universidade de São Paulo. As provas de Fonologia ABFW foram aplicadas tanto na população infantil sem queixa de alteração de fala e linguagem, como na população com queixa. Por ser uma prova com norma e critérios referenciados, é usada em crianças de 3;0 a 8;0 anos de idade e permite o diagnóstico de crianças com TSF. 

Percebemos, com o avanço da obtenção de evidências da avaliação da Fonologia ABFW e também com a observação dos estudos internacionais a respeito da TSF, a necessidade de aplicar provas complementares ao diagnóstico que fornecessem mais informações a respeito da representação fonológica e da produção motora da fala, além da audição e, especialmente, da percepção auditiva. Dessa forma, elaboramos diversas provas complementares que avaliam a consciência fonológica, a taxa articulatória, a estimulabilidade e a inconsistência de fala \citep{Castro2009, Wertzner2009, Castro2011, Wertzner2012}. 

Mais recentemente, introduzimos no processo de avaliação do transtorno a ultrassonografia de fala \citep{Francisco2017}, que, somada à análise acústica da fala e da EGG \citep{PaganNeves2007, PaganNeves2010}, trouxeram contribuições importantes para o entendimento de vários processos fonológicos apresentados por crianças falantes do português brasileiro. Ainda acrescentámos os estudos que envolvem o aspecto perceptivo-auditivo \citep{Goncalves2011, Barrozo2015}.

A avaliação para o diagnóstico do TSF deve incluir amostras de fala obtidas por imitação, nomeação e fala espontânea, que, por sua vez, devem ser analisadas detalhadamente. Em seguida, provas complementares são aplicadas e o conjunto de dados obtidos e analisados tornam-se a base para o planejamento da intervenção fonoaudiológica que busca obter eficácia na proposta para cada criança. Todas as crianças avaliadas no LIFFON passam, também, pela avaliação audiológica e, em função da idade, além da avaliação audiológica, é avaliado o processamento auditivo ou os potenciais evocados da audição.

Neste capítulo são abordadas as provas imitação de palavras e nomeação de figuras da Fonologia do ABFW e de fala espontânea. Duas análises são propostas, a tradicional (inventário fonético) e a dos processos fonológicos. Em seguida são apresentadas as principais provas complementares cognitivo-linguística e de produção motora de fala (Tabela \ref{tab:cap3tab1}) que podem ser utilizadas para obter dados que auxiliarão no planejamento da intervenção.




\begin{Tabela}
\caption{Avaliação do TSF: principais provas usadas no LIFFON}
\label{tab:cap3tab1}


\begin{tabularx}{\textwidth}{QQQ}
\lsptoprule
\textbf{Provas diagnósticas -- cognitivo-linguística} & \textbf{Provas complementares -- cognitivo-linguística} & \textbf{Provas complementares -- produção motora}\\
\midrule
Fonologia ABFW & Vocabulário ABFW & Estimulabilidade de fala\\
\tablevspace
Fala espontânea & Inconsistência de fala & Taxa articulatória\\
\tablevspace
& TSF (aliteração e rima) & Análise acústica da fala\\
\tablevspace
& Leitura e escrita & Análise articulatória -- ultrassom\\
\lspbottomrule
\end{tabularx}
\end{Tabela}


\section{Avaliação fonológica e do inventário fonético}
Na primeira etapa do diagnóstico são aplicadas provas de Fonologia que, em geral, incluem imitação de palavras ou de frase, nomeação de figuras e fala espontânea.

As provas de Fonologia do ABFW foram elaboradas, inicialmente, para serem aplicadas em crianças entre os 3;0 e 7;0 anos, cujo objetivo é verificar o sistema fonológico de crianças falantes do português brasileiro e comparar com a norma esperada para as diversas idades. Vários estudos foram realizados posteriormente estendendo a normatização da idade nas provas de Fonologia ABFW para a faixa etária de 2:00 a 8:11 anos \citep{Galea2003, Galea2008, Wertzner2004a, Wertzner2004b}.

As provas de Fonologia ABFW detectam a presença ou ausência de dificuldades tanto fonéticas como fonológicas, na medida que analisa o inventário fonético e os processos fonológicos. Ainda, por meio da aplicação dessas provas é possível caracterizar a gravidade e a inteligibilidade de fala. 

São aplicadas duas provas: imitação de palavras e nomeação de figuras. A prova de imitação conta com 39 vocábulos e a de nomeação com 34 figuras.  Os dados observados devem ser registrados, nos respectivos protocolos, por meio de transcrição fonética.

A prova de fala espontânea pode ser coletada de várias formas, sendo mais comumente empregadas as   por entrevista ou de produção de uma história. No LIFFON solicitamos à criança que nos conte uma história infantil, em geral, a dos três porquinhos ou a do chapeuzinho vermelho. Para isso, temos 20 pranchas que vão detalhando cada cena da história infantil para que a criança tenha estímulos suficientes para produzir uma história que possa ser analisada. Na prova de fala espontânea, após transcrever foneticamente a história contada pela criança, o fonoaudiólogo seleciona os substantivos e analisa tanto os processos fonológicos quanto o inventário fonético. Além disso, é necessário calcular as possibilidades de ocorrência dos processos fonológicos e dos sons do inventário fonético na amostra de fala.


\subsection{Análise do inventário fonético}
Outra análise realizada após a aplicação das provas de Fonologia ABFW é a tradicional, na qual é feito o levantamento do inventário fonético (dos sons consonantais) nas posições de sílaba inicial e final. Após a transcrição fonética, o fonoaudiólogo calcula a ocorrência de acertos, omissões, substituições e distorções para cada som. Na Fonologia ABFW há a normatização para a idade para cada som do português brasileiro na estrutura CV, bem como para as estruturas CCV e CVC \citep{Wertzner1992, Wertzner1994, Wertzner2004}.

Além da normatização para a idade, há também a indicação da ocorrência de cada som em posição silábica inicial e final. Com essa informação é possível traçar o perfil do inventário fonético da criança. Quando uma criança apresenta algum som consonantal com menos de 75\% de acertos, esses são fortes candidatos à intervenção fonoaudiológica, considerando-se a idade da criança e o tipo de erro mais ocorrente.

A análise do inventário fonético é muito importante, pois fornece dados que permitem ao fonoaudiólogo decidir a respeito da necessidade de aplicação de provas complementares ao diagnóstico, além de serem a base para o cálculo de índices de gravidade do TSF. Também indica que, nos casos em que só se registrarem distorções, a análise fonológica pode ser dispensada, na medida que o levantamento do inventário fonético é suficiente.


\subsection{Análise dos processos fonológicos}
O protocolo de análise propõe a verificação dos processos fonológicos de maior ocorrência durante o período de desenvolvimento fonológico de acordo com o descrito por \citet{Wertzner1992, Wertzner1995}. Os processos fonológicos analisados nas provas de Fonologia ABFW são: redução de sílaba, harmonia consonantal, plosivação de fricativa, posteriorização para velar, posteriorização para palatal, frontalização de velar, frontalização de palatal, simplificação de líquida, simplificação do encontro consonantal, simplificação da consoante final, sonorização de plosiva, sonorização de fricativa, ensurdecimento de plosiva e ensurdecimento de fricativa.

A análise dos processos fonológicos é realizada após a transcrição fonética, na qual são classificados os processos usados em cada uma das provas. É importante destacar que, na análise dos processos fonológicos, o fonoaudiólogo, apesar de analisar cada palavra, deve considerar o total da amostra de palavras para conferir se a classificação dos processos fonológicos está adequada. Isso porque, muitas vezes, em uma mesma palavra ocorrem diferentes processos fonológicos, e é preciso esclarecer quais são esses processos comparando toda a amostra da prova, conforme os exemplos da Tabela \ref{tab:cap3tab2}.



\begin{Tabela}
\caption{Ocorrência de diferentes processos em uma mesma palavra}
\label{tab:cap3tab2}


\begin{tabularx}{.8\textwidth}{lQ}
\lsptoprule
 \textbf{palavra} & \textbf{processos fonológicos identificados} \\
 \midrule
 gola- [ˈtɔlɐ] & frontalização para palatal; \\
  & ensurdecimento de plosiva\\
 \tablevspace
jaca - [ˈtakɐ] & plosivação de fricativa; \\
  & ensurdecimento de fricativa\\
 \lspbottomrule
\end{tabularx}
\end{Tabela}



Para as provas de Fonologia ABFW, além das normas por idade, para cada processo fonológico analisado há, também, a possiblidade de ocorrência para cada um. Assim, o fonoaudiólogo deve calcular a produtividade de cada processo em cada uma das provas. A ocorrência de mais de 25\% de um processo fonológico nas provas de Fonologia ABFW sugere que a criança o aplica numa frequência considerável, e tal processo deve ser alvo de intervenção, caso não seja mais o esperado para a idade.

É importante destacar que os processos indicados na Tabela \ref{tab:cap3tab2} são aqueles mais observados durante o período de desenvolvimento. Como crianças com TSF podem apresentar processos fonológicos idiossincráticos, estes também devem ser analisados, porém, como são muito variáveis entre as crianças com TSF, o fonoaudiólogo deve analisar as características do processo idiossincrático e calcular as possibilidades de sua ocorrência nas provas de Fonologia ABFW e calcular a porcentagem empregada na amostra coletada. 


\subsection{Índices de gravidade}
Na literatura dois índices são, frequentemente, citados e aplicados, tanto no diagnóstico fonoaudiológico, quanto nas avaliações realizadas durante o tratamento para observar a evolução do mesmo. O primeiro índice é a porcentagem de consoantes corretas (PCC) proposto por \citeauthor{Shriberg1982} em \citeyear{Shriberg1982}, no qual contabiliza-se o número de consoantes que foram produzidas corretamente numa amostra de fala. Nesse estudo os autores propuseram os critérios para considerar uma consoante correta, bem como uma classificação qualitativa de gravidade para o PCC considerando os falantes do inglês.

As adaptações desses critérios para a classificação de acertos de consoantes para o português brasileiro, assim como para a classificação da gravidade do PCC, foram propostas por \citet{Wertzner2002} e \citet{Wertzner2005}. 

O índice porcentagem de consoantes corretas revisado (PCC-R) foi proposto por \citet{Shriberg1997}. Este índice se diferencia do PCC por não considerar as distorções como erro. Assim, o PCC-R pode ser usado em amostras de crianças de diferentes idades, pois somente considera como erros substituições e omissões.

O PCC-R, também, tem sido aplicado às provas de Fonologia ABFW. No estudo de \citet{Barrozo2017} com crianças com e sem TSF, foi apresentado o valor de corte obtido na curva \emph{Receiver Operating Characteristic} (ROC), que mostrou valores diferentes para cada uma das provas. Para a prova de nomeação de figuras, o valor de corte foi de 93,4\%, com sensibilidade de 0,89 e especificidade de 0,94 independentemente da idade.  Já na prova de imitação de palavras, os valores obtidos variaram de acordo com a idade, sendo que, para a faixa etária ≤6:5 anos, de 91,0\%, com sensibilidade de 0,77 e especificidade de 0,94, e para a faixa etária >6:5 anos, de 93,9\%, com sensibilidade de 0,93 e especificidade de 0,94.

Outro índice utilizado é o \emph{process density index} (PDI), proposto por \citet{Edwards1992}, que contabiliza o número de processos fonológicos numa amostra de fala. O PDI, também, tem sido aplicado às provas de Fonologia ABFW \citep{Wertzner2002, Wertzner2007}, mostrando ser um índice inversamente proporcional ao PCC e ao PCC-R. O PDI foi aplicado, também, em um estudo com crianças com TSF, entre 7;00 e 9;11 anos de idade, com e sem alteração do processamento auditivo central \citep{Barrozo2015}, no qual foi encontrado um valor de corte de 0,54 com sensibilidade (0,73) e especificidade (0,90), indicando que crianças com TSF e PDI acima de 0,54 apresentam grande possibilidade de ter alteração de processamento auditivo central.

Portanto, nas provas de Fonologia ABFW, a partir do levantamento do inventário fonético, é possível calcular os índices PCC e PCC-R e o da ocorrência de processos fonológicos, o PDI. Vários estudos aplicaram o PCC às provas de Fonologia ABFW \citep{Barrozo2015, Barrozo2017, Castro2011, Wertzner2012}. 


\subsection{Provas de fonologia ABFW em crianças com TSF}
As provas de Fonologia ABFW foram aplicadas para descrever o perfil de crianças falantes do português brasileiro com TSF. \citet{Wertzner2002} analisou as provas de Fonologia ABFW em crianças com diagnóstico de TSF. Observou-se maior ocorrência no sexo masculino, que, por sua vez, manifestaram processos fonológicos de desenvolvimento com maior ocorrência de simplificação de líquidas, simplificação do encontro consonantal e ensurdecimento de fricativas, sendo que em média as crianças apresentaram três processos fonológicos diferentes. Além disso, o estudo indicou que cerca de 10\% das crianças apresentaram processos fonológicos idiossincráticos (por exemplo, o processo fonológico de posteriorização para línguo-alveolar: /f/ → /s/) concomitantemente aos de desenvolvimento. Ao analisar o grau de comprometimento, o estudo mostrou que a maior parte das crianças com TSF têm um grau leve ou levemente moderado.

Os principais fatores que interferem no julgamento qualitativo da gravidade do TSF, por fonoaudiólogos, foram: o tipo de processo fonológico observado, a inteligibilidade de fala e a idade da criança \citep{Wertzner2002, Wertzner2005}.

Portanto, os vários estudos realizados com as provas de Fonologia ABFW demonstraram ser este instrumento adequado para o diagnóstico do TSF, uma vez que, em tais estudos foi constatada a diferença entre crianças com e sem TSF tanto em relação ao inventário fonético, aos processos fonológicos, quanto aos índices PCC, PCC-R e PDI.


\section{Provas complementares}\label{sec:sec3cap3}
Várias pesquisas mostram a inter-relação entre os aspectos cognitivo-linguístico, a produção motora da fala e a percepção auditiva no TSF.  Uma criança com esse transtorno pode apresentar dificuldades nas três áreas embora com intensidades diferentes \citep{Dodd2014, Dodd2018}. Essa é uma das razões pela qual as crianças com tal diagnóstico são heterogêneas. Essa heterogeneidade ocorre tanto em relação às manifestações, quanto às suas causas. Por isso, os estudiosos do TSF têm se dedicado à identificação de subtipos que possam caracterizar se uma criança manifesta, predominantemente, uma dificuldade cognitivo-linguística, de produção motora da fala ou de percepção auditiva \citep{Dodd2014, Dodd2018, Shriberg2019}. 

Partindo dessa premissa da interação entre os aspectos cognitivo-linguístico, a produção motora da fala e a percepção auditiva decorre a preocupação em complementar a avaliação fonoaudiológica após o diagnóstico de TSF, buscando identificar qual é a maior dificuldade da criança. Neste capítulo serão abordadas as provas complementares que exploram o aspecto cognitivo-linguístico e de produção motora de fala.


\subsection{Provas complementares cognitivo-linguísticas}
As provas complementares cognitivo-linguísticas são provas que complementam principalmente as análises das representações fonológicas. As que utilizamos, com maior frequência, no LIFFON, são as que avaliam o vocabulário, a inconsistência de fala, as habilidades de processamento fonológico e as de leitura e escrita. As aplicações dessas provas complementares estão relacionadas aos resultados de pesquisas que mostram que crianças com TSF, que tenham ou não sido submetidas ao tratamento, têm possibilidade de apresentar dificuldades de representação fonológica que pode se refletir no processamento fonológico e na aprendizagem de leitura e escrita \citep{Wertzner2012, Dodd2014}.


\subsubsection{Prova de vocabulário}
O vocabulário é avaliado por meio da prova de vocabulário do Teste de Linguagem Infantil ABFW nas áreas de Fonologia, Vocabulário, Pragmática e Fluência \citep{BefiLopes2004}. A análise dos dados é realizada de acordo com os critérios estabelecidos e o desempenho comparado com a norma por idade.


\subsubsection{Teste de sensibilidade fonológica}
Considerando as habilidades de processamento fonológico, várias provas foram adaptadas para o português brasileiro. O Teste de Sensibilidade Fonológica foi elaborado por \citet{Herrero2001}. A versão auditiva (TSF-A) consta de quatro provas, duas para a habilidade de aliteração (igual e diferente) e as outras duas para rima (igual e diferente). Cada prova apresenta 15 itens, sendo os três primeiros itens para explicação da prova (exemplos, não pontuados) e 12 restantes para aplicação e análise. As provas têm vocábulos distratores que podem ser fonológicos ou semânticos, cujo intuito é verificar como a criança está organizando suas respostas. Os distratores estão presentes somente nas provas de aliteração e rima igual. 

O TSF-A pode ser aplicado em crianças entre 5;00 e 7;11 anos de idade. Em sua pesquisa de mestrado, \citet{Herrero2001} estabeleceu os valores esperados para cada prova em função da idade (Tabela \ref{tab:cap3tab3}). 




\begin{Tabela}
\caption{Média de acertos por prova e idade no TSF-A}
\label{tab:cap3tab3}


\begin{tabularx}{.65\textwidth}{lrrr}
\lsptoprule
 & \multicolumn{3}{c}{\textbf{TSF-A}} \\
 \cmidrule(lr){2-4}
 \textbf{Prova} & \textbf{5 anos} & \textbf{6 anos} & \textbf{7 anos}\\
 \midrule
 \textbf{Aliteração Igual} & 6,0 & 8,4 & 11,0\\
 \textbf{Aliteração Diferente} & 3,5 & 4,3 & 9,3\\
 \textbf{Rima Igual} & 5,0 & 5,3 & 7,1\\
 \textbf{Rima Diferente} & 4,6 & 4,4 & 6,7\\
 \lspbottomrule
\end{tabularx}
\end{Tabela}

Os estudos que aplicaram o TSF-A como prova complementar indicaram que crianças com TSF manifestam maior dificuldade em todas as provas quando comparadas a seus pares sem TSF. Também se verificou que, em geral, apresentam maior dificuldade na identificação da rima tanto igual como diferente \citep{Wertzner2012}.


\subsubsection{Prova de inconsistência de fala}
Outra prova utilizada como complementar é a inconsistência de fala, proposta por \citet{Castro2011}, que consta da nomeação de 25 figuras apresentadas por três vezes em diferentes sequências de apresentação, apoiada na proposta de \citet{Dodd2005}. A análise das respostas é feita comparando-se as três repetições de cada palavra. Quando a palavra é nomeada sempre da mesma forma, independentemente, se certo ou errado, a palavra é considerada consistente e ganha 1 ponto. Quando ocorrem produções diferentes nas três nomeações, a palavra é considerada inconsistente e recebe zero pontos. Em seguida é calculado o índice de inconsistência de fala (IIF) considerando-se o número total de palavras inconsistentes, dividido por 25 e multiplicado por 100.

No estudo de \citet{Castro2011} foram estabelecidos os valores de corte da inconsistência de fala, por meio da curva ROC, que indicou diferenças entre meninos e meninas e em função da idade. Assim, os valores de corte entre crianças com e sem TSF para o IIF são, para a idade entre 5;0 e 7;6 anos, 21,50\% para o sexo feminino e 31,90\% para o masculino; para idade acima de 7;7 anos, 14,50\% para o sexo feminino e 17,60\% para o masculino.

\citet{Castro2011} também observaram que crianças com TSF falantes do português brasileiro apresentaram maior índice de inconsistência de fala que crianças sem TSF. Porém, nem todas as crianças com TSF apresentam inconsistência de fala. Verificaram ainda, em crianças com TSF, uma média de 27.4\% de inconsistência de fala e, nas crianças sem TSF, uma média de 9.8\%, com maior ocorrência em meninos do que meninas e que diminui com a idade nos dois grupos de crianças.

\citet{Wertzner2012} também analisaram a inconsistência de fala em crianças com e sem TSF e correlacionaram seus valores com as habilidades do TSF-visual \citep{Herrero2001} e os índices PCC-R na prova de nomeação da Fonologia ABFW. Obtiveram evidências de diferença significativa entre os dois grupos para as variáveis estudadas, bem como correlação entre a inconsistência de fala e as demais variáveis, porém, somente no grupo de crianças sem TSF.  Esses resultados sugerem que a presença da inconsistência de fala dificulta a representação fonológica em graus variados, apontando a heterogeneidade presente nas crianças com TSF.


\subsection{Provas complementares de produção motora de fala}
As provas complementares mais específicas para a avaliação da produção motora de fala envolvem provas como a estimulabilidade dos sons ausentes, a taxa articulatória, a análise acústica e a análise articulatória.

Na prática clínica, após o diagnóstico do TSF, o fonoaudiólogo, diante da análise do inventário fonético, tem informações para decidir quais as provas complementares que deve aplicar. No caso de se observar sons ausentes do inventário fonético, é de extrema importância aplicar a prova de estimulabilidade. Verificar se um som ausente do inventário fonético pode ser produzido pela criança, mediante diferentes estímulos, fornece informações para o planejamento da intervenção.

Se também observamos no inventário fonético a ausência de vozeamento em plosivas e ou fricativas, precisamos utilizar instrumentos como a análise acústica ou a eletroglotografia para verificar se a criança faz o vozeamento ou não, e se o faz, se este é suficiente para ser percebido pelo ouvinte como vozeado.

Ainda se uma criança apresenta um som distorcido, os denominados erros residuais, pode-se ter uma grande ajuda de instrumentos que propiciam a análise articulatória, como a ultrassonografia de contorno de língua.

\subsubsection{Prova de estimulabilidade de fala}

O Teste de estimulabilidade dos sons da fala (TESF) \citep{Castro2022} dos sons ausentes do inventário fonético foi elaborada, inicialmente, para os sons líquidos do português brasileiro \citep{Castro2009} e, posteriormente, para os demais sons \citep{Castro2012}.

Assim, o TESF consta de duas provas. A primeira envolve a imitação de palavras (IP) e conta com a possibilidade de uso de auxílios. A segunda prova é a imitação de sílabas (IS). A prova IP consta de sete palavras para cada fonema do português brasileiro. O som ausente do inventário fonético é produzido na sílaba inicial (sempre que sua distribuição na língua permitir) combinado com as sete vogais orais.  O TESF conta com palavras para as plosivas surdas /p, t, k/, plosivas sonoras /b, d, g/, fricativas surdas /f, s, ʃ/, fricativas sonoras /v, z, ʒ / nasais /m, n, ɲ/, líquidas /l, ɾ, ʎ/, fricativa velar /X/, /ɾ, s/ em coda e encontros consonantais /pl, kl, bl, gl, fl, pɾ, tɾ, kɾ, bɾ, dɾ, gɾ, fɾ, vɾ/. A prova IS é formada por sete sílabas referentes aos fonemas do português brasileiro combinados com as sete vogais orais. 

Na aplicação do TESF, o fonoaudiólogo deve sentar-se em frente à criança e pedir que ela imite a primeira palavra. Caso a criança não consiga imitar, o terapeuta deve descrever como o som é produzido e solicitar a imitação da palavra. Se a criança não produzir corretamente, o fonoaudiólogo chama a atenção da criança para sua boca e oferece um espelho para que a criança monitore a sua produção após ser solicitada a imitar a mesma palavra. Caso a criança não consiga produzir na segunda tentativa, o terapeuta deve oferecer uma pista tátil do som alvo, usando uma espátula para tocar os pontos articulatórios. Após a terceira tentativa o terapeuta continua solicitando a imitação das demais palavras, independentemente de a criança produzir certo ou errado. Ao final deve-se atribuir pontos a cada uma das produções. Se a criança produziu corretamente, mesmo após as pistas oferecidas, a palavra é pontuada com 2 pontos; se foi produzida de forma distorcida, ganha 1 ponto e, se não conseguiu produzir, zero pontos. Se a criança atingiu 10\% ou mais, o som é considerado estimulável.

O resultado do TESF e auxilia o fonoaudiólogo a selecionar a melhor abordagem de intervenção, além de ser uma evidência importante para a seleção do som alvo para o início da intervenção.

\subsubsection{Prova de taxa articulatória}

Algumas pesquisas mostram que a taxa articulatória de crianças com TSF é menor que em crianças sem alterações \citep{Flipsen2002}. Considerando que esse é um aspecto que pode fornecer indicações a respeito da maturidade motora de fala nos auxiliando na intervenção, construímos no LIFFON um instrumento para avaliar a taxa articulatória, que mede os fones por segundo na produção de fala. Após uma pesquisa com algumas formas de coleta da taxa articulatória, \citet{Wertzner2009} sugerem que a análise seja feita a partir da coleta de três repetições de duas frases padrão, sendo uma curta, contendo 12 fones “O cachorro fugiu” e outra longa, contendo 22 fones “A Maria tem uma bola vermelha”.  Após a coleta é feita a análise, na qual é verificada a duração da sentença. O número de fones por segundo é calculado desconsiderando as pausas superiores a 0,15 segundos, como por exemplo numa respiração profunda. 

\citet{Francisco2016} estudaram a taxa articulatória em crianças sem e com TSF falantes do português brasileiro e observaram a média de fones/segundos de acordo com a idade em meses e o tipo de sentença (Tabela \ref{tab:cap3tab4}). Nota-se que, independentemente do grupo de crianças, há um aumento da taxa articulatória com o aumento da idade, bem como que os valores da taxa articulatória para a sentença longa é maior que para a curta, mostrando que a criança com TSF, apesar de ter uma taxa articulatória menor, já consegue o controle aerodinâmico para essa situação. 



\begin{Tabela}
\caption{Média da taxa articulatória para crianças com e sem TSF em função da do tipo de sentença e da idade.}
\label{tab:cap3tab4}


\begin{tabularx}{.9\textwidth}{llrrrrr}
\lsptoprule
 &&& \multicolumn{2}{c}{\textbf{Sentença curta}}& \multicolumn{2}{c}{\textbf{Sentença longa}}\\
 \cmidrule(lr){4-5}\cmidrule(lr){6-7}
 \textbf{Grupo} & \textbf{Idade em meses} & \textbf{N} & \textbf{Média} & \textbf{DP} & \textbf{Média} & \textbf{DP}\\
 \midrule
sem TSF & 60 a 71 & 21 & 10.24 & 2.4 & 10.99 & 2.1\\
& 72 a 83&	14&	9.93&	2.1&	10.57&	1.7\\
& 84 a 95&	16&	10.82&	2.6&	11.78&	2.9\\
&96 a 107&	10&	12.08&	2.8&	13.27&	3.2\\
&108 a 119&	9&	12.03&	2.1&	12.93&	0.9\\
&Total& 	70&	10.8&	2.5&	11.66&	2.5\\
\midrule
Com&	60 a 71&	22&	8.23&	2.8&	9.29&	2.2\\
\midrule
TSF&	72 a 83&	24&	9.22&	2.4&	9.88&	1.8\\
&84 a 95&	21&	8.86&	1.3&	10.77&	2.1\\
&96 a 107&	11&	9.24&	1.9&	10.78&	3.4\\
&108 a 119&	9&	9.92&	1.8&	10.83&	1.4\\
&Total& 	87&	8.96&	2.2&	10.16&	2.3\\
\lspbottomrule
\end{tabularx}
\parbox{.1\textwidth}{\raggedright\footnotesize\noindent{}}\parbox{.9\textwidth}{\raggedright\footnotesize\noindent{Legenda: N= número de crianças; DP= desvio padrão}}
\end{Tabela}


Os valores mostrados na Tabela \ref{tab:cap3tab4} podem auxiliar o fonoaudiólogo na verificação da influência da taxa articulatória na produção de determinados sons alvo. Assim, se uma criança com TSF apresentar valores abaixo desses encontrados, pode manifestar uma dificuldade maior no controle aerodinâmico para a produção de fricativas, principalmente as vozeadas \citep{Wertzner2016}.

\subsubsection{Análise acústica da fala}

A incorporação da análise acústica como prova complementar ocorreu a partir de 1996, no LIFFON. Foram publicados vários estudos a respeito do uso desse instrumento, e sua contribuição como prova complementar \citep{PaganNeves2007, PaganNeves2010}.

A análise acústica dos sons da fala pode auxiliar o fonoaudiólogo na identificação dos movimentos da língua na produção do som alvo. Isto pode ser observado tanto em relação à altura da língua na cavidade oral (F1) quanto para a posição antero-posterior da língua na cavidade oral (F2), uma vez que a análise acústica da fala reflete um sinal acústico resultante de um movimento articulatório. Para isso, em geral, é usada a espectrografia de banda larga e analisados os formantes, bem como a duração do som. A análise qualitativa da espectrografia também é importante, pois nos indica a distribuição da energia característica de cada som, a presença ou ausência da barra de vozeamento na produção de uma plosiva ou fricativa vozeada. 

Para utilizar a análise acústica, o fonoaudiólogo precisa estar apto a interpretar os dados que o programa escolhido fornece, bem como a imagem da espectrografia. Associada à análise acústica, podemos usar a eletroglotografia, que contribui na observação da presença de vozeamento na produção dos sons plosivos e fricativos.

\subsubsection{Ultrassonografia de contorno de língua}

A ultrassonografia do contorno de língua é outra prova complementar que fornece informações em tempo real da produção de um som que tem como um dos articuladores a língua. Por ser um instrumento em geral autorizado eticamente para ser aplicado em crianças, tem sido cada vez mais utilizado \citep{Preston2017}. As medidas obtidas no ultrassom de contorno de língua podem ser tanto quantitativas como qualitativas. As medidas quantitativas são bastante discutidas na literatura e são muito variáveis \citep{Zharkova2015, Silva2017, Francisco2017}.

As medidas qualitativas, em geral, buscam identificar contrastes encobertos que possam esclarecer os movimentos usados na produção de um som. Muitas vezes tais contrastes encobertos são classificados como um erro fonológico \citep{McAllisterByun2016}.

Para a avaliação em tempo real é importante obter medidas precisas que possam ser comparadas e, para isso, recomenda-se estabilizar a posição da cabeça em relação ao transdutor para que o espaço a ser medido fique, razoavelmente, constante. Porém, a grande questão é o uso do capacete estabilizador em crianças. Alguns pesquisadores têm sugerido que se coletem os dados de avaliação ultrassonográfica sem uso de estabilizador, desde que se siga um protocolo minucioso para registrar as produções dos sons da fala, tanto em posição sagital quanto coronal \citep{Preston2017}.

\section{Conclusão}

A avaliação para o diagnóstico do TSF envolve a aplicação de várias provas que, analisadas em conjunto, além de identificarem sua presença, fornecem informações indispensáveis para o planejamento da intervenção. Esse conjunto de provas busca avaliar os aspectos cognitivo-linguístico de produção de fala. Embora não tenha sido abordada nesse capítulo, a avaliação da audição e da percepção auditiva é imprescindível e completa o processo de diagnóstico do TSF.

As provas fonológicas, que envolvem imitação, nomeação e fala espontânea, permitem tanto a análise fonológica quanto do inventário fonético. A análise dessas provas possibilita determinar o diagnóstico de TSF. Com o diagnóstico definido, o fonoaudiólogo seleciona as provas complementares que podem fornecer mais informações a respeito do tipo de TSF e fundamentar o planejamento da intervenção fonoaudiológica. 


{\sloppy\printbibliography[heading=subbibliography,notkeyword=this]}
\end{document}
