\documentclass[output=paper,colorlinks,citecolor=brown,booklanguage=portuguese]{langscibook} 
\ChapterDOI{10.5281/zenodo.7233251}
\title{Avaliação neurolinguística em pacientes com tumores cerebrais}
\author{Joana Alves \affiliation{Hospital das Forças Armadas – Pólo Porto, Porto, Portugal} and Mafalda Cardoso\affiliation{Unidade de Cuidados Continuados Solar Billadonnes, Penacova, Portugal} and  Mariana Morgado\affiliation{Campus Neurológico Sénior, Torres Vedras, Portugal} and  Luis M. T. Jesus\affiliation{Universidade de Aveiro, Aveiro, Portugal}}

\abstract{Neste capítulo são abordados conceitos como as perturbações adquiridas nos sistemas da linguagem e da fala, mais especificamente, por ocorrência de tumores cerebrais com indicação cirúrgica que requererão, de quem os aborda um conhecimento robusto das áreas neurolinguísticas nas vertentes de avaliação e intervenção. Começamos por fazer uma breve referência à origem de diferentes perturbações neurológicas no adulto e uma revisão de um conjunto de tópicos relacionados com a neurolinguística no adulto, nos domínios semântico, sintático, fonológico e articulatório. Os tumores cerebrais localizados em áreas relevantes para a fala e linguagem são então abordados e os procedimentos relacionados com uma craniotomia acordada. Questões específicas de avaliação pré-, intra- e pós-operatória das competências linguísticas em pacientes com tumores cerebrais, mais especificamente em gliomas de baixo grau, com base numa versão Portuguesa do \emph{Dutch Linguistic Intraoperative Protocol}, são também apresentadas.

\textbf{Palavras-chave:} Language; linguistics; phonology; syntax; semantics; awake brain surgery; low-grade gliomas}

\IfFileExists{../localcommands.tex}{
   \addbibresource{../localbibliography.bib}
   \usepackage{langsci-optional}
\usepackage{langsci-gb4e}
\usepackage{langsci-lgr}

\usepackage{listings}
\lstset{basicstyle=\ttfamily,tabsize=2,breaklines=true}

%added by author
% \usepackage{tipa}
\usepackage{multirow}
\graphicspath{{figures/}}
\usepackage{langsci-branding}

   
\newcommand{\sent}{\enumsentence}
\newcommand{\sents}{\eenumsentence}
\let\citeasnoun\citet

\renewcommand{\lsCoverTitleFont}[1]{\sffamily\addfontfeatures{Scale=MatchUppercase}\fontsize{44pt}{16mm}\selectfont #1}
  
   %% hyphenation points for line breaks
%% Normally, automatic hyphenation in LaTeX is very good
%% If a word is mis-hyphenated, add it to this file
%%
%% add information to TeX file before \begin{document} with:
%% %% hyphenation points for line breaks
%% Normally, automatic hyphenation in LaTeX is very good
%% If a word is mis-hyphenated, add it to this file
%%
%% add information to TeX file before \begin{document} with:
%% %% hyphenation points for line breaks
%% Normally, automatic hyphenation in LaTeX is very good
%% If a word is mis-hyphenated, add it to this file
%%
%% add information to TeX file before \begin{document} with:
%% \include{localhyphenation}
\hyphenation{
affri-ca-te
affri-ca-tes
an-no-tated
com-ple-ments
com-po-si-tio-na-li-ty
non-com-po-si-tio-na-li-ty
Gon-zá-lez
out-side
Ri-chárd
se-man-tics
STREU-SLE
Tie-de-mann
}
\hyphenation{
affri-ca-te
affri-ca-tes
an-no-tated
com-ple-ments
com-po-si-tio-na-li-ty
non-com-po-si-tio-na-li-ty
Gon-zá-lez
out-side
Ri-chárd
se-man-tics
STREU-SLE
Tie-de-mann
}
\hyphenation{
affri-ca-te
affri-ca-tes
an-no-tated
com-ple-ments
com-po-si-tio-na-li-ty
non-com-po-si-tio-na-li-ty
Gon-zá-lez
out-side
Ri-chárd
se-man-tics
STREU-SLE
Tie-de-mann
}
   \boolfalse{bookcompile}
   \togglepaper[20]%%chapternumber
}{}

\begin{document}
\maketitle

\section{Introdução}

O conhecimento de base neurológico sobre domínios como a fonologia, a morfologia, a sintaxe e a semântica, bem como as funções programação e execução motora da fala, teve um amplo crescimento nas últimas décadas e está ultrapassada a visão de que é estanque a relação entre área cerebral e a funcionalidade \citep{Friederici2011}.\footnote{A norma adotada na escrita deste capítulo foi a do português europeu.}
% \footnote{Na sequência de restrições decorrentes do template usado para publicação na editora Language Science Press, os capítulos que integram o volume surgem com a referência [this volume], em inglês.}
Assim, atualmente, sabe-se que diferentes regiões córtico-subcorticais, em ambos os hemisférios, são responsáveis por determinadas funções da linguagem e/ou da fala. A determinação das áreas cerebrais e redes neuronais responsáveis é possível através do avanço da tecnologia, permitindo com alguma precisão sinalizar e destacar essas regiões, aprimorando intervenções médico-terapêuticas aquando de danos neurológicos. Esses poderão ter diversas etiologias e impactar a vários níveis o indivíduo, enquanto ser biopsicossocial.

Para um melhor acompanhamento do paciente neurológico, é fundamental uma intervenção multidisciplinar, na qual deverá ser considerado o Terapeuta da Fala (TF), aquando as áreas da linguagem estejam afetadas, ou se manifestem défices funcionais nas mesmas. Dessa forma, uma avaliação e intervenção especializada, cuidada e precisa poderá fazer a diferença. 

\section{Perturbações neurológicas no adulto}

São várias as etiologias que podem estar na origem das perturbações neurológicas no adulto. Se por um lado estas podem ocorrer devido a condições progressivas, e.g., doenças degenerativas ou tumores cerebrais, por outro lado podem ocorrer por eventos agudos, e.g., acidente vascular cerebral (AVC) ou traumatismo crânio-encefálico (TCE).

Embora uma lesão neurológica possa não afetar somente o domínio da linguagem e/ou fala, este capítulo centra-se nas perturbações neurológicas cujos défices se manifestem nessas áreas. Assim, a linguagem é a função cognitiva necessária para comunicarmos, por meio de um sistema complexo de símbolos e requer integridade das funções cerebrais a ela associadas. Quando uma perturbação da linguagem, ou seja, uma afasia, causa défices que podem envolver uma ou mais modalidades como a compreensão e/ou expressão da linguagem oral e/ou escrita em um ou mais domínios linguísticos, como a fonologia, a semântica e a sintaxe, a função da linguagem encontra-se comprometida \citep{Witte2015}. É ainda de referir que alterações no planeamento e execução motora dos órgãos fonoarticulatórios permitem identificar traços de disartria, uma perturbação motora na qual um ou mais subsistemas da fala podem estar afetados (respiração, fonação, ressonância e articulação). Uma outra causa para as alterações articulatórias é a apraxia do discurso \citep{Stemmer2008}.

Numa perturbação adquirida da linguagem causada por uma lesão focal em determinada área cerebral ou, mais difusa, abrangendo várias áreas, existem défices conhecidos que podem ocorrer na expressão e/ou compreensão como a hesitação na escolha das palavras, erros e distorções articulatórias, bem como dificuldades na construção de frases. Embora também detetáveis, mas menos frequentemente referenciados são os défices na compreensão de frases complexas e dificuldades de escrita e leitura \citep{Bastiaanse2014}.

\section{Neurolinguística no adulto}

A cada um dos domínios da linguagem e à componente articulatória, estão associadas áreas cerebrais específicas, não deixando de salvaguardar a variabilidade interindividual \citep{Friederici2011}. Neste sentido, vários são os estudos que correlacionam as ativações de áreas cerebrais e redes neuronais com os domínios da linguagem e com as tarefas específicas executadas em cirurgia \citep{Coello2013}. Acresce dizer que estes domínios não estão circunscritos ao córtex frontal e temporal, mas também nas vias que interconectam essas regiões. Assim, atualmente, considera-se que a linguagem é organizada por meio de redes córtico-subcorticais distribuídas de forma paralela e dinâmica e não por áreas individuais \citep{Demonet2005}, tal como abaixo sintetizado.

\subsection{Domínio semântico}

Uma capacidade exclusivamente humana, como acedermos ao significado das palavras, objetos, pessoas e factos, só é possível através do processamento semântico \citep{MoritzGasser2013a}. A compreensão da organização semântica é possível por meio de métodos de imagem neurofuncional, permitindo a identificação de regiões cerebrais responsáveis pela organização cortical do sistema semântico \citep{Duffau2005}. 

Foram vários os autores que ao longo dos anos estudaram a organização e armazenamento semântico \citep{Baldo2006, Bello2008, Bello2007, Bertani2009, Coello2013, Witte2015a, Duffau2005, Friederici2009, Gill2015, Huth2016, Lubrano2014, MoritzGasser2013a, Patterson2007, Richardson2011}, concluindo que informação é armazenada em redes constituintes da memória semântica. Vários estudos têm-se focado na organização cortical do processamento semântico, tendo revelado que existe uma rede neuronal abrangente que envolve os lobos frontal, temporal e parietal. Alguns investigadores como \citet{Duffau2005}, descrevem duas áreas no hemisfério esquerdo como de extrema importância: As regiões temporais posteriores esquerdas e o córtex frontal inferior (CFI) esquerdo \citep{Duffau2005}.

Durante tarefas de compreensão, sejam elas induzidas por estímulos visuais, orais ou somestésicos, as regiões da parte posterior do lobo temporal superior são ativadas. Tarefas como a leitura ou a nomeação, i.e., tarefas com estímulos visuais, ativam as áreas temporais inferiores basais, associadas à evocação lexical. Em relação ao papel semântico do CFI, mais especificamente \emph{pars orbitalis}, \emph{pars triangularis} (inferior e anterior) e o córtex pré-frontal dorsolateral, estes podem estar relacionados com os aspetos executivos do processamento semântico. A \emph{pars opercularis} (área posterior) está mais envolvida na tarefa sintática. Outros dados sobre a organização desta rede cortical estão documentados na literatura, mas ainda carecem de informações claras sobre as ligações subcorticais relacionadas com a semântica, i.e., esclarecimentos sobre a conectividade anatomo-funcional subjacente ao sistema semântico \citep{Duffau2005}.  

\citet{Bertani2009} sugerem que existem diferentes vias de substância branca para conectar diferentes áreas do cérebro associadas à semântica, como o fascículo longitudinal inferior (FLI) e o fascículo uncinado (FU), a cápsula extrema e o fascículo fronto-occipital inferior (FFOI). Ainda sobre estas áreas, foi possível concluir através do mapeamento subcortical que o FFOI esquerdo, após a estimulação elétrica, causou alterações semânticas, mostrando a sua pertinência no que concerne o processamento semântico, não podendo ser substituído por outro fascículo subcortical \citep{MoritzGasser2013a}. Num outro estudo é referida a existência de duas vias importantes na semântica: Uma direta, que será o FFOI, e uma indireta, possivelmente o FLI e FU, distinguindo apenas os diferentes significados no processamento semântico \citep{MoritzGasser2013}. A ocorrência de parafasias semânticas foi observada durante a estimulação do FFOI. Em suma, sugere-se que é este fascículo permite gerar e compreender a linguagem, trabalhar conceitos, captar e compreender o mundo que nos cerca \citep{MoritzGasser2013a}.

A nomeação que se integra no domínio da semântica é particularmente relevante na aquisição de linguagem \citep{Fecteau2011} e, de acordo com \citet{Spezzano2010}, é uma das capacidades mais importantes no processamento linguístico envolvendo substratos neuroanatómicos variados.

A nomeação de imagens envolve vários processos cognitivos, desde o processamento/reconhecimento visual, a conceptualização e processamento semântico, a seleção lexical ou evocação e, finalmente, a articulação, que corresponde à ativação da programação e execução motora da palavra. A capacidade de evocação do nome associado ao objeto ou imagem é atribuída a porções do lobo temporal lateral esquerdo e ao córtex temporal anterior \citep{Baldo2013}. Esta informação reforça a ideia de que a nomeação está organizada segundo vários processos e subsistemas que interagem uns com os outros, ainda que mantendo algum grau de independência \citep{Matchin2019}. 

Estudos em pacientes com défices de nomeação, evidenciaram que partes do giro temporal medial posterior e anterior esquerdos e, também, o giro temporal superior são ativados de forma diferente quando isolados das áreas responsáveis pela evocação, mostrando o papel preponderante destas estruturas no ato de nomear \citep{Baldo2013, Strijkers2017}. Tarefas de evocação semântica têm vindo a ser associadas a áreas frontais como o giro frontal inferior esquerdo \citep{Ries2019}.

\subsection{Domínio Sintático}

Apesar da variedade de abordagens imagiológicas terem permitido a caracterização das regiões cerebrais implicadas nos processos sintáticos, a forma como o cérebro humano processa e codifica as estruturas sintáticas não está ainda suficientemente estudado \citep{Matchin2019}. A área de Broca tem sido amplamente mencionada como crucial no processamento sintático de frases complexas. É importante mencionar que é difícil dissociar as áreas envolvidas nas atividades sintáticas e semânticas, pois várias são as evidências de que estas interagem a todo o momento \citep{Friederici2011}. Ainda neste sentido, \citet{Matchin2019} verificaram através da ressonância magnética funcional (RMf) uma crescente ativação combinada de áreas correspondentes ao processamento sintático e semântico nas seguintes áreas: giro angular, lobo temporal anterior, sulco temporal superior posterior e no giro frontal inferior \citep{Matchin2019}.

\citet{Friederici2011} postula que há uma maior ativação na região do giro frontal inferior que acompanha a crescente complexidade do processamento sintático e que o processamento de frases complexas ativa a área de Broca. Existe um papel variável das regiões \emph{pars opercularis} e \emph{pars triangularis} (constituintes da área de Broca), consoante a língua falada, estando a sua ativação e importância comprovada no Português \citep{Friederici2011, Matos2002}.

O estudo de \citet{Thothathiri2012} identificou que lesões que envolvem a região temporoparietal esquerda, incluindo o giro angular, estão associadas a erros de interpretação semântica e sintática de frases reversíveis como <O homem é servido pela mulher> \citep{Thothathiri2012}.

\subsection{Fonologia}

No que concerne o domínio da fonologia, evidencia-se a influência cortical a nível do giro supramarginal (GSM), localizado no lobo parietal, uma vez que quando estimulado se verificam alterações fonológicas nas tarefas de nomeação e nas de repetição \citep{Duffau2007, Duffau2005, Maldonado2011, MoritzGasser2013}. O estudo de \citet{Maldonado2011} especifica a ativação desta estrutura em tarefas de processamento fonológico \citet{Maldonado2011}. Mais ainda, a estimulação da parte inferior do GSM induziu anomias e perturbações fonológicas \citep{MoritzGasser2013}. 

Também o lobo temporal se encontra envolvido em tarefas deste domínio \citep{Duffau2007}, sendo que as porções mediais do lobo temporal são ativadas na descodificação da fala e respetiva compreensão \citep{Hickok2012}. A estimulação do giro temporal posterior, superior e medial induziu a ausência de resposta e perseveração na tarefa de repetição de palavras \citep{Coello2013}. 

O fascículo longitudinal superior (FLS) e respetivas projeções corticais (córtex inferior frontal, lobo parietal inferior e lobo temporal) também estão envolvidos neste domínio, uma vez que quando estimulados elicitam parafasias fonémicas \citep{Coello2013}. O fascículo arqueado (FA) é referenciado como uma estrutura altamente envolvida em tarefas fonológicas \citep{Duffau2007}. Vários são os estudos que mostram a importância desta estrutura subcortical para o domínio fonológico uma vez que, quando estimulada, especialmente na sua porção póstero-superior, verificam-se alterações fonológicas como parafasias fonémicas em tarefas de repetição de palavras \citep{Coello2013, Maldonado2011, MoritzGasser2013}.

Em atividades de fluência fonológica, o córtex frontal é crucial na evocação fonémica, mais concretamente o \emph{pars opercularis}, o córtex motor primário e o córtex pré-motor e motor complementar \citep{Duffau2005}. Adicionalmente, para esta atividade, são ativadas as regiões temporais anteriores, o córtex parietal inferior esquerdo (frequentemente associado à memória de trabalho), a ínsula e o putâmen \citep{Baldo2006, Duffau2007}.

\subsection{Articulação}

Sabe-se que a articulação envolve o córtex pré-motor ventral, i.e., o segmento anterior da parte lateral do FLS \citep{Maldonado2011}. A sua inatividade provoca anartria, i.e., a perda da capacidade articulatória \citep{Geemen2014} e disartria, evidenciando a presença de uma “fala arrastada” \citep{Duffau2005, MoritzGasser2013}. O córtex pré-motor ventral é uma área com plasticidade limitada, sendo, pois importante a sua preservação \citep{Geemen2014, Maldonado2011}. Mais ainda, a parte anterior do FLS lateral induz alterações articulatórias. Refere-se ainda a influência da área complementar motora dominante, ínsula anterior, núcleo lentiforme e córtex pré-motor na componente articulatória, sendo possível evidenciar disartria quando estimuladas estas áreas \citep{Coello2013, MoritzGasser2013}. \citet{Duffau2007} refere que o papel do núcleo lentiforme está associado ao planeamento motor para a fala \citep{Duffau2007}. Adicionalmente perturbações na programação motora para a fala foram reportadas ao fascículo subcaloso \citep{Coello2013}. 

\citet{Bello2007} mencionam o papel da área de Broca nesta competência, sendo que quando é estimulada induz “fala arrastada”. Refere ainda que a estimulação da parte anterior do ventrículo lateral induziu a características disártricas e que, a estimulação da área correspondente à cabeça do núcleo caudado induziu a perseveração discursiva. A estimulação da substância branca perto do bordo anterior da ínsula, medial à área de Broca, induz parafasias fonémicas e, mais superiormente, também “fala arrastada” \citep{Bello2007}. No que concerne ao papel da região insular em tarefas articulatórias, sabe-se que a região anterior da ínsula está associada à capacidade de coordenação articulatória \citep{Baldo2006} e que a via descendente, que conecta a ínsula anterior e o córtex pré-motor ventral ao córtex primário sensório-motor (correspondente à região da boca), também desempenha um papel preponderante. A estimulação destas fibras provoca anartria \citep{Bello2010, Coello2013, Duffau2006}.

\section{Tumores cerebrais}

O tumor cerebral pode ser primário ou secundário/metastático dependendo da sua etiologia. Assim, o tumor cerebral secundário advém de migrações para o cérebro de células cancerígenas de outra parte do corpo, e.g., pulmões \citep{DenBent2012}. Por outro lado, os tumores primários, tal como o próprio nome sugere, têm etiologia primária em células cerebrais. Entre estes (e.g., linfomas, meningiomas, tumores pituitários ou craniofaringiomas), os gliomas são o tipo mais frequente dos tumores primários do sistema nervoso central. Têm a sua origem nas células da glia e podem ser histologicamente classificados em astrocitomas, oligodendrogliomas, oligoastrocitomas mistos (combinam elementos oligodendrogliais e astrocíticos) e ependimomas, consoante a afeção da célula nervosa que lhes dá origem. A par desta classificação, podem ainda ser diferenciados em tumores de baixo ou alto grau, conforme o seu grau de malignidade e agressividade. Assim, os tumores de baixo grau são caracterizados como infiltrativos, de crescimento lento e contínuo, mas que eventualmente podem evoluir para tumores de alto grau, cujo crescimento é rápido e imprevisível e, por isso, mais agressivos \citep{Miotto2011}. Os gliomas representam 60\% de todos os tumores cerebrais e, segundo a Organização Mundial de Saúde, são mais frequentes no sexo masculino.

Normalmente, os gliomas de baixo grau envolvem áreas corticais e subcorticais e manifestam em défices ao nível da funcionalidade. Uma das linhas preferenciais de tratamento passa pela sua resseção, através da craniotomia em paciente acordado. Cada tratamento deve levar em consideração as características do tumor e do paciente \citep{Bertani2009}, sendo os sinais e sintomas clínicos altamente variáveis. Porém, em geral, como resultado destes processos neoplásicos, todos poderão apresentar sintomatologia de alterações cognitivas, e.g., dificuldades de compreensão e desorientação, resultantes da destruição, isquemia ou edemas cerebrais associados das estruturas intracranianas \citep{Miotto2011}.

A resseção tumoral permite aumentar o tempo de sobrevida e a qualidade de vida do paciente, na medida em que são preservadas as suas capacidades. Para este processo é necessária uma avaliação multidisciplinar pré-operatória, detalhando a localização, a extensão da lesão e os défices já manifestados pela presença da lesão tumoral.

Se uma avaliação clínica contínua é crucial no processo do paciente, mais ainda se torna evidente esta necessidade aquando da realização de uma craniotomia realizada com o paciente acordado. Assim, para conseguir preservar a funcionalidade do paciente, além da avaliação pré-operatória, é preciso que, durante a intervenção cirúrgica sejam avaliadas as suas capacidades, para que melhor se delimitem as áreas que podem ser removidas sem causar défices pós-cirúrgicos adicionais, i.e., fazer um mapeamento cerebral individualizado. Por este motivo, o conhecimento sobre a organização neural da linguagem é de extrema importância.

\section{Craniotomias em paciente acordado}

No que concerne aos tumores cerebrais localizados em áreas relevantes para a fala e linguagem, a craniotomia acordada aliada à avaliação neurolinguística e mapeamento intra-cirúrgico, é uma das linhas preferenciais de tratamento \citep{Darder2012}. As \emph{guidelines} atuais sugerem que quando uma pessoa é submetida à cirurgia, o objetivo deve ambicionar o máximo de resseção possível maximizando os ganhos no pós-cirúrgico. 

O paciente tem um papel ativo neste tipo de cirurgia, pelo que existe uma série de especificidades no que concerne à anestesia administrada \citep{Witte2015}. \citet{Pereira2009} mencionam o uso de uma anestesia local que permita a aplicação do método \emph{asleep-awake-asleep}. A primeira fase deste método passa pelo correto posicionamento do crânio por parte da equipa. Posteriormente, fixa-se o mesmo com recurso ao grampo de Mayfield e a craniotomia é executada. Antes de abrir a duramáter o paciente é acordado e o mapeamento cortical é efetuado, através da estimulação cortical direta. Identificam-se as áreas córtico-subcorticais e, depois de removido o tumor, o paciente é re-intubado e a duramáter, o crânio e couro cabeludo são fechados \citep{Duffau2007, Duffau2014, Szelenyi2010}. 

Previamente a todo o aparato cirúrgico há uma avaliação individual e preparação psicológica e emocional da pessoa submetida a este tipo de abordagem para que, na fase crítica seja otimizada a tolerância a este procedimento invasivo.

Com os avanços da tecnologia, particularmente dos sistemas de neuronavegação e de técnicas de anestesia cada vez mais otimizadas, as craniotomias em paciente acordado tornaram-se uma abordagem mais rápida, segura e efetiva, à qual os pacientes apresentam bons níveis de tolerância. Com objetivo de guiar as intervenções cirúrgicas diversos métodos imagiológicos têm vindo a ser usados.

Historicamente, o surgimento da eletroencefalografia (EEG), a magnetoencefalografia (MEG) e a ressonância magnética (RM), especialmente a RMf, permitiram a monitorização \emph{in vivo} das funções cognitivas, que desempenham um papel crucial e servem de mote para um elevado número de estudos da linguagem. Estas técnicas de imagem, como a RMf e \emph{Diffuse Tensor Imaging} (DTI), são usadas para reduzir a duração da cirurgia e contribuir para o melhor delinear da área a remover. Estas técnicas deverão ser realizadas nos momentos pré-, intra- e pós-cirúrgico. No entanto, na última década um crescente número de autores postula o uso da estimulação cortical direta em contexto intra-cirúrgico, especialmente em neuro-oncologia \citep{Mandonnet2010}.  

A estimulação cortical direta consiste numa corrente elétrica bifásica com uma intensidade de 1-4 mA e uma frequência de 60 Hz. Este procedimento mimetiza uma lesão temporária, não só a nível do córtex, mas também, a nível axonal (subcortical) aquando da colocação do elétrodo diretamente em contacto com a substância branca. Importa salientar que o estímulo fornecido não deverá demorar mais do que 4 s, prevenindo crises epiléticas \citep{Duffau2016}.

A estimulação cortical direta é não-focal, já que a corrente entra em toda a rede neuronal que sustenta determinada função. Se a resposta do paciente durante a estimulação for inadequada, i.e., se cessa o movimento/fala ou responde erradamente, o cirurgião evita remover a região estimulada, pois perceberá que a remoção dessa área causará défices pós-cirúrgicos. Este método é considerado um procedimento \emph{gold-standard} em craniotomia em paciente acordado, na medida em que auxilia na tomada de decisão entre o risco/benefício de ressecar uma determinada área. Assim, tem um impacto positivo no tempo de sobrevida do paciente bem como na sua qualidade de vida, já que permite um maior grau de ressecção do tumor, preservando ao máximo as funções das áreas relevantes para a linguagem, prevenindo que perca algum nível de funcionalidade \citep{Duffau2016}. 

\section{Avaliação da fala e da linguagem}

A avaliação da fala e da linguagem é um dos campos de atuação do TF. Assim, quando esta avaliação for realizada em contexto cirúrgico o TF deverá estar presente na equipa, desempenhando o papel de avaliar estes domínios, auxiliando na preservação dos mesmos. Tê-lo na equipa de neurocirurgia é benéfico dado que a sua interação com o neurocirurgião e a restante equipa é crucial para a otimização funcional das resultantes do processo \citep{Rofes2017}. O seu papel passa por reportar e interpretar os resultados obtidos, não só no contexto intra- mas também pré- e pós-cirúrgico, acompanhando o processo de reabilitação \citep{Geemen2014}. 

No contexto intra-cirúrgico deverá reportar em tempo real as alterações de fala do paciente (e.g., algum grau de disartria associado ou “fala arrastada”), alterações nas praxias e/ou simetrias faciais e as alterações da linguagem, e.g., presença de parafasias, classificando o seu tipo ou perseverações \citep{Geemen2014, Maldonado2011}.

\subsection{Avaliação intra-cirúrgica}

Na cirurgia o paciente desempenha várias tarefas sensório-motoras, visuo-espaci\-ais, de linguagem, cognitivas ou emocionais, simultaneamente à disrupção temporária das estruturas cerebrais usando a electroestimulação. 

Considerando a avaliação prévia (pré-cirúrgica) do caso clínico, quer específica da área, quer em discussão com a equipa, é responsabilidade do TF a seleção das provas a utilizar durante a cirurgia. Estas deverão ter em conta as características do tumor (e.g., localização e representação em termos de funções de fala e linguagem), bem como as características individuais do paciente, e.g., níveis de atividade e participação. Ainda neste âmbito, \citet{Duffau2012} refere que todos os estímulos usados para avaliar o paciente, independentemente da área estimulada e respetiva função associada, devem ter em conta o seu contexto social considerando, por isso, as suas ocupações de tempo livre, profissão, gostos pessoais, entre outros. Desta forma, os estímulos apresentados a um advogado ou a um escritor devem ser adaptados a cada um, já que ao primeiro poderá importar mais a capacidade de julgamento, ao segundo a capacidade de leitura \citep{Duffau2012}.

Ademais, para este tipo de avaliação é importante que os testes utilizados englobem os vários domínios de compreensão e expressão da linguagem e que estejam pensados para as condições vivenciadas no bloco operatório. Assim, o tempo para a apresentação e resposta aos estímulos (os 4 s de estimulação cortical direta), o posicionamento geral do corpo e o facto da cabeça estar fixa na mesma posição durante horas, o limite de espaço para equipamento e o próprio \emph{stress} e desconforto do paciente são aspetos importantes a considerar. Além disto, sa\-be-se que os gliomas de baixo grau são caracterizados por um crescimento lento e tal pode conduzir a processos de neuroplasticidade. Tal facto, associado ao de que estes pacientes apresentam défices menos marcados quando comparados, por exemplo, com os pacientes com afasia adquirida após a ocorrência de um AVC \citep{Desmurget2007, Geemen2014}, enfatiza a necessidade da utilização de um teste sensível que detete pequenas variações. 

Ressalva-se a necessidade da escolha minuciosa dos estímulos, evitando o uso dos elementos de sentido ambíguo e de conotação emocional. 

Posto isto, a avaliação da linguagem deverá abranger a componente expressiva e compreensiva dos seus vários domínios. Assim, a semântica pode ser avaliada através de tarefas que permitam averiguar quanto à capacidade de conceptualização, processamento semântico e evocação, e.g., através de tarefas de associação e compreensão de conceitos, frases e imagens, fluência semântica e nomeação \citep{Friederici2011}. No que concerne à avaliação das capacidades de nomeação, as imagens desenvolvidas por \citet{Snodgrass1980} e as imagens do \emph{Boston Naming Test} \citep{Goodglass2001} são as mais utilizadas \citep{Spezzano2010}. Objetiva-se a nomeação de imagens de traço simples preto num fundo branco \citep{Cervenka2013, Radanovic2004}.

Relativamente ao domínio sintático, segundo \citet{Friederici2011} algumas formas de o avaliar são a apresentação de frases cuja estrutura viola as regras naturais de construção frásica, a apresentação de frases com falta de relações semânticas (e.g., troca do uso de um verbo por elemento de uma categoria morfossintática diferente) e a capacidade de análise de frases cuja reversibilidade não fará sentido. 

Quanto às competências fonológicas, uma das provas mais mencionada para a sua avaliação é a repetição de palavras, já que clinicamente participa na classificação do diagnóstico para afasia, podendo informar sobre o seu prognóstico \citep{Hosomi2009} e orientar a reabilitação \citep{MoritzGasser2013, Schlaug2009}. Esta prova permite também aceder a uma importante rota fonológica que avalia a capacidade de descodificação e codificação dos estímulos \citep{Witte2015}. Além desta, outras provas, sobretudo do domínio metalinguístico são sugeridas, a fim de avaliar a capacidade de consciência fonológica do paciente.

Por fim, a englobar numa avaliação da articulação estão as capacidades de planeamento, coordenação e execução motora de todo o sistema de fala, pelo que se devem considerar quer a produção de sons mais anteriores, que envolvam a musculatura orbicular dos lábios, quer os sons mais posteriores, nos quais outros músculos estão envolvidos \citep{Devadiga2012}. 

\subsection{Um protocolo para a avaliação das competências linguísticas em pacientes com tumores cerebrais}

Ao longo dos últimos dez anos foi desenvolvido por uma equipa de investigadores dos Países Baixos \citep{Witte2015a} um protocolo que possibilita a avaliação pré-, intra- e pós-operatória das competências linguísticas em pacientes com tumores cerebrais, mais especificamente em gliomas de baixo grau, o \emph{Dutch Linguistic Intraoperative Protocol} (\emph{DuLIP}).

Os pacientes permanecem acordados durante a cirurgia, o que permite o mapeamento cerebral por meio da estimulação cortical direta. O DuLIP permite a avaliação neurolinguística que ocorre a par da estimulação supracitada. As tarefas aplicadas são selecionadas considerando a localização do tumor e as suas redes funcionais, uma vez que existem ligações/conexões anatómicas funcionais específicas.

O DuLIP foi adaptado para alguns idiomas como espanhol e italiano. A sua tradução, adaptação cultural e estandardização para o Português Europeu (PE) foi realizada de forma a colmatar a falta de instrumentos de avaliação de linguagem durante este tipo de cirurgia em Portugal \citep{Alves2021}. Resulta deste processo um instrumento que contém 17 tarefas: Uma de nomeação de imagens, quatro fonológicas, sete semânticas, quatro sintáticas e uma articulatória. O número de estímulos varia de acordo com a tarefa e podem ser fornecidos via visual e/ou auditiva. Dentro de cada domínio da linguagem existem tarefas com e sem limite de tempo. Tal deve-se ao facto de algumas tarefas terem de ser executadas durante a estimulação cortical direta, i.e., respeitando o limite de 4 s, no entanto, alguns domínios da linguagem de complexidade superior precisam de ser avaliados sem restrições de tempo.

A tradução e adaptação de instrumentos de linguagem acata alguns desafios a cada domínio da linguagem. Dentro destes destacam-se a diferença cultural entre as línguas, diferenças na possibilidade de acesso a bases de dados com atributos linguísticos (e.g., frequência das palavras e imaginabilidade) e os fonemas específicos de cada língua \citep{Fyndanis2017}. De forma a solucionar esses desafios, mais do que a tradução literal do instrumento foi necessário adaptá-lo culturalmente de acordo com as variáveis que foram consideradas na sua origem.

Primeiramente, foi necessário elaborar uma bateria de tarefas a partir da tradução literal do estudo original, tendo sido fornecidos os estímulos em Neerlandês e Inglês pelos autores originais. A sua tradução literal não se demonstrou adequada para todos os estímulos, tendo parte dela sido adaptada culturalmente. Além disso, uma série de itens não respeitou as características linguísticas na base da criação dos estímulos originais, tendo sido os materiais resultantes ajustados individualmente e adaptados de acordo com propriedades psicométricas específicas.

Embora algumas propriedades fossem transversais aos estímulos usados em cada domínio da linguagem, como os dados de frequência de palavra (dados extraídos do Corpus de Referência do Português Contemporâneo [RCCP]) e idade de aquisição de palavra, outros foram considerados especificamente para cada domínio e/ou prova incluída no mesmo. 

A versão portuguesa do DuLIP (DuLIP-EP) engloba tarefas do domínio da semântica tendo sido necessário ter em conta, como anteriormente mencionado, a frequência da palavra, que se refere ao número de vezes em que é utilizada pelos falantes \citep{Leitao2014, Monsell1989}. Considerou-se, também, a idade de aquisição das palavras. Estes dados encontram-se, para o PE, no estudo de \citet{Ventura2003}, do qual também se fez uso das imagens de traço simples preto, em fundo branco. Estas características de imagem foram adotadas em todas as provas que implicassem o seu uso. No estudo original, foi maioritariamente usada a base de imagens de \citet{Snodgrass1980}, cujo estudo de \citet{Ventura2003} validou para o PE. 

De acordo com as diretrizes usadas no estudo original a capacidade com que uma palavra evoca uma experiência ou seja, a imaginabilidade \citep{Leitao2014}, seria uma variável a ter em conta na adaptação, contudo, esta não foi possível considerar por não ter sido encontrado este registo para PE. Tal também se verificou com a variável de prevalência da palavra, definida como o conhecimento de determinada palavra na população \citep{Brysbaert2016}.

Algumas palavras, imagens e frases tiveram de ser culturalmente adaptadas para o contexto português, e.g., na prova de nomeação, estímulos como umas socas neerlandesas ou uma mitra, foram substituídas por um objeto de uso comum no contexto português, neste caso, por um sapato e um colete, respetivamente.

Tanto em tarefas semânticas como sintáticas, em tarefas com uso de frases foi sempre respeitada e mantida a estrutura sintática, voz ativa/passiva, tempo e modo verbal e o tipo de frase. Os nomes próprios neerlandeses foram substituídos por nomes próprios portugueses. 

Na presença de conceitos ambíguos, houve a necessidade de substituição por palavras que não gerassem duplo sentido, e.g., na tarefa pertencente ao domínio sintático, de geração de verbos, a palavra <mente> foi eliminada.

Face ao exposto, como resultado da adaptação, no domínio semântico não foi possível manter todas as imagens usadas no estudo original. Por esta razão várias imagens foram retiradas do DuLIP-EP na sua primeira versão e substituídas por outras validadas para o PE por \citet{Ventura2003}: Cerca de 50, na tarefa específica de nomeação e 23, nas restantes tarefas semânticas. Sempre que possível as imagens eram trocadas por outras da mesma categoria semântica, mas de forma a não repetir estímulos, sendo por vezes necessário utilizar outras categorias. Imagens de ações não foram incluídas por não se encontrarem validadas para o PE. 

Alguns dos critérios considerados de forma transversal a nível das provas de fonologia foram a frequência de palavra, o número de sílabas e a estrutura silábica e o número de fonemas dos itens alvo. Além do mais, nas tarefas que envolvam estruturas frásicas consideraram-se aspetos como o tipo e modo de frase, tempo verbal, voz ativa ou passiva, semelhanças fonémicas ou aliterações e, ainda, a classe das palavras alvo.

Para a prova de repetição de palavras o principal critério foi o nível de complexidade da palavra, que poderia variar de 1 a 6, consoante o número de sílabas e presença/ausência de grupos consonânticos e similaridades fonémicas. O número final de itens para cada nível de complexidade foi o mesmo do instrumento original. No entanto, as palavras contidas em cada nível tiveram de ser alteradas, dado que a sua tradução literal nem sempre cumpria os critérios do mesmo nível, sendo Português ou Neerlandês. Assim, se a tradução da palavra original cumprisse os critérios designados para aquele nível de complexidade poderia ser mantida nesse mesmo nível (e.g., a palavra neerlandesa <gorilla>, correspondente ao Inglês <gorilla> e com tradução portuguesa de <gorila> pode ser mantida no mesmo nível pois não existem diferenças no que concerne ao número de sílabas, às semelhanças fonémicas ou aos grupos consonânticos). Por outro lado, de forma a manter o máximo de estímulos originais possível, moveram-se para o nível de complexidade adequado, os estímulos originais que, ao serem traduzidos, não enquadrassem o nível onde estavam (e.g., a palavra neerlandesa <tijger>, correspondente ao Inglês <tiger>, e com tradução portuguesa de <tigre>, alterou o nível onde se inseria dado que, embora com o mesmo número de sílabas, a sua tradução portuguesa tem um grupo consonântico não presente na mesma palavra na língua neerlandesa). Algumas palavras foram eliminadas e trocadas por outras, por não corresponderem a qualquer nível de complexidade considerado.

Para o domínio da articulação não se realizaram adaptações nas provas de diadococinésias utilizadas no instrumento original, pois estas cumpriam todos as premissas de uma prova articulatória. No entanto, aquando aplicação do instrumento à população normal, 5 estímulos foram eliminados por serem de difícil articulação pela população portuguesa e ultrapassarem o tempo limite proposto para a prova (4 s por estímulo apresentado).

O uso do DuLIP-EP, como qualquer outra avaliação da fala e da linguagem requer um bom conhecimento de como e o que avaliar, com que propósito e qual a finalidade, assim otimizando o papel do TF numa equipa multidisciplinar atuante neste âmbito. O desenvolvimento deste instrumento foi um passo importante no panorama nacional da avaliação e reabilitação dos pacientes oncológicos, neste caso, a nível cerebral. Potencializa uma melhor qualidade de vida do paciente e uma melhor atuação por parte de quem o usa como instrumento de prática clínica, colmatando uma necessidade existente.

\section{Agradecimentos}
Aos autores originais do DuLIP e à Dra. Catarina Viegas, pela confiança depositada. Este trabalho foi financiado pela Fundação para Ciência e Tecnologia (FCT) no âmbito do projeto UIDB/00127/2020.


{\sloppy\printbibliography[heading=subbibliography,notkeyword=this]}
\end{document}
