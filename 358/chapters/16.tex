\documentclass[output=paper,colorlinks,citecolor=brown,booklanguage=portuguese]{langscibook} 
\ChapterDOI{10.5281/zenodo.7233243}
\title{Avaliação com base em narrativas orais}
\author{Stéphanie Vaz\affiliation{Universidade Nova de Lisboa, Faculdade de Ciências Sociais e Humanas, CLUNL} and Maria Lobo\affiliation{Universidade Nova de Lisboa, Faculdade de Ciências Sociais e Humanas, CLUNL}  and  Marisa Lousada\affiliation{CINTESIS.UA, Escola Superior de Saúde, Universidade de Aveiro}}

\abstract{A avaliação da linguagem oral nas crianças baseia-se frequentemente em instrumentos formais direcionados para componentes específicas como a fonologia, a morfossintaxe ou o léxico, podendo encontrar-se vários instrumentos disponíveis para o português europeu. Para além destes instrumentos, a avaliação da linguagem pode ser complementada com a análise de produções espontâneas da criança ou com a análise de discurso estruturado, como a descrição de imagens ou a produção de narrativas. Este capítulo centra-se na avaliação da linguagem com base em narrativas orais. Descreve-se a relevância de analisar estas produções, considerando diferentes parâmetros macro e microestruturais, bem como a relevância de avaliar a compreensão de narrativas. Apresenta-se ainda um instrumento de avaliação de narrativas que está a ser desenvolvido para o português europeu – o (RE)CONTO, bem como os primeiros resultados da sua aplicação a uma pequena amostra de crianças com e sem desenvolvimento típico.

\textbf{Palavras-chave:} Avaliação da linguagem; Narrativas; Produção oral.} 

\IfFileExists{../localcommands.tex}{
   \addbibresource{../localbibliography.bib}
   % add all extra packages you need to load to this file

\usepackage{tabularx,multicol}
\usepackage{url}
\urlstyle{same}

\usepackage{listings}
\lstset{basicstyle=\ttfamily,tabsize=2,breaklines=true}

\usepackage{langsci-basic}
\usepackage{langsci-optional}
\usepackage{langsci-lgr}
\usepackage{langsci-osl}
% \usepackage{./langsci/styles/langsci-lgr}
% \usepackage{./langsci/styles/langsci-osl}
% \usepackage{langsci-gb4e}

\usepackage{tikz}
\usetikzlibrary{patterns,calc}
\pgfdeclarepatternformonly{south east lines}{\pgfqpoint{-0pt}{-0pt}}{\pgfqpoint{3pt}{3pt}}{\pgfqpoint{3pt}{3pt}}{
    \pgfsetlinewidth{0.6pt}
    \pgfpathmoveto{\pgfqpoint{0pt}{3pt}}
    \pgfpathlineto{\pgfqpoint{3pt}{0pt}}
    \pgfpathmoveto{\pgfqpoint{.2pt}{-.2pt}}
    \pgfpathlineto{\pgfqpoint{-.2pt}{.2pt}}
    \pgfpathmoveto{\pgfqpoint{3.2pt}{2.8pt}}
    \pgfpathlineto{\pgfqpoint{2.8pt}{3.2pt}}
    \pgfusepath{stroke}}
    
\usepackage{stmaryrd}
\usepackage{wasysym}
\usepackage{multirow}
\usepackage{caption}
\usepackage{subcaption}
\usepackage{mathrsfs}
\usepackage{qtree}

\usepackage{linguex}


   %pminos do not split footnotes
% \interfootnotelinepenalty=10000 %Footnote in Laporte chapters has to be split SN


%\DeclareIndexNameFormat{default}{%
%\nameparts{#1}%
%\usebibmacro{index:name}%
%{\index[names]}%
%{\namepartfamily}%
%{\namepartgiveni}%
% {}% L1
% {}% L2
%{\namepartprefix}% generates spurious space L3
%{\namepartsuffix}% generates spurious space L4
%}

%  {\DeclareIndexNameFormat{default}{%
%     \usebibmacro{index:name}{\index[names]}{#1}{#3}{#5}{#7}}}

%\DeclareIndexNameFormat{default}{%
%  \usebibmacro{index:name}{\sindex[nom]}{#1}{#3}{#5}{#7}}

%\DeclareIndexNameFormat{default}{%
%  \usebibmacro{index:name}{\sindex[person]}{#1}{#3}{#5}{#7}}
%\DeclareIndexNameFormat{default}{%
%\nameparts{#1} \usebibmacro{index:name}{\sindex[person]]}{\namepartfamily}{‌​\namepartgiven}{\nam‌​epartprefix}{\namepa‌​rtsuffix}}

%\newcommand{\smiley}{:)}

%\renewbibmacro*{index:name}[5]{%
%\usebibmacro{index:entry}{#1}%
%{\iffieldundef{usera}{}{\thefield{usera}\actualoperator}\mkbibindexname{#2}{#3}{#4}{#5}}}

% \newcommand{\noop}[1]{}

%remove for final
%\overfullrule=1mm

\newcommand{\tobi}[2]}}
\renewcommand{\S}[1]{\tobi{#1}{\textsc{*}}}

% this volume references
% puts: [this volume]
% already defined: \citetv
%\newcommand{\citepv}[1]{(\citeauthor{#1} \citeyear*{#1} [this volume])}
\newcommand{\citealtv}[1]{\citeauthor{#1} \citeyear*{#1} [this volume]}

%parentheses around example number
\newcommand{\pref}[1]{(\ref{#1})}

% in-text examples

\newcommand{\lnex}[1]{\textit{#1}} %target lang word
\newcommand{\lnlit}[1]{(lit.: `#1')} %literal reading
\newcommand{\lnlat}[1]{(#1)} % latinization
\newcommand{\lntrans}[1]{`#1'} %translation
\newcommand{\lnexl}[2]%
{\lnex{#1}{} \lnlat{#2}} % ex with latinization
\newcommand{\lnexlat}[3]{\lnex{#1}{} \lnlat{#2}{} \lntrans{#3}} % ex with latinization and tranl.

%ch01
\newcommand{\co}[1]{\mbox{\textbf{#1}}}

%ch09

\newcommand{\cyrbulg}[1]{\begin{otherlanguage*}{bulgarian}#1\end{otherlanguage*}}


%ch10
\newcommand{\nlp}{{\small NLP}}
\newcommand{\mwe}{{\small MWE}}
\newcommand{\rae}{{\small RAE}}
\newcommand{\lvc}{{\small LVC}}
\newcommand{\pos}{{\small P}o{\small S}}
%\newcommand{\todo}[1]{ \textcolor{red}{#1} }

%\renewcommand{\labelenumi}{\theenumi}
%\ainamefmt{{vv}{ll}{, ff}{, jj}} % fullname

\newcommand{\biberror}[1]{{\color{red}#1}}

\newcommand{\osenovaitem}{--~}
   %% hyphenation points for line breaks
%% Normally, automatic hyphenation in LaTeX is very good
%% If a word is mis-hyphenated, add it to this file
%%
%% add information to TeX file before \begin{document} with:
%% %% hyphenation points for line breaks
%% Normally, automatic hyphenation in LaTeX is very good
%% If a word is mis-hyphenated, add it to this file
%%
%% add information to TeX file before \begin{document} with:
%% %% hyphenation points for line breaks
%% Normally, automatic hyphenation in LaTeX is very good
%% If a word is mis-hyphenated, add it to this file
%%
%% add information to TeX file before \begin{document} with:
%% \include{localhyphenation}
\hyphenation{
    Beck-man
    Ngu-yen
    back-chan-nel
    back-chan-nels
    mo-not-o-nous
    ste-reo-typ-i-cal
}

\hyphenation{
    Beck-man
    Ngu-yen
    back-chan-nel
    back-chan-nels
    mo-not-o-nous
    ste-reo-typ-i-cal
}

\hyphenation{
    Beck-man
    Ngu-yen
    back-chan-nel
    back-chan-nels
    mo-not-o-nous
    ste-reo-typ-i-cal
}

   \boolfalse{bookcompile}
   \togglepaper[16]%%chapternumber
}{}

\begin{document}
\maketitle

\section{Por que é relevante fazer uma avaliação com base em narrativas orais?}
A recolha de discurso espontâneo, não planeado e em situação de diálogo, permite analisar diferentes domínios, incluindo o fonológico, o lexical e o sintático.\footnote{A norma adotada na escrita deste capítulo foi a do português europeu.}
% \footnote{Na sequência de restrições decorrentes do template usado para publicação na editora Language Science Press, os capítulos que integram o volume surgem com a referência [this volume], em inglês.}
Contudo, este contexto de produção não é condicionado e é, por conseguinte, mais sensível a variáveis extralinguísticas, como a situação de produção e os intervenientes \citep{Demuth1996, Stromswold1996}. É, contudo, através de discurso espontâneo que mais facilmente se avaliam competências pragmáticas e discursivas (\citealp{Condouris2003}; e.o.). A variabilidade discursiva existente na situação de produção espontânea, porém, não permite tão facilmente o estabelecimento de normas e a comparação entre crianças, sendo a sua análise difícil \citep{Kover2014}. Para além de requerer uma capacidade de análise linguística fina, a recolha de discurso espontâneo não permite estudar a compreensão. Assim, a avaliação da linguagem oral recorre frequentemente a instrumentos formais, havendo diferentes instrumentos disponíveis para o português europeu \citep{Viana2017a}. Estes instrumentos podem focar-se apenas num domínio, que avalia especificamente a sintaxe, ou podem contemplar vários domínios, incluindo o fonológico, o sintático e o lexical. Os testes psicométricos estandardizados permitem avaliar quer a produção (linguagem expressiva), quer a compreensão (linguagem recetiva), avaliando a linguagem oral através de itens que incidem sobre construções linguísticas específicas. Não permitem, contudo, uma avaliação de dimensões mais globais necessárias na produção e compreensão de textos, como mecanismos globais de coesão e coerência, a capacidade de estabelecer relações temporais, a capacidade de estabelecer relações de causa-efeito, a capacidade de estabelecer a progressão temática, entre outros aspetos. Por conseguinte, como complemento à avaliação formal através de provas estandardizadas, pode recorrer-se à avaliação de narrativas orais para avaliar as competências linguísticas da criança, considerando dimensões que as baterias de avaliação da linguagem geralmente não contemplam.

As narrativas desempenham um papel importante a nível cognitivo, social e académico. De acordo com vários autores, a nível cognitivo, são uma forma de estruturação do pensamento e contribuem para consolidar experiências na memória de longo prazo \citep{Wang2015, Leon2016}; a nível social, contribuem para consolidar laços afetivos, quer através de histórias de vida pessoais, quer através de narrativas de tradição oral, desempenhando um papel na identidade de uma comunidade \citep{Labov2010}; e, a nível académico, desempenham um papel importante na aprendizagem da leitura e da escrita. Assim, uma das vantagens de recorrer à avaliação da linguagem através de narrativas orais vem do facto de a produção e a compreensão de textos narrativos serem frequentemente trabalhadas em meio escolar, ainda que sobretudo na modalidade escrita \citep{Pereira2005}. A identificação de dificuldades na produção oral de textos narrativos e na compreensão de textos narrativos orais permite antecipar dificuldades que se manifestarão mais tarde na modalidade escrita \citep{Dickinson2009, Villiers2010, Klop2011}.

A avaliação da linguagem oral através de narrativas constitui, assim, um importante complemento à avaliação da linguagem através de outros instrumentos, permitindo uma avaliação global das competências linguísticas e comunicativas da criança e a análise de diferentes dimensões \citep{Goncalves2011a}. Na próxima secção, faz-se uma caracterização geral de narrativas e descrevem-se as diferentes dimensões que podem ser analisadas através de textos narrativos.

Na secção \ref{sec:doiscap16}, justificamos a relevância de fazer uma análise com base em narrativas orais e, na secção \ref{sec:trêscap16}, descrevemos diferentes dimensões que podem ser avaliadas com base em narrativas orais. Na secção \ref{sec:quatrocap16}, fazemos uma revisão panorâmica de vários instrumentos estandardizados de avaliação de narrativas disponíveis para outras línguas, em particular para o inglês, e, na secção \ref{sec:cincocap16}, damos conta de um instrumento que se encontra em desenvolvimento para o português europeu – o (RE)CONTO. Nas secções \ref{sec:seiscap16} e \ref{sec:setecap16}, apresentamos alguns resultados de investigação sobre o desenvolvimento de competências narrativas por crianças com desenvolvimento típico e atípico, considerando a Perturbação do Desenvolvimento da Linguagem (PDL), em particular, e, na secção \ref{sec:oitocap16}, apresentamos alguns dados comparativos entre crianças com desenvolvimento típico e crianças com PDL na produção e compreensão de narrativas, obtidos através do (RE)CONTO. 

\section{O que se pode avaliar através de narrativas?}\label{sec:doiscap16} \subsection{Características de narrativas} As narrativas são relatos de acontecimentos reais ou imaginários. Incluem-se entre os textos narrativos relatos de acontecimentos pessoais, relatos de acontecimentos vividos por outros, narrativas ficcionais, como contos, romances ou parábolas, entre outros. Uma narrativa prototípica contém uma situação inicial, uma complicação, um conjunto de acontecimentos que visam repor o estado de equilíbrio, a resolução e a situação final ou desfecho \citep{Adam1982, Sousa2010}.

A classificação de um texto como sendo narrativo não quer dizer que ele só inclua sequências narrativas. \citet{Adam1982, Adam1987} propõe que um texto é constituído por sequências que podem ser de vários tipos: narrativas, descritivas, dialogais, explicativas e argumentativas. Um texto pode conter várias destas sequências. Um texto narrativo será um texto em que as sequências narrativas são predominantes, mas pode conter sequências de outros tipos, como sequências dialogais e descritivas, por exemplo. A complexidade de um texto narrativo poderá depender, entre outras coisas, da quantidade de sequências narrativas que este inclui, da sua articulação com sequências de outros tipos, da sua maior ou menor prototipicidade enquanto texto narrativo. Assim, um texto narrativo pode conter uma única sequência narrativa ou múltiplas sequências narrativas. Cada sequência incluirá uma sucessão de acontecimentos ligados por relações de temporalidade e de causalidade \citep{Sousa2010}.

Cada texto pode ser considerado numa dimensão macroestrutural e numa dimensão microestrutural \citep{Beaugrande1983, Dijk1997}, que estão interligadas \citep{Liles1995, Pearson2002}. A dimensão macroestrutural está relacionada com a organização temática do texto e a dimensão microestrutural está relacionada com as estruturas linguísticas usadas (lexicais, morfológicas, sintáticas e semânticas). Ambas as dimensões podem ser avaliadas para determinar a complexidade da narrativa \citep{Westby2005, Monteiro2016}. Consideramos nos próximos pontos cada uma dessas dimensões.

\subsection{Dimensão macroestrutural} A macroestrutura de um texto diz respeito à sua organização temática, às componentes que possibilitam a organização global de sentido do texto e que são responsáveis pela sua significação \citep{Beaugrande1983}. Da macroestrutura do texto, de natureza semântica e cognitiva, fazem parte as representações globais que asseguram a coerência do texto \citep{Sautchuk2003} e que permitem identificar as suas partes constituintes.

Nas narrativas ficcionais, encontra-se geralmente um esquema mais específico com uma situação inicial, que inclui uma localização temporal e espacial e a introdução de personagens, um evento perturbador, uma tentativa de resolução e um resultado dessa ação. Dependendo da maior ou menor complexidade das narrativas, pode haver uma só destas sequências ou várias sequências. O número de sequências narrativas do texto e a inclusão de outras sequências (descritivas ou dialogais, por exemplo) pode contribuir para a complexidade da macroestrutura textual.

A análise macroestrutural considera a coerência da narrativa, incluindo a presença e a organização das partes constituintes da narrativa, com referência aos elementos relevantes da situação inicial, à sequência de eventos ligados por relações de causalidade, manutenção de personagem e tema, e relação entre eventos e desfecho \citep{Miles2002, Heilmann2010}. Diferentes instrumentos de avaliação de narrativas orais usam grelhas de codificação distintas para a análise da componente macroestrutural, podendo verificar se a criança refere: i) a abertura da narrativa, incluindo os elementos do cenário - localização espaciotemporal e diferentes personagens; ii) os fatores que desencadeiam a sucessão de eventos; iii) cada um dos eventos constituintes da narrativa (ação e reação); e iv) o desfecho da narrativa.

\subsection{Dimensão microestrutural} A dimensão microestrutural de um texto engloba o conjunto dos elementos linguísticos, incluindo propriedades lexicais, semânticas, sintáticas e morfológicas, que contribuem para a sua coesão e coerência \citep{Sautchuk2003}. A coesão refe\-re-se às relações estabelecidas através de elementos gramaticais e lexicais que garantem a continuidade de sentido nas dimensões temporal, referencial e lexical \citep{Halliday1976}.

A coerência e a coesão de uma narrativa estão interligadas e a aquisição da narrativa envolve o desenvolvimento integrado desses diferentes aspetos de forma \citep{Halliday1976, Justice2010}.

A análise microestrutural permite avaliar o desenvolvimento linguístico da criança e pode englobar várias componentes linguísticas: i) lexicais – incluindo, por exemplo, a diversidade lexical e o número de palavras diferentes; ii) morfológicas – incluindo, por exemplo, diversidade de morfemas flexionais usados e palavras gramaticais; iii) sintáticas e semânticas – incluindo medidas globais, como a extensão média e máxima de enunciado, ou medidas mais específicas, como a quantidade e diversidade de conectores usados, o número ou a proporção de frases complexas, os tipos de frases complexas produzidos, os tempos verbais produzidos e a coesão no uso de tempos verbais, as marcas de coesão referencial e a manutenção de cadeias de referência, entre outros aspetos.

A análise microestrutural pode considerar uma ampla gama de aspetos linguísticos, incluindo medidas gerais de produtividade e de complexidade linguística, extensão e diversidade lexical, presença de marcação de relações temporais e causais entre eventos expressas através de conectores, congruência dos eventos narrados, entre outros aspetos. Pode também ter em conta desvios relativamente à forma alvo, incluindo, por exemplo, o número de frases agramaticais e o tipo de desvios encontrados.

\subsection{Compreensão} Para além da produção, é possível avaliar a compreensão de narrativas orais. Os processos envolvidos na compreensão de narrativas orais têm mostrado ser determinantes também na compreensão da leitura \citep{Paris2003}. Muitos instrumentos de avaliação de narrativas contêm não só tarefas de produção, mas também tarefas de compreensão. É avaliada a capacidade de a criança compreender uma história narrada, com ou sem o apoio de imagens. Os processos de compreensão (quer da leitura, quer de discurso oral) são muito complexos, envolvendo diferentes planos, incluindo aspetos do código de superfície (microestrutura), de base textual e de construção de um modelo de situação (macroestrutura), que requerem a capacidade de fazer inferências e de relacionar a informação ouvida/lida com os conhecimentos prévios do leitor \citep{Graesser1997}. Os estudos que incidem sobre a compreensão na leitura têm identificado diferentes processos implicados na compreensão, tendo sido propostas várias tipologias \citep{Catala2001, Giasson2005, Ribeiro2010}. Também na compreensão do discurso oral têm sido considerados diferentes processos. Seguindo a nomenclatura de \citealt{Ribeiro2010}, consideram-se aqui processos de compreensão literal, compreensão inferencial e compreensão crítica. A compreensão inferencial, em particular, tem sido explorada em tarefas de compreensão de narrativas orais \citep{Paris2003, Dawes2017}. Alguns estudos têm mostrado a importância de avaliar a compreensão inferencial, que está frequentemente comprometida em crianças com PDL, comparando-a com processos de compreensão literal, bem como com outras dimensões, incluindo tarefas de reconto e conhecimento lexical, que são preditores de sucesso na compreensão inferencial \citep{Dawes2017}.

A compreensão literal corresponde à compreensão de informação explicitamente referida no texto; a compreensão inferencial já requer a capacidade de ir para além da informação expressa no texto, através de processos dedutivos, baseados em raciocínios lógicos a partir de indícios fornecidos no texto e conhecimentos prévios do ouvinte; a compreensão crítica envolve a capacidade de formular juízos próprios e de fazer uma avaliação subjetiva acerca de atitudes, personagens, linguagem, entre outros aspetos \citep{Ribeiro2010}. 

Note-se que, nas tarefas de avaliação da narrativa, produção e compreensão não estão completamente dissociadas. A compreensão é avaliada através da resposta oral da criança, pelo que a tarefa de compreensão requer também produção. Inversamente, a tarefa de reconto, que é sobretudo uma tarefa de produção, requer a compreensão da narrativa escutada. Esta tarefa consiste, assim, numa tarefa multimodal, que requer produção, mas também compreensão: para recontar de forma fiel uma narrativa escutada, a criança terá de a compreender.

\largerpage[2]
\section{Que instrumentos existem para avaliação de narrativas noutras línguas e o que avaliam?}\label{sec:trêscap16} Existem vários instrumentos dedicados à avaliação de narrativas para outras línguas, em particular para o inglês, ainda que nem todos estejam normalizados.

O conto \emph{Frog, Where Are You}? \citep{Mayer1969} é um dos materiais que foi mais usado a nível internacional com a finalidade de desencadear a produção de narrativas, tendo dado origem a inúmeros trabalhos sobre o desenvolvimento de narrativas infantis \citep{Berman1994}. Algumas das recolhas obtidas a partir desta história estão disponíveis no corpus CHILDES de acesso livre (\url{https://childes.talkbank.org/access/Frogs/}).

Existem para outras línguas, e em particular para o inglês, vários instrumentos de avaliação de narrativas orais usados em contexto clínico. Entre estes, encon\-tram-se instrumentos que avaliam a produção de narrativas, instrumentos que avaliam a compreensão de narrativas, e instrumentos que avaliam quer a produção, quer a compreensão, entre os quais estão os seguintes: \emph{The Bus Story} \citep{Renfrew1969}; \emph{Test of Narrative Language} (TNL; TNL2) \citep{Gillam2017}; \emph{Expression, reception and recall of narrative instrument} (ERRNI) \citep{Bishop2004}; \emph{Edmonton Narrative Norms Instrument} (ENNI) \citep{Schneider2005}; \emph{Diagnostic Evaluation of Language Variation - Norm Referenced Test} (DELV-NR) \citep{Seymour2005}.\footnote{Existem vários outros instrumentos não normalizados que são usados em contexto clínico e/ou em contexto educativo para avaliar a produção e a compreensão de narrativas e/ou para estimular o desenvolvimento de competências associadas: \emph{Peter and the Cat Narrative Assessment} \citep{Leitao2003}; \emph{The Squirrel Story Narrative Assessment} \citep{Carey2006}; \emph{Peter and the Cat Narrative Comprehension Assessment; Squirrel Story Narrative Comprehension Assessment} \citep{Dawes2019}). Ainda que não esteja normalizado, o \emph{Multilingual Assessment Instrument for Narratives} (MAIN) \citep{Gagarina2015}, que foi desenvolvido no âmbito da Ação COST IS0804 especialmente para crianças bilingues, tem versões disponíveis para um conjunto alargado de línguas.}

Estes instrumentos distinguem-se quanto a uma série de aspetos. As faixas etárias alvo diferem: a idade mínima varia entre os 3 e os 5 anos e a idade máxima varia entre os 8 e os 15 anos. Nem todos os instrumentos contemplam as mesmas dimensões: \emph{The Bus Story} só avalia produção; os restantes contemplam tanto produção como compreensão. A compreensão inclui processos de compreensão literal e inferencial. A forma como é induzida a narrativa é variável: através de conto, de reconto, incluindo ambas as tarefas, com ou sem o apoio de imagens. Estes instrumentos diferem ainda quanto à diversidade de narrativas contempladas: \emph{The Bus Story} inclui uma única narrativa, os restantes incluem várias narrativas. Apesar das diferenças, todos eles têm como procedimento comum uma tarefa de reconto, com ou sem imagens de suporte que ilustram a sequência de ações que constituem a narrativa.

A codificação e análise diferem de prova para prova. No entanto, incluem habitualmente a análise da macroestrutura (ou esquema narrativo) e a análise de elementos linguísticos, incluindo a análise de mecanismos de coesão. 

\largerpage
Para o português europeu, está em desenvolvimento um instrumento de avaliação de narrativas, que será descrito na próxima secção.\footnote{Este instrumento está a ser desenvolvido no âmbito do projeto de doutoramento de Stéphanie Vaz, sob orientação de Maria Lobo e de Marisa Lousada, com financiamento da Fundação para a Ciência e a Tecnologia (PD/BD/131387/2017).}\clearpage


\section{(RE)CONTO: Uma proposta de instrumento de avaliação de narrativas para o português europeu}\label{sec:quatrocap16} Dada a inexistência de um instrumento de avaliação de narrativas normalizado para o português europeu, que possa ser usado por terapeutas da fala ou professores para complementar a avaliação da linguagem da criança e auxiliar na planificação da intervenção, foi desenvolvido o instrumento (RE)CONTO, que está em processo de validação. Este instrumento contempla tarefas de produção e de compreensão de narrativas e é pensado para crianças falantes de português europeu com idades compreendidas entre os 6 e os 12 anos.

O instrumento é composto por três histórias, correspondendo todas elas a narrativas ficcionais, uma delas original – “A Minhoca e o Pássaro” – e as outras duas adaptadas de livros infantis já publicados: “Os Elefantes Espertalhões”, adaptada de \emph{Chibos Sabichões} \citep{Gonzalez2016}, e “A Fada Arco-Íris”, traduzida e adaptada do livro infantil espanhol \emph{El hada del arco iris} \citep{Yebras2015}. Estas três histórias foram manipuladas para o trabalho experimental, apresentando diferentes graus de complexidade quanto às sequências de eventos e às estruturas linguísticas. Das três histórias, “A Minhoca e o Pássaro” é a que apresenta uma estrutura narrativa mais convencional, com sequências predominantemente narrativas, e a que tem um menor grau de complexidade linguística, com menor número de orações complexas. A história “Os Elefantes Espertalhões” contém não só sequências narrativas, mas também uma sequência descritiva, que inicia a história, e várias sequências dialogais. A história “A Fada Arco-Íris” é a que contém estruturas linguísticas mais complexas, com um maior número e mais diversificado de orações subordinadas, e é a que apresenta relações mais complexas entre as personagens e sequências de eventos, contendo também algumas sequências descritivas. 

A cada história estão associadas uma gravação áudio e sequências de imagens ilustrativas criadas por uma \emph{designer} profissional, de que damos um exemplo na Figura \ref{fig:cap16fig1}:
\begin{Figura}
    \includegraphics[width=\linewidth]{figures/62.png}

    \caption{Sequência de imagens da história “A Minhoca e o Pássaro”}
    \label{fig:cap16fig1}
\end{Figura}

Para cada história, foram criadas perguntas de compreensão que avaliam a compreensão literal, inferencial e crítica. Estas perguntas de compreensão só se aplicam na tarefa de reconto.

O instrumento pode ser aplicado em suporte digital ou em papel. O instrumento é de aplicação individual, devendo todas as produções ser gravadas em áudio e/ou vídeo.

Embora cada sequência de imagens tenha sido testada como estímulo a conto ou como suporte a situação de reconto a diferentes grupos de crianças, no instrumento final optou-se por usar as três histórias em situação de reconto, e unicamente a história “A Minhoca e o Pássaro” como estímulo a conto. Na tarefa de conto, mostra-se à criança a sequência de imagens da história e pede-se que a conte. Na tarefa de reconto, mostra-se a sequência de imagens da história, a história é lida (ou apresentada em formato vídeo). Em seguida, aplicam-se as perguntas de compreensão e, no fim, pede-se à criança que reconte a história. 

No processo de validação do instrumento, foi inicialmente levado a cabo um estudo piloto com um grupo de 38 crianças com desenvolvimento típico, o que permitiu afinar as perguntas de compreensão e os procedimentos a aplicar. Numa fase seguinte, o instrumento e o respetivo manual foram enviados para um painel de sete peritos (profissionais das áreas da Terapia da Fala, Didática e Linguística) para validação de conteúdo, tendo os resultados do painel de peritos demonstrado um índice de validade de conteúdo global de 0.94.

Na validação final do instrumento já revisto com base no estudo piloto e nas observações dos peritos, participaram 623 crianças, com idades compreendidas entre os 6 e os 12 anos, residentes em seis distritos de Portugal Continental (Lisboa, Leiria, Santarém, Portalegre, Guarda e Aveiro) e nas ilhas (Açores e Madeira). As crianças foram caracterizadas segundo idade cronológica, escolaridade, região e género.

As produções das crianças estão a ser transcritas e analisadas de acordo com uma grelha de critérios de desempenho, que contempla parâmetros macroestruturais e microestruturais. A grelha de análise macroestrutural contempla parâmetros relativos às grandes partes constituintes da narrativa (abertura da narrativa; organização e progressão temática da narrativa; desfecho) e parâmetros específicos relativos à localização espacial, introdução de personagens, situação inicial, sequência de eventos, problema e desfecho. Na grelha de análise microestrutural, estão contempladas: a) medidas de produtividade linguística da narrativa, referentes ao número total de palavras, número total de palavras diferentes e número de enunciados; b) medidas globais de complexidade linguística da narrativa, que incluem extensão média e máxima de enunciados em palavras; e c) medidas de complexidade sintático-semântica, que incluem número e proporção de orações simples e complexas, tipo de frases complexas, número e diversidade de orações adverbiais e de conetores que expressam relações de tempo, causa e consequência. Considera-se ainda a manutenção de coesão referencial e eventuais desvios, como problemas de concordância nominal ou verbal, uso de flexão regular e/ou irregular, uso inadequado de tempo e modo, retoma pronominal ou uso de artigos.

\section{O desenvolvimento de competências de produção e compreensão de narrativas no desenvolvimento típico}\label{sec:cincocap16} Há uma extensa investigação sobre o desenvolvimento da produção e da compreensão de narrativas orais. Para o português europeu, diferentes trabalhos \citep{Batoreo1997, EncarnacaoSilva2001, Sousa2007, Sousa2010} focam diferentes dimensões da produção de narrativas, como a expressão do tempo e a expressão do espaço. O corpus de narrativas orais recolhidas por \citet{Batoreo1997} está disponível na base CHILDES (\url{https://childes.talkbank.org/}), e serviu de base à análise do desenvolvimento de produções narrativas de crianças apresentada em \citet{Goncalves2011a}.

As crianças começam desde cedo a relatar sequências de acontecimentos, tor\-nando-se a pouco e pouco capazes de contar narrativas ficcionais progressivamente mais complexas e completas \citep{Silva2000, Bento2010}. O desenvolvimento observado nas narrativas orais é replicado em idade escolar nas narrativas escritas \citep{EncarnacaoSilva2001}.

A nível macroestrutural, as primeiras narrativas das crianças podem não apresentar todas as componentes prototípicas de uma sequência narrativa, faltando, por exemplo, o desfecho ou apresentando uma situação inicial incompleta. Cerca dos 7 anos, a maioria das histórias contadas já contém um evento inicial, ações com vista a um fim e uma consequência \citep{Gillam2017}. A extensão das narrativas vai aumentando com a idade, bem como a complexidade das situações relatadas \citep{Sousa2007, Sousa2010}. No final do ensino primário, as crianças já são capazes de produzir narrativas mais complexas, com múltiplas sequências narrativas e motivações de personagens mais elaboradas \citep{Gillam2017}.

A nível microestrutural, as primeiras narrativas caracterizam-se por fazerem um uso pouco diversificado de conetores, contendo um uso repetido da conjunção \emph{e} e poucas estruturas de subordinação \citep{Goncalves2011a}. Quando surgem estruturas de subordinação adverbial, encontram-se sobretudo orações temporais introduzidas por \emph{quando}, sendo outros tipos de estruturas, como as causais e condicionais, mais raros \citep{EncarnacaoSilva2001, Goncalves2011a}. A complexidade linguística das narrativas continua a desenvolver-se até à adolescência, com a capacidade de encaixar múltiplas orações, vocabulário mais rico e um estilo mais literário \citep{Gillam2017}.

Os mecanismos coesivos estão também sujeitos a desenvolvimento, incluindo os processos de retoma anafórica. Cerca dos 7 anos de idade, as crianças já são capazes de usar adequadamente os artigos indefinidos quando mencionam pela primeira vez uma entidade e artigos definidos ou expressões pronominais quando a retomam, mostrando ser capazes de dominar mecanismos de coesão referencial \citep{Goncalves2011a}. Também os mecanismos de coesão temporal, incluindo o uso de tempos verbais, de conetores temporais e de expressões adverbiais de localização temporal se vão tornando progressivamente mais complexos \citep{Sousa2007}.

O estabelecimento de relações causais está ausente das narrativas de crianças em idade pré-escolar, sendo de desenvolvimento relativamente tardio \citep{Sousa2007, Sousa2010}. As primeiras narrativas das crianças correspondem a justaposições de eventos cronologicamente ordenados. A presença de conetores causais que explicitam a relação entre eventos e que garantem a coerência da narrativa é um indicador de desenvolvimento da competência narrativa \citep{Sousa2010}.

No que diz respeito à compreensão, a investigação tem demonstrado que as perguntas que requerem que a criança relacione a informação explícita com conhecimentos prévios, fazendo inferências e juízos avaliativos, podem colocar dificuldades às crianças com idades inferiores a 5-6 anos \citep{Dawes2017}. Também a compreensão de frases complexas pode colocar problemas sobretudo quando a ordenação temporal das situações não corresponde à ordem linear por que aparecem na frase \citep{Goncalves2011a}.

A investigação tem mostrado também que o desempenho das crianças é sensível ao tipo de tarefa exigida: as crianças têm desempenhos melhores em tarefas que envolvem reconto do que em tarefas de conto (\citealp{Nippold2005, Cavalcante2010, Andrade2017, Balciuniene2016, Vaz2020}; entre outros).

Assim, o desenvolvimento de competências narrativas é um processo gradual e complexo, que mobiliza conhecimentos de múltiplos níveis, que se inicia em idades precoces, mas que continua a desenvolver-se até à adolescência.

\section{O desenvolvimento de competências de produção e compreensão de narrativas no desenvolvimento atípico: O que está alterado?}\label{sec:seiscap16} Esta secção contempla a análise de dados de crianças com perturbação da linguagem no que diz respeito às competências de produção e compreensão de narrativas.

É de referir que existem muitos estudos, sobretudo para outras línguas, centrados na produção e compreensão de narrativas orais, em crianças com desenvolvimento de linguagem atípico, quer com PDL,\footnote{Termo atualmente utilizado para referir as perturbações de linguagem, que ocorrem sem causa aparente \citep{Bishop2017b}. Os termos anteriormente utilizados para referir estas perturbações de linguagem eram Perturbação Específica da Linguagem (PEL) ou Perturbação Específica do Desenvolvimento Linguístico (PEDL).} quer com outro tipo de perturbações de linguagem associadas a diferentes condições biomédicas, tais como: Síndrome de Down \citep{Bysterveldt2012, Segal2015}; Perturbação do Desenvolvimento Intelectual \citep{Bunning2016}; X Frágil \citep{Estigarribia2011, Finestack2012}; Síndrome de Williams \citep{Marini2010}; Perturbação do Espectro do Autismo \citep{Rumpf2012, Gillam2015, Baixauli2016}; Surdez \citep{Crosson2001, Boons2013}; Paralisia Cerebral \citep{Nordberg2015}, entre outras. Em todos estes estudos, foram encontradas dificuldades em produzir narrativas coerentes e coesas, estando as competências macro e microestruturais abaixo do esperado. A nível macroestrutural, podem encontrar-se referências incompletas a personagens, um número inferior de referência a episódios, uma organização menos coerente. A nível microestrutural, pode encontrar-se vocabulário mais restrito, mas sobretudo um número inferior de frases complexas, mais erros gramaticais e problemas de coesão. Na maioria das perturbações, a capacidade de produzir narrativas orais está correlacionada com o desempenho em tarefas de compreensão e produção da linguagem, bem como com funções de memória \citep{Boudreau2008}.
A título de exemplo, as crianças com Perturbação do Espectro do Autismo têm desempenhos mais baixos nos parâmetros de análise macroestrutural. Produzem enunciados com uma extensão média mais baixa e com um menor número de radicais de palavras diferentes, e apresentam dificuldades específicas com o uso da gramática e do vocabulário nas suas produções narrativas \citep{King2013}. As crianças com Perturbação do Espectro do Autismo mostram ainda limitações especialmente no que diz respeito à extensão e à coerência da narrativa \citep{Rumpf2012}.

Já as crianças com Surdez parecem apresentar dificuldades, sobretudo na compreensão de narrativas orais \citep{Crosson2001}. Estudos com crianças com implantes cocleares mostram que estas crianças apresentam bons resultados na quantidade e coerência das suas produções narrativas, mas demonstram problemas na qualidade, no conteúdo e eficácia das histórias recontadas \citep{Boons2013}. 

As crianças com Síndrome de Down têm perfis heterogéneos, en\-contrando-se em geral uma extensão média de enunciado baixa e uma diferenciação lexical baixa \citep{Boudreau2000}. Nas crianças com Síndrome de Down, parece existir uma estreita correlação entre as crianças que produzem narrativas mais coesas e coerentes e as competências de leitura e de escrita \citep{Bysterveldt2012}.

As crianças com PDL produzem maioritariamente frases simples, e menos frases complexas, com poucas orações subordinadas. Os seus enunciados são mais reduzidos e não são mencionados todos os eventos da história, sendo que produzem um menor número de episódios completos. Algumas crianças apresentam erros gramaticais, mais frases agramaticais, apresentando dificuldades na ordenação temporal, entre outros aspetos. Apresentam com menos frequência um discurso do tipo intencional, preferindo um discurso mais descritivo. Tendem ainda a omitir clíticos e marcadores temporais, mostram dificuldades no estabelecimento de cadeias referenciais e, por fim, manifestam pouca coesão na produção das suas narrativas \citep{Fey2004, BefiLopes2008}.

Vários autores têm dedicado os seus estudos à narrativa, comparando as capacidades e competências de crianças com e sem perturbações de linguagem (\citealp{Gillam1992, ContiRamsden2003, Kraljevic2020, Favot2020}; entre outros). Os resultados mostram que as competências narrativas nas crianças com PDL se desenvolvem ao longo do tempo, mas estão abaixo das crianças com desenvolvimento típico com idades semelhantes \citep{Reuterskioeld2011, Kraljevic2020, Favot2020}. Os estudos têm mostrado diferenças significativas nos dois grupos de crianças (crianças com desenvolvimento típico e atípico) no que se refere a tarefas como conto e reconto de histórias, verificando-se geralmente uma maior facilidade nas tarefas de reconto de histórias \citep{Kraljevic2020}.

A compreensão oral de narrativas, em particular os processos que requerem capacidade de fazer inferências e capacidade de formular juízos críticos, tem mostrado estar comprometida em crianças com perturbações da linguagem (\citealt{Bishop1992, Gillam2016, Crais1987, Gillam2009, Dawes2017}; e.o.).

Assim, a avaliação das competências narrativas permite completar o perfil linguístico da criança, auxiliando os terapeutas da fala na avaliação, diagnóstico, planeamento e respetiva intervenção terapêutica. Para além disso, permite eventualmente antecipar dificuldades que se possam vir a manifestar na escrita de narrativas e na compreensão da leitura \citep{Catts2015, Pinto2016}. Assim, a avaliação de narrativas orais tem sido incluída a par de outros instrumentos de avaliação da linguagem, contemplando métodos de análise quantitativos e qualitativos \citep{Justice2010, Petersen2008}.

\section{Alguns dados comparados entre crianças com desenvolvimento típico e crianças com perturbação da linguagem}\label{sec:setecap16} Nesta secção, apresentam-se alguns resultados de um estudo com três crianças com PDL, duas do sexo masculino com seis anos e nove meses e nove anos e oito meses, e uma criança do sexo feminino com doze anos e cinco meses (\emph{cf}. Tabela \ref{tab:cap16tab1}). Estas crianças são falantes de português europeu, vivem na zona centro do país, no distrito de Santarém, e frequentam terapia da fala há mais de três anos. Os resultados destas três crianças foram comparados com os de três crianças com desenvolvimento da linguagem típico (DLT) com as mesmas idades e da mesma região. Todas as crianças estudadas (crianças com desenvolvimento típico e com desenvolvimento atípico) foram sujeitas a uma avaliação de produção oral de histórias quer em tarefa de conto, quer em tarefa de reconto de histórias, e também a perguntas de compreensão. Neste capítulo mostraremos os resultados obtidos através de duas histórias do instrumento (RE)CONTO, a Minhoca e o Pássaro e a história dos Elefantes Espertalhões, para a tarefa de reconto em ambos os grupos de crianças. Após esta última tarefa, foram realizadas as perguntas de compreensão. As produções das narrativas feitas pelos dois grupos de crianças (com as mesmas faixas etárias e mesmo género) foram comparadas relativamente a parâmetros macroestruturais e microestruturais.
\begin{Tabela}
    \caption{{Características das crianças por grupo, sexo, escolaridade e faixa etária.}}
    \label{tab:cap16tab1}

    \begin{tabularx}{.8\textwidth}{Xlllr}
    \lsptoprule
 &\textbf{Sexo}& \textbf{Escolaridade} &\textbf{Idades}  \\
 \midrule
  \multirow{3}{*}{\textbf{GRUPO PDL}}&  L.&	M&	1\textsuperscript{o} ano&	6;9\\
  & M.&	M	&4\textsuperscript{o} ano&	9;8\\
&V.&	F	&6\textsuperscript{o} ano&	12;5\\
\tablevspace
\multirow{3}{*}{\textbf{GRUPO DLT}} & J.&	M&	1\textsuperscript{o} ano&	6;8\\
& G.&	M&	4\textsuperscript{o} ano&	9;9\\
&S.&	F	&5\textsuperscript{o} ano&	11;0\\
\lspbottomrule
\end{tabularx}
\end{Tabela}

\subsection{Produções orais nas tarefas de conto e reconto} Todas as produções das crianças foram transcritas de acordo com o formato Chat \citep{MacWhinney2000} e analisadas de acordo com a grelha de análise do (RE)CONTO (ver secção \ref{sec:cincocap16}). Na Figura seguinte apresentaremos as taxas de acerto obtidas para os dois grupos estudados, 3 crianças com PDL e 3 crianças com DLT, na tarefa de reconto de histórias.
\begin{Figura}
%     \includegraphics[width=\linewidth]{figures/63.png}

        \pgfplotstableread{
1 100 56 67
2 100 67 75
3 100 89 92
4 100 89 92
5 100 100 100
6 100 89 92
}\dataset
    \begin{tikzpicture}
\begin{axis}[ybar,
        width=\textwidth,
        height=.3\textheight,
        ymin=0,
        xtick=data,
        xlabel = {GRUPO PDL \hspace*{1.5cm}|\hspace*{1.5cm} GRUPO DLT},
%         enlarge x limits=.5,
        bar width=4mm,
        nodes near coords,
        nodes near coords style={font=\footnotesize},
        legend cell align=left,
        legend style={font=\footnotesize,at={(.55,-.3)},anchor=north},
        legend columns=3,
        xticklabels = {
            6;9,
            9;8,
            12;5,
            6;8,
            9;9,
            11;0,
        },
      axis y line*=left,
      axis x line*=bottom,
        x tick label style={align=center,text width=2cm},
        ticklabel style = {font=\footnotesize},
        ]
\addplot[draw=black,fill=tmnlpone] table[x index=0,y index=1] \dataset;
\addplot[draw=black,fill=tmnlptwo] table[x index=0,y index=2] \dataset;
\addplot[draw=black,fill=tmnlpthree] table[x index=0,y index=3] \dataset;
\legend{Parâmetros Gerais, Parâmetros Específicos, Score Total}
\end{axis}
\end{tikzpicture}


    \caption{{Percentagem de acerto nos parâmetros gerais (pontuação máxima 3) e específicos (pontuação máxima 9) na tarefa de reconto para cada criança.}}
    \label{fig:cap16fig2}
\end{Figura}

Os resultados mostram um melhor desempenho na tarefa de reconto de histórias sobretudo para o grupo de crianças com DLT, uma tendência para um melhor desempenho nas crianças mais velhas em ambos os grupos, à exceção dos parâmetros gerais, possivelmente pelo facto de as crianças com perturbação constituírem um grupo heterogéneo (\emph{cf}. Figura \ref{fig:cap16fig2}). Na tarefa de reconto, as duas crianças com PDL mais novas (6;9 e 9;8) mostraram dificuldades nos parâmetros específicos, especificamente ao enunciar os diferentes eventos da história.

É sabido que as competências linguísticas da criança a nível microestrutural estão ainda em desenvolvimento em idade escolar, havendo diferenças entre crianças de diferentes faixas etárias. Também há diferenças nas narrativas de crianças com perturbações da linguagem e crianças com desenvolvimento típico no discurso narrativo, que se manifestam na extensão da história, no vocabulário, no desenvolvimento temático, na extensão média dos enunciados e no número de mecanismos/elementos linguísticos \citep{Newman2006}. A avaliação destes aspetos permite ao terapeuta da fala estabelecer um perfil linguístico da criança.

Desta forma, quanto à dimensão microestrutural, analisámos quer para a tarefa de conto, quer para a tarefa de reconto, nos dois grupos estudados, os seguintes parâmetros: i) número de types (formas de palavras diferentes); ii) número de tokens (total de palavras); iii) rácio type/token; iv) extensão média de enunciado medida em palavras (MLUw); v) número de orações simples, de orações coordenadas e de orações subordinadas, e ainda, vi) tipo de orações subordinadas (\emph{cf}. Tabela \ref{tab:cap16tab2}). Achamos interessante analisar comparativamente estes dois grupos. 

Apresentaremos de seguida as diferenças entre as populações estudadas, quan\-to aos quatro primeiros parâmetros de análise microestrutural, nas duas tarefas apresentadas.
\begin{Figura}
%     \includegraphics[width=11cm]{figures/64.png}

        \pgfplotstableread{
1 98 138
2 166 208
3 182 254
4 308 576
}\dataset
    \begin{tikzpicture}
\begin{axis}[ybar,
        width=.8\textwidth,
        height=.3\textheight,
        ymin=0,
        xtick=data,
        xlabel = {~~CONTO \hspace*{1.5cm}|\hspace*{1.5cm} RECONTO},
%         enlarge x limits=.5,
        bar width=4mm,
        nodes near coords,
        nodes near coords style={font=\footnotesize},
        legend cell align=left,
        legend style={font=\footnotesize,at={(.05,.9)},anchor=north west},
        xticklabels = {
            № Types, № Tokens, № Types, № Tokens
        },
      axis y line*=left,
      axis x line*=bottom,
        x tick label style={align=center,text width=2cm},
        ticklabel style = {font=\footnotesize},
        ]
\addplot[draw=black,fill=tmnlpone] table[x index=0,y index=1] \dataset;
\addplot[draw=black,fill=tmnlptwo] table[x index=0,y index=2] \dataset;
\legend{Grupo PDL, Grupo DLT}
\end{axis}
\end{tikzpicture}



    \caption{{Valores absolutos de n\textsuperscript{o} de types e de n\textsuperscript{o} de tokens obtidos no conjunto da amostra (3 crianças com PDL e 3 crianças com DLT) em cada tarefa.}}
    \label{fig:cap16fig3}
\end{Figura}

\begin{Figura}
%     \includegraphics[width=11cm]{figures/65.png}
        \pgfplotstableread{
1 0.61 0.63
2 0.66 0.44
}\dataset
    \begin{tikzpicture}
\begin{axis}[ybar,
        width=.6\textwidth,
        height=.3\textheight,
        ymin=0,
        xtick=data,
        enlarge x limits=.5,
        bar width=5mm,
        nodes near coords,
        nodes near coords style={font=\footnotesize},
        legend cell align=left,
        legend style={font=\footnotesize,at={(1.05,.09)},anchor=west},
        xticklabels = {
            Grupo PDL, Grupo DLT
        },
      axis y line*=left,
      axis x line*=bottom,
        x tick label style={align=center,text width=2cm},
        ticklabel style = {font=\footnotesize},
        ]
\addplot[draw=black,fill=tmnlpone] table[x index=0,y index=1] \dataset;
\addplot[draw=black,fill=tmnlptwo] table[x index=0,y index=2] \dataset;
\legend{CONTO Rácio Type/Token, RECONTO Rácio Type/Token}
\end{axis}
\end{tikzpicture}

    \caption{{ Valores médios de rácio type/token obtidos no conjunto da amostra (3 crianças com PDL e 3 crianças com DLT) em cada tarefa.}}
    \label{fig:cap16fig4}
\end{Figura}

\begin{Figura}
%     \includegraphics[width=11cm]{figures/66.png}
        \pgfplotstableread{
1 5.86 6.27
2 9.07 9.65
}\dataset
    \begin{tikzpicture}
\begin{axis}[ybar,
        width=.6\textwidth,
        height=.3\textheight,
        ymin=0,
        xtick=data,
        enlarge x limits=.5,
        bar width=5mm,
        nodes near coords,
        nodes near coords style={font=\footnotesize},
        legend cell align=left,
        legend style={font=\footnotesize,at={(1.05,.09)},anchor=west},
        xticklabels = {
            Grupo PDL, Grupo DLT
        },
      axis y line*=left,
      axis x line*=bottom,
        x tick label style={align=center,text width=2cm},
        ticklabel style = {font=\footnotesize},
        ]
\addplot[draw=black,fill=tmnlpone] table[x index=0,y index=1] \dataset;
\addplot[draw=black,fill=tmnlptwo] table[x index=0,y index=2] \dataset;
\legend{CONTO MLU, RECONTO MLU}
\end{axis}
\end{tikzpicture}

    \caption{{Valores médios de MLU obtidos no conjunto da amostra (3 crianças com PDL e 3 crianças com DLT) nas tarefas de conto e de reconto.}}
    \label{fig:cap16fig5}
\end{Figura}

Observando as figuras \ref{fig:cap16fig3} e \ref{fig:cap16fig4} podemos verificar que existem diferenças entre o número de types (forma de palavras diferentes) e número de tokens quando comparados por grupo, e por tarefa. Assim, as crianças com desenvolvimento típico apresentam uma produção oral mais extensa. Quanto à extensão de enunciado (cf. Figura \ref{fig:cap16fig5}), as crianças com PDL apresentam uma extensão de enunciado mais baixa em ambas as tarefas (conto e reconto), quando comparadas com as crianças com desenvolvimento típico, como verificado em estudos anteriores \citep{Gillam1992, Fey2004, BefiLopes2008, Balciuniene2016, Kraljevic2020}. 

Quanto à presença de estruturas mais complexas, as crianças com PDL apresentam mais orações simples e menos orações complexas, comparativamente com o grupo com DLT (\emph{cf}. Figura \ref{fig:cap16fig6}), o que vai ao encontro de outros estudos anteriores \citep{Costa2006, Newman2006, Monteiro2016}.
\begin{Figura}
%     \includegraphics[width=\linewidth]{figures/67.png}
        \pgfplotstableread{
1 22 10
2 3 6
3 1 4
4 30 33
5 2 6
6 10 16
}\dataset
    \begin{tikzpicture}
\begin{axis}[ybar,
        width=\textwidth,
        height=.3\textheight,
        ymin=0,
        xtick=data,
        xlabel = {~~CONTO \hspace*{2.2cm}|\hspace*{1.5cm} RECONTO},
        xlabel style={yshift=-0.4cm},
%         enlarge x limits=.5,
        bar width=5mm,
        nodes near coords,
        nodes near coords style={font=\footnotesize},
        legend cell align=left,
        legend style={font=\footnotesize,at={(.01,.9)},anchor=west},
        xticklabels = {
            Orações simples, Orações coordenadas, Orações subordinadas, Orações simples, Orações coordenadas, Orações subordinadas
        },
      axis y line*=left,
      axis x line*=bottom,
        x tick label style={align=center,text width=2cm},
        ticklabel style = {font=\footnotesize},
        ]
\addplot[draw=black,fill=tmnlpone] table[x index=0,y index=1] \dataset;
\addplot[draw=black,fill=tmnlptwo] table[x index=0,y index=2] \dataset;
\legend{Grupo PDL, Grupo DLT}
\end{axis}
\end{tikzpicture}

    \caption{{Comparação entre grupos (3 crianças com PDL e 3 crianças com DLT) e tarefas (conto e reconto) quanto ao n\textsuperscript{o} de orações simples, coordenadas e subordinadas produzidas.}}
    \label{fig:cap16fig6}
\end{Figura}

\begin{Tabela}
    \caption{{Tipo de orações subordinadas produzidas, por grupo e por tarefa.}}
    \label{tab:cap16tab2}

    \begin{tabular}{llrr}
    \lsptoprule
 \textbf{Tarefa}	&\textbf{Tipo de orações}	&\textbf{PDL}&	\textbf{DLT}\\
 \midrule
 & Orações relativas &1 &2\\
 Conto	&Orações subordinadas completivas finitas&	0&	1\\
&	Orações subordinadas adverbiais finais&	0&	1\\
\midrule
\multirow{6}{*}{Reconto}&	Orações relativas&	0&	6\\
&	Orações subordinadas completivas finitas&	3&	3\\
&	Orações subordinadas adverbiais finais	&4	&2\\
&	Orações subordinadas adverbiais causais	&2	&2\\
&	Orações subordinadas adverbiais concessivas&	1&	0\\
&	Orações subordinadas adverbiais temporais	&0	&3\\
\lspbottomrule
    \end{tabular}
    \label{tab:my_label}
\end{Tabela}

Ainda que, dada a dimensão da amostra, estes resultados sejam apenas indicadores de uma diferença entre grupos que deverá ser explorada de forma mais aprofundada em trabalho posterior, eles sugerem que possa haver uma diferença entre grupos quando consideramos as estruturas linguísticas produzidas. Observando a Figura 6 e a Tabela \ref{tab:cap16tab2} percebemos que a produção de orações subordinadas foi superior e mais diversificada na tarefa de reconto. No entanto, o grupo de crianças com PDL apresentou, na tarefa de reconto, quatro tipos de orações subordinadas: orações subordinadas completivas finitas; orações subordinadas adverbiais causais, finais e concessivas. O grupo de crianças com DLT produziu orações relativas; orações subordinadas adverbiais causais, finais e temporais, e ainda orações subordinadas completivas finitas. Estes resultados, ainda que preliminares, vão ao encontro de outros estudos, tendo havido uma maior percentagem de produção de orações complexas nas crianças com DLT, em ambas as tarefas apresentadas, bem como uma maior diversidade de orações, \citep{Gillam1992, Fey2004, Cunha2013, Monteiro2016, Gillam2018}. 


\subsection{Compreensão da história após a tarefa de reconto}
Já na avaliação de compreensão de histórias, as crianças com PDL apresentaram resultados muito abaixo dos das crianças com desenvolvimento de linguagem típico (\emph{cf}. Figura \ref{fig:cap16fig7}). Os resultados mostraram dificuldades nos três tipos de questões, mas com percentagens mais baixas de acerto nas questões que avaliavam compreensão inferencial e crítica, o que tem sido mostrado em outros estudos (\citealp{Cunha2013, Viana2010, Westby2005, Giasson2005, Gillam2018, Favot2020}; \citetv{chapters/15}).
\begin{Figura}
%     \includegraphics[width=11cm]{figures/68.png}
            \pgfplotstableread{
1 67 100
2 56 100
3 0 67
}\dataset
    \begin{tikzpicture}
\begin{axis}[ybar,
        width=.6\textwidth,
        height=.3\textheight,
        ymin=0,
        xtick=data,
        enlarge x limits=.5,
        bar width=5mm,
        nodes near coords,
        nodes near coords style={font=\footnotesize},
        legend cell align=left,
        legend style={font=\footnotesize,at={(1.05,.09)},anchor=west},
        xticklabels = {
            Literal, Inferencial, Crítico
        },
      axis y line*=left,
      axis x line*=bottom,
        x tick label style={align=center,text width=2cm},
        ticklabel style = {font=\footnotesize},
        ]
\addplot[draw=black,fill=tmnlpone] table[x index=0,y index=1] \dataset;
\addplot[draw=black,fill=tmnlptwo] table[x index=0,y index=2] \dataset;
\legend{Grupo PDL, Grupo DLT}
\end{axis}
\end{tikzpicture}


    \caption{{Percentagem de acerto nas perguntas de compreensão (pontuação máxima de questões literais 4; pontuação máxima de questões inferenciais 3, e pontuação máxima de questões críticas 1) na tarefa de reconto para cada criança.}}
    \label{fig:cap16fig7}
\end{Figura}

\section{Conclusão}\label{sec:oitocap16}
\citet{Paul2007} sugere que a avaliação da competência narrativa deve fazer parte dos procedimentos de diagnóstico em terapia da fala, visto que esta permite uma descrição mais detalhada da linguagem comparativamente às tarefas de elicitação habitualmente usadas em instrumentos de avaliação da linguagem. A avaliação das narrativas orais é particularmente importante quando se suspeita de uma PDL. Este instrumento de avaliação das narrativas orais permite obter um perfil linguístico das crianças com perturbação de linguagem a nível dos parâmetros macroestruturais e microestruturais, desde o início do primeiro ano de escolaridade. Esta análise detalhada permitirá ao terapeuta da fala o planeamento de uma intervenção adequada, minimizando o impacto negativo que as alterações de competências narrativas têm no sucesso escolar \citep{Gillam2018}. 

As narrativas orais têm servido de ponte entre a linguagem oral e a literacia \citep{Westby2005} uma vez que são uma forma naturalista de organizar o pensamento abstrato, a linguagem complexa e a sequenciação \citep{Petersen2011}. A capacidade de um indivíduo para gerar uma narrativa está fortemente ligada a questões sociais \citep{Cheshire2000}, bem como a questões académicas \citep{Spencer2018}. A importância das narrativas para a literacia e para o progresso académico é fundamental, sendo que a sua avaliação é muito importante para uma caracterização mais global do perfil linguístico da criança e para a planificação de uma intervenção que tenha em conta esta dimensão \citep{Fey2004, Bliss2008, Favot2020}.

\section*{Agradecimentos} 
Gostaríamos de agradecer de forma especial: a todos os pais que, de boa vontade, aceitaram que os seus filhos pudessem participar neste estudo; ao agrupamento de escolas da cidade do Entroncamento, por aceitarem colaborar connosco neste estudo de investigação; à Terapeuta da Fala Patrícia Oliveira pela sua preciosa ajuda na recolha de dados junto de crianças com PDL, na região de Fátima; ao Eduardo Castro pela colaboração e ajuda com as transcrições e análise dos respetivos dados.

This chapter was supported by National Funds through FCT - Fundação para a Ciência e a Tecnologia within CINTESIS, R\&D Unit (reference UIDB/4255/2020) and PD/BD/131387/2017.


{\sloppy\printbibliography[heading=subbibliography,notkeyword=this]}
\end{document}
