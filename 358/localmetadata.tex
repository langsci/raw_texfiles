\title{Linguística clínica}
\subtitle{Modelos, avaliação e intervenção}
\author{Maria João Freitas and Marisa Lousada and Dina Caetano Alves}
\renewcommand{\lsSeries}{tbls}
\renewcommand{\lsSeriesNumber}{11}
\typesetter{Sybil Vachaudez, Sebastian Nordhoff, Yanru Lu}
\BackBody{A Linguística Clínica reúne profissionais, investigadores e estudantes de diferentes graus académicos cujo foco de trabalho é a exploração da ponte entre a Linguística e a Fonoaudiologia (na tradição brasileira) ou a Terapia da Fala (na tradição portuguesa). Tem como objetivos centrais a construção de conhecimento sobre a natureza dos desempenhos linguísticos atípicos e a preocupação constante em tornar os processos de avaliação e de intervenção em contexto clínico cada vez mais rigorosos e eficazes.

O presente volume destina-se a quem estuda ou investiga aspetos relacionados com desempenhos linguísticos atípicos em países lusófonos, nomeadamente a estudantes e docentes de cursos de graduação e pós-graduação nas áreas da Terapia da Fala/ Fonoaudiologia, da Linguística, da Psicologia e da Educação.

Os capítulos que integram a presente publicação estão organizados em quatro blocos temáticos. O bloco inicial tem como objetivo central fornecer uma perspetiva histórica dos estudos em Fonologia Clínica e em Sintaxe Clínica. O segundo centra-se em aspetos fonológicos e sua relação com a dimensão fonética da língua. O terceiro bloco integra capítulos que exploram o módulo gramatical da sintaxe, a interface gramática-pragmática e o discurso. O último bloco reúne um conjunto de investigações sobre populações específicas.
}
\renewcommand{\lsID}{358}
\renewcommand{\lsISBNdigital}{978-3-96110-400-0}
\renewcommand{\lsISBNsoftcover}{978-3-98554-058-7}
\BookDOI{10.5281/zenodo.7197134}
