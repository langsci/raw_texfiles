%% -*- coding:utf-8 -*-

\section*{Foreword of the fifth edition}

I want to thank Philip Kime for help with biber, the tool that Language Science Press is using for
creating lists of referecnes and for manipulating bibliography databases. The bibliography was
entirely updated and manually checked since this was done for the HPSG Handbook \citep*{HPSGHandbook}. Papers
now have DOIs wherever possible.

Ladis Duffet pointed out a mistake in Section~\ref{sec-valence-classes}, which probably confused many who tried to make
sense of this section in earlier editions.

% added Ajdukiewicz35a-u to DP authors.

% changed some examples in 1-begriffe and other files

% 23.02.2021 changed AdvP -> [very] to AdvP -> [very,roof]

% 23.02.2021 fn regarding triangles in trees

% 26.01.2021 added Lötscher85 to dg chapter.

% 05.11.2021 fixed Peter Cullicover's findings gloss in dg.tex
% and missing "d" in 3-gb.tex

% 03.12.2021 added LS-GRAM for German HPSG implementations

% 06.12.2021 fied typo in index head"complement"=phrase

% fixed typos from email in 1-begriffe.tex (patch)

% 04.04.2022
% typos in 2-psg. Klammern in {ex-einige-kluge-Frauen-und-Maenner}
% 12.04.2022
% typos in 3-gb.tex meet- statt meet
% Titel des zitierten Buches in Leseliste war falsch, noch aus deutscher GT falsch
% 19.04.2022 3-minimalism
% S. 180 \_ fehlte bei ATB extraction
% EPP erklärt und Quelle
% 20.04.22 mögen -> like
%
% 27.04. female person = woman
%        electrical appliance = electrical device
%        PREDspecification war ab Ausgabe 4 kaputt
% 06.05. 7-cg.tex removed lots of brackets around cg-expressions. Thanks to anonymous reader.
%
% 15.05. 8-hpsg.tex Strange brackets in Figure {verb-movement-syn-simple}.
%        loc should be LOC in figure 9.15
%        edits due to comments by anonymous reader.
% replaced Mann by Roman

I fixed a mistake at the beginning of Section~\ref{sec-typeraising}: it now reads ``backward
application'' instead of ``forward application''.

I fixed the Case Principle in the chapter on HPSG. The first two clauses did not mention that they
only apply to verbal heads.

As pointed out to me by an anonymous reader, the type of the AVM in (\ref{avm-woman}) should have
been \type{woman} rather than \type{female person}. The top-most type in Figure~\ref{fig-electric-appliance} has to be
\type{electric device} rather than \type{electrical appliance}, since this is the name used in the text.

I fixed some brackets in the Categorial Grammar derivation in Figure~\ref{Abbildung-CG-isst-der-junge-den-kuchen-jacobs}. There were
just too many brackets to keep track of everything \ldots. Thanks to Matthew Korte and Pascal
Hohmann for spotting this (independently)!
Léonie Cujé found superflous brackets in Figure~\ref{abb-CG-Adjunktion}. They were removed. Thanks!

Figure~\ref{verb-movement-syn-simple} on page~\pageref{verb-movement-syn-simple} contained some strange brackets, which I removed now.

I also want to thank an anonymous reader for sending patches to the \LaTeX{} files correcting some
typos and wrong or missing words in glosses.

Since the last two reviews of the book complained about the classification and new developments
sections referring to material not introduced yet, I decided to make the structure of the book more
explicit by repeating the introductory remark from page~\pageref{page:structure-of-book} at the
beginning of all the advanced sections. I still think that this is the correct structure of the book to introduce a
certain framework and then evaluate it. The only way to fairly evaluate a theory is to compare it to
other theories. This cannot be done without knowledge of the theories to be compared. So readers
interested in such comparisons should read the introductory parts of the chapters and then come back
to the evaluation part and the parts discussing further developments. \citet{Culicover2021a}
remarked that it is unclear how the book is supposed to be used for teaching. The book is already
used at many, many universities worldwide, but those who want to know how I use it may check out my
slides, which are available both as PDF and source code on GitHub:
\url{https://github.com/stefan11/grammatical-theory-slides}. During Corona times I also put
recordings of my lessons online: \url{https://www.youtube.com/watch?v=_W6nVRnC0NA&list=PLXwGGsuPxWRotmEg5LStGTxZWEkqKXmrh&index=1}.

~\medskip

\noindent
Berlin, \today\hfill Stefan Müller



%      <!-- Local IspellDict: en_US-w_accents -->
