%% -*- coding:utf-8 -*-

\begin{abstract}
This paper discusses \citeauthor{AmbridgeBlything2024a}'s claim \citeyearpar{AmbridgeBlything2024a}
that Large Language Models are the best linguistic theory we currently have. It discusses claims
that LLMs are wrong linguistic theories and concludes that they are not linguistic theories at
all. It is pointed out that Chomsky's claims about innateness, about transformations as underlying
mechanisms of the language faculty and about plausible representations of linguistic knowledge are
known to be flawed by quite some time by now and that we would not have needed LLMs for
this. Chomsky's theories are not refuted by LLMs in their current form, since LLMs are different in
many aspects from human brains. However, the tremendous success of LLMs in terms of applications
makes it more plausible to linguists and laymen that the innateness claims are wrong. 

It is argued that the use of LLMs is probably limited when it comes to typological work and
cross-linguistic generalizations. These require work in theoretical linguistics.
\end{abstract}

\section{Are Large Language Models linguistic theories?}

In a recent paper in a special issue of Theoretical Linguistics containing ``Reflections on
Theoretical Linguistics'' on the occasion of the 50th anniversary of the journal,
\citet{AmbridgeBlything2024a} claim that ``large language models are better than theoretical
linguists at theoretical linguistics'' (p.\,33). The authors examine the output of an LLM with
regard to the argument structures of verbs, and are impressed that the model predicts the same as
Ambridge found out in experiments with students. The authors claim that LLMs are a theory of
language. The best one we have right now:
\begin{quote}
large language models (LLMs) are already the leading current theories of how speakers learn and
represent these restrictions. Of course, they are not perfect theories [\ldots] but they’re better
theories than any others that have been proposed. \citet[\page 34]{AmbridgeBlything2024a}
\end{quote}
LLMs are very interesting and you can do a lot of impressive things with them.\footnote{%
For example, ChatGPT can explain prime factorizations in Trump-style \citep[356--357]{Piantadosi2024a}.
} But are they theories? Do they help in any way to get a better understanding of language?

The authors claim that large language models are theories of language acquisition and representation
and that they are instantiations of Construction Grammar approaches \citep{Goldberg2006a}: 
\begin{quote}
Large language models [...] constitute theories of language acquisition and representation; theories
that instantiate exemplar-, input- and construction-based approaches, though only very
loosely. \citep[\page 33]{AmbridgeBlything2024a}
\end{quote}

\noindent
The authors claim that models are a theory (see also \citealt[360]{Piantadosi2024a}):
\begin{quote}
OK, so the model makes the right predictions but -- we hear you ask -- where is the theory? That's
the point: the model is the theory.
\citep[\page 39]{AmbridgeBlything2024a}
\end{quote}
This shows some confusion in terminology. 
A \emph{theory} contains descriptive and explanatory statements about some part of the reality. It contains
laws about the domain that is described by the theory. A \emph{model} is an abstract representation of the
relevant part of the reality under consideration. A theory can be used to build such models.
A theory should use primitives that are appropriate for a certain domain and it should contain statements about these
primitives. Large language models are neuronal nets that have been organized and trained in a
certain way. Nodes of such nets can be examined and we can even find certain information in them
\citep{ManningClarkHewitt2020a,ZhangBowman2018a} but this information is not a theory. The net may
reflect grammatical structures and reject impossible ones, but it does not tell us why this is the case.

Later in the paper and contradicting their earlier claim, the authors argue that the programs that generate the LLMs are a theory:
\begin{quote}
``But'', critics object, ``we have no idea what it's doing'' (e.g., Kodner et al. 2023; Milway
2023). Quite the opposite: Unlike for traditional linguistic theories, every last detail of the
model's assumptions and operation is written out in black and white, in thousands of lines of
computer code. This code is a theory of the acquisition of (among other things) verb argument
structure; it's even -- like traditional linguistic theories -- written in a language, albeit an
artificial programming language, rather than a natural language like English. We know exactly what
the model is doing.  \citep[\page 39--40]{AmbridgeBlything2024a}
\end{quote}
This very quote is an instance of mixing levels. We know what the \emph{code} is doing. We do not
know what the trained net, the model, is doing. This depends on the training data and even if we
knew the training data, we could not predict what specific nodes in the net would do. The issue is
just too complex for us humans and the training data is too vast. This can be compared with Definite
Clause Grammars: this is a notation that can be used to write down phrase structure grammars. Most
Prolog interpreters come with a component that parses such grammars directly (see
\citealt[Chapter~9]{CM2003a} for more information on DCGs and \citealt[Task 10 on
p.\,81]{MuellerGT-Eng5} for more information on working with DCGs online). \citet[268--270]{CM2003a}
provide two pages of code for the translation of DCGs into Prolog code. The resulting Prolog program
does Parsing as Deduction. In this case, we know what the code is doing. It reflects our theory
about language. For a more elaborate example of Parsing as Deduction based on Government \& Binding
see \citew{Johnson89a}. 


\citet{FoxKatzir2024a} published a response to \citename{AmbridgeBlything2024a} in the same issue of
Theoretical Linguistics. They write:
\begin{quote}
The distinction between competence and performance and between correctness and likelihood are parts
of all the best theories of human linguistic cognition, as are the aspects of linguistic
representation that we briefly reviewed (modularity, constituency, and entailment). [\ldots]
the LLM Theory does not even come close to approximating the relevant observations. Obviously it
cannot derive these properties of human linguistic cognition and without doing so it cannot be
considered a scientific theory at all. \citep[\page 75]{FoxKatzir2024a}
\end{quote}
The authors claim that LLMs cannot be a theory, since they do not make the competence-performance
distinction, since they do not adhere to modularity and since they do not capture constituency. If
these failures to capture certain properties of language would indeed entail that LLMs are not
scientific theories, then neither Construction Grammar nor any flavor of Mainstream Generative
Grammar (MGG)\footnote{%
  The term MGG goes back to \citet[\page 3]{CJ2005a}. It refers to all proposals developed by
  Chomsky. Government \& Binding \citep{Chomsky81a} and theories developed under the label of
  Minimalism \citep{Chomsky95a-u} are the most recent incarnations. 
} would be. The distinction between competence and performance is rejected in
Usage-Based Construction Grammar \citep[\page 297]{Diessel2015a-u}. I personally think that this is
a mistake \citep[Chapter~15]{MuellerGT-Eng5}, but nevertheless approaches in Usage-based Construction
Grammar are theories. The alternative approaches in MGG do not fare any better. All basic
architectural assumptions in all of Chomsky's approaches are highly implausible from a
psycholinguistic point of view. The Derivational Theory of Complexity, which assumed that sentences
involving more transformations in their analysis are more difficult to process than sentences with
fewer transformations has been proven wrong (\citealt[\page 320--328]{FBG74a-u}; see
\citealt[Chapter~15.1]{MuellerGT-Eng5} for a recent discussion). The T-model with its autonomous components of 
syntax, phonological and logical form has been
proven wrong resulting in spectacular analyses in Cartography \citep{CR2010a} to circumvent the
autonomy of syntax restriction (see Müller \citeyear[Section~4.6.1.1]{MuellerGT-Eng5}, \citeyear[Section~4.10.2]{MuellerGermanic} on this point and on
problems with Cartographic approaches, for example, \citegen[\page 100]{Cinque94a-u} claim that
categories like Nationality are part of our genetic endowment). Derivation by Phase \citep{Chomsky2008a} and other Minimalist architectures \citep[812, 830]{Richards2015a} are
entirely implausible as architectures for human language \citep[Section~3.6]{BM2021a}, since they
are incompatible with incremental parallel processing of linguistic information at all descriptive
levels 
\iftoggle{long}{\citep{Marslen-Wilson75a,TSKES96a,Labelle2007a}.}{%
\citep{Marslen-Wilson75a,TSKES96a}.}
If the argumentation by
\citename{FoxKatzir2024a} was valid, it would follow that all approaches in 
Usage-based Construction Grammar and MGG were not scientific theories. This would be a very strange
conclusion, but it is not warranted. They are scientific theories, but they are bad ones.

Concerning the other points raised in the above quote: the claims about modularity and interfaces are probably
wrong \parencites{Pulvermueller1999a-u,PCS2013a-u}[\page 22, 27]{Jackendoff2000a}{Kuhn2007a} and there are theories in which constituency does not play
a role but dependency does \citep{Tesniere59a-Eng}. And \citet{ClarkKhandelwalLevaManning2019a,HewittManning2019a,ManningClarkHewitt2020a} show that
dependency information is encoded in LLMs.\footnote{\label{fn-dependencies}%
  It is important to note that \citet{ClarkKhandelwalLevaManning2019a,HewittManning2019a,ManningClarkHewitt2020a} were able to discover the fact that
  dependencies are represented in LLMs because they knew the concept of dependencies, which was
  developed by Tesnière in 1924--1954. So the linguistic theory and related concepts were a prerequisite
  to find linguistic structure in the neural networks. This point will be taken up again
  below in the discussion of typological work.

  Another note on linguistic information in LLMs: Imagine you build a model of a landscape in a
  lab. You have soil and water. The water runs in little rivers, carves valleys into the soil. The
  landscape is formed over time, you get hills, canyons, creeks, rivers. This is like the training
  phase of a neuronal net. After this landscape forming phase you may put liquid into your artificial
  landscape and see what forms rivers will take. But does this tell you something about rivers in
  general? A theory about the way water distributes? No. It gives you concrete examples of how a possible
  river may look like after years and years of forming an artificial landscape. This is what we get
  from LLMs: we train them with lots and lots of data and then get a structure that was shaped by
  the data.
}

% \citet{Milway2023a} quotes Chomsky stating that LLMs have too much power. Since they could learn
% things that humans cannot learn, they are not adequate: ``And if a system works just as well for
% impossible languages as for possible ones by definition not telling us anything about language. And
% that’s the way these systems work it generalizes the other systems too. So the deep problem that
% concerns me is too much strength. I don’t see any conceivable way to remedy that.''. The problem for
% this criterion is that it would also rule out current MGG theories using remnant movement
% \citet{GMueller98a,Wurmbrand}. Haider (p.\,c.\ 2018) has pointed out that theories allowing remnant
% movement can derive a sentence with inverted word order, something that is freuqently claimed to not
% occur in the languages of the world \citep{MMGRRBW2003a}. The details of the derivation are provided
% in \citew[\page 1305]{BM2021a}.

So \citet{FoxKatzir2024a} should not have argued that LLMs are no theories because they do not have
the properties X and Y that some linguistic theories have, but instead they could have argued that
the theory, if it existed in LLMs, would be wrong, since it was missing X and Y. I argue that
there is no theory about language in it. I believe that \citet{AmbridgeBlything2024a} are
fundamentally wrong. To show this let us do a thought experiment. LLMs are neural networks. Their
architecture is inspired by what we find in brains. They differ from brains in various ways, but let
us assume that one could develop a perfect replica of a brain one day. To quote
Norbert Wiener, the founder of cybernetics: ``The best model of a cat is a cat, preferably the same
cat.''
% \footnote{%
% To be clear about this: This was a joke. If your homework is to create a model of the cat Molly and
% you hand in Molly, your supervisor will not be amused. You will fail when it comes to the point when
% your model has to be compared to the original Molly. Similarly, handing in Daisy, the cat of your
% neighbor, will be considered cheating as well.
% } 
So, if we have a perfect copy of somebody's brain, what can we do with it? The artifical brain can then do exactly what
the 48 Liverpool students mentioned in the \citename{AmbridgeBlything2024a} paper can do. Perhaps a
bit more smoothed, because this replica can be fed much, much, much more data than all Liverpool
students will ever see in their 48 lives combined. Now the question is: What does this mean for
linguistics? Is a replica of a brain a theory about language? No. It is a masterpiece of
engineering. Nothing more. To build such masterpieces, you need theories about how brains work. You can then take parts of these theories and use them to build
artificial brains. The code that people write to train the data structures they have created is code
that is motivated by theories about the brain. It is not a theory, and certainly not a theory about
language. The criticism that \citename{AmbridgeBlything2024a} reject is justified: LLMs are not
theories about language; the information contained in LLMs is only indirectly accessible. Just as
you cannot directly access the information in brains. You can only study the behavior of
people. That is what we have been doing for hundreds of years. We look at what people say and write. We
conduct experiments with people. We ask them about the acceptability of sentences. We test where
they look when certain sentences are uttered \citep{TSKES96a}. We measure brain activity (event-related potentials,
cerebral bloodflow, etc.) We investigate what happens when certain areas of the brain are damaged.
This gives us information about the processes and representations of linguistic knowledge in the
brain. From this we can then draw conclusions for plausible theories.  

What is it like with LLMs? They are like brains: black boxes. We could start playing around with
them now and try to find out what is stored where and how. But what good would that do? Actually
such a research field exists already. \citet[\page 5185]{BK2020a} call it ``BERTology'':\footnote{%
  BERT stands for \emph{bidirectional encoder representations from transformers} and is a shorthand
  for a large language model introduced by Google.%
}
Engineers and linguists are playing with LLMs and check what they can do. This is interesting, but irrelevant for
linguistics.\footnote{%
  This was a bit of a hyperbole. LLMs may be used to play around with data and to check what these
  models need as input to get certain facts about language right. This can help linguists to
  discover relations and dependencies between linguistic phenomena that are plausible parts of a
  linguistic theory (generalizations, constructions, families of constructions and the relations
  between constructions). For example,
  \citet{MisraMahowald2024a} show that LLMs perform above chance on phrases like \emph{a five
    beautiful days}, provided certain other constructions are in the training corpus. So, the place of LLMs in
  linguistics seems to be the one of subjects that one can feed arbitrary training material and that
  one can interrogate without them getting tired and without the need of an ethics vote. Since LLMs
  are different from real humans, the resulting theories should be checked with reference to actual
  data and actual human behavior, but they can serve as a first inspiration.%
} 

Conclusion: We (as humanity) have created a technical masterpiece, but we know no more about our
cognitive abilities than we did before.

%\section{Large Language Models, human first language acquisition and ``Chomsky's approach to language''}

\section{LLMs and language acquisition}

Maybe the last sentence in the previous section needs a bit of qualification. \citet{Piantadosi2024a} claims that Chomsky's
approach to language has failed, that it was proven wrong by Large Language Models. As
\citet[366]{Piantadosi2024a} writes himself, LLMs ``are trained on truly titanic datasets compared to children, by a factor of at least a few
thousand''. So, if linguistics is dealing with human capabilities, we are not quite there yet. To
model language acquisition, we would need grounded input, we would need a realistic amount of
training data, we would have to simulate the development of brains and the growth of cognitive
capacities in early childhood.\footnote{
  Children regularize more than adults \citep{HudsonN99a,HKN2005a}, a fact that can be traced back to their
  limited brain capacity (``less is more''-hypothesis, 
\iftoggle{long}{\citealp{Newport90a,Elman93a}}{\citealp{Newport90a}). 
}} But what the success of LLMs suggests is that an elaborated component of Universal
Grammar is not needed, that the argument of the Poverty of the Stimulus was flawed and so on. Above
I wrote that ``we know no more about our cognitive abilities than we did before''. And this is
true. We knew in 1974 (50 years ago) that transformations are psycholinguistically
implausible \citep*[\page 320--328]{FBG74a-u}. Psycholinguists sympathetic with the Chomskyan paradigm suggested that we have our
linguistic knowledge represented as a Transformational Grammar, but that it then gets compiled out
into a set of templates that are equivalent ot the constructions of Construction Grammar \citep[\page 27]{FC96a-u}. But this
of course begs the question why one should not work in Construction Grammar or a related framework
like Constructional HPSG \citep{Sag97a,HPSGHandbook} from the beginning. What
is the evidence for some underlying transformation-based representation of linguistic knowledge? The
various architectures that were proposed over the years were psycholinguistically implausible
too. The T-Model \citep{Chomsky81a,Chomsky86b} was implausible \citep[Section~15.2]{MuellerGT-Eng5} and this got only worse with
Phase-based variants of Minimalism \parencites{Chomsky95a-u,Chomsky2008a}[\page 812,
830]{Richards2015a}[Section~5]{BM2021a}. But if the theories are incompatible with 
empirical facts like incremental processing, how can they tell us anything about human cognition and
inateness? The Principle \& Parameter model of language acquisition \citet[\page 6]{Chomsky81a}
failed in various respects. It was assumed that one parameter was related to many properties of a
language and worked like a switch 
\iftoggle{long}{\parencites{Rizzi86a}[\page 8]{Chomsky2000a-u},}{
\parencites[\page 8]{Chomsky2000a-u},}
 but none of the suggested correlations held 
up 
\iftoggle{long}{%
\parencites{Haider94c-u}[Section~2.2]{Haider2001a}[Section~16.1]{MuellerGT-Eng5}.}{%
(\citealt[Section~2.2]{Haider2001a}; see \citealt[Section~16.1]{MuellerGT-Eng5} for an overview).}
The way
parameterization was conceptualized was biologically implausible. For example, it was assumed that
Subjacency was a universal principle and the parameterization concerned the part of speech of
certain bounding nodes within nonlocal dependencies 
\iftoggle{long}{\parencites[\page 271]{Chomsky73a}[\page 40]{Chomsky86b}{Baltin81a,Baltin2006a}.}{
\parencites[\page 40]{Chomsky86b}{Baltin81a}.}
First, it could be shown that subjacency does not hold in
Dutch, German and English \parencites[\page 52]{Koster78b-u}[\page
211]{Mueller99a}{Mueller2004d}[Section~3]{Mueller2007c}{SS2013b-u} and second, it is biologically
absolutely implausible that part of speech information is encoded in our genes
\iftoggle{long}{\parencites{Bishop2002a}[Section~6.4.2.2]{Dabrowska2004a}{FM2005a}.}{%
\parencites{Bishop2002a}[Section~6.4.2.2]{FM2005a}.}
This was realized by
\citet*{HCF2002a}. What remained as property that was assumed to be part of Universal Grammar was
Merge, an operation for combining linguistic material. Somehow a triviality
\citep[475]{MuellerGT-Eng5}. A triviality that caused another linguistic war \citep{Pullum2024a}. 
 
There is one important aspect of research in the Principles \& Parameters era: The systematic search for
universals, for commonalities and differences lead to a much improved knowledge about variation. We
know much more about language as such, that is, about structures that are similar in principle.
For example, the German sentence in (\ref{ex-quietschen}) is parallel to the English translation. 
\ea
\label{ex-quietschen} 
\gll dass die Straßenbahnen um die Ecke quietschen\\
     that the trams         around the corner squeak\\
\glt `that the trams squeak around the corner'
\z
As \citet{MuellerUnifying} pointed out, it is possible to develop analyses that capture the
commonalities although the linearization of the constituents differs in German and English (English
is an SVO language and German is SOV). 
Typological research is fascinating and requires the comparison of many very different languages on a
theoretical level. I doubt that the results of cross-linguistic research can be derived from LLMs,
without any interaction with theoretical linguistics. Training LLMs on multilingual material will be
non-trivial\footnote{
  See \citet{ChangArnettTu2024a} for comments on the low quality of multilingual language
  models. Note also that a lot of typologically interesting languages are low-resource languages, so
  a massive training like with LLMs is not possible because of the lack of data. See
  \citet{ChangArnettTu2024a} on monolingual models for 350 languages.
} and discovering cross-linguistic generalizations in network representations is probably
impossible without a theoretically informed clue on what to look for (see also footnote~\ref{fn-dependencies}). A suggestion for a
methodological clean way of deriving cross-linguistic generalizations that differs from the MGG
approach is assumed in the CoreGram project \citep{MuellerCoreGram}.

Chomsky claimed that there would be language universals but there are no plausible candidates for
syntactic universals left (\citealt{EL2009a}; see \citealt[Section~13.1]{MuellerGT-Eng5} for an overview). There are tendencies, for sure, but this is not sufficient for positing
innate knowledge of language. 
\iftoggle{long}{Recently, I discovered a universal, but it is not syntactic but rather
on the text level: the Festschrift Universal. The reader is referred to \citew{MuellerFestschrift}
for details.}


The strongest argument for innate linguistic knowledge seemed to be the Argument from the Poverty of the Stimulus, but
it was never actually correctly carried out \citep{PS2002a,SP2002b}. Chomsky repeated his favourite argument
with question formation as auxiliary inversion throughout several decades \parencites[\page 29--33]{Chomsky71a-u}[\page 39]{Chomsky2013a}. \citet{Bod2009a} showed
how frequencies of subtrees can be used to learn structures of auxiliary inversion even though the
examples that Chomsky (wrongly\footnote{%
See \citet[\page 41--45]{PS2002a}.
}) claimed to be non-existent in the data were not used in the learning
procedure. Chomsky ignored these insights (\citealt*{BPYC2011a}, \citealt[\page 39]{Chomsky2013a}) and so we
find the auxiliary inversion claim again two years later in the same journal that also published Bod's paper.
Similarly pattern-based modeling language acquisition research was much more successful in explaining
cross-linguistic differences in acquisition than alternative accounts couched in Chomskyan
frameworks 
\iftoggle{long}{\citep{FPG2006a,FPAG2007a,FPG2009a}.}%
{\citep{FPAG2007a}.} 


Connected to the assumption of Universal Grammar is the assumption of a core/""periphery
distinction 
\iftoggle{long}{%
\parencites[\page 7--8]{Chomsky81a}[\page 150--151]{Chomsky86a}[\page 343]{Fodor98a}}{%
\parencites[\page 7--8]{Chomsky81a}}. The idea is that there is a core of linguistic knowledge that is determined by our
genetic endowment and there is a periphery (e.\,g. idioms) that is learned in another way. There is an interesting and very
simple argument against this stance and it goes like this: If we can learn the idiosyncratic parts of a language that is
assigned to the periphery, we should be able to learn the more regular parts of the assumed core
\iftoggle{long}{%
\parencites[\page 20]{Abney96a}[\page 222]{Goldberg2003b}[\page 14]{Goldberg2006a}[\page 100]{Newmeyer2005a}[\page 36]{Tomasello2006a}[\page
20]{Tomasello2006c}{MuellerKernigkeit}.}{%
\parencites[\page 20]{Abney96a}[\page 14]{Goldberg2006a}[\page 100]{Newmeyer2005a}[\page
20]{Tomasello2006c}{MuellerKernigkeit}.} 
See \citew{MuellerKernigkeit} and the CoreGram project \citep{MuellerCoreGram}
for a method for deriving language-internal and also cross-linguistic generalizations and the notion
of \emph{Kernigkeit} (coriness) that does not refer to Chomsky's core/periphery distinction. 

So, we knew that Chomskyan approaches to language and language acquisition failed in terms of their
basic assumptions (transformations), they failed in terms of their architecture with respect to
psycholinguistic evidence (separation of syntax and phonology and semantics in various forms) and
they failed in respect to assumptions about genetics. Everybody working in non-Chomskyan paradigms
has been aware of this for more than a decade (see the first editions of my \emph{Grammatical theory} textbook
from 2010 and 2014 for a summary of the
respective discussions in German and English, respectively). We did not need LLMs for this, but maybe the actual usefulness of these networks is that
they make the possibility that we do not need any innate domain-specific knowledge plausible to
everybody: linguists and laymen. However, to show that LLMs can acquire languages like humans do, they have
to be more human-like. To reach this goal, we probably need more knowledge about brains. As I
pointed out above, if we manage to reach the goal of creating more human-like models, we know how
brains work, but we do not necessarily know how languages work.

I mentioned many of the failures of Chomskyan theories above, but note that they were very
successful. They contributed to our understanding of language. The reason is that they were
theories. They made predictions and contained claims about languages. We knew how to falsify them
and we did. The era of research on Principles \& Parameters was fruitful in that it caused a
enormous amount of typological research. LLMs on the other hand are black boxes. They make
predictions, some right, some wrong, but this is all we have. There is no explicit law that is falsified.

\section{Linguistic theories}

I believe that linguistic theories should contain rules and symbols. A linguistic theory can to some
extent be derived from large corpora using automatic methods. Both the categories can be obtained
via class formation and rules or valence patterns and the corresponding lexical entries can be
derived automatically. The parts of speech and features like case, gender and
number that are currently used in linguistic theories are basically the outcome of a distributional
analysis that was done ``by hand'' during the last centuries. Grammar rules and also feature-value
pairs may be assigned probabilities \citep{Jurafsky96a}. These can also be derived from corpora. This is complicated and
the mathematics is not fully understood yet. But one can train the system on large amounts of
data. The training procedure contains assumptions about language: there are categories, there are
constituents. There will be a residuum of infrequent phenomena that will not be captured
this way (for example apparent multiple frontings, see \citealt{Mueller2003b}). Some fine-tuning will be
required and this is where the linguist comes in: rare data and complicated interacting phenomena may decide between various alternative
theories of a language \citep[Chapter~6]{MuellerGS}. 

What would be missing in such grammars is the meaning component: a distributional analysis provides
one with distribution classes, with syntactic regularities of the language under consideration. This
is true for LLMs and any other outcome of a distributional analysis unless semantic information is
explicitly encoded in the input and linked to real world experiences.
  LLMs do contain semantic knowledge. \citet[358]{Piantadosi2024a} points out that it is
  interwoven with syntactic information. However, the important point when it comes to human cognition is
  that the semantic knowledge in LLMs is not grounded. \citet[2]{JonesBergenTrott2024a} discuss
  sentence-picture verification tasks. For example, hearers can infer from the sentences ``He
  hammered the nail into the wall.'' and ``He hammered the nail into the floor.'' that the nail is
  horizontal in the situation described by the first sentence and vertical in the second. This
  information is not explicitly coded in the sentences, so LLMs, which are trained on language
  alone, cannot learn this, unless it is made explicit elsewhere in the training material.\footnote{%
See \citet{CM2001a} for computational experiments on language
  acquisition with grounding in the framework of Construction Grammar and \citet{Steels2003a} for
  experiments with grounded communication in robotics. \citet{BeulsVanEcke2024a} extensively
  discuss the shortcomings of LLMs that are due to there representations not being grounded and they
  suggest ways to model grounded language acquisition. \citet{JonesBergenTrott2024a} discuss first
  experiments with Multimodal LLMs and point out some shortcomings of current architectures.%
}
%\itdopt{Add note about semantic knowledge}
Therefore, it is really surprising to see Construction Grammarians
praising large language models as theories of human language. Wasn't it Construction Grammariens who
told everybody in the field that human cognition is grounded
\citep{Barsalou2008a} and that language is not just abstract
% Tomasello2003: Fillmore77, Fillmore77b, Langacker
syntax and cannot be learned as such without a connection to semantics and the real world
\parencites[44]{Klein86a-u}[113]{Tomasello2003a}[Section~4.2.3, 4.2.8]{AL2011a-u}? With grasping the communicative
intention and attention sharing? Already in 1986, Klein pointed out that no human being could learn
Chinese by sitting in a room continually exposed to Chinese from loudspeakers.\footnote{%
Klein speculated that at most phonological regularities can be learned and \citet{NHSA2004a} showed
that humans can detect regularities by just being exposed to a continuous speech stream
of syllables of various forms. \citet[44]{Sogaard2023a} pointed out that two year old infants can learn
from TV, but TV involves another modality, the language is grounded \citep{Rice1983a}.
} This just would not work. But this is how LLMs learn: they just see masses and masses of
text. BERT was trained by guessing masked words in a sentence and by guessing the next
sentence. Children do not play such games. Instead, they have to solve a very hard puzzle on their own:
the segmentation of the speech signal. They have to find out what the units are in order to be able
to discover what they mean. 
% OK. This could be done in principle, the guessing game is more problematic, since it has nothing
% to do with human acquisition.
%
% Any theory of language acquisition has to take into account lifelong
% learning. Humans learn a language by using it. Since LLMs have clearly separated training and
% deployment phases they cannot be considered as ``the leading current theories of how speakers learn
% [\ldots] these restrictions'' \citet[\page 34]{AmbridgeBlything2024a}.\footnote{%
% I thank Remi van Trijp for pointing this out to me.} 
As \citet{BK2020a} pointed
out: BERT and ChatGPT and the like do not have a clue about what they are ``saying''. Their representations do not have any
connection to semantics, they are not grounded \citep{BeulsVanEcke2024a}. ChatGPT is a bullshit machine in the sense of
\citet{HicksHumphriesSlater2024a}, it is not and it does not contain a linguistic theory, not even a
wrong one.
%
% Chinesisch https://www.youtube.com/watch?v=xLg9M_mKrxU


\section{Conclusion}

Large Language Models are not theories of language. To build LLMs, one needs a theory and depending
on the goal to be reached, the theory may be a theory of the human brain. Knowing how a brain is
working does not entail knowledge about language. To do typological research means to compare
thousands of languages. This is done by theoretical linguists on a meta level and not within
neuronal nets trained with input of thousands of languages. Of course, one can imagine typological
research supported by computers, but it would require trained linguists who know what to look for.
The existence and success of LLMs does not entail
that the problem of human language acquisition is solved, since the architecture and the training
process of LLMs is quite different from how human brains develop and how humans acquire
language. However, LLMs show that the data is rich and make it even more plausible that humans are
not born with innate domain specific knowledge about language.

\section*{Acknowledgements}

I thank Rui Chaves, Mark Felfe, Hubert Haider, Joachim Jacobs, Tibor Kiss, Alexander Koller, Roland
Schäfer, Oliver Schallert and Remi van Trijp, Giuseppe
Varaschin for comments and discussion and Konstantin Schulz, Anke Lüdeling, and Martin Klotz for 
discussion and Tine Mooshammer for raising questions about \citew{AmbridgeBlything2024a}.

Oliver Schallert suggested adding more discussion of \citeauthor{Piantadosi2024a}'s paper. Thanks
for this, I think it really improved the paper.



%      <!-- Local IspellDict: en_US -->
