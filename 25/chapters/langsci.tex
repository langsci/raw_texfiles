\section*{Language Science Press: scholar-owned high quality linguistic books}

In 2012 a group of people found the situation in the publishing business so unbearable that they
agreed that it would be worthwhile to start a bigger initiative for publishing linguistics books in
platinum open access, that is, free for both readers and authors. I set up a web page and collected
supporters, very prominent linguists from all over the world and all subdisciplines and Martin
Haspelmath and I then founded Language Science Press. At about the same time the DFG had announced
a program for open access monographs and we applied \citep{MH2013a} and got funded (two out of 18 applications got
funding). The money was used for a coordinator (Dr.\ Sebastian Nordhoff) and an economist (Debora
Siller), two programmers (Carola Fanselow and Dr.\ Mathias Schenner), who worked on the publishing
plattform Open Monograph Press (OMP) and on conversion software that produces various formats (ePub, XML,
HTML) from our \LaTeX{} code. Svantje Lilienthal worked on the documentation of OMP, produced
screencasts and did user support for authors, readers and series editors.

OMP was extended by open review facilities and community-building gamification tools
\citep{MuellerOA,MH2013a}. All Language Science Press books are reviewed by at least two external
reviewers. Reviewers and authors may agree to publish these reviews and thereby make the whole
process more transparent (see also \citew{Pullum84a} for the suggestion of open reviewing of journal
articles). In addition there is an optional second review phase: the open
review (see the blog posts by Sebastian Nordhoff about the reviewing options at Language Science
Press\footnote{%
\url{https://userblogs.fu-berlin.de/langsci-press/2015/05/27/axes-of-open-review/}, 2020-09-03.
}). This second optional reviewing phase is completely open to everybody. The whole community may comment on the document
that is published by Language Science Press. After this second review phase, which usually lasts for
two months, authors may revise their publication and an improved version will be published. The
English version of this book was the first book to go through this open review phase. The Chinese
translation was also open for comments on Paperhive. Readers left more than 2500 comments\footnote{%
\url{https://paperhive.org/documents/items/Zf2Qf47i6nf2}, 2020-09-03.}, which were automatically fed into the version control
and bug tracking system used by Language Science Press\footnote{%
\url{https://github.com/langsci/177/}, 2020-09-03.
}.

Currently, Language Science Press has 26 series on various subfields of linguistics with high
profile series editors from all continents. There are 437 members in the respective editorial boards
coming from 49 countries. We have 134 published books with more than 1 Mio downloads.\footnote{%
Downloads by robots excluded, the English version of this textbook was downloaded over 40,000 times
since 2016.
} 1196 authors from 53 countries have published books or chapters with Language Science Press as of March 2020 and there are
572 expressions of interest. 
%Two multi-volume handbooks, one on HPSG and one on LFG, are in
%preparation \citep{HPSGHandbook,LFGhandbook}.


Series editors are responsible for delivering manuscripts that are typeset in \LaTeX{}, but they are
supported by a web-based typesetting infrastructure that was set up by Language Science Press and
there is also conversion software converting Word manuscripts into \LaTeX{}. Proofreading is
community-based. Until now 224 people helped improve our books. Their work is documented in the
Hall of Fame: \url{http://langsci-press.org/hallOfFame}.


Language Science Press is a community"=based publisher, but apart from the press managers Martin
Haspelmath and me, there are two people who are employed for the central organization and
typesetting: Sebastian Nordhoff, who is also a press manager, and Felix Kopecky, who does 
typesetting. Both have 50\,\% positions. In the period of 2018--2020, these two positions got payed
with the help of financial support by 115 academic institutions including  
Harvard, the MIT, and Berkeley and by societies like EuroSLA.\footnote{%
  A full list of supporting institutions is available at:
  \url{http://langsci-press.org/knowledgeunlatched}.
} The Language Science Press approach is endorsed by the leading scholars Noam Chomsky, Adele
Goldberg, and Steven Pinker, who sent letters of support in 2017.\footnote{%
``Very pleased to learn about this fine initiative, a most valuable way to
bring to the general public the results of scholarly work.  It's a
cliché, but true, that we all stand on the shoulders of giants, and rely
on the cultural wealth provided to everyone by past generations.  It is
only proper that the public should gain access to whatever contemporary
scholarship can contribute, and the ideas outlined here seem to be a
very promising way to realize this ideal.'' Noam Chomsky, 2017-02-01.
 
``Language Science Press is setting a standard for freely accessible
articles and books that are carefully reviewed.'' Adele Goldberg, 2017-05-02. 

``Sharing data and methods is one of the pillars of scholarly inquiry. The knowledge created by
scholars belongs to everyone, and open access publications are a major pathway to realizing that
ideal. Language Science Press, together with Knowledge Unlatched, provides an excellent way for us
to make our findings available to the global public.'' Steven Pinker, 2017-01-22. 
} The fundraising for the period 2021--2023 is ongoing.

If you think that textbooks like this one should be freely available to whoever wants to read them
and that publishing scientific results should not be left to profit-oriented publishers, then you
can join the Language Science Press community and support us in various ways: you can register with Language Science Press and have your name
listed on our supporter page with more than 1000 other enthusiasts, you may devote your time and help
with proofreading. We are also looking for institutional supporters like foundations,
societies, linguistics departments or university libraries. Detailed information on how to support
us is provided at the following webpage: \url{http://langsci-press.org/supportUs}.
In case of questions, please contact me or the Language Science Press coordinator at \href{mailto:contact@langsci-press.org}{contact@langsci-press.org}.


~\medskip

\noindent
Berlin, September 04, 2020\hfill Stefan Müller


%      <!-- Local IspellDict: en_US-w_accents -->
