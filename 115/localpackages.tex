% add all extra packages you need to load to this file
\usepackage{tabularx}

%%%%%%%%%%%%%%%%%%%%%%%%%%%%%%%%%%%%%%%%%%%%%%%%%%%%
%%%                                              %%%
%%%           Examples                           %%%
%%%                                              %%%
%%%%%%%%%%%%%%%%%%%%%%%%%%%%%%%%%%%%%%%%%%%%%%%%%%%%
%% to add additional information to the right of examples, uncomment the following line
% \usepackage{jambox}
%% if you want the source line of examples to be in italics, uncomment the following line
% \renewcommand{\exfont}{\itshape}
\usepackage{./langsci/styles/langsci-optional}
\usepackage{langsci-gb4e}
\usepackage{langsci-lgr}
\usepackage{hhline}
\usepackage{multicol}

% \usepackage{qtree} %http://tug.ctan.org/tex-archive/macros/latex/contrib/qtree/
\usepackage{float}
% \usepackage{tikz-qtree}
% \usepackage[icelandic, swedish, american]{babel}
\usepackage{soul}
\usepackage[justification=raggedright,singlelinecheck=false]{caption}
\usepackage[justification=raggedright,singlelinecheck=false]{subcaption}
\usepackage{todonotes}
% \usepackage{linguex} %breaks gb4e

\usepackage{pifont}

\usepackage[normalem]{ulem}  %normalem option makes \emph give italics, not underline
% \usepackage{chngcntr}
\usepackage{enumitem}
%
% \usepackage{poetrytex}
\usepackage{marvosym}
% \usepackage{xunicode}

%% PACKAGES
%%
\usepackage{tikz}
\usepackage{forest}
\usepackage{forest-animate}
  \usetikzlibrary{arrows.meta}
  \usetikzlibrary{shapes.geometric}
  % Optional PGF libraries
  \usetikzlibrary{arrows}
  \useforestlibrary{linguistics}
  \forestapplylibrarydefaults{linguistics}
% \usepackage[stable]{footmisc} %disabled as this messes with footnote spacing
\usepackage{csquotes}
\usepackage{textglos}

\usepackage{stmaryrd}
\usepackage{siunitx}
\sisetup{detect-weight=true, detect-family=true,output-decimal-marker = {.},table-format=2.1}
\usepackage{colortbl}

%compatibility of pgfplots and forest, see http://tex.stackexchange.com/a/330076
\makeatletter
\let\pgfmathModX=\pgfmathMod@
\usepackage{pgfplots}%
\let\pgfmathMod@=\pgfmathModX
\makeatother

\usepackage{listings}
