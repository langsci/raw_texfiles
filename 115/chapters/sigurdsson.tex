\documentclass[output=paper]{LSP/langsci}
\author{Halldór Ármann Sigurðsson\affiliation{Lund University}}
\title{Who are we -- and who is I? About Person and SELF}
% \epigram{Change epigram}
\abstract{This paper discusses the semantics and syntax of the first person pronouns \textit{we} and \textit{I}, in particular with regard to the Event/Speech Participant Split evidenced in clauses like (i).

(i)  \textbf{We} finally beat Napoleon at Waterloo two centuries ago.

The propositional event participants (the “Napoleon beaters”) are not involved in the speech event (the utterance of (i)), and the speech participants are not involved in the propositional event (“beating of Napoleon”). Nevertheless, the pronoun \textit{we} somehow links the speaker and the theta set (\{\textsubscript{θ} x\textsubscript{1,} … x\textsubscript{n}\} or simply \{θ\})  of “Napoleon beaters”. The paper adopts the idea that person values (1, 2, 3) are computed in syntax, and that the elements entering this computation are: A general abstract Person feature (Pn), vP-internally generated NPs (\{θ\}), and speaker and hearer features (Λ features) at the phase edge. It is this computation that yields the speaker–\{θ\} linking embodied in \textit{we} (mending the Event/Speech Participant Split). Evidence that Pn can be independently computed for each phase comes from self-talk, first discussed as a linguistically relevant phenomenon by Anders \citet{Holmberg2010yourself}. The paper also suggests that the secondary SELF readings seen in logophoric phenomena arise form positive setting of the Pn feature, and that the value +Pn is responsible for the “human bias” of plural pronouns.}
\ChapterDOI{10.5281/zenodo.1116767}

\maketitle

\begin{document}

\section{Introduction}\label{sec:Sigurdsson:1}

Consider the sentence in \REF{ex:Sigurdsson:1}.

\ea%1
  \label{ex:Sigurdsson:1}
	  \textbf{We} beat them!
\z

Who are “we” in this sentence? The question might seem to have a simple and an obvious answer: “Well, you and somebody else, of course!” That answer would accord with the common understanding that the first person plural pronoun has the meaning in \REF{ex:Sigurdsson:2} (see, e.g., \citealt{Cysouw2003}, \citealt[82ff.]{Siewierska2004}).

\ea%2
    \label{ex:Sigurdsson:2}
	  we = ‘the speaker augmented by X’ (‘the speaker + X’ for short)
\z

However, this understanding is incorrect. Certainly, \textit{we} is commonly interpreted as ‘the person who is speaking and someone else’, but it is easy to come up with sentences where this is not the meaning of \textit{we}, such as the one in \REF{ex:Sigurdsson:3}.

\ea%3
    \label{ex:Sigurdsson:3}
	   I’m a Tottenham fan – and \textbf{we} beat Arsenal yesterday!
\z


I have absolutely no relation with Tottenham Hotspurs other than by some coincidence being a fan since I was a kid more than a thousand miles north of London, and yet \REF{ex:Sigurdsson:3} makes perfect sense to me.  Well, in this case you could say: Ok, the pronoun does not actually mean ‘the speaker and someone else’ but rather ‘a set or a group to which the speaker belongs’ – in this case the set containing roughly the Tottenham club and its supporters. However, even this broad understanding is too narrow, as suggested by the example in \REF{ex:Sigurdsson:4}.

\ea%4
    \label{ex:Sigurdsson:4}
	  \textbf{We} finally beat Napoleon at Waterloo two centuries ago.
\z


This sentence is not about the speaker – he or she is obviously not included in the set of individuals and forces that finally beat Napoleon and his army at Waterloo on the 18\textsuperscript{th} of June 1815. Rather, it is about a set of actors (‘Napoleon beaters’), a \textsc{theta set}, with which the speaker identifies himself\slash herself, for whatever reasons. This is explicitly stated in \REF{ex:Sigurdsson:5}.

\ea%5
    \label{ex:Sigurdsson:5}
\ea
  A \textsc{theta set}, \{\textsubscript{θ} x\textsubscript{1,} … x\textsubscript{n}\} (or simply \{θ\}), is the set of individuals/entities that bear   or carry out a theta role, θ.

\ex  The pronoun \textit{we} denotes a theta set with which the speaker identifies himself/herself.
\z
\z

Identifying oneself with some set or group is different from being a member of it. So, if I actually would say the sentence in \REF{ex:Sigurdsson:3}, my friend John, an even more devoted Tottenham fan might respond: “WE who? You never show up on matchdays!” John is in his full right to question my claim to belong to the we-set that beat Arsenal, while I, in turn, am in my full right to empathize or even identify myself with the ‘Arsenal beaters’. Crucially, the use of \textit{we} is based on the speaker’s own judgment and others do not necessarily share that judgment. By using \textit{we} I can even empathize with the whole of humanity (across time and space), as in \REF{ex:Sigurdsson:6}.

\ea%6
    \label{ex:Sigurdsson:6}
     There can be no doubt that \textbf{we} will encounter intelligent beings from other solar systems in the third millennium.
\z

On the other hand, the theta set cannot usually contain anything but humans. Thus, \textit{we} in \REF{ex:Sigurdsson:7} is normally interpreted as referring to humans only and not to, say, humans and bears. Call this \textsc{the human bias}.\footnote{Partly non-human readings can be coerced in certain contexts, in particular under partial coreference\is{reference} as in  “Bears first came to Europe hundreds of thousand years ago and \textbf{we} have been coexisting here for at least the last 40~000 years”. The relevant generalization is that \textit{we} must refer to conscious SELFs, either exclusively (the normal case) or at least partly (under coercion).}

\ea%7
    \label{ex:Sigurdsson:7}
	   \itshape\textbf{We} have lived in Europe for at least 40~000 years.
\z

Many or most of the observations regarding \textit{we} (and \textit{I}) that I will be discussing have parallels for \textit{you}, but, for simplicity, I will for the most part limit my discussion to the first person pronoun. I will also set \isi{Number} aside. It interacts in intriguing ways with \isi{Person} in morphological \isi{agreement} systems, but it does not seem to do so directly in syntax. The pronoun \textit{we} is not the plural of \textit{I}; it does not mean ‘many speakers’ or many ‘Is’ (\citealt{Boas1911,Benveniste1966,Lyons1968,Cysouw2003,Siewierska2004,Bobaljik2008}, among many). “Clusivity”, as we will see, does not involve \isi{Number}.

\sectref{sec:Sigurdsson:2} presents initial thoughts on the relation between the speaker and theta sets. \sectref{sec:Sigurdsson:3} discusses the first person singular pronoun, the notion of primary and secondary SELF, and presents a number of secondary SELF contexts, including the context of self-talk (discussed in \citealt{Holmberg2010yourself}). \sectref{sec:Sigurdsson:4} discusses \isi{Person} and SELF in a neo-performative perspective, developing the central hypothesis that the speaker–\{θ\} linking embodied in \textit{we} is brought about by \isi{Person} computation in syntax, further suggesting that the activation of a secondary SELF arises from a positive setting (+Pn) of the abstract \isi{Person} feature. In addition, it is suggested that +Pn is responsible for the human bias of plural pronouns.

\section{So, more exactly, who are we?}\label{sec:Sigurdsson:2}

The fact that the theta set represented or expressed by \textit{we} does not need to actually \textit{include} the speaker (although it ‘involves’ the speaker), and that the meaning of \textit{we} is thus not ‘the speaker + X’ (or ‘the speaker + \{θ\}’), has not, to my knowledge, been generally noticed or problematized. In the sentence in \REF{ex:Sigurdsson:4}, the speaker certainly identifies himself/herself with a theta set of ‘Napoleon beaters’, but he or she is not one of them – nor are there any ‘Napoleon beaters’ involved in or responsible for the speech act. Refer to this as the \textsc{event\slash speech participant split}, E/SP split, for short.\footnote{It is also found for plural \textit{you} (as in “You finally beat Napoleon at Waterloo two centuries ago”).}

  ‘The speaker + X’ or ‘the speaker + \{θ\}’, then, is not an insightful paraphrase of the meaning of \textit{we}. Consider the reverse paraphrase in \REF{ex:Sigurdsson:8}.

\ea%8
    \label{ex:Sigurdsson:8}   we = ‘\{θ\} augmented by the speaker’  (‘\{θ\}+ speaker’ for short)
\z

This is closer to the mark. However, ``augmented'' in this formula (and the + sign in the short version) is misleading. The relevant relation between the speaker and \{θ\} is not the logical conjunction, \& or ∧, nor is it natural language \textit{and}. Consider the simple \REF{ex:Sigurdsson:9}.

\ea%9
    \label{ex:Sigurdsson:9}
  Mary and John got married yesterday.
\z

Here \textit{Mary and John} make up a homogeneous set in the sense that the potential distinction between \textit{Mary} and \textit{John} is irrelevant. This is clearly not the case for the speaker and the ‘Napoleon beaters’ in \REF{ex:Sigurdsson:4}. The sentence in \REF{ex:Sigurdsson:4} is not about a set of ‘Napoleon beaters’ \textit{and} the speaker, as stated in \REF{ex:Sigurdsson:10}.

\ea%10
    \label{ex:Sigurdsson:10}
      we in \REF{ex:Sigurdsson:4} \textbf{≠} \{Napoleon beaters\} \& the speaker
\z

Rather, as already noted, \REF{ex:Sigurdsson:4} is about a set of ‘Napoleon beaters’ that is somehow related to the speaker in the speaker’s own view. This theta set–speaker relation is an instantiation of a more general event/speech participant linking comprising a theta set–hearer relation as well (embodied by plural \textit{you}). Focusing on the speaker side we can call the theta set–speaker relation \textsc{\{θ\}-S linking} and use the sign ↔ to denote it. We thus replace \REF{ex:Sigurdsson:8} by \REF{ex:Sigurdsson:11}.

\ea%11
    \label{ex:Sigurdsson:11}
   we = ‘\{θ\} linked to the speaker in his or her own judgment’, \{\{θ\} ↔ speaker\} for short
\z

The notion of ``linking'' here is vague, deliberately so, as it is hard to pin down its exact nature: \{\{θ\} ↔ speaker\} is often used to express plain additive readings (the additive relation being subsumed under the more general linking relation), but it crucially involves the speaker’s own judgement, and, as we have seen, it also expresses non-additive E/SP split readings. In \sectref{sec:Sigurdsson:4}, I will suggest that it arises from \isi{Person} computation.

  \{θ\}-S linking applies generally to the pronoun \textit{we}, regardless of its position or function, and it does not necessarily involve sympathy (even though it often does), whereas at least some minimal empathy seems to be required. The essentially non-inclusive relation involved defies the idea (\citealt{Postal1966,Elbourne2005}) that all personal pronouns are complex DPs, with a pronominal head and a deleted or a reconstructed NP. The pronoun \textit{we}, as we have seen, cannot \textit{generally} be analyzed as [we [NP]], for example as [we [Napoleon beaters]] or [we [unspecified people]]; such a DP would wrongly \textit{include} the speaker in the theta set rather than merely linking the speaker and the theta set. Similarly, as we will see shortly, the pronoun \textit{I} cannot always be paraphrased as ‘I, the speaker’ (with roughly the structure [I [the speaker]]). Pronouns obviously \textit{can} have reconstructed complex DP interpretations, but the relevant point here is that they \textit{need} not have any such interpretation. I thus adopt the view that plain pronouns can be pure DPs, without an NP complement: [\textsubscript{DP} we], [\textsubscript{DP} I], etc.

It is obvious that the special nature of \{θ\}-S linking does not stem from the theta set (of Napoleon beaters, or whatever) – it must instead be the case that it stems from the speaker category. In the following I will reflect on the nature of the speaker category and on the intriguing question of how it gets activated or involved in the pronoun \textit{we}.

\section{On the speaker category: Who is I – and SELF?}\label{sec:Sigurdsson:3}

The speaker category is normally represented by the first person singular pronoun, \textit{I}, but it is not equivalent with it. There are certain contexts where the pronoun\textit{ I} does not relate to or denote the speaker, but to what might be referred to as a secondary SELF, overshadowing the primary SELF of the actual speaker. One such context is regular direct speech, as in \REF{ex:Sigurdsson:12}, where “Christer” is the speaker.

\ea%12
    \label{ex:Sigurdsson:12}
   [Christer speaks:]  Halldór said to Anders: “\textbf{I} will cite \textbf{your} paper again!”
\z

It is evident that direct speech somehow embeds a silent secondary SELF that is referred to by the first person singular pronoun. A related and a much-discussed phenomenon (see \citealt{Bianchi2003Syntax,Schlenker2003,Anand2006}) is \textsc{person shift} (indexical shift), as in the Persian clause in \REF{ex:Sigurdsson:13}, where \textit{man} ‘I’ and \textit{tora} ‘you’ refer to \textit{Ali} and \textit{Sara}.\footnote{From \citealt{Sigurðsson2004}, based on pers. comm. with G h. Karimi Doostan. Thanks also to Alireza Soleimani. The sentence is ambiguous between the shifted reading given in \REF{ex:Sigurdsson:16} and the regular non-shifted reading ‘Ali told Sara that I like you’ (irrelevant here).}

\ea%13
    \label{ex:Sigurdsson:13}

\gll  {\upshape [Amir speaks:]}  Ali  be  Sara  goft  [ke  \textbf{man}  \textbf{tora}  doost  daram].\\
      {} Ali  to  Sara  said  that  I  you  friend  have.1\textsc{sg}\\
\glt  ‘Ali told Sara that \textbf{he} likes \textbf{her}.’
\z

Yet another case of the first person singular pronoun not really referring to the speaker of the clause involves bound variable readings, as in the subordinate clause in \REF{ex:Sigurdsson:14}.\footnote{See \citealt{Rullmann2004} for a clear discussion of bound variable readings of first and second person pronouns.}

\ea%14
    \label{ex:Sigurdsson:14}
	  Only I got a question that \textbf{I} understood.
\z

The natural interpretation of this clause is not ‘The speaker of this clause is the only one who got a question that this particular speaker understood’. Rather, it is the bound variable reading ‘There was only one person \textit{x}\textsubscript{i} who got a question that \textit{x}\textsubscript{i} understood (and \textit{x}\textsubscript{i} happens to be me, the speaker of this clause)’. That is: The subject of the subordinate clause does not by itself refer to the speaker, only referring to the actual speaker indirectly, by virtue of being a variable bound by the matrix subject (which in turn does refer to the speaker). Bound first (and second) person variables of this sort are sometimes called ``fake indexicals'' \citep{Kratzer2009}.

  These well-known observations show that the first person singular pronoun does not equal the actual speaker. The pronoun \textit{I} canonically denotes the speaker but that is evidently not all \textit{I} can do. In certain contexts, it can represent a SELF that is different from, albeit somehow dependent on that of the actual speaker’s. In indexical shift and direct speech contexts, as in \REF{ex:Sigurdsson:12} and \REF{ex:Sigurdsson:13}, the distinction between the primary SELF of the actual speaker and the secondary SELF represented by \textit{I} is quite clear. It is less distinct but also discernable in bound variable readings, as in \REF{ex:Sigurdsson:14}.\footnote{Typical \textsc{imposters} are third person expressions, such as \textit{Daddy}, \textit{Mom}, \textit{my boy}, that are used to express a first or a second person relation to the speaker (see \citealt{CollinsPostal2012,WoodSigurðsson2011}), as in “Daddy already told you that” or “How is my boy?” The first person singular pronoun in the direct speech, indexical shift, and the bound variable contexts in (\ref{ex:Sigurdsson:12}–\ref{ex:Sigurdsson:14}) is an inverse imposter of sorts. That is: The pronoun expresses a third person relation (to the actual speaker) in spite of its first person camouflage.}

  A secondary SELF can also hide behind a third person pronoun. This is the case in \textit{de se} (lit. ‘of oneself’)  readings of bound third person pronouns, as \textit{she} in \REF{ex:Sigurdsson:15}.

\ea%15
    \label{ex:Sigurdsson:15}
   Mary looked into the mirror and thought she looked good.
\z

The salient reading of the subordinate clause is the \textit{de se} reading that Mary thought of herself “I look good”. The \textit{de re} reading `she looks good’ is far-fetched but not in principle excluded (in case Mary for some reason, such as insanity or drunkenness, thought she was looking at someone distinct from herself). \textit{De se} is the only possible reading of PRO in control infinitives such as the one in \REF{ex:Sigurdsson:16} \citep{Chierchia1989}.\footnote{Potential \textit{de re} readings in adverbial PRO infinitives (discussed in \citealt[32–33]{Landau2013}) are irrelevant in the present context.}

\ea%16
    \label{ex:Sigurdsson:16}
  Mary hoped to look good.
\z

Here Mary cannot possibly, not even by accident, have someone else’s looks in mind (\textit{de re}). There is no possible world where Mary could be thinking: “I\textsubscript{i} hope she\textsubscript{k} will be looking good”.

  The presence of a secondary SELF is also discernable in contexts that are commonly referred to as ``logophoric'', where the “speech, thoughts, feelings, or general state of consciousness” of someone distinct from the speaker are reported \citep[141]{Clements1975}. I will instead use the term \textsc{secondary selfhood}, reserving ``logophoric'' and ``logophoricity'' for other purposes (see \sectref{sec:Sigurdsson:4}). It seems that most languages do not overtly signal secondary selfhood, but some do, either by using special markers for this purpose (see \citealt{Sells1987} and the references there) or by some specific use of pronouns that are also used for other purposes, commonly reflexive pronouns. \ili{Icelandic} is a language of this latter type, using long-distance reflexives, LDRs, to mark secondary selfhood, as described by \citet{Thráinsson1976,Thráinsson1990,Thráinsson2007}; see also \citealt{Maling1984} and \citealt{Sigurðsson1990}.\footnote{Similar facts are found in other languages. See for example \citealt{Giorgi2006} on \ili{Italian}.} The contrast in \REF{ex:Sigurdsson:17} illustrates this, as will be explained below.

\ea%17
    \label{ex:Sigurdsson:17}
\gll  Aðeins  forsetinn\textsubscript{i}  heldur  að  öll  þjóðin  elski sig\textsubscript{i}/hann\textsubscript{i}.\\
  only  president.the  believes  that  all  country  love  self/him\\
\glt ‘Only the president believes that all the people love him.’
\z

LDR is optional, coreference\is{reference} of the matrix subject and the subordinate object being expressed by either the reflexive \textit{sig} or the pronoun \textit{hann}. There is a subtle difference, though, such that only the reflexive reflects the matrix subject’s point of view or consciousness. While the pronominal reading is that the president is the only one who believes that all the people love the president,\footnote{We can disregard the reading, irrelevant here, where \textit{hann} refers to somebody other than the president. Local coreference of the object with the subordinate subject is usually expressed by the complex reflexive \textit{sjálf- sig} (\textit{sjálfan sig}, \textit{sjálfa sig}, etc.); see \citealt[56]{Sigurjónsdóttir1992} (and \citealt[464]{Thráinsson2007} and further references there).} the reflexive reading is the bound variable \textit{de se} reading that the president is the only person who believes of himself that the whole people love him, as explicitly stated in \REF{ex:Sigurdsson:18}.

\ea%18
    \label{ex:Sigurdsson:18}
    \ea  {HANN}: the president is the only one who believes that all the people
    love the president
\ex {SIG}: the president (\textit{x}\textsubscript{i}) is the only person who believes of himself that all the   people love him (\textit{x}\textsubscript{i})
\z
\z

Notice that the English translation in \REF{ex:Sigurdsson:17} is ambiguous between the two readings. English does not have means to lexically distinguish between these readings – but they are both there, just as in languages with overt markers of secondary selfhood.

The capacity to linguistically reflect someone else’s mind or internal world is a remarkable phenomenon. Let us refer to it as the \textsc{syntactic empathy capacity} (cf.  \citealt{KunoKaburaki1977}).\footnote{This is sufficiently accurate for my present purposes, but it is an oversimplification. As argued elsewhere (see \citealt[49]{Sigurðsson2010mood}), the relevant notion is a “negative” one, namely \textsc{absent speaker truthfulness responsibility} (signaled by the subjunctive mood in \ili{Icelandic}).} In the examples we have been looking at so far, the syntactic empathy is external, so to speak, reflecting secondary SELFs (represented by first or third person pronouns or by PRO) that are distinct from the speaker. However, perhaps not surprisingly, the syntactic empathy can also be internal, directed towards the speaker himself or herself. That is: Speakers can simultaneously (i.e., in a single utterance) talk about their present speech event SELF and another potential SELF of theirs, not present in the speech event. This \textsc{self-split} is nicely illustrated by the indicative/subjunctive contrast in the \ili{Icelandic} \REF{ex:Sigurdsson:19} (see \citealt[325–326]{Sigurðsson1990}).

\ea%19
    \label{ex:Sigurdsson:19}
    \ea     \label{ex:Sigurdsson:19a}
\gll Ég  vissi  að  María  \textbf{kom}  heim.\\
    I  knew  that  Mary  came.\textsc{ind}  home\\
\glt    ‘I knew that Mary came home.’
\ex     \label{ex:Sigurdsson:19b}
\gll  Ég  vissi  að  María  \textbf{kæmi}  heim.\\
    I  knew  that  Mary  came.\textsc{subj}  home\\
\glt     ‘I knew that Mary would come home.’
\z
\z

While \REF{ex:Sigurdsson:19a} simply reports that the speaker was aware of the fact that ‘Mary came home’ at some time point in the past, the subjunctive clause in \REF{ex:Sigurdsson:19b} reflects on the speaker’s past (secondary) SELF, saying that his or her past SELF was confident (rather than actually knew) that ‘Mary would come home’ at some time point later than his or her past time of consciousness. Notice that the existence of the two readings is not dependent on the morphological mood distinction; it is only made extra visible by it.\footnote{Languages that lack inflectional subjunctive have the same semantics, often morphologically unmarked but sometimes marked by other means than mood distinctions, for example by modals as in the English translation of \REF{ex:Sigurdsson:19b}.}

\textsc{Self-talk} is another context with a self-split: a \textsc{secondary speaker SELF}, in addition to the primary speaker SELF. Anders Holmberg has written an essay (\citeyear{Holmberg2010yourself}) about the interesting but hitherto unnoticed properties of self-talk, where self-talk “[is] speaking to yourself, the self being speaker as well as addressee” (\citeyear[57]{Holmberg2010yourself}). Thus, as Holmberg shows, you can refer to yourself either as “I” or as “you” in the context of self-talk. A few of Holmberg’s examples are given below.

\ea%20
    \label{ex:Sigurdsson:20}
    \ea  You’re an idiot.
\ex I’m an idiot.
\z
\z

\ea%21
    \label{ex:Sigurdsson:21}
  \ea  You’re hopeless.
\ex  I’m hopeless.
\z
\z

\ea%22
    \label{ex:Sigurdsson:22}

\ea What’s wrong with you?
\ex  What’s wrong with me?
\z
\z

\ea%23
    \label{ex:Sigurdsson:23}

  \ea[]{I think I’ve/you’ve had it}
\ex[*]{You think I’ve/you’ve had it.  [\textit{*} in self-talk]}
\z
\z

\ea%24
    \label{ex:Sigurdsson:24}
\ea[]{You’re driving me mad.}
\ex[*]{I’m driving you mad.  [\textit{*} in self-talk]}
\z
\z

On the basis of contrasts such as the ones in \REF{ex:Sigurdsson:23} and \REF{ex:Sigurdsson:24} between the “I mode” and the “you mode” Holmberg concludes that self-talk “\textit{you} can’t [usually] refer to the self as holder of thoughts or beliefs, in self-talk”, nor can it “refer to the self as an experiencer of feelings or holder of intentions or plans” (\citeyear[59--60]{Holmberg2010yourself}). This is further demonstrated by the sharp contrast in \REF{ex:Sigurdsson:25}.

\ea%25
    \label{ex:Sigurdsson:25}
\ea[]{I hate you!}
\ex[*]{You hate me!}
\z
\z

All these observations show that language can operate with at least two distinct SELFs, the primary SELF of the speaker and a secondary SELF of either the speaker or of someone else. \citet[60]{Holmberg2010yourself} observes that \textit{you} in (normal) self-talk “never answers back, however much he is insulted … because he can’t think; he is a mindless self. The property shared by the referent of \textit{you} in self-talk and the referent of \textit{you} in dialogue is that they are not controlled by the mind of the speaker: dialogue-you because it has a different mind, self-talk-you doesn’t have a mind”.

I agree, of course, that self-talk \textit{you} has a more limited mind than the speaker and self-talk \textit{I}, but I suspect that Holmberg overstates its “mindlessness”. It cannot be the agent or controller of speech, thought, feelings – cannot answer back as Holmberg notes – but it is not like a lifeless thing. I believe it is more like the other types of secondary SELFs we have been looking at: an incomplete and an inactive SELF with no executive power, verbally or otherwise, but with the capacity of perceiving. Insulting or encouraging it is thus not pointless or an expression of madness, as insulting or encouraging a table or a pen would be in most situations in most cultures. It is thus warranted, I believe, to make a distinction between the fully active primary SELF of the speaker and a less active secondary SELF of either the speaker or someone else.

While the distinction between a primary and a secondary speaker SELF is upheld in normal self-talk, this distinction seems to break down in abnormal self-talk, symptomatic of dementia and madness, such that \textit{you} gains the status of an entirely separate SELF (of an addressee), and “may, for example, answer back when being reproached” (\citealt[63]{Holmberg2010yourself}, building on Crow’s theory (\citeyear{Crow1998,Crow2004}) of schizophrenia as a linguistic disorder).

\section{Person and selfhood: a neo-performative approach}\label{sec:Sigurdsson:4}

There is no way of expressing the word \textit{you} without that being the “responsibility” of some \textit{I}. Holmberg points out (\citeyear[60–61]{Holmberg2010yourself}) that “when addressing yourself as \textit{you}, there is still an \textit{I} linguistically represented in the sentence, covertly if not overtly”, suggesting that the \textsc{performative} \textsc{hypothesis} was on the right track, after all. According to this (much reviled) hypothesis, any declarative sentence is embedded under a silent performative clause, roughly, “I hereby say to you”. Ross famously advocated for this understanding, roughly as sketched in \REF{ex:Sigurdsson:26} for the simple clause “Prices slumped” (see \citeyear[224]{Ross1970}).

\ea%26
    \label{ex:Sigurdsson:26}
    	   [I hereby say to you] Prices slumped.
\z

Translated into modern generative theory this amounts to saying that the C-edge of the clause contains (among other features) a silent speaker feature or operator that takes scope over the clause (\citealt{Sigurðsson2004,Sigurðsson2014}). However, it cannot really be the case that this edge feature gets directly spelled out as the pronoun \textit{I}. Consider the “Ross formula” in \REF{ex:Sigurdsson:27}.

\ea%27
    \label{ex:Sigurdsson:27}
  	   [I hereby say to you] I know that prices will slump.
\z

The spelled-out \textit{I} could not be a plain copy of the silent edge \textit{I} or vice versa. That is: these two “\textit{I}s” cannot be just two occurrences of the same element, or else all occurrences of \textit{I} would simply refer to the actual speaker (precluding person shift as in \ref{ex:Sigurdsson:12}–\ref{ex:Sigurdsson:14}), and also yielding an insoluble infinite regress problem). Rather, the silent “edge \textit{I}” and the overt \textit{I} are distinct but computationally related elements. And when you think about it, it is actually rather obvious that “first person” is a computed value, normally assigned to an NP (a theta set) that somehow relates to the speaker, much as “second person” is normally assigned to an NP that somehow relates to the hearer or the addressee. In other words, “first person” and “second person” are not primitives of language, whereas (roughly) “speaker” and (roughly) “hearer” arguably are basic notions.

In the a neo-performative and neo-Reichenbachian approach developed in previous work (e.g., \citealt{Sigurðsson2004,Sigurðsson2011,Sigurðsson2014,Sigurðsson2016}), any phase edge contains a number of silent features, \textsc{edge linkers}, that link the inner phase to the next phase up or to the speech act context.\footnote{Cf. \citealt[125, n. 17]{Chomsky2004}. This is inspired by Rizzi’s theory of the left periphery (\citeyear{Rizzi1997}, etc.) and by the work of \citet{Bianchi2003Syntax,Bianchi2006,Schlenker2003,Frascarelli2007}, and others. The literature on this is rapidly growing; see for instance \citealt{Giorgi2010,Sundaresan2012,Haddad2014,MartínHinzen2014}. The approach adopted here differs from other structural neo-performative approaches (e.g. \citealt{TennySpeas2003}) in that it claims, in the spirit of \citealt{Ross1970}, that the speaker\slash hearer categories are themselves silent \textit{by necessity}, even though they often have overt correlates somewhere else in the structure (providing indirect evidence for their activeness).}  I will not go deep into the details of this approach here. Suffice it to say that abstract speaker and hearer features, referred to as the logophoric agent and the logophoric patient, \textbf{Λ\textsubscript{A}} and \textbf{Λ\textsubscript{P}}, are among the edge linkers and enter the computation of \isi{Person} (Pn). Any phase that licenses an NP (subject, object, indirect object, etc.) has such linkers as well as an abstract Pn head (and a separate \isi{Number} head, see \citealt{SigurðssonHolmberg2008}). For expository ease, this is sketched in \REF{ex:Sigurdsson:28} for only the simple case of a clause with a defective v (in the sense of \citealt{Chomsky2001}); as defective vP is not a strong phase, the edge linkers are only operative in the C edge in cases of this sort.\footnote{On this approach, as indicated, abstract \isi{Agree} is a computational valuing process (distinct from, albeit related to, morphological \isi{agreement}).}


\ea%28
    \label{ex:Sigurdsson:28}
  	  [\textsubscript{CP}   Λ\textsubscript{A}\tikz[remember picture,baseline=-0.5ex] \node (sigurdash) {-};Λ\textsubscript{P} … \tikz[remember picture,baseline=-0.5ex] \node (sigurPn) {Pn}; …  [\textsubscript{vP}  \tikz[remember picture,baseline=-0.5ex] \node (sigurNP) {NP\textsubscript{α}\textsubscript{Pn}};]]
		\begin{tikzpicture}[remember picture,overlay]
%                   \coordinate [below=5pt of sigurdash.base] (sigurdashbase);
%                   \coordinate [below=5pt of sigurPn.base] (sigurPnbase);
                  \draw[Stealth-Stealth] (sigurdash) -- ++(0,-0.5) -| (sigurPn.250) node[near start, below,align=justify] {\textit{~~Agree}\\(valuing)};
                  \draw[Stealth-Stealth] (sigurNP) -- ++(0,-0.5) -| (sigurPn.290) node[near start, below,align=justify] {\textit{~~Agree}\\(valuing)};
                 \end{tikzpicture}\vspace*{3\baselineskip}
\z

Under \isi{Agree} with the Pn head an NP (NP\textsubscript{α}\textsubscript{Pn}) is valued as either a “personal” or a “non-personal” argument, NP\textsubscript{+Pn} or NP\textsubscript{–Pn}. A “personal” NP (NP\textsubscript{+Pn}), in turn, must get valued in relation to the Λ linkers, as sketched in \REF{ex:Sigurdsson:29} (where the arrow reads ‘gets valued as’).\footnote{The distinction between DPs and NPs is irrelevant in this context, so I am using “NP” as a cover term for both. As seen in \REF{ex:Sigurdsson:29}, NPs may be in the third person either by computation, valued as +Pn, or by default, in which case they are valued as –Pn (“no person” in \citealt{Benveniste1966}). The former typically applies to “personal” definite NPs, while the latter typically applies to indefinite and “non-personal” NPs (see \citealt[168–169]{Sigurðsson2010epp}). So-called “impersonal” pronouns, such as English \textit{one}, \ili{French} \textit{on}, etc., are not “non-personal”. Instead, they are (usually) “non-specifically personal”, valued as +Pn. This extends to arbitrary and generic PRO (inheriting the +Pn valuation under control, see shortly).}

\ea%29
    \label{ex:Sigurdsson:29}
  \begin{tabbing}
   a3. \= NP\textsubscript{+Pn} \= → \= NP\textsubscript{+Pn/+ΛA, –ΛP} \= = \= 2nd person by computation \kill

  a1. \>  NP\textsubscript{+Pn} \> → \> NP\textsubscript{+Pn/+ΛA, –ΛP} \> = \> 1st person by computation\\
  a2. \>  NP\textsubscript{+Pn} \> → \> NP\textsubscript{+Pn/–ΛA, +ΛP} \> = \> 2nd person by computation\\
  a3. \>  NP\textsubscript{+Pn} \> → \> NP\textsubscript{+Pn/–ΛA, –ΛP} \> = \> 3rd person by computation\\
  b.  \>  NP\textsubscript{–Pn} \>   \>                                \> = \> 3rd person by default  (“no person”)
  \end{tabbing}

\z

In passing it is worth noticing that the computation of \isi{Person} largely parallels that of \isi{Tense} (cf. \citealt{Partee1973}). Much as Event Time is computed in relation to Speech Time via Reference\is{reference} Time \citep{Reichenbach1947}, so is an event participant (NP\textsubscript{α}\textsubscript{Pn} or \{θ\}) computed in relation to a speech act participant via abstract \isi{Person} (\citealt{Sigurðsson2004,Sigurðsson2016}). Although I will not do so here, the parallelism could be underlined by talking about Speech \isi{Person}, Event \isi{Person} and Reference \isi{Person}.

The spelled-out pronoun \textit{I}, then, in for example \REF{ex:Sigurdsson:27}, is not (at all) identical with the abstract speaker category. Instead, like the other “truly personal” pronouns, it is the \isi{spell out} of a relation between an NP\textsubscript{α}\textsubscript{Pn} (or a theta set), a general \isi{Person} category (Pn), and the Λ features. Thus, the “speaker” in the “we-formula” in \REF{ex:Sigurdsson:11} is the abstract value +Λ\textsubscript{A}, and its linking to NP\textsubscript{α}\textsubscript{Pn} or \{θ\} yields its theta relatedness. The theta set can also be linked to the hearer feature, +Λ\textsubscript{P}, or to both +Λ\textsubscript{A} and +Λ\textsubscript{P}, as sketched in \REF{ex:Sigurdsson:30} and \REF{ex:Sigurdsson:31} below. It is the computation of the person value (the NP\textsubscript{α}\textsubscript{Pn}/Pn/Λ relation) that “mends” the Event/Speech Participant Split, thereby yielding the speaker–\{θ\} linking embodied in \textit{we}.

The theta set is primary in relation to the speaker and hearer features. Given an event there is always a theta set that saturates it whereas there may or may not be positive speaker or hearer relatedness. This accords with the standard minimalist bottom-to-top approach to the derivation: vP is merged lowest, then \isi{TP}, then \isi{CP} (vP {\textgreater} \isi{TP} {\textgreater} \isi{CP}). While a theta role and therefore some (at least unspecific) theta set is given as soon as the vP predicate has been merged, the speaker and hearer categories are not accessible until at the edge of a phi-complete phase (i.e., not until at the C level in defective v structures like \REF{ex:Sigurdsson:28}). The theta set is open to any interpretation (‘John and Mary’, ‘boat’, ‘God’, etc.) that does not involve the speech participants, including the empty set interpretation \{Ø\}. For the empty set interpretation, the options are as listed in \REF{ex:Sigurdsson:30} (recall that the double pointed arrow denotes the linking between a theta set and a speech participant category, see \REF{ex:Sigurdsson:11}).\footnote{The theta role itself is of course not empty, only the set of individuals or entities (other than the speech act participants) that bear it. Thus, in the sentence “I beat Napoleon” there is the role of a ‘Napoleon beater’ that is carried by \{\{\textsubscript{θ} Ø\} ↔ \{+Λ\textsubscript{A}, –Λ\textsubscript{P}\}\}.}

\ea%30
    \label{ex:Sigurdsson:30}
\ea  \{\{\textsubscript{θ} Ø\} ↔ \{+Λ\textsubscript{A}, –Λ\textsubscript{P}\}\} \hspace{2em} \textit{I}

\ex  \{\{\textsubscript{θ} Ø\} ↔ \{–Λ\textsubscript{A}, +Λ\textsubscript{P}\}\}  \hspace{2em} singular \textit{you}

\ex  \{\{\textsubscript{θ} Ø\} ↔ \{+Λ\textsubscript{A,} +Λ\textsubscript{P}\}\}  \hspace{2em} exhaustively hearer inclusive \textit{we}

\ex  \{\{\textsubscript{θ} Ø\} ↔ \{–Λ\textsubscript{A}, –Λ\textsubscript{P}\}\}  \hspace{2em} expletives
\z
\z

The options for non-empty theta set interpretations are listed in \REF{ex:Sigurdsson:31}.

\ea%31
    \label{ex:Sigurdsson:31}
\ea  \{\{\textsubscript{θ} x\textsubscript{1,} … x\textsubscript{n}\} ↔ \{+Λ\textsubscript{A}, –Λ\textsubscript{P}\}\} \hspace{2em} hearer exclusive \textit{we}

\ex  \{\{\textsubscript{θ} x\textsubscript{1,} … x\textsubscript{n}\} ↔ \{–Λ\textsubscript{A}, +Λ\textsubscript{P}\}\} \hspace{2em} regular plural \textit{you}

\ex  \{\{\textsubscript{θ} x\textsubscript{1,} … x\textsubscript{n}\} ↔ \{+Λ\textsubscript{A,} +Λ\textsubscript{P}\}\} \hspace{2em} general hearer inclusive \textit{we}

\ex  \{\{\textsubscript{θ} x\textsubscript{1,} … x\textsubscript{n}\} ↔ \{–Λ\textsubscript{A}, –Λ\textsubscript{P}\}\} \hspace{2em} computed third person
\z
\z

This exhausts the syntactic options – the speaker and hearer features are thus crucially involved in the computation of both person and “clusivity”.\footnote{For typological overviews of person and “clusivity”, see \citet{Siewierska2004} and \citet{Cysouw2003}. Semantic interpretation at the conceptual-intentional interface is based on both the syntactic computation and post-syntactic pragmatics. I claim that the analysis in \REF{ex:Sigurdsson:30}–\REF{ex:Sigurdsson:31} exhausts the syntactic options, but not all the pragmatically possible ones (cf. \citealt{Bobaljik2008}).} Notice that both inclusive and exclusive readings of \textit{we} are available even in languages like English that do not overtly mark the inclusive/exclusive distinction, as illustrated in \REF{ex:Sigurdsson:32} and \REF{ex:Sigurdsson:33}.

\ea%32
    \label{ex:Sigurdsson:32}
    \upshape Exclusive \textit{we}\\\relax
	  [X speaks]:  Peter, \textbf{we} have decided to help you (Anna and I).
\z

\ea%33
    \label{ex:Sigurdsson:33}
    \upshape Inclusive \textit{we}\\\relax
	  [X speaks]:  Peter, \textbf{we} should go to the movies tonight (the two of us).
\z


The representations in \REF{ex:Sigurdsson:30} and \REF{ex:Sigurdsson:31} are descriptions of pronominal meanings, showing the outcome of pronominal computation (and not the computation process itself). Syntactically, the Λ features at the phase edge normally enter an identity or a control relation with the actual speech event participants \citep{Sigurðsson2004,Sigurðsson2011}. In certain cases, however, they can instead be controlled by overt arguments in a preceding clause. This is what happens in direct speech or quotations, as in \REF{ex:Sigurdsson:12}, and in indexical or person shift examples like \REF{ex:Sigurdsson:13}, as illustrated (for only the C edges) in \REF{ex:Sigurdsson:34} and \REF{ex:Sigurdsson:35}. For simplicity, the \isi{Number} and Pn features involved in argument computations are not shown (but, as stated in \REF{ex:Sigurdsson:28} and \REF{ex:Sigurdsson:29}, only NP\textsubscript{+Pn} feed valuation of Λ\textsubscript{A,} Λ\textsubscript{P}).

\ea%34
    \label{ex:Sigurdsson:34}
	  [Christer speaks]: \textit{Halldor said to Anders: “\textbf{I} will cite \textbf{your} paper again!”}
[\textsubscript{CP} ... \{Λ\textsubscript{A}\}\textsubscript{i} … \{Λ\textsubscript{P}\}\textsubscript{k} … [\textsubscript{TP}  … Halldor\textbf{\textsubscript{j} }… Anders\textbf{\textsubscript{l}} … [\textsubscript{CP} ... \textbf{\{Λ\textsubscript{A}\}\textsubscript{j}} … \textbf{\{Λ\textsubscript{P}\}\textsubscript{l}} … [\textsubscript{TP} … I\textsubscript{j} … you\textsubscript{l} …
\z

\ea%35
    \label{ex:Sigurdsson:35}
\gll	  {\upshape [Amir speaks]:}  Ali  be  Sara  goft  [ke  \textbf{man}  \textbf{tora}  doost  daram].\\
    {} Ali  to  Sara  said  that  I  you  friend  have.1\textsc{sg}\\
\glt  ‘Ali told Sara that \textbf{he} likes \textbf{her}.’

[\textsubscript{CP} ... \{Λ\textsubscript{A}\}\textsubscript{i} … \{Λ\textsubscript{P}\}\textsubscript{k} … [\textsubscript{TP}  … Ali\textbf{\textsubscript{j}} … Sara\textbf{\textsubscript{l}} … [\textsubscript{CP} ... \textbf{\{Λ\textsubscript{A}\}\textsubscript{j}} … \textbf{\{Λ\textsubscript{P}\}\textsubscript{l}} … [\textsubscript{TP} … I\textsubscript{j} … you\textsubscript{l} …
\z


The pronouns themselves are not shifted. Just as in regular unshifted readings they relate to their local Λ features: The meaning of the pronouns \textit{I} and singular \textit{you} is invariably NP\textsubscript{+Pn/+ΛA, –ΛP} and NP\textsubscript{+Pn/–ΛA, +ΛP}, respectively, as stated in \REF{ex:Sigurdsson:29}.\footnote{Contrary to common assumptions, person shift of this sort is cross-linguistically widespread (see for example the general discussion in \citealt{Sigurðsson2014} and the discussion of \ili{Norwegian} in \citealt{Julien2015}); indeed, indexical shift is plausibly a \isi{universal} syntactic option (based on universally available secondary selfhood). It should be noted, though, that quotations have properties that set them apart from regular clauses; they can for instance be pure sound or gesture imitations (see \citealt[80ff]{Anand2006}.). However, the mechanism of person shift as such is the same in quotations as in other person shift contexts: The Λ features at the phase edge are shifted under control by matrix arguments (and \textit{not} in some “semantically free” manner or by discourse antecedents farther away, suggesting that this is a syntactic process subject to locality restrictions).}

  Activation or promotion of a secondary SELF is dependent on positive setting of the person category, +Pn (in the \textit{matrix} clause, see shortly). In contrast, positive setting of the speaker and hearer features is not directly involved (although secondary SELFs may be represented by arguments that are valued as +Λ\textsubscript{A} or +Λ\textsubscript{P}, in addition to +Pn, as in \ref{ex:Sigurdsson:34} and \ref{ex:Sigurdsson:35}).\footnote{But given that negative as well as positive Λ-valuation is fed by +Pn, see \REF{ex:Sigurdsson:29}, \textit{some} valuation of the Λ features (usually a negative one) is indirectly involved.} We can obviously say (or think) a first or a second person pronoun from the point of view of the speaker without activating a secondary SELF, and it is also possible to activate a secondary third person SELF in the presence of a first person or a second person pronoun that simply refers to the speaker vs. the hearer. This is illustrated for the first person in \ili{Icelandic} in \REF{ex:Sigurdsson:36}, where the reflexive \textit{sig} reflects the secondary SELF’s (Anna’s) perspective (the reading being \textit{de se} ‘Anna thought: “X sees me”’).\footnote{The parallel holds for the second person (“Anna thought that \textit{you} saw SIG”). It is even possible to construe examples with a first or a second person pronoun in both the matrix and the subordinate clause, nevertheless letting a matrix third person SELF through. In some languages, though, long-distance secondary selfhood relations are blocked by an intervening first or second person. See \citealt{Jayaseelan1998} on Malayalam and \citealt{Giorgi2006} on Chinese.}

\ea%36
    \label{ex:Sigurdsson:36}
\gll	  Anna  hélt  að  ég  \textbf{sæi}  \textbf{sig}.\\
Anna  tought  that  I  saw.\textsc{subj}  SIG \\
\glt ‘Anna thought/believed that I saw her.’
\z

The matrix clause subject \textit{Anna} is NP\textsubscript{+Pn/–ΛA,–ΛP} and the SELF of this NP takes scope over the event and (the past) tense perspective in the subjunctive subordinate clause, despite the presence of \textit{ég} ‘I’. The activation of a secondary SELF thus requires +Pn, whereas positive setting of the Λ features is not necessarily involved.

  Notice that the relevant +Pn valuation takes place in the matrix clause but takes scope over only the subordinate clause (the perspective in the matrix clause being exclusively the speaker’s) – much as the long-distance reflexive is bound in the matrix clause but does not show until in the subordinate clause. In a similar fashion, the past subjunctive of \textit{sæi} ‘saw’ is triggered by the matrix predicate \textit{hélt} ‘thought, believed’. Thus, both the long-distance reflexive and the subjunctive in the subordinate clause are sanctioned or licensed by factors in the matrix clause (see \citealt{Thráinsson1976,Thráinsson2007,Sigurðsson2010mood}). Compare \REF{ex:Sigurdsson:36} to \REF{ex:Sigurdsson:37}, where indicative \textit{sá} and the regular third person pronoun \textit{hana} are required (\textit{sæi sig} being ungrammatical).

\ea%37
    \label{ex:Sigurdsson:37}
  \gll	  Anna  veit  ekki  að  ég  \textbf{sá}  \textbf{hana}.\\
  Anna  knows  not  that  I  saw.\textsc{ind}  her.\\
\glt   ‘Anna does not know that I saw her.’
\z

As in \REF{ex:Sigurdsson:36}, the matrix subject \textit{Anna} in \REF{ex:Sigurdsson:37} is valued as NP\textsubscript{+Pn/–ΛA,–ΛP}, but here the factive semantics of the matrix predicate \textit{veit ekki} ‘knows not’ blocks the SELF of \textit{Anna} from overshadowing the primary SELF of the speaker (hence not only the main clause but also the subordinate one reflects the perspective of the actual speaker).

Now, consider the \textit{de se} reading of \REF{ex:Sigurdsson:15}, and of the simplified \REF{ex:Sigurdsson:38}, and recall that on this reading Mary thought of herself “I look good”.

\ea%38
    \label{ex:Sigurdsson:38}
  	   Mary thought she looked good.
\z

Again, the relevant +Pn valuation takes place in the matrix clause, taking scope over the subordinate clause. There is of course another +Pn valuation in the subordinate clause, yielding the NP\textsubscript{+Pn/–ΛA, –ΛP} subordinate subject \textit{she}, but this second +Pn valuation has no intervening effects (as also seen in \ref{ex:Sigurdsson:36}). It is evident, (i), that at most one secondary SELF is licensed at the time, and (ii), that it can only have discernible effects in a lower clause (under c-command by a matrix subject). It is remarkable that a secondary SELF can neither have any discernible effects in a main clause, nor, more generally, locally in the clause where it has its +Pn source, be it at matrix clause or a subordinate one (cf. \citealt{Thráinsson1976}). It thus seems that a non-speaker perspective must be mediated via a C-edge that is in the scope of (c-commanded by) an argument that is distinct from the primary, speech event speaker SELF. It also seems that the +Pn valuation of a c-commanding matrix subject (\textit{Mary} in \ref{ex:Sigurdsson:38}) is the factor that activates its secondary SELF (provided that the matrix predicate is an attitude predicate).

The hypothesis that +Pn valuation of a matrix subject is the factor that activates its secondary SELF gains support from the fact that \textit{de se} is the only possible reading of PRO in control infinitives like the one in \REF{ex:Sigurdsson:16} = \REF{ex:Sigurdsson:39}.

\ea%39
    \label{ex:Sigurdsson:39}
  	  Mary hoped to look good.
\z

The reason for this, I believe, is that PRO infinitives differ from \isi{finite} clauses in lacking a subject Pn head, thus lacking “independent” subject person. They can have independent non-subject (e.g. object) person and they can have subject person interpretation under control, as illustrated in \REF{ex:Sigurdsson:40}.

\ea%40
    \label{ex:Sigurdsson:40}

  	  I\textsubscript{i} will try [PRO\textsubscript{i} to convince you].
\z

What PRO infinitives cannot have is independently or locally person-valued PRO (see \citealt[424–425]{Sigurðsson2008Natural}). In contrast, a person value can be transmitted to PRO under control, as in \REF{ex:Sigurdsson:40}. Similarly, the third person value of generic PRO, as in \REF{ex:Sigurdsson:41}, is arguably transferred from a silent \textit{one} (plural in some languages) in the matrix clause, as indicated.

\ea%41
    \label{ex:Sigurdsson:41}
  	  It is always interesting [PRO to discover things about oneself].\\
	  = It is always interesting \{for one\textsubscript{i}\} [PRO\textsubscript{i} to discover things about oneself\textsubscript{i}] \footnote{This analysis implies that there is no non-controlled +human PRO, the +human reading boiling down to control by +Pn of an overt or a silent controller.}
\z

Consider a subordinate clause as the one in \REF{ex:Sigurdsson:42}, stated by, say, Anna.

\ea%42
    \label{ex:Sigurdsson:42}
  	  [Anna speaks]:  John knew that Mary was sick.
\z

On the prominent reading of \REF{ex:Sigurdsson:42} the subordinate clause is a regular factive clause (\textit{de re}), stated from the speaker’s (Anna’s) point of view, not reflecting the perspective or SELF of John (cf. also \ref{ex:Sigurdsson:37}). The reason why this is an option, I believe, is that the subordinate clause contains an independent +Pn subject valuation, capable of shielding it from the matrix clause +Pn valuation, hence from the perspective of the secondary SELF of John’s. This perspective shielding is not forced (as seen in \ref{ex:Sigurdsson:36} and \ref{ex:Sigurdsson:38}), but it is commonly possible in the presence of a local +Pn valuation. There is no such subject valuation in PRO infinitives like the one in \REF{ex:Sigurdsson:39}, hence the inescapable \textit{de se} reading.

\isi{Person} shift, as in \REF{ex:Sigurdsson:34} and \REF{ex:Sigurdsson:35}, and indexical shift more generally, usually works such that all indexicals or deictic elements in a given speech context domain must shift together. \citet{AnandNevins2004} even argue that indexical shift is subject to a general Shift-Together Constraint. A wholesale shift-together is exemplified in \REF{ex:Sigurdsson:43} (modelled on \citealt[25]{Banfield1982}).

\ea%43
    \label{ex:Sigurdsson:43}
  	  [Peter speaks at time X and location Y]:  Mary told me yesterday at the station: “I will meet you here tomorrow.”
\z

While the introductory clause (“Mary told me yesterday at the station”) is stated from the speaker’s (Peter’s) perspective, the perspective in the quotation is completely shifted to that of Mary’s. However, despite the commonness of shift together, there are certain discourse modes that allow split selfhood or two centers of consciousness simultaneously (as discussed in \citealt{Banfield1982} and \citealt{Sigurðsson1990}). Consider \textsc{represented speech and thought} (sometimes called “free indirect discourse”), exemplified in \REF{ex:Sigurdsson:44}.

\ea%44
    \label{ex:Sigurdsson:44}
  	  \emph{John} \emph{was} upset. \textbf{That fool of an actor} always \emph{treated} \emph{him} badly and \textbf{now} \textbf{this idiot} \emph{was} even yelling at \textbf{mama}.
\z

This passage contains both split temporal and anaphoric deixis. The adverbial \textit{now} is anchored in “the moment of the act of consciousness” \citep[99]{Banfield1982}, in which the SELF of John is thinking. The verbal past tense, on the other hand, is anchored with the primary SELF of the author (speaker, in our terms) – it lies in the past relative to the moment of utterance or writing of the passage. Similarly, \textit{that fool of a teacher}, \textit{this idiot} and \textit{mama} all represent John’s view, are anchored in his consciousness, whereas John himself is referred to from outside, in the third person (as \textit{John}, \textit{him}), from the point of view of the author. Represented speech and thought is a literary phenomenon, but it nevertheless illustrates that split selfhood (split \textit{origo} in the sense of \citealt{Bühler1934}) is compatible with natural language grammar.

As we have seen, self-talk is another discourse mode that allows split selfhood, the difference being that the split is speaker internal in self-talk. Importantly also, self-talk, as in “I hate you!”, illustrates that person values can be computed separately for each phase. Nevertheless, shift-together is a pervasive phenomenon, in particular in direct speech or quotations. It would thus seem that the C-edge is more prominent than the v-edge and other “small” phase edges, such that the smaller phase edge computations are usually “coordinated” at the C-edge, by what might be called \textsc{C-edge coordination}.

\section{Concluding remarks}\label{sec:Sigurdsson:5}
\largerpage

An exact analysis of C-edge coordination, just mentioned, has yet to be developed, but self-talk throws some light on what its opposite, absent C-edge coordination, involves. There is a \textsc{relation of sameness} between both the DPs in self-talk examples like “I hate you!”, but not a relation of binding in the sense of (any) binding theory.\footnote{A reviewer asks what the difference between “I hate myself” and self-talk “I hate you” might be. Given the present approach, the reflexive-containing clause involves C-edge coordination, as opposed to the self-talk clause.}  A common sameness integer (cf. \citealt[104]{Baker2003}) is sufficient to link both the subject and the object DP to the speaker, while their separate +Pn valuations activate two distinct SELFs, a primary and a secondary speaker SELF. Double linking to the speaker is reminiscent of temporal Double Access Readings, DAR (see, e.g., \citealt{Giorgi2010,Sigurðsson2016}) – but I will not go into that here.

Finally, recall that there is a human bias in \textit{we} such that it usually refers to humans only, see \REF{ex:Sigurdsson:7} = \REF{ex:Sigurdsson:45}.

\ea%45
    \label{ex:Sigurdsson:45}
      We have lived in Europe for at least 40~000 years.
\z

The human bias is shared by plural \textit{you} (and partly also by \textit{they}). Plausibly, the +Pn valuation involved in the computation of “truly personal” pronouns is the factor that triggers this bias as well as secondary SELF interpretations.

Central issues arise. If the notions of \isi{Universal Grammar} and narrow syntax are understood as narrowly as in much recent minimalist work (including my own work), then there is every reason to assume that “natural language syntax” is a much broader system, based on but not confined to UG and narrow syntax, in turn raising the question of what other conceptual systems are involved in broad syntax. The speaker and the hearer categories and even the central \isi{Person} category might stem from some other subsystem than syntax in the narrowest sense (a plausible thought if the machinery of syntax, \isi{Merge} and abstract \isi{Agree} in recent terms, is “autonomous and independent of meaning” as famously stated by \citealt[17]{Chomsky1957}; the speaker/hearer categories and \isi{Person} are not independent of or unrelated to meaning). These issues, as well as the moot issue of C-edge coordination, will hopefully be subject to much future research that will deepen our understanding of the internal–external language correlation. What seems clear is that the Event/Speech Participant Split as well as the “mending” speaker-\{θ\} linking embodied in \textit{we} are central properties of the human mind and of language in at least the broad sense.

\section*{Abbreviations}
Abbreviations used in this article follow the Leipzig Glossing Rules’ instructions for word-by-word transcription, available at: \url{https://www.eva.mpg.de/lingua/pdf/Glossing-Rules.pdf}.

\section*{Acknowledgements}

The research for this paper is part of a project on pronouns and pronoun features, partly funded by a grant from Riksbankens Jubelumsfond, P15-0389:1. The ideas pursued here were presented at Università Ca’Foscari and Università degli Studi Roma Tre in November 2015. I am grateful to Massimiliano Bampi, Mara Frascarelli, Giuliana Giusti, Roland Hinterhölzl, Nicola Munaro, and other friends and colleagues in Venice and Rome for their hospitality and for their questions and comments. For valuable remarks and discussions, many thanks also to two anonymous reviewers and to Verner Egerland, Jim Wood, and, last but not least, Anders Holmberg himself.



\printbibliography[heading=subbibliography,notkeyword=this]
\end{document}
