\documentclass[output=paper]{LSP/langsci}
\author{Artemis Alexiadou\affiliation{Humboldt-Universität zu Berlin/Leibniz-Center General Linguistics (ZAS)}\lastand Janayna Carvalho\affiliation{Universidade de São Paulo}
}
\title{The role of locatives in (partial) pro-drop languages}
% \epigram{Change epigram}
\ChapterDOI{10.5281/zenodo.1116755}

\abstract{%
It is usually assumed that a difference between pro-drop and non-pro-drop languages is the presence of overt expletives in the latter group, but not in the former (cf. \citealt{Rizzi1982,Rizzi1986,AlexiadouAnagnostopoulou1998}). Compared with this two-way classification, partial pro-drop languages, i.e. languages in which the distribution of pro is more restricted, are intriguing case studies. Unlike in English, for example, the satisfaction of EPP can be done in several ways in this group of languages. Fruitful strategies include remerging deictic elements, such as locatives and temporal adjuncts, or raising of internal arguments. As locatives are elements usually employed by all the languages that fall into this category as a means to satisfy the EPP, our comparison will focus on the use of these elements in two partial pro-drop languages, namely Brazilian Portuguese (BP), and Finnish, and Greek, a full pro-drop language. A comparison with a full pro-drop language will show that the behavior of locatives in partial pro-drop languages is one further characteristic that groups them together in opposition to pro-drop ones, apart from the more constrained distribution of pro. We will be concerned with some structures that contain an overt locative in all three languages, either interpreted as \isi{impersonals} (null impersonals) or not. We will first compare BP to Finnish, and show that while locatives lack an argumental status and simply satisfy the EPP in Finnish as pure expletives, this is not the case in BP. In this language, locatives can both be argumental and expletive-like. By contrast, in Greek, locatives never check the EPP, i.e. they are never expletive-like. Rather they are referential/deictic elements, which perform a function similar to what has been discussed for English locative inversion.
}

\maketitle

\begin{document}
% % % %  \href{mailto:artemis.alexiadou@hu-berlin.de}{artemis.alexiadou@hu-berlin.de}
% % % %
% % % %  \href{mailto:janaynacarvalho@usp.br}{janaynacarvalho@usp.br}
% % % %
% % % %  \textbf{Language Index:}
% % % %
% % % %  \ili{Brazilian Portuguese} (BP): all pages
% % % %
% % % %  \ili{Greek}:2-4; 17-19; 26-27
% % % %
% % % %  \ili{Finnish}: 2-8, 10-12, 14.15, 17-22, 24-27
% % % %
% % % %  \ili{Thai}:24
% % % %
% % % %  \textbf{Subject Index:}
% % % %
% % % %  \isi{EPP}: 2-4; 11-12; 15-18; 20-22; 26-27
% % % %
% % % %  Impersonals: 2-8; 10-15; 17; 22; 25-27
% % % %
% % % %  \isi{INFL}: 22; 25-27
% % % %
% % % %  \isi{Locative}(s): all pages
% % % %
% % % %  Null subject:5
% % % %
% % % %  Null subject in embedded clauses: 4-6, 20-22
% % % %
% % % %  \isi{Generic} (null) subject: 4; 6; 10
% % % %
% % % %  Null generic pronoun: 4; 6; 14; 24
% % % %
% % % %  (Full/Partial/Radical) \isi{pro-drop} language: 2-5; 8; 17-20; 24; 26
% % % %
% % % %  \isi{Tense}: 22; 25; 27

\section{Introduction}\label{§1.alexiadou}\is{impersonal|(}

Locatives have received a considerable amount of attention within generative grammar over the decades. Unlike other circumstantial PPs, it has been shown that these elements have grammatical functions in several languages and constructions. For example, \citet{Stowell1981} noticed that PPs in locative inversion behave as subjects with respect to some tests but not others (see \citealt{Rizzi2007} for a reinterpretation of the data). \citet{Freeze1992} claimed that predicative locative sentences (\textit{The book is on the bench}) and existential sentences (\textit{There is a book on the bench}) are the byproduct of a same underlying structure in which a locative is one of the selected arguments of a complete functional complex, a head that selects both an argument and a specifier \citep{Chomsky1985}. Recently, \citet{Kayne2008} argued that expletive \textit{there} in English is a deictic modifier of the associate, merging low in the structure. \citet{Richards2007,Deal2009}, and \citet{AlexiadouSchaefer2011} reached similar conclusions independently.

In this paper, we explore the role of locatives in \ili{Brazilian Portuguese} (BP), \ili{Finnish}, and \ili{Greek}. By studying these three languages, we provide evidence that the role taken by locatives in different languages is tied to the properties of T in the respective languages. In both BP and \ili{Finnish}, locatives can satisfy the \isi{EPP}. However, in BP, locatives behave as arguments in null \isi{impersonals}, a fact that has not been noticed until now. \ili{Greek} is very different from these two languages in not using locatives to satisfy the \isi{EPP}. We relate this to the full \isi{pro-drop} nature of this language. Full \isi{pro-drop} languages satisfy the \isi{EPP} through V-raising (\citealt{AlexiadouAnagnostopoulou1998}) and locatives are associated with the \isi{CP} domain.

The paper is organized as follows. In \sectref{§2.alexiadou}, we discuss the status of 3\textsuperscript{rd} person subjects in partial \isi{pro-drop} languages. As in other partial \isi{pro-drop} languages, in BP and \ili{Finnish}, 3\textsuperscript{rd} definite subject pronouns can be null in embedded clauses, but not in root clauses. In \isi{impersonal} sentences, however, 3\textsuperscript{rd} generic subject can be null (cf. \citealt{Holmberg2005,Holmberg2009Three}, henceforth HNS;
\citealt{Holmberg2010finnish} and \citealt{HolmbergEtAl2015}; for analyses of BP data, see, e.g., \citealt{Cavalcante2007,Galves2001,FigueiredoSilva1996,Kato1999,Duarte1995,Nunes1990}; among many others). In \sectref{§3.alexiadou}, we compare \ili{Finnish} and BP null \isi{impersonals}, showing that a generic null pronoun is present in the former language but not in the latter.

In order to understand the differences between null \isi{impersonals} in the two languages, in \sectref{§4.alexiadou} we deal with the distribution of locatives in these languages. The comparison shows that while locatives are only licensed if T is specified for either generic or definite 3\textsuperscript{rd} person in BP, they behave as pure expletives in \ili{Finnish}, being licensed whenever \isi{EPP} has to be satisfied. In \sectref{§5.alexiadou}, we briefly turn to \ili{Greek} and show that locatives in this language share properties with English locative alternation. \sectref{§6.alexiadou} ties the properties illustrated throughout the paper to properties of T in these three languages. \sectref{§7.alexiadou} concludes the paper.

\section{Third person in partial pro-drop languages}\label{§2.alexiadou}

As in other partial pro-drop-languages, \ili{Finnish} and \ili{Brazilian Portuguese} 3\textsuperscript{rd} definite subject pronouns cannot be null in root clauses, as shown in \REF{ex:1.alexiadou} and \REF{ex:2.alexiadou}, whereas 3\textsuperscript{rd} \isi{impersonal} pronouns can be null, cf. \REF{ex:3.alexiadou} and \REF{ex:4.alexiadou}.\footnote{A few remarks are in order about the examples. Unless otherwise stated, \ili{Greek} examples are due to the first author and BP examples due to the second. 
The verbal endings glossed as '1, 2, 3' are all singular. The plural verbal endings are indicated in the relevant examples.}


\ea\label{ex:1.alexiadou}
\langinfo{Finnish}{}{\citealt[539]{Holmberg2005}}\\
\gll {\upshape *} (Hän) puhuu englantia.\\
 {} (s/he) speak:3 English:\textsc{par}\\
\glt ‘S/he speaks English.’
\z


\ea\label{ex:2.alexiadou}
{Brazilian Portuguese}{}\\
\gll {\upshape *} (Ele) fala inglês.\\
     {} (he) speak:3 English:\textsc{par}\\
\glt ‘He speaks English.’
\z



\ea\label{ex:3.alexiadou}
\langinfo{Finnish}{}{\citealt[548]{Holmberg2005}}\\
\gll Tässä istuu mukavasti.\\
 here sit:3 comfortably\\
\glt ‘One can sit comfortably here.’
\z


\ea\label{ex:4.alexiadou}
\ili{Brazilian Portuguese}\\
\gll Aqui vende camisa.\\
 here sell:3 shirt.\\
\glt ‘T-shirts are sold here.’
\z


However, 3rd definite subject pronouns can be null in embedded clauses, if there is no topic or locative PP intervening between the \isi{null subject} and the root clause, see \REF{ex:5.alexiadou} from \ili{Finnish}. \REF{ex:6.alexiadou} shows that BP follows the same pattern.


\ea\label{ex:5.alexiadou}
\langinfo{Finnish}{}{\citealt[539]{Holmberg2005}}\\
\gll Pekka\textsubscript{i} väittää [että hän\textsubscript{i/j/}Ø\textsubscript{i}/\textsubscript{*j} puhuu englantia hyvin].\\
 Pekka claim:3 that 3SG/Ø speak:3 English well\\
\glt ‘Pekka claims that he speaks English well.’
\z


\ea\label{ex:6.alexiadou}
\ili{Brazilian Portuguese}\\
\gll João afirma que {ele\textsubscript{i/j}\slash Ø\textsubscript{i}\slash\textsubscript{*j}} fala inglês bem.\\
 João claim:3 that he/Ø speak:3 English well\\
\glt ‘John claims that he speaks English well.’
\z

If a locative PP is fronted, the \isi{null subject} in the embedded clause can only be interpreted as an \isi{impersonal} sentence, having a generic subject, both in BP, example \REF{ex:7.alexiadou}, and \ili{Finnish}, example \REF{ex:8.alexiadou}.


\ea\label{ex:7.alexiadou}
\ili{Brazilian Portuguese}\\
\gll João afirma que no Brasil fala inglês muito bem.\\
 John claim:3 that in.the Brazil speak:3 English very well\\
\glt ‘John claims that in Brazil people speak English very well.’
\z


\ea\label{ex:8.alexiadou}
\langinfo{Finnish}{}{\citealt[73]{Holmberg2009Three}}\\
\gll Jari sanoo että tässä istuu mukavasti.\\
 Jari say:3 that here sit:3 comfortably\\
\glt ‘Jari says that one can sit comfortably here.’
\z


Although there is no overt generic pronoun in the embedded clauses in the sentences \REF{ex:7.alexiadou} and \REF{ex:8.alexiadou}, one can entertain the hypothesis that a generic pronoun is present in these sentences. Indeed, as \citet{Holmberg2005,Holmberg2010finnish} argues in detail, a covert generic pronoun must be present in \ili{Finnish}. In the next section, we draw a quick comparison between \ili{Finnish} and BP null \isi{impersonals} in order to investigate whether BP null impersonals also features a generic null pronoun.

\newpage 
\section{Null impersonals in BP and Finnish}\label{§3.alexiadou}

A first piece of evidence for the presence of a generic pronoun in \ili{Finnish} null \isi{impersonals} is that such pronoun can function as an antecedent for an anaphor.\footnote{An anonymous reviewer, a native speaker of \ili{Finnish}, informs us that this sentence is not completely natural. According to the reviewer an overt subject should be used, e.g.: \textit{Nyt jokaisen [each-one-GEN] täytyy pestä autonsa} ‘Now everyone must wash their cars’ or leave the possessive suffix out: \textit{Nyt täytyy pestä auto} ‘Now it is necessary to was the/a car.’The reviewer comments that: “it may be that the reason has something to do with the fact that the subject of \textit{täytyy} is lexically case marked with genitive. The same goes for other modals with a genitive subject \textit{täytyy, pitää, kuuluu}, all meaning ‘must’. The permissive modal verbs ‘may’ (\textit{saa, voi}) have a nominative subject and they work much better in this context.”}


\ea\label{ex:9.alexiadou}
\langinfo{Finnish}{}{\citealt[550]{Holmberg2005}}\\
\gll Nyt täytyy pestä auntonsa.\\
 Now must:3 wash car:\textsc{poss;rfl}\\
\glt ‘One must wash one’s car now.’
\z


Moreover, the object is assigned accusative \isi{Case}, even though there is no other overt DP, see \REF{ex:10.alexiadou}.\footnote{As \citet{Holmberg2005} points out, in some modal constructions, the subject is assigned genitive \isi{Case} and the object nominative \isi{Case}. Only with these verbs the object can have nominative \isi{Case} in null impersonals.}

\ea\label{ex:10.alexiadou}
\langinfo{Finnish}{}{\citealt[549]{Holmberg2005}}\\
\gll Täällä voi ostaa auton /\,{\upshape *} auto.\\
 Here can:3 buy car:\textsc{acc} / car:\textsc{nom}\\
\glt ‘You can buy a car here.’
\z


Subject-oriented adverbials and purpose clauses are licensed, as shown in \REF{ex:11.alexiadou} and \REF{ex:12.alexiadou}.


\ea\label{ex:11.alexiadou}
\langinfo{Finnish}{}{\citealt[548]{Holmberg2005}}\\
\gll Tässä istuu mukavasti.\\
 Here sit:3 comfortably\\
\glt ‘One can sit comfortably here.’
\z


\ea\label{ex:12.alexiadou}
\langinfo{Finnish}{}{\citealt[205]{Holmberg2010finnish}}\\
\gll Tänne tulee mielellään [PRO ostamaan keramiikkaa].\\
 here come:3  with.pleasure PRO buy.\textsc{inf} pottery\\
\glt ‘It is nice to come here to buy pottery.’
\z

However, even though this analysis has been extended to other partial \isi{pro-drop} languages, it does not seem to work for the canonical BP null \isi{impersonal} data examined in the literature, i.e. null impersonals with generic time \isi{reference}.\footnote{For some comments on other types, see footnote \ref{fn:alexiadou:11} and \sectref{§6.2.alexiadou}.} First, as shown in \REF{ex:13.alexiadou}, anaphors are not licensed in BP null impersonals.\footnote{As Charlotte Galves (p.c) points out, the test in \REF{ex:9.alexiadou} is not replicable in BP, since \textit{seu}, the former possessive generic/3\textsuperscript{rd} pronoun, is nowadays an almost exclusive 2\textsuperscript{nd} definite possessive pronoun, due to changes in the pronominal paradigm. Hence, a version of \REF{ex:9.alexiadou} into BP leads to the interpretation that a generic entity will wash a car possessed by a definite person. (9’) \textit{Agora pode lavar seu carro.} Now can:3 wash:\textsc{inf} your\textsubscript{def} car.}

\ea\label{ex:13.alexiadou}
\ili{Brazilian Portuguese}\\
\gll {\upshape *} Aqui ensina {a si} mesmo.\\
     {} here teach:3 to se:\textsc{obl} self.\\
\glt ‘Here one teaches oneself.’
\z


Also, null \isi{impersonals} in BP do not license inalienable possessors, which require a human antecedent in \ili{Romance}. In \REF{ex:14.alexiadou}, we observe that an inalienable body part ‘\textit{a mão}’ is interpreted as possessed if c-commanded by a human antecedent. Both a definite DP (\textit{João}) and the impersonal morphology (\textit{se}) warrant this interpretation if they c-command an inalienable body part.


\ea\label{ex:14.alexiadou}
\ili{Brazilian Portuguese}\\
\gll João/se levantou a mão na sala para fazer pergunta.\\
 John/one raised:3 the hand in.the classroom to ask:\textsc{inf} question\\
\glt ‘John/one raised his hand to ask questions in the class.’
\z


In \REF{ex:15.alexiadou}, however, this reading does not obtain as no human DP c-commands the inalienable body part.

\ea\label{ex:15.alexiadou}
\ili{Brazilian Portuguese}\\
\gll {\upshape ?*} Na {sala de aula} levanta a mão para fazer pergunta.\upshape\footnotemark{}\\
 {} in.the classroom raise:3 the hand to make:\textsc{inf} question\\
\glt ‘In classrooms, one raises his hand to ask questions.’
\z
\footnotetext{\label{fn:alexiadou:6}Three of four speakers judged this sentence as ungrammatical. One speaker judged it as grammatical under a contrastive reading, something along the lines of: ‘In the classroom, one raises his hand to ask questions, not to argue with the teacher.’ Crucially, under a neutral reading, this sentence is not grammatical for any of our consultants.}

\newpage
Furthermore, subject-oriented adverbials such as \textit{com maestria}/\textit{com atenção} are not licensed, as we see in \REF{ex:16.alexiadou}, and nor are purpose clauses, as \REF{ex:17.alexiadou} shows.\footnote{Charlotte Galves (p.c.) offers as a counterexample the sentence in (i): 
\ea\label{ex:i.alexiadou}
\gll No Brasil só trabalha pra ganhar dinheiro.\\
 In.the Brazil only work:3 to earn money\\
\glt ‘In Brazil one only works to earn money.’
\z
This sentence is indeed grammatical to the second author of this paper and other speakers consulted. However, without the contrastive/emphatic adverb \textit{só}, the judgments are not so sharp. As the discussion in footnote \ref{fn:alexiadou:6} suggests, contrastive contexts improve the grammaticality of the relevant sentences.}


\ea \label{ex:16.alexiadou}
\ili{Brazilian Portuguese}\\
\gll {\upshape *} Naquela escola de culinária prepara doce com maestria/ com atenção.\\
     {} in.that school of culinary prepare:3 sweet with mastery/ with attention\\
\glt ‘One prepares sweets with mastery/with attention in that culinary school.’
\z


\ea\label{ex:17.alexiadou}
\ili{Brazilian Portuguese}\\
\gll {\upshape *} Naquela escola de culinária prepara doce para alimentar criança. \\
    {} in.that school of culinary prepare:3 sweet to feed:\textsc{inf} child.\\
\glt ‘One prepares sweets to feed the children in that culinary school.’
\z

Given these contrasts, it seems that we cannot maintain Holmberg’s analysis for BP, while arguably this captures very nicely the \ili{Finnish} data. The question that arises then is: what ensures the impersonal reading of these sentences in BP?

Before we offer an answer to this question, note that null impersonal sentences in BP are subject to a number of constraints, which further support our conclusion that they differ from their \ili{Finnish} counterparts. As shown in \REF{ex:18.alexiadou}, unaccusative verbs are out in BP null impersonals. In addition, BP null impersonals do not tolerate other circumstantial PPs: a generic reading for the subject is possible only in the presence of a locative element.\footnote{The only apparent counterexample to this generalization is \textit{hoje em dia} ‘nowadays’, as in the sentence \textit{Hoje em dia usa saia} (lit. Nowaday wear:3 skirt), discussed in \citet{Galves2001}. As this is the only temporal element licensed in BP null impersonals, it cannot be said that temporal as locative PPs satisfy the \isi{EPP} in BP null impersonals.}


\ea\label{ex:18.alexiadou}
\ili{Brazilian Portuguese}\\
\gll {\upshape *} Naquele hospital nasce com saúde. \\
     {} in.that hospital born:3 with healthy\\
\glt Intended: ‘One who is born in that hospital is healthy.’
\z

By contrast, these constraints are not found in \ili{Finnish}. Unaccusative verbs appear in null impersonals and a generic \isi{null subject} is generally available, no matter what element satisfies the \isi{EPP}. For example, in \REF{ex:19.alexiadou}, the expletive \textit{sitä} satisfies the \isi{EPP}.\footnote{As BP does not have lexical expletives, \REF{ex:19.alexiadou} has the sole purpose of illustrating that this reading is not dependent on locatives in \ili{Finnish}, but it is in BP.}


\ea\label{ex:19.alexiadou}
\langinfo{Finnish}{}{\citealt{Roberts2015}}\\
\gll Sitä huolestuu helposti.\\
 \textsc{expl} get.worried easily\\
\glt ‘One gets worried easily.’
\z


\REF{ex:20.alexiadou} exemplifies a further constraint in BP null impersonals. Individual-level verbs do not form null impersonals in BP, but they do in \ili{Finnish}, as \REF{ex:21.alexiadou} indicates.\footnote{\label{fn:alexiadou:10}One reviewer argues that the psych verb \textit{temer} in \REF{ex:20.alexiadou} may fall under the same generalization proposed for examples \REF{ex:18.alexiadou} and \REF{ex:19.alexiadou}, since psych verbs are usually analyzed as unaccusatives. Note, however, that \textit{temer} (fear) is usually taken to represent the class of transitive psych verbs in which the experiencer is a ‘deep subject’, hence it is analyzed as a transitive sentence \citep{BellettiRizzi1988}.}


\ea\label{ex:20.alexiadou}
\ili{Brazilian Portuguese}\\
\gll {\upshape *} Naquela casa teme a morte.\\
 {} In.that house fear:3 the death\\
\glt Intended: ‘One fears the death in that house.’
\z


\ea\label{ex:21.alexiadou}
\langinfo{Finnish}{}{\citealt{Roberts2015}}\\
\gll Sitä ei tiedä milloin kuolee.\\
 \textsc{expl} not know:3 when die:3\\
\glt ‘One doesn’t know when one dies.’
\z

\largerpage[-2]
\tabref{tab:1.alexiadou} summarises the differences between BP and \ili{Finnish} null impersonals discussed above.

\begin{table}
\begin{tabular}{lll}
\lsptoprule
{Test} & {Finnish} & {BP}\\
\midrule
{Anaphors} & yes & no\\
\tablevspace
{Subject-oriented adverbials} & yes & no\\
\tablevspace
{Purpose clauses} & yes & no\\
\tablevspace
{Unaccusative verbs} & yes & no\\
\tablevspace
{Individual-level verbs} & yes & no\\
\lspbottomrule
\end{tabular}

\caption{Differences between Finnish and BP null impersonals}
\label{tab:1.alexiadou}
\end{table}


\newpage 
To summarize, we have presented evidence that i) BP null impersonals do not pass any of the tests for the presence of an implicit agent in their structure; ii) only a subset of transitive stage-level verbs is allowed in BP null impersonals. More precisely, the verb at hand must include an agentive external argument in transitive sentences.

While we recognize that the licensing of a subset of transitive stage-level verbs is not a conclusive piece of evidence in favour of the claim that \ili{Finnish} and BP are drastically different, the fact that BP null impersonals do not pass any of the tests for the presence of an implicit argument is quite suggestive of a difference between null impersonals in these two languages.\footnote{\label{fn:alexiadou:11}A reviewer reminded us of the two classes of impersonals in \ili{Italian} discussed in \citet{Cinque1988}. In tensed contexts, several types of verbal classes are licensed (transitives, unergatives, unaccusatives, copulas, and the like). In untensed contexts, however, transitive and unergative verbs are the only ones licensed in some constructions. The reviewer then suggests that BP null impersonals can be a silent counterpart of untensed \ili{Italian} \textit{se}-impersonals. If this were the case, we should be able to detect the presence of this silent pronoun. The tests from \REF{ex:13.alexiadou} to \REF{ex:17.alexiadou}, however, show that BP null impersonals lack an element responsible to license agentive-like elements.}

\largerpage[-2]%longdistance
Recall our question above: what ensures the impersonal reading of the BP examples? We propose that it is the locative element that is responsible for this. Crucially, the locative element in the above sentences  cannot be analyzed as a topic (contra \citealt{Barbosa2011,Barbosatoappear}) or a pure expletive satisfying the \isi{EPP} (contra \citealt{Buthers2009,AvelarEtAl2008}) as the tests from \REF{ex:13.alexiadou} to \REF{ex:17.alexiadou} show that a pronoun is not responsible for the human reading in BP null impersonals. Specifically, we propose that, at least for BP, the locative is the element responsible for deriving the existential interpretation. This proposal is reminiscent of \citegen{Freeze1992} idea that, in several languages, a locative is a subject that generates existential meanings in existential sentences. Likewise, \citet{Brody2013} notes the crucial role of locatives in generating generic readings with personal pronouns. According to this author, locatives have a silent \textit{semantic} person that do not enter into syntactic operations, but contribute to the \isi{semantic} interpretation of some sentences. In order to demonstrate this, consider the contrast between (\ref{ex:22.alexiadou}a) and (\ref{ex:22.alexiadou}b). Whereas (\ref{ex:22.alexiadou}a) can have an impersonal reading, meaning that people in general like to take a nap in the afternoon when in Italy, (\ref{ex:22.alexiadou}b) cannot. In other words, as the locative is absent, (\ref{ex:22.alexiadou}b) can only mean that a definite group of people like to take a nap in the afternoon.


\ea\label{ex:22.alexiadou}
{English}{}{\citealt[34--35]{Brody2013}}\\
\ea[]{
\glt In Italy they like to take a nap in the afternoon.}
\ex[]{
\glt They like to take a nap in the afternoon.}
\z
\z


As we have been arguing that a pronoun is absent in BP null impersonals and it is usually assumed that locatives can give rise to a generic reading, we claim that the locative element is the external argument in these sentences. Under this analysis, we can explain some of the characteristics of BP null impersonals witnessed above, namely: the verbal restriction and the behavior in respect to agentive tests.

Recall that neither individual-level nor unaccusative verbs form null impersonals in BP. Individual-level verbs are argued to lack the event argument, a spatiotemporal argument above vP responsible for, among other things, the licensing of locatives in stage-level but not individual-level verbs \citep{Kratzer1995}. In addition, the impossibility of forming BP null impersonals with unaccusative stage-level verbs is quite revealing. Note that nothing would forbid the licensing of unaccusative stage-level verbs in BP null impersonals if the locative in this construction were a mere adjunct. As transitive stage-level verbs, unaccusative stage-level verbs like \textit{nascer} `born', in \REF{ex:18.alexiadou}, are endowed with an event argument. However, as noted, the reason why this class of verbs is not licensed in BP null impersonals is that this locative can only be in complementary distribution with an argument that is merged on the same region the locative is: above vP.


Finally, concerning the behavior of BP null impersonals in respect to agentive tests, they corroborate an analysis of locatives as having a silent \isi{semantic}, but not syntactic, person. The opposite behavior of \ili{Finnish} in respect to verbal classes licensed and the agentive tests makes it clear that in this language a null pronoun must be present, as argued extensively in Holmberg’s work.\footnote{Anders Holmberg (p.c) observes that the theta-criterion has to be abandoned if this analysis for BP null impersonals is right. Although we will not fully develop this idea here, we believe that a constructionist view for argument structure is the adequate one to explain these facts. Under the view that the argument structure is syntax and, therefore, depends on the specific formatives a language has, theta-criterion is nothing but an epiphenomenon. Finally, adopting the idea that several elements besides verbs have external arguments, including prepositions \citep{Svenonius2010}, \citet{WoodEtAl2017} argue for the existence of a single argument introducer i*, which will be interpreted differently depending on the projection it \isi{merges} with. This proposal can successfully derive the agentive interpretation in BP null impersonals if we assume that i* can s-select for a PP when merging with a vP in this language. Hence, null impersonals in BP would have a quirky subject. For more details, see \citet{Carvalho2016}.}

If the analysis for BP null impersonals in on the right track, we may be able to detect a specific characteristic of BP syntax that allows an external argument to be a locative in these contexts. We turn to this question in the next section.

\section{Locatives as arguments and expletives}\label{§4.alexiadou}

Given the contrasts seen in the above section, we can say that locatives have an expletive function when their only purpose is to satisfy the \isi{EPP} in restricted environments, and are arguments when they yield generic meaning in null impersonals in BP. In \ili{Finnish}, on the other hand, locatives only satisfy the \isi{EPP}, as pure expletives \citep{HolmbergEtAl2002}. In what follows, we provide evidence for this view by showing that in several 3\textsuperscript{rd} person contexts locatives satisfy the \isi{EPP} in BP. By contrast, in \ili{Finnish}, they can remerge to Spec of \isi{TP} whenever necessary, i.e. there is no constraint regarding the specification of T in this language for the satisfaction of the \isi{EPP} by locatives.

% \newpage 
% \subsection{Locatives in BP grammar}\label{§4.1.alexiadou}

The order VS in BP is degraded (cf. \citealt{Berlinck1988} for its loss throughout the centuries). This is a possible order, however, if either locative or temporal elements are fronted. If the temporal or locative element is overt, even unergative verbs can be licensed in VS order (cf. \citealt{AvelarEtAl2008,Avelar2009,AvelarGalves2011}).


\ea\label{ex:23.alexiadou}
\ili{Brazilian Portuguese}\\
\gll Na semana passada entrou um cara na minha casa.\\
 In.the week last enter:\textsc{pst}.3 a man in.the my house\\
\glt ‘Last week a man (= a thief) entered my house.’
\z


If the locative or temporal element is covert, the interpretation is more constrained. In \REF{ex:24.alexiadou}, the only possible interpretation is that the event happened recently, most likely on the same day (see \citealt{Pilati2006,PilatiNaves2013}).


\ea\label{ex:24.alexiadou}
\ili{Brazilian Portuguese}\\
\gll Morre Maria da Silva.\\
 Die.\textsc{prs}:3 Maria da Silva.\\
\glt ‘Maria da Silva died today.’
\z

Consequently, sentence \REF{ex:25.alexiadou}, in which an event that took place some years ago is described, is odd.


\ea\label{ex:25.alexiadou}
\ili{Brazilian Portuguese}\\
\gll {} Você lembra o que aconteceu há 10 anos?\\
 {} You remember:2 the what happened there.is 10 years\\

\gll {\upshape *}  Morreu a Maria da Silva.\\
     {} Died:3 the Maria of.the Silva\\

\glt ‘Do you remember what happened 10 years ago? Maria da Silva died.’
\z


With unaccusative verbs, locatives can be non-canonical subjects (\citealt{Pontes1987,Galves2001,Lunguinho2006,Rodrigues2010}, among many others), as in the \isi{possessor} raising data below shows.\footnote{\citet{Nunes2015} shows that the the object is assigned inherent \isi{Case} in \isi{possessor} raising constructions.}


\ea\label{ex:26.alexiadou}
\ili{Brazilian Portuguese}\\
\gll Cabe muita camisa nessas gavetas.\\
 Fit:3 a.lot T-shirt in.these drawers\\
\glt
\z

\ea\label{ex:27.alexiadou}
\ili{Brazilian Portuguese}\\
\gll [Essas gavetas] cabem muita camisa.\\
 These drawers fit:3\textsc{pl} a.lot T-shirt\\
\glt ‘It fits a lot of things in these drawers.’
\z


A characteristic that unifies all these phenomena is the fact that these locative strategies are fruitful only with 3\textsuperscript{rd} person. Consider, for example, a version of \REF{ex:23.alexiadou} with a 1\textsuperscript{st} person subject. In a neutral context, locatives satisfying the \isi{EPP} in BP are ungrammatical if T bears 1\textsuperscript{st} or 2\textsuperscript{nd} person features.

\ea\label{ex:28.alexiadou}
\ili{Brazilian Portuguese}\\
\gll {\upshape *} Na semana passada entrei eu na minha casa nova.\\
     {} In.the week last enter:\textsc{pst}.1 I in.the my house new\\
\glt ‘I entered my new house last week.’
\z

Even though there is a restriction regarding the grammatical person, locative elements in BP can be said to satisfy \isi{EPP} in VS constructions, for example. Observe, however, that this does not seem to be the case in either null impersonals or in \isi{possessor} raising constructions. For null impersonals, we have demonstrated that the locative PP is in complementary distribution with an agentive external argument (cf. the ungrammaticality of \ref{ex:18.alexiadou} and \ref{ex:20.alexiadou}). In \isi{possessor} raising cases, exemplified in \REF{ex:27.alexiadou}, the assignment of nominative \isi{Case} to the locative is poorly understood, but cannot be solely attributed to a means of satisfying the \isi{EPP}. A more canonical option would be moving the entire DP rather than a part of it.

In \ili{Finnish}, locatives seem to play a different role. They function, as \citet{Holmberg2005} points out, as pure expletives. Hence, they do not occupy Spec,\isi{TP} only in 3\textsuperscript{rd} person contexts, but whenever the \isi{EPP} needs to be satisfied. \REF{ex:29.alexiadou} shows that a locative is satisfying the \isi{EPP} in a context where T is specified for 1\textsuperscript{st} person. We come back to this issue in \sectref{§6.2.alexiadou}.


\ea\label{ex:29.alexiadou}
\langinfo{Finnish}{}{\citealt[547]{Holmberg2005}}\\
\gll Pariisissa minä olen käynyt (mutten Roomassa).\\
 Paris:\textsc{ine} I be:1 visited but.not Rome:\textsc{ine}\\
\glt ‘I've been to PARIS (but not Rome).’
\z

Therefore, our original question of why locatives play a central role in BP null impersonals, but not in \ili{Finnish}, seems to be related to the crucial role of locatives in different types of 3\textsuperscript{rd} person constructions in the first grammar, but not in the second. This question will be discussed in \sectref{§6.alexiadou}.

\section{Greek locatives}\label{§5.alexiadou}

Contrasting with \ili{Finnish} and BP, in \isi{pro-drop} languages locatives only have a discourse function, i.e. they do not satisfy the \isi{EPP} of this type of language. In \ili{Greek}, VS orders are generally acceptable with all sorts of subjects, definite, indefinite, all persons, as well as bare plurals. It has, however, been noted in the literature, that VS orders are degraded with unergative predicates. However, as in other \isi{pro-drop} languages, in \ili{Greek}, VS orders with certain unergative predicates become acceptable when a locative adverbial is added to the sentence (\citealt{Torrego1989,Rigau1997,Borer2005,Alexiadou2010}):


\ea\label{ex:30.alexiadou}
\ili{Greek}\\
\gll \textbf{edo} pezun pedja.\\
 here play:3\textsc{pl} child:\textsc{pl}\\
\glt ‘Children play here.’
\z

\citet{Alexiadou2010} shows that this type of inversion is mainly possible with certain unergative predicates and a sub-class of unaccusatives. This is very different from \ili{Finnish}, where locatives remerge to spec of \isi{TP} regardless of the type of verb, showing, again, the different role of locatives in these two grammars.

\citet{Alexiadou2010} argues in detail that the locative does not occupy the Spec,\isi{TP} position, and that the single DP argument is the external argument of the predicate. For instance, in \REF{ex:31.alexiadou}, taken from \citet{Alexiadou2010}, we see that the predicate retains its agentive characteristics: it is compatible with agentive/instrumental adverbials just like any other unergative predicate.


\ea\label{ex:31.alexiadou}
\ili{Greek}\\
\gll edo epezan pedia prosektika / me ti hrisi bala / epitides\\
 here played:3\textsc{pl} child:\textsc{pl} carefully / with the golden ball / {on purpose}\\
\glt ‘Children play here carefully/with the golden ball/on purpose.’
\z


Instead, \citet{Alexiadou2010} adopts an analysis, according to which the locative is a \textit{stage topic} in \citegen{Cohen2002} terms. It is situated in the \isi{CP} domain, the area in the clause structure that is responsible for discourse features (see \citealt{Rizzi1997}). The presence of a locative in the \isi{CP} area leads to a focus interpretation of the elements following it. Thus full \isi{pro-drop} languages lack expletive locatives. We will maintain that for these languages V-raising always satisfies the \isi{EPP}, and no XP is required to appear in \isi{TP} for \isi{EPP} reasons, as has been argued for in great detail by \citet{AlexiadouAnagnostopoulou1998}.

Below, we offer a syntactic structure for a sentence like \REF{ex:30.alexiadou} in \ili{Greek} \citep[72, (19')]{Alexiadou2010}. This structure will be compared with BP and \ili{Finnish} later on.

\ea\begin{forest} for tree={s sep=20mm}
[\isi{CP}
  [AdvP\\edo,align=left, base=top] [C'
  [C] [\isi{TP}
      [epezan pedia,roof]
  ]
] ]
\end{forest}\z


\section{Towards an analysis}\label{§6.alexiadou}
\subsection{The D feature}\label{§6.1.alexiadou}

In \citegen{Holmberg2005} and \citegen{Holmberg2009Three} analysis, a crucial difference between \isi{pro-drop} and partial-\isi{pro-drop} languages is the feature D in T.\footnote{The feature D is T is inherently specified in \citet{Holmberg2005}, but uninterpretable in \citet{Holmberg2009Three}. In the latter account, D in \isi{pro-drop} languages is valued by an A-topic in the C domain and, in its turn, value the external argument.} D stands for definiteness and its presence in the former group of languages, but not in the latter, accounts for the possibility of having null definite subjects only in \isi{pro-drop} languages.

In the two aforementioned analyses, both definite and generic 3\textsuperscript{rd} person are treated as instances of the same category. Both start out the derivation as phi-pronouns, pronouns smaller than DPs, having only phi-features as their constituents, following \citeauthor{DechaineEtAl2002}’s (\citeyear{DechaineEtAl2002}) typology. After entering into the derivation, the ϕP pronoun \isi{merges} as an external argument at some point. The phi-features in T then agree with the bunch of phi-features merged as external argument. Observe, however, that T, besides also having a bunch of phi-features, corresponding to the verbal morphology, has the feature D in contexts in which the interpretation of the subject is definite (3\textsuperscript{rd} referential person, for example) and information about the time of the utterance, as represented in \REF{ex:33.alexiadou}. The features in T are then a superset of the features merged as an external argument. Therefore, by means of chain reduction, the features in T will end up being the ones pronounced, i.e. the lower chain will be deleted \REF{ex:35.alexiadou}. See the steps of the derivation below, from \citealt[70]{Holmberg2009Three}.


\ea\label{ex:33.alexiadou}
case of external argument to be valued\\\relax
[T, D\textsubscript{k,} uϕ, NOM] [vP [3\textsc{sg}, uCase] v...]
\z

\ea\label{ex:34.alexiadou}
case of external argument is valued\\\relax
[T, D\textsubscript{k,} 3\textsc{sg}, NOM] [vP [3\textsc{sg}, NOM] v...]
\z

\ea\label{ex:35.alexiadou}
chain reduction\\\relax
[T, D\textsubscript{k,} 3\textsc{sg}, NOM] [vP [\st{3\textsc{sg}, NOM}] v...]
\z


In partial \isi{pro-drop} languages, by contrast, the D feature is not present since definite subjects are not null. Nonetheless, recall that 3\textsuperscript{rd} definite person can be null in both languages if they are the subject of an embedded clause. See examples \REF{ex:5.alexiadou} and \REF{ex:6.alexiadou} from both languages repeated below as \REF{ex:36.alexiadou} and \REF{ex:37.alexiadou}.

\ea\label{ex:36.alexiadou}
\langinfo{Finnish}{}{\citealt[539]{Holmberg2005}}\\
\gll Pekka\textsubscript{i} väittää [että hän\textsubscript{i/j/} Ø\textsubscript{i}/\textsubscript{*j} puhuu englantia hyvin]\\
 DP claim:3 that he/ Ø speak:3 English\\
\glt
\z

\ea\label{ex:37.alexiadou}
\ili{Brazilian Portuguese}\\
\gll João afirma que ele\textsubscript{i/j} / Ø\textsubscript{i}/\textsubscript{*j} fala inglês bem.\\
 DP claim:3 that he / Ø speak:3 English well\\
\glt ‘John claims that he speaks English well.’
\z


HNS point out that an alternative derivation must be responsible for the licensing of 3\textsuperscript{rd} person embedded subject in this specific context. Following \citegen{Holmberg2005} analysis, the idea is that the 3\textsuperscript{rd} person definite subject checks \isi{EPP}, because this reading is only available if there is no intervening element between the subject of the embedded clause and the next clause up, as \REF{ex:38.alexiadou} from \ili{Finnish} and \REF{ex:39.alexiadou} from BP exemplify.


\ea\label{ex:38.alexiadou}
\langinfo{Finnish}{}{\citealt[73]{Holmberg2009Three}}\\
\gll Jari sanoo että (hän) istuu mukavasti tässä.\\
 Jari say:3 that (he) sit:3 comfortably here\\
\glt ‘Jari says that he sits comfortably here.’
\z


\ea\label{ex:39.alexiadou}
\langinfo{Brazilian Portuguese}{}{\citealt[142]{Rodrigues2004}}\\
\gll João\textsubscript{1} me contou que (ele\textsubscript{1}) vende cachorro quente na praia.\\
 João\textsubscript{1} me tell:\textsc{pst}.3 that (he\textsubscript{1}) sell:3 hot dog in.the beach\\
\glt ‘João told me that he sells hot dogs at the beach.’
\z


If an adverb checks the \isi{EPP}, for example, the generic reading arises \REF{ex:40.alexiadou} for \ili{Finnish} and \REF{ex:41.alexiadou} for BP.


\ea\label{ex:40.alexiadou}
\langinfo{Finnish}{}{\citealt[73]{Holmberg2009Three}}\\
\gll Jari sanoo että tässä istuu mukavasti.\\
 Jari say:3 that here sit:3 comfortably\\
\glt ‘Jari says that one can sit comfortably here.’
\z


\ea\label{ex:41.alexiadou}
\langinfo{Brazilian Portuguese}{}{\citealt[142]{Rodrigues2004}}\\
\gll João me contou que na praia vende cachorro quente.\\
 João me tell:\textsc{pst}.3 that in.the beach sell:3 dog hot\\
\glt ‘João told me that hot dogs are sold at the beach.’
\z


The generalization then is that subjects can have a definite interpretation only if the subject of the embedded clause is c-commanded by the subject of the matrix clause, whereas the generic reading arises if another constituent, either a PP in both \ili{Finnish} and BP or the object in \ili{Finnish}, are situated in Spec,\isi{TP}. The generic reading is thus obtained if the bunch of phi-features remain inside the vP.

In BP, however, we have seen that locatives seem to be responsible for the generation of an impersonal sentence rather than a covert pronoun. Hence, although\textit{ tässä} (here), in \REF{ex:40.alexiadou}, and \textit{na praia} (at the beach), in \REF{ex:41.alexiadou}, satisfy the \isi{EPP} and preclude the subject of the root clause to control the subject of the embedded one, these two locative elements differ in the sense that \textit{tässä} is non-argumental and \textit{na praia} is argumental. Positing this difference between BP and \ili{Finnish} null impersonals leads us to consider how the valuation of features between T and the locative in the external argument position will take place in BP. If a locative \isi{merges} as external argument in BP null impersonals, the derivation should crash since PP locatives do not have syntactic person features, as the BP data have shown. Alternatively, it could be the case that there are other features on T in BP null impersonals and the use of locatives as arguments reflect this. We explore this possibility in \sectref{§6.2.alexiadou}.

\subsection{Another type of INFL in BP}\label{§6.2.alexiadou}

Following \citet{RitterWiltschko2014}, we assume that in BP locatives anchor the event. In BP, referential T can have a defective set of phi-features (cf. \citealt{Ferreira2000,Nunes2008,Cyrino2011}, among others). Thus, it can be the case that T is devoided of phi-features in BP null impersonals. Null impersonals in this language, we claim, are cases in which \isi{INFL} is specified for location, hence the mandatory presence of a locative, rather than tense. The examples below show the differences on the interpretation when the locatives are present or not. Crucially, whenever T is episodic, locatives are dispensable. In contrast, under a generic tense, they are obligatory in BP null impersonals. In other words, we propose that \isi{INFL} has a location specification in BP when T would have default specification (3\textsuperscript{rd} person, generic tense).

\citet{RitterWiltschko2014} claim that two different \isi{INFL} values cannot coexist as distinctive. As BP null impersonals exemplified above are awkward or entirely out if T is [+past], it seems that location and specified time cannot coexist in BP \isi{INFL}.


\ea\label{ex:42.alexiadou}
\ili{Brazilian Portuguese}\\
\gll {\upshape *} Aqui vendeu camisa.\\
 {} here sell:\textsc{pst}.3 T-shirt\\
\glt ‘One sold T-shirts here.’
\z


\ea\label{ex:43.alexiadou}
\ili{Brazilian Portuguese}\\
\gll {\upshape ?*} Na escola de culinária preparou doce.\\
 {} in.the school of culinary prepare:\textsc{pst}.3 sweet\\
\glt ‘At the culinary school someone prepared sweets.’
\z


Interestingly, as pointed out by Rozana Naves (personal communication) and Charlotte Galves (personal communication), these sentences improve if expressions such as \textit{por muito tempo} (for a long period of time) or \textit{já} (once) are added. \REF{ex:42.alexiadou} becomes grammatical with the addition of these elements.


\ea\label{ex:44.alexiadou}
\ili{Brazilian Portuguese}\\
\gll Aqui já / por muito tempo vendeu camisa.\\
 here once / for much time sell:\textsc{pst}.3 T-shirt\\
\glt ‘One sold T-shirts here for a long period of time/once.’
\z


Observe, however, that an episodic reading for these sentences is not available. They are generic events that stretched for a period of time in the past.

In cases in which a true episodic reading is available, null impersonals are possible, but locatives are not fronted, i.e. they do not have the same role in sentences in which T is not specified, as examples \REF{ex:45.alexiadou} and \REF{ex:47.alexiadou}, from \citet{LunguinhoMedeirosJunior2013}, indicate. If locatives are fronted, as in \REF{ex:46.alexiadou} and \REF{ex:48.alexiadou}, they are at least awkward. 

\ea\label{ex:45.alexiadou}
\langinfo{Brazilian Portuguese}{}{\citealt[16]{LunguinhoMedeirosJunior2013}}\\
\gll Matou um rapaz \textbf{no} \textbf{show} \textbf{do} \textbf{Zezé di Camargo} e Luciano ontem.\\
 Killed:\textsc{pst}.3 a guy in.the show of.the {Zezé di Camargo} e Luciano yesterday\\
\glt ‘A guy was killed at Zezé di Camargo e Luciano’s show yesterday.’
\z


\ea\label{ex:46.alexiadou}
\ili{Brazilian Portuguese}\\
{\upshape ?*} \textbf{No show do Zezé di Camargo} matou um rapaz.\\
\z


\ea\label{ex:47.alexiadou}
\langinfo{Brazilian Portuguese}{}{\citealt[16]{LunguinhoMedeirosJunior2013}}\\
\gll Telefonou \textbf{aí} da CEB pra você.\\
 Telephone:\textsc{pst}.3 there of.the CEB to you\\
\glt ‘Someone from CEB called you.’
\z 

\ea\label{ex:48.alexiadou}
\ili{Brazilian Portuguese}\\
{\upshape *} \textbf{Aí} telefonou da CEB pra você.\\
\z


 
Furthermore, some contrasts found by \citet{HolmbergEtAl2015} between radical \isi{pro-drop} languages and \ili{Finnish} null impersonals are replicable in BP. The authors noticed that the alleged null pronoun in languages like Mandarin and \ili{Thai} can refer to either human or non-human beings if the predicate allows it. Consider example \REF{ex:49.alexiadou} that demonstrates this possibility in \ili{Thai}.


\ea\label{ex:49.alexiadou}
\langinfo{Thai}{}{\citealt[61]{HolmbergEtAl2015}}\\
\gll Rúguo néng huò dé {gèng duo} de {yi´ng y\v{a}ng}, {nà me} huì   zh\v{a}ng de gèng kuài.\\
      if   can get  of more      of  nutrition,         (that) (will) grow      of more fast\\
\glt ‘If one gets a lot of nutrition, one will grow fast.’
\z 


The same interpretation is available for the translation of \REF{ex:49.alexiadou} into BP: \textit{Se pode ter mais nutrição, vai crescer mais rápido}. The null element in both clauses can refer to either plants or humans. \citet{HolmbergEtAl2015} argue that, in the languages in which both interpretations are available, the null pronoun has a referential index -- rather than a human feature -- that is bound by a generic feature located in C. In languages in which T has phi-features, the null pronoun has a human feature, besides a referential index. This warrants that only a human interpretation will be available and that T must enter into an agree relation with the null pronoun, otherwise the derivation clashes.

Abstracting away from the details of \citegen{HolmbergEtAl2015} analysis, the possibility of having a non-human reading in BP for sentence \REF{ex:49.alexiadou} is intriguing, especially taking into consideration that null impersonals in BP have an \isi{INFL} specified for location rather than tense, as we have been arguing. Observe, however, that this reading arises when a subordinate clause is present. Subordinate clauses have operators whose primary function is the temporal binding of the sentence \citep{Gueron1982}. Therefore, we can couple \REF{ex:49.alexiadou} with \REF{ex:45.alexiadou} and \REF{ex:47.alexiadou}. In these three cases, temporality is involved and a locative, if present, is not \isi{INFL} related.

In addition, note that an unaccusative verb, \textit{grow} in \REF{ex:49.alexiadou}, can be used when temporality is involved, showing, once more, that null impersonals with fronted PP locatives and the cases in which there is a temporal interval and this reading is obtained, are different derivations. Remember that unaccusative verbs cannot form null impersonals in BP when locatives are fronted (cf. \tabref{tab:1.alexiadou}). Given the differences, we believe that the reading of a generic entity in \REF{ex:46.alexiadou}, \REF{ex:48.alexiadou} and the BP counterpart of \REF{ex:49.alexiadou} is obtained by operator-binding in BP, which explains two factors: i) as long as the verb allows it, the reading of a human entity is not the only one available; ii) unaccusative verbs are licensed. When locatives are related to \isi{INFL}, by contrast, unaccusative verbs are out, because the locative is a scene-setting modifier that will merge above the vP, as an external argument, and a \isi{semantic} human reading is the only one that this element can contribute.

To summarize, we have seen that other types of null impersonals in BP depend on the specification of tense. BP null impersonals with generic \isi{reference} need a locative as an external argument because the specification of \isi{INFL} in this type of data is location rather than tense. This explains the characteristics of BP null impersonals we have witnessed throughout the discussion.

At this point, we can present two derivations for BP and \ili{Finnish} null impersonals.

\ea BP null impersonals (3\textsuperscript{rd} person, generic tense)\\
\begin{forest}
 [\isi{INFL}
  [LOC]
  [VoiceP
    [PP]
    [\ldots]
  ]
 ]
\end{forest}
\z

\newpage 
\ea \ili{Finnish} null impersonals\\
\begin{forest}
 [T
  [uφ]
  [VoiceP
    [Pronoun] [\ldots]
  ]
 ]
\end{forest}
\z
%%
%%{ BP null impersonals (3\textsuperscript{rd} person, generic tense)}
%%\begin{stylenumeracao}
%% \isi{INFL}
%%\end{stylenumeracao}
%%
%%\begin{stylenumeracao}
%% 3
%%\end{stylenumeracao}
%%
%%\begin{styleListNumber}
%% LOC VoiceP
%%\end{styleListNumber}
%%
%%\begin{styleListNumber}
%% 3
%%\end{styleListNumber}
%%
%%\begin{styleListNumber}
%% PP ....
%%\end{styleListNumber}
%%
%%{\ili{Finnish} null impersonals}
%%\begin{stylenumeracao}
%% T
%%\end{stylenumeracao}
%%
%%\begin{stylenumeracao}
%% 3
%%\end{stylenumeracao}
%%
%%\begin{styleListNumber}
%% u$\varphi $ VoiceP
%%\end{styleListNumber}
%%
%%\begin{styleListNumber}
%% 3
%%\end{styleListNumber}
%%
%%\begin{styleListNumber}
%% Pronoun …..
%%\end{styleListNumber}

\section{Conclusion}\label{§7.alexiadou}

We have compared the role of locatives in \ili{Finnish}, BP, two partial \isi{pro-drop} languages, and \ili{Greek}, a \isi{pro-drop} language. The use of locatives in \ili{Finnish} and BP, despite sharing a substantial number of properties, do not overlap. One of the crucial differences is the role of locatives in null impersonals. In BP, these elements behave as arguments, whereas in \ili{Finnish} they are expletive-like elements. The reason why null impersonals in BP and \ili{Finnish} seem so alike, yet are so different in terms of constituency can be explained in terms of the \isi{INFL} each language has. BP can specify 3\textsuperscript{rd} non-referential person with a locative feature in \isi{INFL}, hence locatives can be arguments and expletives in this language. In \ili{Finnish}, locatives satisfy the \isi{EPP}, i.e. are pure expletives, as T bears no specification for location regardless of time or person specification.

Importantly, the difference between null impersonals in the two languages shows that partial \isi{pro-drop} languages cannot be thought as a coherent group. These languages share some properties, such as the behavior of 3\textsuperscript{rd} person, as discussed in \sectref{§2.alexiadou}, but they seem to have chosen different ways of becoming non-\isi{pro-drop} languages. In particular, BP has chosen a different value to \isi{INFL} in 3\textsuperscript{rd} non-referential contexts. Even when \isi{INFL} is specified for time, as seen in \REF{ex:46.alexiadou} and \REF{ex:48.alexiadou}, no phi-features seem to be present and operator-binding generates the generic reading for an argument. \ili{Finnish}, on the other hand, employs tense in null impersonals and locatives only satisfy \isi{EPP}. In \ili{Greek}, a full \isi{pro-drop} language, none of these options is available, V-raising being the main way to satisfy the \isi{EPP}. The differences among the three languages are summarized in \tabref{tab:2.alexiadou}.\is{impersonal|)}

\begin{table}[H]
\begin{tabularx}{\textwidth}{>{\raggedright}XllX}
\lsptoprule
         & \multicolumn{3}{c}{Language}\\\cmidrule(lr){2-4}
         & \ili{Greek} & \ili{Finnish} & BP\\\midrule
Function & Focusing adverb & \isi{EPP} & \isi{EPP}, argument\\\tablevspace
Nodes to which locatives are associated with in the language & {vP adjunct -} \isi{CP} & vP adjunct – \isi{TP} & {vP adjunct, \isi{TP};} external argument, \isi{TP}\\

\lspbottomrule
\end{tabularx}

\caption{Summary of the properties of locatives in the three languages}
\label{tab:2.alexiadou}
\end{table}

\section*{Abbreviations}
\begin{tabbing}
\textsc{part} \= partitive \kill
% \textsc{obl} \> oblique\\ %% OBL is already part of the LPG and so do not need to be quoted here.
\textsc{part} \> partitive
\end{tabbing}
Abbreviations used in this article follow the Leipzig Glossing Rules’ instructions for word-by-word transcription, available at: \url{https://www.eva.mpg.de/lingua/pdf/Glossing-Rules.pdf}.

\section*{Acknowledgements}
We are indebted to the comments of two anonymous reviewers that greatly improved the readability of this paper. Many thanks to Anders Holmberg for discussions and for being a constant source of inspiration through the years. The authors would also like to acknowledge the support received from the DFG (grant AL 554\slash 8) awarded to the first author and CNPq (grant \#142048\slash 2012-7 and \#229746\slash 2013-6) awarded to the second author.

{\sloppy
\printbibliography[heading=subbibliography,notkeyword=this]
}
\end{document}
