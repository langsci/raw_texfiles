\documentclass[output=paper]{langsci/langscibook}
\author{Laura R. Bailey and Michelle Sheehan \affiliation{University of Kent and Anglia Ruskin University}
}
\abstract{\noabstract}
\title{Introduction: Order and structure in syntax}
\ChapterDOI{10.5281/zenodo.1116787}

\maketitle
\begin{document}
Hierarchical structure and argument structure are two of the most pervasive and widely studied properties of natural language.\footnote{All of the papers in this volume were written on the occasion of Anders Holmberg’s 65\textsuperscript{th} birthday in recognition of the enormous contribution he has made to these issues.} The papers in this set of two volumes further explore these aspects of language from a range of perspectives, touching on a \isi{number} of fundamental issues, notably the relationship between linear order and hierarchical structure and variation in subjecthood properties across languages. The first volume focuses on issues of word order and its relationship to structure. This second volume focuses on argument structure and subjecthood in particular. In this introduction, we provide a brief overview of the content of the 10 papers and seven squibs relating to argument structure and subjecthood, drawing out important threads and questions which they raise.   

Many of the contributions in this volume deal with subjects other than canonical referential DPs, such as expletives with some referential meaning, non-DP subjects, pronouns in \isi{pro-drop} languages, or impersonal subjects of one kind or another. Together they provide a snapshot of cross-linguistic variability in subjecthood. Thráinsson’s contribution considers evidence from \ili{Faroese} that the possibility of quirky subjects is parametrically connected to other surface properties by a deep parameter, and ultimately argues that parameters must be ‘soft’. Greco, Haegeman \& Phan consider the status of overt expletives in Vietnamese and what this implies for the \isi{null subject} parameter. Their expletives are not like the canonical ones as they have some discourse meaning. ‘Non-expletive’ expletives also appear in the contribution from Alexiadou \& Carvalho, who argue that locative subjects in some partial \isi{pro-drop} languages are expletive-like, while in others they are referential. Taraldsen’s chapter also discusses locative subjects, arguing that the PP subjects found in \ili{Norwegian} are genuine subjects and move to canonical subject position. Similarly, Anagnostopoulou uses her contribution to argue for a difference between Movement and \isi{Agree}, arguing that some phenomena which have been argued to involve \isi{Agree} actually involve movement of the subject to Spec,\isi{TP}. 

Both Egerland and Sigurðsson and the squibs from Engdahl and Krzek focus on the interpretation of certain kinds of subjects. Sigurðsson discusses those instances of \textit{we} that cannot be said to include the speaker, and argues for a version of Ross’s performative hypothesis, similar to that defended by Wiltschko (vol. 1). Egerland focuses on first-person impersonal pronouns such as \ili{German} \textit{man} and \ili{Italian} \textit{si} and argues that a plural interpretation is lexically specified in some languages, and must be the interpretation in certain contexts. Krzek returns to \isi{null subject} languages with a squib on null impersonal subjects in \ili{Polish}, while Engdahl discusses expletive \isi{passive} constructions and (un)expected word orders in the \ili{Scandinavian} varieties. Wurmbrand’s squib focuses on the status of \ili{Icelandic} in relation to the \isi{null subject} parameter. Based on the behaviour of fake indexicals, she argues that \ili{Icelandic} is indeed a partial \isi{null subject} language, despite its exceptional behaviour in certain respects. 

A \isi{number} of the contributions focus on object arguments rather than subjects. Van der Wal presents data from \ili{Bantu} languages and shows that they differ with respect to their symmetry and case-licensing properties in ditransitive constructions. She further proposes a novel implicational hierarchy to capture the observed patterns and provides a formalization of this in terms of sensitivity to topicality. It is the absence of ditransitives that fuels Bobaljik’s squib, as he notes that \ili{Icelandic} does not allow ECM distransitives despite lacking the adjacency condition supposed to ban them. This in turn means that \isi{Case} Theory cannot explain this systematic gap. Lee’s squib deals with object drop in Chinese, and returns to the theme of non-specific arguments with indefinite antecedents. Algryani combines the themes of ellipsis and answers to questions with a proposal for fragment answers in \ili{Arabic}. Fassi Fehri focuses on the role of \isi{gender} features on all arguments, arguing that a combination of properties means that \isi{gender} has a range of meanings including diminutive and evaluative, among others. 

Lastly, two of the squibs are about the properties of compounds:\is{compound} \isi{recursive} ones in the case of Mukai, while Johannessen discusses the class of parasynthetic compounds in \ili{Norwegian} of the type \textit{brown-eyed}, whose heads do not surface alone as adjectives. 

\newpage 
This volume, like the first, provides new data and analysis based on a wide range of languages. In all these papers, the influence of the work of Anders Holmberg can be observed, from the typology of \isi{null subject} languages and the status of expletive, locative and generic subjects to the syntax of ditransitives and the status of \isi{V2}. 

\end{document}