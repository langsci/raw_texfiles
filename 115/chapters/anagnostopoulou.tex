\documentclass[output=paper]{LSP/langsci} 
\author{Elena Anagnostopoulou 	\affiliation{University of Crete}}
\title{Defective intervention effects in two Greek varieties and their implications for φ-incorporation as Agree} 
\shorttitlerunninghead{Defective intervention effects in two Greek varieties}
% \epigram{Change epigram}
\abstract{In this paper, I argue that pro-drop configurations cannot be analyzed as formally identical to downward Agree configurations. I take as a starting point the observation that in monoclausal constructions clearly involving downward Agree, as in Icelandic and Dutch, the presence of a dative intervener does not block Agree between T and a lower nominative argument. I then investigate two types of intervention effects in Standard and Northern Greek and argue that intervention effects in the presence of an indirect object arise always, regardless of whether the nominative subject is overt or covert and regardless of whether a subject DP remains in its base position or moves overtly. This leads me to conclude that the relevant constructions always display movement.}
\ChapterDOI{10.5281/zenodo.1116763}

\maketitle

\begin{document}

 
\section{Introduction}
\largerpage[-2]
In his seminal paper on Null Subject Parameters, \citet{Holmberg2010Null} argues that \isi{pro-drop} configurations in consistent and partial Null Subject Languages always involve incorporation of a φP to T.\footnote{Holmberg argues that the two language types differ in whether T contains a D feature or not. In consistent Null Subject Languages, T contains D and therefore \isi{null subjects} can be definite. In partial Null Subject Languages, on the other hand, T lacks D and therefore \isi{null subjects} are either arbitrary/indefinite or expletive but never definite.}  This type of incorporation, however, is claimed not to be movement. Adopting the theory of \citet{Roberts2010}, Holmberg proposes that incorporation of a φP in T is the direct effect of \isi{Agree} \citep{Chomsky2001} and works as follows.  \isi{Finite} T has a set of unvalued φ-features and probes for a category with matching valued features (step 1 in \ref{ex:anagnost:1}). The defective subject pronoun in vP has the required valued φ-features which are copied by T and thus value T’s uφ-features. At the same time, T values the subject’s unvalued case feature (step 2 in \ref{ex:anagnost:1}). As a result, T shares all of φ’s feature values. The result is the same as if φ had moved, by head movement, incorporating into T, but without actual movement taking place. According to Holmberg, the advantage of head-move as \isi{Agree} is that it avoids the problem posed by head movement, namely the lack of c-command between the links of a head chain (but see \citealt{Lechner2006,Lechner2007}). Following \citet{Roberts2010}, \citet{Holmberg2010Null} furthermore proposes that the probe and the goal form a chain, which is subject to chain reduction falling under the rules in \REF{ex:anagnost:2}. The subject φP is therefore not pronounced (by \ref{ex:anagnost:2a}; indicated under step 3 in \ref{ex:anagnost:1}), and the chain is pronounced in the form of an affix on the \isi{finite} verb or \isi{auxiliary}, following incorporation of V+v into T.

\ea
\begin{xlistn}\label{ex:anagnost:1}
\ex  \mbox{[T, D, uφ, NOM] [\textsubscript{vP} [3SG, uCase] v....] →}
\ex  \mbox{[T, D, 3SG, NOM] [\textsubscript{vP} [3SG, NOM] v...] →}
\ex  \mbox{[T, D, 3SG, NOM ] [\textsubscript{vP} \st{[3SG, NOM]} v..]}
\end{xlistn}
\z
\ea\label{ex:anagnost:2}
\begin{xlista}
  \ex \label{ex:anagnost:2a}  Pronounce the highest chain copy.
  \ex  Pronounce only one chain copy.
\end{xlista}
\z


\largerpage[-2]
In this paper, I present an argument based on intervention effects that φ-incorporation in the sense of \citet{Holmberg2010Null} and \citet{Roberts2010} cannot be reduced to downward \isi{Agree}. Specifically, I discuss monoclausal configurations displaying \isi{agreement} between the verb and a subject DP in \ili{Icelandic} and \ili{Dutch} and show that when \isi{agreement} is the result of downward \isi{Agree}, an intervener does not block \isi{Agree} between T/v and the subject. By contrast, constructions in which the subject moves to spec,\isi{TP} are subject to intervention effects in both languages. I then discuss comparable intervention effects in two varieties of \ili{Greek}, Standard and Northern \ili{Greek}, which are both consistent Null Subject Languages. Crucially, intervention effects arise always, regardless of whether the subject is overt or covert, and regardless of the preverbal vs. postverbal position of the subject when this is overt. In view of the \isi{Agree} vs. Move asymmetry regarding monoclausal intervention in non-Null Subject Languages, the presence of intervention effects in Null Subject Languages leads to the conclusion that what Holmberg and Roberts call ``φ-incorporation'' involves actual movement.\footnote{An anonymous reviewer strongly objects to the idea of abandoning Holmberg’s non-move incorporation and suggests that the asymmetry discussed in the paper is not necessarily an argument against it. I am quoting from the reviewer: ``The paper relies crucially on this derivational analysis (or ``hierarchical-structural'') of IE (intervention effect). It does not attempt to explore (not even refer) to potential alternatives, which could ultimately ``save'' Holmberg’s \isi{Agree} analysis. Suppose that IE are not so construed, being rather ``informational'' (prosodic), read off linear strings (and probably subject to variable interpretive judgments). Then the constraints on their presence (or absence) do not depend on \isi{Agree}/Move choices, but crucially on the information structure of the intervener (see e.g. \citealt{Tomioka2007} or \citealt{Eilam2009}, among others). This potential analysis of IE is compatible with the general absence of IE in Amharic, and extendable to alternative questions in which an intervener preceding a disjunctive phrase removes the alternative question reading, leaving the yes/no reading. Other ``\isi{semantic}'' accounts of IE have been brought up by \citet{Beck2006} and others, which may or may not be adequate. The point is not whether or not the Move account of the IE asymmetry is or is not correct; the paper does not show that it is unavoidable, and it does not attempt to look at alternatives that preserve \isi{Agree} incorporation as generally relevant for both IE and non-IE contexts.'' The reviewer is certainly correct that the argument made in the paper crucially relies on a derivational analysis of strong and weak intervention effects (IEs), and might also turn out to be correct that an informational account of IEs could rescue Holmberg’s non-move incorporation. However, \isi{semantic}/pragmatic accounts of IEs along the lines of \citet{Beck2006,Tomioka2007} and \citet{Eilam2009} have been discussed in the context of \isi{wh-movement}, and it is not obvious whether and how they can be extended to capture intervention effects in Move and \isi{Agree} in passives, unaccusatives, raising and expletive-associate constructions of the type discussed here. In the absence of such an account for A movement, I do not see why one should not construct an argument based on the standard view of IEs. Exploring alternatives in order to preserve \isi{Agree} Incorporation is the aim of a different paper.  Note that, as mentioned in the main text, the main advantage of \isi{Agree} incorporation according to Holmberg is that it avoids head movement. In \isi{agreement} with \citet{Lechner2006,Lechner2007,Lechner2009,Baker2009} and others I do not share the view that head movement should be dispensed with.}  



\section{No intervention on local Agree, intervention on local Move: Icelandic and Dutch}
As is widely discussed in recent years (\citealt{HolmbergHróarsdóttir2003} and many others), ``defective intervention effects'' \citep{Chomsky2000} on downward \isi{Agree} arise in biclausal constructions. In \ili{Icelandic}, a matrix raising predicate cannot enter \isi{Agree} with an embedded nominative argument in \isi{number} across an intervening dative experiencer subject, as in \REF{ex:anagnost:3a}, while \isi{agreement} is possible if the intervener moves to the higher clause, as in \REF{ex:anagnost:3b} (\citealt{Watanabe1993,Schütze1997}):

\newpage
\ea\label{ex:anagnost:3}
\ili{Icelandic}\\
\ea \label{ex:anagnost:3a}
\gll  Mér ?*virðast/virðist [Jóni vera taldir t líka hestarnir].\\
Me\textsc{.dat} seemed\textsc{.pl/sg} Jon\textsc{.dat} be believed\textsc{.pl} t like horses\textsc{.nom}\\
\glt ‘I perceive John to be believed to like horses.’
\ex \label{ex:anagnost:3b}
\gll Jóni virðast/?*virðist [t vera taldir t líka hestarnir].\\
Jon\textsc{.dat} seemed\textsc{.pl/sg} t be   believed\textsc{.pl} t   like horses\textsc{.nom}\\
\glt ‘John seems to be believed to like horses.’
\z
\z

But in monoclausal constructions things are different, as stressed by \citet{Bobaljik2008}. In \ili{Icelandic} monoclausal configurations featuring an expletive or a PP in the preverbal position, \isi{number} \isi{agreement} between the inflected verb and a lower nominative argument across an intervening dative is always possible, and generally obligatory, as shown by the data in \REF{ex:anagnost:4} (from \citealt{Jónsson1996} and \citealt{Zmt1985}; \citealt[298, 321]{Bobaljik2008}):

\ea\label{ex:anagnost:4}
\ili{Icelandic}\\
\ea
\gll Það líkuðu einhverjum þessir sokkar.    \\
expl liked\textsc{.pl} someone\textsc{.dat} these  socks\textsc{.nom}\\
\glt ‘Someone liked these socks.’

\ex
\gll Um veturinn  voru konunginum gefnar ambáttir.\\
  {In the} winter  were.\textsc{pl} {the king.\textsc{dat}} given slaves.\textsc{nom}\\
\glt 
  ‘In the winter the king was given (female) slaves.’  

\ex
\gll Það voru konungi gefnar ambáttir í vettur.\\
  \textsc{expl} were.\textsc{pl} king.\textsc{dat} given slaves.\textsc{nom} in winter\\
\glt   ‘There was a king given maidservants this winter.’

\ex
\gll Það voru einhverjum gefnir þessir sokkar.\\
   \textsc{expl} were.\textsc{pl} someone.\textsc{dat} given  these  socks.\textsc{nom}\\
\glt 
  ‘Someone was given these socks.’
\z
\z

Bobaljik concludes that defective intervention on downward \isi{Agree} does not arise in monoclausal configurations. He furthermore proposes to view the contrast between biclausal and monoclausal constructions as an argument for a domain-based characterization of intervention effects according to which, the position of the dative is indicative of the presence of a domain boundary in \REF{ex:anagnost:3a} but not in \REF{ex:anagnost:3b}; cf. \citet{Nomura2005}. 

The conclusion that downward \isi{Agree} in monoclausal constructions is not subject to defective intervention is reinforced by evidence from \ili{Dutch} discussed in \citet{Anagnostopoulou2003}. \ili{Dutch} passives and unaccusatives with an \textit{in} \textit{situ} nominative subject following a dative DP are grammatical, as shown in \REF{ex:anagnost:5} 
% (den Dikken\index{«B»} \citeyear{denDikken1995}: 208, fn 26).  
\citep[208, fn 26]{denDikken1995}.  
Notice that both the dative and the nominative argument are vP internal, since they follow the adverb  \textit{waarschijnlijk} which is taken to mark the left edge of the vP:

\ea\label{ex:anagnost:5}
\ili{Dutch}\\
\ea \label{ex:anagnost:5a}
\gll  dat waarschijnlijk [\textsubscript{vP} Marie het boek gegeven] wordt\\
  That probably {}  Mary.\textsc{dat} the book.\textsc{nom} given is\\

\ex \label{ex:anagnost:5b}
\gll
 dat  waarschijnlijk [\textsubscript{vP} Marie het boek bevallen] zal\\
  that  probably {} Mary.\textsc{dat} the book.\textsc{nom} please will\\

\ex \label{ex:anagnost:5c}
\gll  dat waarschijnlijk [\textsubscript{vP} de jongen de teugels ontglipten]\\
  that probably {} the boys.\textsc{dat} the  reins.\textsc{nom} slipped\\
\z
\z

The facts in \REF{ex:anagnost:5} provide evidence that T, which I take to be situated to the right of the vP where the auxiliaries reside in \REF{ex:anagnost:5a} and \REF{ex:anagnost:5b}, can enter downward \isi{Agree} with an in situ nominative across a higher dative, i.e. the dative does not cause an intervention effect for \isi{Agree} between T and the nominative argument vP-internally. 


Crucially, an intervention effect does arise when the nominative argument undergoes overt NP-movement to spec,\isi{TP} across the vP internal dative.  Consider the following contrast observed by  \citet[207--208]{denDikken1995}:

\ea\label{ex:anagnost:6}
\ili{Dutch}
\ea[?*]{\label{ex:anagnost:6a}
\gll dat [\textsubscript{TP} het boek waarschijnlijk [\textsubscript{vP} Marie \st{het book} gegeven]  wordt] \\
that {} the book.\textsc{nom} probably {} Mary.\textsc{dat} {} given is\\}
\ex[]{\label{ex:anagnost:6b}
\gll dat [\textsubscript{TP}  het boek Marie waarschijnlijk [\textsubscript{vP} \st{Marie het book} gegeven ] wordt] \\
that {} the book.\textsc{nom} Mary.\textsc{dat} probably {} {} given {} is\\
\glt ‘that the book is probably given to Mary’}
\z
\z

In \REF{ex:anagnost:6}, movement of the nominative theme leads to a relatively mild deviance if the DP goal occurs to the right of the adverb \textit{waarschijnlijk}, as in \REF{ex:anagnost:6a},\textit{} and results in a fully well-formed output when it occurs to its left, as in \REF{ex:anagnost:6b}. If argument placement to the left of VP-external adverbs
% \index{«A»}
signifies scrambling%
% \index{«A»}
, then these facts suggest that passivization across an intervening DP goal is subject to an intervention effect in \ili{Dutch}, unless the goal undergoes scrambling. \citet{Anagnostopoulou2003} argues that DP scrambling of the intervener, just like \isi{cliticization} of genitive IO interveners in \ili{Greek} (see \sectref{sec:anagnost:4} below for \isi{cliticization}), is a strategy to obviate intervention effects. The same contrast is found in (non-alternating) unaccusatives, as shown in \REF{ex:anagnost:7} and \REF{ex:anagnost:8}: 

\ea\label{ex:anagnost:7}
\ili{Dutch}  
\ea[?*]{
\gll  dat het boek waarschijnlijk Marie bevallen zal\\
  that the book.\textsc{nom} probably Mary.\textsc{dat} please will\\}
\ex[]{
\gll  dat  het  boek Marie waarschijnlijk bevallen zal\\
  that  the book.\textsc{nom} Mary.\textsc{dat} probably please will \\
\glt ‘that the book will probably appeal to Mary’}
\z
\z

\ea\label{ex:anagnost:8}
\ili{Dutch}\\
\ea[??]{
\gll dat de teugels waarschijnlijk de jongen ontglipten\\
  that  the  reins.\textsc{nom} probably the boys.\textsc{dat} slipped\\}
\ex[]{
\gll  dat de teugels de jongen waarschijnlijk ontglipten\\
  that  the reins.\textsc{nom} the  boys.\textsc{dat} probably slipped\\
\glt ‘that the reins probably slipped out of the boys’ hands’}
\z
\z

While it blocks Move, the vP internal dative does not block \isi{Agree} between the nominative and T, as was shown in \REF{ex:anagnost:5}. In order to account for this difference between Move and \isi{Agree} with respect to intervention, \citet[222]{Anagnostopoulou2003} proposed that the features turning \ili{Dutch} datives into interveners
% \index{«A»} 
are their D/EPP-features%
% \index{«A»}
, and not their \isi{Case}/φ-features. \ili{Icelandic} shows that the Agree-Move asymmetry with respect to intervention is more general. As is well-known and widely discussed in the literature, in the counterparts of \REF{ex:anagnost:4} lacking an expletive or a PP in the preverbal position, it is the higher quirky dative and not the lower nominative DP that is allowed to move to Spec,\isi{TP}. I conclude that defective interveners block Move and not \isi{Agree} because their D features make them interveners, and D features are relevant for Move/\isi{EPP} processes, not for \isi{Agree}/φ-feature valuation processes. 


\section{Pro-drop and case distribution in two varieties of Greek}
As is well known, \ili{Greek} is a language showing all the properties associated with consistent Null Subject Languages. It has definite subject omission \REF{ex:anagnost:9}, lack of expletives with impersonal and weather verbs \REF{ex:anagnost:10}, absence of that-trace effects \REF{ex:anagnost:11}, availability of VS, VSO and VOS orders \REF{ex:anagnost:12}: 

\ea \label{ex:anagnost:9}
De\isi{finite} subject omission  \\
\gll graf-o, graf-is, graf-i, graf-ume, graf-ete, graf-{}-un\\
write.\oldstylenums{1}\textsc{sg}, write.\oldstylenums{2}\textsc{sg}, write.\oldstylenums{3}\textsc{sg}, write.\oldstylenums{1}\textsc{pl}, write.\oldstylenums{2}\textsc{pl}, write.\oldstylenums{3}\textsc{pl}\\
\glt 
  ‘I write, you write, he/she/it writes, we, you, they write’
\z


\ea \label{ex:anagnost:10}
No expletives with impersonal and weather verbs  \\
\gll Fenet-e oti tha vreks-i.\\
Seem.\oldstylenums{3}\textsc{sg} that \textsc{Fut} rain.\oldstylenums{3}\textsc{sg}\\
\glt 
‘It seems that it will rain.’
\z


\ea \label{ex:anagnost:11}
No that-trace effects \\
\gll Pjos ipes oti efige? \\
  Who said.\oldstylenums{2}\textsc{sg} that left\\
\glt 
  ‘*Who did you say that left?’
\z


\ea \label{ex:anagnost:12}
 VS, VSO, VOS orders  \\
\ea\label{ex:anagnost:12a}
\gll Efige o Janis.\\
  left.\oldstylenums{3}\textsc{sg} the Janis.\textsc{nom}\\
\glt 
  ‘John left.’


\ex
\gll  Egrapse o Janis to vivlio.\\
  wrote.\oldstylenums{3}\textsc{sg} the Janis.\textsc{nom} the book.\textsc{acc}\\


\ex
 \gll Egrapse to vivlio o Janis.\\
  wrote.\oldstylenums{3}\textsc{sg} the book.\textsc{acc} the Janis.\textsc{nom}\\

\glt 
  ‘John wrote the book.’
\z
\z

In addition, \ili{Greek} lacks the null indefinite/ arbitrary subject typically found in partial Null Subject Languages \citep{Holmberg2010Null}. It has (i) null exclusive 3\textsuperscript{rd} person plural indefinite subjects (\citealt{BellettiRizzi1988,Pesetsky1995,Condoravdi1989}), (ii) null inclusive 2\textsuperscript{nd} person singular subjects with arbitrary \isi{reference} or (iii) overt expressions with arbitrary \isi{reference} corresponding to English ‘one’:


\ea\label{ex:anagnost:13}
\ili{Greek}: Indefinite Subjects  
\ea
\gll  Su tilefonisan. Prepi na itan o Janis.\\
  Cl.\oldstylenums{2}\textsc{gen} called.\oldstylenums{3}\textsc{pl}\textsc{.} Must \textsc{subj} was.\oldstylenums{3}\textsc{sg}  the Janis.\textsc{nom}\\
\glt    ‘Someone called you. It must have been John.’
\ex
\gll  Dulevis sklira stin Ellada ke xoris na plironese.\\
  Work.\oldstylenums{2}\textsc{sg} hard in-the Greece and without \textsc{subj}     pay.\textsc{nact.\oldstylenums{2}sg}\\
\glt 
  ‘One works hard in \ili{Greek} and without getting paid.’
\ex
\gll  Dulevi kanis  sklira stin Ellada ke  xoris  na plironete.\\
  Work.\oldstylenums{3}\textsc{sg} one hard in-the Greece and  without \textsc{subj} pay.\textsc{nact.\oldstylenums{3}sg}\\
\glt 
  ‘One works hard in \ili{Greek} and without getting paid.’
\z
\z

\ili{Greek} has morphological nominative (NOM), accusative (ACC) and genitive (GEN) case. \isi{Nominative} occurs on subjects, accusative on direct objects (DOs) and most prepositional complements and genitive is the case assigned DP internally. Moreover, Ancient \ili{Greek} datives (DATs) were lost in Medieval \ili{Greek} and have been replaced in ditransitives and two-place unaccusatives by either GENs or ACCs, depending on the dialect (see \citealt{AnagnostopoulouSevdali2015} for discussion and references). Standard Modern \ili{Greek} and many southern dialects have GEN-ACC/NOM constructions, while Northern \ili{Greek} dialects have ACC-ACC/NOM constructions (\citealt{Dimitriadis1999} and references cited there). The IO is not allowed to alternate with NOM in passives, regardless of whether it bears GEN (in Standard \ili{Greek}) or ACC (in Northern \ili{Greek}) in actives:

\ea\label{ex:anagnost:14}
Standard \ili{Greek}: No GEN  – NOM alternations in passives  
\ea[]{
\gll  Edosa  tu Petru ena pagoto.\\
  Gave.\oldstylenums{1}\textsc{sg} the  Peter.\textsc{gen} an icecream.\textsc{acc}\\
\glt 
  ‘I gave Peter an ice-cream.’}

\ex[*]{
\gll O Petros dothike ena pagoto.\\
  The Peter.\textsc{nom} gave.\textsc{nact}  an ice-cream.\textsc{acc}\\
\glt 
  ‘Peter was given an ice-cream.’}
\z
\z

\ea\label{ex:anagnost:15}
Northern \ili{Greek}: No ACC  – NOM alternations in passives\\
\ea[]{
\gll  Edosa  ton Petro ena pagoto.\\
  Gave.\oldstylenums{1}\textsc{sg} the  Peter.\textsc{acc} an ice-cream.\textsc{acc}\\
\glt 
  ‘I gave Peter an ice-cream.’}


\ex[*]{
\gll  O Petros dothike ena pagoto.\\
  The Peter.\textsc{nom} gave.\textsc{nact}  an ice-cream.\textsc{acc}\\
\glt 
  ‘Peter was given an ice-cream.’}
\z
\z

In both varieties, only the DO bearing accusative is allowed to alternate with NOM. Finally, both varieties qualify as consistent Null Subject Languages.

\section{Weak and Strong Intervention in Standard and Northern Greek}\label{sec:anagnost:4}
Both Standard and Northern \ili{Greek} have defective intervention effects in monoclausal \isi{passive} and unaccusative constructions displaying NP-movement of the DO across the IO. However, the two types of intervention have very different properties. Here I will only discuss passivized ditransitives in the two dialects.\footnote{\label{fn:anagnost:3}I thank Sabine Iatridou, Despina Oikonomou and Giorgos Spathas for their judgments on Northern \ili{Greek}. I thank Mark Baker and Ruth Kramer for a discussion that led me to discover the Northern \ili{Greek} intervention pattern.}

Standard \ili{Greek} has a defective intervention effect caused by the GEN IO when the NOM DO undergoes NP-movement across it, as in \REF{ex:anagnost:16a} \citep{Anagnostopoulou2003}. The effect is weak, i.e. the resulting sentence is deviant and not strongly unagrammatical, as is the case with \ili{Dutch} \REF{ex:anagnost:6a}, and can be rescued if the intervener surfaces as a clitic or is clitic doubled, as in \REF{ex:anagnost:16b}, similarly to the \ili{Dutch} scrambling strategy we saw in \REF{ex:anagnost:6b}:   

\ea\label{ex:anagnost:16}
Standard \ili{Greek}: Weak Intervention Effect  
\ea[?*]{\label{ex:anagnost:16a}
\gll To pagoto  dothike tu Petru apo tin Maria.\\
  The ice-cream.\textsc{nom}  gave.\textsc{nact} the Peter.\textsc{gen} by the Mary\\
\glt  ‘The ice-cream was given Peter by Mary.’}
\ex[]{\label{ex:anagnost:16b}
\gll To pagoto tu  dothike (tu Petru) apo tin Maria.\\
The ice-cream.\textsc{nom} cl.\textsc{gen} gave.\textsc{nact} the Peter.\textsc{gen} by the Mary\\
\glt   ‘The ice-cream was given Peter by Mary.’}
\z \z

I will call this ‘a weak defective intervention effect’. \isi{Experimental} evidence in \citet{Georgala2012} supports the view that, even though the deviance of \REF{ex:anagnost:16a} is mild, an intervention effect is indeed present and is obviated in \REF{ex:anagnost:16b}. Specifically, Georgala applies the magnitude estimation experimental method (\citealt{GurmanEtAl1996,Cowart1997,Keller2000}) to such sentences and finds out that sentences like \REF{ex:anagnost:16a} are consistently and systematically scored much lower than their counterparts in \REF{ex:anagnost:16b} by native speakers of Standard \ili{Greek}.   

Northern \ili{Greek} also has a defective intervention effect caused by accusative IOs in passives. The NOM theme is not allowed to move to the subject position across an intervening ACC goal, i.e. the following is ungrammatical:

\largerpage
\ea\label{ex:anagnost:17}
Northern \ili{Greek}: Strong Intervention Effect\\
\gll   * To pagoto dothike  ton Petro.\\
  {} The ice-cream.\textsc{nom}  gave.\textsc{nact} the Peter.\textsc{acc}\\

\glt 
  ‘The ice-cream was given Peter.’    
\z


My consultants (mentioned in footnote \ref{fn:anagnost:3}) are unanimous in judging \REF{ex:anagnost:17} as strongly ungrammatical, and the sentence cannot be rescued by \isi{cliticization} or doubling. The following is equally ungrammatical:


\ea \label{ex:anagnost:18}
Northern \ili{Greek}: no escape strategy with clitics\\
\gll *To pagoto ton dothike (ton Petro).\\
  The ice-cream.\textsc{nom}  \textsc{cl}.\textsc{acc} gave.\textsc{nact} the Peter.\textsc{acc}\\
\glt 
  ‘The ice-cream was given him (Peter).’
\z

I will call this ‘a strong defective intervention effect’. What seems to be crucial for the emergence of weak vs. strong defective intervention in \ili{Greek} is the morphological case of the IO. In both Standard and Northern \ili{Greek} the lower theme cannot undergo movement to spec,\isi{TP} across a higher goal, but the effect is much stronger when the intervener is an ACC argument, as schematized in \REF{ex:anagnost:19b}, than when it is a GEN argument, as in \REF{ex:anagnost:19a}: 



\ea\label{ex:anagnost:19}
\ea\label{ex:anagnost:19a}
\mbox{[\textsubscript{TP}   NOM   T[\textsubscript{vP} [\textsubscript{ApplP} GEN \sout{NOM}]]]   GEN=weak intervener }

\ex\label{ex:anagnost:19b}
\mbox{[\textsubscript{TP}   NOM  T [\textsubscript{vP} [\textsubscript{ApplP} ACC \sout{NOM}]]]  ACC=strong intervener}
\z
\z

It is unclear at this point why exactly morphological case matters, since neither the GEN IO nor the ACC IO alternate with NOM in passives, as was seen in \REF{ex:anagnost:14} and \REF{ex:anagnost:15}, i.e. both are defective interveners, in the sense of \citet{Chomsky2000}. 

Moreover, we saw that GEN intervention is obviated by \isi{cliticization}/clitic doubling of the intervener. The by now standard account for this fact (see e.g. \citealt{Anagnostopoulou2003,Preminger2009} and others) is that the features blocking NP-movement of NOM to T in \REF{ex:anagnost:19a} no longer intervene between NOM and T when \isi{cliticization} takes place, because \isi{cliticization} is movement targeting T, the same position targeted by NP movement, and neither the trace of clitics in \REF{ex:anagnost:20a} nor their DP doubling associate in \REF{ex:anagnost:20b} count anymore as interveners. 



\ea\label{ex:anagnost:20}
\ea\label{ex:anagnost:20a}
\mbox{[\isi{TP}   NOM  cl{}-T  [vP  [\textsubscript{ApplP}  \sout{GEN} \sout{NOM}]]]}

\ex\label{ex:anagnost:20b} 
\mbox{[\isi{TP}   NOM  cl{}-T  [vP  [\textsubscript{ApplP}  GEN \sout{NOM}]]]}
\z
\z

The question is why the same strategy cannot be employed in configurations of strong intervention, as in Northern \ili{Greek} \REF{ex:anagnost:19b}. Speakers agree that the sentences substantially improve if the ACC intervener is a 1st or 2nd person clitic, as in \REF{ex:anagnost:21}, a fact suggesting that there is a problem caused by a 3rd person ACC clitic in sentences like \REF{ex:anagnost:18} (reminiscent of the conditions triggering the spurious \textit{se} rule in Spanish, \citealt{Bonet1991}). 

\ea%21
    \label{ex:anagnost:21}
  	  Northern \ili{Greek}: improvement with 1\textsuperscript{st}/ 2\textsuperscript{nd} person intervener\\
\gll ? To pagoto me/se dothike.  \\
{} The ice-cream.\textsc{nom}  cl.\textsc{acc.\oldstylenums{1}sg/\oldstylenums{2}sg} gave.\textsc{nact}\\
\glt 
‘The ice-cream was given me/you.’
\z

When the intervener is 3rd person, speakers resort to a GEN strategy in order to rescue sentences like \REF{ex:anagnost:17} and \REF{ex:anagnost:18}. Standard \ili{Greek} \REF{ex:anagnost:16a} and \REF{ex:anagnost:16b} are acceptable for Northern \ili{Greek} speakers, and GEN IOs are judged not to be interveners, regardless of whether they are full DPs (though I am skeptical about this; see footnotes 4 and 6 below), clitics or clitic doubled DPs.\footnote{There is more to be said here. It could be that my consultants, which are also speakers of Standard \ili{Greek}, resort to their Standard \ili{Greek} grammar and, at the same time, they belong to those speakers of Standard \ili{Greek} that do not have weak defective intervention at all. Alternatively, the contrast between the sharply ungrammatical Northern \ili{Greek} and the mildly ungrammatical Standard \ili{Greek} version of the sentence is so strong that they judge the NOM-GEN construction as grammatical, while the magnitude estimation experimental method might show that there is still a contrast between a GEN DP and a GEN clitic.} Importantly, a very similar pattern of intervention is found with objects in Northern \ili{Greek}, unlike Standard \ili{Greek}. In a nutshell, ACC DO 3rd person clitics cannot co-occur with ACC IO DPs \REF{ex:anagnost:22a}, two 3rd person clitics are not allowed to form ACC-ACC clusters \REF{ex:anagnost:22b} and speakers have to resort to Standard \ili{Greek} GEN-ACC clusters \REF{ex:anagnost:22c} instead, while 1st and 2nd person ACC IOs can form clusters with 3rd person ACC DOs \REF{ex:anagnost:22d}:



\ea%22
    \label{ex:anagnost:22}
    	 Northern \ili{Greek}: intervention effects with objects  
\ea[*]{\label{ex:anagnost:22a}
\gll To edosa ton Petro (to pagoto).\\
    Cl.\textsc{acc} gave.\textsc{Act.\oldstylenums{1}sg} the Peter.\textsc{acc} the icecream.\textsc{acc}\\
\glt ‘I gave Peter the ice-cream.’}
 
\ex[*]{\label{ex:anagnost:22b}
\gll Ton to edosa (ton Petro)  (to pagoto).\\
Cl.\textsc{acc} cl.\textsc{acc} gave.\textsc{Act.1sg} the Peter.\textsc{acc} the icecream.\textsc{acc}\\
\glt   ‘I gave Peter the ice-cream.’}
 
\ex[]{\label{ex:anagnost:22c}
\gll  Tu to edosa (tu Petru) (to pagoto).\\
  Cl.\textsc{gen} cl.\textsc{acc} gave.\textsc{Act.\oldstylenums{1}sg} the Peter.\textsc{gen} the   icecream.\textsc{acc}\\
\glt  ‘I gave Peter the ice-cream.’}
\ex[]{\label{ex:anagnost:22d}
\gll Me/se  to edose (to pagoto). \\
  Cl.\oldstylenums{1}\textsc{/\oldstylenums{2}.acc} cl.\oldstylenums{3}.\textsc{acc} gave.\textsc{Act.\oldstylenums{3}sg} the icecream.\textsc{acc}\\
\glt  ‘He/she gave me/you the ice-cream.’}
\z
\z

These facts suggest that there is a problem when two 3rd person arguments bearing ACC and/or NOM enter \isi{Agree} with the same head, whether this is T or v, in Northern \ili{Greek}. Here I will not attempt to provide a solution to these puzzles. What matters for present purposes is the very existence of weak and strong defective intervention in Standard and Northern \ili{Greek}, respectively. 



\section{Defective intervention under pro-drop and its implications}


Neither weak defective intervention nor strong defective intervention in passives cease to occur under \isi{pro-drop} of the NOM argument. Consider first the Standard \ili{Greek} pattern: 


\ea%23
\label{ex:anagnost:23}
Standard \ili{Greek}: Weak intervention under pro drop:\\

\gll Apo pjon dothike to vivlio ston Petro?\\
 By whom gave.\oldstylenums{3}\textsc{nact} the  book.\textsc{nom} to-the Peter\\
\glt ‘By whom was the book given to Peter?’ 

\gll  ?? Dothike tu Petru apo ton kathigiti.\\
{} Gave.\textsc{nact.\oldstylenums{3}sg} the Peter.\textsc{gen} by the professor\\
 
\gll Tu dothike apo  ton kathigiti.\\
Cl.\textsc{gen} gave.\textsc{nact.\oldstylenums{3}sg} by the professor\\
 
\gll Tu dothike tu Petru apo ton kathigiti.\\
Cl.\textsc{gen} gave.\textsc{nact.\oldstylenums{3}sg} the Peter.\textsc{gen} by the professor\\

\glt 
‘It was given to Peter by the professor.’
\z 



\ea%24
\label{ex:anagnost:24}
Standard \ili{Greek}: Weak intervention under pro drop:\\
\gll Apo pjon apagoreftike I isodos ston Petro?\\
 By whom forbid.\oldstylenums{3}\textsc{nact} the entrance.\textsc{nom} to Peter\\
\glt ‘By whom was Peter forbidden the entrance?’
 

\gll ?* Apagoreftike   tu Petru apo tin astinomia.\\
{} Forbid.\textsc{nact.\oldstylenums{3}sg} the Peter.\textsc{gen} by the police\\

\gll Tu apagoreftike apo tin astinomia.\\
Cl.\textsc{gen} forbid.\textsc{nact.\oldstylenums{3}sg} by the police.\\

 
\gll Tu apagoreftike tu Petru apo tin astinomia.\\
Cl.\textsc{gen} forbid.\textsc{nact.\oldstylenums{3}sg} the Peter.\textsc{gen} by the police\\

\glt ‘Peter was forbidden the entrance by the police.’
\z 

As shown in \REF{ex:anagnost:23} and \REF{ex:anagnost:24}, a weak intervention effect is caused by undoubled GEN DPs when the subject is null, just as with overt NOM subjects.

The same is shown in Northern \ili{Greek} with strong intervention. The sharp ungrammaticality of an overt ACC IO DP or clitic, persists when the subject is covert, as shown in \REF{ex:anagnost:25} and \REF{ex:anagnost:26}:\footnote{I thank Despina Oikonomou (personal communication) for also providing contexts for all Northern \ili{Greek} sentences below.}

\ea%25
\label{ex:anagnost:25}
Northern \ili{Greek}: Strong intervention under \isi{pro-drop}\\
\ea\label{ex:anagnost:25a}
\isi{Question}.\\
\gll    Pu ine to vivlio mu?\\
 Where   is the book.\textsc{nom} my.\textsc{gen}\\
\glt ‘Where is my book’?
\ex Answer.\\
\gll * Dothike ton  Petro  \\
 {} Gave.\textsc{nact.\oldstylenums{3}sg} the  Peter.\textsc{acc}.\\
\glt ‘It was given to Peter.’
\z
\z

\ea%26
\label{ex:anagnost:26} Northern \ili{Greek}: Strong intervention under \isi{pro-drop}  \\
\ea\label{ex:anagnost:26a}
\isi{Question}.\\
\gll   Dosane to vivlio ston Petro?\\
 Gave.\textsc{act.\oldstylenums{3}pl} the book.\textsc{acc} to-the Peter\\
\glt ‘Did they give the book to Peter?’
\ex Answer.\\  
\gll * Ne, ton dothike xtes.  \\
 {} Yes, cl.\textsc{acc} gave.\textsc{nact.\oldstylenums{3}sg} yesterday        \\
\glt  ‘Yes, it was given to him yesterday.’
\z
\z

And just as with overt NOM subjects, the relevant \isi{null subject} constructions improve when the IO surfaces as a GEN DP\footnote{Note that the question context provided for an undoubled GEN DP in \REF{ex:anagnost:27b} requires emphasis on the GEN DP since it is construed as an answer to a wh-question. In this context, I would also use an undoubled genitive DP, since doubling is incompatible with focus/emphasis. I assume that the undoubled GEN undergoes covert focus movement in \REF{ex:anagnost:27b}, which is another strategy for obviating weak defective intervention. It is therefore more appropriate to check the status of sentences with an undoubled GEN DP in contexts without emphasis, like the ones in \REF{ex:anagnost:23} and \REF{ex:anagnost:24} above. And indeed, Despina Oikonomou (personal communication) confirms that she has a weak intervention effect with an undoubled GEN in contexts like \REF{ex:anagnost:23} and \REF{ex:anagnost:24} and a very strong intervention effect with an ACC IO in the same contexts, regardless of whether the ACC is a DP, a clitic or a clitic doubled DP and regardless of emphasis.}  or clitic:

\ea%27
\label{ex:anagnost:27}
Northern \ili{Greek}: Improvement when IO is GEN (Standard \ili{Greek} pattern)\\
\ea\label{ex:anagnost:27b}
\isi{Question}.\\   
\gll Pu ine to vivlio mu?\\
 Where   is the book.\textsc{nom} my.\textsc{gen}\\
\glt ‘Where is my book’?
\ex Answer.\\
\gll Dothike tu Petru.  \\
 Gave.\textsc{nact.\oldstylenums{3}sg} the  Peter.\textsc{gen}\\
\glt ‘It was given to Peter.’
\z
\z


\ea%28
\label{ex:anagnost:28}
Northern \ili{Greek}: Improvement when IO is GEN (Standard \ili{Greek} pattern)\\
\ea
\isi{Question}.\\
\gll Dosane to vivlio ston Petro?\\
 Gave. \textsc{nact.\oldstylenums{3}pl} the book.\textsc{acc} to-the Peter\\
\glt ‘Did they give the book to Peter?’
\ex 
Answer.\\
\gll  Ne, tu  dothike xtes.  \\
 Yes, \textsc{cl}.\textsc{gen} gave.\textsc{nact.\oldstylenums{3}sg} yesterday      \\
\glt ‘Yes, it was given to him yesterday.’
\z\z

Recall that it was concluded in section 2 on the basis of evidence from \ili{Icelandic} and \ili{Dutch} that defective interveners block Move and not \isi{Agree} because their D features make them interveners, and D features are relevant for Move/\isi{EPP} processes, not for \isi{Agree}/φ-feature valuation processes. If this conclusion is correct, then the presence of weak intervention in Standard \ili{Greek} and strong intervention in Northern \ili{Greek} under \isi{pro-drop} indicates that Null Subject constructions involve not just downward \isi{Agree} between T and the \isi{null subject} but movement of the zero subject to T. In turn, this casts doubt on \citegen{Holmberg2010Null} and \citegen{Roberts2010} proposal that φ-incorporation of \isi{null subjects} is formally indistinguishable from long distance \isi{Agree} configurations. On Holmberg’s account outlined in the introduction, the only difference between the \isi{Agree} derivation in \REF{ex:anagnost:29} for null nominatives in \ili{Greek} and the \isi{Agree} Derivation in \REF{ex:anagnost:30} for overt nominatives in \ili{Icelandic} \REF{ex:anagnost:4} and \ili{Dutch} \REF{ex:anagnost:5} is that the probe and the goal do not form a chain and hence are not subject to chain reduction. And yet, GEN and ACC IOs are interveners in \REF{ex:anagnost:29} while  DAT IOs are not interveners in \REF{ex:anagnost:30}:    

\newpage
\ea%29
\label{ex:anagnost:29}
\begin{xlistn}\let\exfont\small
\ex \mbox{[T, D, uφ, NOM] [\textsubscript{vP} v [\textsubscript{ApplP} ?*\fbox{GEN} /*\fbox{ACC} \isi{Appl} [3SG, uCase]…] →}
\ex \mbox{[T, D, 3SG, NOM] [\textsubscript{vP} v [\textsubscript{ApplP} ?* \fbox{GEN} /* \fbox{ACC} \isi{Appl} [3SG, NOM] →}
\ex \mbox{[T, D, 3SG, NOM ] [\textsubscript{vP} v [\textsubscript{ApplP}  ?*\fbox{GEN} /* \fbox{ACC} \isi{Appl} [\sout{3SG, NOM}]]}
\end{xlistn}
\z

\ea%30
\label{ex:anagnost:30}
\begin{xlistn}\let\exfont\small
\ex \mbox{[T, D, uφ, NOM] [\textsubscript{vP} v [\textsubscript{ApplP}    \fbox{DAT}  \isi{Appl} [\textsubscript{DP} D [3SG, uCase] [\textsubscript{NP}  N..]] →}
\ex \mbox{[T, D, 3SG, NOM] [\textsubscript{vP} v [\textsubscript{ApplP}   \fbox{DAT}  \isi{Appl} [\textsubscript{DP} D [3SG, NOM] [\textsubscript{NP}  N..] ] → }
\ex \mbox{[T, D, 3SG, NOM ] [\textsubscript{vP} v [\textsubscript{ApplP}  \fbox{DAT}   \isi{Appl} [\textsubscript{DP} D [3SG, NOM]  [\textsubscript{NP}  N..] ]}
\end{xlistn}
\z

I therefore propose that the two derivations are not identical. In \isi{pro-drop} configurations, there is movement of the subject from vP to \isi{TP}, while monoclausal \isi{agreement} in \ili{Icelandic} and \ili{Dutch} with a vP internal NOM involves downward \isi{Agree} between T and NOM.\footnote{Mark Baker (personal communication) suggests that one could appeal to the fact that \isi{agreement} with a nominative argument over a dative inside the same clause is weakened, at least in \ili{Icelandic}, so that there is \isi{agreement} in \isi{number} but not in person (\citealt{Taraldsen1995,Sigurðsson1996} and many others) in order to explain why \isi{pro-drop} languages always show defective intervention within Holmberg’s \isi{Agree} approach. Specifically, Mark Baker suggests that person \isi{agreement} is blocked in this configuration, and if there is not a person feature on T, then T and the subject do not share all their features, so that it doesn’t count anymore as a movement chain, and the lower instance does not delete. In such an approach, it is the weakening of \isi{agreement} that prevents \isi{pro-drop} from occurring in the relevant sentences and not locality of movement \textit{per se.} In order for this account to work, one would have to say that person plays a role in \isi{pro-drop} even of third person nominals, despite the fact that they do not have marked person features. Even though an approach along these lines is appealing, I do not think that it will work for \isi{pro-drop} languages which crucially differ from \ili{Icelandic} in never showing a person restriction on nominatives in configurations of downward \isi{Agree}. The constructions showing such an effect in languages like \ili{Greek} are clitic constructions, and the weakening effect only arises with accusative clitics (the well-known \isi{PCC} effect), not with nominatives.} 

What kind of movement is involved in \isi{pro-drop} sentences? Perhaps the simplest analysis would be to follow \citet{Holmberg2010Null} and, more generally, those who assume that pro is syntactically present but not realized at PF (\citealt{Rizzi1986,CardinalettiStarke1999,Roberts2010} and others) and to analyze pro/φ-in\-cor\-po\-ration as actual movement of pro/φ to T. Under the assumption that intervention effects of the type described above are triggered by intervening D-features, it must also be assumed that pro in consistent Null Subject Languages contains a D-layer and not just φ-features. Building on \citet{Tomioka2003,Barbosa2013} argues that this is correct. The different properties of consistent vs. partial Null Subject Languages w.r.t. the definiteness of pro discussed in \citet{Holmberg2010Null} as well as the properties of empty arguments in radical topic drop-lan\-guages (e.g. Japanese) systematically correlate with differences in the internal make-up of DPs and the availability of overt vs. covert definite object pronouns under ellipsis in the languages in question. This correlation can be explained if overt and covert arguments in consistent Null Subject Languages have a D layer missing from overt and covert arguments in partial and radical \isi{pro-drop} languages. 

An alternative I would like to explore, though, is to adopt Alexiadou \& Anagnostopoulou’ proposal (A\& A \citeyear{AlexiadouAnagnostopoulou1998}) that this movement has the form of [v-V]-to-T raising, thus linking the movement nature of \isi{pro-drop} configurations to verb-movement as a way of satisfying the \isi{EPP}. Working in the lexicalist framework of \citet{Chomsky1995}, A\& A proposed that verbal \isi{agreement} morphology in consistent Null Subject Languages is pronominal, i.e. it bears D features. As a result, the \isi{EPP} in these languages is always satisfied via V-to-T raising. For this reason, overt preverbal subjects are \isi{Clitic} Left Dislocated and never the result of \isi{A-movement} to Spec,\isi{TP}. On this view, the NP-movement configurations discussed in \sectref{sec:anagnost:4} for \ili{Greek} do not involve NP-movement of the DP but NP-movement of the zero resumptive subject pro corresponding to overt object clitics in object \isi{CLLD} constructions. This analysis has sometimes been criticized (see e.g \citealt{SpyropoulosRevithiadou2009} for \ili{Greek}), but \citet{Barbosa2009} offers many interesting novel arguments from European vs. \ili{Brazilian Portuguese} in favor of the \isi{CLLD} analysis of preverbal subjects in consistent Null Subject Languages. One such argument that carries over to \ili{Greek} comes from the observation that preverbal subjects in consistent Null Subject Languages are ungrammatical in contexts where \isi{CLLD} is excluded for independent reasons, while they are grammatical in non-pro drop languages. Absolute constructions are the case in point. The subject must precede the Aux-V complex in these environments in English and \ili{French} (from \citealt{Barbosa2009}, ex. 80 and 81, while it follows \isi{Aux} or the Aux-V complex in Spanish, \ili{Italian} and European Portuguese  (Barbosa’s 82--84)):

\ea%31
\label{ex:anagnost:31}
English: S-\isi{Aux}/V\\
Your brother having called, we left.
\z

\ea%32
\label{ex:anagnost:32}
\ili{French}: S-\isi{Aux}/V\\
Ton frère ayant téléphoné, je suis parti.  
\z

\ea%33
\label{ex:anagnost:33}
Spanish: V-S\\
\gll Habiendo (el juez) resuelto (el juez) absolver al acusado el juicio concluyó sin incidentes.\\
having (the judge) decided (the judge) {to acquit} the accused the trial concluded without incidents\\
\glt ‘The judge having decided to acquit the accused, the trial came to an end without further incidents.’
\z

\ea%34
\label{ex:anagnost:34}
\ili{Italian}: \isi{Aux}/V-S \\
\gll Avendo (tuo fratello) telefonato (tuo fratello) (,io sono rimasto a casa).\\
having your brother called I am stayed at home\\
\glt ‘Your brother having called, I stayed at home.’
\z

\ea%35
\label{ex:anagnost:35}
European Portuguese: V - S\\
\gll Aparecendo a Maria, vamos embora.\\
Showing up the Maria, we-leave.\\
\glt ‘As soon as Maria shows up, we leave.’
\z

The same holds in \ili{Greek}, where the preverbal subject is strongly deviant, as shown in \REF{ex:anagnost:36b}:

\ea%36
\label{ex:anagnost:36}
\ili{Greek} V-S\\
\ea[]{\label{ex:anagnost:36a}
\gll Emfanizomeni  i Maria, tha  figume.\\
     {Showing up}  the Mary, \textsc{fut} go.\oldstylenums{1}\textsc{pl}\\
\glt ‘As soon as Maria shows up, we will leave.’}
\ex[?*]{\label{ex:anagnost:36b}
\gll I Maria emfanizomeni, tha figume.\\
The Mary {showing up}, \textsc{fut} go.\oldstylenums{1}\textsc{pl}\\
\glt ‘As soon as Maria shows up, we will leave.’}
\z \z

Updating \citet{AlexiadouAnagnostopoulou1998} in a non-lexicalist model of grammar, I propose that in consistent Null Subject Languages the \isi{null subject} undergoes \isi{merger} with the verbal complex and is spelled out in the form of a [+ pronominal] affix on the main verb or \isi{auxiliary}.\footnote{Following \citet{AlexiadouAnagnostopoulouSchäfer2006,AlexiadouAnagnostopoulouSchäfer2015} I assume that the verbal complex consists of the root, a verbalizing head introducing an event and Voice introducing an external argument. There is evidence that the external argument is introduced below the \isi{auxiliary} head in the \ili{Greek} perfect, because the participle is either active or \isi{passive}, i.e. it contains Voice:
\begin{exe}
\exi{(i)} 
\begin{xlista}
 \ex \gll O Janis   exi lisi  tis askisis.\\
          The Janis.\textsc{nom}  has.\oldstylenums{3}\textsc{sg} solved.\textsc{act} the exercises.\textsc{acc}\\
     \glt ‘John has solved the .
 \ex \gll I askisis exoun  lithi apo ton Jani.\\
          The excercises.\textsc{nom} have.\oldstylenums{3}\textsc{pl} solved.\textsc{nact} by the John\\
      \glt ‘The exercises have been solved by John.’
\end{xlista}
\end{exe}

Since the \isi{auxiliary} shows subject \isi{agreement}, we must assume that in these constructions the \isi{null subject} raises to \isi{Aux} and then \isi{merges} with it. The reason why the subject must merge with the \isi{auxiliary} and is not allowed to merge with the participle has to do with the fact that the \isi{auxiliary} and not the participle is allowed to satisfy the \isi{EPP} property of T since it is closer to T than the participle.} Subsequent raising of the v+V+[pron] affix to T satisfies the \isi{EPP} property of T in the manner suggested by \citet{AlexiadouAnagnostopoulou1998}. I  propose that the mode by which the zero subject combines with the verb is identical to the process by which object clitics combine with the \isi{finite} verb in \isi{cliticization} structures, essentially treating \isi{null subjects} as clitics (see \citealt{Sportiche1996,AlexiadouAnagnostopoulou1998,AlexiadouAnagnostopoulou2001} and others). Following \citet{Nevins2011} I assume that clitics undergo syntactic rebracketing, the Merger operation of \citet{Matushansky2006} which rebrackets two heads that are in a specifier head configuration as a complex head:

\ea%37
\label{ex:anagnost:37}
Rebracketing Merger:\\
\begin{multicols}{2}
\begin{forest}
[YP [X] [Y' [Y] [ZP] ] ]
\end{forest}\\
\begin{forest}
[YP [Y' [Y [X] [Y] ] [ZP] ] ]
\end{forest}
\end{multicols}
\z

Subject pro is a D head bearing φ-features, just like a clitic, and undergoes rebracketing \isi{merger} from its base position in spec,VoiceP (see footnote 8) in transitives and unergatives with the complex Root-v-Voice head created by head movement of the Root to v and Voice:\footnote{In passives and unaccusatives the base position of pro is the position occupied by themes, which is probably outside the projection of the stative Root, i.e. in spec,vP,  in alternating change of state unaccusatives, and a Root-complement in non-alternating unaccusatives, verbs of creation and destruction. This raises non-trivial questions concerning the point at which D[iφ] undergoes Merger with the verbal complex and whether an IO, if present, is expected to cause an intervention effect or not on Merger, if Merger happens after  the verbal complex is formed (which would seem to entail that D[iφ] first moves to the edge of the position hosting the verbal complex and then rebracketing happens). These questions are left open here because they require working out where themes reside in all relevant structures, whether D[iφ] and nominative arguments more generally move to the edge of v/Voice or directly to T in passives and unaccusatives and, if the former, how exactly intervention works when Voice/v is targeted. The two \ili{Greek} varieties sharply differ with respect to the latter issue. In Standard \ili{Greek}, GEN IOs do not block \isi{cliticization} of an ACC DO across them while 3\textsuperscript{rd} person ACC IOs cause a strong intervention effect on \isi{cliticization} of an ACC DO.} 

\parbox{\textwidth}{%
\ea%38
\label{ex:anagnost:38}
\begin{multicols}{2}
\begin{xlist}
\ex \label{ex:anagnost:38a}
      \begin{forest}
	  [VoiceP [D {[}iφ{]}] [Voice' [Voice [v [Root] [v] ] [Voice]] [] ]]
      \end{forest}
\ex \label{ex:anagnost:38b}
     \begin{forest}
      [VoiceP [Voice' [Voice [D {[}iφ{]}] [Voice [v [Root] [v]] [Voice] ] ] [] ] ]
     \end{forest}
\end{xlist}
\end{multicols}
\z
}

If we take suffixal \isi{agreement} morphology to \isi{spell out} D[iφ], then D[iφ] in \REF{ex:anagnost:38b} is right linearized with respect to the verbal complex, while object clitics are left linearized with respect to the verbal complex. Further \isi{verb movement} to T brings along the rebracketed subject which satisfies the \isi{EPP} requirement of T.\textstyleFootnoteSymbol{} 

\section{Defective intervention and NOM \textit{in situ} in Greek}

As a final point, I will briefly discuss intervention effects in sentences where the DP argument bearing nominative \isi{Case} remains \textit{in situ} in \ili{Greek}, and their implications. As already observed in \citet[85]{Anagnostopoulou2003}, Standard \ili{Greek} differs from \ili{Dutch} (and \ili{Icelandic}) in having weak intervention effects in apparent downward \isi{Agree} configurations in monoclausal constructions. Examples with \textit{in situ} subjects still require clitic doubling or \isi{cliticization} in \ili{Greek} passives and unaccusatives: 




\ea%39
\label{ex:anagnost:39}Standard \ili{Greek} : weak intervention with in situ subjects\\
\ea[?*]{
\gll (tu) dhothike tu Petru to vivlio.\\
  {Cl.\textsc{gen}} gave.\textsc{nact.\oldstylenums{3}sg} the Petros.\textsc{gen} the book.\textsc{nom}\\
\glt ‘The book was given to Peter.’}
\ex[?*]{%
\gll (tis) irthe tis Marias to grama. \\
  {Cl.\textsc{gen}} came the Maria.\textsc{gen} the  letter.\textsc{nom}\\
\glt ‘The letter came to Mary.’}
\ex[?*]{%
\gll (tu) aresun tu Petru ta vivlia.\\
  {Cl.\textsc{gen}} please-\oldstylenums{3}\textsc{pl} the Petros.\textsc{gen} the books. \textsc{nom}\\
\glt ‘Peter likes books.’}
\z
\z

The same holds for strong intervention in Northern \ili{Greek}, where a NOM theme is not allowed to co-occur with a 3rd person ACC DP or clitic or clitic doubled IO, as shown in \REF{ex:anagnost:40}:

 
\ea%40
    \label{ex:anagnost:40}
  	 Northern \ili{Greek}: strong intervention with in situ subjects  \\
\ea
\gll *Xthes dothike ton Petro to pagoto.\\
  Yesterday gave.\textsc{nact} the Peter.\textsc{acc} the ice-cream.\textsc{nom}\\
\ex
\gll *Xthes ton dothike to pagoto.\\
  Yesterday cl.\textsc{acc} gave.\textsc{nact} the ice-cream.\textsc{nom}\\
\ex
\gll *Xthes ton dothike ton Petro to pagoto.\\
   Yesterday cl.\textsc{acc} gave.\textsc{nact} the Peter.\textsc{acc} the icecream.\textsc{nom}\\

\glt 
  ‘The ice-cream was given to Peter yesterday.’
\z
\z

In order to account for this difference between \ili{Greek} and \ili{Dutch}/\ili{Icelandic}, in \citet{Anagnostopoulou2003} I appealed to the consistent \isi{pro-drop} and clitic doubling\footnote{Note that not all Null Subject Languages are also clitic doubling languages, for example \ili{Italian} and Catalan are not, at least as far as DO clitic doubling is concerned. \citet{AlexiadouAnagnostopoulou2001} argue that only in clitic doubling languages verbal \isi{agreement} enters a doubling configuration with a full DP. As a result, \ili{Greek}, Romanian and Spanish permit VSO orders with both S and O vP-internal in violation of the \textit{Subject-in-situ Generalization.} In \ili{Italian} and Catalan clitic doubling is not possible, and therefore these languages only allow VOS orders and not VSO orders. But, crucially, in VOS orders the object has moved to the edge of the vP conforming with the \textit{Subject in situ Generalization.} This makes the prediction that if these languages have intervention effects of the type described above for \ili{Greek}, these would be obviated if the nominative remained in its vP internal position, i.e. that \ili{Italian} and Catalan would behave like \ili{Dutch} and \ili{Icelandic} and not like \ili{Greek} w.r.t. intervention effects with \textit{in situ} nominatives. I do not know whether this prediction can be tested since in these languages ‘a-datives’ are not interveners to begin with (presumably because they are ambiguous between a prepositional dative and an applicative\is{Appl} dative).}  nature of \ili{Greek}, as opposed to \ili{Dutch} and \ili{Icelandic}, and I proposed that the relation between subject \isi{agreement} on V and the overt DP subject in \ili{Greek} is an instance of clitic doubling.\footnote{Note that analyzing \isi{agreement} with subjects as an instance of clitic doubling raises the question of why object doubling imposes referentiality conditions on the doubled DP while subject doubling doesn’t. This is a more general question concerning doubling analyses of \isi{agreement} phenomena, as argued for by e.g. \citet{Preminger2009} and \citet{Nevins2011}. I believe that the difference between doubling/\isi{agreement} without interpretational effects vs. doubling/\isi{agreement} displaying such effects should be linked to the obligatoriness of the former vs. optionality of the latter. See \citet{BakerKramer2015} for an alternative view that referentiality conditions constitute the only reliable diagnostic for classifying a dependency as a doubling one.}  It is generally agreed upon that clitic doubling is a movement dependency, which means that some part of the nominative moves to T even when it is pronounced \textit{in situ} (\citealt[224--226]{AlexiadouAnagnostopoulou2001}). Since movement is sensitive to intervention effects, the pattern in \REF{ex:anagnost:39} follows. There are several ways to represent this clitic doubling / movement dependency (see Anagnostopoulou, to appear, for summarizing the relevant literature on clitic doubling and different proposals). Which one to choose depends on how we want to analyze \isi{null subject} constructions to begin with.\footnote{An anonymous reviewer points out that it is unsatisfying not to take a firm position regarding which analysis of \isi{pro-drop} I take to be correct. In view of the complexities and debates on the Null-Subject Parameter, however, (see e.g. \citealt{DAlessandro2015} for an overview of the relevant issues), it is beyond the scope of the present paper to address the syntax and parametrization of \isi{null subject} phenomena in detail. The intervention data I discuss show that movement is a crucial component in \isi{pro-drop} structures; in addition, they provide evidence that covert subjects in \ili{Greek}-type languages have a D-layer and move overtly. In principle, these crucial properties can be expressed both in an A\& A (\citeyear{AlexiadouAnagnostopoulou1998}) style-analysis and in terms of a more conventional analysis, with a null D-pronominal moving to T. In my view, the A\& A analysis has the advantage that it automatically derives both movement and the presence of a D layer by linking them to the EPP-driven movement of the \isi{agreeing} verb. A definitive choice between the two main analytic options, however, would require an in depth investigation of the properties of different Null Subject Languages, the nature of micro- and macro-variation in different types of \isi{null subject} constructions, an analysis of partial pro drop languages, an understanding of the relationship between SVO, VSO and VOS orders in different Null Subject Languages, among other issues.}  For example, if we basically follow \citegen{Holmberg2010Null} analysis with the modifications introduced above (true φ-incorporation combined with the hypothesis that \isi{null subjects} also contain D), then the most adequate analysis for clitic doubling would be that the clitic is a copy of a DP moving to the host, which spelled out as a pronoun (the reverse of a resumptive pronoun chain), a possibility explored by \citet{Harizanov2014} and \citet{Kramer2014}. On this analysis, the copy of a moved subject would be the suffixal verbal \isi{agreement}. On the alternative analysis that verbal subject \isi{agreement} results from \isi{merger} of a subject clitic with the verbal complex, the most compatible analysis of clitic doubling would either be that doubling clitics \isi{spell out} D/φ-features of the DP moving to the host \citep{Anagnostopoulou2003} or a version of the ``big DP hypothesis'' according to which clitics are determiner heads, as in \REF{ex:anagnost:41} (\citealt{Torrego1988,Uriagereka1995} and the literature building on them), with Ds moving to the host:

\ea%41
    \label{ex:anagnost:41}
\begin{forest}
[DP [(double)] [D' [D\\clitic,base=top, align=left] [NP\\pro,base=top, align=left] ] ] 
\end{forest}
\z

A variant of this proposal is that D is adjoined to the DP/KP (similarly to floated quantifiers) and moves to the host stranding the DP/KP \citep{Nevins2011}. On both proposals, the subject doubling clitic would merge with the verbal complex in the way described above for non-doubling subject clitics.\footnote{There are other options not presented here for both \isi{null subject} constructions and clitic doubling constructions. For example, one could adopt a version of \citegen{Sportiche1996} proposal and analyze verbal subject \isi{agreement} as T’s φ-features which are interpretable in \isi{pro-drop} languages. They combine with a zero pro or an overt subject which moves to T covertly. The difference between subject doubling constructions and object doubling constructions would be that the presence of φ-features in T are obligatory, while φ-features on v (object doubling) are optional and associated with interpretive effects.}  



\section{Summary}


In this paper I employed intervention effects in monoclausal constructions as a way of diagnosing whether an \isi{agreement} construction should be analyzed as φ-feature valuation under \isi{Agree} or as the result of movement. I took as a starting point the observation that in monoclausal constructions clearly involving downward \isi{Agree}, as in \ili{Icelandic} and \ili{Dutch}, the presence of a dative intervener does not block \isi{Agree} between T and a lower nominative argument. By contrast, dative arguments in these languages do cause intervention effects blocking movement of the nominative argument to T. I then identified two types of intervention effects in two different varieties of \ili{Greek}, namely weak defective intervention attested in Standard \ili{Greek} and strong defective intervention found in Northern \ili{Greek}. Both are consistent Null Subject Languages. I presented evidence that weak and strong intervention effects in these dialects arise always, regardless of whether the nominative subject is overt or covert and regardless of whether a subject DP remains in its base position or moves overtly. This led me to conclude that the relevant constructions always display movement. I explored some ways in which this movement can be represented. Choosing among the alternatives for \isi{null subject} constructions also has implications for constructions with overt \textit{in situ} nominatives, which necessitate a doubling/movement analysis in \ili{Greek}, in order for intervention effects to be accounted for.        



\section*{Acknowledgments}

Some of the new observations presented here concerning Northern \ili{Greek} intervention effects have been triggered by an e-mail conversation with Mark Baker and Ruth Kramer (Fall 2015). I thank Sabine Iatridou, Giorgos Spathas and Despina Oikonomou for their judgments on Northern (Thessaloniki) \ili{Greek}, Despina Oikonomou for her feed-back on the data, Mark Baker and two anonymous reviewers for their comments. I thank Anders Holmberg for his many important contributions to syntax, his insights, and the generosity he showed towards what later became \citet{AlexiadouAnagnostopoulou1998} which helped tremendously an early career. If Wim Wenders had known Anders, he would have included him in the group of angels in \textit{Der Himmel über Berlin}. This research has been supported by an Alexander von Humboldt Friedrich Wilhelm Bessel Research award which is gratefully acknowledged.

\sloppy
\printbibliography[heading=subbibliography,notkeyword=this]
\end{document}