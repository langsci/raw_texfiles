\documentclass[output=paper]{LSP/langsci} 
\author{Janne Bondi Johannessen\affiliation{MultiLing, Department of Linguistics and Scandinavian Studies, 
University of Oslo
}}
\title{Puzzling parasynthetic compounds in Norwegian}  
% \epigram{Change epigram}
\abstract{This paper describes parasynthetic compounds in Norwegian and questions some recent claims made in the literature about this kind of word formation. In particular, it will be argued that they are not marginal, but productive, and that they are semantically compositional.
}
\ChapterDOI{10.5281/zenodo.1116771}

\maketitle

\begin{document} 

 
%%please move the includegraphics inside the {figure} environment
%%\includegraphics[width=\textwidth]{a20Johannessen-img1.tif}

 
%%please move the includegraphics inside the {figure} environment
%%\includegraphics[width=\textwidth]{a20Johannessen-img2.png}

 
%%please move the includegraphics inside the {figure} environment
%%\includegraphics[width=\textwidth]{a20Johannessen-img3.tif}

 
%%please move the includegraphics inside the {figure} environment
%%\includegraphics[width=\textwidth]{a20Johannessen-img4.png}






\section{Introduction}\label{sec:bondi:1}


The existence of parasynthetic compounds  provides linguistics with some puzzles that I shall discuss, though not solve, in this paper.  Parasynthetic compounds are compounds that consist of three parts, where  any combination of just two of the parts would be ungrammatical, and where there is a bracketing paradox, see the \ili{Norwegian} example in \REF{ex:1}. They can be found in many other Indo-European languages, such as the other mainland North Germanic languages \ili{Swedish} \citep{TelemanEtAl1999} and \ili{Danish}  \citep{Hansen2011}, English \citep{Hirtle1970}, \ili{Greek}, \ili{Slavic} and \ili{Romance} \citep{MelloniBisetto2010}.\footnote{Some examples from other languages are given below.
\begin{exe}
\ex *in+busta *bust(a)+are → im+bust+are \upshape`to put in an envelope’     (\ili{Italian})
\ex *red-blood *blooded → red-blooded     \upshape    (English)
\ex *blauwog+ig *blauw+ogig → blauwogig \upshape `blue-eyed’     (\ili{Dutch})
\ex *kokkino+mal *mal+is → kokkinomalis \upshape `red-haired’     (\ili{Greek})
\ex   *obc(o)kraj+oc obc(o)+*krajoc → obcokrajowiec \upshape `foreigner’   (\ili{Polish}) \citep[199--201]{MelloniBisetto2010}
\end{exe}}

\ea%1
    \label{ex:1}
\ea
 
\textit{rødøyd}\\
red-eyed\\
`red-eyed’

\ex
 \gll \textup{Three parts:} rød\textsubscript{\textup{adj}} + øye\textsubscript{\textup{noun}} +   d\textsubscript{\textup{adj-suffix}}\\
 {} red {} eye {}         d\\

\ex
 
  Ungrammatical combinations of two: \textit{*rødøye, *øyd}

\ex
 
  Bracketing paradox:\\
    Semantically: [[\textit{rød}+\textit{øye}]\textit{d}]\\
    Morphologically: [\textit{rød}+[[\textit{øye}]\textit{d}]]]\footnote{When a lexical stem ends in \textit{–e}, it is deleted under certain morphophonological conditions, thus \textit{øye}, but \textit{øyd}. This process is general and applies in many other contexts than parasynthetic compounding.} 
\z
\z

My perspective will be that of \ili{Norwegian}, and the puzzles, some of which have been raised as claims in the literature,  are these: What do the strict requirements for the parts of speech of the individual compound members mean for syntactic theory? How strict is the category of inalienable possession? Why do they behave morphophonologically as past participles? Why are they often non-compositional semantically? Are they marginal? 

The paper is structured in the following way. An empirical investigation is carried out in \sectref{sec:bondi:2}, using a special compound search interface to the dictionaries and one big corpus. This section also comments on the usability of these empirical resources. \sectref{sec:bondi:3} discusses several aspects of parasynthetic compounds, partly based on claims in the literature. It is discussed whether parasynthetic compounds are a marginal phenomenon, whether they are semantically compositional, why they have the same morphophonological suffix as past participles, to what extent there is a relationship of inalienable possession, and finally their strict categorial restrictions. \sectref{sec:bondi:4} concludes the paper. Using these rich empirical data collections it will be demonstrated that not all claims in the literature can be defended.

\section{Empirical investigation}\label{sec:bondi:2}

Parasynthetic compounds in \ili{Norwegian} have been briefly discussed in \citet{Johannessen2001} and more thoroughly, with a \isi{semantic} focus,  in \citet{Grov2009}.  In order to test claims and get a further basis for the questions posed in \sectref{sec:bondi:1}, a thorough empirical investigation is necessary. There are two types of sources of data that seem particularly appropriate for finding such compounds in \ili{Norwegian}. Both are large electronic data collections, where there is a special option for searching for compounds.  One type of data is dictionaries, more specifically, the \isi{reference} dictionaries \textit{Bokmålsordboka} \citep{Wangensteen2005} and \textit{Nynorskordboka} \citep{Hovdenak2001}. These books are the official dictionaries of the two written varieties of \ili{Norwegian} (Bokmål and Nynorsk).  The other type of empirical source is the NoWaC-corpus (\textit{\ili{Norwegian} Web as a Corpus}) \citep{Guevara2010}. 

These two types of empirical source complement each other. The dictionaries only contain compounds that are sufficiently established for the lexicographers to accept them as worthy of entries or subentries.  The corpus, on the other hand, includes all the compounds that have been coined by the authors of the texts it contains.

\subsection{Parasynthetic compounds in the two dictionaries}

A special search interface for compounds exists for the dictionaries. The compounds in the dictionaries have all been manually annotated, based on the original compounds in the dictionary (Bjørghild Kjelsvik, p.c.). This means that all the compounds are well-formed (in that they represent \ili{Norwegian} words) and have a correct analysis.  

The simple search interface makes it possible to express a search such as:  return all compounds that end in \textit{–t} (one of the  common adjectival derivational suffixes for parasynthetic compounds), and that are adjectives.  The type of results that are returned are illustrated in \figref{fig:bondi:1}, which also illustrates the compound analysis returned by the search interface. 

  
\begin{figure}
% \includegraphics[width=\textwidth]{figures/a20Johannessen-img5.tif}
% \includegraphics[width=\textwidth]{figure/a20Johannessen-img6.png}
\includegraphics[width=.3\textwidth]{figures/bondiscreenshot1.png}
\includegraphics[width=.6\textwidth]{figures/bondiscreenshot2.png}
\caption{Results of a search for compounds that are adjective and that end in \textit{–t}.}
\label{fig:bondi:1}
\end{figure}



The list in \figref{fig:bondi:1} shows that we do not only get parasynthetic compounds. There is also a substantial number of a similar kind of compound where the second member is derived from a verb (and in effect is a past participle). These are not parasynthetic, since past participles can occur on their own. The analysis in \figref{fig:bondi:1} shows that the lexicographers have chosen not to include the original part of speech of the second member (i.e. noun), and have only included the resulting part of speech of the whole second member including the adjectival derivational suffix. To illustrate, the parasynthetic compound \textit{brei-kinnet} ‘broad-cheeked’ has been (wrongly) given the same structure as the past participle \textit{bort-glømt} ‘away-forgotten’:


\ea%2
    \label{ex:2}
 
\ea
 
  \textit{brei}+Adj+Seg+-\textit{kinnet}+Adj+Pos+Sg+Indef

        broad                   cheeked

       ‘broad-cheeked’

\newpage 
\ex
 
  \textit{bort}+\isi{Adv}+Seg+-\textit{glømt}+Adj+Pos+Sg+Indef

         away                   forgotten

    ‘totally forgotten’
\z
\z

It would have been better for our purpose if the analysis had included the part of speech of their original second member, as suggested in (\ref{ex:bondi:3}):

\ea%3
    \label{ex:bondi:3}
 

\textit{brei}+Adj+Seg+-\textit{kinn}+\textbf{Noun}{}-\textit{et}+Adj+Pos+Sg+Indef

\textit{bort}+\isi{Adv}+Seg+-\textit{gløm}+\textbf{Verb}{}-\textit{t}+Adj+Pos+Sg+Indef
\z

For this reason we will not know exactly how many parasynthetic compounds there are in the dictionaries. Some examples of the irrelevant past participles are given in \REF{ex:bondi:4} and (\ref{ex:bondi:bort}). Some of the many parasynthetic compounds are given in \REF{ex:bondi:5}.

\ea%4
    \label{ex:bondi:4}
 

	 Compound past participles

\ea
\textit{bevegelses-hemmet}\\
movement-impaired\\
\glt ‘movement-impaired’

\ex  \textit{bort-bestilt}\\ 
away-booked\\
\glt ‘booked by somebody else’
\z
\z

\ea \label{ex:bondi:bort}
\ea  \textit{bort-glømt}\\ 
away-forgotten\\
\glt ‘totally forgotten’

\ex   \textit{bort-reist} \\ 
away-gone\\
\glt ‘gone away’
\z
\z

\ea%5
    \label{ex:bondi:5}
    
	 Parasynthetic compounds

\ea   \textit{bløt-hjertet} 

soft-hearted

‘soft-hearted’

\ex   \textit{brei-skuldret} \\
 

broad-shouldered

‘broad-shouldered’

\ex   \textit{bred-kinnet} \\
 

broad-cheeked

‘broad-cheeked’

\ex   \textit{brei-kjeftet} \\
 

wide-mouthed

‘wide-mouthed’

\ex  \textit{bar-føtt} \\
 

bare-footed

‘bare-foot’

\ex   \textit{blank-øyd} \\
 

shiny-eyed

‘shiny-eyed’

\ex   \textit{blid-lynt} \\
 

happy-tempered

‘happy-tempered’

\ex   \textit{blid-mælt} \\
 

happy-voiced

‘happy-voiced’

\ex   \textit{brå-lynt} 

quick-tempered

‘quick-tempered’

\ex  \textit{djup-gjengt} 

deep-threaded

 ‘deep-threaded’

\ex   \textit{en-cellet} \\
 

one-celled

 ‘one-celled’

\ex   \textit{fir-beint} \\
 

four-legged

‘four-legged’
\z
\z
\subsection{ Parasynthetic compounds in the NoWaC Corpus}

The NoWaC text corpus of the \ili{Norwegian} Bokmål variety \citep{Guevara2010} contains around 700 million words, and its compounds are tagged. This corpus complements the dictionaries. While the latter contain compounds that lexicographers have chosen to include due to frequency and other factors, the compounds that are marked as such in the NoWaC corpus are those that 1) are not recognized as compounds in the dictionaries, thereby triggering the compound recognizer in the tagger module, 2) satisfy certain characteristics, for example that they have a last member that can be recognized as a word, and at least a couple of letters before that.  The search interface allows the user to specify that the result should be a compound, and that it should end in \textit{–t} (for example). However, unlike the dictionaries, the NoWaC corpus has been annotated automatically and the words marked as compounds therefore also include spelling errors (\textit{difust} ‘vague’, should have been spelt with two f’s), foreign words (\textit{treatment}, English loan) and new words (\textit{ukomprimert} ‘uncompressed’), or rightly as compounds, but not parasynthetic ones: \textit{pårygget} ‘on-backed’, \textit{sesongbetinget} ‘season-dependent’. 

While it is possible to find the appropriate examples in the dictionaries given their careful manual annotation, which includes the grammatical category of the first compound member, the corpus is more difficult to use for somebody interested in the parasynthetic subgroup of compounds. The compounds are only marked by the resulting grammatical category, viz. the adjectival one given by the derivational suffix. A better use of the corpus is searching for a longer sequence, such as a full last member of a parasynthetic compound. The corpus contains compounds that have been used in texts independently of the judgement of lexicographers, and therefore present more and potentially interesting data, and complement the dictionaries. As an example, we have searched for the last member \textit{–beint} ‘–legged’, which gave 10 results in the Bokmål dictionary, and 15 in NoWaC, (\ref{ex:bondi:6}–\ref{ex:bondi:7}).\largerpage[3]

\noindent
\begin{minipage}[t]{.5\textwidth}
\ea\label{ex:bondi:6} {From the dictionaries}\\
 \textit{firebeint} ‘four-legged’                  \\             
 \textit{likebeint} ‘ambi-legged’                  \\
 \textit{lettbeint} ‘light-footed’                 \\
 \textit{stivbeint} ‘stiff-legged’                 \\
 \textit{sårbeint} ‘sore-legged’                   \\
 \textit{tobeint} ‘two-legged’                     \\
 \textit{kalvbeint} ‘calf-legged’ (knock-kneed)    \\
 \textit{langbeint} ‘long-legged’                  \\
 \textit{trebeint} ‘three-legged’                  \\
 \textit{kjappbeint} ‘quick-legged’ (swift-footed) \\
\z
\end{minipage}
\begin{minipage}[t]{.5\textwidth}
\ea\label{ex:bondi:7} {From NoWaC} \\ 
\textit{venstrebeint} ‘left-legged’\\
\textit{stivbeint} ‘stiff-legged’\\
\textit{langbeint ‘long-legged’}\\
\textit{firbeint} ‘four-legged’\\
\textit{breibeint} ‘wide-legged’\\
\textit{kortbeint} ‘short-legged’\textit{} \\
\textit{høyrebeint} ‘right-legged’\\
\textit{tungbeint} ‘heavy-legged’\\
\textit{lavbeint} ‘low-legged’\\
\textit{tibeint} ‘ten-legged’\\
\textit{lettbeint} ‘light-legged’ (light-footed)\\
\textit{jevnbeint} ‘even-legged’\\
\textit{snublebeint} ‘stumble-legged’ (clumsy-footed)\\
\textit{åttebeint} ‘eight-legged’\\
\textit{hjulbeint} ‘wheel-legged’ (bow-legged)\\
\z
\end{minipage}\medskip

We see that both sources are useful for finding examples of this phenomenon. In order to be able to say something general about this kind of compounds, we need to have a wide selection of examples, which we have now.

\section{Some aspects of parasynthetic compounds}\label{sec:bondi:3}

\subsection{A marginal phenomenon?} 

\citet[200]{MelloniBisetto2010} claim that parasynthetic compounds represent ”a marginal phenomenon in most Germanic and \ili{Romance} languages”, in contrast to the \ili{Slavic} languages. This claim is not further substantiated, so it is not clear what they mean by marginal. However, \citet[77]{Johannessen2001} seems to say the opposite\footnote{“Denne typen er produktiv – nye ord lages stadig” \citep[77]{Johannessen2001}.}, she claims that this compound type is productive, and that new words are made all the time. 

If “marginal” refers to quantity, the total number of compounds, we should find an answer by counting.  There are altogether 3795 cases of Bokmål hits and 1594 of Nynorsk in the dictionaries. Without going into each case individually, we do not know how many are genuine examples (recall the list in \figref{fig:bondi:1}), but if we guess that half of them are, this is still a high number, though how to evaluate what it takes to be a high number is not obvious.

If it refers to the strict morpho-syntactic requirements as to their make-up, one could justify calling them marginal. Unlike other compounds, they must have a \isi{number} or an adjective as their first member, a noun as their second member, and an adjective-deriving suffix as their last member. 

However, within those grammatical constraints, there is quite a bit of variation. Extracting the second member of the parasynthetic compounds in the dictionary, there are quite a few and they come from different \isi{semantic} fields, see \REF{ex:bondi:8}, including the human body, animal bodies, vehicles, weapons, poems, clothes etc.  

\ea%8
    \label{ex:bondi:8} 

\textit{aksla} ‘shouldered’, \textit{aldra} ‘aged’, \textit{arma} ‘armed’, \textit{auga} ‘eyed’, \textit{barma} ‘breasted’, \textit{barka} ‘barked’, \textit{beina} ‘legged’, \textit{blada} ‘leaved’, \textit{bottna} ‘bottomed’, \textit{bremma} ‘brimmed’, \textit{bringa} ‘chested’, \textit{brysta} ‘breasted’, \textit{buka} ‘stomached’, \textit{cella} ‘celled’, \textit{egga} ‘edged’, \textit{erma} ‘sleaved’, \textit{farga} ‘coloured’, \textit{fingra} ‘fingered’, \textit{fibra} ‘fibred’, \textit{felta} ‘filed’, \textit{finna} ‘finned’, \textit{folka} ‘peopled’,  \textit{forma} ‘shaped’, \textit{fota} ‘footed’, \textit{greina} ‘branched’, \textit{halsa} ‘throated’, \textit{hjarta} ‘hearted’, \textit{hjula} ‘wheeled’, \textit{horna} ‘horned’, \textit{huda} ‘skinned’, \textit{hæla}  ‘healed’, \textit{høgda} ‘heighted’, \textit{håra} ‘haired’, \textit{kalibra} ‘calibred’, \textit{kanta} ‘edged’, \textit{kinna} ‘ cheeked’ , \textit{kjaka} ‘jawed’, \textit{kjefta} ‘mouthed’, \textit{kjønna} ‘gendered’, \textit{knea} ‘kneed’, \textit{korna} ‘grained’, \textit{lemma} ‘limbed’, \textit{leppa} ‘lipped’,  \textit{lesta} ‘lasted’, \textit{leta} ‘coloured’, \textit{liva} ‘lived’, \textit{linja} ‘lined’, \textit{lugga} ‘haired’, \textit{løpa} ‘barrelled’, \textit{maga} ‘stomached’, \textit{masta} ‘masted’, \textit{munna} ‘mouthed’, \textit{mønstra} ‘patterned’, \textit{nakka} ‘necked’, \textit{nasa} ‘nosed’, \textit{nebba} ‘beaked’, \textit{nerva} ‘nerved’, \textit{pigga} ‘spiked’, \textit{panna} ‘foreheaded’, \textit{rada} ‘rowed’, \textit{rauva} ‘bottomed’, \textit{rumpa} ‘bottomed’,  \textit{rygga} ‘backed’,  \textit{røsta} ‘voiced’, \textit{seila} ‘sailed’, \textit{sida} ‘sided’, \textit{sifra} ‘numbered’, \textit{sinna} ‘minded’, \textit{skafta} ‘shafted’, \textit{skala} ‘shelled’, \textit{skinna} ‘skinned’, \textit{skjefta} ‘shafted’, \textit{skjegga} ‘bearded’, \textit{snuta} ‘snouted’, \textit{spalta} ‘slitted’, \textit{spora} ‘spored’, \textit{stamma} ‘stemmed’, \textit{streak} ‘lined’, \textit{strenga} ‘stringed’, \textit{strofa} ‘versed’, \textit{sylindra} ‘cylindered’, \textit{tagga} ‘spiked’, \textit{tanna} ‘toothed’, \textit{vegga} ‘walled’,  \textit{venga} ‘winged’, \textit{vinkla} ‘angled’, \textit{vomma} ‘stomached’, \textit{ætta} ‘familied’, \textit{øra} ‘eared’, \textit{mælt} ‘voiced’… 
\z

There is also a \isi{semantic} requirement for parasynthetic compounds, as the relationship between the parasynthetic compound and what it modifies, must be inalienable (see \sectref{sec:bondi:3.4}).  Within the constraints given in this section, parasynthetic compounding is productive (see \sectref{sec:bondi:3.4} for this, too). It seems fair to conclude that parasynthetic compounds are both marginal and not marginal, depending on the definition of this word. 

\subsection{Parasynthetic compounds and (non-)compositionality}

\citet[209]{MelloniBisetto2010} refer to \citegen{Ackema2004} theory to argue that some types of parasynthetic compounds are non-compositional. It is quite obvious, though, that whenever we can find productively made compounds, they must have compositional semantics, at least to start out with. The self-made parasynthetic compounds in \REF{ex:bondi:9} all have a completely transparent meaning.

\ea%9
    \label{ex:bondi:9}
\ea
 \textit{spisshanket} ‘pointed-handled’,  \textit{rundhanket} ‘round-handled’,  \textit{ovalhanket} ‘oval-handled’ (about jugs)

\ex  \textit{femlommet} ‘five-pocketed’, \textit{sjulommet} ‘seven-pocketed’, \textit{firkantlommet} \\
 ‘square-pocketed’ (about coats)

\ex  \textit{tohyllet} ‘two-shelved’, \textit{smalhyllet} \\
 ‘narrow-shelved’ (about book-cases)
\z
\z

\newpage 
However, just as the \ili{Slavic} [A+N]\textsc{\textsubscript{n}} compounds \citet[209]{MelloniBisetto2010} discuss, there is a group of parasynthetic compounds that could perhaps be argued to be non-compositional, some examples are given in \REF{ex:bondi:10}.

\ea \label{ex:bondi:10}

\textit{mørkhudet}: lit. ‘dark-skinned', ‘person who originates from Africa or Asia’

\textit{hardhudet}: lit. ‘hard-skinned', ‘person who endures criticism’

\textit{tykkhudet}: lit. ‘thick-skinned', meaning: as above

\textit{gullkantet}:  lit. ‘gold-edged', ‘will give somebody wealth’
\z

However, rather than claiming non-compositionality for these, a better classification is probably as compounds with a metaphorical meaning. They are after all compositional when taking the metaphorical \isi{aspect} into account: A thick-skinned person has such a thick metaphorical skin that the criticisms cannot get through and influence her.

It wouldn’t be surprising, though, if some parasynthetic compounds were non-compositional. All compounds, not just the parasynthetic ones, can be lexicalized and then freeze in a meaning that has appeared at some stage. Many compounds contain words that are no longer in use apart from inside those compounds, and others are impossible to analyse semantically in spite of the known individual members. Some examples are given in \REF{ex:bondi:11}.

\ea%11
    \label{ex:bondi:11} 

\textit{putevar} ‘pillow-case’ (the word \textit{var} is not known any longer)

\textit{tyttebær} ‘x-berry’ (the word \textit{tytte} is unknown today)

\textit{tøffelhelt} lit. ‘slipper-hero', ‘man who has no power in his own home’
\z

The conclusion here is that parasynthetic compounds are compositional when they are productively made and when they are used metaphorically, but it would be surprising if not a few, at least, were also non-compositional.  

\subsection{The phonological form of the parasynthetic compound suffix}

The derivational suffix that changes the noun of the parasynthetic compound into an adjective has the same form as that of the past participle. Their shape depends on the phonological form of the stem they attach to.  When the stem ends in a vowel, the suffix is obligatorily –\textit{d}. When it ends in a lamino-dental stop or labial consonant, the suffix must be either –\textit{et} or \textit{–a} depending on dialect, and finally, after other consonants, \textit{–t} . These are all exemplified in \REF{ex:bondi:12}.

\newpage 
\ea%12
    \label{ex:bondi:12}
   
\ea After a vowel stem: \textit{–d}

Verb stem: \textit{bøy} ‘bend’, participle: \textit{bøyd} ‘bent’

Noun stem: \textit{øy} ‘eye’, parasynthetic compound: \textit{rødøyd} ‘red-eyed’
\ea After a lamino-dental or labial plosive stem: \textit{–et} (some dialects)

Verb stem: \textit{stopp} ‘stop’, participle: \textit{stoppet} ‘stopped’

Verb stem: \textit{varm} ‘warm’, participle: \textit{varmet} ‘warmed’

Noun stem: \textit{hud} ‘skin’, parasynthetic compound: \textit{mørkhudet} `dark-skinned’

Noun stem: \textit{arm} ‘arm’, parasynthetic compound:  \textit{toarmet} ‘two-armed’

\ex After a lamino-dental or labial plosive stem: \textit{–a} (other dialects)

Verb stem: \textit{stopp} ‘stop’, participle: \textit{stoppa} ‘stopped’

Verb stem: \textit{varm} ‘warm’, participle: \textit{varma} ‘warmed’

Noun stem: \textit{hud} ‘skin’, parasynthetic compound: \textit{mørkhuda} ’dark-skinned’

Noun stem: \textit{arm} ‘arm’, parasynthetic compound:  \textit{toarma} ‘two-armed’
\z

\ex \textbf{After other consonant stems:} \textit{\textbf{–t}} 

Verb stem: \textit{spis} ‘eat’, participle: \textit{spist} ‘eaten'

Noun stem: \textit{bein} ‘leg', parasynthetic compound:  \textit{tobeint} ‘two-legged’
\z
\z

The parasynthetic compound suffix clearly does not make the noun into a past participle; there is nothing agentive or verbal about these words. However, both classes of words end up with a word that is or (in the case of participles) can be turned into a different part of speech, and in both cases this is an adjective. Some researchers have tried to find a deeper \isi{semantic} connection between the two. \citet{KoontzGarboden2012} suggests that the meaning of the English –\textit{ed} has the meaning of ‘difference’. For nominals that would entail a possessive relation. 

Maybe related to this is the question why it is impossible to use the noun + derivational suffix without a preposed adjective or \isi{number}. Thus, why is it okay to say about somebody that they are \textit{langbeint} ‘long-legged’, while it is impossible to say that they are *\textit{beint} ‘legged’? \citet[218--219]{Booij2005} claims that such constructions are grammatical, but that they are pragmatically odd, since humans are expected to have the property of legs. There are some problems with such pragmatic constraints, though. One problem is that other pragmatic redundancies are perfectly grammatical, such as \textit{tobeint} ‘two-legged’. Another problem is that we find inalienable possession also in cases where the property is not something to be expected. So we find \textit{tremasta} ‘three-masted’, even if boats are not all expected to have masts. In fact, for small boats it would be more unexpected to find masts at all, yet it would be strange or impossible to say about a small boat with masts that it is *\textit{masta} ‘masted’.  

\subsection{Inalienable possession} \label{sec:bondi:3.4}

It is known that parasynthetic compounds must be part of a relationship of inalienable possession with the noun that they modify, as is also pointed out by \citet{Grov2009}. \citet[210]{MelloniBisetto2010} further claim that the nouns of the compound must not only be inalienably possessed, but must be body-parts of humans or animals.\footnote{It is unclear whether they apply this generalisation to all parasynthetic compounds or to Russian or \ili{Slavic} ones only.} Looking at examples of parasynthetic compounds, it is obviously true that they must involve a relationship of inalienable possession between the compound and the owner. For \ili{Norwegian}, however, it is very clear that any noun from any \isi{semantic} field can occur as long as the special relationship is fulfilled. Some examples of words that use the second member in parasynthetic compounds from \REF{ex:bondi:8} are given in \REF{ex:bondi:13}, together with the kind of \isi{possessor} they would have:

\ea%13
    \label{ex:bondi:13}


Clothes\textit{: blåfarga} ‘blue\textit{{}-}coloured’, \textit{mangefibra} ‘many-fibred’

Containers\textit{: dobbelbottna} ‘double-bottomed’\textit{} 

Hats\textit{: vidbremma} ‘wide-rimmed’

Humans: \textit{brei}\textit{aksla} ‘broad-shouldered’, \textit{berrarma} ‘bare\textit{{}-}armed’, \textit{gråøyd} ‘grey-eyed’ \textit{, breibarma} ‘broad\textit{{}-}breasted’, \textit{kjappbeint} ‘quick\textit{{}-}legged’, \textit{breibringa} ‘broad-chested’, \textit{trongbrysta} ‘narrow-breasted’\textit{} 

Knives\textit{: tviblada} ‘two-bladed’, \textit{kvassegga} ‘sharp-edged’

Numbers: \textit{fleirsifra} ‘several-digited’

Poems: \textit{einstrofa} ‘one-versed’\textit{} 

Trees\textit{: råbarka} ‘raw-barked’\textit{} 
\z

There does seem to be full productivity.  I found some examples in NoWaC that seemed rare, and  googled them, \REF{ex:bondi:14}. There were from one to three hits for these, indicating that they have been productively made. I include some self-made ones as well, \REF{ex:bondi:15}, to illustrate that this is possible and the result grammatical. 

\ea%14
    \label{ex:bondi:14}
   
\textit{trangkjefta} ‘narrow-mouthed’

\textit{skakkjefta} ‘skew-mouthed’ 

\textit{rødkjefta} ‘red-mouthed’
\z

\largerpage
\ea%15
    \label{ex:bondi:15}
   
\textit{kortnegla} ‘short-nailed’

\textit{grønnesa} ‘green-nosed’

\textit{smalpanna} ‘narrow-foreheaded’
\z

The examples all show that parasynthetic compounds require inalienable possession, but the kind of \isi{possessor} can belong to any \isi{semantic} field, not just human or animate. Why they have to obey the inalienability condition remains a puzzle.\footnote{One reviewer, referring to \citet{Myler2014} asks about phrases such as \textit{ragged-trousered philanthropists}, \textit{top-hatted gentleman}, which seem to run contra to the requirement of inalienability for this construction. I don’t know whether these are productive in English, but their equivalents do not seem right in \ili{Norwegian}. One could explain them, perhaps, by claiming that the top hat is an inalienable possession of a gentleman, et cetera, and that \textit{top} is analysed as an adjective.} 

\subsection{Parasynthetic compounds and syntactic theory}

The fact that parasynthetic compounds have very strict categorial requirements makes them very interesting. Consider an example like \REF{ex:bondi:16a},  \textit{seksbeinte} ‘six-legged.\textsc{pl}’. It contains the noun \textit{bein} ‘leg’ modified by the \isi{number} \textit{seks} ‘six’ and the adjectival derivational suffix \textit{–t}. The compound is inflected in the plural. Other parts of speech are not possible (apart from the first member, that could also be an adjective), see (\ref{ex:bondi:16}b–d).  

\ea%16
    \label{ex:bondi:16}
\ea[]{\label{ex:bondi:16a} \textit{seksbeinte} ‘six-legged\textsc{.pl’} (first member: adjective/\isi{number})}

\ex[*]{ \label{ex:bondi:16b} \textit{plastikkbeint} ‘plastic-legged’ (noun instead of adjective/\isi{number})}
\ex[*]{ \label{ex:bondi:16c} \textit{haltebeint} ‘limp-legged’ (verb instead of adjective/\isi{number})     }
\ex[*]{ \label{ex:bondi:16d} \textit{dårligbeint} ‘badly-legged’ (adverb instead of adjective/\isi{number}) }
\z
\z

The second member could be substituted with a verb, in which case all the characteristics of the parasynthetic compounds disappear, consider \REF{ex:bondi:17a} vs. (\ref{ex:bondi:17}b–d).

\ea%17
    \label{ex:bondi:17}
   
\ea \label{ex:bondi:17a} \textit{blåøyd} ‘blue-eyed’ (second member: adjective, followed by derivational suffix) 

\ex \label{ex:bondi:17b}
 
\textit{blåmalt krus} ‘blue-painted cup’ (second member: past participle instead of adj and \textit{–t})

\ex \label{ex:bondi:17c}
 
\textit{børstemalt} ‘brush-painted’ (first member: noun, not adjective)

\ex \label{ex:bondi:17d}
 
\textit{hurtigmalt} ‘quickly-painted’ (first member: adverb, not adjective)
\z
\z

(\ref{ex:bondi:17}b–d) cannot be considered to be parasynthetic compounding, just ordinary synthetic compounding.  First, there is no inalienable possession. In \REF{ex:bondi:17b}, \textit{blåmalt krus} ‘blue-painted mug’, the \isi{possessor} would be \textit{krus} ‘mug’, but there is no noun to be possessed. Second, it has only two members, \textit{blå-malt}, i.e. adjective+past participle, as \textit{malt} ‘painted’ is also a possible word of its own. Third, this entails that there is no bracketing paradox either.  Fourth, it does not have any other restrictions w.r.t. part of speech of the first member, so \textit{børstemalt} ‘brush-painted’ with a noun and \textit{hurtigmalt} ‘quickly-painted’ with an adverb are both ok. 

\citet[79]{Johannessen2001} suggested the analysis in \REF{ex:bondi:18}, in which the adjectival derivational suffix \textit{–t} is attached to the compound stem \isi{number}/adjective+noun. The idea is that this compound stem has a compound feature with information about the individual members, which is percolated up to the combined compound stem. The derivational suffix selects this kind of stem, giving a parasynthetic compound. 

\ea%18
    \label{ex:bondi:18}
\begin{forest} for tree={align=center,base=top}
 [Adjektiv
 [Stamme\\adjektiv
 [Sammensetnings-\\stamme\\substantiv
  [Stamme\\tallord[Rot[seks]]]
  [Stamme\\substantiv[Rot[bein]]]
 ] [Avledningsformativ [t]]
 ]
 [Bøyingsformativ[e]]
 ]
\end{forest}

\z 

\noindent A similar analysis is suggested by \citet[216]{MelloniBisetto2010}, building on \citet{Ackema2004}, for words like \textit{bisillabo} `bisyllabic’, see \REF{ex:bondi:19}. 

\ea%19
    \label{ex:bondi:19}
 \begin{forest} for tree={align=center,base=top}
  [A
  [N
    [\isi{Num}\textsubscript{(bound)}\\\textit{bi-}\\\textit{mono-}]
    [N\\\textit{sillaba}\\\textit{posto}]
  ]
  [A
    [Suf\\\textit{ico\slash Ø}\\\textit{Ø}]
  ]
  ]
 \end{forest}
%%please move the includegraphics inside the {figure} environment
%%\includegraphics[width=\textwidth]{a20Johannessen-img8.png}
 \z
 

\noindent A syntactic theory that has received some interest in recent years is the exo-skeletal theory proposed by \citet{Borer2003} and implemented for \ili{Norwegian} in work such as \citet{Åfarli2007} and \citet{GrimstadEtAl2014}. In this theory syntactic categories are properties of the structure, not of the items themselves. \citet[34–40]{Borer2003} illustrates the theory by taking roots such as \textit{dog}, \textit{sink} and \textit{boat}, and inserting them freely in the syntactic structure yielding sentences such as \textit{The boat will dog three sinks}, as well as \textit{The boat will sink three dogs} etc. 

If the theory is applied to parasynthetic compounds, the skeleton might look like \REF{ex:bondi:18}, but with empty terminals, waiting to be filled. We have already seen in \REF{ex:bondi:16} that there are very strict categorial restrictions on parasynthetic compounds. 

Further, if we substitute the second member, the lexical item \textit{øye} `eye’ (usually used in a noun structure) of a parasynthetic compound such as \textit{blåøyd} `blue-eyed’ with a lexical item often used as a verb \textit{male} `paint’, like we have done in (\ref{ex:bondi:17}a–b), the result is not a parasynthetic compound with an item previously used as a verb now interpreted as a (new) noun. It seems impossible to force a parasynthetic compound reading onto \textit{blåmalt} `blue-painted’, such that for example \textit{blåmalt krus} `blue-painted mug’ would be a mug possessing paint that is blue. This would also have made the prediction that the bracketing paradox would be observed, so that the second item with the suffix should be unacceptable. Again, forcing an unacceptable interpretation onto \textit{malt} `painted’ is beyond what a language user can do. Johannessen (under development) currently investigates a different way of looking at the data; one in which there are \isi{semantic} parallels between parasynthetic compounds and past participles.  

\section{Conclusion}\label{sec:bondi:4}

The paper has investigated parasynthetic compounds using large empirical resources: a searchable dictionary database  especially marked for compounds and a big web-corpus. These turned out to be very useful to garner large amounts of relevant data quickly. It was also discussed whether parasynthetic compounds are a marginal phenomenon, as claimed in the literature. This can hardly be the case since, though there are some syntactico-\isi{semantic} restrictions on their formation, they are productive. Since many are productively made, they clearly cannot be non-compositional, as has also been claimed. One of the clear \isi{semantic} restrictions is that there must be a relationship of inalienable possession, but it is not true that it must only be restricted to body parts of humans and animals, as has been claimed. Finally, with the very strict categorial restrictions on the formation of parasynthetic compounds, syntactic theories that dismiss the idea that lexical items have categorial features have been shown to face a challenge.  


 

 


 
\section*{Acknowledgements}
I would like to thank Bjørghild Kjelsvik and Oddrun Grønvik for inviting me to their workshop on compounds at MONS (Møter om norsk språk) 16, 25th–27th November 2015, University of Agder, Kristiansand. This made me start thinking about the exciting topic of compounding again.  

I would like to thank two anonymous reviewers for very good comments and Kolbe, Steve Pepper and George Walkden for thorough proof reading.

\section*{Web resources}

Compound analysis search interface, {Bokmålsordboka}: 
\url{http://www.edd.uio.no/perl/search/search.cgi?appid=72&tabid=3174}

\noindent
Compound analysis search interface, {Nynorskordboka}: 
\url{http://www.edd.uio.no/perl/search/search.cgi?appid=73&tabid=2562}


\noindent
NoWaC corpus  (\ili{Norwegian} Web as a Corpus): 
\url{http://hf-tekstlab.uio.no/glossa2/?corpus=nowac_1_1}

 


\printbibliography[heading=subbibliography,notkeyword=this]
\end{document}