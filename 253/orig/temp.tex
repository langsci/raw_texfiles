

Greece’s labyrinth of language

A study in the early modern

discovery of dialect diversity

Raf Van Rooy

\textit{To} \textit{my} \textit{parents,} \textit{whose} \textit{love} \textit{speaks} \textit{its} \textit{own} \textit{dialect}

\section{\textstyleKopiChar{Table of contents}}
\hypertarget{Toc19704798}{}\setcounter{tocdepth}{3}
\renewcommand\contentsname{}
\tableofcontents

\section{Acknowledgments}
\hypertarget{Toc19704799}{}
The present book is a thoroughly revised version of a substantial part of my PhD dissertation, defended in \citealt{May2017} (Van \citealt{Rooy2017}). As such, it is an outcome of a four-year PhD fellowship, generously granted by the Research Foundation – Flanders (FWO) and conducted at KU Leuven, my main alma mater. The research leading up to this monograph was carried out in the best possible circumstances at the Leuven Center for the Historiography of Linguistics (CHL), where I could enjoy the invaluable guidance of two of the finest mentors, Toon Van Hal and Pierre Swiggers. This book has greatly benefitted from their feedback as well as from that of the two other members of my supervisory committee, John Considine and Lambert Isebaert. Special thanks are due to the latter, as it was under his supervision that I first embarked on studying the fascinating history of Greek dialectology during my studies at UCLouvain in 2012/2013. Many chapters of the book were written and finalized during a long research stay in the ancient capital of eloquence and elegance, Athens, in the Winter semester of 2018/2019, made possible by an FWO travel grant. I am also much indebted to the editorial care of James McElvenny, who greatly improved the English phrasing at numerous instances, and to Nina Markl for her typesetting efforts.

Leuven,

August 23, 2019

\section{Editorial choices}
\hypertarget{Toc19704800}{}
In order to facilitate reading, I have opted to offer only English translations of quotes and titles in the main text. The original text can be found either in the footnotes in the case of quotes or in the bibliography in the case of titles. Unless otherwise indicated, translations are mine. I have transcribed Greek keywords quoted in the main text into the Latin alphabet (with the original between round brackets), but in order to avoid overloading the footnotes I have refrained from doing the same for Greek citations appearing there. I have regularized Latin orthography, opting for <u> and <i> spellings, but I have preserved the original orthography of early modern vernacular texts, standardizing only <u>/<v> and <i>/<j> alternations in accordance with modern practice. For both Latin and vernacular quotes, I have regularized capitalization and punctuation marks to current practices. Errors in the source texts are marked with “[\textit{sic}]”. Names of Greek, Latin, and early modern authors have been anglicized whenever this is common in secondary literature. Otherwise, I have opted for the most common form. Life dates are provided in the main text when an author is first introduced. Finally, I refer to early modern dissertations by mentioning the name of the chairman – the \textit{praeses} – as well as the student presenting the dissertation – the \textit{respondens} – unless there are sound reasons to suppose that one of both persons should be considered the sole author of the dissertation.\footnote{On the problem of authorship in early modern dissertations, see e.g. \citet{Considine2008b}.}

\section{Preface}
\hypertarget{Toc19704801}{}
In his twenty books on education, the renowned Spanish philologist and humanist pedagogue Juan Luis Vives (1492/1493–1540) warned students of the ancient Greek language of its great difficulty and diversity:

\begin{quote}
In the Greek language, there are great labyrinths and enormously vast recesses, not only in the various dialects, but in every one of them. The Attic dialect and the common one, which is very close to Attic, are especially necessary, because they are also the most eloquent and cultivated. And whatever the Greeks have that is worthy of reading and knowing is recorded in these dialects. The remaining dialects are used by the authors of poems, but it is less important to understand these.\footnote{Vives (1531: e3\textsuperscript{v}): “In Graeca magni sunt labyrinthi et uastissimi recessus, non solum in dialectis uariis, sed in unaquaque illarum. Attica et Atticae proxima communis maxime sunt necessariae, propterea quod et sunt facundissimae atque excultissimae, et quicquid Graeci habent legi ac cognosci dignum istis dialectis est consignatum. Reliquis utuntur auctores carminum, quos non tanti est intelligi”.}
\end{quote}

As a kind of Ariadne, Vives endeavored to guide the reader of his book, his Theseus, through the vast labyrinth of the Greek tongue. In order to make sure that prospective Hellenists learned the language as efficiently as possible, he suggested that they should focus on the Attic dialect and on koine Greek, both for intellectual and esthetic reasons. Dialects such as Doric and Aeolic, primarily poetical media, were deemed to be of lesser importance.

Vives left no doubt as to the immense diversity within the Greek language, which posed an enormous challenge not only to students but also to scholars in the early modern era. Fascinated with the heritage of ancient Greece, early modern intellectuals cultivated a deep interest in its language, the primary gateway to this long-lost culture, rediscovered by Westerners during the Renaissance. The humanist battle cry “Ad fontes!” – Latin for “To the sources!” – forced them to take a detailed look at the Greek source texts in the original language and its different dialects. In so doing, they saw themselves confronted with several major linguistic questions. Is there any order in this great diversity? Can the Greek dialects be classified into larger groups? Is there a hierarchy among the dialects? Which dialect is the oldest? Where should problematic varieties such as Homeric and Biblical Greek be placed? How are the differences between the Greek dialects to be described, charted, and explained? What is the connection between the diversity of the Greek tongue and the Greek homeland? And, last but not least, are Greek dialects similar to the dialects of the vernacular tongues? Why (not)? In the present book, I discuss and analyze the often surprising and sometimes contradictory early modern answers to these questions.

\section{1  Introduction}
\hypertarget{Toc19704802}{}
Early modern scholarship on the Greek dialects has thus far attracted almost no attention at all (cf. Ben-\citealt{Tov2009}: 157–158). It is the subject of only a handful of case studies (e.g. Van \citealt{Rooy2016c}), and a small number of scholars make some cursory comments on the matter (e.g. \citealt{Botley2010}; \citealt{Roelcke2014}: esp. 246–254, 352). The neglect is glaringly apparent from the entry “Classification of dialects” in Brill’s \textit{Encyclopedia} \textit{of} \textit{Ancient} \textit{Greek} \textit{language} \textit{and} \textit{linguistics} \citep{Finkelberg2014}, where a discussion of ancient dialectology is immediately followed by an account of modern classifications of the dialects (see already Van \citealt{Rooy2016a}). It is ancient and medieval ideas on the Greek dialects which have taken center stage in historiographical studies (see Van \citealt{Rooy2018b} for a recent state of the art). As the subject is largely unexplored for the early modern period, it represents an untapped vein of precious new information for the reader interested in language and history. But it also harbors dangers. For instance, the fact that this book deals with a topic that has never been systematically studied before makes it not only impossible but also simply undesirable to attempt to say the last word on the topic. Instead, the reader should regard this book as a first exploration of the subject matter.

In this introduction, I want to achieve two things. To start with, the reader is served a nutshell history of the Greek language, so that \REF{ex:key:s}he understands the central place of dialectal variation in it. In a second step, I briefly outline why early modern intellectuals developed an interest in the Greek dialects and how this manifested itself in their scholarly production.

\subsection{The history of the Greek language in a nutshell}
\hypertarget{Toc19704803}{}
An Indo-European language, Greek was anciently spoken in and around the present-day country of Greece and in Greek colonies scattered across the islands and coasts of the Mediterranean and the Black Sea. Before it was united under Macedonian rule in the fourth century \textsc{bc}, the region constituted a patchwork of polities, lacking a central government and a common language. This allowed for the official and literary use of local varieties of ancient Greek, which most scholars today divide into six main dialect branches: Aeolic, Arcado-Cypriot, Attic–Ionic, Doric, Northwest Greek, and Pamphylian (see e.g. \citealt{Colvin2010}; \citealt{Finkelberg2014}). Yet extant dialect literature and inscriptions reflect the actual spoken varieties only to a limited extent (\citealt{Colvin2010}: 201–202). According to Stephen Colvin (1999: 300, 303), there was mutual intelligibility among most dialects, but Albio Cesare Cassio (2016: 4–5) convincingly casts doubt on this assumption. Over time, certain dialects became tied up with literary genres rather than locality. Aeolic became established as the dialect of lyric poetry, Attic of, among other things, rhetoric and the dialogic parts of tragedy and comedy, Doric of bucolic poetry and the choral odes in tragedy, and Ionic of science and historiography. The literary usage of dialects was, however, not straightforward. Even though most prose authors opted for one dialect – Herodotus (ca. 485–424 \textsc{bc}), for example, used Ionic – poets frequently combined features from different dialects in their work. For this practice, Homer’s epic poems (8th cent. \textsc{bc}) were the main model. Though the \textit{Iliad} and \textit{Odyssey} principally exhibit Ionic properties, they also have Aeolic and Attic features and contain archaisms, traces of earlier phases of the Greek language (\citealt{Hackstein2010}: 401–408; \citealt{Tribulato2010}: 390). As a result, one verse could display features of various dialects. Homer’s artificial language constituted, so to speak, the first literary Greek koine, thus enhancing the Greek feeling of linguistic unity (Morpurgo \citealt{Davies1987}: 15–19; \citealt{Colvin2010}: 200).

In the Hellenistic era, the Greek koine – short for \textit{hē} \textit{koinḕ} \textit{diálektos} ($\text{\textgreek{<h}}$ $\kappa o\iota \nu \text{\textgreek{`h}}$ $\delta \iota \text{\textgreek{'a}}\lambda \varepsilon \kappa \tau o\varsigma $), ‘the common speech’ – developed out of Great Attic. The latter was a written variety of Attic used in administration and influenced by Ionic (\citealt{Horrocks2010}: 73–77, 80–83). The koine normally lacked Attic features that were considered too local or too complex. These included the Attic double tau which corresponded to the less regionally marked double sigma in, for instance, \textit{thálatta} ($\theta \text{\textgreek{'a}}\lambda \alpha \tau \tau \alpha $) versus \textit{thálassa} ($\theta \text{\textgreek{'a}}\lambda \alpha \sigma \sigma \alpha $), ‘sea’. Moreover, some properties of Attic, among which its complicated verbal morphology, were adopted in the koine in a simplified or regularized form (\citealt{Brixhe2010}: 230; \citealt{Horrocks2010}: 75, 82).

The reliance on Attic as the basis for the common language was a consequence of the important political status of Athens in the fifth century \textsc{bc}, the Golden Age of Pericles, and of its immense literary prestige still echoing today; Attic is the variety by which students usually start to learn ancient Greek today. The koine rapidly spread across the Greek-speaking world, vastly enlarged by Alexander the Great’s (356–323 \textsc{bc}) conquests. Over time, local koines came into being, some of which eventually developed into different vernacular Greek dialects (\citealt{Brixhe2010}: 244–249). As a matter of fact, with the exception of Tsakonian, a form of Greek descending from an ancient Laconian Doric variety – though much influenced by the koine – \citep[88]{Horrocks2010}, all modern dialects derive from forms of koine Greek. Indeed, the koine had made the ancient dialects virtually extinct by late antiquity (\citealt{Horrocks2010}: 84, 88). The koine itself was diversified too, not only regionally, but also in terms of social strata and registers, not to mention time. Its conception as a unitary linguistic entity was therefore largely an ideal constructed by grammarians of the period, much like the standard languages of today (\citealt{Brixhe2010}: 230–231; cf. also Van \citealt{Rooy2016b}).

In the Hellenistic and Roman periods, most literary Greek texts were still written in the ancient dialects, with the so-called Atticistic movement flourishing in the second century \textsc{ad}. Authors then consciously took Classical Attic texts as their stylistic, literary, and linguistic models \citep[42]{Whitmarsh2005}. Varieties of the koine were, however, also highly popular, evidenced by the Greek of the New Testament and that of the countless Egyptian papyri (\citealt{EvansObbink2010}). This gradually led to a diaglossic situation, with pure Attic at the highest end of the prestige spectrum and vernacular varieties of the common people at the lowest. Note that I employ the concept of \textsc{diaglossia}, referring to a linguistic context in which there is a spectrum of varieties between the low vernacular dialects, on the one hand, and the high varieties, on the other (see e.g. \citealt{Auer2005}; \citealt{Rutten2016}). This diaglossia continued throughout the Byzantine and early modern era and well into modernity, during which it polarized more radically as a \textit{di}glossia between the low vernacular variety, \textit{Dimotikí} ($\Delta \eta \mu o\tau \iota \kappa \text{\textgreek{'h}}$), and the \textit{Katharévousa} ($K\alpha \theta \alpha \rho \varepsilon \text{\textgreek{'u}}o\upsilon \sigma \alpha $) tongue, reserved for high registers. The \textit{Katharévousa}, ‘the pure tongue’, was a mixed learned language created out of vernacular and ancient Greek, whereas the \textit{Dimotikí} referred to popular varieties of Greek, strongly influenced by centuries of Venetian and especially Ottoman rule.

The diglossia was largely resolved with the replacement of the \textit{Katharévousa} tongue by Standard Modern Greek as the official language of Greece in 1976, two years after the military junta had fallen and the Third Hellenic Republic was installed. This new standard variety had its base in Demotic Greek, but was elaborated by many features of the \textit{Katharévousa}. In the meantime, vernacular Greek dialects of various kinds continue to be spoken all across Greece, whereas the Greek Orthodox Church still makes use of the \textit{Katharévousa}.

In conclusion, dialects have played a major role in the long history of the Greek language, especially in antiquity, when they were eagerly used for epigraphic, administrative, and especially literary purposes.

\subsection{The dialects of ancient Greece in premodern scholarship: A typology of sources}
\hypertarget{Toc19704804}{}
It was for their literary importance that the dialects of the ancient Greek language were primarily studied by scholars of the premodern era. We know that there was a lively tradition of ancient studies on the matter, likely initiated by the first-century \textsc{bc} grammarian Tryphon, active in Alexandria, Egypt. Yet only a distorted picture can be reconstructed of this early history, largely filtered through Byzantine treatises that are not all of the highest quality, to put it mildly. Greek scholarship on the dialects was very much characterized by a hands-on approach. Grammarians devoted their efforts in the first place to describing the features of the canonical literary dialects Attic, Ionic, Doric, Aeolic, and to a lesser extent the koine. Out of these data Tryphon and his successors distilled a framework of word modifications perceivable across different varieties of Greek. One might expect this to have given rise to a comparative approach toward the dialects, but nothing could be farther from the truth. Extant source texts show that the Greeks did not do much more than sum up the features of individual dialects. The main focus was on how they differed from common Greek. This was not any prehistoric Proto-Greek language, but could mean only two things: either what (most) Greek dialects had in common or the Greek koine. The former view was typical of ancient grammarians, whereas the latter conception seems to have prevailed principally among Byzantine scholars. It is, however, not always an easy task to distinguish between both conceptions.

Treatises on the dialects were indispensable instruments for students of the linguistically diverse literature of ancient Greece, and their appearance coincided more or less with the near-extinction of the Greek dialects of antiquity. Yet when fourteenth- and fifteenth-century Italian humanists started to direct their attention to Greek language and literature, knowledge of which had largely vanished in medieval Western Europe, these instruments were inaccessible for decades. They remained in manuscripts within the confines of the crumbling Byzantine empire. Even when these manuscripts gradually made their way to Italy, they did not make popular reading material. They were too difficult for Italian students, who did not have a variety of Greek as their mother tongue as Byzantine students did. Instead, the Italians relied on the teachings of Byzantine teachers who traveled to the West from the end of the fourteenth century onward (see e.g. \citealt{Harris1995}; \citealt{Botley2010}; \citealt{Wilson2016}). These teachers soon realized that the Byzantine language manuals were too complex for their new audience. They met their students halfway and produced simplified grammars of Greek, tailored to the needs of their Italian audience; these described a more or less unitary form of Greek, in fact a mixture of koine, Attic, and Ionic elements \citep[123]{Ciccolella2008}. Strange dialectal features were kept to a minimum in these introductory handbooks, as can be gathered from the concise overview of early Renaissance grammars by Paul \citet{Botley2010}. Yet they were not entirely absent. For example, the first Byzantine scholar to successfully teach Greek in Italy, Manuel Chrysoloras (ca. 1355–1406), explicitly noted in his grammar that in the Attic dialect the nominative and the vocative cases are formally identical. This misinformation he took over from Greek tradition, and in particular from a popular Byzantine treatise on the dialects by Gregory of Corinth, which I discuss at greater length below.\footnote{See Botley (2010: 166 n.70). \citet[20]{Chrysoloras1512}: “$\kappa \alpha \theta \text{\textgreek{'o}}\lambda o\upsilon $ $\mu \text{\textgreek{`e}}\nu $ $o\text{\textgreek{<i}}$ $\text{\textgreek{>A}}\tau \tau \iota \kappa o\text{\textgreek{`i}}$ $\tau \text{\textgreek{`a}}\varsigma $ $\alpha \text{\textgreek{>u}}\tau \text{\textgreek{`a}}\varsigma $ $\text{\textgreek{>'e}}\chi o\upsilon \sigma \iota \nu $ $\text{\textgreek{>o}}\rho \theta \text{\textgreek{`a}}\varsigma $ $\kappa \alpha \text{\textgreek{`i}}$ $\kappa \lambda \eta \tau \iota \kappa \text{\textgreek{'a}}\varsigma $”. I quote from a Renaissance edition, as the grammar has not yet been critically edited to modern standards (see \citealt{Nuti2013}: 241 n.8).} While it is certainly true that Byzantine scholars simplified Greek grammar and reduced dialect information in their handbooks for the benefit of their Italian audience, it seems that this was not motivated by didactic concerns alone. In fact, these Greek teachers were unlikely to have been experts in the matter of the dialects themselves. It is revealing in this regard that the Italian humanist Francesco Filelfo (1398–1481) lamented in 1441 that even in Constantinople no Aeolic was taught (\citealt{Rotolo1973}–1974: 88 n.4; cf. \citealt{Botley2010}: 71–114).

In the second half of the fifteenth century, a change was underway, and it is remarkable that Western students of Greek rather than their Byzantine teachers played a major role in it. The topic of the dialects, marginally present at best in the early Renaissance, drew more and more attention from the 1460s onward. Three events of the final decades of the fifteenth century mark this change. \citealt{In1460}, the Constantinopolitan grammarian Constantine Lascaris (1434–1501) active in Italy published a brief work in which he treated the Greek pronoun from a new angle. Writing for advanced students interested in poetry, he described the way in which the pronominal system varied across different dialects, a matter that must have given many students a headache. The work first circulated in manuscript and was printed only after some forty years (see \citealt{Botley2010}: 26, 124, 175 n.272). Among the Byzantine teachers active in the West, Lascaris was exceptional in providing his students with a treatise related to the thorny issue of the dialects.

It was, however, Western humanists who turned dialectology – though the term was not coined before 1650 – into a subfield of Greek philology. The first step was, however, taken not in Italy, but in Paris, which experienced an extended first flourishing of Greek studies mainly thanks to the émigré George Hermonymus of Sparta (ca. 1430–ca. 1509). In the winter of 1477/1478, Hermonymus’ promising student from Pforzheim, Johann Reuchlin (1455–1522), compiled a \textit{Booklet} \textit{on} \textit{the} \textit{four} \textit{differences} \textit{of} \textit{the} \textit{Greek} \textit{language} (ed. Van \citealt{Rooy2014}). Although presenting it as an original work of his own which he based on Byzantine sources, Reuchlin did in fact not much more than translate a Greek manuscript treatise of questionable quality into Latin. Reuchlin’s booklet did not circulate widely and survives in two manuscripts only, and with good reason, since it is hard to see how a student of Greek would have benefitted from it.

An event of much greater momentum occurred in \citealt{August1496}, when Aldus Manutius (ca. 1449/1451–1515) issued in his Venetian office a large collection of Greek grammatical treatises, deservedly called a “treasure” (\textit{thesaurus}). Toward the end of this collection, the reader could find three treatises on the Greek dialects. The first consisted, in fact, of two abbreviated redactions of a work entitled \textit{Tekhniká} ($T\varepsilon \chi \nu \iota \kappa \text{\textgreek{'a}}$), which can be translated as \textit{Matters} \textit{relating} \textit{to} \textit{the} \textit{art} \textit{of} \textit{grammar}. It is usually attributed to John the Grammarian, likely to be identified with the early Byzantine philosopher John Philoponus (ca. 490–575), and still awaits a critical edition.\footnote{This is also why I quote this text from the first edition of 1496 in the present book.} The second treatise was an excerpt from an anonymous biography of Homer usually ascribed to Plutarch; it was less extended in scope than John the Grammarian’s work, since it focused on the dialects as they are used by Homer.\footnote{See the edition in \citet{Kindstrand1990}. A Greek–English edition is available in \citet{KeaneyLamberton1996}. See also Van \citet{Rooy2018c} on the Renaissance fate of the treatise.} The third treatise, entitled \textit{On} \textit{the} \textit{dialects}, was the longest; its author, the Byzantine grammarian and theologian Gregory of Corinth (11th/12th century), drew on the work of Tryphon and John Philoponus as well as on his personal reading of the classics, as he explained in his proem.\footnote{For a modern critical edition, see the unfortunately unpublished PhD dissertation of Didier \citet{Xhardez1991}.} Manutius cannot be credited with being the first to have published these works – the second and third had been separately printed some years earlier – but he was the first to print them together. In fact, these three texts became something of a dialectological canon in the early modern period (see especially the appendix to \citealt{Trovato1984}). Manutius himself further contributed to their canonization by republishing them in 1512 with an accompanying Latin translation. In no time, these texts enjoyed numerous reissues, frequently in Latin translation and often appended to other helpful instruments for students of Greek such as dictionaries.

The success of these treatises indicates that there was a market for handbooks discussing the Greek dialects in the early sixteenth century. Indeed, not only did the grammars of Western humanists, including the Protestant leader Philipp Melanchthon (1497–1560), increasingly treat the subject matter, but the sixteenth century also witnessed the appearance of the first original monographs on this topic. The earliest of its kind was a popular booklet entitled \textit{On} \textit{the} \textit{diverse} \textit{dialects} \textit{of} \textit{the} \textit{Greek} \textit{inflections} \textit{in} \textit{verbs} \textit{as} \textit{well} \textit{as} \textit{in} \textit{nouns,} \textit{drawn} \textit{from} \textit{Corinth,} \textit{John} \textit{the} \textit{Grammarian,} \textit{Plutarch,} \textit{John} \textit{Philoponus,} \textit{and} \textit{others} \textit{of} \textit{the} \textit{same} \textit{order} \citep{Amerot1530}. Originally part of a 1520 Greek grammar printed in Leuven, the work was published independently for the first time in 1530 with Gérard Morrhe in Paris. Its author was Adrien Amerot (ca. 1495–1560), a young Hellenist from Soissons and professor of Greek in Leuven for the greater part of his life (see \citealt{Hummel1999}; Van \citealt{Rooy2019}). Soon other Hellenists followed suit, and a tradition of Latin handbooks on the Greek dialects quickly emerged in Europe. These include, but are not limited to:

\begin{itemize}
\item \begin{styleListParagraph}
Martin Ruland’s (1532–1602) voluminous \textit{Five} \textit{books} \textit{on} \textit{the} \textit{Greek} \textit{language} \textit{and} \textit{all} \textit{of} \textit{its} \textit{dialects} (Zurich, 1556);
\end{styleListParagraph}
\item \begin{styleListParagraph}
the \textit{Booklet} \textit{on} \textit{the} \textit{Greek} \textit{dialects} (Paris, 1569) by the further unknown Frenchman Robertus Vuidius;
\end{styleListParagraph}
\item \begin{styleListParagraph}
the French Jesuit Guillaume Baile’s (1557–1620) \textit{Booklet} \textit{on} \textit{the} \textit{Greek} \textit{dialects} (Paris, 1588);
\end{styleListParagraph}
\item \begin{styleListParagraph}
the Marburg professor Otto Walper’s (1543–1624) successful \textit{On} \textit{the} \textit{principal} \textit{dialects} \textit{of} \textit{the} \textit{Greek} \textit{language} (Frankfurt am Main, 1589);
\end{styleListParagraph}
\item \begin{styleListParagraph}
Jakob Zwinger’s (1569–1610) \textit{Outline} \textit{of} \textit{the} \textit{Greek} \textit{dialects} (Basel, 1605);
\end{styleListParagraph}
\item \begin{styleListParagraph}
the \textit{Easy} \textit{and} \textit{compendious} \textit{treatise} \textit{of} \textit{the} \textit{dialects} \textit{of} \textit{the} \textit{Greek} \textit{language} (Paris, 1621) of the somewhat enigmatic figure Pierre Bertrand Mérigon;
\end{styleListParagraph}
\item \begin{styleListParagraph}
Caspar Wyss’ (1604/1605–1659) \textit{Sacred} \textit{dialectology} (Zurich, 1650);
\end{styleListParagraph}
\item \begin{styleListParagraph}
the \textit{Greek} \textit{prosody,} \textit{with} \textit{dialectology} (Tübingen, 1684) by the obscure Hellenist Johannes Bregius;
\end{styleListParagraph}
\item \begin{styleListParagraph}
Michael Maittaire’s (1668–1747) influential \textit{Dialects} \textit{of} \textit{the} \textit{Greek} \textit{language} (London, 1706);
\end{styleListParagraph}
\item \begin{styleListParagraph}
the \textit{Dialectology} \textit{in} \textit{paradigms} (Neubrandenburg, 1725) by the poorly known scholar Johann Barthold Nibbe;
\end{styleListParagraph}
\item \begin{styleListParagraph}
Johann Friedrich Facius’ (1750–1825) \textit{Compendium} \textit{of} \textit{Greek} \textit{dialects} (Nuremberg, 1782).
\end{styleListParagraph}
\end{itemize}

It is such handbooks that constitute the core corpus of the present study, alongside Greek grammars which devote substantial chapters to the matter of the dialects. An early example of the latter category is the first Greek grammar by a Spanish Hellenist: Francisco de Vergara’s († 1545) \textit{Five} \textit{books} \textit{on} \textit{the} \textit{grammar} \textit{of} \textit{the} \textit{Greek} \textit{language}, originally published in 1537 in Alcalá de Henares, with several reprints in Paris (1545, 1550, 1554, 1557) and Cologne (1552, 1588). The fifth book of the work, covering 27 pages, is entirely devoted to the dialects (\citealt{Vergara1537}: 209–235). Even though most of these manuals were principally aimed at familiarizing the prospective philologist with the different literary forms of Greek, they also contain many revealing observations that made a lasting impact on Western linguistic thought, well before the study of language was institutionalized in the nineteenth century.

Apart from handbooks, there are also a number of scholarly, mainly philological works elaborating at length on the Greek dialects, including Claude de Saumaise’s (1588–1653) voluminous \textit{Commentary} \textit{on} \textit{the} \textit{Hellenistic} \textit{tongue,} \textit{deciding} \textit{the} \textit{controversy} \textit{on} \textit{the} \textit{Hellenistic} \textit{tongue} \textit{and} \textit{thoroughly} \textit{treating} \textit{the} \textit{origin} \textit{and} \textit{dialects} \textit{of} \textit{the} \textit{Greek} \textit{language} (Leiden, 1643), a result of his dispute with his rival Daniel Heinsius (1580–1655), and Friedrich Gedike’s (1754–1803) German essay \textit{On} \textit{the} \textit{dialects,} \textit{especially} \textit{the} \textit{Greek} (Berlin, 1782). Other texts of various genres in which the Greek dialects occupy a prominent or revealing place are likewise involved in the analysis; these include, most importantly, grammars and lexicons of other languages (e.g. \citealt{Gill1619}), philological studies of non-Greek texts (e.g. \citealt{Schultens1748}), and antiquarian works on Greece or related areas (e.g. \citealt{Castelli1769}).

\subsection{Content in context}
\hypertarget{Toc19704805}{}
When I analyzed the source texts, a number of themes immediately caught my attention, and for this reason it seemed wise to adopt a thematic rather than a strictly chronological approach in presenting the results of my research. This subject-based structure will, I hope, enhance the coherence and readability of the book. Each thematic chapter will, however, have a diachronic dimension, in that where possible I will first sketch ancient and medieval ideas, which were usually the starting point for early modern Hellenists. For the pre-Renaissance era, I will principally draw on existing scholarship. Significant evolutions and major points of disagreement in early modern thought will also be charted.

Which themes have I selected? I have opted to concentrate on the issues that took center stage in early modern discussions; all of these can be framed within broader intellectual currents, either scientific, philological, historiographical, ethno-geographical, or religious, as linguistics – let alone dialectology – was not yet an autonomous research field. The seven main chapters (Chapters 2–8) of this study are devoted to early modern ideas about, and approaches to, the Greek dialects, which are contextualized throughout and especially in Chapters 3, 4, 7, and 8. In Chapter 2, the many different attempts at classifying the Greek dialects are treated. I discuss the main motivation behind the early modern interest in this topic, philology, in the third chapter. Chapter 4 analyzes views on two specific varieties of Greek that posed problems to early modern scholars: the Greek of Homer’s epic poems and that of the Bible, both speech forms still debated by specialists today. The fifth chapter concerns the early modern attempts at writing the linguistic history of the Greek language, which constituted a great challenge due to its many dialects. In Chapter 6, I demonstrate how early modern Hellenists tried to make sense of the great linguistic variation among the Greek dialects in describing their particularities. Chapter 7 treats the way in which the Greek dialects were related to other aspects of ancient Greece, including its literary tradition as well as its diversified geopolitical and ethnic constitution. Finally, in Chapter 8, I zoom out so as to chart the different and often contradictory usages of ancient Greek dialect diversity as a point of reference for understanding the dialects of other languages, especially the vernacular tongues that were emancipating in the early modern era.

My content-in-context approach is indispensable, since the book intends to contribute not only to the history of linguistic thought, but also to intellectual history and especially to the growing subfield of the history of Hellenism. In so doing, it aims to appeal to intellectual historians as well as to linguists and classicists interested in the long and understudied history of their disciplines.

\section{2  Order in chaos? Classifications of Greek dialects}
\hypertarget{Toc19704806}{}
“In the Greek language, there are great labyrinths and enormously vast recesses, not only in the various dialects, but in every one of them”.\footnote{See the Preface for the original quote.} Juan Luis Vives’ judgement on the immense diversity within the ancient Greek language was crystal clear; it posed an enormous challenge to him and his early modern colleagues. Even though not all scholars were excited about tackling this thorny issue, their fascination with the dialectally diverse Greek literature was strong enough to make them reflect at length on the questions: what forms of Greek are there? And how can they be distinguished? Their solutions to these matters constitute the main subject of this chapter, as it is impossible to understand other key aspects of early modern scholarship on the Greek dialects without gaining insight into this sometimes complex matter. Yet before treating early modern classifications of the Greek dialects and the problems central to them, it is indispensable that I elaborate first upon ancient and medieval scholarship, on which their Renaissance successors relied.\footnote{This and the following chapter are an extended, updated, and more detailed version of Van \citet{Rooy2016a}, integrating information from Van \citet{Rooy2018b}.}

\subsection{Between mythology and dialectology}
\hypertarget{Toc19704807}{}
How did ancient writers try to map out and categorize Greek linguistic diversity? Attempts at classifying the Greek dialects appeared relatively early and exhibited a striking mythological-etiological dimension, which I have to discuss at some length here, as it shaped early modern views to a certain extent. From a modern perspective it must seem rather odd to associate dialect with mythology, and it requires some explanation why this was the case in ancient Greece. To understand this connection, I should quote a seminal text fragment of the poet Hesiod (\textit{fl.} late 8th cent. \textsc{bc}) that usually heads outlines of Greek scholarship on the dialects:

\begin{quote}
From Hellen, the warlike king, were born
\end{quote}

\begin{quote}
Dorus, Xuthus, and Aeolus the chariot-fighter.\footnote{\textit{Fragmenta} 9: “$\text{\textgreek{<'E}}\lambda \lambda \eta \nu o\varsigma $ δ’ $\text{\textgreek{>e}}\gamma \text{\textgreek{'e}}\nu o\nu \tau o$ $\varphi \iota \lambda o\pi \tau o\lambda \text{\textgreek{'e}}\mu o\upsilon $ $\beta \alpha \sigma \iota \lambda \text{\textgreek{~h}}o\varsigma $ {\textbar} $\Delta \text{\textgreek{~w}}\rho \text{\textgreek{'o}}\varsigma $ $\tau \varepsilon $ $\Xi o\text{\textgreek{~u}}\theta \text{\textgreek{'o}}\varsigma $ $\tau \varepsilon $ $\kappa \alpha \text{\textgreek{`i}}$ $A\text{\textgreek{>'i}}o\lambda o\varsigma $ $\text{\textgreek{<i}}\pi \pi \iota o\chi \text{\textgreek{'a}}\rho \mu \eta \varsigma $”. See also \citet[64]{Hainsworth1967}.}
\end{quote}

Hellen, the son of Deucalion – or Zeus in a different tradition – and Pyrrha, was the mythological primogenitor of the Greeks, who were divided into four principal tribes, descended from Hellen’s sons Dorus, Xuthus, and Aeolus. Xuthus produced two distinct tribes through his sons Ion and Achaeus, from whom the Ionians and Achaeans derived. This somewhat complex mythological genealogy of the four Greek tribes, which oddly blended different generations of one and the same family, is visualized in \figref{fig:key:1} below.

\begin{stylecaption}
Figure \stepcounter{Figure}{\theFigure}: The genealogy of Deucalion’s children (source Van \citealt{Rooy2016b}: 208).
\end{stylecaption}

%%[Warning: Draw object ignored]
%%[Warning: Draw object ignored]
%%[Warning: Draw object ignored]
%%[Warning: Draw object ignored]
%%[Warning: Draw object ignored]
Deucalion x Pyrrha        Cranaus

\begin{styleListParagraph}
%%[Warning: Draw object ignored]
  \textbf{Hellen} x Orseis  Amphictyon  x  \textbf{Atthis}
\end{styleListParagraph}

\begin{styleListParagraph}
%%[Warning: Draw object ignored]
%%[Warning: Draw object ignored]
%%[Warning: Draw object ignored]

\end{styleListParagraph}

%%[Warning: Draw object ignored]
%%[Warning: Draw object ignored]
%%[Warning: Draw object ignored]
\textbf{Dorus}  Xuthus x Creusa  \textbf{Aeolus}

\textbf{Ion}  Achaeus

As J. B. Hainsworth (1967: 64–65) has rightly remarked, the four tribe model was projected onto the dialect groups of the Greek language. The myth was in other words of an etiological nature, as it explained the existence not only of the different Greek tribes and their names, but also the dialects they spoke. The earliest extant testimony of a dialect classification inspired by the mythological model is a fragment of the geographer Heraclides Criticus’ (\textit{fl.} 3rd cent. \textsc{bc}) \textit{Description} \textit{of} \textit{Greece} (fr. 3.2), which was incorrectly attributed for a long time to his colleague Dicaearchus (\textit{fl.} 4th cent. \textsc{bc}; see \citealt{Brodersen2015}). It suggested a division of the Greek language into Attic, Doric, Aeolic, Ionic, and – atypically – Hellenic, the variety of Hellas, for Heraclides apparently a region in Thessaly.\footnote{On Heraclides’ interpretation of Hellas and its place in his discourse, see McInerney (2012: 257–260).} This early classification, however, occupied a somewhat peculiar position, wielding barely any influence on later thought. Instead, two other dialect classifications dominated Greek scholarship.

\subsection{Four or five dialects? The two major dialect classifications in Greek scholarship}
\hypertarget{Toc19704808}{}
The notable geographer Strabo (ca. 62 \textsc{bc–}ca. \textsc{ad} 24) seems to have been the earliest scholar to propose a classification into four dialects: Ionic, Attic, Doric, and Aeolic (\textit{Geography} 8.1.2). He saw a close kinship between Ionic and Attic, on the one hand, and Doric and Aeolic, on the other. The former connection, corroborated by modern linguistics, might have been inspired by an intuitive comparison of actual linguistic data; the claim that Doric and Aeolic are closely related cannot, however, be backed by dialectal evidence and was probably maintained solely for the sake of analogy with the Ionic–Attic group. Another scholar proposing the fourfold classification was the grammarian Apollonius Dyscolus (1st half 2nd cent. \textsc{ad}), who according to the Byzantine encyclopedia known as the \textit{Suda} (at α.3422) composed a now lost work on the four Greek dialects Doric, Ionic, Aeolic, and Attic. The origin of this influential classification is unclear, but it seems to have been an achievement of Hellenistic scholarship, flourishing especially in Alexandria, Egypt, with its famous library. Strabo passed by the city on his many travels, and Apollonius lived and worked there.

Soon after Apollonius, however, an alternative classification into five dialects appears to have taken root, adding the koine to Attic, Ionic, Doric, and Aeolic. It was presented as a common Greek opinion by the Early Christian author Clement of Alexandria (ca. \textsc{ad} 140/150–before 215/221) in his miscellaneous work entitled \textit{Patchwork} (\textit{Stromata} 1.21.142.4). This widespread classification was later adopted by, among many others, the Byzantine scholar-emperor Constantine \textsc{vii} Porphyrogennetos (905–959; see \textit{De} \textit{thematibus} 17). The classification into five dialects also prevailed in the Byzantine treatises on the Greek dialects by John the Grammarian and Gregory of Corinth, who clearly struggled with the status of the koine and its relationship to the four other dialects.

\subsection{The koine in Greek scholarship}
\hypertarget{Toc19704809}{}
The koine dialect must indeed have been a major problem for Greek scholars working on the dialects, not only in Byzantium but already during antiquity, even though there are no direct sources available proving this. It is, however, hard to believe that ancient scholars were not aware that the koine, a widely used lingua franca, had a status different from the other literary dialects \citep{Consani2000}. As a matter of fact, the early Byzantine author John the Grammarian reported different opinions on the koine, likely reflecting a debate held in earlier works on the matter that are now lost. By means of John’s account and other evidence, often fragmentary, I have been able to reconstruct the most important Greek attitudes toward the koine, five in total:

\begin{itemize}
\item \begin{styleListParagraph}
The four other dialects derived from the koine.\footnote{This overview is an adapted and augmented version of the list in Van \citet[209]{Rooy2016b}. Cf. also Consani (2000: 614–617).}
\end{styleListParagraph}
\item \begin{styleListParagraph}
The koine was the mother of the four other dialects, was formed by mixing them, and therefore consisted of them. In other words, the koine was the variety comprising all the other dialects, since it contained elements of each of them. It embraced, as it were, the dialects, as was befitting for a mother. It can be noted that despite the usage of the mother image, the koine was clearly seen as being posterior in time to the dialects in this view.
\end{styleListParagraph}
\item \begin{styleListParagraph}
The koine was the subject of grammar and characterized by rules, whereas the other dialects deviated from those rules (e.g. \textit{Scholia} \textit{in} \textit{Pindarum} \textit{(scholia} \textit{uetera)} \textsc{o} 3.81c; John the Grammarian in Manutius \textit{et} \textit{al.} 1496: 236\textsc{\textsuperscript{v}}).
\end{styleListParagraph}
\item \begin{styleListParagraph}
The koine was used commonly by all, which implies that this did not hold for the other dialects (e.g. John the Grammarian in Manutius \textit{et} \textit{al.} 1496: 236\textsc{\textsuperscript{v}}).
\end{styleListParagraph}
\item \begin{styleListParagraph}
As a final attitude I can add the usually unexplained addition of the koine as the fifth dialect (e.g. Clement, \textit{Stromata} 1.21.142.4).
\end{styleListParagraph}
\end{itemize}

None of these solutions became generally accepted in Greek scholarship. From a modern perspective, one might regard the third view, according to which the koine was the normative variety from which the other dialects were deviations, as making the most sense. Yet even though there was a tradition of normative thinking separating correct from incorrect forms of Greek, the position of the dialects in this dichotomy remained unclear, to say the least (\citealt{Versteegh1986}; \citealt{Dickey2007}: 235). Overall, the Greek linguistic ideal of \textit{Hellēnismós} ($\text{\textgreek{<E}}\lambda \lambda \eta \nu \iota \sigma \mu \text{\textgreek{'o}}\varsigma $) usually encompassed the canonical literary dialects other than the koine, too, as James \citet{Clackson2015} has convincingly pointed out. This was especially true of the Attic dialect, relaunched as the best form of Greek during the cultural and literary movement known as the Second Sophistic in the first centuries \textsc{ad} (\citealt{Whitmarsh2005}: esp. 41–56).

An important reason why Greek scholars perceived the koine to be problematic was likely the fact that this form of their language could not be linked to a specific Greek tribe or region, two main parameters they put forward in their definitions of the term \textit{diálektos} ($\delta \iota \text{\textgreek{'a}}\lambda \varepsilon \kappa \tau o\varsigma $; see Van \citealt{Rooy2016d}), and that the koine did not seem to have features clearly distinguishing it from the other dialects. The latter was no doubt also the reason why treatises on the dialects usually did not include koine features in their discussion. An additional reason was perhaps that readers of these treatises were expected to already command the koine, the first variety of Greek to be learned in class by a Byzantine student (cf. Van \citealt{Rooy2016b}).

\subsection{Zooming in: Below the level of dialect}
\hypertarget{Toc19704810}{}
Greek scholars did not limit themselves to listing the principal Greek dialects; they were aware that they could be further divided into what modern linguists would perhaps call \textit{subdialects}. In the Byzantine period, varieties of individual Greek dialects were from time to time mentioned in grammatical and philological works, a practice with roots in ancient scholarship (\citealt{Hainsworth1964}: 70–72). The Byzantine theologian and grammarian Gregory of Corinth discussed several “local subdivisions” (\textit{hypodiairéseis} \textit{topikaí}/$\text{\textgreek{<u}}\pi o\delta \iota \alpha \iota \rho \text{\textgreek{'e}}\sigma \varepsilon \iota \varsigma $ $\tau o\pi \iota \kappa \alpha \text{\textgreek{'i}}$) of Doric in his treatise \textit{On} \textit{the} \textit{dialects} (at 3.111). A similar concept was expressed in different terms by scholiasts of the grammar attributed to Dionysius Thrax (170–90 \textsc{bc}; see \citealt{Lambert2009}: 21–22; Van \citealt{Rooy2016d}: 261–263). The multiplicity of individual Greek dialects was occasionally alluded to by ancient authors as well, for instance by Sextus Empiricus (\textit{fl.} ca. \textsc{ad} 190–210), who drew attention to the multitude of Attic and Doric varieties.\footnote{\textit{Aduersus} \textit{mathematicos} 1.89. Cf. current discourse on so-called \textit{Englishes}.} Lower-level varieties of dialects were generally closely connected to the practice of glossography, the collection of rare, often local words, of which Hesychius’ (?5th/6th cent. \textsc{ad}) \textit{Lexicon} is the best known example. Yet the precise relationship between the main dialects, on the one hand, and their “local subdivisions”, on the other, is a matter on which Greek scholars failed to comment.

*

*  *

What were, in a nutshell, the main insights of Greek scholarship on the dialects? Firstly, inspired by a mythological tradition, authors devised different classifications of the Greek dialects which, though certainly not perfect, were still partly accurate and in keeping with actual linguistic evidence. The division into Attic, Ionic, Doric, and Aeolic, either with or without the koine, was overwhelmingly predominant in Greek scholarship and, as I will demonstrate, left an indelible mark on later thought. Secondly, whereas Greek scholars were aware that some of the dialects could be further divided into numerous other varieties, they had great difficulty in adequately determining the exact position of the koine vis-à-vis the other dialects.

\subsection{The Greek dialects in the ancient and medieval Latin world}
\hypertarget{Toc19704811}{}
Greek scholarship, rediscovered by humanists during the Renaissance, was no doubt the most importance source for early modern authors interested in the dialects. It was, however, not the only source, as they could also read relevant observations in a number of Latin works from antiquity and the Middle Ages. The five-way classification seems to have been widely known to Latin authors. In fact, the Roman orator Quintilian (ca. 35–ca. 100 \textsc{ad}), though not referring to the Greek dialects by name, knew that there were five Greek “differences in speech”.\footnote{ \textrm{\textit{Institutio} \textit{oratoria}} \textrm{11.2.50: “quinque Graeci sermonis differentias”.}} He mentioned this when recounting that Publius Crassus Mucianus (ca. 180–130 \textsc{bc}), Roman proconsul in Asia Minor, could speak in all five of them. Before Quintilian, this anecdote had also been related by Valerius Maximus (\textit{fl.} 14–37 \textsc{ad}) in his \textit{Nine} \textit{books} \textit{of} \textit{memorable} \textit{deeds} \textit{and} \textit{sayings} (\textit{Facta} \textit{et} \textit{dicta} \textit{memorabilia} 8.7.6), in which the Latin term \textit{genus}, ‘kind’ or ‘species’, was employed to refer to the Greek dialects. Remarkable is that the testimonies of Valerius and Quintilian predate the appearance of the five-way classification in extant Greek sources. It is, however, not unlikely that Valerius and Quintilian took this anecdote from a common source now lost, perhaps a Greek one, which itself might have been related to Hellenistic scholarship on the dialects.

Quintilian also discussed a fallacy known as \textit{Sardismós} ($\Sigma \alpha \rho \delta \iota \sigma \mu \text{\textgreek{'o}}\varsigma $), consisting of dialect mixture and named after Sardis, a city in Asia Minor (modern-day Turkey) which supposedly had a dialectally mixed population (see \citealt{Gitner2019}). In this context, Quintilian mentioned the four canonical Greek “tongues” (\textit{linguae}):

\begin{quote}
Also, Sardism is the name of a certain speech mixed from a diverging range of tongues, for instance, in case one would confound Doric, Ionic, or even Aeolic words with Attic ones. Yet we have a similar vice in cases where someone mixes lofty with lowly words, old with new ones, and poetic with vulgar ones – that is indeed such a monstrosity, as Horace writes in the first part of his book on the art of poetry: if a painter would want to join a horse’s neck to a human head – and would place other things of different natures under it.\footnote{\textrm{ \textit{Institutio} \textit{oratoria}} \textrm{8.3.59: “$\Sigma \alpha \rho \delta \iota \sigma \mu \text{\textgreek{'o}}\varsigma $ quoque appellatur quaedam mixta ex uaria ratione linguarum oratio, ut si Atticis Dorica, Ionica, Aeolica etiam dicta confundas. Cui simile uitium est apud nos, si quis sublimia humilibus, uetera nouis, poetica uulgaribus misceat – id enim tale monstrum, quale Horatius in prima parte libri de arte poetica fingit: Humano capiti ceruicem pictor equinam iungere si uelit – et cetera ex diuersis naturis subiciat”. See \citet[46]{Carruthers2009} for the relevance of this passage to grasp the ancient and medieval concept of} \textrm{\textit{uarietas}}.}
\end{quote}

Whereas the Greek example referred to regional varieties that had been elevated to literary dialects, the Latin situation did not concern regional linguistic diversity, but mainly register-based variation and – to a certain extent – differences in terms of time and social class. Quintilian’s case is somewhat problematic, in that in the Crassus anecdote he referred to five dialects, whereas in his discussion of \textit{Sardismós} he suggested that there were only four. The solution to this question is probably that Quintilian himself was not very well-versed in the Greek dialects and that the discrepancy in his work is due to the fact that he was drawing on diverging sources.

Several late antique Latin authors too mentioned the division into five dialects. Let me limit myself to the most puzzling example, revealing that knowledge of the dialects was often indirect and incomplete in the Latin world. The Early Christian bishop Isidore of Seville (ca. 560–636), author of an encyclopedic work entitled \textit{Etymologies}, spoke of the fivefold “variety” (\textit{uarietas}) of Greek. The bishop adhered to the view that the koine was the mixed common language, but his remarks on the use of Attic and Doric are rather unusual and betray a clear lack of competence in the Greek language. The former is said to have been used by all literary authors of Greece, whereas the latter is oddly claimed to have been current in Egypt and Syria.\footnote{\textit{Etymologiarum} \textit{siue} \textit{Originum} \textit{libri} \textsc{xx} 9.1.4, on which see Denecker (2017: 227–229).}

Three Roman and early medieval classifications of the Greek tribes and their dialects are somewhat peculiar, which is why they deserve a specific mention here. Firstly, the famous orator Cicero (106–43 \textsc{bc}) asserted that there were three Greek tribes, which later scholars interpreted as referring to a tripartite linguistic classification into Athenian, Aeolic, and Doric (see \textit{Pro} \textit{L.} \textit{Valerio} \textit{Flacco} \textit{oratio} 64). Secondly, the grammarian Diomedes (\textit{fl.} ca. 370–380) associated each of the five Greek “tongues” (\textit{linguae}) with certain linguistic-rhetoric usages and fallacies (\textit{Ars} \textit{grammatica} 2, ed. \citealt{Keil1855}–1880: \textsc{i}.440). This shows the artificial solutions on which some grammarians relied to account for variation in Greek and betrays a transfer of the Latin concept of vices (\textit{uitia}) to the Greek dialects. Ionians, Diomedes claimed, were well-versed in figurative speech – tropes in his terminology – whereas Attic displayed solecisms and Doric was characterized by barbarisms; Aeolic was considered excessive. In the koine, presumably because of its status as common variety, all these elements were allegedly present (\citealt{Consani1991}: 32–33). Thirdly, Pseudacro (\textit{fl.} 7th/8th cent.) offered a peculiar classification of the Greek dialects, claiming that there were “five characters of tongues” of the Greeks: “Attic, Asian, Aeolic, common, Rhodian”, with “Asian” no doubt referring to Ionic and “Rhodian” to Doric.\footnote{ \textrm{See} \textrm{\textit{Scholia} \textit{in} \textit{Horatium:} \textit{Glossarum} \textit{“gamma”} \textit{appendix}} \textrm{4 (i.e. \citealt{Pseudacro1902}–1904:} \textrm{\textsc{ii.}}\textrm{385): “}\textstylehigh{quinque} autem sunt \textstylehigh{caracteres sermonum:} Atticus, Asianus, Aeolius, communis, Rhodius\textrm{”.}} Pseudacro’s alternative glottonymic designations were inspired by the names of three well-known ancient rhetoric trends: exuberant Asianism, traditional Atticism, and the intermediate Rhodian style.

In conclusion, some ancient and early medieval Latin authors were superficially acquainted with the traditional five literary dialects of Greek. Most remarks were of a very general nature, however, with the exception of the Latin grammarian Priscian, who, working in Byzantium around 500 \textsc{ad}, expressed great interest in the Greek dialects in as far as he was able to tie them to Latin (see Conduché fc.). During the greater part of the Middle Ages, Greek was barely studied in the West, as this language was considered either heretic or simply irrelevant (see e.g. \citealt{Boulhol2014}). When copyists encountered Greek words or phrases, they usually had to confess that they were unable to read it: “It is Greek, it is not read” was an often recurring note.\footnote{ “Graecum est, non legitur”. \textrm{See e.g. Bischoff (1961; 1981:} \textrm{\textsc{i.}}\textrm{246–275); Weiss (1977: esp. 3ff);}\textrm{ }\textrm{\citet[36]{Saladin2000}.}} This lack of knowledge was related to the fact that at this time no adequate grammar of Greek composed in Latin existed \citep[215]{Bischoff1961}. The language nevertheless excited considerable practical interest, evidenced by, among other things, the compilation of a number of lexica (\citealt{Bischoff1961}: 217–219; Dahan, \citealt{RosierValente1995}: 267–269). Given the rarity of competence in Greek, it is not surprising that knowledge of the canonical dialects too was highly limited. Even awareness of their existence was rare. The theologian Hugh of Saint Victor (ca. 1096–1141), for instance, was only able to repeat the ill-informed statement of Isidore of Seville in his work on grammar (1966: 79). The twelfth-century \textit{Vatican} \textit{Mythographer}, in turn, made an oddly placed, completely isolated remark about the canonical five “Greek tongues” (\textit{Graecae} \textit{linguae}), whereas Eberhard of Béthune (\textit{fl.} ca. 1212) likewise mentioned the division into five “idioms” (\textit{idiomata}), remarkably substituting, however, the koine with Boeotian.\footnote{\textit{Mythographus} \textit{Vaticanus} 1.192; Eberhard of Béthune, \textit{Graecismus} 8.1–2.}

The only exception seems to have been the English polymath Roger Bacon (ca. 1214/1220–ca. 1292), who accorded generous attention to the dialects in his Greek grammar, which he composed in Latin around 1268. Even though early modern scholars were unable to make use of Bacon’s work – the only edition of Bacon’s grammar appeared in 1902 – his remarkable views deserve to be briefly treated in a history of premodern scholarship on the Greek dialects. How did Bacon classify the Greek dialects? He stated that “there were five and six [\textit{sic}] idioms of the Greek language”.\footnote{\citet[26]{Bacon1902}: “5 et 6 fuerunt idiomata Graecae linguae”.} This phrase is revealing in two ways. It indicates, on the one hand, that Bacon was apparently aware that the Greek dialects were no longer spoken, as he used the perfect indicative form \textit{fuerunt} of the Latin verb \textit{sum}, ‘to be’. On the other hand, he added an additional dialect to the traditional fivefold classification: Boeotian \citep[27]{Bacon1902}. The clumsy formulation “five and six” may suggest that Bacon was hesitant about including it. The koine was clearly perceived as somehow distinct from the other Greek dialects. He regarded it as the variety consisting of what was common to all Greek tribes and which was used for communication by all. It was, Bacon suggested, the core nature and substance of the Greek language, on which the other idioms were mere variations.

In conclusion, Western scholars were usually ill-informed about Greek linguistic diversity. The late medieval polymath Roger Bacon, who expressed a unique interest in the Greek dialects, was the proverbial exception. This state of affairs changed profoundly in the Renaissance, to which I now turn.

\subsection{Tradition and innovation: Old classifications and a new principle}
\hypertarget{Toc19704812}{}
As I have pointed out in Chapter 1 (\sectref{sec:key:2}), the renowned printer Aldus Manutius was responsible for a key turn in the history of Greek dialect studies. This coincided with his issuing an impressive collection of ancient Greek and Byzantine grammatical treatises, intended for the experienced Hellenist. Explaining the range of this reference work in his preface, Manutius boldly stated that

\begin{quote}
it moreover treats the Attic, Ionic, Aeolic, Doric, Boeotian, Cretan, Cypriot, Macedonian, Thessalian, Rhegian, Sicilian, Tarentine, Chalcidian, Argive, Laconian, Syracusan, Pamphylian, and Athenian tongues. These the Greek poets, and Homer in particular, are found to have used. Due to these tongues and their various inflections they have an astonishing liberty. They add, subtract, transmute, invert. What don’t they do? In short, they use words like wax.\footnote{Manutius (1496: *.ii\textsc{\textsuperscript{v}}): “Linguarum praeterea meminit Atticae, Ionicae, Aeolicae, Doricae, Boeticae, Cretensis, Cypriae, Macedonicae, Tessalae [\textit{sic}], Rheginae, Siculae, Tarentinae, Chalcidicae, Argiuae, Laconicae, Syracusanae, Pamphyliae, Atheniensis, quibus usi Graeci poetae inueniuntur, et Homerus praecipue. His linguis ac figuris uariis habent illi miram licentiam. Addunt, detrahunt, transmutant, inuertunt. Quid non faciunt? Denique utuntur dictionibus ut cera”. My translation is inspired, but only very loosely, by the rather free and inadequate rendering of Bean \& \citet[12]{Lemke1958}.}
\end{quote}

Manutius here already mentioned most of the varieties included in the widely accepted modern classification of the ancient Greek dialects into Aeolic, Arcado-Cypriot, Attic–Ionic, Doric, Northwest Greek, and Pamphylian (cf. Chapter 1, \sectref{sec:key:1}), even though he did not offer much more than a mere listing. Manutius listed the dialects Attic, Ionic, Aeolic, and Doric first. This is neither a coincidence nor a surprise; as I have mentioned, these were the four canonized literary dialects of ancient Greek. Manutius did not refer to the koine here, but he was no doubt aware of its existence from the treatises in the collection, which he later translated into Latin. It is immediately apparent from the above passage that Manutius associated the dialects closely with poetry. This reveals the primary reason why humanists studied the dialects; much like their Greek predecessors, they wanted to master them in order to better understand Greek poems. For humanists in the Latin West, however, studying the dialects was initially only a second-degree auxiliary tool. They wanted to know the dialects because they wanted to be able to read Greek literature, which in turn served as a means to gaining deeper insight into Latin literature, as it was modeled on Greek examples. This likely explains why Renaissance Hellenists were content, initially at least, with the two traditional classifications into four or five dialects; they were pursuing philological goals similar to their predecessors.

An early scholar propounding the fourfold classification into Attic, Ionic, Doric, and Aeolic was, for instance, Johann Reuchlin – if, at least, I may presume that he backed the ideas contained in the Byzantine treatise he tried to pass off as his own (see Van \citealt{Rooy2014}: 510–515). Others adhered to the classification including the koine. An early example is Nicolaus Clenardus (1493/1494/1495–1542), a humanist from Diest in modern-day Belgium, whose manual for Greek was so popular in the early modern period that the name \textit{Clenardus} even became synonymous with Greek grammar. From this handbook, first published in 1530 in the university city of Leuven, the student of Greek could gather that “there are five principal tongues among the Greeks: common, Attic, Ionic, Doric, Aeolic”.\footnote{Clenardus (1530: 7 [misprint for 6]): “Quinque Graecorum linguae praecipuae, Communis, Attica, Ionica, Dorica, Aeolica”.} The latter fivefold classification was best-known in early modern linguistic scholarship, most likely because it was the one that predominated in the Byzantine treatises by John the Grammarian and Gregory of Corinth; these works were definitely known to Hellenists of the time, as they were published together by Manutius in 1496 and subsequently in many other handbooks. Gradually, however, scholars felt the need to alter, correct, and supplement traditional Greek dialect classifications. What alternatives did they propose and why? Did they employ the same classificatory principles grounded in mythological and philological assumptions as Greek philologists had done? Or did they break away from Greek tradition?

The insight that some of the traditional four dialects could be further divided into different speech forms, only marginally present in Greek thought, was further developed by early modern philologists. They introduced a distinction between “principal” and “less principal” dialects, to which Clenardus already alluded in his grammar. Principal dialects were those relevant to the study of literature, whereas the less principal dialects were those for which scholars only had fragmentary or indirect evidence from ancient and Byzantine sources and which were not of direct concern for philologists. What was the origin of this new bipartition? It dates without a doubt from the beginning of the Cinquecento and had its roots on the Italian peninsula. The earliest testimony I have been able to trace thus far can be found in a grammatical commentary of 1509, published in Ferrara and authored by the humanist professor Ludovico da Ponte (Ponticus Virunius; ca. 1460–1520), whose contribution to Greek studies merits a closer study.\footnote{For biographical information on Da Ponte, see \citet{Ricciardi1986}, with many further references.} Da Ponte maintained that “even though there are seventeen tongues of the Greeks, there are nevertheless five principal tongues”.\footnote{Da Ponte (1509: 20\textsc{\textsuperscript{v}}–21\textsc{\textsuperscript{r}}): “cum \textsc{xvii} sint linguae Graecorum, tamen principales sunt quinque linguae”.} Da Ponte did not clarify, however, whether the former were subsumed under the five principal ones or stood next to them on the same level; nor did he mention all seventeen dialects by name.\footnote{For similar wordings (esp. the adjective \textit{principalis}) see Oecolampadius (1518: 51–52). Cf. also Canini (1554: 12; 1555: a.3\textsc{\textsuperscript{v}}), using the terms \textit{generalis}, \textit{princeps}, and \textit{superior}; \citet[2]{Walper1589}, speaking of \textit{dialecti} \textit{primariae}.} Another early testimony, this time from north of the Alps, is Adrien Amerot’s Greek grammar of 1520, in which one reads that “there are almost as many tongues of the Greeks as there are tribes, among which nevertheless five are principally employed”.\footnote{Amerot (1520: \textsc{q}.i\textsc{\textsuperscript{v}}): “Graecorum linguae tot paene sunt, quot nationes, ex his tamen praecipue quinque celebrantur”.} The phrase also occurred in Amerot’s popular booklet on the Greek dialects, a separately published excerpt from his grammar which first appeared in 1530 and enjoyed countless reprints during the entire early modern period.\footnote{See Hoven (1985: 5–19) for an extensive list, which can even be augmented by digital searches. See also Chapter 1, \sectref{sec:key:2.}} This greatly contributed to the spread of the idea that there were “principal” and “less principal” Greek dialects. For instance, Amerot certainly inspired the information on the dialects in Michael Neander’s (1525–1595) popular Greek grammar, and he is also likely to have influenced the statement of Nicolaus Clenardus on the Greek dialects.\footnote{See \citet[187]{Neander1553}. For Amerot’s possible influence on Clenardus’ grammar, see Van \citet{Rooy2019}. For the adjective \textit{praecipuus}, see also \citet[42]{Mosellanus1527}, who claimed that there were about 24 dialects in total.}

Early humanist Hellenists did normally not explain why they made this division into “principal” and “less principal” dialects. There are a number of exceptions, however, which are worth a closer look. The earliest justification of the innovation occurred as a passing remark in the Greek grammar of Georg Simler (ca. 1477–1536), a German humanist who was the teacher of, among others, Philipp Melanchthon: “We have called them principal, for they are used by poets, especially Homer”.\footnote{Simler (1512: \textsc{aa.}i\textsc{\textsuperscript{r}}): “Principales diximus, sunt enim quibus utuntur poetae, praesertim Homerus”.} For a more extensive motivation, we have to wait until the middle of the sixteenth century. The French Hellenist Pierre Davantès (Petrus Antesignanus; ca. 1525–1561), an influential commentator of Clenardus’ Greek grammar, explained that the traditional dialects were dubbed “principal”, because these were the varieties of Greek mainly used by literary authors. Other dialects such as Boeotian and Thessalian were labeled “less principal”, since there were no literary works extant which were entirely composed in them. These dialects would have been lost to the ages if Greek authors had not introduced some elements of them into their works.\footnote{\citet[11]{Antesignanus1554}, on which see Van Rooy (2016c: 129–130).} In other words, the mere existence and survival of literary works was employed as a classificatory principle to distinguish between the “principal” and “less principal” dialects of the Greek language. This criterion, proving once again that philology was the primary motivation to study the Greek dialects in the Renaissance, became highly popular and was adopted by numerous early modern Hellenists.\footnote{See e.g. Walper (1589: 2–3); Schmidt (1604: 7–8); Mérigon (1621: 3–4); Rhenius (1626: 83–84); \citet[66]{Busby1696}; [Frisch] (1730: 1132–1133).}

Briefly, the traditional classifications were still vastly important in the early modern era, even though scholars introduced a finer-grained distinction based on philological criteria by opposing “principal” to “less principal” dialects. In this case, the latter were frequently viewed as varieties subsumed under the former.\footnote{See e.g. Canini (1555: a.3\textsc{\textsuperscript{v}}); \citet[439]{Saumaise1643a}; Munthe \& Heiberg (1748: 2–3, 7–9); Valckenaer (1790: 490–491).} The early modern discourse on “principal dialects” came to be extrapolated to other languages too. A notable example is Alexander Gill (1565–1635). This English schoolmaster who taught Greek to John Milton was the author of a grammar of his native tongue, in which he claimed that there were six main dialects in English:

\begin{quote}
There are six principal dialects: Common, Northerners’, Southerners’, Easterners’, Westerners’, Poetic. I neither know nor have heard all their particularities. Yet I will describe as far as I can those I remember.\footnote{\citet[15]{Gill1619}: “Dialecti praecipuae sunt sex: Communis, Borealium, Australium, Orientalium, Occidentalium, Poetica. Omnia earum idiomata nec noui, nec audiui; quae tamen memini, ut potero dicam”. See \citet{Kökeritz1938}. See e.g. also Thomassin (1697: liv) on the three principal (\textit{principes}) dialects of Chaldean; Schultens (1748: \textsc{xciii}) on the principal (\textit{principes}) dialects of the primeval language.}
\end{quote}

Several elements in the above quote suggest influence from the tradition of early modern grammars of ancient Greek, with which Gill, being a distinguished Hellenist, must have been acquainted. He used the designation “common dialect”, reminding of the Greek koine, and included a poetical dialect among his English dialects \citep[18]{Gill1619}, a concept first developed within Greek grammar, as I will demonstrate later (see \sectref{sec:key:7} below). There are, however, also differences. The names of the English dialects were more strictly geographical than their Greek counterparts, which had a link with Greek mythology and tribal history. The English dialects were, moreover, by no means literary varieties. In fact, the Western English dialect, especially in Somerset, was so barbarous that it barely deserved the name “English”. \citet[17]{Gill1619} did grant, however, that it preserved some notable archaic features. He moreover acknowledged that it was useful to know the dialects, since English poets occasionally used dialect elements \citep[18]{Gill1619}. This suggests that his appreciation of English dialects was not unequivocally negative, an attitude for which he might have found support in the prestige of the Greek dialects.\footnote{Cf. also Chapter 8, \sectref{sec:key:1.2.}, on the model status of the Greek dialect context.}

In the present section, I might have created the impression that the two traditional classifications were the only ones proposed by early modern thinkers and that their only contribution was to introduce the distinction between “principal” and “less principal” dialects. Was this really the case? Or did scholars also innovate and design alternative classifications? If so, how and why did they do so?

\subsection{The invention of a poetical dialect}
\hypertarget{Toc19704813}{}
In his \textit{Booklet} \textit{on} \textit{the} \textit{Greek} \textit{dialects} of 1569, a poorly known French Hellenist by the Latin name of Robertus Vuidius, originating from Tonnerre in the center of northern France, explained that he intended “to treat the five idioms or dialects, i.e. Attic, Ionic, Aeolic, Doric, and Poetic”.\footnote{Vuidius (1569: 137\textsc{\textsuperscript{v}}): “Quinque autem idiomata siue $\delta \iota \alpha \lambda \text{\textgreek{'e}}\kappa \tau o\upsilon \varsigma $ tractare sumus ingressi Atticum uidelicet, Ionicum, Aeolicum, Doricum et Poeticum”.} In so doing, Vuidius heralded a new era in early modern classifications of the Greek dialects, during which a set of new varieties was added to the traditional four or five. Before him, influential scholars like Petrus Ramus (Pierre de la Ramée; 1515–1572) and Joseph Justus Scaliger (1540–1609) had already spoken in passing of a Greek “poetical dialect” (\citealt{Ramus1560}: 18–19; \citealt{Scaliger1594}: 56). Scaliger even claimed to have composed a grammar of this dialect when he was about twenty years old, of which no traces remain today, however. What is more, linguistic particularities proper to poetry, often simply explained as “poetical license”, had been noticed well before Ramus, Scaliger, and Vuidius by earlier Greek scholars. The ancient grammarian Tryphon, perhaps the founding father of Greek dialect studies, had already associated procedures such as metathesis with poetry.\footnote{See e.g. Tryphon’s $\Pi \varepsilon \rho \text{\textgreek{`i}}$ $\pi \alpha \theta \text{\textgreek{~w}}\nu $ 3.18. Cf. Da Ponte (1509: 78\textsc{\textsuperscript{v}}); Vergara (1537: 209, 230, 235).} The idea that Greek poetical language was dominated by a far-ranging license, however, received prominence only in Renaissance thought, even if ancient authorities were invoked to back it. A case in point is Manutius’ preface to his collection of Greek grammatical treatises of 1496. He supported his observation that Greeks, especially their poets, used words like wax by the authority of the ancient Roman poet Martial (ca. \textsc{ad} 40–103), who, as he unsuccessfully tried to fit the Greek name Eiarinos into his Latin verses, mused:

\begin{quote}
And yet poets say \textit{Eiarinos}; {\textbar} but they are Greeks to whom nothing is denied, {\textbar} whom it beseems to chant \textit{Āres}, \textit{Ăres}. {\textbar} We, who cultivate more austere Muses, {\textbar} cannot be so clever.\footnote{Martial, \textit{Epigrammata} 9.12.10–14, here quoted in the English translation of the Loeb series. The original Latin verses are: “dicunt Eiarinon tamen poetae, {\textbar} sed Graeci, quibus est nihil negatum {\textbar} et quos $\text{\textgreek{>~A}}\rho \varepsilon \varsigma $ $\text{\textgreek{>'A}}\rho \varepsilon \varsigma $ decet sonare: {\textbar} nobis non licet esse tam disertis, {\textbar} qui Musas colimus severiores”. See Manutius (1496: *.ii\textsc{\textsuperscript{v}}) and e.g. also Énoch (1555: 187\textsc{\textsuperscript{r}}).}
\end{quote}

Robertus Vuidius (1569: 146\textsc{\textsuperscript{v}}–148\textsc{\textsuperscript{v}}) was, however, the first Hellenist to provide a systematic synopsis of the linguistic features of the poetical dialect, mainly consisting – as with the other dialects – in permutations of letters. Its inclusion among the dialects was encouraged by the fact that the problematic language of Greek poetry, apart from being dialectally mixed, also had certain formal characteristics of its own that could not be ascribed to the traditional four or five dialects. Yet Vuidius did not problematize the fact that the texts transmitted in the Aeolic and Doric dialects were almost exclusively poetical in nature. His vague definition of \textit{dialectus} as “particularity of tongue” allowed him to apply the term also to the manner of speaking characteristic of poets (\citealt{Vuidius1569}: 138\textsc{\textsuperscript{r}}–138\textsc{\textsuperscript{v}}). The Greek poetical dialect, an innovation of the 1560s that might strike modern readers as rather odd, became a highly successful construct and was present in numerous later classifications of the Greek dialects.\footnote{See e.g. Dabercusius (1577: \textsc{x}.1\textsc{\textsuperscript{v}}); Camden (1595: \textsc{i.1}\textsc{\textsuperscript{v}}); Köber (1701 [1684]: 376–377); Petisco (1764 [1759]: 113). Cf. also \sectref{sec:key:8} below.} The concept of \textsc{poetical} \textsc{dialect} was gradually applied to varieties of other languages too, as the case of Alexander Gill has already demonstrated. This was frequent especially in the eighteenth century, when philology was much less restricted to the classical languages than it had been in earlier times.\footnote{See e.g. \citet[101]{Hickes1705} on the \textit{dialectus} \textit{poetica} \textit{Dano-Saxonica}; [Verwer] (1707: *.3\textsc{\textsuperscript{r}}) for Dutch; \citet[24]{Wesley1736} and \citet{Vogel1764} for Hebrew; Beattie (1778: 240, 241) for Latin and French; \citet[292]{MacNicol1779} for English.}

The existence of a poetical dialect was, however, not accepted by everyone, and some Hellenists discarded it in the later seventeenth and eighteenth centuries. Critical voices were primarily heard in the Holy Roman Empire, where Greek studies never ceased to flourish in the early modern period. A late seventeenth-century grammarian warned his readers not to forge any new dialects, arguing that it was better not to refer to poetical license in terms of “dialect” \citep[512]{Ursin1691}. Almost a century later, the existence of a poetical dialect was even rejected as absurd by the German Hellenist and pedagogue Friedrich Gedike: “The grammarians speak almost unanimously of an additional fifth dialect, namely a specific poetical dialect. Yet this division brings little honor to their judgment”.\footnote{\citet[21]{Gedike1782}: “Die Grammatiker reden fast insgesammt noch von einem 5ten Dialekt, nehmlich einem besondern poetischen. Allein diese Eintheilung macht ihrer Beurtheilungskraft wenig Ehre”.} Poets introduced linguistic particularities for metrical reasons, Gedike suggested, and it was rather the case that they mixed dialects than that they had one of their own. Another late eighteenth-century German Hellenist, Johann Friedrich Facius, supported his rebuttal of the poetical dialect by means of his definition of \textit{dialectus}:

\begin{quote}
Besides, the poetic dialect, usually added to these four dialects, cannot be called a dialect properly speaking, as it rather is a certain kind of speech not proper to a nation, but to a certain class of writers only.\footnote{Facius (1782: \textsc{v}): “Quae praeterea his quattuor dialectis uulgo additur \textit{poetica}, proprie dialectus dici nequit, cum dictionis potius sit quoddam genus, non genti, sed scriptorum tantum ordini cuidam proprium”.}
\end{quote}

Facius used \textit{dialectus}, understood here as a variety of a language particular to a certain people, to deny dialect status to the amalgam of linguistic particularities restricted to poets.\footnote{Cf. also \citet[67]{Haas1780}: “Manche machen die Freyheit, deren sich die Poeten in ihren Versen bedienen, zu einem Dialekt, und nennen ihn \textit{dialectum} \textit{poet}. Ein solcher Dialekt aber setzet eine poetische Stadt oder Landschaft voraus”.} It was, he maintained, nevertheless justified to discuss poetical particularities together with the dialects, since poetical forms were so common that knowledge of them was indispensable for reading Greek poets.\footnote{\citet[98]{Facius1782}. Other critical voices include Bolius \& Alberti (1689 [1647]: \textsc{a.3}\textsc{\textsuperscript{v}}); Thryllitsch (1709: \textsc{d.2}\textsc{\textsuperscript{v}}), where \citet[147]{Reyher1634} is reproached for adding the poetic dialect to his classification of the Greek dialects; Walch (1772: 136–167).} Likely the best-informed solution to the problem of the poetical dialect was proposed by the theologian Christian Siegmund Georgi (1702–1771), who was also active in the Holy Roman Empire and who specialized in the language of the Greek New Testament. Georgi pointed out that it depended on one’s interpretation of the polysemous term \textit{dialectus} whether one could speak of a “poetical dialect”. If one interpreted the word as “style”, as Aristotle had done, then this was justified. If one understood it as a “variety of one and the same language”, it was most certainly not \citep[169]{Georgi1733}. Remarkably enough, scholars dismissing the existence of the poetical dialect as a rule did not mention any colleagues by name. As a result, the debate on this topic was always indirect.

In conclusion, at the end of the early modern period, many scholars realized that the introduction of a poetical dialect into classifications of the Greek dialects had been a severe setback; it had no historical \textit{raison} \textit{d’être}. In fact, the poetical dialect owed its existence largely to a process of simplification for philological and didactic purposes. It was designed as a unitary rubric for a diverse range of linguistic phenomena with which every would-be Hellenist had to deal in his study of Greek poetry. With the benefit of hindsight, we now know that these poetical particularities have multiple origins. Apart from various dialectal features, they also include archaic, metric, and stylistic properties. The invention of the poetical dialect and the subsequent discussion of its historical validity resulted from the ambiguous polysemy of the word \textit{dialect(us}), as it could signify in very general terms “style” and “manner of speaking”, but also “variety of a language and particular to a certain region and people”. The dismissal of the poetical dialect suggests that the latter meaning eventually prevailed in the eyes of many Hellenists of the period.

Another early modern solution to the problematic status of poetical Greek was more pragmatic. Several grammarians stated that Greek poets intermingled all the dialects, even though they usually made primary use of only one of them. This mixing was variously explained. Some grammarians suggested that the mixture was the result of the poets’ intense traveling across Greece or simply a conscious choice of the poets in order to have more possibilities in versification.\footnote{For a general reflection on the causes of the poets’ mixed usage, see e.g. Gottleber (1765: *.3\textsc{\textsuperscript{r}}–*.4\textsc{\textsuperscript{r}}). The mixed poetic variety was sometimes explicitly identified with the poetic dialect; see e.g. \citet[111]{Bayly1756} and \citet[198]{Peternader1776}.} These ideas were inspired by dominant views on the nature of Homer’s Greek, which I will discuss in the next chapter.

\subsection{Adapting traditional classifications}
\hypertarget{Toc19704814}{}
The invention of a poetical dialect necessarily led to the emergence of new classifications of the ancient Greek dialects, which were, in fact, as a rule adaptations and extensions of the two traditional Greek ones. \figref{fig:key:2} offers by way of demonstration an overview of the most important new classifications, ordered chronologically according to their first appearance. It would be straying too far from the central topic of this book to tease out the details of all these early modern classifications. I will instead focus on the most noteworthy innovations. Looking at the table, one is immediately struck by the fact that early modern scholars augmented traditional classifications by adding newly created dialects – most importantly the poetical and Hebraizing dialects – as well as dialects that had already been recognized in antiquity, but had not yet been canonized despite their being employed in literature – principally Boeotian, the variety in which the enigmatic poetess Corinna composed her verses. In some cases, lesser-known tongues from the margins of Greece, such as Phrygian and Macedonian, were also included among the canonical dialects, either because the scholar had only little acquaintance with the Greek dialects or because he proposed a particular interpretation of the historical status of the koine.\footnote{See e.g. \citet[131]{Kircher1679} for Phrygian, which exemplifies the former reason, and Schwartz \& Helm (1702: \textsc{c.2}\textsc{\textsuperscript{v}}) for Macedonian, an instance of the latter reason. Cf. also Chapter 5, \sectref{sec:key:4.}} The total number of Greek dialects mentioned varied from author to author and was impressively high in the \textit{Polyglot} \textit{thesaurus} of the Stuttgart-born scholar Hieronymus Megiser (ca. 1554/1555–1618/1619), who seems to have attributed a dialect to each Graecophone region or city he knew (1603: )(.7\textsc{\textsuperscript{r-v}}). One Hellenist originating from Taranto in southern Italy even hyperbolically suggested 600 as the number of ancient Greek dialects, even though it is more likely that he used the Latin numeral \textit{sescenti} in the metonymic sense of “innumerable” rather than in its literal meaning of “six hundred” \citep[9]{Giovane1589}. There were indeed other scholars who emphasized the sheer innumerability of the Greek dialects (e.g. \citealt{Bischoff1708}: 127; \citealt{Ries1786} [1782]: 196).

\begin{stylecaption}
Figure \stepcounter{Figure}{\theFigure}: The principal new classifications of the ancient Greek dialects
\end{stylecaption}

\textit{T4} \textit{refers} \textit{to} \textit{the} \textit{traditional} \textit{four} \textit{dialects} \textit{Aeolic,} \textit{Attic,} \textit{Doric,} \textit{and} \textit{Ionic.} \textit{T5} \textit{includes} \textit{all} \textit{of} \textit{these} \textit{and} \textit{the} \textit{koine.}

\tablefirsthead{}

\tabletail{}
\tablelasttail{}
\begin{tabularx}{\textwidth}{XXX}
\lsptoprule

 \textbf{\#} & \textbf{Classification} & \textbf{Example(s)}\\
 5 & T4 \& poetical & Vuidius (1569: 137\textsc{\textsuperscript{v}}); Peternader (1776: 193–198)\\
 6 & T5 \& poetical [+ secondary] & Dabercusius (1577: \textsc{x.1}\textsc{\textsuperscript{r}}\textsc{–x.1}\textsc{\textsuperscript{v}}); \citet[334]{Alsted1630}; \citet[64]{Bregius1684}\\
 6 & 4 proper (= T4) \& 2 less proper (koine + poetical) & Baile (1588: 3\textsc{\textsuperscript{r}}); \citet[4]{Schmidt1604}\\
 7 & 5 (T4 + Boeotian) \& 2 (poetical + Hebraizing) & Pasor (1632: 1–2); \citet[3]{Wyss1650}\\
 5 & T4 \& poetical [+ less principal] & \citet[302]{Opitz1687}; Giraudeau (1739: 100–101)\\
 6 & T4, Boeotian \& poetical & \citet[48]{Wright1691}; \citet[121]{Holmes1735}\\
 3 & Attic, Doric \& Ionic & Busby (1696: 66–67); Maittaire (1706: i–ii)\\
\lspbottomrule
\end{tabularx}
Some scholars reduced the number of principal dialects instead of adding new ones. The English Hellenist Richard Busby (1606–1695) listed only three principal dialects: Attic, Doric, and Ionic. Boeotian was subsumed under Doric, just like Aeolic, although Busby (1696: 66–67) claimed Aeolic also shared features with Ionic. This inspired Michael Maittaire (1668–1747), a French-born pupil of Busby’s, to posit a tripartite division into Attic, Doric, and Ionic, which in turn influenced the views of, among others, Heinrich Ludolf Ahrens (1809–1881), generally regarded as the founding father of modern ancient Greek dialectology.\footnote{Maittaire (1706: i–ii). See e.g. Brekle \textit{et} \textit{al.} (1992–2005: \textsc{viii}.177); \citet[463]{Finkelberg2014}. Maittaire also influenced e.g. \citet[213]{Thompson1732}; \citet[162]{Gesner1774}; Harles (1778: \textsc{xxviii}). Pott (1974 [1884–1890]: 92) still praised Maittaire’s work.} Ahrens (1839–1843: \textsc{i.}1) followed Maittaire, for instance, in leaving out the koine from his dialect classification. At the same time, however, he curiously misinterpreted his predecessor’s division as being quadripartite (Attic, Ionic, Doric, Aeolic) rather than threefold (Attic, Ionic, Doric). Ahrens’ dependency on Maittaire indicates that the so-called founding father of Greek dialectology relied on earlier scholarship for a key aspect of his work; this suggests that his contribution to ancient Greek dialectology needs to be revaluated from a historical perspective, a task which, however, lies outside the scope of this book. In the eighteenth century, Michael Maittaire’s tripartition of the Greek dialects was dismissed by the German Hellenist Johann Friedrich Facius (1782: \textsc{iv–v}), who emphasized the peculiar character of the Aeolic dialect and clearly separated it again from Doric. Some scholars reduced the number of principal dialects even further, primarily for didactic reasons. The French classical scholar and pedagogue Tanneguy Le Fevre (1615–1672) posited two “dominant dialects” (\textit{dialectes} \textit{dominantes}) only, on which Greek courses should focus: Doric and Ionic. The reason Le Fevre (1731 [1672]: 61) provided was that Aeolic was too obscure and very rare in extant literature, and that the Attic dialect was remarkably close to common grammar and therefore did not require separate treatment.

\figref{fig:key:2} shows that the main early modern classifications of the ancient Greek dialects were generally much more detailed than their ancient and medieval sources of inspiration. The best example of this tendency is, however, found in an early eighteenth-century dissertation, presented in the city of Wittenberg on February 9, 1709, in which an idiosyncratic, geographically motivated division consisting of three hierarchical layers was proposed:

\begin{itemize}
\item there were four principal and primary dialects, spoken by an entire tribe – \textit{dialecti} \textit{primariae}, \textit{principales}, or \textit{ethnikaí} ($\text{\textgreek{>e}}\theta \nu \iota \kappa \alpha \text{\textgreek{'i}}$): Ionic, Attic, Doric, and Aeolic;

\item each primary dialect comprised several secondary, regional dialects – \textit{dialecti} \textit{secundariae} or \textit{egkhṓrioi} ($\text{\textgreek{>e}}\gamma \chi \text{\textgreek{'w}}\rho \iota o\iota $) – which emerged as a result of the geographical dispersion of the four principal tribes;

\item each secondary dialect comprised several city or local dialects – \textit{dialecti} \textit{urbicae} or \textit{topikaí} ($\tau o\pi \iota \kappa \alpha \text{\textgreek{'i}}$; \citealt{Thryllitsch1709}: \textsc{d.3}\textsc{\textsuperscript{r}}).

\end{itemize}

This can be seen as a further elaboration of the “principal”/“less principal” dichotomy I have discussed in \sectref{sec:key:6} above; it was, however, also partly inspired by Greek scholarship, as the author drew inspiration for his hierarchical division from Byzantine observations on varieties of Doric (\citealt{Thryllitsch1709}: \textsc{d.2}\textsc{\textsuperscript{v}}\textsc{–d.4}\textsc{\textsuperscript{r}}).

Despite the abundance of different classifications in the early modern period, reflections on the differences among them were rare. Some scholars did, however, feel the need to discuss the correct sequence in which the dialects should be named. The German Hellenist Johann Friedrich Facius (1782: \textsc{iv}) held that the order should be determined by the antiquity of each dialect, which brought him to the following sequence: Doric, Aeolic, Ionic, and Attic (cf. Chapter 5, \sectref{sec:key:4}). The alleged close kinship between Attic and Ionic, on the one hand, and Aeolic and Doric, on the other, an idea current since Strabo, led an eighteenth-century Hellenist from the Dutch Republic to arrange the four principal Greek dialects as Attic, Ionic, Doric, and Aeolic (\citealt{Koen1766}: \textsc{xxix}).

All classifications were to different degrees indebted to Greek scholarship. The type of division proposed greatly depended on a scholar’s aims and the context in which he discussed the Greek dialects. For instance, the so-called Hebraizing dialect was only included by authors interested in New Testament Greek, as I will demonstrate in Chapter 3. In other words, early modern classifications were partly based on the ancient Greek and Byzantine tradition, and partly on the interests of scholars, who tended to focus on Biblical and poetical Greek, both highly problematic varieties. Linguistic principles were largely out of the picture in designing Greek dialect classifications. Indeed, early modern scholars did not engage in systematic historical-comparative research into the relationships among the Greek dialects exclusively or primarily based on linguistic data. Instead, they resorted to ancient authorities and adhered to received views, adjusting them to their scholarly programs on the basis of a priori arguments. In doing so, they only rarely invoked actual linguistic evidence.\footnote{See also Chapter 5, where this generalization will be nuanced.}

\subsection{The koine, an eternal problem?}
\hypertarget{Toc19704815}{}
The sole difference between the two traditional Greek classifications was the absence or presence of the koine. This was a symptom of a larger problem, the inability of ancient and medieval Greek scholars to arrive at an adequate understanding of the historical position of the koine. How did early modern scholars conceive of the koine and its relationship to the Greek dialects? Did they likewise run into trouble when trying to grasp the precise status of this variety or were they more successful? As can be expected, traditional Greek insights persisted. \figref{fig:key:3} shows early modern examples of the way in which traditional Greek views on the koine (outlined in \sectref{sec:key:3} above) were adopted and adapted, usually silently. In some early cases, the koine was not mentioned at all, which might be explained by its absence in the widely read account of Strabo (see e.g. \citealt{Stapleton1566}: 58\textsc{\textsuperscript{v}}–59\textsc{\textsuperscript{r}}).

\begin{stylecaption}
Figure \stepcounter{Figure}{\theFigure}: Early modern uses of traditional Greek views on the koine
\end{stylecaption}

\textit{T4} \textit{refers} \textit{to} \textit{the} \textit{traditional} \textit{four} \textit{dialects} \textit{Aeolic,} \textit{Attic,} \textit{Doric,} \textit{and} \textit{Ionic.}

\tablefirsthead{}

\tabletail{}
\tablelasttail{}
\begin{tabularx}{\textwidth}{XXX}
\lsptoprule
\hhline{~--}
\multicolumn{1}{X}{} & \textbf{Traditional} \textbf{Greek} \textbf{view} & \textbf{Early} \textbf{modern} \textbf{example(s)} \textbf{of} \textbf{adoption} \textbf{and} \textbf{adaptation}\\
 \REF{ex:key:1} & T4 derived from the koine. & Borghini (1971 [before 1580]: 335): “[The koine] was common to all men of that nation and as it were the principal fundament of that language. Subsequently, […] it was divided into four other tongues, which in reality were not languages, but dialects”.

“fu comune a tutti di quella nazione, e come fondamento principale di quella lingua. Di poi, […] si divise in quattro altre, le quali in verità non furono lingue ma \textit{dialetti}”.

See e.g. also \citet[209]{Vergara1537}.\\
 \REF{ex:key:2} & The koine was mixed out of T4 and embraced them as a mother. It therefore consisted of the common properties of the dialects. & \citet[52]{Oecolampadius1518}: “Koine, i.e. common, is collected out of the other dialects and is commonly used by authors”.

“$Ko\iota \nu \text{\textgreek{'h}}$, id est communis, collecticia est ex ceteris, qua scriptores communiter utuntur”.

See e.g. also Girard (1541: 10\textsc{\textsuperscript{r}}\textsc{–10}\textsc{\textsuperscript{v}}). Henri Estienne (1581: 28–34) was exceptional in attempting to substantiate this view with extensive linguistic evidence.\\
 \REF{ex:key:3} & The koine was the subject of grammar and characterized by rules, whereas T4 were variations on it. & The koine’s grammatical and analogical status was accepted as a given by most early modern grammarians (e.g. \citealt{Gaza1495}: α.1\textsc{\textsuperscript{v}}; \citealt{Schmidt1604}: 4; \citealt{Walch1772}: 137), who presented the dialects as deviations from it (cf. \citealt{Ciccolella2008}: 123), even though sometimes a special place was accorded to Attic, especially in the eighteenth century (e.g. \citealt{Luscinius1517}; \citealt{Hemsterhuis1721}: 68; \citealt{Jehne1782}: 288).\\
 \REF{ex:key:4} & The koine was used commonly by all. & Melanchthon (1518: a.i\textsc{\textsuperscript{v}}): “The speech that is common to all is called common language”.

“Qui sermo communis omnibus est, lingua communis dicitur”.\\
 \REF{ex:key:5} & The koine was a fifth dialect (without further explanation). & Beroaldo (1493: 138\textsc{\textsuperscript{v}}): “For the Greeks have five tongues: Ionic, Doric, Attic, Aeolic, and common”.

“nam linguas quinque habent: Ionicam, Doricam, Atthicam [\textit{sic}], Aeolicam et communem”.

See e.g. also Perotti (1489: 85\textsc{\textsuperscript{r}}).\\
\lspbottomrule
\end{tabularx}
Did early modern scholars develop original solutions to the problematic position of the koine too? The answer to this question should be a clear yes. I argue for several reasons that the issue was problematized to a higher degree by early modern Hellenists than by their predecessors. This was for a large part due to evolutions in vernacular language studies during the Renaissance. Grammarians started to develop a norm for these tongues that could rival Latin as a valid and elegant means of communication (see e.g. \citealt{Giard1992}). This fostered the opposition of the normative variety, usually termed “common language”, to the other varieties, the “dialects”, a contrast becoming ever more emphatic in the course of the early modern period. This had enormous repercussions for conceptions of the Greek koine. For instance, a late seventeenth-century Hellenist and grammarian asked himself “whether the koine is likewise to be reckoned as a fifth dialect among those mentioned before”, to which he offered the following straightforward answer:

\begin{quote}
It is not, as this is not so much a dialect as the basis, and like a mother of, the dialects, to which these dialects belong as to a genus its species. It should therefore be designated with the term language rather than dialect.\footnote{\citet[495]{Ursin1691}: “\textit{Annon} \textit{et} \textit{quinta} \textit{dialectus} \textit{communis} \textit{prioribus} \textit{illis} \textit{accensenda} \textit{est?} Non est: quippe haec non tam dialectus, quam dialectorum basis et ueluti mater est, cui hae ut generi species suae accidunt; linguae igitur potius quam dialecti nomine appellanda”.}
\end{quote}

Furthermore, the relationship between the koine and the traditional four dialects was specified in much more detail by early modern scholars. The French grammarian Petrus Antesignanus was convinced that the koine was a variety consisting of the best features of the principal Greek dialects and mainly Attic.\footnote{Antesignanus (1554: 12–13), on which see Van Rooy (2016c: 130–131).} A number of scholars considered the koine a common language based on Attic, thus coming close to the historical truth. This position is adopted, for example, by the Westphalian orientalist Hermann von der Hardt (1660–1746; 1705 [1699]: 17–18). Occasionally, Attic was even identified with the koine, in which case Attic was further specified as a more recent variety of Attic (\citealt{Georgi1733}: 3, 5). Others regarded the koine as the most important prose dialect next to Attic and as opposed to the other dialects principally employed in poetry, which were only to be tackled by brilliant minds and good students (\citealt{Vives1531}: 97\textsc{\textsuperscript{v}}; cf. \citealt{Vuidius1569}: 137\textsc{\textsuperscript{v}}).

There were contradictory opinions about the social status of the koine. Most scholars understood it as a literary variety, and some even explicitly associated it with the higher classes. For instance, the renowned English philologist Richard Bentley (1662–1742), known among other things for his restoration of the digamma in Homer’s epic poems, described the koine as “perfectly a language of the learned, almost as the Latin is now”, emphasizing that it “was never at any time or in any place the popular idiom” (1699: 406). Others advanced exactly the opposite view and associated the koine with lower social classes, among other reasons because it was much easier to learn than the literary dialects. One author, the seventeenth-century French lexicographer of medieval Greek Charles Du Cange (1610–1688), even claimed that reading too many koine books could defile one’s speech (1688: iv). Du Cange mentioned this when discussing the causes of the corruption of the Greek language.

Claude de Saumaise (1643a: esp. 405–406) launched a highly influential interpretation of the koine, claiming that it started out as a dialect historically, specifically that it was a variety peculiar to the people of Thessaly (see also Chapter 5, \sectref{sec:key:4}). It was named Hellenic after Hellen, the forbear of the Greeks. Additionally, there was a city called Hellas in Thessaly. Saumaise was inspired to do so by the idiosyncratic classification propounded by Heraclides Criticus, which included a Hellenic dialect (see \sectref{sec:key:1} above). This Hellenic dialect, also called Thessalian and Macedonian, after giving birth to the four other Greek dialects, developed into a high-end common variety, employed by writers and no longer particular to a specific people, a criterion Saumaise deemed indispensable in order to speak of a “dialect”. A dissertation presented in Wittenberg in 1702, which elaborated on Saumaise’s framework, went as far as claiming that the original Hellenic language – Saumaise’s Thessalian-Hellenic-Macedonian dialect – was extinct, and that scholars later had collected common elements from the surviving dialects and had formed by means of analogy a new common language out of these (\citealt{SchwartzHelm1702}: \textsc{c}.1\textsc{\textsuperscript{v}}; see Van Rooy fc. e).

The problematic status of the koine even led some scholars to doubt and indeed dismiss its very right to existence. This occurred in two Wittenberg dissertations of 1709 to which the obscure German classical scholar Georg Friedrich Thryllitsch (1688–1715) contributed, each time as \textit{respondens} and at least once as the sole author. In the first, a historiographical dissertation on the Greek dialects, Thryllitsch (1709: \textsc{d.1}\textsc{\textsuperscript{v}}) argued that there was no irrefutable proof that something like the koine actually existed. This view was expressed even more emphatically in the second dissertation, which was entirely devoted to the issue of the koine. In it, Thryllitsch, possibly together with the Wittenberg professor of Greek Georg Wilhelm Kirchmaier (1673–1759), tried to convince the reader

\begin{quote}
that there was no Alexandrian dialect, except for a secondary one which proceeded from Attic, that the Macedonian dialect was semi-barbarous and a daughter of Doric rather than Attic, that the Hellenic dialect had already become extinct before or certainly during Homer’s time, and that the common dialect, in truth, was a dream of learned men.\footnote{Kirchmaier \& Thryllitsch (1709: \textsc{c.2}\textsc{\textsuperscript{v}}): “quod Alexandrina non nisi secundaria eque Attica emanans dialectus, Macedonica semibarbara et Doricae potius quam Atticae filia, Hellenica iam ante, aut cum Homero certe abolita, communis uero doctorum hominum somnium fuerit”.}
\end{quote}

The Greek koine, it was argued, did not exist before grammars of Greek started to be composed around Plutarch’s time (\citealt{KirchmaierThryllitsch1709}: \textsc{a.4}\textsc{\textsuperscript{v}}). In fact, the koine was forged so as to account for the presence of common forms in the Greek dialects and to distill grammatical analogy from them.

In conclusion, there was much uncertainty among early modern scholars about the position of the koine vis-à-vis other varieties of Greek and within the general history of the Greek language. This had a double cause. On the one hand, although proposing sometimes highly original answers to the question, scholars always expressed a priori views. In doing so, they were usually in some way or another inspired by Greek ideas, both common and unusual ones. On the other hand, scholars often lacked the necessary historical insight into the genesis of the koine and the Greek dialects. In fact, when trying to sketch the history of the Greek dialects – a theme further developed in Chapter 5 – they usually depended on Strabo’s account, in which, however, the koine did not figure. Still, setting the koine clearly apart from the dialects can be viewed as a major achievement of early modern scholarship, especially since earlier Greek conceptions of it were so blurred.

\subsection{Test case: Classifying vernacular Greek dialects}
\hypertarget{Toc19704816}{}
Inspired by the Greek heritage, early modern scholars offered relatively rigid classifications of the ancient Greek dialects, even though they were somewhat more flexible than their predecessors. Yet how did they map out \textit{vernacular} Greek variation, which they usually distinguished clearly from the ancient dialects?\footnote{Some scholars did, however, regard vernacular Greek as a dialect of Greek. See e.g. Megiser (1603: )(.7\textsc{\textsuperscript{v}}) on “vulgar or new Greek, or Barbaric Greek [Graeca uulgaris, seu noua, uel Barbarograeca]”. Early modern attitudes toward the vernacular Greek language require a more thorough investigation. See, however, already \citet{Caratzas1952}, Rotolo (1973–1974), and \citet{Toufexis2005}.} On what principles did they rely in doing so? Here, their approach was radically different, as interest in this issue was not primarily triggered by philological or historiographical concerns. Rather, early modern scholars only treated vernacular variation when they took a genuine interest in contemporary Greece and its inhabitants, an interest which grew rather slowly. Symptomatically, the earliest rudimentary estimation of the number of vernacular dialects came from the pen of a learned Greek correspondent of the German Philhellene Martin Crusius (1526–1607), Symeon Cabasilas (1546–after 1605), who stated that “there are many different dialects, more than seventy”, of which “that of the Athenians is the worst”.\footnote{Cabasilas in \citet[461]{Crusius1584}: “$\Pi \varepsilon \rho \text{\textgreek{`i}}$ $\delta \text{\textgreek{`e}}$ $\tau \text{\textgreek{~w}}\nu $ $\delta \iota \alpha \lambda \text{\textgreek{'e}}\kappa \tau \omega \nu $, $\tau \text{\textgreek{'i}}$ $\text{\textgreek{>`a}}\nu $ $\kappa \alpha \text{\textgreek{`i}}$ $\text{\textgreek{>'e}}\pi o\iota \mu \iota $ [\textit{sic}]$\text{\textgreek{?}}$ $\Pi o\lambda \lambda \text{\textgreek{~w}}\nu $ $o\text{\textgreek{>u}}\sigma \text{\textgreek{~w}}\nu $, $\kappa \alpha \text{\textgreek{`i}}$ $\delta \iota \alpha \varphi \text{\textgreek{'o}}\rho \omega \nu $, $\text{\textgreek{<u}}\pi \text{\textgreek{`e}}\rho $ $\tau \text{\textgreek{~w}}\nu $ $\text{\textgreek{<e}}\beta \delta o\mu \text{\textgreek{'h}}\kappa o\nu \tau \alpha \text{\textgreek{?}}$ $To\text{\textgreek{'u}}\tau \omega \nu $ δ' $\text{\textgreek{<a}}\pi \alpha \sigma \text{\textgreek{~w}}\nu $, $\text{\textgreek{<h}}$ $\tau \text{\textgreek{~w}}\nu $ $\text{\textgreek{>A}}\theta \eta \nu \alpha \text{\textgreek{'i}}\omega \nu $ $\chi \varepsilon \iota \rho \text{\textgreek{'i}}\sigma \tau \eta $”. In the Latin translation printed together with the original letter, the adverb \textit{fortassis}, ‘perhaps’, is added before “amplius septuaginta”. Perhaps Crusius wanted to mitigate Cabasilas’ claim.} A couple of decades earlier, the Swiss language cataloguer Conrad Gessner (1516–1565) had limited himself to mentioning some vernacular Greek dialects, for instance those of Crete, Cyprus, and the Peloponnese, without classifying them (1555: 47\textsc{\textsuperscript{r}}).

In the early eighteenth century, a number of Western scholars tried to make sense of vernacular Greek and its varieties. The most extensive classification of vernacular dialects was offered by the German academic Johann Tribbechow (1677–1712) in his dissertation on the emergence and nature of vernacular Greek, prefixed to his grammar of that language, published in Jena in 1705. In this interesting text, Tribbechow (1705: a.4\textsc{\textsuperscript{r}}\textsc{–}a.4\textsc{\textsuperscript{v}}) proposed a division into an insular and a continental class of dialects, to which he added, somewhat hesitatingly, the dialect of Constantinople as a third class. It is clear that his classification was principally inspired by geopolitical factors. The geographical contrast between the Greek mainland and the islands was transferred to the linguistic plane. The introduction of the Constantinopolitan dialect into the division was politically motivated, as the city was the seat of the patriarchate and the heart of the Ottoman empire at that time. Despite the presence of speakers of all vernacular Greek dialects in this city, Tribbechow still presented the speech of Constantinople as the purest and best variety. It allegedly shared its purity with the continental tongues of Thessaloniki, the Peloponnese, and the rest of mainland Greece, especially that of the city of Ioannina. There, vernacular speech had remained pure, because of the intensive cultivation of the erudite ancient language by its inhabitants and because of its geographical isolation. Interestingly, Tribbechow claimed to have verified his views with Greek youngsters studying at the Halle Oriental Theological College, with which he was associated.\footnote{On the Greek students at the \textit{Collegium} \textit{Orientale} \textit{Theologicum}, see \citet{Moennig1998} and \citet[283]{Makrides2006}.}

Tribbechow’s division into insular and continental dialects was apparently picked up by a Greek émigré born in Larissa and mainly active in England, Germany, and Russia during his adult life, Alexander Helladius (1686–after \citealt{September1715}).\footnote{See \citet{Helladius1714}, Moennig (1998: 315–317), and Van Rooy (fc. a) for (auto)biographical information.} In a 1714 book published in Altdorf and devoted to the contemporary state of Greece and the Greek church, Helladius politicized the contrast between insular and continental dialects by stressing that the islands were Venetian-occupied, whereas the mainland was Ottoman-occupied. Remarkably enough, he linked this to a linguistic criterion: lexical evidence.\footnote{Before Helladius, other Greek scholars had already tried to offer rudimentary groupings of vernacular Greek dialects based on linguistic evidence such as case variation resulting from the loss of the dative: see Kritopoulos ([1924/]1926 [1627]: 108) and Nikiforos (1908 [writing ca. 1650]: e.g. 1 \& 8) for some interesting but isolated observations.} The insular Italo-Greeks (\textit{insulani}/\textit{Italo-Graeci}), Helladius argued, had many words not used by mainlanders (\textit{continentem} \textit{inhabitantes}). This must be read in close connection with the geopolitical opposition between the islands and the continent; mainlanders used more Turkish words, which they borrowed from their Ottoman occupiers, whereas the insular Greeks under Venetian rule introduced many Italian words into their speech (\citealt{Helladius1714}: 190–191, 194, 203). To exemplify the confusion caused by this vernacular variation, Helladius recounted several amusing anecdotes from his own life (see Van Rooy fc. a).

\subsection{Conclusion}
\hypertarget{Toc19704817}{}
Contrary to ancient and medieval scholars, who roughly agreed on one classification – Attic, Ionic, Aeolic, and Doric with or without the koine – early modern Hellenists proposed a number of diverging classifications in their attempts at creating order in the chaos of Greek linguistic variation. This occurred most importantly in the context of philology and, to a lesser extent, historiography. The principles underlying their dialect classifications were almost without exception of a non-linguistic nature. Instead, they were informed by cultural, historical, and philological circumstances and by the authority of ancient, Byzantine, and early modern scholars. This shows that the study of the ancient Greek dialects was culturally embedded in various ways before the rise of ancient Greek dialectology in modern times, generally connected to the work of Ahrens. It can be stressed in this regard that nineteenth-century scholars did not create ancient Greek dialectology \textit{ex} \textit{nihilo} (cf. \citealt{Colvin2007}: 22). Indeed, Ahrens and his colleagues elaborated on the achievements of their early modern predecessors, such as Michael Maittaire. For this reason, it is necessary to frame their contribution within earlier scholarship, a task awaiting completion. Major innovations of early modern scholarship include the philologically inspired distinction between primary and secondary dialects, the odd introduction and eventual dismissal of a poetical dialect, and the clear setting apart of the koine from the Greek dialects, even though the exact nature and history of the koine was often poorly understood.

The careful early modern attention for the classifications of the canonical ancient Greek dialects contrasts sharply with the lack of interest in grouping vernacular Greek varieties. In the early eighteenth century, however, certain scholars active in Germany did attempt to make general distinctions in vernacular Greek dialectal variation, most importantly by setting up the categories of island and mainland Greek. In doing so, they relied on geographical, political as well as linguistic principles, no doubt because in this case they were not as bound by an authoritative tradition as philologists had been when discussing the revered ancient dialects.

\section{\textsc{3} A true man of letters: Greek dialects and philology}
\hypertarget{Toc19704818}{}\begin{quote}
The Greek language ruled and held great sway through its four principal dialects, which perfect each other in such a manner that no one should rightly be reckoned to be versed in any of them if he has not mastered them all.\footnote{Schröder (1748: 53–54): “Lingua Graeca per quattuor praecipuas dialectos regnauit et amplissimam habuit ditionem, quarum una alteram ita perficit, ut in nulla recte callere censendus sit, qui non omnes fuerit complexus”.}
\end{quote}

The Marburg-born professor Nicolaus Wilhelm Schröder (1721–1798) made this point in 1748 when pronouncing an oration on how to acquire a thorough knowledge of the Hebrew language. Schröder did so in his capacity as professor of Oriental tongues and Greek in Groningen, in the Dutch Republic, while arguing that a comparative approach to Hebrew, Aramaic, Syriac, and Arabic was warranted since they were as closely cognate as the Greek dialects. However noteworthy this methodological consideration on these so-called Oriental tongues may be, I am more interested here in Schröder’s suggestion that mastering the main dialects of Greek appears to have been a requirement to be considered a true Hellenist in the early modern period. Schröder implied that it was otherwise impossible to correctly understand the Greek language, literature, and culture, a view with which most of his early modern colleagues no doubt agreed. The primary motivation to study the dialects was in other words philological in nature, much as it had been in ancient and Byzantine Greece. What main philological needs should knowledge of the dialects fulfill in the eyes of early modern scholars?

\subsection{The basic motivation: Reading Greek poetry}
\hypertarget{Toc19704819}{}
Enabling students to read difficult literary texts from Greek antiquity was the basic motivation for Hellenists to reflect on the dialects and their linguistic features, to which the countless early modern manuals for the Greek dialects bear witness. Sometimes the authors of these handbooks made their goals and readership explicit. The Swiss doctor and Hellenist Martin Ruland (1532–1602) believed that his manual was to be of great use to students of good literature such as Godfrid Seiler, one of the two people to which his handbook was dedicated, and “to other youngsters who likewise just now engage in Greek or also Roman learning”.\footnote{Ruland (1556: α.4\textsc{\textsuperscript{v}}): “Tibi itaque mi carissime ac bonarum litterarum studiosissime Godfrid $\kappa \alpha \text{\textgreek{`i}}$ $\text{\textgreek{>'a}}\lambda \lambda o\iota \varsigma $ $\tau o\text{\textgreek{~i}}\varsigma $ $\mu \varepsilon \iota \rho \alpha \kappa \text{\textgreek{'i}}o\iota \varsigma $ $\kappa \alpha \text{\textgreek{`i}}$ $\nu \varepsilon \omega \sigma \tau \text{\textgreek{`i}}$ $\tau o\text{\textgreek{~u}}$ $\mu \alpha \theta \text{\textgreek{'h}}\mu \alpha \tau o\varsigma $ $\tau o\text{\textgreek{~u}}$ $\text{\textgreek{<E}}$$\lambda \lambda \eta \nu \iota \kappa o\text{\textgreek{~u}}$ $\text{\textgreek{>`h}}$ $\kappa \alpha \text{\textgreek{`i}}$ $\tau o\text{\textgreek{~u}}$ $\text{\textgreek{<R}}$$\omega \mu \alpha \text{\textgreek{"i}}\kappa o\text{\textgreek{~u}}$ $\text{\textgreek{<a}}\pi \tau o\mu \text{\textgreek{'e}}\nu o\iota \varsigma $ profuturum hoc tibi magno labore elaboratum opusculum puto”.} Ruland moreover alluded to the widespread humanist idea that knowledge of the Greek language and literature was indispensable to understand ancient Latin literature.\footnote{Cf. Ben-\citet[139]{Tov2009} for Melanchthon’s expression of this idea. See also Chapter 2, section 6.} The intended readership of manuals for the Greek dialects was frequently a specific group of students, showing that they often catered to very local markets and highly specific audiences. One French handbook of 1588 was, for instance, directed to the youth of Aquitaine (\citealt{Baile1588}: \textsc{a.2}\textsc{\textsuperscript{v}}), whereas a German one published a year later was aimed at the students of the Academy of Marburg \citep{Walper1589}.

The fact that such handbooks started to appear might suggest that these works were considered the primary gateway to mastery of the Greek dialects. Yet was this really the case? A powerful voice in this debate was that of the German Jesuit Jakob Gretser (1562–1625), the author of the standard Greek grammar for Jesuit colleges. Gretser believed that a grammatical work on the Greek dialects was not so much a handbook to be studied in isolation as it was a didactic instrument to which a teacher should refer the student for more information when reading poets (\citealt{Gretser1593}: )(.5\textsc{\textsuperscript{v}}–)(.6\textsc{\textsuperscript{r}}). As a matter of fact, Gretser even contended that “the dialects [can]not be learned better and more easily than by reading poets and others who have inserted in their written works idioms of this kind”.\footnote{Gretser (1593: )(.5\textsc{\textsuperscript{v}}): “pro certo habendum sit, dialectos melius et expeditius non disci, quam lectione poetarum aliorumque, qui suis monumentis huiuscemodi idiomata inseruerunt”.} Gretser’s case thus also demonstrates that the main focus of attention of early modern scholars interested in the Greek dialects was on the reading of poetry. The invention of a poetical dialect should also be viewed in this context.\footnote{See Chapter 2, \sectref{sec:key:6}, especially with regard to Manutius, and \sectref{sec:key:7.}} There were, in fact, only a few Hellenists concentrating on the dialects as they appeared in other literary genres such as orations (see e.g. \citealt{Labbe1639}: 5–7 for an exception).

\subsection{The dialects for the advanced philologist}
\hypertarget{Toc19704820}{}
There were other incentives to pay extensive attention to the Greek dialects, which, however, principally belonged to the domain of the advanced and well-trained philologist. These ranged from etymology through to textual criticism and Neo-Paleo-Greek poetry composition. In this book I can only touch very briefly on this matter, mainly looking at it through the evidence found in manuals for the Greek dialects, as research on these often complex issues is still in its infancy. Interest in Neo-Paleo-Greek texts has, however, started to grow in recent years, so that this may well be considered a subfield of classical scholarship and reception studies (see e.g. \citealt{PällVolt2018}). With the term Neo-Paleo-Greek, which I have Anglicized from German \textit{Neualtgriechisch}, I refer to texts written in varieties of ancient Greek by scholars from the Renaissance and later. I prefer this designation to alternatives such as \textit{Humanist} \textit{Greek} for two main reasons. On the one hand, \textit{Neo-Paleo-Greek} does not carry any ideological connotations. On the other, it captures well the somewhat paradoxical nature of this exceptional type of writings, which served, among other things, to show off one’s erudition.

\subsubsection{Etymology}
\hypertarget{Toc19704821}{}
Dialect forms were viewed as a useful tool to arrive at the correct etymology of a word, an idea only marginally present in the Greek tradition.\footnote{For exceptions, see e.g. Proclus, \textit{In} \textit{Platonis} \textit{Cratylum} \textit{commentaria} 85, and Michael Psellos, \textit{Poemata} 6.187.} The renowned French Hellenist Henri Estienne (1528/1531–1598) demonstrated this by correctly deriving the Ionic noun \textit{apódexis} ($\text{\textgreek{>a}}\pi \text{\textgreek{'o}}\delta \varepsilon \xi \iota \varsigma $), ‘demonstration, exhibition’, from \textit{apodeíknumi} ($\text{\textgreek{>a}}\pi o\delta \varepsilon \text{\textgreek{'i}}\kappa \nu \upsilon \mu \iota $), ‘to show, to demonstrate’, rather than from \textit{apodékhomai} ($\text{\textgreek{>a}}\pi o\delta \text{\textgreek{'e}}\chi o\mu \alpha \iota $), ‘to accept, to receive’ (\citealt{Estienne1581}: 42–43). Typically for early modern scholarship, Estienne supported this view by invoking a letter change process; Ionic could drop the jota (<ι>) from the diphthong \textit{ei} (<$\varepsilon \iota $>) of the koine and the Attic dialect. A couple of decades after Estienne, his German colleague Erasmus Schmidt (1570–1637), professor of Greek in Wittenberg, emphasized the importance of being skilled in the dialects in order to comprehend certain morphological particularities of the Greek language in the dedicatory letter prefixed to his \textit{Treatise} \textit{on} \textit{the} \textit{principal} \textit{dialects} \textit{of} \textit{the} \textit{Greeks} of 1604 (\citealt{Schmidt1604}: ):(.3\textsc{\textsuperscript{v}}–):(.4\textsc{\textsuperscript{r}}). Schmidt exemplified this by means of the koine verbs \textit{klaíō} ($\kappa \lambda \alpha \text{\textgreek{'i}}\omega $, ‘to cry, to lament’) and \textit{kaíō} ($\kappa \alpha \text{\textgreek{'i}}\omega $, ‘to kindle, to burn’) and their respective future indicatives \textit{klaúsō} ($\kappa \lambda \alpha \text{\textgreek{'u}}\sigma \omega $) and \textit{kaúsō} ($\kappa \alpha \text{\textgreek{'u}}\sigma \omega $). He explained the presence of the letter upsilon (<υ>) in both forms by means of two dialect rules. Firstly, Attic dropped the jota in both verbs, and has \textit{kláō} ($\kappa \lambda \text{\textgreek{'a}}\omega $) and \textit{káō} ($\kappa \text{\textgreek{'a}}\omega $). Secondly, Aeolic changed alpha (<α>) into the diphthong alpha upsilon (<$\alpha \upsilon $>), resulting in the verb forms \textit{klaúō} ($\kappa \lambda \alpha \text{\textgreek{'u}}\omega $) and \textit{kaúō} ($\kappa \alpha \text{\textgreek{'u}}\omega $). These two Aeolic forms had a regular future ending in -\textit{aúsō} (-$\alpha \text{\textgreek{'u}}\sigma \omega $) and, though originally Aeolic, they were received into the koine. Needless to say, such etymological experimentation is not always corroborated by modern linguistics, as in Schmidt’s case.\textstyleFootnoteSymbol{} In fact, the diphthong in Attic and koine \textit{k(l)aúsō} (κ(λ)$\alpha \text{\textgreek{'u}}\sigma \omega $) reflects the original presence of a [u̯] sound at the end of the verbal root, normally lost in Attic and thus the koine. It was, in other words, not the result of Aeolic influence, as Schmidt suggested.

\subsubsection{Textual criticism}
\hypertarget{Toc19704822}{}
Mastering the dialects was vital not only for reading Greek literature and gaining better insight into the Greek language, but also for arriving at the correct version of Greek texts, transmitted for centuries by means of manuscript copies that left ample room for mistakes. Even more, as Reynolds \& Wilson (1991: 47–48) have pointed out, Byzantine scribes were often inclined to replace odd dialect forms by more familiar Attic or koine forms, making it impossible for later editors to ever establish a dialect text closely approaching the ancient original. The editorial utility of dialect knowledge was summed up neatly by the zealous editor of Greek texts Henri Estienne in the preface to his extensive commentary on the Attic dialect. There, Estienne drew the following conclusion, after refuting several textual corrections conjectured by various philologists, including the renowned Italian humanist Lorenzo Valla (ca. 1407–1457): “By all means, there is nobody who cannot observe from these examples how dangerous ignorance of the dialects is”.\footnote{Estienne (1573: ¶.iii\textsc{\textsuperscript{v}}): “Equidem uel ex his, quam periculosa sit dialectorum ignoratio, nemo est qui perspicere non possit”.} In this context, Estienne boasted of a correction of his in his edition of Plutarch’s \textit{Lycurgus} (20.2):

\begin{quote}
For in the \textit{Lycurgus}, Demaratus, asked by a certain vile man who was the best among the Spartans, answers “\textit{hóti} \textit{anomoiótatos}” [$\text{\textgreek{<'o}}\tau \iota $ $\text{\textgreek{>a}}\nu o\mu o\iota \text{\textgreek{'o}}\tau \alpha \tau o\varsigma $], as the editions prior to mine read, even though it does not make any sense. In fact, “\textit{ho} \textit{tìn} \textit{anomoiótatos}” [$\text{\textgreek{<o}}$ $\tau \text{\textgreek{`i}}\nu $ $\text{\textgreek{>a}}\nu o\mu o\iota \text{\textgreek{'o}}\tau \alpha \tau o\varsigma $] should be read (as is now written in my edition) with a very evident and suitable meaning, since Demaratus is answering “He who differs most from you”.\footnote{Estienne (1573: ¶.iii\textsc{\textsuperscript{v}}): “In Lycurgo enim Demaratus a quodam improbo homine interrogatus quis esset Spartiatarum optimus, respondet, $\text{\textgreek{<'o}}\tau \iota $ $\text{\textgreek{>a}}\nu o\mu o\iota \text{\textgreek{'o}}\tau \alpha \tau o\varsigma $, ut in editionibus mea prioribus legitur, quamuis nullo sensu; cum legendum sit, $\text{\textgreek{<o}}$ $\tau \text{\textgreek{`i}}\nu $ $\text{\textgreek{>a}}\nu o\mu o\iota \text{\textgreek{'o}}\tau \alpha \tau o\varsigma $ (ut nunc in mea scriptum est) sensu manifestissimo et conuenientissimo; cum respondeat Demaratus, Qui tibi est dissimillimus”. Cf. Estienne (1581: 36, 43–44).}
\end{quote}

Estienne’s correction of \textit{hóti} ($\text{\textgreek{<'o}}\tau \iota $), a common complementizer in Attic and the koine, into \textit{ho} \textit{tìn} ($\text{\textgreek{<o}}$ $\tau \text{\textgreek{`i}}\nu $), the Greek article in the nominative singular followed by the Doric dative singular of \textit{sú} ($\sigma \text{\textgreek{'u}}$), ‘you’, is still accepted today. Interestingly, Estienne intended to devote an entire treatise to the causes of textual mistakes and the importance of the Greek dialects in this context. This work, to which he referred as his \textit{Work} \textit{on} \textit{the} \textit{origin} \textit{of} \textit{errors}, does not seem to have materialized, unfortunately.\footnote{Estienne’s original Latin title for this conceived work was \textit{De} \textit{mendorum} \textit{origine} \textit{opus}.}

In Estienne’s wake, several other philologists relied on their knowledge of Greek dialect rules and particularities to correct ancient Greek texts, with varying degrees of success. The Bavarian classical scholar Gottlieb Christoph Harles (Harleß/Harless; 1738–1815), for instance, tried to do so for the works of the bucolic poet Theocritus. In the process, Harles criticized the changes made to Theocritus’ Doric dialect by Estienne and others, adding, however, that it was difficult to decide when and where to opt for the dialect form and even impossible to know for sure.\footnote{Harles (1780: \textsc{xxii–xxiv}).} Harles moreover believed that it was dangerous to Doricize a word form against the testimony of all manuscripts, all the more since Theocritus’ fatherland Sicily was home not only to Doric varieties but to different Greek dialects (\citealt{Harles1780}: \textsc{xxxi–xxxii}).

\subsubsection{Writing Greek poetry}
\hypertarget{Toc19704823}{}
Competence in the ancient Greek dialects was likewise indispensable for those early modern Hellenists wanting to show off their philological skills by composing ancient Greek texts themselves, especially poetry. This is why the early manual by the Swiss doctor Martin Ruland included several letters in different versions: Latin, Aeolic, Attic, Doric, and Ionic (\citealt{Ruland1556}: 328–335). Ruland composed these texts as examples for students with the ambition of writing in the Greek dialects. The Jesuit grammarian Jakob \citet[35]{Gretser1593} shared Ruland’s concerns but limited himself to emphasizing that Greek dialectal variation was to be carefully noted by students, not only in order to understand ancient Greek poets, but also so as to compose poems in Greek.

One of the best early modern handbooks for writing poetry in the different Greek dialects was a 1610 work entitled \textit{On} \textit{the} \textit{method} \textit{of} \textit{producing} \textit{Greek} \textit{poems} \textit{in} \textit{an} \textit{easy} \textit{and} \textit{skillful} \textit{manner} by Christoph Helwig (1581–1617), professor of Greek and Hebrew at the then recently established university of Giessen.\footnote{The work was republished posthumously in a slightly augmented edition in 1623 in Neurenberg.} \citet[19]{Helwig1610} regarded Greek dialectal diversity as furnishing great abundance, specifying that it was not allowed to use dialectal diversity in prose, but it was necessary and befitting for poetry. There was something like a “legitimate usage of the dialects” in poetry, Helwig explained to his readers.\footnote{Helwig (1610: 21–24), where he speaks of the \textit{legitimus} \textit{usus} \textit{dialectorum}.} Dialects were not to be mixed without any distinction, as this would result in a cento rather than an actual poem. Instead, one should observe certain restrictions. To this end, Helwig distinguished two principal kinds of Greek dialect poetry: Ionic and Doric, the latter also comprising Aeolic. One was not allowed to randomly jump from one to the other, even though there was considerable overlap between both dialects. To enable students to capture this as conveniently as possible, Helwig composed extensive comparative tables of dialectal particularities, which took up the core of his handbook and in which deviations from the koine were noted. In six columns, koine, Attic, Ionic, Doric, Aeolic, and poetical forms were placed next to each other. Helwig was writing mainly for students wanting to improve their understanding of Greek poetry by composing themselves, and this was indeed considered the principal goal of this activity throughout the entire early modern period. More than a century after Helwig, the German classical scholar Johann Matthias Gesner (1691–1761) too regarded it mainly as a student exercise. Gesner argued that students should write in Greek not to show off their erudition but in order to understand the rules of Greek poetry and the mixture of dialects in it. He nonetheless regarded this mixing as a silly undertaking, which he compared to an imaginary case of a German poet intermingling Swiss, Austrian, Low Saxon, and Dutch in his compositions \citep[162]{Gesner1774}.

\subsection{Labyrinths and enigmas}
\hypertarget{Toc19704824}{}
Mastering the Greek dialects is not an easy endeavor, not even today with so many tools available to the student of ancient Greek. In the early modern period, the complexity of the matter was so frequently stressed that it may well be called a topos. I have already mentioned how Juan Luis Vives warned his readers of the “vast labyrinths” in the Greek dialects. A German poet compared the phenomenon of Greek dialectal diversity to the enigma of the sphinx, which required a new Oedipus in order to be solved.\footnote{See the poem by Georg Meisner in Walper (1589: ††.2\textsc{\textsuperscript{r}}), where the Hellenist Otto Walper is dubbed “Oedipus alter”. On Walper, see also Van Rooy (fc. c).} In the early eighteenth century, a French classical scholar characterized the Greek dialects as “difficult nonsense” (\textit{difficiles} \textit{nugae}), the analysis of which constituted a task unappealing to a scholar of standing (\citealt{Maittaire1706}: \textsc{a.4}\textsc{\textsuperscript{r}}). At the same time, he regarded dialectal diversity as a boring topic and – with reference to Juvenal’s \textit{Satires} 7.154 – as “reheated cabbage”, i.e. a topic discussed over and over again by scholars before him (\citealt{Maittaire1706}: \textsc{a.4}\textsc{\textsuperscript{r–v}}). The dialects also troubled early modern interpreters and translators, who often failed to arrive at a correct understanding of Greek texts because they had not mastered the diversity of the language (see \citealt{Facius1782}: \textsc{iii–iv}). What is more, even Ancient and Byzantine Greek scholars had great difficulties with them, which is why they had composed treatises on the subject. That was at least the claim of a late eighteenth-century Dutch Hellenist, who edited several Byzantine works on the dialects (see \citealt{Koen1766}: \textsc{xvii–xviii}).

Only few Hellenists argued that the dialects were easy to learn. The German scholar Erasmus Schmidt (1604: ):(.2\textsc{\textsuperscript{r}}), professor of Greek, Hebrew, and mathematics in Wittenberg stressed that Greek dialectal diversity, if taught well, caused no difficulties. Before Schmidt, it had been put forward that, after mastering the basics of grammar, achieving competence in a dialect would take one or two hours only (\citealt{Caselius1560}: \textsc{e.6}\textsc{\textsuperscript{v}}). This view was shared by an eighteenth-century French Jesuit, who added that it sufficed for a student to know that a certain feature was dialectal, without being able to tell to what dialect it belonged exactly. In fact, the same mutations could pertain to various dialects or could even be transferred to the common language \citep[101]{Giraudeau1739}.

Most scholars did, however, agree on the difficulty of the Greek dialects, which was reflected in their presentation of this subject matter in their handbooks for the language. In fact, grammarians of Greek adopted several strategies in dealing with the issue. The first and most important one was to clearly separate the dialects from the koine, since scholars usually assumed that koine forms were sufficient for beginning students.\footnote{See e.g. Da Ponte (1501: [i]) and \citet[223]{Tavoni1986}. Cf. Glarean (1524: aa.ii\textsc{\textsuperscript{v}}); Metzler (1529: \textsc{a}.ii\textsc{\textsuperscript{v}}); \citet[105]{Rollin1726}. One eighteenth-century German grammarian was exceptional in holding that the dialects were also to be tackled by beginning students, since they were omnipresent (\citealt{Trendelenburg1782}: 174–175).} The renowned French Hellenist Henri Estienne emphasized that knowledge of the dialects was not necessary to correctly decline and conjugate Greek nouns and verbs. The koine/dialect separation could occur in different manners. Some grammars omitted dialectal information altogether in order to avoid overcomplication, whereas others reserved an entirely separate booklet for the issue.\footnote{Caselius (1560: \textsc{b.}iv\textsc{\textsuperscript{r}}) omitted the dialects, whereas Walper (1589: †.6\textsc{\textsuperscript{v}}–†.7\textsc{\textsuperscript{r}}) treated them in a separate booklet.} In other cases, dialect forms were discussed after the koine had been described, which was Philipp Melanchthon’s \textit{modus} \textit{operandi} in his Greek grammar (see e.g. \citealt{Melanchthon1518}: g.iv\textsc{\textsuperscript{v}}). Scholars often opted to typographically distinguish dialect from koine forms, usually by employing fonts of different sizes. As can be expected, dialect forms were as a rule in smaller print than koine forms.\footnote{See the method of presentation in e.g. \citet{Gretser1593}, Anon. \REF{ex:key:1613}, and \citet{Lancelot1655}.}

Another way to separate koine from dialect information was to postpone the matter to later sections of the grammar. A particular case in point is Urbano Bolzanio (1442–1524). In his book on Renaissance grammars of Greek, Paul Botley (2010: 36–40) has described how this Italian Hellenist revised his successful Greek grammar several times, while trying to find a more adequate manner to include dialect forms in his manual. The first edition of his grammar contained information on the dialects throughout (cf. \citealt{Bolzanio1497}: e.vi\textsc{\textsuperscript{v}}). Aware of the difficulties this raised for students, Bolzanio relegated dialect forms to a second part in his revised text of 1512. This was intended for more advanced students who had successfully studied the first introductory part (\citealt{Bolzanio1512}: \textsc{h}.iii\textsc{\textsuperscript{r}}). In the posthumously published second revision, he elaborated further upon this bipartition. He excluded information on the dialects from the first three books, intended for beginners. Advanced students could direct themselves to the six subsequent books. These contained a description of dialect forms, especially book four, “On the variety of tongues” (“De linguarum uarietate”; \citealt{Bolzanio1545}: 60\textsc{\textsuperscript{v}}).

A second strategy consisted in presenting the dialects and their particularities in a way that was as accessible and didactically effective as possible. For instance, in an attempt at facilitating the study of dialect rules, the French Hellenist and Port-Royal professor Claude Lancelot (ca. 1615–1695) composed mnemonic verses describing the most important features (\citealt{Lancelot1655}: \textsc{xiv-xv;} cf. also Anon. 1725). On the Attic dialect, Lancelot mused in absurd French verses, which I refrain from translating:

\begin{quote}
1 Contracter l’Attique aime, 2 et des voix le meslange:
\end{quote}

\begin{quote}
3 Son $\varsigma $ en $\xi \text{\textgreek{~i}}$, $\text{\textgreek{<r}}\text{\textgreek{~w}}$, $\tau \alpha \text{\textgreek{~u}}$, assez souvent il change:
\end{quote}

\begin{quote}
4 Oste ι d’$\alpha \text{\textgreek{"i}}$, $\varepsilon \text{\textgreek{"i}}$; 5 d’\textit{omicron} fait ω grand,
\end{quote}

\begin{quote}
6 $o\text{\textgreek{>~u}}\nu $ à la fin des mots, 7 aux adverbes ι prend.\footnote{\citet[558]{Lancelot1655}. For Ionic, see p. 560, for Doric p. 561, and for Aeolic p. 563.}
\end{quote}

Despite Lancelot’s good intentions, one might wonder how a student of Greek would benefit from these dense and enigmatic verses.

More often, grammarians introduced synoptic and systematized overviews in a schematic form, a method of presentation absent from the Ancient and Byzantine Greek tradition but widespread in early modern grammatical and typographical practice. In this approach, dialectal data were usually presented per linguistic feature – either in separate booklets or scattered throughout grammars – rather than per dialect, as their ancient and medieval predecessors had done.\footnote{For an example of a separately published booklet, see \citet{Amerot1530}, originally part of a grammar \citep{Amerot1520}. For an instance of schematized presentation of dialectal features throughout a grammar, see \citet{Gretser1593}.} This no doubt stimulated a contrastive comparison of the dialects and perhaps also of other languages, as not long after this method of presentation appeared scholars started to compare different languages while trying to assess their degree of kinship.\footnote{On this comparative turn, see e.g. Considine \& Van \citet{Hal2010}, with further references.} A number of scholars combined the per dialect and the per linguistic feature approach, discussing each dialect in separate chapters but structuring every chapter by means of grammatical properties (see e.g. \citealt{Zwinger1605}; \citealt{Mérigon1621}). Grammarians usually did not comment on their motivation in adopting a specific method of presentation. An exception is an early eighteenth-century Hellenist who criticized the per linguistic feature structure and preferred the traditional approach per dialect because he regarded it as more transparent (\citealt{Heupel1712}: ):(.3\textsc{\textsuperscript{r–}}):(.3\textsc{\textsuperscript{v}}).

The variety of strategies adopted in presenting dialect forms triggered some debate. For instance, the schematic presentations in table form could become rather complex, thus losing their didactic perspicuity. This seems to have been one of the reasons for the grammarian Johann Friedrich Facius (1782: \textsc{viii}) to criticize the manuals of his predecessors. To remedy this, Facius composed his own handbook, granting, however, that practical considerations also motivated him to write it, as earlier manuals were difficult to find in bookshops. Around the same time, Friedrich Gedike blamed both ancient and recent grammarians of Greek for having obscured the study of the dialects. Specifically, Gedike criticized existing handbooks, because they “are all a dark chaos of piled up examples” lacking “a philosophical view on the entire matter”.\footnote{\citet[4]{Gedike1782}: “alle sind ein dunkles Chaos aufgehäufter Exempel, nirgends ist ein philosophischer Blik über das Ganze”.}

\subsection{Conclusion: Dialectology as an ancillary subfield of Greek philology}
\hypertarget{Toc19704825}{}
In conclusion, the ancient Greek dialects principally attracted philological interest in the early modern period, albeit not so much as a topic in and for itself. Dialect studies were as a rule subsidiary to philological goals and skills that were considered more important. These principally included the ability to read Greek literature, to achieve more accurate etymological insights into the Greek language, to edit Greek literary texts, and to compose texts in the Greek language and its different dialects. The dialects were widely perceived as difficult subject matter, which grammarians presented in various ways in order to make it as clear as possible to their readership of would-be Hellenists. This led in some cases to a critique of the approach of others but more often to a struggle with presenting the dialects in a didactically effective manner; the opportunities offered by the printing press greatly facilitated this endeavor, as this new technique allowed for convenient schematic visualizations of Greek linguistic diversity.

Even though manuals for the Greek dialects were conceived primarily as auxiliary tools, Hellenists began to regard the study of the ancient Greek dialects as a separate subfield within Greek philology, especially in the eighteenth century. The Bible scholar Christian Siegmund Georgi (1702–1771) explicitly interpreted \textit{dialectologia} in the sense of “analysis and description of the ancient Greek dialects” as a distinct scholarly activity.\footnote{\citet[16]{Georgi1733}. Cf. also Nibbe (1725: b.1\textsc{\textsuperscript{v}}); Hauptmann \& \citet[15]{Schmid1737}. On the history of the term \textit{dialectologia}, see Van Rooy (fc. d).} It is remarkable that this happened at a time when Greek studies were in crisis according to contemporary sources.\footnote{See Reinhard (1724: 86–90), where four causes of this crisis are offered.} There are indications, however, that even before this time scholars considered the study of the Greek dialects to be a separate and specialized branch of learning. Many Hellenists provided a state of the art of Greek dialect studies at the outset of their discussion (cf. Van \citealt{Rooy2014}: 519–520). Initially, only Ancient and Byzantine Greek scholars were mentioned in these accounts, as is to be expected.\footnote{See e.g. Reuchlin (1477/1478 in Van \citealt{Rooy2014}: 509–510); Amerot (1520: \textsc{r.}iii\textsc{\textsuperscript{r}}\textsc{;} \textsc{1530:} title); Ruland (1556: α.3\textsc{\textsuperscript{r}}). Canini (1555: a.3\textsc{\textsuperscript{v}}), however, complained that his predecessors had neglected the problem of Greek dialectal diversity.} From the end of the sixteenth century onward, early modern scholars were also included.\footnote{See e.g. Walper (1589: †.7\textsc{\textsuperscript{r}}) and \citet[1]{Schmidt1604}, referring to, among others, Bolzanio, Clenardus, Antesignanus, and Henri Estienne.} Some scholars, such as the Basel-born physician and Hellenist Jakob Zwinger (1569–1610), carefully indicated their sources, both Greek and early modern, in their handbooks on the dialects \citep{Zwinger1605}\textstyleFootnoteSymbol{.} This likely demonstrates that early modern Hellenists widely regarded the study of the dialects as a well-defined subfield of Greek philology, and that they considered knowledge of the Greek dialects to be indispensable for becoming a true man of letters.

\section{\textsc{4} Dialects in the mixer: Homeric and Biblical Greek}
\hypertarget{Toc19704826}{}\begin{quote}
For the entire Greek tongue is divided into five tongues, into the common, the Ionic, the Doric, the Aeolic, and the Attic. And they say that that sublime genius of Homer has inserted these into his own works in such a manner that each tribe of Greece can recognize its particularities in his work. For his divine genius could not be confined and restrained within the limits of sound and utterly pure Attic speech.\footnote{Oreadini (1525: E.iii\textsc{\textsuperscript{v}}): “Graeca namque uniuersalis lingua in quinque linguas diuiditur, in communem, Ionicam, Doricam, Aeolicam et Acticam, quas ferunt sublime illud Homeri ingenium ita suis opibus inseruisse, ut unaquaeque gens Graeciae sua apud illum idiomata recognoscant. Diuinum enim ingenium non potuit coangustari retinerique intra limites sinceri purissimique Attici sermonis”.}
\end{quote}

The obscure early sixteenth-century Italian humanist Vincenzo Oreadini was concerned about Italian orthography and the problems dialectal diversity caused in this regard. More specifically, Oreadini published a book in 1525 on the question of whether new letters should be added to the Italian alphabet. In this work, published in Perugia, he compared the variation in his native tongue with the ancient Greek dialects. In passing, he praised the linguistic genius of Homer, which transcended Greek tribal divisions and allowed his \textit{Iliad} and \textit{Odyssey} to be enjoyed by all Greeks. Oreadini’s was only one of many premodern explanations of the peculiar nature of Homeric Greek. Most scholars did, however, agree that his language was a mixture of different dialects. A similar idea was put forward to account for the peculiar character of the Greek of the Bible. What is the history behind these mixed conceptions of Homeric and Biblical Greek, both of which still cause problems to present-day linguists?

\subsection{Homeric Greek: Puzzling scholars since antiquity}
\hypertarget{Toc19704827}{}
In ancient and medieval scholarship, the mixed nature of Homeric Greek was widely accepted. One of the oldest testimonies dates back to the late first century \textsc{ad}. In the orations of Dio Chrysostom (ca. \textsc{ad} 40–after 112), a long passage was devoted to Homer’s Greek, treated as part of a larger argument on the function of poetry and other arts. Dio’s ideas deserve to be quoted in full here:

\begin{quote}
Very great indeed is the ability and power of man to express in words any idea that comes into his mind. But the poets’ art is exceedingly bold and not to be censured therefor; this was especially true of Homer, who practiced the greatest frankness and freedom of language; and he did not choose just one variety of diction, but mingled together every Hellenic dialect which before his time were separate – that of the Dorians and Ionians, and also that of the Athenians – mixing them together much more thoroughly than dyers do their colors – and not only the languages of his own day but also those of former generations; if perchance there survived any expression of theirs taking up this ancient coinage, as it were, out of some ownerless treasure-store, because of his love of language; and he also used many barbarian words as well, sparing none that he believed to have in it anything of charm or vividness. Furthermore, he drew not only from things which lie next door or near at hand, but also from those quite remote, in order that he might charm the hearer by bewitching and amazing him; and even these metaphors he did not leave as he first used them, but sometimes expanded and sometimes condensed them, or changing them in some other way. And, last of all, he showed himself not only a maker of verses but also of words, giving utterance to those of his own invention, in some cases by simply giving his own names to the things and in others adding his new ones to those current, putting, as it were, a bright and more expressive seal upon a seal. He avoided no sound, but in short imitated the voices of rivers and forests, of winds and fire and sea, and also of bronze and of stone, and, in short, of all animals and instruments without exception, whether of wild beasts, birds, or pipes and reeds. […] As a result of this epic art of his, he was able to implant in the soul any emotion he wished.\footnote{Dio Chrysostom, \textit{Orationes} 12.66–69: “$\pi \lambda \varepsilon \text{\textgreek{'i}}\sigma \tau \eta $ $\mu \text{\textgreek{`e}}\nu $ $o\text{\textgreek{>~u}}\nu $ $\text{\textgreek{>e}}\xi o\upsilon \sigma \text{\textgreek{'i}}\alpha $ $\kappa \alpha \text{\textgreek{`i}}$ $\delta \text{\textgreek{'u}}\nu \alpha \mu \iota \varsigma $ $\text{\textgreek{>a}}\nu \theta \rho \text{\textgreek{'w}}\pi \text{\textgreek{w|}}$ $\pi \varepsilon \rho \text{\textgreek{`i}}$ $\lambda \text{\textgreek{'o}}\gamma o\nu $ $\text{\textgreek{>e}}\nu \delta \varepsilon \text{\textgreek{'i}}\xi \alpha \sigma \theta \alpha \iota $ $\tau \text{\textgreek{`o}}$ $\pi \alpha \rho \alpha \sigma \tau \text{\textgreek{'a}}\nu $. $\text{\textgreek{<h}}$ $\delta \text{\textgreek{`e}}$ $\tau \text{\textgreek{~w}}\nu $ $\pi o\iota \eta \tau \text{\textgreek{~w}}\nu $ $\tau \text{\textgreek{'e}}\chi \nu \eta $ $\mu \text{\textgreek{'a}}\lambda \alpha $ $\alpha \text{\textgreek{>u}}\theta \text{\textgreek{'a}}\delta \eta \varsigma $ $\kappa \alpha \text{\textgreek{`i}}$ $\text{\textgreek{>a}}\nu \varepsilon \pi \text{\textgreek{'i}}\lambda \eta \pi \tau o\varsigma $, $\text{\textgreek{>'a}}\lambda \lambda \omega \varsigma $ $\tau \varepsilon $ $\text{\textgreek{<O}}\mu \text{\textgreek{'h}}\rho o\upsilon $, $\tau o\text{\textgreek{~u}}$ $\pi \lambda \varepsilon \text{\textgreek{'i}}\sigma \tau \eta \nu $ $\text{\textgreek{>'a}}\gamma o\nu \tau o\varsigma $ $\pi \alpha \rho \rho \eta \sigma \text{\textgreek{'i}}\alpha \nu $, $\text{\textgreek{<`o}}\varsigma $ $o\text{\textgreek{>u}}\chi $ $\text{\textgreek{<'e}}\nu \alpha $ $\varepsilon \text{\textgreek{<'i}}\lambda \varepsilon \tau o$ $\chi \alpha \rho \alpha \kappa \tau \text{\textgreek{~h}}\rho \alpha $ $\lambda \text{\textgreek{'e}}\xi \varepsilon \omega \varsigma $, $\text{\textgreek{>a}}\lambda \lambda \text{\textgreek{`a}}$ $\pi \text{\textgreek{~a}}\sigma \alpha \nu $ $\tau \text{\textgreek{`h}}\nu $ $\text{\textgreek{<E}}\lambda \lambda \eta \nu \iota \kappa \text{\textgreek{`h}}\nu $ $\gamma \lambda \text{\textgreek{~w}}\tau \tau \alpha \nu $ $\delta \iota \text{\textgreek{h|}}\rho \eta \mu \text{\textgreek{'e}}\nu \eta \nu $ $\tau \text{\textgreek{'e}}\omega \varsigma $ $\text{\textgreek{>a}}\nu \text{\textgreek{'e}}\mu \iota \xi \varepsilon $, $\Delta \omega \rho \iota \text{\textgreek{'e}}\omega \nu $ $\tau \varepsilon $ $\kappa \alpha \text{\textgreek{`i}}$ $\text{\textgreek{>I}}\text{\textgreek{'w}}\nu \omega \nu $, $\text{\textgreek{>'e}}\tau \iota $ $\delta \text{\textgreek{`e}}$ $\tau \text{\textgreek{`h}}\nu $ $\text{\textgreek{>A}}\theta \eta \nu \alpha \text{\textgreek{'i}}\omega \nu $, $\varepsilon \text{\textgreek{>i}}\varsigma $ $\tau \alpha \text{\textgreek{>u}}\tau \text{\textgreek{`o}}$ $\kappa \varepsilon \rho \text{\textgreek{'a}}\sigma \alpha \varsigma $ $\pi o\lambda \lambda \text{\textgreek{~w|}}$ $\mu \text{\textgreek{~a}}\lambda \lambda o\nu $ $\text{\textgreek{>`h}}$ $\tau \text{\textgreek{`a}}$ $\chi \rho \text{\textgreek{'w}}\mu \alpha \tau \alpha $ $o\text{\textgreek{<i}}$ $\beta \alpha \varphi \varepsilon \text{\textgreek{~i}}\varsigma $, $o\text{\textgreek{>u}}$ $\mu \text{\textgreek{'o}}\nu o\nu $ $\tau \text{\textgreek{~w}}\nu $ $\kappa \alpha \theta \text{\textgreek{>}}$ $\alpha \text{\textgreek{<u}}\tau \text{\textgreek{'o}}\nu $, $\text{\textgreek{>a}}\lambda \lambda \text{\textgreek{`a}}$ $\kappa \alpha \text{\textgreek{`i}}$ $\tau \text{\textgreek{~w}}\nu $ $\pi \rho \text{\textgreek{'o}}\tau \varepsilon \rho o\nu $, $\varepsilon \text{\textgreek{>'i}}$ $\pi o\text{\textgreek{'u}}$ $\tau \iota $ $\text{\textgreek{<r}}\text{\textgreek{~h}}\mu \alpha $ $\text{\textgreek{>e}}\kappa \lambda \varepsilon \lambda o\iota \pi \text{\textgreek{'o}}\varsigma $, $\kappa \alpha \text{\textgreek{`i}}$ $\tau o\text{\textgreek{~u}}\tau o$ $\text{\textgreek{>a}}\nu \alpha \lambda \alpha \beta \text{\textgreek{`w}}\nu $ $\text{\textgreek{<'w}}\sigma \pi \varepsilon \rho $ $\nu \text{\textgreek{'o}}\mu \iota \sigma \mu \alpha $ $\text{\textgreek{>a}}\rho \chi \alpha \text{\textgreek{~i}}o\nu $ $\text{\textgreek{>e}}\kappa $ $\theta \eta \sigma \alpha \upsilon \rho o\text{\textgreek{~u}}$ $\pi o\theta \varepsilon \nu $ $\text{\textgreek{>a}}\delta \varepsilon \sigma \pi \text{\textgreek{'o}}\tau o\upsilon $ $\delta \iota \text{\textgreek{`a}}$ $\varphi \iota \lambda o\rho \rho \eta \mu \alpha \tau \text{\textgreek{'i}}\alpha \nu $, $\pi o\lambda \lambda \text{\textgreek{`a}}$ $\delta \text{\textgreek{`e}}$ $\kappa \alpha \text{\textgreek{`i}}$ $\beta \alpha \rho \beta \text{\textgreek{'a}}\rho \omega \nu $ $\text{\textgreek{>o}}\nu \text{\textgreek{'o}}\mu \alpha \tau \alpha $, $\varphi \varepsilon \iota \delta \text{\textgreek{'o}}\mu \varepsilon \nu o\varsigma $ $o\text{\textgreek{>u}}\delta \varepsilon \nu \text{\textgreek{`o}}\varsigma $ $\text{\textgreek{<'o}}$ $\tau \iota $ $\mu \text{\textgreek{'o}}\nu o\nu $ $\text{\textgreek{<h}}\delta o\nu \text{\textgreek{`h}}\nu $ $\text{\textgreek{>`h}}$ $\sigma \varphi o\delta \rho \text{\textgreek{'o}}\tau \eta \tau \alpha $ $\text{\textgreek{>'e}}\delta o\xi \varepsilon \nu $ $\alpha \text{\textgreek{>u}}\tau \text{\textgreek{~w|}}$ $\text{\textgreek{<r}}\text{\textgreek{~h}}\mu \alpha $ $\text{\textgreek{>'e}}\chi \varepsilon \iota \nu \text{\textgreek{;}}$ $\pi \rho \text{\textgreek{`o}}\varsigma $ $\delta \text{\textgreek{`e}}$ $\tau o\text{\textgreek{'u}}\tau o\iota \varsigma $ $\mu \varepsilon \tau \alpha \varphi \text{\textgreek{'e}}\rho \omega \nu $ $o\text{\textgreek{>u}}$ $\tau \text{\textgreek{`a}}$ $\gamma \varepsilon \iota \tau \nu \iota \text{\textgreek{~w}}\nu \tau \alpha $ $\mu \text{\textgreek{'o}}\nu o\nu $ $o\text{\textgreek{>u}}\delta \text{\textgreek{`e}}$ $\text{\textgreek{>a}}\pi \text{\textgreek{`o}}$ $\tau \text{\textgreek{~w}}\nu $ $\text{\textgreek{>e}}\gamma \gamma \text{\textgreek{'u}}\theta \varepsilon \nu $, $\text{\textgreek{>a}}\lambda \lambda \text{\textgreek{`a}}$ $\tau \text{\textgreek{`a}}$ $\pi \lambda \varepsilon \text{\textgreek{~i}}\sigma \tau o\nu $ $\text{\textgreek{>a}}\pi \text{\textgreek{'e}}\chi o\nu \tau \alpha $, $\text{\textgreek{<'o}}\pi \omega \varsigma $ $\kappa \eta \lambda \text{\textgreek{'h}}\sigma \text{\textgreek{h|}}$ $\tau \text{\textgreek{`o}}\nu $ $\text{\textgreek{>a}}\kappa \rho o\alpha \tau \text{\textgreek{`h}}\nu $ $\mu \varepsilon \tau \text{\textgreek{>}}$ $\text{\textgreek{>e}}\kappa \pi \lambda \text{\textgreek{'h}}\xi \varepsilon \omega \varsigma $ $\kappa \alpha \tau \alpha \gamma o\eta \tau \varepsilon \text{\textgreek{'u}}\sigma \alpha \varsigma $, $\kappa \alpha \text{\textgreek{`i}}$ $o\text{\textgreek{>u}}\delta \text{\textgreek{`e}}$ $\tau \alpha \text{\textgreek{~u}}\tau \alpha $ $\kappa \alpha \tau \text{\textgreek{`a}}$ $\chi \text{\textgreek{'w}}\rho \alpha \nu $ $\text{\textgreek{>e}}\text{\textgreek{~w}}\nu $, $\text{\textgreek{>a}}\lambda \lambda \text{\textgreek{`a}}$ $\tau \text{\textgreek{`a}}$ $\mu \text{\textgreek{`e}}\nu $ $\mu \eta \kappa \text{\textgreek{'u}}\nu \omega \nu $, $\tau \text{\textgreek{`a}}$ $\delta \text{\textgreek{`e}}$ $\sigma \upsilon \nu \alpha \iota \rho \text{\textgreek{~w}}\nu $, $\tau \text{\textgreek{`a}}$ $\delta \text{\textgreek{`e}}$ $\text{\textgreek{>'a}}\lambda \lambda \omega \varsigma $ $\pi \alpha \rho \alpha \tau \rho \text{\textgreek{'e}}\pi \omega \nu $. $T\varepsilon \lambda \varepsilon \upsilon \tau \text{\textgreek{~w}}\nu $ $\delta \text{\textgreek{`e}}$ $\alpha \text{\textgreek{<u}}\tau \text{\textgreek{`o}}\nu $ $\text{\textgreek{>a}}\pi \text{\textgreek{'e}}\varphi \alpha \iota \nu \varepsilon \nu $ $o\text{\textgreek{>u}}$ $\mu \text{\textgreek{'o}}\nu o\nu $ $\mu \text{\textgreek{'e}}\tau \rho \omega \nu $ $\pi o\iota \eta \tau \text{\textgreek{'h}}\nu $, $\text{\textgreek{>a}}\lambda \lambda \text{\textgreek{`a}}$ $\kappa \alpha \text{\textgreek{`i}}$ $\text{\textgreek{<r}}\eta \mu \text{\textgreek{'a}}\tau \omega \nu $, $\pi \alpha \rho \text{\textgreek{>}}$ $\alpha \text{\textgreek{<u}}\tau o\text{\textgreek{~u}}$ $\varphi \theta \varepsilon \gamma \gamma \text{\textgreek{'o}}\mu \varepsilon \nu o\varsigma $, $\tau \text{\textgreek{`a}}$ $\mu \text{\textgreek{`e}}\nu $ $\text{\textgreek{<a}}\pi \lambda \text{\textgreek{~w}}\varsigma $ $\tau \iota \theta \text{\textgreek{'e}}\mu \varepsilon \nu o\varsigma $ $\text{\textgreek{>o}}\nu \text{\textgreek{'o}}\mu \alpha \tau \alpha $ $\tau o\text{\textgreek{~i}}\varsigma $ $\pi \rho \text{\textgreek{'a}}\gamma \mu \alpha \sigma \iota $, $\tau \text{\textgreek{`a}}$ $\delta \text{\textgreek{>}}$ $\text{\textgreek{>e}}\pi \text{\textgreek{`i}}$ $\tau o\text{\textgreek{~i}}\varsigma $ $\kappa \upsilon \rho \text{\textgreek{'i}}o\iota \varsigma $ $\text{\textgreek{>e}}\pi o\nu o\mu \text{\textgreek{'a}}\zeta \omega \nu $, $o\text{\textgreek{<~i}}o\nu $ $\sigma \varphi \rho \alpha \gamma \text{\textgreek{~i}}\delta \alpha $ $\sigma \varphi \rho \alpha \gamma \text{\textgreek{~i}}\delta \iota $ $\text{\textgreek{>e}}\pi \iota \beta \text{\textgreek{'a}}\lambda \lambda \omega \nu $ $\text{\textgreek{>e}}\nu \alpha \rho \gamma \text{\textgreek{~h}}$ $\kappa \alpha \text{\textgreek{`i}}$ $\mu \text{\textgreek{~a}}\lambda \lambda o\nu $ $\varepsilon \text{\textgreek{>'u}}\delta \eta \lambda o\nu $, $o\text{\textgreek{>u}}\delta \varepsilon \nu \text{\textgreek{`o}}\varsigma $ $\varphi \theta \text{\textgreek{'o}}\gamma \gamma o\upsilon $ $\text{\textgreek{>a}}\pi \varepsilon \chi \text{\textgreek{'o}}\mu \varepsilon \nu o\varsigma $, $\text{\textgreek{>a}}\lambda \lambda \text{\textgreek{`a}}$ $\text{\textgreek{>'e}}\mu \beta \rho \alpha \chi \upsilon $ $\pi o\tau \alpha \mu \text{\textgreek{~w}}\nu $ $\tau \varepsilon $ $\mu \iota \mu o\text{\textgreek{'u}}\mu \varepsilon \nu o\varsigma $ $\varphi \omega \nu \text{\textgreek{`a}}\varsigma $ $\kappa \alpha \text{\textgreek{`i}}$ $\text{\textgreek{<'u}}\lambda \eta \varsigma $ $\kappa \alpha \text{\textgreek{`i}}$ $\text{\textgreek{>a}}\nu \text{\textgreek{'e}}\mu \omega \nu $ $\kappa \alpha \text{\textgreek{`i}}$ $\pi \upsilon \rho \text{\textgreek{`o}}\varsigma $ $\kappa \alpha \text{\textgreek{`i}}$ $\theta \alpha \lambda \text{\textgreek{'a}}\tau \tau \eta \varsigma $, $\text{\textgreek{>'e}}\tau \iota $ $\delta \text{\textgreek{`e}}$ $\chi \alpha \lambda \kappa o\text{\textgreek{~u}}$ $\kappa \alpha \text{\textgreek{`i}}$ $\lambda \text{\textgreek{'i}}\theta o\upsilon $ $\kappa \alpha \text{\textgreek{`i}}$ $\xi \upsilon \mu \pi \text{\textgreek{'a}}\nu \tau \omega \nu $ $\text{\textgreek{<a}}\pi \lambda \text{\textgreek{~w}}\varsigma $ $\zeta \text{\textgreek{'w|}}\omega \nu $ $\kappa \alpha \text{\textgreek{`i}}$ $\text{\textgreek{>o}}\rho \gamma \text{\textgreek{'a}}\nu \omega \nu $, $\tau o\text{\textgreek{~u}}\tau o$ $\mu \text{\textgreek{`e}}\nu $ $\theta \eta \rho \text{\textgreek{'i}}\omega \nu $, $\tau o\text{\textgreek{~u}}\tau o$ $\delta \text{\textgreek{`e}}$ $\text{\textgreek{>o}}\rho \nu \text{\textgreek{'i}}\theta \omega \nu $, $\tau o\text{\textgreek{~u}}\tau o$ $\delta \text{\textgreek{`e}}$ $\alpha \text{\textgreek{>u}}\lambda \text{\textgreek{~w}}\nu $ $\tau \varepsilon $ $\kappa \alpha \text{\textgreek{`i}}$ $\sigma \upsilon \rho \text{\textgreek{'i}}\gamma \gamma \omega \nu $ […]. $\text{\textgreek{<u}}\varphi \text{\textgreek{>}}$ $\text{\textgreek{<~h}}\varsigma $ $\text{\textgreek{>e}}\pi o\pi o\iota \text{\textgreek{'i}}\alpha \varsigma $ $\delta \upsilon \nu \alpha \tau \text{\textgreek{`o}}\varsigma $ $\text{\textgreek{>~h}}\nu $ $\text{\textgreek{<o}}\pi o\text{\textgreek{~i}}o\nu $ $\text{\textgreek{>e}}\beta o\text{\textgreek{'u}}\lambda \varepsilon \tau o$ $\text{\textgreek{>e}}\mu \pi o\iota \text{\textgreek{~h}}\sigma \alpha \iota $ $\tau \text{\textgreek{~h|}}$ $\psi \upsilon \chi \text{\textgreek{~h|}}$ $\pi \text{\textgreek{'a}}\theta o\varsigma $”. The English translation is taken over from the Loeb series.}
\end{quote}

According to Dio, Homer’s intricate mix of dialects was more perfect than the way in which dyers dyed clothes in various colors. Apart from different dialects, Homer’s speech was also marked by archaisms, barbarian words, and neologisms. Using these various linguistic devices, Homer was able to evoke whatever emotion he wanted. Dio, in other words, believed Homer to have resorted to linguistic mixing for psychological effect rather than to transcend Greek tribal divisions, as Vincenzo Oreadini suggested in the early sixteenth century.

Dio’s characterization of Homeric Greek roughly matches current views, even though the fundamental assumptions of Dio and modern research are quite different.\footnote{For modern views, see e.g. \citet{Hackstein2010} and \citet{Ruijgh2011}.} Much like Dio, scholars today assume that Homeric Greek is a mixed, multilayered, and artificial literary koine, but they do so against the background of historical-comparative linguistics rather than that of artistic functionality, as Dio had done. Contemporary linguists have demonstrated that Homer’s language is principally Ionic in nature, which likely shows that an important phase of redaction took place in central Ionic territory. There are, however, many Aeolic features too, perhaps because there was an earlier or parallel epic tradition in this dialect on which the poet(s) behind the \textit{Iliad} and the \textit{Odyssey} elaborated. In the sixth century \textsc{bc}, the Athenian ruler Peisistratus commissioned the production of a definitive version of Homer’s poems, which likely explains the presence of some distinctively Attic features. Like Dio, modern scholars have also identified archaisms in Homer’s speech, such as the absence of the definite article.

The rhetorician Hermogenes of Tarsus, active during the reign of Marcus Aurelius (r. \textsc{ad} 161–180), agreed with Dio that Homer’s speech was mixed, but added, much like modern linguists now, that Ionic predominated, as this dialect was both poetic and sweet in nature. Hermogenes did so in his treatise on style, while commenting on the language of the Ionic historian Herodotus rather than that of Homer.\footnote{Hermogenes of Tarsus, $\Pi \varepsilon \rho \text{\textgreek{`i}}$ $\text{\textgreek{>i}}\delta \varepsilon \text{\textgreek{~w}}\nu $ $\lambda \text{\textgreek{'o}}\gamma o\upsilon $ 2.4: “$\text{\textgreek{<h}}$ $\gamma \text{\textgreek{`a}}\rho $ $\text{\textgreek{>I}}\text{\textgreek{`a}}\varsigma $ $o\text{\textgreek{>~u}}\sigma \alpha $ $\pi o\iota \eta \tau \iota \kappa \text{\textgreek{`h}}$ $\varphi \text{\textgreek{'u}}\sigma \varepsilon \iota $ $\text{\textgreek{>e}}\sigma \tau \text{\textgreek{`i}}\nu $ $\text{\textgreek{<h}}\delta \varepsilon \text{\textgreek{~i}}\alpha $. $\varepsilon \text{\textgreek{>i}}$ $\delta \text{\textgreek{`e}}$ $\kappa \alpha \text{\textgreek{`i}}$ $\text{\textgreek{>'a}}\lambda \lambda \omega \nu $ $\delta \iota \alpha \lambda \text{\textgreek{'e}}\kappa \tau \omega \nu $ $\text{\textgreek{>e}}\chi \rho \text{\textgreek{'h}}\sigma \alpha \tau \text{\textgreek{'o}}$ $\tau \iota \sigma \iota $ $\lambda \text{\textgreek{'e}}\xi \varepsilon \sigma \iota \nu $, $o\text{\textgreek{>u}}\delta \text{\textgreek{`e}}\nu $ $\tau o\text{\textgreek{~u}}\tau o$, $\text{\textgreek{>e}}\pi \varepsilon \text{\textgreek{`i}}$ $\kappa \alpha \text{\textgreek{`i}}$ $\text{\textgreek{<'O}}\mu \eta \rho o\varsigma $ $\kappa \alpha \text{\textgreek{`i}}$ $\text{\textgreek{<H}}\sigma \text{\textgreek{'i}}o\delta o\varsigma $ $\kappa \alpha \text{\textgreek{`i}}$ $\text{\textgreek{>'a}}\lambda \lambda o\iota $ $o\text{\textgreek{>u}}\kappa $ $\text{\textgreek{>o}}\lambda \text{\textgreek{'i}}\gamma o\iota $ $\tau \text{\textgreek{~w}}\nu $ $\pi o\iota \eta \tau \text{\textgreek{~w}}\nu $ $\text{\textgreek{>e}}\chi \rho \text{\textgreek{'h}}\sigma \alpha \nu \tau o$ $\mu \text{\textgreek{`e}}\nu $ $\kappa \alpha \text{\textgreek{`i}}$ $\text{\textgreek{>'a}}\lambda \lambda \alpha \iota \varsigma $ $\tau \iota \sigma \text{\textgreek{`i}}$ $\lambda \text{\textgreek{'e}}\xi \varepsilon \sigma \iota \nu $ $\text{\textgreek{<e}}\tau \text{\textgreek{'e}}\rho \omega \nu $ $\delta \iota \alpha \lambda \text{\textgreek{'e}}\kappa \tau \omega \nu $, $\tau \text{\textgreek{`o}}$ $\pi \lambda \varepsilon \text{\textgreek{~i}}\sigma \tau o\nu $ $\mu \text{\textgreek{`h}}\nu $ $\text{\textgreek{>i}}\text{\textgreek{'a}}\zeta o\upsilon \sigma \iota $, $\kappa \alpha \text{\textgreek{`i}}$ $\text{\textgreek{>'e}}\sigma \tau \iota \nu $ $\text{\textgreek{<h}}$ $\text{\textgreek{>I}}\text{\textgreek{`a}}\varsigma $ $\text{\textgreek{<'o}}\pi \varepsilon \rho $ $\text{\textgreek{>'e}}\varphi \eta \nu $ $\pi o\iota \eta \tau \iota \kappa \text{\textgreek{'h}}$ $\pi \omega \varsigma $, $\delta \iota \text{\textgreek{`a}}$ $\tau o\text{\textgreek{~u}}\tau o$ $\delta \text{\textgreek{`e}}$ $\kappa \alpha \text{\textgreek{`i}}$ $\text{\textgreek{<h}}\delta \varepsilon \text{\textgreek{~i}}\alpha $”.} In fact, Hermogenes did not only regard Homer’s speech as mixed, but also that of, among others, the ancient didactic poet Hesiod and, oddly, that of Herodotus himself, whose language scholars today regard as straightforwardly Ionic.

However, not all ancient authors agreed that Homer’s language was mixed. The orator Aelius Aristides (\textsc{ad} 117–after 177), active in the decades between Dio and Hermogenes, conceived of Homeric speech as essentially Attic. Aristides expressed this idea in his \textit{Panathenaicus}, a speech held on the occasion of the Panathenaeic games of the year 155. As can be expected, the city of Athens was extravagantly praised in this oration, declaimed in Atticizing diction – one of his main models was the ancient rhetorician Demosthenes. Aristides maintained that Athens could stake a claim to Homer’s poetry as well, for two reasons. Not only did the great poet originate from Smyrna, a colony of Athens, but his language too was Attic, a statement not further substantiated by linguistic evidence.\footnote{Aelius Aristides, $\Pi \alpha \nu \alpha \theta \eta \nu \alpha \text{\textgreek{"i}}\kappa \text{\textgreek{'o}}\varsigma $ Jebb page 181: “$\varepsilon \text{\textgreek{>i}}$ $\delta \text{\textgreek{`e}}$ $\delta \varepsilon \text{\textgreek{~i}}$ $\kappa \alpha \text{\textgreek{`i}}$ $\tau \text{\textgreek{~h}}\varsigma $ $\text{\textgreek{<O}}\mu \text{\textgreek{'h}}\rho o\upsilon $ $\mu \nu \eta \sigma \theta \text{\textgreek{~h}}\nu \alpha \iota $, $\mu \varepsilon \tau \text{\textgreek{'e}}\chi \varepsilon \iota $ $\kappa \alpha \text{\textgreek{`i}}$ $\tau \alpha \text{\textgreek{'u}}\tau \eta \varsigma $ $\tau \text{\textgreek{~h}}\varsigma $ $\varphi \iota \lambda o\tau \iota \mu \text{\textgreek{'i}}\alpha \varsigma $ $\text{\textgreek{<h}}$ $\pi \text{\textgreek{'o}}\lambda \iota \varsigma $, $o\text{\textgreek{>u}}$ $\mu \text{\textgreek{'o}}\nu o\nu $ $\delta \iota \text{\textgreek{`a}}$ $\tau \text{\textgreek{~h}}\varsigma $ $\text{\textgreek{>a}}\pi o\text{\textgreek{'i}}\kappa o\upsilon $ $\pi \text{\textgreek{'o}}\lambda \varepsilon \omega \varsigma $, $\text{\textgreek{>a}}\lambda \lambda $’ $\text{\textgreek{<'o}}\tau \iota $ $\kappa \alpha \text{\textgreek{`i}}$ $\text{\textgreek{<h}}$ $\varphi \omega \nu \text{\textgreek{`h}}$ $\sigma \alpha \varphi \text{\textgreek{~w}}\varsigma $ $\text{\textgreek{>e}}\nu \theta \text{\textgreek{'e}}\nu \delta \varepsilon $”.}

The most important ancient source on the language of Homer was, however, a double biography of the poet, for a long time incorrectly attributed to the prolific writer and biographer Plutarch of Chaeronea (ca. \textsc{ad} 45–before 125). The work likely dates to the Roman period, but this question is complicated by the fact that it received later additions. In a part of this biography, Homer’s mixed use of the dialects was treated. Having visited each tribe of Greece, the poet inserted forms of every dialect into his compositions, according to the biographer. Indeed, Pseudo-Plutarch imagined Homer’s speech as a kind of linguistic potpourri: “In using a variegated diction, he mingled together the distinctive forms of each of the Greek dialects, out of which it is clear that he has visited the whole of Greece and each tribe”.\footnote{Pseudo-Plutarch, \textit{De} \textit{Homero} \textit{2} 8 (ed. \citealt{Kindstrand1990}: 9–10): “$\lambda \text{\textgreek{'e}}\xi \varepsilon \iota $ $\delta \text{\textgreek{`e}}$ $\pi o\iota \kappa \text{\textgreek{'i}}\lambda \text{\textgreek{h|}}$ $\kappa \varepsilon \chi \rho \eta \mu \text{\textgreek{'e}}\nu o\varsigma $ $\tau o\text{\textgreek{`u}}\varsigma $ $\text{\textgreek{>a}}\pi \text{\textgreek{`o}}$ $\pi \text{\textgreek{'a}}\sigma \eta \varsigma $ $\delta \iota \alpha \lambda \text{\textgreek{'e}}\kappa \tau o\upsilon $ $\tau \text{\textgreek{~w}}\nu $ $\text{\textgreek{<E}}\lambda \lambda \eta \nu \text{\textgreek{'i}}\delta \omega \nu $ $\chi \alpha \rho \alpha \kappa \tau \text{\textgreek{~h}}\rho \alpha \varsigma $ $\text{\textgreek{>e}}\gamma \kappa \alpha \tau \text{\textgreek{'e}}\mu \iota \xi \varepsilon \nu $, $\text{\textgreek{>e}}\xi $ $\text{\textgreek{<~w}}\nu $ $\delta \text{\textgreek{~h}}\lambda \text{\textgreek{'o}}\varsigma $ $\text{\textgreek{>e}}\sigma \tau \iota \nu $ $\pi \text{\textgreek{~a}}\sigma \alpha \nu $ [$\mu \text{\textgreek{`e}}\nu $] $\text{\textgreek{<E}}\lambda \lambda \text{\textgreek{'a}}\delta \alpha $ $\text{\textgreek{>e}}\pi \varepsilon \lambda \theta \text{\textgreek{`w}}\nu $ $\kappa \alpha \text{\textgreek{`i}}$ $\pi \text{\textgreek{~a}}\nu $ $\text{\textgreek{>'e}}\theta \nu o\varsigma $”.} In contrast to his ancient colleagues, who did not do much more than briefly comment on Homer’s Greek, Pseudo-Plutarch tried to substantiate his claims by means of linguistic evidence. He mentioned actual dialect features in his treatment of Homer’s mixed speech. Three Doric features were discussed, among which was the shortening of words, claimed to be typical of that dialect. Six Aeolic, nine Ionic, and twelve Attic characteristics were likewise described (see Van \citealt{Rooy2018c} for a more detailed overview). In keeping with these numbers, Attic was claimed to be the principal dialect of Homer, which, Pseudo-Plutarch argued on unclear grounds, was not unsurprising since that dialect had a mixed nature itself (see also Chapter 7, \sectref{sec:key:2}, \figref{fig:key:4}). An account of two syntactic particularities in Homer’s speech, one from Attic and the other from Doric, rounded off the linguistic analysis of Homer’s language. Pseudo-Plutarch subsequently concluded:

\begin{quote}
It is clear, then, how, in mustering the sounds of all Greeks, he creates a richly varied discourse and sometimes employs unusual utterances, as the aforementioned are, and sometimes ancient ones, as for example each time he says \textit{áor} [‘sword’] and \textit{sákos} [‘shield’], and sometimes common and usual ones, such as each time he says \textit{ksíphos} [‘sword’] and \textit{aspís} [‘shield’]. And one might wonder that even common words preserve with him the elevation of his style.\footnote{Pseudo-Plutarch, \textit{De} \textit{Homero} \textit{2} 14 (ed. \citealt{Kindstrand1990}: 14–15): “$\text{\textgreek{<'o}}\pi \omega \varsigma $ $\mu \text{\textgreek{`e}}\nu $ $o\text{\textgreek{>~u}}\nu $ $\tau \text{\textgreek{`a}}\varsigma $ $\pi \text{\textgreek{'a}}\nu \tau \omega \nu $ $\text{\textgreek{<E}}\lambda \lambda \text{\textgreek{'h}}\nu \omega \nu $ $\varphi \omega \nu \text{\textgreek{`a}}\varsigma $ $\text{\textgreek{>a}}\theta \rho o\text{\textgreek{'i}}\zeta \omega \nu $ $\pi o\iota \kappa \text{\textgreek{'i}}\lambda o\nu $ $\text{\textgreek{>a}}\pi \varepsilon \rho \gamma \text{\textgreek{'a}}\zeta \varepsilon \tau \alpha \iota $ $\tau \text{\textgreek{`o}}\nu $ $\lambda \text{\textgreek{'o}}\gamma o\nu $ $\kappa \alpha \text{\textgreek{`i}}$ $\chi \rho \text{\textgreek{~h}}\tau \alpha \iota $ $\pi o\tau \text{\textgreek{`e}}$ $\mu \text{\textgreek{`e}}\nu $ $\tau \alpha \text{\textgreek{~i}}\varsigma $ $\xi \text{\textgreek{'e}}\nu \alpha \iota \varsigma $, $\text{\textgreek{<'w}}\sigma \pi \varepsilon \rho $ $\varepsilon \text{\textgreek{>i}}\sigma \text{\textgreek{`i}}\nu $ $\alpha \text{\textgreek{<i}}$ $\pi \rho o\varepsilon \iota \rho \eta \mu \text{\textgreek{'e}}\nu \alpha \iota $, $\pi o\tau \text{\textgreek{`e}}$ $\delta \text{\textgreek{`e}}$ $\tau \alpha \text{\textgreek{~i}}\varsigma $ $\text{\textgreek{>a}}\rho \chi \alpha \text{\textgreek{'i}}\alpha \iota \varsigma $, $\text{\textgreek{<w}}\varsigma $ $\text{\textgreek{<'o}}\tau \alpha \nu $ $\lambda \text{\textgreek{'e}}\gamma \text{\textgreek{h|}}$ $\text{\textgreek{>'a}}o\rho $ $\kappa \alpha \text{\textgreek{`i}}$ $\sigma \text{\textgreek{'a}}\kappa o\varsigma $, $\pi o\tau \text{\textgreek{`e}}$ $\delta \text{\textgreek{`e}}$ $\tau \alpha \text{\textgreek{~i}}\varsigma $ $\kappa o\iota \nu \alpha \text{\textgreek{~i}}\varsigma $ $\kappa \alpha \text{\textgreek{`i}}$ $\sigma \upsilon \nu \text{\textgreek{'h}}\theta \varepsilon \sigma \iota \nu $, $\text{\textgreek{<w}}\varsigma $ $\text{\textgreek{<'o}}\tau \alpha \nu $ $\lambda \text{\textgreek{'e}}\gamma \text{\textgreek{h|}}$ $\xi \text{\textgreek{'i}}\varphi o\varsigma $ $\kappa \alpha \text{\textgreek{`i}}$ $\text{\textgreek{>a}}\sigma \pi \text{\textgreek{'i}}\delta \alpha $, $\delta \text{\textgreek{~h}}\lambda o\nu $. $\kappa \alpha \text{\textgreek{`i}}$ $\theta \alpha \upsilon \mu \text{\textgreek{'a}}\sigma \varepsilon \iota \text{\textgreek{'e}}$ $\tau \iota \varsigma $ $\text{\textgreek{<'o}}\tau \iota $ $\kappa \alpha \text{\textgreek{`i}}$ $\kappa o\iota \nu \alpha \text{\textgreek{`i}}$ $\lambda \text{\textgreek{'e}}\xi \varepsilon \iota \varsigma $ $\pi \alpha \rho $’ $\alpha \text{\textgreek{>u}}\tau \text{\textgreek{~w|}}$ $\sigma \text{\textgreek{'w|}}\zeta o\upsilon \sigma \iota $ $\tau \text{\textgreek{`o}}$ $\sigma \varepsilon \mu \nu \text{\textgreek{`o}}\nu $ $\tau o\text{\textgreek{~u}}$ $\lambda \text{\textgreek{'o}}\gamma o\upsilon $”.}
\end{quote}

In his final paragraph on Homer’s language, the author stressed once more the composite nature of Homer’s speech, but he did not only point to dialectal features, here somewhat oddly termed “foreign, unusual” (\textit{ksénos}/$\xi \text{\textgreek{'e}}\nu o\varsigma $). He also noticed archaisms and the use of common words. Pseudo-Plutarch, in other words, seems to have suggested that Homer’s speech also contained koine elements. If so, this would betray an ahistorical conception of the koine, for Homer is today usually placed in the eighth century \textsc{bc}, whereas the koine only emerged in the wake of Alexander the Great’s (356–323 \textsc{bc}) conquests. Such anachronistic ideas about the Greek language were, however, not unusual in premodern scholarship.

Pseudo-Plutarch’s discussion of the dialects turned out to be very welcome to Renaissance humanists, who eagerly read it in their attempts at deciphering Homer’s difficult poems. What is more, they excerpted it from the biography and printed it separately from the original work, most often together with two Byzantine treatises on the Greek dialects by John the Grammarian and Gregory of Corinth (see Van \citealt{Rooy2018c}). The latter two works did not comment extensively on Homer’s Greek, even though Gregory suggested that the poet composed in Ionic (\textit{De} \textit{dialectis} 1), without elaborating on this statement.

\subsection{In Plutarch’s footsteps: Renaissance ideas on Homer’s speech}
\hypertarget{Toc19704828}{}
All in all, ancient and Byzantine scholars were not too much concerned by the peculiar nature of Homeric Greek. In the Renaissance, however, scholars problematized the matter to a far greater extent. The speech of Homer was part of the larger issue of the language of Greek poets in general, for which the concept of \textsc{poetical} \textsc{dialect} was devised, as I have demonstrated earlier (Chapter 2, \sectref{sec:key:7}). Still, ideas on Homer’s Greek deserve a separate treatment here, all the more so since his work occupied a prominent position in early modern teaching of, and scholarship on, Greek language and literature (see e.g. \citealt{Botley2010}: 81–85). What is more, scholars tended to linger on Homer’s speech at greater length than on the speech of other poets.

When Renaissance scholars were able to move beyond the basics of the Greek tongue and started to become interested in the diversity of this language, Pseudo-Plutarch’s widely known analysis of Homeric Greek was one of their primary starting points. The case of Vincenzo Oreadini, cited at the outset of this chapter, may stand as an example of this, especially since, much like Pseudo-Plutarch, he seems to have presupposed a traveling poet, who through his mixed speech neutralized Greek tribal divisions. This should come as no surprise, as Plutarch was considered a trustworthy ancient authority, and only few scholars disputed his authorship of the treatise (Van \citealt{Rooy2018c}). In fact, there seems to have been a consensus among humanists, in Plutarch’s alleged tracks, that Homer mixed the four canonized dialects with common words in his epic poems. An early example can be retrieved from the \textit{Oration} \textit{in} \textit{the} \textit{course} \textit{of} \textit{explaining} \textit{Homer}, held in 1486/1487 by the pioneering Hellenist Angelo Poliziano (1454–1494) in Florence, the primary crib of Greek studies in Italy. A professor of Greek, Poliziano read Homer with his students, to whom he explained that

\begin{quote}
both [the \textit{Iliad} and the \textit{Odyssey}] were produced from all the tongues the Greeks call “dialects”, so that every tribe of Greece could discover his own proper features with him. Yet he does not reject common words either.\footnote{See Poliziano’s \textit{Oratio} \textit{in} \textit{expositione} \textit{Homeri}, first printed in 1553 (= \citealt{Poliziano1553}: 479): “utraque […] linguis [...] ex omnibus quas $\delta \iota \alpha \lambda \text{\textgreek{'e}}\kappa \tau o\upsilon \varsigma $ Graeci uocant, conflata est, sic ut unaquaeque Graeciae gens sua apud illum idiomata deprehendat. Neque tamen communia respuit uerba”.}
\end{quote}

Poliziano’s words obviously echoed Pseudo-Plutarch’s comments, even though he did not explicitly refer to the ancient author. In fact, it is not inconceivable that Poliziano had a hand in excerpting the section on the Greek dialects in Homer from the biography associated with Plutarch. For the extant manuscripts suggest that the extraction occurred toward the end of the fifteenth century in northern Italy in the humanist circles to which Poliziano belonged (Van \citealt{Rooy2018c}). What is more, Poliziano contributed to editing the collection of grammatical treatises issued in Venice by Aldus Manutius, which contained the first separate edition of the excerpt, where it was, however, attributed to the Byzantine commentator of Homer Eustathius of Thessalonica. In a later edition, Manutius changed this to Plutarch. Like Poliziano, the Venetian printer-scholar claimed that Homer’s speech was mixed, stating hyperbolically that he used all dialects and not only the principal ones (\citealt{Manutius1496}: *.ii\textsc{\textsuperscript{v}}).

The idea that Homer mixed the principal dialects in his poetry remained a common opinion throughout the entire early modern period; these principal dialects could be either four or five in number, depending on whether the koine was included.\footnote{See e.g. Simler (1512: \textsc{aa.}i\textsc{\textsuperscript{r}}); Liburnio (1546: \textsc{a}.viii\textsc{\textsuperscript{r}}); Lancelot (1655: xxxiv); Grosch (1753: 20–23); Ries (1786 [1782]: 196).} Some scholars brought the Pseudo-Plutarchan view to a head by claiming that Homer could speak five dialects in one single verse.\footnote{See e.g. Furetière (1701: \textit{s.u.} \textit{dialecte}); Chambers (1728: \textsc{i.}203 [3rd sequence of pagination]); \citet[934]{Dumarsais1754}.} The dialect mixture was often explained by means of Pseudo-Plutarch’s image of Homer as a traveling poet who had visited the whole of Hellas in order to be understood by all Greeks, which was also why he had introduced features common to all dialects into his speech. Poliziano was very explicit about this. Some humanists provided additional explanations. The quote from Oreadini heading this chapter, for example, might reflect Poliziano’s insistence on the idea that Homer wanted his diverse Greek audience to recognize features of their own dialects in his work. Oreadini was, however, idiosyncratic in arguing that Homer’s genius could not be contained within the limits of the elegant Attic dialect. The Spanish scholar in exile Juan Luis Vives Vives (1533: \textsc{x}.iiii\textsc{\textsuperscript{r}}) offered a different but vague explanation, as he claimed that Homer’s mixed language was the result of the fact that he considered all dialects to be one and could therefore draw no boundaries between them. Amalgamating the dialects was, in other words, natural for Homer, Vives suggested. Later scholars simply classified Homer as one of the poets using the mixed poetical dialect which they had introduced into their dialect divisions.\footnote{See e.g. Baile (1588: 6\textsc{\textsuperscript{r-v}}); \citet[333]{Alsted1630}; \citet[161]{Gesner1774}.} The great classical scholar Joseph Justus Scaliger (1540–1609), for instance, claimed that he had learned Greek in twenty-one days by studying Homer, during which time he had composed for himself a grammar of the poetical dialect based on his reading of this author (\citealt{Scaliger1594}: 56; see also Chapter 2, \sectref{sec:key:7}).

The idea that Homer mixed different dialects was by far the most popular explanation of the odd outlook of his speech. Like some of their ancient predecessors, a number of humanists claimed that one specific dialect predominated in Homer’s poems. In Hermogenes’ tracks, a large number of scholars assumed that Homer favored his allegedly native Ionic in mixing the dialects, whereas others followed Pseudo-Plutarch in proposing that he mainly used Attic.\footnote{For Ionic, see e.g. Da Ponte (1509: 47\textsc{\textsuperscript{r}}); \citet[215]{Ringelbergh1541}; \citet[167]{Labbe1639}; Kirchmaier \& Crusius (1684: \textsc{b.3}\textsc{\textsuperscript{v}}); Nibbe (1725: b.2\textsc{\textsuperscript{r}}, 334); \citet[161]{Gesner1774}. For Attic, see e.g. Codro (1502 [?1491/1492]: \textsc{f.}v\textsc{\textsuperscript{r}}); Waser (1610: 96\textsc{\textsuperscript{r}}); \citet[514]{Fabricius1711}.} Still others suggested a combined solution, stating that Homer mainly used both Attic and Ionic (e.g. \citealt{Schmidt1604}: 2; \citealt{Rhenius1626}: 84). One eighteenth-century author assumed that Homer principally mixed Ionic and Aeolic because he was born in the Ionian city of Smyrna to an Aeolic family and lived for many years on Chios, an Ionic island close to Aeolia (\citealt{Reynolds1752}: vi). Their suggestions were, however, usually not backed by linguistic evidence but by speculation, in which respect they differed from Pseudo-Plutarch’s account.

\subsection{Toward a historical solution}
\hypertarget{Toc19704829}{}
Not all early modern scholars were convinced that Homer purposely mixed different dialects in his speech. Especially in the eighteenth century, scholars sought for more convincing alternatives. Why and in which context did they do so? In the eighteenth century, much progress was achieved in Homeric scholarship, especially in Britain and German-speaking areas, where Greek philology still flourished, unlike in many other regions of Europe. In this period, the so-called Homeric question emerged: who was Homer? Scholars increasingly tried to contextualize this mystified poet in historical terms.\footnote{Primary sources central to the genesis of the Homeric question include [Blackwell] \REF{ex:key:1735}, \citet{Wood1775}, and \citet{Wolf1795}. On Blackwell and Wood, see Bauman \& Briggs (2003: 90–108). On Wolf see Sandys (1908: \textsc{iii.}55–57).} Concomitantly, they gained ever better insight into many aspects of Homer’s language too. Most notably, the English philologist Richard Bentley (1662–1742) solved a major metrical problem by restoring the \textit{digamma} in the Homeric text. This ancient letter, representing a [w] sound, had been lost in most canonical dialects of Greek, but not in Aeolic, which is why it was often called “the Aeolic digamma” in premodern scholarship. The sound must originally also have been present in the original Homeric text, but was lost in one of the redactions the poems underwent. As a matter of fact, in many cases, one should presuppose the presence of a digamma in order to have a metrically correct verse.\footnote{On Richard Bentley and the digamma, see e.g. Sandys (1908: \textsc{ii.}407) and especially Haugen (2011: 182–186).}

A heightened sense of the historicity of Homer’s epic poems and their language stimulated the idea that Homeric speech was an archaic variety of the Greek language. This view came in different guises. For instance, in early \citealt{November1709}, a disputation on the Greek koine was presented in Wittenberg by the young Hellenist Georg Friedrich Thryllitsch under the supervision of the professor of Greek Georg Wilhelm Kirchmaier. It is unknown who exactly authored the disputation – Thryllitsch, Kirchmaier, or both of them together – but nine months earlier Thryllitsch had presented another disputation on the Greek dialects from a historical perspective. This text was certainly authored by Thryllitsch alone, and since its content shows some similarities with that of the later dissertation, one might argue that Thryllitsch and not Kirchmaier composed the later one as well (see already Chapter 2, \sectref{sec:key:9}). Whatever the case, the text made an interesting, historically informed suggestion about the nature of Homeric Greek. There was, it claimed, a very ancient variety of Greek, which was called \textit{Hellenic} or also \textit{ancient} \textit{Attic} and was taken to be a kind of Proto-Greek language, to use a term from modern linguistics. This form of Greek was extinct by Homer’s time at the latest, and the specificity of the poet’s language should be partly explained by the fact that “residues” (\textit{rudera}) of this no longer extant Hellenic variety were still noticeable in his work (\citealt{KirchmaierThryllitsch1709}: \textsc{b.4}\textsc{\textsuperscript{v}}).

Others identified Homer’s tongue with a variety of ancient Ionic. The German Hellenist Friedrich Gedike elaborated most extensively on this idea in his article on the Greek dialects of 1782. Gedike (1782: 22–23) argued that Homer wrote in an ancient Ionic variety that had not yet been differentiated clearly from Attic, its mother dialect. It was moreover influenced by the speech of Dorians and Aeolians who roamed in Attica before migrating to other regions around the same time as the Ionians did. In doing so, Gedike made Attica the heartland of the Greek people and its language and rejected the idea attributed to Pseudo-Plutarch that Homer, as a traveling poet, purposely mixed different dialects in his language.\footnote{For ideas similar to Gedike’s, see e.g. Fréret (1809 [1746–47]: 115–116); \citet[202]{Beattie1778}; Trendelenburg (1782: 175–176).}

A final ingenuous solution was proposed by the English classical scholar Robert Wood (1717–1771), a major figure in the history of the Homeric question. Wood claimed that in Homer’s time the dialects had not yet been clearly distinguished, as there was not yet a cultivated state of language, a “standard” in his terms. This made it to Wood’s mind anachronistic to state that Homer mixed various dialects, as it was impossible for him to use one clearly demarcated form of speech.\footnote{\citet[238]{Wood1775}. See also Harles (1778: \textsc{xxiiii–xxv)}, who elaborated upon this view; Facius (1782: \textsc{v}).} Wood thus suggested that at that time Greek was more or less a dialect continuum, to use modern terminology.

In sum, several eighteenth-century scholars broke away from the traditional ideas of Pseudo-Plutarch and others. Instead, they viewed Homer’s Greek as representing an early stage of the Greek language rather than as an artificially mixed entity, which was, however, not really a step forward. In their attempt at understanding Homeric Greek in historical terms, they did not consider the idea, now widely accepted, that it was a \textit{Kunstsprache} that was never a native tongue. The fact that the historization of Homeric Greek occurred only in the eighteenth century suggests that the widespread early modern interest in language change and diversification, with roots in the sixteenth century, penetrated discussions of the language of Homer rather late.

\subsection{The struggle with Biblical Greek}
\hypertarget{Toc19704830}{}
For the peculiar nature of Biblical Greek, early modern scholars had no real precedent to follow, as ancient, Byzantine, and early Renaissance scholars had expressed limited interest in this issue.\footnote{Pre-early modern ideas about Biblical Greek deserve, however, a closer analysis.} Yet before moving to premodern ideas, I should clarify what exactly is meant by Biblical Greek here. Scholars today usually distinguish between the Greek of the Septuagint, a translation of the Hebrew Old Testament produced in Ptolemaic Egypt in the third and second centuries \textsc{bc} for Greek-speaking Jews, and the Greek of the New Testament, which was originally composed in this language in the first two centuries \textsc{ad}. Both varieties are, however, usually considered to have more or less the same nature, in that they represent lower, substandard varieties of the Greek koine, into which Semitisms have been introduced, primarily in vocabulary, syntax, and idiom. In the case of Septuagint Greek, these are principally due to the influence of the original Hebrew text, whereas the Semitic character of New Testament Greek remains somewhat of a mystery.\footnote{On Septuagint Greek, see e.g. Horrocks (2010: 106–108). On New Testament Greek, see e.g. \citet{Janse2007} and \citet{PorterPitts2013}.} A very likely explanation is that Semitisms were introduced in imitation of Septuagint Greek. Additionally, there might have been interference from Biblical and Mishnaic Hebrew as well as from Aramaic, like Greek an important \textit{lingua} \textit{franca} in Palestine and elsewhere during the first centuries \textsc{ad} \citep{Janse2007}.

Research on the history of ideas on Biblical Greek has been limited, and existing scholarship has largely restricted itself to some passing comments on the matter. For this reason, it is difficult to provide a satisfying answer here to the question as to how this variety of Greek has been perceived throughout the ages. There was, however, a vague awareness that New Testament Greek was “somewhat peculiar” and simpler than classical literary Greek. In fact, it was characterized as the language of fishermen or sailors by some Latin and Greek Early Christian authors \citep[647]{Janse2007}. No author writing before the early modern period, however, seems to have argued that Biblical Greek was a mixed variety consisting of different dialects. This idea was an early modern innovation, on which I will concentrate in this section and the next. For reasons of space and focus, I will not provide here a discussion of all interpretations suggested in this period. A thorough, comprehensive study of this matter therefore has to remain a desideratum for now, even though some scholars have already analyzed certain episodes of this history, usually from a theological or historical point of view. Henk J. de Jonge, for instance, has treated the study of the New Testament at early modern Dutch universities. One of the debates in this context concerned the purity of New Testament Greek. A number of scholars regarded it as impure, which caused serious theological problems. After all, how could the linguistic medium of the divine message be void of purity (De \citealt{Jonge1980}: 35–38; 1981: 117–118)?

I will focus here on the ways in which early modern scholars employed the dialectal reality of ancient Greece to account for the problematic nature of Biblical and especially New Testament Greek. The study of this variety of Greek was fostered by the interest of several leading humanists in the earliest Christians and their desire to return to a purer form of Christianity, fueled by their disappointment in contemporary religion. It was made possible by the return to the original Greek text advocated most sedulously by Desiderius Erasmus, inspired in his endeavor by his rediscovery of Lorenzo Valla’s notes on the Greek New Testament.\footnote{See e.g. Bentley (1983: esp. 31) for the innovativeness of the \textit{ad} \textit{fontes} approach closely associated with Erasmus.} This innovative approach gained ground mainly in Protestant areas, where scholars were eager to reach a correct vernacular translation of the Bible by all means possible, including the study of the original Greek New Testament. In Catholic areas, however, the sanctity of the Latin Vulgate seems to have largely obstructed the systematic study of the original Greek New Testament and its language. It therefore comes as no surprise that the first systematic dialectological solution to New Testament Greek was proposed by a Calvinist scholar, Georg Pasor (1570–1637), a German philologist and theologian mainly active in the Dutch Republic who compiled the first lexicon and grammar of New Testament Greek. Pasor’s activity must be viewed in connection with the creation of a course “New Testament Greek for theologians” at different Dutch universities, particularly in the Frisian city of Franeker, where he held the Greek chair.\footnote{De Jonge (1980: 29–31), where Pasor’s work is discussed in its Dutch context.}

\subsection{New Testament Greek as a dialect mixture}
\hypertarget{Toc19704831}{}
In his \textit{Form} \textit{of} \textit{the} \textit{Greek} \textit{dialects} \textit{of} \textit{the} \textit{New} \textit{Testament} of 1632, Pasor outlined his interpretation of New Testament Greek as follows: “There are without doubt seven dialects of the New Testament […], i.e. Attic, Ionic, Doric, Aeolic, Boeotian, Poetic, and the Hebraizing”.\footnote{Pasor (1632: 1–2): “\textit{Sunt} \textit{uero} \textit{dialecti} \textit{Noui} \textit{Testamenti} […], nempe \textit{Attica}, \textit{Ionica}, \textit{Dorica}, \textit{Aeolica}, \textit{Boeotica}, \textit{Poetica} et $\text{\textgreek{<h}}$ $\text{\textgreek{<E}}\beta \rho \alpha \text{\textgreek{'"i}}\zeta o\upsilon \sigma \alpha $.”} Pasor thus posited the four canonical dialects to be present in the Greek New Testament, to which he added the Boeotian and poetical dialects; these were frequently listed in early modern classifications of Greek dialects, also outside of discussions of Biblical Greek (see Chapter 2, \sectref{sec:key:8}). The Hebraizing dialect, however, was introduced by Pasor himself to account for the many Semitisms present in the New Testament. This means that Pasor must have presupposed the existence of a kind of Hebraizing or Jewish Greek nation, since he had defined \textit{dialectus} as “speech peculiar to whatever people it may be, and that in the same language”.\footnote{\citet[1]{Pasor1632}: “$\Delta \iota \text{\textgreek{'a}}\lambda \varepsilon \kappa \tau o\varsigma $ \textit{est} \textit{sermo} \textit{cuique} \textit{populo} \textit{peculiaris} \textit{idque} \textit{in} \textit{eadem} \textit{lingua}”.} Pasor did not, however, further comment on this, and in the remainder of his treatise he frequently spoke of “Hebraisms” (\textit{Hebraismi}) rather than of a Hebraizing dialect. In another debate, which started around the time Pasor published his work, such a typically Jewish Greek dialect was indeed explicitly posited by one of the parties involved, as I will demonstrate in the next section.

The mixed nature of New Testament Greek did not imply, however, that all dialects were equally represented in it. In fact, after discussing Attic features, Pasor added that, compared to Attic, “the remaining Greek dialects are by far rarer in the New Testament”.\footnote{\citet[24]{Pasor1632}: “Ceterae dialecti Graecae in N. T. sunt longe rariores.”} He claimed that communicative reasons were behind this dialectal diversity in New Testament Greek. The apostles wanted to announce the gospel not only to Jews, who read the Septuagint, but also to the remaining peoples speaking a variety of Greek dialects \citep[143]{Pasor1650}. This resembles premodern ideas on Homeric Greek in several ways. Firstly, both were considered to constitute a mixture of dialects. Secondly, many authors believed one dialect, often identified with Attic, to predominate in both varieties. Thirdly, intelligibility across ethnic divisions was frequently perceived as the main motivation behind the mixed nature of both Homeric and Biblical Greek. Yet, unlike some of his contemporaries, Pasor did not realize that by the time the New Testament was being composed, most Greek dialects had become extinct in favor of the koine.

Pasor’s treatise discussed the supposed linguistic features of New Testament Greek per dialect and with extensive exemplification. The typically Attic double tau instead of double sigma, for instance, was frequently found in the New Testament, he claimed (\citealt{Pasor1632}: 11–12). Some particularities he compared to dialectal variation in contemporary German. The <s>/<t> alternation had a parallel in High German \textit{Wasser} as opposed to Low German \textit{Water}, “water”. On the frequency of the letter alpha in Doric and the peculiar Dorian pronunciation of this letter, Pasor remarked:

\begin{quote}
\emph{\textup{Besides,} \emph{much} \emph{as} \emph{the} \emph{Ionians} \emph{love} \emph{the} \emph{eta,} \emph{the} \emph{Dorians} \emph{love} \emph{the} \emph{alpha} \emph{and} \emph{the} \emph{Attics} \emph{the} \emph{omega.} \emph{The} \emph{Dorians} \emph{pronounce} \emph{the} \emph{alpha} \emph{with} \emph{an} \emph{open} \emph{mouth,} \emph{just} \emph{as} \emph{among} \emph{the} \emph{Germans} \emph{too} \emph{there} \emph{are} \emph{some,} \emph{primarily} \emph{the} \emph{Bavarians} \emph{and} \emph{the} \emph{Austrians,} \emph{who} \emph{usually} \emph{do} \emph{this,} \emph{as} \emph{is} \emph{well-known.}}\footnote{\citet[28]{Pasor1632}: “Ceterum sicut Iones amant $\tau \text{\textgreek{`o}}$ η, ita Dores $\tau \text{\textgreek{`o}}$ α et Attici $\tau \text{\textgreek{`o}}$ ω. Dores $\tau \text{\textgreek{`o}}$ α ore diducto pronuntiant, uti apud Germanos quoque quosdam, imprimis Bauaros et Austriacos, factitare notum est”. Cf. Chapter 8, \sectref{sec:key:1.1.}}
\end{quote}

Pasor drew on sixteenth-century scholarship to retrieve dialectal features in the Greek of the New Testament. For example, he repeatedly referred to Joachim Camerarius’ (1500–1574) 1541 notes on Herodotus’ Ionic dialect, whereas for the so-called Hebraizing dialect he made use of Santes Pagnino’s (1470–1541) \textit{Treasure} \textit{of} \textit{the} \textit{holy} \textit{language}, first published in 1529 in Lyon.\footnote{For Camerarius, see Pasor (1632: 24–25, 27–28). For Pagnino, see \citet[36]{Pasor1632}.}

Pasor’s \textit{Form} was frequently reprinted. It moreover gave rise to the emergence of a philological subdiscipline which could be dubbed “biblical dialectology” for two main reasons. On the one hand, the alleged multidialectal nature of New Testament Greek provoked a considerable number of writings entirely dedicated to this theory, flourishing especially in Protestant scholarship.\footnote{\citet{Wyss1650}, \citet{Olearius1668}, \citet{Leusden1670}, and \citet{Nibbe1755}. Cf. \citet[347]{Parr1686}; Von der Hardt (1705 [1699]: 18–19); Florinus (1706: 9–10); Thryllitsch (1709: \textsc{d.2}\textsc{\textsuperscript{r}}; \textsc{d.5}\textsc{\textsuperscript{v}}); \citet[18]{Reinhard1724}; Holmes (1735: 121–122), a school grammar, suggesting that the theory was also taught; Walch (1772: 136–137).} The idea that “without knowledge of the dialects, the New Testament cannot be accurately understood” was indeed a commonplace.\footnote{Thryllitsch (1709: \textsc{d.5}\textsc{\textsuperscript{v}}): “Sine cognitione dialectorum Nouum Testamentum accurate intelligi non potest”.} On the other hand, the term \textit{dialectologia} was coined in 1650 by the Zurich Hellenist Caspar Wyss (1605–1659) to designate the study of the Greek dialects as they figured in the Greek New Testament. The word featured prominently in the title of his work on this matter: \textit{Sacred} \textit{dialectology}, or \textit{Dialectologia} \textit{sacra} in the original Latin. Wyss’ book is of interest for other reasons as well. He added the koine as a geographically neutral variety to the varieties of Greek which Pasor had recognized in the New Testament \citep[3]{Wyss1650}. The koine was opposed to the five principal and regional Greek dialects, Attic, Ionic, Doric, Aeolic, and Boeotian, to which, on Pasor’s authority, the poetic and the Hebraizing dialects needed to be added. However, these latter two were, Wyss explained, less important, since they were not tied to any province and exhibit idiomatically non-Greek properties (\citealt{Wyss1650}: 289–290, 295). Wyss was, in other words, trying to formulate a solution for a mismatch he had discovered in Pasor’s work. Wyss’ predecessor had provided a definition of \textit{dialectus} in ethnic terms which was difficult to reconcile with the Hebraizing and poetical dialects which he had posited for New Testament Greek. Interestingly, Wyss consciously arranged his discussion of the New Testament dialects in terms of frequency, an approach implicit at best in Pasor’s book. Attic, the most prominent dialect in the New Testament, was described first; Boeotian, the least prominent, last \citep[4]{Wyss1650}.

Pasor’s theory that New Testament Greek was an amalgam of different Greek dialects did not come out of nowhere. More or less simultaneously to Pasor, a Scottish exegete observed in passing: “So in the New Testament there are sundry dialects as \textit{Ionick}, \textit{Dorick}, \textit{Attick}, \textit{etc}.” \citep[102]{Weemes1632}. It appears that Pasor was systematizing a tradition already found in earlier work. Desiderius \citet[270]{Erasmus1519} relied as early as 1519 on his knowledge of the Greek dialects to refute a judgment of St Jerome’s about an alleged syntactic irregularity – a so-called solecism – in the New Testament. This rebuttal by Erasmus presupposed the idea that certain features of the language of the New Testament could be explained by appealing to a Greek dialect. Such an assumption became even more clearly apparent in the second half of the sixteenth century. The French-Swiss Protestant theologian Theodore Beza (1519–1605), for example, explained several linguistic particularities of the New Testament by referring to the Attic dialect.\footnote{See e.g. Beza (1594: \textsc{i.}226, \textsc{ii}.355), where Hebrew influence was also mentioned.} Grammars of Greek and manuals for the dialects, too, started to contain occasional references to the Greek New Testament to exemplify certain dialectal particularities. \citealt{In1589}, the Marburg professor of Greek Otto Walper illustrated the alleged Attic feature of using comparatives and superlatives interchangeably by citing 1 Corinthians 13.13.\footnote{\citet[32]{Walper1589}: “Superlatiuis pro comparatiuis utuntur frequenter, et contra.1.Cor.13. $\mu \varepsilon \text{\textgreek{'i}}\zeta \omega \nu $ $\delta \text{\textgreek{`e}}$ $\tau o\text{\textgreek{'u}}\tau \omega \nu $ $\text{\textgreek{<h}}$ $\text{\textgreek{>a}}\gamma \text{\textgreek{'a}}\pi \eta $, id est, $\mu \varepsilon \gamma \text{\textgreek{'i}}\sigma \tau \eta $”. See also already \citet[251]{Ruland1556}, explaining an Attic particularity with reference to, among other texts, the Bible.} The idea that the Greek New Testament exhibited dialectal features may moreover have been enhanced by poetical adaptations of the Greek Bible appearing in the second half of the sixteenth century, such as, for instance, Johannes Posselius the Younger’s (1565–1623) Greek versification of parts of the New Testament, which contained many dialect forms that were occasionally explained in the margins.\footnote{\citet{Posselius1599}. Cf. also \citet{Jamot1593} and \citet{Keimann1649}.}

The multidialectal interpretation of New Testament Greek was criticized early on, most notably by the prominent French humanist-printer Henri Estienne (1581: 32–33, 138). Some even saw the misuse of the Greek dialects as a danger for the vernacular translator of the New Testament, who might be able to distort the sense of a word by referring to the Greek dialects, thus introducing heresies into the text.\footnote{See \citet[429]{Rainolds1583}, on which see Van Rooy \& Considine (2016: 654–655).} Much later, biblical dialectology came to be rebutted by the eighteenth-century German professor of theology and philology Christian Siegmund Georgi. Georgi emphasized that the authors of the New Testament wrote in pure Attic, a medium befitting the divine message. He explained the presence of non-Attic elements by claiming that they had become Attic in the course of history (\citealt{Georgi1733}: 6–7). He moreover reacted against scholars inventing “pseudo-dialects” to account for the particularity of New Testament Greek, no doubt thinking of the so-called Hebraizing dialect as well as the Hellenistic dialect proposed by Daniel Heinsius, which I will treat in the next section.\footnote{Georgi coined the term “$\psi \varepsilon \upsilon \delta o\delta \iota \text{\textgreek{'a}}\lambda \varepsilon \kappa \tau o\iota $” to make his point. For New Testament Greek as Attic, see e.g. also Georgi \& Graun (1729: 3, 10–12) and Fischer (1754: b.7\textsc{\textsuperscript{r}}–b.8\textsc{\textsuperscript{v}}).} Georgi’s ideas must be viewed in the context of the eighteenth-century debate between Hebraists and Purists which took place primarily in the Northern Low Countries (the modern Netherlands), Germany, and England. The Purists, including Georgi, argued that the New Testament was written in pure Greek, whereas the Hebraists contended that Hebrew elements were unmistakably present, which, however, were not barbarisms but adornments (De \citealt{Jonge1980}: 35). A thesis similar to Georgi’s had already been proposed for public discussion in 1702 in Wittenberg by Georg Wilhelm Kirchmaier and Christian Gottlieb Schwartz (see \citealt{KirchmaierSchwartz1702}: [2], thesis \textsc{i}). It intelligently suggested that a mixed use of dialects was highly unlikely, as this would have made the New Testament unintelligible to the populace. However, as late as 1765, the framework of biblical dialectology was still presented as canonical knowledge by several scholars (e.g. \citealt{Gottleber1765}: *.2\textsc{\textsuperscript{r}}), which indicates that the efforts of Georgi and others had not yet disparaged the awkward idea of a dialectally mixed New Testament Greek.

\subsection{Biblical Greek, a Hellenistic dialect?}
\hypertarget{Toc19704832}{}
A different dialectal solution to the problematic status of Biblical Greek was proposed by the philologist Daniel Heinsius (1580–1655), born in Ghent but mainly active at Leiden university. Heinsius assumed the existence of a clearly distinct dialect used in translating the Hebrew Old Testament into Greek and writing down the Greek New Testament. This dialect, he claimed, was spoken by the so-called Hellenists, a Greek nation separate from the others. The Burgundian classical scholar Claude de Saumaise (1588–1653) reacted sharply against this Hellenistic tongue, which he dismissed as an invention of Heinsius. Their fierce controversy was as much a matter of personal rivalry as it was a scholarly disagreement.\footnote{De \citet[32]{Jonge1980}. For this well-known and much-studied controversy, see De Jonge (1980: 32–34); Muller (1984: 391–392); \citet{Considine2010}; Van Hal (2010a: 350–351); Hardy (2012: 207–209). Daniel Georg Morhof (1708: \textsc{ii.}74–77) already summarized the controversy.} Heinsius and Saumaise were in fact archenemies because of their competing ambitions to succeed Joseph Justus Scaliger at Leiden university. Saumaise formulated his main attack of Heinsius and his Hellenistic dialect in two books, which he wrote in France but sent to Leiden to be printed.\footnote{Saumaise (1643a \& 1643b). The preface of \citet{Saumaise1639} already touched on the issue too.} He did so in order to avoid his absence endangering his position at the university. Both works appeared in 1643 and were centered around the argument that there was no such thing as a Hellenistic people, let alone a Hellenistic dialect.

In Saumaise’s rebuttal of Heinsius’ Hellenistic dialect, the correct interpretation of the Greek term \textit{diálektos} played a pivotal role. Saumaise emphasized – repeatedly and ad nauseam – that in order to speak of a Hellenistic dialect, the existence of a Hellenistic people was required, which was not corroborated by historical evidence. This seems to indicate that Heinsius and Saumaise, their personal differences and their insistence on terminology aside, “were arguing over a serious scientific problem”, as Henk J. de \citet[117]{Jonge1981} has put it (see also De \citealt{Jonge1980}: 34–35). They had different views on the linguistic history of the Greek language, and their debate was, in consequence, not simply a matter of word choice, as has been maintained (see \citealt{Simon1689}: 318–319; cf. \citealt{Considine2012}: 298). The alternative Saumaise (1643a: 98–99, 240–266) suggested for Heinsius’ Hellenistic dialect indeed seems to support this interpretation. He argued that instead of a Hellenistic dialect, the authors of the New Testament used the uneducated vulgar variety of koine Greek – the \textit{stylus} \textit{idioticus} – of their times.\footnote{See De Jonge (1980: 34–35). In the early sixteenth century, Erasmus had already proposed a similar solution to the issue \citep[181]{Bentley1983}.} Saumaise thus stressed the link with contemporary non-Biblical Greek and proposed an answer that, from a modern perspective, seems more correct. This vulgar variety, Saumaise proceeded, was influenced by Aramaic, thus obtaining a “translational” (\textit{hermēneutikós}/$\text{\textgreek{<e}}\rho \mu \eta \nu \varepsilon \upsilon \tau \iota \kappa \text{\textgreek{'o}}\varsigma $) character. The language of the Septuagint was likewise, and more understandably, said to be of a translational nature, but it was also claimed to have Macedonian characteristics due to Alexander the Great’s heritage \citep[264]{Saumaise1643a}. The notorious Heinsius–Saumaise controversy stirred up many subsequent discussions of the matter.\footnote{The “notoriousness” of the issue was already underlined by Morhof (1708: \textsc{ii.}74), who used the Greek adjective \textit{poluthrúl<l>ētos}  ($\pi o\lambda \upsilon \theta \rho \text{\textgreek{'u}}\lambda $<λ>$\eta \tau o\varsigma $) in this context.} Some scholars preferred to speak of “Biblical Graecism” (\textit{Biblicus} \textit{Graecismus}) rather than of a dialect peculiar to this text, whereas others rejected the Hellenistic dialect as a product of Heinsius’ imagination.\footnote{For \textit{Biblicus} \textit{Graecismus}, see Bolius \& Alberti (1689 [1647]: \textsc{b.3}\textsc{\textsuperscript{v}}). For Heinsius’ \textit{dialectus} \textit{Hellenistica} as a “dream” (\textit{somnium}), see \citet{Croy1644}.} It does not, however, lie within the scope of this book to provide an exposition of the extensive early modern debate on the issue in its entirety, as this would require a separate study of its own.

In summary, both the term \textit{dialect} and the fact that the Greek language had different dialects were exploited by early modern Hellenists to make sense of the peculiar form of Greek they encountered in reading the Septuagint and especially the New Testament. Whereas the framework of biblical dialectology, according to which New Testament Greek was a mixture of dialects, strikes the modern reader as highly artificial and even clumsy, the debate between Heinsius and Saumaise led to a relatively accurate hypothesis about the nature of Septuagint Greek. Biblical dialectologists were eager to attribute peculiar forms which they encountered in the New Testament to specific dialects, and the description of these particularities was their main concern. In the controversy about Hellenistic Greek, linguistic features were confined to the margins of the main argument. Saumaise focused on the interpretation of the technical term \textit{diálektos} as well as on the linguistic history of the Greek language and its speakers to dismiss Heinsius’ Hellenistic tongue. He did, however, claim that it was impossible for one and the same word to have different meanings in one and the same dialect (\citealt{Saumaise1643a}: 41–42 of the dedicatory letter). This impossibility of polysemy was employed by Saumaise as a supporting argument to refute the existence of a Hellenistic dialect, in which according to Heinsius and his followers certain words could have several interpretations.

\subsection{Conclusion}
\hypertarget{Toc19704833}{}
In classifying the Greek dialects, early modern scholars encountered two major difficulties: the speech of Homer and the Greek Bible. Incidentally, these were also among the Greek texts that were read most eagerly in the early modern period. For Homer’s peculiar Greek, Renaissance Hellenists first followed the idea that it was a dialectally mixed variety, which was backed by a text attributed to the authoritative ancient polygraph Plutarch. An increased awareness of the historicity of the figure of Homer and the emergence conditions of the epic poems associated with him led eighteenth-century scholars to a revaluation of his language in historical terms. Even though their solutions were certainly not wholly satisfying, they paved the way for later interpretations of Homer’s language that took into account its historical evolution more fully.

Perhaps by analogy with Homer’s Greek, scholars from Protestant areas developed the idea that the language of that other great Greek textual corpus, the New Testament, was also dialectally mixed. In this case, philologists were not backed by ancient scholarship, and it is hard to understand why this seemed such an appealing solution. They appear to have believed that mixing dialects implied reaching a larger audience, an assumption that seems counterintuitive to most modern readers. The discussion about the right to existence of a Hellenistic dialect initiated by the rival colleagues Heinsius and Saumaise, in contrast, did lead to a better understanding of Biblical Greek, grounded, like eighteenth-century views on Homeric Greek, in an appreciation of the historical conditions under which koine Greek emerged.

\section{\textsc{5} Old, older, oldest: Writing the linguistic history of Greek}
\hypertarget{Toc19704834}{}
An event of great importance in the history of Greek studies occurred in 1518, when Philipp Melanchthon was appointed as the first professor of the language at the university of Wittenberg. His teaching there laid the groundwork for the strong Protestant tradition in this discipline. In the century or so after the installment of the Greek chair in Luther’s city, countless Hellenists were educated in humanist spirit. A major goal they pursued was to arrive at a fuller understanding of the New Testament in its original language, as I have pointed out in the previous chapter. An exponent of Protestant Hellenism was Lorenz Rhodoman (1546–1606). A student of several of Melanchthon’s pupils, Rhodoman later became professor at the Wittenberg academy himself. He was a prolific scholar and poet, who showed off his mastery of Greek more than once in his compositions. When in 1604 one of his promising students left the city, he pronounced a lengthy oration on the Greek language and its historical development, which was printed in Strasbourg the next year \citep{Rhodoman1605}. Part praise, part history, the text constituted a precursor to later histories of the Greek language, a genre flourishing particularly in the Holy Roman Empire. This can be regarded as a symptom of the wider interest in the historical development of languages during the early modern period. A major achievement of humanist scholars in this regard was the formulation of the idea that many European and Asian languages, including Greek, were related and, in fact, descendants of a lost original language, often dubbed “Scythian”. This so-called Scythian hypothesis foreshadowed to some extent the later modern concept of \textsc{Proto-Indo-European}.\footnote{On the Scythian hypothesis, see e.g. Metcalf (2013 [1974]: 34–39); \citet{Droixhe1980}; Swiggers (1984; 1998); \citet{Villani2003}; \citet{Considine2010}; Van Hal (2010a: esp. 335–401, 473–475; 2010b).} The increasing interest in language history forced humanists to think about the place of the Greek language and its dialects in it. Yet before moving to early modern ideas, I have to briefly consider the very few earlier remarks on the matter that are extant. How did ancient Greek and Byzantine scholars picture the history of the Greek language and its dialects?

\subsection{The linguistic history of Greek in ancient and medieval scholarship}
\hypertarget{Toc19704835}{}
In Chapter 2, I showed that Strabo, likely inspired by Alexandrian scholarship, proposed a classification into four dialects – Ionic, Attic, Doric, and Aeolic – and that he saw a close kinship between Ionic and Attic, on the one hand, and Doric and Aeolic, on the other. Strabo did more than merely suggesting kinship, however, as he framed the Greek dialects into a historical scheme. He claimed that initially old Attic was the same as Ionic, and that Doric was identical to Aeolic, suggesting that there were initially only two dialects (\textit{Geographica} 8.1.2). In the Byzantine period, the Homer commentator Eustathius of Thessalonica (ca. 1115–1195/1196) took over Strabo’s language-historical scheme (\textit{Commentarii} \textit{ad} \textit{Homeri} \textit{Iliadem} 1.14). Strabo’s brief account is the most extensive consideration of the historical position of the Greek dialects found in ancient and medieval texts, which indicates that scholars of these eras were barely interested in this question. It is, besides, a little surprising that Strabo did not go further back. One might have expected him to point out that the two branches, Attic–Ionic and Doric–Aeolic, were also originally one language, as they went back to one and the same mythological ancestor, Hellen, and Greeks were aware that they spoke in essence a single tongue (see Morpurgo \citealt{Davies1987}). The idea of a Greek protolanguage was, however, usually not made explicit by Greek authors, perhaps because they regarded it as obvious. An exception is the early Byzantine scholar John Philoponus, who assumed that Greek was originally unitary and believed that a process of geographical dispersion was responsible for ethnic and linguistic diversification. Philoponus argued this in the following anacoluthic sentence:

\begin{quote}
For when [the children of Hellen] were dispersed toward multiple places and no longer preserved the same speech, but changed along with their migration also their speech, it happened that they were called dialects.\footnote{John the Grammarian (ed. Manutius \textit{et} \textit{al.} 1496: 236\textsc{\textsuperscript{v}}): “$\delta \iota \alpha \sigma \pi \alpha \rho \text{\textgreek{'e}}\nu \tau \omega \nu $ $\gamma \text{\textgreek{`a}}\rho $ $\tau o\text{\textgreek{'u}}\tau \omega \nu $, $\varepsilon \text{\textgreek{>i}}\varsigma $ $\pi \lambda \varepsilon \text{\textgreek{'i}}o\nu \alpha \varsigma $ $\tau \text{\textgreek{'o}}\pi o\upsilon \varsigma $, $\kappa \alpha \text{\textgreek{`i}}$ $\tau \text{\textgreek{`h}}\nu $ $\alpha \text{\textgreek{>u}}\tau \text{\textgreek{`h}}\nu $ $\varphi \omega \nu \text{\textgreek{`h}}\nu $, $o\text{\textgreek{>u}}\kappa $ $\text{\textgreek{>'e}}\tau \iota $ $\varphi \upsilon \lambda \alpha \xi \text{\textgreek{'a}}\nu \tau \omega \nu \text{\textgreek{;}}$ $\text{\textgreek{>a}}\lambda \lambda \text{\textgreek{`a}}$ $\tau \text{\textgreek{~h|}}$ $\tau \text{\textgreek{~w}}\nu $ $\tau o\text{\textgreek{'u}}\tau \omega \nu $ $\mu \varepsilon \tau \alpha \beta o\lambda \text{\textgreek{~h|}}$ $\text{\textgreek{'a}}\mu \alpha $ [\textit{sic}] $\kappa \alpha \text{\textgreek{`i}}$ $\tau \text{\textgreek{`h}}\nu $ $\varphi \omega \nu \text{\textgreek{`h}}\nu $ $\mu \varepsilon \tau \alpha \beta \alpha \lambda \lambda \text{\textgreek{'o}}\nu \tau \omega \nu $, $\sigma \upsilon \nu \text{\textgreek{'e}}\beta \eta $ $\delta \iota \alpha \lambda \text{\textgreek{'e}}\kappa \tau o\upsilon \varsigma $ $\lambda \text{\textgreek{'e}}\gamma \varepsilon \sigma \theta \alpha \iota $”.}
\end{quote}

Strabo distinguished between old and new forms of a dialect, especially with reference to Attic. Other Greek scholars did so too. Around the same time, the literary critic Dionysius of Halicarnassus (ca. 60–after 8/7 \textsc{bc}) expressed the view that Plato and Thucydides wrote in an older variety of Attic (\textit{De} \textit{Lysia} 2). The philosopher Sextus Empiricus (\textit{fl.} ca. \textsc{ad} 190–210) also made a distinction between old and current Athenian speech (\textit{Aduersus} \textit{mathematicos} 1.228). A Byzantine commentator on the Hellenistic poet Theocritus’ work distinguished between the old, harsh Doric of older poets and the new, mellower Doric of Theocritus. He seemingly suggested that the latter was influenced by other dialects, most importantly the effeminate dialect of the Ionians.\footnote{\textit{Scholia} \textit{in} \textit{Theocritum} \textit{(scholia} \textit{uetera)} \textsc{f} a.–d. For this attitude toward Doric and Ionic, cf. Chapter 7, \sectref{sec:key:2.}}

Strabo, Eustathius, and John the Grammarian put the four Greek dialects on the same chronological scale. Attic, Ionic, Doric, and Aeolic derived from the legendary ancestor Hellen. This happened through an unspecified process of change, and as a result different stages of dialects could be distinguished. Other Greek thinkers, however, preferred to ignore the traditional ethno-mythological scheme and projected one particular dialect as the oldest, thus introducing an imbalance into the chronological relationship between the dialects. The mysterious philosopher Pythagoras (\textit{fl.} 6th/5th cent. \textsc{bc}) went so far as to claim that his preferred medium of communication, Doric, was not only the most harmonious but also the oldest Greek dialect, at least if one is to believe his biographer Iamblichus (ca. \textsc{ad} 240–325; \textit{De} \textit{uita} \textit{Pythagorica} 34.242–243). The Early Christian author Epiphanius of Salamis (ca. 310/320–403) seems to have reserved this honor for Ionic, which he associated with the biblical figure of Javan, a son of Japheth and grandson of Noah whom he identified with Ion, the mythological forefather of the Ionians (Van \citealt{Rooy2013}: 44 n.43).

Greek scholarship on the dialects was largely Hellenocentric; other languages were not invoked in treatments of this theme. In Roman times, however, a clearly distinct dimension to the historization of the Greek dialects manifested itself. Roman authors acknowledged that their culture was greatly indebted to the Greek world, a realization that made them consider the idea that this was perhaps also the case in terms of language. In the first centuries \textsc{bc} and \textsc{ad}, several scholars, including Dionysius of Halicarnassus, Varro, and Quintilian, claimed that Latin descended – at least partly – from Aeolic, an idea which some modern scholars have dubbed “Aeolism”, even though it was hardly the full-fledged theory this term might suggest it was.\footnote{\textit{Antiquitates} \textit{Romanae} 1.90.1. On Aeolism, see especially Stevens (2006/2007). For Quintilian, see \citet[149]{Fögen2000}. Cf. also Schöpsdau (1992: 117–119).}

Scarce though language-historical ideas in Greek scholarship may be, early modern scholars gratefully took them as their starting point, quickly going beyond them. Not only did they systematize earlier thought, but – more importantly – they also greatly contributed to a better historical understanding of the history of the Greek language and the place of the dialects in it. They looked farther back in time and asked themselves: “How do the dialects relate to earlier stages of Greek?” They also looked into later developments: “What happened with the dialects after antiquity?” They zoomed out even further still to connect the Greek language in various ways to other tongues, often by means of mechanisms involving either a specific Greek dialect or the concept of \textsc{dialect}. In the remainder of this chapter, I will demonstrate the main contributions of early modern scholars to the better historical understanding of the Greek tongue and its variability.

\subsection{In Strabo’s wake}
\hypertarget{Toc19704836}{}
Strabo’s idea of an original binary division between Attic–Ionic and Doric–Aeolic was an influential one; it was the classical answer humanists offered when treating the question of the historical relationships between the Greek dialects. The Dutch polymath and experienced Hellenist Hugo Grotius (1583–1645) formulated it as follows in a 1622 letter to one of his French contacts:

\begin{quote}
The most ancient division of the Greeks is into Ionians and Dorians, whence a variety of dialects spread itself into several branches, but all of them are to be reduced to these stocks. Just like the Attic dialect is part of the Ionic, but separated from the commonality with Ionic in certain properties, in like manner Aeolic pertains to Doric.\footnote{\citet[143]{Grotius1648}: “Graecorum antiquissima diuisio est, in Iones et Dores; unde dialectorum uarietas in plures se ramos fudit; sed qui omnes ad illas stirpes deferendi sunt. Sicut Attica dialectus pars est Ionicae, sed a communitate Ionicae certis proprietatibus distincta, ita Aeolica ad Doricam pertinet”.}
\end{quote}

This view, truly ubiquitous throughout the entire early modern period, was only discarded after linguistics emerged as a separate field of research in the nineteenth century.\footnote{See e.g. Sabellicus (1490: 64\textsc{\textsuperscript{r}}); \citet[235]{Estienne1573}; \citet[563]{Lancelot1655}; Schwartz \& Helm (1702: \textsc{c.2}\textsc{\textsuperscript{r}}–\textsc{c.2}\textsc{\textsuperscript{v}}); Maittaire (1706: i); \citet[82]{Vitringa1712}; Castelli (1769: \textsc{xv}); Hauptmann (1776: \textsc{a.2}\textsc{\textsuperscript{r}}).} One author held that there were originally four dialects, but that through mixture Doric and Aeolic eventually merged, resulting in three main dialects, thus historically reversing Strabo’s Doric–Aeolic unity \citep[20]{Gedike1782}.

The authority of Strabo eclipsed actual empirical evidence. Indeed, unlike Ionic–Attic unity, Doric–Aeolic identity could not be convincingly corroborated by linguistic facts, even though a few early modern scholars tried to do exactly that. Most notably, Henri Estienne attempted to substantiate such claims of kinship by pointing to a number of alleged linguistic similarities between Aeolic and Doric. Estienne remarked, among other things, that the nominative plural of nouns like \textit{hippeús} ($\text{\textgreek{<i}}\pi \pi \varepsilon \text{\textgreek{'u}}\varsigma $), ‘knight’, was the same in both dialects: \textit{hippêis} ($\text{\textgreek{<i}}\pi \pi \text{\textgreek{~h|}}\varsigma $), as opposed to koine \textit{hippeîs} ($\text{\textgreek{<i}}\pi \pi \varepsilon \text{\textgreek{~i}}\varsigma $).\footnote{Estienne (1581: 25–26): “Sicut enim $\text{\textgreek{<i}}\pi \pi \text{\textgreek{~h|}}\varsigma $ et $\text{\textgreek{<i}}\varepsilon \rho \text{\textgreek{~h|}}\varsigma $ et $\beta \alpha \sigma \iota \lambda \text{\textgreek{~h|}}\varsigma $ pro $\text{\textgreek{<i}}\pi \pi \varepsilon \text{\textgreek{~i}}\varsigma $ et $\text{\textgreek{<i}}\varepsilon \rho \varepsilon \text{\textgreek{~i}}\varsigma $ et $\beta \alpha \sigma \iota \lambda \varepsilon \text{\textgreek{~i}}\varsigma $ dicunt, ita dialectus Aeolica necnon Dorica in infinitiuis hac mutatione utuntur”. Cf. also \citet[179]{Trendelenburg1782}.} The assumption of an originally binary division implied for many scholars, as it had done for Strabo, that there were two forms of each dialect: an older, rougher and a newer, more elegant form (see e.g. \citealt{Mazzocchi1754}: 119). This was emphasized especially often for Attic and Ionic.\footnote{See e.g. Hauptmann \& \citet[18]{Schmid1737}; \citet[137]{Walch1772}; Facius (1782: \textsc{iv–v}).} Some scholars even associated specific linguistic and alphabetic properties with the different diachronic stages of a dialect. The usage of the letter xi (<ξ>) instead of sigma (<σ>), the absence of the letter upsilon (<υ>) in long vowels, and the epigraphic usage of capital eta (<H>) to denote aspiration were associated with Old Attic, claimed to be identical to Ionic and to “degenerate not very much from ancient Hellenic”, conceived as a kind of ancestral Greek language.\footnote{Munthe \& Heiberg (1748: 4–5): “nec ualde degenerans a prisca Hellenica”.}

\subsection{The dialects between Greek and biblical genealogy}
\hypertarget{Toc19704837}{}
Like their ancient and medieval predecessors, early modern scholars too tried to answer the question as to how the dialects and the tribes speaking them fitted into traditional genealogical schemes. As I have repeatedly pointed out, the Greek dialects were usually linked closely – for etiological reasons – with the history of the four main Greek tribes and their mythological forbears. This connection persisted in the early modern period, even though a number of eighteenth-century scholars rationalized the issue and rejected the link with the mythological figures but not that with the Greek tribes.\footnote{See e.g. Walper (1589: 3–4); Labbe (1639: 166–167); \citet[73]{Vitringa1689}; Harles (1778: \textsc{xxiiii–xxvi}). The mythological link was rejected by Thryllitsch (1709: \textsc{c.4}\textsc{\textsuperscript{v}}\textsc{–d.1}\textsc{\textsuperscript{r}}) and Hemsterhuis (2015 [ca. 1740–1765]: 108–110), who stressed that dialectal diversification requires a large time span (see \citealt{Gerretzen1940}: 151–152).} In the very same attempt at demythologizing the Greek dialects, the diversification of the Greek dialects was sometimes related to colonization movements of the Greek tribes. For instance, the Enlightenment pedagogue Friedrich \citet[12]{Gedike1782} pointed out that the dialects could greatly contribute to elucidating the initial stages of the Greek states and their colonies and vice versa.

Scholars also tried to fit the history of the Greek dialects into the genealogical framework of the Bible. This endeavor was still rare in the Greek tradition, the most significant exception being the Early Christian scholar Epiphanius of Salamis, who identified Javan with Ion and claimed that Ionic was the oldest dialect (see \sectref{sec:key:1} above). In the early modern period, attempts at framing the Greek dialects into biblical history were more intensive.\footnote{Schwartz \& Helm (1702: \textsc{c.1}\textsc{\textsuperscript{v}}) were aware of such attempts at reconciliation.} Let me demonstrate this by means of a striking example. In his annotations on a Greek inscription and its peculiar dialect, the Oxford chronologer Thomas Lydiat (1572–1646) initially tried to prove the Hebrew origin of the “founders” (\textit{auctores}) of the three Greek dialects, Dorus, Aeolus, and Xuthus (Ion’s father; Lydiat in \citealt{Prideaux1676}: \textsc{ii}.21). In his later notes, however, Lydiat offered an account that was more in agreement with Greek tradition and seemed to be an expanded version of Strabo’s scheme. At first, he now claimed, there was only one dialect, as long as the Greeks lived in Thessaly. This original dialect subsequently broke up into Aeolic and Ionic. Ionic then disintegrated into Attic and Ionic, whereas Aeolic developed into Aeolic, Boeotian, and Doric (Lydiat in \citealt{Prideaux1676}: \textsc{ii.134}, \textsc{ii.}155). Lydiat failed to see that he was offering contradictory outlines of Greek linguistic history, grounded in different traditions.

\subsection{The early stages of the Greek language}
\hypertarget{Toc19704838}{}
Greek scholars were not very concerned over the origin and early stages of their language, which they usually approached from a static and synchronic perspective. Humanists, however, developed a broad interest in the diachronic development of language and linguistic diversity, from which scholarship on the Greek tongue also benefitted. How did early modern authors sketch the early stages of the Greek language? And what was the place of the dialects in them?

A frequently proposed solution consisted in propagating a specific dialect as the oldest form of Greek. Following biblical genealogy, early modern scholars often claimed Ionic primacy, as Epiphanius had done in late antiquity (e.g. \citealt{Alsted1630}: 2019; Von der \citealt{Hardt1705} [1699]: 17). Due to the assumed close kinship between Attic and Ionic, Ionic primacy came to be equated with Ionic–Attic primacy by some Hellenists (see e.g. \citealt{Schmidt1604}: 5–7). Occasionally, Attic was claimed to be the pristine dialect, from which Ionic, and later on Doric and Aeolic, originated (\citealt{Baile1588}: 4\textsc{\textsuperscript{r}}–5\textsc{\textsuperscript{r}}); this view was possibly motivated by the common idea that Attic was the most elegant dialect, used by the most valued prose authors. Several scholars, often inspired by Iamblichus’ biography of Pythagoras, proposed Doric as the oldest Greek dialect.\footnote{See e.g. Goropius \citet[860]{Becanus1569}; \citet[29]{Burton1657}; \citet[118]{Mazzocchi1754}; Facius (1782: \textsc{iv}); \citet[21]{Gedike1782}.} Aeolic was only rarely suggested to be the oldest dialect. A French orientalist did so in 1697 while oddly stating that this dialect stemmed from Elisa, the son of Javan, from whom he thought the Ionic dialect to have originated \citep[110]{Thomassin1697}. He apparently did not realize that this idea obviously compromised the chronology of biblical genealogy.

In short, scholars often relied on ancient or biblical authorities to propagate one dialect or another as the oldest form of Greek. These proposals were usually not motivated by any linguistic evidence. A major exception to this tendency was, however, the case of Doric primacy. The antiquity of this dialect was often allegedly proved by, among other things, the prevalence of monosyllabic words and the low frequency of double consonants claimed to be inherent to it.\footnote{For the former alleged piece of evidence, see Munthe \& \citet[17]{Heiberg1748}. For the latter, see Harles (1778: \textsc{xxvi).}} Here, the common early modern idea that monosyllabicity indicated antiquity was applied to the Greek language (for this idea, see \citealt{Jansen1995}: 297–300). One German author cited the intrinsic ruggedness of the Doric dialect as evidence for its antiquity, probably presupposing that linguistic cultivation and polishing was a time-consuming process \citep[21]{Gedike1782}.

In the seventeenth and eighteenth century, a number of scholars were not content with simply positing a specific dialect as the most ancient form of the Greek tongue. Instead, they suggested that there was some kind of prehistoric, unitary Greek before the emergence of dialectal diversification. They were in other words gaining deeper insight into the early stages of this language. The French classical scholar Claude de Saumaise, in his 1643 monograph on the Greek language, its origin, and its dialects \citep{Saumaise1643a}, posited an original Hellenic tribe speaking an ancient variety of Greek he dubbed “Hellenic”. This tongue, Saumaise claimed, first came to be divided into the different Greek dialects before evolving into the Greek koine. In Saumaise’s wake, a dissertation presented in 1702 at the Wittenberg academy argued that there was a now extinct ancestral Greek language, which grammarians have been able to distill out of the common features of the different dialects. In other words, the koine was a grammatical reconstruction of the Greek protolanguage, according to this text (see Chapter 2, \sectref{sec:key:9}). Other eighteenth-century scholars likewise presupposed a now lost Greek protolanguage, termed “Pelasgian” and closely associated with Ionic. Let me look at two instructive examples. Firstly, according to the Lutheran theologian Valentin Ernst Löscher (1673–1749), original Pelasgian Greek may be lost, but it partly lives on in the dialects descending from it. These have preserved the original Pelasgian roots to varying degrees of accuracy, with Ionic safeguarding them best. Indeed, out of this dialect, the roots can be reconstructed, Löscher explicitly stated.\footnote{Löscher (1705: 24–25, 84–85), where the Latin verb \textit{restituere} is used to express the notion “to reconstruct”.} Secondly, the Dutch orientalist Albert Schultens frequently compared Oriental (Semitic) with ancient Greek linguistic diversity and emphasized that both have four dialects deriving from one – now lost – common ancestor, called “Pelasgian” or “Ionic” in the case of Greek.\footnote{Schultens (1748: \textsc{lxxv–lxxvi,} \textsc{xcii–xciv,} \textsc{civ}). For Schultens’ concept of a \textsc{Semitic} \textsc{protolanguage}, see e.g. Eskhult (2015: esp. 84–86). See also Chapter 8, \sectref{sec:key:3.1.} of this book.} Schultens formulated a key methodological principle in this context: comparing related dialects helps to penetrate into the nature of the extinct tongue.\footnote{See Schultens in Eskhult (fc. [ca. 1748–1750]: §§\textsc{cxv–cxx}). Cf. also Schultens (1738a: 19–20; in Eskhult fc. [ca. 1748–1750]: §\textsc{cxvi}, §§\textsc{xc–xcii,} where also a Proto-Germanic tongue is suggested).} There were several other eighteenth-century scholars who assumed a prehistoric, extinct Greek language.\footnote{See e.g. Munthe \& Heiberg (1748: 1–2); Hemsterhuis (2015 [ca. 1740–1765]: 104–106); \citet[15]{Wise1758}.} Considering all the evidence cited here, it seems safe to conclude that the concept of \textsc{Proto-Greek} was an achievement of early modern rather than modern language studies, even if no straightforward terminology was coined to express it and no rigorous comparative method was designed to prove it by means of Greek dialect data.

\subsection{The later fate of the Greek dialects: Extinction and vestiges}
\hypertarget{Toc19704839}{}
How did early modern scholars picture the fate of the ancient Greek dialects in late antiquity and beyond? As with the Greek protolanguage, the idea of extinction was central to discussions of this question, as several authors argued that the ancient Greek dialects had perished in late antiquity. This idea was explored most influentially by Claude de Saumaise, who described the Greek language situation at the time of Justinian’s reign (ruled 527–565) as follows:

\begin{quote}
all varieties of the dialects were abolished. […] Finally, it came through a progression of time to the point that all differences of dialects were done away with among the Greeks and a uniform shape of the Greek language spread over the whole of Greece, and an extremely corrupt one.\footnote{Saumaise (1643a: 446–447): “omnes dialectorum uarietates abolefactae sunt. […] Eo postremo deuentum est temporis progressu, ut tollerentur omnes dialectorum differentiae apud Graecos et uniformis facies Graeci sermonis per uniuersam Graeciam diffunderetur eaque corruptissima”.}
\end{quote}

It would be in vain, Saumaise (1643a: 447–449) added, to retrieve the dialectal variation of ancient Greek in the vernacular tongue. As a consequence of the extinction in late antiquity, Byzantine grammarians were not in a position to assign the Greek dialects to the different regions of Greece, as they were no longer spoken. Instead, they linked them to the names of literary authors. This, Saumaise (1643a: 450, 453–455) ingenuously asserted, is also why Byzantine scholars such as John the Grammarian and Gregory of Corinth changed the Greek word \textit{tópos} ($\tau \text{\textgreek{'o}}\pi o\varsigma $) into \textit{túpos} ($\tau \text{\textgreek{'u}}\pi o\varsigma $) in the traditional definition “A dialect is speech showing the particular character of a region (\textit{tópos})/model (\textit{túpos})”. Even though Saumaise may well have been right on this point, it is difficult to back up this conjecture with actual evidence; a thorough study of the complex transmission of these texts could clarify the matter (see Van \citealt{Rooy2016d}: 264 n.47). In Saumaise’s tracks, several scholars suggested similar evolutions for the dialects in late antiquity. For example, in one dissertation, it was claimed that “the common language resembles the Attic dialect most closely, the Attic dialect has entirely obscured the remaining dialects, and, finally, the common language has abolished all” after Alexander the Great’s conquests. The dialects were all “absorbed into the common language”.\footnote{Schörling \& Michaelis (1678: \textsc{a.5}\textsc{\textsuperscript{r}}): “Lingua communis proxime ad Atticam accedit, Attica plane obscurauit reliquas dialectos, communis tandem omnes aboleuit. […] Ita fuere absorptae dialecti sub imperio Seleucidarum in Syria et Ptolomaeorum in Aegypto et in linguam communem redactae”.}

The French philologist Charles Du Cange (1610–1688) refuted Saumaise’s claim that the dialects had perished entirely. Du Cange did so by quoting the account of the Greek scholar Symeon Cabasilas (1546–after 1605), according to which vernacular Greek contained remnants of the four ancient dialects.\footnote{Du Cange (1688: viii), referring to Cabasilas in \citet[462]{Crusius1584}.} Inspired by ideas such as Cabasilas’, a few early modern scholars tried to trace properties of contemporary vernacular Greek to the ancient dialects. The German theologian and grammarian Johann Tribbechow (1677–1712) set out to prove that the “vulgar Greek language” (\textit{lingua} \textit{Graeca} \textit{uulgaris}) had taken elements from all the ancient dialects. Tribbechow did so in a dissertation on the origin and nature of vernacular Greek, prefixed to his grammar of this tongue.\footnote{Tribbechow (1705: a.3\textsc{\textsuperscript{r}}). The Greek grammarian Romanos Nikiforos, writing ca. 1650, likewise relied on the traditional ancient Greek dialects to account for vernacular Greek forms (see e.g. \citealt{Nikiforos1908}: 40, 45).} In contrast to Cabasilas and Du Cange, he took great pains to support his hypothesis by means of empirical – but largely faulty – linguistic evidence. For example, vulgar Greek allegedly followed Attic in supplementing certain verbs – e.g. \textit{aréskō} ($\text{\textgreek{>a}}\rho \text{\textgreek{'e}}\sigma \kappa \omega $) – with the accusative instead of the dative case. Ionic influence was allegedly visible in the accusative and nominative feminine plural forms of the vernacular Greek definitive article; instead of ancient \textit{taîs} ($\tau \alpha \text{\textgreek{~i}}\varsigma $, actually dative case) and \textit{hai} ($\alpha \text{\textgreek{<i}}$), Greeks now wrote in Ionic fashion \textit{têis} ($\tau \text{\textgreek{~h|}}\varsigma $) and \textit{hē} ($\text{\textgreek{<h}}$), respectively – forms likely pronounced [tis] and [i] as in Modern Greek. The high frequency of the letter alpha, in turn, was supposedly inherited from Doric – e.g. in vernacular Epirotic verbal endings such as \textit{epígaman} ($\text{\textgreek{>e}}\pi \text{\textgreek{'h}}\gamma \alpha \mu $αν), ‘we went away’, instead of more usual \textit{epígamen} ($\text{\textgreek{>e}}\pi \text{\textgreek{'h}}\gamma \alpha \mu $$\varepsilon $ν). The addition of the particle -\textit{ske} (-$\sigma \kappa \varepsilon $) was likewise Doric, Tribbechow claimed. The Aeolic dialect allegedly surfaced in the accusative and nominative feminine plural adjective \textit{kalaîs} ($\kappa \alpha \lambda \alpha \text{\textgreek{~i}}\varsigma $), ‘good’ – i.e. Modern Greek \textit{kalés} ($\kappa \alpha \lambda \text{\textgreek{'e}}\varsigma $) – replacing ancient \textit{kalás} ($\kappa \alpha \lambda \text{\textgreek{'a}}\varsigma $) and \textit{kalaí} ($\kappa \alpha \lambda \alpha \text{\textgreek{'i}}$). Tribbechow attributed the absence of aspiration at the beginning of words in vernacular Greek – so-called psilosis – likewise to this dialect. Uncovering the dialectal origin of a vernacular form often demanded great effort, he emphasized (\citealt{Tribbechow1705}: a.3\textsc{\textsuperscript{v}}). Tribbechow’s precise reasons for putting forward this hypothesis are unclear, but it might have been a strategy to elevate the status of vernacular Greek by narrowing the gap with its ancient counterpart. Demonstrating continuity between both forms of Greek would, in this scenario, accord prestige to the vernacular variant he was describing in his grammar. In the nineteenth century, Greek scholars entertained the idea of ancient–vernacular continuity in an adapted form, known as Aeolodorism, the Romantic hypothesis that vernacular Greek derived from the ancient Aeolic and Doric dialects rather than from medieval Greek. This idea, put forward in a number of Greek grammars of the time, was definitively refuted by the Greek linguist Georgios Hatzidakis (1843–1941; see \citealt{Argyropoulos2009}: 289; \citealt{Mackridge2009}: 264–265).

Before Du Cange and Tribbechow, other scholars had already suggested continuity between ancient and vernacular Greek dialects, but in a different fashion. Most significantly, the Tsakonian tongue was correctly accorded a privileged relationship with ancient Greek. The German theologian Stephan Gerlach (1546–1612), a friend of Martin Crusius (1526–1607), was the first to elaborate on its particular status, even though he wrongly labeled speakers of Tsakonian \textit{Ionians} rather than \textit{Dorians}, as one would expect:

\begin{quote}
And all [Greeks], whatever areas they are from, understand each other, with the exception of the Ionians who, inhabiting fourteen villages in the Peloponnese between Nafplio and Monemvasia, use the ancient language, which, however, violates grammar in many respects. They understand a grammatical speaker, but a speaker of the vulgar language only very poorly. These are commonly called Tsakonians.\footnote{Gerlach in \citet[489]{Crusius1584} “Et omnes, quorumcumque locorum, se mutuo intelligunt, exceptis Ionibus, qui in Peloponneso inter Naupliam et Monembasiam, 14. pagos inhabitantes, antiqua lingua, sed multifariam in grammaticam peccante, utuntur, qui grammatice loquentem intelligunt, uulgarem uero linguam minime. Hi Zacones uulgo dicuntur”.}
\end{quote}

This brief remark remained the main source of information on Tsakonian for the remainder of the early modern period, until its rediscovery at the end of the eighteenth century.\footnote{Cf. \citet[44]{Howell1650a} and Du Cange (1688: vii). For its rediscovery, see \citet{Famerie2007}.} Still other scholars seem to have downplayed the differences between ancient and vernacular Greek by intuitively comparing the diachronic variation existing among them to dialect-level differences; vernacular Greek was, in other words, a dialect of the language just like Attic and Doric were.\footnote{See e.g. Castillo (1678: 47–48); Chambers (1728: \textsc{i.}184 [4th sequence of pagination]); Fréret (1809 [1746–1747]: 127–128).} This idea was especially common among Greek scholars active in the late eighteenth century, who posited vernacular Greek as an additional dialect next to the four or five traditional ones (\citealt{Mackridge2009}: 264; 2014: 138–139).

However, the bulk of early modern scholars assumed, usually silently, that there was great \textit{dis}continuity between ancient and vernacular Greek. The fact that vernacular Greek was often characterized as “vulgar” or “barbarous” can be taken to imply that its dialects – in contrast to their ancient counterparts – were also regarded as defective speech forms. Was this view indeed advanced in early modern times? This occurred to a certain extent. For example, it became customary to contrast Attic, the ancient Greek literary variety par excellence, to vernacular Athenian, conceived as the most barbarous and ridiculous form of vernacular Greek. This idea, linked to the decay of Athens, which had become a small provincial town, was first expressed by three acquaintances of the German Philhellene Martin Crusius: the two Greek scholars Theodosius Zygomalas (1544–1607) and Symeon Cabasilas, and the German theologian Stephan Gerlach.\footnote{For Zygomalas, see Crusius (1584: 99, 216). For Cabasilas, see \citet[461]{Crusius1584}. See also Rotolo (1973–1974: 91); Rhoby (2002: 185, 189–190). For Gerlach, see \citet[489]{Crusius1584}. Cf. Ben-\citet[194]{Tov2013}, quoting Michael Neander.} It was repeated several times throughout the early modern period, primarily in German-speaking territory, where Crusius’ \textit{Turcograecia} \REF{ex:key:1584}, a book of miscellanea containing the relevant texts, was best-known.\footnote{See e.g. \citet[215]{Becman1673}; Rodigast (1685: \textsc{a.3}\textsc{\textsuperscript{v}}); Hofmann (1698: \textsc{ii}.824); [Frisch] (1730: 1135); \citet[9]{Gedike1782}.} This does not mean that all vernacular varieties were considered vicious. Scholars promoted several different dialects as the best variety of the contemporary tongue. Most commonly, the speech of the Ottoman capital, Constantinople, was granted this status.\footnote{See e.g. Gerlach in \citet[489]{Crusius1584}; \citet[215]{Becman1673}; \citet[74]{Blount1680}; Du Cange (1688: vii).} Tribbechow (1705: a.4\textsc{\textsuperscript{v}}, a.7\textsc{\textsuperscript{r}}) mentioned the speech of Ioannina, at that time a flourishing intellectual center in Epirus (northwestern Greece), alongside the Constantinopolitan dialect, contrasting both of them to the inferior Greek spoken in the Ceraunian Mountains (modern-day southwestern Albania). The Swiss doctor and language cataloguer Conrad Gessner (1555: 47\textsc{\textsuperscript{r}}), however, believed the speech of the Peloponnese to be the purest, without providing any further specifications.

In short, several scholars seem to have agreed that the ancient dialects perished as native forms of speech in late antiquity. There was, however, discord about the degree of continuity between ancient and vernacular Greek. Some intellectuals argued that there were no traces whatsoever of the ancient dialects in contemporary Greek, whereas others, most importantly Johann Tribbechow, tried to prove that there were clear vestiges of the literary dialects in the vernacular tongue. In the latter case, ennobling vernacular Greek by associating it more closely with its illustrious ancient predecessor may have been an underlying incentive.

\subsection{Aeolism and its early modern transformations}
\hypertarget{Toc19704840}{}
Already during antiquity, Latin had been incorporated into the history of the Greek language and its dialects thanks to the idea known as Aeolism. As I have mentioned above, certain ancient scholars assumed, without relying on much linguistic evidence, if any at all, that Latin descended at least in part from the Aeolic dialect. Early modern scholars usually relied on ancient authorities such as Dionysius of Halicarnassus when claiming that Latin was principally or entirely derived from the Aeolic dialect, generally without adducing any additional proof.\footnote{See e.g. \citet[84]{Crinesius1629}; Bentley (1726: \textsc{xvii}); Hemsterhuis (2015 [ca. 1740–1765]: 76, 106); Munthe \& \citet[30]{Heiberg1748}; Simonis (1752: 215–216). On Renaissance views on the relationship of Latin to Greek, see \citet{Tavoni1986}.} The Dutch jurist Hugo Grotius (1648: 144–146) was exceptional in trying to demonstrate by means of an extensive array of linguistic arguments that Latin derived from Aeolic. Grotius referred, for instance, to the short [a] sound allegedly present in both Aeolic and Latin words such as \textit{pháma} ($\varphi \text{\textgreek{'a}}\mu \alpha $) and \textit{fama}, ‘rumor, reputation’, which contrasted to the Doric long alpha and the Attic–Ionic long eta. Grotius also pointed out that the letter digamma, <F>, is present in the alphabets of both Aeolic and Latin, and that these varieties have similar morphological properties in certain verbal endings. The presenters of an early eighteenth-century dissertation, defended in Wittenberg, went even further than Grotius by systematically comparing and tracing back Latin pronunciation to that of Aeolic.\footnote{\citet{ThryllitschBrunner1709}, which deserves further study.}

Interestingly, a number of authors preferred to adapt Greek history rather than to provide linguistic evidence in order to corroborate Aeolism. The grammarian of Greek Georg Heinrich Ursin (1647–1707), for instance, asked himself, “Where was the Aeolic dialect in usage?”, a question he answered as follows:

\begin{quote}
At first among the Aeolians, a Greek tribe, who left their fatherland, crossed to Asia, and, after establishing a settlement there and founding the region of Aeolis, instituted this dialect. Yet just like also the Doric dialect, which was cognate to it, Aeolic thrived in that part of Italy which is called Magna Graecia. This is also why the Aeolic tongue is, above all, the mother of Latin.\footnote{\citet[509]{Ursin1691}: “\textit{Aeolica} \textit{dialectus} \textit{ubi} \textit{in} \textit{usu} \textit{fuit?} Apud Aeoles Graeciae gentem primum, qui relicta patria in Asiam traiecere sedibusque ibi captis et Aeolide regione condita dialectum hanc instituerunt; quae tamen, ut et Dorica, ei cognata, in Italiae parte illa, quae magna Graecia dicitur, uiguit, unde et Aeolica lingua Latinae potissimum mater est”.}
\end{quote}

In other words, Ursin felt compelled to locate the Aeolians on the Italian peninsula in order to account for the alleged derivation of Latin from Aeolic, even though there were no Aeolic colonies in this area historically.\footnote{Cf. [Schulze] (1711: 289); Ten Kate (1723: \textsc{i}.69); Munthe \& \citet[30]{Heiberg1748}; \citet[89]{Facius1782}; Ries (1786 [1782]: 199) for similar suggestions.} The idea that Aeolians migrated to Italy was no doubt inspired by the ancient myth about the settlement of Arcadians, who allegedly spoke Aeolic Greek, on the peninsula (see \citealt{Lamers2019}: 30, with further references).

Aeolism was not the only solution put forward to account for the supposedly Greek origin of Latin. Some scholars transformed it into what could be dubbed Graecism, the idea that Latin derived from Greek as a whole (see \citealt{Tavoni1986}: 214, 218; \citealt{Lamers2015}: 173–180, 190–192; 2019). \citealt{In1493}, the Byzantine émigré Janus Lascaris (ca. 1445–1535) argued exactly this in his \textit{Florentine} \textit{oration} by means of an elaborate etymological method, citing Doric words – e.g. \textit{pháma} ($\varphi \text{\textgreek{'a}}\mu \alpha $) as the equivalent of Latin \textit{fama} – while also relying on the Aeolic hypothesis \citep[179]{Lamers2015}. The popular \textit{pháma}–\textit{fama} example was probably taken from Priscian (\citealt{Lamers2019}: 36 n.34). Other authors proposed a different dialectal connection, which did make sense geographically: Doric, varieties of which were spoken in Magna Graecia in the south of Italy. The Protestant Hellenist Philipp Melanchthon, for instance, described Doric as follows, also making use of the \textit{pháma}–\textit{fama} example:

\begin{quote}
Doric is Sicilian, very close to Italy, familiar to Theocritus. It changes eta into alpha, e.g. \textit{phḗmē} ($\varphi \text{\textgreek{'h}}\mu \eta $) \textit{pháma} ($\varphi \text{\textgreek{'a}}\mu \alpha $), something which we Latins are also accustomed to when making use of Greek words, as we are neighbors of the Dorians.\footnote{Melanchthon (1518: a.i\textsc{\textsuperscript{v}}): “Dorica Sicula est, Italiae proxima, Theocrito familiaris, η in α mutat, $\varphi \text{\textgreek{'h}}\mu \eta $ $\varphi \text{\textgreek{'a}}\mu \alpha $, id quod et Latini solemus cum Graeca usurpamus, quippe Doribus uicini”.}
\end{quote}

Melanchthon, however, seemingly invoked geographical vicinity and language contact rather than genealogical derivation to explain the parallels between Doric and Latin, even though the term \textit{uicinus} was often used in the Renaissance to express not only closeness in location but also genealogical kinship. Later scholars posited a Doric origin for Latin in a more straightforward fashion, some of whom invoked particular linguistic features to support their claims.\footnote{See e.g. \citet[10]{Sylvius1531}; Estienne (1572: \textsc{i.}14); \citet[208]{Merula1605}; [Frisch] (1730: 1191); Gedike (1782: 20–21). Both Frisch and Gedike adduced linguistic evidence.} Apart from Graecism and Dorism, a number of scholars proposed dialectally mixed solutions for the Greek origin of Latin. This usually consisted in claiming a Doric–Aeolic foundation for the language of ancient Rome.\footnote{E.g. Anon. (1613: 11–12). Cf. Verwey (1684: 304–305); Maittaire (1706: a.2\textsc{\textsuperscript{v}}, 159); \citet[161]{Gesner1774}. Casaubon (1650: 270, 379) suggested Aeolic and Sicilian Greek as the origins of Latin.} Some added to Aeolic and Doric other source languages such as Pelasgian or Etruscan and Umbrian.\footnote{For Pelasgian, see Canini (1555: a.4\textsc{\textsuperscript{r}}). For Etruscan and Umbrian, see \citet[39]{Rüdiger1782}.} One author posited a shared Aeolic–Celtic origin.\footnote{\citet[13]{Nicolson1715}, who did so on the authority of Wolfgang Lazius (1514–1565).}

Not every early modern scholar assumed that there was a close relationship between Latin and Greek. The renowned Hellenist Joseph Justus Scaliger (1540–1609) clearly separated both tongues in his classification of the languages of Europe and believed that they did not show any kinship at all \citep{Scaliger1610}. The orientalist Albert Schultens (1686–1750), in turn, actively refuted the idea that Latin was a dialect of Greek when reflecting on the interrelationships of the so-called Oriental tongues \citep[109]{Schultens1738b}. Still, the predominant early modern view remained that Latin was somehow connected to the Greek language and especially to one or more of its dialects, usually Aeolic and Doric; this was without a doubt connected to ideas about the alleged close kinship of these two Greek dialects and to the ancient presence of Doric varieties in southern regions of the Italian peninsula.

\subsection{The Greek dialects in relation to other tongues}
\hypertarget{Toc19704841}{}
Languages other than Latin were also claimed to have a special connection to certain ancient Greek dialects. Alleged Doric broadness was occasionally used as an argument for positing a special relationship with Syriac, allegedly broad itself (\citealt{Saumaise1643a}: 415–417). The Greek dialects were retraced to the Egyptian language as well, and Attic and especially Doric were even proclaimed to be dialects of it by Lord Monboddo (1714–1799), who accorded a pivotal role to Egyptian civilization in the evolution of human language (\citealt{Monboddo1774}: e.g. 637, 655). The great polymath Gottfried Wilhelm (von) Leibniz (1646–1716), who was also an enthusiastic language scholar, stipulated a particular link between Laconian and German, which he tried to corroborate by pointing to their predilection for ending nouns with the so-called dog’s letter <r> (\citealt{Leibniz1991} [ca. 1712]: 253). Philologists moreover conducted a debate on the status of certain undocumented or poorly known tongues: should Lycaonian and Phrygian be considered Greek dialects or separate languages?\footnote{For Phrygian, compare \citet[465]{Rijcke1684} to \citet[16]{Jablonski1714}. For Lycaonian, see \citet[2]{Jablonski1714}. See also Van Rooy (fc. d).} Such discussions, however, usually revolved around interpretations of ancient texts rather than actual linguistic evidence and were fueled by different presuppositions about what constituted a dialect. Finally, as a result of the emerging early modern interest in the origin and diversity of language, Greek was frequently framed within larger language families. In this context, scholars sometimes pictured it as a dialect of a protolanguage, often termed “Scythian”, and claimed that it differed only in dialectal terms from such tongues as Saxon, Gothic, and Celtic.\footnote{For Greek as a Scythian dialect, see e.g. Court de Gébelin (1778: xxxiv). For the Scythian hypothesis, see the introduction to this chapter. For Saxon, see Casaubon (1650: 139; 190). For Gothic, see Junius (1665: *.3\textsc{\textsuperscript{v}}). For Celtic, see Martin (1727: \textsc{i}.44).} These suggestions were, however, usually not grounded in thorough and systematic linguistic research but rather in an intuitive and sporadic comparison of words and letter changes.

\subsection{By way of conclusion: Linguistic histories of Greek}
\hypertarget{Toc19704842}{}
All in all, attempts at writing the linguistic history of the Greek language and especially that of its dialects remained relatively futile in the early modern period. It is true that there was a great variety of ideas on the Greek dialects and their historical interrelationships as well as their connection with other tongues, especially Latin. However, few Hellenists tried to offer a full and detailed history of the Greek language from prehistoric to contemporary times. A rare and late counterexample is the German Hellenist Daniel Christoph Ries (1741–1825), who proposed an elaborate periodization of the Greek language in his unusually encyclopedic school grammar, published in Mainz. Ries (1786 [1782]: 199–202) organized the history of Greek into three eras. At first, there was one Greek language, which gradually developed into different dialects following political diversification. Subsequent political unification went hand in hand with the elaboration of a common language and linguistic change for the worse, the four main dialects being nevertheless preserved in literary works. Finally, barbarian, principally Ottoman, invasions further corrupted the Greek language. Unlike his predecessors, Ries succeeded in providing his readers with a comprehensive account, but as was common in the early modern period, it focused on language-external circumstances. In fact, the interrelationships of the Greek dialects were usually not corroborated in the first place by an independent study of linguistic data, but by the authority of ancient scholars, especially Strabo. As a result, the geographer’s ill-informed suggestion that Aeolic and Doric were closely related lived on for several centuries.

Histories of the Greek tongue other than Ries’ were usually incomplete, in that they omitted several episodes, often those regarding the later fate of the language.\footnote{Cf. Saumaise (1643a: 267–464), \citet{Burton1657}, \citet{LagerlööfPalmroot1685}, \citet{Rodigast1685}, \citet{Eling1691}, \citet{Florinus1707}, \citet{Reinhard1724}, \citet{MuntheHeiberg1748}, and \citet{Harles1778}.} Still, the authors of these texts often reflected at length on the place of the ancient Greek dialects in the development of the language. The earliest example of an entirely freestanding history of the Greek language seems to be William Burton’s (1609–1657) \textit{History} \textit{of} \textit{the} \textit{Greek} \textit{language}, an oration published in London in 1657 but already held twenty-six years earlier in Oxford.\footnote{The \textit{Historia} \textit{linguae} \textit{Graecae} \textit{methodica} by Stephanus (Étienne) \citet{Simon1615} is nothing more than a grammar of ancient Greek.} In his history, \citet[27]{Burton1657} suggested a mixed dialectal origin for the ancient Greek language on the authority of Strabo. A second important but somewhat peculiar example of a history of the Greek language is Claude de Saumaise’s \textit{Commentary} \textit{on} \textit{the} \textit{Hellenistic} \textit{tongue}. As I have mentioned in the previous chapter, Saumaise tried in this lengthy work to disprove the existence of a so-called Hellenistic dialect. To be able to do so, an elaborate description of the history of the Greek language and its dialects was indispensable to Saumaise’s mind. Most of these histories still require a more thorough and systematic investigation, going beyond the place they accord to the Greek dialects, yet this does not lie within the scope of the present book.

\section{6 Using words like wax: The many mutations of the Greek dialects}
\hypertarget{Toc19704843}{}\begin{styleCatalogusnotities}
“They add, subtract, transmute, invert. What don’t they do? In short, they use words like wax”.\footnote{ \textrm{For the original Latin quote, see Chapter 2, \sectref{sec:key:6.}}} The Venice-based printer Aldus Manutius, the first great publisher of Greek texts, left no doubt about it: the Greeks, especially their poets, could do almost anything with their language thanks to the dialects, a liberty not granted to their Latin colleagues. The sheer endlessness of linguistic variation did not, however, scare early modern Hellenists, who did their best to accurately chart it in their handbooks for the Greek language and its dialects. How did they tackle this thorny issue? And on what sources did they rely? Let me start by briefly sketching the way in which ancient and Byzantine scholars described the Greek language and its diversity.
\end{styleCatalogusnotities}

\subsection{Dialects and the pathology of words}
\hypertarget{Toc19704844}{}\begin{styleCatalogusnotities}
In ancient and medieval treatises on the Greek dialects, linguistic variation was almost as a rule discussed per dialect and its characteristic letter mutations, and not per linguistic category such as phonology or the verbal system.\footnote{ \textrm{A rare exception is Herodian’s $\Pi \varepsilon \rho \text{\textgreek{`i}}$ $\pi \alpha \rho \alpha \gamma \omega \gamma \text{\textgreek{~w}}\nu $ $\gamma \varepsilon \nu \iota \kappa \text{\textgreek{~w}}\nu $ $\text{\textgreek{>a}}\pi \text{\textgreek{`o}}$ $\delta \iota \alpha \lambda \text{\textgreek{'e}}\kappa \tau \omega \nu $, which discusses dialectal deviations within one category: the genitive case.}} This method of presentation was a consequence of the widespread assumption among Greek grammarians that individual dialects showed an inclination toward certain types of letter mutations; this made it self-evident for them to describe these changes per dialect and not per morphological feature. These letter mutations were framed within a methodological framework now known as pathology, because the mutations were usually called “modifications of the word”, \textit{páthē} \textit{tês} \textit{léxeōs} ($\pi \text{\textgreek{'a}}\theta \eta $ $\tau \text{\textgreek{~h}}\varsigma $ $\lambda \text{\textgreek{'e}}\xi \varepsilon \omega \varsigma $) in Greek.\footnote{ \textrm{See \citet{Wackernagel1876}, \citet[150]{Siebenborn1976}, and Lallot (1995: esp. 118) for this framework of pathology and its link with ancient Greek dialect studies.}} A letter was, in this context, conceived in its traditional ancient meaning, conjoining its form and the sound it represented, but the emphasis seems to have been on its formal appearance, as the treatises were conceived as an aid for philologists to understand literary texts written in different dialects.
\end{styleCatalogusnotities}

\begin{styleCatalogusnotities}
One of the many kinds of word modifications was, for instance, \textit{pleonasmós} ($\pi \lambda \varepsilon o\nu \alpha \sigma \mu \text{\textgreek{'o}}\varsigma $), meaning the insertion of an additional letter into a word. In Aeolic, the letter upsilon (<υ>) was usually inserted before a vowel or a rho (<ρ>), according to one ancient grammarian.\footnote{ \textrm{See $\Pi \varepsilon \rho \text{\textgreek{`i}}$ $\kappa \lambda \text{\textgreek{'i}}\sigma \varepsilon \omega \varsigma $ $\text{\textgreek{>o}}\nu o\mu \text{\textgreek{'a}}\tau \omega \nu $ (ed. \citealt{Lentz1870}: 640), a work ascribed to Herodian.}} A systematic discussion of pathology was provided by the ancient grammarian Tryphon in his treatise “On modifications”, \textit{Perì} \textit{pathôn} ($\Pi \varepsilon \rho \text{\textgreek{`i}}$ $\pi \alpha \theta \text{\textgreek{~w}}\nu $), extant in several different versions and eagerly read by early modern Hellenists. The work offered a classification of the various word modifications and exemplified some of them by referring to features of certain Greek dialects. Apart from the four canonical dialects, Tryphon also mentioned Boeotian and Laconian. Sometimes a certain word modification was assigned exclusively to one specific dialect. \textit{Parénthesis} ($\pi \alpha \rho \text{\textgreek{'e}}\nu \theta \varepsilon \sigma \iota \varsigma $), which designated the insertion of a vowel in the middle of a word without creating a new syllable, was supposedly typical of the Ionic dialect (Tryphon, $\Pi \varepsilon \rho \text{\textgreek{`i}}$ $\pi \alpha \theta \text{\textgreek{~w}}\nu $ 1.16). Other word modifications were attributed to several different dialects. Both Ionic and Aeolic were, for example, said to exhibit \textit{diplasiasmós} ($\delta \iota \pi \lambda \alpha \sigma \iota \alpha \sigma \mu \text{\textgreek{'o}}\varsigma $), the doubling of a consonant in the middle of a word without causing an additional syllable to emerge ($\Pi \varepsilon \rho \text{\textgreek{`i}}$ $\pi \alpha \theta \text{\textgreek{~w}}\nu $ 1.17). The point of reference for the modifications was always common Greek, usually identified with the koine or also with what was believed to be common to most or all dialects. These two interpretations of common Greek were not clearly distinguished by Greek grammarians, many of whom viewed the koine as an amalgam of the different dialects (see Chapter 2, \sectref{sec:key:3}).
\end{styleCatalogusnotities}

\begin{styleCatalogusnotities}
Pathology also constituted the background against which the early Byzantine author John the Grammarian developed his three levels of variation to describe the Greek dialects. They differed from each other, John claimed, on the level of entire words, parts of words, and word accidents such as accent and spiritus (see Manutius \textit{et} \textit{al}. 1496: 237\textsc{\textsuperscript{r}}). Greek dialectal variation, in other words, manifested itself in the lexicon and in small modifications within words, either in terms of letters or superficial features, to which John referred using the Aristotelian concept of “accidents”.\footnote{The Greek word John used was $\tau \text{\textgreek{`o}}$ $\sigma \upsilon \mu \beta \varepsilon \beta \eta \kappa \text{\textgreek{'o}}\varsigma $, “the accident”, as opposed to $\text{\textgreek{<h}}$ $o\text{\textgreek{>u}}\sigma \text{\textgreek{'i}}\alpha $, “the essence”.} John focused in his treatise on the latter category, describing in the first place letter mutations in the tradition of pathology as well as accent and spiritus deviations. The description of lexical variation was usually reserved for separate works. These focused either on rare dialectal words – so-called glosses – attributed to regions or cities rather than to the canonical dialects, as in the case of Hesychius, or on Attic words, like the wordlists of Phrynichus (2nd cent. \textsc{ad}), Moeris (?2nd/3rd cent. \textsc{ad}), and Thomas Magister (?1275–1350/1351).\footnote{ \textrm{For Magister’s life dates, see \citet[417]{Baloglou1998}.}} The latter should be explained by the Atticistic movement, which emerged during the Second Sophistic in the first centuries \textsc{ad}; the lexica were intended to cater to the needs of those aspiring to write pure Attic Greek, which functioned as a kind of high-end shibboleth. It served to distinguish true scholars from would-be intellectuals.\footnote{On the Second Sophistic and its fascination with Attic, see \citet{Whitmarsh2005}.}
\end{styleCatalogusnotities}

\begin{styleCatalogusnotities}
Pathology is reminiscent of the Roman framework known as \textit{permutatio} \textit{litterarum}, ‘permutation of letters’, designed for etymological purposes by the polymath Marcus Terentius Varro (116–27 \textsc{bc}) and also outlined by the rhetorician Quintilian (ca. \textsc{ad} 35–100).\footnote{ \textrm{On the} \textrm{\textit{permutatio} \textit{litterarum} }\textrm{and its link with pathology, see Ax (1987: esp. 25–28, 37). On Varro’s etymological method, see e.g. \citet{Pfaffel1981} and Taylor (1996: 7–10,} \textrm{\textit{passim}}).} The number of letter permutations was, however, much more limited than the word modifications in pathology and amounted to four: addition (\textit{adiectio}), omission (\textit{detractio}), transposition (\textit{metathesis}), and permutation (\textit{permutatio}). The precise relationship between both frameworks remains, however, obscure, and further research is required to cast light on it (cf. \citealt{Ax1987}: 25), which lies outside the scope of the present book. Latin authors were, besides, not very interested in Greek dialect variation, with the exception of Priscian, who worked in early sixth-century Byzantium and tried to demonstrate the close connection between Latin and Greek, also through the dialects. He too believed in the Greek (Aeolic) origin of Latin and moreover pointed out the importance of Attic syntax for Latin.\footnote{ \textrm{See in particular Conduché (fc.) and Chapter 2, \sectref{sec:key:5.}}}
\end{styleCatalogusnotities}

\subsection{The heritage of pathology}
\hypertarget{Toc19704845}{}
In the early Renaissance, the description of Greek dialectal features was initially restricted to occasional cursory remarks in grammatical handbooks and limited to the bare necessities for two reasons. On the one hand, these manuals were intended for beginners. The dialect particularities which a grammarian deemed indispensable knowledge for this specific audience were included; all else was wisely omitted. After all, why scare off students by overemphasizing a major difficulty of this language, which in itself was suspect enough because of, among other things, its association with the Orthodox Church? On the other hand, Greek philology was not yet so advanced as to enable scholars to produce adequate descriptions of the Greek dialects. From the second half of the Quattrocento onward, however, scholars became increasingly acquainted with the Greek language and its manifold forms. This was made possible by the wide availability of the triad of dialectological works attributed to John the Grammarian, Plutarch, and Gregory of Corinth (see Chapter 1, \sectref{sec:key:2}). The relative difficulty of these Greek texts, which were inconveniently arranged from a didactic perspective, soon urged scholars to produce treatments of their own, which were more transparent and instructive. In doing so, early modern grammarians frequently opted to discuss dialectal particularities per linguistic feature – usually per part of speech – and not per dialect as was normally the case in the Greek tradition. In other words, these Hellenists adopted a more comparative-contrastive approach, as they treated the different dialectal realizations of a feature in one and the same paragraph and not scattered throughout their work, as their predecessors had done. An early example is Adrien Amerot’s successful and pioneering booklet on the Greek dialects, which focused on variation in nominal and verbal morphology.\footnote{ \textrm{\citet{Amerot1530}. See Hoven (1985: 1–19) on this treatise and its success. See also \citet{Hummel1999}; Chapter 1, \sectref{sec:key:2.}}} This new structure developed naturally out of Renaissance grammars of Greek; these works increasingly contained dialectal information, which was inserted into the sections discussing the relevant part of speech.\footnote{ \textrm{On Renaissance Greek grammars before 1530, see \citet{Botley2010}.}} In fact, it is telling that Amerot’s booklet was actually a separately published extract from his Greek grammar, printed ten years earlier.

Renaissance grammarians of Greek attempted to discover a certain regularity in the dialectal variation they were describing, which led them to formulate rules of change and exceptions to them. The French Hellenist Petrus Antesignanus made the following comment on Attic in his \textit{Appendix} \textit{on} \textit{the} \textit{dialects}, included among his remarks on Nicolaus Clenardus’ Greek grammar:

\begin{quote}
Attic puts tau instead of sigma, as in \textit{glôtta} instead of \textit{glôssa}, “tongue”. This is always observed when there is a double \textit{ss} and sometimes when there is a simple, as in \textit{tḗmeron} instead of \textit{sḗmeron}, “today”.\footnote{ \textrm{\citet[13]{Antesignanus1554}: “Attica ponit τ pro σ, $\gamma \lambda \text{\textgreek{~w}}\tau \tau \alpha $, pro $\gamma \lambda \text{\textgreek{~w}}\sigma \sigma \alpha $, lingua; hoc semper obseruatur, ubi est duplex $\sigma \sigma $; atque interdum ubi est simplex, ut $\tau \text{\textgreek{'h}}\mu \varepsilon \rho o\nu $, pro $\sigma \text{\textgreek{'h}}\mu \varepsilon \rho o\nu $, hodie”.}}
\end{quote}

Antesignanus added that Attic “rejoices” (“gaudet”) in vowel contractions. In the margin, he called such rules “general precepts about the dialects” (“generalia praecepta de dialectis”). These were understood as guiding principles for understanding and recognizing the Greek dialects, mainly through letter changes, and should not be taken as strict grammatical rules without exceptions, let alone as precursors of later sound laws. Antesignanus’ description of the dialects served, in other words, as a philological tool allowing its readers to reach a better understanding of the variability of the Greek tongue rather than as a scientific linguistic account. It is especially revealing what he stated toward the end of his \textit{Appendix}:

\begin{quote}
But do not believe that these things we have said here are observed everywhere in all words. In fact, these do not take place, except in certain words and in certain cases of the parts of speech, which are inflected through cases, and in certain persons and tenses of verbs. We will also add some other, less general rules, if the context allows it.\footnote{ \textrm{\citet[15]{Antesignanus1554}: “Ne uero credas ea quae hic diximus passim in omnibus dictionibus obseruari. Non enim ista locum habent, nisi in certis quibusdam uocibus et certis casibus partium orationis, quae per casus inflectuntur, atque in certis quibusdam personis et temporibus uerborum [...]. Adiciemus quoque nonnullas alias regulas minus generales iuxta locorum opportunitatem” (translation adapted from Van \citealt{Rooy2016c}: 129).}}
\end{quote}

Antesignanus thus stressed the lack of regularity in the precepts he was offering. Some decades later, Henri Estienne (1581: 46–47) stressed the limitations of such dialect rules in much the same manner, even though he suggested that one ought to look for rules that are as general as possible. The fallibility of dialect rules was widely acknowledged by Hellenists.\footnote{ \textrm{Cf. e.g. also Schmidt (1604: 38–39); \citet[2]{Heupel1712}; [Frisch] (1730: 1136, 1139); \citet[299]{Jehne1782}.}} The frequent use of adverbs meaning “sometimes”, “occasionally”, or “frequently” in the formulation of such rules, inherited from the Greek tradition, is therefore not surprising.\footnote{ \textrm{See \citet[53]{Förstel1999}; Van \citet[516]{Rooy2014}. Cf. the adverbs} \textrm{\textit{aliquando}} \textrm{(e.g. \citealt{Walper1589}: 41),} \textrm{\textit{frequenter}} \textrm{(e.g. \citealt{Walper1589}: 42),} \textrm{\textit{interdum}} \textrm{(e.g. \citealt{Antesignanus1554}: 13),}\textrm{ }\textrm{and} \textrm{\textit{quandoque}} \textrm{(e.g. \citealt{Walper1589}: 64).}} Some scholars went so far as to deny the possibility of formulating rules altogether (e.g. \citealt{Camden1595}: \textsc{i}.1\textsc{\textsuperscript{v}}). The Jena academic Johann Andreas Grosch (1717–1796) explicitly opposed grammatical rules to dialects, which he associated with anomaly and irregularity (\citealt{Grosch1753}: 17–18, 24–25). This was no doubt a consequence of the fact that dialects were increasingly seen as anomalous deviations of the analogical standard from the seventeenth century onward. This interpretation of \textsc{dialect} as opposed to \textsc{language} was fostered by ideas on vernacular dialects, which could not boast such a rich literary tradition as the Greek dialects had and which scholars associated with the lower classes and their allegedly depraved speech.\footnote{ \textrm{See Van Rooy (fc. d) for the history of this interpretation of the} \textrm{\textsc{dialect}} \textrm{concept.}}

A number of Hellenists debated the validity of individual dialect rules at some length. Some even raised fundamental objections against the methods and approaches of their predecessors. The German scholar Georg Heinrich \citet[512]{Ursin1691} deplored that the Greek dialects were a source of discord among grammarians up to the point that some of them even contradicted themselves. Ursin moreover warned his readers not to forge new dialects – he was no doubt thinking of the so-called poetical dialect (see Chapter 2, \sectref{sec:key:7}) – stressing the importance of considering actual usage in ancient Greek literary texts. This was part of a broader tendency, as many scholars claimed to rely on their own reading of Greek texts when formulating and exemplifying dialect rules, even though some of them were simply drawing on their predecessors for the greater part.\footnote{ \textrm{See e.g. Walper (1589: †.7}\textrm{\textsc{\textsuperscript{r}}}\textrm{); Portus (1603: )(.4}\textrm{\textsc{\textsuperscript{r}}}\textrm{); Mérigon (1621: 5}\textrm{\textsc{\textsuperscript{[a]}}}\textrm{); Nibbe (1725: b.4}\textrm{\textsc{\textsuperscript{v}}}\textrm{–b.5}\textrm{\textsc{\textsuperscript{r}}}\textrm{, 432).}} An in-depth analysis of Greek grammatical works, outside the scope of this book, could shed more light on the innovativeness of each scholar and his exact method in describing Greek dialectal variation.

Early modern Hellenists conducted their analyses of dialectal variation principally in the spirit of their ancient Greek and Byzantine predecessors, whose methodology they largely appropriated, even if they opted to present the matter differently. As a matter of fact, the Greek framework of pathology proved to be keenly used by early modern scholars to account for dialectal differences, not only by specialists of Greek, but also by grammarians of vernacular tongues such as German and English who had mastered Greek.\footnote{ \textrm{For Greek, see e.g. Melanchthon (1518: b.iv}\textrm{\textsc{\textsuperscript{v}}}\textrm{); Baile (1588: 7}\textrm{\textsc{\textsuperscript{v}}}\textrm{–11}\textrm{\textsc{\textsuperscript{v}}}\textrm{); \citet[11]{Schmidt1604}; Hill (1658: 2–8, 20–22). For German, see e.g. \citet{Wolf1578}. For English, see e.g. Gill (1619: 130–133).}} Early on, a cross-fertilization with the Roman letter permutation framework seems to have taken place, which surfaces in the terminology used by certain grammarians. The Marburg Hellenist Otto Walper, for instance, discussed the “permutation of vowel letters” of the Doric dialect.\footnote{ \textrm{\citet[62]{Walper1589}:} \textrm{\textit{permutatio} \textit{litterarum} \textit{uocalium}}.} Walper noted, among other things, the omission of the letter jota and the addition of the very same letter in other contexts.\footnote{ \textrm{\citet[63]{Walper1589}: “Deinde iota frequenter detrahunt” \& “Rursum iota ad $o$ apponunt”.}} He moreover mentioned letter transpositions as well as a considerable number of letter changes.\footnote{ \textrm{See e.g. \citet[63]{Walper1589}, “per metathesin litterarum” (i.e. in $\text{\textgreek{<r}}\text{\textgreek{'e}}\zeta \omega $ becoming $\text{\textgreek{>'e}}\rho \zeta \omega $), and \citet[64]{Walper1589}, “Item θ quandoque mutatur in χ, ut $\text{\textgreek{>'o}}\rho \nu \iota \chi \alpha $, pro $\text{\textgreek{>'o}}\rho \nu \iota \theta \alpha $”, respectively.}} These corresponded to the four main letter change processes of the \textit{permutatio} \textit{litterarum}, which were all present in the framework of pathology too.\footnote{ \textrm{See Tryphon’s $\Pi \varepsilon \rho \text{\textgreek{`i}}$ $\pi \alpha \theta \text{\textgreek{~w}}\nu $ and especially Amerot’s (1520:} \textrm{\textsc{p.}}\textrm{iv}\textrm{\textsc{\textsuperscript{v}}}\textrm{)} \textrm{adaptation of Tryphon’s treatise.}}

Many of the letter changes in the Greek dialects were so widely known that they were sometimes used to formulate generally fictitious etymologies of words in various languages. Scholars assumed that if a letter change could occur among Greek dialects, the very same change could take place in other linguistic contexts, distinct in place and time, as well. So it became possible for Philipp Clüver (1580–1622), a geographer and historian from Danzig, to derive the English toponym \textit{Thetford} from Celtic \textit{Sitomagus} by appropriating a letter change known from Greek: “\textit{Sit}, moreover, could have been as easily changed into \textit{Thet} by a variation of dialect as the Greeks’ \textit{Theós} [“god”] into \textit{Siós}”.\footnote{ \textrm{\citet[64]{Clüver1616}: “}\textrm{\textit{Sit}} \textrm{autem tam facile, uariatione dialecti, mutari potuit in} \textrm{\textit{Thet}}\textrm{, quam Graecorum $\Theta \varepsilon \text{\textgreek{`o}}\varsigma $ in $\Sigma \iota \text{\textgreek{`o}}\varsigma $”. See Metcalf (2013 [1972]: 114–115).}} What is more, knowledge of Greek letter changes seems to have heightened scholars’ awareness of similar variations in their own vernaculars. In fact, they sometimes tried to justify vernacular differences by stressing parallel changes in Greek.\footnote{ \textrm{For more details on Greek–vernacular parallels, see Chapter 8, especially \sectref{sec:key:1.2.}}} Comparisons in the other direction could serve to help the reader understand the nature of Greek dialect changes. In his well-known dialogue on the ancient pronunciation of Latin and Greek, Desiderius Erasmus mentioned the change of the letter <r> into <s> in the French of Parisian women as a way of clarifying a similar phenomenon in Greek.\footnote{ \textrm{\citet[52]{Erasmus1528}: “Hanc asperitatem quidam mitigant supposito σ, ut $\theta \alpha \rho \sigma \varepsilon \text{\textgreek{~i}}\nu $ pro $\theta \alpha \rho \rho \varepsilon \text{\textgreek{~i}}\nu $. Idem faciunt hodie mulierculae Parisinae, pro} \textrm{\textit{Maria}} \textrm{sonantes} \textrm{\textit{Masia}}\textrm{, pro} \textrm{\textit{ma} \textit{mere} \textit{ma} \textit{mese}}\textrm{”. Cf. Chapter 8, especially \sectref{sec:key:1.1.}}} Similarities between letter changes were also adduced to support the link between a contemporary people – e.g. the Venetians – and an ancient Greek tribe – e.g. the Ionians.\footnote{ \textrm{See Da Ponte (1509: 97}\textrm{\textsc{\textsuperscript{r}}}\textrm{). Cf. Reitz’ (1730: e.g., 122, 125, 126–127) efforts to establish kinship between Germanic and ancient Greek (for which, see Van \citealt{Hal2016}).}} Early modern Greeks also relied on ancient Greek letter changes to account for properties of the vernacular Greek language. \citealt{Around1650}, a Greek grammarian even oddly claimed that the Turks Doricized the prepositional phrase \textit{stḕn} \textit{pólin} ($\sigma \tau \text{\textgreek{`h}}\nu $ $\pi \text{\textgreek{'o}}\lambda \iota \nu $), ‘to the city’, at that time pronounced as \textit{stimbolin}, into \textit{stampól}/\textit{stamból} ($\sigma \tau \alpha \mu \pi \text{\textgreek{'o}}\lambda $), which allegedly resulted in the toponym of the well-known city of Istanbul.\footnote{ \textrm{\citet[14]{Nikiforos1908}. See also \citet[35]{Nikiforos1908}, referring to Ionic and Attic letter particularities. Cf.} also Rodigast (1685: \textsc{a.4}\textsc{\textsuperscript{r}}).}

More at the margins, the intense early modern debate over the correct pronunciation of ancient Greek also provoked analyses of specific letter changes across the dialects. There were two main camps in this discussion: those defending an itacist pronunciation, largely corresponding to vernacular Greek pronunciation and connected with the Pforzheim Hellenist Johann Reuchlin, and those propounding a reconstructed etacist pronunciation, resembling that of fifth-century \textsc{bc} Attic and closely associated with the proposals of Desiderius Erasmus.\footnote{See Sandys (1908: \textsc{ii.}130). See \citet{Bywater1908} on the Erasmian pronunciation and its precursors.} Already in Erasmus’ dialogue on Greek pronunciation of 1528, frequent allusion was made to Greek dialectal features (e.g. \citealt{Erasmus1528}: 52, 106). He did so for various reasons, among other things to prove the cognate nature of certain letters, such as alpha and eta, which often changed across dialects \citep[62]{Erasmus1528}. Early modern, especially seventeenth- and eighteenth-century, views on ancient Greek pronunciation deserve closer attention, as does the role of dialectal evidence in this context.

\subsection{Beyond letter changes}
\hypertarget{Toc19704846}{}\begin{styleCatalogusnotities}
Letter changes, although vastly important, are not the entire story. Early modern Hellenists often included other types of dialectal differences in their descriptions, ranging from accent and spiritus through alphabet, morphology, and lexicon to syntax and even style. The view that dialectal variation affected style – often termed \textit{syntaxis} \textit{figurata} – was, much like ideas about other levels of variation, to a large extent inherited from the Greek tradition, in which certain rhetorical figures were claimed to be specific to a dialect.\footnote{ \textrm{See, most importantly, Lesbonax’} \textrm{\textit{De} \textit{figuris}}\textrm{, the source of, among others, Saumaise (1643a: 145–146).}} This link between dialects and stylistic peculiarities was related to the literary status of the canonical dialects. As to the alphabet, scholars were aware of certain peculiarities across the dialects, most notably the digamma. This letter was generally seen as exclusive to Aeolic, in spite of the fact that it was also used in codifications of non-Aeolic varieties of Greek. This can be explained by the fact the philological focus was initially on literary dialects and only texts in Aeolic contained this letter. The ancient grammarian Tryphon, however, already referred to the wider application of the digamma ($\Pi \varepsilon \rho \text{\textgreek{`i}}$ $\pi \alpha \theta \text{\textgreek{~w}}\nu $ 1.11), later confirmed by inscriptional evidence. The phonetic value of this Greek letter was correctly explained by such ancient authors as Dionysius of Halicarnassus, who identified it with a [w] sound, the voiced labiovelar approximant (\textit{Antiquitates} \textit{Romanae} 1.20.3). Yet early modern scholars experienced great difficulty in trying to discover the value of this letter with the limited evidence available to them. Erasmus (1528: 68–69, 108), in his aforementioned dialogue on the correct pronunciation of Latin and Greek, first accorded the digamma a value between [w] and [p\textsuperscript{h}], but then claimed that it stood for a [w] sound only (see Kramer in \citealt{Erasmus1978}: 177 n.361). Other scholars took it to express a [f] or a [v] sound or presumed that it had several different values.\footnote{ \textrm{For [f], see e.g. \citet[4]{Sylvius1531} and \citet[5]{Rhenius1626}. For [v], see Fréret (1809 [1746–1747]: 108–109), who noted the presence of the digamma in inscriptions on ancient medallions of Aeolic cities. For the idea that the digamma had different values, see e.g. Canini (1555: 107–108) and Thryllitsch \& Brunner (1709: b.1}\textrm{\textsc{\textsuperscript{r}}}\textrm{–b.2}\textrm{\textsc{\textsuperscript{v}}}).} Still others correctly recognized its [w] value, sometimes inspired by ancient authors such as Dionysius.\footnote{ \textrm{See e.g. Kirchmaier \& Crusius (1684: b.4}\textrm{\textsc{\textsuperscript{r}}}\textrm{) }\textrm{and \citet[19]{Reynolds1752}.}} In the eighteenth century, epigraphic evidence made a number of philologists and antiquarians realize that there were other graphemes, such as <${\sqsubset}$>, which could express the [w] sound in varieties of ancient Greek.\footnote{ \textrm{See e.g. Mazzocchi (1754: 128–130), where, besides, the digamma was interpreted as having multiple values.}}
\end{styleCatalogusnotities}

\subsection{Debating dialectal features}
\hypertarget{Toc19704847}{}
The case of the digamma raises the question as to whether it became a more frequent occurrence that scholars discussed a specific dialect feature at greater length and offered various interpretations of it. Looking at early modern views on letter changes believed to occur among the Greek dialects and at ideas about other dialectal particularities – on the level of syntax, for instance – one is left to conclude that the case of the digamma was relatively exceptional. Indeed, early modern discussions of dialectal features were usually of a highly rigid nature, in that they generally complied with Greek tradition even if the dialect rule in question could not be supported by actual evidence found in extant Greek texts. For example, according to the traditional view, it was peculiar to people from Attica to use the vocative where one would expect a nominative and vice versa.\footnote{See e.g. Apollonius Dyscolus, \textit{De} \textit{constructione} 3, Uhlig page 301; Gregory of Corinth, \textit{De} \textit{dialectis} 2.41 \& 2.53, where only examples from poetry – Homer and tragedy – are offered. Cf. also Priscian, \textit{Institutiones} \textit{grammaticae}, book 17 (ed. Martin Hertz in \citealt{Keil1855}–1880: III.208).} Less frequently, the feature was attributed to Macedonian and Thessalian too.\footnote{Apollonius Dyscolus, \textit{De} \textit{constructione} 3, Uhlig page 301. Cf. also Priscian, \textit{Institutiones} \textit{grammaticae}, book 17 (ed. Martin Hertz in \citealt{Keil1855}–1880: III.208).} This idea was adopted unquestioningly by most early modern scholars. The Swiss Hellenist and physician Martin Ruland the Elder (1532–1602) dubbed it a “rule” (\textit{regula}) in his handbook for the Greek language and its dialects, while trying to demonstrate it not only with examples from pagan literature, but also by means of passages from the Septuagint and the New Testament (\citealt{Ruland1556}: 251; cf. Chapter 4, \sectref{sec:key:5}). \citet[302]{Ruland1556} extrapolated the feature to Thessalian, for which he was likely inspired by Apollonius Dyscolus or Priscian.\footnote{Cf. e.g. also Da Ponte (1509: 36\textsc{\textsuperscript{r}}); \citet[216]{Vergara1537}; Núñez (1555/1556: 50\textsc{\textsuperscript{v}}); Dabercusius (1577: \textsc{x.1}\textsc{\textsuperscript{v}}); Rhenius (1626: 5 [second pagination sequence]); Pasor (1632: 8–9); Wyss (1650: 85–87); \citet[88]{Leusden1670}. Kirchmaier \& Thryllitsch (1709: \textsc{b.3}\textsc{\textsuperscript{r}}) regarded it as a Macedonian feature, referring to Priscian.} Other scholars took it to be particular to poetry or even to the koine, which was claimed to imitate Attic.\footnote{For poetry, see e.g. Amerot (1520: \textsc{q.}i\textsc{\textsuperscript{v}}); \citet[129]{Antesignanus1554}; \citet[34]{Gretser1593}; \citet[157]{Schmidt1604}. For the koine, see \citet[54]{Lancelot1655}.} The syntactic particularity nevertheless remained closely associated with Attic and even led the influential grammarian Nicolaus Clenardus to posit that, in Attic, the nominative and vocative were morphologically identical, a misconception prominent in early modern Greek grammars.\footnote{Clenardus (1530: 7 [misprint for 6]). See e.g. also \citet[534]{Crusius1558}; Baile (1588: 12\textsc{\textsuperscript{r}}); Walper (1589: 11, 37); Gretser (1593: 32, 34); Lancelot (1655: 53, 453); \citet[101]{Giraudeau1739}; \citet[20]{Facius1782}.}

The renowned printer and Hellenist Henri Estienne fiercely refuted the widespread idea that Attic authors used the nominative instead of the vocative in his \textit{Comments} \textit{on} \textit{the} \textit{particularities} \textit{of} \textit{the} \textit{Attic} \textit{language} \textit{or} \textit{dialect}, for which he relied on the actual usage of Attic authors such as Thucydides and other writers. Instead, Estienne ascribed this feature to the Boeotian and Aeolic dialects, a view he oddly supported by referring to the Byzantine grammarian Eustathius of Thessalonica rather than by actual usage in Boeotian and Aeolic texts.\footnote{\citet[15]{Estienne1573}. For Aeolic, see also Schmidt (1604: ):(.4\textsc{\textsuperscript{v}}). For Boeotian, see also \citet[71]{Mérigon1621}, who also believed it to be a Doric feature (cf. \citealt{Maittaire1706}: 257–258).} Clearly, Estienne was not making progress here. Although correctly refuting on the basis of empirical evidence the faulty idea that the nominative was used instead of the vocative in Attic, he at the same time attributed the feature to other dialects by invoking only the authority of a Byzantine scholar. Even more, a little further on, Estienne argued, in agreement with Clenardus and others, that, in Attic, the nominative and vocative were morphologically identical, while dismissing the idea that the vocative was replaced by the nominative on the syntactic level.\footnote{\citet[17]{Estienne1573}. See also Estienne (1573: 29, 42–43, 150), where his views are recapitulated.} Estienne thus showed himself to be more critical toward the canonical dialectal features transmitted by Greek tradition, which were blindly adopted by many early modern Hellenists, even though this inquisitive attitude still failed to bring him to correct insights.

Estienne himself was reproached in the early eighteenth century by the Hellenist Georg Friedrich Thryllitsch (1709: d.3\textsuperscript{v}) for failing to be consistent in his attitude toward dialectal particularities. Thryllitsch noticed a contradiction in Estienne’s work. In his commentary on Attic, \citet[13]{Estienne1573} contended that Thucydides wrote \textit{thálassa} ($\theta \text{\textgreek{'a}}\lambda \alpha \sigma \sigma \alpha $), even though he had printed \textit{thálatta} ($\theta \text{\textgreek{'a}}\lambda \alpha \tau \tau \alpha $) in his edition of the Greek historiographer’s work, thus failing to put his views into actual practice. This critique seems somewhat unfair in view of the fact that Estienne published his commentary on \citet{Attic1573} nine years after his edition of \citet{Thucydides1564}, in which period of time he might have studied Thucydides’ language more closely.

Even though Estienne’s case reveals that there could be detailed discussions of the validity of certain dialect rules, this remained relatively rare and one could say that early modern analyses of the Greek dialects lacked the empirical focus and dialogic interaction necessary for achieving considerable scholarly progress. However, there were a number of critical voices other than Estienne’s as well. Some scholars were cautious about attributing certain properties to a specific dialect. As a matter of fact, the authors of a dissertation on Atticism presented at Leipzig in 1737, Johann Gottfried Hauptmann (1712–1782) and Christian Ernst Schmid (1715–1786), even formulated a kind of methodological precept stipulating that a feature was not to be labeled Attic straight away if it was used by only one author writing in that dialect (\citealt{HauptmannSchmid1737}: 16). Moreover, if only one example of a certain linguistic feature could be found in an author – however eminent – or if it was typical of poets, it should be considered an idiomatic peculiarity or a poetic feature rather than an Atticism. Some decades earlier, a German Hellenist had suggested a similar methodological precept, albeit from the reverse perspective; determining the identity of a Greek author’s dialect must be based on the general appearance of his language and not on one or two particular words and their features.\footnote{Ursin (1691: 495–496). See e.g. also the critical approach toward earlier sources of \citet{Walper1589}; Portus (1603: )(.3\textsc{\textsuperscript{r}}); Gedike (1782: 10–11).} In both cases, a thorough knowledge of the Greek literary dialects was presupposed, and one can once again see right away how crucial the philological incentive was in studying Greek linguistic diversity (see Chapter 3).

When scholars described Greek dialectal particularities, the same examples tended to recur, since they were often taken from traditional Greek treatises on the subject and from the grammatical work of their early modern predecessors. They could be supplemented by a scholar’s own reading of Greek literary texts, including the Septuagint and the New Testament, and – at a later stage and much less frequently so – inscriptions.\footnote{For pagan literary texts, see e.g. Amerot (1520; 1530). For the Greek Bible, see e.g. \citet{Pasor1632}.} The French-born Hellenist Michael Maittaire (1668–1747), who worked as a teacher in England, relied on inscriptional evidence from steles and coins to describe and exemplify certain Doric particularities as well as to reconstruct the ancient orthography of Greek, which according to him was close to that of Latin.\footnote{Maittaire (1706: e.g. 161–167, 170, 184, 205–206, 211–212, 221, 240, 243).} On rare occasions, scholars tried to introduce new dialectal features into the canon on the basis of inscriptional evidence. The English clergyman Thomas Lydiat assumed that the change of [n] into [m] at the end of a word before [b], [m], [p], or [p\textsuperscript{h}] was a particularity of Ionic, since he had found this feature in an Ionic inscription (Lydiat in \citealt{Prideaux1676}: \textsc{ii}.116). This was, in fact, nothing more than a somewhat clumsy solution to account for what we would today consider a straightforward case of phonological assimilation in front of a labial sound.

In sum, discussions of Greek dialectal features were normally not very animated. Some Hellenists did put forward more innovative views, even though these usually remained at the margins of early modern scholarship. Let me round off by citing two final intriguing examples from the sixteenth century. Firstly, Henri Estienne innovatively tried to map out, in some detail, currents of interdialectal influence (i.e. the introduction of certain dialectal features of one dialect into another); Estienne (1581: 22–28) did so with specific attention to Attic and its alleged adoption of Ionic, Doric, and Aeolic elements. It shows that he did not consider the Greek dialects, including the revered Attic dialect, to be stable closed systems but forms of speech susceptible to external influence. Secondly, the first Spanish grammarian of Greek, Francisco de Vergara, made an interesting idiosyncratic remark on what is now called the “deictic iota”, for instance, in \textit{toutí} ($\tau o\upsilon \tau \text{\textgreek{'i}}$), “this here”, instead of \textit{toûto} ($\tau o\text{\textgreek{~u}}\tau o$), “this”. Although following the traditional faulty idea that the deictic iota was an exclusively Attic feature, Vergara was at the same time uniquely aware of its pragmatic function, as he revealingly suggested that it was used “to indicate an object more clearly and as if with a certain gesture”.\footnote{\citet[218]{Vergara1537}: “clarius ac ueluti gestu quodam rem indicent”.}

\subsection{Conclusion}
\hypertarget{Toc19704848}{}
There seems to have been a consensus among early modern Hellenists that the Greek dialects exhibited certain regular variations and that rules could be formulated to grasp dialectal changes even if these were, scholars widely agreed, by no means without exception. Letter changes constituted the focus of attention, a tendency enhanced by the seeming merger of two ancient, letter-centered frameworks: Greek pathology and Roman letter permutation, the interplay between which deserves further study. Despite the focus on letter variation, the Greek dialects were believed to differ from each other on every possible linguistic level, including accent, lexicon, syntax, and even style. Dialect rules – often dubbed \textit{regulae} or \textit{leges} in Latin – were largely adopted from Greek scholarship, with relatively limited room for adaptation and innovation. The rigidity of Greek dialect descriptions can be at least partly explained by the fact that the Greek language and its many forms were not studied in and for themselves, but nearly always with reference to reading and understanding ancient Greek literature (see Chapter 3). In other words, early modern scholars continued Greek tradition in the sense that they also primarily associated the Greek dialects with literature. Critical voices such as Henri Estienne’s were exceptional and did not always bring about a change for the better. To put it differently, the contribution of early modern scholars to ancient Greek dialectology seems to have been rather modest on the micro-level of linguistic description. However, as I have stressed before, the mere fact that modern scholars like Ahrens relied in part on early modern scholarship when they were laying the foundations of modern forms of ancient Greek dialectology calls for a more systematic and comprehensive investigation of early modern approaches toward Greek dialectal features. This holds especially true if one reckons that scholars like Thomas Lydiat and Michael Maittaire, both active in England, where one of the first extensive collections of Greek inscriptions was published (i.e. \citealt{Prideaux1676}), increasingly included data from non-literary sources in their discussions of the dialects. Even though they usually tried to understand such data within the traditional framework as designed by Greek and maintained by early modern scholarship, this evolution was of paramount importance for the development of ancient Greek dialectology in the nineteenth century, as it helped to break up the absolute monopoly of literary texts in the corpus of dialectal source material.

\section{\textsc{7} Picturing ancient Greece through the dialects}
\hypertarget{Toc19704849}{}
When in 1579 Franciscus Junius the Elder (1545–1602) held his \textit{Discourse} \textit{on} \textit{the} \textit{antiquity} \textit{and} \textit{excellence} \textit{of} \textit{the} \textit{Hebrew} \textit{language} at the short-lived Reformed academy of Neustadt, he could not resist emphasizing the merits of this sacred tongue vis-à-vis the Greek language:

\begin{quote}
Indeed, as to individual words, fluency of expression is achieved by the fact that there are neither innumerous words nor so many dialects [in Hebrew] as among the verbose and mendacious Greeks, since almost every single author among them seems to have forged himself his own language because of a certain malicious rivalry.\footnote{Junius (1579: \textsc{b.3}\textsc{\textsuperscript{v}}): “In uocibus enim singulis pertinet ad facilitatem istud, quod non habentur innumerae uoces neque dialecti tam multae, ut apud uerbosos et mendaces Graecos, quorum singuli paene auctores suam sibi linguam cacozelo quodam uidentur fabricasse”. This discourse was reprinted in Junius’ Hebrew grammar (\citealt{Junius1580}: ẽ.ii\textsc{\textsuperscript{v}}–ẽ.iii\textsc{\textsuperscript{r}}). The word \textit{cacozelus} (< Greek $\kappa \alpha \kappa \text{\textgreek{'o}}\zeta \eta \lambda o\varsigma $) can mean both ‘using a bad, affected style’ and – in the neuter ($\tau \text{\textgreek{`o}}$ $\kappa \alpha \kappa \text{\textgreek{'o}}\zeta \eta \lambda o\nu $) – ‘unhappy imitation; rivalry’ (\citealt{LiddellScott1940}: s.v.). Here, “cacozelo quodam” must be interpreted as an ablative of the substantivized adjective expressing a cause.}
\end{quote}

Junius’ observation on the uniformity of Hebrew, favorably compared to the endlessly affected variation of Greek, betrays his negative ideas about the countless differences existing among the Greek dialects. It moreover shows that he connected the Greek dialects to other aspects of Greekness, in this case the Greeks’ innate verbosity, mendacity, and malicious competitiveness. Junius was not the only scholar to do so. Numerous early modern thinkers related dialectal differences existing in Greek to language-external aspects of ancient Greece. How and why did they do so? And to what extent were they inspired by ancient and medieval sources?

\subsection{Texts and tribes}
\hypertarget{Toc19704850}{}
As the Greek dialects were principally studied for philological reasons, scholars associated them closely with the literary texts composed in them (see already Chapter 3). As a matter of fact, ever since antiquity, it had been customary to link a dialect primarily to an author or a group of authors. Aeolic was written by authors such as Alcaeus and Sappho, Attic by Plato and Thucydides, Doric by Alcman and Theocritus, and Ionic by Herodotus and Hippocrates. Oddly enough, several Greek scholars mistook Pindar’s language for the koine, a misconception definitively corrected only in the Renaissance. The Italian Hellenist Angelo Canini (1521–1557) was already able to rightly identify the poet’s speech as principally Doric (\citealt{Canini1555}: a.4\textsc{\textsuperscript{r}}). The dialects were moreover tied up with specific literary genres. Doric was, for example, the usual dialect of bucolic poetry and tragic choral odes. At the same time, the dialects were also associated with the homogeneously conceived tribes speaking them. Aeolic was the dialect of the Aeolian Greeks, Doric of the Dorians, Ionic of the Ionians, and Attic of the inhabitants of Attica. This coincidental close linking of the dialects with literature, on the one hand, and the people speaking them, on the other, made authors prone to transferring evaluative labels associated with literary genres and Greek tribes to the dialects themselves. In this and the following sections I will focus on such dialect attitudes.

Research into language and dialect attitudes in general is a recent, though well-established field of investigation (see e.g. \citealt{Edwards2009}: 73–98; \citealt{Garrett2010}: 19–29). It studies what qualities and vices are ascribed to specific speech forms, and how and why this happens. In other words, it endeavors to map out the impressions languages and dialects convey on speakers. Such impressions are often construed or reinforced by cultural stereotypes – i.e. assumptions about the alleged characteristics of specific regions and ethnic groups – so that the study of language and dialect attitudes may be considered a contribution to imagology as well (on imagology, see \citealt{BellerLeerssen2007}). Early modern attitudes to other languages and dialects have already received considerable attention. William J. \citet{Jones1999}, for instance, has studied the attitudes of early modern German scholars toward European languages. However, no systematic treatment of early modern attitudes toward the ancient Greek dialects exists, which is why I aim to offer a first exploration of the matter here, with a focus on attitudes toward the canonical four dialects: Aeolic, Attic, Doric, and Ionic.\footnote{For attitudes toward Attic in early modern German works, see already the brief account of Roelcke (2014: 251–252).} Here, too, it is impossible to understand early modern views separately from ancient Greek and Byzantine ideas. For this reason, I will very briefly delve into Greek views first.

\subsection{Dialect attitudes from antiquity to early modernity}
\hypertarget{Toc19704851}{}
Ancient and Byzantine Greek authors expressed their assessments of individual dialects at various occasions in their works, almost as a rule in passing. This occurred in diverging genres, including works of grammar, philosophy, history, geography, rhetoric, and even poetry. As most relevant comments are of a cursory nature, there was no canonized, generally accepted evaluation of the Greek dialects. Some ancient Roman authors also attributed labels to Greek dialects in the same sporadic fashion. \figref{fig:key:4} offers a synoptic overview of the most important ancient and medieval attitudes toward the dialects. It suggests that negative labels were more numerous than positive ones. This does not indicate, however, that the canonical dialects were predominantly assessed in a negative way. Many of the unfavorable evaluations were only mentioned by one author, such as the label of “barbarian” in the case of (Lesbian) Aeolic, whereas some of the positive labels were widespread, in particular the eloquence and elegance of Attic.

\begin{stylecaption}
Figure \stepcounter{Figure}{\theFigure}: Ancient and medieval attitudes toward the canonical Greek dialects
\end{stylecaption}

\tablefirsthead{}

\tabletail{}
\tablelasttail{}
\begin{tabularx}{\textwidth}{XXX}
\lsptoprule

\multicolumn{1}{X}{\textbf{\textsc{Dialect}}} & \textbf{\textsc{Label}} & \textbf{\textsc{Sources} \textbf{(\&} \textbf{early} \textbf{modern} \textbf{authors} \textbf{relying} \textbf{on} \textbf{them)}}\\
\multicolumn{1}{X}{Aeolic} & barbarian & Plato, \textit{Protagoras} 341c, said specifically of Lesbian Aeolic.\\
& obscure & Dionysius of Halicarnassus, \textit{De} \textit{imitatione} 31.2.8.\\
\hhline{~--} & unusual, affected, insolent & Apuleius, \textit{Apologia} \textit{(Pro} \textit{se} \textit{de} \textit{magia} \textit{liber)} 9; Athenaeus, \textit{Deipnosophistae} 14.19. (See e.g. \citealt{MuntheHeiberg1748}: 3.)\\
\hhline{~--} & old-fashioned, archaic & \textit{Scholia} \textit{Vaticana} (ed. \citealt{Hilgard1901}: 117).\\
\multicolumn{1}{X}{Attic} & mixed & Pseudo-Xenophon, \textit{Atheniensium} \textit{respublica} 2.8; Athenaeus, \textit{Deipnosophistae} 3.94; Pseudo-Plutarch, \textit{De} \textit{Homero} \textit{2}. (See e.g. \citealt{Schwartz1721}: 223; \citealt{Maittaire1706}: iii; \citealt{Saumaise1643a}: 437–438, respectively.)\\
& (too) elaborate & Heraclides Criticus, \textit{Descriptio} \textit{Graeciae} 1.4. (See e.g. \citealt{Estienne1573}: ¶.ii\textsc{\textsuperscript{v}}–¶.iii\textsc{\textsuperscript{r}}, referring to “Artemidori geographiae fragmentum”.)\\
\hhline{~--} & concise, popular, fitting for pleasantries & Demetrius, \textit{De} \textit{elocutione} 177; Cicero, \textit{Orator} 89. (See e.g. \citealt{MuntheHeiberg1748}: 3.)\\
\hhline{~--} & excellent, charming, eloquent & Quintilian, \textit{Institutio} \textit{oratoria} 6.3.107, 8.1.2 \& 10.1.100; Cicero, \textit{Orator} 25 \& 28 and \textit{Brutus} 172; Velleius Paterculus, \textit{Historiae} \textit{Romanae} 1.18.1. (See e.g. \citealt{Duret1613}: 690; \citealt{Rollin1726}: 118–119.)\\
\hhline{~--} & artificial & \textit{Scholia} \textit{Vaticana} (ed. \citealt{Hilgard1901}: 117).\\
\multicolumn{1}{X}{Doric} & broad, flat & Theocritus, \textit{Idyllia} 15.87–88 and \textit{Scholia} \textit{in} \textit{Theocritum} (\textit{scholia} \textit{uetera}) on this passage; Hermogenes, $\Pi \varepsilon \rho \text{\textgreek{`i}}$ $\text{\textgreek{>i}}\delta \varepsilon \text{\textgreek{~w}}\nu $ $\lambda \text{\textgreek{'o}}\gamma o\upsilon $ 1.6; Demetrius, \textit{De} \textit{elocutione} 177. (See e.g. Caelius \citealt{Rhodiginus1542}: 465; \citealt{Estienne1573}: ¶.ii\textsc{\textsuperscript{r-v}}; \citealt{Saumaise1643a}: 77.)\\
& annoying, affected & Suetonius, \textit{De} \textit{uita} \textit{Caesarum}, \textit{Tiberius}, 56.1.\\
\hhline{~--} & obscure & Porphyry, \textit{Vita} \textit{Pythagorae} 53. (See e.g. \citealt{Bentley1699}: 317; \citealt{Mazzocchi1754}: 119 n.5.)\\
\hhline{~--} & rustic & Pseudo-Probus, \textit{Commentarius} \textit{in} \textit{Vergilii} \textit{Bucolica} \textit{et} \textit{Georgica}, \textit{praefatio}. Marcus Manilius (\textit{Astronomica} 767) associated Dorians with rusticity in general terms. (See \citealt{Rapin1659}: 121; cf. \textit{infra}.)\\
\hhline{~--} & [\textit{said} \textit{of} \textit{old} \textit{Doric:}] rough, difficult & \textit{Scholia} \textit{in} \textit{Theocritum} \textit{(scholia} \textit{uetera)} F.a.–c. (For old Doric, see e.g. \citealt{Mazzocchi1754}: 118–119; for new Doric, see e.g. \citealt{Valckenaer1773}: 208.)\\
\hhline{~--} & [\textit{said} \textit{of} \textit{new} \textit{Doric:}]

gentler, easier & \\
\hhline{~--} & magnificent & \textit{Scholia} \textit{Vaticana} (ed. \citealt{Hilgard1901}: 117). (See e.g. \citealt{Estienne1581}: 15–16.)\\
\multicolumn{1}{X}{Ionic} & fluent, pleasant & Quintilian, \textit{Institutio} \textit{oratoria} 9.14.18. (See e.g. \citealt{MuntheHeiberg1748}: 9.)\\
& relaxed, frivolous & \textit{Scholia} \textit{Vaticana} (ed. \citealt{Hilgard1901}: 117).\\
\hhline{~--}
\lspbottomrule
\end{tabularx}
It can be noted here that ancient scholars were prone to link the Greek tribes and their dialects to styles within certain arts as well. The Greek dialects were in other words not approached in isolation, but viewed as an undeniable characteristic of the Greek world, pervading numerous dimensions of it. Modes of music were called Doric and Aeolic because they reminded of certain features of these dialects, and a similar association occurred in scholarship on architecture. It would lead me too far to treat this complex extrapolation of the traditional Greek tribal-dialectal scheme to music and architecture in detail here, all the more since its impact on early modern views was highly limited.\footnote{See \citet{Cassio1984}. \citet[118]{Mazzocchi1754} was exceptional in connecting the canonical dialects and architectural styles with the same evaluative properties. In the case of Doric, this was coarseness and roughness. In doing so, he no doubt relied on Vitruvius, \textit{De} \textit{architectura} 4.1.6–8.} Yet it is important to keep in mind that the dialects were intertwined with other domains of knowledge, and that they were able to evoke strong sensual associations going beyond the level of language already in ancient and medieval times.

As can be expected, early modern scholars relied to a considerable degree on ancient and Byzantine sources when attributing evaluative labels to the canonical dialects; this can be gathered from \figref{fig:key:4}, which offers a rudimentary chart of this dependence of early modern Hellenists on earlier sources. There are nonetheless three major differences between ancient and medieval texts, on the one hand, and early modern works, on the other. Firstly, scholars introduced numerous new assessments, as \figref{fig:key:5} reveals. These were very often a direct consequence of the literary usage of the dialect in question. For instance, the frequent characterization of Doric as “boorish” or “rustic” seems to have largely been an early modern innovation. Pseudo-Probus already called Doric \textit{rusticus} in his commentary on Vergil’s \textit{Bucolics} and \textit{Georgics}, but this is an isolated instance, which barely influenced early modern authors. The early modern emphasis on Doric rusticity is likely to have been due to a stronger association of Doric with the bucolic poetry of authors such as Theocritus, a very popular poet among humanists and in their schools. This is in agreement with a broader tendency in language attitudes. Indeed, Brigitte Schlieben-\citet{Lange1992} has shown that it is not uncommon for properties of texts to be transferred to the variety in which they are written. For example, in a letter dating to \citealt{November1511}, a German student learning Greek in Paris characterized the Doric dialect as “scabrous” or “filthy” (\textit{scaber}) and “somewhat rustic” (\textit{subrusticus}). He complained that his teacher, the polyglot humanist and later cardinal Girolamo Aleandro (1480–1542), kept focusing on the Doric poetry of Theocritus instead of reading texts in the \textit{lingua} \textit{communis}, the Greek koine. The student did admit, however, that this dialect was very apt for rustic subject matter.\footnote{The letter, written by a certain Johannes Kierher, is cited in Botley (2010: 220 n.435; cf. also p. 103).} The idea of Doric roughness was also fostered by its close association with the rugged Peloponnese and the rather unrefined mores of its inhabitants, not in the least those of warlike Sparta. The Dutch philologist Isaac Vossius (1618–1698) indeed linked harshness and rusticity to Doric in his 1673 treatise on ancient poetry, claiming that the Ionians laughed at the Dorians for this reason. The Dorians, in turn, allegedly mocked the Ionians for their effeminacy \citep[55]{Vossius1673}. Vossius was, in a sense, fictitiously reconstructing the mutual social behavior of two ancient Greek tribes by relying on widespread stereotypes about them.

\begin{stylecaption}
Figure \stepcounter{Figure}{\theFigure}: Early modern attitudes toward the canonical ancient Greek dialects
\end{stylecaption}

The number of examples offered in the right column can be taken as an indication of the frequency of each label.

\tablefirsthead{}

\tabletail{}
\tablelasttail{}
\begin{tabularx}{\textwidth}{XXX}
\lsptoprule

\multicolumn{1}{X}{\textbf{\textsc{Dialect}}} & \textbf{\textsc{Label}} & \textbf{\textsc{Testimonies}}\\
\multicolumn{1}{X}{Aeolic} & sweet, adequate for lyric poems & Canini (1555: a.4\textsc{\textsuperscript{r}}) called Aeolic “melicis apta”. Cf. \citet[103]{Hoius1620}. \citet[106]{Giraudeau1739} regarded it as “pronuntiatu suauissima”.\\
& heavy, weighty, serious & \citet[16]{Estienne1581} believed it to display a certain \textit{grauitas}, ‘seriousness’, which is central to his discussion of the qualities of French vis-à-vis Italian as well \citep[71]{Swiggers2009}.\\
\hhline{~--} & rough, uncultivated, unpleasant & Walper (1589: 61; 1590: 415) labeled it together with the allegedly cognate Doric dialect \textit{incultior}, \textit{ingratus} \textit{auribus}, \textit{minus} \textit{politus}, and \textit{insuauis}. See e.g. also Fabricius (1711: 515 [\textit{asper}]); Georgi \& Graun (1729: 6 [\textit{rudis}]); Munthe \& Heiberg (1748: e.g. 28 [\textit{inamoenus}]).\\
\hhline{~--} & broad, rather thick & By analogy with Doric, to which Aeolic was believed to be closely cognate, \citet[582]{Nibbe1725} called Aeolic \textit{breit}. See e.g. also Hauptmann (1776: \textsc{a.2}\textsc{\textsuperscript{v}}), where the verb \textit{platustoméo\={} } ($\pi \lambda \alpha \tau \upsilon \sigma \tau o\mu \text{\textgreek{'e}}\omega $), ‘to speak with a broad mouth’, is applied to Aeolic. Von der Hardt (1705 [1699]: 17) characterized Aeolic pronunciation as \textit{obtusior}.\\
\multicolumn{1}{X}{Attic} & (most) elegant, noble, polished, cultivated, tender, fine, pure, neat, honey-sweet, etc. & Melanchthon (1518: a.i\textsc{\textsuperscript{v}}) called Attic “elegantissima”. See e.g. also \citet[209]{Vergara1537}; Baile (1588: 5\textsc{\textsuperscript{r}}); \citet[334]{Alsted1630}. \citet[226]{Ruland1556} attributed \textit{concinnitas} to Attic and characterized it as beautiful and charming. See e.g. also Oreadini (1525: \textsc{e}.iii\textsc{\textsuperscript{v}}); Saumaise (1643a: 76, 112, 424), who linked this label to a round-mouthed pronunciation. \citet[96]{Hoius1620} called Attic \textit{mellitus}.\\
& copious & Canini (1555: a.3\textsc{\textsuperscript{v}}) dubbed Attic \textit{copiosus}.\\
\hhline{~--} & manly, weighty & Georgi \& \citet[6]{Graun1729} applied the adjectives \textit{uirilis} and \textit{grauis} to Attic. See e.g. also \citet[515]{Fabricius1711}.\\
\multicolumn{1}{X}{Doric} & boorish, rustic & \citet[317]{Bentley1699} e.g. labeled Doric \textit{rustic}. This property led the translator of \citet[117]{Rapin1659} to call Doric “sometimes scarce true grammar” \citep[31]{Rapin1684}. See also the main text below.\\
& pleasant, adequate for smoother poets & Canini (1555: a.4\textsc{\textsuperscript{r}}) dubbed it “suauissima” and “poetis mollioribus accommodatissima”. See e.g. also Vuidius (1569: 139\textsc{\textsuperscript{r}}).\\
\hhline{~--} & rough, uncultivated, unpleasant & See above on Aeolic and Gessner (1555: 46\textsc{\textsuperscript{r}}), labelling Doric \textit{crassissimus}. \citet[54]{Vossius1673} characterized Laconian, a variety of Doric, as rough, threatening, and “doglike”. This last property was linked to the frequency of the letter rho, the dog’s letter, at the end of many Laconian words. Cf. Munthe \& \citet[24]{Heiberg1748}.\\
\hhline{~--} & short in speech & Attributed to Laconian Doric by Plato (\textit{Leges} 641e), it was extrapolated to Doric as a whole by \citet[393]{Saumaise1643a}. Cf. Beroaldo (1493: 138\textsc{\textsuperscript{v}}).\\
\hhline{~--} & magnificent, warlike, manly & \citet[55]{Vossius1673} described the Doric dialect as “magnifica et bellica, sed absque iracundia”. He also associated it with manliness.\\
\hhline{~--} & distinguished, flourishing & \citet[161]{Gesner1774} called Doric \textit{florentissimus}.\\
\multicolumn{1}{X}{Ionic} & long in speech, slow, redundant & Caelius \citet[677]{Rhodiginus1542} opposed Ionic lengthiness in speech to Laconian brevity (he called the Ionians “\textit{makrológoi} [$\mu \alpha \kappa \rho o\lambda \text{\textgreek{'o}}\gamma o\iota $]”). \citet[75]{Saumaise1643a} spoke of Ionic slowness and redundancy.\\
& elegant, polished, neat, honey-sweet & Hauptmann (1776: \textsc{a.2}\textsc{\textsuperscript{r}}) ascribed \textit{mundities} to Ionic. \citet[290]{Verwey1684} spoke of the \textit{mel} \textit{Ionicum}.\\
\hhline{~--} & faint, delicate, womanish & \citet[75]{Saumaise1643a} linked the \textit{genius} of Ionic to the mores and the “long and fluid” clothing style of the Ionians, which he characterized as both faint and womanish. He pointed to the migration to Asia as the cause of their effeminacy. See e.g. also Vuidius (1569: 139\textsc{\textsuperscript{r}}).\\
\hhline{~--}
\lspbottomrule
\end{tabularx}
A second major difference is that the sources and motivations of early modern scholars to propose dialect evaluations are more transparent than those of their ancient and Byzantine predecessors. Early modern attitudes toward the Ionic dialect provide a good example of the various ways in which Hellenists supplemented the ancient and Byzantine sources. To start with, philologists introduced new properties by quoting ancient testimonies that did not so much concern the dialects as the tribes speaking them. These ancient text passages encouraged early modern scholars to construe a specific mental picture of these tribes, their customs, and their speech. Claude de Saumaise, for example, characterized Ionic as \textit{mollis}, ‘effeminate, delicate’, by referring to a verse of the Roman poet Martial (ca. \textsc{ad} 40–103): “nor let the \textit{delicate} \textit{Ionians} be praised for their temple of Trivia”.\footnote{\citet[75]{Saumaise1643a}, citing Martial, \textit{Spectaculorum} \textit{liber} 1.3: “nec Triuiae templo \textit{molles} laudentur \textit{Iones}” (my emphasis).} Saumaise moreover linked Ionic effeminacy to their clothing style and, more fundamentally, to their migration from Greece to Asia, thus presenting a classic case of an Orientalist attitude (cf. \citealt{Said2003} [1978]). The Danish philologist and professor Caspar Frederik Munthe (1704–1763) and his colleague Ludvig Heiberg (1723–1760), in turn, relied on the Byzantine scholia on Thucydides for their opposition of Ionic delicacy to Doric manliness.\footnote{See Munthe \& \citet[15]{Heiberg1748}, relying on \textit{Scholia} \textit{in} \textit{Thucydidem} (\textit{Scholia} \textit{uetera} \textit{et} \textit{recentiora}), commentary at 1.124.1. On the Doric–Ionic opposition in antiquity, see \citet{Cassio1984}.} Here, the alleged properties of the people speaking a dialect were transferred to the dialect itself, a procedure very common throughout history. Indeed, John Edwards (2009: 66–68) has pointed out that there exists a clear causative link between stereotypes about certain social groups and the esthetic qualities attributed to the varieties they speak (see also \citealt{Silverstein2003}; \citealt{Preston2018}: 200). Some early modern philologists even argued that certain tribal characteristics manifested themselves in specific dialectal features. Isaac Vossius linked Ionic delicacy and effeminacy to concrete features of the dialect: the frequency of the letter eta (<η>) in it, its lack of contractions, its many diminutives, and other linguistic “flatteries”, such as the alleged usage of feminine articles with male objects and animals, even with the most “monstrous” ones.\footnote{\citet[55]{Vossius1673}: “Nihil hac mollius et effeminatius, siue ubique occurrentem litteram $\text{\textgreek{>~h}}\tau \alpha $, siue frequentes uocalium hiatus, siue etiam crebra diminutiua aliaque spectes blandimenta. Adeo huic populo terrori fuit, quidquid esset uirile, ut quibusque fere rebus masculis et beluis etiam quantumuis immanibus, sequioris sexus articulos praeposuerint”.} Before the early modern period, the link between linguistic features and evaluative properties was practically non-existent with one sole exception; the idea of Doric broadness was sometimes connected to the frequency of the letter alpha (<α>) in this dialect.\footnote{See \textit{Scholia} \textit{in} \textit{Theocritum} (\textit{scholia} \textit{uetera}) at \textit{Idyllia} 15.87–88.} Finally, one scholar, Henri Estienne, created new authoritative documentation himself in order to establish the smooth character of Ionic. In his commentary on Attic of 1573, Estienne (1573: ii\textsc{\textsuperscript{r}}) quoted – somewhat pretentiously, one might say – a Greek epigram of his own invention to prove the historical primacy of Ionic as well as its sweet and delicate character. He had prefixed this poem to his edition of the Ionian historian Herodotus, published three years earlier:

\begin{quote}
The Ionic dialect is indeed sweet, far above all,
\end{quote}

\begin{quote}
and utters delicate noises, but certainly,
\end{quote}

\begin{quote}
as far as Ionic surpasses all, so far
\end{quote}

\begin{quote}
does Herodotus surpass those speaking Ionic.\footnote{\citet[8]{Estienne1570}: “$\text{\textgreek{>'E}}\sigma \tau \iota $ $\mu \text{\textgreek{`e}}\nu $ $\text{\textgreek{>'e}}\sigma \tau \iota \nu $ $\text{\textgreek{>I}}\text{\textgreek{`a}}\varsigma $ $\lambda \iota \gamma \upsilon \rho \text{\textgreek{`h}}$ $\delta \iota \text{\textgreek{'a}}\lambda \varepsilon \kappa \tau o\varsigma $ $\text{\textgreek{<a}}\pi \alpha \sigma \text{\textgreek{~w}}\nu $ / $\text{\textgreek{>'e}}\xi o\chi \alpha $, $\kappa \alpha \text{\textgreek{`i}}$ $\mu \alpha \lambda \alpha \kappa o\text{\textgreek{`u}}\varsigma $ $\text{\textgreek{>e}}\xi \alpha \varphi \iota \varepsilon \text{\textgreek{~i}}\sigma \alpha $ $\theta \rho \text{\textgreek{'o}}o\upsilon \varsigma \text{\textgreek{;}}$ / $\text{\textgreek{>a}}\lambda \lambda \text{\textgreek{`a}}$ $\gamma \text{\textgreek{`a}}\rho $ $\text{\textgreek{<'o}}\sigma \sigma o\nu $ $\text{\textgreek{>I}}\text{\textgreek{`a}}\varsigma $ $\pi \alpha \sigma \text{\textgreek{~w}}\nu $ $\pi \rho o\varphi \varepsilon \rho \varepsilon \sigma \tau \text{\textgreek{'a}}\tau \eta $ $\text{\textgreek{>e}}\sigma \tau \text{\textgreek{'i}}$, / $\tau \text{\textgreek{'o}}\sigma \sigma o\nu $ $\text{\textgreek{>i}}\alpha \zeta \text{\textgreek{'o}}\nu \tau \omega \nu $ $\text{\textgreek{<H}}\rho \text{\textgreek{'o}}\delta o\tau o\varsigma $ $\pi \rho o\varphi \text{\textgreek{'e}}\rho \varepsilon \iota $”.}
\end{quote}

A third difference is that early modern Hellenists tried to organize their evaluations in a much more systematic manner. In Greek scholarship, there had been only one isolated attempt at doing so. A Byzantine scholiast, commenting on the ancient grammar attributed to Dionysius Thrax, was exceptional in trying to systematize the characteristic properties of the Greek dialects, linking them to the customs of the individual Greek tribes:

\begin{quote}
The Greeks indeed differ from the barbarians with respect to customs, speech as well as ways of life. One has to know, however, that, among the Greeks, there are the Dorians, the Aeolians, the Ionians, and the Attics. And we are explaining qualities occurring among these, for even these [tribes] do differ from one another in their ways as well as their customs. In fact, the Doric tribe seems to be manlier in its ways of life, and magnificent in the sounds of its names and in the tone of its voice, whereas the Ionic is relaxed in all these aspects, since the Ionians are frivolous. The Attic tribe seems to differ as regards way of life and artificiality of speech, whereas the Aeolic is distinctive through the austerity of its way of life and the old fashion of its speech.\footnote{\textit{Commentaria} \textit{in} \textit{Dionysii} \textit{Thracis} \textit{Artem} \textit{Grammaticam}, \textit{Scholia} \textit{Vaticana} \textit{(partim} \textit{excerpta} \textit{ex} \textit{Georgio} \textit{Choerobosco,} \textit{Georgio} \textit{quodam,} \textit{Porphyrio,} \textit{Melampode,} \textit{Stephano,} \textit{Diomede)} (ed. \citealt{Hilgard1901}: 117): “$\kappa \alpha \text{\textgreek{`i}}$ $\gamma \text{\textgreek{`a}}\rho $ $\text{\textgreek{>'h}}\theta \varepsilon \sigma \iota $ $\kappa \alpha \text{\textgreek{`i}}$ $\delta \iota \alpha \lambda \text{\textgreek{'e}}\kappa \tau \text{\textgreek{w|}}$ $\kappa \alpha \text{\textgreek{`i}}$ $\text{\textgreek{>a}}\gamma \omega \gamma \alpha \text{\textgreek{~i}}\varsigma $ $\delta \iota \alpha \varphi \text{\textgreek{'e}}\rho o\upsilon \sigma \iota \nu $ <$o\text{\textgreek{<i}}$> $\text{\textgreek{<'E}}\lambda \lambda \eta \nu \varepsilon \varsigma $ $\tau \text{\textgreek{~w}}\nu $ $\beta \alpha \rho \beta \text{\textgreek{'a}}\rho \omega \nu $. $\Gamma \iota \nu \text{\textgreek{'w}}\sigma \kappa \varepsilon \iota \nu $ $\delta \text{\textgreek{`e}}$ $\chi \rho \text{\textgreek{`h}}$ $\text{\textgreek{<'o}}\tau \iota $ $\tau \text{\textgreek{~w}}\nu $ $\text{\textgreek{<E}}\lambda \lambda \text{\textgreek{'h}}\nu \omega \nu $ $o\text{\textgreek{<i}}$ $\mu \text{\textgreek{'e}}\nu $ $\varepsilon \text{\textgreek{>i}}\sigma \iota $ $\Delta \omega \rho \iota \varepsilon \text{\textgreek{~i}}\varsigma $, $o\text{\textgreek{<i}}$ $\delta \text{\textgreek{`e}}$ $A\text{\textgreek{>i}}o\lambda \varepsilon \text{\textgreek{~i}}\varsigma $, $o\text{\textgreek{<i}}$ $\delta \text{\textgreek{`e}}$ $\text{\textgreek{>'I}}\omega \nu \varepsilon \varsigma $, $o\text{\textgreek{<i}}$ $\delta \text{\textgreek{`e}}$ $\text{\textgreek{>A}}\tau \tau \iota \kappa o\text{\textgreek{'i}}$. $\sigma \upsilon \mu \beta \varepsilon \beta \eta \kappa \upsilon \text{\textgreek{'i}}\alpha \varsigma $ $\delta \text{\textgreek{`e}}$ $\delta \iota \text{\textgreek{`a}}$ $\tau o\text{\textgreek{'u}}\tau \omega \nu $ $\delta \eta \lambda o\text{\textgreek{~u}}\mu \varepsilon \nu $ $\pi o\iota \text{\textgreek{'o}}\tau \eta \tau \alpha \varsigma $, $\kappa \alpha \text{\textgreek{`i}}$ $\gamma \text{\textgreek{`a}}\rho $ $\kappa \alpha \text{\textgreek{`i}}$ $o\text{\textgreek{<~u}}\tau o\iota $ $\tau \rho \text{\textgreek{'o}}\pi o\iota \varsigma $ $\kappa \alpha \text{\textgreek{`i}}$ $\text{\textgreek{>'h}}\theta \varepsilon \sigma \iota $ $\delta \iota \alpha \varphi \text{\textgreek{'e}}\rho o\upsilon \sigma \iota \nu $ $\text{\textgreek{>a}}\lambda \lambda \text{\textgreek{'h}}\lambda \omega \nu $· $\delta o\kappa \varepsilon \text{\textgreek{~i}}$ $\gamma \text{\textgreek{`a}}\rho $ $\tau \text{\textgreek{`o}}$ $\Delta \text{\textgreek{'w}}\rho \iota o\nu $ $\text{\textgreek{>a}}\nu \delta \rho \omega \delta \text{\textgreek{'e}}\sigma \tau \varepsilon \rho \text{\textgreek{'o}}\nu $ $\tau \varepsilon $ $\varepsilon \text{\textgreek{>~i}}\nu \alpha \iota $ $\tau o\text{\textgreek{~i}}\varsigma $ $\beta \text{\textgreek{'i}}o\iota \varsigma $, $\kappa \alpha \text{\textgreek{`i}}$ $\mu \varepsilon \gamma \alpha \lambda o\pi \rho \varepsilon \pi \text{\textgreek{`e}}\varsigma $ $\tau o\text{\textgreek{~i}}\varsigma $ $\varphi \theta \text{\textgreek{'o}}\gamma \gamma o\iota \varsigma $ $\tau \text{\textgreek{~w}}\nu $ $\text{\textgreek{>o}}\nu o\mu \text{\textgreek{'a}}\tau \omega \nu $ $\kappa \alpha \text{\textgreek{`i}}$ $\tau \text{\textgreek{~w|}}$ $\tau \text{\textgreek{~h}}\varsigma $ $\varphi \omega \nu \text{\textgreek{~h}}\varsigma $ $\tau \text{\textgreek{'o}}\nu \text{\textgreek{w|}}$, $\tau \text{\textgreek{`o}}$ $\delta \text{\textgreek{`e}}$ $\text{\textgreek{>I}}\omega \nu \iota \kappa \text{\textgreek{`o}}\nu $ $\text{\textgreek{>e}}\nu $ $\pi \text{\textgreek{~a}}\sigma \iota $ $\tau o\text{\textgreek{'u}}\tau o\iota \varsigma $ $\text{\textgreek{>a}}\nu \varepsilon \iota \mu \text{\textgreek{'e}}\nu o\nu $ – $\chi \alpha \text{\textgreek{~u}}\nu o\iota $ $\gamma \text{\textgreek{`a}}\rho $ $o\text{\textgreek{<i}}$ $\text{\textgreek{>'I}}\omega \nu \varepsilon \varsigma $ – $\tau \text{\textgreek{`o}}$ $\delta \text{\textgreek{`e}}$ $\text{\textgreek{>A}}\tau \tau \iota \kappa \text{\textgreek{`o}}\nu $ $\varepsilon \text{\textgreek{>'i}}\varsigma $ $\tau \varepsilon $ $\delta \text{\textgreek{'i}}\alpha \iota \tau \alpha \nu $ $\kappa \alpha \text{\textgreek{`i}}$ $\varphi \omega \nu \text{\textgreek{~h}}\varsigma $ $\text{\textgreek{>e}}\pi \iota \tau \text{\textgreek{'e}}\chi \nu \eta \sigma \iota \nu $ $\text{\textgreek{>a}}\varepsilon \text{\textgreek{`i}}$ $\delta \iota \alpha \varphi \text{\textgreek{'e}}\rho \varepsilon \iota \nu $, $\tau \text{\textgreek{`o}}$ $\delta \text{\textgreek{`e}}$ $A\text{\textgreek{>i}}o\lambda \iota \kappa \text{\textgreek{`o}}\nu $ $\tau \text{\textgreek{~w|}}$ τ’ $\alpha \text{\textgreek{>u}}\sigma \tau \eta \rho \text{\textgreek{~w|}}$ $\tau \text{\textgreek{~h}}\varsigma $ $\delta \iota \alpha \text{\textgreek{'i}}\tau \eta \varsigma $ $\kappa \alpha \text{\textgreek{`i}}$ $\tau \text{\textgreek{~w|}}$ $\tau \text{\textgreek{~h}}\varsigma $ $\varphi \omega \nu \text{\textgreek{~h}}\varsigma $ $\text{\textgreek{>a}}\rho \chi \alpha \iota o\tau \rho \text{\textgreek{'o}}\pi \text{\textgreek{w|}}$”.}
\end{quote}

Such general accounts are as a rule absent from ancient and Byzantine treatises on the dialects. During the early modern period, however, dialect evaluations were frequently included in handbooks for the Greek dialects as a piece of standard information, especially from the seventeenth century onward. This is in keeping with a more general development in early modern discourse on stereotypes of ethnic groups, as, Joep \citet[17]{Leerssen2007} argues,

\begin{quote}
the cultural criticism of early-modern Europe […] began, in the tradition of Julius Caesar Scaliger (1484–1558), to sort European cultural and societal patterns into national categories, thereby formalizing an older, informal tradition of attributing essential characteristics to certain national or ethnic groups.
\end{quote}

In early modern formalized discussions of the Greek dialects and their properties, many of the same qualities and vices recurred, thus encouraging the canonization of a number of properties. Numerous instances of this tendency could be cited, but let me limit myself here to listing three representative examples from different centuries, which all have a clear link with philology:

\begin{quote}
Attic is the most elegant and copious of all and the cherisher of eloquence, which most of the noblest writers employed.
\end{quote}

\begin{quote}
Related to this is Ionic, which the oldest authors used, Democritus, Hippocrates, Herodotus; Homer also for a large part and Hesiod.
\end{quote}

\begin{quote}
Doric is the most pleasant and the most adequate for smoother poets, which the choruses of tragedians have also received so as to moderate the bitterness of the subject. This dialect was used by the Pythagoreans, Pindar, Epicharmus, Sophron, and Theocritus.
\end{quote}

\begin{quote}
Similar to this is Aeolic, adequate for lyric poems, which Alcaeus, Sappho, and many others expressed in their writings, of whom fortune has left nothing at all, except for those passages that are cited by others.\footnote{Canini (1555: a.3\textsc{\textsuperscript{v}}–a.4\textsc{\textsuperscript{r}}): “Attica omnium elegantissima et copiosissima eloquentiaeque altrix, quam plurimi nobilissimi scriptores celebrarunt. Huic affinis Ionica, quam uetustissimi auctores usurparunt, Democritus, Hippocrates, Herodotus; Homerus etiam magna ex parte atque Hesiodus. Dorica suauissima est et poetis mollioribus accommodatissima, quam etiam tragicorum chori ad temperandam argumenti acerbitatem receperunt. Ea usi sunt Pythagorici, Pindarus, Epicharmus, Sophron et Theocritus. Huic similis Aeolica, melicis apta, quam scriptis expressere Alcaeus, Sappho aliique permulti, e quibus, praeter pauca quae ab aliis citantur, nihil omnino fortuna reliquum fecit”. For another sixteenth-century example, see Vuidius (1569: 138\textsc{\textsuperscript{v}}–139\textsc{\textsuperscript{r}}). Cf. also already Lopad (1536: 12\textsc{\textsuperscript{v}}) and Gessner (1543: \textsc{a.6}\textsc{\textsuperscript{v}}\textsc{–a.7}\textsc{\textsuperscript{r}}).}
\end{quote}

\begin{quote}
The first is Attic, which indeed must be preferred as the noblest above all others. It was mainly in this dialect that Thucydides, Demosthenes, Isocrates, and the majority of the historiographers wrote.
\end{quote}

\begin{quote}
The second is Ionic, which has a wonderful grace and charm, which mainly Herodotus, Hippocrates, and the poets, even Doric ones, used.
\end{quote}

\begin{quote}
The third is Doric, a little rougher and harder because of the pronunciation, as the Dorians are said ‘to pronounce broadly’ (that is, to speak with a wide and open mouth). This dialect was employed by, among others, Theocritus and Pindar.
\end{quote}

\begin{quote}
The fourth, finally, is Aeolic, which no authors have followed avowedly, but the poets have interspersed it hither and thither in their writings, especially, however, Alcaeus, Sappho, what is more, Theocritus himself and Pindar (as it has many things in common with Doric), also Homer and therefore others.\footnote{Mérigon (1621: 2–3): “Prima est Attica, quae quidem ut nobilior omnibus aliis praeponi debet; hac autem scripsere praecipue Thucydides, Demosthenes, Isocrates et maior pars historiographorum. Secunda, Ionica, quae mirificum habet leporem et uenustatem, qua usi sunt praecipue Herodotus, Hippocrates et poetae, etiam Dorici. Tertia Dorica, quae paulo asperior et durior propter pronuntiationem, quippe $\pi \lambda \alpha \tau \upsilon \text{\textgreek{'a}}\zeta \varepsilon \iota \nu $ (hoc est lato et diducto ore loqui) dicuntur Dores; hanc autem dialectum celebrauit inter alios Theocritus et Pindarus. Quarta denique est Aeolica, quam nulli auctores ex professo sectati sunt, sed eam huc illuc in suis scriptis insperserunt poetae, praecipue uero Alcaeus, Sappho, immo Theocritus ipse et Pindarus (ut pote cum Dorica multa communia habentem) tum Homerus aliique ideo”. See \citet{Hoius1620} and \citet{Rhenius1626} for other seventeenth-century examples.}
\end{quote}

\begin{quote}
In this way, it happened that neither the Ionic nor the Doric nor any other dialect was similar to the Attic dialect, but that Attic surpassed all these dialects, as it is not too delicate, like the Ionic, nor too hard, like the Doric, nor too rude, like the Aeolic, but moderate, manly, weighty, and most shining of all.\footnote{Georgi \& \citet[6]{Graun1729}: “[…] quo contigit, ut Atticae dialecto neque Ionica neque Dorica neque alia quaedam, similis fuerit, sed eas omnes superaret, cum neque nimis mollis sit, ut Ionica, neque nimis dura, ut Dorica, neque nimis rudis, ut Aeolica, sed temperata, uirilis, grauis atque omnium nitidissima”. Cf. also Ries (1786 [1782]: 197–199).}
\end{quote}

The above three accounts also exhibit differences. In the first case, the emphasis is on the link between a dialect and its literary usage by different authors and in distinct genres. The primacy of Attic is also suggested, but this stands out much more clearly in the second account, which seems to construe a kind of evaluative ranking of the dialects: Attic first, Ionic second, Doric third, and Aeolic fourth. In the third passage, the superiority of Attic is likewise maintained, but it seems that the other three dialects were believed to be on more or less the same level.

Evaluative attitudes toward the ancient Greek dialects were, for the greater part, the product of post factum projections of virtues and vices on these literary varieties. Indeed, in the early modern period and even in antiquity, attitudes were usually based on an esthetic sensation during the act of reading. The ancient Roman rhetorician Quintilian experienced the fluency and pleasantness of Ionic in this fashion, as he added to his judgment the following reservation: “at least as I perceive it”.\footnote{Quintilian, \textit{Institutio} \textit{oratoria} 9.14.18: “ut ego quidem sentio”.} To demonstrate the different impressions distinct Greek dialects conveyed, the French Hellenist Henri Estienne even transposed a Doric verse of the Hellenistic poet Callimachus (4th/3rd cent. \textsc{bc}) into Ionic as follows:

\begin{quote}
Original Doric: \textit{Tòn} \textit{dè} \textit{kholōsaménā} \textit{per} \textit{hómōs} \textit{proséphēsen} \textit{Athā\'{} nā} [$T\text{\textgreek{`o}}\nu $ $\delta \text{\textgreek{`e}}$ $\chi o\lambda \omega \sigma \alpha \mu \text{\textgreek{'e}}\nu \alpha $ $\pi \varepsilon \rho $ $\text{\textgreek{<'o}}\mu \omega \varsigma $ $\pi \rho o\sigma \text{\textgreek{'e}}\varphi \alpha \sigma \varepsilon \nu $ $\text{\textgreek{>A}}\theta \text{\textgreek{'a}}\nu \alpha $].
\end{quote}

\begin{quote}
Ionicized version: \textit{Tòn} \textit{dè} \textit{kholōsaménē} \textit{per} \textit{hómōs} \textit{proséphēsen} \textit{Athḗnē} [$T\text{\textgreek{`o}}\nu $ $\delta \text{\textgreek{`e}}$ $\chi o\lambda \omega \sigma \alpha \mu \text{\textgreek{'e}}\nu \eta $ $\pi \varepsilon \rho $ $\text{\textgreek{<'o}}\mu \omega \varsigma $ $\pi \rho o\sigma \text{\textgreek{'e}}\varphi \eta \sigma \varepsilon \nu $ $\text{\textgreek{>A}}\theta \text{\textgreek{'h}}\nu \eta $].\footnote{Estienne (1581: 15–16), with reference to Callimachus, \textit{In} \textit{lauacrum} \textit{Palladis} 5.79.}
\end{quote}

\begin{quote}
English translation according to the Loeb series: “And Athena was angered, yet said to him”.
\end{quote}

The Doric verse allegedly became, when transposed to Ionic, feeble and inadequate and lost its seriousness and majesty, and this solely through the replacement of the letter alpha by eta. An evaluative label could also result from a conscious critical review of the style in which a literary work was composed. The Doric texts known to the Neoplatonist philosopher Porphyry (ca. 234–305/310) seemed to be written in an obscure style, which is why in his biography of Pythagoras he labeled the dialect itself obscure (\textit{Vita} \textit{Pythagorae} 53). In other words, it was not direct, oral contact with a dialect that triggered evaluative attitudes, but indirect confrontation through reading, either as an immediate sensation or as the result of a conscious assessment of the style of a text. This distinguishes premodern attitudes toward the ancient Greek dialects from those toward vernacular languages and dialects, which were usually at least partly informed by direct exposure to the variety in its spoken form.

Apart from encounters with literary texts, it was the link that scholars frequently made between the customs of a tribe and its language – \textit{lingua} \textit{et} \textit{mores} in Latin – which led them to conjure up evaluative labels for Greek dialects.\footnote{Van \citet{Hal2013} offers a preliminary historical survey of the \textit{lingua} \textit{et} \textit{mores} link, while pointing out that it deserves further study.} Indeed, many attitudes were motivated by stereotypes about the four canonical Greek tribes, as I have shown throughout this section.\footnote{See e.g. the \textit{Scholia} \textit{Vaticana} quote above as well as the ideas of Saumaise and Vossius.} Early modern scholars took ancient and Byzantine attitudes as their starting point and complemented them in various ways. This materialized not only in the form of new evaluative statements and an increased emphasis on certain properties, especially Attic elegance and Doric rusticity, but – most notably – it also resulted in a tendency toward canonizing dialect attitudes. Even though there remained some variation in the early modern perception of the Greek dialects, it is nonetheless safe to state that the evaluation of the four traditional dialects became a canonized format. Indeed, it constituted an almost inherent part of the study of the ancient Greek language and its literary dialects and was for this reason integrated into many Greek language manuals. Since scholars usually sensed the koine to be of a particular nature, they did not assign specific properties to it, neither in antiquity and the Byzantine era nor in the early modern period.

\subsection{Evaluative discourse between Greek and the vernacular}
\hypertarget{Toc19704852}{}
The evaluative discourse on the Greek dialects must have been widely known in learned circles, as it apparently influenced attitudes toward vernacular speech forms to some extent. The terminology used to label vernacular tongues and their dialects sometimes resembled that found in evaluations of the Greek dialects. This emerges most clearly from cases in which scholars assigned labels to both Greek and vernacular speech forms in their works. Let me look at two noteworthy examples from the seventeenth and the eighteenth century, respectively: Isaac Vossius and Friedrich Gedike.

In his widely read treatise on ancient poetry and its original rhythm, published in Oxford in 1673, the Dutch philologist Isaac Vossius (1673: 54–55) opposed the effeminate Ionic dialect to virile Doric, for which he may have relied on a Byzantine commentary on Thucydides.\footnote{Cf. \textit{Scholia} \textit{in} \textit{Thucydidem} (\textit{Scholia} \textit{uetera} \textit{et} \textit{recentiora}), commentary at 1.124.1.} Directly after that, he provided a brief outline of the qualities of a number of vernacular tongues of his time. Especially relevant to my purposes is his characterization of English, with which he was very well acquainted, having moved to England in 1670. Vossius described the language as “delicate” (\textit{mollis}) and “effeminate” (\textit{muliebris}). To exemplify this linguistically, he referred to the English preference for the letter \textit{êta} (“$\text{\textgreek{>~h}}\tau \alpha $”) and its avoidance of the letter <a>. Vossius’ views on the Ionic dialect, cited earlier in this chapter, irrefutably informed his assessment of English (see \sectref{sec:key:2} above). Ionic was also known for having the letter eta where the other dialects had a long alpha, and Vossius spoke of the Greek letter eta rather than the English letter <e> in characterizing this supposed property of English. He did add, however, that English “delicacy” (\textit{mollities}) was somewhat tempered by the harshness of its syllables and the frequency of consonants in this language (Vossius: 1673: 56). After that, Vossius praised French for its strength and its many war-related words, which is reminiscent of his description of the Doric dialect.

Friedrich Gedike (1779: \textsc{xx}), a German scholar from the late Enlightenment, drew a detailed comparison between the Greek and German dialect contexts in his \textit{Thoughts} \textit{on} \textit{purism} \textit{and} \textit{language} \textit{enrichment}. Gedike modeled his threefold classification of Greek on his perception of vernacular German diversity, thus proceeding in a direction opposite to Vossius, who had moved from Greek to the vernacular. First, Gedike compared Ionic with Low German (\textit{Niederdeutsch} or \textit{Plattdeutsch}), both of which he described as being “smooth” (\textit{sanft}) and “delicate” (\textit{weich}). He associated this characteristic with the absence of aspirations and rough diphthongs, and emphasized the obviousness of the parallel he was pointing out. He proceeded by treating the similarity of Doric and Upper German (\textit{Oberdeutsch}), which was situated in the broadness with which they were pronounced. They moreover contained, Gedike argued, many hissing sounds, aspirations, and diphthongs. This gave them a “solemn” (\textit{feierlich}) and “splendid” (\textit{prunkvoll}) air. Gedike thus assessed Doric in distinctly positive terms. Finally, the “middle dialects” were discussed: Attic and High German (\textit{Hochdeutsch}). They were, however, not exactly in the middle, because both inclined toward the respective “solemn” varieties: Doric and Upper German. Gedike refrained from elaborating more extensively on the properties of Attic and High German in his 1779 work. However, three years later, in an article on the Greek dialects, he stated that Attic was less rough than Doric and less fluid, yet more consistent than Ionic. Something similar held true for High German, he suggested \citep[25]{Gedike1782}. Gedike (1779: \textsc{xx–xxi)} rounded off his comparison by stating that, just like the ancient Greek dialects, the three German dialects also used to be “book languages” (\textit{Büchersprachen}), until the High German speech of the Lutheran Reformation expelled the two others from writing. Gedike’s comparison of Greek and German dialects was applauded by several of his contemporaries, including the famed grammarian of German Johann Christoph Adelung (1732–1806; see \citealt{Adelung1781}: 56 and also \citealt{Moritz1781}: 20).

\subsection{Beyond the early modern era}
\hypertarget{Toc19704853}{}
The evaluative discourse on the Greek dialects did not end with the arrival of modernity. On the contrary, it persisted until very late. In the nineteenth century, the distinguished German philologist Heymann Steinthal (1823–1899) noted the following on the Greek dialects in general and Attic in particular:

\begin{quote}
Each dialect counts as a phase in time and an interior moment of the spirit. In the Attic dialect, the Greek spirit manifested itself last, but also most perfectly, and, to be sure, in such an encompassing manner that one may rightly say that the other dialects have been neutralized in it. This is also why all Greek dialects have perished in and with it.\footnote{\citet[9]{Steinthal1891}: “Jeder Dialekt gilt als ein Abschnitt in der Zeit und ein inneres Moment des Geistes. Im attischen Dialekt offenbarte sich der griechische Geist am spätesten, aber auch am vollkommensten, und zwar in so umfassender Weise, dass man wol sagen darf, in ihm seien die andren Dialekte aufgehoben gewesen. Darum sind auch in und mit ihm alle griechischen Dialekte zu Grunde gegangen”.}
\end{quote}

Steinthal’s underlying assumptions were, however, different from those of early modern evaluative discourse. He presumed the existence of a Greek \textit{Volksgeist}, which has to be viewed against the background of his interest in the psychology of tribes and nations (\textit{Völkerpsychologie}), and he supposed that some Greek tribes represented that \textit{Geist} better than others. Still, it is telling that, as with his early modern predecessors, evaluating the Greek dialects came naturally to him. Today, the idea of Attic elegance and primacy is still latent in the sense that it is taught as the principal variety of ancient Greek in most high school and university curricula. This is largely a modern innovation, as early modern grammars tended to describe “the Greek language”, usually a form of the koine with typically Attic and Ionic elements interspersed, as Federica \citet[123]{Ciccolella2008} has rightly suggested. Be that as it may, literary Attic was generally valued most highly even by early modern Hellenists (cf. \citealt{Roelcke2014}: 251). In other words, the shift from the early modern to the modern period coincided with a shift in the prototypical form of Greek: from a hybrid form of koine Greek to Attic Greek.\footnote{Differences in the prototypicalization of Greek throughout history require further study (Van Rooy fc. b).}

Early modern scholars approached and evaluated the Greek dialects principally against the backdrop of reading and understanding Greek literature, even though stereotypes about the traditional four Greek tribes likewise constituted an important trigger for dialect attitudes. The authors sometimes also assumed a connection between the dialects and certain other aspects of ancient Greece, albeit in a much looser way than with Greek literature and the Greek tribes. What are these other aspects?

\subsection{Geography, politics, and natural disposition}
\hypertarget{Toc19704854}{}
First of all, in keeping with the idea, widespread in early modern times, that geography was responsible for dialectal diversification, the terrain of Greece was frequently appealed to in order to account for the existence of Greek dialects.\footnote{On the link between geography and dialectal diversity, see Van Rooy (fc. d).} The Protestant theologian and renowned Hellenist Philipp Melanchthon (1518: a.1\textsc{\textsuperscript{v}}; 1520: \textsc{a.}i\textsc{\textsuperscript{v}}) described in his grammar of the language Greece as “spacious” (\textit{ampla}) and “wide” (\textit{lata}), while presenting dialectal diversity as a self-evident consequence of this aspect of Greek geography (cf. also \citealt{Ruland1556}: 1). The dialects were linked to the many islands of ancient Greece in particular, most notably by the Anglo-Welsh writer James Howell (ca. 1594–1666). Inspired by the prominent philologist Josephus Justus Scaliger, \citet[89]{Howell1650b} emphasized that “the cause why from the beginning ther wer so many differing dialects in the \textit{Greek} tongue was because it was slic’d into so many islands” (cf. \citealt{Howell1642}: 138–139; cf. \citealt{Scaliger1610}: 121). Howell’s treatment of Greek diversity was actually triggered by a comment on Italian dialects, which he subsequently compared to their Greek counterparts. He claimed that, in the case of Italian, dialectal variation was caused by “multiplicity” or “diversity of governments” rather than geography. This brings me to a second major link made by early modern scholars, that between the dialects and the political diversity of ancient Greece, which, in turn, was often viewed as a consequence of the rugged geography of the area. The humanist Lorenzo Valla’s famous praise of the Latin language cannot be left unmentioned in this context:

\begin{quote}
Just as the Roman law is one law for many peoples, so is the Latin language one for many. The language of Greece, a single country, is shamefully not single, but as various as there are factions in the state.\footnote{Valla in \citet[122]{Regoliosi1993}: “multarum gentium, uelut una lex, una est lingua Romana: unius Graeciae, quod pudendum est, non una sed multae sunt, tamquam in republica factiones”. The translation is adopted from \citet[10]{Trapp1990}. On this passage, see e.g. Tavoni in Benvoglienti (1975: 90 n.55) and Trovato (1984: 212–213).}
\end{quote}

The polyhistor Daniel Georg Morhof (1639–1691) made a similar point, emphasizing the inability of Athens to impose its dialect on neighboring city-states \citep[146]{Morhof1685}. The German classical scholar Johann Matthias Gesner (1691–1761) similarly suggested that Greek dialectal diversity was caused by the fact that “ancient Greece did not have a capital and dominant city, but several cities had the same and equal rights”.\footnote{Gesner (1774: 160–161): “Origo autem dialectorum uariarum haec est; quia Graecia antiqua non habuit caput et dominam urbem, sed plures urbes eadem habebant et paria iura”. Cf. Rollin (1731: 395–396); Priestley (1762: 136–138); Ries (1786 [1782]: 204).} The poet Pierre de Ronsard (1524–1585), for his part, contrasted Greek diversity to his native French context, connecting at the same time Greek linguistic abundance to the fragmented political landscape of ancient Greece (\citealt{Ronsard1565}: 5\textsc{\textsuperscript{r}}). Ronsard interestingly added that if there still were political diversity in France, each ruler would desire, for reasons of honor, that their subjects wrote in the language of their native country.\footnote{Cf. Court de Gébelin (1778: lxviii) for a similar observation. See also Chapter 8, \sectref{sec:key:1.3.}} An odd characterization of ancient Greece was proposed by the Bohemian Protestant scholar Christoph(orus) Crinesius (1584–1629). Operating within a biblical framework and deriving Greek from Hebrew, \citet[77]{Crinesius1629} held that the Greek dialects were the varieties spoken in the different provinces of the kingdom of Javan, a grandson of Noah and traditionally associated with the Ionians. In other words, he incorrectly claimed that the linguistic variation of ancient Greece coincided with the regional-administrative division of a politically unitary empire. Apart from political diversity, the dialects were often also connected to the many colonies established by the Greeks (see e.g. \citealt{Simonis1752}: 207). It is worthwhile recalling here that certain early eighteenth-century scholars believed the geopolitical diversity of early modern Greece to correlate with vernacular Greek dialectal variation as well (see Chapter 2, \sectref{sec:key:10}). In other words, ancient and vernacular dialects of Greek were thought to have emerged under similar circumstances.

Certain scholars associated the dialects with the Greeks’ natural disposition and innate character. This connection was, however, much rarer. In this case, the Greek dialects were taken as a symptom of a negative characteristic of the Greek people as a whole: their inconstancy. This emerges most clearly from the words of Franciscus Junius the Elder, quoted at the outset of this chapter and labeling the Greeks as “verbose” and “mendacious” because of a certain “malicious rivalry” that led them to forge so many different dialects. This view was silently copied by the Dutch biblical scholar Johannes Leusden (1624–1699).\footnote{Leusden (1656: a.4\textsc{\textsuperscript{r}}\textsc{–}a.4\textsc{\textsuperscript{v}}, 167). Schultens (in Eskhult fc.: §\textsc{xlix.}δ) quoted Leusden, without realizing that Leusden relied on Junius. For the rivalry among speakers of different dialects, cf. also Baile (1588: 5\textsc{\textsuperscript{r}}); Schörling \& Michaelis (1678: \textsc{b.3}\textsc{\textsuperscript{r}}).} It was moreover implicit in Lorenzo Valla’s ridiculing of Greek multiplicity as opposed to Roman uniformity, quoted earlier in this section.

\subsection{Reconstructing ancient Greece: Antiquarians on the dialects}
\hypertarget{Toc19704855}{}
As the previous section has shown, Renaissance Hellenists realized that the phenomenon of Greek linguistic diversity was not only relevant for the study of language and literature, but could also help a scholar shed light on other aspects of ancient Greece, especially the character of its tribes and its geopolitical constitution. This realization motivated many authors to devote attention to the Greek dialects outside of philological contexts in the strict sense, especially in the not always clearly distinguished fields of historiography, antiquarianism, and geography. How did scholars active in these branches fit dialectal diversity into their descriptions and reconstructions of ancient Greece and its regions and colonies? Let me provide a brief and necessarily eclectic answer to this question, which deserves further study.

\citealt{In1589}, the obscure Taranto philologist and antiquarian Giovanni Giovane (Latinized: Johannes Juvenis) published his \textit{Eight} \textit{books} \textit{on} \textit{the} \textit{antiquity} \textit{and} \textit{changing} \textit{fortune} \textit{of} \textit{the} \textit{people} \textit{of} \textit{Taranto}. One of the first sections of this historiographical-antiquarian monograph comprised a short lexicon of the ancient Greek dialect spoken in the city of Taranto or \textit{Táras} ($T\text{\textgreek{'a}}\rho \alpha \varsigma $), its original Greek name, situated in modern-day southern Italy (\citealt{Giovane1589}: 9–18). Giovane was, however, aware that not all words he included were specific to Taranto. Yet he still presented Tarentine as a distinct Greek dialect and recognized it as a variety of Doric. Giovane (1589: 8–9) did so on the authority of Aristotle as well as by pointing out the Doric character of the extant fragments attributed to the Pythagorean philosopher Archytas of Tarentum (5th/4th cent. \textsc{bc}). What is more, Giovane believed it to be common knowledge that grammarians have reckoned Tarentine Greek among the countless dialects of the language (cf. Chapter 2, \sectref{sec:key:8}).

The Dutch antiquarian Johannes Meursius (1579–1639) inserted information on specific Greek dialects in a fashion similar to Giovane in two posthumously published works: firstly, a book on ancient Laconia, in which the Doric character and particularities of its dialect were outlined (\citealt{Meursius1661}: 216–233), and secondly, a treatise on ancient Crete and other Greek islands, in which the Doric Cretan dialect was described and Cretan words were listed (\citealt{Meursius1675}: 254–258). Apart from such antiquarian writings, the Greek dialects attracted attention in more general works on the history of ancient Greece and neighboring areas, especially in the eighteenth century, for instance in Charles Rollin’s (1661–1741) popular multivolume account of ancient history and in Nicolas Fréret’s (1688–1749) dissertation on the first inhabitants of ancient Greece. Rollin (1731: 395–396) linked the dialects to the enormous geopolitical diversity of ancient Greece, whereas Fréret (1809 [1746–1747]: esp. 107–129) framed Greek within a larger family of dialects anciently spoken over an area stretching from Celtic lands to those of the Syrians and Medes; against this background, he described the development of Greek and its dialects out of a now lost protolanguage.

Clearly written out of historiographical interest was the \textit{Brief} \textit{dissertation} \textit{on} \textit{the} \textit{settlements} \textit{and} \textit{colonies} \textit{of} \textit{the} \textit{dialects} \textit{of} \textit{the} \textit{Greek} \textit{language} \REF{ex:key:1620} of the Bruges humanist Andreas Hoius (1551–1635). A professor of Greek and history at the university of Douai, today in northern France, \citet[95]{Hoius1620} principally attempted to trace the history of the Greek tribes and their migrations, which he held responsible for the variation in the Greek tongue, as well as to map out the geography of Greece. The dialects themselves hovered in the background of this dissertation, and Hoius mentioned only some of their linguistic particularities explicitly. One of his main theses was that all the Greek dialects were originally spoken in Greece in the strict sense, from which he excluded Asia Minor, part of modern-day Turkey.\textsuperscript{} What is more, there were initially only two tribes in Greece, which Hoius asserted on the authority of Herodotus: the migratory “Pelasgians”, equated with the Aeolians, from whom the Romans derived, and the stationary “Hellenists” (\citealt{Hoius1620}: 102, referring to Herodotus 1.57–58).

The history of the Greek tribes and the historical status of the dialects also served as the principal focus of a dissertation defended by the Hellenist Georg Friedrich Thryllitsch in 1709 at the university of Wittenberg.\footnote{Cf. the dissertation presented (likewise at Wittenberg) by Georg Caspar Kirchmaier and Johannes Crusius (= \citealt{KirchmaierCrusius1684}), even though here the history of the Greek alphabet (chapters \textsc{i}{}-\textsc{iii}), the correct pronunciation of Greek (chapter \textsc{iv}), and the particularities of the Greek dialects (most of chapter \textsc{v}) were the main focus of attention.} Its title neatly summed up Thryllitsch’ goal, which consisted in presenting “some historical-technical suggestions about the Greek dialects collected on the basis of a consideration of the origins and migrations of the Greek tribes”.\footnote{“Suspiciones quasdam historico-technicas de dialectis Graecis ex consideratione originum migrationumque Graecarum nationum collectas”.} One of the main aims of this dissertation consisted in reconciling the biblical account with that of Greek historiographers, for which the traditional association of Javan with Ion – making Ionic the oldest Greek dialect – was invoked (\citealt{Thryllitsch1709}: \textsc{a.4}\textsc{\textsuperscript{r}}\textsc{–b.3}\textsc{\textsuperscript{r}}; cf. Chapter 5, \sectref{sec:key:4}). The French etymologist Gilles Ménage (1613–1692) apparently planned to compose seven books on the ancient Greek dialects, as Leibniz (1991 [ca. 1712]: 252) informs us; this might have been the culmination of the early modern historiographical interest in the dialects, since Ménage’s work was not only intended to include – like Hoius’ and Thryllitsch’ accounts – information on Greek geography, tribes, and colonies, but also an extensive description of the linguistic particularities of the dialects. It was, unfortunately, never realized.

Historiographers and especially antiquarians sometimes put their philological knowledge of the Greek dialects into practice when analyzing the language of Greek inscriptions discovered in the Mediterranean area. This application was, however, relatively rare, probably because the discipline of epigraphy was only nascent – the first collections containing Greek inscriptions were published in the late sixteenth century – and the Greek dialects remained predominantly tied up with the study of literary texts.\footnote{On Greek inscriptions in the early modern period, see Stenhouse (fc.). \citet{Stenhouse2005} discusses the occasional usage of Greek inscriptions by sixteenth-century Italian historiographers. \citet{Liddel2014} briefly elaborates on the usage of the so-called Parian Marble in early modern chronology. A more comprehensive study of the early interest in Greek epigraphy remains a desideratum.} Yet the antiquarian editors of Greek inscriptions did their best to identify the dialect of the pieces they were publishing, with varying success. Thomas Lydiat relied on his knowledge of the Greek dialects and of Greek history to identify the language of the Parian Chronicle as a mixed koine–Ionic variety; scholars now agree, however, that it is composed in Attic, even if on the island of Paros, where the chronicle was found, a variety of central Ionic was originally spoken.\footnote{See Lydiat in Prideaux (1676: \textsc{ii}.116–117). On the dialect of Paros, see e.g. Alonso \citet[531]{Déniz2018}.} Lydiat’s observation featured in the epigraphic collection edited by Humphrey Prideaux (1648–1724) in 1676 and centered around the so-called Arundel marbles. These were named after the eager art and antiquities collector Thomas Howard (1586–1646), Earl of Arundel, who had acquired the marble sculptures and inscriptions through his contacts in the Ottoman empire, thus laying the foundation of the first major collection of Greek inscriptions in England, now principally preserved at the Ashmolean Museum in Oxford (on the eventful history of the marbles, see \citealt{Vickers2006}). Prideaux (1676: \textsc{i.}a.1\textsc{\textsuperscript{v}}, 123) himself drew attention to an inscription in the collection regarding a treaty between two Cretan cities because of its unusual dialect. Even though he cautiously pointed out some Doric features in his notes to this inscription, he did not feel confident enough to identify its language as Doric. In summary, Lydiat and Prideaux activated their philological knowledge of the Greek dialects for antiquarian-epigraphic purposes, but not always successfully so.

In the seventeenth century, inscriptional evidence was occasionally also invoked by scholars tackling typical philological questions such as the variety and history of the Greek language and the literary usage of the dialects. Claude de \citet[430]{Saumaise1643a} saw the Doric character of Cretan Greek confirmed by epigraphic data, whereas Richard \citet[311]{Bentley1699} combined his knowledge of the Greek dialects and inscriptional evidence to correctly identify the dialect of Sicily as Doric. Bentley did so in his well-known dismissal of the authenticity of a collection of letters written in Attic and attributed to Phalaris, the tyrant of Akragas on Sicily (modern-day Agrigento) in the sixth century \textsc{bc}. How could a Sicilian tyrant ever have written letters in Attic, especially considering that this dialect had not yet eclipsed all the others in Phalaris’ lifetime? If the letters were indeed authored by Phalaris, Bentley convincingly pointed out, they would have been written in a variety of Doric.

The eighteenth century witnessed an increasing interest in Greek inscriptions, especially among antiquarians who had enjoyed a decent philological education. Hellenists finally started to consider inscriptions to be a valuable source of dialectal data (cf. \citealt{Walch1772}: 87). This growing fascination with epigraphical documents also resulted in lengthier discussions of the dialectal identity of specific inscriptions or collections of inscriptions. Let me take a look here at two notable Italian examples. The priest and early archeologist Alessio Simmaco Mazzocchi (1684–1771) was the first to edit in their entirety the so-called Heraclean Tablets, two bronze plates discovered separately in 1732 and 1735 near the ancient city of Heraclea Lucania in the very south of modern-day Italy and currently preserved in the archeological museum of Naples. One side of the tablets contains a Latin legal inscription from the first century \textsc{bc}, which Michael Maittaire had already published in 1735; the other has two Greek inscriptions from the late fourth or early third century \textsc{bc}.\footnote{See \citet{UguzzoniGhinatti1968} for a modern edition and discussion of the Heraclean Tablets. See also \citet{Weiss2016}, who argues that the dating of the tablets should be reconsidered.} Mazzocchi included an extensive commentary on the tablets in his edition, which appeared in 1754 at the Naples printing press of Benedetto Gessari and also touched on linguistic aspects of the inscriptions. Thanks to his excellent philological education, he was able to correctly identify the dialect of the Greek inscriptions as Doric, which he believed to be the oldest variety of Greek. However, misguided by the obscure ancient and medieval accounts on the Greek dialects as well as by the odd-looking alphabet of the inscriptions, Mazzocchi (1754: 118–120) further specified the language as “Old Doric” as opposed to the “New Doric” dialect. This New Doric was allegedly introduced by Sicilian poets such as Epicharmus and Sophron in the fifth century \textsc{bc}. Mazzocchi contended, however, that New Doric did not spread to all regions at the same time, and some regions, like Magna Graecia in Italy, preserved Old Doric for a longer period. This complex argument allowed Mazzocchi to situate the two Greek inscriptions in approximately the correct time frame – i.e. around 250 \textsc{bc} – as well as to account for its unusual orthography. He even proposed a relative chronology for the two inscriptions, based on orthographic and linguistic data \citep[135]{Mazzocchi1754}. In conclusion, Mazzocchi’s philological schooling enabled him to formulate a detailed and well-founded assessment of the language of the Heraclean Tablets, even if his results were still firmly grounded in traditional ideas on the Greek dialects and his views have been surpassed by modern scholarship (see \citealt{Weiss2016} for a state of the art).

My second example is the Sicilian antiquarian and numismatist Gabriele Lancillotto Castelli (1727–1794), who relied on established dialectal features to prove that not only Doric, but also Attic and Ionic were spoken on his native island, contrary to what was commonly believed (\citealt{Castelli1769}: \textsc{xv}). The language of inscriptions of various types, including coins, constituted one of Castelli’s principal pieces of evidence for his hypothesis (\citealt{Castelli1769}: \textsc{xv–xvi,} \textsc{xxi}). At the same time, however, he also made ample use of ancient authorities to substantiate his views. For example, inspired by the historian Thucydides, he claimed that a kind of intermediate Doric–Chalcidian Ionic variety was in use among the inhabitants of the Sicilian city of Himera (see \citealt{Castelli1769}: \textsc{xxxiii} for a neat overview of his theses). He still wavered, in other words, between evidence and authority as he was exploring the new inscriptional data available to him.

Moving beyond historiographical and antiquarian works focusing on ancient Greece, I cannot leave unmentioned here that the Greek dialects of antiquity were often the only ones mentioned at some length in geographical descriptions of Europe or the world, especially before the eighteenth century. The English churchman Peter Heylyn (1599–1662) referred to them in his long description of Greece, included in his \textit{Microcosmus,} \textit{or} \textit{A} \textit{little} \textit{description} \textit{of} \textit{the} \textit{great} \textit{world} of 1621: “The language they spake was the \textit{Greeke}, of which were five dialects, \textit{1} \textit{Atticke.} \textit{2} \textit{Doricke.} \textit{3} \textit{Aeolicke.} \textit{4} \textit{Beoticke.} \textit{5} The \textit{common} dialect or phrase of speech” (\citealt{Heylyn1621}: 205; see e.g. also \citealt{Speed1676}: 15, 60, 63). He claimed to be relying on Nicolaus Clenardus’ grammar of Greek, but Heylyn’s classification into Attic, Doric, Aeolic, Boeotian, and koine does not feature in Clenardus’ work and has no parallels in the early modern period. In the revised edition of 1625, \citet[375]{Heylyn1625} replaced Boeotian by Ionic, most likely because he had realized his idiosyncrasy.

In summary, the dialects occupied an important place in a number of early modern historiographical and antiquarian works concentrating on parts of ancient Greece, and they were often discussed in close conjunction with the history, geography, and tribes of Greece. In the eighteenth century, antiquarians increasingly involved epigraphic dialect evidence in their attempts at providing encompassing descriptions of ancient Greece and its many different settlements, especially those in regions of modern-day Italy. The Greek dialect inscriptions from these areas were, after all, better accessible to Western scholars than the ones hidden away in Ottoman Greece. The dialects, finally, also figured in comprehensive geographical works covering more than Greece alone, albeit more marginally so. These accounts tended to be rather unoriginal in their information regarding the dialects, as Heylyn’s case demonstrates.

\subsection{Conclusion}
\hypertarget{Toc19704856}{}
Before the modern period, scholars eagerly applied evaluative labels to the canonical ancient Greek dialects. Most of these attitudes must be understood against the background of the study of Greek literature and resulted in particular from the perceptions readers had of texts and their form. This holds for ancient and medieval times as well as for the early modern period, even though early modern philologists also relied to a considerable extent on the attitudes of their predecessors. Scholars linked the Greek dialects with other aspects of ancient Greece and Greek culture as well, and increasingly so from the Renaissance onward. Assumptions about the customs of individual Greek tribes triggered specific attitudes toward their respective dialects, and, on a more general level, the fickleness of the Greek people in its entirety was believed to have caused the vast dialectal diversity of its language. Put another way, early modern stereotypes about Greeks in general and the tribes of ancient Greece in particular played a pivotal role in evaluating Greek linguistic diversity. In addition, the authors perceived a close connection between the Greek dialects and the ethnic and geopolitical constitution of Greece. To sum up, early modern scholars attempted to fit the dialects into the larger picture of ancient Greece. Even though they principally had a philologically colored view of the matter, they frequently related the dialects to other, non-textual aspects of Greek culture and Greekness.

\section{\textsc{8} The Greek dialects in confrontation}
\hypertarget{Toc19704857}{}
“It is common knowledge that there are nowhere better-known and more distinct dialects than in the Greek language”.\footnote{\citet[23]{Wesley1736}: “In propatulo est quod nullibi notiores aut distinctiores sint dialecti quam in lingua Graeca”.} This is how the English clergyman and poet Samuel Wesley (1662–1735) introduced his concern that it was difficult to formulate rules of dialectal change. Wesley did so when discussing the style and language of the Old Testament book of Job, which he regarded as a kind of Hebrew that had features of related dialects. By dialects he mainly meant Arabic and Syriac. He attempted to discover a certain regularity in Oriental variation and referred in this context to the ancient Greek dialects. In fact, Wesley assumed that the letter mutations among the Greek dialects could be transposed to the Oriental context without any problem. This implies a presupposition on Wesley’s part that both linguistic contexts were comparable, which also emerges from his explicit connecting of specific Greek dialects to individual Oriental tongues. Indeed, \citet[24]{Wesley1736} attributed similar linguistic properties to Doric Greek and Syriac, on the one hand, and to Attic Greek and Arabic, on the other.

Samuel Wesley was not alone in comparing ancient Greek dialectal diversity to other contexts of dialectal or dialect-like variation. Indeed, it was common early modern practice to assert that the Greek dialects were either comparable with, or clearly different from, diversity within other languages or language families, especially the Western European vernaculars, Latin, and the close-knit group of so-called Oriental tongues, now known as the Semitic language family. What arguments did early modern scholars invoke when claiming comparability or lack thereof? And how do their views relate to the intellectual and linguistic context in which they operated? It is these two major questions I want to address in the final chapter of this book.

\subsection{The vernaculars of Western Europe and the Greek reflex}
\hypertarget{Toc19704858}{}
It comes as no surprise that scholars from early modern Western Europe compared the ancient Greek dialects most frequently to their native vernaculars. The confrontation with Greek triggered a reflex among Western European scholars to relate Greek variation to the regional diversity which they encountered in their mother tongues. It is, however, remarkable that they did so in various ways and for various purposes. What were their most significant incentives to emphasize or dismiss the comparability of ancient Greek with vernacular dialects?

\subsubsection{Explanation: The Greek dialects in need of clarification}
\hypertarget{Toc19704859}{}
When Greek studies started to develop on the Italian peninsula from the end of the Trecento onward, Renaissance Hellenists were initially compelled to focus primarily on one principal form of the language, consisting in the koine interspersed with some occasional features typical of Attic and Ionic. Toward the end of the Quattrocento, however, Hellenists developed an ever growing interest in the Greek dialects per se and their individual features (see also Chapter 1, \sectref{sec:key:2}). In this process, the dialects obtained a more clearly defined position in the teaching of the Greek language, being usually reserved for more advanced students, often in connection with the study of poetry and its dialectally diversified genres. Grammarians soon realized that if they wanted to efficiently explain the nature of the ancient Greek dialects to their students, they needed to appeal to a situation more familiar to their audience, in particular the regional diversity in their native vernacular tongue. As Greek studies boomed first in the states of northern Italy, it is not hard to see why vernacular dialects were first invoked by Italian grammarians to explain the existence of different forms of ancient Greek. For instance, in his updated commentary on Guarino’s abridgement of Manuel Chrysoloras’ Greek grammar, published in Ferrara in 1509, the professor of Greek Ludovico da Ponte noted that there were five principal tongues among the Greeks: the koine, Doric, Aeolic, Ionic, and Attic, the most pre-eminent among them. Da Ponte (1509: 20\textsc{\textsuperscript{v}}–21\textsc{\textsuperscript{r}}, 46\textsc{\textsuperscript{v}}–47\textsc{\textsuperscript{r}}) compared these dialects at two different occasions to the varieties of Italian spoken by the Venetians, the Bergamasques, the Florentines etc. (on Da Ponte, see also Chapter 2, \sectref{sec:key:6}). Originally from the city of Belluno in the Veneto region, he related his native Venetian with elegant Attic speech, even claiming that Venetian was “the most beautiful and learned speech of all, scented with the entire majesty of the Greek language”.\footnote{Da Ponte (1509: 47\textsc{\textsuperscript{r}}): “pulcherrimus et doctissimus omnium sermo, in quo redolet tota linguae Graecae maiestas”.} Such explanatory comparisons, in this case with a distinctly patriotic touch, occurred very frequently from the early sixteenth century onward, usually in a didactic context.

The procedure was quickly picked up by grammarians outside of the Renaissance heartland of Italy. It happened particularly early in Philipp Melanchthon’s successful Greek grammar, first published in 1518, in which the Protestant Hellenist assumed the existence of a certain south-western High German common language in Bavaria and Swabia. Melanchthon might have been thinking of the southern German print language, one of the three regional print languages emerging after 1500 (see \citealt{Mattheier2003}: 216), or some other form of a regional koine. The reference to his native German context served to explain the status of the Greek koine to his readership of prospective Hellenists (\citealt{Melanchthon1518}: a.i\textsc{\textsuperscript{v}}). The first Greek grammar composed by a Spanish scholar, Francisco de Vergara, adopted the same technique; a brief description of native regional varieties was offered to help the Spanish reader understand ancient Greek diversity (\citealt{Vergara1537}: 209–210). Revealing in this context is the 1561 edition of the Greek grammar composed by the German pedagogue Michael Neander (1525–1595), who silently copied the bulk of Vergara’s discussion of the Greek dialects. In doing so, however, Neander (1561: 340–343) left out the reference to Spanish variation, as this would not have been very helpful to a reader with a German background.\footnote{The first edition of Neander’s work (i.e. \citealt{Neander1553}) did not yet contain the passage in question.}

The explanatory use of German dialects in Greek handbooks occurred extremely frequently.\footnote{See e.g. Schmidt (1604: 3–4); \citet[83]{Rhenius1626}; Schörling \& Michaelis (1678: \textsc{b.4}\textsc{\textsuperscript{r}}); Kirchmaier \& Crusius (1684: \textsc{b.2}\textsc{\textsuperscript{v}}); Köber (1701 [1684]: 376); Thryllitsch (1709: \textsc{c.2}\textsc{\textsuperscript{v}}); Nibbe (1725: b.2\textsc{\textsuperscript{v}}\textsc{–}b.3\textsc{\textsuperscript{r}}); \citet[141]{Georgi1733}; Schuster \& \citet[13]{Lauterbach1737}; Simonis (1752: 207–209); Peternader (1776: 191–192); Harles (1778: \textsc{xxvi}).} It is summed up neatly by the renowned Saxon lexicographer of Latin Immanuel Johann Gerhard Scheller (1735–1803), who, though not a grammarian of Greek, briefly discussed the Greek dialects in his reflections on the properties of the German \textit{Schriftsprache}. In this context, Scheller remarked:

\begin{quote}
I want to adduce only a few examples that demonstrate the similarity of the German and Greek dialects, so that in this manner a young person, if he knows it in German, will not be so astonished at it in Greek.\footnote{\citet[229]{Scheller1772}: “Ich will nur wenige Beyspiele anführen, die die Aehnlichkeit der deutschen und griechischen Dialecte beweisen: daß also ein junger Mensch, wenn er es im Deutschen wüste, im Griechischen nicht sich so verwundern würde”.}
\end{quote}

The intensive Greek–German comparison seems to be related to two main historical circumstances: the continuous early modern interest in the history, language, and literature of ancient Greece in German-speaking areas and the flourishing of regional dialects there, which from the end of the seventeenth century onward received monograph-length studies, with a focus on lexical particularities (see Haßler 2009: 877). Clarifying the Greek dialects by referring to native vernacular diversity also occurred in grammars by native speakers of French and English, albeit much less frequently.\footnote{For French, see e.g. Antesignanus (1554: 11–12), on which see Van \citet{Rooy2016c}. For English, see e.g. Milner (1734: 191–192) and \citet[121]{Holmes1735}.} This might be related to the fact that in these politically unified areas grammarians more easily reached a consensus on the vernacular standard to be adopted. As a result, Hellenists in these regions might have sensed that French and English dialects, conceived as corrupt deviations from the revered standard, could not be so easily compared with the highly valued literary dialects of ancient Greek.

Early modern Hellenists did not only fall back on their native context when Greek dialectal variation needed to be explained as a general phenomenon. It was also employed as a point of reference for clarifying the different evaluative attitudes toward the Greek dialects (cf. Chapter 7, \sectref{sec:key:3}). Notably, in his monograph on the Greek dialects, the German professor Otto Walper presented Attic and Ionic as more polished and smooth, whereas he claimed Aeolic and Doric to be less cultivated and not as pleasant to the ears. This, Walper explained, was not very different in “our German language”, which “some provinces speak more smoothly, elegantly, and neatly than others”.\footnote{\citet[61]{Walper1589}: “Vt autem superiores dialecti politiores et suauiores fuere; ita hae duae (Dorica et Aeolica) incultiores et auribus ingratiores existimantur, haud secus atque in lingua nostra Germanica prouinciae aliae aliis loquuntur suauius, concinnius atque politius”.} Also a specialist of Hebrew, Walper went on to suggest that Hebrew resembled Attic and Ionic, whereas Syriac and Aramaic had properties similar to Aeolic and Doric.

Hellenists addressing a more international audience referred to various vernacular contexts when explaining features of the sociolinguistic situation of ancient Greece. In his comprehensive Greek grammar, destined for Jesuit schools in various parts of Europe, the Jesuit Jakob \citet[20]{Gretser1593} referred to the German, Italian, and French “common languages”, the allegedly geographically neutral standard languages that were being developed, to explain the status of the Greek koine to his student readers (for his intended audience, see \citealt{Gretser1593}: )(.4\textsc{\textsuperscript{r}}). The French Hellenist Petrus Antesignanus (1554: 11–12), one of Gretser’s main sources of methodological inspiration, also clarified the status of the Greek koine by a more familiar situation, his native French context. Antesignanus’ case is revealing in that it shows that the explanation did not occur in an entirely unidirectional manner from the vernacular to the ancient Greek context. Instead, certain aspects of the French context seem to have been forced into the Greek straitjacket, as, for instance, the idea that the French common language could be adorned by features of certain approved French dialects. Not all grammarians of French would have agreed with this rather bold claim by Antesignanus. Something similar happened when the eighteenth-century Frisian Hellenist Tiberius Hemsterhuis (1685–1766) took the comparability of Greek and Dutch for granted, using his native context to clarify the status of Greek variation and the Greek common language. In order to explain what the koine was, “I will use”, Hemsterhuis said, “the example of our fatherland”.\footnote{Hemsterhuis (2015 [ca. 1740–1765]: 102): “Mirabitur quis quae sit illa $\kappa o\iota \nu \text{\textgreek{'h}}$. Exemplo utar nostrae patriae, ut id possim explicare”.} This led him to boldly present both the Greek and the Dutch common languages as the standard speech of high society composed out of different dialects and not bound to a specific region (\citealt{Hemsterhuis2015} [ca. 1740–1765]: 102–104). In doing so, he neglected the fact that the Greek koine and the Dutch standard were based principally on specific dialects: Attic in the case of the koine, and Brabantian and Hollandic in the case of Dutch.

In summary, Hellenists widely assumed that it was possible to explain and clarify the foreign as well as ancient phenomenon of Greek dialectal diversity by means of a more familiar context. This usually coincided with the dialects of the native language of the early modern Hellenist grammarian and – more importantly – of his intended readership. Needless to say, this practice emerged out of didactic concerns. As such, it was a neat realization of Juan Luis Vives’ pedagogical insight that a teacher was better equipped to give instruction in Latin and Greek if he also possessed a thorough knowledge of his mother tongue and that of his audience (see \citealt{Padley1985}: 146).

The explanatory usage also appeared outside of strictly grammaticographic and didactic contexts, in which case no thorough knowledge of the Greek dialects was required on the part of the author. For example, in his Latin–Polish dictionary of 1564, Jan Mączyński (ca. 1520–ca. 1587) invoked variation among Slavic tongues alongside the Greek dialects to explain the Latin term \textit{dialectus}, without mentioning, however, any Greek dialect by name:

\begin{quote}
The Greeks call \textit{dialects} species of languages, \textit{A} \textit{property} \textit{of} \textit{languages,} \textit{like} \textit{in} \textit{our} \textit{Slavic} \textit{language,} \textit{the} \textit{Pole} \textit{speaks} \textit{differently,} \textit{the} \textit{Russian} \textit{differently,} \textit{the} \textit{Czech} \textit{differently,} \textit{the} \textit{Illyrian} \textit{differently,} \textit{but} \textit{it} \textit{is} \textit{nevertheless} \textit{still} \textit{one} \textit{language.} \textit{Only} \textit{does} \textit{every} \textit{region} \textit{have} \textit{its} \textit{own} \textit{property,} \textit{and} \textit{likewise} \textit{it} \textit{was} \textit{in} \textit{the} \textit{Greek} \textit{language}.\footnote{Mączyński (1564: \textit{s.v.} “dialectus”): “Dialectos Graeci uocant linguarum species, Vlasność yęzyków yáko w nászim yęzyku Slawáckim ynáczey mowi Polak ynáczey Ruśyn, ynáczey Czech ynaczey Ilyrak, á wzdy yednak yeden yęzyk yest. Tylko ysz każda ziemiá ma swę wlasność, y tákże też w Greckim yęzyku bylo”. The form “wzdy” should be “wżdy”, but the diacritic dot above the <z> does not appear in the original text. I kindly thank Herman Seldeslachts for this information and for helping me translate this early modern Polish passage.}
\end{quote}

Before Mączyński, Thomas Eliot (1538: \textsc{xxxiii}\textsc{\textsuperscript{v}}) had likewise defined \textit{dialectus} with reference to his native context. However, unlike Mączyński, Eliot made no mention at all of Greek variation. This suggests that Eliot preferred to explain the Latin word \textit{dialectus} by means of a familiar situation instead of troubling his reader with the diversity of ancient Greek, far distant in time and space from the sixteenth-century English audience of his dictionary. Later dictionaries focusing on English did, however, include references to both English and Greek dialects.\footnote{See e.g. Bullokar (1616: \textit{s.v.} “dialect”) and Blount (1656: \textit{s.v.} “dialect”). See \citet[7]{Blank1996}.}

In a sixteenth-century English controversy on early Church practices, including the language used during Mass in the east of the Roman Empire, a more familiar linguistic situation was invoked to make claims about ancient Greek diversity. In this so-called Challenge controversy – so named because it started out as a challenge mounted by the Protestant John Jewell (1522–1571) – the English recusant John Rastell (1530/1532–1577) assumed a certain degree of comparability between Greek and English variation, claiming that in both cases there was no mutual intelligibility. He did so as he was trying to demonstrate that not all speakers of Greek would have understood the learned Greek used in Mass.\footnote{On the controversy, see e.g. Jenkins (2006: 115–154). On the use of the English word \textit{dialect} in this context, see Van Rooy \& Considine (2016: 647–651).} Since an Englishman could not understand a Scotsman, there was no reason to stipulate that speakers of different Greek dialects were able to comprehend each other, Rastell (1566: 68\textsc{\textsuperscript{r}}) argued. Rastell’s native English context thus clearly informed his views on the lack of mutual intelligibility among the ancient Greek dialects to make a point in a theological controversy.

The explanatory comparison of vernacular with ancient Greek dialectal diversity occurred in various genres other than Greek grammars, dictionaries, and theological invectives too. These ranged from philological commentaries on classical works and monographs on New Testament Greek to geographical publications, prefaces to lexica, and various historiographical works.\footnote{For a philological commentary, see e.g. Casaubon (1587: 68; French–Greek comparison). For a monograph on New Testament Greek, see e.g. Cottière (1646: 212–213; also French–Greek). For a geographical publication, see e.g. Speed (1676: 60; English–Greek comparison). For a preface to a lexicon, see e.g. Phillips (1658: (b.3)\textsuperscript{v}\textsc{;} also English–Greek). For a historiographical work, see e.g. Fréret (1809 [1746–1747]: 108, 117; French/Italian–Greek comparison).} A particular case in point is John Williams (?1636–1709), who, in his discourse on the language of church service, mentioned English and Greek variation alongside each other when explaining the concept of \textsc{dialect} to his readership, interestingly adding that the Greek dialects were “well known to the learned” \citep[5]{Williams1685}. Does this imply that Williams was providing a reference to the readers’ native context for those who were not as learned? Whatever the case, Williams drew a direct parallel between the Greek koine “standard” – he used this exact term – and court English, projecting along the way his conception of the English standard back onto the Greek koine.

Before proceeding to the next early modern trend in comparing ancient Greek with vernacular dialectal diversity, I want to point out briefly here that the explanatory function did not come into being with the Renaissance revival of Greek studies. As a matter of fact, explaining one linguistic context of variation by means of another occasionally occurred in the late Middle Ages as well, for instance in exegetical works on biblical passages alluding to regional linguistic differences, in particular the shibboleth incident in Judges 12.\footnote{See Van Rooy (2018b: 199–200) for the views of Nicholas of Lyra (1265–1349).} This was especially frequent in travel writings. Chinese variation was compared with Gallo-Romance diversity in the \textit{Book} \textit{of} \textit{the} \textit{marvels} \textit{of} \textit{the} \textit{world}; this work constitutes the written version of what the famous Venetian traveler Marco Polo (1254–1324) dictated to his cell mate in Genoa, Rustichello da Pisa, in 1298–1299. Rustichello, drawing up Marco Polo’s words in Old French, wanted to explain Chinese diversity by referring to the native context of his intended readership. Interestingly, an Italian translator substituted the allusion to diversity in France by referring to Italo-Romance variation, clearly adapting the text to his Italian audience.\footnote{See \citet[157]{Polo1938}, where both the French original and the Italian rendering are offered in an English translation. Cf. Borst (1957–1963: 855).} Mutual intelligibility was explicitly posited for Chinese variation as well as the Italo-Romance dialects, but not for the Gallo-Romance context. This kind of comparison, omitting any reference to ancient Greek, continued to be drawn throughout the early modern period, even though these comparisons were far less frequent than those between ancient Greek and the vernacular. The French explorer and diplomat Pierre Belon (1517–1564), for example, employed his native linguistic context to explain to his readers that inhabitants of Constantinople mocked the vernacular Greek spoken by outsiders. Just as the French laughed at Picard speech and any other Gallo-Romance variety that was not true French, residents of Constantinople jibed at other varieties of vernacular Greek, Belon (1553: 5\textsc{\textsuperscript{v}}) remarked.\footnote{On Belon as a traveler in Greece, see Vingopoulou (2004: esp. 122).} Such comparisons of different contexts of variation excluding ancient Greek also occurred outside of travel writings. The Spanish Dominican Domingo de Santo Tomás (1499–1570) explained Quechua diversity by referring to Romance differences in pronouncing Latin in his grammar of the South-American language.\footnote{See Santo Tomás (1560: 1\textsc{\textsuperscript{v}}). On his eye for variation, see Calvo \citet[140]{Pérez2005}.} Another noteworthy example stems from the correspondence of the Parisian humanist Claude Dupuy (1545–1594). \citet[274]{Dupuy2001} clarified Provençal diversity to his Neapolitan colleague Gian Vincenzo Pinelli (1535–1601) by comparing it to the variation in his addressee’s native Italian language in a letter dated December 12, 1579.

\subsubsection{Justification and description: Greek as a polyvalent model}
\hypertarget{Toc19704860}{}
It came as a relief to many humanists that, unlike Latin, the revered ancient Greek language was not a monolithic linguistic whole. This reminded them of the situation in their native vernaculars and made them at the same time aware of the fact that dialectal variation was not necessarily an insurmountable obstacle to the regulation and grammatical codification of their mother tongue. An observation in the first printed grammar of Dutch is revealing in this regard. This language, its authors argued, could be regarded as one entity, even though there were regional differences in pronunciation, “but not in such a manner that they do not understand each other very well”. Interestingly, they added that “in like manner the Greek language, which enjoys such high esteem, also had its different ‘dialects’”.\footnote{[Spieghel \textit{et} \textit{al.}] (1584: 110): “Ick spreeck […] int ghemeen vande Duytse taal, die zelve voor een taal houdende, […] wel iet wat inde uytspraack verschelende, maar zó niet óf elck verstaat ander zeer wel, tis kenlyck dat de Griexe taal, die zó waard gheacht is, óóck haar verscheyden \textit{dialectos} had”. For the authorship of this grammar, see \citet{Peeters1982}.} The addition of the relative clause “which enjoys such high esteem” clearly points to a justificatory use of ancient Greek diversity. This suggests that an acquaintance with the Greek literary dialects, however slight, catalyzed the emancipation of the vernaculars from Latin, which, certainly in the fourteenth and fifteenth centuries, was often conceived of as a highly uniform language synonymous with grammar.\footnote{On the catalyzing effect, see e.g. already Bonfante (1953–1954: 688); \citet[9]{Trapp1990}; \citet[67]{Rhodes2015}. On Latin as an allegedly uniform tongue, see \sectref{sec:key:2} below.} The catalyzing effect seems confirmed by the fact that early comparisons of Greek with Italian diversity sometimes included an explicit contrast with the unity of Latin.\footnote{See e.g. Landino (1974 [1481]: \textsc{ii}.41) and Manutius (1496: *.ii\textsc{\textsuperscript{v}}). See Alinei (1984 [1981]: 172–173); Trovato (1984: 209–210, 215). For the justificatory use of the Greek model in Italy, see Tavoni (1998: 46, 50).} It comes as no wonder then that the diversified linguistic patchwork of ancient Greece widely functioned as a model for scholars engaged in elevating, standardizing, and describing their native vernacular language. This intriguing tendency manifested itself in various ways.

To begin with, many sixteenth-century scholars saw in the Greek dialects a literary model, which must be framed in the tradition of claiming a close link between ancient Greek and one’s native vernacular.\footnote{On the Greek–vernacular link, see \citet{Demaizière1982}, with a focus on the French context; \citet{Trapp1990}; \citet{Dini2004}, with reference to Prussian. For a late example, see Van Hal (2016: esp. 435–436), who concentrates on Reitz (1730: 119–132) and his linking of Dutch to Greek.} The most telling example of this use of the Greek dialects can be found in the work of the renowned French printer and Hellenist Henri Estienne.\footnote{On Estienne’s comparison of French and Greek diversity, see already \citet[70]{Demaizière1988}.} In his \textit{Treatise} \textit{on} \textit{the} \textit{conformity} \textit{of} \textit{the} \textit{French} \textit{language} \textit{with} \textit{the} \textit{Greek}, Estienne defended the usage of dialect words in French literary works, adding that dialect words needed to be adapted to the common French tongue, just like meat imported from elsewhere must be prepared in the French manner and not as it was cooked in the land of origin.\footnote{Estienne (1565: ¶¶.ii\textsc{\textsuperscript{v}}). Cf. Ronsard (1550: ✠\textsc{\textsuperscript{r}}), on which see Alinei (1984 [1981]: 170]); Barbier-\citet[24]{Mueller1990}; \citet[14]{Trapp1990}. Cf. also Mambrun (1661: 456, 458). Similar views were expressed by scholars from other areas: see e.g. Oreadini (1525: \textsc{e.}iii\textsc{\textsuperscript{v}}–\textsc{e.}iv\textsc{\textsuperscript{r}}) for Italian and Craige (1606: \textsc{a}.vi\textsc{\textsuperscript{r}}) for English.} Estienne (1579: 133; 1582: *.iii\textsc{\textsuperscript{v}}–*.iiii\textsc{\textsuperscript{r}}) propagated the usage of the ancient Greek satirical author Lucian as a model for this practice. Inspired by the Greek heritage, he regarded French dialectal diversity as a source of richness that could adorn the French language (\citealt{Estienne1582}: *.iii\textsc{\textsuperscript{v}}; see \citealt{AurouxClerico1992}: 366–367). The fact that \citet[143]{Estienne1579} allowed for dialect words and even dialect endings in French implies that to his mind “the pure and native French language” (\textit{le} \textit{pur} \textit{et} \textit{nayf} \textit{langage} \textit{françois}) did not entirely correspond to Parisian speech, the variety on which the French norm was primarily based. He explained this by drawing a comparison with the Attic dialect, in which not every Athenian feature was allegedly approved. Estienne (1579: 133–134) denied the same flexibility to Italian, since its Tuscan-based standard was much less prone to adopt features from other dialects. To sum up, Estienne, inspired by the Greek dialects he knew so well, viewed the French dialects as a source of richness that could embellish French language and literature. He perceived esthetic and typological similarities between French and Greek dialects, even though he did not go so far as to make any claims about the genealogical dependency of French on Greek (\citealt{Droixhe1978}: 99; \citealt{Considine2008a}: 62).

Such ideas also appeared outside of France. The great German grammarian Justus Georg Schottel (1612–1676), for instance, argued that not everything outside of the selected dialect – in particular Attic Greek and the German of Meissen – was faulty (\citealt{Schottel1663}: 176; cf. \citealt{Roelcke2014}: 250). What is more, not all dialect words must be avoided, since some could be current in certain technical jargons. These considerations led Schottel to conclude that frequent and important dialect words needed to be included in a dictionary. The value he attached to dialectal material clashes somewhat with his view, expressed only some pages earlier, that dialects were inherently incorrect and unregulated \citep[174]{Schottel1663}. William J. \citet[1110]{Jones2001} has summed up this contradiction nicely:

\begin{quote}
Himself a native speaker of Low German, Schottelius was caught between admiration for a[n] […] etymologically valuable dialect, and an awareness that prestige and currency precluded any choice but High German.
\end{quote}

Other German scholars stressed the richness of vernacular dialects as well, often with reference to the ancient Greek context.\footnote{See e.g. Chytraeus (1582: \textsc{a.3}\textsc{\textsuperscript{r}}\textsc{–a.3}\textsc{\textsuperscript{v}}); Meisner (1705: \textsc{c.1}\textsc{\textsuperscript{r}}); \citet[73]{Hertling1708}.}

Occasionally, patriotic sentiments tempted scholars to accord a special status to the dialects of their native vernacular tongue. This happened in Manuel de Larramendi’s (1690–1766) Basque grammar, which contains a section “On the dialects of the Basque language” (“De los dialectos del bascuenze”; \citealt{Larramendi1729}: 12–15). Larramendi’s views were clearly informed by early modern scholarship on the Greek dialects. He emphasized that, much like Greek, Basque had a common language, a “body of language common and universal to all its dialects”.\footnote{Larramendi (1729: 12–13): “cuerpo de lengua, comun y universal à todos sus dialectos”.} Further, he seems to have projected the distinction between principal and minor dialects from early modern grammars of Greek onto the Basque context (\citealt{Larramendi1729}: 12; see Chapter 2, \sectref{sec:key:6}). Greek and Basque diversity was, however, not comparable on every level, claimed \citet[12]{Larramendi1729}:

\begin{quote}
The difference is that the dialects of the Basque language are very regulated and consistent, as if they were invented with devotion, discretion, and expediency, which the Greek dialects did not have and others in many other languages do not have.\footnote{“La diferencia está que los dialectos del bascuenze son muy arreglados y consiguientes, como inventados con estudio, discrecion y oportunidad: lo que no tenian, ni tienen los dialectos griegos, y otros en otras muchas lenguas”. On this passage, see also Haßler (2009: 876).}
\end{quote}

In other words, the Greek dialects served as a model for Larramendi in several respects, but were at the same time valued less highly than their Basque counterparts, a quite unusual idea in the early modern period. In a work published a year earlier, however, \citet[142]{Larramendi1728} had presented the Greek dialects as also being regulated. It is unclear why exactly he had this change of heart, but patriotic sentiment no doubt played a role.

Not all scholars associated the Greek dialects with spoken varieties of the vernaculars. The Dutch grammarian Adriaen Verwer (ca. 1655–1717) was aware of the literary character of the Greek dialects and compared them with different written registers of his native vernacular rather than with spoken regional dialects. Verwer (1707: 53–54) divided written Dutch into three main forms: \REF{ex:key:1} the common language (\textit{lingua} \textit{communis}), \REF{ex:key:2} the dialect used in government (\textit{dialectus} \textit{curiae} \textit{senatuique} \textit{familiaris}), and \REF{ex:key:3} the poetical dialect (\textit{dialectus} \textit{poetis} \textit{familiaris}). Verwer also mentioned a court dialect (\textit{dialectus} \textit{forensis}), a variety closely cognate to the common language, from which it only differed in rhetorical – and not in grammatical – terms. The focus on register variation is also apparent from his definition of the Latin term \textit{dialectus}; dialects were “various particular speech forms in our written language”.\footnote{[Verwer] (1707: 53): “dese ende gene, bysondere spraekvormen in onse schrijftaele”.}

The situation of ancient Greece also functioned as a model for selecting a variety to be codified as the vernacular norm. A very straightforward example of such an approach can be found in Nathan Chytraeus’ (1543–1598) preface to his Latin–Low Saxon lexicon of 1582. In it, Chytraeus (1582: \textsc{a.3}\textsc{\textsuperscript{r}}\textsc{–a.3}\textsc{\textsuperscript{v}}) described the constitution and elevation of a German common language as a process awaiting completion and stressed the model function of the Greek koine in this context. He moreover saw a key role for the dialects, which could beautify the common language. More theoretical still were the proposals by certain early Cinquecento Italian scholars to create a mixed common language after the example of the Greek koine as an artificial solution to the \textit{questione} \textit{della} \textit{lingua}.\footnote{See Vincenzo Colli’s ideas as quoted by Pietro Bembo (1525: \textsc{xii}\textsc{\textsuperscript{v}}\textsc{–xiii}\textsc{\textsuperscript{r}}). See \citet[119]{Melzi1966}; Trovato (1984: 215–218); \citet[12]{Trapp1990}.} Not all humanists limited themselves to mere reflection. The Dutch scholar and priest Pontus de Heuiter (1535–1602) put the active creation of a vernacular common language through mixture to actual practice in his \textit{Dutch} \textit{orthography}. De Heuiter explicitly mentioned his debt to the ancient Greek model for his initiative:

\begin{quote}
I have taken the Greeks as an example, who, having the four good tongues of the country in usage, namely \textit{Ionic}, \textit{Attic}, \textit{Doric}, and \textit{Aeolic}, have created a fifth one out of them, which they called the \textit{common} \textit{language}. Thus I have created my Dutch over a period of twenty-five years out of Brabantian, Flemish, Hollandic, Guelderish, and Kleverlandish.\footnote{De \citet[93]{Heuiter1581}: “[…] heb ic exempel ande Grieken genomen, die vier lants goude talen in ufenijng hebbende, te weten: \textit{Ionica,} \textit{Attica,} \textit{Dorica,} \textit{Aeolica}, die vijfste noh daer uit gesmeet hebben, die zij nommen \textit{gemeen} \textit{tale}: aldus heb ic mijn Nederlants over vijf en twintih jaren gesmeet uit Brabants, Flaems, Hollants, Gelders en Cleefs”. See also \citet[110]{Dibbets2008} and De Vooys (1917: 13–14). The latter has linked this passage to Hieronymus Wolf’s reference to Greek in his discussion of German dialects. However, Wolf did not explicitly take the Greek context as a model and seems to have stressed, instead, the incomparability of both contexts. See \sectref{sec:key:1.3} below.}
\end{quote}

Not all scholars using the Greek koine as a model for their vernacular norm believed the koine to be created out of the different dialects. The grammarian Kaspar von Stieler (1632–1707) held that the Greek koine, which he saw as a model for his High German norm, was exempt from dialectal elements \citep[2]{Stieler1691}. Interestingly, later authors emphasized the frequently drawn parallel between the Greek koine and the German norm by referring to the former as “High Greek” (\textit{Hoch-Griechisch}) by analogy to “High German” (\textit{Hochdeutsch,SchusterLauterbach1737}: 13).

Not everybody regarded the Greek koine as the model for the selected, normative variety of their vernacular tongue. Almost equally often, scholars put forward the Attic dialect as the main form of Greek and the principal model after which one’s mother tongue should be developed. This holds especially true in cases where scholars emphasized the literary function of the selected variety. A telling example is Henri Estienne (1582: *.iii\textsc{\textsuperscript{v}}), who put French in the capital city of the kingdom; just as Athens was the “Greece of Greece” in terms of speech, Paris was the “France of France”. Estienne added, however, that this was the case not because the French capital was frequently visited by the royal court, but because it had a parliament – he was perhaps inspired here by the example of Athenian democracy. He was thus comparing the French language to Attic rather than to the Greek koine. This was surely prompted by his emphasis on the codification of French as a respected literary norm similar to Attic rather than a language understood by all inhabitants of the kingdom. In fact, Estienne (1582: *.iii\textsc{\textsuperscript{r}}) seems to have regarded pure French as a social privilege which the lower classes could never attain.\footnote{Cf. Marineo Sículo ([ca. 1497]: \textsc{xxxiii}\textsc{\textsuperscript{v}}) for an early comparison of Castilian Spanish with Attic Greek.}

Taking Attic and especially the Greek koine as the model for selection had far-going glottonymic consequences. Indeed, the designation “common language” was widely used to refer to the selected variety of a vernacular language in imitation of the Greek koine, usually termed \textit{lingua} \textit{communis} in Latin. What is more, some even referred to the vernacular norm, by the procedure of antonomasia, as “Attic”. The Greek scholar Alexander Helladius (1686–after mid-1714) attributed the label of “Attic” to what he called the “High German par excellence” (“$\kappa \alpha \tau $’ $\text{\textgreek{>e}}\xi o\chi \text{\textgreek{`h}}\nu $ \textit{das} \textit{Hochteutsche}”; \citealt{Helladius1714}: 187). Attic or koine Greek were not, however, the only speech forms that could serve as the model for selecting a vernacular norm. In cases where a vernacular variety was described that was not or not yet fully established as the selected norm but which an author wanted to see established, it was sometimes compared to varieties of languages other than Greek that were widely accepted as the standard form. One scholar writing in 1595 wanted to promote his native Croatian dialect as the Slavic norm, for which Tuscan Italian constituted his model (\citealt{Veranzio1595}: *.3\textsc{\textsuperscript{v}}; cf. also \citealt{Schoppe1636}: 46).

Apart from selection, Greek could also be the model for another key standardization process in vernacular tongues: codification in spite of the presence of dialectal variation. Early in the sixteenth century, the French humanist Geoffroy Tory (ca. 1480–before late 1533) commented as follows on the regulation and grammatical codification of French, which he regarded more as a set of varieties rather than a unitary language with a single norm:

\begin{quote}
Our language is as easy to regulate and put in good order as the Greek language once was, in which there are five speech varieties, which are the Attic, Doric, Aeolic, Ionic, and common language. These have certain mutual differences in their noun declensions, verb conjugations, orthography, accents, and pronunciation.\footnote{Tory (1529: \textsc{iv}\textsc{\textsuperscript{v}}\textsc{–v}\textsc{\textsuperscript{r}}): “Nostre langue est aussi facile a reigler et mettre en bon ordre, que fut jadis la langue grecque, en la quelle ya cinq diversites de langage, qui sont la langue attique, la dorique, la aeolique, la ionique et la commune, qui ont certaines differences entre elles en declinaisons de noms, en conjugations de verbes, en orthographe, en accentz et en pronunciation”. See Trudeau (1983: 466–467) for Tory’s “pandialectal” conception of French. Cf. Defaux (2003: 19–20), where the passage is contextualized within the French grammatical tradition; \citet[23]{Cordier2006}, who frames it in Tory’s general reception of antiquity.}
\end{quote}

Tory proceeded by mentioning a number of French speech forms: the court variety, Parisian (which he seems to have associated closely with the court variety), Picard, Lyonnais, Limousin, and Provençal. Inspired by the Greek model, he did not view dialectal variation as a negative property hindering the regulation of the vernacular. Other scholars were not as optimistic about the codification of dialect-ridden tongues. The Hellenist Erasmus \citet[239]{Schmidt1615} emphasized the impossibility of reducing the dialects of both Greek and his native German to a norm. It goes without saying that not only Greek was used as a model for the selection and codification of a norm. Latin or other vernacular contexts were a major source of inspiration as well. The renowned grammarian Johann Christoph Gottsched (1700–1766), for instance, was inspired by the example of the Latin tongue in declaring it necessary to ban dialectal features from the German norm \citep[334]{Gottsched1748}.

The ancient Greek dialect context also served as a descriptive model, taken here in a very broad sense and therefore encompassing a range of approaches. To start with, the Greek prototype was projected onto the linguistic situation on the Iberian peninsula by the Spanish humanist Bartolomé Jiménez Patón (1569–1640). More particularly, Jiménez Patón relied on the traditional classification of Greek into five dialects to map out variation in his native land:

\begin{quote}
And thus we say that among the Greeks there are five manners of tongue with different dialects, which are the Attic, Ionic, Doric, Aeolic, and common tongue. And in Spain there are five others, which are the Valencian, Asturian, Galician, and Portuguese. All of these derive from this fifth, or principal and first Original Spanish of ours, different from the Cantabrian.\footnote{Jiménez Patón (1604: 10\textsc{\textsuperscript{r}}\textsc{–10}\textsc{\textsuperscript{v}}):“Y asi entre los Griegos decimos aver cinco maneras de lengua con differentes dialectos que son la lengua attica, ionica, dorica, aeolica y comun. Y en España ay otros cinco, que son la valenciana, asturiana, gallega, portuguesa. Las quales todas se an derivado de esta nuestra, quinta o principal y primera, originaria española differente de la cantabria”.}
\end{quote}

Jiménez Patón’s circumscription of the historical position of “Original Spanish” vis-à-vis the four other dialects may suggest that he envisioned the relationship of the koine to the Greek dialects in much the same terms. If so, the projection did not happen solely from Greek to Spanish, but partly also vice versa. In other cases, the Greek dialects were unmistakably forced into a vernacular straitjacket, reversing the directionality of the comparison. For example, Friedrich Gedike’s analysis and classification of the Greek dialects were modeled on his tripartite conception of the German dialects (see Chapter 7, \sectref{sec:key:3}).

The Greek dialects were also eagerly used as a descriptive point of reference by scholars wanting to sketch the degree of kinship among certain vernacular varieties, even among varieties that today are usually considered to be distinct but related languages. The preacher from Dordrecht Abraham Mylius (1563–1637) compared in his \textit{Belgian} \textit{language} the superficial variation among some of the languages now known as Germanic to differences between Aeolic and Ionic, stressing that, in both cases, the root and character of speech had remained the same (\citealt{Mylius1612}: 90; cf. e.g. also \citealt{Boxhorn1647}: 75–76). This also occurred on a lower level, as in Sven Hof’s (1703–1786) pioneering monograph on the dialect of Västergötland, a province in the west of modern-day Sweden. In this work, Hof (1772: esp. 12–13, 23) relied on his familiarity with the Greek context in seeing dialects as classifiable entities and in describing individual dialect features. For some scholars, using the Greek dialects as a model context had glottonymic consequences. The Italian humanist Claudio Tolomei (ca. 1492–1556), writing around 1525, contended that in much the same way as it was justified to group the Greek dialects together and designate them with one and the same label, the varieties of Italian should be seen as one linguistic class and should called by one and the same name (\citealt{Tolomei1555}: 14; see \citealt{Trovato1984}: 216).

Individual Greek dialects were frequently proposed as a point of comparison for clarifying the status and position of a vernacular dialect in its broader linguistic landscape. Attic was said to be similar to Misnian – the German of Meissen – often presented as the standard variety of German (see e.g. \citealt{Börner1705}: \textsc{b.4}\textsc{\textsuperscript{v}}; \citealt{Simonis1752}: 214–215). Henri Estienne perceived parallel features in individual French and Greek dialects. For instance, Estienne (1582: 3–4) compared the broadness of Franco-Provençal speech – \textit{sermo} \textit{Romantius} he termed it in Latin – to that of Doric Greek, pointing out that both varieties were characterized by the prominence of the vowel [a]; examples he cited were Franco-Provençal \textit{cla} and Doric \textit{kláks} ($\kappa \lambda \text{\textgreek{'a}}\xi $), both words meaning ‘key’. In a similar vein, the Enlightenment scholar Ferdinando Galiani (1728–1787), in his monograph on his native Neapolitan dialect, stressed its archaism and contended that it had phonetic properties – open vowels, a great expressivity of words, and strong consonants – similar to Doric, the Greek dialect spoken by the ancient inhabitants of Naples and surroundings. In sum, Galiani claimed, “Neapolitan could well be called the Doric dialect of the Italian tongue”.\footnote{\citet[16]{Galiani1779}: “il napoletano potrebbe ben dirsi il dorico della favella italiana”.} His glottonymic suggestion did not, however, enjoy any success.

Things are very different with an early modern comparison of a Greek with an English dialect. As a matter of fact, a development with consequences that resonate today began around the mid-seventeenth century, when the church historian Thomas Fuller (1608–1661) linked Scots with Doric Greek. According to Fuller, “the speech of the modern Southern-\textit{Scot} [was] onely a \textit{Dorick} dialect of, no distinct language from \textit{English}” \citep[81]{Fuller1655}. Forty years later, Patrick \citet[20]{Hume1695}, a commentator of John Milton’s \textit{Paradise} \textit{Lost}, remarked on Milton’s use of the verb \textit{to} \textit{rouse} that it signified ‘to get up’, being “a more northern pronunciation of rise, like the Dorick dialect”. Around the same time, the writer John Dryden (1631–1700) characterized the English poet Edmund Spenser’s (1552/1553–1599) language as follows:

\begin{quote}
But Spencer, being master of our Northern dialect and skill’d in Chaucer’s English, has so exactly imitated the Doric of Theocritus, that his love is a perfect image of that passion which God infus’d into both sexes, before it was corrupted with the knowledge of arts and the ceremonies of what we call good manners. (Dryden in \citealt{Vergil1697}: \textsc{a.2}\textsc{\textsuperscript{r}})
\end{quote}

Why was there such a close association between Doric and Scots? This parallel seems to have been informed not only by certain shared linguistic features, such as the frequency of [a] and a presumed broad pronunciation, but also – and probably primarily – by the alleged rustic nature and status of both dialects as well as their being used in bucolic poetry. This practice continued into the modern period (\citealt{Colvin1999}: v). A vestige of this early modern tradition is reflected in current glottonymic practice; the variety of Scots spoken in the Aberdeen area, now known as Mid-Northern or North-East Scots among linguists, is still labeled \textit{Doric} in popular usage to this day.\footnote{See McColl \citet[116]{Millar2007}: “In the course of the twentieth century, the North-East variety became known as The Doric, a term previously applied to all Scots varieties”.} The history of the association of Scots with Doric, which I have shown to go back at least to the seventeenth century, deserves a closer investigation, but this lies outside the scope of this book.

Yet another important manner in which Greek diversity was used as a descriptive point of reference was the extrapolation of letter permutations closely and prototypically associated with Greek to the diversity among the tongues of Western Europe. Greek letter changes were already around the turn of the sixteenth century a source of inspiration to describe similar variations in Italo-Romance.\footnote{See e.g. Manutius (1496: *.ii\textsc{\textsuperscript{v}}) and Da Ponte (1509: 97\textsc{\textsuperscript{r}}). See also Chapter 6, \sectref{sec:key:2.}} Especially in West Germanic-speaking Europe, this was a prominent phenomenon; there, the sigma–tau alternation present in, for instance, koine \textit{glôssa} ($\gamma \lambda \text{\textgreek{~w}}\sigma \sigma \alpha $) and Attic \textit{glôtta} ($\gamma \lambda \text{\textgreek{~w}}\tau \tau \alpha $), meaning ‘tongue’, was very often understood as somehow cognate to the <s>–<t> alternation among varieties of West Germanic, as in High German \textit{Wasser} vs. Dutch \textit{Water}.\footnote{See e.g. \citet[21]{Mylius1612}. Cf. also Althamer (1536: \textsc{m}.ii\textsc{\textsuperscript{r}}); Chytraeus (1582: \textsc{a.3}\textsc{\textsuperscript{r}}); Reitz (1730: 119–132); Ruhig (1745: 61–62); Hof (1772: 23–24).}

A final and somehow peculiar use of Greek diversity as a model can be found in the work of the Enlightenment pedagogue Friedrich \citet[7]{Gedike1782}, who assumed that the Greek context could assist in predicting dialectal evolution in other languages. Gedike’s knowledge of the history of Greek colonization and its impact on dialect formation led him to prophesize the emergence of a new English dialect in the United States, which at his time of writing in 1782 had just recently declared independence from Great \citet{Britain1776}, even though this was officially recognized by Great Britain only in \citealt{September1783} through the Treaty of Paris. Gedike was, however, probably not very familiar with the linguistic situation in the US; otherwise he would have realized that his prediction was, in fact, already becoming a reality at his time of writing.

In summary, Greek variation was eagerly used as a model by early modern scholars engaged in the elevation, standardization, and description of the vernacular tongues of Western Europe, usually their native ones. This happened in various ways, which can be placed under three main, not always easily distinguishable headings; the Greek linguistic context with its characteristic dialectal diversity was employed as \REF{ex:key:1} a literary \textit{exemplum}, \REF{ex:key:2} a model for standardization, and \REF{ex:key:3} a descriptive point of reference, this in very broad terms. The fascination with the Greek model was sometimes so intense that one could speak of a true Hellenomania, as with the printer-philologist Henri Estienne. An intimate acquaintance with the Greek language and its dialects was not always an indispensable prerequisite, even though it usually stimulated the exemplary use of the Greek language strongly, as again in Estienne’s case.

\subsubsection{Dissociation: The particularity of the Greek dialects foregrounded}
\hypertarget{Toc19704861}{}
At first, humanist scholars seem to have largely agreed upon the comparability of Greek and vernacular dialectal variation, which for them seems to have been a kind of uncontested assumption. Gradually, however, different voices were heard, especially from the end of the sixteenth century onward, when the selection of the linguistic norm was more or less settled for many Western European vernaculars, even though this process was completed at different moments for each language.\footnote{See e.g. Mattheier (2003: 217–222), who points out that Luther’s German and so-called general German (an East Upper German koine) competed for most of the early modern period, even though the former eventually gained the upper hand.} Two early scholars with a particularly outspoken opinion on the issue were Benedetto Varchi (1503–1565) and Vincenzo Borghini (1515–1580), both Italian humanists involved with the \textit{questione} \textit{della} \textit{lingua}.

Benedetto \citet[95]{Varchi1570} regarded the Greek dialects as “equal” (\textit{eguali}) – they were of the same noblesse and dignity – whereas there was inequality among Italian varieties, since Florentine speech was elevated above the rest. This seems to be reflected in Varchi’s usage of the term \textit{dialetto}, which he restricted to varieties of the Greek language. He nevertheless reserved a particular place for Attic, which he claimed to be similar to Italian, by which he meant Tuscan \citep[141]{Varchi1570}. Siding with Pietro Bembo (1470–1547) against Baldassare Castiglione (1478–1529) and Gian Giorgio Trissino (1478–1550), Varchi was fiercely opposed to the use of the Greek koine as a model for a common Italian language.\footnote{Varchi (1570: 269–271), with reference to \citet{Bembo1525}, \citet{Castiglione1528}, and \citet{Trissino1529}.} Varchi argued that there were only four Greek dialects, out of which the Greeks easily created a common tongue, but the varieties in Rome were innumerable, making it impossible to produce an Italian koine out of them.

Like Varchi, the Italian monk and exceptional Hellenist Vincenzo Borghini was convinced that Greek and Italo-Romance variation were incomparable, a train of thought he developed in a manuscript treatise entirely devoted to this problem – it bears the title \textit{Whether} \textit{the} \textit{diversity} \textit{of} \textit{the} \textit{Greek} \textit{language} \textit{is} \textit{the} \textit{same} \textit{as} \textit{the} \textit{Italian} and was likely composed in the first half of the 1570s (edition in \citealt{Borghini1971}; see \citealt{Alinei1984} [1981]: 171, 191). \citet[335]{Borghini1971} argued instead that if the Greek context really needed to be compared with variation on the Italian peninsula, it should be with variation in the Tuscan subgroup rather than with Italian as a whole. After all, Italo-Romance tongues differed from each other to a far greater extent than the Greek dialects did. The Tuscan–Greek comparison was all the more preferable, Borghini continued, since the varieties of both linguistic groups were approved speech forms, in contrast to other Italian varieties such as Lombard. Borghini (1971: 338–340) dismissed the comparison of Italian and Greek also for historical reasons. Speakers of Italian did not have a common tongue because, unlike the ancient Greeks, there was originally no unitary Italian people speaking a common language. In fact, Italian emerged out of the mixture and corruption of the tongues of several different peoples. This was why constructing a common Italian language was a bad idea. What is more, much like Varchi, Borghini contrasted the approved and written Greek dialects, which only showed slight mutual differences, with the innumerous Italo-Romance varieties, which could not be reduced to writing and which exhibited substantial divergences.\footnote{\citet[341]{Borghini1971}. See Alinei (1984 [1981]: 171); \citet[210]{Trovato1984}; Benincà (1988: 32–37). Cf. Salviati (1588: 253–254) for an argument similar to Borghini’s.} During the sixteenth century, voices similar to Varchi’s and Borghini’s were heard outside of Italy as well.\footnote{See e.g. Wolf (1578: 595–596), on whom see Von \citet{Raumer1856}, Jellinek (1898; 1913–1914: \textsc{i}.58–59), and Mattheier (2003: esp. 214–218). Cf. also Palsgrave (1530: xiii.\textsc{\textsuperscript{v}}).} This continued throughout the seventeenth century and reached its summit in the eighteenth century, especially in France, to which I turn now.\footnote{For seventeenth-century examples, see Mambrun (1661: 458–459) and Morhof (1685: 146–147).}

The stress on incomparability was particularly prominent in the widely read works of the French historian and classical scholar Charles Rollin, who distinguished between the dialects of the Greek language, termed \textit{idiomes} and \textit{dialectes}, and the patois of the different provinces of France, called \textit{jargons}. Rollin characterized these latter as vulgar and corrupted manners of speaking not deserving the label of \textit{language} (\textit{langage}). A dialect, in contrast, was “a language perfect in its own right”, apt for literary use, having its own rules and elegant features.\footnote{\citet[117]{Rollin1726}: “Chaque dialecte étoit un langage parfait dans son genre”. See also \citet[395]{Rollin1731}.} In a later work, \citet[395]{Rollin1731} connected this to the political fragmentation of Greece as opposed to the high degree of centralization in France (cf. Chapter 7, \sectref{sec:key:5}). The comparability was subsequently denied in Greek grammars composed by French scholars, as in the 1752 edition of a lengthy \textit{Introduction} \textit{to} \textit{the} \textit{Greek} \textit{language} by the French Jesuit Bonaventure Giraudeau (1697–1774). This grammar, composed in Latin, was first published in Rome thirteen years earlier, but that edition lacked a reference to the French dialects, as it would not have been useful to its Italian audience. Only when it was published in French-speaking territory – the edition of 1752 appeared in La Rochelle and was sold in Paris – did a comment about French linguistic diversity become relevant \citep[117]{Giraudeau1752}.

The criticism of the comparability of French and ancient Greek regional diversity reached an apogee in the “Langue” article included in the ninth volume of Diderot and d’Alembert’s \textit{Encyclopédie}, published in 1765. The author of the entry was the French grammarian Nicolas Beauzée (1717–1789). In his lengthy article, \citet[249]{Beauzée1765} elaborated on two types of regional language variation, correlating with political differences. He contrasted Latin and French diversity with variation in ancient Greece, Italy, and Germany. Greeks, Italians, and Germans were made up of “several equal and mutually independent peoples” (“plusieurs peuples égaux et indépendans les uns des autres”), which was why their dialects were “equally legitimate” (“également légitimes”) forms of their respective national language. The situation was different for Latin, which was the language of a politically unified empire. It therefore had only “one legitimate usage” (“un usage légitime”), while everything deviating from it did not deserve the label “dialect of the national language” (“dialecte de la langue nationale”). Instead, it should be circumscribed as “a patois abandoned to the populace of the provinces” (“un patois abandonné à la populace des provinces”).\footnote{Cf. Priestley (1762: 135–139), who expressed a view similar to Beauzée’s in the English context.} The same held true for his contemporary French context, claimed Beauzée. Yet not every contributor to the \textit{Encyclopédie} seems to have been convinced of the differences between French and Greek diversity. The anonymous author of the “Patois” entry asked himself: “What are the different dialects of the Greek language other than the patois of the different areas of Greece?”\footnote{Anon. (1765: 174): “Qu’est-ce que les différens dialectes de la langue greque, sinon les patois des différentes contrées de la Grece?”}

The emphasis on the incomparability of vernacular and Greek variation also occurred outside of France, especially in German-speaking territories.\footnote{See e.g. Nibbe (1725: b.2\textsc{\textsuperscript{v}}\textsc{–}b.3\textsc{\textsuperscript{r}}), who stressed differences in literary usage; [Frisch] (1730: 1131–1132), who opposed the literary Greek dialects to the German dialects of the lower social classes (\textit{Pöbel-Sprach}); [Frederick the Great] (1780: 6–8); Ries (1786 [1782]: 203–204). For an example from England, see Bayly (1756: 13–14).} Of particular interest is the work of the eighteenth-century German classical scholar Johann Matthias Gesner, who provided an insightful account of the comparability of German and ancient Greek diversity. In the past, Gesner argued, they were comparable. The absence of a centralized government and capital caused dialectal variation in both areas.\footnote{Gesner (1774: 160–161). Cf. Court de Gébelin (1778: lxviii), who limited the comparability to the period before France had a centralized government.} Moreover, Greek as well as German dialects were initially used in writing. Starting with the Lutheran era, the German dialects lost their prominence and social prestige, leading them to be ridiculed and to attain a status different from the ancient Greek dialects. \citet[162]{Gesner1774} likewise considered it unacceptable to compose dialectally mixed poetry in German, arguing at the same time that this was equally inappropriate for Greek authors writing in or after late antiquity.

In conclusion, scholars frequently stressed the incomparability of Greek and vernacular dialects, especially toward the end of the early modern period, when most vernacular dialects had slipped into the shadows of their overarching standard varieties and the comparison must have appeared less convincing. In assessing this lack of comparability, authors were generally inspired by language-external circumstances, usually geopolitical and sociocultural. On some occasions, however, incomparability was maintained on a more strictly linguistic basis, for instance, when attempting to map out different degrees of linguistic kinship. This is what happened when certain eighteenth-century Scottish scholars compared the Greek dialects with the relationship among a number of tongues known today as Celtic. The early eighteenth-century Scottish antiquarian David Malcolm stressed the incomparability of both contexts, leading him to propose a different terminology for each situation:

\begin{quote}
Many indeed say that the \textit{Welsh} and \textit{Irish} are but different dialects of the same language, but those who have enquired into them will easily see that they differ more widely than the dialects of the \textit{Greeks}. Perhaps it may not be amiss to call them sister languages. (\citealt{Malcolm1738}: 46–47; cf. \citealt{MacNicol1779}: 311)
\end{quote}

The Greek dialects were not always directly involved when scholars emphasized the incomparability of two dialect contexts. Comparisons of different Western European vernaculars sometimes served to devalue the dialects of one language in favor of the dialects of another. Henri Estienne (1579: 133–134), for example, praised the richness and utility of French dialectal diversity, both properties he denied to Italian (see \citealt{Swiggers1997}: 306; 2009: 73). Also, when comparing two or more vernacular dialect contexts, scholars noticed different degrees of mutual intelligibility and variation.\footnote{For mutual intelligibility, see e.g. Hosius (1560: 158\textsc{\textsuperscript{r}}\textsc{–158}\textsc{\textsuperscript{v}}); Högström (1748: 77 – I refer to the German translation of the Swedish original, published in 1746/1747). For different degrees of variation, see e.g. Sajnovics (1770: 27, 57).}

\subsubsection{Synthesis}
\hypertarget{Toc19704862}{}
Vernacular diversity was very often compared to the ancient Greek dialects during the early modern period. This happened for various purposes, most importantly, \REF{ex:key:1} to explain the nature of Greek dialectal diversity, mainly to would-be Hellenists or to an intended readership unacquainted with the Greek language, \REF{ex:key:2} to justify and describe (certain uses of) dialectal variation in the Western European vernaculars, and \REF{ex:key:3} to emphasize differences between Greek and vernacular variation, especially in literary and sociopolitical terms. I have visualized the directionality of the comparisons in \figref{fig:key:6} below.

\begin{stylecaption}
Figure \stepcounter{Figure}{\theFigure}: Directionality of comparison of ancient Greek with vernacular dialects
\end{stylecaption}

\tablefirsthead{}

\tabletail{}
\tablelasttail{}
\begin{tabularx}{\textwidth}{XX}
\lsptoprule

\ea%1
    \label{ex:key:1}
    \gll\\
        \\
    \glt
    \z

         & %%[Warning: Draw object ignored]
ancient Greek                vernacular\\
\ea%2
    \label{ex:key:2}
    \gll\\
        \\
    \glt
    \z

         & %%[Warning: Draw object ignored]
%%[Warning: Draw object ignored]
ancient Greek                vernacular\\
\ea%3
    \label{ex:key:3}
    \gll\\
        \\
    \glt
    \z

         & %%[Warning: Draw object ignored]
ancient Greek                vernacular\\
\lspbottomrule
\end{tabularx}
In the cases of \REF{ex:key:1} and \REF{ex:key:2}, the figure suggests a strictly unidirectional movement. However, as I have argued, especially in \sectref{sec:key:1.1.} above, this is too simple a picture. Scholars often suppressed, usually silently, the differences between both dialect contexts in order to underline the similarities, and they sometimes even forced one situation into the straitjacket of the other. This could happen either consciously or subconsciously. It is, however, difficult to tell the degree of consciousness from the actual evidence, as the suppressing of the differences was nearly always left unmentioned. The reason for this is obvious; mentioning differences would invalidate the scholar’s claim of comparability.

The enumeration above may be taken to carry chronological implications as well. At first, the tendency to explain the phenomenon of ancient Greek dialectal diversity prevailed, soon after which the directionality was reversed with the Greek linguistic context functioning as a model for justifying and describing vernacular variation. The third element, dissociation, came about as a reaction against this latter use of the Greek dialects in the second half of the sixteenth century and culminated in the eighteenth century. This occurred especially in France, where the devalued patois were emphatically opposed to the literary Greek dialects. Even though it is possible to distinguish certain tendencies throughout the early modern period, one must be aware that, once the three main approaches toward Greek vis-à-vis vernacular diversity were established, they often coexisted. Even more, one scholar could reflect and reunite different approaches in his work, even as seemingly contradictory attitudes as \REF{ex:key:2} and \REF{ex:key:3}. For example, in Henri Estienne’s work, the model function of Greek took center stage, as I have established above in \sectref{sec:key:1.2.} Elsewhere in his work, however, Estienne (1587: 93–94) granted that the literary use of dialects was much more restricted in French than it had been in ancient Greek, thus displaying an awareness of differences between both dialect contexts. He noticed that Homer was allowed to mix different dialects in his epic poems, but in French this primarily happened in comic pieces and was uncommon in more serious writings, with the exception of certain dialect words in the poetry of Pierre de Ronsard and Joachim du Bellay.

What vernacular varieties were compared most intensively to the ancient Greek dialects? It should come as no surprise that Italian humanists were the first to compare ancient Greek diversity with their vernacular context, as they were at the cradle of Renaissance Greek studies.\footnote{On the comparison of the Greek and Italian contexts, see also Dionisotti (1968: 2–3, 51), Alinei (1984 [1981]: 179), \citet[215]{Trovato1984}, and Lepschy (2002: 36–37).} Indeed, Italian diversity was frequently compared to the Greek dialects, primarily in the sixteenth century. After the selection of the normative variety was more or less settled, comparisons of Greek and Italian variation became less frequent. It seems to have occurred only occasionally in the seventeenth and eighteenth century, mainly to stress the similarities both contexts displayed (e.g. Salvini in \citealt{MuratoriSalvini1724}: 99–100). Almost immediately after the revival of Greek studies reached the other side of the Alps, intuitive comparisons of the Greek and German dialect contexts started to appear. Soon, they occurred in the work of Frenchmen too, in which it seems to have been related to the patriotic claim that French derived from Greek and not from Latin. Paradoxically, it turned out to be French scholars who stressed most strongly the incomparability of Greek and French variation in the eighteenth century. This was no doubt related to the purist and prescriptivist attitudes current in French linguistic thought at the time as well as to a reverence for the literary dialects of Greek.\footnote{On French purism in the eighteenth century, with specific reference to the \textit{Académie} \textit{française}, see François \REF{ex:key:1905}.} In England, comparisons were frequent too, albeit less so than in Italy, Germany, and France, and the comparability of Greek and English variation was usually taken for granted. It was somewhat less customary to compare the ancient Greek dialects with variation in Dutch, Spanish, and North Germanic, and much less so with varieties of Baltic, Basque, Celtic, Portuguese, and Slavic. This is not really astonishing; intense comparisons of Greek with vernacular variation were principally conducted by scholars active in areas and cities that were centers of Greek studies, including most importantly Italy, Germany, and France. Comparative approaches toward ancient and vernacular Greek variation were exceptional, most likely because Western European scholars did not feel the need to justify or describe the dialectal variation of a foreign language they considered barbarous and because they approached the matter largely in terms of discontinuity rather than incomparability (see Chapter 2, \sectref{sec:key:10}; Chapter 5, \sectref{sec:key:5}). A notable exception was the Italian Jesuit missionary Girolamo Germano (1568–1632), who tried to justify his focus on the dialect of Chios in his vernacular Greek grammar by referring to the central status of Attic among the ancient dialects.\footnote{\citet[10]{Germano1622}. Cf. Du Cange (1688: vi–vii), who reminded his readers of ancient Greek dialectal diversity in order to explain vernacular Greek variation.}

Early modern scholars positioned the ancient Greek dialects in various ways vis-à-vis those of the Western European vernaculars. Yet how did they relate the Greek dialects to other languages they eagerly studied, primarily Latin and the so-called Oriental tongues, including Hebrew and Arabic?

\subsection{Latin: Uniquely uniform or diversified like Greek?}
\hypertarget{Toc19704863}{}
In the early stages of the Renaissance, there was a common belief that, in contrast to ancient Greek, Latin was uniform and therefore exempt from dialectal variation. This view was most famously championed by the Italian humanist Lorenzo Valla. For Valla, as I have shown, the unifying power of Latin was a great advantage in sharp contrast to the internal linguistic discord among the Greeks. Later humanist scholars such as Aldus Manutius and Juan Luis Vives also adhered to the idea of Latin uniformity, which lived on throughout the early modern period.\footnote{See Manutius (1496: *.ii\textsc{\textsuperscript{v}}); Vives (1533: \textsc{x}.iii\textsc{\textsuperscript{v}}): “Romana dialectos non habet, unica est et simplex”. See \citet[11]{Trapp1990}. Cf. Erasmus (1528: 34–35); \citet[121]{Rapin1659}; \citet[29]{Wesley1736}; Primatt (1764: 113–114).} Unlike Valla, however, Manutius regarded it as a cause of poetical poverty. Vives, on the other hand, denied the existence of diversity in classical Latin, but at the same time felt compelled to grant that Latin had clearly changed over time – he was no doubt thinking of the traditional four-stage periodization offered by the Early Christian author Isidore of Seville (ca. 560–636).\footnote{See Denecker (2017: 229–232) on Isidore’s division of the history of Latin into ancient, Latin, Roman, and mixed.} Valla, Manutius, and Vives all opposed Greek diversity directly to Latin uniformity. The illusion of Latin internal harmony seems to have obstructed an early recognition of the universality of dialectal variation and perhaps also a more avid interest in language-internal diversity in general. Regional variation in Latin was nevertheless gradually recognized in the sixteenth century.\footnote{For a modern linguistic study of regional variation within Latin, see the detailed account of \citet{Adams2007}.} A telling early example is the Flemish nobleman Georgius Haloinus’ (ca. 1470–1536/1537) \textit{Restauration} \textit{of} \textit{the} \textit{Latin} \textit{language}, a strong plea for usage and against grammar in learning correct Latin; this work was first published in 1533, but Haloinus had already composed it several decades earlier, around 1508. Haloinus (1978 [1533]: 55) stressed that Latin, too, was internally diversified and pointed to the alleged Paduan touch to Livy’s speech, his so-called “Patavinity” (\textit{Patauinitas}), to prove this. Livy’s Patavinity became a prototype and leitmotiv in demonstrating the existence of Latin dialects.\footnote{See e.g. also Castiglione (1528: b.viii\textsc{\textsuperscript{v}}); Estienne (1582: *.iii\textsc{\textsuperscript{r}}); Schottel (1663: 174, 176); \citet[311]{Rice1765}; Mazzarella-Farao (1779: \textsc{lix}); Ries (1786 [1782]: 203–204). See Van \citet{Rooy2018a} for a more extensive discussion of sixteenth and seventeenth-century ideas about Livy’s Patavinity.} Some scholars even posited the existence of several other Latin varieties by analogy with Patavinity. In an eighteenth-century dissertation presented in Copenhagen, reference was made to Vergil’s alleged Mantuan dialect, his “Mantuanity” (\textit{Mantuanitas,MuntheHeiberg1748}: 22). Scholars went further than simply varying on the Patavinity theme, however. The Dutch scholar and politician Ernst Brinck (1582/1583–1649) even made a list of Latin dialects in his manuscript catalogue of linguistic specimens. Brinck referred to “dialects” (\textit{dialecti}) specific to a certain social or gender group – peasants or women, for instance – as well as to “dialects” characteristic of a certain locality, including Praeneste and Tusculum, noting some particular words for each variety.\footnote{Brinck (1615–1635: 56\textsc{\textsuperscript{v}}). Cf. also \citet[43]{Stubbe1657}, where a list of Latin dialects is provided, albeit mixed up with Isidore of Seville’s four-stage periodization of Latin.}

Once it had been established that Latin also must have had its dialects, seventeenth-century scholars began to compare the Latin dialect context with its ancient Greek counterpart, always resulting in the a priori affirmation that they showed great differences. In his monograph on Livy’s Patavinity, Daniel Georg \citet[146]{Morhof1685} emphasized that the Greek language had greater dialectal variation and license than Latin because of the political diversity of ancient Greece, which he opposed to the highly centralized Roman Empire. This did not mean, however, that Latin did not have any dialects at all, and Morhof (1685: 148–149) indeed listed several dialects of the language. About a decade later, the Hebraist Louis Thomassin (1619–1695) stressed that Latin, in comparison to ancient Greek, “had few or no dialects”, with the exception of “a number of native and vernacular tongues of certain cities”. Thomassin attributed this to the Roman desire for unity and simplicity.\footnote{Thomassin (1697: xix): “Graeca rursus lingua dialectis etiam statim ab initio luxuriata est. Quattuor quidem ex iis eminent; sed plurium supersunt uestigia. Porro singulae dialecti de iure mutandi uetera nouaque superstruendi uocabula cum suis dicendi modis, inter se quasi certatim contenderunt. Latina uero suae tum unitatis tum simplicitatis tenacior, paucas aut nullas habuit dialectos, si aliquot excipias quarundam ciuitatum patrios uernaculosque sermones”.}

It was, however, only in the eighteenth century that Latin dialects were described in explicitly negative terms in comparison to the ancient Greek dialects. The German theologian (Johannes) Nicolaus Hertling (1666–1710) contrasted Greek dialects with Latin varieties in esthetic terms. Greek had various dialects pleasant to the ears, which Latin and most other languages lacked, as they only contained corrupt dialects \citep[73]{Hertling1708}. The English grammarian Joseph Priestley (1733–1804) provided a more neutral and down-to-earth account. \citet[138]{Priestley1762} stressed that, in Latin, “dialects are unknown”, since these were not introduced into writings. “The \textit{Patavinity} of Livy is not to be perceived”. Put differently, “the \textit{Romans}, having one seat of power and of arts, allowed of no dialects”.\footnote{\citet[280]{Priestley1762}. See \citet[52]{Amsler1993}. Cf. \citet[49]{Galiani1779}; Ries (1786 [1782]: 203–205) for similar views.} In sum, Priestley did not deny that Latin dialects existed, but pointed out that they were no longer knowable, since, unlike the Greek dialects, they had not received written codification.

The diversity of the Romance languages that developed out of Latin was sometimes compared to the Greek dialects. The sixteenth-century Hellenist and orientalist Angelo Canini even forced the Romance tongues into the straitjacket of Greek as well as Oriental diversity. This involved Canini (1554: \textsc{a}.iii\textsc{\textsuperscript{r}}) interpreting both Greek and Latin as linguistic tetrads, the former consisting of Attic, Ionic, Doric, and Aeolic, and the latter encompassing Latin, Italian, French, and Spanish (cf. also \citealt{Canini1555}: a.3\textsc{\textsuperscript{v}}). Oddly enough, he did not elaborate on the precise relationship of Latin to the three Romance tongues he mentioned. Together with the Hebrew tetrad, consisting of Hebrew, Syriac, Arabic, and Ethiopian, the Greek and Latin tetrads constituted a linguistic triad, Canini suggested. This makes it clear that Canini’s scheme, into which Latin and three Romance tongues descending from it were forced in an ahistorical way, was very much numerologically inspired and not based on much linguistic evidence.

In a nutshell, Latin was regarded as uniform by many scholars throughout the entire early modern period. However, an alternative view emerged in the early sixteenth century, attributing regional variation to Latin, a realization which paved the way for the insight that regional variation was a universal phenomenon. In the seventeenth century, some scholars even attempted to list Latin dialects despite the scarcity of the evidence available to them. At the same time, they started to intuitively compare Latin to Greek variation with a focus on language-external, sociopolitical differences. In the eighteenth century, the superiority of Greek over Latin dialects was explicitly stressed on account of the literary value of the former. In other words, the main aim of the comparison was dissociation (cf. \sectref{sec:key:1.3} above). Exceptionally, Greek dialectal variation was put forward as a descriptive model for Romance diversity (cf. \sectref{sec:key:1.2} above).

\subsection{The Oriental language family and the Greek dialects}
\hypertarget{Toc19704864}{}\subsubsection{The Oriental dialects}
\hypertarget{Toc19704865}{}
Early modern scholars compared the ancient Greek dialects very frequently to the Oriental tongues, up to the point that it seems to have become a refrain. Why was this the case? A large part of the answer can be found by looking at what the Swiss humanist Theodore Bibliander (1504/1509–1564) had to say about the interrelationship of a number of Oriental languages:

\begin{quote}
By means of a diligent investigation one knows that the Chaldean, Assyrian, Arabic, and Syriac tongues are so cognate that some take them to be one, which is true if the matter would be understood in terms of all dialects of the Greeks, which are called one Greek language.\footnote{\citet[58]{Bibliander1542}: “diligentique inquisitione cognitum est Chaldaeum, Assyrium, Arabicum, Syriacum sermonem ita finitimos, ut pro uno quidam accipiant, quod uerum est, si, ut omnes Graecorum dialecti una lingua Graeca dicuntur, ea res intelligatur”.}
\end{quote}

Bibliander (1542: 58–59) proceeded by elaborating on the close connection between these Oriental languages and the primeval Hebrew tongue. It is obvious that he employed the example of the Greek context to justify the idea that these Oriental tongues actually constituted one language (see also \citealt{Metcalf2013} [1980]: 61). Bibliander used Greek dialectal variation as a touchstone and a descriptive point of reference to analyze and approach Oriental diversity, a method omnipresent in early modern descriptions of this language family.\footnote{Semitic variation was often also explained by referring to one’s native or another more familiar linguistic context. See e.g. \citet[41]{Purchas1613}; \citet[197]{Kircher1679}; Le Clerc (1696: b.1\textsc{\textsuperscript{v}}); Chambers (1728: \textsc{i.}230 [4th sequence of pagination]); Kals (1752: 57–58).} Consider, for instance, how Bibliander’s pupil Conrad Gessner described Aramaic and its relationship to Hebrew:

\begin{quote}
Today, the more erudite men use the Chaldean language in Egypt and Ethiopia, as far as I hear. It is close to Hebrew and, perhaps, does not differ much more from it than Doric from the common Greek.\footnote{Gessner (1555: 15\textsc{\textsuperscript{r}}): “Chaldaica lingua hodie eruditiores in Aegypto et Aethiopia utuntur, ut audio. Hebraicae confinis est, nec forte multo amplius differt quam Dorica a Graeca communi”. See Peters (1970–1971: \textsc{i.}43). Cf. e.g. also \citet[325]{Rocca1591}, silently adopting Gessner’s phrase; \citet[459]{Saumaise1643a}; \citet[88]{Bagnati1732}; \citet[24]{Wesley1736}; \citet[22]{Eichhorn1780}.}
\end{quote}

The comparability of the Greek and Oriental contexts was especially prominent in the work of the Dutch orientalist Albert Schultens (1686–1750), who held that the four Oriental tongues Hebrew, Aramaic, Syriac, and Arabic derived from a now lost primeval tongue just like the four Greek dialects descended from a common Greek, “Pelasgian” mother language (\citealt{Schultens1739}: 234–235). Schultens (1748: \textsc{xcvi)} also believed that Attic and Hebrew were similar because of their tendency toward contractions, whereas Ionic and Arabic shared the property of being conservative varieties (see \citealt{Eskhult2015}: 85). Other scholars likewise paired a Greek dialect with an Oriental tongue. Like Schultens, some perceived similarities between Attic and Hebrew, whereas others connected Doric to Syriac because of their alleged broadness.\footnote{See \citet[425-432]{Lakemacher1730} for the Attic–Hebrew comparison. For Doric and Syriac broadness, see Chapter 5, \sectref{sec:key:7.}} Comparing Greek to Oriental variation is truly a topos throughout Schultens’ work, in which Greek diversity always served as a point of reference for understanding Oriental variation.\footnote{See e.g. Schultens (1769 [1706]: 490; 1732: 4; 1737: 5; 1738a: 19–21; 1738b: 106–107 [stressing that the Oriental and the Germanic contexts were less comparable]; 1739: 187; 1748: \textsc{xcvi}; in Eskhult fc. [ca. 1748–1750]: §\textsc{xxvii}). On this topos in Schultens’ work, see also \citet[105]{Fück1955}; \citet[707]{Covington1979}; Eskhult (fc.). Cf. in Schultens’ tracks Polier de \citet[5]{Bottens1739}; \citet{GroddeckTreuge1747}.} This procedure occurred in the work of other scholars as well, whether or not in combination with a reference to vernacular variation (see e.g. \citealt{Bochart1646}: 778; \citealt{Blount1680}: 71–72).

Some scholars even claimed that Greek dialects differed more from each other than the Oriental tongues, thus dissociating both linguistic contexts (cf. \sectref{sec:key:1.3} above). Angelo \citet[34]{Canini1554} already did so when discussing verb conjugations in his 1554 comparative grammar of a number of Oriental tongues (see \citealt{Contini1994}: 50; Kessler-\citealt{Mesguich2013}: 211). The idea was expressed more clearly still by the orientalist Christian Ravis (Raue/Ravius; 1613–1677).\footnote{Ravis (1646: *.2\textsc{\textsuperscript{r}}). See e.g. also Hunt (1739: 51–52); Groddeck \& Treuge (1747: \textsc{xxvi}).} \citet[48]{Ravis1650} also emphasized that even though there were separate chairs for each Oriental language at universities, but not for the Greek dialects, this institutional fact should not lead to the conclusion that Hebrew, Syriac, Arabic, and so on were truly “divers tongues”. In fact, just like the Greek language, they were “only one”. The practice of comparing Oriental to Greek diversity was criticized by Johann Heinrich Hottinger (1620–1667), who explicitly reacted against his colleague Christian Ravis’ views on the matter. Hottinger’s (1661: a.3\textsc{\textsuperscript{v}}–a.4\textsc{\textsuperscript{r}}) two main points were that Hebrew was not an Oriental dialect, but the primeval language, and that the differences among the Oriental tongues were much greater than those among the Greek dialects. The Dutch orientalist Sebald Rau (Sebaldus Ravius; ca. 1725–1818) adopted a similar perspective. Rau (1770: 20–21) argued that the Greek dialects were spoken by one nation, whereas the “Oriental dialects” (\textit{dialecti} \textit{Orientales}) were current among different nations, living in various climates and having diverging ways of living, customs, and rites. This resulted in greater linguistic differences, he argued.

In rare instances, the Oriental context served as a reference point to understand developments in the history of the Greek language (cf. \sectref{sec:key:1.1} above). A late seventeenth-century Hellenist active in Leipzig used the alleged decay and dialectal diversification of the Hebrew language during the Babylonian captivity in the sixth century \textsc{bc} to clarify the decline of the Greek language (\citealt{Eling1691}: 318–319). In a sixteenth-century handbook on the Greek literary dialects, the Oriental context was cited as an additional example, next to the grammarian’s native one, to explain differences in elegance among the ancient Greek varieties (\citealt{Walper1589}: 61–62).

\subsubsection{Hebrew dialects}
\hypertarget{Toc19704866}{}
As to variation within Hebrew, identified by many authors as the primeval language spoken by Adam and Eve and confused at the Tower of Babel, early modern opinions greatly differed.\footnote{In the present section I discuss views on variation within Hebrew, thus excluding cases in which Semitic tongues such as Arabic were dubbed \textit{dialects} of Hebrew (see e.g. \citealt{Bochart1646}: 56; \citealt{Martin1737}: 134–135).} Some scholars were eager to claim that Hebrew did not have any dialects. The Leipzig theologian Bartholomaeus Mayer (1598–1631) did so while citing Lorenzo Valla’s comparison of Latin and ancient Greek \citep[10]{Mayer1629}. Franciscus Junius the Elder Junius (1579: \textsc{b.3}\textsc{\textsuperscript{v}}) took a more moderate stance, as he contrasted the immense variability of Greek to the relatively uniform Hebrew tongue, claiming that the latter did not have as many dialects as ancient Greek (cf. Chapter 7). The English orientalist Thomas Greaves (1612–1676) also attributed dialects to both Hebrew and ancient Greek, while praising Arabic for lacking them in his \textit{Oration} \textit{on} \textit{the} \textit{utility} \textit{and} \textit{preeminence} \textit{of} \textit{the} \textit{Arabic} \textit{language}, held at Oxford in 1637 and published there in 1639.\footnote{See Greaves (1639: 19–20), who inspired \citet[60]{Leigh1656} and \citet[73]{Blount1680}.}

Scholars often found it sufficient to prove regional variation within Hebrew by simply referring to the shibboleth incident in the Old Testament at Judges 12.5–6 or to the supposed Galilean character of St Peter’s speech, alluded to in the New Testament at Matthew 26.73.\footnote{See e.g. \citet[6]{Bovelles1533}; Bachmann \& Ludovici (1625: \textsc{b.3}\textsc{\textsuperscript{v}}); \citet[102]{Weemes1632}; \citet[2]{Wyss1650}; \citet[18]{Walton1657}; \citet[180]{Webb1669}; Kiesling \& \citet[6]{Bause1712}; Salvini in Muratori \& \citet[103]{Salvini1724}; \citet[30]{Hauptmann1751}; Hof (1772: 13–14). For the relevant biblical passages, see also Van Rooy (2018b: 199–200).} Gradually, however, philologists focusing on the Bible started to recognize that St Peter was more likely to have spoken a variety of Aramaic or – in early modern terms – of (Chaldeo-)Syriac (e.g. \citealt{PfeifferMartini1663}), whereas others denied that the shibboleth incident was evidence of variation within Hebrew (e.g. \citealt{Mayer1629}: 10–11). Sometimes, they developed historically nuanced answers to the question of whether Hebrew was dialectally diversified. In a dissertation entirely devoted to the question of St Peter’s speech and presented in Wittenberg, a periodization of Hebrew was designed in order to show the development of the language. The authors of the dissertation argued, among other things, that Hebrew was originally a unitary language like Latin, but underwent dialectal diversification after the Babylonian captivity (\citealt{PfeifferMartini1663}: \textsc{a.4}\textsc{\textsuperscript{v}}). In these more focused investigations into the question of whether Hebrew had dialects, the Greek dialects occupied a marginal position at best.

\subsubsection{Summary}
\hypertarget{Toc19704867}{}
Briefly put, the Greek dialects were frequently used as a point of reference to map out the close genealogical relationship among the Oriental tongues, which aroused great philological interest in the early modern era. Scholars were struck by the close kinship between these languages and tried to find an adequate way to express it. Since most orientalists were also trained as Hellenists, many of them thought of the Greek dialects as a revealing parallel. These too were closely cognate, despite their many formal differences. What is more, the Greek dialects had received written codification, just like the Oriental tongues. These two similarities made ancient Greek diversity a helpful reference point for early modern orientalists. Some of them went a step further and claimed that the Oriental tongues were even more alike than the Greek dialects. Such an exaggerated conception was usually rejected by orientalists in the seventeenth and eighteenth century, who preferred to maintain the comparability of both contexts. This stance culminated in the work of the Dutch philologist Albert Schultens, who formulated the Greek–Oriental simile in nearly every one of his publications. Finally, as with Latin, scholars struggled to assert Hebrew uniformity, even though from the sixteenth century onward there were voices admitting that Hebrew, too, that sacred tongue often identified with the language of Adam, had its dialects just like Greek, Latin, and the vernaculars.

\subsection{Conclusion: Between exemplarity and particularity}
\hypertarget{Toc19704868}{}
In the present chapter, I have attempted to demonstrate that early modern scholars compared and contrasted the linguistic diversity of ancient Greek to dialectal or dialect-like variation in a wide range of other languages and language families. This occurred most frequently with reference to Oriental diversity and dialectal variation in Western European vernaculars, especially Italian, German, French, and English. Modern scholars have often emphasized the exemplarity of the Greek context to grasp or ennoble vernacular diversity. For example, Peter Burke (2004: 35–36) states that, for the early modern awareness of dialectal variation, “the model situation was that of Ancient Greece with its Ionic, Doric, Attic, and other varieties of speech”.\footnote{Cf. \citet[923]{Haugen1966}; \citet[216]{Giard1992}: “la question des dialectes portée au passif des vernaculaires est considérée autrement dès lors qu’on remarque la signification et l’usage positifs qu’ils avaient en grec”.} In selecting the variety to be adopted as the literary standard in the so-called language questions during the Renaissance, the Greek context indeed seems to have functioned as a paradigmatic touchstone and was taken as a noble and close parallel to vernacular dialectal diversity (\citealt{Alinei1984} [1981]; \citealt{Trovato1984}; \citealt{Trapp1990}). Moreover, the Greek example with its allegedly dialectally mixed koine suggested that vernacular dialects, too, could contribute to the literary standard language under construction. However, as I have endeavored to demonstrate in this chapter, this is only part of the picture, albeit a very important one. The situation was very different in early modern manuals for ancient Greek. As a matter of fact, there, the Greek context did not serve as a model at all. Instead, the grammarians needed to explain it by referring to the native vernacular context of their intended readership. In other words, ancient Greek diversity constituted a phenomenon that very often required elucidation. This was especially common in works published in German-speaking areas, where Greek studies flourished throughout the entire early modern period and vernacular dialectal diversity was not easily transcended by an established standard language. In order to maintain the comparability of Greek with other dialect contexts, early modern scholars could tone down some of the differences between them so as to emphasize their similarities. To this end, they projected certain characteristics of one context onto the other, a process of which they were not always fully aware (cf. \citealt{Alinei1980}: 20).

Even though most early modern scholars seem to have assumed that ancient Greek dialectal diversity was highly similar to variation within other languages, the point of comparison being the close kinship among the dialects, there were nonetheless also a considerable number of authors who emphasized the particular place of ancient Greek diversity, certainly during later stages of the early modern period and particularly in eighteenth-century France. In the large majority of cases, the incomparability of ancient Greek with another dialect context was mainly motivated by language-external circumstances. This included, most importantly, the political diversity of ancient Greece and the literary and codified status of the canonical Greek dialects. Several scholars contrasted this to cases of political centralization, as in France, or to the existence of a sole written standard, as in the case of German. Authors emphasizing comparability likewise concentrated on language-external circumstances, but less exclusively so. The relative lack of reference to specific linguistic features in this discussion may seem remarkable at first sight, but this should be seen in connection with the main goal of the early modern discourse on comparability; this consisted in making a statement – either explanatory, justificatory, descriptive, or dissociating – about the precise status of a specific dialect situation in its broader sociolinguistic and cultural context rather than about the actual linguistic forms of the dialects.

A scholar’s emphasis on comparability or lack thereof depended to a large extent on his discursive intentions as well as his underlying language ideology. For instance, when explaining ancient Greek diversity in a grammar, comparability was usually stressed, since the grammarian hoped to help his readers understand the status of the Greek dialects by referring to a similar and more familiar context. Early modern literary critics, however, tended to deny comparability, as they emphasized the literary insignificance of vernacular dialects, which stood in glaring contrast to the high esteem of the ancient Greek dialects. This lack of comparability made it to the mind of certain scholars impossible to apply the term \textit{dialect} to any linguistic context other than ancient Greek. Put differently, early modern scholars vacillated between exemplarity and particularity. On the one hand, the ancient Greek linguistic situation was used as a model to approach variation within other languages or language families or turned out to be the situation in need of clarification by means of a more familiar vernacular example. On the other hand, a scholar could stress, whenever it suited him, the extreme idiosyncrasy of the Greek dialects and the exceptional historical coincidence that they have been eagerly used as literary mediums.

The level of competence in ancient Greek was also of relevance for the discourse on comparability. It seems that the better a scholar’s competence was, the more detailed his comparison tended to be and the more likely it was that his ideas were picked up by later scholars, as in the cases of Henri Estienne, Charles Rollin, and Albert Schultens. Inspired by their thorough knowledge of Greek – Estienne even claimed it to be his second language before Latin – they put forward various ideas on the (in)comparability of Greek with vernacular or Oriental diversity, all with considerable influence.

As a final point, I want to add that not all comparisons of different dialect contexts involved the ancient Greek dialects, even though this Greek-free approach occurred with a noticeably lower frequency. The relative rarity of such instances demonstrates the tremendous importance of ancient Greek diversity in triggering early modern interest in dialectal variation as a general phenomenon affecting every language. It also clearly indicates that the widespread comparison of dialect contexts was largely an early modern development, catalyzed by the Renaissance revival of Greek studies, all the more since the procedure was so exceptional in the Middle Ages. In sum, the well-chosen words of the Austrian Germanist Max Hermann Jellinek (1868–1938), which pertained specifically to German grammarians, may well be generalized: early modern scholars “cannot speak of dialects and written language without calling in Attic, Ionic, Doric, and the koine”.\footnote{Jellinek (1913–1914: \textsc{i}.21): “Diese Männer können nicht von Dialekten und Schriftsprache reden, ohne das Attische, Jonische, Dorische, Aeolische und die $\kappa o\iota \nu \text{\textgreek{'h}}$ aufmarschieren zu lassen”.}

\section{Conclusion} %9 /
\hypertarget{Toc19704869}{}
The ancient Greek dialects, it should have become clear by now, were eagerly studied in early modern Western Europe, ever since they were put on the scholarly agenda by humanists in the second half of the fifteenth century. The main motivation to do so was philological, as without mastering the different literary varieties of ancient Greek it was impossible to get a firm grip of the ancient Greek world, its literature, culture, and history. This was also why scholars believed it necessary to devote entire handbooks to the Greek dialects. Knowledge of them was, in other words, largely auxiliary and instrumental, and they were never studied in and for themselves. After this initial stage of philological focus on the great pagan classics of Greek literature, the dialects were soon introduced into related fields of interest, most importantly biblical philology as well as historiography with a strong antiquarian touch. In particular, awareness of the Greek dialects urged them to investigate two additional matters, especially from the seventeenth century onward: the peculiar nature of biblical Greek, on the one hand, and the language of Greek inscriptions which were being uncovered in large numbers all over the Mediterranean, on the other. In a more stereotypical fashion, the Greek dialects were also introduced in historiographical and ethno-geographical accounts of ancient Greece. Outside of Greek studies, the dialects proved to be a welcome orientation point for grammarians and philologists interested either in describing and codifying other languages or in gaining insight into language history, change, and diversity. This was, in very general terms, the context in which early modern scholars, exclusively men, studied the Greek dialects. I have tried to demonstrate how in this setting knowledge about the Greek dialects evolved from the end of the fifteenth until the end of the eighteenth century and how it related to ancient and medieval ideas. In doing so, I have focused on two aspects: contents and valorization. On the one hand, I have charted actual early modern ideas on the Greek dialects, their origins and development, and their successes as much as their failures. I have, on the other hand, also laid out in which ways scholars put their knowledge of the Greek dialects to use in dealing with linguistic themes and problems outside the scope of Greek philology. These most often related to the codification and description of Oriental and especially vernacular tongues, for the standardization of which the greatly admired Greek language was often hailed as a welcome reference point.

Early modern scholars devoted considerable effort to untangling the mystery of the Greek dialects, a focus of interest gradually conceived as a separate subfield within philology, as I have shown in Chapter 3. Hellenists attempted to develop accurate classifications of the complex linguistic situation of ancient Greece. These were not only more diverse but also finer-grained than those of their ancient and medieval predecessors, on whose ideas they as a rule elaborated. A major innovation of early modern times was the division between principal and less principal dialects, a distinction based on the language-external criterion of literary usage. The principal dialects were used in writing, the minor ones were not. This bipartition was so widely known and popular that some vernacular grammarians with a Hellenist background transposed it to their native context, even if the criterion of literary usage was not so easily applicable to it. A major setback was the invention of a poetical Greek dialect by early modern Hellenists, who often included it in their classifications; this resulted from the semantic ambivalence of the term \textit{dialect(us)}, which could mean not only ‘regional form of a language’ but also more generically ‘manner of expression’. The concept of \textsc{poetic} \textsc{dialect} was, however, far from unsuccessful in the early modern landscape of linguistic thought, since vernacular grammarians adopted it to describe the particularities and liberties of language found in their native poetical traditions. A notable success of early modern scholarship was the clear separation of the koine from the other dialects. Many Hellenists recognized its particular position as a common language transcending regional diversity, an insight likely fostered by the emergence of similar common languages in their times, for which the koine was often cited as a model. The establishment of the peculiar nature of the koine did not, however, mean that its history and its emergence conditions were adequately understood. Apart from the koine, Hellenists experienced great difficulties also in understanding the place of Homeric as well as Biblical Greek within the Greek language and its history. These were usually interpreted as an artificial mix of the canonical Greek dialects. Some Hellenists in the seventeenth and eighteenth century had, however, flashes of remarkable insight, proposing solutions approximating those of modern philologists. These resulted from a better appreciation of the historical conditions under which both Homeric and Biblical Greek emerged. Homeric Greek was correctly identified as representing earlier stages of Greek, whereas scholars like Saumaise recognized that Biblical Greek was a vernacular variety of the koine. These solutions were part of a wider tendency to describe the evolution of the Greek language in detail, a tendency formalized in the new genre of linguistic histories of Greek, which emerged in the seventeenth century. Early modern scholars tried to fit the different dialects of Greek, their historical development, and their interrelationships into this historical puzzle. Heavy with the burden of tradition and Strabo’s authority, they perceived a close link between Aeolic and Doric by analogy with the strong bond Attic and Ionic shared, a misconception definitively corrected only in modern times. They were similarly misled in their claim that Latin was closely associated with Greek and especially two of its dialects: first Aeolic and later also Doric. This faulty idea was likewise only abandoned in modern scholarship, even though pioneering linguists like Rasmus Rask (1787–1832) still believed in this age-old link (\citealt{Rask2013} [1818]: 152–153).

When outlining the precise differences among the Greek dialects, Hellenists recognized in the wake of their ancient and medieval predecessors certain regularities in this variation, noticing at the same time that these were not so much stable linguistic laws as they were fickle letter changes. However, they did not blindly focus on the level of the letter and looked at differences in terms of accent, lexicon, syntax, and style too as well The source material from which they worked was initially, like that of their Greek predecessors, restricted to literary texts, usually the pagan classics and from the sixteenth century also the Greek Bible. When, however, Greek inscriptions became more widely known to Hellenists in the seventeenth and eighteenth century, dialect specialists also broadened their perspective and introduced this new type of evidence into their manuals. This was a foundational step toward the development of a modern dialectology of ancient Greek in the early nineteenth century, usually associated – too narrowly – with Heinrich Ludolf Ahrens, even though the contributions of other pioneers like Albert Giese (1803–1834) also deserve further study \citep{Giese1837}. These nineteenth-century dialectologists, moreover, recognized that they were indebted to the efforts of their early modern predecessors, in particular those of Michael Maittaire.

The dialects were, finally, considered to offer a window on the ancient Greek world. They were not isolated linguistic media, but embedded forms of speech that allowed early modern Hellenists to construe a more lively image of this distant society. The close connection between the dialects, on the one hand, and Greek texts and tribes, on the other, tempted them to continue the long-standing tradition of projecting properties of the latter onto the former. Doric, for example, was a rustic and harsh dialect, because it was the dialect of bucolic poetry and the rough Dorians. The existence of dialects in Greek was moreover believed to betray the versatility and maliciousness of the Greek people as a whole and to reflect also the diversified geographical and political landscape of ancient Greece.

Early modern Hellenists were greatly indebted to the work and ideas of their predecessors from ancient and Byzantine Greece and, to a much lesser extent, the ancient Roman world. Symptomatic of this fact is that in each chapter I have had to outline earlier conceptions before tackling early modern views. Scholars in the Renaissance and later were not mere parrots, however, and produced many original contributions of their own. They did not blindly adopt ancient Greek and Byzantine ideas when they recognized that they were inadequate, but rather tried to formulate more consistent solutions, grounded not only in the authority of ancient authors but also in an assessment of the Greek language itself, its different varieties, and their history as well as other pieces of evidence. In other words, they systematized the ideas of ancient and medieval grammarians, while at the same time surpassing them. They saw a wider application for the knowledge of the dialects, which their predecessors had limited more strictly to grammar and philology and which they extended, first and foremost, to Bible studies and antiquarian investigations. They designed, in addition, more transparent methods of presenting Greek dialectal features. In contrast to their predecessors, who principally discussed Greek diversity per dialect, many early modern Hellenists arranged their treatment of the matter per grammatical category, often making use of elucidating tables, intended to facilitate memorization. This was, however, a more superficial innovation, since descriptions of dialectal features remained fairly traditional throughout the early modern period; Hellenists usually operated with the traditional frameworks of pathology and letter permutations, while taking over information found in ancient and medieval treatises. Yet gradually other linguistic domains such as accent, the lexicon, morphology, and syntax were introduced into handbooks for the Greek dialects. This development was based primarily on a more thorough reading of ancient grammatical works, where Hellenists encountered scattered comments on various features of the Greek dialects. A handful of seasoned philologists went beyond the dialectological accounts contained in earlier treatises to make an actual contribution of their own. The sixteenth-century Hellenist prodigy Henri Estienne was one of the most intriguing exceptions in this regard, as he systematically relied on his own reading of Greek literary texts in order to cast doubt on the dialectal features transmitted by earlier scholarship.

The Greek dialects were a subject that appealed to scholars across Western Europe, and it is difficult, if not impossible, to discern regional traditions or focuses. The dialects belonged to the transnational Republic of Letters (cf. \citealt{BotsWaquet1997}). This does not mean, of course, that no hotbeds of Hellenism can be pinpointed. What were the main centers of interest in the Greek dialects throughout the early modern period? The Greek turn was initiated in the north of the Italian peninsula in cities like Florence and Venice. The latter played a major role in the early modern history of ancient Greek dialectology, since it was there that in 1496 Aldus Manutius issued for the first time three Greek treatises on the dialects that were to be read enthusiastically for decades to come, before being superseded by new handbooks by Western Hellenists. These were published in different parts of Western Europe, from Italy and Spain to Denmark and from England to the Holy Roman Empire. It is not easy to identify true centers of reflection on the Greek dialects, because of the nature of the evidence. The manuals appeared scattered across Europe and often catered to local didactic needs, their composition being encouraged by a lack of available textbooks rather than by a scientific interest in the matter. I believe that it is nevertheless possible to label some cities as major centers, including not least of all Paris and Wittenberg, both cities where the teaching of Greek was well established and outstanding throughout most of the early modern period. Several handbooks on the Greek dialects were published in these two cities.\footnote{See e.g. \citet{Schmidt1604}, published in Wittenberg, and \citet{Mérigon1621}, printed in Paris.} In the Protestant stronghold of Wittenberg, Hellenists did not limit themselves to didactic publications, as was largely the case in Paris, but also ventured original investigations into the Greek dialects, their nature, and their history. This often occurred in the form of academic disputations (see e.g. \citealt{Thryllitsch1709}), as Greek had been incorporated into the university curriculum ever since Philipp Melanchthon’s appointment in 1518. More scientific concerns were also behind the publication of Claude de \citegen{Saumaise1643a} commentary on the Greek language, triggered by Daniel Heinsius’ positing of a Hellenistic dialect, which Saumaise keenly contested. This academic dispute took place in the city of Leiden, the main hub of Hellenism in the Dutch Republic. Given the didactic goals of most manuals on the Greek dialects, direct debate was limited in them, and their main concern was either to accumulate all known dialect features or to select the most important properties, depending on their aims and intended student readership. In philological and historiographic dissertations, on the other hand, conceptions of the Greek dialects were sometimes heavily discussed and could even stir up fierce controversies, as in the case of Heinsius and Saumaise; their rivalry was, however, spurred as much by personal hostilities as it was by intellectual disagreement.

One of the main arguments I have tried to deploy in the present book is that knowledge and awareness of the ancient Greek dialects were frequently valorized by grammarians concerned with a wide range of different languages. This usually happened to stress either the similarities or the differences between Greek and vernacular dialects. In the former case, Greek was a model context; its variation encouraged scholars to grammatically codify their native tongues in spite of the enormous dialectal diversity they encountered in it. In the latter case, vernacular grammarians stressed the particularity of the Greek situation, where the dialects were canonized for literary reasons; this was considered impossible or undesirable for vernacular tongues, which they were trying to mold into a uniform whole. The Greek dialects were taken as a point of comparison not only for language-ideological purposes. The close kinship among them was also a useful descriptive reference point for many philologists, in particular those engaged in charting the interrelationships of the so-called Oriental tongues and in describing various aspects of non-Greek languages. In contrast, the Greek dialects were often in need of explanation themselves, especially in a didactic context. Teachers broaching the topic of Greek linguistic diversity frequently felt compelled to draw their students’ attention to their native dialects in order to clarify what the Greek dialects were. These cross-linguistic comparisons often involved the suppression of differences between Greek dialects and dialects in other languages, usually in sociolinguistic terms, as the Greek dialects were rather exceptional in having reached literary status.

Given the frequency of the comparison of the Greek dialects with other contexts of language-internal diversity, I would like to venture two final thoughts on the broader impact of the Renaissance rehabilitation of the Greek language and its dialects on the early modern study of linguistic diversity. Firstly, it does not seem inconceivable that the intense early modern confrontation with the Greek dialects greatly stimulated the comparative study of languages as it started to emerge in the latter half of the sixteenth century.\footnote{This is often dubbed “precomparativism”. See e.g. Considine \& Van \citet{Hal2010}, with further references.} In fact, in the early sixteenth century, Greek language teachers introduced a new presentation of dialect data that likely contributed to triggering a comparative reflex among philologists. At that time, Hellenists started to put the different dialectal realizations of one and the same form next to each other in large tables printed in their manuals. This made it possible to assess at a single glance the similarities and differences between related word forms. It is moreover not farfetched to assume that the cross-dialectal letter variations philologists perceived in Greek increased their awareness of similar variations cross-linguistically. This conclusion becomes very tangible indeed when one thinks of the frequent early modern paralleling of sigma–tau variation in Greek with s–t alternation in Germanic. It is further corroborated, I believe, by the inclusion of Latin in the Greek linguistic sphere; the language of the ancient Romans was frequently claimed to be in a close relationship with especially Aeolic and Doric, which some scholars even attempted to prove by means of concrete linguistic correspondences. It led, in other words, to an active comparison of Latin with varieties of Greek. The comparative reflex of early modern philologists was, in short, partly fostered by their rediscovery of Greek dialectal diversity.

Secondly, the confrontation with the Greek dialects contributed to exciting awareness of, and curiosity about, language-internal diversity of other tongues. From the second half of the seventeenth century onward, non-Greek dialects likewise received book-length treatments. Apart from the popular format of dialect wordlists and lexica, numerous descriptions of vernacular dialects and their peculiarities were published, usually written out of patriotic sentiment or antiquarian interest.\footnote{For patriotic sentiment, see e.g. \citet{Meisner1705} on Silesian. For antiquarianism, see e.g. \citet{Oberlin1775}. For curiosity-driven dialect wordlists, see the excellent account of \citet{Considine2017}.} Several of the authors of these works were trained as Hellenists and actively put their knowledge of the Greek dialects to use in charting the unique features of individual dialects. These included, most notably, Michael Richey (1678–1761), compiler of a dialect lexicon of Hamburg speech \citep{Richey1743} and author of a \textit{Hamburg} \textit{dialectology} \citep{Richey1755}, and Sven Hof, who published in 1772 a monograph on the Västergötland dialect in Sweden, standing out for its frequent usage of Greek terminology and examples. Early sixteenth-century humanists such as Juan Luis Vives were, one might conclude, getting lost only in the vast labyrinth of the Greek dialects, but by the end of the eighteenth century scholars were wandering in a much larger one, that of dialect diversity \textit{tout} \textit{court}. For the first time in history, the universality of the phenomenon was widely recognized, a feat which may count as a fundamental achievement of early modern dialect studies.

*

*  *

In this book, I have only provided a first exploration of the history of Greek dialectology. Much work remains to be done. More case studies are needed to deepen our understanding of the aspects I have highlighted here and to adjust my conclusions wherever necessary. It would, for instance, be interesting to analyze more closely the use of Greek in treatises on vernacular dialects. Closer studies of individual Greek dialect manuals and their unique book-historical aspects are needed in order to understand how these books were actually used by early modern readers and students of Greek. This includes investigating the annotations contained in many copies of these handbooks. It would moreover be fruitful to find out to what extent the insights I have formulated here can be extrapolated to ideas scholars have expressed in the large body of extant manuscript material, which I have included only very marginally in my discussion for reasons of feasibility. Finally, I have pointed out that the so-called pioneers of ancient Greek dialectology were partly indebted to some of their eighteenth-century predecessors, a remarkable continuity which has hitherto remained under the radar of historians of linguistics and which provides an intriguing example that counters the nineteenth-century trend to neglect early modern insights.

\section{Bibliography}
\hypertarget{Toc19704870}{}\subsection{Primary sources}
\hypertarget{Toc19704871}{}
Adelung, Johann Christoph. 1781. \textit{Kurzer} \textit{Begriff} \textit{menschlicher} \textit{Fertigkeiten} \textit{und} \textit{Kenntnisse,} \textit{so} \textit{fern} \textit{sie} \textit{auf} \textit{Erwerbung} \textit{des} \textit{Unterhalts,} \textit{auf} \textit{Vergnügen,} \textit{auf} \textit{Wissenschaft,} \textit{und} \textit{auf} \textit{Regierung} \textit{der} \textit{Gesellschaft} \textit{abzielen}. Vol. 4. 4 vols. Leipzig: bey Christian Gottlieb Hertel.

Ahrens, Heinrich Ludolf. 1839–1843. \textit{De} \textit{Graecae} \textit{linguae} \textit{dialectis}. 2 vols. Gottingae: apud Vandenhoeck et Ruprecht.

Alsted, Johann Heinrich. 1630. \textit{Encyclopaedia} \textit{septem} \textit{tomis} \textit{distincta} [...]. Herbornae Nassouiorum: [Christoph Corvin].

Althamer, Andreas. 1536. \textit{Commentaria} \textit{Germaniae} \textit{in} \textit{P.} \textit{Cornelii} \textit{Taciti} \textit{equitis} \textit{Rom.} \textit{libellum} \textit{de} \textit{situ,} \textit{moribus} \textit{et} \textit{populis} \textit{Germanorum}. Norimbergae: apud Io. Petreium.

Amerot, Adrien. 1520. \textit{Compendium} \textit{Graecae} \textit{grammatices,} \textit{perspicua} \textit{breuitate} \textit{complectens,} \textit{quicquid} \textit{est} \textit{octo} \textit{partium} \textit{orationis}. Louanii: apud Theodoricum Martinum.

Amerot, Adrien. 1530. \textit{De} \textit{dialectis} \textit{diuersis} \textit{declinationum} \textit{Graecanicarum} \textit{tam} \textit{in} \textit{uerbis} \textit{quam} \textit{nominibus,} \textit{ex} \textit{Corintho,} \textit{Ioan.} \textit{Grammatico,} \textit{Plutarcho,} \textit{Ioan.} \textit{Philopono} \textit{atque} \textit{aliis} \textit{eiusdem} \textit{classis}. Parisiis: ex officina Gerardi Morrhii.

Anon. 1613. \textit{Grammatica} \textit{Graeca,} \textit{praecipue} \textit{quatenus} \textit{a} \textit{Latina} \textit{Bremensi} \textit{differt}. Bremae: apud Ioannem Wesselium.

Anon. 1725. \textit{Abregé} \textit{de} \textit{la} \textit{grammaire} \textit{grecque} \textit{de} \textit{Clenard,} \textit{des} \textit{accens,} \textit{de} \textit{la} \textit{syntaxe} \textit{et} \textit{des} \textit{dialectes:} \textit{Peu} \textit{de} \textit{préceptes,} \textit{mais} \textit{beaucoup} \textit{de} \textit{lecture,} \textit{de} \textit{réflexion} \textit{et} \textit{d’exercice}. A Paris: chez Jean-Luc Nion.

Anon. 1765. Patois. In Denis Diderot \& Jean le Rond d’Alembert (eds.), \textit{Encyclopédie,} \textit{ou} \textit{dictionnaire} \textit{raisonné} \textit{des} \textit{sciences,} \textit{des} \textit{arts} \textit{et} \textit{des} \textit{métiers} [...], vol. 12, 174. A Neufchastel: chez Samuel Faulche.

Antesignanus, Petrus. 1554. De dialectis appendix. \textit{Institutiones} \textit{linguae} \textit{Graecae,} \textit{N.} \textit{Cleonardo} \textit{auctore,} \textit{cum} \textit{scholiis} \textit{P.} \textit{Antesignani} \textit{Rapistagnensis}, 11–16. Lugduni: apud Matthiam Bonhomme.

Bachmann, Andreas [praeses] \& Michael Ludovici [respondens]. 1625. \textit{$\Pi} \textit{\varepsilon} \textit{\rho} \textit{\text{\textgreek{`i}}$ $\tau \text{\textgreek{~w}}\nu $ $\delta \iota \alpha \lambda \text{\textgreek{'e}}\kappa \tau \omega \nu $: $\Delta \iota \text{\textgreek{'a}}\lambda \varepsilon \xi \iota \varsigma $ $\varphi \iota \lambda o\lambda o\gamma \iota \kappa \text{\textgreek{`h}}$, siue De linguis}. Lipsiae: excudebat Gregorius Ritzsch.

Bacon, Roger. 1902. \textit{The} \textit{Greek} \textit{Grammar} \textit{of} \textit{Roger} \textit{Bacon} \textit{and} \textit{a} \textit{Fragment} \textit{of} \textit{His} \textit{Hebrew} \textit{Grammar}. (Ed.) Edmond Nolan \& S. A. Hirsch. Cambridge: at the University Press.

Bagnati, Simone. 1732. \textit{Lezioni} \textit{sacre} \textit{sopra} \textit{i} \textit{fatti} \textit{illustri} \textit{della} \textit{Divina} \textit{Scrittura} \textit{predicate} \textit{nel} \textit{Gesu’} \textit{di} \textit{Napoli} [...]. Vol. 1. In Venezia: appresso Cristoforo Zane.

Baile, Guillaume. 1588. \textit{De} \textit{Graecorum} \textit{dialectis} \textit{libellus} \textit{numquam} \textit{excussus}. Burdigalae: ex officina Simonis Millangii.

Bayly, Anselm. 1756. \textit{An} \textit{introduction} \textit{literary} \textit{and} \textit{philosophical} \textit{to} \textit{languages,} \textit{especially} \textit{to} \textit{the} \textit{English,} \textit{Latin,} \textit{Greek} \textit{and} \textit{Hebrew,} \textit{exhibiting} \textit{at} \textit{one} \textit{view} \textit{their} \textit{grammar,} \textit{rationale,} \textit{analogy} \textit{and} \textit{idiom,} \textit{in} \textit{three} \textit{parts}. Vol. 1 \& 2. 3 vols. London: printed for John and James Rivington.

Beattie, James. 1778. \textit{Essays} \textit{on} \textit{the} \textit{nature} \textit{and} \textit{immutability} \textit{of} \textit{truth,} \textit{in} \textit{opposition} \textit{to} \textit{sophistry} \textit{and} \textit{scepticism;} \textit{on} \textit{poetry} \textit{and} \textit{music,} \textit{as} \textit{they} \textit{affect} \textit{the} \textit{mind;} \textit{on} \textit{laughter,} \textit{and} \textit{ludicrous} \textit{composition;} \textit{and} \textit{on} \textit{the} \textit{utility} \textit{of} \textit{classical} \textit{learning}. A new edition, revised and carefully corrected. Vol. 2. 2 vols. Dublin: printed for C. Jenkin.

Beauzée, Nicolas. 1765. Langue. In Denis Diderot \& Jean le Rond d’Alembert (eds.), \textit{Encyclopédie,} \textit{ou} \textit{dictionnaire} \textit{raisonné} \textit{des} \textit{sciences,} \textit{des} \textit{arts} \textit{et} \textit{des} \textit{métiers} [...], vol. 9, 249–266. A Neufchastel: chez Samuel Faulche.

Becman, Johann Christoph. 1673. \textit{Historia} \textit{orbis} \textit{terrarum} \textit{geographica} \textit{et} \textit{ciuilis,} \textit{de} \textit{uariis} \textit{negotiis} \textit{nostri} \textit{potissimum} \textit{et} \textit{superioris} \textit{saeculi} \textit{aliisue} \textit{rebus} \textit{selectioribus}. Francofurti ad Oderam: sumptibus heredum Iobi Wilhelmi Fincelii, praelo Becmaniano.

Belon, Pierre. 1553. \textit{Les} \textit{observations} \textit{de} \textit{plusieurs} \textit{singularitez} \textit{et} \textit{choses} \textit{memorables,} \textit{trouvées} \textit{en} \textit{Grece,} \textit{Asie,} \textit{Iudée,} \textit{Egypte,} \textit{Arabie} \textit{et} \textit{autres} \textit{pays} \textit{estranges,} \textit{redigées} \textit{en} \textit{trois} \textit{livres}. A Paris: en la boutique de Georges Corrozet.

Bembo, Pietro. 1525. \textit{Prose} [...] \textit{nelle} \textit{quali} \textit{si} \textit{ragiona} \textit{della} \textit{volgar} \textit{lingua} [...]. Vinegia: per Giovan Tacuino.

Bentley, Richard. 1699. \textit{A} \textit{dissertation} \textit{upon} \textit{the} \textit{epistles} \textit{of} \textit{Phalaris:} \textit{With} \textit{an} \textit{answer} \textit{to} \textit{the} \textit{objections} \textit{of} \textit{the} \textit{honourable} \textit{Charles} \textit{Boyle}. London: printed by J. H. for H. Mortlock and J. Hartley.

Bentley, Richard. 1726. De metris Terentianis $\sigma \chi \varepsilon \delta \text{\textgreek{'i}}\alpha \sigma \mu \alpha $. In Richard Bentley (ed.), \textit{Publii} \textit{Terentii} \textit{Afri} \textit{Comoediae,} \textit{Phaedri} \textit{Fabulae} \textit{Aesopiae,} \textit{Publii} \textit{Syri} \textit{et} \textit{aliorum} \textit{ueterum} \textit{sententiae}, \textsc{i}–\textsc{xix}. Cantabrigiae \& Londini: apud Cornelium Crownfield \& apud Iacobum Knapton, Robertum Knaplock, Paulum Vaillant.

Benvoglienti, Bartolomeo. 1975. \textit{Il} \textit{discorso} \textit{linguistico} \textit{di} \textit{Bartolomeo} \textit{Benvoglienti} (Biblioteca degli Studi Mediolatini e Volgari. Nuova serie 3). (Ed.) Mirko Tavoni. Pisa: Pacini.

Beroaldo, Filippo. 1493. \textit{Commentationes} [...] \textit{in} \textit{Suetonium} \textit{Tranquillum,} \textit{dicatae} \textit{inclyto} \textit{Annibali} \textit{Bentiuolo}. Bononiae: impressit Benedictus Hectoris Bononiensis.

Beza, Theodore. 1594. \textit{Annotationes} \textit{maiores} \textit{in} \textit{Nouum} \textit{Dn.} \textit{nostri} \textit{Iesu} \textit{Christi} \textit{Testamentum}. Noua autem haec editio multo correctior et emendatior priore. 2 vols. [Geneva]: [Jérémie des Planches].

Bibliander, Theodore. 1542. \textit{De} \textit{optimo} \textit{genere} \textit{grammaticorum} \textit{Hebraicorum} \textit{commentarius}. Basileae: per Hieronymum Curionem.

Bischoff, August. 1708. \textit{Cadmus,} \textit{siue} \textit{Lingua} \textit{Graeca} \textit{e} \textit{suis} \textit{eruta} \textit{natalibus,} \textit{fundamentis} \textit{superstructa} \textit{firmioribus} \textit{analogia} \textit{quadam} \textit{numquam} \textit{audita} \textit{ad} \textit{legendos} \textit{Graecos} \textit{accessuris} \textit{apprime} \textit{accommodata}. Editio secunda priori magis perspicua, auctior atque emendatior. Ienae: sumptibus auctoris; litteris Io. Adolphi Mulleri.

[Blackwell], [Thomas]. 1735. \textit{An} \textit{enquiry} \textit{into} \textit{the} \textit{life} \textit{and} \textit{writings} \textit{of} \textit{Homer}. London: [s.n.].

Blount, Charles. 1680. Illustrations on Chap. 13. In Charles Blount (ed.), trans. Charles Blount, \textit{The} \textit{two} \textit{first} \textit{books} \textit{of} \textit{Philostratus,} \textit{concerning} \textit{the} \textit{life} \textit{of} \textit{Apollonius} \textit{Tyaneus,} \textit{written} \textit{originally} \textit{in} \textit{Greek,} \textit{and} \textit{now} \textit{published} \textit{in} \textit{English,} \textit{together} \textit{with} \textit{philological} \textit{notes} \textit{upon} \textit{each} \textit{chapter}, 68–76. London: printed for Nathaniel Thompson.

Blount, Thomas. 1656. \textit{Glossographia,} \textit{or} \textit{A} \textit{dictionary,} \textit{interpreting} \textit{all} \textit{such} \textit{hard} \textit{words,} \textit{whether} \textit{Hebrew,} \textit{Greek,} \textit{Latin,} \textit{Italian,} \textit{Spanish,} \textit{French,} \textit{Teutonick,} \textit{Belgick,} \textit{British} \textit{or} \textit{Saxon,} \textit{as} \textit{are} \textit{now} \textit{used} \textit{in} \textit{our} \textit{refined} \textit{English} \textit{tongue} [...]. London: printed by Tho. Newcomb, and are to be sold by Humphrey Moseley [...] and George Sawbridge [...].

Bochart, Samuel. 1646. \textit{Geographiae} \textit{sacrae} \textit{pars} \textit{prior:} \textit{Phaleg} \textit{seu} \textit{de} \textit{dispersione} \textit{gentium} \textit{et} \textit{terrarum} \textit{diuisione} \textit{facta} \textit{in} \textit{aedificatione} \textit{turris} \textit{Babel} [...]. \textit{Geographiae} \textit{sacrae} \textit{pars} \textit{altera:} \textit{Chanaan} \textit{seu} \textit{de} \textit{coloniis} \textit{et} \textit{sermone} \textit{Phoenicum} [...]. Cadomi: typis Petri Cardonelli.

Bolius, Jacobus [praeses] \& Nicolaus Alberti [respondens]. 1689 [1647]. \textit{Iusta} \textit{et} \textit{modesta,} \textit{absentiae} \textit{ab} \textit{exequiis} \textit{et} \textit{iustis} \textit{funebribus,} \textit{funeri} \textit{linguae} \textit{Hellenisticae} \textit{paratis,} \textit{excusatio} \textit{et} \textit{uenia,} \textit{quam} \textit{ab} \textit{iis,} \textit{qui} \textit{idiotismum} \textit{Ebraeum} \textit{in} \textit{tabulis} \textit{Testamentariis} \textit{Noui} \textit{Foederis,} \textit{Saluatoris} \textit{nostri} \textit{Christi} \textit{IESV} \textit{negando,} \textit{funus} \textit{linguae} \textit{Hellenisticae} \textit{procurant,} \textit{iis} \textit{uero,} \textit{qui} \textit{soloecismos} \textit{et} \textit{barbarismos} \textit{eisdem} \textit{impingendo,} \textit{prorsus} \textit{efferunt} \textit{et} \textit{contumulant} [...]. Regiomonti: prelo Reusneriano.

Bolzanio, Urbano. 1497. \textit{Institutiones} \textit{Graecae} \textit{grammatices}. Venetiis: in aedibus Aldi Manutii.

Bolzanio, Urbano. 1512. \textit{Grammaticae} \textit{institutiones} \textit{iterum} \textit{perquam} \textit{diligenter} \textit{elaboratae,} \textit{quippe} \textit{quod} \textit{alias} \textit{unum} \textit{ac} \textit{satis} \textit{in} \textit{compositum} \textit{fuerat} \textit{corpus} \textit{in} \textit{duo} \textit{nunc} \textit{politissima} \textit{membra} \textit{defluxit:} \textit{Quorum} \textit{alterum} \textit{per} \textit{compendia} \textit{ducet} \textit{adulescentes} \textit{alterum} \textit{iam} \textit{artis} \textit{arcana} \textit{consulet} \textit{indagaturis}. Venetiis: impressum [...] sumptu miraque diligentia Ioannis de Tridino alias Tacuino.

Bolzanio, Urbano. 1545. \textit{Grammaticae} \textit{institutiones} \textit{in} \textit{Graecam} \textit{linguam} \textit{ultima} \textit{ipsius} \textit{censura} \textit{editioneque} \textit{probatae,} \textit{ac} \textit{post} \textit{longam} \textit{suppressionem} \textit{tandem} \textit{in} \textit{lucem} \textit{emissae}. Venetiis: apud heredes Petri Rabani et socios.

Borghini, Vincenzio. 1971 [\textit{ante} 1580]. Se la diversità della lingua greca è la medesima come la italiana. In John Robert Woodhouse (ed.), \textit{Scritti} \textit{inediti} \textit{o} \textit{rari} \textit{sulla} \textit{lingua} (Collezione di opere inedite o rare 132), 335–347. Bologna: commissione per i testi di lingua.

Börner, Christian Friedrich. 1705. \textit{Dissertatio} \textit{altera} \textit{de} \textit{exulibus} \textit{Graeciae} \textit{iisdemque} \textit{litterarum} \textit{in} \textit{Italia} \textit{instauratoribus} \textit{qua} \textit{illustres} \textit{duumuiros} \textit{in} \textit{celeberrima} \textit{academia} \textit{Lipsiensi} \textit{primi} \textit{gradus} \textit{ad} \textit{locum} \textit{in} \textit{amplissimo} \textit{sapientum} \textit{ordine} \textit{capessendum} \textit{more} \textit{maiorum} \textit{struendi} \textit{causa} \textit{[…]} \textit{in} \textit{memoriam} \textit{reuocat} \textit{[…]} \textit{Boernerus} \textit{[…]}. Lipsiae: litteris Goezianis.

Bovelles, Charles de. 1533. \textit{Liber} \textit{de} \textit{differentia} \textit{uulgarium} \textit{linguarum} \textit{et} \textit{Gallici} \textit{sermonis} \textit{uarietate:} \textit{Quae} \textit{uoces} \textit{apud} \textit{Gallos} \textit{sint} \textit{factitiae} \textit{et} \textit{arbitrariae} \textit{uel} \textit{barbariae;} \textit{quae} \textit{item} \textit{ab} \textit{origine} \textit{Latina} \textit{manarint}. Parisiis: ex officina Roberti Stephani.

Boxhorn, Marcus Zuerius van. 1647. \textit{Antwoord} \textit{van} \textit{Marcus} \textit{Zuerius} \textit{van} \textit{Boxhorn,} \textit{gegeven} \textit{op} \textit{de} \textit{vraaghen,} \textit{hem} \textit{voorgestelt} \textit{over} \textit{de} \textit{Bediedinge} \textit{van} \textit{de} \textit{afgodinne} \textit{Nehalennia,} \textit{onlancx} \textit{uytghegeven} \textit{[…]}. Tot Leyden: by Willem Christiaens vander Boxe.

Bregius, Johannes. 1684. \textit{Prosodia} \textit{Graeca,} \textit{cum} \textit{dialectologia,} \textit{in} \textit{usum} \textit{et} \textit{tironum} \textit{et} \textit{prouectiorum} \textit{elaborata}. Tubingae: excusa […] sumptibus Ioan. Gotofredi Zubroti, typis Ioannis Henrici Reisii.

Brinck, Ernst. 1615–1635. Album amicorum 2. The Hague, Koninklijke Bibliotheek, ms. n° 135 K 4, fol. 56\textsc{\textsuperscript{v}}–58\textsc{\textsuperscript{r}} [linguistic notes].

Bullokar, John. 1616. \textit{An} \textit{English} \textit{expositor,} \textit{teaching} \textit{the} \textit{interpretation} \textit{of} \textit{the} \textit{hardest} \textit{words} \textit{used} \textit{in} \textit{our} \textit{language:} \textit{With} \textit{sundry} \textit{explications,} \textit{descriptions,} \textit{and} \textit{discourses}. London: printed by John Legatt.

Burton, William. 1657. \textit{Graecae} \textit{linguae} \textit{historia,} \textit{siue} \textit{Oratio} \textit{de} \textit{eiusdem} \textit{linguae} \textit{origine} \textit{progressu} \textit{atque} \textit{ad} \textit{ipsam} \textit{$\text{\textgreek{>a}}\kappa \mu \text{\textgreek{`h}}\nu $ incremento, de latissimo denique ipsius, omnibus prope saeculis, per uniuersum terrarum orbem, usu, habita olim Oxoniis in Aula Gleuocestrensi, ante XX et VI annos}. Londinii (Augustae Trinobantum): apud Thomam Roycroft.

Busby, Richard. 1696. \textit{Grammatica} \textit{Busbeiana} \textit{auctior} \textit{et} \textit{emendatior,} \textit{i.e.} \textit{rudimentum} \textit{grammaticae} \textit{Graeco-Latinae} \textit{metricum:} \textit{In} \textit{usum} \textit{nobilium} \textit{puerorum} \textit{in} \textit{schola} \textit{regia} \textit{Westmonasterii}. Opus posthumum. Londini: ex officina Eliz. Redmayne.

Caelius Rhodiginus, Lodovicus. 1542. \textit{Lectionum} \textit{antiquarum} \textit{libri} \textit{XXX}. Basileae: per Hier. Frobenium et Nicol. Episcopium.

Camden, William. 1595. \textit{Institutio} \textit{Graecae} \textit{grammatices} \textit{compendiaria,} \textit{in} \textit{usum} \textit{Regiae} \textit{Scholae} \textit{Westmonasteriensis:} \textit{Scientiarum} \textit{ianitrix} \textit{grammatica}. Londini: excudebat Edm. Bollifant pro Simone Waterson.

Canini, Angelo. 1554. [...] \textit{Institutiones} \textit{linguae} \textit{Syriacae,} \textit{Assyriacae} \textit{atque} \textit{Thalmudicae,} \textit{una} \textit{cum} \textit{Aethiopicae} \textit{atque} \textit{Arabicae} \textit{collatione}. Parisiis: apud Carolum Stephanum.

Canini, Angelo. 1555. \textit{$\text{\textgreek{<E}}\lambda \lambda \eta \nu \iota \sigma \mu \text{\textgreek{'o}}\varsigma $: In quo quicquid uetustissimi scriptores de Graecae linguae ratione praecipiunt atque adeo omnia quae ad dialectos intelligendas et poetas penitus cognoscendos pertinent, facili methodo exponuntur. Eo accedit plurimorum uerborum originis explicatio}. Parisiis: apud Guil. Morelium.

Casaubon, Isaac. 1587. Commentarius et castigationes in librum IIII. Geographicorum Strabonis. \textit{$\Sigma} \textit{\tau} \textit{\rho} \textit{\text{\textgreek{'a}}\beta \omega \nu o\varsigma $ $\Gamma \varepsilon \omega \gamma \rho \alpha \varphi \iota \kappa \text{\textgreek{~w}}\nu $ $\beta \text{\textgreek{'i}}\beta \lambda o\iota $ $\iota \zeta $. {\textbar} Strabonis Rerum geographicarum libri XVII}, 67–83. Isaacus Casaubonus recensuit summoque studio et diligentia, ope etiam ueterum codicum, emendauit ac Commentariis illustrauit. [Geneva]: excudebat Eustathius Vignon Atrebat.

Casaubon, Méric. 1650. \textit{De} \textit{quattuor} \textit{linguis} \textit{commentationis} \textit{pars} \textit{prior,} \textit{quae} \textit{de} \textit{lingua} \textit{Hebraica} \textit{et} \textit{de} \textit{lingua} \textit{Saxonica}. Londini: typis I. Flesher, sumptibus Ric. Mynne.

Caselius, Johannes. 1560. \textit{Graecae} \textit{grammaticae} \textit{progymnasmata} \textit{tanta} \textit{facilitate} \textit{et} \textit{dexteritate} \textit{tradita,} \textit{ut} \textit{ex} \textit{iis} \textit{adolescens} \textit{priuata} \textit{etiam} \textit{opera} \textit{paucis} \textit{diebus} \textit{huius} \textit{linguae} \textit{initia} \textit{sine} \textit{magno} \textit{negotio} \textit{recte} \textit{haurire} \textit{et} \textit{feliciter} \textit{ad} \textit{irregularia,} \textit{inquisitionem} \textit{dialectorum} \textit{et} \textit{auctorum} \textit{lectionem} \textit{transire} \textit{possit}. Witebergae: in officina typographica Laurentii Schwenck.

Castelli, Gabriele Lancillotto. 1769. Prolegomena: I. De Graecis Siculorum dialectis historica disquisitio. \textit{Siciliae} \textit{et} \textit{obiacentium} \textit{insularum} \textit{ueterum} \textit{inscriptionum} \textit{noua} \textit{collectio} \textit{prolegomenis} \textit{et} \textit{notis} \textit{illustrata}, \textsc{xv}–\textsc{xxxiii}. Panormi: excudebat Caietanus Maria Bentiuenga.

Castiglione, Baldassarre. 1528. \textit{Il} \textit{libro} \textit{del} \textit{Cortegiano}. In Venetia: nelle case d’Aldo Romano et d’Andrea d’Asola suo suocero.

Castillo, R. P. F. Martin del. 1678. \textit{$\Gamma} \textit{\rho} \textit{\alpha} \textit{\mu} \textit{\mu} \textit{\alpha} \textit{\tau} \textit{\iota} \textit{\kappa} \textit{\text{\textgreek{`h}}$ $\tau \text{\textgreek{~h}}\varsigma $ $\gamma \lambda \text{\textgreek{'w}}\sigma \sigma \eta \varsigma $ $\text{\textgreek{<E}}\lambda \lambda \eta \nu \iota \kappa \text{\textgreek{`h}}\varsigma $ $\text{\textgreek{>e}}\nu $ $\tau \text{\textgreek{~h|}}$ $\delta \iota \alpha \lambda \text{\textgreek{'e}}\kappa \tau \text{\textgreek{w|}}$ $\text{\textgreek{>I}}\beta \varepsilon \rho \iota \kappa \text{\textgreek{~h|}}$ […]}. En Leon de Francia: a costa de Florian Anisson.

Chambers, Ephraim. 1728. \textit{Cyclopaedia,} \textit{or} \textit{An} \textit{universal} \textit{dictionary} \textit{of} \textit{arts} \textit{and} \textit{sciences}. 2 vols. London: printed for James and John Knapton, John Darby, Daniel Midwinter, Arthur Bettesworth, John Senex [...].

Chrysoloras, Manuel. 1512. $\text{\textgreek{>E}}\rho \omega \tau \text{\textgreek{'h}}\mu \alpha \tau \alpha $ $\tau o\text{\textgreek{~u}}$ $X\rho \upsilon \sigma o\lambda \omega \rho \text{\textgreek{~a}}$. \textit{$\text{\textgreek{>E}}\rho \omega \tau \text{\textgreek{'h}}\mu \alpha \tau \alpha $ $\tau o\text{\textgreek{~u}}$ $X\rho \upsilon \sigma o\lambda \omega \rho \text{\textgreek{~a}}$· $\Pi \varepsilon \rho \text{\textgreek{`i}}$ $\text{\textgreek{>a}}\nu \omega \mu \text{\textgreek{'a}}\lambda \omega \nu $ $\text{\textgreek{<r}}\eta \mu \text{\textgreek{'a}}\tau \omega \nu $ [...]}, 3–115. Venetiis: in aedib. Aldi.

Chytraeus, Nathan. 1582. \textit{Nomenclator} \textit{Latinosaxonicus:} \textit{Multo} \textit{aliis} \textit{locupletior}. Rostochii: typis Stephani Myliandri.

Clenardus, Nicolaus. 1530. \textit{Institutiones} \textit{in} \textit{linguam} \textit{Graecam}. Louanii: ex officina chalcographica Rutgeri Rescii ac Ioannis Sturmii.

Clüver, Philipp. 1616. \textit{Germaniae} \textit{antiquae} \textit{libri} \textit{tres:} \textit{Opus} \textit{post} \textit{omnium} \textit{curas} \textit{elaboratissimum,} \textit{tabulis} \textit{geographicis} \textit{et} \textit{imaginibus,} \textit{priscum} \textit{Germanorum} \textit{cultum} \textit{moresque} \textit{referentibus,} \textit{exornatum.} \textit{Adiectae} \textit{sunt} \textit{Vindelicia} \textit{et} \textit{Noricum,} \textit{eiusdem} \textit{auctoris}. Lugduni Batauorum: apud Ludouicum Elzeuirium.

Codro, Antonio Urceo. 1502. \textit{In} \textit{hoc} \textit{Codri} \textit{uolumine} \textit{haec} \textit{continentur:} \textit{Orationes,} \textit{seu} \textit{Sermones} \textit{ut} \textit{ipse} \textit{appellabat.} \textit{Epistolae.} \textit{Siluae.} \textit{Satyrae.} \textit{Eglogae.} \textit{Epigrammata}. Bononiae: emendate ac curateque impressum [...] per Ioannem Antonium Platonidem.

Cottière, Matthieu. 1646. \textit{De} \textit{Hellenistis} \textit{et} \textit{lingua} \textit{Hellenistica} \textit{exercitationes} \textit{secundariae}. [Argentorati]: typis Rihelianis.

Court de Gébelin, Antoine. 1778. \textit{Monde} \textit{primitif,} \textit{analysé} \textit{et} \textit{comparé} \textit{avec} \textit{le} \textit{monde} \textit{moderne,} \textit{considéré} \textit{dans} \textit{les} \textit{origines} \textit{Françoises,} \textit{ou} \textit{Dictionnaire} \textit{étymologique} \textit{de} \textit{la} \textit{langue} \textit{Françoise}. A Paris: chez l’auteur, Boudet, Valleyre, Veuve Duchesne, Saugrain, Ruault.

Craige, Alexander. 1606. \textit{The} \textit{amorose} \textit{songs,} \textit{sonets,} \textit{and} \textit{elegies}. At London: imprinted [...] by William White.

Crinesius, Christoph. 1629. \textit{בָּבֶל} \textit{siue} \textit{Discursus} \textit{de} \textit{confusione} \textit{linguarum,} \textit{tum} \textit{orientalium:} \textit{Hebraicae,} \textit{Chaldaicae,} \textit{Syriacae,} \textit{scripturae} \textit{Samariticae,} \textit{Arabicae,} \textit{Persicae,} \textit{Aethiopicae;} \textit{tum} \textit{occidentalium,} \textit{nempe} \textit{Graecae,} \textit{Latinae,} \textit{Italicae,} \textit{Gallicae,} \textit{Hispanicae,} \textit{statuens} \textit{Hebraicam} \textit{omnium} \textit{esse} \textit{primam} \textit{et} \textit{ipsissimam} \textit{matricem}. Noribergae: typis et sumptibus Simonis Halbmayeri.

Croy, Jean de. 1644. \textit{Sacrarum} \textit{et} \textit{historicarum} \textit{in} \textit{Nouum} \textit{Foedus} \textit{obseruationum} \textit{pars} \textit{prior} [...]. Geneuae: ex typographia Petri Chouet.

Crusius, Martin. 1558. \textit{Puerilis} \textit{in} \textit{lingua} \textit{Graeca} \textit{institutio} \textit{pars} \textit{altera,} \textit{continens} \textit{omnium} \textit{grammaticae} \textit{partium} \textit{tractationem} \textit{satis} \textit{absolutam:} \textit{Pro} \textit{scholae} \textit{Memmingensis} \textit{classe} \textit{prima}. Basileae: per Ioannem Oporinum.

Crusius, Martin. 1584. \textit{Turcograeciae} \textit{libri} \textit{octo}. Basileae: per Leonardum Ostenium, Sebastiani Henricpetri impensa.

da Ponte, Ludovico. 1501. Ponticus Virunius litterarum Graecarum studioso cuicumque benefacere. \textit{Erotemata} \textit{Guarini}, [i–iii]. Rhegii Lingobardiae: impensis nobilis Simonis Bombasii; et sociorum Pontici Virunii; et presbyteri Dionysii Berthochi; Benedictus Mangius Carpensis impressit.

da Ponte, Ludovico. 1509. Declarationes quaedam ad magnificum Antonium uicecomitem Lod. Sfor. Subrorum ducis consiliarium ac oratorem Ferrariae in Erotemata Guarini tumultuariae. \textit{Erotemata} \textit{Guarini} \textit{cum} \textit{multis} \textit{additamentis,} \textit{et} \textit{cum} \textit{commentariis} \textit{Latinis}, 14\textsc{\textsuperscript{r}}–172\textsc{\textsuperscript{r}}. Ferrariae: impressum [...] per Ioannem Mazochum.

Dabercusius, Matthias Marcus. 1577. \textit{Quaestionum} \textit{de} \textit{grammatica} \textit{Graeca} \textit{libri} \textit{duo}. Rostochii: apud Iacobum Lucium.

de Heuiter, Pontus. 1581. \textit{Nederduitse} \textit{orthographie,} \textit{dat} \textit{is:} \textit{Maniere} \textit{houmen} \textit{opreht} \textit{Nederduits} \textit{spellen} \textit{ende} \textit{schriven} \textit{zal,} \textit{niet} \textit{alleen} \textit{nut} \textit{ende} \textit{nootelic} \textit{die} \textit{opreht} \textit{begeren} \textit{te} \textit{schriven,} \textit{maer} \textit{al} \textit{die} \textit{zulx} \textit{de} \textit{ioincheit} \textit{zouken} \textit{te} \textit{leren}. T’Antwerpen: by Christoffel Plantijn.

de Ronsard, Pierre. 1550. \textit{Les} \textit{quatre} \textit{premiers} \textit{livres} \textit{des} \textit{Odes}. A Paris: chez Guillaume Cavellart.

de Ronsard, Pierre. 1565. \textit{Abbregé} \textit{de} \textit{l’art} \textit{poëtique} \textit{franc̜ois}. A Paris: chez Gabriel Buon.

de Vergara, Francisco. 1537. \textit{De} \textit{Graecae} \textit{linguae} \textit{grammatica} \textit{libri} \textit{quinque:} \textit{Opus} \textit{nunc} \textit{primum} \textit{natum} \textit{et} \textit{excusum}. Compluti: apud Michaelem de Eguia.

Du Cange, Charles. 1688. Praefatio ad glossarium: De causis corruptae Graecitatis. \textit{Glossarium} \textit{ad} \textit{scriptores} \textit{mediae} \textit{et} \textit{infimae} \textit{Graecitatis} [...], vol. 1, i–xviii. Lugduni: apud Anissonios, Ioan. Posuel et Claud. Rigaud.

Dumarsais, César Chesneau. 1754. Dialecte. In Denis Diderot \& Jean le Rond d’Alembert (eds.), \textit{Encyclopédie,} \textit{ou} \textit{Dictionnaire} \textit{raisonné} \textit{des} \textit{sciences,} \textit{des} \textit{arts} \textit{et} \textit{des} \textit{métiers} [...], vol. 4, 933–934. A Paris: chez Briasson [...], David [...], Le Breton [...], Durand [...].

Dupuy, Claude. 2001. À Pinelli. Paris, le 12 décembre 1579. In Anna Maria Raugei (ed.), \textit{Gian} \textit{Vincenzo} \textit{Pinelli} \textit{et} \textit{Claude} \textit{Dupuy:} \textit{Une} \textit{correspondance} \textit{entre} \textit{deux} \textit{humanistes} (Le corrispondenze letterarie, scientifiche ed erudite dal Rinascimento all’Età Moderna 8), vol. 1, 273–284. Firenze: Leo S. Olschki.

Duret, Claude. 1613. \textit{Thresor} \textit{de} \textit{l’histoire} \textit{des} \textit{langues} \textit{de} \textit{cest} \textit{univers:} \textit{Contenant} \textit{les} \textit{origines,} \textit{beautés,} \textit{perfections,} \textit{decadences,} \textit{mutations,} \textit{changements,} \textit{conuersions,} \textit{et} \textit{ruines} \textit{des} \textit{langues} [...]. A Cologny: imprimé [...] par Matth. Berjon.

Eichhorn, Johann Gottfried. 1780. \textit{Einleitung} \textit{in} \textit{das} \textit{alte} \textit{Testament}. Vol. 1. Leipzig: bey Weidmanns Erben und Reich.

Eling, Lorenz Ingewald. 1691. \textit{Historia} \textit{Graecae} \textit{linguae}. Cum praefatione Adami Rechenbergii. Lipsiae: impensis Io. Frid. Gleditsch.

Eliot, Thomas. 1538. \textit{The} \textit{dictionary} \textit{of} \textit{syr} \textit{Thomas} \textit{Eliot} \textit{knyght}. Londini: in aedibus Thomae Bertheleti typis impress.

Énoch, Louis. 1555. \textit{De} \textit{puerili} \textit{Graecarum} \textit{litterarum} \textit{doctrina} \textit{liber}. [Geneva]: oliua Roberti Stephani.

Erasmus, Desiderius. 1519. \textit{In} \textit{Nouum} \textit{Testamentum} [...] \textit{denuo} \textit{recognitum} \textit{annotationes,} \textit{ingenti} \textit{nuper} \textit{accessione} \textit{per} \textit{auctorem} \textit{locupletatae}. Basileae: apud Ioannem Frobenium.

Erasmus, Desiderius. 1528. \textit{De} \textit{recta} \textit{Latini} \textit{Graecique} \textit{sermonis} \textit{pronuntiatione} [...] \textit{dialogus}. Apud inclytam Basilaeam: in officina Frobeniana.

Erasmus, Desiderius. 1978. \textit{De} \textit{recta} \textit{Latini} \textit{Graecique} \textit{sermonis} \textit{pronuntiatione} \textit{dialogus} \textit{{\textbar} Dialog über die richtige Aussprache der lateinischen und griechischen Sprache} (Beiträge zur klassischen Philologie 98). (Ed.) Johannes Kramer. Als Lesetext herausgegeben, übersetzt und kommentiert. Meisenheim am Glan: Anton Hain.

Estienne, Henri. 1565. \textit{Traicté} \textit{de} \textit{la} \textit{conformité} \textit{du} \textit{language} \textit{François} \textit{avec} \textit{le} \textit{Grec,} \textit{divisé} \textit{en} \textit{trois} \textit{livres} [...]. [Genève]: [Henri Estienne].

Estienne, Henri (ed.). 1570. \textit{$\text{\textgreek{<H}}\rho o\delta \text{\textgreek{'o}}\tau o\upsilon $ $\tau o\text{\textgreek{~u}}$ $\text{\textgreek{<A}}\lambda \iota \kappa \alpha \rho \nu \alpha \sigma \sigma \text{\textgreek{'e}}\omega \varsigma $ $\text{\textgreek{<I}}\sigma \tau o\rho \text{\textgreek{'i}}\alpha $, $\text{\textgreek{>`h}}$ $\text{\textgreek{<I}}\sigma \tau o\rho \iota \text{\textgreek{~w}}\nu $ $\lambda \text{\textgreek{'o}}\gamma o\iota $ θ’, $\text{\textgreek{>e}}\pi \iota \gamma \rho \alpha \varphi \text{\textgreek{'o}}\mu \varepsilon \nu o\iota $ $Mo\text{\textgreek{~u}}\sigma \alpha \iota $ […]}. [Geneva]: excudebat Henricus Stephanus.

Estienne, Henri. 1572. \textit{$\Theta} \textit{\eta} \textit{\sigma} \textit{\alpha} \textit{\upsilon} \textit{\rho} \textit{\text{\textgreek{`o}}\varsigma $ $\tau \text{\textgreek{~h}}\varsigma $ $\text{\textgreek{<E}}\lambda \lambda \eta \nu \iota \kappa \text{\textgreek{~h}}\varsigma $ $\gamma \lambda \text{\textgreek{'w}}\sigma \sigma \eta \varsigma $. Thesaurus Graecae linguae}. 5 vols. [Geneva]: excudebat Henr. Stephanus.

Estienne, Henri. 1573. De Atticae linguae seu dialecti idiomatis comment. \textit{Glossaria} \textit{duo,} \textit{e} \textit{situ} \textit{uetustatis} \textit{eruta:} \textit{Ad} \textit{utriusque} \textit{linguae} \textit{cognitionem} \textit{et} \textit{locupletationem} \textit{perutilia} \textit{[…]}. [Geneva]: excudebat Henr. Stephanus.

Estienne, Henri. 1579. \textit{Project} \textit{du} \textit{livre} \textit{intitulé} \textit{De} \textit{la} \textit{precellence} \textit{du} \textit{langage} \textit{François}. A Paris: par Mamert Patisson.

Estienne, Henri. 1581. \textit{Paralipomena} \textit{grammaticarum} \textit{Gr.} \textit{linguae} \textit{inst.} \textit{Item} \textit{animaduersiones} \textit{in} \textit{quasdam} \textit{grammaticorum} \textit{Gr.} \textit{traditiones}. [Geneva]: [Henri Estienne].

Estienne, Henri. 1582. \textit{Hypomneses} \textit{de} \textit{Gall.} \textit{lingua,} \textit{peregrinis} \textit{eam} \textit{discentibus} \textit{necessariae,} \textit{quaedam} \textit{uero} \textit{ipsis} \textit{etiam} \textit{Gallis} \textit{multum} \textit{profuturae:} \textit{Inspersa} \textit{sunt} \textit{nonnulla,} \textit{partim} \textit{ad} \textit{Graecam,} \textit{partim} \textit{ad} \textit{Lat.} \textit{linguam} \textit{pertinentia,} \textit{minime} \textit{uulgaria}. [Geneva]: [Henri Estienne].

Estienne, Henri. 1587. \textit{Dialogus} \textit{de} \textit{bene} \textit{instituendis} \textit{Graecae} \textit{linguae} \textit{studiis}. [Geneva]: [Henri Estienne].

Fabricius, Johann Albert. 1711. \textit{Bibliothecae} \textit{Graecae} \textit{libri} \textit{IV} \textit{pars} \textit{altera} \textit{quae} \textit{praeter} \textit{scriptores} \textit{de} \textit{numerorum} \textit{doctrina} \textit{et} \textit{alios} \textit{nonnullos} \textit{philosophos} \textit{recensentur} \textit{rhetores} \textit{ac} \textit{sophistae} \textit{lexicorumque} \textit{ueterum} \textit{Graecorum} \textit{notitia} \textit{traditur}. Hamburgi: sumptu Christiani Liebezeit.

Facius, Johann Friedrich. 1782. \textit{Compendium} \textit{dialectorum} \textit{Graecarum}. Norimbergae: sumptibus E. C. Grattenaueri.

Fischer, Johann Friedrich. 1754. Praefatio. \textit{De} \textit{dialectis} \textit{N.} \textit{T.,} \textit{singulatim} \textit{de} \textit{eius} \textit{Hebraismis} \textit{libellus} \textit{singularis}, a.5\textsc{\textsuperscript{r}}–b.8\textsc{\textsuperscript{v}}. Denuo edidit Io. Frider. Fischerus. Lipsiae: apud uiduam B. Gasp. Fritschii.

Florinus, Johann Matthias. 1707. \textit{Exercitationum} \textit{historico-philologicarum} \textit{fasciculus,} \textit{de} \textit{origine} \textit{et} \textit{propagatione} \textit{linguae} \textit{Graecae} \textit{et} \textit{uitis} \textit{quattuor} \textit{Euangelistarum} [...]. Francofurti ad Moenum: litteris Ioannis Philippi Andreae.

[Frederick the Great]. 1780. \textit{De} \textit{la} \textit{littérature} \textit{allemande:} \textit{Des} \textit{défauts} \textit{qu’on} \textit{peut} \textit{lui} \textit{réprocher,} \textit{quelles} \textit{en} \textit{sont} \textit{les} \textit{causes} \textit{et} \textit{par} \textit{quels} \textit{moyens} \textit{on} \textit{peut} \textit{les} \textit{corriger}. A Berlin: chez G. J. Decker.

Fréret, Nicolas. 1809. Observations générales sur l’origine et sur l’ancienne histoire des premiers habitans de la Grèce. \textit{Histoire} \textit{de} \textit{l’Académie} \textit{royale} \textit{des} \textit{inscriptions} \textit{et} \textit{belles-lettres,} \textit{avec} \textit{les} \textit{mémoires} \textit{de} \textit{littérature} \textit{tirés} \textit{des} \textit{registres} \textit{de} \textit{cette} \textit{Académie,} \textit{depuis} \textit{l’année} \textit{M.} \textit{DCCLXXXIV} \textit{jusqu’au} \textit{8} \textit{août} \textit{M.} \textit{DCCXCIII} 47. 1–133 [Section: \textit{Mémoires} \textit{de} \textit{littérature}].

[Frisch], [Johann Leonhard]. 1730. \textit{Vollständigere} \textit{Griechische} \textit{Grammatik,} \textit{nach} \textit{der} \textit{Lehr-Ordnung} \textit{der} \textit{Lateinischen} \textit{Märkischen} \textit{Grammatik} \textit{eingerichtet}. Berlin: zu finden bey Christoph Gottlieb Nicolai.

Fuller, Thomas. 1655. \textit{The} \textit{church-history} \textit{of} \textit{Britain:} \textit{From} \textit{the} \textit{birth} \textit{of} \textit{Jesus} \textit{Christ,} \textit{until} \textit{the} \textit{year} \textit{M.} \textit{DC.} \textit{XLVIII.} London: printed for John Williams.

Furetière, Antoine. 1701. \textit{Dictionnaire} \textit{universel,} \textit{contenant} \textit{generalement} \textit{tous} \textit{les} \textit{mots} \textit{François} \textit{tant} \textit{vieux} \textit{que} \textit{modernes,} \textit{et} \textit{les} \textit{termes} \textit{des} \textit{sciences} \textit{et} \textit{des} \textit{arts} [...]. (Ed.) Henri Basnage de Beauval. Seconde édition, revuë, corrigée et augmentée. 3 vols. A la Haye et à Rotterdam: chez Arnoud et Reinier Leers.

Galiani, Ferdinando. 1779. \textit{Del} \textit{dialetto} \textit{Napoletano}. Napoli: per Vincenzo Mazzola-Vocola.

Gaza, Theodore. 1495. \textit{In} \textit{hoc} \textit{uolumine} \textit{haec} \textit{insunt:} \textit{Theodori} \textit{Introductiuae} \textit{grammatices} \textit{libri} \textit{quattuor.} \textit{Eiusdem} \textit{De} \textit{mensibus} \textit{opusculum} \textit{sanequam} \textit{pulchrum.} [...]. Venetiis: impressum [...] in aedibus Aldi Romani.

Gedike, Friedrich. 1779. \textit{Gedanken} \textit{über} \textit{Purismus} \textit{und} \textit{Sprachbereicherung}. Berlin: gedrukt bei George Jakob Dekker.

Gedike, Friedrich. 1782. Ueber Dialekte, besonders die griechischen. \textit{Berlinsches} \textit{Magazin} \textit{der} \textit{Wissenschaften} \textit{und} \textit{Künste} 1(2). 1–26.

Georgi, Christian Siegmund. 1733. \textit{Hierocriticus} \textit{N.} \textit{Testamenti,} \textit{siue} \textit{De} \textit{stylo} \textit{Noui} \textit{Testamenti} \textit{libri} \textit{tres} [...]. Wittebergae et Lipsiae: ex officina Io. Michaelis Teubneri.

Georgi, Christian Siegmund [praeses] \& Wolfgang Heinrich Graun [respondens]. 1729. \textit{Dissertationem} \textit{criticam} \textit{de} \textit{dialecto} \textit{Noui} \textit{Testamenti} \textit{Attica} \textit{Ionismos} \textit{atque} \textit{Aeolismos} \textit{non} \textit{admittente} [...] \textit{exponet} [...] \textit{Graunius}. Vitembergae: litteris heredum Gerdesianorum.

Germano, Girolamo. 1622. \textit{Vocabolario} \textit{Italiano} \textit{et} \textit{Greco,} \textit{nel} \textit{quale} \textit{si} \textit{contiene} \textit{come} \textit{le} \textit{voci} \textit{Italiane} \textit{si} \textit{dicano} \textit{in} \textit{Greco} \textit{volgare}. In Roma: per l’herede di Bartolomeo Zannetti.

Gesner, Johannes Matthias. 1774. \textit{Primae} \textit{lineae} \textit{isagoges} \textit{in} \textit{eruditionem} \textit{uniuersalem} \textit{nominatim} \textit{philologiam,} \textit{historiam} \textit{et} \textit{philosophiam} \textit{in} \textit{usum} \textit{praelectionum} \textit{ductae}. Accedunt nunc praelectiones ipsae per Io. Nicolaum Niclas. Vol. 1. 2 vols. Lipsiae: sumptibus Caspari Fritsch.

Gessner, Conrad. 1543. De utilitate ac praestantia Graecae linguae, in omni genere studiorum, ad candidos lectores praefatio. \textit{Lexicon} \textit{Graecolatinum}, A.2\textsc{\textsuperscript{r}}–A.7\textsc{\textsuperscript{v}}. Basileae: per Hieronymum Curionem.

Gessner, Conrad. 1555. \textit{Mithridates:} \textit{De} \textit{differentiis} \textit{linguarum} \textit{tum} \textit{ueterum} \textit{tum} \textit{quae} \textit{hodie} \textit{apud} \textit{diuersas} \textit{nationes} \textit{in} \textit{toto} \textit{orbe} \textit{terrarum} \textit{in} \textit{usu} \textit{sunt} [...] \textit{obseruationes}. Tiguri: excudebat Froschouerus.

Giese, Albert. 1837. \textit{Ueber} \textit{den} \textit{Aeolischen} \textit{Dialekt:} \textit{Zwei} \textit{Bücher}. Berlin: in der G. Finckeschen Buchhandlung.

Gill, Alexander. 1619. \textit{Logonomia} \textit{Anglica:} \textit{Qua} \textit{gentis} \textit{sermo} \textit{facilius} \textit{addiscitur}. Londini: excudit Ioannes Beale.

Giovane, Giovanni. 1589. \textit{De} \textit{antiquitate} \textit{et} \textit{uaria} \textit{Tarentinorum} \textit{fortuna} \textit{libri} \textit{octo}. Neapoli: apud Horatium Saluianum.

Girard, Charles. 1541. \textit{Graecarum} \textit{institutionum} \textit{libelli} \textit{undecim}. [Paris \& Bourges]: apud Simonem Colinaeum et Ioannem Engellier.

Giraudeau, Bonaventure. 1739. \textit{Introductio} \textit{ad} \textit{linguam} \textit{Graecam} \textit{complectens} \textit{regulas} \textit{grammaticae,} \textit{radices} \textit{uocum} \textit{et} \textit{exercitationem} \textit{seu} \textit{poema,} \textit{in} \textit{quo} \textit{regulae} \textit{radicesque} \textit{omnes} \textit{ad} \textit{usum} \textit{et} \textit{praxim} \textit{rediguntur}. Romae: typis heredum Ferri.

Giraudeau, Bonaventure. 1752. \textit{Introductio} \textit{ad} \textit{linguam} \textit{Graecam,} \textit{complectens} \textit{regulas} \textit{grammaticae,} \textit{radices} \textit{uocum} \textit{et} \textit{exercitationem} \textit{seu} \textit{poema,} \textit{in} \textit{quo} \textit{regulae} \textit{radicesque} \textit{omnes} \textit{ad} \textit{usum} \textit{et} \textit{praxim} \textit{rediguntur}. Rupellae: ex typographia R. J. Desbordes.

Glarean, Heinrich. 1524. Glareanus poeta laureatus lectori S. \textit{Grammaticae} \textit{institutiones,} \textit{iam} \textit{tanta} \textit{adhuc} \textit{iterum} \textit{cura} \textit{excussae,} \textit{ut} \textit{maiore} \textit{uix} \textit{potuerint,} \textit{in} \textit{quibus} \textit{quid} \textit{operae} \textit{ultra} \textit{priorem} \textit{editionem} \textit{expectes,} \textit{uersa} \textit{pagella} \textit{et} \textit{deinde} \textit{in} \textit{mox} \textit{sequente} \textit{epistola} \textit{reperies}, aaiir-aaiiv. Basileae: [Valentin Curio].

Goropius Becanus, Johannes. 1569. \textit{Origines} \textit{Antwerpianae,} \textit{siue} \textit{Cimmeriorum} \textit{Becceselana} \textit{nouem} \textit{libros} \textit{complexa:} \textit{Atuatica,} \textit{I.} \textit{Gigantomachia,} \textit{II.} \textit{Niloscopium,} \textit{III.} \textit{Cronia,} \textit{IV.} \textit{Indoscythica,} \textit{V.} \textit{Saxsonica,} \textit{VI.} \textit{Gotodanica,} \textit{VII.} \textit{Amazonica,} \textit{VIII.} \textit{Venetica} \textit{et} \textit{Hyperborea,} \textit{IX.} Antuerpiae: ex officina Christophori Plantini.

Gottleber, Johann Christoph. 1765. \textit{De} \textit{causis} \textit{dialectorum} \textit{uariarum} \textit{poetis} \textit{Graecis} \textit{obuiarum} \textit{prolusio,} \textit{qua} \textit{orationem} \textit{Hofmannianam} \textit{in} \textit{Schola} \textit{Annaemontana} \textit{d.} \textit{XXV.} \textit{Ian.} \textit{AO.} \textit{R.} \textit{S.} \textit{MDCCLXV} \textit{habendam} \textit{indicit} \textit{M.} \textit{Io.} \textit{Christoph.} \textit{Gottleber,} \textit{scholae} \textit{rector}. Annabergae: ex officina Aug. Valent. Frisii.

Gottsched, Johann Christoph. 1748. \textit{Grundlegung} \textit{einer} \textit{Deutschen} \textit{Sprachkunst,} \textit{nach} \textit{den} \textit{Mustern} \textit{der} \textit{besten} \textit{Schriftsteller} \textit{des} \textit{vorigen} \textit{und} \textit{jetzigen} \textit{Jahrhunderts} \textit{abgefasset}. Leipzig: verlegts Bernh. Christoph Breitkopf.

Greaves, Thomas. 1639. \textit{De} \textit{linguae} \textit{Arabicae} \textit{utilitate} \textit{et} \textit{praestantia} \textit{oratio} \textit{Oxonii} \textit{habita} \textit{Iul.} \textit{19.} \textit{1637}. Oxonii: excudebat Leonardus Lichfield.

Gretser, Jakob. 1593. \textit{Institutionum} \textit{de} \textit{octo} \textit{partibus} \textit{orationis,} \textit{syntaxi} \textit{et} \textit{prosodia} \textit{Graecorum,} \textit{libri} \textit{tres}. Ingolstadii: excudebat Dauid Sartorius.

Groddeck, Benjamin [praeses] \& Michael Gottlieb Treuge [respondens]. 1747. \textit{Commentatio} \textit{academica} \textit{de} \textit{natura} \textit{dialectorum} \textit{ad} \textit{linguam} \textit{Hebraicam} \textit{et} \textit{Arabicam} \textit{applicata} [...]. Vitembergae: typis Schlomachianis.

Grosch, Johann Andreas. 1753. \textit{Erweiß} \textit{es} \textit{sey} \textit{dem} \textit{Zwecke} \textit{und} \textit{der} \textit{Natur} \textit{der} \textit{griechischen} \textit{Grammatic} \textit{schnurstracks} \textit{zuwider} \textit{in} \textit{selbiger} \textit{die} \textit{Dialecte} \textit{unter} \textit{ihre} \textit{Regeln} \textit{zu} \textit{mischen} \textit{und} \textit{in} \textit{ihr} \textit{mit} \textit{abzuhandeln}. Jena: bey Johann Christoph Crökern.

Grotius, Hugo. 1648. Viro ampliss. Nic. Peirescio, senatori apud Aquas Sextias. \textit{Epistolae} \textit{ad} \textit{Gallos,} \textit{nunc} \textit{primum} \textit{editae}, 143–146. Lugduni Batauorum: ex officina Elzeuiriorum.

Haas, Johann Gottfried. 1780. \textit{Grichische} \textit{Grammatik} \textit{kurz} \textit{und} \textit{deutlich} \textit{insonderheit} \textit{den} \textit{Anfängern} \textit{zum} \textit{Besten} \textit{abgefasset}. Leipzig: im Schwickertschen Verlage.

Haloinus, Georgius. 1978 [1533]. \textit{De} \textit{restauratione} \textit{linguae} \textit{latinae} \textit{libri} \textit{III} (Bibliotheca scriptorum Graecorum et Romanorum Teubneriana). (Ed.) Constantinus Matheeussen. Leipzig: B.G. Teubner.

Harles, Gottlieb Christoph. 1778. \textit{Introductio} \textit{in} \textit{historiam} \textit{linguae} \textit{Graecae}. Altenburgi: in officina libraria Richteria.

Harles, Gottlieb Christoph. 1780. De Dorismo Theocriteo commentatio. \textit{Theocriti} \textit{reliquiae} \textit{Graece} \textit{et} \textit{Latine}, \textsc{xix}–\textsc{xxxix}. Lipsiae: sumpt. hered. Weidmann et Reich.

Hauptmann, Johann Gottfried. 1751. \textit{Historia} \textit{linguae} \textit{Ebraeae} \textit{primis} \textit{lineis} \textit{descripta}. Lipsiae: apud Io. Samuelem Heinsium.

Hauptmann, Johann Gottfried. 1776. \textit{Ad} \textit{iterum} \textit{experiendos} \textit{A.} \textit{R.} \textit{D.} \textit{MDCCLXXVI.} \textit{in} \textit{illustri} \textit{Rutheneo} \textit{commilitones} \textit{inuitans} \textit{de} \textit{Laconica} \textit{dialecto} \textit{pauca} \textit{disputat} \textit{director}. Gerae: ex officina Rothii.

Hauptmann, Johann Gottfried \& Christian Ernst Schmid. 1737. \textit{De} \textit{Atticismo}. Lipsiae: Ioannis Christiani Langenhemii.

Helladius, Alexander. 1714. \textit{Status} \textit{praesens} \textit{ecclesiae} \textit{Graecae:} \textit{In} \textit{quo} \textit{etiam} \textit{causae} \textit{exponuntur} \textit{cur} \textit{Graeci} \textit{moderni} \textit{Noui} \textit{Testamenti} \textit{editiones} \textit{in} \textit{Graeco-barbara} \textit{lingua} \textit{factas} \textit{acceptare} \textit{recusent.} \textit{Praeterea} \textit{additus} \textit{est} \textit{in} \textit{fine} \textit{status} \textit{nonnullarum} \textit{controuersiarum}. [Altdorf]: [s.n.].

Helwig, Christoph. 1610. \textit{De} \textit{ratione} \textit{conficiendi} \textit{facile} \textit{et} \textit{artificiose} \textit{Graeca} \textit{carmina,} \textit{in} \textit{qua} \textit{praecipua} \textit{artificia} \textit{et} \textit{compendia} \textit{Graecae} \textit{poeseos} \textit{perspicue} \textit{et} \textit{breuiter} \textit{indicantur} [...]. Giessae Hassorum: Casparus Chemlinus imprimebat.

Hemsterhuis, Tiberius. 1721. \textit{Oratio} \textit{inauguralis} \textit{de} \textit{Graecae} \textit{linguae} \textit{praestantia,} \textit{ex} \textit{ingenio} \textit{Graecorum} \textit{et} \textit{moribus} \textit{probata} [...]. Franequerae: excudit Henricus Halma.

Hemsterhuis, Tiberius. 2015. \textit{Lectio} \textit{publica} \textit{de} \textit{analogia} \textit{linguae} \textit{Graecae} \textit{(ca} \textit{1750)} (Cahiers voor Taalkunde 31). With a translation and a commentary by Bouke Slofstra. Amsterdam \& Münster: Stichting Neerlandistiek VU \& Münster.

Hertling, Nicolaus. 1708. \textit{Scientia} \textit{Latinitatis} \textit{ex} \textit{eius} \textit{originis,} \textit{miscellae} \textit{et} \textit{interpolationis} \textit{triplici} \textit{Graecismo} \textit{[…]}. Moguntiae: typis et sumptibus Ioannis Mayeri.

Heupel, Georg Friedrich. 1712. \textit{Canones} \textit{de} \textit{Graecarum} \textit{dialectorum} \textit{proprietatibus} \textit{exemplis} \textit{ex} \textit{optimis} \textit{Graecis} \textit{auctoribus} \textit{depromptis} \textit{illustrati} \textit{et} \textit{confirmati,} \textit{cum} \textit{indice} \textit{duplici,} \textit{uno} \textit{auctorum} \textit{qui} \textit{citantur,} \textit{altero} \textit{rerum} \textit{et} \textit{uerborum}. Argentorati: typis et sumptibus uiduae Io. Friderici Spoor.

Heylyn, Peter. 1621. \textit{Microcosmus,} \textit{or} \textit{A} \textit{little} \textit{description} \textit{of} \textit{the} \textit{great} \textit{world:} \textit{A} \textit{treatise} \textit{historicall,} \textit{geographicall,} \textit{politicall,} \textit{theologicall}. Oxford: printed by John Lichfield and James Short.

Heylyn, Peter. 1625. \textit{$M\iota} \textit{\kappa} \textit{\rho} \textit{\text{\textgreek{'o}}\kappa o\sigma \mu o\varsigma $: A little description of the great world}. Augmented and revised. Oxford: printed by John Lichfield and William Turner.

Hickes, George. 1705. \textit{Linguarum} \textit{uett.} \textit{Septentrionalium} \textit{thesaurus} \textit{grammatico-criticus} \textit{et} \textit{archaeologicus}. Vol. 1. 2 vols. Oxoniae: e Theatro Sheldoniano.

Hilgard, Alfredus (ed.). 1901. \textit{Scholia} \textit{in} \textit{Dionysii} \textit{Thracis} \textit{Artem} \textit{grammaticam} (Grammatici Graeci 1). Vol. 3. Lipsiae: in aedibus B. G. Teubneri.

Hill, William. 1658. \textit{Grammaticarum} \textit{in} \textit{Dionysii} \textit{$\Pi} \textit{\varepsilon} \textit{\rho} \textit{\iota} \textit{\text{\textgreek{'h}}\gamma \eta \sigma \iota \nu $ annotationum systema, in usum tironum concinnatum: Vbi dialecti et ceterae anomaliae, quae in singulis dictionibus aut syntaxi occurrunt, ad figuras, methodice hic digestas, reducuntur}. Londini: excudebat Tho. Newcomb.

Hof, Sven. 1772. \textit{Dialectus} \textit{Vestrogothica,} \textit{ad} \textit{illustrationem} \textit{aliquam} \textit{linguae} \textit{Suecanae,} \textit{ueteris} \textit{et} \textit{hodiernae,} \textit{dissertatione} \textit{philologica} \textit{et} \textit{uocabulorum} \textit{Vestrogothicorum} \textit{indice} \textit{explanata}. Stockholmiae: typis et impensis Io. Aru. Carlbohm.

Hofmann, Johann Jakob. 1698. \textit{Lexicon} \textit{uniuersale} \textit{[…]}. Editio absolutissima [...]. 4 vols. Lugduni Batauorum: apud Iacob. Hackium, Cornel. Boutesteyn, Petr. vander Aa et Iord. Luchtmans.

Högström, Pehr. 1748. \textit{Beschreibung} \textit{des} \textit{Crone} \textit{Schweden} \textit{gehörenden} \textit{Lapplandes} [...]. Aus dem Schwedischen übersetzt. Copenhagen und Leipzig: verlegts Gabriel Christian Rothe.

Hoius, Andreas. 1620. Dissertatiuncula de linguae Graecae dialectorum sedibus et coloniis. \textit{Orthoepeia,} \textit{siue} \textit{De} \textit{Germana} \textit{ac} \textit{recta} \textit{linguae} \textit{Graecae} \textit{et} \textit{obiter} \textit{Latinae} \textit{pronuntiatione}, 95–104. Duaci: ex officina Ioannis Bogardi.

Holmes, John. 1735. \textit{The} \textit{Greek} \textit{Grammar,} \textit{or} \textit{A} \textit{short,} \textit{plain,} \textit{critical,} \textit{and} \textit{comprehensive} \textit{method} \textit{of} \textit{teaching} \textit{and} \textit{learning} \textit{the} \textit{Greek} \textit{tongue}. London: printed for the author; and sold by A. Bettesworth and C. Hitch [...] and A. Feazer [...].

Hosius, Stanislaus. 1560. \textit{De} \textit{expresso} \textit{Dei} \textit{uerbo,} \textit{libellus} \textit{his} \textit{temporibus} \textit{accommodatissimus}. Parisiis: apud Io. Foucherium.

Hottinger, Johann Heinrich. 1661. \textit{Etymologicum} \textit{Orientale,} \textit{siue} \textit{Lexicon} \textit{harmonicum} \textit{$\text{\textgreek{<e}}\pi \tau \text{\textgreek{'a}}\gamma \lambda \omega \tau \tau o\nu $ [...]}. Francofurti: sumptibus Io. Wilhelmi Ammonii et Wilhelmi Serlini.

Howell, James. 1642. \textit{Instructions} \textit{for} \textit{forreine} \textit{travell:} \textit{Shewing} \textit{by} \textit{what} \textit{cours,} \textit{and} \textit{in} \textit{what} \textit{compasse} \textit{of} \textit{time,} \textit{one} \textit{may} \textit{take} \textit{an} \textit{exact} \textit{survey} \textit{of} \textit{the} \textit{kingdomes} \textit{and} \textit{states} \textit{of} \textit{Christendome,} \textit{and} \textit{arrive} \textit{to} \textit{the} \textit{practicall} \textit{knowledge} \textit{of} \textit{the} \textit{languages,} \textit{to} \textit{good} \textit{purpose}. London: printed by T. B. for Humprey Mosley.

Howell, James. 1650a. \textit{A} \textit{new} \textit{volume} \textit{of} \textit{familiar} \textit{letters,} \textit{partly} \textit{philosophicall,} \textit{partly} \textit{politicall,} \textit{partly} \textit{historicall}. The second edition, with additions. London: printed by W. H. for Humphrey Moseley.

Howell, James. 1650b. \textit{Epistolae} \textit{Ho-Elianae:} \textit{Familiar} \textit{letters} \textit{domestic} \textit{and} \textit{forren,} \textit{divided} \textit{into} \textit{sundry} \textit{sections,} \textit{partly} \textit{historicall,} \textit{partly} \textit{politicall,} \textit{partly} \textit{philosophicall,} \textit{upon} \textit{emergent} \textit{occasions}. The second edition, enlarged with divers supplements, and the dates annexed which were wanting in the first, with an addition of a third volume of new letters. London: printed by W. H. for Humphrey Moseley.

Hugh of Saint Victor. 1966. \textit{Opera} \textit{propaedeutica:} \textit{Practica} \textit{geometriae,} \textit{De} \textit{Grammatica,} \textit{Epitome} \textit{Dindimi} \textit{in} \textit{philosophiam} (Publications in Mediaeval Studies: The University of Notre-Dame 20). (Ed.) Roger Baron. Notre Dame: University of Notre Dame Press.

Hume, Patrick. 1695. \textit{Annotations} \textit{on} \textit{Milton’s} \textit{Paradise} \textit{lost}. London: printed for Jacob Tonson.

Hunt, Thomas. 1739. \textit{De} \textit{antiquitate,} \textit{elegantia,} \textit{utilitate,} \textit{linguae} \textit{Arabicae} \textit{oratio} \textit{habita} \textit{Oxonii,} \textit{in} \textit{schola} \textit{linguarum,} \textit{VII} \textit{Kalend.} \textit{Augusti,} \textit{MDCCXXXVIII}. Oxonii: e Theatro Sheldoniano, impensis Ricardi Clements.

Jablonski, Paul Ernst. 1714. \textit{Disquisitio} \textit{de} \textit{lingua} \textit{Lycaonica} \textit{ad} \textit{locum} \textit{Actor.} \textit{XIV.} \textit{u.} \textit{11.} \textit{$\text{\textgreek{>E}}\pi \text{\textgreek{~h|}}\rho \alpha \nu $ $\tau \text{\textgreek{`h}}\nu $ $\varphi \omega \nu \text{\textgreek{`h}}\nu $ $\alpha \text{\textgreek{<u}}\tau \text{\textgreek{~w}}\nu $ $\Lambda \upsilon \kappa \alpha o\nu \iota \sigma \tau \text{\textgreek{`i}}$ $\lambda \text{\textgreek{'e}}\gamma o\nu \tau \varepsilon \varsigma $. Sustulerunt uocem suam Lycaonice dicentes etc. speciminis loco placido eruditorum examini oblata}. Berolini: litteris Wesselianis.

Jamot, Frédéric. 1593. \textit{Varia} \textit{poemata} \textit{Graeca} \textit{et} \textit{Latina:} \textit{Hymni,} \textit{idyllia,} \textit{funera,} \textit{odae,} \textit{epigrammata,} \textit{anagrammata}. Antuerpiae: ex officina Plantiniana.

Jehne, Lebrecht Heinrich Samuel. 1782. \textit{Griechische} \textit{Sprachlehre} \textit{zum} \textit{Gebrauch} \textit{des} \textit{Königlichen} \textit{Christianeums} \textit{zu} \textit{Altona}. Hamburg und Altona: verlegt von Benjamin Gottlob Hoffmann und gedruckt bei Johann David Adam Eckhardt.

Jiménez Patón, Bartolomé. 1604. \textit{Eloquencia} \textit{española} \textit{en} \textit{arte}. En Toledo: por Thomas de Guzman.

Junius (the Elder), Franciscus. 1579. \textit{De} \textit{linguae} \textit{Hebraeae} \textit{antiquitate} \textit{praestantiaque} \textit{oratio} \textit{habita} \textit{in} \textit{illustri} \textit{schola} \textit{Neapolitana}. Neapoli: impressa [...] per heredes Ioannis Meyeri.

Junius (the Elder), Franciscus. 1580. \textit{Grammatica} \textit{Hebraeae} \textit{linguae} \textit{nunc} \textit{primum} \textit{iustae} \textit{artis} \textit{methodo} \textit{quam} \textit{accuratissime} \textit{breuissimeque} \textit{fieri} \textit{potuit} \textit{conformata} \textit{et} \textit{in} \textit{lucem} \textit{edita}. Francofurti: apud Andream Wechelum.

Junius (the Younger), Franciscus. 1665. \textit{Quattuor} \textit{D.} \textit{N.} \textit{Iesu} \textit{Christi} \textit{euangeliorum} \textit{uersiones} \textit{perantiquae} \textit{duae,} \textit{Gothica} \textit{scil.} \textit{et} \textit{Anglo-Saxonica:} [...]. Dordrechti: typis et sumptibus Iunianis, excudebant Henricus et Ioannes Essaei.

Kals, Jan Willem. 1752. \textit{Dissertatio} \textit{philologica} \textit{de} \textit{linguae} \textit{Hebraeae} \textit{natalibus} \textit{punctisque} \textit{uocalibus} \textit{ad} \textit{eam} \textit{docendam} \textit{discendamque} \textit{summe} \textit{necessariis}. Vol. 1. Oxonii: e Theatro Sheldoniano, impensis Ioannis Barrett.

Kate, Lambert ten. 1723. \textit{Aenleiding} \textit{tot} \textit{de} \textit{kennisse} \textit{van} \textit{het} \textit{verhevene} \textit{deel} \textit{der} \textit{Nederduitsche} \textit{sprake}. 2 vols. Tot Amsterdam: by Rudolph en Gerard Wetstein.

Keaney, J. J. \& Robert Lamberton (eds.). 1996. \textit{Essay} \textit{on} \textit{the} \textit{Life} \textit{and} \textit{Poetry} \textit{of} \textit{Homer} (American Classical Studies 40). Atlanta: Scholars Press.

Keil, Heinrich (ed.). 1855–1880. \textit{Grammatici} \textit{Latini}. 7 vols. Lipsiae: in aedibus B. G. Teubneri.

Keimann, Christian. 1649. \textit{Tabulae} \textit{declinationum,} \textit{motionis,} \textit{comparationis,} \textit{coniugationum,} \textit{contractionum} \textit{et} \textit{accentuum} \textit{in} \textit{encliticis} \textit{apud} \textit{Graecos} \textit{compendiosae,} \textit{cum} \textit{canonibus} \textit{necessariis,} \textit{in} \textit{usum} \textit{discipulorum} \textit{concinnatae:} \textit{Accesserunt} \textit{regulae} \textit{de} \textit{accentu} \textit{nominatiui,} \textit{de} \textit{dialectis} \textit{deque} \textit{prosodia} \textit{Graecorum} \textit{breues} \textit{notae}. Lipsiae: apud hered. Henningi Grossii, charactere Sengenwaldiano.

Kiesling, Johann Daniel [praeses] \& Johann Georg Bause [respondens]. 1712. [...] \textit{De} \textit{dialectis} \textit{Ebraeorum} \textit{puris,} \textit{dissertatio} \textit{1.} \textit{generalis} [...]. Lipsiae: litteris Brandenburgerianis.

Kindstrand, Jan Fredrik (ed.). 1990. \textit{[Plutarchi]} \textit{de} \textit{Homero} (Bibliotheca scriptorum Graecorum et Romanorum Teubneriana). Leipzig: Teubner.

Kircher, Athanasius. 1679. \textit{Turris} \textit{Babel,} \textit{siue} \textit{Archontologia} [...]. Amstelodami: ex officina Ianssonio-Waesbergiana.

Kirchmaier, Georg Caspar [praeses] \& Johannes Crusius [respondens]. 1684. \textit{De} \textit{originibus} \textit{et} \textit{causis} \textit{litteraturae} \textit{Graecae} \textit{eiusdemque} \textit{dialectis} [...]. Wittenbergae: typis Christiani Schrödteri.

Kirchmaier, Georg Wilhelm [praeses] \& Christian Gottlieb Schwartz [respondens]. 1702. \textit{Ex} \textit{philologia} \textit{Graeca,} \textit{de} \textit{dialecto} \textit{Noui} \textit{Testamenti,} \textit{propositiones} \textit{quaedam,} \textit{ad} \textit{disputandum} \textit{publice} \textit{selectae} [...]. Vitembergae: prelo Gerdesiano.

Kirchmaier, Georg Wilhelm [praeses] \& Georg Friedrich Thryllitsch [respondens]. 1709. \textit{Eruditorum} \textit{de} \textit{dialecto} \textit{Graecorum} \textit{communi} \textit{sententias} [...] \textit{excutiendas} \textit{proponet} [...] \textit{Thryllitius}. Witembergae: litteris Christiani Schroedteri.

Köber, Johann Friedrich. 1701. \textit{Grammatica} \textit{Graeca} \textit{harmonica} \textit{Golio-Welleriana} [...]. Lipsiae: apud Thomam Fritsch.

Koen, Gijsbert. 1766. Praefatio ad aequum et eruditum lectorem. \textit{$\Gamma} \textit{\rho} \textit{\eta} \textit{\gamma} \textit{o\rho} \textit{\text{\textgreek{'i}}o\upsilon $, $\mu \eta \tau \rho o\pi o\lambda \text{\textgreek{'i}}\tau o\upsilon $ $Ko\rho \text{\textgreek{'i}}\nu \theta o\upsilon $, $\Pi \varepsilon \rho \text{\textgreek{`i}}$ $\delta \iota \alpha \lambda \text{\textgreek{'e}}\kappa \tau \omega \nu $ {\textbar} Gregorius, Corinthi metropolita, De dialectis. E codicibus mss. emendauit et notis illustrauit Gisbertus Koen [...]}, \textsc{v}–\textsc{xliv}. Lugduni Batauorum: apud Petrum van der Eyk et Cornelium de Pecker.

Kritopoulos [$K\rho \iota \tau \text{\textgreek{'o}}\pi o\upsilon \lambda o\varsigma $], Metrophanes [$M\eta \tau \rho o\varphi \text{\textgreek{'a}}\nu \eta \varsigma $]. [1924/]1926 [1627]. $M\eta \tau \rho o\varphi \text{\textgreek{'a}}\nu o\upsilon \varsigma $ $K\rho \iota \tau o\pi o\text{\textgreek{'u}}\lambda o\upsilon $ $\text{\textgreek{>a}}\nu \text{\textgreek{'e}}\kappa \delta o\tau o\varsigma $ $\gamma \rho \alpha \mu \mu \alpha \tau \iota \kappa \text{\textgreek{`h}}$ $\tau \text{\textgreek{~h}}\varsigma $ $\text{\textgreek{<a}}\pi \lambda \text{\textgreek{~h}}\varsigma $ $\text{\textgreek{<e}}\lambda \lambda \eta \nu \iota \kappa \text{\textgreek{~h}}\varsigma $ $\text{\textgreek{>e}}\kappa \delta \iota \delta o\mu \text{\textgreek{'e}}\nu \eta $. (Ed.) Konstantinos I. [$K\omega \nu \sigma \tau \alpha \nu \tau \text{\textgreek{'i}}\nu o\varsigma $ $\text{\textgreek{>I}}$.] Dyovouniotis [$\Delta \upsilon o\beta o\upsilon \nu \iota \text{\textgreek{'w}}\tau \eta \varsigma $]. \textit{$\text{\textgreek{>E}}\pi \iota \sigma \tau \eta \mu o\nu \iota \kappa \text{\textgreek{`h}}$ $\text{\textgreek{>e}}\pi \varepsilon \tau \eta \rho \text{\textgreek{`i}}\varsigma $ $\tau \text{\textgreek{~h}}\varsigma $ $\Theta \varepsilon o\lambda o\gamma \iota \kappa \text{\textgreek{~h}}\varsigma $ $\Sigma \chi o\lambda \text{\textgreek{~h}}\varsigma $ $\tau o\text{\textgreek{~u}}$ $\text{\textgreek{>A}}\theta \text{\textgreek{'h}}\nu \eta \sigma \iota $ $\Pi \alpha \nu \varepsilon \pi \iota \sigma \tau \eta \mu \text{\textgreek{'i}}o\upsilon $} 1. 97–123.

Labbe, Philippe. 1639. \textit{Regulae} \textit{accentuum} \textit{et} \textit{spirituum} \textit{Graecorum,} \textit{nouo} \textit{ordine} \textit{in} \textit{faciliores} \textit{et} \textit{difficiliores,} \textit{pro} \textit{captu} \textit{scholasticorum,} \textit{distributae:} \textit{Quibus} \textit{additae} \textit{sunt} \textit{nonnullae} \textit{obseruationes,} \textit{omnibus} \textit{Graecae} \textit{linguae} \textit{studiosis} \textit{utilissimae.} \textit{Item} \textit{dialecti} \textit{apud} \textit{oratores} \textit{usurpatae} \textit{a} \textit{poeticis} \textit{seiunctae} [...]. Tertia hac editione accesserunt regulae syntaxeos Graecae. Parisiis: apud Mathurinum Henault.

Lagerlööf, Petrus [praeses] \& Johannes Palmroot [respondens]. 1685. \textit{Historiola} \textit{linguae} \textit{Graecae} [...]. Vpsaliae: excudit Henricus Curio.

Lakemacher, Johann Gottfried. 1730. \textit{Obseruationes} \textit{philologicae} \textit{quibus} \textit{uaria} \textit{praecipue} \textit{S.} \textit{Codicis} \textit{loca} \textit{ex} \textit{antiquitatibus} \textit{illustrantur:} \textit{Pars} \textit{IV.} \textit{V.} \textit{et} \textit{VI} \textit{cum} \textit{indicibus}. Helmstadii: impensis Christ. Fried. Weygandi.

Lancelot, Claude. 1655. \textit{Nouvelle} \textit{methode} \textit{pour} \textit{apprendre} \textit{facilement} \textit{la} \textit{langue} \textit{Greque} [...]. À Paris: de l’Imprimerie d’Antoine Vitré, chez Pierre le Petit.

Larramendi, Manuel de. 1728. \textit{De} \textit{la} \textit{antiguedad} \textit{y} \textit{universalidad} \textit{del} \textit{bascuenze} \textit{en} \textit{España,} \textit{de} \textit{sus} \textit{perfecciones} \textit{y} \textit{ventajas} \textit{sobre} \textit{otras} \textit{muchas} \textit{lenguas,} \textit{demonstracion} \textit{previa} \textit{al} \textit{arte,} \textit{que} \textit{se} \textit{dará} \textit{a} \textit{luz} \textit{desta} \textit{lengua}. En Salamanca: por Eugenio Garcia de Honorato.

Larramendi, Manuel de. 1729. \textit{El} \textit{impossible} \textit{vencido:} \textit{Arte} \textit{de} \textit{la} \textit{lengua} \textit{bascongada}. En Salamanca: por Antonio Joseph Villargordo Alcaráz.

Le Clerc, Jean. 1696. Dissertatio de lingua Hebraica. \textit{Pentateuchus,} \textit{siue} \textit{Mosis} \textit{prophetae} \textit{libri} \textit{quinque.} [...], b.1\textsc{\textsuperscript{r}}–c.2\textsc{\textsuperscript{v}}. Amstelodami: sumptibus auctoris, prostat apud Henricum Wetstenium.

Le Fevre, Tanneguy. 1731 [1672]. \textit{Méthode} \textit{pour} \textit{commencer} \textit{les} \textit{humanités} \textit{Grecques} \textit{et} \textit{Latines} [...]. Avec des notes et des lettres sur la maniere de les enseigner dans les colleges, par M. Gaullyer [...]. A Paris: chez la veuve Jean Baptiste Brocas [...] et Claude Simon [...].

Leibniz, Gottfried Wilhelm (von). 1991 [ca. 1712]. Epistolica de historia etymologica dissertatio. \textit{Il} \textit{naturale} \textit{e} \textit{il} \textit{simbolico:} \textit{Saggio} \textit{su} \textit{Leibniz} (Biblioteca di cultura 436), 201–271. Roma: Bulzoni.

Leigh, Edward. 1656. \textit{A} \textit{treatise} \textit{of} \textit{religion} \textit{and} \textit{learning,} \textit{and} \textit{of} \textit{religious} \textit{and} \textit{learned} \textit{men:} \textit{Consisting} \textit{of} \textit{six} \textit{books}. London: printed by A. M. for Charles Adams.

Lentz, Augustus (ed.). 1870. \textit{Herodiani} \textit{technici} \textit{reliquiae} (Grammatici Graeci 3.2). 2 vols. Lipsiae: in aedibus B. G. Teubneri.

Leusden, Johannes. 1656. \textit{Philologus} \textit{Hebraeus,} \textit{in} \textit{quo} \textit{pleraeque} \textit{quaestiones} \textit{generales} \textit{philologico-Hebraicae,} \textit{concernentes} \textit{textum} \textit{Hebraeum} \textit{Veteris} \textit{Test.} \textit{dilucide} \textit{pertractantur}. Vltraiecti: apud Meinardum à Dreunen.

Leusden, Johannes. 1670. Dissertatio undecima, de dialectis Noui T. in genere et in specie. \textit{Philologus} \textit{Hebraeo-Graecus} \textit{generalis,} \textit{continens} \textit{quaestiones} \textit{Hebraeo-Graecas,} \textit{quae} \textit{circa} \textit{Nouum} \textit{Testamentum} \textit{Graecum} \textit{fere} \textit{moueri} \textit{solent}, 83–91. Vltraiecti: ex officina Anthonii Smytegelt.

Liburnio, Niccolò. 1546. \textit{Le} \textit{occorrenze} \textit{humane}. In Vinegia: in casa de’ figliuoli di Aldo.

Lopad, Ludwig. 1536. \textit{Graecae} \textit{linguae} \textit{rudimenta,} \textit{praeter} \textit{hactenus} \textit{edita} \textit{singulari} \textit{et} \textit{diligentia} \textit{et} \textit{breuitate} \textit{in} \textit{puerorum} \textit{usum} \textit{conscripta}. Basileae: [Thomas Platter \& Balthasar Lasius].

Löscher, Valentin Ernst. 1705. \textit{Ion,} \textit{siue} \textit{Originum} \textit{Graeciae} \textit{restauratarum} \textit{ad} \textit{uirum} \textit{celeberrimum,} \textit{D.} \textit{Ottonem} \textit{Menkenium} \textit{libri} \textit{II:} \textit{In} \textit{quibus} \textit{ad} \textit{penetralia} \textit{Graecae} \textit{antiquitatis,} \textit{contra} \textit{praeiudicatas} \textit{ueterum} \textit{ac} \textit{recentiorum} \textit{opiniones,} \textit{aditus} \textit{conciliatur}. Lipsiae \& Delitii: sumptibus heredum Grossianorum, [...] excudebat Christianus Kobersteinius.

Luscinius, Ottomarus. 1517. \textit{Progymnasmata} \textit{Graecanicæ} \textit{literaturæ}. Argentorati: Excusum [...] typis solertis uiri Ioanni Knoblouch.

MacNicol, Donald. 1779. \textit{Remarks} \textit{on} \textit{Dr.} \textit{Samuel} \textit{Johnson’s} \textit{Journey} \textit{to} \textit{the} \textit{Hebrides:} \textit{In} \textit{which} \textit{are} \textit{contained,} \textit{observations} \textit{on} \textit{the} \textit{antiquities,} \textit{language,} \textit{genius,} \textit{and} \textit{manners} \textit{of} \textit{the} \textit{Highlanders} \textit{of} \textit{Scotland}. London: printed for T. Cadell.

Mączyński, Jan. 1564. \textit{Lexicon} \textit{Latinopolonicum} \textit{ex} \textit{optimis} \textit{Latinae} \textit{linguae} \textit{scriptoribus} \textit{concinnatum}. Regiomonti Borussiae: excudebat [...] Ioannes Daubmann.

Maittaire, Michael. 1706. \textit{Graecae} \textit{linguae} \textit{dialecti:} \textit{In} \textit{usum} \textit{Scholae} \textit{Westmonasteriensis}. Londini: typis Edm. Powell, impensis Eliz. Bennet.

Malcolm, David. 1738. \textit{An} \textit{essay} \textit{on} \textit{the} \textit{antiquities} \textit{of} \textit{Great} \textit{Britain} \textit{and} \textit{Ireland}. Edinburgh: printed by T. and W. Ruddimans.

Mambrun, Pierre. 1661. \textit{Opera} \textit{poetica:} \textit{Accessit} \textit{dissertatio} \textit{de} \textit{epico} \textit{carmine}. Fixae Andecauorum: ex officina Geruasii Laboe.

Manutius, Aldus. 1496. Aldus Manutius Basianas Romanus studiosis omnibus S. P. D. In Aldus Manutius \textit{et} \textit{al.} (eds.), \textit{$\Theta} \textit{\eta} \textit{\sigma} \textit{\alpha} \textit{\upsilon} \textit{\rho} \textit{\text{\textgreek{'o}}\varsigma $. $K\text{\textgreek{'e}}\rho \alpha \varsigma $ $\text{\textgreek{>a}}\mu \alpha \lambda \theta \varepsilon \text{\textgreek{'i}}\alpha \varsigma $ $\kappa \alpha \text{\textgreek{`i}}$ $\kappa \text{\textgreek{~h}}\pi o\iota $ $\text{\textgreek{>A}}\delta \text{\textgreek{'w}}\nu \iota \delta o\varsigma $ {\textbar} Thesaurus cornu copiae et horti Adonidis}, *.ii\textsc{\textsuperscript{r}}–*.iii\textsc{\textsuperscript{v}}. Venetiis: in domo Aldi Romani.

Manutius, Aldus, \textit{et} \textit{al.} (eds.). 1496. \textit{$\Theta} \textit{\eta} \textit{\sigma} \textit{\alpha} \textit{\upsilon} \textit{\rho} \textit{\text{\textgreek{'o}}\varsigma $. $K\text{\textgreek{'e}}\rho \alpha \varsigma $ $\text{\textgreek{>a}}\mu \alpha \lambda \theta \varepsilon \text{\textgreek{'i}}\alpha \varsigma $, $\kappa \alpha \text{\textgreek{`i}}$ $\kappa \text{\textgreek{~h}}\pi o\iota $ $\text{\textgreek{>A}}\delta \text{\textgreek{'w}}\nu \iota \delta o\varsigma $. Thesaurus Cornu copiae et horti Adonidis}. Venetiis: in domo Aldi Romani.

Marineo Sículo, Lucio. [ca. 1497]. \textit{[De} \textit{Hispaniae} \textit{laudibus]}. [Burgos]: [Fadrique de Basilea].

Martin, Benjamin. 1737. \textit{Bibliotheca} \textit{technologica,} \textit{or} \textit{A} \textit{philological} \textit{library} \textit{of} \textit{literary} \textit{arts} \textit{and} \textit{sciences}. London: printed by S. Idle for John Noon.

Martin, Jacques. 1727. \textit{La} \textit{religion} \textit{des} \textit{Gaulois,} \textit{tirée} \textit{des} \textit{plus} \textit{pures} \textit{sources} \textit{de} \textit{l’antiquité}. 2 vols. A Paris: chez Saugrain Fils.

Mayer, Bartholomaeus. 1629. \textit{Philologiae} \textit{sacrae} \textit{pars} \textit{prima} \textit{continens} \textit{prodromum} \textit{Chaldaismi} \textit{sacri,} \textit{in} \textit{quo} \textit{eiusdem} \textit{causa} \textit{eruitur,} \textit{ac} \textit{sylloge} \textit{uocabulorum} \textit{Aegyptiacorum,} \textit{Graecorum} \textit{et} \textit{Latinorum,} \textit{quae} \textit{in} \textit{Veteris} \textit{Instrumenti} \textit{authentico} \textit{codice,} \textit{partim} \textit{reuera,} \textit{partim} \textit{opinione} \textit{quorundam} \textit{habentur,} \textit{exhibetur} [...]. Lipsiae: sumptibus Gothofredi Grosii [...]; excudebat Johann-Albertus Minzelius.

Mazzarella-Farao, Francesco. 1779. \textit{La} \textit{Neoellenopedia,} \textit{o} \textit{sia} \textit{Il} \textit{nuovo} \textit{metodo} \textit{per} \textit{erudire} \textit{la} \textit{gioventù} \textit{nel} \textit{greco} \textit{linguaggio}. Vol. 1. Napoli: nella stamperia Porsiliana.

Mazzocchi, Alessio Simmaco. 1754. \textit{Commentariorum} \textit{in} \textit{regii} \textit{Herculanensis} \textit{musei} \textit{aeneas} \textit{tabulas} \textit{Heracleenses} \textit{pars} \textit{I}. Neapoli: ex officina Benedicti Gessari.

Megiser, Hieronymus. 1603. \textit{Thesaurus} \textit{polyglottus,} \textit{siue} \textit{Dictionarium} \textit{multilingue,} \textit{ex} \textit{quadringentis} \textit{circiter} \textit{tam} \textit{ueteris,} \textit{quam} \textit{noui} \textit{(uel} \textit{potius} \textit{antiquis} \textit{incogniti)} \textit{orbis} \textit{nationum} \textit{linguis,} \textit{dialectis,} \textit{idiomatibus} \textit{et} \textit{idiotismis} \textit{constans}. Francofurti ad Moenum: sumptibus auctoris.

Meisner, Christian [auctor et respondens]. 1705. \textit{Silesiam} \textit{loquentem} [...] \textit{praeside} \textit{Conrado} \textit{Samuele} \textit{Schurzfleischio} [...] \textit{protulit} [...] \textit{Christianus} \textit{Meisnerus} [...]. Vitembergae: litteris Schulzianis.

Melanchthon, Philipp. 1518. \textit{Institutiones} \textit{Graecae} \textit{grammaticae}. Tubingae \& Hagnoae: ex Academia Anshelmiana.

Melanchthon, Philipp. 1520. \textit{Integrae} \textit{Graecae} \textit{grammatices} \textit{institutiones,} [...] \textit{pluribus} \textit{in} \textit{locis} \textit{auctae}. Haganoae: in aedibus Thomae Anshelmi Badensis.

Mérigon, Pierre Bertrand. 1621. \textit{Facilis} \textit{et} \textit{compendiarius} \textit{tractatus} \textit{dialectorum} \textit{linguae} \textit{Graecae:} \textit{Vna} \textit{cum} \textit{tabulis} \textit{illarum,} \textit{quibus} \textit{accessit} \textit{alia} \textit{tabula} \textit{licentiam} \textit{poetarum} \textit{complectens,} \textit{perquam} \textit{utilis} \textit{studiosis} \textit{poeticae} \textit{lectionis}. Parisiis: sumptibus auctoris.

Merula, Paulus. 1605. \textit{Cosmographiae} \textit{generalis} \textit{libri} \textit{tres:} \textit{Item} \textit{Geographiae} \textit{particularis} \textit{libri} \textit{quattuor} [...]. Lugduni Batauorum: ex officina Plantiniana Raphelengi F.

Metzler, Johannes. 1529. \textit{Primae} \textit{grammatices} \textit{Graecae} \textit{partis} \textit{rudimenta}. Haganoae: per Ioannem Secerium.

Meursius, Johannes. 1661. \textit{Miscellanea} \textit{Laconica,} \textit{siue} \textit{Variarum} \textit{antiquitatum} \textit{Laconicarum} \textit{libri} \textit{IV:} \textit{Nunc} \textit{primum} \textit{editi} \textit{cura} \textit{Samuelis} \textit{Pufendorfii}. Amstelodami: apud Iudocum Pluymer.

Meursius, Johannes. 1675. \textit{Creta,} \textit{Cyprus,} \textit{Rhodus,} \textit{siue} \textit{De} \textit{nobilissimarum} \textit{harum} \textit{insularum} \textit{rebus} \textit{et} \textit{antiquitatibus} \textit{commentarii} \textit{postumi,} \textit{nunc} \textit{primum} \textit{editi}. Amstelodami: apud Abrahamum Wolfgangum.

Milner, John. 1734. \textit{A} \textit{practical} \textit{grammar} \textit{of} \textit{the} \textit{Greek} \textit{tongue:} \textit{Wherein} \textit{all} \textit{the} \textit{rules} \textit{are} \textit{express’d} \textit{in} \textit{English} [...]. London: printed for John Gray.

Monboddo [Burnett, James]. 1774. \textit{Of} \textit{the} \textit{origin} \textit{and} \textit{progress} \textit{of} \textit{language}. Second edition: With large additions and corrections. Vol. 1. 6 vols. Edinburgh: printed for J. Balfour [...] and T. Cadell [...].

Morhof, Daniel Georg. 1685. \textit{De} \textit{Patauinitate} \textit{Liuiana} \textit{liber:} \textit{Vbi} \textit{de} \textit{urbanitate} \textit{et} \textit{peregrinitate} \textit{sermonis} \textit{Latini} \textit{uniuerse} \textit{agitur}. Kiloni: litteris ac sumptibus Ioachimi Reumanni.

Morhof, Daniel Georg. 1708. \textit{Polyhistor,} \textit{in} \textit{tres} \textit{tomos} \textit{[…]} \textit{diuisus}. Opus posthumum [...]. Lubecae: sumptibus Petri Böckmanni.

Moritz, Karl Philipp. 1781. \textit{Ueber} \textit{den} \textit{märkischen} \textit{Dialekt:} \textit{In} \textit{Briefen}. Vol. 1. Berlin: bei Arnold Wever.

Mosellanus, Petrus. 1527. \textit{In} \textit{M.} \textit{Fab.} \textit{Quintiliani} \textit{Rhetoricas} \textit{institutiones} \textit{annotationes}. Basileae: apud Adamum Petrum.

Munthe, Caspar Frederik [praeses] \& Ludvig Heiberg [respondens]. 1748. \textit{Historia} \textit{Graecae} \textit{linguae} [...]. Vol. 1. Hafniae: typis [...] Ioannis Georg. Høpffneri.

Muratori, Lodovico Antonio \& Anton Maria Salvini. 1724. \textit{Della} \textit{perfetta} \textit{poesia} \textit{italiana} \textit{spiegata} \textit{e} \textit{dimostrata} \textit{con} \textit{varie} \textit{osservazioni} \textit{e} \textit{con} \textit{vari} \textit{giudizi} \textit{sopra} \textit{alcuni} \textit{componimenti} \textit{altrui} \textit{da} \textit{Lodovico} \textit{Antonio} \textit{Muratori} \textit{[...]:} \textit{Con} \textit{le} \textit{annotazioni} \textit{critiche} \textit{dell’Abate} \textit{Anton} \textit{Maria} \textit{Salvini} [...]. Vol. 2. 2 vols. In Venezia: appresso Sebastiano Coleti.

Mylius, Abraham. 1612. \textit{Lingua} \textit{Belgica,} \textit{siue} \textit{De} \textit{linguae} \textit{illius} \textit{communitate} \textit{tum} \textit{cum} \textit{plerisque} \textit{aliis,} \textit{tum} \textit{praesertim} \textit{cum} \textit{Latina,} \textit{Graeca,} \textit{Persica;} \textit{deque} \textit{communitatis} \textit{illius} \textit{causis;} \textit{tum} \textit{de} \textit{linguae} \textit{illius} \textit{origine} \textit{et} \textit{latissima} \textit{per} \textit{nationes} \textit{quamplurimas} \textit{diffusione;} \textit{ut} \textit{et} \textit{de} \textit{eius} \textit{praestantia:} \textit{Qua} \textit{tum} \textit{occasione,} \textit{hic} \textit{simul} \textit{quaedam} \textit{tractantur} \textit{consideratu} \textit{non} \textit{indigna,} \textit{ad} \textit{linguas} \textit{in} \textit{uniuersum} \textit{omnes} \textit{pertinentia}. Lugduni Batauorum: pro Bibliopolio Commeliniano, excudebant [...] Vlricus Cornelii et G. Abrahami.

Neander, Michael. 1553. \textit{Graecae} \textit{linguae} \textit{erotemata,} \textit{quae} \textit{hoc} \textit{ordine} \textit{explicata} \textit{complectuntur} [...]. Cum praefatione Philippi Melanchthonis. Basileae: per Ioannem Oporinum.

Neander, Michael. 1561. \textit{Graecae} \textit{linguae} \textit{erotemata,} \textit{quae} \textit{hoc} \textit{ordine} \textit{explicata} \textit{complectuntur} [...]. Basileae: per Ioannem Oporinum.

Nibbe, Johann Barthold. 1725. \textit{Dialectologia} \textit{paradigmatica} \textit{oder} \textit{Gründliche} \textit{Anweisung} \textit{zu} \textit{den} \textit{4.} \textit{Haupt-dialectis} \textit{der} \textit{Griechischen} \textit{Sprache} [...]. Neu-Brandenburg: verlegts Joh. George Schönborn.

Nibbe, Johann Barthold. 1755. \textit{Dialectologia} \textit{Graeca} \textit{in} \textit{tabellis} \textit{exhibens} \textit{mutationes} \textit{uocalium,} \textit{diphthongorum} \textit{et} \textit{consonantium} [...]. Rostochii et Wismariae: sumptibus Io. Andr. Bergeri et Iac. Boedneri.

Nicolson, William. 1715. Io. Chamberlaynio armigero: S. et O. In John Chamberlayne (ed.), \textit{Oratio} \textit{dominica} \textit{in} \textit{diuersas} \textit{omnium} \textit{fere} \textit{gentium} \textit{linguas} \textit{uersa} \textit{et} \textit{propriis} \textit{cuiusque} \textit{linguae} \textit{characteribus} \textit{expressa,} \textit{una} \textit{cum} \textit{dissertationibus} \textit{nonnullis} \textit{de} \textit{linguarum} \textit{origine} \textit{uariisque} \textit{ipsarum} \textit{permutationibus}, 1–21. Amstelaedami: typis Guilielmi et Dauidis Goerei.

Nikiforos, Romanos. 1908. \textit{Grammatica} \textit{linguae} \textit{Graecae} \textit{uulgaris} \textit{communis} \textit{omnibus} \textit{Graecis} \textit{ex} \textit{qua} \textit{alia} \textit{artificialis} \textit{deducitur} \textit{peculiaris} \textit{eruditis} \textit{et} \textit{studiosis} \textit{tantum} \textit{per} \textit{patrem} \textit{Romanum} \textit{Nicephori} \textit{Thessalonicensem,} \textit{Macedonem,} \textit{éditée} \textit{d’après} \textit{le} \textit{ms.} \textit{2604} \textit{(Fonds} \textit{grec)} \textit{de} \textit{la} \textit{B.} \textit{N.} \textit{de} \textit{Paris} (Bibliothèque de la Faculté de Philosophie et Lettres de l’Université de Liège 18). (Ed.) J. Boyens. Liège: H. Vaillant-Carmanne.

Núñez, Pedro Juan. 1555/1556. \textit{Institutiones} \textit{grammaticae} \textit{linguae} \textit{Graecae}. Valentiae: ex Ioannis Mey Flandri typographia.

Oberlin, Jérémie-Jacques. 1775. \textit{Essai} \textit{sur} \textit{le} \textit{patois} \textit{lorrain} \textit{des} \textit{environs} \textit{du} \textit{comté} \textit{du} \textit{Ban} \textit{de} \textit{la} \textit{Roche,} \textit{fief} \textit{royal} \textit{d’Alsace}. Strasbourg: chez Jean Fred. Stein.

Oecolampadius, Johannes. 1518. \textit{Dragmata} \textit{Graecae} \textit{litteraturae}. Basileae: ex aedibus Andreae Cratandri, et Seruatii Cruft.

Olearius, Johannes. 1668. \textit{Dissertatio} \textit{philologico-theologica} \textit{de} \textit{stylo} \textit{Noui} \textit{Testamenti} [...]. Lipsiae: sumptibus Ioannis Grosii et socii, litteris Ioannis Coleri.

Opitz, Heinrich. 1687. \textit{Graecismus} \textit{facilitati} \textit{suae} \textit{restitutus,} \textit{methodo} \textit{noua} \textit{eaque} \textit{cum} \textit{Orientalibus} \textit{suis} \textit{quam} \textit{proxima} \textit{harmonica} \textit{et} \textit{sic} \textit{regulis} \textit{XL} \textit{succincte,} \textit{sed} \textit{plene} \textit{absolutus} \textit{in} \textit{gratiam} \textit{auditorum} \textit{suorum} \textit{antehac} \textit{editus}. Nunc multorum desiderio secunda uice quantum per alias curas licuit recognitus. Lipsiae \& Francofurti: sumptibus Ioann. Casp. Meyeri.

Oreadini, Vincenzo. 1525. \textit{Opusculum} \textit{in} \textit{quo} \textit{agitur} \textit{utrum} \textit{adiectio} \textit{nouarum} \textit{litterarum} \textit{Italicae} \textit{linguae} \textit{aliquam} \textit{utilitatem} \textit{pepererit}. Perusiae: in aedibus Hieronymi Francisci Chartularii.

Palsgrave, John. 1530. \textit{Lesclarcissement} \textit{de} \textit{la} \textit{langue} \textit{francoyse}. [S.l.]: the imprintyng fynysshed by Johan Haukyns.

Parr, Richard. 1686. \textit{The} \textit{life} \textit{of} \textit{the} \textit{most} \textit{reverend} \textit{father} \textit{in} \textit{God,} \textit{James} \textit{Usher,} \textit{late} \textit{lord} \textit{arch-bishop} \textit{of} \textit{Armagh,} \textit{primate} \textit{and} \textit{metropolitan} \textit{of} \textit{all} \textit{Ireland}. London: printed for Nathanael Ranew.

Pasor, Georg. 1632. Idea Graecarum Noui Testamenti dialectorum. \textit{Syllabus} \textit{Graeco-Latinus} \textit{omnium} \textit{Noui} \textit{Testamenti} \textit{uocum,} \textit{ubi} \textit{in} \textit{praefatione} \textit{agitur} \textit{de} \textit{singulari} \textit{compendio} \textit{discendi} \textit{linguam} \textit{Graecam:} \textit{Cui} \textit{adiecta} \textit{est} \textit{Idea} \textit{utilis} \textit{et} \textit{necessaria,} \textit{de} \textit{septem} \textit{Noui} \textit{Testamenti} \textit{dialectis}. Amsterodami \& Franequerae: apud Ianssonium \& apud Heynsium.

Pasor, Georg. 1650. Idea Graecarum Noui Testamenti dialectorum. \textit{Syllabus} \textit{Graeco-Latinus} \textit{omnium} \textit{Noui} \textit{Testamenti} \textit{uocum,} \textit{quae} \textit{ordine} \textit{alphabetico} \textit{recensentur:} \textit{Vbi} \textit{in} \textit{praefatione} \textit{agitur} \textit{de} \textit{singulari} \textit{compendio} \textit{discendi} \textit{linguam} \textit{Graecam:} \textit{Cui} \textit{adiecta} \textit{est} \textit{Idea} \textit{utilis} \textit{et} \textit{necessaria,} \textit{de} \textit{septem} \textit{Graecis} \textit{Noui} \textit{Testamenti} \textit{dialectis}, 143–178. De nouo diligenter recognita et edita. Amstelodami: apud Ioannem Ianssonium.

Perotti, Niccolò. 1489. \textit{[Cornucopia} \textit{seu} \textit{commentarii} \textit{linguae} \textit{Latinae]}. Venetiis: per magistrum Paganinum de Paganinis Brixiensem.

Peternader, Leo. 1776. \textit{Einleitung} \textit{zur} \textit{griechischen} \textit{Sprache} \textit{für} \textit{die} \textit{Kremsmünsterischen} \textit{Schulen}. Steyr: gedruckt bey Abraham Wimmer.

Petisco, Joseph. 1764 [1759]. \textit{Gramatica} \textit{griega}. En Villagarcia: en la Imprenta del Seminario.

Pfeiffer, August [praeses] \& Johann Georg Martini [respondens]. 1663. \textit{Diatribe} \textit{philologica} \textit{de} \textit{lingua} \textit{Galilaea,} \textit{per} \textit{quam} \textit{D.} \textit{Petrus} \textit{agnitus} \textit{fuisse} \textit{legitur} [...]. Wittebergae: typis Ioannis Haken.

Phillips, Edward. 1658. \textit{The} \textit{new} \textit{world} \textit{of} \textit{English} \textit{words,} \textit{or} \textit{A} \textit{general} \textit{dictionary:} \textit{Containing} \textit{the} \textit{interpretations} \textit{of} \textit{such} \textit{hard} \textit{words} \textit{as} \textit{are} \textit{derived} \textit{from} \textit{other} \textit{languages} [...]. London: printed by E. Tyler, for Nath. Brooke.

Polier de Bottens, Antoine-Noé [auctor et defendens]. 1739. \textit{Dissertatio} \textit{philologica} \textit{qua} \textit{disquiritur} \textit{de} \textit{puritate} \textit{dialecti} \textit{Arabicae,} \textit{comparate} \textit{cum} \textit{puritate} \textit{dialecti} \textit{Hebraeae,} \textit{in} \textit{relatione} \textit{ad} \textit{antediluuianam} \textit{linguam} [...]. Lugduni in Batauis: apud Ioannem Luzac.

Poliziano, Angelo. 1553. Oratio in expositione Homeri. \textit{Opera,} \textit{quae} \textit{quidem} \textit{extitere} \textit{hactenus,} \textit{omnia,} \textit{longe} \textit{emendatius} \textit{quam} \textit{usquam} \textit{antehac} \textit{expressa} [...], 477–492. Basileae: apud Nicolaum Episcopium Iuniorem.

Polo, Marco. 1938. \textit{The} \textit{description} \textit{of} \textit{the} \textit{world}. (Trans.) A. C. Moule \& Paul Pelliot. Vol. 1. London: George Routledge \& Sons.

Portus, Aemilius. 1603. \textit{$\Lambda} \textit{\varepsilon} \textit{\xi} \textit{\iota} \textit{\kappa} \textit{\text{\textgreek{`o}}\nu $ $\text{\textgreek{>i}}\omega \nu \iota \kappa \text{\textgreek{`o}}\nu $ $\text{\textgreek{<e}}\lambda \lambda \eta \nu o\rho \rho \omega \mu \alpha \iota \kappa \text{\textgreek{`o}}\nu $, hoc est Dictionarium Ionicum Graecolatinum [...]}. Francofurti: ex officina Paltheniana, sumptibus heredum Petri Fischeri.

Posselius, Johannes. 1599. \textit{Euangelia} \textit{et} \textit{epistolae,} \textit{quae} \textit{diebus} \textit{Dominicis} \textit{et} \textit{festis} \textit{sanctorum} \textit{in} \textit{ecclesia,} \textit{usitato} \textit{more} \textit{proponi} \textit{solent,} \textit{Graecis} \textit{uersibus} \textit{reddita} \textit{et} \textit{postremo} \textit{diligenter} \textit{multis} \textit{in} \textit{locis} \textit{recognita}. Lipsiae: sumptibus Henningi Grosii.

Prideaux, Humphrey (ed.). 1676. \textit{Marmora} \textit{Oxoniensia,} \textit{ex} \textit{Arundellianis,} \textit{Seldenianis} \textit{aliisque} \textit{conflata}. Oxonii: e theatro Sheldoniano.

Priestley, Joseph. 1762. \textit{A} \textit{course} \textit{of} \textit{lectures} \textit{on} \textit{the} \textit{theory} \textit{of} \textit{language} \textit{and} \textit{universal} \textit{grammar}. Warrington: printed by W. Eyres.

Primatt, William. 1764. \textit{Accentus} \textit{rediuiui,} \textit{or} \textit{A} \textit{defence} \textit{of} \textit{an} \textit{accented} \textit{pronunciation} \textit{of} \textit{Greek} \textit{prose,} \textit{shewing} \textit{it} \textit{to} \textit{be} \textit{conformable} \textit{to} \textit{all} \textit{antiquity}. Cambridge: printed by J. Bentham [...], for the author.

Pseudacro. 1902–1904. \textit{Scholia} \textit{in} \textit{Horatium} \textit{uetustiora}. (Ed.) Otto Keller. 2 vols. Lipsiae: in aedibus B. G. Teubneri.

Purchas, Samuel. 1613. \textit{Purchas} \textit{his} \textit{pilgrimage,} \textit{or} \textit{Relations} \textit{of} \textit{the} \textit{world} \textit{and} \textit{the} \textit{religions} \textit{observed} \textit{in} \textit{all} \textit{ages} \textit{and} \textit{places} \textit{discovered,} \textit{from} \textit{the} \textit{creation} \textit{unto} \textit{this} \textit{present:} \textit{In} \textit{foure} \textit{partes}. London: printed by William Stansby for Henrie Fetherstone.

Rainolds, William. 1583. \textit{A} \textit{refutation} \textit{of} \textit{sundry} \textit{reprehensions,} \textit{cavils,} \textit{and} \textit{false} \textit{sleightes,} \textit{by} \textit{which} \textit{M.} \textit{Whitaker} \textit{laboureth} \textit{to} \textit{deface} \textit{the} \textit{late} \textit{English} \textit{translation,} \textit{and} \textit{Catholike} \textit{annotations} \textit{of} \textit{the} \textit{New} \textit{Testament}. Paris: [?Richard Verstegan].

Ramus, Petrus. 1560. \textit{Grammatica} \textit{Graeca,} \textit{quatenus} \textit{a} \textit{Latina} \textit{differt}. Parisiis: apud Andream Wechelum.

Rapin, René. 1659. \textit{Eclogae} \textit{cum} \textit{dissertatione} \textit{de} \textit{carmine} \textit{pastorali}. Parisiis: apud Sebastianum Cramoisy.

Rapin, René. 1684. \textit{The} \textit{Idylliums} \textit{of} \textit{Theocritus} \textit{with} \textit{Rapin’s} \textit{Discourse} \textit{of} \textit{pastorals} \textit{done} \textit{into} \textit{English}. Oxford: printed by L. Lichfield [...] for Anthony Stephens.

Rask, Rasmus. 2013. \textit{Investigation} \textit{of} \textit{the} \textit{origin} \textit{of} \textit{the} \textit{Old} \textit{Norse} \textit{or} \textit{Icelandic} \textit{language} (Amsterdam classics in linguistics, 1800–1925 18). New edition of the 1993 English translation by Niels Ege, with an introduction by Frans Gregersen. Amsterdam \& Philadelphia: John Benjamins.

Rastell, John. 1566. \textit{A} \textit{treatise} \textit{intitled,} \textit{beware} \textit{of} \textit{M.} \textit{Jewel}. Antuerpiae: ex officina Ioannis Fouleri.

Rau, Sebald. 1770. \textit{Oratio} \textit{de} \textit{iudicio} \textit{in} \textit{philologia} \textit{Orientali} \textit{regundo,} \textit{dicta} \textit{publice} \textit{die} \textit{11.} \textit{Aprilis} \textit{MCCLXX} [sic pro \textit{MDCCLXX]}. Traiecti ad Rhenum: ex officina Ioannis Broedelet.

Ravis, Christian. 1646. \textit{Orthographiae} \textit{et} \textit{analogiae} \textit{(uulgo} \textit{etymologiae)} \textit{Ebraicae} \textit{delineatio} \textit{iuxta} \textit{uocis} \textit{partes} \textit{abstractas} \textit{I.} \textit{consonas.} \textit{II.} \textit{uocales.} \textit{III.} \textit{accentus,} \textit{qua} \textit{uia} \textit{centenae} \textit{singularum} \textit{anomaliae} \textit{in} \textit{analogiam} \textit{conuertuntur}. Amstelodami: apud Ioannem Ianssonium.

Ravis, Christian. 1650. \textit{A} \textit{generall} \textit{grammer} \textit{for} \textit{the} \textit{ready} \textit{attaining} \textit{of} \textit{the} \textit{Ebrew,} \textit{Samaritan,} \textit{Calde,} \textit{Syriac,} \textit{Arabic,} \textit{and} \textit{the} \textit{Ethiopic} \textit{languages:} \textit{With} \textit{a} \textit{pertinent} \textit{discourse} \textit{of} \textit{the} \textit{orientall} \textit{tongues}. London: printed by W. Wilson for Tho. Slater and Tho. Humington.

Reinhard, Lorenz. 1724. \textit{Historia} \textit{Graecae} \textit{linguae} \textit{critico-litteraria} \textit{uiam} \textit{pandens} \textit{ad} \textit{philologiam} \textit{Graecam}. Lipsiae: sumpt. heredum Io. Friderici Braunii.

Reitz, Wilhelm Otto. 1730. \textit{Belga} \textit{Graecissans}. Rotterodami: apud Io. Hofhout.

Reyher, Andreas. 1634. \textit{Synopsis} \textit{grammaticae} \textit{Graecae} \textit{in} \textit{lectiones} \textit{XXIV.} \textit{secta,} \textit{cui} \textit{praemissa} \textit{est} \textit{praefatio} \textit{docendae} \textit{discendaeque} \textit{huius} \textit{linguae} \textit{modum} \textit{breuiter} \textit{adumbrans:} \textit{Pro} \textit{illustri} \textit{Gymnasio} \textit{et} \textit{Scholis} \textit{Hennebergicis} \textit{ex} \textit{iussu} \textit{et} \textit{auctoritate} \textit{superiorum} \textit{conscripta}. Schleusingae: excusa typis Steinmannianis.

Reynolds, John. 1752. \textit{$\text{\textgreek{<H}}\rho \text{\textgreek{'o}}\delta o\tau o\varsigma $ $\text{\textgreek{<o}}$ $\text{\textgreek{<A}}\lambda \iota \kappa \alpha \rho \nu \alpha \sigma \sigma \varepsilon \text{\textgreek{`u}}\varsigma $: $\Pi \varepsilon \rho \text{\textgreek{`i}}$ $\text{\textgreek{<O}}\mu \text{\textgreek{'h}}\rho o\upsilon $ $\gamma \varepsilon \nu \text{\textgreek{'e}}\sigma \iota o\varsigma $, $\kappa \alpha \text{\textgreek{`i}}$ $\text{\textgreek{<h}}\lambda \iota \kappa \text{\textgreek{'i}}\eta \varsigma $, $\kappa \alpha \text{\textgreek{`i}}$ $\beta \iota o\tau \text{\textgreek{~h}}\varsigma $. Et item Historia Graecarum et Latinarum litterarum}. Etonae: ueneunt [...] apud Iosephum Pott.

Rhenius, Johannes. 1626. \textit{Graeca} \textit{grammatica} \textit{maior,} \textit{continens} \textit{doctrinam} \textit{accentuum,} \textit{dialectorum,} \textit{syntaxeos} \textit{et} \textit{prosodiae:} \textit{Accurata} \textit{diligentia} \textit{et} \textit{quanta} \textit{fieri} \textit{potuit} \textit{perspicuitate} \textit{elaboratam} \textit{in} \textit{usum} \textit{adultiorum}. Lipsiae: sumptibus Zachariae Schüreri iun. Matthiae Götzii et Friderici Lanckisch, excudebat Fridericus Lanckisch.

Rhodoman, Lorenz. 1605. \textit{Oratio} \textit{de} \textit{lingua} \textit{Graeca;} \textit{ut} \textit{ab} \textit{initio} \textit{huc} \textit{usque} \textit{propagata} \textit{sit} \textit{et} \textit{dehinc} \textit{etiam} \textit{propagari} \textit{queat.} [...] \textit{Item} \textit{oratio,} \textit{de} \textit{uita} \textit{philosophica} [...]. Argentinae: typis Antonii Bertrami.

Rice, John. 1765. \textit{An} \textit{introduction} \textit{to} \textit{the} \textit{art} \textit{of} \textit{reading} \textit{with} \textit{energy} \textit{and} \textit{propriety}. London: printed for J. and R. Tonson.

Richey, Michael. 1743. \textit{Idioticon} \textit{Hamburgense,} \textit{siue} \textit{Glossarium} \textit{uocum} \textit{Saxonicarum} \textit{quae} \textit{populari} \textit{nostra} \textit{dialecto} \textit{Hamburgi} \textit{maxime} \textit{frequentantur} [...]. Hamburgi: apud Conr. Koenigium.

Richey, Michael. 1755. \textit{Idioticon} \textit{Hamburgense,} \textit{oder} \textit{Wörter-Buch,} \textit{zur} \textit{Erklärung} \textit{der} \textit{eigenen,} \textit{in} \textit{und} \textit{üm} \textit{Hamburg} \textit{gebräuchlichen,} \textit{Nieder-Sächsischen} \textit{Mund-Art}. Jetzo vielfältig vermehret, und mit Anmerckungen und Zusätzen Zweener berühmten Männer, nebst einem Vierfachen Anhange, ausgefertiget. Hamburg: verlegt von Conrad König.

Ries, Daniel Christoph. 1786 [1782]. \textit{Lehrbuch} \textit{für} \textit{das} \textit{griechische} \textit{Sprachstudium}. Zweyte Auflage. Vol. 1. 2 vols. Mainz: verlegt auf Kösten des Schulfonds, gedruckt in der kurfürstl. privileg. Buchdruckerey des St. Rochus-Hospitals durch Andreas Craß.

Rijcke, Theodorus. 1684. Dissertatio de primis Italiae colonis et Aeneae aduentu. In Theodorus Rijcke (ed.), \textit{Notae} \textit{et} \textit{castigationes} \textit{posthumae} \textit{in} \textit{Stephani} \textit{Byzantii} \textit{$\text{\textgreek{>E}}\theta \nu \iota \kappa \text{\textgreek{`a}}$, quae uulgo $\Pi \varepsilon \rho \text{\textgreek{`i}}$ $\pi \text{\textgreek{'o}}\lambda \varepsilon \omega \nu $ inscribuntur: Post longam doctorum exspectationem editae [...]}, 399–467. Lugd. Batauorum: apud Iacobum Hackium.

Ringelbergh, Joachim Sterck van. 1541. \textit{Lucubrationes,} \textit{uel} \textit{potius} \textit{Absolutissima} \textit{$\kappa} \textit{\upsilon} \textit{\kappa} \textit{\lambda} \textit{o\pi} \textit{\alpha} \textit{\text{\textgreek{'i}}\delta \varepsilon \iota \alpha $: Nempe liber de ratione studii, utriusque linguae, grammatice, dialectice, rhetorice, mathematice et sublimioris philosophiae multa}. Basileae: apud Bartholomeum Westhemerum.

Rocca, Angelo. 1591. Appendix de dialectis, hoc est de uariis linguarum generibus. \textit{Bibliotheca} \textit{Apostolica} \textit{Vaticana} \textit{a} \textit{Sixto} \textit{V.} \textit{Pont.} \textit{Max.} \textit{in} \textit{splendidiorem} \textit{commodioremque} \textit{locum} \textit{translata} \textit{et} [...] \textit{commentario} \textit{uariarum} \textit{artium} \textit{ac} \textit{scientiarum} \textit{materiis} \textit{curiosis} \textit{ac} \textit{difficillimis} \textit{scituque} \textit{dignis} \textit{refertissimo,} \textit{illustrata}, 291–376. Romae: ex Typographia Apostolica Vaticana.

Rodigast, Samuel. 1685. \textit{Meletema} \textit{historico-philologicum} \textit{de} \textit{fatis} \textit{Graecae} \textit{linguae}. Ienae: apud Io. Bielkium [...]; typis Io. Dauidis Wertheri.

Rollin, Charles. 1726. \textit{De} \textit{la} \textit{maniere} \textit{d’enseigner} \textit{et} \textit{etudier} \textit{les} \textit{belles} \textit{lettres,} \textit{par} \textit{raport} \textit{à} \textit{l’esprit} \textit{et} \textit{au} \textit{cœur}. Vol. 1. A Paris: chez Jacques Estienne.

Rollin, Charles. 1731. \textit{Histoire} \textit{ancienne} \textit{des} \textit{Egyptiens,} \textit{des} \textit{Carthaginois,} \textit{des} \textit{Assyriens,} \textit{des} \textit{Babyloniens,} \textit{des} \textit{Medes} \textit{et} \textit{des} \textit{Perses,} \textit{des} \textit{Macedoniens,} \textit{des} \textit{Grecs}. Vol. 2. 13 vols. A Amsterdam: aux dépens de la Compagnie.

Rüdiger, Johann Christian Christoph. 1782. \textit{Grundriß} \textit{einer} \textit{Geschichte} \textit{der} \textit{menschlichen} \textit{Sprache} \textit{nach} \textit{allen} \textit{bisher} \textit{bekannten} \textit{Mund-} \textit{und} \textit{Schriftarten} \textit{mit} \textit{Proben} \textit{und} \textit{Bücherkenntniß}. Vol. 1 [\textit{Von} \textit{der} \textit{Sprache}]. Leipzig: bey P. G. Kummer.

Ruhig, Philipp. 1745. \textit{Betrachtung} \textit{der} \textit{littauischen} \textit{Sprache} \textit{in} \textit{ihrem} \textit{Ursprunge,} \textit{Wesen} \textit{und} \textit{Eigenschaften:} \textit{Aus} \textit{vielen} \textit{Scribenten} \textit{und} \textit{eigener} \textit{Erfahrung,} \textit{mit} \textit{Fleiß} \textit{angestellet} \textit{und} \textit{zu} \textit{reiferer} \textit{Beurtheilung} \textit{der} \textit{Gelehrten} \textit{zum} \textit{Druck} \textit{gegeben}. Königsberg: druckts und verlegts Johann Heinrich Hartung.

Ruland, Martin. 1556. \textit{De} \textit{lingua} \textit{Graeca} \textit{eiusque} \textit{dialectis} \textit{omnibus} \textit{libri} \textit{V}. Tiguri: apud Andream Gesnerum F. et Iacobum Gesnerum, fratres.

Sabellicus, Marcus Antonius. 1490. \textit{Suetonius} \textit{cum} \textit{commento}. Venetiis: per Baptistam de Tortis.

Sajnovics, János. 1770. \textit{Demonstratio} \textit{idioma} \textit{Vngarorum} \textit{et} \textit{Lapponum} \textit{idem} \textit{esse}. Hafniae: typis orphanotrophii regii, excudit Gerhard Giese Salicath.

Salviati, Leonardo. 1588. \textit{Lo’nfarinato} \textit{secondo} \textit{ovvero} \textit{Dello’nfarinato} \textit{accademico} \textit{della} \textit{Crusca,} \textit{risposta} \textit{al} \textit{libro} \textit{intitolato} Replica di Camillo Pellegrino ec. In Firenze: per Anton Padovani.

Santo Tomás, Domingo de. 1560. \textit{Grammatica,} \textit{o} \textit{Arte} \textit{de} \textit{la} \textit{lengua} \textit{general} \textit{de} \textit{los} \textit{Indios} \textit{de} \textit{los} \textit{Reynos} \textit{del} \textit{Peru:} \textit{Nuevamente} \textit{compuesta} [...]. En Valladolid: impresso [...] por Francisco Fernandez de Cordova.

Saumaise, Claude. 1643a. \textit{De} \textit{Hellenistica} \textit{commentarius,} \textit{controuersiam} \textit{de} \textit{lingua} \textit{Hellenistica} \textit{decidens} \textit{et} \textit{plenissime} \textit{pertractans} \textit{originem} \textit{et} \textit{dialectos} \textit{Graecae} \textit{linguae}. Lugduni Batauorum: ex officina Elseuiriorum.

Saumaise, Claude. 1643b. \textit{Funus} \textit{linguae} \textit{Hellenisticae,} \textit{siue} \textit{Confutatio} \textit{exercitationis} \textit{de} \textit{Hellenistis} \textit{et} \textit{lingua} \textit{Hellenistica}. Lugduni Batauorum: ex officina Ioannis Maire.

Scaliger, Joseph Justus. 1594. \textit{Epistola} \textit{de} \textit{uetustate} \textit{et} \textit{splendore} \textit{gentis} \textit{Scaligerae} \textit{et} \textit{Iul.} \textit{Caes.} \textit{Scaligeri} \textit{uita}. Lugduni Batauorum: ex officina Plantiniana.

Scaliger, Joseph Justus. 1610. Diatriba de Europaeorum linguis. In Isaac Casaubon (ed.), \textit{Opuscula} \textit{uaria} \textit{antehac} \textit{non} \textit{edita}, 119–122. Parisiis: apud Hadrianum Beys.

Scheller, Immanuel Johann Gerhard. 1772. \textit{Gedanken} \textit{von} \textit{den} \textit{Eigenschaften} \textit{der} \textit{deutschen} \textit{Schreibart} \textit{und} \textit{Empfehlungen} \textit{der} \textit{deutschen} \textit{Sprache} \textit{in} \textit{Predigten,} \textit{im} \textit{Reden} \textit{und} \textit{Schreiben,} \textit{bey} \textit{der} \textit{Philologie} \textit{und} \textit{in} \textit{Schulen}. Halle: gedruckt und verlegt von Joh. Jac. Curt.

Schmidt, Erasmus. 1604. \textit{Tractatus} \textit{de} \textit{dialectis} \textit{Graecorum} \textit{principalibus,} \textit{quae} \textit{sunt} \textit{in} \textit{parte} \textit{$\lambda} \textit{\text{\textgreek{'e}}\xi \varepsilon \omega \varsigma $: Cum rerum et uerborum indice locupletissimo}. Wittebergae: imprimebat Laurentius Seuberlich, impensis Samuel Selfisch.

Schmidt, Erasmus. 1615. Discursus de pronuntiatione Graeca, contra $N\varepsilon \text{\textgreek{'o}}\varphi \upsilon \tau o\nu $. \textit{Cyrilli,} \textit{uel} \textit{ut} \textit{alii} \textit{uolunt,} \textit{Ioannis} \textit{Philoponi} \textit{opusculum} \textit{utilissimum} \textit{De} \textit{differentiis} \textit{uocum} \textit{Graecarum} \textit{quoad} \textit{tonum,} \textit{spiritum,} \textit{genus} \textit{etc.} \textit{Plus} \textit{quintuplo} \textit{auctum} \textit{et} \textit{in} \textit{gratiam} \textit{$\tau} \textit{\text{\textgreek{~w}}\nu $ $\varphi \iota \lambda \varepsilon \lambda \lambda \text{\textgreek{'h}}\nu \omega \nu $ editum ab Erasmo Schmidt [...]}, 213–255. Witebergae: typis Richterianis, sumptibus B. Coruini.

Schoppe, Caspar. 1636. \textit{Consultationes} \textit{de} \textit{scholarum} \textit{et} \textit{studiorum} \textit{ratione} \textit{deque} \textit{prudentiae} \textit{et} \textit{eloquentiae} \textit{parandae} \textit{modis}. Patauii: apud Paulum Frambottum.

Schörling, Ernst Theophil [praeses] \& Georg Michaelis [respondens]. 1678. \textit{Exercitium} \textit{philologicum} \textit{de} \textit{Graecae} \textit{linguae} \textit{pronuntiatione} \textit{eiusdemque} \textit{uariis} \textit{dialectis}. Wittenbergae: typis Matthaei Henckelii.

Schottel, Justus Georg. 1663. \textit{Ausführliche} \textit{Arbeit} \textit{von} \textit{der} \textit{teutschen} \textit{Haubt} \textit{Sprache} [...] \textit{abgetheilet} \textit{in} \textit{5} \textit{Bücher}. Braunschweig: gedrukt und verlegt durch Christoff Friederich Zilligern.

Schröder, Nicolaus Wilhelm. 1748. \textit{Oratio} \textit{de} \textit{fundamentis,} \textit{quibus} \textit{solida} \textit{linguae} \textit{Hebraeae} \textit{cognitio} \textit{superstruenda,} \textit{habita} \textit{Groningae} \textit{Frisiorum,} \textit{A.} \textit{D.} \textit{XIV.} \textit{Iunii,} \textit{MDCCXLVIII}. Groningae: apud Georgium Spandaw.

Schultens, Albert. 1732. \textit{Oratio} \textit{altera} \textit{de} \textit{linguae} \textit{Arabicae} \textit{antiquissima} \textit{origine,} \textit{intima} \textit{ac} \textit{sororia} \textit{cum} \textit{lingua} \textit{Hebraea} \textit{cognatione} \textit{nullisque} \textit{saeculis} \textit{praeflorata} \textit{puritate:} \textit{Habita} \textit{a.} \textit{d.} \textit{20.} \textit{Iunii} \textit{a.} \textit{MDCCXXXII}. Lugduni Batauorum: apud Samuelem Luchtmans.

Schultens, Albert. 1737. \textit{Institutiones} \textit{ad} \textit{fundamenta} \textit{linguae} \textit{Hebraeae:} \textit{Quibus} \textit{uia} \textit{panditur} \textit{ad} \textit{eiusdem} \textit{analogiam} \textit{restituendam} \textit{et} \textit{uindicandam.} \textit{In} \textit{usum} \textit{collegii} \textit{domestici} [...]. Lugduni Batauorum: apud Ioannem Luzac.

Schultens, Albert. 1738a. \textit{Originum} \textit{Hebraearum} \textit{tomus} \textit{secundus:} \textit{Cum} \textit{uindiciis} \textit{tomi} \textit{primi,} \textit{nec} \textit{non} \textit{libri} \textit{de} \textit{defectibus} \textit{hodiernis} \textit{linguae} \textit{Hebraeae} \textit{aduersus} \textit{Cl.} \textit{dissertatorem}. Lugduni Batauorum: apud Samuelem Luchtmans.

Schultens, Albert. 1738b. \textit{Vetus} \textit{et} \textit{regia} \textit{uia} \textit{Hebraizandi,} \textit{asserta} \textit{contra} \textit{nouam} \textit{et} \textit{metaphysicam} \textit{hodiernam}. Lugduni Batauorum: apud Ioannem Luzac.

Schultens, Albert. 1739. \textit{Excursus} \textit{tertius} \textit{ad} \textit{editionem} \textit{primam} \textit{et} \textit{secundam} \textit{Dissertationis} \textit{de} \textit{lingua} \textit{primaeua} \textit{eiusque} \textit{additamentum} \textit{apologeticum}. Lugduni Batauorum: apud Ioannem Luzac.

Schultens, Albert. 1748. Praefatio. \textit{Prouerbia} \textit{Salomonis:} \textit{Versionem} \textit{integram} \textit{ad} \textit{Hebraeum} \textit{fontem} \textit{expressit} \textit{atque} \textit{commentarium} \textit{adiecit} \textit{Albertus} \textit{Schultens}, \textsc{i}–\textsc{cviii}. Lugduni Batauorum: apud Ioannem Luzac.

Schultens, Albert. 1769 [1706]. Dissertatio theologico-philologica de utilitate linguae Arabicae. \textit{Opera} \textit{minora,} \textit{animaduersiones} \textit{eius} \textit{in} \textit{Iobum,} \textit{et} \textit{ad} \textit{uaria} \textit{loca} \textit{V.} \textit{T.} \textit{nec} \textit{non} \textit{uarias} \textit{dissertationes} \textit{et} \textit{orationes,} \textit{complectentia,} \textit{antehac} \textit{seorsum} \textit{in} \textit{lucem} \textit{emissa,} \textit{nunc} \textit{in} \textit{unum} \textit{corpus} \textit{collecta} \textit{et} \textit{coniunctim} \textit{edita,} \textit{una} \textit{cum} \textit{indicibus} \textit{necessariis}, 487–510. Lugduni Batauorum: apud Io. le Mair et H. A. de Chalmot.

[Schulze], [Johann Heinrich]. 1711. \textit{Erleichterte} \textit{griechische} \textit{grammatica,} \textit{oder} \textit{Gründliche} \textit{Anführung} \textit{zur} \textit{griechischen} \textit{Sprache} \textit{[…]}. Halle: in Verlegung des Wäysenhauses.

Schuster, Johann Heinrich \& Johann Michael Lauterbach. 1737. \textit{Disputatio} \textit{philologico-critica} \textit{in} \textit{naturam} \textit{quattuor} \textit{linguarum} \textit{cardinalium} \textit{Germanicae,} \textit{Latinae,} \textit{Graecae} \textit{ac} \textit{Hebraeae,} \textit{nec} \textit{non} \textit{methodum} \textit{in} \textit{libris} \textit{grammaticis} \textit{uulgaribus} \textit{adhibitam} \textit{inquirens} \textit{delineata} [...]. Ienae: litteris Io. Friderici Ritteri.

Schwartz, Christian Gottlieb [praeses] \& Abraham Helm [respondens]. 1702. \textit{Disputatio} \textit{de} \textit{causis} \textit{dialectorum,} \textit{speciatim} \textit{Graecarum} [...]. Vitembergae: prelo Gerdesiano.

Schwartz, Johann Conrad. 1721. [Notae ad aphorismum VIII: \REF{ex:key:6} In dialectorum uarietate]. \textit{De} \textit{stilo} \textit{Noui} \textit{Testamenti} \textit{liber} \textit{philologico-theologicus} [...], 222–230. Coburgi: impensis Pauli Guntheri Pfotenhauers.

Simler, Georg. 1512. Isagogicum in litteras Graecanicas. \textit{Quae} \textit{hoc} \textit{libro} \textit{continentur.} \textit{Georgii} \textit{Simler} \textit{Vuimpinensis} \textit{Obseruationes} \textit{de} \textit{arte} \textit{grammatica.} \textit{De} \textit{litteris} \textit{Graecis} \textit{ac} \textit{diphthongis} \textit{et} \textit{quemadmodum} \textit{ad} \textit{nos} \textit{ueniant,} \textit{abbreuiationes} \textit{quibus} \textit{frequentissime} \textit{Graeci} \textit{utuntur} \textit{[…]}, AA.i\textsc{\textsuperscript{r}}–EE.iij\textsc{\textsuperscript{v}}. Tubingae: in aedibus Thomae Anselmi Badensis.

Simon, Richard. 1689. \textit{Histoire} \textit{critique} \textit{du} \textit{texte} \textit{du} \textit{Nouveau} \textit{Testament,} \textit{où} \textit{l’on} \textit{établit} \textit{la} \textit{verité} \textit{des} \textit{actes} \textit{sur} \textit{lesquels} \textit{la} \textit{religion} \textit{chrêtienne} \textit{est} \textit{fondée}. A Rotterdam: chez Reinier Leers.

Simon, Stephanus. 1615. \textit{Historia} \textit{linguae} \textit{Graecae} \textit{methodica} \textit{[…]}. Parisiis: apud Claudium Morellum.

Simonis, Johann. 1752. \textit{Introductio} \textit{grammatico-critica} \textit{in} \textit{linguam} \textit{Graecam,} \textit{qua} \textit{de} \textit{linguae} \textit{illius} \textit{origine} \textit{et} \textit{antiquitate,} \textit{natura,} \textit{fatis} \textit{ac} \textit{subsidiis,} \textit{de} \textit{praecipuis} \textit{grammatices} \textit{Graecae} \textit{momentis,} \textit{lingua} \textit{et} \textit{textu} \textit{N.} \textit{T.} \textit{Graeco,} \textit{de} \textit{uersionibus} \textit{denique} \textit{V.} \textit{T.} \textit{Graecis} \textit{disseritur,} \textit{in} \textit{usum} \textit{iuuentutis} \textit{$\varphi} \textit{\iota} \textit{\lambda} \textit{\text{\textgreek{'e}}\lambda \lambda \eta \nu o\varsigma $}. Halae Magdeburgicae: impensis Orphanotrophei.

Speed, John. 1676. \textit{An} \textit{epitome} \textit{of} \textit{Mr.} \textit{John} \textit{Speed’s} \textit{Theatre} \textit{of} \textit{the} \textit{empire} \textit{of} \textit{Great} \textit{Britain,} \textit{and} \textit{of} \textit{his} \textit{Prospect} \textit{of} \textit{the} \textit{most} \textit{famous} \textit{parts} \textit{of} \textit{the} \textit{world}. London: printed for Tho. Basset [...] and Ric. Chiswel.

[Spieghel], [Hendrik Laurensz], \textit{et} \textit{al}. 1584. \textit{Twe-spraack} \textit{vande} \textit{Nederduitsche} \textit{letterkunst} \textit{ofte} \textit{Vant} \textit{spellen} \textit{ende} \textit{eyghenscap} \textit{des} \textit{Nederduitschen} \textit{taals}. Tot Leyden: by Christoffel Plantyn.

Stapleton, Thomas. 1566. \textit{A} \textit{returne} \textit{of} \textit{untruthes} \textit{upon} \textit{M.} \textit{Jewelles} \textit{replie}. In Antwerpe: printed [...] by John Latius.

Steinthal, Heymann. 1891. \textit{Geschichte} \textit{der} \textit{Sprachwissenschaft} \textit{bei} \textit{den} \textit{Griechen} \textit{und} \textit{Römern} \textit{mit} \textit{besonderer} \textit{Rücksicht} \textit{auf} \textit{die} \textit{Logik}. Zweite vermehrte und verbesserte Auflage. Vol. 2. Berlin: Ferd. Dümmlers Verlagsbuchhandlung. (12 August, 2014).

Stubbe, Henry. 1657. \textit{Clamor,} \textit{rixa,} \textit{ioci,} \textit{mendacia,} \textit{furta,} \textit{cachini,} \textit{or} \textit{A} \textit{severe} \textit{enquiry} \textit{into} \textit{the} \textit{late} \textit{Oneirocritica} \textit{published} \textit{by} \textit{John} \textit{Wallis,} \textit{grammar-reader} \textit{in} \textit{Oxon.} [Second title page:] \textit{$\Theta} \textit{\varepsilon} \textit{\rho} \textit{\sigma} \textit{\text{\textgreek{'i}}\tau \eta \varsigma $ $\text{\textgreek{>a}}\kappa \rho \iota \tau \text{\textgreek{'o}}\mu \upsilon \theta o\varsigma $. Or an exact account of the grammatical part of the controversy betwixt Mr. Hobbes and J. Wallis [...]}. London: [s.n.].

Sylvius, Jacobus. 1531. \textit{In} \textit{linguam} \textit{Gallicam} \textit{isagωge,} \textit{una} \textit{cum} \textit{eiusdem} \textit{grammatica} \textit{Latino-Gallica,} \textit{ex} \textit{Hebraeis,} \textit{Graecis} \textit{et} \textit{Latinis} \textit{auctoribus}. Parisiis: ex officina Roberti Stephani.

Thomassin, Louis. 1697. \textit{Glossarium} \textit{uniuersale} \textit{Hebraicum,} \textit{quo} \textit{ad} \textit{Hebraicae} \textit{linguae} \textit{fontes} \textit{linguae} \textit{et} \textit{dialecti} \textit{paene} \textit{omnes} \textit{reuocantur}. Parisiis: e Typographia Regia.

Thompson, George. 1732. \textit{Apparatus} \textit{ad} \textit{linguam} \textit{Graecam} \textit{ordine} \textit{nouo} \textit{ac} \textit{facili} \textit{digestus:} \textit{In} \textit{quo} \textit{defectus} \textit{aliorum} \textit{quamplurimi} \textit{supplentur,} \textit{flexiones} \textit{nominum} \textit{et} \textit{uerborum} \textit{fusius} \textit{quam} \textit{apud} \textit{alios} \textit{tractantur,} \textit{dialecti} \textit{etiam,} \textit{quantitates} \textit{et} \textit{accentus} \textit{dilucidius} \textit{explicantur} \textit{quaeque} \textit{magis} \textit{uel} \textit{minus} \textit{necessariu} \textit{sunt,} \textit{typis} \textit{diuersis} \textit{in} \textit{tyronum} \textit{gratiam} \textit{distinguuntur}. Londini: typis Gul. Bowyer; impensis autem Io. Osborn et Tho. Longman.

Thryllitsch, Georg Friedrich [auctor et respondens]. 1709. \textit{Suspiciones} \textit{quasdam} \textit{historico-technicas} \textit{de} \textit{dialectis} \textit{Graecis} \textit{ex} \textit{consideratione} \textit{originum} \textit{migrationumque} \textit{Graecarum} \textit{nationum} \textit{collectas} \textit{praeside} [...] \textit{eruditorum} \textit{examini} \textit{modeste} \textit{subicio,} [...] \textit{Thryllitius} [...]. Vitembergae: litteris Schroedterianis.

Thryllitsch, Georg Friedrich [praeses] \& Johann Gottfried Brunner [respondens]. 1709. \textit{Pronuntiationem} \textit{Latinam} \textit{ex} \textit{Aeolica} \textit{repetendam} \textit{esse} \textit{explicandamque} [...] \textit{Thryllitius} \textit{et} [...] \textit{Brunnerus} [...] \textit{edisserent}. Vitembergae: litteris Christiani Schroedteri.

Tolomei, Claudio. 1555. \textit{Il} \textit{Cesano,} [...] \textit{nel} \textit{quale} \textit{da} \textit{piu} \textit{dotti} \textit{huomini} \textit{si} \textit{disputa} \textit{del} \textit{nome,} \textit{col} \textit{quale} \textit{si} \textit{dee} \textit{ragionevolmente} \textit{chiamare} \textit{la} \textit{volgar} \textit{lingua}. In Vinegia: appresso Gabriel Giolito de Ferrari et fratelli.

Tory, Geoffroy. 1529. \textit{Champ} \textit{fleury:} \textit{Au} \textit{quel} \textit{est} \textit{contenu} \textit{lart} \textit{et} \textit{science} \textit{de} \textit{la} \textit{deue} \textit{et} \textit{vraye} \textit{proportion} \textit{des} \textit{lettres} \textit{attiques,} \textit{quon} \textit{dit} \textit{autrement} \textit{lettres} \textit{antiques,} \textit{et} \textit{vulgairement} \textit{lettres} \textit{romaines} \textit{proportionnees} \textit{selon} \textit{le} \textit{corps} \textit{et} \textit{visage} \textit{humain}. A Paris: par Maistre Geofroy Tory de Bourges [...] et par Giles Gourmont.

Trendelenburg, Johann Georg. 1782. \textit{Anfangsgründe} \textit{der} \textit{griechischen} \textit{Sprache}. Danzig: gedruckt bey Daniel Ludwig Wedel.

Tribbechow, Johann. 1705. \textit{Breuia} \textit{linguae} \textit{$\text{\textgreek{<r}}\omega \mu \alpha \iota \kappa \text{\textgreek{~h}}\varsigma $, siue Graecae uulgaris elementa, quibus differentia antiquum inter et recentiorem Graecismum praecipue ostenditur: Praemissa est Dissertatio de ortu et natura huius linguae}. Ienae: impensis Ioannis Bielckii, typis Nisianis excudebat Henricus Beyerus.

Trissino, Gian Giorgio. 1529. \textit{La} \textit{p$\omega} \textit{\varepsilon} \textit{$tica}. In Vic$\varepsilon $nza: stampata [...] per Tωlωm$\varepsilon \omega $ Ianiculω.

Ursin, Georg Heinrich. 1691. \textit{Grammatica} \textit{Graeca} \textit{ex} \textit{aliis} \textit{accurato} \textit{ordine} \textit{ac} \textit{sollicito} \textit{quorumuis} \textit{examine} \textit{collecta} \textit{inque} \textit{sectiones} \textit{et} \textit{capita} \textit{et} \textit{haec} \textit{in} \textit{quaestiones} \textit{ac} \textit{responsiones} \textit{digesta,} \textit{ad} \textit{usus} \textit{Gymnasii} \textit{Ratisponensis} \textit{Poetici} [...]. Norimbergae: sumptibus Wolfgangi Mauritii Endteri.

Valckenaer, Lodewijk Caspar. 1773. Digressio IV, Laconica quaedam exhibens. \textit{Theocriti} \textit{Decem} \textit{eidyllia,} \textit{Latinis} \textit{pleraque} \textit{numeris} \textit{a} \textit{C.} \textit{A.} \textit{Wetstenio} \textit{reddita,} \textit{in} \textit{usum} \textit{auditorum} \textit{cum} \textit{notis} \textit{edidit} \textit{eiusdemque} \textit{Adoniazusas} \textit{uberioribus} \textit{annotationibus} \textit{instruxit} \textit{L.} \textit{C.} \textit{Valckenaer}, 271–300. Lugduni Batauorum: apud Ioann. le Mair.

Valckenaer, Lodewijk Caspar. 1790. [Nota ad $\lambda \text{\textgreek{'e}}\gamma \omega $]. \textit{Etymologicum} \textit{linguae} \textit{Graecae} \textit{siue} \textit{Obseruationes} \textit{ad} \textit{singulas} \textit{uerborum} \textit{nominumque} \textit{stirpes} \textit{secundum} \textit{ordinem} \textit{lexici} \textit{compilati} \textit{olim} \textit{a} \textit{Ioanne} \textit{Scapula}, vol. 1, 490–491. Editionem curauit atque animaduersiones cum aliorum, tum suas adiecit Euerardus Scheidius [...]. Traiecti ad Rhenum […]: apud G. T. a Paddenburg et filium […].

Varchi, Benedetto. 1570. \textit{L’Hercolano:} \textit{Dialogo} [...], \textit{nel} \textit{qual} \textit{si} \textit{ragiona} \textit{generalmente} \textit{delle} \textit{lingue} \textit{et} \textit{in} \textit{particolare} \textit{della} \textit{Toscana} \textit{e} \textit{della} \textit{Fiorentina}. In Vinetia: appresso Filippo Giunti e Fratelli.

Veranzio, Fausto. 1595. \textit{Dictionarium} \textit{quinque} \textit{nobilissimarum} \textit{Europae} \textit{linguarum,} \textit{Latinae,} \textit{Italicae,} \textit{Germanicae,} \textit{Dalmatiae} \textit{et} \textit{Vngaricae}. Venetiis: apud Nicolaum Morettum.

Verwer, Adriaen. 1707. \textit{Linguae} \textit{Belgicae} \textit{idea} \textit{grammatica,} \textit{poetica,} \textit{rhetorica}. Amstelaedami: excudit Franciscus Halma.

Verwey, Johannes. 1684. \textit{Noua} \textit{uia} \textit{docendi} \textit{Graeca} [...]. Gaudae: sumptibus auctoris exscripsit Iustus van der Hoeve.

Vergil. 1697. \textit{The} \textit{works} \textit{of} \textit{Virgil} \textit{containing} \textit{his} \textit{Pastorals,} \textit{Georgics,} \textit{and} \textit{Aeneis}. (Trans.) John Dryden. London: printed for Jacob Tonson.

Vitringa, Campegius. 1689. \textit{Sacrarum} \textit{obseruationum} \textit{libri} \textit{duo} \textit{in} \textit{quorum} \textit{altero,} \textit{de} \textit{confusione} \textit{linguarum} [...], \textit{altero} \textit{autem,} \textit{de} \textit{cultu} \textit{Molechi} \textit{in} \textit{deserto} [...]. Franequerae: apud Ioannem Gyselaar.

Vitringa, Campegius. 1712. \textit{Obseruationum} \textit{sacrarum} \textit{libri} \textit{sex,} \textit{in} \textit{quibus} \textit{de} \textit{rebus} \textit{uarii} \textit{argumenti} \textit{et} \textit{utilissimae} \textit{inuestigationis,} \textit{critice} \textit{ac} \textit{theologice,} \textit{disseritur:} \textit{Sacrorum} \textit{imprimis} \textit{librorum} \textit{loca} \textit{multa} \textit{obscuriora} \textit{noua} \textit{uel} \textit{clariore} \textit{luce} \textit{perfunduntur} [...]. Franequerae: ex officina Wibii Bleck.

Vives, Juan Luis. 1531. \textit{De} \textit{disciplinis} \textit{libri} \textit{XX}. Antuerpiae: excudebat [...] Michael Hillenius.

Vives, Juan Luis. 1533. \textit{De} \textit{ratione} \textit{dicendi} \textit{libri} \textit{tres.} \textit{De} \textit{consultatione}. Louanii: ex officina Rutgeri Rescii [...], sumptibus eiusdem ac Bartholomaei Grauii.

Vogel, Georg Johann Ludwig. 1764. \textit{Libellus} \textit{singularis} \textit{de} \textit{dialecto} \textit{poetica} \textit{scripturarum} \textit{Ebraicarum} \textit{Veteris} \textit{Testamenti} \textit{cum} \textit{epistola} \textit{Guilielmi} \textit{Abrahami} \textit{Telleri} \textit{D.} Helmstadii: apud Ioannem Fridericum Weygand.

von der Hardt, Hermann. 1705 [1699]. \textit{Studiosus} \textit{Graecus}. Editio secunda. Helmstadii: typis G. W. Hammii.

von Stieler, Caspar. 1691. Kurze Lehrschrift von der hochteutschen Sprachkunst {\textbar} Breuis grammaticae imperialis linguae Germanicae delineatio. \textit{Der} \textit{teutschen} \textit{Sprache} \textit{Stammbaum} \textit{und} \textit{Fortwachs,} \textit{oder} \textit{teutscher} \textit{Sprachschatz} [...] \textit{{\textbar} Teutonicae linguae semina et germina, siue Lexicon Germanicum [...]}, 1–243. Nürnberg \& Altdorf {\textbar} Noribergae \& Altdorfii: in Verlegung Johann Hofmanns [...]. Gedruckt [...] von Heinrich Meyern {\textbar} Impensis Ioannis Hofmanni [...]. Typis Henrici Meyeri.

Vossius, Isaac. 1673. \textit{De} \textit{poematum} \textit{cantu} \textit{et} \textit{uiribus} \textit{rythmi}. Oxonii: e Theatro Sheldoniano.

Vuidius, Robertus. 1569. De dialectis Graecis libellus. \textit{De} \textit{ratione} \textit{quantitatis} \textit{syllabariae} \textit{liber} [...], 136\textsc{\textsuperscript{v}}–148\textsc{\textsuperscript{v}}. Parisiis: apud Hieronymum de Marnef et Gulielmum Cauellat.

Walch, Johann Ernst Immanuel. 1772. \textit{Introductio} \textit{in} \textit{linguam} \textit{Graecam}. Editio secunda auctior. Ienae: sumptibus uiduae I. R. Croeckeri.

Walper, Otto. 1589. \textit{De} \textit{dialectis} \textit{Graecae} \textit{linguae} \textit{praecipuis,} \textit{Attica,} \textit{Ionica,} \textit{Dorica,} \textit{Aeolica} \textit{et} \textit{coronidis} \textit{uice} \textit{nonnulla} \textit{de} \textit{proprietate} \textit{poetica:} \textit{Libellus} \textit{methodice} \textit{conscriptus} \textit{et} \textit{in} \textit{gratiam} \textit{tironum} \textit{$\varphi} \textit{\iota} \textit{\lambda} \textit{\varepsilon} \textit{\lambda} \textit{\lambda} \textit{\text{\textgreek{'h}}\nu \omega \nu $ in Academia Marpurgensi propositus}. Francofurti ad Moenum: ex officina typographica Ioannis Spiessii.

Walper, Otto. 1590. \textit{Grammatica} \textit{Graeca,} \textit{ex} \textit{optimis} \textit{quibusque} \textit{auctoribus,} \textit{in} \textit{usum} \textit{Academiae} \textit{Marpurgensis} \textit{ceterarumque} \textit{Scholarum} \textit{Hussiacarum,} \textit{per} \textit{quaestiones} \textit{et} \textit{responsiones} \textit{concinnata} \textit{tribusque} [...]. Marpurgi: typis Pauli Egenolphi.

Walton, Brian. 1657. In Biblia polyglotta prolegomena. In Brian Walton (ed.), \textit{Biblia} \textit{sacra} \textit{polyglotta,} \textit{complectentia} \textit{textus} \textit{originales,} \textit{Hebraicum,} \textit{cum} \textit{Pentateucho} \textit{Samaritano,} \textit{Chaldaicum,} \textit{Graecum,} \textit{uersionumque} \textit{antiquarum,} \textit{Samaritanae,} \textit{Graecae} \textit{LXXII} \textit{interp.,} \textit{Chaldaicae,} \textit{Syriacae,} \textit{Arabicae,} \textit{Aethiopicae,} \textit{Persicae,} \textit{Vulg.} \textit{Lat.}, 1–102. Londini: imprimebat Thomas Roycroft.

Waser, Caspar. 1610. Ad Mithridatem Gesneri libellus commentarius. \textit{Mithridates} \textit{Gesneri,} \textit{exprimens} \textit{differentias} \textit{linguarum,} \textit{tum} \textit{ueterum,} \textit{tum} \textit{quae} \textit{hodie} \textit{per} \textit{totum} \textit{terrarum} \textit{orbem} \textit{in} \textit{usu} \textit{sunt:} \textit{Caspar} \textit{Waserus} \textit{recensuit} \textit{et} \textit{libello} \textit{commentario} \textit{illustrauit}, 86\textsc{\textsuperscript{r}}–140\textsc{\textsuperscript{v}}. Editio altera. Tiguri: typis Wolphianis.

Webb, John. 1669. \textit{An} \textit{historical} \textit{essay} \textit{endeavoring} \textit{a} \textit{probability} \textit{that} \textit{the} \textit{language} \textit{of} \textit{the} \textit{empire} \textit{of} \textit{China} \textit{is} \textit{the} \textit{primitive} \textit{language}. London: printed for Nath. Brook.

Weemes, John. 1632. \textit{Exercitations} \textit{divine:} \textit{Containing} \textit{diverse} \textit{questions} \textit{and} \textit{solutions} \textit{for} \textit{the} \textit{right} \textit{understanding} \textit{of} \textit{the} \textit{Scriptures}. London: printed by T. Cotes for John Bellamie.

Wesley, Samuel. 1736. \textit{Dissertationes} \textit{in} \textit{librum} \textit{Iobi}. Londini: typis Gulielmi Bowyer.

Williams, John. 1685. \textit{A} \textit{discourse} \textit{concerning} \textit{the} \textit{celebration} \textit{of} \textit{divine} \textit{service} \textit{in} \textit{an} \textit{unknown} \textit{tongue}. London: printed for Richard Chiswell.

Wise, Francis. 1758. \textit{Some} \textit{enquiries} \textit{concerning} \textit{the} \textit{first} \textit{inhabitants} \textit{language} \textit{religion} \textit{learning} \textit{and} \textit{letters} \textit{of} \textit{Europe}. Oxford: printed at the Theatre for J. Fletcher, S. Parker, D. Prince [...], and Messrs Rivington and Fletcher.

Wolf, Friedrich August. 1795. \textit{Prolegomena} \textit{ad} \textit{Homerum} \textit{siue} \textit{de} \textit{operum} \textit{Homericorum} \textit{prisca} \textit{et} \textit{genuina} \textit{forma} \textit{uariisque} \textit{mutationitubs} \textit{et} \textit{probabili} \textit{ratione} \textit{emendandi}. Vol. 1. Halis Saxonum: e libraria Orphanotrophei.

Wolf, Hieronymus. 1578. De orthographia Germanica, ac potius Sueuica nostrate. \textit{Institutionum} \textit{grammaticarum} [...] \textit{libri} \textit{octo} [...], 594–615. Augustae Vindelicorum: Michael Manger excudebat.

Wood, Robert. 1775. \textit{An} \textit{essay} \textit{on} \textit{the} \textit{original} \textit{genius} \textit{and} \textit{writings} \textit{of} \textit{Homer,} \textit{with} \textit{a} \textit{comparative} \textit{view} \textit{of} \textit{the} \textit{ancient} \textit{and} \textit{present} \textit{state} \textit{of} \textit{the} \textit{Troade}. [Second expanded edition.] London: printed by H. Hughs; for T. Payne [...] and P. Elmsly [...].

Wright, Joseph. 1691. \textit{Folly} \textit{detected,} \textit{or} \textit{Some} \textit{animadversions} \textit{on} \textit{a} \textit{book} \textit{called,} A brief discourse concerning singing in the public worship of God, \textit{put} \textit{forth} \textit{by} \textit{one} \textit{Mr.} \textit{Isaac} \textit{\citealt{Marlow1690} [...]}. London: [...] sold by John Harris.

Wyss, Caspar. 1650. \textit{Dialectologia} \textit{sacra:} \textit{In} \textit{qua} \textit{quicquid} \textit{per} \textit{uniuersum} \textit{Noui} \textit{Foederis} \textit{contextum} \textit{in} \textit{Apostolica} \textit{et} \textit{uoce} \textit{et} \textit{phrasi} \textit{a} \textit{communi} \textit{Graecorum} \textit{lingua} \textit{eoque} \textit{grammatica} \textit{analogia} \textit{discrepat,} \textit{methodo} \textit{congrua} \textit{disponitur,} \textit{accurate} \textit{definitur} \textit{et} \textit{omnium} \textit{sacri} \textit{contextus} \textit{exemplorum} \textit{inductione} \textit{illustratur}. Tiguri: typis Ioannis Iacobi Bodmeri.

Zwinger, Jakob. 1605. Graecarum dialectorum hypotyposis, seorsim primum singularum, tum coniunctim omnium, tabulis methodicis, iudicio memoriaque seruientibus, proposita. \textit{Lexicon} \textit{Graeco-Latinum} \textit{nouum} \textit{in} \textit{quo} \textit{ex} \textit{primitiuorum} \textit{et} \textit{simplicium} \textit{fontibus} \textit{deriuata} \textit{atque} \textit{composita} \textit{ordine} \textit{non} \textit{minus} \textit{naturali,} \textit{quam} \textit{alphabetico,} \textit{breuiter} \textit{et} \textit{dilucide} \textit{deducuntur}, TT.1\textsc{\textsuperscript{r}}–AAa.6\textsc{\textsuperscript{r}}. Editio ultima, priori locupletior et correctior. Cum auctario dialectorum omnium [...] in expeditas succinctasque tabulas compendiose redactarum. Basileae: per Sebastianum Henricpetri.

\subsection{Secondary literature}
\hypertarget{Toc19704872}{}
Adams, James Noel. 2007. \textit{The} \textit{regional} \textit{diversification} \textit{of} \textit{Latin} \textit{200} \textit{BC–AD} \textit{600}. Cambridge: Cambridge University Press.

Alinei, Mario. 1980. Dialect: A dialectical approach. In Joachim Göschel, Pavle Ivić \& Kurt Kehr (eds.), \textit{Dialekt} \textit{und} \textit{Dialektologie:} \textit{Ergebnisse} \textit{des} \textit{internationalen} \textit{Symposions} \textit{“Zur} \textit{Theorie} \textit{des} \textit{Dialekts”} \textit{Marburg/Lahn,} \textit{5.–10.} \textit{\citealt{September1977}} (Zeitschrift für Dialektologie und Linguistik: Beihefte. Neue Folge 26), 11–42. Wiesbaden: Franz Steiner.

Alinei, Mario. 1984 [1981]. “Dialetto”: Un concetto rinascimentale fiorentino. \textit{Lingua} \textit{e} \textit{dialetti:} \textit{Struttura,} \textit{storia} \textit{e} \textit{geografia} (Studi linguistici e semiologici), 169–199. Bologna: Il Mulino.

Alonso Déniz, Alcorac. 2018. The dialect of Thasos and the transmission of Archilochus’ fragments. In Georgios K. Giannakis, Emilio Crespo \& Panagiotis Filos (eds.), \textit{Studies} \textit{in} \textit{Ancient} \textit{Greek} \textit{dialects:} \textit{From} \textit{central} \textit{Greece} \textit{to} \textit{the} \textit{Black} \textit{Sea} (Trends in classics – Supplementary volumes 49), 531–560. Berlin \& Boston: De Gruyter.

Amsler, Mark. 1993. History of linguistics, “standard Latin”, and pedagogy. In Vivien Law (ed.), \textit{History} \textit{of} \textit{linguistic} \textit{thought} \textit{in} \textit{the} \textit{Early} \textit{Middle} \textit{Ages} (Studies in the history of the language sciences 71), 49–66. Amsterdam \& Philadelphia: John Benjamins.

Argyropoulos, Vassileios. 2009. Chatzidakis, Georgios. In Harro Stammerjohann (ed.), \textit{Lexicon} \textit{grammaticorum:} \textit{A} \textit{bio-bibliographical} \textit{companion} \textit{to} \textit{the} \textit{history} \textit{of} \textit{linguistics}, vol. 1, 288–290. Second edition, revised and enlarged. Tübingen: Niemeyer.

Auer, Peter. 2005. Europe’s sociolinguistic unity, or: A typology of European dialect/standard constellations. In Nicole Delbecque, Johan Van der Auwera \& Dirk Geeraerts (eds.), \textit{Perspectives} \textit{on} \textit{variation:} \textit{Sociolinguistic,} \textit{historical,} \textit{comparative} (Trends in linguistics: Studies and monographs 163), 7–42. Berlin \& New York: Mouton de Gruyter.

Auroux, Sylvain \& Geneviève Clerico. 1992. Les traditions nationales: \sectref{sec:key:4.} France. In Sylvain Auroux (ed.), \textit{Histoire} \textit{des} \textit{idées} \textit{linguistiques} (Philosophie et langage), vol. 2 [\textit{Le} \textit{développement} \textit{de} \textit{la} \textit{grammaire} \textit{occidentale}], 359–386. Liège: Mardaga.

Ax, Wolfram. 1987. \textit{Quadripertita} \textit{ratio}. Bemerkungen zur Geschichte eines aktuellen Kategoriensystems (Adiectio – Detractio – Transmutatio – Immutatio). In Daniel J. Taylor (ed.), \textit{The} \textit{History} \textit{of} \textit{Linguistics} \textit{in} \textit{the} \textit{Classical} \textit{Period} (Studies in the History of the Language Sciences 46), 17–40. Amsterdam \& Philadelphia: John Benjamins.

Baloglou, Christos P. 1998. Economic thought in the last Byzantine period. In S. Todd Lowry \& Barry Gordon (eds.), \textit{Ancient} \textit{and} \textit{medieval} \textit{economic} \textit{ideas} \textit{and} \textit{concepts} \textit{of} \textit{social} \textit{justice}, 405–438. Leiden, New York \& Köln: Brill.

Barbier-Mueller, Jean Paul. 1990. \textit{Ma} \textit{bibliothèque} \textit{poétique.} \textit{Deuxième} \textit{partie:} \textit{Ronsard}. Genève: Droz.

Bean, Donald P. \& Antje Lemke. 1958. \textit{Aldus} \textit{Manutius} \textit{and} \textit{his} Thesaurus cornucopiae \textit{of} \textit{1496} \textit{[…].} Syracuse: Syracuse University Press.

Beller, Manfred \& Joep Leerssen (eds.). 2007. \textit{Imagology.} \textit{The} \textit{cultural} \textit{construction} \textit{and} \textit{literary} \textit{representation} \textit{of} \textit{national} \textit{characters:} \textit{A} \textit{critical} \textit{survey}. Amsterdam \& New York: Rodopi.

Benincà, Paola. 1988. \textit{Piccola} \textit{storia} \textit{ragionata} \textit{della} \textit{dialettologia} \textit{italiana} (Quaderni patavini di linguistica: Monografie 3). Padova: Unipress.

Bentley, Jerry H. 1983. \textit{Humanists} \textit{and} \textit{Holy} \textit{Writ:} \textit{New} \textit{Testament} \textit{scholarship} \textit{in} \textit{the} \textit{Renaissance}. Princeton: Princeton University Press.

Ben-Tov, Asaph. 2009. \textit{Lutheran} \textit{humanists} \textit{and} \textit{Greek} \textit{antiquity:} \textit{Melanchthonian} \textit{scholarship} \textit{between} \textit{universal} \textit{history} \textit{and} \textit{pedagogy} (Brill’s studies in intellectual history 183). Leiden: Brill.

Ben-Tov, Asaph. 2013. \textit{Turco-Graecia}: German humanists and the end of Greek antiquity – Cultural exchange and misunderstanding. In Anna Contadini \& Claire Norton (eds.), \textit{The} \textit{Renaissance} \textit{and} \textit{the} \textit{Ottoman} \textit{world}, 181–195. Farnham \& Burlington: Ashgate.

Bischoff, Bernhard. 1961. The study of foreign languages in the Middle Ages. \textit{Speculum:} \textit{A} \textit{journal} \textit{of} \textit{mediaeval} \textit{studies} 36(2). 209–224.

Bischoff, Bernhard. 1981. \textit{Mittelalterliche} \textit{Studien:} \textit{Ausgewählte} \textit{Aufsätze} \textit{zur} \textit{Schriftkunde} \textit{und} \textit{Literaturgeschichte}. Stuttgart: Hiersemann.

Blank, Paula. 1996. \textit{Broken} \textit{English:} \textit{Dialects} \textit{and} \textit{the} \textit{politics} \textit{of} \textit{language} \textit{in} \textit{Renaissance} \textit{writings} (The politics of language). London \& New York: Routledge.

Bonfante, Giuliano. 1953–1954. Ideas on the kinship of the European languages from 1200 to 1800. \textit{Cahiers} \textit{d’histoire} \textit{mondiale} \textit{{\textbar} Journal of world history {\textbar} Cuadernos de historia mundial} 1. 679–699.

Borst, Arno. 1957–1963. \textit{Der} \textit{Turmbau} \textit{von} \textit{Babel:} \textit{Geschichte} \textit{der} \textit{Meinungen} \textit{über} \textit{Ursprung} \textit{und} \textit{Vielfalt} \textit{der} \textit{Sprachen} \textit{und} \textit{Völker}. 4 vols. Stuttgart: Hiersemann.

Botley, Paul. 2010. \textit{Learning} \textit{Greek} \textit{in} \textit{Western} \textit{Europe,} \textit{1396–1529:} \textit{Grammars,} \textit{lexica,} \textit{and} \textit{classroom} \textit{texts} (Transactions of the American Philosophical Society held at Philadelphia for promoting useful knowledge 100.2). Philadelphia: American Philosophical Society.

Bots, Hans \& Françoise Waquet. 1997. \textit{La} \textit{république} \textit{des} \textit{lettres}. Paris: Belin.

Boulhol, Pascal. 2014. \textit{«Grec} \textit{langaige} \textit{n’est} \textit{pas} \textit{doulz} \textit{au} \textit{françois»:} \textit{L’étude} \textit{et} \textit{l’enseignement} \textit{du} \textit{grec} \textit{dans} \textit{la} \textit{France} \textit{ancienne} \textit{(IVe} \textit{siècle} \textit{–} \textit{1530)} (Héritages méditerranéens). Aix-en-Provence: Presses Universitaires de Provence.

Brekle, Herbert E., \textit{et} \textit{al.} (eds.). 1992–2005. \textit{Bio-bibliographisches} \textit{Handbuch} \textit{zur} \textit{Sprachwissenschaft} \textit{des} \textit{18.} \textit{Jahrhunderts:} \textit{Die} \textit{Grammatiker,} \textit{Lexikographen} \textit{und} \textit{Sprachtheoretiker} \textit{des} \textit{deutschsprachigen} \textit{Raums} \textit{mit} \textit{Beschreibungen} \textit{ihrer} \textit{Werke}. 8 vols. Tübingen: Max Niemeyer.

Brixhe, Claude. 2010. Linguistic diversity in Asia Minor during the empire: \textit{Koine} and non-Greek languages. In Egbert J. Bakker (ed.), \textit{A} \textit{companion} \textit{to} \textit{the} \textit{Ancient} \textit{Greek} \textit{language} (Blackwell companions to the ancient world), 228–252. Malden, Oxford \& Chichester: Wiley-Blackwell.

Brodersen, Kai. 2015. Heraclides: [18] H. Creticus/Criticus Greek periegete, 3rd cent. BC. In Hubert Cancik \& Helmuth Schneider (eds.), \textit{Brill’s} \textit{new} \textit{Pauly}. Brill online. Last accessed February 27, 2019.

<http://referenceworks.brillonline.com/entries/brill-s-new-pauly/heraclides-e508320>.

Burke, Peter. 2004. \textit{Languages} \textit{and} \textit{communities} \textit{in} \textit{early} \textit{modern} \textit{Europe}. Cambridge: Cambridge University Press.

Bywater, Ingram. 1908. \textit{The} \textit{Erasmian} \textit{pronunciation} \textit{of} \textit{Greek} \textit{and} \textit{its} \textit{precursors:} \textit{Jerome} \textit{Aleander,} \textit{Aldus} \textit{Manutius,} \textit{Antonio} \textit{of} \textit{Lebrixa}. London \& Oxford: Henry Frowde \& Oxford University Press.

Calvo Pérez, Julio. 2005. Fonología y ortografía de las lenguas indígenas de América del Sur a la luz de los primeros misioneros gramáticos. \textit{Missionary} \textit{linguistics} \textit{II} \textit{{\textbar} Lingüística misionera II: Orthography and phonology. Selected papers from the Second International Conference on Missionary Linguistics, São Paulo, 10–13 \citealt{March2004}} (Studies in the history of the language sciences 109), 137–170. Amsterdam \& Philadelphia: John Benjamins.

Caratzas, Stamatios C. 1952. La dissimilation du \textit{t} dans le dialecte du vieil Athènes et la valeur du témoignage de Kavassilas et Zygomalas (16e siècle). \textit{Istituto} \textit{Lombardo} \textit{di} \textit{scienze} \textit{e} \textit{lettere:} \textit{Rendiconti.} \textit{Classe} \textit{di} \textit{lettere} \textit{e} \textit{scienze} \textit{morali} \textit{e} \textit{storiche} 85. 288–297.

Carruthers, Mary J. 2009. Varietas: A word of many colours. \textit{Poetica} 41(1). 11–32.

Cassio, Albio Cesare. 1984. Il “carattere” dei dialetti greci e l’opposizione Ioni-Dori: Testimonianze antiche e teorie di età romantica (su Arist. Quint. 2. 13, Iambl. \textit{v.} \textit{Pyth.} 241 sgg., \textit{sch.} \textit{in} Dion. Thr. p. 117, 18 sgg. Hilgard). \textit{$AI\Omega} \textit{N$:} \textit{Annali} \textit{del} \textit{Dipartimento} \textit{di} \textit{Studi} \textit{del} \textit{Mondo} \textit{Classico} \textit{e} \textit{del} \textit{Mediterraneo} \textit{Antico.} \textit{Sezione} \textit{linguistica} 6. 113–136.

Cassio, Albio Cesare. 2016. Parte I: Introduzione generale. In Albio Cesare Cassio (ed.), \textit{Storia} \textit{delle} \textit{lingue} \textit{letterarie} \textit{greche}, 1–133. Seconda edizione. Firenze: Le Monnier Università.

Ciccolella, Federica. 2008. \textit{Donati} \textit{Graeci:} \textit{Learning} \textit{Greek} \textit{in} \textit{the} \textit{Renaissance} (Columbia studies in the classical tradition 32). Leiden \& Boston: Brill.

Clackson, James. 2015. Latinitas, $\text{\textgreek{<E}}\lambda \lambda \eta \nu \iota \sigma \mu \text{\textgreek{'o}}\varsigma $ and standard languages. \textit{Studi} \textit{e} \textit{Saggi} \textit{Linguistici} 53(2). 309–330.

Colvin, Stephen. 1999. \textit{Dialect} \textit{in} \textit{Aristophanes} \textit{and} \textit{the} \textit{politics} \textit{of} \textit{language} \textit{in} \textit{Ancient} \textit{Greek} \textit{literature}. Oxford: Clarendon Press.

Colvin, Stephen. 2007. \textit{A} \textit{historical} \textit{Greek} \textit{Reader:} \textit{Mycenaean} \textit{to} \textit{the} \textit{koiné}. Oxford: Oxford University Press.

Colvin, Stephen. 2010. Greek dialects in the archaic and classical ages. In Egbert J. Bakker (ed.), \textit{A} \textit{companion} \textit{to} \textit{the} \textit{Ancient} \textit{Greek} \textit{language} (Blackwell companions to the ancient world), 200–212. Malden, Oxford \& Chichester: Wiley-Blackwell.

Conduché, Cécile. Fc. Latin et dialectes grecs chez Priscien: De la périphérie au centre. In Raf Van Rooy (ed.), \textit{Essays} \textit{in} \textit{the} \textit{history} \textit{of} \textit{dialect} \textit{studies:} \textit{From} \textit{ancient} \textit{Greece} \textit{to} \textit{modern} \textit{dialectology}. Münster: Nodus.

Consani, Carlo. 1991. \textit{$\Delta} \textit{IA\Lambda} \textit{EKTO\Sigma} \textit{$:} \textit{Contributo} \textit{alla} \textit{storia} \textit{del} \textit{concetto} \textit{di} \textit{“dialetto”} (Testi linguistici 18). Pisa: Giardini.

Consani, Carlo. 2000. La nozione di “lingua comune”/“varietà dialettale” nei grammatici tardo-antichi. In Cristina Vallini (ed.), \textit{Le} \textit{parole} \textit{per} \textit{le} \textit{parole:} \textit{I} \textit{logonimi} \textit{nelle} \textit{lingue} \textit{e} \textit{nel} \textit{metalinguaggio.} \textit{Atti} \textit{del} \textit{Convegno} \textit{Napoli,} \textit{Istituto} \textit{Universitario} \textit{Orientale} \textit{18–20} \textit{dicembre} \textit{1997}, 605–618. Roma: Il Calamo.

Considine, John. 2008. Did Andreas Jäger or Georg Caspar Kirchmaier write the dissertation \textit{De} \textit{lingua} \textit{vetustissima} \textit{Europae} \REF{ex:key:1686}? \textit{Historiographia} \textit{linguistica} 35(1–2). 13–22.

Considine, John. 2010. Why was Claude de Saumaise interested in the Scythian hypothesis? \textit{Language} \textit{\&} \textit{history} 53(2). 81–96.

Considine, John. 2012. Claudius Salmasius and the deadness of Neo-Latin. In Astrid Steiner-Weber (ed.), \textit{Acta} \textit{Conventus} \textit{Neo-Latini} \textit{Upsaliensis:} \textit{Proceedings} \textit{of} \textit{the} \textit{fourteenth} \textit{International} \textit{Congress} \textit{of} \textit{Neo-Latin} \textit{Studies} \textit{\citep{Uppsala2009}}, vol. 1, 295–305. Leiden \& Boston: Brill.

Considine, John. 2017. \textit{Small} \textit{dictionaries} \textit{and} \textit{curiosity:} \textit{Lexicography} \textit{and} \textit{fieldwork} \textit{in} \textit{post-medieval} \textit{Europe}. Oxford: Oxford University Press.

Considine, John \& Toon Van Hal. 2010. Introduction: Classifying and comparing languages in post-Renaissance Europe (1600–1800). \textit{Language} \textit{\&} \textit{history} 53(2). 63–69.

Contini, Riccardo. 1994. I primordi della linguistica semitica comparata nell’Europa rinascimentale: Le \textit{Institutiones} di Angelo \citet{Canini1554}. \textit{Annali} \textit{di} \textit{Ca’} \textit{Foscari:} \textit{Rivista} \textit{della} \textit{Facoltà} \textit{di} \textit{lingue} \textit{e} \textit{letterature} \textit{straniere} \textit{dell’Università} \textit{di} \textit{Venezia} (Serie orientale) 33(3). 39–56.

Cordier, Pierre. 2006. Geoffroy Tory et les leçons de l’Antique. \textit{Anabases:} \textit{Traditions} \textit{et} \textit{réceptions} \textit{de} \textit{l’Antiquité} 4. 11–32.

Covington, Michael A. 1979. Short note: Albert Schultens on language relationship. \textit{Linguistics} 17(7–8). 707–708.

Dahan, Gilbert, Irène Rosier \& Luisa Valente. 1995. L’arabe, le grec, l’hébreu et les vernaculaires. In Sten Ebbesen (ed.), \textit{Sprachtheorien} \textit{in} \textit{Spätantike} \textit{und} \textit{Mittelalter} (Geschichte der Sprachtheorie 3), 265–321. Tübingen: G. Narr.

de Jonge, Henk J. 1980. \textit{De} \textit{bestudering} \textit{van} \textit{het} \textit{Nieuwe} \textit{Testament} \textit{aan} \textit{de} \textit{Noordnederlandse} \textit{universiteiten} \textit{en} \textit{het} \textit{Remonstrants} \textit{Seminarie} \textit{van} \textit{1575} \textit{tot} \textit{1700} (Verhandelingen der Koninklijke Nederlandse Akademie van Wetenschappen, afdeling letterkunde, nieuwe reeks 106). Amsterdam, Oxford \& New York: Noord-Hollandse Uitgevers Maatschappij.

de Jonge, Henk J. 1981. The study of the New Testament in the Dutch universities, 1575–1700. \textit{History} \textit{of} \textit{universities} 1. 113–129.

de Vooys, C. G. N. 1917. Pontus de Heuiter, een taal- en spelling-hervormer uit de zestiende eeuw. \textit{De} \textit{nieuwe} \textit{taalgids} 11. 1–18.

Defaux, Gérard. 2003. Présentation. In Gérard Defaux (ed.), \textit{Lyon} \textit{et} \textit{l’illustration} \textit{de} \textit{la} \textit{langue} \textit{française} \textit{à} \textit{la} \textit{Renaissance}, 13–35. Lyon: ENS.

Demaizière, Colette. 1982. La langue à la recherche de ses origines: La mode des étymologies grecques. \textit{Bulletin} \textit{de} \textit{l’Association} \textit{d’étude} \textit{sur} \textit{l’humanisme,} \textit{la} \textit{réforme} \textit{et} \textit{la} \textit{renaissance} 15(1). 65–78.

Demaizière, Colette. 1988. Deux aspects de l’idéal linguistique d’Henri Estienne: Hellénisme et parisianisme. \textit{Henri} \textit{Estienne} (Collection de l’Ecole Normale Supérieure de Jeunes Filles 43: Cahiers V.L. Saulnier 5), 63–75. Paris: Ecole Normale Supérieure de Jeunes Filles.

Denecker, Tim. 2017. \textit{Ideas} \textit{on} \textit{language} \textit{in} \textit{early} \textit{Latin} \textit{Christianity:} \textit{From} \textit{Tertullian} \textit{to} \textit{Isidore} \textit{of} \textit{Seville} (Supplements to Vigiliae Christianae 142). Leiden \& Boston: Brill.

Dibbets, G. R. W. 2008. Taalgebruik in De Heuiters \textit{Nederduitse} \textit{orthographie} \REF{ex:key:1581} en Spiegels \textit{Twe-spraack} \textit{vande} \textit{Nederduitsche} \textit{letterkunst} \REF{ex:key:1584}. \textit{Voortgang:} \textit{Jaarboek} \textit{voor} \textit{de} \textit{neerlandistiek} 26. 107–144.

Dickey, Eleanor. 2007. \textit{Ancient} \textit{Greek} \textit{scholarship:} \textit{A} \textit{guide} \textit{to} \textit{finding,} \textit{reading,} \textit{and} \textit{understanding} \textit{scholia,} \textit{commentaries,} \textit{lexica,} \textit{and} \textit{grammatical} \textit{treatises,} \textit{from} \textit{their} \textit{beginnings} \textit{to} \textit{the} \textit{Byzantine} \textit{period} (American Philological Association classical resources series 7). Oxford \& New York: Oxford University Press.

Dini, Pietro U. 2004. Baltic palaeocomparativism and the idea that Prussian derives from Greek. In Philip Baldi \& Pietro U. Dini (eds.), \textit{Studies} \textit{in} \textit{Baltic} \textit{and} \textit{Indo-European} \textit{linguistics:} \textit{In} \textit{honor} \textit{of} \textit{William} \textit{R.} \textit{Schmalstieg} (Current issues in linguistic theory 254), 37–50. Amsterdam \& Philadelphia: John Benjamins.

Dionisotti, Carlo. 1968. \textit{Gli} \textit{umanisti} \textit{e} \textit{il} \textit{volgare} \textit{fra} \textit{Quattro} \textit{e} \textit{Cinquecento} (Bibliotechina del saggiatore 29). Firenze: Felice Le Monnier.

Droixhe, Daniel. 1978. \textit{La} \textit{linguistique} \textit{et} \textit{l’appel} \textit{de} \textit{l’histoire} \textit{(1600–1800):} \textit{Rationalisme} \textit{et} \textit{révolutions} \textit{positivistes} (Langue et cultures 10). Genève: Droz.

Droixhe, Daniel. 1980. Le prototype défiguré: L’idée scythique et la France gauloise (XVIIe–XVIIIe siècles). In E. F. K. Koerner (ed.), \textit{Progress} \textit{in} \textit{linguistic} \textit{historiography:} \textit{Papers} \textit{from} \textit{the} \textit{International} \textit{Conference} \textit{on} \textit{the} \textit{History} \textit{of} \textit{the} \textit{Language} \textit{Sciences} \textit{(Ottawa,} \textit{28–31} \textit{\citealt{August1978})} (Studies in the history of the language sciences 20), 123–137. Amsterdam: John Benjamins.

Edwards, John. 2009. \textit{Language} \textit{and} \textit{identity:} \textit{An} \textit{introduction} (Key topics in sociolinguistics). Cambridge: Cambridge University Press.

Eskhult, Josef. 2015. Albert Schultens (1686–1750) and primeval language: The crisis of a tradition and the turning point of a discourse. In Gerda Haßler (ed.), \textit{Metasprachliche} \textit{Reflexion} \textit{und} \textit{Diskontinuität:} \textit{Wendepunkte} \textit{–} \textit{Krisenzeiten} \textit{–} \textit{Umbrüche}, 72–94. Münster: Nodus.

Eskhult, Josef. Fc. \textit{Albert} \textit{Schultens’} \textit{(1686–1750)} \textit{comparative} \textit{Semitics:} \textit{Edition} \textit{of} \textit{main} \textit{treatises} \textit{and} \textit{texts} \textit{with} \textit{introduction,} \textit{translation,} \textit{and} \textit{commentary}.

Evans, T. V. \& D. D. Obbink (eds.). 2010. \textit{The} \textit{language} \textit{of} \textit{the} \textit{papyri}. Oxford \& New York: Oxford University Press.

Famerie, Étienne. 2007. Villoison et la redécouverte du dialecte tsakonien. \textit{Anabases:} \textit{Traditions} \textit{et} \textit{réceptions} \textit{de} \textit{l’Antiquité} 6. 235–248.

Finkelberg, Margalit. 2014. Dialects, classification of. In Georgios K. Giannakis (ed.), \textit{Encyclopedia} \textit{of} \textit{Ancient} \textit{Greek} \textit{language} \textit{and} \textit{linguistics}, vol. 1, 461–468. Leiden: Brill.

Fögen, Thorsten. 2000. \textit{Patrii} \textit{sermonis} \textit{egestas.} \textit{Einstellungen} \textit{lateinischer} \textit{Autoren} \textit{zu} \textit{ihrer} \textit{Muttersprache:} \textit{Ein} \textit{Beitrag} \textit{zum} \textit{Sprachbewusstsein} \textit{in} \textit{der} \textit{römischen} \textit{Antike}. München: K. G. Saur.

Förstel, Christian. 1999. Die griechische Grammatik im Umkreis Reuchlins: Untersuchungen zur >Wanderung< der griechischen Studien von Italien nach Deutschland. In Gerald Dörner (ed.), \textit{Reuchlin} \textit{und} \textit{Italien} (Pforzheimer Reuchlinschriften 7), 45–56. Stuttgart: Jan Thorbecke.

François, Alexis. 1905. \textit{La} \textit{grammaire} \textit{du} \textit{purisme} \textit{et} \textit{l’Académie} \textit{française} \textit{au} \textit{XVIII\textsuperscript{e}} \textit{siècle:} \textit{Introduction} \textit{à} \textit{l’étude} \textit{des} \textit{commentaires} \textit{grammaticaux} \textit{d’auteurs} \textit{classiques}. Paris: Bellais.

Fück, Johann. 1955. \textit{Die} \textit{arabischen} \textit{Studien} \textit{in} \textit{Europa} \textit{bis} \textit{in} \textit{den} \textit{Anfang} \textit{des} \textit{20.} \textit{Jahrhunderts}. Leipzig: Otto Harrassowitz.

Garrett, Peter. 2010. \textit{Attitudes} \textit{to} \textit{language} (Key topics in sociolinguistics). Cambridge: Cambridge University Press.

Gerretzen, Jan Gerard. 1940. \textit{Schola} \textit{Hemsterhusiana:} \textit{De} \textit{herleving} \textit{der} \textit{Grieksche} \textit{studiën} \textit{aan} \textit{de} \textit{Nederlandsche} \textit{universiteiten} \textit{in} \textit{de} \textit{achttiende} \textit{eeuw} \textit{van} \textit{Perizonius} \textit{tot} \textit{en} \textit{met} \textit{Valckenaer}. Nijmegen \& Utrecht: Dekker \& van de Vegt.

Giard, Luce. 1992. L’entrée en lice des vernaculaires. In Sylvain Auroux (ed.), \textit{Histoire} \textit{des} \textit{idées} \textit{linguistiques} (Philosophie et langage), vol. 2 [\textit{Le} \textit{développement} \textit{de} \textit{la} \textit{grammaire} \textit{occidentale}], 206–225. Liège: Mardaga.

Gitner, Adam. 2019. \textit{Sardismos}: A rhetorical term for bilingual or plurilingual interaction? \textit{The} \textit{classical} \textit{quarterly} 1–16.

Hackstein, Olav. 2010. The Greek of epic. In Egbert J. Bakker (ed.), \textit{A} \textit{companion} \textit{to} \textit{the} \textit{Ancient} \textit{Greek} \textit{language} (Blackwell companions to the ancient world), 401–423. Malden, Oxford \& Chichester: Wiley-Blackwell.

Hainsworth, J. B. 1967. Greek Views of Greek Dialectology. \textit{Transactions} \textit{of} \textit{the} \textit{Philological} \textit{Society} 65. 62–76.

Hardy, N. J. S. 2012. The enlightenments of Richard Bentley. \textit{History} \textit{of} \textit{universities} 26(2). 196–219.

Harris, Jonathan. 1995. \textit{Greek} \textit{emigres} \textit{in} \textit{the} \textit{West,} \textit{1400–1520}. Camberley: Porphyrogenitus.

Haßler, Gerda. 2009. Dialekt. In Gerda Haßler \& Cordula Neis (eds.), \textit{Lexikon} \textit{sprachtheoretischer} \textit{Grundbegriffe} \textit{des} \textit{17.} \textit{und} \textit{18.} \textit{Jahrhunderts}, vol. 1, 866–881. Berlin \& New York: Walter de Gruyter.

Haugen, Einar. 1966. Dialect, language, nation. \textit{American} \textit{anthropologist} 68(4). 922–935.

Haugen, Kristine Louise. 2011. \textit{Richard} \textit{Bentley:} \textit{Poetry} \textit{and} \textit{Enlightenment}. Cambridge \& London: Harvard University Press.

Horrocks, Geoffrey. 2010. \textit{Greek:} \textit{A} \textit{history} \textit{of} \textit{the} \textit{language} \textit{and} \textit{its} \textit{speakers}. Second edition. Malden, Oxford \& Chichester: Wiley-Blackwell.

Hoven, René. 1985. \textit{Bibliographie} \textit{de} \textit{trois} \textit{auteurs} \textit{de} \textit{grammaires} \textit{grecques} \textit{contemporains} \textit{de} \textit{Nicolas} \textit{Clénard:} \textit{Adrien} \textit{Amerot,} \textit{Arnold} \textit{Oridryus,} \textit{Jean} \textit{Varennius} (Livre – idées – société 7). Aubel: P. M. Gason.

Hummel, Pascale. 1999. Un opuscule-relais: Le \textit{De} \textit{dialectis} (1520/1530) d’Adrien Amerot. \textit{Bibliothèque} \textit{d’humanisme} \textit{et} \textit{Renaissance:} \textit{Travaux} \textit{et} \textit{documents} 61(2). 479–494.

Janse, Mark. 2007. The Greek of the New Testament. In A.-F. Christidis, Maria Arapopoulou \& Maria Chriti (eds.), \textit{A} \textit{history} \textit{of} \textit{Ancient} \textit{Greek:} \textit{From} \textit{the} \textit{beginnings} \textit{to} \textit{late} \textit{antiquity}, 646–653. Cambridge: Cambridge University Press.

Jansen, Jeroen. 1995. \textit{Brevitas:} \textit{Beschouwingen} \textit{over} \textit{de} \textit{beknoptheid} \textit{van} \textit{vorm} \textit{en} \textit{stijl} \textit{in} \textit{de} \textit{renaissance}. Vol. 1. 2 vols. Hilversum: Verloren.

Jellinek, Max Hermann. 1898. Über die schrift des Hieronymus Wolf De orthographia Germanica, ac potius Suevica nostrate. \textit{Zeitschrift} \textit{für} \textit{deutsche} \textit{Philologie} 30. 251–255.

Jellinek, Max Hermann. 1913–1914. \textit{Geschichte} \textit{der} \textit{neuhochdeutschen} \textit{Grammatik} \textit{von} \textit{den} \textit{Anfängen} \textit{bis} \textit{auf} \textit{Adelung}. 2 vols. Heidelberg: Carl Winter.

Jenkins, Gary W. 2006. \textit{John} \textit{Jewel} \textit{and} \textit{the} \textit{English} \textit{national} \textit{church:} \textit{The} \textit{dilemmas} \textit{of} \textit{an} \textit{Erastian} \textit{reformer}. Aldershot \& Burlington: Ashgate.

Jones, William Jervis. 1999. \textit{Images} \textit{of} \textit{language:} \textit{Six} \textit{essays} \textit{on} \textit{German} \textit{attitudes} \textit{to} \textit{European} \textit{languages} \textit{from} \textit{1500} \textit{to} \textit{1800} (Studies in the history of the language sciences 89). Amsterdam \& Philadelphia: John Benjamins.

Jones, William Jervis. 2001. Early dialectology, etymology and language history in German-speaking countries. In Sylvain Auroux \textit{et} \textit{al.} (eds.), \textit{History} \textit{of} \textit{the} \textit{language} \textit{sciences:} \textit{An} \textit{international} \textit{handbook} \textit{on} \textit{the} \textit{evolution} \textit{of} \textit{the} \textit{study} \textit{of} \textit{language} \textit{from} \textit{the} \textit{beginnings} \textit{to} \textit{the} \textit{present} (Handbücher zur Sprach- und Kommunikationswissenschaft 18), vol. 2, 1105–1115. Berlin \& New York: Walter de Gruyter.

Kessler-Mesguich, Sophie. 2013. \textit{Les} \textit{études} \textit{hébraïques} \textit{en} \textit{France:} \textit{De} \textit{François} \textit{Tissard} \textit{à} \textit{Richard} \textit{Simon} \textit{(1508–1680)} (Travaux d’humanisme et Renaissance 517). Avant-propos de Max Engammare. Genève: Droz.

Kökeritz, Helge. 1938. Alexander \citet{Gill1621} on the dialects of South and East England. \textit{Studia} \textit{neophilologica} 11(1–2). 277–288.

Lallot, Jean. 1995. Analogie et pathologie dans la grammaire alexandrine. \textit{Lalies} 15. 109–123.

Lambert, Frédéric. 2009. Les noms des langues chez les Grecs. \textit{Histoire} \textit{épistémologie} \textit{langage} 31(2). 15–27.

Lamers, Han. 2015. \textit{Greece} \textit{reinvented:} \textit{Transformations} \textit{of} \textit{Byzantine} \textit{Hellenism} \textit{in} \textit{Renaissance} \textit{Italy} (Brill’s studies in intellectual history 247). Leiden: Brill.

Lamers, Han. 2019. Janus Lascaris’ Florentine oration and the “reception” of ancient Aeolism. In Giancarlo Abbamonte \& Stephen Harrison (eds.), \textit{Making} \textit{and} \textit{rethinking} \textit{the} \textit{Renaissance:} \textit{Between} \textit{Greek} \textit{and} \textit{Latin} \textit{in} \textit{15th–16th} \textit{century} \textit{Europe} (Trends in classics), 27–50. Berlin: De Gruyter.

Lepschy, Giulio C. 2002. \textit{Mother} \textit{tongues} \textit{and} \textit{other} \textit{reflections} \textit{on} \textit{the} \textit{Italian} \textit{language} (Toronto Italian studies). Toronto, Buffalo \& London: University of Toronto Press.

Liddel, Peter. 2014. From chronology to liberal imperialism: Greek inscriptions, the history of Greece, and historiography from Selden to Grote. \textit{Journal} \textit{of} \textit{the} \textit{history} \textit{of} \textit{collections} 26(3). 387–398.

Mackridge, Peter. 2009. Mothers and daughters, roots and branches: Modern Greek perceptions of the relationship between the ancient and modern languages. In Alexandra Georgakopoulou \& Michael Silk (eds.), \textit{Standard} \textit{languages} \textit{and} \textit{language} \textit{standards:} \textit{Greek,} \textit{past} \textit{and} \textit{present} (The Centre for Hellenic Studies, King’s College London, publications 12), 259–276. Farnham \& Burlington: Ashgate.

Mackridge, Peter. 2014. The Greek language since 1750. In Caterina Carpinato \& Olga Tribulato (eds.), \textit{Storia} \textit{e} \textit{storie} \textit{della} \textit{lingua} \textit{greca} (Antichistica: Filologia e letteratura 5/1), 133–164. Venezia: Ca’ Foscari.

Makrides, Vasilios N. 2006. Greek Orthodox compensatory strategies towards Anglicans and the West at the beginning of the eighteenth century. In Peter M. Doll (ed.), \textit{Anglicanism} \textit{and} \textit{Orthodoxy} \textit{300} \textit{years} \textit{after} \textit{the} \textit{“Greek} \textit{College”} \textit{in} \textit{Oxford}, 249–287. Oxford: Peter Lang.

Mattheier, Klaus J. 2003. German. In Ana Deumert \& Wim Vandenbussche (eds.), \textit{Germanic} \textit{standardizations:} \textit{Past} \textit{to} \textit{present} (Impact: Studies in language and society 18), 211–244. Amsterdam \& Philadelphia: John Benjamins.

McColl Millar, Robert. 2007. \textit{Northern} \textit{and} \textit{Insular} \textit{Scots} (Dialects of English). Edinburgh: Edinburgh University Press.

McInerney, Jeremy. 2012. Heraclides Criticus and the problem of taste. In Ineke Sluiter \& Ralph M. Rosen (eds.), \textit{Aesthetic} \textit{value} \textit{in} \textit{classical} \textit{antiquity} (Mnemosyne supplements: Monographs on Greek and Latin language and literature 350), 243–264. Leiden \& Boston: Brill.

Melzi, Robert C. 1966. \textit{Castelvetro’s} \textit{Annotations} \textit{to} \textit{the} \textit{Inferno:} \textit{A} \textit{new} \textit{perspective} \textit{in} \textit{sixteenth} \textit{century} \textit{criticism} (Studies in Italian literature 1). The Hague \& Paris: Mouton.

Metcalf, George J. 2013. \textit{On} \textit{language} \textit{diversity} \textit{and} \textit{relationship} \textit{from} \textit{Bibliander} \textit{to} \textit{Adelung} (Studies in the history of the language sciences 120). (Ed.) Toon Van Hal \& Raf Van Rooy. Amsterdam \& Philadelphia: John Benjamins.

Moennig, Ulrich. 1998. Die griechischen Studenten am Hallenser Collegium orientale theologicum. In Johannes Wallmann \& Udo Sträter (eds.), \textit{Halle} \textit{und} \textit{Osteuropa:} \textit{Zur} \textit{europäischen} \textit{Ausstrahlung} \textit{des} \textit{hallischen} \textit{Pietismus} (Hallesche Forschungen 1), 299–329. Halle \& Tübingen: Verlag der Franckeschen Stiftungen \& Max Niemeyer.

Morpurgo Davies, Anna. 1987. The Greek notion of dialect. \textit{Verbum} 10. 7–28.

Muller, Jean-Claude. 1984. Saumaise, Monboddo, Adelung: Vers la grammaire comparée. In Sylvain Auroux \textit{et} \textit{al.} (eds.), \textit{Matériaux} \textit{pour} \textit{une} \textit{histoire} \textit{des} \textit{théories} \textit{linguistiques} {\textbar} \textit{Essays} \textit{toward} \textit{a} \textit{history} \textit{of} \textit{linguistic} \textit{theories} {\textbar} \textit{Materialien} \textit{zu} \textit{einer} \textit{Geschichte} \textit{der} \textit{sprachwissenschaftlichen} \textit{Theorien} (Travaux et recherches), 389–396. Lille: Université de Lille III, diffusion P.U.L.

Nuti, Erika. 2013. Reconsidering Renaissance Greek grammars through the case of Chrysoloras’ \textit{Erotemata}. \textit{Greek,} \textit{Roman,} \textit{and} \textit{Byzantine} \textit{studies} 53(1). 240–268.

Padley, G. A. 1985. \textit{Grammatical} \textit{theory} \textit{in} \textit{Western} \textit{\citealt{Europe1500}–1700: Trends in vernacular grammar I}. Cambridge: Cambridge University Press.

Päll, Janika \& Ivo Volt. 2018. \textit{Hellenostephanos.} \textit{Humanist} \textit{Greek} \textit{in} \textit{early} \textit{modern} \textit{Europe:} \textit{Learned} \textit{communities} \textit{between} \textit{antiquity} \textit{and} \textit{contemporary} \textit{culture} (Morgensterni Seltsi Toimetised {\textbar} Acta Societatis Morgensternianae 6–7). Tartu: University of Tartu Press.

Peeters, Leopold. 1982. Auteurschap en tekst van “Spiegels” “Twe-spraack” \REF{ex:key:1584}. \textit{Tijdschrift} \textit{voor} \textit{Nederlandse} \textit{taal-} \textit{en} \textit{letterkunde} 98(1). 117–129.

Peters, Manfred. 1970–1971. \textit{Conrad} \textit{Gessner} \textit{als} \textit{Germanist} \textit{und} \textit{Linguist}. 4 vols. Gent: unpublished PhD dissertation (Rijksuniversiteit Gent).

Pfaffel, Wilhelm. 1981. Quartus gradus etymologiae\textit{:} \textit{Untersuchungen} \textit{zur} \textit{Etymologie} \textit{Varros} \textit{in} \textit{“De} \textit{lingua} \textit{Latina”} (Beiträge zur klassischen Philologie 131). Königstein/Ts.: Anton Hain.

Porter, Stanley E. \& Andrew W. Pitts (eds.). 2013. \textit{The} \textit{language} \textit{of} \textit{the} \textit{New} \textit{Testament:} \textit{Context,} \textit{history,} \textit{and} \textit{development}. Leiden \& Boston: Brill.

Pott, August Friedrich. 1974 [1884–1890]. Einleitung in die Allgemeine Sprachwissenschaft \textit{(1884–1890)} \textit{together} \textit{with} Zur Literatur der Sprachenkunde Europas \textit{(Leipzig,} \textit{1887)} (Amsterdam classics in linguistics, 1800–1925 10). (Ed.) E. F. K. Koerner. Amsterdam: John Benjamins.

Preston, Dennis R. 2018. Perceptual dialectology. In Charles Boberg, John Nerbonne \& Dominic Watt (eds.), \textit{The} \textit{handbook} \textit{of} \textit{dialectology} (Blackwell handbooks in linguistics), 177–203. Hoboken, NJ: Wiley Blackwell.

Regoliosi, Mariangela. 1993. \textit{Nel} \textit{cantiere} \textit{del} \textit{Valla:} \textit{Elaborazione} \textit{e} \textit{montaggio} \textit{delle} \textit{«Elegantiae»} (Humanistica 13). Roma: Bulzoni.

Reynolds, L. D. \& N. G. Wilson. 1991. \textit{Scribes} \textit{and} \textit{scholars:} \textit{A} \textit{guide} \textit{to} \textit{the} \textit{transmission} \textit{of} \textit{Greek} \textit{and} \textit{Latin} \textit{literature}. Third edition. Oxford: Clarendon Press.

Rhoby, Andreas. 2002. Beitrag zur Geschichte Athens im späten 16. Jahrhundert: Untersuchung der Briefe des Theodosios Zygomalas und Symeon Kabasilas an Martin Crusius. \textit{Medioevo} \textit{greco} 2. 177–191.

Rhodes, Neil. 2015. Pure and common Greek in early Tudor England. In Tania Demetriou \& Rowan Tomlinson (eds.), \textit{The} \textit{culture} \textit{of} \textit{translation} \textit{in} \textit{early} \textit{modern} \textit{England} \textit{and} \textit{France,} \textit{1500–1660} (Early modern literature in history), 54–70. Basingstoke \& New York: Palgrave Macmillan.

Ricciardi, Roberto. 1986. Da Ponte, Ludovico. \textit{Dizionario} \textit{Biografico} \textit{degli} \textit{Italiani}, vol. 32. Online edition. Last accessed February 27, 2019.

<http://www.treccani.it/enciclopedia/ludovico-da-ponte\_\%28Dizionario-Biografico\%29/>.

Roelcke, Thorsten. 2014. \textit{Latein,} \textit{Griechisch,} \textit{Hebräisch:} \textit{Studien} \textit{und} \textit{Dokumentationen} \textit{zur} \textit{deutschen} \textit{Sprachreflexion} \textit{in} \textit{Barock} \textit{und} \textit{Aufklärung}. Berlin \& Boston: de Gruyter.

Rotolo, Vincenzo. 1973–1974. L’opinione di F. Filelfo sul greco volgare. \textit{Rivista} \textit{di} \textit{studi} \textit{bizantini} \textit{e} \textit{neoellenici} 20-21 (n.s. 10–11). 85–107.

Ruijgh, Cornelis J. 2011. Mycenaean and Homeric language. In Yves Duhoux \& Anna Morpurgo Davies (eds.), \textit{A} \textit{companion} \textit{to} \textit{Linear} \textit{B:} \textit{Mycenaean} \textit{Greek} \textit{texts} \textit{and} \textit{their} \textit{world} (Bibliothèque des cahiers de l’Institut de Linguistique de Louvain 127), vol. 2, 253–298. Louvain-la-Neuve: Peeters.

Rutten, Gijsbert. 2016. Historicizing diaglossia. \textit{Journal} \textit{of} \textit{sociolinguistics} 20(1). 6–30.

Said, Edward W. 2003 [1978]. \textit{Orientalism}. London et al.: Penguin.

Saladin, Jean-Christophe. 2000. \textit{La} \textit{bataille} \textit{du} \textit{grec} \textit{à} \textit{la} \textit{Renaissance}. Paris: Les belles lettres.

Sandys, John Edwin. 1908. \textit{A} \textit{history} \textit{of} \textit{classical} \textit{scholarship}. 3 vols. Cambridge: at the University Press.

Schlieben-Lange, Brigitte. 1992. Reichtum, Energie, Klarheit und Harmonie: Die Bewertung der Sprachen in Begriffen der Rhetorik. In Susanne R. Anschütz (ed.), \textit{Texte,} \textit{Sätze,} \textit{Wörter} \textit{und} \textit{Moneme:} \textit{Festschrift} \textit{für} \textit{Klaus} \textit{Heger} \textit{zum} \textit{65.} \textit{Geburtstag}, 571–586. Heidelberg: Heidelberger Orientverlag.

Schöpsdau, Klaus. 1992. Vergleiche zwischen Lateinisch und Griechisch in der antiken Sprachwissenschaft. In Carl Werner Müller, Kurt Sier \& Jürgen Werner (eds.), \textit{Zum} \textit{Umgang} \textit{mit} \textit{fremden} \textit{Sprachen} \textit{in} \textit{der} \textit{griechisch-römischen} \textit{Antike}, 115–136. Stuttgart: Franz Steiner.

Siebenborn, Elmar. 1976. \textit{Die} \textit{Lehre} \textit{von} \textit{der} \textit{Sprachrichtigkeit} \textit{und} \textit{ihren} \textit{Kriterien:} \textit{Studien} \textit{zur} \textit{antiken} \textit{normativen} \textit{Grammatik}. Amsterdam: B. R. Grüner.

Silverstein, Michael. 2003. Indexical order and the dialectics of sociolinguistic life. \textit{Language} \textit{\&} \textit{communication} 23. 193–229.

Stenhouse, William. 2005. \textit{Reading} \textit{inscriptions} \textit{and} \textit{writing} \textit{ancient} \textit{history:} \textit{Historical} \textit{scholarship} \textit{in} \textit{the} \textit{late} \textit{Renaissance} (Bulletin of the Institute of Classical Studies, supplement 86). London: Institute of Classical Studies.

Stenhouse, William. Fc. The Greekness of Greek inscriptions: Ancient inscriptions in early modern scholarship. In Natasha Constantinidou \& Han Lamers (eds.), \textit{Receptions} \textit{of} \textit{Hellenism} \textit{in} \textit{early} \textit{modern} \textit{Europe:} \textit{Transmission,} \textit{representation,} \textit{and} \textit{exchange}. Leiden \& Boston: Brill.

Stevens, Benjamin. 2006/2007. Aeolism: Latin as a dialect of Greek. \textit{The} \textit{classical} \textit{journal} 102(2). 115–144.

Swiggers, Pierre. 1984. Adrianus Schrieckius: De la langue des Scythes à l’Europe linguistique. \textit{Histoire} \textit{épistémologie} \textit{langage} 6(2). 17–35. [Special issue: Daniel Droixhe (ed.), \textit{Genèse} \textit{du} \textit{comparatisme} \textit{indo-européen}.]

Swiggers, Pierre. 1997. Français, italien (et espagnol): Un concours de “précellence” chez Henri Estienne. In Günter Holtus, Johannes Kramer \& Wolfgang Schweickard (eds.), \textit{Italica} \textit{et} \textit{Romanica:} \textit{Festschrift} \textit{für} \textit{Max} \textit{Pfister} \textit{zum} \textit{65.} \textit{Geburtstag}, vol. 2, 297–311. Tübingen: Max Niemeyer.

Swiggers, Pierre. 1998. \textit{Van} \textit{t’beghin} \textit{der} \textit{eerster} \textit{volcken} \textit{van} \textit{Europen} \REF{ex:key:1614}: Kelten en Scythen bij Adrianus Schrieckius. In Lauran Toorians (ed.), \textit{Kelten} \textit{en} \textit{de} \textit{Nederlanden} \textit{van} \textit{prehistorie} \textit{tot} \textit{heden}, 123–147. Leuven \& Paris: Peeters.

Swiggers, Pierre. 2009. Le français et l’italien en lice: L’examen comparatif de leurs qualités chez Henri Estienne. \textit{Synergies} \textit{Italie} 5. 69–76.

Tavoni, Mirko. 1986. On the Renaissance idea that Latin derives from Greek. \textit{Annali} \textit{della} \textit{Scuola} \textit{normale} \textit{superiore} \textit{di} \textit{Pisa:} \textit{Classe} \textit{di} \textit{lettere} \textit{e} \textit{filosofia} 16(1). 205–238.

Taylor, Daniel J. 1996. \textit{De} \textit{lingua} \textit{Latina} \textit{X:} \textit{A} \textit{new} \textit{critical} \textit{text} \textit{and} \textit{English} \textit{translation} \textit{with} \textit{prolegomena} \textit{and} \textit{commentary} (Studies in the history of the language sciences 85). Amsterdam \& Philadelphia: John Benjamins.

Toufexis, Panagiotis. 2005. \textit{Das} Alphabetum vulgaris linguae graecae \textit{des} \textit{deutschen} \textit{Humanisten} \textit{Martin} \textit{Crusius} \textit{(1526–1607):} \textit{Ein} \textit{Beitrag} \textit{zur} \textit{Erforschung} \textit{der} \textit{mündlichen} \textit{griechischen} \textit{Sprache} \textit{im} \textit{16.} \textit{Jh.} (Neograeca medii aevi 8). Köln: Romiosini.

Trapp, Joseph Burney. 1990. The conformity of Greek with the vernacular: The history of a Renaissance theory of languages. \textit{Essays} \textit{on} \textit{the} \textit{Renaissance} \textit{and} \textit{the} \textit{Classical} \textit{Tradition} (Variorum collected studies 323), 8–21. Hampshire \& Vermont: Variorum.

Tribulato, Olga. 2010. Literary dialects. In Egbert J. Bakker (ed.), \textit{A} \textit{companion} \textit{to} \textit{the} \textit{Ancient} \textit{Greek} \textit{language} (Blackwell companions to the ancient world), 388–400. Malden, Oxford \& Chichester: Wiley-Blackwell.

Trovato, Paolo. 1984. “Dialetto” e sinonimi (‘idioma’’, “proprietà”, ’lingua’) nella terminologia linguistica quattro- e cinquecentesca (con un’appendice sulla tradizione a stampa dei trattatelli dialettologici bizantini).’ \textit{Rivista} \textit{di} \textit{letteratura} \textit{italiana} 2. 205–236.

Trudeau, Danielle. 1983. L’ordonnance de Villers-Cotterêts et la langue française: Histoire ou interprétation? \textit{Bibliothèque} \textit{d’humanisme} \textit{et} \textit{Renaissance:} \textit{Travaux} \textit{et} \textit{documents} 45(3). 461–472.

Uguzzoni, Arianna \& Franco Ghinatti. 1968. \textit{Le} \textit{tavole} \textit{greche} \textit{di} \textit{Eraclea} (Pubblicazioni dell’Istituto di storia antica 7). Roma: “L’Erma” di Bretschneider.

Van Hal, Toon. 2010a. \textit{“Moedertalen} \textit{en} \textit{taalmoeders”:} \textit{Het} \textit{vroegmoderne} \textit{taalvergelijkende} \textit{onderzoek} \textit{in} \textit{de} \textit{Lage} \textit{Landen} (Verhandelingen van de Koninklijke Vlaamse Academie van België voor Wetenschappen en Kunsten: Nieuwe reeks 20). Brussel: Koninklijke Vlaamse Academie voor Wetenschappen en Kunsten.

Van Hal, Toon. 2010b. On “the Scythian theory”: Reconstructing the outlines of Johannes Elichmann’s (1601/1602–1639) planned \textit{Archaeologia} \textit{Harmonica}. \textit{Language} \textit{\&} \textit{history} 53(2). 70–80.

Van Hal, Toon. 2013. $\Gamma \lambda \text{\textgreek{~w}}\tau \tau \alpha $ $\kappa \alpha \text{\textgreek{`i}}$ $\delta \text{\textgreek{'i}}\alpha \iota \tau \alpha $ {\textbar} Lingua et mores {\textbar} Sprache und Sitten: Eine jahrhundertelange Verbindung. In S. Große \textit{et} \textit{al.} (eds.), \textit{Angewandte} \textit{Linguistik} \textit{{\textbar} Linguistique appliquée: Zwischen Theorien, Konzepten und der Beschreibung sprachlicher Äußerungen {\textbar} Entre théories, concepts et la description des expressions linguistiques}, 21–30. Frankfurt am Main, Berlin, Bern, Bruxelles, New York, Oxford \& Wien: Peter Lang.

Van Hal, Toon. 2016. Bevoorrechte betrekkingen tussen het Grieks en het Germaans: Taalvergelijkende verkenningen in het Vroegduitse humanisme en in Wilhelm Otto Reitz’ \textit{Belga} \textit{graecissans} \REF{ex:key:1730}. \textit{Leuvense} \textit{bijdragen} \textit{–} \textit{Leuven} \textit{contributions} \textit{in} \textit{linguistics} \textit{and} \textit{philology} 99–100. 427–443. [Special issue: Kurt Feyaerts \textit{et} \textit{al.} (eds.), \textit{Sprache} \textit{in} \textit{Raum} \textit{und} \textit{Geschichte,} \textit{System} \textit{und} \textit{Kultur:} \textit{Festschrift} \textit{für} \textit{Luc} \textit{Draye}.]

Van Rooy, Raf. 2013. “πó$\theta \varepsilon \nu $ o$\text{\textgreek{>~u}}\nu $ $\text{\textgreek{<h}}$ τo$\sigma \alpha \text{\textgreek{'u}}\tau \eta $ $\delta \iota \alpha \varphi \omega \nu \text{\textgreek{'i}}\alpha $;” Greek patristic authors discussing linguistic origin, diversity, change and kinship. \textit{Beiträge} \textit{zur} \textit{Geschichte} \textit{der} \textit{Sprachwissenschaft} 23(1). 21–54.

Van Rooy, Raf. 2014. A first stumbling step toward Ancient Greek dialectology in Western Europe: An edition and brief discussion of Johann Reuchlin’s \textit{De} \textit{quattuor} \textit{Graecae} \textit{linguae} \textit{differentiis} \textit{libellus} (1477/1478). \textit{Bibliothèque} \textit{d’humanisme} \textit{et} \textit{Renaissance:} \textit{Travaux} \textit{et} \textit{documents} 76(3). 501–526.

Van Rooy, Raf. 2016a. Struggling to order diversity: The variegated classifications of Greek dialects before the rise of modern linguistics. \textit{Studies} \textit{in} \textit{Greek} \textit{linguistics} 36. 465–473.

Van Rooy, Raf. 2016b. Teaching Greek grammar in 11th-century Constantinople: Michael Psellus on the Greek “dialects.” \textit{Byzantinische} \textit{Zeitschrift} 109(1). 207–222.

Van Rooy, Raf. 2016c. The diversity of Ancient Greek through the eyes of a forgotten grammarian: Petrus Antesignanus (ca. 1524/1525–1561) on the notion of «dialect». \textit{Histoire} \textit{épistémologie} \textit{langage} 38(1). 123–140.

Van Rooy, Raf. 2016d. “What is a ‘dialect’?” Some new perspectives on the history of the term $\delta \iota \text{\textgreek{'a}}\lambda \varepsilon \kappa \tau o\varsigma $ and its interpretations in ancient Greece and Byzantium. \textit{Glotta:} \textit{Zeitschrift} \textit{für} \textit{griechische} \textit{und} \textit{lateinische} \textit{Sprache} 92. 244–279.

Van Rooy, Raf. 2017. \textit{Through} \textit{the} \textit{vast} \textit{labyrinth} \textit{of} \textit{languages} \textit{and} \textit{dialects:} \textit{The} \textit{emergence} \textit{and} \textit{transformations} \textit{of} \textit{a} \textit{conceptual} \textit{pair} \textit{in} \textit{the} \textit{early} \textit{modern} \textit{period} \textit{(ca.} \textit{1478–1782)}. Leuven: unpublished PhD dissertation (KU Leuven).

Van Rooy, Raf. 2018a. Latin as a variable language: Livy’s \textit{Patauinitas} through early modern eyes. In Astrid Steiner-Weber \& Franz Römer (eds.), \textit{Acta} \textit{Conventus} \textit{Neo-Latini} \textit{Vindobonensis:} \textit{Proceedings} \textit{of} \textit{the} \textit{sixteenth} \textit{International} \textit{Congress} \textit{of} \textit{Neo-Latin} \textit{Studies} \textit{\citep{Vienna2015}}, 767–777. Leiden \& Boston: Brill.

Van Rooy, Raf. 2018b. Regional language variation in European thought before 1500: A historical sketch reflecting current research. \textit{Beiträge} \textit{zur} \textit{Geschichte} \textit{der} \textit{Sprachwissenschaft} 28(2). 167–216.

Van Rooy, Raf. 2018c. « Plutarque dialectologue »: La pseudo-autorité de Plutarque dans le discours sur les dialectes grecs à la Renaissance (ca. 1400–1670). \textit{Belgisch} \textit{tijdschrift} \textit{voor} \textit{filologie} \textit{en} \textit{geschiedenis} \textit{{\textbar} Revue belge de philologie et d’histoire} 96. 1115–1134.

Van Rooy, Raf. Fc. a. A Latin defence of early modern Greek culture: Alexander Helladius’s \textit{Status} \textit{praesens} \REF{ex:key:1714} and its linguistic arguments. In Ioannis Deligiannis, Vasileios Pappas \& Vaios Vaiopoulos (eds.), \textit{Post-Byzantine} \textit{Latinitas:} \textit{Latin} \textit{in} \textit{post-Byzantine} \textit{scholarship} \textit{(15th–19th} \textit{cent.)}.

Van Rooy, Raf. Fc. b. A professor at work: Hadrianus Amerotius (1490s–1560) and the study of Greek in sixteenth-century Louvain. In Natasha Constantinidou \& Han Lamers (eds.), \textit{Receptions} \textit{of} \textit{Hellenism} \textit{in} \textit{early} \textit{modern} \textit{Europe:} \textit{Transmission,} \textit{representation,} \textit{and} \textit{exchange}. Leiden \& Boston: Brill.

Van Rooy, Raf. Fc. c. The riddle of the Greek language unraveled by a Renaissance Oedipus: Otto Walper and his manual on the Greek dialects \REF{ex:key:1589}. In Raf Van Rooy (ed.), \textit{Essays} \textit{in} \textit{the} \textit{history} \textit{of} \textit{dialect} \textit{studies:} \textit{From} \textit{ancient} \textit{Greece} \textit{to} \textit{modern} \textit{dialectology}. Münster: Nodus.

Van Rooy, Raf. Fc. d. \textit{Language} \textit{or} \textit{dialect?} \textit{The} \textit{history} \textit{of} \textit{a} \textit{conceptual} \textit{pair}. Oxford: Oxford University Press.

Van Rooy, Raf. Fc. e. Tradition, synthesis, and innovation: An early eighteenth-century dissertation on dialects. In Marti Hanspeter \& Robert Seidel (eds.), \textit{Early} \textit{modern} \textit{disputations} \textit{and} \textit{dissertations} \textit{in} \textit{an} \textit{interdisciplinary} \textit{and} \textit{European} \textit{context}.

Van Rooy, Raf \& John Considine. 2016. Between homonymy and polysemy: The origins and career of the English form \textit{dialect} in the sixteenth century. \textit{Anglia:} \textit{Journal} \textit{of} \textit{English} \textit{philology} 134(4). 639–667.

Versteegh, Kees. 1986. Latinitas, Hellenismos, ’Arabiyya. \textit{Historiographia} \textit{linguistica} 13(2–3). 425–448.

Vickers, Michael. 2006. \textit{The} \textit{Arundel} \textit{and} \textit{Pomfret} \textit{Marbles} (Ashmolean handbooks). Oxford: Ashmolean Museum.

Villani, Francesco Paolo. 2003. Scythae: Un problema linguistico, etnografico e culturale dell’età moderna. \textit{Linguistica} 1. 443–491.

Vingopoulou, Ioli. 2004. \textit{Le} \textit{monde} \textit{grec} \textit{vu} \textit{par} \textit{les} \textit{voyageurs} \textit{du} \textit{XVI\textsuperscript{e}} \textit{siècle} (Collection histoire des idées 4). Athènes: Institut de Recherches Néohelléniques.

von Raumer, Rudolf. 1856. Die Schrift des Hieronymus Wolf De orthographia Germanica, ac potius Suevica nostrate in ihrer Beziehung zur neuhochdeutschen Schriftsprache. \textit{Germania:} \textit{Vierteljahrsschrift} \textit{für} \textit{deutsche} \textit{Alterthumskunde} 1. 160–165.

Wackernagel, Jacob. 1876. \textit{De} \textit{pathologiae} \textit{ueterum} \textit{initiis:} \textit{Dissertatio} \textit{inauguralis} [...]. Basileae: typis expressit Officina Schweighauseriana.

Weiss, Emmanuel. 2016. \textit{Les} \textit{Tables} \textit{d’Héraclée:} \textit{Étude} \textit{historique} \textit{et} \textit{linguistique} (Travaux et mémoires: Etudes anciennes 63). Nancy \& Paris: A.D.R.A. \& De Boccard.

Weiss, Roberto. 1977. \textit{Medieval} \textit{and} \textit{humanist} \textit{Greek:} \textit{Collected} \textit{essays} (Medioevo e umanesimo 8). Padova: Antenore.

Whitmarsh, Tim. 2005. \textit{The} \textit{Second} \textit{Sophistic} (Greece \& Rome: New surveys in the classics 35). Oxford: Oxford University Press.

Wilson, N. G. 1992. \textit{From} \textit{Byzantium} \textit{to} \textit{Italy:} \textit{Greek} \textit{studies} \textit{in} \textit{the} \textit{Italian} \textit{Renaissance}. London: Duckworth.

Xhardez, Didier. 1991. \textit{$\Pi} \textit{\varepsilon} \textit{\rho} \textit{\text{\textgreek{`i}}$ $\delta \iota \alpha \lambda \text{\textgreek{'e}}\kappa \tau \omega \nu $: Le traité des dialectes de Grégoire de Corinthe édité et traduit. Contribution aux méthodes d’analyse d’une tradition textuelle}. 4 vols. Louvain-la-Neuve: unpublished PhD dissertation (Université catholique de Louvain).

\section{Indices}
\hypertarget{Toc19704873}{}
planned:

\begin{itemize}
\item \begin{styleListParagraph}
historical persons
\end{styleListParagraph}
\item \begin{styleListParagraph}
languages \& dialects
\end{styleListParagraph}
\item \begin{styleListParagraph}
(subjects, if desired, but the book is already structured thematically)
\end{styleListParagraph}
\end{itemize}

\begin{verbatim}%%move bib entries to  localbibliography.bib
\end{verbatim}