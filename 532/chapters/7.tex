\chapter{\textit{Ein} and number}\label{sec:7}

\section{Introduction}\label{sec:7.1}

In this chapter, I continue the investigation begun in \chapref{sec:5} and turn to the second and final consequence of the proposals about \textit{ein}. I argue that \textit{ein} is not a reflex of number, either morphologically or semantically. This is consistent with the claim that \textit{ein} is semantically vacuous (Hypothesis 1a).

\subsection{A brief review}\label{sec:7.1.1}

Recall from \chapref{sec:6} that true minimal pairs between the DP and TP are not possible in singular, out-of-the-blue contexts. Specifically, although a bare role noun such as \textit{Bauer} is possible in both the DP and TP, it crucially has different interpretations: It is emotive/figurative in the DP (meaning ‘peasant’) and neutral/literal in the TP (meaning ‘farmer’). In other words, bare nouns are fairly restricted with regard to (non-)emotiveness in the two domains. I proposed that factors involving both pragmatics and semantics explain these restrictions, the former (Gricean maxims) being particularly relevant for the DP and the latter (the operator CAP) for the TP. When \textit{ein} appears in front of a role noun, the facts fall differently.



\largerpage[1]
In particular, I noted that \textit{ein} cannot surface in pronominal DPs. Compare \REF{ex:7:1a} to \REF{ex:7:1b}. In order to indicate that \textit{ein} is the indefinite article here, I provide the latter in its reduced, unstressable form \textit{’n}.\footnote{As might be expected, the singularity numeral \textit{EIN} ‘one’ is also ungrammatical here \REF{ex:7:1:ia}, but adjectival \textit{eine} is not \REF{ex:7:1:ib}. Note that the second example involves a contrast cancelling the duality presupposition.

  	\ea
  	\ea[*]{ \label{ex:7:1:ia}
  	\gll  du   EIN Bauer\\
  	you one peasant.\textsc{masc}\\
  	}
  	\ex[]{ \label{ex:7:1:ib}
  	\gll wir zwei Bauern und du   einer Bauer\\
  	we two  farmers and you one   farmer.\textsc{masc}\\
  	\glt ‘us two farmers and you one farmer’
  	}
  	\z
  	\z

    In the main text below, \REF{ex:7:1b} is ruled out due to the presence of two determiners in one DP. Given the composite analysis of \textit{EIN} ‘one’ in German (\chapref{sec:5}), the same explanation extends to \REF{ex:7:1:ia}. As adjectival \textit{eine} is a lexical item, morphologically unrelated to \textit{ein}, it can occur after a definite determiner as in \REF{ex:7:1:ib}.
  	}

  	 \ea%1
    \label{ex:7:1}
\ea \label{ex:7:1a}
\gll du            Bauer\\
you(\textsc{sg}) peasant.\textsc{masc}\\
\glt ‘you (peasant)’
\ex[*]{ \label{ex:7:1b}
\gll du           ’n Bauer\\
you(\textsc{sg}) {\db}a peasant.\textsc{masc}\\
}
\z
\z

However, \textit{ein} is possible in predicate nominals in the TP. Specifically, if \textit{ein} appears in these copular cases, the role noun may undergo an interpretative extension from its neutral/literal meaning in \REF{ex:7:2a} to an emotive/figurative reading in \REF{ex:7:2b}.

\ea%2
    \label{ex:7:2}
\ea\label{ex:7:2a}
\gll Du           bist Bauer.\\
you(\textsc{sg}) are farmer.\textsc{masc}\\
\glt ‘You are a farmer.’
\ex\label{ex:7:2b}
\gll Du           bist ’n Bauer.\\
you(\textsc{sg}) are   {\db}a peasant/farmer.\textsc{masc}\\
\glt ‘You are a peasant.’

    \%‘You are a farmer.’
\z
\z

I proposed that this effect is not due to the presence of \textit{ein} itself. Rather, this extension in interpretation possibilities is a function of the null operator REL, the presence of which is flagged by \textit{ein}. This is consistent with the claim that \textit{ein} is a semantically vacuous element.

\subsection{Number in the DP and TP}\label{sec:7.1.2}

This chapter abstracts away from the different meanings of the noun. Below, I focus on issues related to number (for general typological discussion, see \citealt{Corbett2000} and \citealt{Hurford2003}; for a survey of the formal semantics, see \citealt{Link1998}). To start with some simple examples, consider plural contexts where nouns and pronouns of the same number can combine in both the DP and TP.

\ea%3
    \label{ex:7:3}
\ea\label{ex:7:3a}
\gll ihr         Schweine\\
you(\textsc{pl}) pigs\\
\glt ‘you idiots’
\ex\label{ex:7:3b}
\gll Ihr         seid  Ärzte.\\
you(\textsc{pl}) are   doctors\\
\glt ‘You are doctors.’

\z
\z
These distributions are as we would expect.

Turning to singular nouns, it is also expected that such nouns cannot combine with plural pronouns to form pronominal DPs. This holds independently of the presence of \textit{ein}.

\ea%4
    \label{ex:7:4}
\ea[*]{\label{ex:7:4a}
\gll  ihr         Schwein\\
you(\textsc{pl}) pig.\textsc{neut}\\
}
\ex[*]{\label{ex:7:4b}
\gll ihr        ’n Schwein\\
you(\textsc{pl}) {\db}a pig.\textsc{neut}\\
}
\z
\z

However, singular nouns behave differently in the TP. Surprisingly, a singular noun is possible in copular contexts \REF{ex:7:5a}. At face value, this presents a case of dis-agreement between a singular noun and a plural pronoun. Crucially though, when \textit{ein} appears, restrictions in number reveal themselves again \REF{ex:7:5b}.

\ea%5
    \label{ex:7:5}
\ea\label{ex:7:5a}
\gll Ihr         seid alle Arzt.\\
you(\textsc{pl}) are  all   doctor.\textsc{masc}\\
\glt ‘You are all doctors.’
\ex[ ?*]{\label{ex:7:5b}
\gll  Ihr         seid alle ’n Arzt.\\
you(\textsc{pl}) are  all     {\db}a doctor.\textsc{masc}\\
}
\z
\z

To sum up thus far, as seen in \chapref{sec:6}, bare nouns are restricted to non-emotiveness in singular TPs. As just seen though, they are unrestricted in number in TPs: Bare nouns can occur not only in singular but also plural contexts – they are number-neutral. To repeat, bare nouns in TPs are restricted as regards emotiveness but unrestricted as regards number.

It is clear that the grammaticality status of \REF{ex:7:5a} vs. \REF{ex:7:5b} correlates with the absence vs. presence of \textit{ein}. It could be claimed then that this difference is due to the morphological and/or semantic number specifications of \textit{ein}. However, I proposed in \chapref{sec:5} that \textit{ein} itself is not specified for morphological number; that is, it has the specification [$\alpha$PL morph]. Furthermore, I argued that \textit{ein} has no semantics. If so, we cannot rule out \REF{ex:7:5b} by resorting to problems in morphological or semantic number induced by the presence of \textit{ein}.

  To anticipate the following discussion, I reiterate the claim that morphological number is actually not specified on nouns in German. In \chapref{sec:6}, I followed  \citet{deSwartEtAl2007} in assuming that number morphology has to do with NumP. Specifically, bare nouns such as the role noun \textit{Arzt} ‘doctor’ are proposed to be unmarked for number and only project NP \REF{ex:7:6a}. In contrast, plural nouns like \textit{Ärzte} ‘doctors’ are argued to involve NumP \REF{ex:7:6b}, and singular nominals such as \textit{(ei)n Arzt} ‘a doctor’ are claimed to project more structure. I proposed in \chapref{sec:6} that, \textit{pace} de Swart \textit{et al.}, they may involve ArtP as in \REF{ex:7:6c}. In order to differentiate between bare role nouns \REF{ex:7:6a} and singular nouns \REF{ex:7:6c}, I label the first as non-plural reserving the term singular for the second. In other words, what has traditionally been classified as a “singular” noun may have two structural interpretations depending on the presence of \textit{ein}.

  \TabPositions{3.5cm,4.5cm}
\ea%6
    \label{ex:7:6}
\ea\label{ex:7:6a}   bare/non-plural \textit{Arzt}:\tab  N  \tab$\rightarrow$ [ NP ]
\ex\label{ex:7:6b}   plural \textit{Ärzte}:\tab    N-\textsc{pl}  \tab$\rightarrow$ [ NumP [ NP ]]
\ex\label{ex:7:6c}   singular \textit{’n Arzt}:\tab  \textit{ein} N  \tab$\rightarrow$ [[[ ArtP [ NumP [ NP ]]]
\z
\z

To explain the difference in grammaticality in \REF{ex:7:5}, I make use of the structural proposal in \REF{ex:7:6}, where NumP is absent with non-plural nouns \REF{ex:7:6a} but present with plural and singular nouns \REF[b-c]{ex:7:6}. Specifically, \REF{ex:7:5a} involves a non-plural noun as in \REF{ex:7:6a}, but \REF{ex:7:5b} contains a singular noun as in \REF{ex:7:6c}. I propose that \REF{ex:7:5a} is grammatical as NumP is absent in the predicate nominal. In other words, the predicate noun is number-neutral as no agreement relation between the predicate nominal and the subject pronoun needs to be established.\footnote{That bare nouns as in \REF{ex:7:5a} are neither singular nor plural in interpretation (i.e., they are number-neutral) can also be seen in contexts other than the ones discussed in the main text.

\ea \label{ex:7:2:i}
    \ea  flea(*s) infestation
    \ex  stone carving
    \z
\z
For the discussion of \REF{ex:7:2:i} and some other cases, see \citet[132-35]{Borer2005}.} In contrast, \REF{ex:7:5b} is ungrammatical as NumP is present in the predicate nominal. Here, an agreement relation between the predicate nominal and the subject pronoun must be established, but the elements involve a mismatch in number. In other words, it is the absence vs. presence of NumP (but not \textit{ein} itself) that is proposed to explain the difference in grammaticality in \REF{ex:7:5}.

As for the DP cases in \REF{ex:7:4}, I propose that NumP is always present; that is, predicate nominals within DPs always involve NumP. Here, NumP is present for cartographic reasons. Unlike in the TP, where the presence of NumP is indicated by overt elements, in the DP higher, obligatory phrases like ArtP entail the presence of lower NumP. If so, the presence of NumP rules out \REF{ex:7:4a} due to a mismatch in number between a plural pronoun and a singular noun. Note now that what may look like a non-plural noun in \REF{ex:7:4a} actually involves a singular noun – NumP entails REL and thus \textit{ein} should be present as in \REF{ex:7:4b} (NB: I usually provide the label for the surface forms given the terminology provided in \REF{ex:7:6}).

I suggest that the reason \textit{ein} can never be present inside pronominal DPs, be they singular as in \REF{ex:7:1b} or plural as in \REF{ex:7:4b}, is that the null operator REL can be flagged not only by expletive \textit{ein} but also by other, substantive elements (e.g., pronominal determiners). Assuming that both types of determiner elements are merged in ArtP, only one of them can be inserted and consequently appear. If these considerations are on the right track, then I can maintain the claim that \textit{ein} does not determine number, either morphologically or semantically. As in the previous chapter, I suggest that \textit{ein} indicates more structure on top of NP (Hypothesis 3a) and that it flags the presence of a null operator (Hypothesis 3b).

  The chapter is organized as follows. First, I present the data illustrating the restrictions on morphological and semantic number in more detail. In \sectref{sec:7.3}, I lay out my proposal to account for these facts. \sectref{sec:7.4} summarizes the main findings.

\section{Data}\label{sec:7.2}

I start with nominal and clausal contexts involving agreement in number. After that, I consider different instances of dis-agreement. The data are summarized in \tabref{tab:7:1} and \ref{tab:7:2} below. Given the discussion of the previous chapter, I present the data of the DP with figuratively extended nouns and the examples of the TP with role nouns. Recall that I refer to “singular” forms of the noun without \textit{ein} as non-plural and to those with \textit{ein} as singular.

\subsection{Cases of agreement}\label{sec:7.2.1}

To begin, consider cases involving agreement in number. As already seen in the introduction, a non-plural noun can combine with a singular pronoun \REF{ex:7:7a} but a singular noun cannot \REF{ex:7:7b}. Furthermore, a plural noun can combine with a plural pronoun \REF{ex:7:7c}.

\ea%7
    \label{ex:7:7}
\ea \label{ex:7:7a}
\gll du            Schwein\\
you(\textsc{sg}) pig.\textsc{neut}\\
\glt ‘you (idiot)’
\ex[*]{ \label{ex:7:7b}
\gll du           ’n Schwein\\
you(\textsc{sg}) {\db}a pig.\textsc{neut}\\
}
\ex \label{ex:7:7c}
\gll ihr         Schweine\\
you(\textsc{pl}) pigs\\
\glt ‘you idiots’
\z
\z

As expected, \REF{ex:7:7a} can only be singular in interpretation and \REF{ex:7:7c} only plural. In other words, the interpretations in \REF{ex:7:7a} and \REF{ex:7:7c} are parallel to the morphology on the noun. I discuss the ungrammatical case of \REF{ex:7:7b} in \sectref{sec:7.3.2.5} in detail proposing that \REF{ex:7:7b} involves two determiners in the same DP.

  Turning to the TP, we find similar facts with the exception of singular nouns. Specifically, a non-plural as well as a singular noun can combine with a singular pronoun \REF[a-b]{ex:7:8}. Again, a plural noun can occur with a plural pronoun \REF{ex:7:8c}.\footnote{Recall from \chapref{sec:6} that a predicative [-figurative] role noun in German can, for many speakers, take an indefinite article. In other words, these speakers allow both \REF{ex:7:8a} and \REF{ex:7:8b}. Below, I focus on those speakers.}

\ea%8
    \label{ex:7:8}
\ea \label{ex:7:8a}
\gll Du          bist Arzt.\\
you(\textsc{sg}) are doctor.\textsc{masc}\\
\glt ‘You are a doctor.’
\ex \label{ex:7:8b}
\gll Du            bist ’n Arzt.\\
you(\textsc{sg}) are    {\db}a doctor.\textsc{masc}\\
\glt ‘You are a doctor.’
\ex \label{ex:7:8c}
\gll Ihr         seid Ärzte.\\
you(\textsc{pl}) are  doctors\\
\glt ‘You are doctors.’
\z
\z

Again, the semantics runs parallel to the morphology. Next, I illustrate the various cases involving dis-agreement in number in more detail.

\subsection{Cases of dis-agreement}\label{sec:7.2.2}

Starting with the DP, a plural noun cannot combine with a singular pronoun \REF{ex:7:9a}. Conversely, a singular or a non-plural noun cannot combine with a plural pronoun either \REF[b-c]{ex:7:9}.\footnote{In \sectref{sec:7.3.3.2}, I discuss special cases such as \textit{Sie Schwein} ‘you (pig =) idiot’, which are characterized by semantic agreement but morphological dis-agreement (the latter, though, crucially holds between two different nominals as explicated in detail in that section).}

\ea%9
    \label{ex:7:9}
\ea[*]{\label{ex:7:9a}
\gll du            Schweine\\
you(\textsc{sg}) pigs\\
}
\ex[*]{\label{ex:7:9b}
\gll ihr  ’n Schwein\\
you(\textsc{pl}) {\db}a  pig.\textsc{neut}\\
}
\ex[*]{\label{ex:7:9c}
\gll ihr     Schwein\\
you(\textsc{pl}) pig.\textsc{neut}\\
}
\z
\z

So far, the facts are as we might expect.

  Turning to the TP, it may not be surprising either that a plural noun cannot combine with a singular pronoun \REF{ex:7:10a} or a singular noun with a plural pronoun \REF{ex:7:10b}. However, it is unexpected that a non-plural noun can combine with a plural pronoun \REF{ex:7:10c}. Importantly, the interpretation here is plural (for similar facts in Dutch, see \citealt{deSwartEtAl2005}: 451).\footnote{There seem to be some interesting cross-linguistic differences with the data in \REF{ex:7:10c}. For example, the Romance languages are different in that they require agreement between the pronoun and the nominal (see \citealt{Zamparelli2008}: 107 fn. 4; also \citealt{MunnSchmitt2005}: 839).}

\ea%10
    \label{ex:7:10}
\ea[*]{  \label{ex:7:10a}
\gll Du           bist Ärzte.\\
you(\textsc{sg}) are doctors\\
}
\ex[?*]{  \label{ex:7:10b}
\gll Ihr         seid alle ’n Arzt.\\
you(\textsc{pl}) are   all    {\db}a doctor.\textsc{masc}\\
}
\ex  \label{ex:7:10c}
\gll Ihr      seid alle Arzt.\\
you(\textsc{pl}) are   all doctor.\textsc{masc}\\
\glt ‘You are all doctors.’
\z
\z
\largerpage[1]
Note that there is a floating quantifier in the grammatical instance \REF{ex:7:10c} above. This element can also be in a different position \REF{ex:7:11a} or be exchanged by another quantifier \REF{ex:7:11b}. Interestingly, when such an element is missing, the example becomes somewhat marked, and judgments are a little bit unstable \REF{ex:7:11c} (I indicate this by \textsuperscript{m} here).\footnote{Below I argue that the (overt) quantifiers in \REF[a-b]{ex:7:11} are distributivity operators: They “single out” the individuals inside the set presupposed by the plural pronoun, and they distribute the relevant predicate (i.e., \textit{Arzt} 'doctor') over those individuals. As to \REF{ex:7:11c}, I tentatively suggest below that there is a null distributivity operator present (also \citealt{deSwartEtAl2007}: 218). The difference in overtness of the operator may explain the more stable judgments in \REF[a-b]{ex:7:11} vs. \REF{ex:7:11c}.}

\judgewidth{\textsuperscript{m}}
\ea%11
    \label{ex:7:11}
\ea[]{\label{ex:7:11a}
\gll Ihr        alle seid Arzt.\\
you(\textsc{pl}) all   are  doctor.\textsc{masc}\\
\glt ‘You are all doctors.’
}
\ex[]{\label{ex:7:11b}
\gll Ihr         seid jeder Arzt.\\
you(\textsc{pl}) are   each  doctor.\textsc{masc}\\
\glt ‘You are each a doctor.’
}
\ex[\textsuperscript{m}]{\label{ex:7:11c}
\gll  Ihr         seid Arzt.\\
you(\textsc{pl}) are   doctor.\textsc{masc}\\
\glt ‘You are doctors.’}
\z
\z
\judgewidth{\%}

Similar facts hold in capacity constructions involving \textit{als} ‘as’. This is what I turn to next.

\subsection{More cases of dis-agreement: Instances involving \textit{als}-nominals}\label{sec:7.2.3}

Recall that I refer to nouns preceded by \textit{als} ‘as’ as \textit{als}-nominals. As we will see, \textit{als}-nominals can occur in the context of pronominal DPs and non-copular TPs. First, I discuss cases of agreement and then instances of dis-agreement. Starting with the DPs, \textit{als}-nominals agree with their preceding pronoun, independently of whether the pronoun is in the singular \REF{ex:7:12a} or in the plural \REF{ex:7:12c}, with the qualification that the presence of \textit{ein} leads to slight markedness \REF{ex:7:12b}.

\ea%12
    \label{ex:7:12}
\ea\label{ex:7:12a}
\gll du            als Arzt\\
you(\textsc{sg}) as  doctor.\textsc{masc}\\
\glt ‘you as a doctor’
\ex[?]{\label{ex:7:12b}
\gll du            als ’n Arzt\\
you(\textsc{sg}) as    {\db}a doctor.\textsc{masc}\\
\glt ‘you as a doctor’
}
\ex\label{ex:7:12c}
\gll ihr         als Ärzte\\
you(\textsc{pl}) as  doctors\\
\glt ‘you as doctors’
\z
\z

  The same grammaticality judgments hold for the corresponding sentences involving a non-copular verb.\footnote{The sentences with non-copular verbs involve \textit{als} ‘as’ and have capacity readings. In contrast, the sentences with copular verbs discussed in \sectref{sec:7.2.1} and \ref{sec:7.2.2} above do not involve ALS ‘as’ and do not have capacity readings (recall from \chapref{sec:6}, \sectref{sec:6.3.4.1} though that copular TPs \textit{can} have a capacity reading – they involve null ALS licensed by the modal particle \textit{vielleicht} ‘really’). The different readings do not play a role here – the focus is on number.}

\ea%13
    \label{ex:7:13}
\ea\label{ex:7:13a}
\gll Du           sprichst als Arzt.\\
you(\textsc{sg}) speak     as  doctor.\textsc{masc}\\
\glt ‘You speak as a doctor.’
\ex[?]{\label{ex:7:13b}
\gll Du           sprichst als ’n Arzt.\\
you(\textsc{sg}) speak     as   {\db}a doctor.\textsc{masc}\\
\glt ‘You speak as a doctor.’
}
\ex\label{ex:7:13c}
\gll Ihr         sprecht als Ärzte.\\
you(\textsc{pl}) speak    as  doctors\\
\glt ‘You speak as doctors.’
\z
\z

  Turning to the cases involving dis-agreement, consider nominals where a singular pronoun is combined with a plural noun \REF{ex:7:14a}, and a plural pronoun is followed by a singular noun \REF{ex:7:14b}. While these two combinations lead to ungrammaticality, there is one exception: A plural pronoun can be followed by an \textit{als}-nominal with a non-plural noun \REF{ex:7:14c}.

\ea%14
    \label{ex:7:14}
\ea[*]{\label{ex:7:14a}
\gll  {du}            als Ärzte\\
you(\textsc{sg}) as  doctors\\
}
\ex[?*]{\label{ex:7:14b}
\gll ihr   alle als ’n Arzt\\
you(\textsc{pl}) all   as    {\db}a doctor.\textsc{masc}\\
}
\ex\label{ex:7:14c}
\gll ihr         alle  als Arzt\\
you(\textsc{pl}) all    as  doctor.\textsc{masc}\\
\glt ‘you all as doctors’
\z
\z

Note that the interpretation of the non-plural nominal in \REF{ex:7:14c} is indeed plural. The same holds for the clausal counterparts (the datum in \REF{ex:7:15c} is adapted from \citealt{deSwartEtAl2007}: 206).

\ea%15
    \label{ex:7:15}
\ea[*]{\label{ex:7:15a}
\gll Du           sprichst als Ärzte.\\
you(\textsc{sg}) speak    as   doctors\\
}
\ex[??]{\label{ex:7:15b}
\gll Ihr      sprecht alle als ’n Arzt.\\
you(\textsc{pl}) speak   all    as    {\db}a doctor.\textsc{masc}\\
}
\ex\label{ex:7:15c}
\gll Ihr        sprecht alle als Arzt.\\
you(\textsc{pl}) speak   all   as   doctor.\textsc{masc}\\
\glt ‘You speak all as doctors.’
\z
\z


To review, this last section discussed \textit{als}-nominals in the context of pronominal DPs and non-copular TPs. These two cases show the same syntactic distributions and corresponding semantic interpretations as the copular TPs illustrated in the two preceding sections – these three cases may involve morphological dis-agreement. This is in stark contrast to simple DPs, that is, DPs without \textit{als}-nominals, which have to exhibit agreement. Before moving on, I provide a more detailed summary of the individual cases in tabular form.

\subsection{Summary of the data}\label{sec:7.2.4}

I summarize the syntactic distributions in the following charts. I mark the unexpected instances by (!), the latter all restricted to the occurrence of non-plural nouns. Note that if there is a difference in judgments between nominals with or without \textit{als}, an explicit reference to \textit{als} is provided. \tabref{tab:7:1} shows the facts involving the DP (the shadings in \tabref{tab:7:1} are commented on below).

\begin{table}
\caption{Summary of the judgments in the DP}
\label{tab:7:1}

% \begin{tabularx}{\textwidth}{XXXX}
% \lsptoprule
%  & Non-plural N & \textit{(ei)n} N & N-\textsc{pl}\\
% \midrule
%  \textit{du} ‘you(\textsc{sg})’ & ${\surd}$ & \shadecell { * [without \textit{als}]}
%
%  ? [with \textit{als}] & *\\
%
% \textit{ihr} ‘you(\textsc{pl})’ & \shadecell { * [without \textit{als}]}
%
%  ${\surd}$ (!) [with \textit{als}] & { * [without \textit{als}]}
%
%  ?* [with \textit{als}] & ${\surd}$\\

\begin{tabular}{lccc}
\lsptoprule
 & Non-plural N & \textit{(ei)n} N & N-\textsc{pl}\\
 \midrule
\textit{du} ‘you(\textsc{sg})’ & ${\surd}$ & \shadecell { * [without \textit{als}]} & * \\
& & \shadecell{  ? [with \textit{als}]} & \\
\midrule
\textit{ihr} ‘you(\textsc{pl})’ & \shadecell { * [without \textit{als}]} & { * [without \textit{als}]} & ${\surd}$ \\
& \shadecell{ ${\surd}$ (!) [with \textit{als}]} & ?* [with \textit{als}] & \\
\lspbottomrule
\end{tabular}
\end{table}

\tabref{tab:7:2} shows the facts of the TP. Observe again that non-plural nouns are surprising as they are grammatical in basically all clausal contexts.

\begin{table}
\caption{Summary of the judgments in the TP}
\label{tab:7:2}
\begin{tabular}{lccc}
\lsptoprule
 & Non-plural N & \textit{(ei)n} N & N-\textsc{pl}\\
\midrule
\textit{du} ‘you(\textsc{sg})’ & ${\surd}$ & { ${\surd}$ [without \textit{als}]} & *\\
& &  ? [with \textit{als}] & \\
\midrule
\textit{ihr} ‘you(\textsc{pl})’ & ${\surd}$ (!) & { ?* [without \textit{als}]} & ${\surd}$\\
& &  ?? [with \textit{als}] & \\
\lspbottomrule
\end{tabular}
\end{table}

To repeat, the unexpected distributions occur when a plural pronoun is combined with a non-plural noun, both in the DP mediated by \textit{als} ‘as’ and in the TP more generally. Furthermore, the judgments are basically the same with or without \textit{als} (and these cases are treated below as the same). There are two exceptions to this that both hold in the DP (indicated by shading of the relevant cells in \tabref{tab:7:1}): Depending on the presence of \textit{als}, the judgments differ markedly if a singular pronoun combines with a singular noun or a plural pronoun occurs with a non-plural noun.

  In view of these data, I conclude that number is very restricted in simple DPs such that pronouns and nouns have to match in number. Syntactically, singular pronouns cannot combine with plural nouns, and plural pronouns cannot combine with singular or non-plural nouns.\footnote{The only potential exception that I am aware of comes from discontinuous noun phrases in split topicalizations, where a plural noun is compatible with both a plural and a singular determiner element.
  	\ea
  	\gll Hemden habe ich \{kein-e   / ?kein-es\}  getragen.\\
  	shirts     have  I      {\db}none-\textsc{pl} / {\db}none-\textsc{sg} worn\\
  	\glt ‘As for shirts, I have worn none.’
  	\z 
  	
  	I argued in \chapref{sec:4}, \sectref{sec:4.3} that the source and the split-off involve two separately base-generated nominals. Furthermore, \citet{OttNicolae2010} suggest that the plural in the split-off presumably has to do with frame-setting in topicalization, and as such the “dis-agreement” here is a different phenomenon.} Semantically, singular and plural pronouns are interpreted as singular and plural, respectively; singular/non-plural and plural nouns are understood as singular and plural, respectively.

  In contrast, copular clauses, non-copular clauses, and pronominal DPs, where the latter two involve \textit{als}, are less restricted, both syntactically and semantically. Syntactically, plural pronouns can, under certain conditions, occur with non-plural nouns. Semantically, non-plural nouns are compatible with a plural interpretation. In the next section, I turn to an explanation of these syntactic and semantic facts.

\section{Proposal}\label{sec:7.3}

In this section, I account for the different morphological and semantic possibilities licensed in the different syntactic domains. Just as in \chapref{sec:6}, I follow \citet{deSwartEtAl2007} and extend their proposal to pronominal DPs explaining the differences between predicate nominals in TPs and those in DPs.

\subsection{Basic assumptions}\label{sec:7.3.1}

To begin, I lay out my assumptions of the simple structures; the instances involving \textit{als}-nominals are discussed at the end of this chapter. Starting with the syntax and simplifying somewhat for now, I assume again that all DPs, including pronominal DPs, consist of a head noun projecting an NP, a Number head projecting a NumP, and a (pronominal) determiner surfacing in the DP-level \REF{ex:7:16a}. In keeping with the discussion in \chapref{sec:6}, the predicate nominal inside the DP always involves NumP (where REL maps an element of type \textlangle e\textrangle\  to type \textlangle e,t\textrangle ). TPs are different: While they all involve a subject DP, a copular verb, and a predicate nominal, the latter may vary in size. As in \chapref{sec:6}, I assume that NumP is syntactically optional in predicate nominals following a copular auxiliary. Specifically, a non-plural nominal involves NP \REF{ex:7:16b}, a plural nominal consists of NumP \REF{ex:7:16c}, and a singular nominal projects ArtP \REF{ex:7:16d} (Aux = stands for the auxiliary \textit{sein} ‘to be’).

\TabPositions{1cm}
\ea%16
    \label{ex:7:16}
\ea\label{ex:7:16a}   DP\textsubscript{\textlangle e\textrangle}:\tab  [\textsubscript{DP} [\textsubscript{NumP} [\textsubscript{NP} ]]]\textsubscript{\textlangle e\textrangle}
\ex\label{ex:7:16b}   TP\textsubscript{\textlangle t\textrangle}:\tab  [\textsubscript{DP} [\textsubscript{NumP} [\textsubscript{NP} ]]]\textsubscript{\textlangle e\textrangle} Aux [\textsubscript{NP} ]\textsubscript{\textlangle e,t\textrangle}
\ex\label{ex:7:16c}   TP\textsubscript{\textlangle t\textrangle}:\tab  [\textsubscript{DP} [\textsubscript{NumP} [\textsubscript{NP} ]]]\textsubscript{\textlangle e\textrangle} Aux [\textsubscript{NumP} Num\textsubscript{[+PL]} [\textsubscript{NP} ]]\textsubscript{\textlangle e,t\textrangle}
\ex\label{ex:7:16d}   TP\textsubscript{\textlangle t\textrangle}:\tab  [\textsubscript{DP} [\textsubscript{NumP} [\textsubscript{NP} ]]]\textsubscript{\textlangle e\textrangle} Aux [\textsubscript{ArtP} \textit{ein} [\textsubscript{NumP} Num\textsubscript{[--PL]} [\textsubscript{NP} ]]]\textsubscript{\textlangle e,t\textrangle}
\z
\z


These are the basic assumptions about the syntax.

As for the semantics, I again follow \citegen{deSwartEtAl2007} discussion of the clause in \REF[b-d]{ex:7:16} and assume that the predicate nominal must be of type \textlangle e,t\textrangle\  to combine with the subject DP (type \textlangle e\textrangle ). Recalling that the copular auxiliary is semantically vacuous, this means that NP, NumP, and ArtP must all be of type \textlangle e,t\textrangle . As already discussed above, \citeauthor{deSwartEtAl2007} propose that role nouns combine with CAP, which is in NP, and that kind nouns, inherent or coerced, combine with REL, which is in NumP. The first option holds for \REF{ex:7:16b}; the second option applies to plural \REF{ex:7:16c} and to singular \REF{ex:7:16d}, with singular involving an additional structural layer (ArtP). To be clear, these types of predicate nominals are all syntactically different but semantically the same (i.e., type \textlangle e,t\textrangle ). As such, they can combine with the subject DP to yield a truth value (type \textlangle t\textrangle ). Note again that \textit{ein} indicates the presence of a certain amount of structure on top of NP (Hypothesis 3a) and that it flags the presence of the operator REL (Hypothesis 3b).

As for the DP in \REF{ex:7:16a}, I suggested that NumP (and thus REL) is always present. I argue in more detail below that this is so for cartographic reasons – higher, obligatory phrases entail the presence of lower ones. Semantically, combining the head noun and REL yields an element of type \textlangle e,t\textrangle . As determiners, including pronominal determiners, are of type \textlangle \textlangle e,t\textrangle e\textrangle , predicate nominals and determiners can combine to return an entity (type \textlangle e\textrangle ).

With these general points in place, I turn to the cases from the data section. First, I discuss DPs and TPs involving agreement and dis-agreement. Then, I turn to some special cases involving the pronoun \textit{Sie} ‘you(\textsc{formal})’, identifying the parts of the structure where morphological and semantic number originate. In the final section, I address the more complex DPs and TPs involving \textit{als}-nominals.

\newpage
\subsection{Agreement in constructions without \textit{als}}\label{sec:7.3.2}

First, I review some general facts about agreement in the TP. This is followed by proposing that NumP plays a crucial role in agreement. Finally, I detail the account of agreement as regards plural nouns, non-plural nouns, and singular nouns.

\subsubsection{Agreement and dis-agreement}\label{sec:7.3.2.1}
\largerpage
Consider the data involving copular TPs below. Recall from above that non-plural nouns are fine in all contexts \REF{ex:7:17}. In contrast, singular nouns can only combine with singular pronouns \REF{ex:7:18} and plural nouns only with plural pronouns \REF{ex:7:19}.

\ea%17
    \label{ex:7:17}
\ea\label{ex:7:17a}
\gll Du           bist Arzt.\\
you(\textsc{sg}) are   doctor.\textsc{masc}\\
\glt ‘You are a doctor.’
\ex\label{ex:7:17b}
\gll Ihr        seid alle Arzt.\\
you(\textsc{pl}) are  all  doctor.\textsc{masc}\\
\glt ‘You are all doctors.’
\z
\z

\ea%18
    \label{ex:7:18}
\ea\label{ex:7:18a}
\gll Du           bist ’n Arzt.\\
you(\textsc{sg}) are   {\db}a doctor.\textsc{masc}\\
\glt ‘You are a doctor.’

\ex[?*]{
\gll Ihr        seid alle ’n Arzt.\\
you(\textsc{pl}) are  all    {\db}a doctor.\textsc{masc}\\
}
\z
\z

\ea%19
    \label{ex:7:19}
\ea[*]{ \label{ex:7:19a}
\gll Du           bist Ärzte.\\
you(\textsc{sg}) are  doctors\\
}
\ex \label{ex:7:19b}
\gll Ihr        seid Ärzte.\\
you(\textsc{pl}) are  doctors\\
\glt ‘You are doctors.’
\z
\z

To be clear, non-plural nouns can occur in both singular and plural contexts, but singular and plural nouns are in complementary distribution as regards number.

  To see whether this is a morphological or semantic restriction, I discuss \textit{jeder} ‘each’ in this regard. This quantificational element is morphologically singular but semantically plural; that is, it takes a non-plural restriction but presupposes a plurality of entities. As might be expected, this element can combine with a non-plural noun via an auxiliary \REF{ex:7:20a}. However, while \textit{jeder} can also occur with a singular noun \REF{ex:7:20b}, it cannot occur with a plural one \REF{ex:7:20c}.\footnote{Other quantifiers also presuppose a plurality of entities, but they are morphologically plural and can combine with plural nouns.
  	\ea
  	\gll Einige (Männer hier) sind Ärzte.\\
  	some    {\db}men       here  are  doctors\\
  	\glt  ‘Some (men here) are doctors.’
    \z }

\ea%20
    \label{ex:7:20}
\ea \label{ex:7:20a}
\gll Jeder (Mann hier) ist Arzt.\\
each    {\db}man   here  is  doctor.\textsc{masc}\\
\glt ‘Each (man here) is a doctor.’
\ex \label{ex:7:20b}
\gll Jeder (Mann hier) ist ’n Arzt.\\
each    {\db}man   here  is   { }a doctor.\textsc{masc}\\
\glt ‘Each (man here) is a doctor.’
\ex[*]{ \label{ex:7:20c}
\gll Jeder (Mann hier) ist Ärzte.\\
each    {\db}man   here  is  doctors\\
}
\z
\z

Considering that \textit{jeder} is semantically plural, the incompatibility with a plural noun in \REF{ex:7:20c} is somewhat surprising. I make the strongest claim and propose that the number restrictions shown in \REF{ex:7:17} through \REF{ex:7:20} are morphological \textit{and} semantic in nature. The lack of morphological agreement between singular \textit{jeder} and the plural predicate nominal explains the ungrammaticality of \REF{ex:7:20c}. As for the grammatical \REF[a-b]{ex:7:20}, I propose that \textit{jeder} is semantically compatible with non-plural and singular nouns. Specifically, I suggest that \textit{jeder}, itself a distributivity operator, distributes the predicate over the set presupposed by itself.\footnote{Alternatively, there could be an additional null distributivity operator involved here. Note in this regard that  \citet[218]{deSwartEtAl2007} propose a null distributivity operator for cases like \textit{Jan und Sofie sind Arzt} ‘Jan and Sofie are (doctor =) doctors’ in Dutch (more on null distributivity operators in \sectref{sec:7.3.3.2}).}  In what follows, I focus on morphological number, but I return to semantic number in \sectref{sec:7.3.3}.

\subsubsection{Agreement and NumP}\label{sec:7.3.2.2}

Starting with the TP, recall that predicate nominals in the clause involve NP, NumP, or ArtP depending on the morphology inside the predicate nominal.\linebreak Specifically, while all nouns involve at least NP, plural morphology on the noun indicates the presence of NumP, and the presence of \textit{ein} shows ArtP (containing NumP). Furthermore, I propose that when NumP is present, an agreement relation must be established between the predicate nominal and the subject DP. In contrast, a non-plural noun does not project NumP – it is a predicate nominal involving NP. I propose that with NumP absent, such a predicate nominal does not have to undergo an agreement relation, and consequently its distribution is much freer (see also  \citealt{denDikken2006}: 210).

DPs are different. I propose that NumP must be present for cartographic reasons: Higher phrases such as ArtP entail the presence of lower phrases such as NumP. Note in this regard that DPs involve determiners and that these determiners originate in ArtP and move to the DP-level. Now, as is well known, elements in the DPs must exhibit concord in agreement features like case, number, and gender. I follow the literature in making the standard assumption that number features originate in NumP (e.g., \citealt{deSwartEtAl2007}, \citealt{Julien2005a}, \citealt{Ritter1991}, \citealt{Roehrs2006b}). I assume that the values on Num are morphological (i.e., [$\alpha$PL morph]) and that NumP mediates concord within the DP. For concreteness, I assume here that the value of the Num head “percolates” up the nominal tree by some concord mechanism.\footnote{As mentioned in \chapref{sec:1}, \sectref{sec:1.4.1.1}, there are different mechanisms that have been claimed to bring about concord. Note also that the obligatory presence of NumP is presumably not due to concord itself (perhaps due to the need to specify [$\alpha$PL morph] on Num). This is so because I suggest below that mass nouns receive singular morphology by default (as [$\alpha$PL morph] is proposed to be absent on Num in those cases).}  As to the head noun, I proposed in \chapref{sec:1}, \sectref{sec:1.4} that it moves to adjoin to Num (also \citealt{Julien2005a}), where the head noun establishes an agreement relation with Num. To repeat, NumP is always present in the DP.

Taking stock thus far, there are two types of agreement: concord in agreement features within the DP, indicated below by subscript alphas \REF{ex:7:21a}, and agreement between the subject and the predicate nominal within the copular TP, marked below by subscript betas \REF{ex:7:21b}. The latter type of agreement only holds if NumP is present and if the morphological number of the predicate nominal agrees with that of the subject. Note that \REF{ex:7:21b} also involves concord in agreement features, just like \REF{ex:7:21a}, if NumP and higher phrases are present within the predicate nominal.

\ea%21
    \label{ex:7:21}
\ea \label{ex:7:21a}  DP:  [ Det\textsubscript{$\alpha$} Num\textsubscript{$\alpha$} N\textsubscript{$\alpha$} ]
\ex \label{ex:7:21b}  TP:  Subj\textsubscript{$\beta$} Aux [ … (Num\textsubscript{$\alpha$}) N\textsubscript{$\alpha$} ]\textsubscript{$\beta$}     (where $\beta$ = $\alpha$ on Num)
\z
\z

It is clear that NumP is crucial in establishing agreement relations both within the nominal domain and the clausal domain (involving copulas). For the TP, I focus on the agreement relation between the subject and the predicate nominal.

With these points about NumP in mind, we can observe that there are two notions of dis-agreement. Starting with the DP, all ungrammatical cases in the nominal domain involve (true) morphological dis-agreement where the combination of singular and plural elements cannot establish an agreement relation \REF{ex:7:22}. Indeed, such mismatches are ruled out due to the obligatory presence of NumP in the DP, which is either specified as singular or plural (for details, see \sectref{sec:7.3.3}).

\ea%22
    \label{ex:7:22}
\ea[*]{ \label{ex:7:22a}
\gll  du            Ärzte\\
you(\textsc{sg}) doctors\\
}
\ex[*]{\label{ex:7:22b}
\gll ihr         Arzt\\
you(\textsc{pl}) doctor.\textsc{masc}\\
}
\z
\z


This is different for the TP. Given certain conditions, we may find instances of true dis-agreement \REF{ex:7:23a} but also cases of \textit{apparent} dis-agreement \REF{ex:7:23b}.

\ea%23
\label{ex:7:23}
\ea[*]{\label{ex:7:23a}
\gll Du          bist Ärzte.\\
you(\textsc{sg}) are doctors\\
}
\ex\label{ex:7:23b}
\gll Ihr         seid alle Arzt.\\
you(\textsc{pl}) are   all  doctor.\textsc{masc}\\
\glt ‘You are all doctors.’
\z
\z


Importantly, true dis-agreement in copular contexts also involves the fact that NumP is present, here in the plural predicate nominal in \REF{ex:7:23a}. This leads to ungrammaticality as an agreement relation between the singular subject pronoun and the plural predicate nominal cannot be established. Cases of apparent dis-agreement are different. In \REF{ex:7:23b}, NumP is absent with non-plural nominals, and no agreement relation has to be established. This allows the combination of a plural subject pronoun and a non-plural noun to surface. I discuss the individual derivations involving the different types of nouns.

\subsubsection{Plural nouns}\label{sec:7.3.2.3}

DPs involve structures where NumP is always present. Assuming again that determiners orginate in ArtP, the tree diagram of \REF{ex:7:24b} is given in \figref{figex:7:24} (in the discussion, I abstract away from the traces/copies left by movement).

\ea%24
\label{ex:7:24}
\ea[*]{ \label{ex:7:24a}
	\gll du            Schweine\\
	you(\textsc{sg}) pigs\\
}
\ex \label{ex:7:24b}
\gll ihr Schweine\\
you(\textsc{pl}) pigs\\
\glt ‘you idiots’
\z
\z

 \begin{figure}
 	\caption{Agreement in the DP}
 	\label{figex:7:24}
 	\begin{forest}
 		[DP\textsubscript{\parbox{0mm}{\mbox{\textlangle e\textrangle}}}
 			[\textit{ihr}\textsubscript{\parbox{0mm}{\mbox{\textlangle\textlangle e,t\textrangle e\textrangle $_k$}}}]
 			[ArtP
 				[ t\textsubscript{$k$}]
 				[NumP\textsubscript{\parbox{0mm}{\mbox{\textlangle e,t\textrangle}}}, s sep=20mm
 					[Num\textsubscript{\parbox{0mm}{\mbox{\textsc{rel}\textlangle e\textlangle e,t\textrangle\textrangle}}}]
 					[NP\\\textit{Schweine}\textsubscript{\textlangle e\textrangle}]
 				]
 			]
 		]
 	\end{forest}
 \end{figure}

As mentioned in the previous subsection, all elements inside DP have to establish an agreement relation mediated by NumP. If so, it is easy to rule out \REF{ex:7:24a}, where such a relation between the pronominal determiner, Num, and the noun cannot be established. In contrast, \REF{ex:7:24b} is fine as the relevant elements can establish an agreement relation in number. I turn to the clausal domain.

As in \chapref{sec:6}, I follow \citet{Bowers1993} in assuming that copular structures involve a Predication Phrase (PrP). The latter is embedded under a TP. I propose that the head Pr can take different elements as its complement. With plural morphology present on the nouns in \REF[a-b]{ex:7:25}, I suggest that Pr takes NumP as its predicative complement. I provide the tree diagram of \REF{ex:7:25b} in \figref{figex:7:25}.

\ea%25
    \label{ex:7:25}
\ea[*]{\label{ex:7:25a}
\gll Du           bist Ärzte.\\
you(\textsc{sg}) are doctors\\
}
\ex\label{ex:7:25b}
\gll Ihr         seid Ärzte.\\
you(\textsc{pl}) are  doctors\\
\glt ‘You are doctors.’
\z
\z

\begin{figure}
	\caption{Agreement in the TP}
	\label{figex:7:25}
	\begin{forest}
		[TP\textsubscript{\parbox{0mm}{\mbox{\textlangle t\textrangle}}}
			[{[\textit{ihr} REL \textit{e\textsubscript{N}}]}\textsubscript{\textlangle e\textrangle $_k$}]
			[ T$'$
				[\textit{seid}\textsubscript{$i$}]
				[PrP
					[t\textsubscript{$k$}]
					[Pr$'$
						[t\textsubscript{$i$}]
						[NumP\textsubscript{\parbox{0mm}{\mbox{\textlangle e,t\textrangle}}}\\\textit{Ärzte}]
					]
				]
			]
		]
	\end{forest}
\end{figure}

Recall from the previous chapter that pronominal determiners (\textit{ihr}) are taken to be of type \textlangle\textlangle e,t\textrangle e\textrangle\ but that pronominal DPs as a whole (e.g., \textit{ihr} REL \textit{e\textsubscript{N}}) are of type \textlangle e\textrangle. As regards morphological number, I proposed above that the NumP of the predicate nominal has to establish an agreement relation with the subject DP. As plural \textit{Ärzte} ‘doctors’ can enter into such a relation with plural \textit{ihr} but not singular \textit{du}, the difference in grammaticality in \REF[a-b]{ex:7:25} is accounted for. As just seen, we can observe that the cases involving plural nouns are straightforward.

\subsubsection{Non-plural nouns}\label{sec:7.3.2.4}

Again, all DPs contain NumP. As such, a singular pronoun is grammatical with a non-plural noun \REF{ex:7:26a}, but a plural pronoun is not \REF{ex:7:26b}. I derive \REF{ex:7:26a} as in \figref{figex:7:26}.

\ea%26
    \label{ex:7:26}
    \ea\label{ex:7:26a}
    \gll du            Schwein\\
    you(\textsc{sg}) pig.\textsc{neut}\\
    \glt ‘you (idiot)’
    \ex[*]{\label{ex:7:26b}
    \gll ihr      Schwein\\
    you(\textsc{pl}) pig.\textsc{neut}\\
    }
\z
\z

\begin{figure}
	\caption{Agreement in the DP}
	\label{figex:7:26}
	\begin{forest}
		[DP\textsubscript{\parbox{0mm}{\mbox{\textlangle e\textrangle}}}
			[\textit{du}\textsubscript{\parbox{0mm}{\mbox{\textlangle\textlangle e,t\textrangle e\textrangle $_k$}}}]
			[ArtP
				[t\textsubscript{$k$}]
				[NumP\textsubscript{\parbox{0mm}{\mbox{\textlangle e,t\textrangle}}}, s sep=20mm
					[Num\textsubscript{\parbox{0mm}{\mbox{\textsc{rel}\textlangle e\textlangle e,t\textrangle\textrangle}}}]
					[NP\\\textit{Schwein}\textsubscript{\parbox{0mm}{\mbox{\textlangle e\textrangle}}}]
				]
			]
		]
	\end{forest}
\end{figure}

The ungrammaticality in \REF{ex:7:26b} is due to the failure of establishing the relevant agreement relation between the pronominal determiner, Num, and the noun. Note now that with NumP present, REL is present too, and we would expect \textit{ein} to occur. In the next subsection, I discuss why singular nouns are not possible here.

  As regards the TP, I followed  \citet{deSwartEtAl2007} in that NumP is absent if no number-related elements are present in the predicate nominal. This is the case for \REF[a-b]{ex:7:27}. I propose that the head Pr takes NP as its predicative complement here. Recall that bare role nouns combine with CAP. I derive \REF[a-b]{ex:7:27} as in \figref{figex:7:27}.

\ea\label{ex:7:27}
\ea\label{ex:7:27a}
\gll Du           bist Arzt.\\
you(\textsc{sg}) are doctor.\textsc{masc}\\
\glt ‘You are a doctor.’
\ex\label{ex:7:27b}
\gll Ihr         seid alle Arzt.\\
you(\textsc{pl}) are  all   doctor.\textsc{masc}\\
\glt ‘You are doctors.’
\z
\z

\begin{figure}
	\caption{Apparent (dis-)agreement in the TP}
	\label{figex:7:27}
	\begin{forest}
		[TP\textsubscript{\parbox{0mm}{\mbox{\textlangle t\textrangle}}}
			[\textit{du}/\textit{ihr}\textsubscript{\textlangle e\textrangle $_k$}]
			[ T$'$
				[\textit{seid}\textsubscript{$i$}]
				[PrP
					[t\textsubscript{$k$}]
					[Pr$'$\textsubscript{\parbox{0mm}{\mbox{\textlangle e,t\textrangle}}}%, s sep=20mm
						[t\textsubscript{$i$}]
						[NP\textsubscript{\parbox{0mm}{\mbox{\textsc{cap}\textlangle e\textlangle e,t\textrangle\textrangle}}}\\\textit{Arzt}\textsubscript{\parbox{0mm}{\mbox{\textlangle e\textrangle}}}]
					]
				]
			]
		]
	\end{forest}
\end{figure}

With NumP absent, the predicate nominal does not have to enter into an agreement relation. In other words, no agreement relation between the predicate nominal and the subject DP has to be established, and both singular and plural subject pronouns are grammatical here. \REF{ex:7:27b}, then, is a case of apparent dis-agreement as no relevant agreement relation is established in the first place. In order to obtain a plural interpretation of the non-plural noun in \REF{ex:7:27b}, I follow \citet{deSwartEtAl2007} in that this case involves a distributivity operator. I take the floating quantifier in \REF{ex:7:27b} to be this element.

  This discussion makes an interesting prediction: If a predicate nominal in a copular TP does not involve NumP but just NP, then the head noun can only combine with CAP to yield an element of type \textlangle e,t\textrangle~(see \figref{figex:7:27} again). This in turn should allow only a neutral/literal meaning of a role noun that can, at least potentially, undergo figurative extension. In keeping with what we saw in \chapref{sec:6}, we find the same interpretation restriction in the (apparent) dis-agreement cases discussed here.

\ea%28
    \label{ex:7:28}
    \gll Ihr         seid jeder Bauer.\\
  you(\textsc{pl}) are   each  farmer.\textsc{masc}\\
  \glt ~~‘You (each) are farmers.’\\
  \#‘You (each) are peasants.’
\z

In the next section, I discuss singular nouns, and I provide an explanation as to why \textit{ein} is not possible in pronominal DPs.

\subsubsection{Singular nouns}\label{sec:7.3.2.5}

Starting with the DP, I return to the question left unanswered in the previous subsection, namely why \textit{ein} cannot occur in pronominal DPs despite the fact that REL is present. Note that this not only applies to pronominal DPs \REF{ex:7:29a} but also to ordinary DPs \REF{ex:7:29b}.

\ea%29
    \label{ex:7:29}
\ea[*]{\label{ex:7:29a}
\gll   du          ’n Schwein\\
you(\textsc{sg}) {\db}a pig.\textsc{neut}\\
}
\ex[*]{\label{ex:7:29b}
\gll  das ’n Schwein\\
the   {\db}a pig.\textsc{neut}\\
}
\z
\z


I proposed in the previous chapter that REL is not only flagged by \textit{ein} but also by other elements. Specifically, definite determiners, pronominal or ordinary, can also flag the presence of REL. Recall that this is consistent with \citegen{deSwartEtAl2007} assumption about regular DPs like \textit{the child}. Now, assuming that determiners and \textit{ein} originate in the same position (i.e., ArtP), only one such element can occur. This explains the ungrammaticality of the cases in \REF{ex:7:29}, where two determiners are present in each case. Finally, given the current analysis, the distributional interaction between \textit{ein} and personal pronouns in pronominal DPs provides evidence that the latter are indeed determiners. I turn to the TP.

  In \chapref{sec:6}, I discussed in detail the proposal that for nouns to be predicates, kind nouns combine with REL directly, but role nouns do so indirectly (via kind coercion and REL). REL triggers the presence of \textit{ein} in singular contexts, which I explained by proposing that \textit{ein} flags the presence of REL. Consider \REF[a-b]{ex:7:30}. Given the absence of a feature for definiteness on \textit{ein}, I also proposed that this element may surface in ArtP (and not necessarily in DP). Given that, I propose that besides NP and NumP, the head Pr can also take ArtP as its predicative complement. The tree diagram of \REF{ex:7:30a} is provided in \figref{figex:7:30}.

\ea%30
    \label{ex:7:30}
\ea  \label{ex:7:30a}
\gll Du           bist ’n Arzt.\\
you(\textsc{sg}) are   {\db}a doctor.\textsc{masc}\\
\glt ‘You are a doctor.’
\ex[?*]{  \label{ex:7:30b}
\gll   Ihr        seid alle ’n Arzt.\\
you(\textsc{pl}) are  alle  {\db}a doctor.\textsc{masc}\\
}
\z
\z

\begin{figure}
	\caption{Agreement in the TP}
	\label{figex:7:30}
	\begin{forest}
		[TP\textsubscript{\parbox{0mm}{\mbox{\textlangle t\textrangle}}}
			[\textit{du}\textsubscript{\textlangle e\textrangle $_k$}]
			[T$'$
				[\textit{bist}\textsubscript{$i$}]
				[PrP
					[t\textsubscript{$k$}]
					[Pr$'$
						[t\textsubscript{$i$}]
						[ArtP\textsubscript{\parbox{0mm}{\mbox{\textlangle e,t\textrangle}}}
							[\textit{’n}]
							[NumP\textsubscript{\parbox{0mm}{\mbox{\textlangle e,t\textrangle}}}\\\textit{Arzt}]
						]
					]
				]
			]
		]
	\end{forest}
\end{figure}

Observe that the subject pronoun \textit{du} ‘you’ and \textit{ein} in \REF{ex:7:30a} are part of different nominals – the pronominal one, illustrated in \figref{figex:7:30} in simplified form, and the predicate one, where the former is separated from the latter by an auxiliary. As such, \textit{du} and \textit{ein} do not compete for insertion in the same ArtP, and they can co-occur. Now, notice that with NumP present, a singular predicate nominal can establish an agreement relation with a singular, but not a plural, subject pronoun. As such, the latter case presents an instance of true dis-agreement, and the failure to establish an agreement relation explains the ungrammaticality of \REF{ex:7:30b}. I return to the main focus of this book summarizing the discussion so far and drawing some conclusions.

Starting with the DP, I proposed that NumP is always present, and a relevant agreement relation between a determiner, Num, and a noun has to be established. Furthermore, although REL is always present in the DP, \textit{ein} does not surface when another determiner occurs. As suggested above, elements other than \textit{ein} can flag the presence of REL in singular contexts. As for the TP, I proposed that the head Pr can take NP, NumP, or ArtP as its predicative complement. Importantly, when NumP is present in the predicative complement, an agreement relation with the subject DP has to be established. In contrast, when NumP is absent, no agreement relation has to be established, and a number-neutral predicate comes about.

More generally, these assumptions allow us to account for the diverse agreement phenomena illustrated above while maintaining the claim that \textit{ein} has no specifications for number. In keeping with \chapref{sec:5}, I provide more details in the next section that morphological number is not due to \textit{ein} but due to the number specification on NumP. In addition, I relate morphological number to semantic number and the mass/count distinction of nouns.

\subsection{Morphological and semantic number}\label{sec:7.3.3}

Considerations regarding number and the related mass/count distinction are notoriously complex and difficult. For convenience, I employ my own system developed in \citet{Roehrs2006b}.\footnote{There are many other proposals. To name just a few: \citet{Borer2005}, \citet{vanEynde2020}, \citet{Ott2011b}, \citet{Watanabe2006}, and H. \citet{WieseMaling2005} (see also references cited in these works). Also, \citet{Hachem2015} and \citet[181-86]{Rehn2019} argue that a simple dichotomy of count vs. mass nouns is not correct (also S. \citealt{Grimm2012}, \citealt{Zhang2012}). On the basis of diachronic facts, \citet{Hachem2015} proposes that grammatical gender functions as a mass quantifier. It means that gender is a semantically meaningful category that creates different types of mass distinctions (her page 100): Neuter involves unbounded mass (e.g., \textit{snow}), masculine forms individuative mass (e.g., \textit{frost}), and feminine is collective mass (e.g., \textit{winter}); plural is a separate category. \citet{Rehn2019} points out that these fine-grained distinctions present a problem for \citet{Borer2005} (for other issues with Borer’s work, see also \citealt{Ott2011b}, \citealt{Zhang2012}). To the extent that these finer distinctions are correct, the current analysis could break NumP into several phrases (as is done in \citealt{Hachem2015} and \citealt{Rehn2019}). Having said that, Hachem states that this proposed correlation between gender of the noun and its interpretation is no longer obvious in Modern German – the correlation has been regrammaticalized (see \citealt{Leiss1997}). Furthermore, these finer distinctions do not seem to have an influence on the distribution of adjectival inflections or \textit{ein} in Standard German. As such, I do not take them into consideration here.} My primary goal here is not to argue that this is the correct way to account for morphological and semantic number. Rather, I intend to show that elements other than \textit{ein} can be held responsible for morphological and semantic number thus defending the claim that \textit{ein} is not a reflex of number.

  In the first subsection, I relate morphological and semantic number. With this in place, I turn to cases of dis-agreement in pronominal DPs containing \textit{Sie} ‘you(\textsc{formal})’. In the third subsection, I provide more details of the derivation of these cases as regards number, arguing that these cases involve non-canonical structures. The discussion of these non-canonical structures reveals a number of interesting points.

\subsubsection{Relating morphological and semantic number}\label{sec:7.3.3.1}

I begin by illustrating the relevant issues with an ordinary noun like \textit{Schwein} ‘pig’. As is well known, this type of noun has three readings. It can have a mass \REF{ex:7:31a}, a singular count \REF{ex:7:31b}, or a plural count \REF{ex:7:31c} interpretation.

\ea%31
    \label{ex:7:31}
\ea\label{ex:7:31a}
\gll Schwein   schmeckt gut.\\
pig.\textsc{neut} tastes       good\\
\glt ‘Pork tastes good.’
\ex\label{ex:7:31b}
\gll du            Schwein\\
you(\textsc{sg}) pig.\textsc{neut}\\
\glt ‘you (idiot)’
\ex\label{ex:7:31c}
\gll ihr         Schweine\\
you(\textsc{pl}) pigs\\
\glt ‘you idiots’
\z
\z

In order to account for these readings, I propose that there is an intricate interplay between the number specifications on the noun and the Number head (Num). Specifically, head nouns have a statement for semantic number, and the latter interacts with the morphological number on Num.

  In more detail, I assume with \citet[94]{Borer2005} that nouns are unmarked for the mass-count distinction. Unlike Borer, I assume that all regular nouns involve two statements as regards number (to be collapsed into one statement below). The first consists of morphological and semantic number with unspecified values \REF{ex:7:32a}; the second states that both of these values have to coincide \REF{ex:7:32b}; that is, either both values for number are positive or both are negative.\footnote{It is not likely that these two statements are acquired for each and every noun. Rather, they are probably provided to all regular nouns by a default mechanism. Note in this regard that we will see below that there are certain elements where the specifications for morphological and semantic number can diverge. Presumably, these specifications are lexically marked on those elements and have to be acquired individually.} As for NumP, its head has the morphological number feature [$\alpha$PL morph], which enters the derivation specified as positive or negative. Regarding the syntax, recall from \chapref{sec:1}, \sectref{sec:1.4} that the head noun moves to adjoin to Num. With these points in mind, the lower part of the DP structure can be fleshed out as in \figref{figex:7:32}, where the two statements of the noun in \REF[a-b]{ex:7:32} are collapsed into one.

\ea%32
    \label{ex:7:32}
\ea\label{ex:7:32a} [$\alpha$PL morph; $\beta$PL sem]
\ex\label{ex:7:32b}    (where $\alpha$ = $\beta$)
\z
\z

\begin{figure}
	\caption{General makeup of nouns}
	\label{figex:7:32}
	\begin{forest}
		[NumP
			[~]
			[Num$'$
				[Num, s sep=50mm
					[N\textsubscript{\parbox{0mm}{\mbox{[$\alpha$PL morph; $\beta$PL sem; $\alpha$ = $\beta$]$_k$}}}]
					[Num\textsubscript{\parbox{0mm}{\mbox{[$\alpha$PL morph]}}}]
				]
				[NP\\t\textsubscript{$k$}]
			]
		]
	\end{forest}
\end{figure}

With the head noun and Num in a local relation, I assume that the morphological value for number on Num and that on the noun establish an agreement relation. After the morphological number of the noun has agreed with Num, the equality statement in \REF{ex:7:32b} brings about the semantic number of the noun. This means that there is an interplay between Num and the head noun as regards number, morphologically and consequently semantically.

  Before providing the detailed derivations of \REF[a-c]{ex:7:31} above, I state the following vocabulary entries for second-person informal pronouns, the pronominal determiners in \REF[b-c]{ex:7:31} above (see also \chapref{sec:3}, \sectref{sec:3.5}).

\ea%33
    \label{ex:7:33}
\ea\label{ex:7:33a} [+D; --AUTH, +PART; INFORMAL]      $\rightarrow$  \textit{ihr}  /    [+F, +N, --O, --S]
\ex\label{ex:7:33b} [+D; --AUTH, +PART; INFORMAL]      $\rightarrow$  \textit{du}   /    [--O, --S]
\z
\z


The formal pronoun of address (i.e., \textit{Sie} ‘you(\textsc{formal})’) is discussed in the next subsection.

Returning to the data in \REF{ex:7:31}, recall that \textit{Schwein} can have three manifestations: It can be a mass, a singular count, or a plural count noun. I assume that the morphological feature [$\alpha$PL morph] on NumP is syntactically optional. If it is absent, \textit{Schwein} is interpreted as a mass noun, and it occurs with a null article. The subject DP in \REF{ex:7:31a} can be illustrated in the simplified structure seen in \figref{figex:7:34}.

\begin{figure}
	\caption{\emph{Schwein} as a mass noun}
	\label{figex:7:34}
	\begin{forest}
		[DP
			[$\emptyset$\textsubscript{D}]
			[NumP
				[Num
					[\textit{Schwein}\textsubscript{$k$}\\{[$\alpha$PL morph; $\beta$PL sem; $\alpha$ = $\beta$]}]
					[Num]
				]
				[NP\\t\textsubscript{$k$}]
			]
		]
	\end{forest}
\end{figure}

\newpage
Note that the number values of the noun are not specified. I suggest that this yields a (number-neutral) mass interpretation. The apparent singularity of the mass nominal in \REF{ex:7:31a} is due to singular being a default value here.

If [$\alpha$PL morph] is present on Num, it can have two values. [--PL] results in a singular count noun, and [+PL] brings about a plural count noun. The singular pronominal DP in \REF{ex:7:31b} is illustrated as seen in \figref{figex:7:35}.

\begin{figure}
	\caption{\emph{Schwein} as a singular count noun}
	\label{figex:7:35}
	\begin{forest}
		[DP
			[\textit{du}]
			[NumP
				[Num
					[\textit{Schwein}\textsubscript{$k$}\\{[--PL morph; --PL sem; $\alpha$ = $\beta$]}]
					[Num\\{[--PL morph]}]
				]
				[NP\\t\textsubscript{$k$}]
			]
		]
	\end{forest}
\end{figure}

\largerpage[-1]
The plural counterpart in \REF{ex:7:31c} is similar, but has positive values for number. This feature specification on Num percolates up to the other terminal nodes of the structure bringing about concord in agreement features. Depending on this specification (singular or plural), the corresponding pronominal determiner in \REF{ex:7:33a} or \REF{ex:7:33b} is inserted. Observe now that in the singular and plural cases, the morphological and semantic number coincide by \REF{ex:7:32b}. Note that this yields countability as a side effect and makes individuative or mass quantifiers (e.g., \textit{mehrere} ‘several’ vs. \textit{viel} ‘much’) compatible with individuative or mass nominals, respectively.\footnote{Again, there are other proposals. H.  \citet[8]{WieseMaling2005} claim that countability stems from the head noun. Second, \citet[122]{Crisma1999} argues that the indefinite article does not mark indefiniteness but brings about countability. Third, developing ideas of \citet{DelfittoSchroten1991}, \citeauthor{Panagiotidis2002} (\citeyear{Panagiotidis2002}: 58, \citeyear{Panagiotidis2003}: 421) suggests that countability originates with Num. The latter is similar to what is claimed in the main text. Also, as far as I can see, the current discussion does not need to make use of the countability feature [±COUNT]. However, the adoption of such a feature makes the formulation of the vocabulary insertion rules of determiners more straightforward (see \chapref{sec:8}, \sectref{sec:8.2.2.4}).}

  I argued in \chapref{sec:6} that bare role nouns in copular clauses do not involve NumP. Now, if there is no NumP in these instances at all, then the number values on these types of head nouns do not get specified resulting in a number neutral element (cf. also \citealt{MunnSchmitt2005}: 827). This makes bare role nouns similar to the mass nominals discussed in \figref{figex:7:34} above.\footnote{\label{foot:7:15} Presumably, the pragmatics will force a bare role noun to have an individuative (rather than mass) interpretation in copular TPs. Furthermore, one might suggest that nouns on their mass interpretation also lack NumP (cf. \citealt{Borer2005}). Note, however, that these nouns can combine with adjectives. With AgrP – the location of adjectives – high in the structure, NumP is also present. Notice though that the mass interpretation of the modified noun is not necessarily turned into a count interpretation (perhaps due to the mere presence of NumP). Thus, I continue to assume that, with the exception of bare role nouns, NumP is present and that the number feature on Num is syntactically optional. Having said that, note that the pronominal DP \textit{du Schwein} does not mean ‘you pork', but rather ‘you (pig =) idiot'. Given concord in agreement features, this means that NumP (and its number feature) is present here. Related to that, we utilized the number feature on Num to account for cases of true dis-agreement in the previous section. It seems clear then that nouns in pronominal DPs do not get an individuative interpretation by the pragmatics.}

With these general remarks in mind, I continue illustrating the relevant issues with some special cases involving the pronoun \textit{Sie} ‘you(\textsc{formal})’, an element rarely discussed in the literature in this respect. As before, I follow the idea that non-canonical cases may reveal the true nature of the phenomenon.

\subsubsection{Dis-agreement revisited}\label{sec:7.3.3.2}

As detailed in \sectref{sec:7.2}, DPs involving \textit{als}-nominals and TPs more generally may exhibit morphological dis-agreement in number. As discussed above, these cases, though, only involved apparent mismatches. Extending this empirical discussion, it is clear from the verbal agreement in \REF{ex:7:36a} that \textit{Sie} ‘you’ is morphologically plural. As is well known, \textit{Sie} is semantically ambiguous between singular and plural; that is, \textit{Sie} can be used to address one or several individuals. It is interesting to note that when this pronoun occurs with a non-plural noun in a copular sentence, it can have both interpretations \REF{ex:7:36a}. The presence of the floating quantifier disambiguates \textit{Sie} as having plural semantics and makes the latter reading more easily available. The DP is different. Although both \textit{Sie} and a non-plural noun can co-occur, this string only has a singular interpretation \REF{ex:7:36b}. This interpretative difference is indicated in the respective translations.\footnote{Note that the combination of formal \textit{Sie} ‘you’ and (negatively) emotive \textit{Schwein} ‘(pig =) idiot’ leads to a stylistic clash. This makes \REF{ex:7:36b} marked for some speakers. Cases similar to \REF{ex:7:36b} involve the fairly common strings \textit{Sie Armleuchter} ‘you (chandelier =) idiot’ or \textit{Sie Arsch} ‘you ass’ (see also \citealt{Darski1979}, \citealt{Roehrs2006b}).}

\ea%36
    \label{ex:7:36}
\ea  \label{ex:7:36a}
\gll Sie  sind   (alle) Arzt.\\
you are.\textsc{pl} {\db}all    doctor.\textsc{masc}\\
\glt ‘You are a doctor.’\\
\glt ‘You are (all) doctors.’
\ex  \label{ex:7:36b}
\gll Sie  Schwein\\
you pig.\textsc{neut}\\
\glt ‘you (idiot)’

    \#‘you idiots’
\z
\z

This difference in the semantics between \REF{ex:7:36a} and \REF{ex:7:36b} resembles the cases from \sectref{sec:7.2}. However, here \textit{both} the TP and the DP involve morphological dis-agreement. This is of special interest for the nominal domain as \REF{ex:7:36b} does not involve an \textit{als}-nominal.

In keeping with the discussion above, I suggest below that the semantic difference between \REF{ex:7:36a} and \REF{ex:7:36b} follows from the optional presence of NumP in the clausal domain but its obligatory presence in the nominal domain. Specifically, I proposed above that semantic number is due to an interaction between Num and the head noun. I propose that NumP is absent in \REF{ex:7:36a}. Consequently, the predicate noun is number-neutral. In contrast, NumP is present in \REF{ex:7:36b}, and the unexpected grammaticality is accounted for by a structure involving two separate nominals on a par with cases discussed in \chapref{sec:2}, \sectref{sec:2.3.5}. I consider this in more detail.

  Starting with the copular case in \REF{ex:7:36a}, recall from above that if there is no number morphology on the noun, then there is no NumP in the predicate nominal. Consequently, no agreement relation is established, and there is no true dis-agreement. To derive the difference between the singular and plural interpretations of the predicate nominal, I assume that there is a null counterpart of the distributivity operator \textit{alle} ‘all’ (\citealt{deSwartEtAl2007}: 218). I label this null element as DIST. I suggest that it is optionally present (when \textit{alle} is absent).

\ea%37
    \label{ex:7:37}
\gll     Sie  sind (DIST) Arzt.\\
  you are.\textsc{pl}      {}      doctor.\textsc{masc}\\
\z

If this operator is absent, we derive the singular interpretation; if the operator is present, we obtain the plural reading. Note in passing that this distributivity operator is incompatible with subjects that have singular semantics (i.e., the operator cannot distribute the relevant predicate over just one entity). As such, unlike \textit{Sie}, \textit{du} ‘you(\textsc{sg})’ combined with a non-plural noun (e.g., \textit{Du bist Arzt.} ‘You are a doctor.’) can only be singular in interpretation.

As for \REF{ex:7:36b}, I argued above that DPs always contain NumP. At first glance, this then appears to be a case of true morphological dis-agreement between the plural pronominal determiner and the singular nominal. In \sectref{sec:7.3.2}, similar cases were ruled out by suggesting that the singular nominal cannot establish an agreement relation with a plural pronominal determiner. Now, in order to explain the unexpected grammaticality of this example, I suggested in \citet{Roehrs2006b} that these types of DPs have a structural analysis different from the canonical DPs discussed earlier.

Specifically, it is proposed in that paper that this construction involves two nominals, a matrix nominal containing the pronominal determiner and an embedded nominal containing the dis-agreeing predicate nominal. The latter is proposed to be located in a specifier inside the matrix pronominal DP. For lack of a better term, this position was labeled as specifier of a Disagreement Phrase (DisP) in \chapref{sec:2}, \sectref{sec:2.3.5}, where cases like \textit{ihr junges} \textit{Gemüse} ‘you young (vegetable =) folks’ were discussed. Abstracting away from movement, the pronominal DP in \REF{ex:7:36b}, repeated below, can be illustrated as a first approximation as in \figref{figex:7:38}.

\glltree[\label{figex:7:38}]{
	Apparent dis-agreement in the DP\\
	\gll Sie  Schwein\\
	you pig.\textsc{neut}\\
	\glt {‘you (idiot)’}
}{
	[DP\textsubscript{\parbox{0mm}{\mbox{\textlangle e\textrangle}}}
		[\textit{Sie}]
		[DisP
			[{[\textsubscript{NumP} Num\textsubscript{[--PL]} [\textsubscript{NP} \textit{Schwein}]]}\textsubscript{\textlangle e,t\textrangle}]
			[Dis$'$
				[Dis]
				[NumP\textsubscript{\parbox{0mm}{\mbox{\textlangle e,t\textrangle}}}, s sep=10mm
					[Num\textsubscript{\parbox{0mm}{\mbox{[+PL]}}}]
					[NP\\\textit{e\textsubscript{N}}]
				]
			]
		]
	]
}

Having claimed above that Num has a specification for morphological number, I propose that both the matrix and the embedded nominal have a Num where the value of the former is plural, but the value of the latter is singular. In order to account for the absence of a featural clash between the two NumPs in \figref{figex:7:38}, I blame the lack of obligatory concord between these two nominals on the absence of a head (Mod) that mediates concord in agreement features between these nominals (for more details and other cases, see \chapref{sec:8}, \sectref{sec:8.3.1}). To be clear, then, both nominals in \figref{figex:7:38} form independent agreement domains, and consequently, there is no true morphological dis-agreement. These are the basic assumptions of pronominal DPs involving \textit{Sie} ‘you’. Next, I flesh out some of the details of the derivation in \figref{figex:7:38}, specifically those about number.

\subsubsection{Pronominal DPs involving \textit{Sie} in more detail}\label{sec:7.3.3.3}

As briefly alluded to above, the two number values, morphological and semantic, can also diverge; for instance, pluralia tantum nouns such as \textit{Ferien} ‘vacation/holidays’ are morphologically plural but semantically singular (i.e., they involve one entity or event).\footnote{For  \citet[263]{PesetskyTorrego2007}, English pluralia tantum nouns like \textit{scissors} are also taken to have a lexically valued feature for number (see also \citealt{Borer2005}: 105).} In other words, they are lexically specified such that they can only occur in the corresponding context in \REF{ex:7:39a}. Furthermore, I assume that \textit{Sie} ‘you’ has a specification for morphological number but that it does not have a fixed value for semantic number – it can be used to address one or several individuals \REF{ex:7:39b}.\footnote{Note that it might be possible to restate the restriction to a certain morphological number in \REF[a-b]{ex:7:39} by employing restriction features (see the discussion of uninflected \textit{dies} ‘this’ in \chapref{sec:4}, \sectref{sec:4.5} and \chapref{sec:8}, \sectref{sec:8.2.2.5} and \ref{sec:8.2.2.6}). However, I continue with the system laid out in \citet{Roehrs2006b}. Notice also that since \textit{Sie} can be merged in semantically singular or plural contexts, the specification for semantic number involving a variable in \REF{ex:7:39b} could presumably be left out. For reasons of parsimony, I continue with the statement in \REF{ex:7:39b}. Interestingly, the Dutch counterpart \textit{u} ‘you(\textsc{formal})’ is morphologically singular and semantically singular or plural. On current assumptions, its feature specification would be [--PL morph; $\beta$PL sem].}

\ea%39
    \label{ex:7:39}
\ea\label{ex:7:39a}   pluralia tantum noun

    [+PL morph; --PL sem]
\ex\label{ex:7:39b}
  {Sie}

    [+PL morph; $\beta$PL sem]
\z
\z


To be clear, compared to regular nouns (\sectref{sec:7.3.3.1}), these elements have different values for morphological and semantic number, and (consequently) they lack the equality statement “$\alpha$ = $\beta$”. Furthermore, while both values are fixed with pluralia tantum nouns, the value for semantic number of \textit{Sie} is unspecified. I assume that this value is determined by the head noun in the complement position of \textit{Sie} mediated by NumP as discussed in \sectref{sec:7.3.3.1}. To be clear, these lexically specified elements can only be inserted in contexts that are compatible with the featural specification of their vocabulary entries. Before I return to the intriguing case of \textit{Sie} occurring with a non-plural noun, I discuss the derivation of \textit{Sie} in the environment of a plural noun (where both elements agree in morphological number). This lays the groundwork for the remaining discussion.

  Considering \figref{figex:7:40}, note that all number values of \textit{Sie} are positive. The morphological number [+PL morph] on \textit{Sie} is compatible with the morphological number on Num. The semantic number [$\beta$PL sem] on \textit{Sie} is positive. Specifically, due to the interaction of the head noun and Num, the semantic number on the noun is positive and, consequently, only a semantically plural number on \textit{Sie} is compatible with a semantically plural number on the noun.

\glltree[\label{figex:7:40}]{
		\gll Sie         Schweine\\
		you(\textsc{pl}) pigs\\
		\glt {‘you idiots’}
}{
	[DP
		[\textit{Sie}\\{[+PL morph; +PL sem]}]
		[NumP
			[Num
				[\textit{Schweine}\textsubscript{$k$}\\{[+PL morph; +PL sem; $\alpha$ = $\beta$]}]
				[Num\\{[+PL morph]}]
			]
			[NP\\t\textsubscript{$k$}]
		]
	]
}

This derivation involving formal \textit{Sie} ‘you’ is similar to informal \textit{ihr Schweine} ‘you idiots’, briefly discussed in \sectref{sec:7.3.3.1}. With this in place, I provide more details of the derivation in \figref{figex:7:38}, where \textit{Sie} occurs in the context of a non-plural noun.

For easier readability, I illustrate the pronominal DP in \figref{figex:7:41} in two parts: \figref{figex:7:41} shows the matrix nominal, and \figref{figex:7:42} further below illustrates the embedded part XP located in Spec,DisP in \figref{figex:7:41}. Starting with the matrix nominal, I assume that the complement of \textit{Sie} is not \textit{Schwein} itself but involves a null pluralia tantum noun. Since both \textit{Sie} and this type of noun are lexically specified for [+PL morph], both elements are morphologically compatible. However, \textit{Sie} is only compatible with a pluralia tantum noun if \textit{Sie} has a negative value on its [$\beta$PL sem] statement. This leads to singular semantics of \textit{Sie}.\footnote{This is different from the copular case \textit{Sie sind Arzt} ‘(you are doctor =) you (all) are doctors’ discussed earlier. Here, \textit{Sie}, the subject DP, has an ordinary null noun as part of its complement structure. This null noun has the familiar specification [$\alpha$PL morph; $\beta$PL sem; $\alpha$ = $\beta$]. Mediated by NumP, the subject DP (i.e., \textit{Sie} Num \textit{e\textsubscript{N}}) gets plural semantics (note also that given the null elements, this makes \textit{Sie} structurally similar to \textit{Sie Schweine} ‘you idiots’). The subject DP is then combined with the non-plural predicate noun \textit{Arzt} ‘doctor’ via the head Pr and the auxiliary (\chapref{sec:6}). An assumed null distributivity operator accounts for the compatibility of plural \textit{Sie} and the non-plural noun (cf. \REF{ex:7:37}).}

\glltree[\label{figex:7:41}\small]{
	\gll Sie  Schwein\\
	you pig.\textsc{neut}\\
	\glt {‘you (idiot)’}
}{
	[DP\textsubscript{\parbox{0mm}{\mbox{\textlangle e\textrangle}}}
		[\textit{Sie}\textsubscript{\parbox{0mm}{\mbox{\textlangle\textlangle e,t\textrangle e\textrangle}}}\\{[+PL morph; --PL sem]}]
		[DisP\textsubscript{\parbox{0mm}{\mbox{\textlangle e,t\textrangle}}}
			[XP\textsubscript{\parbox{0mm}{\mbox{\textlangle e,t\textrangle}}}]
			[Dis$'$
				[Dis]
				[NumP\textsubscript{\parbox{0mm}{\mbox{\textlangle e,t\textrangle}}}
					[Num
						[\textit{e\textsubscript{N}}\textsubscript{$i$}\\{[+PL morph; --PL sem]}]
						[Num\\{[+PL morph]}]
					]
					[NP\\t\textsubscript{$i$}]
				]
			]
		]
	]
}

As for the embedded nominal, this NumP – distinguished by the subscript 2 below – is specified as [--PL morph], and the ordinary noun \textit{Schwein} comes out as a singular count element, see \figref{figex:7:42}.

\begin{figure}
	\caption{\emph{Schwein} as an embedded nominal}
	\label{figex:7:42}
	\begin{forest}
		[NumP\textsubscript{2}\parbox{0mm}{\mbox{{(= XP)}}}
			[~]
			[Num$'$\textsubscript{2}
				[Num\textsubscript{2}
					[\textit{Schwein}\textsubscript{$k$}\\{[--PL morph; --PL sem; $\alpha$ = $\beta$]}]
					[Num\textsubscript{2}\\{[--PL morph]}]
				]
				[NP\\t\textsubscript{$k$}]
			]
		]
	\end{forest}
\end{figure}

To recapitulate, the matrix nominal involves a plural NumP and a (null) pluralia tantum head noun, and the embedded nominal contains a singular NumP and an (overt) ordinary head noun. In other words, although there are two NumPs in \figref{figex:7:41} and \figref{figex:7:42} with different (morphological) specifications for number, both nominals are semantically singular. I now turn to the question as to how the two nominals in \figref{figex:7:41} and \figref{figex:7:42} are semantically combined.

  I assume that the null noun in the matrix nominal is of the kind type. For this element to function as a predicate, it must combine with REL in NumP. Suppose now that the matrix nominal and the embedded nominal in \figref{figex:7:41} and \figref{figex:7:42} are combined by Predicate Modification, which conjoins two elements of the same semantic type (i.e., \textlangle e,t\textrangle). In order to avoid type mismatch, the embedded predicate nominal must be of the same type. This is true. With the kind noun \textit{Schwein} ‘(pig =) idiot’ present there, the latter must also involve REL in NumP (to bring about a predicate), which is in line with the assumptions of \chapref{sec:6}. Thus, we have justification for the assumption of two NumPs, one in the matrix nominal and one in the embedded one. Furthermore, assuming that Predicate Modification only combines two nominals of the same general semantics, I can also account for the fact that the semantically singular nominal in Spec,DisP is only compatible with a semantically singular matrix nominal (for more details, see \citealt{Roehrs2006b}: 168). This explains the lack of a plural reading in \textit{Sie Schwein} ‘you (pig =) idiot’.

To sum up, following the literature, I suggested that morphological number originates with NumP. Furthermore, I proposed that semantic number is the result of an intricate interplay between Num and the head noun. In addition, I considered some non-canonical cases in copular TPs and pronominal DPs, where \textit{Sie} occurs in the context of a non-plural noun. I explained the grammaticality of \textit{Sie} in the copular cases in the same fashion as in \sectref{sec:7.3.2}, namely by the syntactic optionality of NumP in the clause: In this particular instance, NumP is absent. As for the pronominal DPs, I accounted for the grammaticality of \textit{Sie} here by proposing a different structure where the pronominal determiner is in the matrix DP, but the non-plural noun forms a complex specifier embedded in the matrix DP.

More generally, I proposed that NumP is present for cartographic reasons in noun phrases bigger than NP. This means that all noun phrases (except those with mass and bare role nouns) involve morphological number. Semantic number may come about in one of three ways: (i) It is lexically specified on the head noun (pluralia tantum), (ii) it originates with the (regular) kind noun if [$\alpha$PL morph; $\beta$PL sem] is specified via the morphological value of Num and the equality statement “$\alpha$ = $\beta$”, or (iii) the semantic number of bare role nouns in copular sentences stems from the pragmatics (in the singular cases; see Footnote \ref{foot:7:15}) or from a distributivity operator (in the plural cases; see \REF{ex:7:37} above).

If correct, this discussion leads to some interesting issues.\footnote{I mention just one issue in this context. As pointed out in the main text, the kind noun in the embedded nominal in \figref{figex:7:42} requires the presence of REL to supply a predicate. REL, in turn, should trigger the presence of \textit{ein}. However, this yields an ungrammatical example.
	\ea \label{ex:7:20:i}
	\gll* Sie [\textsubscript{ArtP} \textit{’n} \textit{Schwein}] e\textsubscript{N}\\
	 {} you      {}   {\db}a pig.\textsc{neut}\\
	 \z
	 
	 To explain the ungrammaticality, there is more involved here than in cases like *\textit{du ’n Schwein} ‘[intended:] you a (pig =) idiot’ (see \sectref{sec:7.3.2.5}) – notice that \textit{Sie} and \textit{ein} in \REF{ex:7:20:i} are in different nominals. Without going into too much detail, I suggest that \textit{ein} is deleted in the presence of the adjacent pronominal determiner. Note in this regard that presumably, the latter is retained as it has semantics unlike vacuous \textit{ein}.} I will not pursue these questions here. My main goal in this section was to show that elements other than \textit{ein} can be held responsible for morphological and semantic number. If this is tenable, then I can continue claiming that \textit{ein} is not a reflex of number, be it morphological or semantic – it is a semantically vacuous element (Hypothesis 1a). Next, I consider constructions involving \textit{als}-nominals in more detail. While these instances have certain similarities to the dis-agreement cases discussed above, they are different in a number of ways.

\subsection{Agreement in constructions involving \textit{als}-nominals}\label{sec:7.3.4}

Finally, I discuss \textit{als}-constructions, nominal and clausal constructions involving \textit{als} ‘as’ and a following noun, the latter two elements labeled as \textit{als}-nominals. I begin by showing how \textit{als}-nominals combine with the preceding part of their structures. This includes a brief discussion of the semantic types and \textit{als} as a realization operator. Then I address the agreement facts with regard to number.

\subsubsection{Structure and semantics}\label{sec:7.3.4.1}

In \chapref{sec:6}, \sectref{sec:6.3.4.2}, I offered the tentative claim that \textit{als} ‘as’ in the DP is a generalized capacity operator that combines with a noun, be it a role or a kind noun, to return an element of type \textlangle e,t\textrangle . I suggested there that syntactically, the \textit{als}-nominal is right-adjoined to its preceding structure and that semantically, this \textit{als}-nominal combines with another element by Predicate Modification (which combines two elements of type \textlangle e,t\textrangle ). As such, I proposed that the \textit{als}-nominal is adjoined to an element of type \textlangle e,t\textrangle . Recalling that \textit{du} ‘you’ itself is argued to have internal structure, NumP is such an element. I also suggested in that section that \textit{als} is the head of ModP, and I proposed that this ModP is adjoined to NumP of the pronominal DP. The resultant structure is slightly adapted here as in \figref{figex:7:43}.%, deriving \REF{ex:7:43a} as in \REF{ex:7:43b}:

\glltree[\label{figex:7:43}]{
	\gll du           als Arzt\\
	you(\textsc{sg}) as  doctor.\textsc{masc}\\
	\glt {‘you as a doctor’}
}{
	[DP\textsubscript{\parbox{0mm}{\mbox{\textlangle e\textrangle}}}
		[\textit{du}\textsubscript{\parbox{0mm}{\mbox{\textlangle\textlangle e,t\textrangle e\textrangle}}}]
		[NumP\textsubscript{\parbox{0mm}{\mbox{\textlangle e,t\textrangle}}}
			[NumP\textsubscript{\parbox{0mm}{\mbox{\textlangle e,t\textrangle}}}, s sep=20mm
				[Num\textsubscript{\parbox{0mm}{\mbox{\textsc{rel}\textlangle e\textlangle e,t\textrangle\textrangle}}}]
				[NP\\\textit{e\textsubscript{N}}\textsubscript{\textlangle e\textrangle}]
			]
			[ModP\textsubscript{\parbox{0mm}{\mbox{\textlangle e,t\textrangle}}}, edge label={node[above right]{(Predicate Modification)}}
				[\textit{als}]
				[NP\\\textit{Arzt}\textsubscript{\textlangle e\textrangle}]
			]
		]
	]
}

As for the non-copular TP,   \citet[218]{deSwartEtAl2007} assume that \textit{als}-nominals are verb modifiers. I propose that the \textit{als}-nominal in the clausal cases is adjoined to the VP. Note that intransitive verbal predicates are of type \textlangle e,t\textrangle . Semantically, this is the same type as that of \textit{als}-nominals. The simplified structure is provided in \figref{figex:7:44}.

\glltree[\label{figex:7:44}]{
	\gll Du          sprichst als Arzt.\\
	you(\textsc{sg}) speak     as  doctor.\textsc{masc}\\
	\glt {‘You speak as a doctors.’}
}{
	[TP
		[\textit{du}\textsubscript{$k$}]
		[T$'$
			[\textit{sprichst}\textsubscript{$i$}]
			[\textit{v}P
				[t\textsubscript{$k$}]
				[VP\textsubscript{\parbox{0mm}{\mbox{\textlangle e,t\textrangle}}}, s sep=10mm
					[VP\textsubscript{\parbox{0mm}{\mbox{\textlangle e,t\textrangle}}}\\t\textsubscript{$i$}]
					[{[\textsubscript{ModP} \textit{als Arzt}]}\textsubscript{\textlangle e,t\textrangle }]
				]
			]
		]
	]
}

These are the basic assumptions about the syntax and semantics. Before moving on to the agreement facts, I specify the role of \textit{als} in more detail.

In \chapref{sec:6}, we saw that CAP and REL are both possible with role nouns. Recall that CAP involves direct predication but that REL involves an indirect path to predication. Note again that these realization operators are mutually exclusive – only one is needed to supply a predicate, and the presence of a second would lead to type mismatch. In other words, CAP must be, at least in principle, optional. I now make the assumption that all realization operators are optional in type but not in number; that is, while there is a potential choice between different operators, the presence of one such operator is required to supply a predicate. This might lead us to expect that there are other realization operators.

As tentatively suggested in \chapref{sec:6}, \sectref{sec:6.3.4.2}, I take \textit{als} to be such an element. Similar to the two other elements, if \textit{als} is present, both CAP and REL cannot be. I reiterate the proposal that \textit{als} takes all elements of type \textlangle e\textrangle , that is, role nouns or kind nouns alike, as arguments and returns predicate nominals (type \textlangle e,t\textrangle ). In other words, \textit{als} is a general(ized) capacity operator of type \textlangle e,\textlangle e,t\textrangle \textrangle . While of the same semantic type, this makes this element different from CAP and REL. I turn to the agreement facts.

\subsubsection{Agreement}\label{sec:7.3.4.2}
\largerpage
Starting with the (non-copular) TPs, I assume that agreement in morphological number between the subject and the \textit{als}-nominal is established independently of the presence of the verb. In a way, this makes the adjunction of the \textit{als}-nominal in the clause similar to a displaced relative clause – in both instances, the two related elements are separated by a verb.\footnote{Below is an example of a displaced relative clause.
	\ea
	\gll Er hat das Haus gekauft, das nicht so teuer war.\\
	he has the house bought  that not    so expensive was\\
	\glt ‘He bought the house that was not so expensive.’
	\z } Note that this parallel pattern fits well with the discussion in \chapref{sec:6}, \sectref{sec:6.3.4.1}, where I argued that \textit{als}-nominals in pronominal DPs are similar to relative clauses. Consequently, I treat the \textit{als}-nominals in the non-copular TPs and the ones in the pronominal DPs in a related way. Below, I put the verb in parentheses discussing both the TPs and the DPs at the same time.

We saw in \sectref{sec:7.3.1} that plural morphology on the noun indicates the presence of NumP. Assuming that the number specification of the predicate nominal inside the \textit{als}-nominal has to establish an agreement relation with the pronominal element, this rules out \REF{ex:7:45a} but allows \REF{ex:7:45b}.

\ea%45
    \label{ex:7:45}
\ea[*]{ \label{ex:7:45a}
\gll du          (sprichst) als Ärzte\\
you(\textsc{sg}) {\db}speak      as   doctors\\
}
\ex \label{ex:7:45b}
\gll ihr      (sprecht) als Ärzte\\
you(\textsc{pl}) {\db}speak     as  doctors\\
\glt ‘you (speak) as doctors’
\z
\z

In contrast, a non-plural noun in the \textit{als}-nominal does not project NumP. It involves NP and does not enter into an agreement relation with the pronominal element. As such, both \REF{ex:7:46a} and \REF{ex:7:46b} are fine.

\ea%46
    \label{ex:7:46}
\ea \label{ex:7:46a}
\gll du          (sprichst) als Arzt\\
you(\textsc{sg}) {\db}speak      as  doctor.\textsc{masc}\\
\glt ‘you (speak) as a doctor’
\ex \label{ex:7:46b}
\gll ihr        (sprecht) als Arzt\\
you(\textsc{pl}) {\db}speak     as   doctor.\textsc{masc}\\
\glt ‘you (speak) as doctors’
\z
\z

I assume that the plural interpretation in \REF{ex:7:46b} follows from the presence of a null distributivity operator just like in the copular cases. So far, the agreement facts are similar to what we saw above.

  Turning to singular nouns, I observed in \sectref{sec:7.2.3} that the presence of \textit{ein} leads to awkwardness in a singular context \REF{ex:7:47a}. As also pointed out in that section, a singular \textit{als}-nominal is ungrammatical with a plural pronominal \REF{ex:7:47b}, where the presence of a non-copular verb and quantifier yields a somewhat better result than the counterpart without those two elements (this correlation in \REF{ex:7:47b} is indicated by two sets of parentheses).

\ea%47
    \label{ex:7:47}
\ea[?]{
\label{ex:7:47a}
\gll du            (sprichst) als ’n Arzt\\
you(\textsc{sg}) {\db}speak      as    {\db}a doctor.\textsc{masc}\\
\glt ‘you (speak) as a doctor’
}
\ex[(??)?*]{\label{ex:7:47b}
\gll ihr     (sprecht alle) als ’n Arzt\\
you(\textsc{pl}) {\db}speak    all    as    {\db}a doctor.\textsc{masc}\\
}
\z
\z

The agreement facts in \REF{ex:7:47} are similar to the copular constructions discussed in \sectref{sec:7.3.2} (cf. also \tabref{tab:7:1} and \ref{tab:7:2}). There is only one exception: The presence of \textit{ein} in a copular construction such as \textit{Du bist ’n Arzt} ‘You are a doctor’ is completely fine for many speakers, but the presence of \textit{ein} in an \textit{als}-nominal like \REF{ex:7:47a} leads to a certain degree of awkwardness. Note though that the latter case involves \textit{als} ‘as’.

In order to account for \REF{ex:7:47}, I propose that syntactically, \textit{als} has intermediate characteristics as regards CAP and REL. To see this, I briefly return to \REF{ex:7:45} and \REF{ex:7:46}. I suggest for \REF{ex:7:45} that \textit{als} is in NumP with a plural noun (similar to REL) but I assume for \REF{ex:7:46} that it is in NP with a non-plural noun (similar to CAP) – the first case will force morphological agreement, the second will not.\footnote{This means that Mod takes different complements (NP, NumP) and that \textit{als} might move to Mod.} This rules out the plural noun in \REF{ex:7:45a} but allows the non-plural noun in \REF{ex:7:46b}. Turning to the singular nouns in \REF{ex:7:47}, I propose that \textit{als} does not trigger \textit{ein}. This makes \textit{als} similar to CAP and explains the markedness of \REF{ex:7:47a}. As to the more marked \REF{ex:7:47b}, \textit{als} does not bring about \textit{ein} either, and additionally, no agreement relation between the plural pronominal element and the singular noun in the \textit{als}-nominal is established.

  To sum up the entire section, the preceding discussion revealed an interesting syntax-semantics correlation: Cases of morphological agreement are very restricted in their interpretations, but instances of apparent morphological dis-agreement (copular cases and \textit{als-}nominals) are fairly free. Given the proposed interaction between Num and the head noun, this difference follows from the presence of NumP in the first cases but its absence in the second. Importantly, the role of \textit{ein} is only indirect. Its presence allows us to draw conclusions about the syntactic structure and the type of realization operator present in that structure (i.e., REL). As seen throughout this section, \textit{ein} does not specifify number, neither morphologically nor semantically.

\section{Conclusion}\label{sec:7.4}

This chapter provided the second and final consequence of the proposals involving \textit{ein}, here focusing on number. Continuing the investigation from the previous chapter, I provided a survey of the data and discussed issues pertaining to morphological and semantic number in the nominal and clausal domains. I showed that the DP is more restricted in this regard than the TP. I proposed that this is accounted for by the obligatory presence of NumP in the DP (a reflex of the cartography of the DP) and its optional syntactic presence in the TP.

Specifically, I proposed that morphological number resides in Num and semantic number is a result of an intricate interaction between Num and the head noun. With current purposes in mind, I can conclude that \textit{ein} itself does not determine number, neither morphologically nor semantically. Rather, we have seen that \textit{ein} indicates the presence of a certain amount of structure on top of NP (Hypothesis 3a) and that it flags the presence of the operator REL (Hypothesis 3b). I also showed that elements other than \textit{ein} can flag the presence of REL precluding \textit{ein} from occurring. Again, these facts find a natural explanation if \textit{ein} is semantically vacuous (Hypothesis 1a); that is, \textit{ein} is an element that can be absent without a loss in meaning.

More generally, to the extent that the discussion here proves tenable, it provides further support for \citegen{deSwartEtAl2007} proposal, and more generally, it strengthens the parallels between the DP and TP.

