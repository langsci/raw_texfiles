\chapter{Discussion and concluding remarks}\label{sec:8}

\section{The bigger picture}\label{sec:8.1}

Focusing on German, one goal of this book was to provide a more comprehensive discussion of adjectival inflections and \textit{ein} compared to that which exists in the literature. Beside an overview of these important empirical subdomains of the noun phrase, the second goal was to find commonalities and differences between these two types of elements and to discuss certain theoretical issues that arise in these contexts. It is important to point out again that both goals go in partially different directions.

Providing an overview aims to be fairly exhaustive, but painting the big picture is an attempt to see what different domains, empirical or theoretic, have in common. As such, discussing issues to some comprehensive degree in one domain tends to move the focus away from traits shared by the different domains. In this final chapter, I try to make more headway toward the second goal; that is, toward comparing both adjectival inflections and \textit{ein} more directly. In the second part of this chapter, I discuss some further consequences of the current analysis, show avenues for future research, and draw some more general conclusions. Before I compare adjectival inflections and \textit{ein} in more detail, recall the general contents of the previous chapters.

In \chapref{sec:1}, I laid out my general assumptions and formulated the main claims of this book. \chapref{sec:2} focused on adjectival inflections and included a detailed investigation of the special role of \textit{ein}-words in this regard. In \chapref{sec:3}, variation was discussed leading to the postulation of two secondary mechanisms that regulate adjectival endings, and in \chapref{sec:4}, some consequences of the analysis for other accounts were addressed. \chapref{sec:5} provided an overview of the different types of \textit{ein}. This was followed by \textit{ein} being discussed with regard to emotiveness and number in \chapref{sec:6} and \ref{sec:7}, respectively. The analysis of adjectival inflections developed earlier helped narrow down the choices of plausible analyses in later chapters.

Contrasting these phenomena, we observed that the analysis of adjectival inflections and that of \textit{ein} are closely connected. In fact, I argued that adjectival inflections and \textit{ein} are similar in certain aspects but different in others. I review the main hypotheses discussed throughout this book starting with the similarities followed by the differences. Besides providing a summary and drawing some conclusions, I add a few other details.

\subsection{Commonalities of adjectival inflections and \textit{ein}}
\label{sec:8.1.1}
I formulated the following two main claims.

\ea%1
    \label{ex:8:1}

         Hypothesis 1

Adjectival inflections and \textit{ein}:

\ea \label{ex:8:1a}    are expletive elements and

\ex\label{ex:8:1b}     indicate abstract structure in the noun phrase.
\z
\z


Beginning with Hypothesis 1a, we have seen that both adjectival endings and \textit{ein} are not a reflex of (in-)definiteness. While both elements share this characteristic, evidence for Hypothesis 1a has also come from other, different empirical domains. Specifically, adjectival inflections have nothing to do with the restrictiveness of the interpretation of modifiers or “referentiality”, and \textit{ein} is not an exponent of emotiveness and number. These points were summarized in the statement that adjectival inflections and \textit{ein} are semantically vacuous.

  As for Hypothesis 1b, both adjectival inflections and \textit{ein} indicate abstract structure. However, they do so in different ways. Compare Hypothesis 2a to Hypothesis 3a.

\ea%2
    \label{ex:8:2}

         Hypothesis 2

  Adjectival inflections:
\ea \label{ex:8:2a}  indicate abstract structure in the higher layers of the noun phrase (DP vs. LPP),       [and] they provide clues about structures involving various degrees of embedding       of adjectives (simple vs. complex DPs).
\z
\z

\ea%3
    \label{ex:8:3}

          Hypothesis 3

  \textit{Ein}:
\ea \label{ex:8:3a} indicates abstract structure in the lower layers of the noun phrase (NP vs. ArtP).
\z
\z


While both adjectival inflections and \textit{ein} indicate certain sizes of abstract structure, the former also provides clues about different types of embeddings.

In what follows, I briefly review some of the evidence for Hypotheses 1, 2a, and 3a below. Hypotheses 2b (adjectival endings make features visible) and 3b (\textit{ein} supports or flags operators), which deal with differences between adjectival inflections and \textit{ein}, are discussed in \sectref{sec:8.1.2}. I begin with the intersecting semantic concept of (in-)definiteness, where both types of elements not only show a similar lack of semantic import but also directly interact with one another. Following that, I repeat one more argument from the other, different empirical domains showing that both types of elements are semantically vacuous: Adjectival inflections do not indicate the restrictiveness of the interpretation of modifiers, and \textit{ein} is not related to number. Finally, I review the main facts that show that both types of elements provide clues about abstract structure (the indication of different types of embeddings by adjectival inflections is left for the section dedicated to the differences – \sectref{sec:8.1.2}).

\subsubsection{Some evidence for Hypothesis 1a}\label{sec:8.1.1.1}

Starting with (in-)definiteness, consider again cases in the nominative, the (a)-examples, and in the dative, the (b)-examples.

\ea%4
    \label{ex:8:4}
\ea\label{ex:8:4a}
\gll ein gut-er    Wein\\
a    good-\textsc{st} wine.\textsc{masc}\\
\glt ‘a good wine’
\ex\label{ex:8:4b}
\gll mit   ein-em gut-en     Wein\\
with a-\textsc{st}     good-\textsc{wk} wine.\textsc{masc}\\
\glt ‘with a good wine’
\z
\z

\ea%5
    \label{ex:8:5}
\ea\label{ex:8:5a}
\gll sein gut-er    Wein\\
his   good-\textsc{st} wine.\textsc{masc}\\
\glt ‘his good wine’
\ex\label{ex:8:5b}
\gll mit   sein-em gut-en     Wein\\
with his-\textsc{st}    good-\textsc{wk} wine.\textsc{masc}\\
\glt ‘with his good wine’
\z
\z

Traditionally, the article \textit{ein} ‘a’ is taken to be an indefinite element, but the possessive article \textit{sein} ‘his’ is a definite one. With that in mind, compare the (a)-examples to their (b)-counterparts. We observe that indefinite contexts exhibit strong and weak adjectives \REF{ex:8:4} but that definite environments also show strong and weak adjectives \REF{ex:8:5}. Put differently and contrasting the (a)-examples to each other and the (b)-examples to each other, strong adjectives occur in both indefinite and definite contexts, and weak adjectives also surface in both indefinite and definite environments.

  Furthermore, what is sometimes left unmentioned in this context is that determiners can have the exact same inflections as adjectives (and that is why I referred to these inflections as adjectival, rather than adjective, endings). Considering the (b)-examples above, we note that both the (indefinite) article and the (definite) possessive article have strong inflections as their own endings. Indeed, I discussed one case in \chapref{sec:3}, \sectref{sec:3.4}, where certain (definite) \textit{der}-words may optionally have a strong or a weak ending. Considering the distributions of these inflections in all these cases, we can observe again that there is no association of adjectival inflections with (in-)definiteness.

  Recall though that I proposed in \chapref{sec:5} that possessive articles consist of a possessive component and \textit{ein} (e.g., \textit{s-ein} ‘his’). If we assume that \textit{ein} itself determines the adjectival inflections, then we can observe that \textit{ein} occurs with a strong adjective in the nominative but with a weak adjective in the dative. This would be consistent with the claim that \textit{ein} itself is indefinite, and the inflections on adjectives would vary according to morphological case. However, weak adjectives are also possible in the nominative \REF{ex:8:6a}, and strong adjectives can also appear in the dative \REF{ex:8:6b}.

\ea%6
    \label{ex:8:6}
\ea\label{ex:8:6a}
\gll seine gut-en     Weine\\
his    good-\textsc{wk} wines\\
\glt ‘his good wines’
\ex\label{ex:8:6b}
\gll mit   gut-em   Wein\\
with good-\textsc{st} wine.\textsc{masc}\\
\glt ‘with good wine’
\z
\z

Thus, there is no correlation between case and the strong/weak alternation on the adjective either.   These contradictory points disappear once we claim that neither adjectival inflections nor \textit{ein} are related to (in-)definiteness (or case). Rather, both of these elements are semantically vacuous and interact with each other morpho-syntactically. I proposed in \chapref{sec:2} that determiners including \textit{ein} trigger Impoverishment on the adjective yielding the weak endings. There is one exception: Instances of \textit{ein} specified as [--N, --O] are special – in this featural context (along with the absence of positive features for definiteness and deixis), [+D] does not trigger Impoverishment on the adjective. It is clear that adjectival inflections and \textit{ein} are directly related albeit not by the semantics.

  Turning to the other, differing empirical domains, I point out again that adjectival endings do not indicate the restrictiveness of the interpretation of modifiers. Considering \REF{ex:8:7}, notice that both a restrictive and a non-restrictive adjective can have a weak ending.

\ea%7
    \label{ex:8:7}
\ea \label{ex:8:7a}
\gll der alt-e    Mann\\
the  old\textsc{-wk} man.\textsc{masc}\\
\glt ‘the man that is old’
\ex \label{ex:8:7b}
\gll der (übrigens)  alt-e     Mann\\
the   {\db}incidentally old-\textsc{wk} man.\textsc{masc}\\
\glt ‘the man, who is (by the way) old’
\z
\z

Recalling that determiners move from ArtP to DP, I proposed in \chapref{sec:4} that the determiner is interpreted above the adjective in \REF{ex:8:7a} but below the adjective in \REF{ex:8:7b}. The latter also involves a coindexed \textit{pro}. The weak endings on the adjectives are the result of Impoverishment occurring in a canonical DP structure. This analysis is consistent with the fact that adjectival endings involve no semantics.

  As for \textit{ein}, I claimed that this element is not associated with number (and related countability). To illustrate this claim again, notice that the example in \REF{ex:8:8a} denotes a singularity, and the ones in \REF[b-c]{ex:8:8}, taken from the Appendix, involve pluralities.

\ea%8
    \label{ex:8:8}
\ea\label{ex:8:8a}
\gll Nur  EINE Katze    war auf dem Hof!\\
only one  cat.\textsc{fem} was in   the  yard\\
\glt ‘Only one cat was in the yard!’
\ex[\%]{\label{ex:8:8b}
\gll Och so n-e  süssen Katzis$\dots$ :-)\\
oh,  so a-\textsc{pl} cute    kittens\\
\glt ‘Oh, such cute kittens!’
}
\ex[\%]{\label{ex:8:8c}
\gll Gerade gesehen$\dots$ was   für eine Idioten, die   so Aufmerksamkeit wollen\\
just       seen          what for  a-\textsc{pl} idiots    who so attention             want\\
\glt ‘Just seen… what kind of idiots that want attention like that.’
}
\z
\z

Contrasting these two types of examples, we observe that \textit{ein} can appear in very diverse semantic contexts. Again, given that \textit{ein} appears to be associated with different, contradictory semantics, I proposed that this element is also semantically vacuous. I suggested in \chapref{sec:5} that \textit{EIN} in \REF{ex:8:8a} consists of \textit{ein} and the null operator $\emptyset$\textsubscript{[--PL]} inducing singularity, that \textit{ein} makes null operators visible \REF[b-c]{ex:8:8}, and more generally that number is due to an interaction between the head noun and Num (\chapref{sec:7}). I return to the discussion of \REF[b-c]{ex:8:8} in \sectref{sec:8.2.2.3} below.

Given these facts, we expect that adjectival inflections and \textit{ein} can be left out without a change in meaning. I showed that this is indeed the case in certain lexical contexts, with specific sets of adjectives in non-elliptical contexts \REF{ex:8:9a} and with role nouns in predicate nominals \REF{ex:8:9b}.

\ea%9
    \label{ex:8:9}
\ea\label{ex:8:9a}
\gll Das ist ein lila(-nes)      Buch.\\
this is   a    purple-\textsc{infl} book.\textsc{neut}\\
\glt ‘This is a purple book.’
\ex\label{ex:8:9b}
\gll Er ist (ein) Lehrer.\\
he is    {\db}a     teacher.\textsc{masc}\\
\glt ‘He is a teacher.’
\z
\z

Furthermore, the suggestion that both types of elements are semantically vacuous also allows us to avoid the conclusion that certain semantic features are redundantly present when these elements do co-occur. Again, this holds for both adjectival inflections and \textit{ein} (I return to \REF{ex:8:10b} in \sectref{sec:8.2.2.2}).

\ea%10
    \label{ex:8:10}
\ea\label{ex:8:10a}
\gll gut-e         teur-e                Autos\\
good-\textsc{infl} expensive-\textsc{infl} cars\\
\glt ‘good expensive cars’
\ex[\%]{\label{ex:8:10b}
\gll k-eine so’ne Autos\\
\textsc{neg}-a {so a}   cars\\
\glt ‘no such cars’
}
\z
\z

If these elements are indeed semantically vacuous, then the actual semantics involved is brought about in a different way. Discussing the various cases, I made remarks to such effect in the previous chapters.

\subsubsection{Some evidence for Hypothesis 1b: Hypothesis 2a vs. 3a}\label{sec:8.1.1.2}

Hypothesis 1b, the indication of abstract structure, consists of two subclaims: Hypothesis 2a and Hypothesis 3a. I proposed that adjectival inflections and \textit{ein} indicate differences in nominal structures with regard to size. For instance, the strong/weak alternation supports the claim that synactic arguments are not only DPs \REF{ex:8:11a} but also LPPs \REF{ex:8:11b}. Note in this regard that \textit{alle} ‘all’ is followed by an element with a weak inflection in \REF{ex:8:11a} but by an element with a strong ending in \REF{ex:8:11b}.

\ea%11
    \label{ex:8:11}
\ea\label{ex:8:11a} 
\gll[\textsubscript{DP} \textit{all-e} \textit{nett-en}   \textit{Studenten} ]\\
         {} all-\textsc{st} nice-\textsc{wk} students\\
\glt ‘all nice students’
\ex\label{ex:8:11b}
\gll [\textsubscript{LPP} {\textit{all-e}} [\textsubscript{DP} \textit{dies-e} \textit{nett-en} \textit{Studenten}]]\\
         {} all-\textsc{st}      {} these-\textsc{st} nice-\textsc{wk} students\\
\glt ‘all these nice students’
\z
\z


I argued in \chapref{sec:2} that Impoverishment only occurs in simple, canonical DPs \REF{ex:8:11a}, the domain in which the determiner undergoes movement. The strong ending on \textit{diese} ‘these’ in \REF{ex:8:11b} followed from the assumption that this nominal involves a non-canonical structure. Unlike in \REF{ex:8:11a}, \textit{alle} ‘all’ in \REF{ex:8:11b} is base-generated in LPP. Consequently, Impoverishment does not occur.

Turning to structural differences indicated by \textit{ein}, I sided with \citet{deSwartEtAl2007} in arguing that predicate nominals without \textit{ein} involve NPs \REF{ex:8:12a} and that those with \textit{ein} are bigger – under my assumptions (at least) ArtPs \REF{ex:8:12b}.

\ea%12
    \label{ex:8:12}
\ea \label{ex:8:12a}
\gll Er ist [\textsubscript{NP} \textit{Lehrer}].\\
    he is      {}  teacher.\textsc{masc}\\
\glt ‘He is a teacher.’
\ex \label{ex:8:12b}
\gll Er ist [\textsubscript{ArtP} \textit{ein} \textit{Mann}].\\
    he is  {} a man.\textsc{masc}\\
\glt ‘He is a man.’
\z
\z

I followed the above-mentioned authors in that \REF{ex:8:12a} involves a role noun with CAP in NP but that \REF{ex:8:12b} contains a kind noun with REL in NumP, the latter being flagged by \textit{ein} in ArtP. More generally, observe again that adjectival inflections indicate differences in structure in the higher layers of the noun phrase but that \textit{ein} shows distinctions in the lower projections. This is consistent with another fact.

  As just seen, role nouns can appear in predicative contexts without an article. Importantly, role nouns require a determiner when they appear in argument position.

\ea[]{%13
    \label{ex:8:13}
\gll *(Ein) Lehrer          braucht Unterstützung.\\
 {\db}a      teacher.\textsc{masc} needs    support\\
  \glt { }‘A teacher needs support.’
  }
\z

Considering \REF{ex:8:13}, it could be claimed that \textit{ein} brings about the semantic change from a predicate nominal \REF{ex:8:12a} to an argumental expression \REF{ex:8:13} and that \textit{ein} would then be tied to the semantics after all. However, there is an independent explanation for the occurrence of \textit{ein} in \REF{ex:8:13}.

\citet{Longobardi1994} proposed that argumental nominals project DPs (also \citealt{Borer2005}: 65-66, \citealt{Stowell1989}). For role nouns to occur in argumental expressions, more structure beyond NP must be projected. As discussed in \chapref{sec:6}, part of this structure is NumP which involves REL. The latter is flagged by \textit{ein}. We can point out that it is not \textit{ein} itself that turns a predicate nominal into an argumental expression – it is more structure. Having said that, \textit{ein} does not indicate the presence of the DP-level either: Many speakers allow \textit{ein} to appear with role nouns in predicative contexts, and kind nouns appear with \textit{ein} in such contexts more generally. As discussed above, both of the latter nominals only involve ArtP (not DP) and yet, they contain \textit{ein} (due to REL). This means that \textit{ein} does not necessarily indicate a DP. The reason why argumental noun phrases require a DP-level is that this phrasal layer involves a person feature \citep{Longobardi1994}. In other words, it is the structure involving the person feature (and not \textit{ein}) that brings about an argumental noun phrase.

  Finally, I restate the observation that adjectival inflections and \textit{ein} interact morpho-syntactically calling into question certain structural claims of other proposals. In \chapref{sec:4}, \sectref{sec:4.2}, I pointed out that plural \textit{ein} is possible in certain \textit{wh}-exclamatives \REF{ex:8:14a} and in constructions involving \textit{so} ‘such’ \REF{ex:8:14b}. Now, while not all speakers of German allow \textit{ein} to occur in these contexts, those who do have a weak ending on the adjective as shown in the following attested examples.

\ea%14
    \label{ex:8:14}
\ea[\%]{ \label{ex:8:14a}
\gll   NFU und NFSU2 (was  für n-e   dumm-en  Abkürzungen)\\
NFU and NFSU2  {\db}what for a-\textsc{pl} stupid-\textsc{wk} abbreviations\\
\glt ‘NFU and NFSU2 (what stupid abbreviations!)’
}

\ex[\%]{ \label{ex:8:14b}
\gll   So n-e  süß-en    klein-en   Pfoten\\
so a-\textsc{pl} cute-\textsc{wk} little-\textsc{wk} paws\\
\glt ‘Such cute little paws!’
}
\z
\z

I demonstrated in the preceding chapters that a simple, surface-oriented account of the strong/weak alternation cannot explain all the cases. Rather, determiners including \textit{ein} bring about a weak ending on the adjective in a specific structural constellation – simple DPs. In other words, \textit{ein} interacts with the inflection on the adjective morpho-syntactically.

If this is so, then the current account also raises some questions for other proposals. Specifically, the weak inflections on the adjectives above raise issues for a Predicate Inversion analysis of \REF{ex:8:14a} as discussed in \citet{BennisEtAl1998}, and they argue against the assumption of a null noun in \REF{ex:8:14b}, either after \textit{ein} or after the adjectives (cf.  \citealt{vanRiemsdijk2005}). Furthermore, the strong inflections on adjectives in the split-off of discontinuous noun phrases are only consistent with split topicalization being analyzed as involving two separately base-generated nominals (e.g., \citealt{Fanselow1988}). In the latter context, I also argued that \textit{ein} (i.e., its feature bundles) is unlikely to be inserted late in split topicalizations involving indefinite pronouns.

Returning briefly to \REF{ex:8:14}, it is interesting to observe that \textit{ein} does not only determine the inflection on the adjectives but that \textit{ein} also seems to have some semantic effect there. For instance,  \citet{vanRiemsdijk2005} points out for similar cases in Dutch that these types of constructions can express some relatively excessive property of the entities under discussion. These cases were not addressed in detail in \chapref{sec:4}. In \sectref{sec:8.2.2.3} below, I suggest again that \textit{ein} itself does not have an influence on the interpretation in these cases. Rather, an operator is held responsible for this effect, and this operator is flagged by \textit{ein}.

More generally, given these shared properties and the interwoven argumentation, I reiterate my proposal that adjectival inflections and \textit{ein} deserve to be discussed in tandem. Importantly, these elements also differ in certain ways.

\subsection{Differences between adjectival inflections and \textit{ein}}
\label{sec:8.1.2}
In the previous section, I reviewed the claim that adjectival inflections and \textit{ein} indicate abstract structure. I left open there that the former also provide clues about the different types of embeddings, namely clues that indicate simple vs. complex DPs. To illustrate this again, consider \REF{ex:8:15}. The example in \REF{ex:8:15a} involves a simple, canonical DP where the adjective is in Spec,AgrP. In contrast, the example in \REF{ex:8:15b} is a case of morphological dis-agreement. I argued in \chapref{sec:2}, \sectref{sec:2.3.5} that it involves a different structure where both the adjective and the noun are in Spec,DisP, and the head noun of the larger structure is a null element. This yields a non-canonical structure (see also the discussion of pronominal DPs involving \textit{Sie} ‘you’ and non-plural nouns like \textit{Schwein} ‘(pig =) idiot’ in \chapref{sec:7}, \sectref{sec:7.3.3.2}).

\ea%15
    \label{ex:8:15}
\ea\label{ex:8:15a}
\gll ihr [\textsubscript{AgrP} \textit{blöd-en} [\textsubscript{NP} \textit{Schweine}]]\\
    you     {}   stupid-\textsc{wk} {} pigs\\
\glt ‘you stupid idiots’
\ex\label{ex:8:15b}
\gll ihr [\textsubscript{DisP} [\textsubscript{AgrP} \textit{blöd-e} [\textsubscript{NP} \textit{Bande}]] [\textsubscript{NP} \textit{{e\textsubscript{N}}}]]\\
    you      {} {}         stupid-\textsc{st} {} gang.\textsc{fem}\\
\glt ‘you stupid gang’
\z
\z

Importantly, although the adjectives in both \REF{ex:8:15a} and \REF{ex:8:15b} are in Spec,AgrP, the one in \REF{ex:8:15b} is more deeply embedded – this AgrP is in Spec,DisP. Consequently, Impoverishment cannot occur in the latter case resulting in a strong ending on the adjective. In addition to complex specifiers, it was also shown in \chapref{sec:2} that strong adjectival endings surface in several other non-canonical structures, for instance, in different kinds of adjunction. In each case, Impoverishment does not occur, and I concluded that weak inflections only surface in simple, regular DPs. More generally, the strong/weak alternation on the adjective was employed throughout the book as a means to find plausible structures for related constructions (many containing \textit{ein}) and as briefly mentioned just above, to evaluate other proposals as regards their structural claims.

Both adjectival inflections and \textit{ein} also make contributions in their own right. While both types of elements indicate the presence of certain elements, they differ in that adjectival endings make morphological features visible but that \textit{ein} supports or flags semantic operators. I discuss the remaining claims of Hypotheses 2 and 3 in more detail.

\subsubsection{Some evidence for Hypothesis 2b}\label{sec:8.1.2.1}

I put forth the following hypothesis for adjectival inflections.

\ea%16
    \label{ex:8:16}
        Hypothesis 2\\
  Adjectival inflections:
\begin{xlist}
%\ea \label{ex:8:16a}
\exi{b.} \label{ex:8:16b} […] make nominal features like case, number, and gender visible.
\z
\z

To briefly illustrate, this claim receives credence from the comparison of a regular noun phrase as in \REF{ex:8:17a} and its discontinuous counterpart as in \REF{ex:8:17b}.

\ea%17
    \label{ex:8:17}
\ea  \label{ex:8:17a}
\gll ein lila(-n-es)   Kleid\\
a    purple-n-\textsc{st} dress.\textsc{neut}\\
\glt ‘a purple dress’
\ex  \label{ex:8:17b}
\gll Kleid          habe ich ein lila*(-n-es).\\
dress.\textsc{neut} have  I     a    purple-n-\textsc{st}\\
\glt ‘As for dresses, I have a purple one.’
\z
\z

I proposed that the adjectival ending in \REF{ex:8:17b} makes the features of the displaced noun visible; that is, the adjectival inflection is an exponent of case, number, and gender (specifically for \REF{ex:8:17b}: nominative, singular, neuter). In fact, due to concord inside the noun phrase, adjectival inflections make features of the entire DP visible. The same holds for the inflection on \textit{ein}.

\ea%18
    \label{ex:8:18}
\ea\label{ex:8:18a}
\gll ein Kleid\\
a    dress.\textsc{neut}\\
\glt ‘a dress’
\ex\label{ex:8:18b}
\gll \textit{Kleid}          habe ich nur   \textit{ein*(-es)}.\\
dress.\textsc{neut} have  I     only one-\textsc{st}\\
\glt ‘As for dresses, I have only one.’
\z
\z

Comparing \REF{ex:8:17b} and \REF{ex:8:18b}, note again that adjectival inflections and \textit{ein} interact in that \textit{ein} determines the ending on the following adjective \REF{ex:8:17b} but that this very ending may also appear on \textit{ein} itself if the adjective (and noun) is absent \REF{ex:8:18b}. Recall from \chapref{sec:1}, \sectref{sec:1.3.2.2} that the additional adjectival ending in \REF{ex:8:18b} is only possible on \textit{ein} under very specific conditions (which makes \textit{ein} different from other determiners). This was made formal in \chapref{sec:2}, \sectref{sec:2.2.2} postulating certain vocabulary insertion rules for \textit{ein}.

\subsubsection{Some evidence for Hypothesis 3b}\label{sec:8.1.2.2}

The element \textit{ein} also makes certain elements visible. Unlike adjectival inflections, \textit{ein} indicates the presence of semantic operators.\footnote{It may turn out that adjectival inflections are similar to \textit{ein} here as well. Specifically, adjectival inflections might also be able to flag the presence of an operator in certain contexts. In this regard, \citet[183]{Merchant1996} points out that uninflected \textit{all} ‘all’ tends to induce a collective reading \REF{ex:8:1:ia} but that its inflected counterpart seems to yield a distributive reading \REF{ex:8:1:ib}.
	\ea
	\ea \label{ex:8:1:ia}
	\gll all diese guten Freunde\\
	all these good  friends\\
	\glt ‘all these good friends’
	\ex \label{ex:8:1:ib}
	\gll all-e diese guten Freunde\\
	all-\textsc{infl} these  good  friends\\
	\glt ‘all these good friends’
	\z
	\z
	
	Taking this at face value, the presence or absence of the inflection seems to correlate with different readings. Rather than suggesting that adjectival inflections are associated with the semantics after all (e.g., \citealt{Roehrs2015}: 264), we might tentatively suggest that they make a distributivity operator in \REF{ex:8:1:ib} visible.} I formulated the following general claim.

\ea%19
    \label{ex:8:19}

      Hypothesis 3\\\textit{Ein}:
      \begin{xlist}
\exi{b.} \label{ex:8:19b} […] supports overt semantic operators (e.g., NEG \textit{k}-) and flags the presence of       covert semantic operators (e.g., REL).
\z
\z

I argued above that the possessive articles, the negative article, and the singularity numeral are composite forms. These complex elements involve possessor, negator, or singularity components that, if supported by \textit{ein}, receive the following spell-out forms in a noun phrase.

\ea%20
    \label{ex:8:20}
\ea\label{ex:8:20a}
\gll m-eine Frau\\
\textsc{poss}-a woman.\textsc{fem}\\
\glt ‘my woman’
\ex\label{ex:8:20b}
\gll k-eine Frau\\
\textsc{neg}-a woman.\textsc{fem}\\
\glt ‘no woman’
\ex\label{ex:8:20c}
\gll EINE    Frau\\
$\emptyset$\textsubscript{[--}\textsc{\textsubscript{pl}}\textsubscript{]}+a woman.\textsc{fem}\\
\glt ‘one woman’
\z
\z


Note again that supporting involves overt operators. Overtness of the operator is a convenient shorthand for stating that the relevant semantic components have a detectable manifestation: Possessors and negation have an overt segmentable element (e.g., \textit{m}-, \textit{k}-), but the singularity numeral involves a non-segmentable element, namely stress. I suggested above that these three components are operators. This not only allows us to relate the three instances to each other but also to other cases.

In particular, basically following \citet{deSwartEtAl2007}, I proposed that \textit{ein} indicates the presence of the realization operator REL, an element that does not have an independent overt manifestation \REF{ex:8:21}. With \textit{ein} appearing in this context, this is what I referred to as flagging. Notice also that the flagging of an operator and the indication of a certain amount of structure are related here. Assuming with \citet{deSwartEtAl2007} that REL is in NumP, \textit{ein} indicates the presence of REL and a certain abstract structure in singular contexts (i.e., ArtP).

\ea%21
    \label{ex:8:21}
    \gll [\textsubscript{ArtP} \textit{eine} [\textsubscript{NumP} REL\textsubscript{\textlangle e\textlangle e,t\textrangle\textrangle} [\textsubscript{NP} \textit{Frau}\textsubscript{\textlangle e\textrangle}]]]\textsubscript{\textlangle e,t\textrangle}\\
   {}   a                  {} {} {}                   woman\\
  \glt ‘a woman’
\z

Finally, it is worth pointing out that the cases in \REF{ex:8:20} above also involve REL – the head nouns (type \textlangle e\textrangle ) must be mapped to sets of individuals (type \textlangle e,t\textrangle ). In other words, \textit{ein} does not only support an overt operator there, but it also flags a covert one at the same time. In \sectref{sec:8.2.2.2} and \ref{sec:8.2.2.3} below, I list more cases involving supporting and flagging and discuss some of the related issues in more detail.

\subsection{Summary of the main claims}\label{sec:8.1.3}

In this last subsection, I provide a convenient summary of the main insights in bullet-point format.

Adjectival inflections are not a reflex of:

\begin{itemize}
\item (in-)definiteness
\item (non-)restrictiveness of interpretation of modifiers
\item referentiality
\end{itemize}

The article \textit{ein} is not a reflex of:

\begin{itemize}
\item indefiniteness
\item emotiveness
\item singular number/countability
\end{itemize}

Both adjectival inflections and \textit{ein} are semantically vacuous and can:

\begin{itemize}
\item appear in contradictory semantic contexts
\item be left out in certain well-defined contexts without loss of meaning
\item have multiple occurrences without leading to semantic redundancy
\end{itemize}

Furthermore, both adjectival inflections and \textit{ein} can each be in different positions:

\begin{itemize}
\item strong endings are in AgrP, CardP, DP, and LPP; weak endings are in AgrP, CardP, and DP
\item \textit{ein} is in Art, Card, and D
\end{itemize}

Abstracting away from adjectival inflections indicating different degrees of embedding, these two types of elements indicate abstract structure:

\begin{itemize}
\item adjectival inflections:  DP vs. LPP
\item \textit{ein}:       NP vs. ArtP
\end{itemize}

These two types of items make other elements visible:

\begin{itemize}
\item adjectival inflections:  morphological features

\item \textit{ein}:       semantic operators
\end{itemize}

Miscellaneous for adjectival inflections:

\begin{itemize}
\item concord is a necessary (but not sufficient) condition for weak endings
\item Impoverishment occurs locally and in a bottom-up fashion
\end{itemize}

Miscellaneous for \textit{ein}:

\begin{itemize}
\item it is the least specified determiner\footnote{Besides having no semantics, there are some other advantages of assuming that \textit{ein} has no feature for definiteness. On the one hand, \textit{ein} does not have to move to the DP-level allowing us to claim that predicates can be of a smaller structural size; on the other hand, if \textit{ein} can stay low in the structure, we can propose that \textit{ein} always surfaces on the right side of the operator it supports (e.g., with \textit{EIN} ‘one’, see also \sectref{sec:8.2.2.5}). This, in turn, allows us to use the same mechanism (Local Dislocation) to instantiate this support.}
\item it is not inserted late\footnote{Recall that with Vocabulary Items inserted late in DM, this statement actually has to do with the feature bundles that make up \textit{ein(-er)} where the stem involves [+D] and its adjectival inflection contains features for case, number, and gender. In other words, both the stem and the inflection are treated in the same way – no features are inserted late, only the related Vocabulary Items.}
\item it is triggered by REL (but not by CAP or \textit{als} ‘as’)
\end{itemize}

Overall, I conclude that although semantically vacuous, these two types of elements provide clues about abstract structure and the presence of covert elements (features and operators) in the German noun phrase.

\section{Some extensions and further consequences} \label{sec:8.2}

In this section, I consider some possible extensions and more consequences.\linebreak Given the current state of the investigation, I only briefly discuss the relevant issues and tentatively hint at some solutions. A full-blown account of all the relevant aspects must await another occasion. I start with adjectival inflections and then move on to \textit{ein}.

\subsection{Extensions and consequences of adjectival inflections: German dialects}\label{sec:8.2.1}

In \chapref{sec:3}, \sectref{sec:3.7}, I provided my account of adjectival inflections in one dialectal variety – Mannheim German. I have not investigated and analyzed other regional forms of German in this regard. In what follows, I make some brief remarks about adjectival inflections in one well-studied dialect, Alemannic German, showing how this type of dialect could be accommodated in the current discussion.

\citet{Rehn2019} provides an interesting overview and an in-depth analysis of prenominal adjectives in Alemannic (for different subdialects of Alemannic, see \citealt{Baechler2017}).\footnote{\citet{Baechler2017} provides convenient paradigms of adjectival inflections (her pages 305-13) and definite articles and demonstratives (her pages 339-44) in a number of different Alemannic subdialects (note though that unlike the current analysis, she does not consider definite articles and demonstratives as bipartite forms).} She observes that they can appear without an inflection in both indefinite and definite contexts. \citet[114]{Rehn2019} points out that this cannot be a continuation of ENHG as uninflected adjectives in Alemannic occur in more contexts than in ENHG. Rather than involving a null ending, she argues that these adjectives are truly uninflected. \citet[103-04]{Rehn2019} argues that uninflected adjectives and nouns could be analyzed as compound-like elements in some instances but crucially not in all cases and dialects. As such, the distribution of the inflections on adjectives (and other elements in the noun phrase) is proposed to follow from certain features that Alemannic (and German more generally) must mark overtly. In what follows, I briefly summarize some of the main aspects of Rehn’s work (but I am not able here to do full justice to all the intricacies of her multi-faceted proposal).

\citet{Rehn2019} observes that inflections on adjectives are basically optional. This is shown with examples in the nominative/accusative in \REF[a-b]{ex:8:22}. There is no difference in the interpretation that corelates with the presence or absence of the inflections, and there are no morpho-syntactic restrictions. There is one exception. In the absence of a determiner, the inflection on the adjective is obligatory. I illustrate this with an example in the nominative in \REF{ex:8:22c} (examples are from \citealt{Rehn2019}: 133, 121).

\ea%22
Alemannic German
    \label{ex:8:22}
\ea \label{ex:8:22a}
\gll des lang(-e)  Soil  \\
    the long-\textsc{wk} rope.\textsc{neut}\\
\glt ‘the long rope’
\ex \label{ex:8:22b}
\gll a guat(-s) Buach\\
a good-\textsc{st} book.\textsc{neut}\\
\glt ‘a good book’
\ex \label{ex:8:22c}
\gll guad*(-er) Wei\\
good-\textsc{st}     wine.\textsc{masc}\\
\glt ‘good wine’
\z
\z

In order to explain the distribution of the adjectival inflections in \REF{ex:8:22}, \citet[122-26]{Rehn2019} proposes that German DPs must mark features for number and oblique case overtly.

  Considering the distribution of the indefinite and definite articles, Rehn observes that certain features are consistently marked overtly (as different). To exemplify with Standard German, number is marked differently with the indefinite article in the singular (\textit{ein}) and plural ($\emptyset$\textsubscript{D}). Similarly, the definite article involves \textit{der}, \textit{das}, \textit{die} in the singular and \textit{die} in the plural (NB: Feminine singular and plural forms are not entirely straightforward as the nominative/accusative forms are all \textit{die}, and the dative forms are both \textit{der}). Turning to case, Rehn points out that there is syncretism in the nominative and accusative in both the singular and plural. In fact, only the oblique cases are consistently different, for instance, nominative/accusative neuter \textit{das} vs. dative neuter \textit{dem} and genitive neuter \textit{des} (NB: Feminine singular has the form \textit{der} in both the dative and genitive). As for gender, this category is not consistently different. Masculine and neuter are the same with \textit{ein} in the nominative, and masculine and neuter are the same more generally with \textit{einem/dem} in the dative and \textit{eines/des} in the genitive. Rehn draws the conclusion that only number and oblique case must be marked overtly in German DPs.\footnote{Given the \textit{nota bene} remarks in the main text, the difference between number and oblique case vs. non-oblique case and gender is not entirely clearcut. I put this potential issue aside here.}

  \citet[125]{Rehn2019} formulates the following principle.

\ea%23
    \label{ex:8:23}
Feature Specification in German DPs:\\
Within a DP in German, number and oblique case must be marked overtly either\\
\ea on an article,\\
\ex through inflection realized in Mod\textsuperscript{0} [here Agr, DR] when no article is realized,\\
\ex or on the noun itself when neither of the above is realized.
\z
\z

This principle is meant to hold for both Standard German and the regional dialects alike. Note that the obligatory inflections in Standard German are, in \citeauthor{Rehn2019}’s (\citeyear{Rehn2019}: 135-39) view, due to a standardization process in the 18\textsuperscript{th} and early 19\textsuperscript{th} centuries. For Rehn, inflections on adjectives are generally optional in German, and the obligatory inflections in Standard German are “a mere\linebreak PF-phenomenon”.\footnote{\citet[133]{Rehn2019} observes that inflections on adjectives are also optional in other dialectal varieties of German. Note though that this does not hold for all dialects: Berlin German has obligatory inflections.}

  Returning to Alemannic, the principle in \REF{ex:8:23} accounts for the data in \REF{ex:8:22}. In the non-oblique cases provided in \REF{ex:8:22}, only number must be marked overtly. This marking is supplied by the articles in \REF[a-b]{ex:8:22}, and as a consequence, the inflections on the adjectives are optional. As for the obligatory inflection on the adjective in \REF{ex:8:22c}, there is no article, and the inflection on the adjective itself is the only way to mark number \citep[124]{Rehn2019}.

While the principle in \REF{ex:8:23} may account for the data in \REF{ex:8:22}, it raises questions about DPs involving bare mass nouns. In the non-oblique cases as in \REF{ex:8:24a}, number must be overtly marked; in the oblique cases as in \REF{ex:8:24b}, both number and the oblique case must be overtly marked. As far as I am aware, there is no overt marking in \REF[a-b]{ex:8:24}, and yet, the DPs are grammatical.

\ea%24
Alemannic German
    \label{ex:8:24}
\ea \label{ex:8:24a}
\gll Wei       \\
    wine.\textsc{masc}\\
\glt ‘wine’
\ex \label{ex:8:24b}
\gll mit   Wei\\
with wine.\textsc{masc}\\
\glt ‘with wine’
\z
\z

In my view, the grammaticality of these cases is telling. To explain their status, we have to assume that features for case, number, and gender are only spelled out by adjectival inflections (but not endings on nouns). An alternative to Rehn’s proposal could be to extend the current analysis to Alemannic German.

In the current system, null articles and nouns have no feature bundles for case, number, and gender – only overt determiners and adjectives do (I follow \citealt{Rehn2019}: 155 in that case inflections on nouns are under Cl, in the current system Num, signaling case and number at the same time). I assume the following feature realization rule for German DPs, for both the standard and regional varieties.

\ea%25
    \label{ex:8:25}
Feature Specification of German DPs:\\
If CNG bundles are present, at least one of them must be made overt.
\z

Note that this is a stronger claim than Hypothesis 2b (i.e., adjectival inflections make nominal features like case, number, and gender visible) in that this conditional is meant to help explain the distribution of adjectival inflections. Consider how Standard and Alemannic German can be accounted for by \REF{ex:8:25}.

  As discussed above, overt determiners have CNG features as part of their structures. Furthermore, (regular) adjectives project InflP as part of their extended projection. As such, the rule in \REF{ex:8:25} is adhered to in all cases involving overt determiners and/or adjectives in the counterparts of \REF{ex:8:22} in Standard German – CNG features are present and made overt (I comment on the obligatory presence of the inflection below). Bare mass nominals as in \REF{ex:8:24} are predicted to be fine too as they do not involve CNG bundles in the first place. As for Alemannic, recall that there is no difference in the interpretation that corelates with the presence or absence of the inflections on adjectives (for such a case in Dutch, see \citealt{Evans2021}). As such, I agree with \citet[194]{Rehn2019} that there is no difference in the syntactic position of inflected vs. uninflected adjectives. One way to account for the obligatory presence of inflections on adjectives in Standard German and the optional absence of inflections on adjectives in Alemannic German is to assume that like in Standard German, InflP is projected with (regular) adjectives in Alemannic German. Unlike in Standard German, the inflections on the adjectives are optionally deleted in Alemannic German unless the deletion violates the rule in \REF{ex:8:25} – the latter scenario involves the presence of only one CNG feature bundle as in \REF{ex:8:22c}.

  Finally, I make some remarks on the indefinite article. Note that DPs like \textit{ein Buch} or \textit{a Buach} ‘a book’ are grammatical in both Standard and Alemannic German. Recall that I claimed in \chapref{sec:2}, \sectref{sec:2.2.2.2} that the indefinite article spells out CNG features. This includes the non-oblique instances where this element has no overt inflections in three feature combinations. Given the vocabulary insertion rules provided in that section, the uninflected indefinite article \textit{ein} spells out both [+D] and [CNG] at the same time. If this is so, then this means that the non-oblique cases are not special. Rather, we can maintain that all features of the CNG bundles must be spelled out. In other words, non-oblique case features are on a par with oblique case features making the account of the inflectional distribution more general. If the sketch above turns out to be a viable alternative account of the Alemannic data, it supports the current analysis of the indefinite article.

To sum up, while I have not examined adjectival inflections in dialectal varieties of German in much detail here, I believe the current system can be extended to those cases. Above, this was exemplified by a brief discussion of Alemannic German. Admittedly, the details of this extension still need to be worked out more carefully. This goes beyond the scope of this book. Besides certain theoretical questions that arise, there are also empirical points that need to be clarified. For instance, it is important to determine the distribution of inflections on multiple adjectives in cases where determiners are absent. Such data will help narrow down the analytical options. Next, I consider \textit{ein} in more detail.

\subsection{Extensions and consequences of the indefinite article: Other contexts where \textit{ein} appears}\label{sec:8.2.2}

In this section, I turn to some extensions and consequences of the discussion of \textit{ein}. At the end, I provide the updated vocabulary insertion rules of the main types of articles. I begin by reviewing certain parts of the previous discussion.

\subsubsection{Reviewing cases of supporting and flagging}\label{sec:8.2.2.1}

As argued in \chapref{sec:5}, \textit{ein} can support POSS of the possessive articles, NEG of the negative article, and $\emptyset$\textsubscript{[--PL]} of the singularity numeral.

\ea%26
    \label{ex:8:26}
\ea\label{ex:8:26a}
\gll m-eine Frau\\
\textsc{poss}-a woman.\textsc{fem}\\
\glt ‘my woman’
\ex\label{ex:8:26b}
\gll k-eine Frau\\
\textsc{neg}-a woman.\textsc{fem}\\
\glt ‘no woman’
\ex\label{ex:8:26c}
\gll EINE    Frau\\
$\emptyset$\textsubscript{[--}\textsc{\textsubscript{pl}}\textsubscript{]}+a woman.\textsc{fem}\\
\glt ‘one woman’
\z
\z

In \chapref{sec:6}, I followed work by \citet{deSwartEtAl2007} such that kind nouns like \textit{Frau} ‘woman’ combine with the realization operator REL returning predicate nominals. Note though that combinations of REL(\textit{Frau}) can occur not only as part of predicate nominals \REF{ex:8:27a} but also as part of argumental expressions \REF{ex:8:27b}. In either case, REL is flagged by \textit{ein}.

\ea%27
    \label{ex:8:27}
\ea\label{ex:8:27a}
\gll Die Person da     hinten ist eine Frau.\\
the  person there back    is  a      woman.\textsc{fem}\\
\glt ‘The person back there is a woman.’
\ex\label{ex:8:27b}
\gll Eine Frau            kam   durch    die Tür.\\
a      woman.\textsc{fem} came through the door\\
\glt ‘A woman came through the door.’
\z
\z

Indeed, as briefly pointed out in \sectref{sec:8.1.2}, one instance of \textit{ein} can make two operators visible at the same time. To see this again, consider the example in \REF{ex:8:28a}. Having added an adjective, it is clear that the kind noun \textit{Frau} ‘woman’ must combine with REL in NumP to yield a predicate. With both elements of type \textlangle e,t\textrangle, Predicate Modification can combine the predicate nominal and its adjectival modifier. REL triggers the presence of \textit{ein}. Furthermore, the singularity operator must be supported by \textit{ein}. Thus, \REF{ex:8:28a} has the more abstract representation in \REF{ex:8:28b}.

\ea%28
    \label{ex:8:28}
\ea\label{ex:8:28a}
\gll EINE blonde Frau\\
one  blonde woman.\textsc{fem}\\
\glt ‘one blonde woman’
\ex\label{ex:8:28b}
$\emptyset$\textsubscript{[--}\textsc{\textsubscript{pl}}\textsubscript{]} Adjective REL Noun
\z
\z

To be clear, \textit{ein} flags the presence of REL and supports the singularity operator at the same time. The same applies to \textit{mein} ‘my’ and \textit{kein} ‘no’ discussed above. This means that cases of supporting also involve flagging. In fact, cases of supporting seem to entail instances of flagging in that all overt operators occur with REL.\footnote{Currently, I am not aware of cases where \textit{ein} only supports an operator but does not flag a covert operator at the same time. In other words, all nouns in strings involving “… \textit{ein} … Noun” seem to involve REL (or another covert operator, see below).}

\ea%29
    \label{ex:8:29}

          Supporting and flagging:

\TabPositions{2.5cm}
\ea \label{ex:8:29a} (\textit{m-eine})         \tab REL
\ex \label{ex:8:29b} (\textit{k-eine})         \tab  REL
\ex \label{ex:8:29c} ($\emptyset$\textsubscript{[--}\textsc{\textsubscript{pl}}\textsubscript{]}\textit{+eine})\tab  REL
\z
\z


Before moving on, consider some other points about REL.

  I proposed in \chapref{sec:7} that other determiners can also flag REL. Besides regular definite determiners like \textit{der} ‘the’ as in \citet{deSwartEtAl2007}, I proposed that pronominal determiners like \textit{du} ‘you’ can do so as well (e.g., \textit{der Wagen} ‘the car’; \textit{du Schwein} ‘you (pig =) idiot’). This means that \textit{ein} surfaces in indefinite contexts but \textit{der} or \textit{du} in definite environments. Given that only one determiner can be merged in ArtP, the distinction followed from the different vocabulary entries where \textit{ein} is not marked for (in-)definiteness but \textit{der} has a definiteness feature and \textit{du} has features for author and participant (entailing definiteness). More generally, it seems clear that different elements can flag the same operator and that the presence of REL is not a sufficient condition for the occurrence of \textit{ein} (we see in \sectref{sec:8.2.2.3} below that the presence of REL is not a necessary condition for the occurrence of \textit{ein} either). With this brief review in place, consider more complex cases.

\subsubsection{Nominals with two overt operators}\label{sec:8.2.2.2}


As just discussed, there are two ways for \textit{ein} to make operators visible: supporting and flagging. Interestingly, the cases discussed above can be more complex – they can involve two overt operators in one nominal, both supported by \textit{ein}. To set the stage, recall that we saw above that \textit{so} ‘such’ can precede \textit{ein} where both unreduced and reduced forms of \textit{ein} are possible (for \REF{ex:8:30b}, cf. \citealt{ElmentalerRosenberg2015}: 389).\footnote{\citet{HoleKlumpp2000} propose
  that \textit{son}, that is, \textit{so’n} in \REF{ex:8:30}, is an article that simultaneously expresses definite type reference and indefinite token reference. One of their main arguments is that since \textit{son} can appear in plural contexts but the indefinite article \textit{ein} cannot, \textit{son} must be an atomic element. They wind up suggesting that German has three articles: the definite article, the indefinite article, and \textit{son}. Note, however, that German allows the following strings.

  \ea \label{ex:8:8:i}
    \ea
    \gll so jemand\\
    so someone.\textsc{masc}\\
    \glt ‘someone like that’
    \ex
    \gll so (et)was   \\
    so {\db}something.\textsc{neut} \\
    \glt ‘something like that’
    \z
  \z

  There are different kinds of \textit{so} \citep[20-21]{Heusinger2011}. Taking into consideration the different semantic contributions of \textit{jemand} ‘someone’ and \textit{(et)was} ‘something’ (and abstracting away from the independent non-reducibility of \textit{jemand} and \textit{(et)was}), the element \textit{so} in \REF{ex:8:8:i} appears to be, as far as I can tell, the same as in \REF{ex:8:30} (admittedly, this intuition has to be investigated in more detail). If so, this parallel of \REF{ex:8:8:i} and \REF{ex:8:30} makes atomicity of \textit{so’n} less likely. More importantly, \textit{ein} can also surface as an unreduced form after \textit{so} (also in the plural, \chapref{sec:1}, \sectref{sec:1.2.2}), and \textit{so’n} can follow another article and/or a numeral in the same noun phrase (as shown in \REF{ex:8:31} below). Thus, the occurrenc of two instances of \textit{ein}, one after \textit{k}- and one after \textit{so}, and the low position of \textit{so’n} itself indicate that \textit{son} is not an atomic article but a bipartite form (as indicated in the alternative spelling). Finally, as shown in \chapref{sec:1}, \sectref{sec:1.2.2}, plural \textit{ein} can also appear in \textit{was-für} contexts, both in reduced and unreduced form. As far as I know, there are no claims that \textit{(was) für} and \textit{ein} form an atomic article. For cross-linguistic discussion of \textit{so} and the indefinite article, see \citealt{WoodVikner2011,WoodVikner2013}.
}

\ea%30
    \label{ex:8:30}
\ea\label{ex:8:30a}
\gll so (ei)ne (nette) Frau\\
so  {\db}a         {\db}nice   woman.\textsc{fem}\\
\glt ‘such a nice woman’
\ex\label{ex:8:30b}
\gll irgend so (ei)ne Frau\\
any     so   {\db}a       woman.\textsc{fem}\\
\glt ‘some woman or other’
\z
\z

The element \textit{so} can roughly be rendered as ‘such’ or, perhaps more telling, as ‘of that type’. If we interpret this element as an overt type operator, then a number of other interesting points can be made.

Putting the possessive articles aside for a moment (but see below), we have seen that \textit{ein} can support NEG and $\emptyset$\textsubscript{[--PL]}. I assume that this also holds for \textit{so} ‘such’. Considering \REF{ex:8:31a} and \REF{ex:8:31b}, note that there are two operators in each case and that \textit{ein} follows both in each case yielding two instances of the indefinite article. Notice that the example in \REF{ex:8:31b} is parallel to an example from the Appendix, \REF{ex:8:31c}, where a numeral other than \textit{EIN} ‘one’ precedes \textit{so} suggesting that \textit{EIN} is the regular singularity numeral in \REF{ex:8:31b}. Indeed, \textit{so’n} can follow another article and a numeral as in the constructed example in \REF{ex:8:31d} (the example in \REF{ex:8:31b} is adapted from \citealt{HoleKlumpp2000}: 238, also \citealt{ElmentalerRosenberg2015}: 383, 389; for Dutch, see \citealt{Barbiers2008}: 6; note that the numeral \textit{zwei} ‘two’ should be stressed in \REF{ex:8:31d}).

\ea%31
    \label{ex:8:31}
\ea \label{ex:8:31a}
\gll k-eine so’ne Frau\\
\textsc{neg}-a {so a}   woman.\textsc{fem}\\
\glt ‘no such woman’
\ex \label{ex:8:31b}
\gll EINE    so’ne Frau\\
$\emptyset$\textsubscript{[--}\textsc{\textsubscript{pl}}\textsubscript{]}+a {so a}   woman.\textsc{fem}\\
\glt ‘one such woman’
\ex[\%]{ \label{ex:8:31c}
\gll drei  {so ne} geilen  Mädels\\
three {so a}   awesome girls\\
\glt ‘three such awesome girls’
}
\ex[\%]{ \label{ex:8:31d}
\gll Es gibt   keine zwei so’ne Frauen auf der Welt.\\
it   gives \textsc{neg}-a two  {so a}    women in   the world\\
\glt ‘There are not two such women in the world.’
}
\z
\z

The current account is able to throw some light on these multiple occurrences of \textit{ein}. Before I provide the derivations, I identify some restrictions.

First, unlike \REF{ex:8:30}, \REF{ex:8:31} cannot involve an unreduced form of \textit{ein} as a second instance. This is particularly clear with the singularity numeral \REF{ex:8:32a}. Furthermore, the first instance of \textit{ein} cannot be a reduced form either \REF{ex:8:32b} (\REF[a]{ex:8:32} is based on \citealt{ElmentalerRosenberg2015}: 383).

\ea%32
    \label{ex:8:32}
\ea[*]{  \label{ex:8:32a}
\gll EINE so eine Frau\\
one    so a     woman.\textsc{fem}\\
}
\ex[?*]{  \label{ex:8:32b}
\gll ’ne so eine Frau\\
 {\db}a   so a      woman.\textsc{fem}\\
 }
\z
\z

Second, \textit{so} ‘such’ cannot be left out.

\ea%33
    \label{ex:8:33}
\ea[*]{ \label{ex:8:33a}
\gll keine ’ne Frau\\
     no    {\db}a        woman.\textsc{fem}\\
}
\ex[*]{ \label{ex:8:33b}
\gll EINE ’ne Frau\\
one       {\db}a      woman.\textsc{fem}\\
}
\z
\z

Third, the second instance of \textit{ein} cannot be left out.

\ea%34
    \label{ex:8:34}
\ea[*]{\label{ex:8:34a}
\gll keine so Frau\\
no     so woman.\textsc{fem}\\
}
\ex[*]{\label{ex:8:34b}
\gll EINE so Frau\\
one    so woman.\textsc{fem}\\
}
\z
\z

Fourth, \textit{so’n} is not possible in definite contexts such as possessives \REF[a-b]{ex:8:35} or definite determiners \REF[c-d]{ex:8:35}.

\ea%35
    \label{ex:8:35}
\ea[*]{\label{ex:8:35a}
\gll meine so’ne Frau\\
my     {so a}   woman.\textsc{fem}\\
}
\ex[*]{\label{ex:8:35b}
\gll Peters  so’ne Frau\\
Peter’s {so a}   woman.\textsc{fem}\\
}
\ex[*]{\label{ex:8:35c}
\gll die so’ne Frau\\
the {so a}   woman.\textsc{fem}\\
}
\ex[*]{\label{ex:8:35d}
\gll diese so’ne Frau\\
this   {so a}    woman.\textsc{fem}\\
}
\z
\z

Summarizing these sets of data, note that these multiple occurrences of \textit{ein} only surface in indefinite contexts where one instance of \textit{ein} occurs before \textit{so} and one after it. Furthermore, when both instances of \textit{ein} are present, \textit{so} cannot be left out; that is, both instances of \textit{ein} cannot be adjacent. When the first instance of \textit{ein} and \textit{so} are present, the second instance of \textit{ein} must be present as well. Finally, the first instance of \textit{ein} involves an unreduced form, but the second instance of \textit{ein} is a reduced form. To repeat, this is different when the first instance of \textit{ein} is missing as in \REF{ex:8:30}. In the latter case, \textit{ein} can appear in both reduced or unreduced forms.

  I argued in \chapref{sec:1}, \sectref{sec:1.4} that determiners are base-generated in ArtP and move successive cyclically up the tree. I propose that the multiple occurrences of \textit{ein} above are also related by movement such that two copies of \textit{ein} are pronounced. For concreteness, I assume that the operator \textit{so} 'such' is in the specifier position of a Type Phrase (TypeP). I repeat \REF{ex:8:31b} below and derive it as in \figref{figex:8:36}. I suggest that $\emptyset$\textsubscript{[--PL]} and \textit{so} each get supported by \textit{ein}. The relevant elements are linearly adjacent.\footnote{For the counterpart in Dutch, \citet[171]{Barbiers2005} suggests that NumP (CardP in the current system) is base-generated above DP. Such a strong (non-standard) claim does not have to be made in the current account.}
  
\glltree[\label{figex:8:36}]{
	  \gll EINE    so’ne Frau\\
	  $\emptyset$\textsubscript{[--}\textsc{\textsubscript{pl}}\textsubscript{]}+a {so a}   woman.\textsc{fem}\\
	  \glt {`one such woman'}
}{
	[CardP
		[$\emptyset$\textsubscript{[--\textsc{pl}]}]
		[Card$'$
			[\textit{eine}\textsubscript{$i$}]
			[TypeP
				[\textit{so}]
				[Type$'$
					[\textit{’ne}\textsubscript{$i$}]
					[ArtP
						[\sout{\textit{eine}}\textsubscript{$i$}]
						[NumP\\\textit{Frau}]
					]
				]
			]
		]
	]
}

Some remarks are in order here.

I argued in \chapref{sec:2}, \sectref{sec:2.2.1.6} that \textit{ein} involves a complex head consisting of the categorial feature [+D] and a separate feature bundle for case, number, and gender. Notice that this complex head undergoing movement immediately explains that the second instance of \textit{ein} is inflected and that it is inflected in the same way as the first instance. More tentatively, I assume that the operator \textit{so} is a deficient element (i.e., a bound stem) in that it must be supported by \textit{ein}. This accounts for the fact that \textit{ein} must follow \textit{so} – just like with the other overt operators. There are two kinds of derivation. Starting with \REF{ex:8:30}, where only one overt operator is present, either an unreduced or a reduced form of \textit{ein} follows \textit{so}. I assume that in the reduced instances, \textit{ein} has encliticized to the operator.

Second, unlike \REF{ex:8:30}, \REF[a-b]{ex:8:31} involve two instances of \textit{ein}, and the \textit{ein} following \textit{so} is a reduced form. \citet{Nunes2001} argues that Move is not a primitive operation but rather the output of Copy, Merge, Form Chain, and Copy Reduction (\chapref{sec:1}). Assuming that Copy Reduction applies to free, unbound copies, unreduced forms of lower copies are deleted. Extending this discussion, if \textit{ein} is an unreduced copy, it gets deleted by Copy Reduction. This explains the ungrammaticality of \REF[a-b]{ex:8:32} as the second instance of \textit{ein} should have been deleted. Furthermore, it explains the ungrammaticality of \REF{ex:8:34} as \textit{ein} has been deleted but \textit{so} remains unsupported here. The examples in \REF{ex:8:33} are presumably out as \textit{ein} cannot encliticize to a copy of itself. Note though that \textit{ein} can encliticize to \textit{so}. This reduced form of \textit{ein} is a bound copy and becomes invisible to Copy Reduction (cf. \citealt{Nunes2001}: 311-12). This derivation allows a second \textit{ein} to surface, and this lower copy can support \textit{so}.\footnote{Discussing similar cases, \citealt{WoodVikner2011,WoodVikner2013} also locate lower instances of \textit{ein} in a lower positon of the extended projection of the noun.}

Note that I have not discussed cases where both instances of \textit{ein} are reduced. While perhaps not fully grammatical, they seem to be possible.

\judgewidth{?}
\ea[?]

Similar to the discussion above, these cases can be accounted for by movement and enclitization of the second instance of \textit{ein} to \textit{so}. Additionally, we can suggest that the first instance of \textit{ein} is reduced to \textit{’n}, just like other instances of the (reduced) article.

Finally, consider the data in \REF{ex:8:35} again. I assume that \textit{mein} ‘my’ is not possible in the construction in \REF{ex:8:35a} as this type of \textit{so} is incompatible with definiteness (more on this in \sectref{sec:8.2.2.5}). This also rules out the cases of the other definite determiner(-like) elements in \REF[b-d]{ex:8:35} in addition to the fact that they do not involve (a higher instance of) \textit{ein} to begin with; that is, the cases in \REF[b-d]{ex:8:35} are also ungrammatical as there are two independent determiners in the noun phrase at the same time (e.g., $\emptyset$\textsubscript{D}/\textit{der}/\textit{dieser} and \textit{ein}).\footnote{We saw above that the singularity numeral cannot be followed by \textit{so} ‘such’ and unreduced \textit{ein} \REF{ex:8:32a}. The same holds for possessive articles \REF{ex:8:11:1a}. The question arises as to why the negative article followed by \textit{so} and unreduced \textit{ein} as in \REF{ex:8:11:1b} sounds better than its two counterparts.
	
	\ea
	\ea[*]{\label{ex:8:11:1a}
	\gll meine so eine Frau\\
	my so a woman\\
	\glt `my such woman'
}
	\ex[?]{\label{ex:8:11:1b}
	\gll keine so eine Frau\\
	no so a woman\\
	\glt `no such woman'
}
\z
\z 
		
Note that an informal Google-search retrieved hits such as \REF{ex:8:11:1b}. The fairly good status of this example is surprising as regards the previous discussion. While I do not have a definite answer here, there appears to be a tension between the need to support \textit{so} and the requirement to delete lower unreduced copies of \textit{ein} (as discussed in the main text). This presumably accounts for the markedness of this type of example. While this tension also holds for the singularity numeral and the possessive articles, there are additional factors that make the latter two cases worse: The case with the singularity numeral involves two (almost-) homophonous instances of \textit{ein} and the possessive articles induce definite contexts.}

  To sum up thus far, \textit{ein} appears to be able to support multiple instances of overt operators, and it can flag a single instance of a covert one.

  \TabPositions{2.5cm,4cm}
\ea%38
    \label{ex:8:38}

          Supporting and flagging:
\ea \label{ex:8:38a} (\textit{m-eine})      \tab 		\tab   REL
\ex \label{ex:8:38b} (\textit{k-eine})       \tab   (\textit{so-ne}) \tab  REL
\ex \label{ex:8:38c} ($\emptyset$\textsubscript{[--PL]}\textit{+eine}) \tab  (\textit{so-ne}) \tab  REL
\z
\z

Considering \REF{ex:8:38}, we observe that \textit{ein} can support (at least) four different kinds of overt operators. The question arises if \textit{ein} can also flag different kinds of covert operators. I tentatively suggest in the next section that besides REL, \textit{ein} can also indicate the presence of EXIST, QUAL, QUANT, SORT, and CONT. However, so far, I have not been able to identify a clear case indicating that \textit{ein} can flag multiple instances of covert operators at the same time.

\subsubsection{Flagging of other covert operators}\label{sec:8.2.2.3}

There are some other well-known cases where \textit{ein} seems to make a semantic contribution, at least at first glance. I discuss split-scope phenomena, I return to exclamatives, and I briefly discuss \textit{ein} in mass nominals. For each of these cases, I suggest that it is not \textit{ein} that brings about the different interpretations. Rather, different covert operators are taken to be responsible, and \textit{ein} flags the presence of these null elements. One result of this discussion is that I identify more cases involving flagging. Furthermore, I draw the tentative conclusion that there do not seem to be cases where \textit{ein} flags multiple covert operators at the same time.

  I have argued at length that \textit{ein} is not a reflex of indefiniteness. It could be suggested that the absence of definite elements such as \textit{der} ‘the’ or \textit{dieser} ‘this’ leads to indefiniteness, perhaps as a default option. However, this is, most likely, not correct. There is evidence that indefiniteness is not a default option but rather that it involves an actual element. This evidence comes from split-scope phenomena (e.g., \citealt{Bech1955}, \citealt{Jacobs1980}, data taken from \citealt{Zeijlstra2011}: 113).

\ea%39
    \label{ex:8:39}
\gll Du  musst keine Krawatte tragen.\\
  you must  no     tie.\textsc{fem}     wear\\

  \TabPositions{7.5cm}
\ea \label{ex:8:39a}   ‘It is not required that you wear a tie.’    \tab ($\lnot$ > must > {{${\exists}$})}
\ex \label{ex:8:39b}   ‘There is no tie that you are required to wear.’ \tab  ($\lnot$ > {{${\exists}$}} > must{)}
\ex  \label{ex:8:39c}  ‘It is required that you don’t wear a tie.’   \tab  (must > $\lnot$ > {{${\exists}$})}
\z
\z

The most salient reading is paraphrased in \REF{ex:8:39a}, where the negation has the widest scope, and the existential operator has the narrowest scope. In other words, the negative article \textit{kein} ‘no’ is split into two parts, and both elements have semantic import. Note first that split scope provides more evidence for the composite analysis of the negative article (for a different view, see \citealt{JägerPenka2012}). Now, given that the different scope options are derived from the various positions of the existential operator in \REF[a-c]{ex:8:39}, I conclude that this involves an actual element. In keeping with the claims made thus far, I suggest that this existential operator is a null element and that this element is responsible for indefiniteness. For convenience, I relabel {{${\exists}$}} as EXIST(ence). I assume that the presence of this covert operator is flagged by \textit{ein}.

Second, as briefly discussed in \chapref{sec:4}, \sectref{sec:4.2} special contexts such as exclamatives license the presence of \textit{ein} with plural nouns as in the (constructed) example in \REF{ex:8:40a}. To repeat, \citet[167]{vanRiemsdijk2005} points out that these – often called – emotive or affective constructions can express some relatively excessive property of the entities under discussion. Mass nouns can also occur with \textit{ein} \REF{ex:8:40b}. Note that both examples have a clear quality reading expressed in the translation as ‘great’. While the latter reading may be dominant, a quantity interpretation also seems possible rendered in the translation as ‘a lot of’.

\ea%40
    \label{ex:8:40}
\ea[\%]{\label{ex:8:40a}
\gll  Was  für (ei)ne Autos!\\
what for  {\db}a       cars\\
\glt ‘Such great cars!’\\
\glt ‘Such a lot of cars!’}
\ex[]{\label{ex:8:40b}
\gll Was  für (ei)n Bier!\\
what for  {\db}a      beer.\textsc{neut}\\
\glt ‘Such great beer!’\\
\glt ‘Such a lot of beer!’
}
\z
\z


The presence of \textit{ein} is unexpected with plural nouns in indefinite contexts and with mass nouns more generally.\footnote{\label{foot:8:12} As briefly discussed in \chapref{sec:1}, \sectref{sec:1.2.2.1}, southern dialects of German allow indefinite articles to occur in plural contexts as part of argumental expressions (without \textit{so} ‘such’). The same holds true of mass nouns (\citealt{Haider1988}: 52 fn. 14).
	
	\ea Bavarian German\\
	\gll wos   moch i den mid  an wossa\\
	what make I \textsc{prt} with a   water\\
	\glt ‘What do I do with water?’
	\z 

Unlike Standard German, the indefinite article here is not restricted to singular count contexts.} I argued in \chapref{sec:4}, \sectref{sec:4.2.2} that cases like \REF{ex:8:40a} cannot involve a null noun between the article and the overt noun; that is, \textit{ein} cannot be due to the presence of a null noun. I assume the same for \REF{ex:8:40b}. While \textit{was für} ‘what kind’ is an (interrogative) operator in questions, the examples in \REF{ex:8:40} involve exclamatives. Thus, it seems less likely that \textit{was} \textit{für} is the operator here. I suggest that there are two covert semantic operators at work, QUAL(ity) and QUANT(ity) (cf. \chapref{sec:4}, \sectref{sec:4.2}, where \citet{BennisEtAl1998} and  \citealt{vanRiemsdijk2005} also claimed null operators, their [+EXCL] and !!!, respectively, to be present in the related constructions in Dutch).\footnote{Note that besides \textit{was für} ‘what kind’, the cases in \REF{ex:8:40} are also possible with \textit{so} ‘such’ instead. It might be claimed that \textit{so} is the relevant operator in these instances (similar to \sectref{sec:8.2.2.2} above).}  Depending on which operator is present, a different interpretation obtains. The presence of these operators is flagged by \textit{ein}.

  Third, mass nouns are interesting in other ways. When they occur with an indefinite article, they can also have an effect different from \REF{ex:8:40b} above. Specifically, \REF{ex:8:41a} has a sortal reading, and \REF{ex:8:41b} has a container reading as is clear from the translation involving \textit{loaf}.

\ea%41
    \label{ex:8:41}
\ea\label{ex:8:41a}
\gll Pumpernickel ist (ei)n Brot.\\
pumpernickel is    {\db}a     bread.\textsc{neut}\\
\glt ‘Pumpernickel is a bread (= certain kind of that substance).’
\ex\label{ex:8:41b}
\gll Ich möchte      (ei)n Brot.\\
I     would.like  {\db}a     bread.\textsc{neut}\\
\glt ‘I would like a loaf of bread (= a certain quantity of that substance).’
\z
\z

Notice that unlike \REF{ex:8:40b}, these instances involve non-exclamative contexts. I suggest that semantic operators also account for these data. To be concrete, I simply label these elements as SORT(er) and CONT(ainer) (cf. ‘univeral sorter’ and ‘universal packer’, respectively; see H. \citealt{WieseMaling2005} for discussion). Again, I take it that \textit{ein} flags the presence of the operators that bring about the different readings.

  To sum up, I made the claim more general that \textit{ein} indicates the presence of covert operators. In particular, I extended this claim to split-scope phenomena (EXIST), to plural indefinite nouns in exclamatives (QUAL, QUANT), and to mass nouns (QUAL, QUANT, SORT, CONT). This means that besides REL, other covert operators appear to be flagged by \textit{ein}.

\TabPositions{1.2cm}
\ea%42
    \label{ex:8:42}

          Flagging:
\ea\label{ex:8:42a}  \textit{(ei)n}  \tab REL
\ex\label{ex:8:42b}  \textit{(ei)n}  \tab EXIST
\ex\label{ex:8:42c}  \textit{(ei)n}  \tab QUAL
\ex\label{ex:8:42d}  \textit{(ei)n}  \tab QUANT
\ex\label{ex:8:42e}  \textit{(ei)n}  \tab SORT
\ex\label{ex:8:42f}  \textit{(ei)n}  \tab CONT
\z
\z

Thus far, I have not identified any cases where \textit{ein} flags two covert operators at the same time.\footnote{A case with two covert operators might involve the complex determiner \textit{ein jeder} ‘(an every =) each’, where \citet{Roehrs2012} suggests that \textit{ein} intensifies the distributive reading of \textit{jeder}. Accepting the proposal that \textit{ein} itself is not responsible for the semantics, a null distributivity operator (DIST) could be postulated. The example in \REF{ex:8:14:ia} could then be schematically analyzed as in \REF{ex:8:14:ib}.
	\ea
	\ea \label{ex:8:14:ia}
	\gll ein jeder  nette Student\\
	an  every nice  student.\textsc{masc}\\
	\glt ‘each nice student’
	\ex \label{ex:8:14:ib} \textit{ein} DIST \textit{jeder} Adjective REL Noun
	\z
	\z
	This presents a case of multiple covert operators in one nominal. However, it is unlikely that \textit{ein} flags both DIST and REL at the same time, as the definite determiner \textit{jeder} is present; that is, \textit{jeder} presumably flags REL on par with \textit{der} ‘the’ or \textit{du} ‘you’ as discussed above. Notice also that DIST is different from the other covert operators above in that it intensifies the reading of a specific determiner (rather than determines the reading of the nominal as a whole). As such, I do not add it to the list in \REF{ex:8:42}.} More generally, if the interpretation of the above empirical observations is tenable, then I can state again that the traditional term indefinite article is a misnomer. The fact though that one covert operator (CAP) is not indicated by \textit{ein}, but the operators of the rather long list in \REF{ex:8:42} are, is not very satisfying. Some more remarks are in order.

That CAP is not flagged by \textit{ein} but that the types of operator in \REF{ex:8:42} are, should ideally be related to an independent factor. If possible, we might like to claim that the respective operators have something in common. Given my current understanding of this issue, I think it is not impossible to make such a claim. In particular, we could suggest that the position the operator appears in is related to it being flagged by \textit{ein}. \Citet{deSwartEtAl2007} proposed that CAP is in NP and that REL is in NumP, the latter being indicated in singular contexts by \textit{ein}. Note that NumP is closer to the base-position of \textit{ein} (ArtP) than NP. Assuming that locality plays a role here, this could explain why REL (but not CAP) is flagged by \textit{ein}. If so, we could suggest further that all covert operators in \REF{ex:8:42} are in NumP. This, in turn, might open another avenue of research.

  It is linguistically neither very interesting nor very elegant to postulate a different operator for every different reading. This is basically what I did at the beginning of this subsection. In my view, it is desirable to pursue the idea that these operators can be related or even collapsed into fewer elements. While I have to leave the ultimate solution of this issue to semanticists, there are a couple of morpho-syntactic observations worth making here.

  REL and EXIST typically occur with count nouns. While REL was originally motivated for predicative contexts, EXIST (or $\exists$) is postulated for argumental expressions. At the very least, these cases seem structurally related in that predicative NumP (REL) is part of argumental DP (EXIST). At best, these two operators could be collapsed into one if the syntax of predicate vs. argumental nominals could explain the remaining properties (e.g., figurative extension, scope, etc.).

  By contrast, QUAL, QUANT, SORT, and CONT typically occur with mass nouns. If these four covert operators are indeed all in NumP as tentatively suggested above, then they could be collapsed into one more general operator that gets specified in a given context. In fact, it might turn out that the pragmatics specifies this general operator as regards the different readings. While I have to leave a more robust discussion of these issues for future research, it does not seem impossible to simplify the account above. Pending a solution of these issues, I continue with the operators as presented in \REF{ex:8:42}.

The result of the discussion above is that we can observe that all nominals containing \textit{ein} involve at least one operator. I believe this is an interesting claim. Crucially though, \textit{ein} occurs with a number of different operators. This means that stating the vocabulary insertion rule for \textit{ein} and accounting for the presence or absence of \textit{ein} in the various contexts is not trivial. As seen below, a number of complications arise. As such, the discussion in the next three subsections is, admittedly, somewhat involved. However, I think that it serves well to highlight the complexities that arise. Indeed, as should be clear, any account espousing the composite analysis of complex \textit{ein}-words has to deal with these complications. First, I address \textit{ein} in the context of covert operators, then I add overt operators to the discussion, and finally, I provide my account involving feature deletion, which may widen the distribution of \textit{ein}.

\subsubsection{Contexts of vocabulary insertion for \textit{ein}: covert operators}\label{sec:8.2.2.4}

In the previous chapters, I argued that the indefinite article does not have semantic import. Rather, it flags and supports operators. I proposed that \textit{ein} involves the categorial feature [+D] and a separate feature bundle for case, number, and gender. I took [+D] to be the semantically vacuous element that is spelled out as the stem \textit{ein}; the features for case, number, and gender are spelled out as its inflection. If features in addition to [+D], case, number, and gender are present, different elements are spelled out. To give some examples: If [+DEF] is present, the definite article \textit{der} ‘the’ is spelled out; if [+DEF; +DEIX] are present, the demonstrative \textit{dieser} ‘this’ is realized; if features for [AUTH] and [PART] are present, pronominal determiners surface. Note though that the distribution of determiners is more complicated when we consider the indefinite articles $\emptyset$\textsubscript{D} and \textit{ein} in the context of operators.

For instance, if \textit{ein} flags REL, it only occurs in singular contexts. However, if \textit{ein} supports an overt operator (in addition to flagging REL), it can occur in mass, singular, and plural contexts. Leaving the discussion of overt operators to the next subsection, note that all covert operators flagged by \textit{ein} involve indefinite contexts (see previous subsection). Before I turn to a more detailed discussion of the vocabulary insertion rules of the articles in indefinite contexts, I review the cases above and add a few minor points to the investigation. This lays the foundation for the account of Vocabulary Insertion of determiners. Given the complex sets of data, the discussion is summarized at various points in \tabref{tab:8:1} through \ref{tab:8:3} below.

Nouns become instantiated as mass, singular count, or plural count nominals during the derivation. I proposed in \chapref{sec:7} that the interplay between the head noun and Num as regards morphological and consequently semantic number leads to the three manifestations. Importantly, independent of number (and thus manifestation of the noun as mass, singular, or plural), all types of nominals involve an operator. Focusing on kind nouns, recall that they are of type \textlangle e\textrangle . In order to become predicates (type \textlangle e,t\textrangle ), they combine with REL. This step is clearly needed so that these nominals can combine with adjectival modifiers (type \textlangle e,t\textrangle ) by Predicate Modification. \Citet{deSwartEtAl2007} did not discuss mass nouns in this regard. However, since they can also be modified, I assume that they also involve REL. Furthermore, mass nouns can also occur in the context of existential operators. As such, they involve EXIST – just like count nouns. As is well known, in indefinite, non-exclamative contexts, mass and plural count nominals involve null articles, and singular count nominals contain \textit{ein}. This is summarized in \tabref{tab:8:1} (recall that mass nouns involve Num without a specification for morphological number).

\begin{table}
\caption{Covert operators in indefinite, non-exlamative contexts}
\label{tab:8:1}
\begin{tabular}{ll}
\lsptoprule
 Type of nominal & Non-exclamative\\
\midrule
mass (Num) & {$\emptyset$\textsubscript{D} (REL/EXIST)}\\
{singular (Num\textsubscript{[--PL]})} & {\textit{ein} (REL/EXIST)}\\
{plural (Num\textsubscript{[+PL]})} & {$\emptyset$\textsubscript{D} (REL/EXIST)}\\
\lspbottomrule
\end{tabular}
\end{table}

More needs to be said about mass nouns.

Mass nouns on their sortal (SORT) and container (CONT) reading as in \REF[a-b]{ex:8:41} occur in non-exclamative contexts and most likely involve nominals where Num is specified as [--PL morph]. This is so as these nominals can be pluralized (e.g., \textit{zwei verschiedene Zucker} ‘two different sugars’; \textit{zwei Brote} ‘two loaves of bread’). As such, I put the singular instances of these readings with the singular nominals in \tabref{tab:8:2} below. I assume that the distinction between the ordinary count reading of \textit{ein Schwein} ‘a pig’ and the special sortal reading of \textit{ein Zucker} ‘a (type of) sugar’ follows from the different lexical items involved and their preferred respective interpretation (\citealt{Borer2005}: 102, see also H. \citealt{WieseMaling2005}: 8). In \tabref{tab:8:2}, the singular sortal and container readings are put in a different row from the ordinary readings involving REL/EXIST. To be clear, compared to \tabref{tab:8:1}, \tabref{tab:8:2} involves an additional row in the non-exclamative column.

As to exclamative contexts (see column three of \tabref{tab:8:2}), it is not obvious if the quantity (QUANT) or quality (QUAL) reading of mass nouns involving \textit{ein} as in \REF{ex:8:40b} is singular. The pluralization (i.e., \textit{Was für eine Biere!} ‘Such beers!’) seems a bit marked and has a different, sortal reading. Furthermore, it does not seem plausible to relate the unspecific quantity reading, rendered in \REF{ex:8:40b} above as ‘a lot of’, to semantic singularity and countability. Rather, it seems more straightforward to interpret these cases as actual mass nominals; that is, as nominals that involve Num with no morphological specification at all. If so, these nominals are number neutral and can be interpreted in different ways. These nominals are put in the row of mass nominals in the exclamative column of \tabref{tab:8:2}. Plural count nominals involving \textit{ein} as in \REF{ex:8:40a} are put with the plural instances. Perhaps unsurprisingly, singular cases also occur in exclamative contexts. Given the singular context, they only have a quality reading (e.g., \textit{Was für ein Auto!} ‘Such a great car!’). The remarks above can be summarized as shown in \tabref{tab:8:2}.

\begin{table}
\caption{Covert operators in indefinite contexts}
\label{tab:8:2}
\begin{tabular}{lll}
\lsptoprule
 Type of nominal & Non-exclamative & Exclamative\\
\midrule
mass (Num) & {$\emptyset$\textsubscript{D} (REL/EXIST)} & {\textit{ein} (QUANT, QUAL)}\\

{singular (Num\textsubscript{[--PL]})} & {\textit{ein} (REL/EXIST)} & {\textit{ein} (QUAL)}\\

& {\textit{ein} (SORT, CONT)} &\\

{plural (Num\textsubscript{[+PL]})} & {$\emptyset$\textsubscript{D} (REL/EXIST)} & {\textit{ein} (QUANT, QUAL)}\\
\lspbottomrule
\end{tabular}
\end{table}

Considering \tabref{tab:8:2}, note again that the articles occur in the context of certain operators. To repeat, articles in non-exclamative contexts are inserted in their usual contexts: The null articles are inserted in mass and plural contexts, and \textit{ein} in singular environments. Exclamative contexts are different. Here, \textit{ein} occurs in all number environments. Given this complexity, the vocabulary insertion rules for the indefinite articles $\emptyset$\textsubscript{D} and \textit{ein} are developed in two steps, the first step is discussed in the remaining part of this subsection and the second step in the next subsection.

All determiners including articles involve the categorial feature [+D]. \textit{Ein} is the least specified determiner – it only has the categorial feature (abstracting away from the inflection). As mentioned above, other determiners including the null articles have more features. In order to restrict the null articles to non-exlamative contexts, I assume for now that exclamative contexts involve the feature [+EXCL(amative)] (but see the next subsection where this feature is made unnecessary). Postulating that the null articles have the negative counterpart of [+EXCL] as part of their vocabulary insertion rules, this feature will block the occurrence of the null articles in exclamative contexts. Before I provide the preliminary set of vocabulary insertion rules, I need to make some other remarks.

I argued above that countability is a side effect of Num being specified for singular or plural. While the feature [±COUNT] has not done any work in the current system so far, I make use of it here as it allows me to state the vocabulary insertion rules for determiners more straightforwardly.\footnote{Note that with features having only two values (e.g., [±PL]), a three way distinction such as mass/singular/plural cannot be expressed by just one feature. As will become clear, underspecification for mass contexts does not work in these cases – a feature like [--COUNT] seems to be needed.} In order to restrict the occurrence of the articles to their usual contexts (i.e., the non-exclamative cases in \tabref{tab:8:1}), I assume a special type of feature for all indefinite articles. I label this specification as restriction feature. This restriction feature can be of two types: [±COUNT] and [±PL]. This special feature restricts the occurrence of the articles to contexts as regards morphological number, so that the mass article only occurs in mass contexts, the singular article only in singular contexts, and the plural article only in plural contexts. I assume that this specification is on the categorial feature [+D], and I put it after the colon following the categorial feature (i.e. [+D:\_\_\_]). Note that this feature is not a regular concord feature (the latter being spelled out by the inflection) – it is a compatibility feature (see also the discussion of uninflected \textit{dies} ‘this’ in \chapref{sec:4}, \sectref{sec:4.5}).

Again, restricting the discussion to indefinite contexts for now, I assume that the mass article $\emptyset$\textsubscript{D} involves [+D: --COUNT] and that the plural article $\emptyset$\textsubscript{D} has [+D: +PL]. The above-mentioned feature [--EXCL] confines the null articles to non-exclamative contexts. The first versions of the vocabulary insertion rules for mass and plural $\emptyset$\textsubscript{D} are provided in \REF[a-b]{ex:8:43}. The singular article \textit{ein} involves the feature [+D: --PL], and its preliminary vocabulary insertion rule is given in \REF{ex:8:43c} (NB: This number restriction is neither on Num nor on N and will not lead to semantic singularity of \textit{ein}).\footnote{Recall from \chapref{sec:2}, \sectref{sec:2.2.2.2} that the vocabulary insertion rules for \textit{ein} consisted of three entries. \REF{ex:8:43c} in the main text presents the elsewhere case. The remaining two rules are repeated here for convenience and updated with the restriction feature [--PL].
	\TabPositions{3cm}
	\ea
	\ea\label{ex:8:16:ia} [+D: --PL]   	    \tab$\rightarrow$   \textit{ein-} / [--F, --N, --O, +S]
	\ex\label{ex:8:16:ib} [+D: --PL][--F, --O]  \tab$\rightarrow$   \textit{ein} / \_\_ word {]}\textsubscript{φ}
	\z
	\z 
Also, recall from Footnote \ref{foot:8:12} that \textit{ein} in southern German dialects has a wider distribution, something I will not make more formal here. 
}

\ea%43
    \label{ex:8:43}

      Preliminary vocabulary insertion rules for:
\TabPositions{5cm}
\ea\label{ex:8:43a} [+D: --COUNT; --DEF; --EXCL]   \tab$\rightarrow$   $\emptyset$\textsubscript{D}
\ex\label{ex:8:43b} [+D: +PL; --DEF; --EXCL]    \tab$\rightarrow$   $\emptyset$\textsubscript{D}
\ex\label{ex:8:43c} [+D:  --PL]        \tab $\rightarrow$   \textit{ein-}
\z
\z


Returning to the co-occurrence of the operators and articles (\tabref{tab:8:2}), the null articles $\emptyset$\textsubscript{D} and \textit{ein} are inserted if the features of their vocabulary entries in \REF{ex:8:43} are present in the syntactic representation, provided their restriction feature on [+D] is compatible with the concord feature on Num of the noun phrase as a whole. Note in this regard that I assume that [--COUNT] in \REF{ex:8:43a} is only compatible with the absence of [$\alpha$PL morph] on Num. This accounts for the distribution of the different articles occurring with REL/EXIST in the three number contexts (mass, singular, plural) in non-exclamative environments. SORT, CONT, QUANT, and QUAL are different. They only occur with \textit{ein} (\tabref{tab:8:2}). SORT and CONT occur with \textit{ein} in singular contexts; that is, they occur with \textit{ein} when Num is specified as [--PL morph] – note that only \textit{ein} is compatible with this number specification on Num.\footnote{As mentioned above, SORT and CONT also occur in plural contexts (e.g., \textit{zwei verschiedene Zucker} ‘two different sugars’; \textit{zwei Brote} ‘two loaves of bread’). Here, Num is specified as [+PL morph] leading to the insertion of different determiners, provided they are compatible with this number specification (e.g., plural $\emptyset$\textsubscript{D} and \textit{diese} ‘these’).}

As to QUANT and QUAL, they occur in mass, singular, and plural contexts in exclamative environments. This means that null articles cannot be inserted here – they have the feature [--EXCL]. Note though that \textit{ein} cannot be inserted in mass and plural contexts either, as \textit{ein} is restricted to singular environments by [--PL] on [+D]. The question arises how \textit{ein} can surface in mass and plural contexts with QUANT and QUAL (but not with REL/EXIST, SORT, or CONT in non-exclamative mass and plural contexts). In the next subsection, I propose that certain operators can delete the restriction feature on [+D]. This allows \textit{ein} (and $\emptyset$\textsubscript{D}) to occur in other number contexts.

\subsubsection{Contexts of vocabulary insertion for \textit{ein}: Covert and overt operators}\label{sec:8.2.2.5}

As discussed above, in non-exclamative contexts, the null articles are restricted to mass and plural contexts, and \textit{ein} is restricted to singular contexts. Note again that these articles can be used in other environments when certain operators are present, operators that occur in exclamative contexts (e.g., QUANT), but also operators that occur more generally (e.g., POSS). This means that certain combinations of operators and articles have to be allowed while others have to be excluded. \tabref{tab:8:3} repeats the covert operators occurring in exclamative contexts (i.e., QUANT, QUAL), and it also contains the overt operators and their co-occurring articles. Note that POSS (\textit{m}-) and POSS (-\textit{s}) stand for possessive articles and Saxon Genitives, respectively.

\begin{table}
\caption{Certain operators and their co-occurring articles}
\label{tab:8:3}
\begin{tabular}{llllllll}
\lsptoprule
 & QUANT/ & \textit{so} & $\emptyset$\textsubscript{[--PL]} & POSS & POSS & NEG & NEG\\
 & QUAL & & & {(\textit{m}-)} & {(-\textit{s})} & {(\textit{k}-)} & {(\textit{nicht})}\\
\midrule
mass & {\itshape ein} & {\itshape ein} & {\itshape -} & {\itshape ein} & $\emptyset$\textsubscript{D} & {\itshape ein} & {\itshape der}\\
singular & {\itshape ein} & {\itshape ein} & {\itshape ein} & {\itshape ein} & $\emptyset$\textsubscript{D} & {\itshape ein} & {\itshape der}\\
plural & {\itshape ein} & {\itshape ein} & {\itshape -} & {\itshape ein} & $\emptyset$\textsubscript{D} & {\itshape ein} & {\itshape der}\\
\lspbottomrule
\end{tabular}
\end{table}

These are the main combinations of operators and articles. This means that the following account has to rule out strings with combinations like *\textit{so-$\emptyset$\textsubscript{D}} \textit{Milch/Autos} ‘such milk/cars’, *\textit{m-der Freund} ‘my friend’, *\textit{Peters der Freund} ‘Peter’s friend’, *\textit{k-$\emptyset$\textsubscript{D}} \textit{Milch/Autos} ‘no milk/cars’, and others.

  I propose that the deletion of features under Art in the syntactic representation may widen and narrow the distribution of determiners.\footnote{This deletion could be instantiated by Impoverishment.} Specifically, deleting the restriction feature on [+D] will allow all relevant articles to occur in all three number contexts (i.e., mass, singular, plural). To narrow the distribution of articles again, another feature under Art in the syntactic representation can also be deleted. For instance, deleting the categorial feature [+D] under certain conditions will prevent $\emptyset$\textsubscript{D} and \textit{der} from surfacing but not \textit{ein}.

The account of REL/EXIST, SORT and CONT is as above – no features are deleted. However, the analysis of QUANT and QUAL is different. Given the suggestion just made, this alternative will account not only for the occurrence of \textit{ein} with these two operators but also for the distribution of determiners with the overt operators more generally. In addition, this deletion of features will make the use of the feature [EXCL] unnecessary. Before I turn to the details, I lay out some other assumptions.

All argument DPs have determiners. In what follows, I focus on the three types of articles: the null articles as in \figref{figex:8:44} below, \textit{ein} provided in \figref{figex:8:45}, and the definite article given in \figref{figex:8:46}. As proposed in \chapref{sec:2}, \sectref{sec:2.2.1.6}, these three types of articles have slightly different structures and features under Art (see the (a)-instances below). Each article has its own corresponding vocabulary insertion rule (see the (b)-instances below). This is consistent with the general assumptions of DM, where a difference is made between the syntactic representation and subsequent Vocabulary Insertion. To be clear, Art with its varying feature bundles is in the derivation at the beginning, but the corresponding vocabulary insertion rules of the articles supply the spell-out forms later. Recall that all these articles are merged in ArtP and move to DP (triggering Impoverishment as discussed in \chapref{sec:2}). These are the final structures and vocabulary insertion rules of the three types of articles (for the vocabulary insertion rules of adjectival inflections, see \chapref{sec:2}, \sectref{sec:2.2.1.5}; the forward slash sign in \figref{figex:8:44a} is explained below).
  
\begin{figure}
	\subfigure[Structure of the indefinite null articles]{
		\label{figex:8:44a}
		\begin{forest}
			[Art\textsubscript{\parbox{0mm}{\mbox{[+D: --COUNT/+PL; --DEF]}}}
				[{[+D: --COUNT/+PL; --DEF]}]
			]
		\end{forest}
	}

	\subfigure[Feature spell-out of the indefinite null articles]{
		\label{figex:8:44b}
		\begin{forest}
			[,phantom
				[{[+D; --DEF]}, tier=A]
				[$\rightarrow$  $\emptyset$\textsubscript{D}, no edge, tier=A]
			]
		\end{forest}
	}
\caption{Indefinite null articles}
\label{figex:8:44}
\end{figure}

\begin{figure}
	\subfigure[Structure of the indefinite article \emph{ein}]{
		\label{figex:8:45a}
		\begin{forest}
			[Art\textsubscript{\parbox{0mm}{\mbox{[+D: --PL][CNG]}}}
				[{[+D: --PL]}]
				[{[CNG]}]
			]
		\end{forest}
	}

	\subfigure[Feature spell-out of the indefinite article \emph{ein}]{
		\label{figex:8:45b}
		\begin{forest}
			[,phantom
				[{[+D]}, tier=A]
				[$\rightarrow$  \textit{ein-}, no edge, tier=A]
			]
		\end{forest}
	}
\caption{Indefinite overt article}
\label{figex:8:45}
\end{figure}

\begin{figure}
	\subfigure[Structure of the definite article]{
		\label{figex:8:46a}
		\begin{forest}
			[Art\textsubscript{\parbox{0mm}{\mbox{[+D; +DEF][CNG]}}}
				[{[+D; +DEF]}]
				[{[CNG]}]
			]
		\end{forest}
	}

	\subfigure[Feature spell-out of the definite article]{
		\label{figex:8:46b}
	\begin{forest}
		[,phantom
			[{[+D; +DEF]}, tier=A]
			[$\rightarrow$  \textit{d-}, no edge, tier=A]
		]
	\end{forest}
	}
\caption{Definite article}
\label{figex:8:46}
\end{figure}


Recall that the feature bundles under Art undergo feature union. I comment on the feature bundles under Art and on the feature bundles in the vocabulary insertion rules.

Recall that the definiteness feature [DEF] is not a concord feature of the noun phrase as a whole. Rather, it is part of specific determiners. The two null indefinite articles involve [--DEF] as in \figref{figex:8:44}. For convenience, I provided both null articles in \figref{figex:8:44}, but it should be kept in mind that they differ in their restriction feature on [+D] (the two articles are distinguished by the different restriction feature separated by a forward slash sign in \figref{figex:8:44a}). As to the two other articles, \textit{ein} lacks a definiteness feature as in \figref{figex:8:45}, and the definite article has [+DEF] as in \figref{figex:8:46}. Furthermore, unlike the two types of indefinite articles ($\emptyset$\textsubscript{D}, \textit{ein}-), there is only one definite article (\textit{d}-). The latter element can be inserted in all number contexts. As such, I assume that there is no restriction feature on [+D] with the definite article. Importantly, observe that none of the vocabulary insertion rules involve restriction features. Other than that, the features of the vocabulary insertion rules are identical to the features of their corresponding determiner stems under Art. Finally, recall that CNG stands for the inflectional feature bundle involving case, number, and gender.

Crucially, I assume that the determiner structures are all merged freely; that is, either the structure of a null article in \figref{figex:8:44a}, the structure of \textit{ein} in \figref{figex:8:45a}, or the structure of the definite article in \figref{figex:8:46a} is merged (projecting ArtP in the process). As discussed in detail below, this will allow some articles – specifically the null articles and \textit{ein} – to appear in some other contexts.\footnote{For instance, this will allow \textit{ein}, usually restricted to singular contexts, to appear in plural environments (where Num involves a positive feature for [PL]). As made more precise below, this means that the compatibility of determiners and Num in number is checked after the deletion of the restriction feature on [+D]. This will also result in some cases of overgeneration. The latter are accounted for by the deletion of the feature [+D] on the determiner in certain contexts leading to bad derivations.}  Furthermore, I assume that all features of a determiner structure must be spelled out or supported. This means that the categorial feature [+D] and, if present, the definiteness feature must be spelled out by a determiner stem. Furthermore, the CNG feature bundle, if present, must be supported by an (overt) determiner stem. Since every type of Art above involves features that must be spelled out, every nominal that involves ArtP has a determiner.

In addition, I assume that all operators surface in the left periphery of the noun phrase: Either they are base-generated there (e.g., NEG), or they move there (e.g., POSS).\footnote{This high position also licenses the occurrence of adjectival inflections and \textit{ein} as expletive elements (see \sectref{sec:8.3.2} below).} As a consequence, all operators precede the determiner. This can schematically be illustrated as follows where OP stands for operator, and $\alpha$ may be XP in case the operator is adjoined (e.g., NEG) or X’ in case the operator is in a specifier position (e.g., POSS).

\begin{figure}
	\caption{Structural relation between the operator and the indefinite article \emph{ein}}
	\label{figex:8:47}
	\begin{forest}
		[XP
			[\textit{OP}]
			[$\alpha$
				[Art]
				[$\dots$]
			]		
		]
	\end{forest}
\end{figure}

Now, in order to widen the distribution of articles in certain contexts, I assume that all operators (except REL/EXIST, SORT and CONT) delete the restriction feature on [+D] under Art before the determiner structure is checked for compatibility with Num of the larger noun phrase and before Vocabulary Insertion occurs. While this does not affect the distribution of the definite article in \figref{figex:8:46}, this allows the indefinite articles in \figref{figex:8:44} and \figref{figex:8:45} to occur in other number contexts with specific operators: the null article(s) with POSS (Saxon Genitives), and \textit{ein} in most of the other cases. With this in place, I continue the discussion of covert QUANT and QUAL and extend it to the overt operators.

\subsubsection{Contexts of vocabulary insertion for \textit{ein}: Feature deletions}\label{sec:8.2.2.6}

As just proposed, the deletion of the restriction feature on [+D] under Art widens the distribution of the indefinite articles. To constrain this widening in the relevant way, there are several options to narrow the distribution of articles. As far as I can see, the most straightforward option is to assume that the categorial feature [+D] is deleted in certain featural contexts by specific operators.\footnote{Note that if other features than [+D] were deleted, then this would, in certain contexts, allow the least specified element \textit{ein} in \figref{figex:8:45b} to be inserted in the null article structure in \figref{figex:8:44a} or in the definite article structure in \figref{figex:8:46a}.} For instance, QUANT and QUAL trigger the deletion of [+D] in the context of a definiteness feature, where [DEF] includes [--DEF] and [+DEF]. This is shown in \REF{ex:8:48a}, along with the other operators that work the same way. \REF[b-c]{ex:8:48} show the remaining cases. In \REF{ex:8:48b}, [+D] is deleted in the context of the [CNG] feature bundle and the operator POSS (-\textit{s}); in \REF{ex:8:48c}, [+D] is deleted in the context of the feature [--DEF] and the operator NEG (deletion of a feature is marked below by strikethrough).

\ea%48
    \label{ex:8:48}
Deletion of [+D] under Art in the Contexts of […]:
\ea\label{ex:8:48a} \sout{[+D]}   /   [DEF] triggered by QUANT, QUAL, \textit{so}, $\emptyset$\textsubscript{[--PL]}, POSS (\textit{m}-)
\ex\label{ex:8:48b} \sout{[+D]}   /   [CNG] triggered by POSS (-\textit{s})
\ex\label{ex:8:48c} \sout{[+D]}   /   [--DEF] triggered by NEG
\z
\z


These are the deletions that narrow the distributions of the articles. As explicated in more detail below, \REF{ex:8:48a} only allows \textit{ein} to be inserted in the relevant contexts, \REF{ex:8:48b} only allows $\emptyset$\textsubscript{D}, and \REF{ex:8:48c} only allows \textit{ein} or \textit{d}- to be inserted in the relevant context. The articles that cannot be inserted into their corresponding structure will lead to bad derivations as the undeleted features of their corresponding structure will not be spelled out or supported. Also, note that each of the “surviving” determiners in \REF{ex:8:48} still has the feature [+D] in tact under Art. This means that Impoverishment will not be affected (by this feature deletion on the other determiner structures). I discuss the effects of the deletion of the restriction feature and other features in more detail.

  As mentioned above, QUANT and QUAL delete the restriction feature on [+D]. This brings about [+D: \sout{--COUNT}], [+D: \sout{--PL}], or [+D: \sout{+PL}] in the relevant syntactic representations (Art). Consequently, all three article structures in \figref{figex:8:44a}, \figref{figex:8:45a}, and \figref{figex:8:46a} can, at least in principle, occur in the three morphological contexts as regards number (mass, singular, plural). In other words, this deletion widens the distribution of the indefinite articles (recall that the definite aricle has no restriction feature to begin with). In addition, I suggested in \REF{ex:8:48a} above that these two operators delete the categorial feature [+D] under Art in the context of [DEF], which includes [--DEF] and [+DEF]. This allows the structure in \figref{figex:8:45a} to be spelled out, but not the structures in \figref{figex:8:44a} and \figref{figex:8:46a}. This, in turn, narrows the distribution of articles.

  Specifically, if Art in the syntactic representation involves [\sout{+D: --COUNT}; --DEF] or [\sout{+D: +PL}; --DEF] as in \figref{figex:8:44a} above (modulo the deletion), then $\emptyset$\textsubscript{D} cannot be inserted by \figref{figex:8:44b} as [+D] was deleted by the relevant operator. Similar assumptions hold for the definite article where Art involves [\sout{+D}; +DEF] [CNG] as in \figref{figex:8:46a} above, again modulo the deletion. Note that these two cases lead to bad derivations as the remaining definiteness feature under Art cannot be spelled out by a determiner. This is different for \textit{ein}, where Art involves [+D: \sout{--PL}] [CNG] as in \figref{figex:8:45a}. Here, there is no definiteness feature under Art and in the corresponding vocabulary insertion rule. Consequently, \textit{ein} can be inserted by the rule in \figref{figex:8:45b}. This type of account carries over to the overt operators and their co-occuring determiners.

Unless independently ruled out (e.g., $\emptyset$\textsubscript{[--PL]}), all operators occur in all three number contexts (\tabref{tab:8:2} and \ref{tab:8:3}). Like QUANT and QUAL, I suggest that \textit{so}, POSS, and NEG also delete the restriction feature on [+D]. Like above, this allows all three article structures to occur in the three number contexts with these operators as well. Furthermore, similar to QUANT and QUAL just discussed, overt operators also delete [+D] under Art in certain contexts narrowing the distribution of the articles, as stated in \REF{ex:8:48} above.

In more detail, \textit{so} ‘such’ also deletes the categorial feature [+D] under Art in the context of [DEF]. As with QUANT and QUAL, if \figref{figex:8:44a} or \figref{figex:8:46a} is merged in ArtP and moves to DP, their corresponding vocabulary items cannot be inserted. As the definiteness feature under Art cannot be spelled out, this leads to bad derivations. By contrast, if \figref{figex:8:45a} is merged in ArtP and moves to DP, \textit{ein} can be inserted by the rule in \figref{figex:8:45b}, and all the features under Art are spelled out. This results in a good derivation. Similar considerations hold for $\emptyset$\textsubscript{[--PL]}, with the proviso that this element itself is restricted to occur in singular contexts (\chapref{sec:5}). There are two types of derivations for POSS and NEG each.

Like \textit{so} above, the possessive component (POSS) of the possessive articles deletes [+D] under Art in the context of [DEF]. This allows \figref{figex:8:45a} to be spelled out, but not \figref{figex:8:44a} or \figref{figex:8:46a}. The latter two lead to bad derivations as the definiteness feature under Art cannot be spelled out. In contrast, I suggest that the possessive marker -\textit{s} (POSS) of Saxon Genitives deletes [+D] under Art in the context of the [CNG] feature bundle. This allows \figref{figex:8:44a} to be spelled out (NB: There is no deletion of [+D] here – the null articles have no CNG features; only the restriction feature on [+D] is deleted). In contrast, the articles in \figref{figex:8:45b} and \figref{figex:8:46b} cannot be inserted – [+D] has been deleted under Art. These derivations are bad as the [CNG] feature bundles of \figref{figex:8:45a} and \figref{figex:8:46a} cannot be supported by the Saxon Genitive (e.g., *\textit{Peters-er}). In addition, the definiteness feature in \figref{figex:8:46a} cannot be spelled out.\footnote{Recall from \chapref{sec:1}, Footnote \ref{foot:1:51} that definiteness spread, if instantiated by Spec-head agreement, leads to incompatibility between definite Saxon Genitives and the null articles (which are specified as [--DEF]). There are two possible solutions. First, similar to the head D, which involves [uDEF] (see \chapref{sec:1}, \sectref{sec:1.4.1.1}), we could assume that the null articles do not have an inherent feature for definiteness but rather that they get specified in this regard during the derivation. Note though that if the definiteness feature of the null articles gets specified as positive, this presumably turns $\emptyset$\textsubscript{D} into triggers for Impoverishment, resulting in weak adjectives also in nominative/accusative/dative contexts, contrary to fact (see again \chapref{sec:2}, \sectref{sec:2.2.2.2}). Clearly, such a specification process must be prevented. Alternatively, the feature [DEF] on $\emptyset$\textsubscript{D} could be of a different kind. Similar to some vocabulary insertion rules discussed above, this feature could involve an (unvaluable) variable (i.e., [$\alpha$DEF]), which is compatible with both a positive or a negative value of definiteness. In other words, $\emptyset$\textsubscript{D} could occur in either indefinite or definite contexts (note that similar to \textit{ein}, the (in-)definiteness of the larger noun phrase would depend on the presence of possessors and other elements). Since this type of feature does not get specified, Impoverishment would not be triggered. I do not attempt here to decide between these two solutions. Note though that either solution involves a definiteness feature on $\emptyset$\textsubscript{D}, and the structures of the null articles, their vocabulary insertion rules, and the relevant deletion rules above would only have to be changed from [--DEF] to [uDEF] or [$\alpha$DEF].}

Finally, I assume that NEG deletes [+D] under Art in the context of [--DEF]. This allows both \figref{figex:8:45a} and \figref{figex:8:46a} to be spelled out. Note that \figref{figex:8:46b} is more specific than \figref{figex:8:45b}. This means that \figref{figex:8:46a} is spelled out by the rule in \figref{figex:8:46b} and \figref{figex:8:45a} by the rule in \figref{figex:8:45b}. The null articles in \figref{figex:8:44a} lead to bad derivations. In the latter cases, the definiteness feature under Art cannot be spelled out by the rule in \figref{figex:8:44b}. The vocabulary insertion rules from \chapref{sec:5}, \sectref{sec:5.4.1.2}, repeated below for convenience, account for the overt realization of NEG depending on the context: If \figref{figex:8:45a} is merged and spelled out by the rule in \figref{figex:8:45b}, then NEG is realized as \textit{k}-; if \figref{figex:8:46a} is merged and spelled out by the rule in \figref{figex:8:46b}, then NEG is spelled out as \textit{nicht}.

\ea%49
    \label{ex:8:49}

          \gll NEG   $\rightarrow$   \textit{k}-   /{\_\_ (unstressed) \textit{ein}}\\
          {} $\rightarrow$   \textit{nicht}   (elsewhere)\\
\z

To sum up this section on \textit{ein}, I reviewed the combinations of operators and articles and accounted for them. It is probably fair to state that this discussion provided a somewhat involved solution, and certain issues remain. Having said that, I hope that it makes clear what complexities arise in the discussion. As already mentioned above, any account adopting a composite analysis of complex \textit{ein}-words such as \textit{kein} ‘(NEG+a =) no’ has to come to terms with these issues. Some other conclusions can be drawn.

Given the discussion above, it appears that determiners are not only inserted to flag and support operators. Rather, Art in the syntactic representation involves features, and these features must be spelled out by a determiner to yield a good derivation. All determiners including \textit{ein} have [+D] as part of their vocabulary entries. Consequently, if Art is present in the noun phrase, [+D] is spelled out yielding the fact that all nominals bigger than NumP have a determiner (in fact, since REL and other operators flagged by determiners reside in NumP, it may turn out that all nominals bigger than NP involve determiners).

As to \textit{ein} itself, this element is usually inserted in singular contexts given the restriction feature [--PL] on [+D]. However, when certain operators are present, the restriction feature is deleted, and \textit{ein} can occur in other number contexts. The side effect is that \textit{ein} can not only flag and support operators in singular but also in mass and plural environments. This means that \textit{ein} is probably not due to a simple Last-Resort operation – its occurrence (and that of other determiners) follows from the specific features present under Art, depending on the presence of certain operators.

To sum up the entire \sectref{sec:8.2}, I addressed some further consequences of the current proposal. Specifically, I briefly discussed how the current analysis could be extended to the distribution of adjectival inflections in Alemannic German. Furthermore, I briefly examined more cases of supporting and flagging. Thus far, I have identified cases that only involve flagging, but there do not seem to be instances that only show supporting. In addition, I have found instances where overt and covert operators are supported and flagged by \textit{ein} at the same time. This included cases where two overt operators are supported (and a covert operator is flagged) by \textit{ein}. However, two covert operators do not seem to be flagged by \textit{ein} at the same time. Finally, I provided an account of the insertion of \textit{ein} (and other articles) in the contexts of overt and covert operators. The postulation of a restriction feature and the deletion of it and that of [+D] accounted for the varying distribution of \textit{ein} (and other articles) as regards the presence of different operators.

\section{More general considerations}\label{sec:8.3}

In this final section, I return to complex nominals in the context of concord in agreement features, and I make suggestions as to what all expletives, clausal and nominal, may have in common.

\subsection{Nominal structure and concord}\label{sec:8.3.1}

In the previous chapters, a variety of syntactic phrases played a role. These phrases can be assembled into one hierarchical, abstract structure.

\ea%50
    \label{ex:8:50}

          [ LPP [ D [ Card [ Agr [ Art [ Num [ N ]]]]]]]
\z

I note two things about this structure: I assumed that nouns move to Num and that articles merged in Art (or more precisely, merged \textit{as} Art) also undergo movement to higher positions in certain cases. With minor differences, this is consistent with the proposals by \citet{Julien2005a}, \citet{Roehrs2009a}, and others. Importantly, the structure in \REF{ex:8:50} received further application in this book. I take this to indicate that this structure shows good promise for future investigations. I turn to some issues related to agreement in non-canonical noun phrases.

  As is well known, elements in the noun phrase are subject to concord in agreement features. For example, determiners, adjectives, and nouns in structures like \REF{ex:8:50} agree in case, number, and gender. In the course of this book, I also discussed a number of more complex cases where a second nominal is combined with the matrix DP by way of an embedded complex specifier or right adjunction. In the literature, these non-canonical nominals have not received much attention. I return to some of them briefly here focusing on agreement. We will see that all these constructions agree in semantic number but not necessarily in morphological number. The relevant structures are illustrated here again but in a simplified way.

  In the context of capacity readings, I discussed DPs that involve \textit{als} ‘as’ and its null counterpart ALS. To recapitulate, the \textit{als-}nominal in \REF{ex:8:51a} involves adjunction to the pronoun where \textit{als} is the head of ModP, and the role noun \textit{Arzt} ‘doctor’ is an NP complement of that head. I proposed that the adjoined nominal does not contain NumP, and as such the number mismatch with a plural pronoun is only apparent (the plural interpretation follows from the presence of a distributivity operator as postulated by \citealt{deSwartEtAl2007}: 218). In contrast, while the ALS-nominal in \REF{ex:8:51b} also contains ModP, this contruction does exhibit concord. I suggested that the ALS-nominal must contain NumP and that it is located in the specifier of AgrP.

\ea%51
    \label{ex:8:51}
\ea \label{ex:8:51a}
\gll ihr [\textsubscript{ModP} \textit{als}\textsubscript{Mod} \textit{Arzt}]\\
    you(\textsc{pl})  {} as       doctor.\textsc{masc}\\
\glt ‘you as doctors’
\ex  \label{ex:8:51b}
\gll (vielleicht) ein [\textsubscript{ModP} ALS\textsubscript{Mod} \textit{Landwirt}] Agr {e\textsubscript{N}}\\
 {\db}really        a             {} {}            farmer.\textsc{masc}\\
\glt ‘(really) some farmer’
\z
\z

Two other types of nominals should be discussed in this context.

  Indefinite pronoun constructions also involve complex structures \REF{ex:8:52a}. As in the first two cases, these structures also involve ModP. Like \REF{ex:8:51a}, this ModP was also argued to involve adjunction. However, unlike \REF{ex:8:51a}, this nominal exhibits concord. In view of the presence of an inflected adjective and a null noun, I argued that REL (and thus NumP) must be in the structure. Finally, I discussed a case of morphological dis-agreement but semantic agreement \REF{ex:8:52b}. Given the figuratively extended meaning of the noun, I concluded that REL (and thus NumP) must be present. I proposed that the dis-agreeing nominal is embedded in the specifier of DisP.

\ea%52
    \label{ex:8:52}
\ea\label{ex:8:52a}
\gll jemand [\textsubscript{ModP} $\emptyset$\textsubscript{Mod} \textit{ander-er e\textsubscript{N}}]\\
    someone.\textsc{masc}   {} {}   different-\textsc{st}\\
\glt ‘someone different’
\ex \label{ex:8:52b}
\gll Sie [\textsubscript{NumP} \textit{Bauer}] Dis {e\textsubscript{N}}\\
you     {}    peasant.\textsc{masc}\\
\glt ‘you peasant’
\z
\z

I summarize the properties of these different cases in \tabref{tab:8:4}.

\begin{table}
\caption{Summary of the constructions and their properties}
\label{tab:8:4}
\small
\begin{tabularx}{\textwidth}{QQQQl}
\lsptoprule
Construction & Structure & Agreement between nominals & NumP & Head\\
\midrule
\textit{als}-capacity & adjunction & apparent morphological and semantic disagreement & no NumP  (but distributivity OP)\footnote{The \textit{als}-capacity construction involves NumP in the plural cases (e.g., \textit{ihr als Ärzte} ‘you as doctors’).} & \textit{als}\textsubscript{Mod}\\
ALS-capacity & specifier & morphological and semantic agreement & NumP & ALS\textsubscript{Mod}\\
Indefinite pronoun construction & adjunction & morphological and semantic agreement & NumP & $\emptyset$\textsubscript{Mod}\\
DisP & specifier & morphological disagreement but semantic agreement & NumP & --\\
\lspbottomrule
\end{tabularx}
\end{table}

It is interesting to note that all four constructions agree in semantic number, either mediated through NumP or a distributivity operator. As for concord, agreement in phi-features is not an issue if NumP is absent in the embedding (\textit{als}-capacity).\footnote{Recall from \chapref{sec:7} that this is similar to clausal structures involving copular verbs and non-plural role nouns (e.g., \textit{Ihr seid alle Arzt.} ‘You are all (doctor =) doctors.’), where NumP is also absent.} Such agreement holds though if NumP is present (ALS-capacity, indefinite pronoun construction). The only true exception is DisP. The latter involves NumP but does not show concord in agreement features. I briefly explore why this might be so.

  It is clear that the position of the embedded nominal \textit{per se} cannot account for these agreement patterns as both adjuncts and specifiers may exhibit concord (indefinite pronoun construction, ALS-capacity) or not (\textit{als}-capacity, DisP). Rather, it seems to be a property of the head of the embedded nominal that determines the agreement possibilities. In particular, ALS\textsubscript{Mod} and $\emptyset$\textsubscript{Mod} could be taken to mediate concord. I also argued that \textit{als}\textsubscript{Mod} is an operator that may take an NP complement, as in the case in \REF{ex:8:51a}, but it may also embed a NumP complement, as in the case of plural nouns (see superscript a in \tabref{tab:8:4}). With NumP present in the plural, the \textit{als}-capacity construction also shows agreement. All these three constructions involve the head Mod.

Unlike the first three cases, the nominal embedded in Spec,DisP does not contain the head Mod. Given this distinction, I would like to suggest that all three Mod heads mediate concord between the two nominals provided NumP is present.\footnote{I briefly discussed in \chapref{sec:2}, \sectref{sec:2.3.2} that Mod is instantiated by the prepositional element \textit{de} ‘of’ in certain Romance languages. As suggested in \citet[21-23]{Roehrs2008}, this element mediates concord between different nominals, for instance, in French indefinite pronoun constructions of the type \textit{quelque chose}\textsubscript{MASC} \textit{de grand}\textsubscript{MASC} ‘something big’.} Due to the absence of such a head in the nominal inside Spec,DisP, the latter does not have to undergo concord with the matrix nominal. Currently, it is not clear to me how to implement this idea. As such, I have to leave the details of this suggestion for future research.

\subsection{Expletive elements more generally}\label{sec:8.3.2}

As mentioned at the beginning, this book is not meant to contribute to the theory of expletive elements \textit{per se}. Rather, it seeks to identify more elements that share some of the properties of generally accepted expletives. \textit{There} in existential constructions and, to a lesser degree, definite articles in the context of proper names seem to be such established elements. This book argued that adjectival inflections and \textit{ein} are also expletive elements. In this final section, I compare these four elements in a bit more detail with the goal of identifying more common properties.

Recall that \citet{Chomsky1995} pointed out that expletive elements seem to violate the Principle of Full Interpretation. He proposed that the associate noun phrase in existential constructions moves at LF to license \textit{there}. In other words, a substantive element moves to license a vacuous one. \citet{Longobardi1994} suggested something similar for proper names and proprial articles. To motivate the movement of the proper name to the article, I follow Chomsky by claiming that all expletives are deficient elements (note that Chomsky takes \textit{there} to be a LF-affix).

Making this more general, this means that adjectival inflections and \textit{ein} also need to be licensed by a contentful element. For adjectives, I proposed in \chapref{sec:2}, \sectref{sec:2.2.4} that the adjective stem moves to provide a host for the inflection. As to \textit{ein}, I tentatively suggested in \sectref{sec:8.2.2.5} above that operators surface in the left periphery of the noun phrase: Either they are base-generated there, or they move there to precede \textit{ein}. I assume that this local relation licenses \textit{ein}. This means that \textit{ein} makes an operator visible but gets licensed by it at the same time. To repeat from \chapref{sec:1}, there is a division of labor: Syntactically, the expletive indicates the substantive element; semantically, the substantive element identifies the expletive. These relations are shown in \REF{ex:8:53} before movement of the licenser.


\eabox{%53
 \label{ex:8:53}
 \begin{tikzpicture}
 	\node[] (EXPL) {EXPL};
 	\node[] (SUBST) [right=of EXPL, xshift=3cm] {SUBST};
 	
 	\draw[->] ([yshift=2.5pt] EXPL.east) -- node[anchor=south] {Indicate (syntax)} ([yshift=2.5pt] SUBST.west);
 	\draw[<-] ([yshift=-2.5pt] EXPL.east) -- node[anchor=north] {Identify (semantics)} ([yshift=-2.5pt] SUBST.west);
 \end{tikzpicture}
}

There is a second point worth making.

Recall also that one of the conclusions of this book is that adjectival inflections and \textit{ein} make different abstract structures visible. Again, I suggest that this is a general property of all expletives. In other words, I would like to claim that both \textit{there} and the proprial article also indicate abstract structure. I consider this point in more detail.

Above, I made a distinction between supporting and flagging. Starting with the latter, this mechanism seems to exhibit certain similarities to what is involved with \textit{there} and its associate and with proprial articles and proper names. Specifically, in all these cases, the expletive element is a free, unbound morpheme (at least in the overt component of the derivation). I observe now that the expletives and the contentful elements involve two different positions where the expletives are higher in the structure than the substantive elements as shown in \figref{figex:8:54}.

\begin{figure}
	\caption{Structural relation between expletives and substantive elements}
	\label{figex:8:54}
	\begin{forest}
		[$\alpha$
			[EXPL
				[\textit{there}\\proprial article\\\textit{ein}]
			]
			[$\beta$
				[associate\\proper name\\operator]
				[~]
			]
		]
	\end{forest}
\end{figure}

The contentful licensers are merged lower in the structure due to semantico-syntactic factors; that is, they are base-generated low for independent reasons. Specifically, associates are at the bottom of existential constructions given local theta-role assignment \REF{ex:8:55a}, nouns including proper names form the bottom of their extended projections \REF{ex:8:55b}, and semantic operators like REL turn nouns into predicates before these nominals can combine with other elements \REF{ex:8:55c}. Notice that the expletives and their licensers are not linearly adjacent in their positions as other elements can intervene.

\ea%55
    \label{ex:8:55}
\ea\label{ex:8:55a} [There] is [a man] in the garden.
\ex\label{ex:8:55b}
\gll [der] alte [Peter]\\
      {\db}the    old    {\db}Peter.\textsc{masc}\\
\glt ‘Peter, who is old’
\ex\label{ex:8:55c}
\gll [ein] netter [REL] Freund\\
  {\db}a      nice      {}           friend.\textsc{masc}\\
\glt ‘a nice friend’
\z
\z

Given these data, I reiterate the claim that these expletives and their licensers are related by movement. Recall that expletives are deficient elements and that the substantive elements are in a lower position for independent reasons. Now, as syntactic movement only proceeds upwards, we have an explanation as to why the expletives must be in the higher position (and not the lower one) – they are licensed by the movement of the substantive elements. Being in a high position, expletives indicate abstract structure. This includes \textit{there} and proprial articles.

  The mechanism of supporting is partially different. It, too, makes operators visible but forms overt composites of the two relevant elements under linear adjacency. This involves the complex \textit{ein}-words and inflected adjectives. Like flagging, the substantive element is base-generated low in the structure and undergoes movement to the left periphery. This is clearly the case with the possessive component. After movement, it combines with expletive \textit{ein} \REF{ex:8:56a}. Similary, the adjective stem is base-generated below its inflection and moves to combine with it \REF{ex:8:56b}.

\ea%56
    \label{ex:8:56}
\ea\label{ex:8:56a}
\gll [\textsubscript{DP} \textit{m}\textsubscript{i}-\textit{ein}  \textit{Auto} t\textsubscript{i}]\\
     {}     \textsc{poss}-a car.\textsc{neut}\\
\glt ‘my car’
\ex\label{ex:8:56b}  
\gll das [\textsubscript{InflP} \textit{groß}\textsubscript{i}-\textit{e} t\textsubscript{i}] Auto\\
the    {}     big-\textsc{infl}  {}  car.\textsc{neut}\\
\glt ‘the big car’
\z
\z


In each case, the substantive part originates low in the structure, and the expletive element is located higher in the structure.

This leaves the three overt operators that are part of the negative article \textit{kein} ‘no’, the singularity numeral \textit{EIN} ‘one’, and \textit{so’n} ‘such a’ to be discussed. Unlike the cases above, these three operators are base-generated higher in the structure than \textit{ein}: NEG is outside the DP proper, $\emptyset$\textsubscript{[--PL]} is in Spec,CardP, and \textit{so} is in Spec,TypeP. Recall from \sectref{sec:8.2.2.2} though that I have, thus far, not been able to identity cases that only involve supporting. In other words, all cases involve flagging and may, additionally, also show supporting. If this turns out to be correct, then we can claim that flagging is the primary mechanism to indicate operators, and supporting is a secondary, additional mechanism.\footnote{This could mean that flagging is a universal process but supporting is not.} Consequently, I take flagging, as discussed above, to manifest the typical properties of the relationship between expletives and their licensers. To be clear, the cases involving supporting do not seem to be telling as regards the relation between expletives and licensers.

With this in mind, I summarize the discussion of all expletives. I start by providing a list of properties of \textit{there}, an established expletive discussed in the introduction of this book. Besides the properties mentioned there, other features have been identified. \citet[134, 150--51]{LasnikEtAl2005} point out the following traits in their overview discussion of the relevant construction:\footnote{There are other properties of \textit{there} that are not relevant to the discussion of adjectival inflections and \textit{ein} (nor are they relevant to proprial articles): For instance, a third element – the verb – does not agree with \textit{there}, but with the associate.} (i) \textit{There} does not have a meaning and has to be licensed, (ii) \textit{there} and the associate are in two different positions, (iii) \textit{there} is in the highest site and the associate is in the lowest site, (iv) \textit{there} stands in a certain formal relation with the associate such that both sites are related by movement, and (v) the higher position is made visible, either by \textit{there} or the associate indicating the abstract subject position.

Comparing these properties to those of adjectival inflections and \textit{ein}, we notice that they are basically identical. I would like to point out though that the last point, indicating abstract structure, is often mentioned but does not seem to have attracted much attention (unlike Hypothesis 1b in the current discussion). It is possible that this property can ultimately be related to other phenomena.\footnote{One possible avenue of research might be to relate the indication of abstract structure to language acquisition.} Be that as it may, there are some further characteristics that have been pointed out for established expletives: (vi) Unlike the current cases, the associate moves at LF to license the expletive, (vii) unlike the current cases, the expletive can be null in some languages (e.g., Italian), and (viii) unlike the current cases, there is an alternative construction whereby the associate occupies the higher position and the expletive does not appear at all.

Of particular interest here is the last property in (viii). As far as I am aware, adjectival inflections or \textit{ein} are never in complementary distribution with their licensers as regards the higher position. In my view, this is a difference that might indicate that there are two types of expletives after all: \textit{there} and proprial articles vs. adjectival inflections and \textit{ein}. Having said that, it might turn out that independent factors such as the requirement to spell out CNG features or to overtly indicate operators could explain the obligatory presence of the latter two elements. I leave a more detailed exploration of these ideas for future research.

Finally, the main empirical goal of this book was to provide a more comprehensive discussion of adjectival inflections and \textit{ein}. I believe this pursuit has led to some interesting theoretical proposals. At this point, I cannot claim to have resolved all issues surrounding adjectival inflections and \textit{ein}. However, I hope to have made a good argument for the main hypothesis of this book – these two elements are semantically vacuous in German. Undoubtedly, further investigations and the detailed discussion of other languages will yield new insights into these empirical domains.
