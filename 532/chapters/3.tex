\chapter{Variation and secondary mechanisms}\label{sec:3}

\section{Introduction}\label{sec:3.1}

This chapter and the next continue the discussion of adjectival inflections begun in the previous chapter. While the current chapter focuses on consequences for and extensions of the present proposal, the next chapter discusses implications for other analyses. In more detail, this chapter addresses data involving variation and how this variation can be accounted for in the current system. I examine two types: (i) variation within colloquial Standard German and (ii) variation as regards different regional dialects with special focus on Mannheim German. I provide a brief overview of the relevant cases before I engage in a more detailed discussion.\footnote{It is not an easy task to explain variation, especially optionality, in a formal system. Thus, while certain parts of the discussion below are admittedly somewhat tentative, I discuss new data and offer novel (and I hope, interesting) generalizations and proposals.}

The first type of variation involves both canonical and non-canonical constructions. In canonical DPs, strong, weak, or no inflections are possible on adjectives and determiners. This variation can only occur in certain structures. Strong or weak inflections are possible on the second adjective of two stacked modifiers when a determiner is missing, on adjectives in pronominal DPs, and on certain \textit{der}-words. Strong or no inflections occur on certain other \textit{der}-words (e.g., \textit{d-er} vs. \textit{die} both meaning ‘the/that’).

Importantly, this type of variation only surfaces in specific, well-defined featural contexts in these structures; that is, this variation within colloquial Standard German is restricted to certain combinations of “case + gender” or “case + number”. I argue that there is another mechanism besides Impoverishment: a phonetic rule. Furthermore, the application of Impoverishment Rule 2 is extended, a phonological constraint (avoidance of a hiatus) is held to explain certain unexpected determiner forms, and Impoverishment is proposed not to be triggered by pronominal determiners. In addition, I address the optional inflections on certain adjectives (e.g., \textit{lila} ‘purple’), and I discuss the absence of inflections on some other adjectives (e.g., \textit{prima} ‘great’).

There is also variation within colloquial Standard German in non-canonical structures. In this respect, I discuss the inflectional options on the predeterminer \textit{all-} ‘all’, where in the plural, the ending can be present or absent in all four morphological cases. All these constructions, their restricted variation, and their analyses are briefly summarized in \tabref{tab:3:1} (\textsc{infl} indicates where the variation occurs).

\begin{table}
\small
\caption{Restricted variation in colloquial Standard German}
\label{tab:3:1}
\begin{tabularx}{\textwidth}{lp{19mm}QQ}
\lsptoprule
 Construction & Ending & Restrictions & Analysis\\
\midrule
{ adj + adj-\textsc{infl} + noun} & weak\newline (unexpected) & only in \textsc{dat} \textsc{masc/neut} (in \textsc{sg}) & phonetic rule\\
\tablevspace
{ det-\textsc{infl} + noun} & weak\newline (unexpected) & { only in \textsc{gen} \textsc{masc/neut} (in \textsc{sg}); certain \textit{der}-words} & Impoverishment Rule 2\\
\tablevspace
{ det-\textsc{infl} + noun} & absent & { only in \textsc{nom/acc} in \textsc{fem/pl}; \textit{die} ‘the/that/those’}  & deletion of inflection
(avoid hiatus)\\
\tablevspace
{ det + adj-\textsc{infl} + noun}  & (optional/)\newline absent & { lexically restricted to a few adjectives (e.g., \textit{lila} ‘purple’)} & (optional) deletion of inflection\\
\tablevspace
{ pron + adj-\textsc{infl} + noun} & strong\newline (unexpected) & (expected) weak endings only occur in \textsc{dat} \textsc{sg} and \textsc{nom} \textsc{pl} & ST: pronominal determiner; \newline WK: phonetic rule, analogy\\
\tablevspace
{ predet-\textsc{infl} + noun phrase} & optional & { \textit{all}- ‘all’} & (optional) deletion of inflection in LPP\\
\lspbottomrule
\end{tabularx}
\end{table}

As part of the discussion of the absence of inflections on \textit{die} ‘the/that/those’, I also address the different stem forms of the definite article and the almost-homophonous distal demonstrative (i.e., \textit{de}-\textit{r}, \textit{da}-\textit{s}, \textit{di}-), and I relate the distal and proximal demonstratives to the definite article (the different parsing of \textit{der} ‘the’ as \textit{d-er} vs. \textit{de-r} will be commented on below).

Studying \tabref{tab:3:1}, we note that some variation is found in the oblique cases in masculine and neuter contexts and that other variation occurs in the structural cases in the feminine and plural. Indeed, the workings of the phonetic rule and Impoverishment Rule 2 are restricted to certain constructions and specific featural combinations, making them different from Impoverishment Rule 1. Consequently, I conclude below that Impoverishment Rule 1 is the primary mechanism and that the other two are secondary ones. Based on the discussion of this variation, I review the traditional generalizations of the strong/weak alternation.

The second type of variation concerns regional varieties, here focusing on one dialect – Mannheim German. This variation can be captured in the current system: While it involves Impoverishment Rule 1 as formulated in \chapref{sec:2}, some of the vocabulary insertion rules provided there are deleted here or have a different featural context of application and, as a consequence, are reordered. These adjustments to the vocabulary insertion rules account for the inflectional distribution in this dialect (for some brief remarks on Alemannic German, see \chapref{sec:8}, \sectref{sec:8.2.1}).

More generally, I propose in this chapter that adjectival inflections in German do not receive a homogeneous account but rather are the result of (at least) three different mechanisms, one primary (Impoverishment Rule 1) and two secondary ones (phonetic rule, Impoverishment Rule 2). Furthermore, this chapter provides more evidence that adjectival inflections do not mark (in-)definiteness; that is, they are semantically vacuous elements (Hypothesis 1a).

The chapter is organized as follows. First, I turn to instances involving different inflections on two unpreceded adjectives. \sectref{sec:3.3} is dedicated to the inflectional properties of certain adjectives and certain \textit{der}-words, and to the stem forms of the definite article and the demonstratives. In \sectref{sec:3.4}, I take up the variation of inflections on certain other \textit{der}-words. \sectref{sec:3.5} discusses the variation in pronominal DPs. In \sectref{sec:3.6}, I address the presence or absence of the strong inflection on the predeterminer \textit{all-} ‘all’. \sectref{sec:3.7} returns to the discussion of the traditional generalizations Weak After Strong and the Principle of Monoinflection. In \sectref{sec:3.8}, I provide a comparison between Standard German and the German dialect of Mannheim, and \sectref{sec:3.9} closes the chapter with a summary of the main inflectional patterns in colloquial Standard German and their respective analyses.

\section{Canonical DPs with unexpected weak adjectives: Phonetic rule}\label{sec:3.2}

I start the investigation with the variation that has seen the most interest in the theoretical literature. As such, I begin by discussing noun phrases that lack an overt determiner but involve two adjectives in a row where the first adjective exhibits a strong ending, but the second shows a weak one. This inflectional alternation concerns nasal sounds, and I provide a phonetic rule suggesting that this involves phonetic markedness reduction. In the second subsection, this discussion is extended to indefinite pronoun constructions.

\subsection{Two adjectives without a determiner}\label{sec:3.2.1}

In the previous chapter, we saw examples where two co-occurring adjectives have the same ending, with either both endings being strong \REF{ex:3:1a} or both endings being weak \REF{ex:3:1b}.

\ea%1
    \label{ex:3:1}
\ea\label{ex:3:1a}
\gll frisch-er schwarz-er Kaffee\\
fresh-\textsc{st}  black-\textsc{st}     coffee.\textsc{masc}\\
\glt ‘fresh black coffee’
\ex\label{ex:3:1b}
\gll d-er     frisch-e schwarz-e Kaffee\\
the-\textsc{st} hot-\textsc{wk}  black-\textsc{wk}  coffee.\textsc{masc}\\
\glt ‘the fresh black coffee’
\z
\z

Recall that \textit{der} ‘the’ in \REF{ex:3:1b} involves [+DEF], a context where [+D] triggers Impoverishment. With the determiner moving from ArtP to DP in a successive-cyclic fashion, the CNG features of the two adjectives undergo Impoverishment and are spelled out as weak. As for \REF{ex:3:1a}, Impoverishment does not occur, and the underlying features are spelled out as the strong endings.

There is one much-discussed exception to the pattern in \REF{ex:3:1a} above: In the dative masculine (and neuter), there is the option of the second adjective exhibiting a weak ending. Comparing \REF[a-b]{ex:3:2}, we observe that \REF{ex:3:2a} exhibits the expected pattern but that \REF{ex:3:2b} shows two adjectives in a row that have different endings. The latter is often referred to as non-parallel distribution. This is in stark contrast to all the other patterns I discussed above. As is clear from \REF[c-d]{ex:3:2}, patterns with a weak ending on the first adjective are not possible (also \citealt{Nübling2011}, \citealt{Sahel2021}), independently of whether the second adjective is strong or weak. I call the inflectional option of -\textit{em} vs. -\textit{en} nasal alternation. Note also that contrasting the examples involving the different nasal inflections yields less sharp grammaticality judgments (\% indicates variation with speakers; see \tabref{tab:3:2} below).\footnote{\citet[182]{Nübling2011} states that the coexistence of the parallel and non-parallel adjective strings has a long history, possibly several hundred years. Note that this co-occurrence seems to have existed already in ENHG (see \citealt{EbertEtAl1993}: 201-02). However, it is hard to tell when this phenomenon started as there is some independent variation with adjectives in the earlier stages of German (e.g., \citealt{Peter2013}, \citealt{Sahel2021}). \citet{Sahel2021} observes that \REF{ex:3:2b} is the only non-parallel string that continues occurring after the 18\textsuperscript{th} century.  Also, \citet[183]{Nübling2011} finds in her corpus study that the parallel string occurs slightly more often (57\%) and that all classes of adjectives have this variation (albeit adjectives in coordinations much less so; more on this in Footnote \ref{foot:3:5}). Interestingly, there is a slight correlation such that adjective classes that occur closer to the noun (in the nominal extended projection) tend to be inflected weak more often than those further away (her page 187), provided they are preceded by a (strong) adjective. The corpus studies by \citet{Peter2013} and \citet{Sahel2021} offer partially different results, but both Peter and Sahel point out that non-parallel strings seem to be on the rise.}

\judgewidth{*?}
\ea\label{ex:3:2}
  \ea[\%]{\label{ex:3:2a}
  \gll  mit   frisch-em schwarz-em Kaffee\\
    with fresh-\textsc{st}   black-\textsc{st}      coffee.\textsc{masc}\\
    \glt ‘with fresh black coffee’
    }
  \ex[\%]{\label{ex:3:2b}
    \gll   mit   frisch-em schwarz-en Kaffee\\
    with fresh-\textsc{st}   black-\textsc{wk}    coffee.\textsc{masc}\\
    \glt ‘with fresh black coffee’
    }
  \ex[*?]{\label{ex:3:2c}
    \gll mit   frisch-en schwarz-em Kaffee\\
    with fresh-\textsc{wk} black-\textsc{st}      coffee.\textsc{masc}\\
    }
  \ex[*?]{\label{ex:3:2d}
    \gll   mit   frisch-en schwarz-en Kaffee\\
    with fresh-\textsc{wk} black-\textsc{wk}    coffee.\textsc{masc}\\
    }
  \z
\z
\judgewidth{\%}

\citet[290]{Sternefeld2004} points out that the alternation in \REF[a-b]{ex:3:2} is also possible with Saxon Genitives.

\ea%3
    \label{ex:3:3}
  \ea\label{ex:3:3a}
  \gll Johanns  gut-em   alt-em Wein\\
  Johann’s good-\textsc{st} old-\textsc{st} wine.\textsc{masc}\\
  \glt ‘Johann’s good old wine’
  \ex\label{ex:3:3b}
  \gll Johanns  gut-em   alt-en   Wein\\
  Johann’s good-\textsc{st} old-\textsc{wk} wine.\textsc{masc}\\
  \glt ‘Johann’s good old wine’
\z
\z

As in the canonical cases (e.g., \textit{d-em frisch-en Kaffee} ‘the-\textsc{st} fresh-\textsc{wk} coffee’), the strong ending precedes the weak one yielding a left-to-right asymmetry in \REF{ex:3:2b} and \REF{ex:3:3b}. Notice also that the weak ending in \REF{ex:3:2b} and similar instances below cannot mark definiteness.

\citet[38]{Duden2007} claims that the weak adjective occurs most frequently when the second adjective and noun form a whole concept (“Gesamtbegriff”). \citet{Nübling2011} suggests that the weak adjective strengthens the tendency to mark the left bracket in the noun phrase, the right bracket being the head noun (cf. the sentence bracket in German). However, \citet[9-14]{Sahel2021} argues convincingly that there is no semantic or functional difference between the two variants. Having said that, these patterns do have a number of peculiar properties. First, as just mentioned, the grammaticality judgments are less sharp than in the cases discussed in the previous chapter. Second, these inflectional distributions are restricted to the two instances above (i.e., dative masculine/neuter). For instance, a weak ending on a second adjective is not possible in the feminine gender.

\ea%4
    \label{ex:3:4}
  \ea\label{ex:3:4a}
  \gll mit   gut-er    rot-er  Sauce\\
  with good-\textsc{st} red-\textsc{st} sauce.\textsc{fem}\\
  \glt ‘with good red sauce’
  \ex[*]{\label{ex:3:4b}
  \gll  mit   gut-er    rot-en   Sauce\\
  with good-\textsc{st} red-\textsc{wk} sauce.\textsc{fem}\\
  }
\z
\z

Third, as noted in \citet{Roehrs2009b}, different authors have reported different possibilities for \REF[a-b]{ex:3:2}. I refer to this inter-speaker variation as dialects. Considering the two inflectional options in \tabref{tab:3:2} (Column 1), \citeauthor{Gallmann1996} (\citeyear{Gallmann1996}: 296, \citeyear{Gallmann2004}: 156), \citet[119]{Schlenker1999}, and \citet[53]{Demske2001} describe Dialect 1 where a strong or a weak ending is equally possible on the second adjective. G. \citet[139]{Müller2002a} discusses Dialect 2 where the strong ending on the second adjective is preferred. Finally, \citet{Schlenker1999} reports that some speakers have Dialect 3 where the weak ending on the second adjective is strongly preferred. Dialect 4 does not exist -- all speakers allow patterns of either Dialect 1, 2, and/or 3. In other words, all speakers have at least one of the two inflectional distributions in their language.

\begin{table}
\caption{Different dialects of multiple adjectives without determiners in the dative masculine/neuter}
\label{tab:3:2}
\begin{tabularx}{\textwidth}{Xcccc}
\lsptoprule
String & Dialect 1 & Dialect 2 & Dialect 3 & Dialect 4\\
\midrule
Adj+\textit{em} Adj+\textit{em} & ${\surd}$ & ${\surd}$ & ?? & -\\
Adj+\textit{em} Adj+\textit{en} & ${\surd}$ & ? & ${\surd}$ & -\\
\lspbottomrule
\end{tabularx}
\end{table}

The weak ending -\textit{en} cannot be the result of Impoverishment. Recall that Impoverishment is a local process that occurs within a phrase (AgrP). Non-local effects are brought about by determiner movement (Impoverishment Rule 1) or a specific featural context (Impoverishment Rule 2). Both options affect all adjectives in the same way. In the cases discussed here, the weak ending occurs only on the second adjective and seems to be parasitic on the presence of the first adjective. I return to this statement in \sectref{sec:3.2.2}.

  An account of this alternation should answer the questions as to why this is only possible with adjectives ending in -\textit{em} but not, for instance, with adjectives in -\textit{es} instead or why this alternation is not possible with adjectives in -\textit{em} and -\textit{es}. In this regard, \citet[19-20]{Sahel2021} points out that -\textit{m} and -\textit{n} have the most similarities as compared to all other strong/weak inflectional pairs (e.g., -\textit{er} vs. -\textit{en} as in \REF[a-b]{ex:3:4}). In order to explain the above properties, I proposed in \citet{Roehrs2009b} that the alternation in \REF[a-b]{ex:3:2} and \REF[a-b]{ex:3:3} follows from a phonetic rule, according to which the inflection on the second adjective is changed from -\textit{m} to -\textit{n} (+ indicates a morpheme boundary; \# signifies a word boundary).

\ea%5
    \label{ex:3:5}

     Rule (preliminary version):

m  $\rightarrow$  n  / […]\textsubscript{A} […]\textsubscript{A} +{\longrule}{\longrule}\#
\z

This rule turns a labial sound into its coronal (/alveolar) counterpart; that is, the place of articulation of this inflection is altered. Note in this regard that \citet[165]{Wiese1996a} states that “alveolar is taken to be the default place of articulation for consonants”. Utilizing this, R. Wiese formulates rules where coronal is the default value (his page 219). In a similar vein,  \citet[35-42]{deLacy2006} argues that coronal is a less marked place of articulation than labial or dorsal. Specifically, -\textit{n} is considered the unmarked nasal sound as compared to -\textit{m}. I suggest that the rule above is a reflex of markedness reduction, where a more marked sound is changed into a less marked sound (e.g., \citealt{deLacy2006}: 23, 75-76). For similar explanations of the data in \REF[a-b]{ex:3:2}, see also \citet[78-79]{Evans2019} and \citet{Sahel2021}.\footnote{In an earlier version of this work, I took the rule in \REF{ex:3:5} to involve phonetic simplification, presumably a reflex of ease of articulation. Joe Salmons (p.c.) pointed out to me that there are issues with such an idea (see \citealt{Salmons2021}: 39-41, also references cited therein). He and Tracy Hall suggested to me that reduction in markedness might be more appropriate. In this vein, note that \citet[156]{Gallmann2004} formulates a phonological markedness constraint in his optimality-theoretic proposal that bans German words that end in schwa + /m/ (cf. also \citealt{GunkelEtAl2017}: 1308).  Note also that -\textit{m} and -\textit{n} are diachronically and dialectally related. \citet[120-22]{BrauneReiffenstein2004} point out for OHG that diachronically -\textit{m} changes to -\textit{n} in suffixal endings (if \textit{n} changes, it tends to change to \textit{l}). \citet[146]{PaulEtAl1989} show for MHG that final -\textit{m} also changes to -\textit{n}, except with strong adjectival endings. In certain dialects, though, specifically in Middle Franconian, adjectival -\textit{em(e)} is changed to -\textit{en} in dative masculine/neuter contexts (\citealt{PaulEtAl1989}: 211-12). ENHG also shows the relatedness of these two elements with certain nouns changing their final sound (e.g., \textit{besem} > \textit{besen} ‘broom’), but there is also general variation with articles, pronominal elements, and adjectives (\citealt{EbertEtAl1993}: 140, 192-93). With this in mind, we might expect that these two sounds are also dialectally related. This is borne out in that contemporary Southern Hessian dialects retain \textit{Besem} for \textit{Besen} ‘broom’ \citep[433]{Schirmunski2010}. In my view, this diachronic and dialectal relatedness of -\textit{m} and -\textit{n} strengthens the proposal that the nasal alternation is a phonetic/phonological phenomenon (also \citealt{Peter2013}, \citealt{Sahel2021}: Footnote 6). I thank Tracy Hall, Joe Salmons, and Laura Smith for help with this section including providing references.}

  Returning to \tabref{tab:3:2}, the application of the rule is optional in Dialect 1,\linebreak “costly” in Dialect 2, and obligatory in Dialect 3.\footnote{The grammaticality judgments in \tabref{tab:3:2} are taken from the original sources. It is not clear to me if the degree of the markedness/ungrammaticality of the respective bad patterns in Dialect 2 or 3 is the same. If the relevant case in Dialect 2 were marked more than “?”, we could suggest that the rule in \REF{ex:3:5} simply does not apply in that dialect at all.} Thus, this rule accounts for the unexpected weak adjectives in canonical DPs of Dialects 1 and 3. Given this rule, non-nasal inflections are immediately excluded, explaining why this pattern is so restricted in occurrence. The left-to-right asymmetry follows from the way the rule above is stated – only the second adjectives may undergo the phonetic change.\footnote{\label{foot:3:5} Besides \citet{Nübling2011}, \citet[194]{Peter2013} points out that parallel inflections are much more frequent in coordinations, including two adjectives separated by a comma. Specifically, parallel inflections occur 39\% in non-coordinations vs. 79\% in coordinations. Björn Köhnlein (p.c.) makes an interesting observation in this regard. In non-coordinated noun phrases, main phrasal stress usually goes to the head noun, but there is a secondary stress on the initial element (this is probably determined rhythmically). In the current cases, the initial element is the first adjective. This means that the second adjective~has the lowest degree of prominence. This low degree of prominence may facilitate the reduction from -\textit{m} to -\textit{n}. In contrast, if there is a pause between the two adjectives as in coordinations, then the second adjective is not in a weak prosodic position but tends to have its own stress. This may make the retention of -\textit{m} more likely.}

There are two possible alternatives: First, the left-to-right asymmetry might also have to do with easier parsing because a disambiguating, strong ending occurs first in the nominal string. Matthias Schlesewsky (p.c.) points out to me that this claim about more effective processing is in agreement with work by John Hawkins. In particular, \citeauthor{Hawkins1994} (\citeyear{Hawkins1994}: 404-05; \citeyear{Hawkins2004}: 49, 92-93) explains the left-to-right asymmetry seen with the examples above by general processing advantages (for an early intuition in this regard, see \citealt{Esau1973}: 139). As a second alternative, we could follow G. \citet[140]{Müller2002a} claiming that this asymmetry has to do with analogy with the canonical pattern \textit{d-em nett-en Mann} ‘the-\textsc{st} nice-\textsc{wk} man’.

The cases in the next subsection present a challenge to both of these alternatives. Indeed, what often goes unmentioned is that this nasal alternation is more general; for instance, it can also be found with indefinite pronoun constructions (and pronominal DPs, see \sectref{sec:3.5}). In my view, these are related phenomena that should be discussed in the same context and receive the same account (note again that all other authors only discuss the contrast in \REF{ex:3:2}, with the qualification that \citealt{Sternefeld2004} also discusses \REF{ex:3:3}). Unlike the two alternatives, the phonetic rule above is able to capture the above-mentioned cases too once it is slightly modified.

\subsection{Indefinite pronoun constructions revisited}\label{sec:3.2.2}

In \chapref{sec:2}, \sectref{sec:2.3.2}, I discussed indefinite pronoun constructions. To extend this discussion, consider cases involving the pronoun \textit{etwas} ‘something’ and a following adjective. In either the nominative/accusative \REF[a-b]{ex:3:6} or the dative \REF[c-d]{ex:3:6}, there is no variation – only a strong adjective is possible.

\ea%6
    \label{ex:3:6}
\ea\label{ex:3:6a}
\gll etwas                  ander-es\\
something.\textsc{neut} different-\textsc{st}\\
\glt ‘something different’
\ex[*]{\label{ex:3:6b}
\gll   etwas                  ander-e\\
something.\textsc{neut} different-\textsc{wk}\\
}
\ex\label{ex:3:6c}
\gll mit   etwas                  ander-em\\
with something.\textsc{neut} different-\textsc{st}\\
\glt ‘with something different’
\ex[?*]{\label{ex:3:6d}
\gll   mit   etwas                  ander-en\\
with something.\textsc{neut} different-\textsc{wk}\\
}
\z
\z

It was proposed in the previous chapter that the nominative string \textit{jemand anderer} ‘someone different’ is analyzed as in \figref{figex:3:7}. I now add \REF{ex:3:6a} to the analysis (recall that IPRP stands for Indefinite Pronoun Restrictor Phrase; for the internal makeup of indefinite pronouns, see also Footnote \ref{foot:3:7}).

\begin{figure}
	\caption{Indefinite pronoun construction}
	\label{figex:3:7}
	\begin{forest}
		[DP
			[D [\textit{je}\\\textit{et},tier=bottom]]
			[IPRP
				[IPRP
					[IPR[\textit{-mand}\\\textit{-was},tier=bottom]]
					[NP [\textit{e}$_N$\\\textit{e}$_N$]]
				]
				[ModP
					[Mod]
					[AgrP [{[\textit{anderer e}$_N$]}\\{[\textit{anderes e}$_N$]}]]
				]
			]
		]
	\end{forest}
\end{figure}

Given right adjunction to the pronoun, the adjective is only expected to have a strong ending. However, \citet[491]{Duden2007} points out that there is variation of the adjective in the dative when the adjective follows \textit{jemand} ‘someone’ and \textit{niemand} ‘no one’, and the same goes for the two pronouns themselves.\footnote{There is also variation with indefinite pronouns in the accusative (e.g., \textit{jemanden} vs. \textit{jemand}), but the following adjective always ends in the expected inflection -\textit{en}. Furthermore, as far as I know, there is no variation with the dative pronoun \textit{ihm} ‘him’; in other words, this pronoun cannot appear as \textit{ihn} in dative contexts. In addition, this element is not part of the discussion of inflections on adjectives, as third-person pronominal elements cannot be followed by adjectives (\sectref{sec:3.5.1}). Given these remarks, the cases mentioned in this footnote are not relevant to the discussion of the nasal alternation.} I start by discussing the pronoun \textit{jemand} in more detail. This is followed by addressing the inflections on the adjective.

  Although \textit{jemand} ‘someone’ is the first element in the noun phrase, this pronoun shows the nasal alternation \REF[a-b]{ex:3:8}. In addition, it can also occur without an ending \REF{ex:3:8c}.

\ea%8
    \label{ex:3:8}
\ea\label{ex:3:8a}
  \gll mit jemand-em\\
  with someone.\textsc{masc}-\textsc{st}\\
  \glt ‘with someone’
  \ex\label{ex:3:8b}
  \gll mit  jemand-en\\
  with someone.\textsc{masc}-\textsc{wk}\\
  \glt ‘with someone’
  \ex\label{ex:3:8c}
  \gll mit jemand\\
  with someone.\textsc{masc}\\
  \glt ‘with someone’
  \z
\z

The same options can be found with \textit{niemand} ‘no one’. It is clear that this inflectional alternation is not due to the phonetic rule proposed above. On the one hand, this element occurs in first position; on the other hand, it is not an adjective. Syntactically, \textit{jemand} can appear in argument position. Morphologically, it can have a strong, a weak, or no ending at all. Rather, \citet{Roehrs2009b} argues that \textit{jemand} belongs to three different morphological paradigms. Besides the older declensions that are similar to those of strong nouns and determiners, there seems to be a third paradigm in the progress of developing. In the latter case, \textit{jemand} has features of a weak noun where the ending -\textit{en} is generalized throughout the non-nominative cases. The three paradigms are summarized in \tabref{tab:3:3}.\footnote{\label{foot:3:7}Note that \citet[491]{Duden2007} claims that dative pronouns in -\textit{en} are not part of the standard language. It is clear though that such pronouns occur in the colloquial standard language. Notice also that the different inflections in \tabref{tab:3:3} presumably indicate different internal makeups of the pronoun. For instance, \textit{jemand} may involve a determiner component \REF{ex:3:7:1a}, but it may also involve a determiner component with an indefinite pronoun restrictor where the latter has features of a weak noun \REF{ex:3:7:1b} or strong noun \REF{ex:3:7:1c}.
	\ea
	\ea\label{ex:3:7:1a}  [\textsubscript{D} \textit{jemand}]-em
	\ex\label{ex:3:7:1b}  [\textsubscript{D} \textit{je}] + [\textsubscript{IPR} \textit{mand}]-en
	\ex\label{ex:3:7:1c}  [\textsubscript{D} \textit{je}] + [\textsubscript{IPR} \textit{mand}]
	\z
	\z
	
	The analyses in \REF{ex:3:7:1b} and \REF{ex:3:7:1c} indicate how I interpret the terms weak noun and strong noun in \tabref{tab:3:3} – the latter are grammaticalized nouns under IPR in \figref{figex:3:7}.}

\begin{table}
\caption{Different paradigms of the indefinite pronoun \textit{jemand}}
\label{tab:3:3}
\begin{tabularx}{.85\textwidth}{Xlll}
\lsptoprule
 & Strong noun & Determiner & Weak noun\\
 \midrule
Nominative & jemand & jemand (/w-er) & jemand\\
Accusative & jemand & jemand-en & jemand-en\\
Dative & jemand & jemand-em & jemand-en\\
Genitive & jemand-s & jemand-(e)s & jemand-en\\
\lspbottomrule
\end{tabularx}
\end{table}

  Adding an adjective after the pronoun retains the variation on the pronoun itself. In addition, the adjective also alternates between a strong and weak ending yielding six different options. Below, I provide my own grammaticality judgments.
\judgewidth{(?)}

\ea%9
    \label{ex:3:9}
\ea[(?)]{\label{ex:3:9a}
\gll mit   jemand-em             ander-em\\
with someone.\textsc{masc}-\textsc{st} different-\textsc{st}\\
\glt ‘with someone different’
}
\ex[]{\label{ex:3:9b}
\gll mit   jemand-em             ander-en\\
with someone.\textsc{masc}-\textsc{st} different-\textsc{wk}\\
\glt ‘with someone different’
}
\z
\z

\ea%10
    \label{ex:3:10}
\ea[(?)]{\label{ex:3:10a}
\gll   mit   jemand-en               ander-em\\
with someone.\textsc{masc}-\textsc{wk} different-\textsc{st}\\
\glt ‘with someone different’
}
\ex[(?)]{\label{ex:3:10b}
\gll   mit   jemand-en               ander-en\\
with someone.\textsc{masc}-\textsc{wk} different-\textsc{wk}\\
\glt ‘with someone different’
}
\z
\z

\ea%11
    \label{ex:3:11}
  \ea[]{\label{ex:3:11a}
  \gll mit   jemand              ander-em\\
  with someone.\textsc{masc} different-\textsc{st}\\
  \glt ‘with someone different’
  }
  \ex[(?)]{\label{ex:3:11b}
  \gll   mit   jemand              ander-en\\
  with someone.\textsc{masc} different-\textsc{wk}\\
  \glt ‘with someone different’
  }
  \z
\z
\judgewidth{*}

All these patterns occur naturally but with different frequencies. \tabref{tab:3:4a} presents the numeric results of an informal Google-search involving three dative prepositions: \textit{mit} ‘with’, \textit{von} ‘from’, and \textit{bei} ‘at’ (note that while I checked quite a few examples for relevance, the totals were too large to examine all the individual instances). While these numbers are, admittedly, not precise, the tendencies are clear. Whereas all patterns above occur, a weak adjective is more frequent with a pronoun involving an inflection, be it -\textit{em} or -\textit{en} (cf. \REF{ex:3:9b}, \REF{ex:3:10b}). In contrast, a strong adjective is more frequent with a pronoun involving no inflection (cf. \REF{ex:3:11a}). Notice also that the latter is the most frequent pattern overall (more on these points below). Finally, it is worth pointing out that the positive pronoun, followed by a forward slash sign in \tabref{tab:3:4a}, is more frequent than its negative counterpart.

\begin{table}
\caption{\label{tab:3:4a} Numeric results of IPC in the dative identified by Google (October 7, 2020)}
\begin{tabularx}{.9\textwidth}{lY@{ / }rY@{ / }r}
\lsptoprule
Preposition and pronoun & \multicolumn{2}{c}{anderem} & \multicolumn{2}{c}{anderen}\\
\midrule
{mit jemandem/niemandem} & 35\,000 & 425 & 95\,000 & 5\,000\\
{von jemandem/niemandem} & 8\,000 & 711 & 63\,000 & 50\,000\\
{bei jemandem/niemandem} & 3\,000 & 9 & 10\,000 & 347\\
\tablevspace
{mit jemanden/niemanden} & 9\,000 & 238 & 75\,000 & 3\,000\\
{von jemanden/niemanden} & 7\,000 & 350 & 32\,000 & 16\,000\\
{bei jemanden/niemanden} & 1\,000 & 109 & 14\,000 & 503\\
\tablevspace
{mit jemand/niemand}     & 755\,000 & 129\,000 & 408\,000 & 69\,000\\
{von jemand/niemand}     & 762\,000 & 195\,000 & 362\,000 & 19\,000\\
{bei jemand/niemand}     & 141\,000 & 19\,000 & 108\,000 & 816\\
\lspbottomrule
\end{tabularx}
\end{table}

In addition to the Google-search, I examined these constructions in \textit{Datenbank für Gesprochenes Deutsch} (DGD; Database of Spoken German). The results are provided in \tabref{tab:3:4b}. Note that \tabref{tab:3:4b} is similar to \tabref{tab:3:4a} above. However, the prepositions were removed from the first column and added to the relevant cells in parentheses under the total numbers. The number of occurrences of these prepositions in these constructions is provided behind the preposition (the one instance of R after the preposition \textit{von} ‘of’ indicates a speaker in the corpus \textit{Australiendeutsch}, who is probably the interviewer, but no information is available here; IO stands for indirect object). Since the number of hits involving the two indefinite pronouns and the adjective \textit{ander}- ‘different’ was fairly low, I extended the search to other adjectives. These cases are provided in the third row of each form of the two pronouns (as above, the forward slash sign separates the results of the positive pronoun from those of the negative pronoun). I consider the results in more detail.

While all three different forms of the positive and negative indefinite pronouns (strong, weak, uninflected) can be found in dative contexts when the adjective is absent, this differs when the adjective is present. For instance, in this search, I did not find any examples of a pronoun with a weak ending followed by an adjective with a strong ending. With that in mind, note though that the numeric distributions in \tabref{tab:3:4b} are similar to those in \tabref{tab:3:4a}: Inflected pronouns occur, with one exception, with weak adjectives, but uninflected pronouns occur most frequently with strong adjectives (18 strong vs. 5 weak). Also similar to above, examples involving uninflected pronouns are more frequent than those with inflected pronouns.

\begin{table}\small
\caption{\label{tab:3:4b} Numeric results of IPC in the dative identified in DGD (October 10, 2024)}
\begin{tabularx}{\textwidth}{lQQ}
\lsptoprule
Pronoun & {anderem} & {anderen}\\
\midrule
{jemandem/} &   & 2 / 1 \\
niemandem & & (bei 2) / (aus)\\
& {mit jemandem Deutschem} & {vor jemandem Fremden}\\
\midrule
{jemanden/} & & {1 /-}\\
niemanden & & ({von} R) /-\\
& & {mit jemanden Fremden}\\
\midrule
{jemand/} & {12 /-} & {1 /-}\\
niemand         & ({bei} 2; {mit} 4; {von} 4; {zu}; IO) /- & ({mit}) /-\\
                & {mit jemand neuem/ Deutschem/ Fremdem} (2); {von jemand Fremdem/ fremdem} & {mit jemand andern} (3); {von jemand fremden}\\
\lspbottomrule
\end{tabularx}
\end{table}

Note that all constructions involving weak adjectives were checked.\footnote{There were two additional examples in the dative with a weak adjective (\textit{bei jemand anderen}, \textit{jemand anderen}). These were uttered by speakers whose first language is not German, and they were not counted in \tabref{tab:3:4b} above.} There is one example where the pronoun and the adjective both have a strong inflection. This instance was uttered by a speaker of Swiss German from Zurich. Finally, the weak form \textit{andern} ‘different’ (where schwa is missing in the inflection) was also found in another weak context (with at least one speaker). Now, while the numbers are not large, there are two types of instances where the distribution of the strong and weak adjectives can be directly compared.

First, the weak adjective \textit{fremd-en} ‘strange-\textsc{wk}’ occurs one time each with strong \textit{jemandem}, weak \textit{jemanden}, and uninflected \textit{jemand}. In other words, the weak adjective occurs independently of the inflection on \textit{jemand}. In contrast, the strong adjective \textit{fremd-em} ‘strange-\textsc{st}’ occurs four times with uninflected \textit{jemand}. In fact, there is a minimal pair where \textit{jemand} occurs either with weak \textit{fremd}-\textit{en} or strong \textit{fremd}-\textit{em}. Note that the latter scenario is more general with the second type of instance, which involves \textit{jemand} and \textit{ander}- ‘different’. In these minimal pairs, (uninflected) \textit{jemand} occurs several times each with weak \textit{ander-(e)n} (4 times) and with strong \textit{ander}-\textit{em} (12 times).

Analyzing the results of \tabref{tab:3:4a} and \ref{tab:3:4b}, it is clear that the adjectives exhibit the nasal alternation basically independently of whether the pronoun has a strong, a weak, or no inflection. For the cases discussed in \sectref{sec:3.2.1}, this means that the second adjective in -\textit{en} should not be taken to be parasitic on a preceding adjective ending in -\textit{em} (for such an idea, see \citeauthor{Murphy2018}’s 2018 proposal of partial copying). Rather, the weak adjective depends on a preceding element more generally. Furthermore, indefinite pronoun constructions have a structure different from canonical DPs. This suggests that the nasal alternation is independent of a specific structure. If these cases are indeed related by the nasal alternation, then no strong morpho-syntactic claims should be made on the basis of the data in \sectref{sec:3.2.1}. Note in this regard that \citet{Schlenker1999} argues for top-down derivations and \citet[288-90]{Sternefeld2004} for recursive DP-levels.

\newpage
  As seen above, the word preceding the (lower) adjective may not only be an adjective but also a pronoun. What these elements have in common is that they are nominal in nature. I modify the phonetic rule above slightly (the feature [+N] stands for nominal).

\ea%12
    \label{ex:3:12}
      Rule (final version):\\
	m  $\rightarrow$  n  / […]\textsubscript{[+N]} […]\textsubscript{A} +{\longrule}{\longrule}\#
\z

In words, an adjective preceded by another nominal element (in the same DP) may occur with the inflection -\textit{en}. Notice that the rule does not apply to adjectives that have already undergone Impoverishment due to a preceding determiner. The latter has changed the ending on the adjective from -\textit{em} to -\textit{en} independently. The varying inflections on the pronouns are due to the three paradigms above.\footnote{As my focus is on adjectives, I do not reformulate the three paradigms of the indefinite pronouns as vocabulary insertion rules.}

  At first glance, the different frequencies in \tabref{tab:3:4a} and \ref{tab:3:4b} seem to confirm the left-to-right asymmetry noted in \sectref{sec:3.2.1} (i.e., a strong inflection precedes a weak one). While \REF{ex:3:13a} below is generally preferred over \REF{ex:3:13b}, it is worth repeating that the strong ending on the adjective is preferred over the weak one in \REF{ex:3:13a} but vice versa in \REF{ex:3:13b}. In other words, recalling the three paradigms of \textit{jemand} ‘someone’ in \tabref{tab:3:3}, a strong inflection is preferred on the first element that can take an inflection (the percentage sign indicates a less frequent option, which I interpret as a less-preferred string).

\ea%13
    \label{ex:3:13}
\ea\label{ex:3:13a}
\gll jemand ander-em/\textsuperscript{\%}-{en}\\
someone.\textsc{masc} different-\textsc{st}/-\textsc{wk}\\
\glt ‘someone different’

\ex[\%]{\label{ex:3:13b}
\gll jemand-em/-en ander-en/\textsuperscript{\%}-em\\
someone.\textsc{masc}-\textsc{st}/-\textsc{wk} different-\textsc{wk}/-\textsc{st}\\
\glt ‘someone different’
}
\z
\z

However, unlike in \sectref{sec:3.2.1}, the different frequencies seen with the indefinite pronoun construction are only tendencies, rather than clearcut left-to-right asymmetries. In fact, an element with a weak ending can precede one with a strong ending (e.g., \textit{jemand-en ander-em} ‘someone-\textsc{wk} different-\textsc{st}’), and a weak adjective can occur despite the fact that there is no element with a strong inflection preceding it (e.g., \textit{jemand(-en) ander-en} ‘someone(-\textsc{wk}) different-\textsc{wk}’). Recall that these inflectional distributions are not possible with two adjectives in a row. This means that explanations involving parsing or analogy as mentioned in \sectref{sec:3.2.1} cannot be the whole story. I submit that the rule in \REF{ex:3:12} is better able to capture these inflectional distributions. Indeed, this rule is extended in \sectref{sec:3.5} to pronominal DPs like \textit{mir nett-em/\textsuperscript{\%}}\textit{-en Studenten} ‘me (nice-\textsc{st/}\textsc{\textsuperscript{\%}}\textsc{-wk} student)’.

To sum up, we arrive then at a first secondary mechanism – a phonetic rule – to explain inflectional variation, here accounting for unexpected weak adjectives in canonical DPs. Note that the weak endings here occur in a combination of certain featural and lexical contexts (i.e., an adjective in the genitive masculine/neuter is preceded by another nominal element). Next, I turn to some special cases that share a phonological constraint but are restricted to certain lexical items.

\section[Canonical DPs and adjectives/determiners with optional/no inflections]{Canonical DPs involving adjectives with optional or no inflections and definite   determiners with no inflections}\label{sec:3.3}

In this section, I discuss two types of variation. I address the special inflectional behavior of certain adjectives, and I analyze some uninflected forms of the definite article and its related distal demonstrative. On the basis of the second discussion, I relate the distal and proximal demonstratives to the definite article in terms of their decompositions. More generally, I propose that all these cases have something in common: They exhibit the avoidance of a hiatus – the occurrence of two adjacent vowels in a prosodic word.

  In \chapref{sec:1}, \sectref{sec:1.3.1.1}, I briefly illustrated some adjectives exhibiting special inflectional behaviors. Recall that there are three types of cases. First, the adjectives \textit{lila} ‘purple’ and \textit{rosa} ‘pink’ participate in the strong/weak alternation, but the inflection is optional: It can be absent, and it can be present \REF[a-b]{ex:3:14}. In order to account for the presence of -\textit{n}-, I assume for convenience that the latter consonant has been epenthesized.\footnote{Björn Köhnlein (p.c.) points out that -\textit{n}- is not a typical epenthetic consonant (the glottal stop being the usual element).} Inflected forms without \textit{n}-epenthesis are not possible \REF{ex:3:14c}.

\ea%14
    \label{ex:3:14}
  \ea\label{ex:3:14a}
  \gll das lila(-n-e)      Kleid\\
  the purple-n-\textsc{wk} dress.\textsc{neut}\\
  \glt ‘the purple dress’
  \ex\label{ex:3:14b}
  \gll ein lila(-n-es)   Kleid\\
        a    purple-n-\textsc{st} dress.\textsc{neut}\\
  \glt ‘a purple dress’
  \ex[*]{\label{ex:3:14c}
  \gll ein lila-es      Kleid\\
  a    purple-\textsc{st} dress.\textsc{neut}\\
  }
  \z
\z

There are two ways to analyze this optionality. On the one hand, we could assume that InflP of the extended projection of the adjective is built optionally: If InflP is not projected, we obtain the uninflected forms; if InflP is projected, we get the inflected forms, provided \textit{n}-epenthesis occurs. On the other hand, we could suggest that InflP is built in every instance. In this scenario, the inflection is deleted yielding the uninflected forms, or the inflection is not deleted, provided \textit{n}-epenthesis occurs.\footnote{\label{foot:3:11}This deletion could be made more formal by assuming that Impoverishment deletes either all the features of the Infl head, or it deletes a feature that all inflections have in common (perhaps a nominal feature like [+N]).}  Note that either option involves \textit{n}-epenthesis and results in the avoidance of a hiatus.

Turning to the second case, certain adjectives never take inflections. This holds for \textit{prima} ‘great’, \textit{sexy} ‘sexy’, \textit{super} ‘super’, and others \citep[256-57]{Duden1995}. Note that \textit{n}-epenthesis cannot rescue these cases (e.g., *\textit{prima-n-es}, *\textit{sexy-n-es}). Observe that many of these cases seem to be borrowings from other languages. Following the idea of a reviewer, I assume that these cases are not fully integrated into the language yet. To make this idea more formal, I suggest that InflP is not projected in these instances (note in this regard that the emergence of InflP on top of an adjective takes time; see \citealt{SappRoehrs2016} for the discussion of \textit{viele} ‘many’ changing from the uninflected noun \textit{vil} in OHG to the inflected quantificational adjective emerging in the 16\textsuperscript{th} century).

Third, toponymic formations where adjectives are derived from place names by adding -\textit{er} never take inflections either, for instance \textit{Berlin-er-(*es)} ‘(from) Berlin’. Again, there are two options to analyze this. We could suggest that either this derivational suffix blocks the projection of InflP, or alternatively, we could assume that the derivational ending -\textit{er} does double duty as an invariant inflectional element under Infl. Note in this regard that this -\textit{er} can license a genitive noun phrase, for instance, \textit{der Verkauf Berlin-er Bier-es} ‘the sale of Berlin-\textsc{infl} beer-\textsc{gen}’ \citep{Fuhrhop2003} (for more discussion of the Genitive Rule, see \sectref{sec:3.4}).

\newpage
Notice that these three sets of cases have other intriguing properties that may help us decide between the different analytical options.\footnote{I have not investigated adjectives in ellipsis contexts in detail. As far as I am aware, all ordinary adjectives have the same inflections in ellipsis and non-ellipsis contexts. However, the three types of adjectives discussed in the main text are also special in ellipsis contexts. First, \citet[101]{Fanselow1988} observes that the inflectional ending with \textit{lila} ‘purple’ and \textit{rosa} ‘pink’ is obligatory in ellipsis contexts.
	\ea 
	\gll ein lila*(-n-es)\\
	a    purple-n-\textsc{st}\\
	\glt ‘a purple one’
	\z
	
Second, uninflected adjectives like \textit{prima} ‘great’, \textit{sexy} ‘sexy’, and \textit{super} ‘super’ cannot appear with an elided noun. Third, derived adjectives like \textit{Berliner} ‘(from) Berlin’ can occur in ellipsis contexts (as also noted by \citealt{Rehn2019}: 221-24 for Alemannic German). Given that ordinary adjectives have the same forms in ellipsis contexts and that only the special cases behave differently, I leave the analysis of ellipsis for future research. Note though that a detailed analysis of these facts may help narrow down the analytical options mentioned in the main text, which were offered to account for the optionality or absence of the inflections in non-ellipsis contexts.} While I cannot investigate these three special cases of adjectives in more detail here, it seems clear that the avoidance of a hiatus plays a role here and in the cases I turn to next.

Definite articles also have some special forms. Consider \tabref{tab:3:5}, where the inflections are set apart by hyphens. The parsing follows the one proposed in R. \citet[33]{Wiese1988} closely.\footnote{Note that the presentation of the definite article might be a simplification as reduced forms like \textit{s} as in \textit{s Haus} ‘the house’ can be found in speech. For Swiss German, \citet{Studler2011} shows that there are two paradigms of definite articles – the reduced and the full forms. Besides their different morphological manifestations, they also involve different semantics. Since the current investigation focuses on adjectival inflections and \textit{ein}, I will not investigate this for (colloquial) Standard German.}

\begin{table}
\caption{Definite articles}
\label{tab:3:5}
\begin{tabular}{lllll}
\lsptoprule
 & Masculine & Neuter & Feminine & Plural\\
\midrule
 Nominative & de-r & da-s & die & die\\
Accusative & de-n & da-s & die & die\\
Dative & de-m & de-m & de-r & de-n\\
Genitive & de-s & de-s & de-r & de-r\\
\lspbottomrule
\end{tabular}
\end{table}

As can be observed in \tabref{tab:3:5}, there are two points of interest (see also \citealt{GunkelEtAl2017}: 1297-98). First, there is no inflection on \textit{die} ‘the (/that)’. In other words, this form is pronounced [di] as an article and [di:] as its related distal demonstrative (where stress and the slightly different vowel quality of the demonstrative are often indicated by capitalization as in \textit{DIE}). Second, the stem forms vary between \textit{de}- in most instances vs. \textit{da}- in the nominative/accusative neuter vs. \textit{di}- in the nominative/accusative in feminine/plural contexts.\footnote{A reviewer asks why \textit{die} cannot be analyzed as \textit{d-i}. One advantage would be that now all articles would have an inflection. Note though that now one of the determiner stems would have no vowel and that this new inflection would be the only full-vowel inflection in Standard German. Furthermore, it would increase the inflectional inventory considering that *\textit{dies-i} ‘this/these’ or *\textit{gut-i} ‘good’ are not possible in Standard German. Having said that, there may be a phonetic explanation of the absence of a full vowel in second (unstressed) syllables. Be that as it may, I proceed on the basis of the traditional assumptions regarding the inflectional inventory. Notice that assuming no inflection on \textit{die} allows me to relate \textit{wir} ‘we’ and \textit{ihr} ‘you(\textsc{pl})’ to this element in the discussion of adjectival inflections in pronominal DPs (\sectref{sec:3.5.3.1}).} 

  Starting with the first issue, it is clear that schwa cannot occur with the form \textit{die} ‘the (/that)’. On the one hand, it cannot be added to the stem directly; compare \REF{ex:3:15a} to \REF{ex:3:15b}. On the other hand, it cannot be added, mitigated by \textit{n}-epenthesis \REF{ex:3:15c}.

\ea%15
    \label{ex:3:15}
\ea\label{ex:3:15a}
\gll die   [di(:)]\\
    the\\
\glt ‘the (/that)’

\ex[*]{\label{ex:3:15b}
\gll  \textit{di-e}   [di(:)ə]\\
  the-\textsc{st}\\
  }
\ex[*]{ \label{ex:3:15c}
\gll   \textit{di-n-e}\\
the-n-\textsc{st}\\
}
\z
\z

We can suggest that \REF{ex:3:15b} follows from the avoidance of a hiatus, and that \REF{ex:3:15c} indicates that \textit{n}-epenthesis is not possible. As seen with the proximal demonstrative below, the avoidance of a hiatus and the non-availability of \textit{n}-epenthesis also play a role in the account of other determiners. Indeed, this type of analysis is even more general accounting also for the third-person pronoun \textit{sie} ‘she, her; they, them’, which is pronounced as [zi:] (for the discussion of pronouns, see \sectref{sec:3.5}). If so, then the final -\textit{e} on \textit{die} and \textit{sie} can be taken as an orthographic element marking the quality of the stem vowel (cf. also \textit{nie} ‘never’, \textit{wie} ‘how’, \textit{Knie} ‘knee’, \textit{Chemie} ‘chemistry’, etc.).

  As to the second issue, the different stem forms, I assumed in \chapref{sec:2}, \sectref{sec:2.2.1.2} that the schwas in adjectival inflections are due to epenthesis. This is presumably different for -\textit{e}- in the relevant stem forms of the definite article: \textit{der}, \textit{den}, \textit{dem}, \textit{des}. R. \citet[34]{Wiese1988} observes that the stem vowels are short and lax if an obstruent follows (\textit{des}, also \textit{das}) but long and tense otherwise (\textit{der}, \textit{den}, \textit{dem}, also \textit{die}). Again, this is more general and applies to the different vowel qualities in \textit{es} ‘it’ vs. \textit{er} ‘he’, \textit{ihn} ‘him(\textsc{acc})’, \textit{ihm} ‘him(\textsc{dat})’, \textit{sie} ‘she, her; they, them’, \textit{ihr} ‘her(\textsc{dat})’, and \textit{ihnen} ‘them(\textsc{dat})’.

  Given this regularity and to make the vocabulary entries of the different definite article forms parallel, I include the stem vowel -\textit{e}- in the vocabulary insertion rule. I analyze the different stem forms as another instance of contextually conditioned allomorphy. The vocabulary insertion rule for the nominative/accusative form in the neuter is given in \REF{ex:3:16a}, the one for the nominative/accusative in the feminine and plural is in \REF{ex:3:16b}, and the elsewhere case is provided in \REF{ex:3:16c} (adapted here from \citealt{Roehrs2009a}: 132).\footnote{Referencing unpublished work by B. Wiese, G. \citet[125]{Müller2002a} derives both issues (absence of schwa on \textit{die}, different stem forms) by assuming one stem form (\textit{de}-), where the stem vowel /e/ is realized as [a] in nominative/accusative neuter contexts and as [i:] when the inflection schwa is added to \textit{de}- yielding a sequence of /e/-/e/. As far as I know, these are not regular phonetic realizations of an underlying /e/.}

\ea%16
    \label{ex:3:16}
\ea\label{ex:3:16a} [+D; +DEF]   $\rightarrow$   \textit{da}- / {\longrule} [--O, --F, +N]
\ex\label{ex:3:16b} [+D; +DEF]   $\rightarrow$   \textit{di}- / {\longrule} [--O, +F]
\ex\label{ex:3:16c} [+D; +DEF]   $\rightarrow$   \textit{de}-
\z
\z

Note that the stem form spells out the features on the left of the arrow; the features on the right specify the context of application of the vocabulary insertion rules. As just discussed, I assume that the phonological environment (i.e., elements such as obstruents) will determine the specific properties of the stem vowels in \REF{ex:3:16}. To bring about the inflected forms of the definite article, recall that overt articles involve complex heads consisting of two separate feature bundles. As such, inflections are added to the stems in \REF{ex:3:16} by the vocabulary inflection rules, discussed in \chapref{sec:2}, \sectref{sec:2.2.1.5}. These rules spell out the CNG features of the bipartite article structure. With these insertion rules in mind, consider demonstratives.

It is well known that demonstratives and definite articles are diachronically related in that the latter have evolved from the former (e.g., \citealt{Greenberg1978}, \citealt{vanGelderen2007}). As briefly mentioned in \chapref{sec:1}, \sectref{sec:1.4.1.2}, it is also often argued that they are synchronically related such that the demonstratives consist of the definite article and an additional deictic component (e.g., \citealt{Leu2007}; \citealt{Roehrs2010,Roehrs2013a}; and references cited therein). Starting with the proximal demonstrative, its decomposition is exemplified in \REF{ex:3:17a}. This set of examples illustrates the masculine, neuter, and feminine forms in the nominative, where \textit{d}- spells out [+D; +DEF], -\textit{ies}- spells out [+DEIX], and the varying inflections spell out the different CNG feature bundles. Note though that the vocabulary insertion rules proposed in \REF{ex:3:16} should actually yield the strings in \REF{ex:3:17b} or \REF{ex:3:17c}, contrary to fact.

\ea%17
    \label{ex:3:17}
  \ea\label{ex:3:17a}
  \gll d-ies-er,        d-ies-es, d-ies-e\\
  the-\textsc{deix}-\textsc{st}, the-\textsc{deix}-\textsc{st}, the-\textsc{deix}-\textsc{st}\\
  \glt ‘this’
  \ex[*]{\label{ex:3:17b}
  \gll de-ies-er, da-ies-es,  di-ies-e\\
  the-\textsc{deix}-\textsc{st}, the-\textsc{deix}-\textsc{st}, the-\textsc{deix}-\textsc{st}\\
  }
  \ex[*]{\label{ex:3:17c}
  \gll de-n-ies-er, da-n-ies-es, di-n-ies-e\\
  the-n-\textsc{deix}-\textsc{st}, the-n-\textsc{deix}-\textsc{st}, the-n-\textsc{deix}-\textsc{st}\\
  }
  \z
\z


If demonstratives and definite articles are indeed related, then certain adjustments have to occur. To generate the correct forms in \REF{ex:3:17a}, I follow \citet{Roehrs2013a} in suggesting that \REF{ex:3:17b} shows that a hiatus is avoided (by deleting the stem vowel of the article component). Furthermore, \REF{ex:3:17c} indicates that similar to definite articles, \textit{n}-epenthesis cannot rescue the cases in \REF{ex:3:17b}. As to the distal demonstrative \textit{DER} ‘that’, I follow \citet{Roehrs2013a} in assuming that the deictic component involves a null element here (i.e. \textit{de-Ø-r}). Given this null element, no adjustments have to be made. Moving forward, I abstract away from these finer points continuing with \textit{d}- as the stem for the definite article and its related distal demonstrative and with \textit{dies}- as the stem for the proximal demonstrative.

  To sum up, this section accounted for some unexpected forms of certain adjectives, definite articles and their related demonstratives. All these cases show that hiatuses do not occur. While it seems clear that a hiatus is generally avoided in the determiner system, the exact conditions that induce the avoidance of a hiatus with adjectives are less clear: A hiatus is avoided with certain disyllabic adjectives ending in -\textit{a}, but other instances tolerate a hiatus (e.g., \textit{blau-e} ‘blue’, \textit{roh-e} ‘raw’). I leave the detailed investigation of these differences for future research. Next, I discuss a type of variation that has received very little attention in the theoretical literature.

\section{Canonical DPs with unexpected weak determiners: Impoverishment Rule 2}\label{sec:3.4}

In this section, I continue discussing variation that concerns the inflections on the determiners themselves. Like in the previous section, this variation is restricted, but here to genitive masculine/neuter environments. Furthermore, like in the previous section, this variation involves inflections on \textit{der}-words, but here on \textit{der}-words other than the definite article and its related distal demonstrative. Unlike in the previous section, the inflection on these determiners is present, and it alternates between strong and weak. Now, given the discussion of \chapref{sec:2}, it is surprising that determiners may have a weak ending at all. I propose that this variation is the result of Impoverishment; in particular, it is due to a more general application of Impoverishment Rule 2. First, I present the data, and then I provide my account along with an additional restriction.

  To begin, certain determiners show inflectional variation in genitive masculine/neuter contexts. Considering the data below, the (a)-examples exhibit the (expected) strong endings on the determiners \textit{dieser} ‘this’, \textit{jeder} ‘every’, and \textit{aller} ‘all’; the (b)-examples show these elements with the corresponding weak endings.

\ea%18
    \label{ex:3:18}
\ea \label{ex:3:18a}
\gll im      Sommer dies-es Jahr-es\\
in.the summer this-\textsc{st} year.\textsc{neut}-\textsc{gen}\\
\glt ‘in the summer of this year’
\ex \label{ex:3:18b}
\gll im      Sommer dies-en Jahr-es\\
in.the summer this-\textsc{wk} year.\textsc{neut}-\textsc{gen}\\
\glt ‘in the summer of this year’
\z
\z

\ea%19
    \label{ex:3:19}
\ea\label{ex:3:19a}
\gll die Verarbeitung jed-es      Holz-es\\
the processing     every-\textsc{st} wood.\textsc{neut}-\textsc{gen}\\
\glt ‘the processing of every (type of) wood’
\ex\label{ex:3:19b}
\gll die Verarbeitung jed-en      Holz-es\\
the processing     every-\textsc{wk} wood.\textsc{neut}-\textsc{gen}\\
\glt ‘the processing of every (type of) wood’
\z
\z

\ea%20
    \label{ex:3:20}
\ea\label{ex:3:20a}
\gll der Beginn     all-es Schön-en\\
the beginning all-\textsc{st} beautiful.\textsc{neut}-\textsc{wk}\\
\glt ‘the beginning of everything beautiful’
\ex\label{ex:3:20b}
\gll die Wege all-en Übel-s\\
the ways all-\textsc{wk} evil.\textsc{neut}-\textsc{gen}\\
\glt ‘the ways of all evil’
\z
\z

As noted by other scholars, this variation is possible with the following \textit{der}-words: \textit{dieser} ‘this’ (G. \citealt{Müller2002a}: 137 fn. 38), \textit{jener} ‘that’ \citep[154]{Gallmann2004}, \textit{jeder} ‘every’ (also \textit{jedweder} ‘every’, \textit{jeglicher} ‘every’), \textit{aller} ‘all’, \textit{mancher} ‘some’, \textit{solcher} ‘such’, \textit{welcher} ‘which’ (the latter are all mentioned in \citealt{ZifonunStrecker1997}: 1936-48). In contrast, there is no inflectional variation with the definite article \textit{der} ‘the’, its related demonstrative \textit{DER} ‘that’, and \textit{ein}-words in general – the latter must all have strong inflections.

\ea%21
    \label{ex:3:21}
\ea\label{ex:3:21a}
\gll der Verkauf d-es   Wagen-s\\
the sale       the-\textsc{st} car.\textsc{masc}-\textsc{gen}\\
\glt ‘the sale of the car’
\ex[*]{\label{ex:3:21b}
  \gll der Verkauf d-en    Wagen-s\\
  the sale       the-\textsc{wk} car.\textsc{masc}-\textsc{gen}\\
}
\ex\label{ex:3:21c}
\gll der Verkauf ein-es Wagen-s\\
the sale        a-\textsc{st}   car.\textsc{masc}-\textsc{gen}\\
\glt ‘the sale of a car’
\ex[*]{\label{ex:3:21d}
  \gll der Verkauf ein-en Wagen-s\\
  the sale        a-\textsc{wk}  car.\textsc{masc}-\textsc{gen}\\
}
\z
\z

B. \citet[184-85]{Wiese2009} states that there is no variation in this context with regular adjectives either. However, unlike the determiners in \REF{ex:3:21}, adjectives are always weak in this context. Rather than explaining the variation on the relevant \textit{der}-words by analogy with adjectives, I make the stronger and more interesting claim that the weak endings on these determiners are also due to Impoverishment.

I argued in \chapref{sec:2}, \sectref{sec:2.2.3} that adjectives in genitive masculine/neuter contexts have weak inflections due to Impoverishment Rule 2. Impoverishment affects elements that have their InflP in the specifier of AgrP.

\begin{figure}
	\caption{Impoverishment Rule 2}
	\label{figex:3:22}
	\begin{forest}
		[AgrP
			[InflP\\{[--F, $\alpha$N, +O, \sout{+S}]}]
			[Agr$'$\\{[-F, $\alpha$N, +O, +S]}]
		]
	\end{forest}
\end{figure}

Observe again that genitive masculine/neuter contexts are the exact same featural environments where the relevant \textit{der}-words from above can have weak inflections too. By contrast, note again that \textit{der} ‘the’, \textit{DER} ‘that’, and \textit{ein}-words do not exhibit this variation. The same goes for the head noun itself – its ending is not changed from -\textit{(e)s} to -\textit{(e)n} (e.g., \textit{Stuhl-(e)s} '(of the) chair' vs. *\textit{Stuhl-(e)n} '(of the) chair'). The difference between these two types of elements seems to involve structural size: The first group (\textit{der}-words other than \textit{der} and \textit{DER}) involves phrasal elements but the second group (\textit{der}, \textit{DER}, \textit{ein}-words, nouns) consists of heads. I propose that Impoverishment only applies to the inflections on the phrasal elements.\footnote{The demonstrative \textit{DER} ‘that’ presumably patterns with the article \textit{der} ‘the’ due to their obvious relatedness.} Consider this in more detail.

 I proposed in \chapref{sec:2}, \sectref{sec:2.2.1.6} that demonstratives have a more complex structure than articles in that they have InflP at the top of their structure, as shown in \figref{figex:3:23}.

\begin{figure}
	\caption{Structure of demonstratives}
	\label{figex:3:23}
	\begin{forest}
		[InflP\textsubscript{\parbox{0mm}{\mbox{[+D; +DEF, +DEIX][F, N, O, S]}}}
			[{[F, N, O, S]}]
			[DemP
				[{[+D; +DEF, +DEIX]}]
			]
		]
	\end{forest}
\end{figure}

This yields a phrasal determiner, which occurs in phrasal (e.g., specifier) positions. Extending this discussion, I propose that besides \textit{dieser} ‘this’, some other items such as \textit{jener} ‘that’, \textit{jeder} ‘every’ (also \textit{jedweder} ‘every’, \textit{jeglicher} ‘every’), \textit{aller} ‘all’, \textit{mancher} ‘some’, \textit{solcher} ‘such’, and \textit{welcher} ‘which’ also have structures similar to \figref{figex:3:23}.

With that in mind, I argue that Impoverishment Rule 2 has a more general context of application, updated as Impoverishment Rule 2$'$. Rather than just applying in AgrP, this rule operates in all phrases indicated by XP below.

\begin{figure}
	\caption{Impoverishment Rule 2$'$}
	\label{figex:3:24}
	\begin{forest}
		[XP
		[InflP\\{[--F, $\alpha$N, +O, \sout{+S}]}]
		[X$'$\\{[--F, $\alpha$N, +O, +S]}]
		]
	\end{forest}
\end{figure}

\largerpage
Given this, the more general Impoverishment rule in \figref{figex:3:24} now applies to all determiners in Spec,ArtP, adjectives in Spec,AgrP, quantifiers/numerals in Spec,CardP, and determiners in Spec,DP.\footnote{Note that Impoverishment Rule 2$'$ does not apply to all elements in Spec,CardP. For instance, while it is applicable to inflected singular quantifiers (e.g., \textit{vieler} ‘much’), it is not relevant to the singularity numeral \textit{EIN} ‘one’, which consists of a null element in Spec,CardP and \textit{ein} in Card (\chapref{sec:5}).} More generally, if Impoverishment occurs in this featural context, then only the least specified ending -\textit{en} can be inserted as discussed with adjectives in \chapref{sec:2}. This accounts for the weak ending on the relevant determiners. Importantly, note that this Impoverishment rule presents a second, secondary mechanism that accounts for the strong/weak alternation.

It is not entirely clear to me how to derive the optionality of the strong/weak alternation on these determiners. Currently, I see two analytical options. First, it would be possible to stipulate that there are actually two independent Impoverishment rules, Rule 2 and Rule 2$'$, where the former applies obligatorily, but the latter does not. Note though that both rules have the same basic context of application. Assuming the existence of two separate rules applying in the same context runs the risk of losing an important generalization. Alternatively, it would be possible to suggest that – what I have proposed to be – phrasal determiners can actually vacillate between involving phrases or heads (cf. \citealt{Borer2005}: 169-74). If the relevant determiners merged as heads, Impoverishment Rule 2$'$ would not apply to them (but only to their phrasal counterparts and adjectives more generally). Whichever is the right solution to pursue, note that determiners, be they phrases or heads, still trigger Impoverishment on the adjective (by Impoverishment Rule 1).\footnote{This makes the predication that a determiner with a weak inflection can be followed by an adjective with a weak inflection. While I have not found any attested examples yet, such cases seem possible to me.
	\ea
	\ea
	\gll im      Verlauf dies-en schön-en        Sommer-s\\
	in.the course  this-\textsc{wk} beautiful-\textsc{wk} summer.\textsc{masc}-\textsc{gen}\\
	\glt ‘in the course of this beautiful summer’
	\ex
	\gll im      Sommer jed-en      heiß-en Jahr-es\\
	in.the summer every-\textsc{wk} hot-\textsc{wk} year.\textsc{neut}-\textsc{gen}\\
	\glt  ‘in the summer of every hot year’
	\z
	\z
}

I close this section by discussing a final restriction. The occurrence of weak endings on the determiners is constrained by the ending on the following noun: If the ending on the noun is -\textit{(e)s}, both strong or weak endings on the determiners are usually possible \REF[a-b]{ex:3:25}; if the ending on the noun is -\textit{(e)n}, only strong endings on the determiners are possible; compare \REF{ex:3:25c} to \REF{ex:3:25d} (data are from \citealt{Gallmann1996}: 293, also \citealt{Gallmann1990}: 268-75).

\ea%25
    \label{ex:3:25}
\ea\label{ex:3:25a}
\gll der Traum manch-es Schüler-s\\
the dream  some-\textsc{st}   pupil.\textsc{masc}-\textsc{gen}\\
\glt ‘the dream of some pupil’
\ex\label{ex:3:25b}
\gll der Traum manch-en Schüler-s\\
the dream  some-\textsc{wk}  pupil.\textsc{masc}-\textsc{gen}\\
\glt ‘the dream of some pupil’
\ex\label{ex:3:25c}
\gll der Traum manch-es Student-en\\
the dream  some-\textsc{st}   student.\textsc{masc}-\textsc{gen}\\
\glt ‘the dream of some student’
\ex[*]{\label{ex:3:25d}
\gll der Traum manch-en Student-en\\
the dream  some-\textsc{wk}  student.\textsc{masc}-\textsc{gen}\\
}
\z
\z

Note that this restriction has nothing to do with the strong/weak alternation per se. Rather, it involves a dependency between the inflection on the determiner (or adjective, not discussed here separately; for some remarks, see \chapref{sec:2}, \sectref{sec:2.2.3}) and the inflection on the noun.

In \chapref{sec:2}, I provided a specificity-based account of the inventory and distribution of adjectival inflections. \citet{Gallmann1996,Gallmann1998,Gallmann2018} utilizes specificity of inflections to account for the cases above. Given the interplay between the inflection on the determiner element and that on the noun, \citet{Gallmann1996} formulates the Genitive Rule: A noun phrase in the genitive must contain at least one element that is sufficiently specific as regards case. Gallmann states that -\textit{es} and -\textit{er} are more specific than -\textit{en}. To be precise, the first two are [+Genitive], which consists of [+Oblique] and other features, but the third is only [+Oblique] (see \citealt{Gallmann1998}: 146, 153). This is consistent with the feature specifications in \chapref{sec:2}, where the strong inflections -\textit{es} and -\textit{er} are also more specific than the weak inflection -\textit{en} (the latter is interpreted as the elsewhere case and has no feature specification at all).\footnote{Although case suffixes on nouns and adjectival inflections on determiners and adjectives are not the same elements (\chapref{sec:2}, \sectref{sec:2.2.3}), the case suffix -\textit{(e)s} presumably has a similar specification as the adjectival inflection -\textit{es}.} Since there is no sufficiently specific element in \REF{ex:3:25d}, this example violates the Genitive Rule, and its ungrammaticality is accounted for.

  To sum up, this section illustrated some inflectional variation on certain \textit{der}-words in genitive masculine/neuter contexts. I suggested that Impoverishment Rule 2 from \chapref{sec:2} appears to have a more general context of application. I made two tentative suggestions to account for the (restricted) optionality of this variation. Given its restricted application, Impoverishment Rule 2 is interpreted as a second, secondary mechanism. More generally, besides Impoverishment Rule 1, adjectival inflections in German are, in certain contexts, regulated by a phonetic rule and Impoverishment Rule 2.

\section{Canonical DPs with unexpected strong adjectives: Pronominal determiners}\label{sec:3.5}

In this section, I return to the discussion of pronominal DPs. Recalling that these are definite constructions, I show that there is variation as regards the inflections on adjectives: In some cases, both (expected) weak and (unexpected) strong adjectives are possible; in others, only (unexpected) strong adjectives occur. I propose that pronominal DPs involve regular structures. However, pronominal determiners lack features that trigger Impoverishment. This accounts for the general option of strong adjectives. Weak adjectives are due to the phonetic rule discussed in \sectref{sec:3.2.2} and to (phonetically constrained) analogy with ordinary definite DPs.

I also discuss briefly preferences for strong adjectives over weak adjectives and vice versa. I suggest that, if possible, strong adjectives are preferred as they are more specified as regards CNG features than their corresponding pronominal determiners. Note that I often refer to pronominal determiners (or pronouns) as pronominal elements or, for short, as pronominals. Given the complexity of the data and the fact that this variation has not received much attention in the theoretical literature, I discuss pronominal DPs in quite some detail. At the end of this section, I summarize the data and analysis.

\subsection{Data}\label{sec:3.5.1}

First, I present the data under discussion (see also \citealt{Bhatt1990}: 154-55; \citealt{Darski1979}: 198; \citealt{Duden1995}: 280, \citeyear{Duden2007}: 39; \citealt{GunkelEtAl2017}: 1308). I focus on the first-person pronominal elements. Second-person pronominals show the same distribution and are only briefly discussed in what follows. Starting with the singular, there is no variation in the nominative and accusative in that only strong adjectives are possible.\footnote{Recall that adjectival inflections in the nominative/accusative feminine involve -\textit{e} (which is ambiguous between a strong and a weak inflection). Such a case is not provided in \REF{ex:3:26}. Also, strings like \REF[a-b]{ex:3:26} are ungrammatical in English. Given that there are no straightforward translations of such German examples, I put the adjective and noun in parentheses in the English translations. For simplicity, I translate the figurative meaning of animal names as ‘idiot’. Note though that different animal names have different implications; for instance, \textit{Schwein} ‘pig’ implies that the person is, in some sense, dirty whereas \textit{Esel} ‘donkey’ implies that the person is not very bright.}

\ea%26
    \label{ex:3:26}
\ea\label{ex:3:26a}
\gll ich      dumm-er Idiot\\
I.\textsc{nom} stupid-\textsc{st} idiot.\textsc{masc}\\
\glt ‘I (stupid idiot)’
\ex\label{ex:3:26b}
\gll mich      dumm-es Schwein\\
me.\textsc{acc} stupid-\textsc{st} pig.\textsc{neut}\\
\glt ‘me (stupid idiot)’
\z
\z

The second-person pronominal element \textit{du} ‘you(\textsc{nom.sg})’ shows the pattern in \REF{ex:3:26a}, and \textit{dich} ‘you(\textsc{acc.sg})’ has the same distribution as in \REF{ex:3:26b}.

  There is variation in the dative in all three genders. Strong adjectives are preferred over weak ones in the masculine and neuter but vice versa in the feminine (the less preferred option is indicated by \%).

\judgewidth{\%}
\ea%27
    \label{ex:3:27}
  \ea[\%]{ \label{ex:3:27a}
  \gll  mir        groß-en   Esel\\
  me.\textsc{dat} great-\textsc{wk} donkey.\textsc{masc}\\
  \glt ‘me (stupid idiot)’
  }
  \ex[]{ \label{ex:3:27b}
  \gll mir        groß-em Esel\\
  me.\textsc{dat} great-\textsc{st} donkey.\textsc{masc}\\
  \glt ‘me (stupid idiot)’
  }
  \z
\z

\ea%28
    \label{ex:3:28}
\ea[\%]{\label{ex:3:28a}
  \gll   mir        groß-en   Schwein\\
me.\textsc{dat} great-\textsc{wk} pig.\textsc{neut}\\
\glt ‘me (stupid idiot)’
}
\ex[]{\label{ex:3:28b}
\gll mir        groß-em Schwein\\
me.\textsc{dat} great-\textsc{st} pig.\textsc{neut}\\
\glt ‘me (stupid idiot)’
}
\z
\z

\ea%29
    \label{ex:3:29}
\ea[]{\label{ex:3:29a}
\gll mir        groß-en   Gans\\
me.\textsc{dat} great-\textsc{wk} goose.\textsc{fem}\\
\glt ‘me (stupid idiot)’
}
\ex[\%]{\label{ex:3:29b}
\gll mir        groß-er  Gans\\
me.\textsc{dat} great-\textsc{st} goose.\textsc{fem}\\
\glt ‘me (stupid idiot)’
}
\z
\z

\judgewidth{*}

The same patterns hold for the second-person pronominal element \textit{dir}\linebreak ‘you(\textsc{dat.sg})’. The cases in the singular are summarized in \tabref{tab:3:6}, where, if important, [--F] indicates pronominal DPs in the masculine/neuter and [+F] does so as regards feminine/plural.

\begin{table}
\caption{Summary of adjectival inflections in singular pronominal DPs}
\label{tab:3:6}
\begin{tabularx}{\textwidth}{lcCc}
\lsptoprule
Pronominal element & Gender of the N & Weak & Strong\\
\midrule
{\textit{ich} ‘I(\textsc{nom})’ \newline /\textit{du} ‘you(\textsc{nom})’} &  & - & ${\surd}$\\
{\textit{mich} ‘me(\textsc{acc})’ \newline /\textit{dich} ‘you(\textsc{acc})’} &  & - & ${\surd}$\\
 \textit{mir} ‘me(\textsc{dat})’/\textit{dir} ‘you(\textsc{dat})’   & [--F] & \%        & ${\surd}$\\
                                                                    &  [+F] & ${\surd}$ & \%\\
\lspbottomrule
\end{tabularx}
\end{table}

In the plural, there is variation in the nominative where, like in the dative feminine singular, weak adjectives are preferred over strong ones \REF[a-b]{ex:3:30}. There is no variation in the accusative and dative \REF[c-d]{ex:3:30}, with the qualification that the inflections on the adjectives in the dative are ambiguous between weak and strong \REF{ex:3:30d}.

\ea%30
    \label{ex:3:30}
\ea\label{ex:3:30a}
\gll wir         nett-en   Studenten\\
we.\textsc{nom} nice-\textsc{wk} students\\
\glt ‘us nice students’

\ex[\%]{\label{ex:3:30b}
\gll  wir         nett-e    Studenten\\
we.\textsc{nom} nice-\textsc{st} students\\
\glt ‘us nice students’
}
\ex\label{ex:3:30c}
\gll für uns       nett-e    Studenten\\
for us.\textsc{acc} nice-\textsc{st} students\\
\glt ‘for us nice students’

\ex\label{ex:3:30d}
\gll von   uns       nett-en        Studenten\\
from us.\textsc{dat} nice-\textsc{wk/st} students\\
\glt ‘from us nice students’
\z
\z

The nominative plural pronominal element \textit{ihr} ‘you(\textsc{nom.pl})’ has the same distribution as in \REF[a-b]{ex:3:30}, and the dative/accusative pronominal element \textit{euch} ‘you(\textsc{acc/dat.pl})’ shows the patterns as in \REF[c-d]{ex:3:30}. The plural cases are summarized in \tabref{tab:3:7}, where the specified features for [O] and [S] indicate the different morphological cases of the pronominal DPs involved.

\begin{table}
\caption{Summary of adjectival inflections in plural pronominal DPs}
\label{tab:3:7}
\begin{tabular}{lccc}
\lsptoprule
Pronominal element & Case of the DP & Weak & Strong\\
\midrule
{ \textit{wir} ‘we(\textsc{nom})’ \newline /\textit{ihr} ‘you(\textsc{nom})’} & [--O, --S] & ${\surd}$ & \%\\
{ \textit{uns} ‘us(\textsc{acc})’ \newline /\textit{euch} ‘you(\textsc{acc})’} & [--O, +S] & - & ${\surd}$\\
{ \textit{uns} ‘us(\textsc{dat})’ \newline /\textit{euch} ‘you(\textsc{dat})’} & [+O, --S] & ambiguous & ambiguous\\
\lspbottomrule
\end{tabular}
\end{table}

\newpage
Finally, as is well known, third-person pronominals do not take overt complements in the singular or plural (for recent discussion, see \citealt{Höhn2020}).\footnote{A brief note on older varieties of German is of interest here. MHG appears to have tolerated forms of the third person, for instance, \textit{er süezer man vil guoter} ‘he (nice man), very good’ (see \citealt{PaulEtAl1989}: 359). However, these strings may actually involve appositives (as suggested by the continuation with \textit{vil guoter}). Be that as it may, it is clear that MHG is different in that pronominal elements in the nominative singular could be followed by a weak adjective, for instance, \textit{ich arme tore} ‘I (poor-\textsc{wk} fool)’. For similar facts in ENHG, see \citet[197]{EbertEtAl1993}.}

\ea%31
    \label{ex:3:31}
\ea[*]{\label{ex:3:31a}
\gll er Idiot\\
he idiot.\textsc{masc}\\
}
\ex[*]{\label{ex:3:31b}
\gll sie    Linguisten\\
they linguists\\
}
\z
\z

Note also that pronominal elements in the genitive are not commonly used in German and do not do double duty as (transitive) pronominal determiners – similar to third-person pronominals.\footnote{Recall that I assume that transitive determiners have overt complements but that intransitive determiners have covert complements.}

Taking stock thus far, pronominal determiners are definite elements. Given their similarity to ordinary definite determiners, we would expect all adjectives to have weak inflections. However, this is not the case. In fact, all instances allow strong adjectives. Crucially, weak adjectives only occur in the dative singular and in the nominative plural, but not in the nominative/accusative singular and not in the accusative plural. Adjectives in the dative plural are inflectionally ambiguous. Finally, third-person and genitive pronominals do not take adjectives and nouns.

Observe again that the strong endings cannot mark indefiniteness. Furthermore, similar to the distribution of strong adjectival inflections where, for instance, -\textit{em} only occurs in the masculine/neuter, and -\textit{e} only surfaces in the feminine/plural, here masculine also patterns with neuter and feminine also with plural. In the current contexts though, the similarity concerns the preference of the relevant endings: Strong endings are preferred in the dative masculine/neuter but weak endings in the dative feminine and in the nominative plural.

  Before moving on, consider the complete set of pronominal elements in \tabref{tab:3:8} (for detailed discussion of the internal structure, see \citealt{Fischer2006}). Starting with the first and second-person pronominals, notice that these elements do not involve inflections. Furthermore, they do not distinguish gender in the singular, and they do not have different forms in the accusative and dative case in the plural (\textit{uns}, \textit{euch}).

\begin{table}
\caption{Pronominal elements in German}
\label{tab:3:8}
\fittable{\begin{tabular}{lllllllll}
\lsptoprule
  & \multicolumn{5}{c}{ Singular} & \multicolumn{3}{c}{Plural}\\
  \cmidrule(lr){2-6}\cmidrule(lr){7-9}
  & 1 & 2  & 3m & 3n & 3f& 1   & 2  & 3 \\
\midrule
\scshape nom & ich    & du     & e-r & e-s & sie & wir & ihr & sie\\
\scshape acc & mich   & dich   & ih-n & e-s & sie & uns & euch & sie\\
\scshape dat & mir    & dir    & ih-m & ih-m & ih-r & uns & euch & ih-n-en\\
\scshape gen & meiner & deiner & seiner & seiner & ihrer & unserer & eurer & ihrer\\
\lspbottomrule
\end{tabular}
}
\end{table}

Third-person pronominal elements are different in a number of ways. First, they have the typical determiner endings, and the singular instances of these pronominals also distinguish gender. Second, note that the form \textit{sie} is special. On the one hand, it is pronounced as [zi:]; that is, it does not have an inflection (see \sectref{sec:3.3}). On the other hand, it is the only pronominal element unspecified for number: It translates as ‘she/her’ and ‘they/them’. Third, as mentioned above, third-person elements, like genitive pronominals more generally, do not take overt complements.\footnote{There is another type of \textit{sie}: When capitalized in writing, \textit{Sie} ‘you(\textsc{sg/pl.formal})’ is morphologically third-person plural, but semantically, it is second-person singular/plural. Also, unlike lower case \textit{sie}, capitalized \textit{Sie} can take overt complements (for detailed discussion, see \chapref{sec:7}).} Finally, note that dative plural \textit{ihnen} has the additional inflection -\textit{en} (similar to the dative plural pronominal \textit{d-en-en} ‘those (ones)’) and that genitive pronominal elements do not involve (regular) adjectival inflections. This is particularly clear in masculine/neuter contexts where the strong inflection would be -\textit{es} or the weak one would be -\textit{en}, both contrasting with -\textit{er}. I provide the vocabulary insertion rules for all these elements in \sectref{sec:3.5.3.2}.

The lack of gender specification in the first and second person singular and the fact that case is not uniquely differentiated by all these pronominal elements in the plural can be linked to the general preference for strong adjectives (provided there is variation). I propose in \sectref{sec:3.5.3.3} that \textit{wir} ‘we’ and \textit{ihr} ‘you(\textsc{nom.pl})’ are the only first and second-person pronominals that are specified for all CNG features. Consequently, they are preferably followed by weak adjectives. By contrast, all other relevant pronominal determiners are not fully specified for CNG features (note again that one pronominal form spells out multiple feature combinations), and the following adjective is recruited to indicate those features more uniquely with a strong inflection.

  In the next section, I briefly discuss three types of proposals that might be offered to account for the inflectional variation on the adjectives inside the pronominal DPs. Given certain assumptions, I point out some shortcomings for all of them (if the reader is not interested in the detailed discussion of other proposals, they are invited to proceed to \sectref{sec:3.5.2.4}). In \sectref{sec:3.5.3}, I propose a different analysis that avoids these issues. \sectref{sec:3.5.4} summarizes the discussion.

\subsection{Three types of proposals}\label{sec:3.5.2}

As discussed in \chapref{sec:2}, \sectref{sec:2.3.5}, personal pronouns are determiners in that they take the adjective and noun as their complement (e.g., \citealt{DéchaineWiltschko2002}: 421-22, \citealt{Höhn2020}, \citealt{Pesetsky1978}, \citealt{Postal1966}, \citealt{Roehrs2005}); consider \REF{ex:3:30a} repeated below as \REF{ex:3:32}. This pronominal DP can be represented in the simplified structure in \figref{figex:3:32} (AgrP abbreviates the rest of the canonical structure discussed at length in \chapref{sec:1} and \ref{sec:2}).

%\glltree[\label{figex:3:32}]{
%	(Pro)nominal DP -- Complementation\\
%	\gll wir nett-en  Studenten\\
%	we nice-\textsc{wk} students\\
%	\glt ‘us nice students’
%}{
%	[(LPP)
%		[~]
%		[DP
%			[\textit{wir}]
%			[CardP
%				[~]
%				[AgrP\\\textit{netten Studenten}]
%			]
%		]
%	]
%}

\ea%32
    \label{ex:3:32}
\gll wir nett-en  Studenten\\
we nice-\textsc{wk} students\\
\glt ‘us nice students’
\z

\begin{figure}
	\caption{(Pro)nominal DP -- complementation}
	\label{figex:3:32}
	\begin{forest}
		[(LPP)
			[~]
			[DP
				[\textit{wir}]
				[(CardP)
					[~]
					[AgrP\\\textit{netten Studenten}]
				]
			]
		]
	\end{forest}
\end{figure}

If personal pronouns are indeed determiners, then they should trigger Impoverishment. From this perspective, the weak ending on the following adjective in \REF{ex:3:32} is expected (in \sectref{sec:3.5.2.4}, I discuss the absence of weak adjectives in certain instances; this will lead to the ultimate proposal that pronominal determiners do not trigger Impoverishment). In order to account for the variation, that is, the unexpected strong ending on the following adjective as in \textit{wir nette Studenten} (cf. \REF{ex:3:30b}), I entertain three basic types of proposals. All these proposals involve the structural analysis in \figref{figex:3:32} above and some additional assumptions.

\subsubsection{One structure: Agreeing vs. non-agreeing pronominal elements}\label{sec:3.5.2.1}

The first type of proposal involves one structure where the pronominal element is in the same position in \figref{figex:3:32} above. This position could be Spec,DP or D.\footnote{Note that in the present account, determiners in either of these positions trigger Impoverishment. Having said that, I point out in \sectref{sec:3.5.3.1} that pronominals of the first and second person seem to be associated with a certain kind of deixis (in fact, this kind of deixis will be crucial for the discussion of Impoverishment). Given the deixis, these pronominals are similar to demonstratives and are assumed to project phrasal structures. If so, these pronominal determiners are presumably in Spec,DP.} This is what I assumed in \chapref{sec:2}, \sectref{sec:2.3.5} to account for the weak ending on the following adjective. To explain the strong ending, \citet[259]{Roehrs2005} makes a distinction between “agreeing” and “non-agreeing” determiners. Note first that these pronominal elements share concord features with the following adjective and noun; that is, a pronominal in the nominative combines with an adjective and noun in the nominative (and not, say, in the dative). This means that non-agreement cannot mean non-participation of these pronominal elements as regards concord in agreement features. Rather, we could assume that (non-)agreement has to do with the presence or absence of agreement morphology on the pronominal element itself. This idea could be made more concrete by assuming a null inflection for the agreeing type \REF{ex:3:33a} and the lack of such an inflection for the non-agreeing type \REF{ex:3:33b}.

\ea%33
    \label{ex:3:33}
\ea\label{ex:3:33a}
\gll wir-$\emptyset$\\
  we-\textsc{infl}\\
  \glt ‘we’
\ex\label{ex:3:33b}
\gll wir\\
we\\
\glt ‘we’
\z
\z

Roehrs proposes that an agreeing determiner takes a weak adjective; a non-agreeing determiner takes a strong adjective (see again \REF[a-b]{ex:3:30}). Elements like this exist elsewhere in German, with the qualification that they show an overt inflection in the agreeing cases; compare \REF{ex:3:34a} to \REF{ex:3:34b} (the same distribution holds with \textit{solch(e)} ‘such’ and \textit{welch(e)} ‘which).\footnote{Recall from \chapref{sec:2}, Footnote \ref{foot:2:23} that I analyze uninflected \textit{manch} in \REF{ex:3:34b} as a modificational element.}

\ea%34
    \label{ex:3:34}
\ea\label{ex:3:34a}
\gll manch-e gut-en     Freunde\\
some-\textsc{st} good-\textsc{wk} friends\\
\glt ‘some good friends’
\ex\label{ex:3:34b}
\gll manch gut-e      Freunde\\
some   good-\textsc{st} friends\\
\glt ‘some good friends’
\z
\z

Note though that the notion of (non-)agreeing determiner is not entirely\linebreak straightforward. Unlike with \textit{manch}- ‘some’, \textit{solch}- ‘such’, and \textit{welch}- ‘which’, agreeing pronominals involve an (assumed) null inflection. However, given that first and second-person pronominal elements never have an overt inflection in German, this raises the question whether the postulation of a null inflection is warranted here.

  To sum up, given the issue of a true distinction between agreeing vs. non-agreeing pronominal elements, this proposal seems to lack proper motivation. Furthermore, this distinction would not work in the current system at all as inflections on adjectives are a function of the determiner stem – its categorial feature [+D] – but not its inflection.

\subsubsection{One structure: Pronominal elements in different positions}\label{sec:3.5.2.2}

Similar to the previous subsection, this type of proposal also involves one structure – the one in \figref{figex:3:32}. However, the pronominal element could be proposed to be in different phrasal levels in that structure. Like in the previous subsection, the pronominal could be in the DP-level. In this case, the pronominal would be a determiner trigging Impoverishment and, consequently, it would occur with a weak adjective. In addition, the pronominal could be in a different structural level. There are two options: Either it could be a quantifier in CardP, or it could be a predeterminer in LPP. As explicated in more detail below, Impoverishment would not be triggered in either case, and the following adjective would surface with a strong ending.

  As a first option to explain a strong adjective, the pronominal element could do double duty as a quantifier in CardP. This option was entertained in a previous version of this work. As a quantifier, the pronominal element would not have the feature [+D], and as such, it would not trigger Impoverishment (for details, see \citealt{Roehrs2021}, also \citealt{Roehrs2009a}: 166-67). However, there does not seem to be evidence that pronominal elements can also appear as quantifiers. In fact, they can co-occur with numerals, a type of quantifier located in CardP.

\ea%35
    \label{ex:3:35}
\gll  wir drei  Freunde\\
  we three friends\\
  \glt ‘us three friends’
\z

As such, this possibility seems unlikely to be on the right track.

The second option to explain a strong adjective involves the claim that pronominal elements can do double duty as predeterminers in LPP. Unlike quantifiers, predeterminers are determiner-like elements with the feature [+D]. Unlike determiners, these elements are base-generated in LPP and neither trigger nor undergo Impoverishment (\chapref{sec:2}, \sectref{sec:2.4}). Ambiguous elements like this exist elsewhere in German. For instance, \textit{all-} ‘all’ and \textit{dies-} ‘this’ can be in two positions. When in the DP-level, these elements are determiners triggering Impoverishment \REF{ex:3:36}.

\ea%36
    \label{ex:3:36}
\ea \label{ex:3:36a}
\gll alle gut-en      Freunde\\
all   good-\textsc{wk} friends\\
\glt ‘all good friends’
\ex \label{ex:3:36b}
\gll diese gut-en      Freunde\\
these good-\textsc{wk} friends\\
\glt ‘these good friends’
\z
\z

When in the LPP-level, they are predeterminers, which do not bring about Impoverishment, neither on a following determiner \REF{ex:3:37} nor on a lower adjective \REF{ex:3:38}.

\ea%37
    \label{ex:3:37}
\ea\label{ex:3:37a}
\gll alle mein-e gut-en Freunde\\
all   my-\textsc{st}  good-\textsc{wk} friends\\
\glt ‘all my good friends’
\ex\label{ex:3:37b}
\gll diese mein-e gut-en     Freunde\\
these my-\textsc{st} good-\textsc{wk} friends\\
\glt ‘these good friends of mine’
\z
\z

\ea%38
    \label{ex:3:38}
\ea\label{ex:3:38a}
\gll all mein groß-es  Glück\\
all my    great-\textsc{st} happiness.\textsc{neut}\\
\glt ‘all my great happiness’
\ex\label{ex:3:38b}
\gll dieses mein groß-es  Glück\\
this     my    great-\textsc{st} happiness.\textsc{neut}\\
\glt ‘this my great happiness’
\z
\z

Before moving on, note that this conjecture is more promising as pronominals of the first and second person seem to be associated with deixis; that is, these pronominals are similar to the demonstrative \textit{dies-} ‘this’ above, which can appear as either a determiner or a predeterminer.

Again, it is hard to provide direct independent evidence for the claim that pronominal elements can be in two different positions – in this case, the DP-layer for pronominals as determiners and the LPP-layer for pronominals as predeterminers. On the one hand, these pronominal elements are distributionally restricted (they must occur before numerals, adjectives, and nouns); on the other hand, first and second-person pronominals are inflectionally restricted (they exhibit no endings). Thus far, I have come across only one distributional restriction: These pronominals cannot co-occur with ordinary predeterminers in German \REF[a-b]{ex:3:39}.

\ea%39
    \label{ex:3:39}
\ea[*]{\label{ex:3:39a}
\gll   all(-e)   wir Studenten\\
  all(-\textsc{st}) we  students\\
  \glt ‘all us students’
}
\ex[*]{\label{ex:3:39b}
\gll diese wir Studenten\\
  these we  students\\
  \glt ‘(these) us students’
  }
\z
\z

Let us consider how the data in \REF{ex:3:39} bear on the question of whether pronominals can be determiners or predeterminers. As we see below, the explanation of the ungrammaticality of \REF{ex:3:39} under the option of pronominals as determiners has an independent (and indeed language-specific) explanation. In contrast, the explanation of the ungrammaticality of \REF{ex:3:39} under the option of pronominals as predeterminers receives a more general explanation. We will wind up concluding that pronominals are indeed determiners but do not do double duty as predeterminers.

Starting with the option of pronominal elements as determiners, \citet[53]{Haider1988} discusses the contrast between the possessive article in \REF{ex:3:40a} and the Saxon Genitive in \REF{ex:3:40b} (see also \citealt{Haider1992}: 314-15, \citealt{Bayer2015}). Both of these elements are usually assumed to be in the DP-level.

\ea%40
    \label{ex:3:40}
\ea\label{ex:3:40a}
\gll all(-e)   ihr-e    Autos\\
all(-\textsc{st}) her-\textsc{st} cars\\
\glt ‘all her cars’
\ex[*]{\label{ex:3:40b}
\gll all(-e)   Marias Autos\\
all(-\textsc{st}) Mary’s cars\\
\glt ‘all Mary’s cars’
}
\z
\z

To the constrast in \REF{ex:3:40}, we can add a similar distinction involving the predeterminer \textit{dies-} ‘this’ \REF{ex:3:41}. While the (a)-example may sound slightly marked, there is a clear contrast between \REF{ex:3:41a} and \REF{ex:3:41b}, just as in \REF{ex:3:40} above.

\ea%41
    \label{ex:3:41}
\ea[?]{\label{ex:3:41a}
\gll  dies-e     ihr-e   Autos\\
these-\textsc{st} her-\textsc{st} cars\\
\glt ‘these cars of hers’
}
\ex[*]{\label{ex:3:41b}
\gll  dies-e     Marias Autos\\
these-\textsc{st} Mary’s cars\\
\glt ‘(these) Mary’s cars’
}
\z
\z

The difference between \textit{ihre} ‘her’ in \REF{ex:3:40a} and \REF{ex:3:41a}, on the one hand, and \textit{wir} ‘we’ in \REF[a-b]{ex:3:39} and \textit{Marias} ‘Mary’s’ in \REF{ex:3:40b} and \REF{ex:3:41b}, on the other hand, is that the former has an overt inflection but latter do not. Indeed, \textit{wir} and \textit{Marias} never take an overt adjectival inflection (unlike \textit{mein} in \REF{ex:3:38}, which, for instance, may appear as \textit{mein-em} in dative masculine/neuter contexts). The failure to (ever) take an inflection may account for this constrast (for fuller discussion, see \citealt{Roehrs2021}). If this is on the right track, then we can account for the ungrammaticality of \REF[a-b]{ex:3:39} under the option of pronominals as determiners on par with the ungrammaticality of the determiner-like element \textit{Marias} in \REF{ex:3:40b} and \REF{ex:3:41b}. Given that the ungrammaticality receives an independent explanation, we can continue maintaining that pronominals are determiners.\footnote{This is confirmed by the fact that this is a language-specific restriction. Nominals similar to \REF{ex:3:39a} above are possible in English, as pointed out by \citet[353]{Pesetsky1978}. 
	\ea all us linguists
\z} Indeed, this explanation is based on the very assumption that pronominal elements are in fact determiners in the DP-level, just like possessive articles and Saxon Genitives. To repeat, what is crucial here is that determiner(-like) elements in German have to be inflected if they follow a predeterminer.

  As to the option of pronominal elements as predeterminers, it could be suggested that pronominals can also be in LPP. As seen in \REF{ex:3:42}, ordinary predeterminers can be stacked.

\ea%42
    \label{ex:3:42}
\gll  all-e   dies-e    mein-e Freunde\\
  all-\textsc{st} these-\textsc{st} my-\textsc{st} friends\\
  \glt ‘all these friends of mine’
\z

As such, there is no structural competition between predeterminers – each can be housed in its own LPP. In other words, the occurrence of multiple predeterminers should be fine. However, as we see below, while analyzing the strings in \REF[a-b]{ex:3:39} in such a way provides an explanation of the ungrammaticality of these cases, this type of account also rules out the simpler, grammatical cases as in \REF{ex:3:30b}.

Recall that pronominal DPs are definite in interpretation. If pronominal elements as predeterminers are in LPP, then the DP-level does not involve an overt determiner (e.g., a definite article) or determiner-like element (e.g., a Saxon Genitive). This, however, is something that is required for definite noun phrases. Note in this regard that \textit{Studenten} ‘students’ by itself cannot have a definite interpretation – the DP-level must be filled with a relevant overt element (see \citealt{Alexiadou2005}, \citealt{Longobardi1994}). Thus, the option of pronominals as predeterminers can also explain the ungrammaticality of \REF{ex:3:39}, but, unlike the option of pronominals as determiners, this option receives a more general explanation. To repeat, the DP-layer must have an overt determiner(-like) element when the nominal is definite in interpretation.

With this in mind, I return to the simple cases like \textit{wir nette Studenten} ‘us nice-\textsc{st} students’ as in \REF{ex:3:30b}. If the discussion above is on the right track, then this means that such grammatical strings cannot be analyzed as \textit{wir} involving a predeterminer, as the DP-layer would not contain an overt element although the nominal is definite in interpretation. Such a structure would lead to ungrammaticality. However, since this simple pronominal DP is fine, we can argue that the pronominal must be a determiner in the DP-level here.

  To conclude, both options of pronominal as quantifier and pronominal as predeterminer are unlikely to be correct. As such, the variation of the inflection on the adjective remains unclear on those assumptions.

\subsubsection{Two different structures: Complementation and adjunction}  \label{sec:3.5.2.3}

Besides the structure in \figref{figex:3:32}, a second structure could be proposed to account for the possibility of a strong ending on the adjective in \REF{ex:3:43} below. Beside being the complement of the pronominal determiner, we could assume that the adjective and the noun are right-adjoined to the pronominal determiner. There are two basic options: The adjective and noun could be adjoined either high or in the middle of the structure. High right-adjunction was utilized in the discussion of appositional DPs (\chapref{sec:2}, \sectref{sec:2.3.4}). The structure of the pronominal DP would look as shown in \figref{figex:3:43}.

\ea%43
    \label{ex:3:43}
\gll wir nett-e   Studenten\\
we nice-\textsc{st} students\\
\glt ‘us nice students’
\z

\begin{figure}
\caption{Pronominal DP -- high right-adjunction}
\label{figex:3:43}
\begin{forest}
	[DP
		[DP\\\textit{wir}]
		[AgrP\\\textit{nette Studenten}]
	]
\end{forest}
\end{figure}

Importantly, unlike appositional DPs, pronominal DPs do not involve comma intonation, and their adjectives may, given the right conditions, receive a restrictive interpretation (in the plural cases). Alternatively, the adjunction could be located in the middle, shown in \figref{figex:3:44}, where XP might range over CardP, AgrP, ArtP, NumP, or NP (recall that \textit{e\textsubscript{N}} is the null head noun of a pronominal DP with an intransitive determiner).

\begin{figure}
	\caption{Pronominal DP -- mid right-adjunction}
	\label{figex:3:44}
	\begin{forest}
		[DP
			[\textit{wir}]
			[XP
				[XP [\textit{e}$_N$]]
				[AgrP\\\textit{nette Studenten}]
			]
		]
	\end{forest}
\end{figure}

Mid right-adjunction was employed in the discussion of indefinite pronoun constructions. The latter construction does not involve comma intonation and its adjective may have a restrictive interpretation. These properties fit better with pronominal DPs. Thus, unlike high adjunction, this structure presents a more promising candidate. Note now that this is a non-canonical structure. As such, Impoverishment would not occur, and a strong ending would surface on the adjective. Observe that assuming two structures (complementation, adjunction) would essentially result in the more general claim that pronominals are equally transitive and intransitive (cf. \citealt{Bhatt1990}: 155), where complementation of the adjective and noun would instantiate a transitive pronominal element, but adjunction of the adjective and noun would manifest an intransitive pronominal element. This claim is interesting and requires further investigation, something I will not pursue here. However, as we see below, there is another, independent issue.

  To take stock thus far, note that despite the issues raised for the first two types of accounts above, all three proposals can potentially account for the strong adjectives, which are possible in all contexts. However, given these proposals, it is not clear why weak adjectives can occur in only some contexts. I make this general issue of the three proposals more explicit in the next subsection.

\subsubsection{Absence of weak adjectives in certain instances}\label{sec:3.5.2.4}

Recall that the different structural options above were entertained to account for the variation of the inflections on the adjective following the pronominal determiner. As mentioned above, pronominal determiners are definite elements. As such, we expect weak adjectives to occur with them on a par with regular definite determiners. However, as documented above, all featural combinations pronominal DPs appear to exhibit strong adjectives but only some combinations allow weak adjectives. Weak adjectives are missing in three instances: nominative and accusative singular (which is very clear in the masculine/neuter) and accusative plural. All three proposals discussed above have to come to terms with the variation in some contexts but the lack of it in others – the three instances just mentioned.

  As I show below, all three proposals can account for the absence of weak adjectives in the nominative/accusative singular. Given certain assumptions, they all can account for the lack of variation here. It will turn out, though, that including the dative singular in the discussion will bring out issues for all three proposals. The absence of weak adjectives in the accusative plural is special. As far as I can see, none of the three proposals can rule them out without ad-hoc stipulations. At the end of this subsection, I provide a functional explanation for the absence of weak adjectives here.

  In order to account for the absence of weak adjectives in the nominative/accusative singular, we could follow \citet{Roehrs2005} in observing that the strong adjectives in these contexts are mirrored by certain strings involving \textit{ein}-words (also \citealt{Bhatt1990}: 155). The first proposal could suggest that the pronominal elements involve non-agreeing determiners. Like \textit{ein}, these instances of the pronominal determiner would have no inflections at all, and consequently the adjectives would be strong. The second proposal could also assume that these instances of the pronominal determiner are like \textit{ein}-words. However, unlike the first proposal, this account could follow the assumptions in \chapref{sec:2} claiming that these pronominal determiners do not induce Impoverishment. Consequently, the adjectives would be strong. The third proposal could take a different route. It could claim that all instances in the singular (which includes the dative), only involve adjunction (and not complementation) of the adjective and noun.

If so, the question arises why the dative singular does involve variation. The first two proposals have a straightforward answer: \textit{Ein}-words are only special in the nominative and accusative singular. The third proposal has to make additional assumptions to allow weak adjectives in the dative. Now, while this may indicate an advantage for the first two proposals, an account involving \textit{ein}-words is not straightforward. As discussed in \chapref{sec:2}, \sectref{sec:2.2.2.3} and \chapref{sec:5}, \textit{ein}-words are composite forms consisting of semantically vacuous \textit{ein} and another component. As such, it is \textit{ein} itself that has special properties. Crucially, \textit{ein} contrasts with pronominal elements as regards definiteness. This makes these two types of elements rather different from each other, and an extension of the properties of \textit{ein} to these pronominal elements does not seem straightforward. To be clear, given these assumptions, all three proposals face some challenges: Either the parallel to \textit{ein}-words is not straightforward or additional assumptions have to be made to explain all the data. It appears then as if a different solution needs to be found (see next section).

  Turning to the instances in the accusative plural, recall that a weak adjective cannot occur in this context \REF{ex:3:45a}. Furthermore, observe again that the dative counterpart in \REF{ex:3:45b} exhibits the ending -\textit{en} on the adjective, which is ambiguous between strong and weak.

\ea%45
    \label{ex:3:45}
\ea\label{ex:3:45a}
\gll für uns nett-e(*n)     Schüler\\
for us   nice-\textsc{st}/*\textsc{wk} pupils\\
\glt ‘for us nice pupils’

\ex\label{ex:3:45b}
\gll  von   uns nett-en        Schüler-n\\
from us   nice-\textsc{st/wk} pupils-\textsc{dat}\\
\glt ‘from us nice pupils’
\z
\z

Note now that case is not overtly manifested by a unique form of the pronominal element – it is \textit{uns} ‘us’ (or \textit{euch} ‘you(\textsc{pl})’) in both the accusative and dative cases. In addition, notice also that the weak ending in the accusative plural would be -\textit{en}, the same as the (ambiguous) ending in the dative plural. As such, it is the different inflections on the adjectives in \REF[a-b]{ex:3:45} that distinguish the accusative from the dative.\footnote{Note that the head noun has the case inflection -\textit{n} in the dative in \REF{ex:3:45b}, but not in the accusative in \REF{ex:3:45a} This inflection may potentially indicate case in the dative. There are two points that need to be made here. First, case in this instance would be manifested on the noun (rather than on the pronominal element). Second, \citet[154-63]{Wegener1995} and \citet[289]{Gallmann1996} point out that this case inflection on the noun has become somewhat unstable. Furthermore, this case marking cannot occur at all if the plural of the noun is formed in -\textit{n} or -\textit{s}, for instance, \textit{Junge-n-(*en)} ‘boy-s’ or \textit{Jung-s-(*en)} ‘(coll.) guy-s’. Given these points, I put the case marking on the noun aside.}

It is not clear how the three proposals can account for the lack of variation in the accusative plural: \textit{Ein}-words are only special in the singular instances, and complementation (which involves weak adjectives) should not be restricted to specific morphological cases like nominative and dative. As such, the accusative plural instances seem to be special. Given this state-of-affairs, I follow \citet[39]{Duden2007} in providing a functional explanation of the absence of the weak inflection here.

I assume that morphological case needs to be distinguished (if possible). Since the cases at hand involve dative and accusative, this may be related to their typical grammatical functions, indirect object and direct object, respectively. Now, as is well known, German has relatively free word order due to the availability of scrambling its objects. \citet[261-62]{Roehrs2005} provides the following examples where the pronominal DPs are in the order of indirect object > direct object in \REF{ex:3:46a} but in the sequence of direct object > indirect object in \REF{ex:3:46b}.

\ea%46
    \label{ex:3:46}
\ea\label{ex:3:46a}
\gll Sie   haben uns nett-en                Jungen euch klug-e            Mädchen vorgestellt.\\
they have   us   nice-\textsc{dat.st/wk} boys      you  smart-\textsc{acc.st} girls         introduced\\
\glt ‘They introduced you smart girls to us nice boys.’
\ex\label{ex:3:46b}
\gll Sie   haben uns nett-e           Jungen euch klug-en                 Mädchen vorgestellt.\\
they have   us   nice-\textsc{acc.st} boys     you   smart-\textsc{dat.st/wk} girls         introduced\\
\glt ‘They introduced us nice boys to you smart girls.’
\z
\z

Note that despite the two orderings of the objects in \REF{ex:3:46a} vs. \REF{ex:3:46b}, the different inflections on the adjectives inside the objects make it clear who was introduced to whom in each case. This functional explanation of the strong inflection in the accusative plural – the strong adjective clearly distinguishes accusative objects from dative objects – can apply to all three (and other) proposals.

Returning to the first part of this section, given the issues with the postulation of (non-)agreeing pronominals in the first proposal, the problems with the suggestion that pronominals can do double duty as determiners and quantifiers or predeterminers in the second account, and the issues with the claim of two co-existing types of structures in the third analysis, I make a different suggestion.

\subsection{Pronominal determiners and complementation} \label{sec:3.5.3}

In this section, I offer my proposal for the structure of pronominal DPs. I also discuss the structure and feature specifications of the pronominal determiners themselves, I provide the vocabulary insertion rules for pronominal elements, and I account for the preferences for strong or weak adjectives in certain pronominal DPs.

\subsubsection{Structures and feature specifications}\label{sec:3.5.3.1}

I propose that pronominal DPs involve complementation and that pronominal elements are determiners that lack features that make [+D] a trigger of Impoverishment. In other words, this proposal postulates one structure and one type of pronominal element. The fact that Impoverishment does not occur accounts for the general option of strong adjectives. The restricted occurrence of weak adjectives follows from certain phonetic considerations – the phonetic rule discussed in \sectref{sec:3.2.2} and phonetically conditioned analogy with ordinary definite DPs.

As far as I am aware, the proposal sketched just above is the simplest account with the fewest additional assumptions. Furthermore, proposing that pronominal DPs involve complemenation is compatible with the discussion in later chapters. Specifically, \chapref{sec:6} and \ref{sec:7} investigate similarities and differences between noun phrases (DPs) and clauses (TPs) of the form “pronoun + (copular verb +) noun”. TPs involve complementation and if DPs also constitute cases of complementation, where the string “pronoun + noun” is analyzed as a determiner taking the noun as its complement, then this allows a straightforward comparison between the nominal and clausal domains.

The proposal that pronominal elements are determiners is compatible with work by \citet{Postal1966} and others, briefly discussed in \chapref{sec:2}, \sectref{sec:2.3.5}. As determiners, pronominals have the categorial feature [+D]. They are merged in ArtP and take overt complements. I propose that the difference in inflections on adjectives in ordinary vs. pronominal DPs lies in the different structure and feature specifications of the pronominal determiners themselves. At first glance, there seem to be two options.

Pronominal elements could involve structures of demonstratives as in \figref{figex:3:47a}. Recalling that the relevant pronominal determiners are of the first and second person, they are associated with deixis, similar (but not identical) to that of third-person demonstratives. I follow \citet[429]{Halle1997} and \citet[288]{Nevins2007} in that grammatical person can be broken down into the features [±AUTH(or)] and [±PART(icipant)]: First person consists of [+AUTH, +PART], second person involves [--AUTH, +PART], and third person has [--AUTH, --PART].\footnote{It is possible that third person only includes [--PART].} As an alternative to \figref{figex:3:47a}, pronominal determiners could project reduced structures where InflP is missing at the top, see \figref{figex:3:47b}. This yields the following feature specifications for the two structural options of, say, the first-person pronominal \textit{wir} ‘we’.

\begin{figure}
	\subfigure[Pronominal determiner (Variant 1)]{
		\label{figex:3:47a}
		\begin{forest}
			[,phantom
			[InflP\textsubscript{\parbox{0mm}{\mbox{[+D; +AUTH, +PART][F, N, O, S]}}}
				[{[F, N, O, S]}]
				[DemP
					[Dem\\{[+D; +AUTH, +PART]}, tier=Dem]
				]
			]
			[\\$\rightarrow$  \textit{wir}, no edge, tier=Dem]
			]
		\end{forest}	
	}

	\subfigure[Pronominal determiner (Variant 2)]{
		\label{figex:3:47b}
		\hspace*{1cm}\begin{forest}
			[,phantom
			[DemP\textsubscript{\parbox{0mm}{\mbox{[+D; +AUTH, +PART]}}}
				[Dem\\{[+D; +AUTH, +PART]}, tier=Dem]
			]
			[\\$\rightarrow$  \textit{wir}, no edge, tier=Dem]
			]
		\end{forest}\hspace*{1cm}
	}
\caption{Pronominal determiners}
\label{figex:3:47}
\end{figure}

The immediate advantage of the variant in \figref{figex:3:47a} is that all phrasal determiners have the same basic structure. As for \figref{figex:3:47b}, this reduced demonstrative structure provides a straightforward account of the absence of inflections on first and second-person elements. Note that neither of the two structures involves features for definiteness or deixis. However, first and second person (i.e., [+PART]) presumably entail definiteness, and the category person is associated and indeed relatable to the category of deixis (see, e.g., \citealt{Lyons1999}: 18-19, \citealt{Roehrs2019}: 378-79). Crucially, with [+DEF] and [+DEIX] absent, the presence of the CNG feature bundle in \figref{figex:3:47a} but its absence in \figref{figex:3:47b} have different consequences for the occurrence of Impoverishment. Thus, depending on which of the two structural variants in \figref{figex:3:47} is adopted, we need to find different ways to account for the general variation and the varying preferences as regards the inflectional distribution on the adjectives.

If we adopt \figref{figex:3:47a}, we expect the adjectives after the pronominal determiners to pattern with those after \textit{ein}-words – the featural specification in \figref{figex:3:47a} is similar to the article \textit{ein} in that Impoverishment is not triggered in [--F, --O] contexts. In other words, the masculine/neuter instances in the nominative/accusative are expected to surface with strong adjectives. By contrast, Impoverishment is triggered in all other featural contexts, yielding weak adjectives in the remaining cases (including the nominative/accusative feminine, where both the strong and the weak inflections are -\textit{e}). However, it is not clear how to account for the strong inflections in the oblique and/or plural instances. By assumption, there is no other structural option available as regards the larger DP (as all these pronominal DPs involve canonical structures). Furthermore, note that an explanation involving analogy with other, related DPs is not straightforward as definite DPs of the form “determiner + adjective + noun” do not have strong inflections in these instances either. This means that the strong inflections in the dative singular and in the plural instances remain unaccounted for. The variant in \figref{figex:3:47b} fares better.

If \figref{figex:3:47b} is adopted, all adjectives should be strong. With no feature present that makes [+D] a trigger of Impoverishment, feature deletion does not occur at all. This accounts for the general option of strong adjectives. The restricted occurrence of weak adjectives requires a different account. There are three cases to consider. Note that all three can be related by certain phonetic considerations.

First, recall that instances in the dative masculine/neuter involve strong adjectives ending in -\textit{em} or weak adjectives ending in -\textit{en}. Notice that the weak inflection has a phonetic resemblance to the strong inflection – both involve nasal sounds. Extending the discussion in \sectref{sec:3.2.2}, I assume that this presents another case of nasal alternation such that -\textit{em} can be changed into -\textit{en}, provided a nominal element (here, the pronominal) precedes the adjective.

Second, the dative feminine instances involve phonetic resemblance between – what sounds like – strong inflections on the pronominal determiners \textit{mi-r} ‘me(\textsc{dat})’ and \textit{di-r} ‘you(\textsc{dat.sg})’ and the strong inflection -\textit{er} on the following adjectives. This brings about a special distribution such that two strong inflections occur on two different lexical categories at the same time (i.e., the pronominal determiner and adjective). Sequences of two strong inflections in cases like \textit{mi-r} \textit{groß-er Gans} ‘me (stupid idiot)’ do not exist elsewhere in the system. Following \citeauthor{Duden1995} (\citeyear{Duden1995}: 281, \citeyear{Duden2007}: 39), I assume that speakers tend to avoid such strings. Note again that the final sound on \textit{mir} and \textit{dir} is the same as the inflection on the dative feminine article \textit{der} ‘the’. I suggest that the weak adjectives are the result of analogy with other definite DPs (e.g., \textit{d-er nett-en Frau} ‘the-\textsc{st} nice-\textsc{wk} woman’).

Before moving on to the third case, the plural instances, notice that there is no phonetic rememblance in the nominative/accusative singular, neither as regards the nasal alternation on the adjectives nor as regards the inflections on the adjectives and the final sound on the pronominal determiners: The strong inflections -\textit{er}, -\textit{en}, -\textit{es}, -\textit{e} are very different from (or identical to) the weak inflection -\textit{e}, and all these inflections are different from the final sound on the pronominal determiners \textit{ich} ‘I’, \textit{mich} ‘me(\textsc{acc})’, \textit{du} ‘you(\textsc{nom.sg})’, \textit{dich} ‘you(\textsc{acc.sg})’. This means that the phonetic rule and analogy do not play a role in the nominative/accusative singular instances – as mentioned above, there is no variation here at all.

The third case involves the instances in the plural. Starting with the dative, recall that the inflection is -\textit{en}, which is ambiguous between strong and weak. Given my analysis, I interpret this inflection as strong. Continuing with the nominative, note that we cannot easily state that the weak endings here are also due to the phonetic considerations mentioned above: The strong inflection -\textit{e} is different from the weak inflection -\textit{en}, and both of these inflections are different from the final sound on the pronominals \textit{wir} ‘we’ and \textit{ihr} ‘you(\textsc{nom.pl})’. However, like in the dative feminine, the forms of the ordinary and pronominal determiners are related. Unlike in the dative feminine, here the relatedness concerns the fact that \textit{die} ‘the/those’ has no inflection, just like \textit{wir} and \textit{ihr}. I suggest that the weak adjectives in the nominative plural are also due to analogy with other definite DPs (e.g., \textit{die nett-en Frauen} ‘the nice-\textsc{wk} women’). Finally, observe that uninflected \textit{die} also occurs in the accusative plural. However, as discussed in \sectref{sec:3.5.2.4}, pronominal DPs in the accusative plural are special. I assume that the analogy is blocked here – the absence of these weak adjectives has a functional explanation.

Above, I made use of analogy with ordinary definite DPs in two instances, in the dative feminine and in the nominative plural. In order to constrain the application of analogy, I have assumed that analogy is only possible if there are special phonetic circumstances as regards the determiners: related endings on \textit{mi-r/di-r} and \textit{d-er}, and no inflections on \textit{wir/ihr} and \textit{die}. Note in passing that the presence or absence of endings on determiners seems to play a role here for the inflections on adjectives. This is reminiscent of the traditional generalization of Weak After Strong. However, both cases bring about weak adjectives by analogy, making these cases different from \textit{ein}-words (which have strong endings in the exceptional cases).

To sum up, all pronominal DPs may involve strong adjectives. Assuming complementation, this is explained by the reduced demonstrative structure in \figref{figex:3:47b} and the feature specifications of the first and second-person pronominals. As a consequence, Impoverishment does not occur, resulting in strong adjectives. Weak adjectives in the dative masculine/neuter are due to the fairly general phonetic rule that accounts for the nasal alternation on adjectives. Weak adjectives in the dative feminine and nominative plural are due to analogy (constrained by the similarity of the final sounds of the pronominal determiners and the inflections of the ordinary determiners). In other words, weak adjectives in all these cases only occur under specific phonetic conditions. Also, note again that masculine/neuter pattern in opposition to feminine/plural, which, as seen before, is a more general property of the nominal system in German. Before turning to the discussion of the different preferences as regards strong or weak adjectives, I provide the vocabulary insertion rules for pronominal elements.

\subsubsection{Vocabulary insertion rules for pronominal elements}\label{sec:3.5.3.2}

I illustrate the vocabulary insertion rules with first-person elements (second-person items are like the first-person ones but have a negative value for [AUTH]). Third-person pronominals are provided for the sake of completeness (for some discussion, see also \citealt{Höhn2020}). Starting with the first-person pronominals, I take genitive pronouns to be the most specific elements. Recall though that they do not take overt complements. This is indicated in \REF[a-b]{ex:3:48} such that no overt element can be in the right edge of the noun phrase. Note also that the CNG features after the forward slash sign in \REF{ex:3:48} specify the featural context of insertion. As to the pronominals that do take overt complements, I pointed out in \sectref{sec:3.5.1} that \textit{wir} ‘we’ is the only (transitive) pronominal that has a unique form for the relevant CNG features. Indeed, singular pronominals do not distinguish different genders, and \textit{uns} ‘us(\textsc{acc/dat})’ is the only form that does not distinguish (accusative/dative) case. Importantly, note again that accusative and dative case do not form a general natural class in German (i.e., they are not syncretic except here). Given these points, I take \textit{wir} as the most specific transitive element \REF{ex:3:48c}, and \textit{uns} as the least specific transitive item \REF{ex:3:48g}. The latter presents the elsewhere case. The remaining, singular pronouns are provided in \REF[d-f]{ex:3:48}. The vocabulary insertion rules below are ordered by decreasing specificity and are grouped according to case. Similar to the weak adjectival ending -\textit{e} (\chapref{sec:2}, \sectref{sec:2.2.1.5}), I utilize the category variable [$\gamma$], where a negative value indicates singular \REF[e-f]{ex:3:48}.

\ea%48
    \label{ex:3:48}
\ea\label{ex:3:48a} [+D; +AUTH, +PART]  $\rightarrow$   \textit{unserer} / {\longrule}]\textsubscript{φ} [+F, +N, +O, +S]
\ex\label{ex:3:48b} [+D; +AUTH, +PART]  $\rightarrow$  \textit{meiner}  / {\longrule}]\textsubscript{φ} [+O, +S]
\ex\label{ex:3:48c} [+D; +AUTH, +PART]  $\rightarrow$  \textit{wir}    /          [+F, +N, --O,  --S]
\ex\label{ex:3:48d} [+D; +AUTH, +PART]  $\rightarrow$  \textit{ich}        /          [--O,  --S]
\ex\label{ex:3:48e} [+D; +AUTH, +PART]  $\rightarrow$  \textit{mich}     /          [--$\gamma$,  --O]
\ex\label{ex:3:48f} [+D; +AUTH, +PART]  $\rightarrow$  \textit{mir}   /          [--$\gamma$]
\ex\label{ex:3:48g} [+D; +AUTH, +PART]  $\rightarrow$  \textit{uns}
\z
\z

To round off the discussion, I also provide the vocabulary insertion rules for the third-person pronominals.

Without going into much detail here, it is clear that pronominals like \textit{er} ‘he’ have to be distinguished from ordinary determiners like \textit{der-}words\textit{, ein-}words, or the null articles. Note that pronominals are inherently definite but that ordinary determiners are specified for definiteness (see again the different determiner structures in \chapref{sec:2}, \sectref{sec:2.2.1.6}). As such, I assume that the features [--AUTH] and/or [--PART] also entail definiteness. This allows me to leave out the feature for definiteness in the vocabulary insertion rules, provided in \REF{ex:3:49}.\footnote{If this turns out to be untenable, then the feature [+DEF] could be added to these pronominals. Note that this does not affect the discussion of Impoverishment as third-person pronominals do not take adjectives (and nouns) as their complement.} As above, the rules are ordered with decreasing specificity and are grouped according to case (if possible). Note that unlike first and second-person pronominals, their third-person counterparts involve adjectival inflections. Consequently, they have separate CNG feature bundles. As briefly discussed in \chapref{sec:2}, \sectref{sec:2.2.2.3}, I single out exceptional \textit{ihn} ‘him’ as the most specific case \REF{ex:3:49a}. Recall also that I pointed out above that the genitive pronominals do not have regular inflections. As such, I assume that like uninflected \textit{ein}, they spell out the stem features and the CNG features by one morpheme \REF[b-c]{ex:3:49}. Furthermore, \REF{ex:3:49c} and \REF{ex:3:49d} are equally specified but differ in the value of [O]. The remaining pronominals are given in \REF[e-f]{ex:3:49}, with \REF{ex:3:49f} forming the elsewhere case. The inflections for \REF{ex:3:49a} and \REF[d-f]{ex:3:49} are given in \REF[g-h]{ex:3:49} (for some discussion, see also \citealt{GunkelEtAl2017}: 1298).\footnote{The
  elsewhere case in \REF{ex:3:49f} involves an additional -\textit{en} in the dative plural. This makes \textit{ih-n-en} ‘them’ similar to the dative plural pronominal \textit{d-en-en} ‘those (ones)’ (for the historical development of this additional -\textit{en}, see \citealt{Lühr1991}). Interestingly, this additional -\textit{en} is not possible with \textit{ein}-words (e.g., *\textit{mein-en-en} ‘mine’, *\textit{kein-en-en} ‘none’). Again, what the latter elements have in common is that they consist of vacuous \textit{ein} and another component. The obvious difference between the first type of element and \textit{ein} is that the first type involves definiteness. There are different ways to account for this additional -\textit{en} on the definite pronominals. We could follow, for instance, work by \citet{CorvervanKoppen2010,CorvervanKoppen2011b}, who propose that -\textit{en} in Dutch is generated in a lower position, or we could formulate a late insertion rule. I leave the decision between these options for future research.}

\newpage
\TabPositions{5.5cm}
\ea%49
    \label{ex:3:49}
\ea\label{ex:3:49a} [+D; --AUTH, --PART]              \tab  $\rightarrow$   \textit{ih-}         / {\longrule}]\textsubscript{φ} [--F, --N, --O, +S]
\ex\label{ex:3:49b} [+D; --AUTH, --PART][--F, +O, +S] \tab  $\rightarrow$   \textit{seiner}    / {\longrule}]\textsubscript{φ}
\ex\label{ex:3:49c} [+D; --AUTH, --PART][+O, +S]    \tab $\rightarrow$   \textit{ihrer}    / {\longrule}]\textsubscript{φ}
\ex\label{ex:3:49d} [+D; --AUTH, --PART]           \tab  $\rightarrow$   \textit{e-}    / {\longrule}]\textsubscript{φ} [--F, --O]
\ex\label{ex:3:49e} [+D; --AUTH, --PART]           \tab  $\rightarrow$   \textit{sie-}        / {\longrule}]\textsubscript{φ} [--O]
\ex\label{ex:3:49f} [+D; --AUTH, --PART]           \tab  $\rightarrow$   \textit{ih-}  / {\longrule}]\textsubscript{φ}
\ex\label{ex:3:49g} [+F, --N, +O, $\alpha$S]      \tab    $\rightarrow$  \textit{-r}
\ex\label{ex:3:49h} etc.
\z
\z

Recall from \sectref{sec:3.3} that the inflection on \textit{sie} ‘she, her; they, them’ is later deleted to avoid a hiatus.

\subsubsection{Preferences for strong or weak adjectives}\label{sec:3.5.3.3}

As discussed above, there is no variation as regards the inflections on the adjectives after \textit{ich} ‘I’, \textit{mich} ‘me(\textsc{acc})’, \textit{du} ‘you(\textsc{nom.sg})’, \textit{dich} ‘you(\textsc{acc.sg})’, \textit{uns} ‘us(\textsc{acc/dat})’, \textit{euch} ‘you(\textsc{acc/dat.pl})’. All these adjectives are strong – there are no specific phonetic conditions that induce the phonetic rule from \sectref{sec:3.2.2} or facilitate analogy with other definite DPs. In contrast, dative singular \textit{mir} ‘me(\textsc{dat})’ and \textit{dir} ‘you(\textsc{dat.sg})’, and nominative plural \textit{wir} ‘we’ and \textit{ihr}\linebreak ‘you(\textsc{nom.pl})’ do involve variation where both strong and weak adjectives are possible. The account of this variation was based on certain phonological conditions. Interestingly, the preference for either strong or weak adjectives is not the same. I propose that this has to do with the underlying features of the larger noun phrase and the specificity of the vocabulary insertion rules of the pronominal determiners.

Recall that elements in DPs, including pronominal DPs, share concord features in case, number, and gender. As is well known, ordinary DPs involving count nouns require a determiner. In fact, as has been observed for German, it is the determiner that typically spells out the features for case, number, and gender on its inflection.\footnote{This is particularly clear with gender, which is inherently specified on the noun but has no overt manifestation on it.} The inflection on the determiner is typically strong. This is different for the cases under discussion here. While first and second-person pronominals also combine with count nouns, they do not have inflections. In fact, they do not have spell-out forms specified for the different genders in the singular, and with the exception of \textit{wir} ‘we’ and \textit{ihr} ‘you(\textsc{nom.pl})’, they do not have spell-out forms specified for the different cases in the plural (\sectref{sec:3.5.3.1}). I propose that if an adjective is present, it is the latter that preferably surfaces with a strong ending thereby providing a more-specified spell-out form of the underlying features of the noun phrase.

\tabref{tab:3:9} lists the four pronouns that involve varying inflections on their following adjectives. It shows which case, number, or gender the pronominal forms distinguish, and it contains the featural contexts in which their corresponding vocabulary insertion rules apply (\sectref{sec:3.5.3.2}). Finally, the table repeats the general variation and the varying preferences, pointing out again that strong adjectives are preferred in dative masculine/neuter contexts and weak adjectives in dative feminine and nominative plural environments (the varying specification of [F] in the last column distinguishes the cases in the masculine/neuter and feminine).

\begin{table}
\small
\caption{Specifications of pronominal elements and their relation to the inflection on the following adjective}
\label{tab:3:9}
\begin{tabularx}{\textwidth}{p{19mm}@{}lp{25mm}@{}Q@{}Q}
\lsptoprule
Pronominal \newline element & \multicolumn{2}{c}{Unique form for} & Featural contexts\newline of vocabulary insertion rules & Ending on the following adjective\\
\cmidrule(r){2-3}
& Case & Number/gender &  &\\
\midrule
\textit{mir}, \textit{dir} & dative & unspecified\newline (gender) & [-$\gamma$] & \mbox{[--F]:   strong/\textsuperscript{\%}weak}\\
&  &  &  & \mbox{[+F]:       weak/\textsuperscript{\%}strong}\\
\textit{wir}, \textit{ihr} & nominative & plural & \mbox{[+F, +N, --O, --S]} &\relax \hphantom{[--F]:} weak/\textsuperscript{\%}strong\\
\lspbottomrule
\end{tabularx}
\end{table}

It is important to state that, while adjectival inflections spell out CNG features (\chapref{sec:2}, \sectref{sec:2.2.1.5}), pronominals of the first and second person do not. As seen in the previous subsection, these pronominals involve a reduced demonstrative structure and are inserted in the context of CNG features. Note though that the vocabulary insertion rules of adjectival inflections and pronominal determiners do have something in common: They are ordered as regards the specificity of insertion such that strong inflections take precedence over weak inflections, and \textit{wir} ‘we’ and \textit{ihr} ‘you(\textsc{nom.pl})’ take precedence over the other relevant pronominal determiners.

Abstracting away from the dative feminine instances for a moment, \tabref{tab:3:9} suggests a certain correlation provided in \REF{ex:3:50a} and \REF{ex:3:50b}. This correlation can be collapsed into the statement in \REF{ex:3:50c}.

\ea%50
    \label{ex:3:50}
\ea\label{ex:3:50a}   less-specified pronominal {\textasciitilde} preference for more-specified (/strong) adjective
\ex\label{ex:3:50b}   more-specified pronominal {\textasciitilde} preference for less-specified (/weak) adjective
\ex\label{ex:3:50c}  DPs prefer one element to be more specified (where the element is of the lexical       category determiner or adjective).
\z
\z

Some comments are in order here. In both \REF{ex:3:50a} and \REF{ex:3:50b}, a more-specified and a less-specified element combine (cf. again the last two columns of \tabref{tab:3:9}). Put differently, it appears as if more-specified pronominals and strong adjectives have something in common, and less-specified pronominals and weak adjectives have something in common – the two groups differ in their degree of specificity and stand in opposition to each other. Note again that this correlation is not about the spell-out of features. Rather, it is about the specificity of the vocabulary insertion rules involved.

Furthermore, the unified, more general correlation in \REF{ex:3:50c} states a preference, not a requirement. Since the form of the pronominal is determined by the morpho-syntactic and semantic context, the actual choice has to do with the vocabulary insertion rule that yields the preferred inflection on the adjective (NB: The use of the term preference is also compatible with the fact that adjectives are generally optional). Observe that this is different from, for instance, the generalization Weak After Strong, which describes an either/or distribution of the different inflectional forms. Finally, note also that \REF{ex:3:50c} is not in conflict with Weak After Strong as \REF{ex:3:50c} involves a weaker statement. With these preliminary remarks in place, consider the three individual cases distinguished in the last column of \tabref{tab:3:9}.

I start with \textit{wir} ‘we’ and \textit{ihr} ‘you(\textsc{nom.pl})’. These elements are the only relevant first and second-person pronominals whose vocabulary insertion rules involve a fully specified context as regards CNG features (i.e., [+F, +N, --O, --S]). I propose that since all the underlying features of the larger noun phrase are part of the vocabulary insertion rules for \textit{wir} and \textit{ihr}, the less-specified weak inflection is preferred. The latter involves -\textit{en}, the vocabulary insertion rule of which is the elsewhere case.

Singular pronominals of the first and second person are not specified for gender. The vocabulary insertion rules of the dative singular pronominals \textit{mir}\linebreak ‘me(\textsc{dat})’ and \textit{dir} ‘you(\textsc{dat.sg})’ are specified as [-$\gamma$]. Note that this involves a low degree of specificity. Starting with the dative masculine/neuter instances, I propose that the inflectable adjective is recruited to indicate the gender of the larger noun phrase. Specifically, since gender is not specified in the vocabulary insertion rules of the relevant pronominals, a more-specified inflection on the adjective is preferred. Note in this regard that the vocabulary insertion rule for the strong inflection -\textit{em} is specified as [-F, +O, -S] (see \chapref{sec:2}, \sectref{sec:2.2.1.5}). This explains the preference for the strong adjective in the dative masculine/neuter instances.

Finally, I turn to the feminine singular cases. Given that the dative pronominal in feminine contexts is also specified as [-$\gamma$], the question arises why these instances prefer weak adjectives. I suggest that this involves the same type of explanation that was offered in \sectref{sec:3.5.3.1} to account for the very existence of weak adjectives in this context. Again, I follow \citeauthor{Duden1995} (\citeyear{Duden1995}: 281, \citeyear{Duden2007}: 39) in observing that the feminine instances lead to – what sounds like – the sequence of two strong endings (e.g., \textit{mi-r} \textit{groß-er Gans}). It seems as if speakers tend to avoid similar endings on co-occurring but lexically different elements (i.e., the pronominal element and adjective). Note though, if -\textit{r} on the pronominal element were to be interpreted as a strong inflection, then it would be possible to claim that this element provides the more-specified form (observe in this regard that the vocabulary insertion rule of -\textit{er} is specified as [+F, --N, +O, $\alpha$S]). Unlike in the dative masculine/neuter instances, here the “inflected” pronominal element itself would provide the more-specified form. This would make these instances similar to \textit{wir} ‘we’ and \textit{ihr} ‘you(\textsc{nom.pl})’ as regards a higher degree of specificity. Indeed, this parallelism would be in line with the general observation that feminine and plural often behave alike.

Note that the preference for strong adjectives discussed above is consistent with the instances that do not involve variation. Recall that the pronominal determiners \textit{ich} ‘I’, \textit{mich} ‘me(\textsc{acc})’, \textit{du} ‘you(\textsc{nom.sg})’, \textit{dich} ‘you(\textsc{acc.sg})’, \textit{uns} ‘us(\textsc{acc/ dat})’, and \textit{euch} ‘you(\textsc{acc/dat.pl})’ only take strong adjectives. Importantly, none of these elements involve forms specified for gender in the singular or case in the plural. With that in place, I briefly comment on the different degrees of specificity of the vocabulary insertion rules for adjectival inflections and pronominal elements in the last part of this subsection. This includes a brief discussion of how these vocabulary insertion rules relate to the terminal nodes in the syntactic representation.

The different adjectival inflections and pronominal elements are not specified for all the features present in the underlying syntactic structure. Having adopted DM, vocabulary insertion rules involve underspecification, which regulates the order of the application of the rules. In other words, the absence of features in the vocabulary insertion rules has to do with the very assumptions of how Vocabulary Insertion works in DM. Indeed, the varying numbers of features in rules lead to different degrees of specificity of those rules. For instance, the strong inflection -\textit{er} has more features and is thus more specified than the strong inflection -\textit{em}, and both of these are more specified than the weak inflection -\textit{en}, the elsewhere case. Similar considerations hold for the contexts of vocabulary insertion of pronominal determiners.

Returning to the correlations in \REF{ex:3:50}, note that the more-specified elements involve three features (-\textit{em}: [--F, +O, --S]) or four features (e.g., \textit{wir}: [+F, +N, --O, --S]). By contrast, the less-specified elements contain no feature (-\textit{en}: []), one feature (e.g., \textit{mir}: [--$\gamma$]), or two features (-\textit{e}: [--$\gamma$, --O]). Comparing adjectival inflections to each other and pronominal forms to each other, they each involve a difference of at least one CNG feature. More importantly, comparing the preferred co-occuring elements to each other, they each involve a difference of (at least) two CNG features: \textit{mir} (1) + Adj-\textit{em} (3); \textit{wir} (4) + Adj-\textit{en} (0). In contrast, the dispreferred co-occurring elements have other distinctions: The difference in the degree of specificity is reversed but still smaller (\textit{mir} (1) + Adj-\textit{en} (0)) or just smaller (\textit{wir} (4) + Adj-\textit{e} (2)). As is clear, the various combinations constitute different degrees of specificity that find overt manifestation in the varying preferences for the different inflections on the adjectives. 

Finally, note that the specification of the vocabulary insertion rules differs from that of the terminal nodes in the syntactic representation. Provided that the relevant terminal nodes are present, they involve all CNG features. In other words, all CNG features are present in the abstract representation, but only a subset of those features are specified in the vocabulary insertion rules to be applied. Given the discussion above, it appears then as if German prefers one more and one less-specified element to co-occur, provided such a choice is available to begin with (i.e., Impoverishment does not occur).

\subsection{Summary of the discussion}\label{sec:3.5.4}

All pronominal DPs involve canonical DPs; that is, pronominal determiners take adjectives and nouns as part of their complement structure. If modifiers are present, all pronominal DPs may exhibit strong adjectives. This is explained by the reduced demonstrative structure and the feature specifications involving [±AUTH] and [±PART]. With CNG, definiteness, and deixis features absent on the pronominal determiners, Impoverishment cannot occur, resulting in the strong adjectives. Besides strong adjectives, some instances may also involve weak adjectives. The weak adjectives in the dative masculine/neuter are part of the nasal alternation and can be accounted for by the phonetic rule from \sectref{sec:3.2.2}. The weak adjectives in the dative feminine and nominative plural follow from analogy with ordinary definite DPs (constrained by the similarity of the final sounds of the pronominal determiners and the inflections on the ordinary determiners). In other words, all the weak adjectives surface under specific phonetic conditions only.

The variation in the dative singular and in the nominative plural shows varying preferences as regards strong or weak inflections on adjectives. This preference is a reflex of the different degrees of specification of two co-occurring elements, the pronominal determiner and the inflection on the following adjective. The preference for strong adjectives in the dative masculine/neuter is due to the preference for more-specified forms provided by the adjectival inflection. In contrast, the preference for weak adjectives in the dative feminine and in the nominative plural is also due to the preference for more-specified forms, but here provided by the pronominal determiners themselves. As the above discussion has made clear again, masculine and neuter pattern in opposition to feminine and plural. In the next section, I turn to variation in a non-canonical structure. Like in \sectref{sec:3.3} and \ref{sec:3.4}, it is lexically restricted.

\section{Non-canonical structures with optional inflections on predeterminer \textit{alle}} \label{sec:3.6}

In this section, I discuss the presence and absence of the strong endings on the predeterminer \textit{alle} ‘all’ in the plural. Recall from \chapref{sec:1}, \sectref{sec:1.4.1.2} that a predeterminer is a type of determiner-like element that intensifies the meaning of the following DP in a certain way.

As is well known, both uninflected \textit{all} and inflected \textit{alle} can appear before elements such as possessive determiners \REF{ex:3:51} or demonstratives \REF{ex:3:52}. Interestingly, there is a restriction such that the co-occurrence of inflected \textit{alle} and a definite article results in slight degradedness \REF{ex:3:53b}. Note that in this section, I gloss determiners like \textit{die} ‘the/those’ differently. Recall that they are pronounced as [di(:)]; that is, they are monosyllabic. This will become important below.

\ea%51
    \label{ex:3:51}
\ea  \label{ex:3:51a}
\gll all mein-e Freunde\\
all my-\textsc{st}  friends\\
\glt ‘all my friends’
\ex\label{ex:3:51b}
\gll all-e   mein-e Freunde\\
all-\textsc{st} my-\textsc{st}  friends\\
\glt ‘all my friends’
\z
\z

\ea%52
    \label{ex:3:52}
\ea\label{ex:3:52a}
\gll all dies-e    Leute\\
all these-\textsc{st} people\\
\glt ‘all these people’
\ex\label{ex:3:52b}
\gll all-e   dies-e     Leute\\
all-\textsc{st} these-\textsc{st} people\\
\glt ‘all these people’
\z
\z

\ea%53
    \label{ex:3:53}
\ea\label{ex:3:53a}
\gll all die         Leute\\
all the.\textsc{nom} people\\
\glt ‘all the people’
\ex[?]{\label{ex:3:53b}
\gll all-e   die          Leute\\
all-\textsc{st} the.\textsc{nom} people\\
\glt ‘all the people’
}
\z
\z

Before proceeding, a remark on these sets of data is necessary. \citet[197-200]{Cirillo2016} states that inflected \textit{alle} in the (b)-examples above is marked or even ungrammatical for some speakers; that is, uninflected \textit{all} is the dominant or correct form. While I agree that \REF{ex:3:53b} is indeed marked, \REF{ex:3:51b} and \REF{ex:3:52b} are completely fine for me. An informal Google-search has revealed that uninflected \textit{all} is actually less frequent than inflected \textit{alle}. The results are provided in \tabref{tab:3:10a} (M stands for million).

\begin{table}
\caption{\label{tab:3:10a} Numeric results of \textit{all(e)} identified by Google (ca. 2020)}
\begin{tabularx}{0.4\textwidth}{XYY}
\lsptoprule
 & {\itshape all} & {\itshape alle}\\
\midrule
 {\itshape meine} & 1.4M & 8.8M\\
{\itshape diese} & 16.5M & 21.2M\\
\lspbottomrule
\end{tabularx}
\end{table}

While I checked many instances for relevance, these numbers were too large to check every single example. Note though that while there are some false positives, it is clear that inflected \textit{alle} is fine for many speakers. A reviewer points out that the inflected forms may be more frequent in written German.

I also conducted a search in the DGD, specifically, the corpus of spoken German called \textit{Forschungs- und Lehrkorpus Gesprochenes Deutsch} (FOLK) (Research and Teaching Corpus of Spoken German), and I found similar results. Consider \tabref{tab:3:10b}.

\begin{table}
\caption{\label{tab:3:10b} Numeric results of \textit{all(e)} identified in FOLK (October 31, 2024)}
\begin{tabularx}{0.4\textwidth}{XYY}
\lsptoprule
 & {\itshape all} & {\itshape alle}\\
\midrule
 {\itshape meine} & 1 & 9\\
{\itshape diese} & 9 & 12\\
\lspbottomrule
\end{tabularx}
\end{table}

These numbers are much smaller. Note that every example was checked for relevance, and a few examples were removed – either they were not relevant or their status was somewhat unclear. Observe though that the relative frequencies of the uninflected and inflected forms of \textit{all}- here are the same as in \tabref{tab:3:10a} above. In other words, inflected \textit{alle} seems to be more frequent than uninflected \textit{all}, in both written and spoken contexts. As such, I point out that \citegen{Cirillo2016} description of the data is not entirely accurate. Given the data above, there are two questions that arise here. First, why is the inflection on \textit{all}- optional in \REF{ex:3:51} and \REF{ex:3:52}? Second, why is the inflection in \REF{ex:3:53b} dispreferred?

Starting with the first question, I proposed in \chapref{sec:2}, \sectref{sec:2.4} that \textit{alle} ‘all’ in the cases above is located in LPP. Not being part of the DP proper at any point, the predeterminer \textit{alle} can neither trigger nor undergo Impoverishment, explaining the strong ending on \textit{alle} itself and on the following determiner. Consider the following simplified structure.

\glltree[\label{figex:3:54}]{
	\gll alle meine Leute\\
        all my people\\
    \glt ‘all my people’
}{
		[LPP
			[\textit{alle}]
			[DP
				[D\\\textit{meine}]
				[NP\\\textit{Leute}]
			]
		]
}

\citet[264-66]{Pafel1994} makes a similar syntactic proposal for inflected \textit{alle} such that this predeterminer selects a DP complement.\footnote{\citet[217-18]{Bhatt1990} claims that \textit{alle}, \textit{diese}, and \textit{meine} in \textit{alle diese meine Freunde} ‘all these friends of mine’ are heads. She proposes that \textit{alle} is adjoined to \textit{diese} and the resulting complex is adjoined to \textit{meine}. The entire complex is inside Spec,DP. Note that it is not a standard assumption that heads reside in phrasal positions.} Several options have been proposed to account for the absence of the inflection on \textit{all}-. There are two basic types.

  First, a number of authors have proposed that uninflected \textit{all} is in a different position. For instance, \citet{Pafel1994} proposes that \textit{all} is adjoined to the determiner in D. \citet[701]{Bošković2004} argues that while inflected \textit{alle} is either left-adjoined to DP or in a higher position, uninflected \textit{all} is in that higher position (left unspecified by Bošković, see his Footnote 23). Given the discussion above, this higher position can be related to the LPP-layer.

The second type of proposal involves the claim that inflected and uninflected \textit{all}- start out in the same position but that uninflected \textit{all} undergoes some further operation. To name just two ideas, we could build on Pafel suggesting that uninflected \textit{all} is a deficient element that starts out in LPP but undergoes adjunction to D late in the derivation. Alternatively, we could formulate an Impoverishment rule that deletes the inflection optionally (cf. Footnote \ref{foot:3:11}). Since the account of the inflection on \textit{all}- has repercussions for the analysis of floating quantifiers (where \textit{all}- must have an inflection), I will not pursue this question further here and turn to the second query.\footnote{Notice also that it is sometimes claimed that uninflected \textit{all} and inflected \textit{alle} involve different semantics (see \citealt{Merchant1996}: 183 and \citealt{KobeleZimmermann2012}: 249). For some brief discussion, see \chapref{sec:8}.}

  As far as I am aware, there is no explanation of the contrast in \REF{ex:3:53}, neither in Pafel’s nor in any other work. Following an idea from Björn Köhnlein (p.c.), I make a tentative suggestion at the end of this section. To begin, note that the contrast above is found in the other morphological cases as well, here illustrated with the dative.\footnote{Like \textit{all}-, the demonstrative \textit{dies}- ‘this/these’ can also do double duty as a determiner and as a predeterminer. Unlike \textit{all}-, the inflection of \textit{dies}- can only be left out in the nominative/accusative neuter, yielding \textit{dieses} vs. \textit{dies} (see \chapref{sec:4}, \sectref{sec:4.5}).}

\ea%55
    \label{ex:3:55}
\ea\label{ex:3:55a}
\gll mit   all den       Leuten\\
with all the.\textsc{dat} people\\
\glt ‘with all the people’
\ex[?]{\label{ex:3:55b}
\gll mit   all-en den        Leuten\\
with all-\textsc{st} the.\textsc{dat} people\\
\glt ‘with all the people’
}
\z
\z

If the (lower) DP is replaced by a disyllabic pronoun, the counterparts of \REF{ex:3:55a} and \REF{ex:3:55b} are both fine.

\ea%56
    \label{ex:3:56}
\ea\label{ex:3:56a}
\gll mit   all denen\\
with all those\\
\glt ‘with all those’
\ex\label{ex:3:56b}
\gll mit   all-en denen\\
with all-\textsc{st} those\\
\glt ‘with all those’
\z
\z

The obvious difference between these plural elements is that possessive determiners and demonstratives are disyllabic but that definite articles are monosyllabic. Assuming that LPP constitutes the left periphery of the noun phrase, I formulate a preliminary version of the relevant generalization.

\ea%57
    \label{ex:3:57}
Generalization (preliminary version):\\
A disyllabic determiner-like element in the left periphery cannot be followed by a monosyllabic determiner.
\z

Note that the generalization is silent about sequences starting with a monosyllabic element; that is, it says nothing about the grammatical (a)-examples above.

At first glance, it might be suggested that German prefers trochaic feet, which yield a certain phonotactic rhythm in the pronunciation. This suggestion seems to get confirmation when three determiner elements are present \REF{ex:3:58a}. However, we have already seen above that two monosyllabic elements do co-occur (\textit{all die} ‘all the’). Furthermore, as pointed out to me by David Fertig (p.c.), the string in \REF{ex:3:53b} above becomes better with stress on \textit{die} \REF[b]{ex:3:58}; that is, when \textit{die} functions as a demonstrative. Note that cases as in \REF{ex:3:58b} are often followed by a relative clause (also \citealt{Pafel1994}: 238).

\ea%58
    \label{ex:3:58}
\ea\label{ex:3:58a}
\gll all-e   dies-e     mein-e Freunde\\
all-\textsc{st} these-\textsc{st} my-\textsc{st}  friends\\
\glt ‘all these friends of mine’
\ex\label{ex:3:58b}
\gll all-e   DIE           Leute (,die   in einer Stadt wohnen)\\
all-\textsc{st} those.\textsc{nom} people  {\db\db}who in a        city   live\\
\glt ‘all those people (who live in a city)’
\z
\z

While \textit{die} is stressed now, it is still a monosyllabic element. I interpret multiple syllabicity and stress as related measures of heaviness. In other words, I assume that multiple syllables in an element, on the one hand, or stress on an element, on the other hand, leads to heaviness of a word. This groups disyllabic and stressed monosyllabic words together in opposition to unstressed monosyllabic words. With this in mind, I finalize the generalization as follows.

\ea%59
    \label{ex:3:59}
          Generalization (final version):\\
          A polysyllabic determiner-like element in the left periphery cannot be followed by a lighter determiner.
\z

Put differently, the relevant elements must be of the same weight or the relevant elements must increase (but not decrease) in weight from left to right. The question arises as to how to deal with this generalization.

Björn Köhnlein (p.c.) makes the interesting suggestion that the generalization above actually involves an epiphenomenon of the fact that articles are clitics. In more detail, phonology tends to avoid lapses, that is, sequences of two adjacent unstressed syllables in the same phonological word. Assuming that definite articles are clitics, the phonological word [\textit{all-die}] consists of a sequence of a stressed and one unstressed syllable. This is fine in the phonology. In contrast, the phonological word [\textit{alle-die}] consists of a sequence of a stressed and two adjacent unstressed syllables, something that is marked in the phonology (yielding the slight degree of grammatical markedness). Unlike articles, demonstratives and possessive determiners are not clitics, and they do not form a phonological word with \textit{all(e)}. Consequently, those sequences are fine, with or without the inflection on \textit{all}-. Similarly, when \textit{die} is stressed, it is no longer a clitic and forms a different phonological word.

In the last couple of sections, I discussed variation of adjectival inflections in noun phrases in Standard German. With this in mind, I return to the traditional generalizations Weak After Strong and the Principle of Monoinflection.

\section{Revisiting two traditional generalizations of adjectival inflections}\label{sec:3.7}

In this section, I return to the two traditional generalizations that describe the strong/weak alternation in German. To recapitulate, compare the canonical cases in \REF{ex:3:60}, where the adjective is weak provided it is preceded by a determiner with a strong inflection \REF{ex:3:60a}, but the adjective is strong in all other cases \REF[b-c]{ex:3:60}.

\ea%60
    \label{ex:3:60}
\ea\label{ex:3:60a}
\gll d-er    gut-e        Kaffee\\
the-\textsc{st} good-\textsc{wk} coffee.\textsc{masc}\\
\glt ‘the good coffee’
\ex\label{ex:3:60b}
\gll ein gut-er    Kaffee\\
a    good-\textsc{st} coffee.\textsc{masc}\\
\glt ‘a good coffee’
\ex\label{ex:3:60c}
\gll gut-er    Kaffee\\
good-\textsc{st} coffee.\textsc{masc}\\
\glt ‘good coffee’
\z
\z

This inflectional distribution can be summarized in the generalization in \REF{ex:3:61}, repeated from \chapref{sec:2}, \sectref{sec:2.2.1.1} (again, for similar statements, see \citealt{Bierwisch1967}: 257, \citealt{Eisenberg1998}: 171-73, \citealt{Gallmann1998}: 144, \citealt{Giusti2015}: 207, G. \citealt{Müller2002a}: 129, \citealt{Petrova2024}: 184-85, \citealt{Pfaff2017}: 286, \citealt{Rehn2019}: 58, \citealt{Sauerland1996}: 34, \citealt{Schoorlemmer2009}: 53, and many others).

\ea%61
    \label{ex:3:61}
Weak After Strong\\
          	An adjective with a weak inflection is preceded by a determiner with a strong inflection.
\z

Note again that this formulation does not explicitly mention adjectives that are not preceded by (inflected) determiners. However, since prenominal adjectives alternate only between weak and strong, the distribution of strong adjectives can be inferred from \REF{ex:3:61}.

  It is clear that reference to lexical categories like adjective vs. determiner or determiner-like element is essential to the proper workings of this generalization. As documented above, inflections on elements of the same lexical category are the same. Specifically, independently of the presence or absence of \textit{ein}, an adjective that follows a strong adjective must also be strong \REF{ex:3:62a}. As for determiners and determiner-like elements, they typically exhibit strong inflections. In fact, as long as the determiner(-like) elements appear in the order in \REF{ex:3:62b}, any combination of them yields a grammatical string.

\ea%62
    \label{ex:3:62}
\ea\label{ex:3:62a}
\gll (ein) gut-er    frisch-er Kaffee\\
 {\db}a     good-\textsc{st} fresh-\textsc{st}  coffee.\textsc{masc}\\
\glt ‘(a) good fresh coffee’
\ex\label{ex:3:62b}
\gll (all-e) (dies-e)  (mein-e) Freunde\\
 {\db}all-\textsc{st}  {\db}these-\textsc{st} {\db}my-\textsc{st}   friends.\textsc{masc}\\
\glt ‘all these friends of mine’
\z
\z

To be clear, the generalization Weak After Strong must take the different lexical categories of the elements involved into consideration. Recall that determiners and determiner-like elements have the categorial feature [+D], but adjectives do not. This lexical distinction prevents us from expecting the second adjective in \REF{ex:3:62a} and the non-initial determiner(-like) elements in \REF{ex:3:62b} to exhibit weak inflections.

Note that this generalization is surface-oriented, making reference to the notion of precedence. While I already pointed out in \chapref{sec:2}, \sectref{sec:2.2.1.1} that this generalization is, in and of itself, not an explanation, there are also two basic types of exceptions: (i) Strong adjectives can be preceded by determiners with strong inflections; (ii) weak adjectives can occur without determiners with strong inflections preceding them.

The first type of exception manifests itself when complex proper names are embedded in larger nominals. In these cases, strong adjectives that are part of proper names are preceded by the determiners of the larger nominals, where the determiners also have strong inflections.

\ea%63
    \label{ex:3:63}
\ea\label{ex:3:63a}
\gll d-em [Deutsch-e  Bahn]        Personal\\
the-\textsc{st} {\db}German-\textsc{st} Railroad.\textsc{fem} personnel.\textsc{neut}\\
\glt ‘(to) the German-Railroad personnel’
\ex\label{ex:3:63b}
\gll d-es  [Deutsch-e   Bank]      Chef-s\\
the-\textsc{st} {\db}German-\textsc{st} Bank.\textsc{fem} boss.\textsc{masc}-\textsc{gen}\\
\glt ‘(of) the German-Bank boss’
\z
\z

It is clear that any generalization (or proposal) regarding the strong/weak alternation must take structure into account. I argued above that the determiners are in nominals different from those of the complex proper names in \REF{ex:3:63}. As the determiners do not adjoin to any parts of the AgrPs containing the adjectives, Impoverishment does not occur. This explains the strong adjectives in \REF{ex:3:63}.

As to the second type of exception, weak adjectives routinely occur in the genitive masculine/neuter without any determiners being present \REF{ex:3:64a}. Furthermore, they also occur in the presence of Saxon Genitives, items often taken to be determiner(-like) elements \REF{ex:3:64b}. Observe that – what looks like – strong inflections (i.e., -\textit{s}) are present in both \REF{ex:3:64a} and \REF{ex:3:64b}. Crucially though, these inflections are on elements that follow (but not precede) the relevant adjectives – they are on the head nouns. Third, weak adjectives can be preceded by pronominal determiners, which do not have inflections in the first and second person \REF{ex:3:64c}. Finally, weak adjectives can also be preceded by uninflected \textit{dies} ‘this’ \REF{ex:3:64d}.\footnote{I argue in detail in \chapref{sec:4}, \sectref{sec:4.5} that uninflected \textit{dies} is not based on inflected \textit{dieses} such that the ending -\textit{es} has been deleted. This is different for \textit{die} ‘the/that/those’. Recall from \sectref{sec:3.3} that schwa has been deleted here. Note that, if the generalization above only applies to surface strings, then nominals involving \textit{die} present another exception to it.}

\ea%64
    \label{ex:3:64}
\ea\label{ex:3:64a}
\gll trotz     heiß-en Kaffee-s\\
despite hot-\textsc{wk}  coffee.\textsc{masc}-\textsc{gen}\\
\glt ‘despite hot coffee’
\ex\label{ex:3:64b}
\gll trotz     Peters  heiß-en Kaffee-s\\
despite Peter’s hot-\textsc{wk}  coffee.\textsc{masc}-\textsc{gen}\\
\glt ‘despite Peter’s hot coffee’
\ex\label{ex:3:64c}
\gll wir nett-en  Studenten\\
we nice-\textsc{wk} students\\
\glt ‘us nice students’
\ex\label{ex:3:64d}
\gll dies schön-e    Kleid\\
this pretty-\textsc{wk} dress.\textsc{neut}\\
\glt ‘this pretty dress’
\z
\z

The cases in \REF{ex:3:64} involve canonical structures. Above, I argued that Impoverishment Rule 2 is at work in \REF[a-b]{ex:3:64}. Turning to \REF{ex:3:64c}, the weak adjective is due to analogy with related definite DPs. As for \REF{ex:3:64d}, I argued that it is the feature [+D] on the determiner stem that triggers Impoverishment on the adjective, not the inflection on the determiner – as is clear from \REF{ex:3:64d}, such an inflection can be absent and yet, the adjective is weak. There are other issues with this generalization.

Strictly speaking, Weak After Strong is a generalization about inflections on adjectives. More precisely, it describes the dependency of the inflections on the adjectives with regard to the inflections on determiners. The inflectional behavior of determiners or determiner-like elements themselves is not covered. Since inflections on determiners, determiner-like elements, and adjectives are identical, it is desirable to put forth a generalization that captures the distribution of the inflections on all these elements.

A more general statement than the first generalization has come to be known as the Principle of Monoinflection, also repeated here from \chapref{sec:2},\linebreak \sectref{sec:2.2.1.1} (see also, e.g., \citealt{Darski1979}, \citealt{Esau1973}, \citealt{HelbigBuscha2001}: 274-76, \citealt{Murphy2018}: 344, \citealt{Roehrs2009a}: 135, \citealt{Wegener1995}: 105, 153, cf. \citealt{Nübling2011}: 178).\footnote{Note that a similar principle is also mentioned in the Duden reference grammar (\textit{Monoflexion}), but with a slightly different formulation, where the strong ending does not necessarily precede the weak one (e.g., \citealt{Duden2016}: 954).}

\ea%65
    \label{ex:3:65}
Principle of Monoinflection\\
The first inflected element within a noun phrase carries the strong and the second one the   weak ending.
\z

As with the first generalization, reference to lexical categories must be made. In other words, the first element of a nominal string is taken to be of a different lexical category than the second element. Given that determiner(-like) elements and adjectives can each be stacked, the notion of element (of a certain lexical category) must be understood as group of elements (of the same category). To briefly illustrate, given that \REF{ex:3:66} involves determiner(-like) elements (which all have the categorial feature [+D]), they are all considered to be one type of element as regards the relevant lexical category. As such, they are all taken to be first in the relevant sense, and the generalization correctly describes the fact that all of them have a strong ending.

\ea%66
    \label{ex:3:66}
\gll (all-e) (dies-e)  (mein-e) Freunde\\
   {\db}all-\textsc{st}  {\db}these-\textsc{st} {\db}my-\textsc{st}   friends\\
  \glt ‘all these friends of mine’
\z

By contrast, if determiner(-like) elements are absent, then adjectives are the first inflected element. As expected under \REF{ex:3:65}, they appear with strong inflections (cf. \REF{ex:3:60c} above).

While this generalization covers more empirical ground, it faces the same general challenges as Weak After Strong. Besides the issues already mentioned above, a determiner-like element with a strong ending may precede an adjective with a strong ending \REF{ex:3:67a}.\footnote{Note
  that this case cannot simply be discounted by stating that the adjective is not immediately preceded by \textit{dieses} ‘this’. This is so because weak adjectives can be preceded by determiners with a strong inflection even if an uninflected element intervenes.
	\ea
	\gll dies-e     zehn klein-en    Autos\\
	these-\textsc{st} ten    small-\textsc{wk} cars\\
	\glt ‘these ten small cars’
	\z
	
	 Again, a surface-oriented generalization involving precedence cannot capture cases like \REF{ex:3:67a} and others to be discussed below.} Similarly, taking indefinite pronouns to be determiner-like elements (\sectref{sec:3.2.2}), indefinite pronoun constructions present a similar problem \REF{ex:3:67b}.

\judgewidth{\%}
\ea%67
    \label{ex:3:67}
\ea[]{\label{ex:3:67a}
\gll dies-es mein groß-es   Glück\\
this-\textsc{st} my    great-\textsc{st}  happiness.\textsc{neut}\\
\glt ‘this great happiness of mine’
    }

\ex[\%]{\label{ex:3:67b}
\gll mit jemand-em ander-em\\
 with someone.\textsc{masc}-\textsc{st} different-\textsc{st}\\
\glt ‘with someone different’
  }
\z
\z


In \chapref{sec:2}, I argued that the demonstrative in \REF{ex:3:67a} is base-generated in LPP. As such, it can neither trigger nor undergo Impoverishment. As to \REF{ex:3:67b}, I formulated in \sectref{sec:3.2.2} three different paradigms for \textit{jemand} ‘someone’ and an optional phonetic rule, which has not applied to the inflection on the adjective in \REF{ex:3:67b}.

  The second type of exception concerns a weak ending on the first inflected element. This may hold with determiners \REF{ex:3:68a}, indefinite pronouns \REF{ex:3:68b}, and adjectives \REF{ex:3:68c}.

\ea%68
    \label{ex:3:68}
\ea[]{\label{ex:3:68a}
\gll im      Sommer dies-en Jahr-es\\
in.the summer this-\textsc{wk} year.\textsc{neut}-\textsc{gen}\\
\glt ‘in the summer of this year’
   }

\ex[\%]{\label{ex:3:68b}
\gll    mit {jemand-en}                ander-en\\
with someone.\textsc{masc}-\textsc{wk} different-\textsc{wk}\\
\glt ‘with someone different’
}

\ex[\%]{\label{ex:3:68c}
\gll   mit {jemand}               ander-en\\
with someone.\textsc{masc} different-\textsc{wk}\\
\glt ‘with someone different’
}
\z
\z


While the occurrence of these strings is admittedly restricted, these distributions do exist. In \sectref{sec:3.4}, I argued that determiners can optionally undergo Impoverishment in genitive masculine/neuter contexts \REF{ex:3:68a}. In \sectref{sec:3.2.2}, I provided three paradigms for \textit{jemand} ‘someone’ \REF[b-c]{ex:3:68} and an optional phonetic rule, which has applied to the inflections on the adjectives in \REF[b-c]{ex:3:68}. Finally, notice that both generalizations are silent about prenominal adjectives that can appear without inflections (e.g., \textit{das lila/prima/Berliner Auto} ‘the purple/great/Berlin car’).

\largerpage
Given these issues, I propose the following generalization. Note that the generalization below is not based on precedence but on structure, where canonical DPs play a crucial role.

\ea%69
    \label{ex:3:69}
    Strong/Weak Alternation in German:\\
	Disregarding a few (mostly) lexical exceptions, both adjectives and determiners have   strong or weak inflections. Weak inflections only occur in canonical DPs, either

    \ea   \label{ex:3:69a} in the context of certain lexical items, or
    \ex   \label{ex:3:69b} in certain featural contexts, or
    \ex   \label{ex:3:69c}  in a combination of (a) and (b).
    \z
Strong inflections occur in all other environments.
\z

To briefly illustrate, \REF{ex:3:69a} refers to adjectives following \textit{der}-words, \REF{ex:3:69b} involves adjectives and determiners in genitive masculine/neuter contexts, and \REF{ex:3:69c} is instantiated by adjectives following \textit{ein}-words that occur in the feminine, plural, and in the two oblique cases. Furthermore, \REF{ex:3:69c} is also manifested by adjectives in dative masculine/neuter contexts that are preceded by nominal elements, and by adjectives in the dative singular and in the nominative plural preceded by pronominal determiners. These are the main contexts where weak inflections occur. The lexical exceptions alluded to in \REF{ex:3:69} involve adjectives like \textit{lila}, \textit {prima}, etc; formations like \textit {Berliner} are more general exceptions.

To sum up, the generalization Weak After Strong is a good initial description of the canonical facts, but it does not capture the whole spectrum. The Principle of Monoinflection is more general but faces the same general issues. Many (but not all) of the issues with these generalizations reveal themselves when we consider non-canonical constructions. An analysis that seeks to be more comprehensive has to include the latter cases as well. If so, the different structures involved must be part of the account of the strong/weak alternation. As mentioned before, I argue that the non-canonical constructions reveal the true nature of adjectival inflections.

Thus far, I have discussed variation (and generalization) in the same dialect – colloquial Standard German. Next, I turn to variation of a different kind by discussing a regional variety. I chose Mannheim German for two reasons. First, the account developed above can, without adding too many details, be extended to this dialect, showcasing its more general applicability. Second and more importantly, the discussion of this regional dialect will reiterate the special status of \textit{ein}-words in the four exceptional cases, delineated in \chapref{sec:2}, \sectref{sec:2.2.2.2} as masculine/neuter in the nominative and accusative.

\section{Dialectal variation: Mannheim German}\label{sec:3.8}

In his critique of \citet{Bierwisch1967}, \citet{Blevins1995}, \citet{Wunderlich1997}, and B. \citet{Wiese1999}, G. \citet{Müller2002b} discusses a regional variety of German spoken in and around Mannheim, \textit{Mannheimer Regionalsprache} ‘Mannheim regional dialect’, a Standard German dialect that is influenced by the local Palatinate variety. For convenience, I refer to this dialect as Mannheim German. G. Müller observes that regular nominative masculine forms are used in accusative masculine contexts (data from G. \citealt{Müller2002b}: 354).\footnote{G. \citet[353-55]{Müller2002b} provides more details about this locally colored Standard German variety. There are three points of interest here: The accusative masculine personal pronoun is \textit{en} ‘him’ (not \textit{er}), endings including weak adjectival endings seem to be systematically reduced from -\textit{en} to -\textit{e} – a general feature of the dialect, and the weak ending -\textit{e} in the nominative masculine is often left out. This means that personal pronouns have their own insertion rules (at least in this dialect), that phonological rules may obscure the underlying inflectional paradigms, and that some endings may be optional (possibly analyzed as instances of Impoverishment). I abstract away from these finer points here. Also, as \citet[210-14]{Russ1989} points out, this Mannheim German dialect should not be equated with regional varieties that are spoken in Mannheim or near this geographical area (see, for instance, \citealt{Russ1989}: 231, 251 on the German spoken in Michelstadt in southern Hesse and in Kaulbach in western Palatinate). I thank a reviewer for making me aware of this issue.}

\ea%70
Mannheim German
    \label{ex:3:70}
\ea\label{ex:3:70a}
\gll Ich kenn [d-er   ander-e    Mann].  \\
    I    know  {\db}the-\textsc{st} other-\textsc{wk} man.\textsc{masc}\\
\glt ‘I know the other man.’
\ex\label{ex:3:70b}
\gll Der hat [ein groß-er Hubser] gemacht.\\
he   has   {\db}a    big-\textsc{st}   jump.\textsc{masc} done\\
\glt ‘He did a big jump.’
\ex\label{ex:3:70c}
\gll Ich mag  [gut-er    Wein].\\
    I    like  {\db}good-\textsc{st} wine.\textsc{masc}\\
\glt ‘I like good wine.’
\ex\label{ex:3:70d}
\gll Ich seh [ein-er].\\
    I     see   {\db}one-\textsc{st}\\
\glt ‘I see one.’
\z
\z

G. Müller concludes that all analyses including his own can straightforwardly account for this dialect by deleting a rule or changing the featural specification of a rule. We see below that the current proposal can also explain this variety. Unlike G. Müller’s discussion, here I also address the weak inflections and \textit{ein}-words.

I start by updating the two inflectional paradigms. Considering \REF{ex:3:70a}, note that the strong and weak ending -\textit{en} in the accusative masculine is replaced by the strong ending -\textit{er} and the weak ending -\textit{e}, respectively (\REF[b-d]{ex:3:70} provide additional evidence that the nominative masculine form occurs in accusative contexts). The inflectional paradigm of Mannheim German is summarized in \tabref{tab:3:11}.

\begin{table}
\caption{Strong and weak endings in Mannheim German}
\label{tab:3:11}
\fittable{
    \begin{tabular}{llllllll}
    \lsptoprule
    {\bfseries STRONG} & [--F, --N]\textsubscript{M} & [--F, +N]\textsubscript{N} & [+F, --N]\textsubscript{F} & [+F, +N]\textsubscript{PL} & {\bfseries WEAK} & [--$\gamma$]\textsubscript{SG} & [+F, +N]\textsubscript{PL}\\
    \midrule
    {}[--O, --S]\textsubscript{NOM} & -er & -es & (-e) & -e      & [--O] & (-e)    &  \\
    {}[--O, +S]\textsubscript{ACC} & -er & -es & (-e) & -e      &  &     &  \\
    {}[+O, --S]\textsubscript{DAT} & -em & -em & -er  & \{-en\} & [+O]  &  \multicolumn{2}{c}{\{-en\}} \\
    {}[+O, +S]\textsubscript{GEN} & -es & -es & -er  & -er     &   &  &  \\
    \lspbottomrule
    \end{tabular}
}
\end{table}

The rules for vocabulary insertion are provided below. There are three differences from Standard German as regards adjectival inflections:
(i) the rule inserting -\textit{en} in the accusative masculine has been deleted,
(ii) both rules inserting -\textit{er} have been changed in their feature specifications, and
(iii) due to the latter change, these two rules have been reordered using the same specificity metrics as above (in \tabref{tab:3:12} below, I juxtapose the rule systems of Standard and Mannheim German).

\TabPositions{3.5cm}
\ea%71
    \label{ex:3:71}
          Strong (except for feminine \textit{-e} and plural \textit{-en}):
\ea\label{ex:3:71a} [+F, +N, +O, +S]                      \tab$\rightarrow$  \textit{-er}
\ex\label{ex:3:71b} [+F, +N, --O, $\alpha$S] \           \tab$\rightarrow$  \textit{-e}
\ex\label{ex:3:71c} [$\alpha$F, --N, $\alpha$O, $\beta$S]  \tab$\rightarrow$  \textit{-er}
\ex\label{ex:3:71d} [--F, +O, --S]                        \tab$\rightarrow$  \textit{-em}
\ex\label{ex:3:71e} [--F, $\alpha$O, $\beta$S]             \tab$\rightarrow$  \textit{-es}
\z
\z

\ea%72
    \label{ex:3:72}

             Weak (including strong feminine \textit{-e} and plural \textit{-en}):

\ea\label{ex:3:72a} [--$\gamma$, --O]      \tab$\rightarrow$  \textit{-e}
\ex\label{ex:3:72b} []                    \tab$\rightarrow$  \textit{-en}
\z
\z

As regards \textit{ein}-words, the rule that inserts accusative masculine \textit{einen} ‘a’ has been deleted yielding the following insertion rules. Recall that if \REF{ex:3:73a} does not apply, \REF{ex:3:73b} will, in conjunction with the inflectional rules in \REF{ex:3:73c} or \REF{ex:3:73d}.

\ea%73
    \label{ex:3:73}
\ea\label{ex:3:73a} [+D][--F, --O]    \tab$\rightarrow$   \textit{ein} / {\longrule} word ]\textsubscript{φ}
\ex\label{ex:3:73b} [+D]             \tab$\rightarrow$   \textit{ein}-

\ex\label{ex:3:73c} [+F, +N, +O, +S]  \tab$\rightarrow$  \textit{-er}
\ex\label{ex:3:73d} etc.
\z
\z

Besides the three differences in adjectival inflections, this adds a fourth difference to Standard German.

More generally, deleting the vocabulary insertion rules for accusative masculine -\textit{en} and for accusative masculine \textit{einen} ‘a’ highlights the fact that the feature combination of “nominative/accusative + masculine/neuter” has a special status. This is directly stated by the features [--F, --O] in \REF{ex:3:73a}.

The vocabulary insertion rules for the two varieties are provided in \tabref{tab:3:12}, juxtaposing the strong endings in (1), the weak endings in (2), and \textit{ein} in (3) in Standard German and their counterparts in (1’), (2’), and (3’) in Mannheim German.

\begin{table}
\caption{Comparison between Standard and Mannheim German}
\label{tab:3:12}
\fittable{
\begin{tabular}{llll}
\lsptoprule
 Standard German                                  &                               & Mannheim German &\\
 \midrule
1a.  [+F, --N, +O, $\alpha$S]                     & $\rightarrow$  \textit{-er}                       & 1a’.  [+F, +N, +O, +S]                         & $\rightarrow$  \textit{-er}\\
~~~~~~{[+F, +N, --O, $\alpha$S]}                  & $\rightarrow$  \textit{-e}                        & ~~b’.  [+F, +N, --O, $\alpha$S]                &  $\rightarrow$  \textit{-e}\\
~~b. [$\alpha$F, $\alpha$N, $\alpha$O, $\alpha$S] & $\rightarrow$  \textit{-er}                       & ~~c’.  [$\alpha$F, --N, $\alpha$O, $\beta$S]   &  $\rightarrow$  \textit{-er}\\
~~c. [--F, --N, --O]                              & $\rightarrow$  \textit{-en}                       & ~~d’.  [--F, +O, --S]                          &  $\rightarrow$  \textit{-em}\\
 ~~~~~~{[--F, +O, --S]}                           & $\rightarrow$  \textit{-em}                       &~~e’.  [--F, $\alpha$O, $\beta$S]               &  $\rightarrow$  \textit{-es}\\
~~d. [--F, $\alpha$O, $\beta$S]                   & $\rightarrow$  \textit{-es}                       &\\
\tablevspace
2a.  [--$\gamma$, --O]                            & $\rightarrow$  \textit{-e}                        &2a’.  [--$\gamma$, --O]                         &  $\rightarrow$  \textit{-e}\\
~~b. []                                           & $\rightarrow$  \textit{-en}                       & ~~b’.  []                                      &    $\rightarrow$  \textit{-en}\\
\tablevspace
3a.  [+D]                                         & $\rightarrow$   \textit{ein-}/[--F, --N, --O, +S]             &3a’.  [+D][--F, --O]                            &  $\rightarrow$   \textit{ein}/{\longrule}word]\textsubscript{φ}\\
~~b.  [+D][--F, --O]                              & $\rightarrow$   \textit{ein}/{\longrule}word]\textsubscript{φ}&~~b’.  [+D]                 &   $\rightarrow$   \textit{ein}-\\
~~c.  [+D]                                        & $\rightarrow$   \textit{ein}- &&\\

%									1a’.  [+F, +N, +O, +S] $\rightarrow$  \textit{-er}
%
%                                      b’.  [+F, +N, -O, $\alpha$S]   $\rightarrow$  \textit{-e}
%
%                                      c’.  [$\alpha$F, -N, $\alpha$O, $\beta$S]   $\rightarrow$  \textit{-er}
%
%                                      d’.  [-F, +O, -S]   $\rightarrow$  \textit{-em}
%
%                                      e’.  [-F, $\alpha$O, $\beta$S]   $\rightarrow$  \textit{-es}
%
%                                    2a’.  [-$\gamma$, -O]     $\rightarrow$  \textit{-e}
%
%                                      b’.  []       $\rightarrow$  \textit{-en}\\
%
%                                    { 3a.  [+D]                   $\rightarrow$   \textit{ein-} / [-F, -N, -O, +S]}
%
%                                    {   b.  [+D][-F, -O]       $\rightarrow$   \textit{ein} / {\longrule} word ]\textsubscript{φ}}
%
%                                    {   c.  [+D]         $\rightarrow$   \textit{ein}-} & { 3a’.  [+D][-F, -O]   $\rightarrow$   \textit{ein} / {\longrule} word ]\textsubscript{φ}}
%
%                                    {   b’.  [+D]     $\rightarrow$   \textit{ein}-}\\
\lspbottomrule
\end{tabular}
}
\end{table}

Note again that the rule inserting -\textit{en} in (1c) in \tabref{tab:3:12} has no counterpart in (1’). Given that, all vocabulary insertion rules for adjectival inflections in Mannheim German in (1a’-e’) have the feature [S]. Impoverishment applies as in \chapref{sec:2} but now to all (unambiguously) strong inflections.

To sum up, like the five other analyses, the current system can also account for the regional variety of Mannheim German, including the variation seen with weak inflections and \textit{ein}-words. Notice that I focused here on one regional dialect to show that the current system can capture dialectal variation as well. For the description of other dialects, see \citet{Schirmunski2010}; for some tentative discussion of Alemannic German, see \chapref{sec:8}, \sectref{sec:8.2.1}.

\section{Conclusion}\label{sec:3.9}

This chapter continued the discussion of adjectival inflections begun in \chapref{sec:2}. Here I focused on variation in (mostly) canonical DPs where elements with unexpected strong or unexpected weak inflections occur. The unexpected strong adjectives are due to pronominal determiners not triggering Impoverishment. The unexpected weak adjectives follow from a phonetic rule, presumably a reflex of markedness reduction. The unexpected weak inflections on determiners are accounted for by a more general application of Impoverishment Rule 2. We arrive then at three different mechanisms that explain the weak inflections: one primary mechanism (Impoverishment Rule 1) and two secondary mechanisms (phonetic rule, Impoverishment Rule 2). Unlike the primary mechanism, the secondary ones apply in restricted contexts, that is, in certain constructions and/or specific featural contexts. Note that the secondary mechanisms are brought to light when the primary one does not occur. The main patterns of \chapref{sec:2} and \ref{sec:3} along with their analyses are summarized in \tabref{tab:3:13} (the inflection under consideration is set apart by a hyphen).

\begin{table}
\caption{Summary chart}
\label{tab:3:13}
\begin{tabularx}{\textwidth}{QQ}
\lsptoprule
 Construction: Example & Inflection: Analysis\\
 \midrule
Regular DP: \textit{der heiß-e schwarz-e Kaffee} & Weak: Impoverishment Rule 1\\
Regular DP: (i) \textit{heiß-en Kaffees}; (ii) \textit{dies-en Jahres} & Weak: Impoverishment Rule 2\\
Regular DP: \textsuperscript{\%}\textit{mit frischem schwarz-en Kaffee} & Weak: Phonetic Rule\\
Regular DP: \textit{ein heiß-er schwarz-er Kaffee} & Strong: \textit{ein}-words\\
Close Appositions: \textit{der Indianer Groß-er Bär} & Strong: Low Right-Adjunction\\
Indefinite Pronoun Constructions: \textit{wer ander-er} & Strong: Mid Right-Adjunction\\
Noun-Adjective Exclamatives: \textit{Schwein, schwarz-es!} & Strong: Mid Right-Adjunction\\
Loose Appositions: \textit{wir, begeistert-e Linguisten} & Strong: High Right-Adjunction\\
Agreement in Pronominal DPs: \textit{ich dumm-er Idiot} & Strong: Pronominal Determiner\\
Dis-agreement in Pronominal DPs: \textit{ihr jung-es Gemüse} & Strong: Complex Specifier inside DP\\
Split Topicalizations: \textit{bunt-e Hemden {\dots} diese da} & Strong: Separate Base-generation\\
Nominal Compounds: \textit{dem Deutsch-e Bank Chef} & Strong: Complex Compound Modifier\\
Vocatives: \textit{Dumm-er Idiot!} & Strong: Voc Head and Absence of Determiner\\
Predeterminers: \textit{dieses mein groß-es Glück} & Strong: Determiner Outside DP Proper\\
\lspbottomrule
\end{tabularx}
\end{table}

Besides these cases, I also discussed inflectional variation on certain adjectives (e.g., \textit{lila} ‘purple’, \textit{prima} ‘great’, \textit{Berliner} ‘(from) Berlin’), and I addressed the uninflected forms of the definite article and its related distal demonstrative. Furthermore, I discussed the optional inflection on the predeterminer \textit{all-} ‘all’. On the basis of these cases, I demonstrated that the generalization Weak After Strong and the more general Principle of Monoinflection face a number of empirical challenges. Finally, I showed that the current system can be extended to Mannheim German, one of the dialects in Germany. The overall conclusion is that a purely surface-oriented account does not work – the consideration of structure appears to be an essential part of any proposal. The next chapter discusses consequences of the current proposal for other accounts.
