\chapter{The structural nature of adjectival inflections}\label{sec:2}

\section{Introduction}\label{sec:2.1}

This chapter provides detailed evidence in support of Hypothesis 1, namely that adjectival inflections are semantically vacuous and indicate abstract structure. As regards the latter, the discussion supports Hypothesis 2a: Adjectival inflections indicate structure in the higher layers of the noun phrase, and they provide clues about various degrees of embeddings of adjectives.

\subsection{The strong/weak alternation of adjectives}\label{sec:2.1.1}

As briefly discussed in \chapref{sec:1}, adjectives can take strong or weak endings. Compare the following typical inflectional alternation on adjectives where the ending is strong when no determiner is present, and it is weak when there is a determiner. Below, I show these patterns in the nominative case using a singular masculine mass noun and a plural count noun (for a complete inventory of all the combinations of case, number, and gender, see \sectref{sec:2.2.1.2}).

\ea%1
    \label{ex:2:1}
\ea
\gll frisch-er Kaffee\\
    fresh-\textsc{st}  coffee.\textsc{masc}\\
\glt ‘fresh coffee’
\ex
\gll d-er     frisch-e   Kaffee\\
    the-\textsc{st} fresh-\textsc{wk} coffee.\textsc{masc}\\
\glt ‘the fresh coffee’
\z
\z

\ea%2
    \label{ex:2:2}
\ea
\gll nett-e   Frauen\\
     nice-\textsc{st} women\\
\glt ‘nice women’
\ex
\gll di-e     nett-en   Frauen\\
    the-\textsc{st} nice-\textsc{wk} women\\
\glt ‘the nice women’
\z
\z

These are the basic patterns. Recall that the phenomenon involving the two related forms of the adjective is labeled the strong/weak alternation.

These inflectional patterns have received a good amount of attention in the literature (e.g., \citealt{Murphy2018}, \citealt{Olsen1991b}, \citealt{Schoorlemmer2009}; for a discussion of these works, see \sectref{sec:2.5}).\footnote{To anticipate the discussion in \sectref{sec:2.5}, previous accounts of adjectival inflections in German have focused on the canonical cases, that is, the alternation of the adjective in simple noun phrases of the schematic type “determiner + adjective(s) + noun”, where the three elements agree in case, number, and gender. As argued in that section, it is not clear how these proposals can explain some of the non-canonical instances. In my view, considerations of elegance cannot be the sole or decisive measure to evaluate a proposal – the empirical coverage must be taken into account to evaluate the plausibility of an analysis.} As pointed out in \chapref{sec:1}, it could be claimed that the strong and the weak endings correlate with an indefinite or definite interpretation of the noun phrase. In the (a)-examples above, the nominals with the strong adjectives are indefinite, but in the (b)-examples, the nominals with the weak adjectives are definite. This is often referred to as the semantic distribution of adjectival inflections. At first glance then, these different inflections seem to be a reflex of the varying semantics of the noun phrase.

\citet{Demske2001}, \citet[60-62]{Lohrmann2010}, and \citet{Rehn2019} show that this is indeed the case in the older Germanic languages including older varieties of German.\footnote{Focusing on the development from Proto-Indo-European to Old Norse, \citet{Pfaff2020} notes that weak adjectives occur indeed in definite contexts but that strong adjectives have nothing to do with indefiniteness. \citet[119-47]{Evans2019} makes an even stronger claim for the older Germanic languages such that weak adjectives only occur in the context of determiners (for the lack of a clear semantic distribution in OHG, see \citealt{Petrova2024}).}  Interestingly, the older literature on Modern German (e.g., \citealt{Curme1910}, \citealt{Lockwood1968}, \citealt{Prokosch1939}) also claims that adjective inflections are regulated by the semantics. Indeed, this claim can occasionally be encountered in the more recent literature as well (e.g., \citealt{KarnowskiPafel2004}: 177-78, \citealt{Pafel1994}: 257-64).\footnote{To explain certain exceptions, these authors claim that \textit{ein} ‘a’, \textit{kein} ‘no’, and possessives involve idiosyncracies.} To briefly illustrate one such proposal, \citet[259]{Abraham2013} seems to claim that different adjectival inflections mark anaphoric/thematic vs. non-anaphoric/rhematic relations, especially in dialectal German. It should be pointed out though that not many details are provided.

\citet[130-37]{Harbert2007} points out that the strong/weak alternation is different in some of the contemporary languages (also \citealt{Evans2019}, \citealt{Gallmann1996}: 302-03, \citealt{RoehrsJulien2014}, \citealt{Schoorlemmer2009}: 53 fn. 55). In fact, he claims that diachronically, the two sets of endings developed from a semi-regular distribution toward regularization and functionalization \citep[131]{Harbert2007}. This development occurred in different ways: In the Scandinavian languages, the alternation signals (in-)definiteness (e.g., \citealt{Julien2005a}, \citealt{Lohrmann2011}); in German, it developed into an economy principle such that the strong ending is limited to one per noun phrase (e.g., \citealt{Esau1973}). Looking at the literature, it is probably fair to state that a consensus seems to have emerged that the strong/weak alternation in Modern German is regulated by the morpho-syntax (and not the semantics).

While I basically concur with Harbert and others, I argue that the German facts are more complicated than often presented (I briefly discuss earlier proposals and the related traditional generalizations in \sectref{sec:2.2.1.1}). Before I turn to a detailed discussion of the strong/weak alternation in German, I provide more evidence that adjectival inflections in German do not correlate with (in-)definiteness. This lack of correlation can be shown on the basis of diachronic and synchronic considerations.

\subsection{Adjectival inflections and (in-)definiteness}\label{sec:2.1.2}

Prenominal adjectives in the Germanic languages vary as regards a correlation between the different types of inflections and (in-)definiteness. As briefly discussed in \chapref{sec:1}, \sectref{sec:1.2.1.2}, German, Yiddish, and Norwegian exhibit a number of distinctions in this regard. Specifically, adjectival inflections in German clearly lack a correlation regarding (in-)definiteness, and this point can be strengthened by a side-by-side diachronic comparison of Modern German and Old High German (OHG). I provide two types of examples (the OHG instances are taken from \citealt{Demske2001}: 67). First, adjectives behave differently in vocatives: In Modern German, adjectives are strong; in OHG, adjectives are weak.\footnote{As pointed out by \citet[203-04, 211-12]{Petrova2024}, vocatives could also involve strong adjectives in OHG, and the same goes for possessive pronominals (see below). Note that this variation is not possible in Modern German.}

\ea%3
    \label{ex:2:3}
\ea
Modern German\\
\gll Dumm-er Idiot!\\
    stupid-\textsc{st}  idiot.\textsc{masc}\\
  \glt ‘Stupid idiot!’
\ex
OHG\\
\gll líob-o     man\\
    dear-\textsc{wk} man.\textsc{masc}\\
\glt ‘dear man’
\z
\z

Second, possessive elements also occur with adjectives showing different inflections. As above, Modern German exhibits strong adjectives but OHG weak ones.

\ea%4
    \label{ex:2:4}
\ea
Modern German\\
\gll mein lieb-er  Sohn\\
    my    dear-\textsc{st} son.\textsc{masc}\\
\glt ‘my dear son’
\ex
OHG\\
\gll mîn liob-o     sun\\
    my  dear-\textsc{wk} son.\textsc{masc}\\
\glt ‘my dear son’
\z
\z

I know of no evidence that indicates that the definiteness involved in these vocatives and possessives is different in the two language varieties. As such, Modern German and OHG show different endings on adjectives in the same definite contexts. Recalling the cross-linguistic discussion from \chapref{sec:1}, \sectref{sec:1.2.1.2}, I conclude that adjectival endings do not uniformly indicate the definiteness of noun phrases across the different Germanic languages, synchronically (German vs. Norwegian) and diachronically (Modern German vs. OHG).\footnote{There is no agreement in the literature on when adjectival inflections changed from a semantic to a morpho-syntactic distribution. \citet{Demske2001} claims that this change occurred during ENHG. In contrast, \citet[79-88]{Rehn2019} argues that this happened between OHG and MHG (for discussion, see also \citealt{Evans2019}: 147-57, \citealt{Klein2007}, \citealt{Petrova2024}). Observing an increase in the occurrence of the indefinite article \textit{ein}, \citet[88-90]{Rehn2019} proposes that the marking of indefiniteness by the article triggered the change in the way adjectival inflections are regulated in German.}  This lack of correlation in Modern German between inflection and interpretation can be made even more forcefully when we consider certain synchronic data of that language.

  I also briefly showed in \chapref{sec:1}, \sectref{sec:1.3.1.1} that noun phrases involving indefinite articles or possessive articles take either a strong adjective or a weak adjective in (Modern) German. This depends on the morphological case the noun phrase appears in. Compare the nominals in the nominative in the (a)-examples to those in the dative in the (b)-examples, where the former have strong adjectives but the latter weak ones.

\ea%5
    \label{ex:2:5}
\ea
\gll ein groß-es Auto\\
    a     big\textsc{-st}    car.\textsc{neut}\\
\glt ‘a big car’
\ex
\gll mit  ein-em groß-en Auto\\
    with a-\textsc{st}      big-\textsc{wk}  car.\textsc{neut}\\
\glt ‘with a big car’
\z
\z

\ea%6
    \label{ex:2:6}
\ea
\gll mein groß-es Auto\\
my    big-\textsc{st}   car.\textsc{neut}\\
\glt ‘my big car’
\ex
\gll mit mein-em groß-en Auto\\
with my-\textsc{st}     big-\textsc{wk}  car.\textsc{neut}\\
\glt ‘with my big car’
\z
\z


As shown in \REF{ex:2:5} and \REF{ex:2:6}, adjectives pattern the same after indefinite articles and possessive articles in German; that is, adjectives can take a strong or weak ending in both indefinite and definite contexts. I argue in detail in \chapref{sec:5} that possessive articles in Modern German consist of a possessive component and vacuous \textit{ein} (e.g., \textit{m-ein} ‘my’). This composite analysis of possessive articles directly relates the latter to \textit{ein}. If so, it is \textit{ein} that seems to be responsible for the regulation of the inflectional endings in \REF{ex:2:5} and \REF{ex:2:6}. In fact, the distinguishing factor appears to be case. However, it is undisputed that the different morphological cases in \REF{ex:2:5} and \REF{ex:2:6} have no impact on the (in-)definiteness of those instances. Indeed, Saxon Genitives also involve strong and weak adjectives, similar to \REF{ex:2:6} above, but in different morphological cases (for detailed discussion, see \sectref{sec:2.2.3}).

  There is more evidence that morphological case alone cannot explain the\linebreak strong/weak alternation in this language. This can be exemplified with the same lexical item where the morphological case remains unchanged. For instance, the quantifying element \textit{manch}- ‘some’ is indefinite. If it is inflectionless, the adjective is strong \REF{ex:2:7a}; if it has a strong ending, the adjective is weak \REF{ex:2:7b}. Similarly, the pronominal determiner \textit{wir} ‘we’ is definite but can occur with both types of endings \REF[a-b]{ex:2:8}. Note that \% below marks a less preferred but certainly possible form (see \citealt{Bhatt1990}: 154-55; \citealt{Darski1979}: 198; \citealt{Duden1995}: 280, \citeyear{Duden2007}: 39).\footnote{The strong ending in \REF{ex:2:8b} becomes more acceptable if the overt noun is missing (\citealt{Bhatt1990}: 179, \citealt{Duden2007}: 39). Presumably, a strong ending is better able to license a null noun.}

\ea%7
    \label{ex:2:7}
\ea \label{ex:2:7a}
\gll manch nett-e   Studenten\\
some   nice-\textsc{st} students\\
\glt ‘some nice students’
\ex \label{ex:2:7b}
\gll manch-e nett-en   Studenten\\
some-\textsc{st} nice-\textsc{wk} students\\
\glt ‘some nice students’
\z
\z

\ea%8
    \label{ex:2:8}
\ea  \label{ex:2:8a}
\gll wir nett-en    Studenten \\
we  nice-\textsc{wk}  students\\
\glt ‘us nice students’
\ex[\%]{  \label{ex:2:8b}
\gll wir nett-e    Studenten\\
we  nice-\textsc{st} students\\
\glt ‘us nice students’
}
\z
\z

As will become clear below, these are not isolated cases: They also apply to \textit{solch}- ‘such’, \textit{welch}- ‘which’, \textit{mir} ‘me(\textsc{dat})’, \textit{dir} ‘you(\textsc{dat.sg})’, and \textit{ihr}\linebreak ‘you(\textsc{nom.pl})’.


To summarize, it is, in my view, beyond doubt that at least in (Modern) German, different adjectival endings do not correlate with differences in definiteness. Again, the majority of the recent literature on German concludes the same. However, despite this emerged consensus, the fact that the strong/weak alternation is regulated by the semantics in the related Scandinavian languages and given the occasional claims that this is also the case in German, I will continue arguing in favor of a morpho-syntactic distribution making this point in different ways throughout this book. This fits well with my overall hypothesis that adjectival inflections are semantically vacuous: They neither have semantics of their own nor do they make semantic features visible -- they are not a reflex of the semantics.

Returning briefly to the diachronic data, it was pointed out by \citet[82]{Demske2001} that adjectives in OHG are weak after the possessive element but strong after the indefinite article. As illustrated above, this is different in Modern German, where adjectives pattern the same after the possessive element and the indefinite article. If so, this implies that the possessive element and the indefinite article were not related to each other in the older varieties of German as compared to Modern German. It seems clear that German underwent a number of changes in this regard during its diachronic development (for the discussion of some diachronic issues, see \chapref{sec:5}; also \citealt{Alexiadou2004}, \citealt{Demske2001}, \citealt{Evans2019}, \citealt{Rehn2019}, and \citealt{Wood2007}).

\subsection{Outlook}\label{sec:2.1.3}

The following discussion focuses on (Modern) German with a few cross-linguis-tic remarks if the phenomenon under discussion shows an interesting parallel or contrast between German and another language. Restricting the discussion to prenominal adjectives, I reach the conclusion that, at least in German, adjectival inflections are semantically vacuous (Hypothesis 1a) and that the strong/weak alternation is not simply a surface phenomenon. Rather, it is, in part, a reflex of different structures of the noun phrase (Hypothesis 2a). In fact, this alternation is argued to provide a good diagnostic of the structure, but not the semantics, of the nominal domain.

To anticipate the discussion, I propose that the derivation of the weak endings on adjectives must take a certain structure into account. I argue that adjectives are only weak if they occur in canonical DPs and undergo a certain reduction process. This process is argued to be feature deletion (Impoverishment), and it is triggered by certain determiners (Impoverishment Rule 1) or in a specific feature context (Impoverishment Rule 2). The diverse distribution of the strong adjectives in other canonical DPs and non-canonical DPs more generally is explained by the fact that Impoverishment does not occur, either because there is no determiner, no relevant determiner, or because the determiner or the adjective is in a position that is different from that in canonical DPs. In a sense, the strong endings form the elsewhere case.

If this is on the right track, then a number of other cases involving canonical DPs appear to be exceptional. I argue in \chapref{sec:3} that the inflectional patterns of adjectives (and determiners) in German are due to several mechanisms. It is important to point out though that the other mechanisms only occur in specific, well-defined contexts; that is, unlike Impoverishment Rule 1, they are restricted to certain combinations of case and gender, or case and number. \chapref{sec:4} discusses some consequences of the current proposal for other analyses. I utilize the system developed for adjectival inflections as a diagnostic of the plausibility of the structures put forth in those works.

This chapter is organized as follows. In \sectref{sec:2.2}, I provide the basic proposal deriving the inflections on adjectives in the canonical cases. I propose that the operation of Impoverishment explains these cases. \sectref{sec:2.3} turns to the discussion of adjectives in non-canonical DPs, and it is proposed that these cases involve different structures where Impoverishment does not apply. In \sectref{sec:2.4}, I address the inflections on determiners and determiner-like elements arguing that Impoverishment does not apply to these elements either. \sectref{sec:2.5} discusses three previous proposals of the strong/weak alternation, and \sectref{sec:2.6} concludes the chapter.

\section{Adjectival inflections in canonical DPs}\label{sec:2.2}

The basic proposal is as follows. Adopting the framework of Distributed Morphology (\chapref{sec:1}, \sectref{sec:1.4.2.1}), I assume that adjectival inflections are the result of the spell-out of the features for case, number, and gender on the terminal heads in the syntactic representation. I propose that the weak endings follow from the operation of Impoverishment. Arguing that this mechanism only occurs in canonical DPs, it is taken to reduce the fully specified feature bundles in the tree representation. These reduced feature bundles are then spelled out as the weak endings. Given the diverse distribution of the strong endings, I propose that the occurrence of the latter instances is not subject to any specific conditions. Rather, the strong endings surface in the absence of Impoverishment as the fully specified (underlying) feature bundles. In the current account, the strong inflections involve the elsewhere case. Below, I develop this proposal in more detail.

  This section is organized as follows. First I discuss adjectives in the context of \textit{der}-words. This is followed by noun phrases involving \textit{ein}-words, null articles, and Saxon Genitives. At the end of this section, I turn to adjectives occurring in more complex canonical DPs, canonical DPs that contain extended-adjective constructions and numerals.

\subsection{Weak adjectives: Impoverishment}\label{sec:2.2.1}

I start by repeating the basic alternation of adjectival inflections involving \textit{der}-words, briefly reviewing some traditional generalizations that have been put forth to describe these patterns. This is followed by providing an inventory of the adjectival inflections and discussing a previous proposal. After these preliminaries, I lay the foundation of my own analysis of weak adjectives and spell out some of the details of my account.

\subsubsection{Basic alternation and traditional generalizations}\label{sec:2.2.1.1}

Recall the basic alternation as regards adjectival inflections: If a determiner is not present, then adjectives surface with strong endings \REF{ex:2:9a}. If a determiner is present, then the determiner has a strong ending, and the adjectives appear with weak inflections \REF{ex:2:9b}.

\ea%9
    \label{ex:2:9}
\ea \label{ex:2:9a}
\gll frisch-e*(r)    schwarz-e*(r) Kaffee\\
fresh-\textsc{st}/*\textsc{wk} black-\textsc{st}/*\textsc{wk} coffee.\textsc{masc}\\
\glt ‘fresh black coffee’
\ex\label{ex:2:9b}
\gll d-er    frisch-e(*r) schwarz-e(*r) Kaffee\\
the-\textsc{st} hot-\textsc{wk}/*\textsc{st} black-\textsc{wk}/*\textsc{st}  coffee.\textsc{masc}\\
\glt ‘the fresh black coffee’
\z
\z

Note again that the definite article in \REF{ex:2:9b} has the same ending as the adjectives in \REF{ex:2:9a}. In fact, following \citet{MilnerMilner1972}, \citet{Leu2015} and many others have pointed out that the endings on determiners (D) are the same as the strong endings on adjectives (for detailed discussion, see the next subsection). Assuming a null article for \REF{ex:2:9a}, I summarize this determiner-adjective interaction in the following schematic patterns (> stands for precede; * means multiple occurrences).

\ea%10
    \label{ex:2:10}
\ea \label{ex:2:10a} {Ø\textsubscript{D}} > STRONG\textsubscript{ADJ*}
\ex \label{ex:2:10b}   STRONG\textsubscript{D} > WEAK\textsubscript{ADJ*}
\z
\z

Observe that this abstract distribution is instantiated by the canonical cases, simple noun phrases of the schematic type “determiner + adjective(s) + noun”, where the three elements agree in case, number, and gender.

Different generalizations have been offered to describe the distribution in \REF{ex:2:10}. They basically fall into two types. One type of generalization focuses on the inflectional behavior of the adjectives. I provide one such statement in \REF{ex:2:11}, which is often conveniently referred to as Weak After Strong (for similar statements, see \citealt{Bierwisch1967}: 257, \citealt{Eisenberg1998}: 171-73, \citealt{Gallmann1998}: 144, \citealt{Giusti2015}: 207, G. \citealt{Müller2002a}: 129, \citealt{Petrova2024}: 184-85, \citealt{Pfaff2017}: 286, \citealt{Rehn2019}: 58, \citealt{Sauerland1996}: 34, \citealt{Schoorlemmer2009}: 53, and many others).

\ea%11
    \label{ex:2:11}
Weak After Strong\\
          	An adjective with a weak inflection is preceded by a determiner with a strong inflection.
\z

Note that this particular formulation does not explicitly mention adjectives that are not preceded by (inflected) determiners. However, since prenominal adjectives are only subject to a two-way alternation, the distribution of strong adjectives can be inferred from \REF{ex:2:11}.

The other type of generalization is broader in that it focuses on inflections more generally, that is, on the inflections of both the adjective and the preceding determiner. It highlights the facts that the strong inflection occurs only once in the noun phrase and that it appears in first position. \citet[135]{Roehrs2009a} provides the following formulation of the so-called Principle of Monoinflection (see also, e.g., \citealt{Darski1979}, \citealt{Esau1973}: 140, \citealt{HelbigBuscha2001}: 274-76, \citealt{Murphy2018}: 344, \citealt{Wegener1995}: 105, 153, cf. \citealt{Nübling2011}: 178).\footnote{There is a variant of this generalization that includes the inflection of the head noun. \citet[58]{Evans2019} provides the following statement of the so-called “Once-per-DP” Principle: In the maximal projection of a noun (i.e., in a DP), the strong ending occurs on at most one lexical type (determiner, adjective, or noun). \citet[Chapter 3]{Evans2019} proposes that the order of strong endings occurring before weak endings is derived from c-command. However, this leads to some issues. One problem is the relevance of the head noun in the calculation of the adjectival inflections (see \citealt{Roehrs2021}).}

\ea%12
    \label{ex:2:12}
        Principle of Monoinflection\\
The first inflected element within a noun phrase carries the strong and the second one the   weak ending.
\z

These are the two main types of traditional generalizations for adjectival inflections in German. Unlike the semantic distribution, they describe a morpho-syntactic pattern. Both types seek to capture the fact that some combinations of the strong and weak endings do not occur; for instance, both exclude patterns where a weak determiner in -\textit{e} or \textit{-en} is followed by a strong adjective in -\textit{er}, \textit{-es,} or \textit{-em} (e.g., *\textit{d-en gut-em Wagen} ‘the-\textsc{wk} good-\textsc{st} car’).

While the more general statement in \REF{ex:2:12} should, in my view, be preferred, both generalizations face two types of exceptions. As briefly discussed in \chapref{sec:1}, \sectref{sec:1.3.1.1}, these issues have to do (i) with weak elements occurring without the presence of preceding strong elements, and (ii) with strong elements surfacing despite the presence of other preceding strong elements. To my mind, these issues are the result of stating the generalizations in terms of surface precedence and the inflections involved (rather than in terms of the structure and the lexical and featural triggers involved, as made more precise below). The relevant problematic cases are addressed in detail in the course of the following discussion and are summarized in \chapref{sec:3}, \sectref{sec:3.7}. Having said that, the generalizations above are good approximations of the canonical facts in German and provide a useful heuristic of the investigation.

The generalization that I will work towards is the following.

\ea%13
    \label{ex:2:13}
        Strong/Weak Alternation in German:\\
	Disregarding a few (mostly) lexical exceptions, both adjectives and determiners have   strong or weak inflections. Weak inflections only occur in canonical DPs, either

        \ea   \label{ex:2:13a} in the context of certain lexical items, or
        \ex   \label{ex:2:13b} in certain featural contexts, or
        \ex   \label{ex:2:13c}  in a combination of (a) and (b).
        \z

Strong inflections occur in all other environments.
\z
While this generalization is, admittedly, more complex than its predecessors, it will be shown that it is more in line with the empirical facts.

  Returning to the data in \REF{ex:2:9}, it is important to point out that the grammaticality judgments of these instances are very sharp. I propose that the morphological mechanism of Impoverishment explains these data. I closely follow the discussion of \citet{RoehrsJulien2014}, who base their proposal on \citet{Sauerland1996}. Before I turn to my analysis, I provide an inventory of the relevant elements, and I briefly review \citet{Sauerland1996}.

\subsubsection{Inventory of determiners and adjectival inflections}\label{sec:2.2.1.2}

In this subsection, I discuss the inventory of determiners and adjectival inflections. This allows me to formulate the range of Impoverishment and specific vocabulary insertion rules. Starting with determiners, they can be categorized into three groups: \textit{der}-words, \textit{ein}-words, and null articles.

\ea%14
    \label{ex:2:14}
\ea   \label{ex:2:14a} \textit{der}-words:\\

\textit{der} ‘the’, (stressed) \textit{DER} ‘that’, \textit{dieser} ‘this’, \textit{jener} ‘that’, \textit{jeder} ‘every’, \textit{mancher} ‘some’, \textit{solcher} ‘such’, \textit{welcher} ‘which’, and \textit{alle} ‘all’\footnote{These are the most frequent \textit{der}-words (usually provided in reference works).}
\ex
\textit{ein}-words:\\
\textit{ein} ‘a’, (stressed) \textit{EIN} ‘one’, \textit{kein} ‘no’, and possessive articles like \textit{mein} ‘my’,   \textit{dein} ‘your’, etc.
\ex    null articles:

$\emptyset$\textsubscript{D}
\z
\z

The stem forms of the definite article \textit{der} ‘the’ may vary between \textit{de}-, \textit{da}-, and \textit{di}- depending on case and gender. Furthermore, the form \textit{die} is pronounced [di] as the article or [di:] as the distal demonstrative; that is, this form has no inflection. I dedicate a separate section to these and related points in \chapref{sec:3}, \sectref{sec:3.3}. Given that, I illustrate the \textit{der}-words with the proximal demonstrative \textit{dieser} ‘this’ separating its endings by a hyphen in \tabref{tab:2:1} (I discuss \textit{ein}-words and null articles in Sections \ref{sec:2.2.2} and \ref{sec:2.2.3}). The arrangement of case, number, and gender features of this table follows \citet{Bierwisch1967}. This alignment allows me to state syncretisms based on natural classes (more on this below). I also follow \citet[250]{Bierwisch1967} and others in assuming that the schwa before a consonantal inflection is due to schwa-epenthesis (\citealt{Wiese1996a}: 109-10, 243). This is not indicated separately in the tables below.

\begin{table}
\caption{Determiner inflections}
\label{tab:2:1}
\begin{tabularx}{\textwidth}{XXXXl}
\lsptoprule
 & Masculine & Neuter & Feminine & Plural\\
 \midrule
Nominative & dies-er & dies-es & dies-e & dies-e\\
Accusative & dies-en & dies-es & dies-e & dies-e\\
Dative & dies-em & dies-em & dies-er & dies-en\\
Genitive & dies-es & dies-es & dies-er & dies-er\\
\lspbottomrule
\end{tabularx}
\end{table}

As already noted above, the endings on the determiners are the same as the strong endings on the adjectives. The latter are exemplified with the adjective \textit{gut}- ‘good’ in \tabref{tab:2:2}. Comparing Tables \ref{tab:2:1} and \ref{tab:2:2}, there are only two instances (in the genitive masculine/neuter) where the endings on the determiners are different from the endings on the adjectives (-\textit{es} vs. -\textit{en}).

\begin{table}
\caption{Strong adjective inflections}
\label{tab:2:2}

\begin{tabularx}{\textwidth}{XXXXl}
\lsptoprule
 & Masculine & Neuter & Feminine & Plural\\
 \midrule
Nominative & gut-er & gut-es & gut-e & gut-e\\
Accusative & gut-en & gut-es & gut-e & gut-e\\
Dative & gut-em & gut-em & gut-er & gut-en\\
Genitive & gut-en & gut-en & gut-er & gut-er\\
\lspbottomrule
\end{tabularx}
\end{table}

Studying Tables \ref{tab:2:1} and \ref{tab:2:2} further, we also notice that in many instances, masculine and neuter pattern together (e.g., in the dative and genitive) and feminine and plural do too (e.g., in the nominative, accusative, and genitive). The weak inflections are provided in \tabref{tab:2:3}.

\begin{table}
\caption{Weak adjective inflections}
\label{tab:2:3}
\begin{tabularx}{\textwidth}{XXXXl}
\lsptoprule
 & Masculine & Neuter & Feminine & Plural\\
 \midrule
Nominative & gut-e & gut-e & gut-e & gut-en\\
Accusative & gut-en & gut-e & gut-e & gut-en\\
Dative & gut-en & gut-en & gut-en & gut-en\\
Genitive & gut-en & gut-en & gut-en & gut-en\\
\lspbottomrule
\end{tabularx}
\end{table}

In view of \tabref{tab:2:2} and \tabref{tab:2:3}, German is traditionally taken to have an inventory of five strong and two weak endings \citep{Duden1995}. The latter (-\textit{e}, -\textit{en}) form a proper subset of the former (-\textit{e}, -\textit{en}, -\textit{er}, -\textit{es}, -\textit{em}). Considering all three tables, we notice that the nominative is the same as the accusative within each gender and plural, with one well-known exception (e.g., \citealt{Blevins1995}: 145-46, \citealt{Eisenberg1998}: 173-74): The masculine accusative has a different form (-\textit{en}) from the masculine nominative (-\textit{er}, -\textit{e}), something I return to below. Given the overall similarity of the inflections on determiners and adjectives, I treat them the same and refer to them as adjectival endings.

To reiterate, strong endings have more different forms than weak ones. In fact, comparing the tables above, they express more distinctions in case, number and gender. One of the main insights of \citet{Sauerland1996} is that strong endings involve more features than weak ones. As such, there is no real difference between the traditional strong and weak endings (although I retain this distinction in name for expository purposes). The vocabulary insertion rules formulated in \sectref{sec:2.2.1.5} reflect this. Indeed, taking the number of features as the relevant metrics, we will see that the (less-specified) weak endings form a proper subset of the (more-specified) strong endings.\footnote{Rather than an account based on feature specificity as developed below, G. \citet{Müller2002a} provides an optimality-theoretic proposal (also \citealt{Gallmann2004}). In order to formulate simpler constraints, Müller analyzes noun phrases as NPs although DPs (e.g., \textit{dieser} ‘this’) exist in the account. Furthermore, and also unlike the current account, Müller’s proposal is that inflections have no independent status as morphemes in the lexicon.} \citet{Sauerland1996} proposes to derive the weak inflections from the strong endings by a deletion process. Next, I consider his proposal in more detail.

\subsubsection{\citet{Sauerland1996}}\label{sec:2.2.1.3}

Adopting the general framework of Distributed Morphology (see \chapref{sec:1}, \sectref{sec:1.4.2.1}), \citet{Sauerland1996} proposes to derive the basic alternation illustrated in \sectref{sec:2.2.1.1} by a feature deletion process called Impoverishment (focusing here on German only, I make some minor adjustments to the presentation of his account). He assumes that inflections involve fully specified, abstract feature bundles underlyingly. These feature bundles make up the terminal nodes of the syntactic representation. If these feature bundles are reduced by Impoverishment, they are spelled out as weak endings; if not, they surface as strong endings. Impoverishment is triggered morpho-syntactically: The inflectional feature bundle of the adjective undergoes feature deletion if the preceding determiner has a strong inflection. Sauerland combines Impoverishment and Feature Hierarchy in his account.

  In more detail, the weak endings occur, unsurprisingly, in all featural contexts. Sauerland observes though that like in Norwegian, in German they surface in the same featural contexts as their homophonous strong endings; for instance, -\textit{e} is not only the corresponding weak inflection of strong -\textit{er} in the nominative masculine but also of strong -\textit{e} in the nominative feminine (see Tables \ref{tab:2:2} and \ref{tab:2:3} above). Similar considerations hold for the other weak inflection; for instance, -\textit{en} is not only the corresponding weak inflection of strong -\textit{em} in the dative masculine but also of strong -\textit{en} in the accusative masculine. In other words, the two weak endings -\textit{e} and -\textit{en} also occur as strong endings. Now, since they occur in more contexts (weak and strong), -\textit{e} and -\textit{en} are taken to be the least marked strong endings. This means that there are actually no weak endings \textit{per se} – only (strong) inflections that differ in their specificity.

  As regards -\textit{e} and -\textit{en} themselves, they are also clearly distinguished as regards their individual specificity. Considering the relevant feature combinations, strong -\textit{e} can appear as weak -\textit{e} (e.g., in the nominative feminine), and strong -\textit{en} can surface as weak -\textit{en} (e.g., in the dative plural). However, while strong -\textit{e} is changed to weak -\textit{en} (e.g., in the nominative plural), strong -\textit{en} is never altered to weak -\textit{e}. Sauerland concludes that this indicates that -\textit{e} is more specified than -\textit{en}.

  In order to account for the strong/weak alternation, Sauerland adopts the Hierarchy of Case features from \citet{Harris1994} as shown in \figref{figex:2:15}.

\begin{figure}
	\caption{Hierarchy of case features}
    \label{figex:2:15}
\begin{forest}
  [Case
    [direct
        [accusative]
    ]
    [oblique
        [genitive]
    ]
  ]
\end{forest}
\end{figure}
\largerpage[-2]
Evidence in German that the direct cases (i.e., nominative and accusative) pattern together in opposition to the oblique cases (i.e., dative and genitive) comes from the weak inflections in the neuter and feminine where -\textit{e} occurs in the direct cases but -\textit{en} in the oblique ones (see \tabref{tab:2:3}).

Sauerland formulates the following seven vocabulary insertion rules.

\TabPositions{5cm,6cm}
\ea%16
    \label{ex:2:16}
    \ea\label{ex:2:16a} [+plural, +oblique, --genitive]    \tab $\rightarrow$  -\textit{en}
    \ex\label{ex:2:16b} [+masc, +direct, --accusative]     \tab $\rightarrow$  -\textit{er}
    \ex\label{ex:2:16c} [--fem, +oblique, --genitive]       \tab $\rightarrow$  -\textit{em}
    \ex\label{ex:2:16d} [--fem]                            \tab $\rightarrow$  -\textit{es}
    \ex\label{ex:2:16e} [+oblique]                        \tab $\rightarrow$  -\textit{er}
    \ex\label{ex:2:16f} [+direct]                         \tab $\rightarrow$  -\textit{e}
    \ex\label{ex:2:16g} []                                \tab $\rightarrow$  -\textit{en}
\z
\z

He proposes that feature deletion widens the distribution of -\textit{e} in \REF{ex:2:16f} and -\textit{en} in \REF{ex:2:16g}. In his account, there are three types of deletion rules yielding a total of six.\footnote{I added the feature [+genitive] to \REF{ex:2:17b} to allow -\textit{em} to surface on strong adjectives in dative contexts.}

\ea%17
    \label{ex:2:17}
\ea\label{ex:2:17a}  Impoverishment rule applying to inflections on determiners and adjectives:

    (i) Delete Agr with [+masc, +direct, +accusative]
\ex\label{ex:2:17b}  Impoverishment rule applying to inflections on adjectives:

    (ii) Delete Agr with [--fem, +oblique, +genitive]
\ex\label{ex:2:17c}  Impoverishment rules applying to adjectives in weak positions:

    (iii) Delete [Gender]

    (iv) Delete [accusative]

    (v)  Delete [oblique]

    (vi) Delete [Case] in the environment of [+plural]
\z
\z

After briefly commenting on the role of the various deletion rules, I discuss some shortcomings and point out how the analysis to be developed below deals with these issues.

  As mentioned above, accusative masculine -\textit{en} is special with both determiners and adjectives: It is the only instance where strong and weak inflections differ in the nominative and accusative cases. This exception is captured by formulating the deletion rule in \REF{ex:2:17a}. As a consequence, only \REF{ex:2:16g} can apply.\footnote{\citet[235]{Schoorlemmer2009} formulates an Impoverishment rule to account for the weak ending -\textit{en} (on adjectives) in the accusative masculine.}  Similarly, adjectives in the genitive masculine/neuter are exceptional in that their ending is -\textit{en} (rather than -\textit{es} as compared to the determiners). This follows from the deletion rule in \REF{ex:2:17b} such that only \REF{ex:2:16g} can apply.

  Finally, the set of rules in \REF{ex:2:17c} accounts for the (traditional) weak inflections. The weak positions of adjectives are defined by adjectives occurring after determiners with strong inflections (no further details are provided). Adopting additional Feature Hierarchies, Sauerland assumes that features with a capital letter subsume all instances of that category. This means that the deletion of Gender removes masculine, neuter, and feminine features and that the deletion of Case removes direct (which includes nominative and accusative) and oblique (which consists of dative and genitive) cases. After the deletion of the features in \REF{ex:2:17c} from the syntactic representation, the vocabulary items in \REF[a-e]{ex:2:16} can no longer be inserted. Rather, only the rule in \REF{ex:2:16f}, restricted now to singular contexts (due to (vi) in \REF{ex:2:17c}), and the rule in \REF{ex:2:16g} can apply.

Given certain interpretations (e.g., which exact features are present on terminal nodes), the system works, and I think it is basically on the right track. However, there are also some issues. Pointing out these shortcomings is not meant to be a criticism of Sauerland’s proposal. Rather, the following points are intended to be understood as issues that any proposal that works on a similar set of assumptions needs to address. There are three general and three technical issues.

Starting with the general points, there are three types of deletion rules (feature deletion for both determiners and adjectives \REF{ex:2:17a}, feature deletion for adjectives only \REF{ex:2:17b}, and feature deletion to bring about weak inflections \REF{ex:2:17c}). In the system to be developed below, the effect of \REF{ex:2:17a} does not follow from an Impoverishment rule but rather from the way the vocabulary insertion rule for accusative masculine -\textit{en} is stated: Impoverishment deletes a feature, but that feature is not part of the vocabulary insertion rule for this -\textit{en}. The effects of \REF{ex:2:17b} and \REF{ex:2:17c} do involve Impoverishment rules in the system to be developed (but in modified form): The effect of \REF{ex:2:17b} is capured by Impoverishment Rule 2 and that of \REF{ex:2:17c} by Impoverishment Rule 1.

Second, there are six deletion rules in total. Below, only two Impoverishment rules are proposed. In fact, it is suggested that the application of \REF{ex:2:17b} is more general: Impoverishment Rule 2 also accounts for the weak inflections on certain determiners in genitive masculine/neuter contexts (\chapref{sec:3}, \sectref{sec:3.4}). Third, since the weak inflections on adjectives depend on the strong inflections of determiners, the latter must be inserted first. It is left unspecified by Sauerland why this must be so (and cannot be the other way around). Below, no such ordering is crucial. It is proposed that it is not the inflection of the determiner but rather its stem (specifically, its categorial feature [+D] present in the syntax) that triggers Impoverishment.

Turning to some technical questions, it is not clear (except by stipulation) why \REF{ex:2:16d} must be ordered before both \REF{ex:2:16e} and \REF{ex:2:16f} – all these vocabulary insertion rules are equally specified (note that if \REF{ex:2:16e} were ordered before \REF{ex:2:16d}, this would yield the incorrect forms \textit{der} ‘the’ or \textit{guter} ‘good’ in genitive masculine/neuter contexts; if \REF{ex:2:16f} were ordered before \REF{ex:2:16d}, this would result in the incorrect forms \textit{die} ‘the’ or \textit{gute} ‘good’ in nominative/accusative neuter contexts).\footnote{\citet[281]{HalleMarantz1994} assume “a reasonable feature hierarchy that treats Case features as more specific than gender features”. If so, then both \REF{ex:2:16e} and \REF{ex:2:16f} should indeed be ordered before \REF{ex:2:16d} yielding incorrect forms as discussed in the main text. This means that Sauerland must stipulate the ordering of the vocabulary items (see \citealt{EmbickNoyer2007}: 298 fn. 14).} Second, as far as I can tell, the deletion rule (iv) in \REF{ex:2:17c} is not needed. Finally, if Agr, presumably the head in the syntactic representation that involves the inflection, is deleted by \REF{ex:2:17a} or \REF{ex:2:17b}, it is not clear where or how the default ending -\textit{en} will be attached.

The following proposal draws heavily on the insights of \citet{Sauerland1996} but tries to avoid the general and technical issues raised above. In addition, it aims at spelling out in detail the morpho-syntactic (and lexical) conditions the strong/\linebreak weak alternation is subject to.

\subsubsection{Basic proposal}\label{sec:2.2.1.4}

I propose that determiners, \textit{der}-words and \textit{ein}-words, form triggers for Impoverishment. I assume that Impoverishment operates in a local fashion. More precisely, I propose that the feature bundles that are to be realized as the overt inflections on the adjective undergo Impoverishment inside the phrase that hosts them (i.e., AgrP). This means that the trigger for Impoverishment, the determiner, and its target, the inflection on the adjective, must be in a local relation at some point in the derivation. With the determiners surfacing in the DP-level, this implies that determiners must have originated in a lower position.

In \chapref{sec:1}, \sectref{sec:1.4.1.1}, I argued that there are two determiner positions in the noun phrase, ArtP and DP (e.g., \citealt{Julien2002,Julien2005a}; \citealt{Taraldsen1990}), and that both are connected by movement (e.g., see again \citealt{Borer2005}; \citealt{Nykiel2015}; \citealt{Rehn2019}; \citealt{Roehrs2009a,Roehrs2015,Roehrs2019}; \citealt{Schoorlemmer2009,Schoorlemmer2012}; \citealt{vanGelderen2007}). I provided some distributional evidence for a lower position of determiners in that section. Adopting the structure and derivation in \figref{figex:2:18}, I reiterate the proposal that the determiner moves from ArtP to DP in a successive-cyclic fashion. This is illustrated below with an article (for the DP structure with a demonstrative undergoing movement, see \sectref{sec:2.2.4}).

\glltree[\small\label{figex:2:18}]{
\gll d-er    frisch-e schwarz-e Kaffee\\
the-\textsc{st} hot-\textsc{wk} black-\textsc{wk}  coffee.\textsc{masc}\\
\glt ‘the fresh black coffee’
}{
[DP
    [~]
    [D$'$
      [{D\\\textit{der}$_i$}, name=d]
      [AgrP
        [{InflP\\\textit{frische}}]
        [Agr$'$
            [{Agr\\\textit{\sout{der}}$_i$}, name=agr2]
            [AgrP
                [{InflP\\\textit{schwarze}}]
                [Agr$'$
                    [{Agr\\\textit{\sout{der}}$_i$}, name=agr1]
                    [ArtP
                        [~]
                        [Art$'$
                            [{Art\\\textit{\sout{der}}$_i$}, name=art]
                            [NumP
                                [~]
                                [Num$'$
                                    [Num
                                        [\textit{Kaffee}$_k$]
                                        [Num]
                                    ]
                                    [{NP\\t$_k$}]
                                ]
                            ]
                        ]
                    ]
                ]
            ]
        ]
      ]
    ]
]
\draw[->](art)  to [bend left=45] (agr1);
\draw[->](agr1) to [bend left=45] (agr2);
\draw[->](agr2) to [bend left=45](d);
}

Recall also that definite determiners have the feature [+DEF] and move to the DP-level to specify the definiteness feature on D. In contrast, \textit{ein} is semantically vacuous and does not have a feature for definiteness. Consequently, it can surface in different positions (i.e., Art, Card, or D). To anticipate the discussion in \chapref{sec:5}, this makes it possible to claim that \textit{ein} supports overt operators or flags the presence of covert operators. To be clear, I assume that \textit{ein} remains in ArtP unless it has to move up for a reason. Importantly, as documented in detail below, weak inflections only occur in regular, simple DPs like \figref{figex:2:18}.

I propose that determiners move via adjunction to the DP-level. Articles (Art) are heads, and they move by head adjunction to D; demonstratives (Dem) are phrases, and they move by phrasal adjunction to Spec,DP.\footnote{Recall from \chapref{sec:1}, \sectref{sec:1.4.1.2} that demonstratives involve complex structures with InflP at the top. To avoid confusion, I use Dem (for demonstrative) in the tree representation in \figref{figex:2:19b} below.} I assume that all elements including structural elements are fully specified for case, number, and gender yielding concord. Impoverishment involves the deletion of a certain feature (to be made more precise below). Taking determiners to be triggers for Impoverishment, I illustrate this feature reduction by [+I(mpoverishment)] for now. The relevant portions of the trees are given in \figref{figex:2:19}. Specifically, I propose that adjunction by the determiner reduces the feature bundle of the element adjoined to. Assuming Percolation within the same phrase, this feature reduction spreads from the head to the phrase in case of an article \figref{figex:2:19a}, or from the phrase to the head in case of a demonstrative \figref{figex:2:19b}.\footnote{\citet{Olsen1991b} assumes Percolation from the phrase to the head; \citet[136-37]{Norris2014} points out that Percolation from the head to the phrase immediately follows from Bare Phrase Structure \citep{Chomsky1995}.} Finally, assuming Spec-head agreement and Percolation, [+I] spreads to InflP in Spec,AgrP and to its head Infl. The latter contains the feature bundle for case, number, and gender ([CNG]) and will be spelled out as the ending on the adjective stem in AP in both structures in \figref{figex:2:19}. 

\begin{figure}
\subfigure[Adjunction of article]{
	\label{figex:2:19a}
    \begin{forest}
    [AgrP$_{[+I]}$
        [InflP$_{[+I]}$
            [{[CNG]$_{[+I]}$}]
            [AP]
        ]
        [Agr$'_{[+I]}$
            [Agr$_{[+I]}$
                [Art$_k$]
                [Agr$_{[+I]}$]
            ]
            [{\dots}t$_k$\dots]
        ]
    ]
    \end{forest}
}
\subfigure[Adjunction of demonstrative]{
	\label{figex:2:19b}
    \begin{forest}
    [AgrP$_{[+I]}$
        [Dem$_{k}$]
        [AgrP$_{[+I]}$
        [InflP$_{[+I]}$
                [{[CNG]$_{[+I]}$}]
                [AP]
            ]
            [Agr$'_{[+I]}$
                [Agr$_{[+I]}$]
                [{\dots}t$_k$\dots]
            ]
        ]
    ]
    \end{forest}
}
\caption{Adjunctions}
\label{figex:2:19}
\end{figure}


Note that the adjunction to heads and phrases, feature Percolation within the same phrase, and Spec-head agreement are all local relations; that is, Impoverishment occurs in a locally restricted domain. Given a recursive AgrP in \figref{figex:2:18} above and successive-cyclic movement of the determiner from ArtP to DP, this analysis provides an immediate explanation of why either all adjectives are strong or all are weak. Next, I flesh out this basic proposal. The following discussion is based on   \citegen{RoehrsJulien2014} analysis of German with one modification (Footnote \ref{foot:20}). In the course of the discussion, I add many details to that proposal. As mentioned above, \citet{RoehrsJulien2014} was heavily influenced by \citet{Sauerland1996}.

\subsubsection{The proposal in more detail: Vocabulary insertion rules}\label{sec:2.2.1.5}

Following \citet{Bierwisch1967} and others, I argue that case, number, and gender are not primitive features. Rather, I assume that case involves a two-category system and can be represented by the features [O(blique)] and [S(tructural)].\footnote{The feature [S(tructural)] was chosen over [D(irect)] to avoid confusion. Below, I propose that the feature [S] is deleted by Impoverishment, which is triggered by the presence of the categorial feature of determiners [+D]. This choice allows me to clearly distinguish between the feature that undergoes deletion and the one that involves the trigger of that deletion.} Each of these features can have a negative or a positive value (i.e., [--, +]) yielding the following decomposition.

\TabPositions{2cm,4cm,6cm,8cm}
\ea%20
    \label{ex:2:20}
\ea\label{ex:2:20a}   nominative: \tab  [--O,  --S]
\ex\label{ex:2:20b}   accusative: \tab  [--O, +S]
\ex\label{ex:2:20c}   dative:     \tab  [+O,  --S]
\ex\label{ex:2:20d}   genitive:   \tab  [+O, +S]
\z
\z
\largerpage[-1]
Similarly, gender consists of the features [F(eminine)] and [N(euter)] and can be broken down as in \REF[a-c]{ex:2:21}. Plural is the neutralization of gender \REF{ex:2:21d}, something \citet[80]{Sternefeld2008} refers to as the “fourth gender”.\footnote{\citet[163-65]{Krifka2009} points out that there are advantages in assuming that plural involves a separate gender (for the relatedness of feminine and plural, see also \citealt{Leiss1994}: 291-94).}

\ea%21
    \label{ex:2:21}
\ea \label{ex:2:21a}   masculine: \tab [--F,  --N]
\ex \label{ex:2:21b}  neuter:     \tab [--F, +N]
\ex \label{ex:2:21c}  feminine:   \tab [+F,  --N]
\ex \label{ex:2:21d}   plural:    \tab [+F, +N]
\z
\z

Unlike plural, all singular forms involve at least one negative value for [F] or [N].

\citet[250 fn. 5]{RoehrsJulien2014} provide motivation for the system above. Most importantly, following \citegen[468]{Harbert2007} Case Hierarchy, they assume that nominative is the least and genitive the most oblique case. Based on \citet{Steinmetz2001}, they assume that masculine is the default gender in German, and following \citet[80]{Sternefeld2008}, they take plural to be the neutralization of gender. Note that taking the masculine gender and the nominative case as the least marked categories – they each have two negative values in their feature decompositions – is consonant with \citegen[253]{Bierwisch1967} statement that there is more syncretism within marked categories (observe that with the strong inflections, masculine has the most different forms in \tabref{tab:2:4}; conversely, genitive has the fewest different forms).\footnote{There are many different systems to decompose case, number, and gender – too many to review here in detail. For other, partially similar decompositions, see \citet[246-50]{Bierwisch1967}, \citet[51]{Evans2019}, \citet[124-25]{Gallmann2004}, G. \citet[119]{Müller2002a}, \citet{Trommer2005}, \citet{Wiese1996,Wiese1999}, \citet[48-49]{Wunderlich1997}, and many others. \citet{Blevins1995} captures syncretism as part of Feature Hierarchy, with primary (e.g., feminine vs. non-feminine) and subsidiary (e.g., masculine vs. neuter) features. This hierarchy as well as Feature Geometry more generally will not play a role here (for some brief remarks on hierarchies as regards DM, see \citealt{ArregiNevins2012}: 204-05, \citealt{HarleyNoyer1999}: 6).}

  Before moving on, I point out that the decomposition of gender and number in \REF{ex:2:21} allows us to state certain well-known syncretisms that we find in the masculine/neuter vs. in the feminine/plural, that is, in the opposition [--F] vs. [+F]: (i) strong adjectival inflections in the nominative/accusative (C = consonant): -\textit{eC} vs. -\textit{e}, (ii) strong adjectival inflections in the dative: -\textit{em} vs. -\textit{er/-en}, (iii) strong adjectival inflections in the genitive: -\textit{es} vs. -\textit{er}, (iv) third-person pronouns in the nominative/accusative (abstracting away from \textit{ihn} ‘him’): \textit{er}/\textit{es} ‘he/it’ vs. \textit{sie} ‘she, her; they, them’, (v) \textit{ein}-words in the nominative/accusative (abstracting away from \textit{einen} ‘a’): uninflected \textit{ein} vs. inflected \textit{eine}, (vi) definite article stems in the nominative/accusative: \textit{de}-/\textit{da}- ‘the’ vs. \textit{di}- ‘the’, (vii) possessive articles consisting of bound and free morphemes: \textit{sein} ‘(POSS+\textit{ein} =) his, its’ vs. \textit{ihr} ‘her, their’ (\chapref{sec:5}, \sectref{sec:5.4.1.2}), and perhaps less widely known (viii) pronominal DPs in the dative masculine/neuter prefer strong adjectives vs. pronominal DPs in the dative feminine and nominative plural prefer weak adjectives (\chapref{sec:3}, \sectref{sec:3.5}).

  As to case, syncretism is much less prevalent here. As should be clear from the discussion of syncretism in gender and number provided just above, there is syncretism in the nominative/accusative vs. the oblique cases. Specifically, while nominative and accusative have the same inflections in the neuter, feminine, and plural (e.g., plural -\textit{e}), the oblique cases dative and genitive have the same inflections only in the feminine (i.e., -\textit{er}). It is also worth pointing out that there are very few instances of syncretism between the structural and oblique cases; for instance, the forms of the personal pronouns of the first and second person plural are the same in the accusative and dative cases (i.e., \textit{uns} ‘us’, \textit{euch} ‘you’). Note though that the latter instances are the exception, rather than the rule. This exceptional syncretism may have to do with the fact that these items are non-third person elements.\footnote{The lack of syncretism between structural and oblique case may also be related to the fact that the structural cases behave differently from the oblique cases in syntactic terms (\citealt{BayerEtAl2001}). To account for these differences, these authors propose that the oblique cases project a K(ase)P on top of the DP, an idea that I will not follow here.}
\largerpage
  In order to state the syncretisms above and to reduce the number of vocabulary insertion rules, I utilize value variables ($\alpha$, $\beta$), whereby $\alpha$ or $\beta$ in a vocabulary insertion rule is compatible with a positive or negative value in the syntactic representation (note that the variable does not get specified for such values – it is a compatibility feature). To give an example, the specification [$\alpha$O] in a vocabulary insertion rule is compatible with a positive or a negative value in the syntactic representation (but not both at the same time). Among others, this captures the syncretism of nominative and accusative case (cf. \REF[a-b]{ex:2:20} above). Similarly, I employ a category variable ($\gamma$), which ranges over [F] or [N]; that is, over one of these two features (but not both at the same time). To illustrate, [--$\gamma$] in a vocabulary insertion rule is compatible with either [--F] or [--N] in the syntactic represenation. This means that [--$\gamma$] corresponds to singular (as opposed to plural, which has a positive value for both [F] and [N]; for the discussion of the explicit use of disjunctions, see G. \citealt{Müller2002a}: 120 fn. 14).\footnote{The category variable [$\gamma$], with a negative value, will find more application with first and second-person pronouns, which lack gender (\chapref{sec:3}, \sectref{sec:3.5}).}

Following \citet{Sauerland1996}, I propose that the two weak endings have the same feature specifications as certain strong endings (cf. also \citealt{Gallmann2004}: 140). I assume that these instances are related and have the same vocabulary insertion rules (see below). For clarity, these related instances are respectively marked by round and curly brackets in \tabref{tab:2:4}. Using the feature decomposition above, all strong and weak inflections can be analyzed as follows (for convenience, I added the traditional labels as subscripts after the featural decomposition).

\begin{table}
\caption{Strong and weak inflections in German}
\label{tab:2:4}
\fittable{
    \begin{tabular}{llllllll}
    \lsptoprule
    {\bfseries STRONG} & [--F, --N]\textsubscript{M} & [--F, +N]\textsubscript{N} & [+F, --N]\textsubscript{F} & [+F, +N]\textsubscript{PL} & {\bfseries WEAK} & [--$\gamma$]\textsubscript{SG} & [+F, +N]\textsubscript{PL}\\
    \midrule
    {}[--O, --S]\textsubscript{NOM} & -er & -es & (-e) & -e      & [--O] & (-e)    & \{-en\} \\
    {}[--O, +S]\textsubscript{ACC} & -en & -es & (-e) & -e      & [--O] & (-e)    & \{-en\} \\
    {}[+O, --S]\textsubscript{DAT} & -em & -em & -er  & \{-en\} & [+O]  & \{-en\} & \{-en\} \\
    {}[+O, +S]\textsubscript{GEN} & -es & -es & -er  & -er     & [+O]  & \{-en\} & \{-en\} \\
    \lspbottomrule
    \end{tabular}
}
\end{table}

Comparing \tabref{tab:2:4} to \tabref{tab:2:3} from above, we notice that the accusative masculine ending -\textit{en} is missing from the weak endings here. The special status of this ending follows from the way the strong ending for accusative masculine is stated in \REF{ex:2:22c}.\footnote{\label{foot:20} In \citet{RoehrsJulien2014}, the exceptional accusative masculine form -\textit{en} contained the specification [+S]. Removing this feature from the vocabulary entry as in \REF{ex:2:22c} still allows me to specify this ending disambiguously (as nominative masculine -\textit{er} is more specific). In fact, this is now the only strong ending without a specification for [S]. This will become significant during the discussion of Impoverishment (which deletes [S]).} Assuming that features can be underspecified, the rules for vocabulary insertion of the endings can be stated as follows.\footnote{In his critique of some previous – what he calls – constructive analyses (\citealt{Bierwisch1967}, \citealt{Blevins1995}, B. \citealt{Wiese1999}, and \citealt{Wunderlich1997}), G. \citet{Müller2002b} points out that all these types of accounts have rules that show different contexts for the insertion of (some of) the same strong inflections, for instance, -\textit{er} in \REF{ex:2:22a} and \REF{ex:2:22b}. This leads to systematic as well as accidental syncretism. According to G. \citet{Müller2002a,Müller2002b}, destructive type of analyses involving rules that ban the insertion of certain inflections fare better.  This criticism seems valid and can also be leveled against the current proposal. However, I think such multiple vocabulary insertion rules for one and the same inflectional form(s) cannot completely be avoided (also \citealt{Sauerland1996}: 24). Considering that certain endings occur in other, unrelated domains as well (e.g., plural inflections on nouns: -\textit{e}, -\textit{en}, -\textit{er}, -\textit{s}; personal endings on verbs: -\textit{e}, -\textit{en}), it is not clear how these inflections can be related yielding just one rule of vocabulary insertion (or ban) for each inflection. Note in this regard that unspecified -\textit{en} in \REF{ex:2:23b} could potentially do multiple duty as a plural inflection for nouns and verbs. An advantage of the current proposal is that unlike some of the previous analyses, here the weak inflections are explicitly discussed; in fact, they are integrated with the strong inflections, and there is no inherent difference between these types of inflections (for the discussion of quite different types of analyses of weak inflections, see the end of \sectref{sec:2.5.3}).}

\TabPositions{3cm,5cm,7cm}
\ea%22
    \label{ex:2:22}
          Strong (except for feminine \textit{-e} and plural \textit{-en}):
\ea\label{ex:2:22a} [+F, --N, +O, $\alpha$S]  \tab $\rightarrow$  \textit{-er}

    [+F, +N, --O, $\alpha$S]  \tab $\rightarrow$  \textit{-e}
\ex\label{ex:2:22b} [$\alpha$F, $\alpha$N, $\alpha$O, $\alpha$S]  \tab $\rightarrow$  \textit{-er}
\ex\label{ex:2:22c} [--F, --N, --O]    \tab $\rightarrow$  \textit{-en}

    [--F, +O, --S]    \tab $\rightarrow$  \textit{-em}
\ex\label{ex:2:22d} [--F, $\alpha$O, $\beta$S]    \tab $\rightarrow$  \textit{-es}
\z
\z
\newpage
\ea%23
    \label{ex:2:23}
             Weak (including strong feminine \textit{-e} and plural \textit{-en}):
\ea\label{ex:2:23a} [--$\gamma$, --O]      \tab $\rightarrow$  \textit{-e}
\ex\label{ex:2:23b} []             \tab $\rightarrow$  \textit{-en}
\z
\z

Comparing \REF{ex:2:22} to \REF{ex:2:23}, note first that the (unambiguously) strong endings are stated in more specific terms than the weak endings. In other words, weak endings are underspecified to a greater degree. This means that there is no real difference between \REF{ex:2:22} and \REF{ex:2:23}. This is confirmed by the fact that some of the strong endings (i.e., certain instances of -\textit{e} and -\textit{en}) share the same vocabulary insertion rules as the weak inflections -\textit{e} and -\textit{en} \REF{ex:2:23}. This was one of \citegen{Sauerland1996} main insights. Second, I assume that a vocabulary item involving the same number of features with a variable is less specific than a corresponding vocabu\-lary item without such a variable. Accordingly, I have listed the vocabulary items with descending degrees of specificity in \REF{ex:2:22} and \REF{ex:2:23}. Note that equally specified items differ from one another. Specifically, in both \REF{ex:2:22a} and \REF{ex:2:22c}, the two respective items vary in the feature [O]. Finally, as is often stated, note that the two paradigms, strong and weak, in \tabref{tab:2:4} are simply generalizations. They have been replaced by the vocabulary insertion rules above; that is, paradigms are an epiphenomenon without independent status. Before moving on, I comment on the relationship between the vocabulary insertion rules and the terminal nodes in the syntactic representation.

Starting with the weak inflections, note that the weak ending -\textit{e} never occurs in the plural. Thus, it distinguishes number as it only occurs in the singular. However, it does not distinguish individual cases or genders. This is particularly clear in the current account where -\textit{e} is specified [--$\gamma$, --O]. The weak ending -\textit{en} is not specified for any CNG features. In the current account, it is the elsewhere case. What both of these inflections have in common is that they do not distinguish case and gender. In contrast, the strong endings are specified for more features. Note first that strong endings are more specific in their occurrence (e.g., -\textit{em} only occurs in the dative masculine/neuter; -\textit{s} only in the masculine/neuter). This is reflected by the contexts in which their vocabulary insertion rules apply; for instance, the inflection -\textit{em} is specified as [--F, +O, --S]. Compared to weak inflections, strong inflections distinguish case, number, and gender better.
\largerpage
The feature specifications on the terminal nodes in the syntactic representation are different. If present, they always involve specifications for the four features [F, N, O, S]. In other words, all features are present in the abstract syntactic representation, but only a subset of those features are specified in the vocabulary insertion rules – the latter involve underspecification allowing competition between some of the vocabulary insertion rules. Note again that these are basic tenets of DM.

\subsubsection{The proposal in more detail: Structural position of inflections}\label{sec:2.2.1.6}

In DM, vocabulary items are inserted late. Matching the maximum number of features on the terminal head, the availability of a more specific vocabulary item precludes the insertion of a less specific one. With the system laid out above in mind, the abstract feature bundle in Infl ([CNG]) of the larger adjective structure can now be restated (for convenience, I continue referring to the feature bundle as CNG if the decomposition is not relevant). Taking dative masculine as an example, the terminal head Infl is specified as in \figref{figex:2:24}. Turning to Vocabulary Insertion, note that [--F, +O, --S] in \REF{ex:2:22c} is the most specific, matching vocabulary item. It is inserted and spelled out as the strong ending -\textit{em} on \textit{klein}- ‘small’ resulting in the form after the arrow.

\begin{figure}
	\caption{Structure and spell-out of adjectives}
    \label{figex:2:24}
\begin{forest}
[,phantom
  [InflP
    [~]
    [Infl$'$
        [{[--F, --N, +O, --S]}]
        [AP
            [~]
            [A$'$
                [A\\\textit{klein}, tier=A]
                [~]
            ]
        ]
    ]
  ]
  [\\$\rightarrow$  \textit{klein}-\textit{em}, no edge, tier=A]
]
\end{forest}
\end{figure}

The structure in \figref{figex:2:24} is a convenient shorthand for a more detailed derivation. More specifically, after movement of the adjective stem \textit{klein} to Spec, InflP, Linearization, and Vocabulary Insertion, we obtain the string in \figref{figex:2:25}. Following \citet{Murphy2018}, I assume that the adjectival suffix undergoes Local Dislocation (instantiated as a type of leaning) combining with the adjective stem to yield the spell-out form after the arrow (see also \sectref{sec:2.5.3}).

\begin{figure}
	\caption{Spell-out of adjectives}
    \label{figex:2:25}
\begin{forest}
[,phantom
  [
  [\textit{klein}, tier=A]
  [\textit{-em}, tier=A]
  ]
  [$\rightarrow$  \textit{kleinem}, no edge, tier=A]
]
\end{forest}
\end{figure}
\newpage
Unless these details are of signifance, I usually illustrate the analysis by a syntactic representation as in \figref{figex:2:24}. As briefly discussed in \chapref{sec:1}, \sectref{sec:1.4.1.2}, determiners also have internal structure. Since they are proposed to be the trigger of Impoverishment, I discuss their inner makeup in more detail now.

I focus here on four common types of determiners: the indefinite null articles (in the contexts of mass and plural nouns), the indefinite article \textit{ein}, the definite article, and the demonstrative. Illustrating with the dative masculine again, these elements have the following internal structure. The terminal head Art where the indefinite null articles later surface has a categorial feature [+D(eterminer)] and a negative feature for definiteness. As null elements, they have no features for case, number, and gender that are later spelled out as inflections, see \figref{figex:2:26a}. The indefinite article \textit{ein} only has the categorial feature [+D] and features for case, number, and gender. Given these different types of features (category vs. CNG), I assume that these make up two individual feature bundles. These two separate bundles undergo feature union yielding Art as in \figref{figex:2:26b}.\footnote{There
    is evidence that inflections are generated with their determiner heads. As mentioned in \chapref{sec:1} and discussed in more detail in \chapref{sec:8}, \sectref{sec:8.2.2.2}, there are cases where \textit{ein} occurs in two positions. Crucially, \textit{ein} in the lower position has the same, varying inflection as related \textit{kein} ‘no’. Compare \REF{ex:2:22:a} to \REF{ex:2:22:b}.
    \ea
        \ea \label{ex:2:22:a}
        \gll k-ein-e     so’n-e  Leute\\
            \textsc{neg}-a-\textsc{st} {so a}-\textsc{st} people\\
        \glt ‘no such people’

        \ex \label{ex:2:22:b}
        \gll   mit   k-ein-en   so’n-en Leuten\\
                with \textsc{neg}-a-\textsc{st} {so a}-\textsc{st} people\\
        \glt ‘with no such people’
        \z
    \z
    This clearly shows that inflected \textit{ein} has a bipartite structure. As indefinite articles are usually assumed to be heads, this is captured in the main text by two separate feature bundles under Art.
} Like the indefinite overt article, the definite article has a feature bundle for case, number, and gender; like the indefinite null articles, the definite article has a definiteness feature, but it is specified as positive as in \figref{figex:2:26c}.

\begin{figure}[t]
\subfigure[Indefinite null articles]{
\label{figex:2:26a}
 \begin{forest}
[,phantom
    [Art\textsubscript{\parbox{0mm}{\mbox{[+D; --DEF]}}},
        [{[+D; --DEF]}, tier=bottom]
    ]
    [ $\rightarrow$  $\emptyset$\textsubscript{D}, no edge, tier=bottom]
]
 \end{forest}
}

\subfigure[Indefinite overt article]{
\label{figex:2:26b}
 \begin{forest}
[,phantom
    [Art\textsubscript{\parbox{0mm}{\mbox{[+D][--F, --N, +O, --S]}}}
            [{[+D]}]
            [{[--F, --N, +O, --S]}, tier=bottom]
    ]
    [$\rightarrow$  \textit{ein-em}, no edge, tier=bottom]
]
\end{forest}
}

\subfigure[Definite article]{
\label{figex:2:26c}
 \begin{forest}
 [,phantom
   [{Art\textsubscript{\parbox{0mm}{\mbox{[+D; +DEF][--F, --N, +O, --S]}}}}
    [{[+D; +DEF]}]
    [{[--F, --N, +O, --S]}, tier=bottom]
   ]
    [$\rightarrow$  \textit{d-em}, tier=bottom]
]
\end{forest}
}
\caption{Structure and spell-out of articles}
\label{ex:2:26}
\end{figure}

The spell-out forms are as follows: [+D; --DEF] is realized as $\emptyset$\textsubscript{D}, [+D] as \textit{ein}-, and [+D; +DEF] as \textit{d}-. As for the inflection, only [--F, +O, --S] matches the CNG bundle yielding -\textit{em}. Taken together, this spells out the forms of the determiners provided after the arrows.
\largerpage[-1]
Unlike the first three elements, the demonstrative is phrasal. I assume there are two terminal heads as in \figref{figex:2:27}. Dem involves the features [+D; +DEF, +DEIX] and builds its own extended projection with InflP at the top (see \citealt{Leu2007,Leu2015}; \citealt{Roehrs2010,Roehrs2013a}). The features for case, number, and gender are in Infl. Dem moves to adjoin to Infl (not shown). Given feature union and Percolation, all features spread to InflP.

\begin{figure}[t]
	\caption{Structure and spell-out of demonstratives}
    \label{figex:2:27}
\begin{forest}
[,phantom
    [{InflP\textsubscript{\parbox{0mm}{\mbox{[+D; +DEF, +DEIX][--F, --N, +O, --S]}}}}, no edge
        [{[--F, --N, +O, --S]}]
        [DemP
            [{Dem\\{[+D; +DEF, +DEIX]}}, tier=bottom]
        ]
    ]
    [\\$\rightarrow$  \textit{dies-em}, tier=bottom]
]
\end{forest}
\end{figure}

The features [+D; +DEF, +DEIX] and [--F, --N, +O, --S] are spelled out as \textit{dies}- and -\textit{em}, respectively, yielding the form after the arrow.

Besides the demonstrative \textit{dieser} ‘this’, other phrasal \textit{der}-words have this general structure: (stressed) \textit{DER} ‘that’, \textit{jener} ‘that’, \textit{jeder} ‘every’, \textit{mancher} ‘some’, \textit{solcher} ‘such’, \textit{welcher} ‘which’, and \textit{alle} ‘all’ (see also \chapref{sec:3}, \sectref{sec:3.4}). While I cannot specify (and motivate) all the relevant features of these individual words, I assume that all these elements have the categorial feature [+D]. Furthermore, like \textit{dieser} ‘this’, the demonstratives \textit{DER} ‘that’ and \textit{jener} ‘that’ involve the features [+DEF, +DEIX]. The quantifiers \textit{jeder} ‘every’ and \textit{alle} ‘all’ and the (presuppositional) interrogative \textit{welcher} ‘which’ have the feature [+DEF] but lack the deixis feature. In contrast, \textit{mancher} ‘some’ and \textit{solcher} ‘such’ lack the definiteness feature but have the feature [+DEIX] in their feature makeup.\footnote{\label{foot:2:23}The presence of the deixis feature is most straightforward with \textit{solch} ‘such’. For simplicity, I also assume this for \textit{manch} ‘some’ (although this element may turn out to have a different positively valued (relevant) feature in its makeup). Notice that these positively valued features play a role later in the discussion (\sectref{sec:2.2.2.1}). On a different note, recall that \textit{mancher} ‘some’, \textit{solcher} ‘such’, and \textit{welcher} ‘which’ can also occur without an inflection (i.e., \textit{manch}, \textit{solch}, \textit{welch}). Given the lack of inflection, I assume that they have a different structure and analysis than in \figref{figex:2:27}. Observe in this regard that they cannot occur directly before nouns (in non-formal contexts) but rather require \textit{ein} and/or an adjective to be present: \textit{manch *(ein/guter) Student} ‘some (good) student’. Given this dependency, I assume that they are not determiners but modificational elements, which do not involve the categorial feature [+D].}

Note that all determiners have the categorial feature [+D]. I propose that this is the trigger for Impoverishment. In addition, some determiners have features for definiteness and/or deixis, which also play a role in the account below. The indefinite article \textit{ein} is different – it has no features for definiteness and deixis. Note already here that considering the features of the various determiner stems, \textit{ein} is the least specified element. In \chapref{sec:5}, I propose in more detail that \textit{ein} is a semantically vacuous element.

\subsubsection{The proposal in more detail: Impoverishment}\label{sec:2.2.1.7}

Impoverishment involves a rule deleting a specific feature. Note that in the current context, Impoverishment cannot be stated in terms of the features [F] or [N] as these features are still relevant for the weak ending -\textit{e} given the category variable $\gamma$ in \REF{ex:2:23a}. Furthermore, it cannot be stated in terms of the feature [O] as that feature is also mentioned in \REF{ex:2:23a}. I propose that Impoverishment deletes the feature [S] if a determiner adjoins to a head or a phrase in the nominal structure.

\ea%28
    \label{ex:2:28}
          {Impoverishment Rule 1:}\\{}

  [\textsubscript{δ} Determiner [\textsubscript{δ} [\sout{S}] ]], where δ = X, XP
\z

Note that this is a language-specific rule that applies to German only (for the discussion of Impoverishment in Yiddish, see \citealt{Roehrs2015}). Furthermore, recall the partial ordering of \{Lowering, Impoverishment\} >> Vocabulary Insertion from \chapref{sec:1}, \sectref{sec:1.4.2.1}. First, as observed by \citet{Sauerland1996}, Impoverishment widens the distribution of the least specified (i.e., weak) inflections. This is consistent with the ordering of Impoverishment preceding Vocabulary Insertion. Second, Impoverishment Rule 1 involves the structural relation of adjunction. Bearing in mind that Lowering also has access to structural relations and precedes Vocabulary Insertion, this also fits with the claim that Impoverishment precedes Vocabulary Insertion. In other words, current assumptions are in agreement with the general layout of DM provided earlier.\footnote{Notice that Impoverishment Rule 1 and Rule 2 (see below) bear some resemblance to  \citeapo{ArregiNevins2012} syntagmatic neutralization rules, Impoverishment rules that involve two distinct nodes where a rule affecting one node makes reference to the (external) morpho-syntactic environment of a second node. Note though that both nodes in their system are inside the same M-word (defined as a X\textsuperscript{0} that is not immediately dominated by another X\textsuperscript{0}; note that M-words are typically complex heads assembled by head movement). While the Impoverishment rules in the main text also apply in a local context, they do not occur inside M-words. If it turns out that the context of application of Impoverishment only involves M-words, then there are several options to update the proposal above; for instance, we could assume that all determiners and adjectives are heads and that determiners move by head adjunction (now including adjunction to adjectives) and excorporation to D.}

  To illustrate the workings of Impoverishment Rule 1, I begin by discussing the determiners, the triggers of Impoverishment, and how they relate to the rest of the nominal structure. With \REF{ex:2:28} in place, the property [+I(mpoverishment)] from \sectref{sec:2.2.1.4} can now be restated. I propose that the categorial feature [+D] triggers Impoverishment. Recall that the definite article moves to adjoin to Agr. Leaving out the positive/negative values of the CNG features, the general constellation is as in \figref{figex:2:29}. Impoverishment Rule 1 deletes [S], marked by strikethrough, from Agr, the adjunction site. In fact, due to Percolation and Spec-head agreement, this feature is deleted from the feature bundles on all elements except the terminal head Art of the definite article itself (for the indefinite articles \textit{ein} and $\emptyset$\textsubscript{D}, see \sectref{sec:2.2.2} and \sectref{sec:2.2.3}).

\begin{figure}
	\caption{Impoverishment by definite article}
    \label{figex:2:29}
\begin{forest}
  [AgrP\textsubscript{\parbox{0mm}{\mbox{[F, N, O, \sout{S}]}}}
    [InflP\textsubscript{\parbox{0mm}{\mbox{[F, N, O, \sout{S}]}}}
        [{[F, N, O, \sout{S}]}]
        [AP]
    ]
    [Agr$'$\textsubscript{\parbox{0mm}{\mbox{[F, N, O, \sout{S}]}}}
        [Agr\textsubscript{\parbox{0mm}{\mbox{[F, N, O, \sout{S}]}}}
            [Art\textsubscript{\parbox{0mm}{\mbox{[+D; +DEF][F, N, O, S]$_k$}}}, s sep=30mm
                [{[+D; +DEF]}]
                [{[F, N, O, S]}]
            ]
            [Agr\textsubscript{\parbox{0mm}{\mbox{[F, N, O, \sout{S}]}}}]
        ]
        [{\dots}t$_k$\dots]
    ]
  ]
\end{forest}
\end{figure}

Operations in DM occur after syntax. Given the successive-cyclic movement of the determiner from ArtP to DP, there are several copies of this element in the nominal structure. Each of these copies triggers Impoverishment deleting the feature [S]. This explains why multiple adjectives in a noun phrase all undergo Impoverishment (provided there is a determiner). Note also that after Impoverishment and Copy Reduction, there is still (at least) one fully specified feature bundle in the structure – it is on the terminal head Art later to be spelled out as the appropriate form of the definite article. Surfacing in the DP-level, this element is accessible to DP-external operations (e.g., case checking/valuing).

  The updated analysis of the demonstrative is similar, but this element undergoes adjunction to AgrP triggering Impoverishment (the demonstrative structure in \figref{figex:2:27} above is simplified in \figref{figex:2:30}  for reasons of space).

\begin{figure}
	\caption{Impoverishment by demonstrative}
    \label{figex:2:30}
\begin{forest}
  [AgrP\textsubscript{\parbox{0mm}{\mbox{[F, N, O, \sout{S}]}}}
    [Dem\textsubscript{\parbox{0mm}{\mbox{[+D; +DEIX][F, N, O, S]$_k$}}}
        [{[+D; +DEIX]}]
        [{[F, N, O, S]}]
    ]
    [AgrP\textsubscript{\parbox{0mm}{\mbox{[F, N, O, \sout{S}]}}}
        [InflP\textsubscript{\parbox{0mm}{\mbox{[F, N, O, \sout{S}]}}}
            [{[F, N, O, \sout{S}]}]
            [AP]
        ]
        [Agr$'$\textsubscript{\parbox{0mm}{\mbox{[F, N, O, \sout{S}]}}}, s sep=20mm
            [Agr\textsubscript{\parbox{0mm}{\mbox{[F, N, O, \sout{S}]}}}]
            [{\dots}t$_k$\dots]
        ]
    ]
]
\end{forest}
\end{figure}

Finally, I turn to the discussion of the inflection on the adjective, the target of Impoverishment.

  If Impoverishment applies, and the feature [S] is deleted in the nominal structure, then this impacts the inflection on the adjective with visible effect – InflP in \figref{figex:2:29} and \figref{figex:2:30} undergoes feature deletion. For instance, the removal of the feature [S] yields \figref{figex:2:31} in the dative masculine. With [S] absent, only [] in \REF{ex:2:23b} above is a match here spelling out the weak ending -\textit{en}.

\begin{figure}
	\caption{Structure and spell-out of adjectives}
    \label{figex:2:31}
\begin{forest}
[,phantom
  [InflP
    [~]
    [Infl$'$
        [{[--F, --N, +O]}]
        [AP
            [~]
            [A$'$
                [{A\\\textit{klein}}, tier=bottom]
                [~]
            ]
        ]
    ]
  ]
  [\\{$\rightarrow$  \textit{klein}-\textit{en}}, tier=bottom, no edge]
]
\end{forest}
\end{figure}

I discuss the inflections on the determiners themselves in \sectref{sec:2.2.4}. Before moving on, note again that all strong endings in \REF{ex:2:22} have the feature [S], except for one: -\textit{en}. In other words, the accusative masculine ending is left untouched by Impoverishment and surfaces as the (so-called) exceptional case in strong and weak contexts alike. In the current account, this inflection is not exceptional: Like the vocabulary entries in \REF{ex:2:23}, -\textit{en} in \REF{ex:2:22c} lacks the feature [S]; unlike the entries in \REF{ex:2:23}, -\textit{en} in \REF{ex:2:22c} is (simply) more specified as regards other features.

The above discussion explains the weak inflections in canonical constructions. Importantly, note that the trigger of Impoverishment, the categorial feature [+D], is on the stem of the determiner (rather than on the inflection). In other words, the current account of the strong/weak alternation is independent of the presence of adjectival inflections on determiners. With this in place, I turn to the discussion of strong adjectives in canonical structures. In the next section, I address adjectives in the context of \textit{ein}-words. In \sectref{sec:2.2.3}, I discuss adjectives following null articles and Saxon Genitives. \sectref{sec:2.2.4} is dedicated to the discussion of more complex canonical DPs – those involving extended-adjective constructions and numerals. It is shown that these cases also follow from the system developed above. In the course of the discussion, I refine the current proposal further.
\subsection{Strong or weak adjectives in canonical DPs: \textit{Ein}-words}\label{sec:2.2.2}

In this section, I provide my analysis of strong adjectives after certain \textit{ein}-words, and I discuss the inflectional behavior of the \textit{ein}-words themselves.

\subsubsection{Adjectives after \textit{ein}-words: Strong inflections}\label{sec:2.2.2.1}

In this subsection, I consider canonical DPs that involve adjectives preceded by \textit{ein}-words. Recall that \textit{ein}-words consist of the indefinite article \textit{ein} ‘a’, the singularity numeral \textit{EIN} ‘one’, the negative article \textit{kein} ‘no’, and possessive articles like \textit{mein} ‘my’, \textit{dein} ‘your’, etc. As is well known, these elements occur with strong adjectives in three instances: in the nominative masculine and in the nominative/accusative neuter. This is exemplified in the neuter with an indefinite article and a possessive article in \REF{ex:2:32a}. In all the other instances, the adjective must be weak as shown with the dative in \REF{ex:2:32b}.

\ea%32
    \label{ex:2:32}
\ea\label{ex:2:32a}
\gll (m-)ein groß-es Auto\\
{(my) a}  big-\textsc{st}   car.\textsc{neut}\\
\glt ‘a / my big car’
\ex\label{ex:2:32b}
\gll mit (m-)einem groß-en Auto\\
with {(my) a}       big-\textsc{wk} car.\textsc{neut}\\
\glt ‘with a / my big car’
\z
\z

This inflectional distribution is often referred to as mixed pattern. Considering \REF{ex:2:32}, it is clear that the presence of possessive \textit{m}- does not make a difference for the inflection on the adjective. Furthermore, given this pattern, I need to say something about the determiners that in some instances, do not trigger Impoverishment \REF{ex:2:32a} but in others, do \REF{ex:2:32b}.

Note first that it is not possible to propose that \REF{ex:2:32a} involves no concord in features, and consequently, there is a strong ending on the adjective. This is so as uninflected \textit{ein} is restricted to specific featural combinations, and the following adjective has a regular strong ending. In other words, the agreement features need to be present underlyingly so that the correct forms of the article and the adjectival inflection can be inserted.\footnote{Note in this regard that the generalization Weak After Strong makes crucial reference to the agreement morpheme of a determiner. In other words, determiners are not unanalyzed word forms that involve some abstract features independently of the overt inflection. As discussed above, I also assume that determiners consist of morphemes. However, I argue that the underlying features are crucial, not the overt inflection.} More generally, this means that concord is not a sufficient condition for the occurrence of weak endings (see also \sectref{sec:2.3}). In order to explain the distribution in \REF{ex:2:32}, I do not modify the general system laid out above but derive these patterns by a certain property of the relevant \textit{ein}-words themselves. I focus on \textit{ein} and later briefly comment on possessive articles, the negative article \textit{kein} ‘no’, and the singularity numeral \textit{EIN} ‘one’.

Some scholars claim that uninflected \textit{ein} is, in a sense, invariable: For some reason, it does not have an ending in the three above-mentioned cases (\citealt{Demske2001}: 33, \citealt{Eisenberg1999}: 233, \citealt{Olsen1991b}: 47 fn. 14). Consequently, the following adjective must be strong to spell out the relevant features for case, number, and gender. These authors seem to suggest that there are two types of lexical items: a determiner involving uninflected \textit{ein}, and a pronoun involving inflected \textit{einer} (more on this in the next subsection).\footnote{Note that it is not possible to simply claim that the adjectival inflection on \textit{ein} emerges to make case, number, and gender features visible. This is so because there are cases where the adjectival inflection is optional (e.g., \textit{ein lila(nes) Kleid} ‘a purple dress’) and instances where no CNG features are visible (e.g., \textit{zehn (Kleider)} ‘ten (dresses)’).} Other authors propose a timing mechanism to explain the three exceptional instances of \textit{ein}. For instance, \citet[Chapter 4]{Roehrs2009a} suggests that certain cases of \textit{ein} move to the DP-level later in the derivation. As a consequence, they do not trigger Impoverishment on the adjective, and the ending on \textit{ein} itself is not licensed. In order to avoid such late syntactic movement, I make a different proposal here.

I break down the behavior of \textit{ein} into two issues: (i) \textit{Ein} itself has no inflection, and (ii) the following adjective is strong (i.e., \textit{ein} does not trigger Impoverishment). At first glance, these points seem to be related (e.g., \citealt{Murphy2018} and references cited therein). Indeed, they have given rise to the traditional generalizations discussed in \sectref{sec:2.2.1.1}: Weak After Strong and the Principle of Monoinflection. Note though that we have already indicated that these generalizations do not cover all the facts (e.g., \textit{wir nett-en Studenten} ‘us nice-\textsc{wk} students’; for a more detailed discussion, see \chapref{sec:3}, \sectref{sec:3.7}). Second, generalizations are descriptions of the facts, not explanations.

Starting with the second issue (i.e., the strong inflection on the adjective after \textit{ein}), I proposed in the previous section that the categorial feature [+D] of the determiner is the trigger for Impoverishment. To explain the special properties of \textit{ein}, I refine this proposal by claiming that the categorial feature [+D] of determiners may be a trigger for Impoverishment but only under certain (featural) conditions. In order to establish a natural group, I propose that four (not three) instances of \textit{ein} are special. Besides the three traditional cases (i.e., nominative masculine and nominative/accusative neuter), I add accusative masculine to the group.\footnote{There
    are two points that deserve mentioning here. First, grouping the accusative masculine of \textit{ein} with the (traditional) three exceptions means that -\textit{en} on the adjective is a strong (rather than weak) ending, an interpretation that is possible given that -\textit{en} is both a strong and weak ending in this feature combination. Indeed, I formulated a vocabulary insertion rule in \REF{ex:2:22c} above that captures this identity directly. Second, \citet[234]{PaulEtAl1989} provide the following forms of \textit{ein} in the nominative/accusative singular in MHG.
    \ea
    \begin{tabular}{lllll}
                 & MASC &  NEUT &  FEM  \\
            NOM  &  ein & ein & ein   & (MHG)\\
            ACC  & ein-en&   ein &  ein(e)\\
    \end{tabular}
    \z
    It appears as if over time, the feminine form of \textit{ein} has taken on -\textit{e}, possibly to mark feminine gender more consistently (cf. \textit{dies-e Lamp-e} ‘this lamp’; for discussion involving markedness and syncretism formation, see \citealt{Bittner2006} and \citealt{Krifka2009}). This change in the feminine may have led to the establishment of a natural group containing the four other instances of \textit{ein} in (i), as suggested in the main text.
} In the system laid out above, nominative/accusative masculine and nominative/accusative neuter are the least marked items in terms of the features [O] and [F]: Both have a negative value in these four instances (recall from \sectref{sec:2.2.1.5} that I take negative values to represent less marked instances).

I propose that [+D] is only a trigger for Impoverishment if it occurs in the context of positively valued [O], [F], [DEF], or [DEIX]. Note now that the terminal heads that are later to be spelled out as the four relevant instances of \textit{ein} have [+D] but lack [DEF], [DEIX]. Furthermore, they have no positive value on [O] and [F]. Consequently, these cases of \textit{ein} will not trigger Impoverishment. This is illustrated with \textit{ein} in the nominative masculine.

\begin{figure}
	\caption{No Impoverishment by \textit{ein}}
    \label{ex:2:33}
\fittable{
\begin{forest}
[AgrP\textsubscript{\parbox{0mm}{\mbox{[--F, --N, --O, --S]}}}
    [InflP\textsubscript{\parbox{0mm}{\mbox{[--F, --N, --O, --S]}}}
        [{[--F, --N, --O, --S]}]
        [AP]
    ]
    [Agr$'$\textsubscript{\parbox{0mm}{\mbox{[--F, --N, --O, --S]}}}
        [Agr\textsubscript{\parbox{0mm}{\mbox{[--F, --N, --O, --S]}}}, s sep=20mm
            [Art\textsubscript{\parbox{0mm}{\mbox{[+D][--F, --N, --O, --S]$_k$}}}, s sep=20mm
                [{[+D]}]
                [{[--F, --N, --O, --S]}]
            ]
            [Agr\textsubscript{\parbox{0mm}{\mbox{[--F, --N, --O, --S]}}}]
        ]
        [{{\dots}t$_k$\dots}]
    ]
]
\end{forest}
}
\end{figure}

To be clear, like the other determiners, the four instances of \textit{ein} undergo movement up the syntactic structure by way of adjunction. Unlike the other determiners, they do not trigger Impoverishment given the feature specification under Art. In keeping with the tradition, I usually refer to the special cases as the three (not four) instances of \textit{ein}.

\subsubsection{Adjectives after \textit{ein}-words: Uninflected \textit{ein}}\label{sec:2.2.2.2}

Turning to the first issue (i.e., \textit{ein} may have no inflection), note that the endings on \textit{ein} do appear in elliptical contexts when no noun or adjective is present. Compare \REF[a-b]{ex:2:34} to \REF{ex:2:34c}.\footnote{Observe
    that ellipsis is irrelevant for Impoverishment; that is, the ellipsis of a noun in a regular DP does not impact the strong/weak alternation on the adjective itself.
    \ea
    \gll der gut-e\\
        the good-\textsc{wk}\\
        \glt ‘the good one’
    \z
    Also, recall from \chapref{sec:1}, Footnote \ref{foot:1:35} that \textit{der}-words used pronominally are different from \textit{ein} in \REF{ex:2:34c} in that they have two inflections in certain feature combinations, the usual strong inflection and an additional -\textit{en} (e.g., \textit{d-en-en} ‘the-\textsc{st}-\textsc{infl} in the dative plural). This means that the additional strong inflection on \textit{ein} in \REF{ex:2:34c} has its own explanation.
}

\ea%34
    \label{ex:2:34}
\ea\label{ex:2:34a}
\gll Das ist ein(*-er) Wagen. \\
this is  a*-\textsc{st}       car.\textsc{masc}\\
\glt ‘This is a car.’
\ex\label{ex:2:34b}
\gll Das ist ein(*-er) gut-er. \\
this is  a*-\textsc{st}       good-\textsc{st}\\
\glt ‘This is a good one.’
\ex\label{ex:2:34c}
\gll Das ist ein*(-er). \\
this is   one-\textsc{st}\\
\glt ‘This is one.’
\z
\z

With \citet{Panagiotidis2002,Panagiotidis2003} and others, I assume that \REF[b-c]{ex:2:34} involve null nouns (cf. also \textit{pro} as in \citealt{Kester1996a,Kester1996b}; \citealt{Lobeck1995}; \citealt{Olsen1987}; \citealt{Rehn2019}: 209--12). This makes these two nominals structurally parallel to \REF{ex:2:34a}, which involves an overt noun.

  Inflected \textit{einer} as in \REF{ex:2:34c} can be followed by overt elements provided these elements are outside the noun phrase that contains \textit{ein} and the (null) matrix noun. Such elements include genitive DPs, PPs, and relative clauses.

\ea%35
    \label{ex:2:35}
\ea\label{ex:2:35a}
\gll ein-er  mein-er  Freunde\\
one-\textsc{st} my-\textsc{gen} friends\\
\glt ‘one of my friends’
\ex\label{ex:2:35b}
\gll ein-er  von mein-en Freunden\\
one-\textsc{st} of   my-\textsc{dat} friends\\
\glt ‘one of my friends’
\ex\label{ex:2:35c}
\gll ein-er, den ich gesehen habe\\
one-\textsc{st} that I    seen       have\\
\glt ‘one that I have seen’
\z
\z

I summarize the complete set of \textit{ein}-words in \tabref{tab:2:5}, where the inflections in brackets indicate the forms occurring in the absence of the relevant overt material (e.g., adjectives and nouns in the matrix nominals). Given the restrictions on the occurrence of \textit{ein} in the plural (\chapref{sec:1}, \sectref{sec:1.2.2}), I illustrate this with the possessive article \textit{mein}- ‘my’.\footnote{The two neuter forms of \textit{meines} are often realized without a schwa as \textit{meins} (the same goes for the other \textit{ein}-word forms in these instances: \textit{eins} ‘a’, \textit{EINS} ‘one’, and \textit{keins} ‘no’).}

\begin{table}
\caption{Endings on \textit{ein}-words}
\label{tab:2:5}
\begin{tabularx}{\textwidth}{XXXXl}
\lsptoprule
 & Masculine & Neuter & Feminine & Plural\\
Nominative & mein-[er] & mein-[es] & mein-e & mein-e\\
Accusative & mein-en & mein-[es] & mein-e & mein-e\\
Dative & mein-em & mein-em & mein-er & mein-en\\
Genitive & mein-es & mein-es & mein-er & mein-er\\
\lspbottomrule
\end{tabularx}
\end{table}

Abstracting away from the brackets for a moment, the inflections in \tabref{tab:2:5} are the same as those on the \textit{der}-words discussed earlier. Having said that, recall that the inflection on \textit{ein} in the three special instances is a function of the presence or absence of following overt material such as an adjective and/or a noun. Given the large overlap of forms, it is, in my view, undesirable to state two types of vocabulary items, one involving \textit{ein} as a determiner and one with \textit{einer} as a pronominal form. Rather, since the presence of the inflection depends on overt material, I propose that the alternation between uninflected and inflected \textit{ein} is brought about by certain vocabulary insertion rules. These are late operations that occur after lexical elements such as adjectives and nouns have been inserted.

Inspired by \citegen[13--16]{Höhn2020} analysis of the third-person gap in pronominal DPs (e.g., *\textit{they linguists}), I provide a – what he calls – PF-analysis, an account of uninflected \textit{ein} that occurs after syntax. I propose that the different forms of \textit{ein} present another case of contextually conditioned allomorphy. Höhn states that the relevant overt material must be linearly adjacent and in the same spell-out domain. As to the latter, I assume that the DP as a whole involves an independent domain – a phase in \citeauthor{Chomsky2000} (\citeyear{Chomsky2000} and subsequent work). Rather than formulating three separate, exceptional instances of \textit{ein} (in the nominative masculine and nominative/accusative neuter), I use the feature system from \sectref{sec:2.2.1.5} and propose the following vocabulary insertion rules.

The insertion rules are provided in \REF{ex:2:36}: \REF[a-c]{ex:2:36} state the vocabulary entries for \textit{ein} and \REF[d-e]{ex:2:36} for the adjectival inflections on \textit{ein} (note that \REF{ex:2:36d} shows the entry for the nominative masculine inflection -\textit{er}, and \REF{ex:2:36e} stands for the remaining vocabulary entries of the endings from \sectref{sec:2.2.1.5}). As mentioned several times before, instances in the accusative masculine are exceptional in the nominal system in German. As such, I single out this feature combination as the most specific one \REF{ex:2:36a}. In order to make sure that this element is inserted in the right feature combination, the context of the application of the rule is specified by the relevant features after the forward slash sign. Note also that \REF{ex:2:36a} only spells out the stem feature – the CNG features are spelled out by the vocabulary insertion rules for the adjectival inflections. In contrast, \REF{ex:2:36b} spells out the stem feature and the CNG features by one element. This explains the lack of inflection on \textit{ein} in these cases. Furthermore, \REF{ex:2:36b} is sensitive to the presence of overt material, the following adjective and/or noun, at the right edge of the same phase. This adheres to Höhn’s condition of linear adjacency. Given the relevance of word order, this is consistent with the assumption from \chapref{sec:1} that Vocabulary Insertion follows Linearization. Notice that taken together, \REF{ex:2:36a} and \REF{ex:2:36b} form the four special cases of \textit{ein}. Finally, the vocabulary entry in \REF{ex:2:36c} presents the elsewhere case for \textit{ein} (word in \REF{ex:2:36b} stands for a prosodic word, and the closing square bracket followed by φ indicates the right edge of the DP phase, more on this below).\footnote{For the use of a forward slash sign to define the contextual condition of vocabulary items, see \citet[299]{EmbickNoyer2007}. Also, the setup in \REF[a-c]{ex:2:36} is inspired by \citegen[146]{Blevins1995} treatment of weak adjectives in German where the form \textit{kleinen} ‘small’ is both the most and least specific, and \textit{kleine} is in between (also \citealt{Evans2019}: 56). Note that in an earlier version of this work, I formulated \REF{ex:2:36a} as [+D][--F, --N, --O, +S] \tab $\rightarrow$ \textit{einen}, where the latter spells out both the categorial feature and the CNG features by one element. Observe though that \textit{einen} would now be an unanalyzed form; that is, the ending -\textit{en} would no longer be a regular adjectival inflection. Also, it is worth pointing out that the specification of a feature context after the forward slash sign as in \REF{ex:2:36a} will find wider application with inflected third-person pronouns (see next subsection), inflected definite articles (see \chapref{sec:3}, \sectref{sec:3.3}), and first and second-person pronouns, which do not have CNG features as part of their structure (\chapref{sec:3}, \sectref{sec:3.5}). Finally, note again that \REF{ex:2:36b} involves two separate feature bundles spelled out by one morpheme. This has to do with the earlier observation that articles, despite being heads, have internal structure consisting of stems and inflections. In some cases, these individual bundles are spelled out by one morpheme \REF{ex:2:36b}; in others, they are realized by two morphemes where \REF[a,c]{ex:2:36} and \REF[d-e]{ex:2:36} are later combined by Local Dislocation (for sample derivations, see the main text below). That two feature bundles are spelled out by one morpheme is independently needed for (uninflected) third-person pronouns in the genitive (see next subsection). More generally, the different types of insertion rules in \REF{ex:2:36} will be used for other cases as well.}

\ea%36
    \label{ex:2:36}
\ea\label{ex:2:36a} [+D]      \tab $\rightarrow$   \textit{ein-} / [--F, --N, --O, +S]
\ex\label{ex:2:36b} [+D][--F, --O]    \tab $\rightarrow$   \textit{ein} / {\longrule} word ]\textsubscript{φ}
\ex\label{ex:2:36c} [+D]      \tab $\rightarrow$   \textit{ein}-

\ex\label{ex:2:36d} [+F, --N, +O, $\alpha$S]  \tab $\rightarrow$  \textit{-er}
\ex\label{ex:2:36e} etc.
\z
\z

Some more comments are in order.

I assume that a rule with specifications for both the categorial feature [+D] and the CNG features is more specific than a rule with either the categorial feature [+D] or the CNG features alone; that is, \REF[a-b]{ex:2:36} are more specific than \REF[c-e]{ex:2:36}. Note again that \REF{ex:2:36a} is more specific in its feature specification than \REF{ex:2:36b}: [--F, --N, --O, +S] vs. [--F, --O]. This means that \REF{ex:2:36a} applies first, provided its feature context matches that of the noun phrase. In conjunction with \REF{ex:2:36e}, this brings about the inflected form \textit{einen}. If \REF{ex:2:36a} does not apply, \REF{ex:2:36b} may, again provided its features and context match the syntactic representation. Specifically, \REF{ex:2:36b} spells out uninflected \textit{ein} in the context of a following overt word inside the phase φ. However, if both \REF{ex:2:36a} and \REF{ex:2:36b} do not apply, \REF{ex:2:36c} does. In conjunction with \REF{ex:2:36d} or \REF{ex:2:36e}, this brings about the remaining inflected forms of \textit{ein} including the ones in elliptical contexts. Note that the latter cases also concern the three exceptional instances where \textit{ein} surfaces with a strong inflection in elliptical contexts. Finally, observe that the insertion rules in \REF[a-c]{ex:2:36} involve three related vocabulary items and that there is no inherent difference as regards determiner vs. pronominal forms. Next, I provide more details as regards the morpho-syntax involved.

  After movement of \textit{ein} to the DP-level (provided the nominal is an argument), the simplified structure of the noun phrase can be represented as shown in \figref{figex:2:37}.

\begin{figure}
	\caption{Movement of the article to D}
    \label{figex:2:37}
\begin{forest}
  [DP
    [~]
    [D$'$
        [D
            [Art$_k$
                [{[+D]}]
                [{[F, N, O, S]}]
            ]
            [D]
        ]
        [ArtP
            [~]
            [Art$'$
                [t$_k$]
                [NumP
                    [~]
                    [Num$'$
                        [Num
                            [N$_i$]
                            [Num]
                        ]
                        [{NP\\t$_i$}]
                    ]
                ]
            ]
        ]
    ]
  ]
\end{forest}
\end{figure}

Like \citet{Höhn2020}, I assume that lexical elements like adjective and nouns are inserted before functional elements like determiners and inflections. This is consistent with \citegen[295]{EmbickNoyer2007} assumption that lexical elements are merged as terminal nodes that involve phonological features.\footnote{This early presence of lexical elements in the derivation is presumably needed independently as nouns in German have inherent grammatical gender. Mediated by a syntactic mechanism that brings about concord in agreement features, this gender determines the form of the inflected determiner (e.g., \textit{der}\textsubscript{MASC} vs. \textit{das}\textsubscript{NEUT} vs. \textit{die}\textsubscript{FEM}).} Now, after Linearization, functional elements and lexical elements like, for instance, \textit{Wagen} ‘car’ or a null noun, yield strings such as in \figref{figex:2:38} below. To finalize the derivation, I highlight the interaction of the rules in \REF{ex:2:36} with the presence of overt material, specifically, the noun. I use examples in the nominative and dative masculine for illustration (I comment on \REF{ex:2:36a}, which involves accusative masculine, at the end).

First, when the features later to be spelled out as \textit{ein} are in the nominative masculine and appear in the context of an overt noun, insertion rule \REF{ex:2:36b} applies yielding the spell-out form after the arrow in \figref{figex:2:38a}. Specifically, \REF{ex:2:36b} spells out the adjacent feature bundles for category and case, number, and gender by one element. Second, when the same features occur with a null noun, insertion rules \REF{ex:2:36c} and \REF{ex:2:36d} apply bringing about the spell-out in \figref{figex:2:38b}. In contrast to the first instance, here both feature bundles of the article are realized separately, and the inflection undergoes Local Dislocation onto the determiner stem. Third, similar assumptions hold for the dative masculine. Here, \REF{ex:2:36c} and \REF{ex:2:36e} apply. The presence of a relevant overt element makes no difference in this feature combination as shown in \figref{figex:2:38c}.

\begin{figure}
    \subfigure[Spell-out of nominative \emph{ein} with overt noun]{   
\label{figex:2:38a}
\begin{forest}
[, phantom
  [
    [{[+D]}]
    [{[--F, --N, --O, --S]}]
    [\textit{Wagen}~~, tier=bottom]
  ]
  [$\rightarrow$   \textit{ein Wagen}, tier=bottom]
]
\end{forest}
}

\subfigure[Spell-out of nominative \emph{ein} without overt noun]{
\label{figex:2:38b}
\begin{forest}
[, phantom
  [
    [{[+D]}]
    [{[--F, --N, --O, --S]}]
    [~~~e$_N$~~~, tier= bottom]
  ]
  [$\rightarrow$  \textit{einer}, tier=bottom]
]
\end{forest}
}

\subfigure[Spell-out of dative \emph{einem}]{
\label{figex:2:38c}
\begin{forest}
[, phantom
  [
    [{[+D]}]
    [{[--F, --N, +O, --S]}]
    [\textit{Wagen}/e$_N$, tier=bottom]
  ]
  [$\rightarrow$  \textit{einem (Wagen)}, tier=bottom]
]
\end{forest}
}
\caption{Spell-outs of \emph{ein}}
\label{figex:2:38}
\end{figure}

Briefly commenting on \REF{ex:2:36a}, it works as in \figref{figex:2:38c}, only in a different featural context.

To be clear, the vocabulary insertion rules only supply the phonetic form of \textit{ein} and that of the inflection – the abstract features have been present during the entire syntactic derivation. Furthermore, unlike adjectives and nouns as in \REF{ex:2:36b}, elements outside the DP (genitive DPs, PPs, or relative clauses) are not part of the DP phase, delinated above by φ, and do not have an impact on the calculation of the right edge. This means that \REF{ex:2:36b} does not apply. Hence, genitive DPs, PPs, and relative clauses occur with inflected \textit{ein} if an adjective and/or noun is absent in the matrix DP. Before turning to the other \textit{ein}-words, I briefly discuss two extensions of the insertion rules in \REF[a-b]{ex:2:36}.

\subsubsection{Adjectives after \textit{ein}-words: Some extensions}\label{sec:2.2.2.3}

First, note that these insertion rules may also be relevant to another, very recent development in German. \citet{Vogel2006} observes that the reduced form of the indefinite article \textit{’n} is replaced in the nominative masculine and nominative/accusative neuter – the three exceptional cases – by \textit{nen}. This is particularly clear in chat contexts (notice that the example below makes it clear that \textit{nen} is not an accusative form).

\ea%39
    \label{ex:2:39}
\gll  das  waere     nen gut-er    preis\\
  that would.be a    good-\textsc{st} price.\textsc{masc}\\
  \glt ‘That would be a good price.’
\z

Vogel proposes that speakers might take \textit{’n} to be too short filling it with more phonetic material. With an added initial \textit{n} and schwa, this makes \textit{nen} more similar to other reduced forms of the article (e.g., accusative masculine \textit{nen}).

I interpret \textit{nen} as the reduced form of \textit{einen} (see \chapref{sec:5}). With this in mind, notice that this new form \textit{nen} occurs in the featural context of \REF{ex:2:36b}, that is, in the masculine/neuter in the nominative and accusative cases. Thus, as an alternative to Vogel’s explanation, we could suggest that the featural context of the insertion rule \REF{ex:2:36a} has become less specific yielding the context in \REF{ex:2:36b}: [+D] \tab $\rightarrow$ \textit{(ei)n-} / [--F, --O].

Vogel also states that unreduced \textit{einen} is sometimes replaced by reduced \textit{ein} (rather than \textit{nen}) in the accusative masculine. This makes the accusative identical to the nominative here. This holds for Mannheim German, more generally. I argue in detail in \chapref{sec:3}, \sectref{sec:3.8} that in this dialect, \REF{ex:2:36a} has been deleted. What seems to be clear then is that the featural contexts of \REF[a-b]{ex:2:36} – masculine/neuter in the nominative/accusative – are special. I take this as confirmation that these four instances of \textit{ein} form a natural group, in Standard German and in other varieties. We see in the next paragraphs that the forms of the third-person personal pronouns are also delineated by this feature combination.

The second extension of the analysis of \textit{ein} involves pronominal determiners (for fuller discussion, see \chapref{sec:3}, \sectref{sec:3.5}). As is well known, third-person personal pronouns have the same inflections as definite determiners. Consider \tabref{tab:2:6}. As mentioned in \sectref{sec:2.2.1.5} above, there is syncretism in the nominative and accusative such that masculine and neuter pattern together, and feminine and plural do too: \textit{e-/ih}- ‘he, him; it’ vs. \textit{sie} ‘she, her; they, them’. Note that like with determiners (and adjectives), the accusative masculine form (\textit{ihn}) is exceptional. Also, all dative forms start in \textit{ih}-, and there is an additional ending (-\textit{en}) on the dative plural pronoun \textit{ihnen} (which I abstract away from here; for the historical development of this -\textit{en}, see \citealt{Lühr1991}). Finally, notice that unlike the other forms, the genitive pronouns do not have regular inflections.

\begin{table}
\caption{Third-person personal pronouns in German}
\label{tab:2:6}
\begin{tabularx}{\textwidth}{XXXXl}
\lsptoprule
Gen/num & Masc & Neut & Fem & PL\\
\midrule
Nom & e-r & e-s & sie & sie\\
Acc & ih-n & e-s & sie & sie\\
Dat & ih-m & ih-m & ih-r & ih-n-en\\
Gen & seiner & seiner & ihrer & ihrer\\
\lspbottomrule
\end{tabularx}
\end{table}

There are some interesting parallels to the distribution of \textit{ein}.

Recall that the three exceptional cases of \textit{ein} involve nominative masculine and nominative/accusative neuter, where \textit{ein} does not have an inflection if an adjective and/or noun follows. The corresponding third-person personal pronouns are also special: These three cases are the only ones that start in \textit{e}-, and as is well known, none of the third-person personal pronouns can be followed by an adjective and/or noun (e.g., \textit{er} \textit{(*Idiot)} ‘he (*idiot)’; \citealt{Höhn2020}). In other words, both uninflected \textit{ein} and the stem \textit{e}- occur in exactly the same contexts, and both are sensitive to the overt material that follows them (although in opposite ways).

Similar to the vocabulary insertion rules of \textit{ein}, I propose that the exceptional accusative masculine form \textit{ihn} ‘him’ is singled out as the most specific element \REF{ex:2:40a}.\footnote{In \chapref{sec:3}, \sectref{sec:3.5}, I propose in detail that personal pronouns do not involve features for definiteness and deixis but rather for author (AUTH) and participant (PART) (for a full list of the vocabulary insertion rules for personal pronouns, see \chapref{sec:3}, \sectref{sec:3.5.3.2}).} Note that it only spells out the stem form (but not the CNG features; see also \citealt[1297-98]{GunkelEtAl2017}). In order to make sure that this form (and others like it) is inserted in the right feature combinations, an indication to that effect is provided after the forward slash sign. The genitive pronouns are the second most specified elements \REF[b-c]{ex:2:40}. Unlike \textit{ihn} ‘him’ (and the other forms), they spell out the stem features and the CNG features by one element. This explains their irregular endings. The pronouns starting in \textit{e}- are the third most specified forms \REF{ex:2:40d}. Note now that [--F, --O] is the same feature combination as that of uninflected \textit{ein}. As to the remaining rules, \REF{ex:2:40e} provides the insertion rule for the feminine/plural form in the nominative/accusative, and \REF{ex:2:40f} involves the elsewhere case. Observe that all vocabulary insertion rules specify that there cannot be an overt word in the right edge of the DP adopting an idea from \citet[16]{Höhn2020}. Finally, \REF{ex:2:40g} and \REF{ex:2:40h} specify the vocabulary insertion rules for the adjectival inflections (see \REF{ex:2:22} and \REF{ex:2:23} above; recall that the schwa in the insertion rules in \REF[g-h]{ex:2:40} is due to epenthesis, not applying here).

\TabPositions{6cm,8cm}
\ea%40
    \label{ex:2:40}
\ea\label{ex:2:40a} [+D; --AUTH, --PART]     \tab $\rightarrow$   \textit{ih-}         / {\longrule}]\textsubscript{φ} [--F, --N, --O, +S]
\ex\label{ex:2:40b} [+D; --AUTH, --PART] [--F, +O, +S]  \tab $\rightarrow$   \textit{seiner}    / {\longrule}]\textsubscript{φ}
\ex\label{ex:2:40c} [+D; --AUTH, --PART] [+O, +S]  \tab $\rightarrow$   \textit{ihrer}    / {\longrule}]\textsubscript{φ}
\ex\label{ex:2:40d} [+D; --AUTH, --PART]      \tab $\rightarrow$   \textit{e-}    / {\longrule}]\textsubscript{φ} [--F, --O]
\ex\label{ex:2:40e} [+D; --AUTH, --PART]      \tab $\rightarrow$   \textit{sie-}        / {\longrule}]\textsubscript{φ} [--O]
\ex\label{ex:2:40f} [+D; --AUTH, --PART]      \tab $\rightarrow$   \textit{ih-}  / {\longrule}]\textsubscript{φ}
\ex\label{ex:2:40g} [+F, --N, +O, $\alpha$S]       \tab $\rightarrow$  \textit{-r}
\ex\label{ex:2:40h} etc.
\z
\z

Note that the extension of the discussion of \textit{ein} to third-person pronouns highlights several points. First, one morpheme can spell out two feature bundles at the same time \REF[b-c]{ex:2:40}.\footnote{For another advantage of \textit{ein} spelling out two feature bundles, see \chapref{sec:8}, \sectref{sec:8.2.1}, where I briefly discuss \citegen{Rehn2019} analysis of adjectival inflections in Alemannic German.} Second, the exceptional accusative masculine form is part of the four special cases (masculine/neuter in the nominative/accusative). In order to capture this, the set-up of the vocabulary insertion rules in \REF{ex:2:36} and \REF{ex:2:40} is basically the same: \REF{ex:2:40a} corresponds to \REF{ex:2:36a}, \REF{ex:2:40d} relates to \REF{ex:2:36b}, and \REF{ex:2:40f} is the elsewhere case just like \REF{ex:2:36c}. Finally, I comment on the other types of \textit{ein}-words.

Recall that all \textit{ein}-words behave the same as regards their inflection (also \chapref{sec:5}). It would be undesirable though to formulate the same type of vocabulary insertion rules in \REF[a-c]{ex:2:36} for possessive articles, the negative article, and the singularity numeral, respectively. Rather, the composite analysis of these \textit{ein}-words (consisting of \textit{ein} and another component) allows us to use the vocabulary items in \REF[a-c]{ex:2:36} here as well. While I provide a more detailed discussion in \chapref{sec:5}, I point out briefly here that possessive elements undergo some late interaction with \textit{ein} (i.e., after the insertion of \textit{ein}). Note in this regard that possessive articles fall into two groups: elements that are phonetically similar to \textit{ein}, and those that are not. Decomposing \textit{mein} ‘my’, \textit{dein} ‘your(\textsc{sg})’, and \textit{sein} ‘his’ into \textit{ein} and possessive components, the latter parts surface as in \REF{ex:2:41a}. The second type of the possessive articles involves free forms as in \REF{ex:2:41b}.

\ea%41
    \label{ex:2:41}
\ea \label{ex:2:41a}
\gll m-,  d-,              s-\\
    my, your(\textsc{sg}), his\\
\ex\label{ex:2:41b}
\gll ihr,                                  unser, euer\\
her/their/your(\textsc{formal}), our,     your(\textsc{pl})\\
\z
\z

Recall also that \textit{ein} is proposed to be a semantically vacuous element. After Vocabulary Insertion, the elements in \REF{ex:2:41a} combine with \textit{ein}; that is, the possessive components are supported by \textit{ein}. The same holds for the negative article \textit{kein} ‘(NEG+a =) no’ and the singularity numeral \textit{EIN} ‘($\emptyset$\textsubscript{[--}\textsc{\textsubscript{pl}}\textsubscript{]}+a =) one’. As for \REF{ex:2:41b}, I assume that these forms suppress the pronunciation of \textit{ein} (for more details, see \chapref{sec:5}). In the next section, I turn to another set of cases where strong or weak adjectives are possible.
\subsection{Strong or weak adjectives in canonical DPs: Null articles and Saxon Genitives}\label{sec:2.2.3}

I briefly argued in \chapref{sec:1}, \sectref{sec:1.4.1.2} that prenominal possessives are in Spec,DP. This means that possessives can structurally occur with articles. In the previous section, this is what I tacitly assumed for possessive articles where the possessive component (e.g., \textit{m}-) is in Spec,DP and \textit{ein} is in D yielding \textit{mein} ‘my’ (more on this in \chapref{sec:5}). Similarly, I suggested in \chapref{sec:1} that Saxon Genitives are also in Spec,DP. Unlike possessive articles, they do not occur with an overt determiner, and I assumed that Saxon Genitives are followed by a null article.

There are several advantages of this idea: On the one hand, all possessives occur with articles; on the other hand, we can capture the fact that adjectives that occur after Saxon Genitives behave in the same way as adjectives that are not preceded by an (overt) determiner or determiner-like element. This is illustrated with the nominative in \REF{ex:2:42a} and with the dative in \REF{ex:2:42b}.

\ea%42
    \label{ex:2:42}
    \ea\label{ex:2:42a}
    \gll (Marias) kalt-es  Bier\\
    {\db}Mary’s   cold-\textsc{st} beer.\textsc{neut}\\
    \glt ‘(Mary’s) cold beer’
    \ex\label{ex:2:42b}
    \gll mit (Marias) kalt-em Bier\\
    with {\db}Mary’s   cold-\textsc{st} beer.\textsc{neut}\\
    \glt ‘with (Mary’s) cold beer’
    \z
\z

On my assumptions then, both \REF{ex:2:42a} and \REF{ex:2:42b} involve null articles. We can observe that like in the previous section, the presence of the possessive does not have an impact on the inflection of the adjective. However, unlike above, the adjective here is also strong in the dative. As such, Saxon Genitives are in stark contrast to possessive articles such as \textit{mein} ‘my’ (and the other \textit{ein}-words). Before proceeding, note also that \textit{wessen} ‘whose’ and \textit{dessen} ‘his’ behave like Saxon Genitives as seen in the dative.

\ea%43
    \label{ex:2:43}
    \ea\label{ex:2:43a}
    \gll mit (wessen) kalt-em Bier\\
    with {\db}whose    cold-\textsc{st} beer.\textsc{neut}\\
    \glt ‘with (whose) cold beer’
    \ex\label{ex:2:43b}
    \gll mit  (dessen) kalt-em Bier\\
    with {\db}his        cold-\textsc{st} beer.\textsc{neut}\\
    \glt ‘with (his) cold beer’
    \z
\z

This means that \textit{wessen} ‘whose’ is morpho-syntactically related to Saxon Genitives and \textit{dessen} ‘his’ (but not to possessive articles). Note that this might be relatable to the proposal that unlike possessive articles, Saxon Genitives and \textit{dessen} do not involve composite forms (which consist of a possessive component and the article \textit{ein}). Be that as it may, given the different inflections on the following adjectives, Saxon Genitives and possessive articles must have a different analysis. Interestingly, the inflectional behavior of adjectives in the genitive is revealing here.

  As mentioned in \sectref{sec:2.2.1.2}, (overtly) unpreceded adjectives have strong inflections, with the exception of two instances: In genitive masculine/neuter contexts, these adjectives are weak. Note that the noun has an inflection in these very contexts. Compare \REF{ex:2:44a} to \REF{ex:2:44b}. Once again, the presence of a Saxon Genitive does not make a difference.

\ea%44
    \label{ex:2:44}
    \ea[*]{\label{ex:2:44a}
        \gll statt          kalt-es  Bier-es\\
    instead.of  cold-\textsc{st} bier.\textsc{neut}-\textsc{gen}\\
    \glt ‘instead of cold beer’
    }

    \ex\label{ex:2:44b}
    \gll statt (Marias) kalt-en   Bier-es\\
    instead.of {\db}Mary’s  cold-\textsc{wk} bier.\textsc{neut}-\textsc{gen}\\
    \glt ‘instead of Mary’s cold beer’
    \z
\z

Recall that the inflections on the determiners themselves do not undergo Impoverishment (i.e., they are strong). This means that the ending -\textit{es} on, for instance, the determiners \textit{des} ‘the’ or \textit{eines} ‘a’ in the genitive masculine/neuter, is directly related to the underlying features. Now, if the endings on the determiners and those on the adjectives are indeed the same, then -\textit{en} on the adjectives in the genitive masculine/neuter must be the result of Impoverishment.\footnote{Historically, adjectives in this context had the strong ending -\textit{es}. According to \citet[84]{Demske2001}, the weak ending -\textit{en} started to spread in the 15\textsuperscript{th} century, and according to \citet[29]{Sahel2021}, this change was basically completed during the 18\textsuperscript{th} century.}

  On the current analysis, a weak adjective in these contexts could imply the presence of a determiner triggering Impoverishment. One could suggest that this determiner is an(other) null article $\emptyset$\textsubscript{D} with the feature specification [+D][--F, +O, +S] yielding $\emptyset$\textit{-es}. Following \citet{Roehrs2009a}, another timing mechanism could be suggested. It could be claimed that the null article in the genitive masculine/neuter moves to the DP-level in the regular fashion triggering Impoverishment (the remaining instances of the null article would move later in the derivation). Since null articles cannot support overt suffixes, a repair mechanism could be formulated. It could be suggested that the lower copy of the null article combines with the head noun by partial N-raising, and the higher copy of the null article is deleted under Recoverability of Deletion.

\ea%45
    \label{ex:2:45}
          [\textsubscript{DP} \sout{$\emptyset$-\textit{es}} [\textsubscript{AgrP} \textit{kalten} [\textsubscript{ArtP} \textit{Bier}\textsubscript{k}-$\emptyset$-\textit{es} [\textsubscript{NP} t\textsubscript{k}]]]
\z

Considering that nouns are usually not inflected for case in German, this derivation would explain the weak ending of the adjective and the presence of a suffix on the noun. However, I would like to avoid a timing mechanism involving late syntactic movement (cf. the discussion of \textit{ein} in the previous section).

Note that null articles cannot be triggers of Impoverishment in the current analysis. Like the other determiners, null articles have the categorial feature [+D] and, additionally, they have the feature [--DEF]. Unlike the other determiners, they do not have feature bundles for CNG. While the absence of CNG features avoids the issue of stranded affixes – after all, null elements cannot provide overt hosts for affixes, note that there are also no positively valued features for definiteness or deixis on the null articles that make [+D] a trigger for Impoverishment.

  I propose that Impoverishment in the current cases occurs independently of the presence of determiners. I formulate a second Impoverishment rule that has to do with the features on the terminal head Agr and its projections; that is, Impoverishment operates in a certain featural context. This rule says: When InflP is a daughter of AgrP, delete [S] on InflP in genitive masculine/neuter contexts.

\begin{figure}
	\caption{Impoverishment Rule 2}
    \label{figex:2:46}
\begin{forest}
  [AgrP
    [{InflP\\{}[--F, $\alpha$N, +O, \sout{+S}]}]
    [{Agr$'$\\{}[--F, $\alpha$N, +O, +S]}]
  ]
\end{forest}
\end{figure}

Notice that the feature specification in \figref{figex:2:46} excludes [+D]; that is, this rule applies independently of the presence of a determiner. In fact, it is the specific featural contexts and the structural relation of motherhood between AgrP and InflP that trigger the rule. Observe also that the same feature (i.e., [S]) is deleted here as in Impoverishment Rule 1.\footnote{In \chapref{sec:3}, \sectref{sec:3.4}, we see that this rule is more general applying to elements not only in AgrP but also to elements in other phrases. Note also that this rule has the looks of a dissimilation operation where one of two identical features (here [S]) is deleted (cf. \citeauthor{ArregiNevins2012}’ 2012: 213 Participant Dissimilation). As stated above, Impoverishment Rule 1 also deletes [S]. Assuming that the two Impoverishment rules are on the right track, it is currently not clear to me why [S] has this special status in German.}

To illustrate the application of this rule, note first that the null article moves to adjoin to Agr but does not trigger Impoverishment. Rather, the features on Agr percolate to AgrP and induce Impoverishment on InflP by the rule in \figref{figex:2:46}. This reduced feature set on InflP percolates to Infl as shown in \figref{figex:2:47}.

\begin{figure}
	\caption{Percolation of reduced features}
    \label{figex:2:47}
\begin{forest}
  [AgrP\textsubscript{\parbox{0mm}{\mbox{[--F, $\alpha$N, +O, +S]}}}
    [InflP\textsubscript{\parbox{0mm}{\mbox{[--F, $\alpha$N,+O, +\sout{S}]}}}
        [{[--F, $\alpha$N, +O, +\sout{S}]}]
        [AP]
    ]
    [Agr$'$\textsubscript{\parbox{0mm}{\mbox{[--F, $\alpha$N, +O, +S]}}}
        [Agr\textsubscript{\parbox{0mm}{\mbox{[--F, $\alpha$N, +O, +S]}}}, s sep=30mm
            [{Art$_k$\\{}[+D; --DEF]}]
            [Agr\textsubscript{\parbox{0mm}{\mbox{[--F, $\alpha$N, +O, +S]}}}]
        ]
        [{\dots}t$_k$\dots]
    ]
  ]
\end{forest}
\end{figure}

There are only three vocabulary items in \REF{ex:2:22} and \REF{ex:2:23} that do not involve [S]: Two have a negative specification for [O], and one is the elsewhere case in \REF{ex:2:23b}. The latter is the only vocabulary item that matches [--F, $\alpha$N, +O] in Infl yielding the desired ending -\textit{en} on the adjective. Some other remarks are in order.

Null articles are in complementary distribution with other determiners. If a null article is present, then there is no overt determiner. Consequently, there is no (additional) Impoverishment (triggered by another determiner). Conversely, if an overt determiner is present, then there is no null article. However, in this scenario, there is an overlap in the application of the two Impoverishment rules. Given that feature deletion occurs postsyntactically, if Rule 1 applies first, Rule 2 does not (as [S] has already been removed in the entire structure); if Rule 2 applies first, Rule 1 still applies to the structural elements in AgrP (without visible effect), but not to the adjective in Spec,AgrP (where the feature [S] has already been removed by Rule 2). In either scenario, only the least specified element -\textit{en} can be inserted under Infl.

Impoverishment Rule 2 is stated independently of the suffix on the noun. Two more remarks are in order here. First, there are nouns that do not take the genitive suffix -\textit{es}; for instance, \textit{des Faschismus(*-es)} ‘of fascism’ or \textit{des Jazz(*-es)} ‘of Jazz’ (see also \citealt{Olsen1991b}: 41). Furthermore, there is another type of genitive suffix on singular nouns: -\textit{en}. It is sometimes claimed (e.g., \citealt{Rehn2019}: 116) that weak adjectival inflections also occur on nouns, specifically on so-called weak masculine nouns. Note that -\textit{en} occurs in all instances on these nouns except in the nominative singular – just like with weak adjectives in the masculine singular. However, there are some important differences. While the endings on these nouns are obligatory in the genitive singular, they are optional in the accusative/dative singular, at least for some speakers (note that \citealt{Eisenberg1998}: 153 claims that this optionality holds more generally; \citealt{Gallmann1996}: 289, \citeyear[143]{Gallmann1998} points out some interesting restrictions on this optionality).\footnote{\citet{Krischke2012} notes that besides -\textit{en}, the endings -\textit{s}, -\textit{ens}, and quite rarely no ending can appear on weak masculine nouns in the genitive in non-standard contexts. He concludes that these forms are not established rival variants yet and are far from becoming part of the standard language (his page 66). Rather, they are due to analogy formations with similar forms in local contexts (\textit{lokale Analogien}, his page 72). The relation of these genitives to accusative and dative forms is only very briefly discussed.} In contrast, weak endings on adjectives are not optional at all (and there is no variation). I conclude that these two cases should not be collapsed (although they are diachronically related, \chapref{sec:1}, \sectref{sec:1.3.1.1}). In other words, adjectival endings appear on determiners and adjectives (but not on nouns). Note that the Impoverishment rule in \figref{figex:2:46} is independent of these types of points -- the head noun with its inflection is not part of this rule.

As to the second remark, note that although the genitive suffix -\textit{es} on nouns does not play a role in the current account of the strong/weak alternation, this does not mean that this suffix is synchronically not relevant. The so-called Genitive Rule states that a noun phrase in the genitive must contain at least one element that is sufficiently specific as regards case (i.e., it must contain -\textit{es} in the masculine/neuter or -\textit{er} in the feminine/plural). In conjunction with the analysis above, this explains the ungrammaticality of the following data where both weak and strong adjectives are ungrammatical with nouns that have -\textit{en} as their genitive suffix \REF[a-b]{ex:2:48} (data from \citealt[1308]{GunkelEtAl2017}). Indeed, this also accounts for the ungrammaticality of nouns that do not have a genitive ending at all \REF[c-d]{ex:2:48}.

\judgewidth{?(?)}
\ea%48
    \label{ex:2:48}
    \ea[*]{ \label{ex:2:48a}
        \gll der Geruch gebraten-en Ochse-n\\
        the smell     fried-\textsc{wk}      ox.meat.\textsc{masc}-\textsc{gen}\\
        \glt ‘the smell of roast ox’
        }
    \ex[*]{ \label{ex:2:48b}
        \gll der Geruch gebraten-es Ochse-n\\
        the smell     fried-\textsc{st}       ox.meat.\textsc{masc}-\textsc{gen}\\
        \glt ‘the smell of roast ox’
        }
    \ex[?(?)]{ \label{ex:2:48c}
        \gll eine Stunde toll-en     Jazz\\
        an    hour    great-\textsc{wk} jazz\\
        \glt ‘an hour of great jazz’
        }
    \ex[*]{ \label{ex:2:48d}
        \gll eine Stunde toll-es    Jazz\\
        an    hour    great-\textsc{st} jazz\\
        \glt ‘an hour of great jazz’
        }
    \z
\z
\judgewidth{*}

Specifically, the weak adjectives in \REF[a,c]{ex:2:48} are ungrammatical as the Genitive Rule is violated (for more discussion of the Genitive Rule, see \chapref{sec:3}, \sectref{sec:3.4}, where it is shown that this rule also applies to determiners);\footnote{Note that \REF{ex:2:48c} has a fairly good status. This presumably has to do with the fact that the second nominal in pseudo-partitive constructions can, for many speakers, be in all four morphological cases in German \citep{Löbel1989}. Note in this regard that the second nominal in \REF{ex:2:48c} is morphologically ambiguous between genitive and accusative.}  the strong adjectives in \REF[b,d]{ex:2:48} are ungrammatical as Impoverishment Rule 2 should have applied in this context.

  Before moving on, note that there are some special, rare cases where adjectives appear to have a strong inflection in genitive masculine/neuter contexts (\REF[a]{ex:2:49} is from \citealt[1308]{GunkelEtAl2017}, \REF{ex:2:49b} is from \citealt{Durrell2002}: 126).

\ea%49
    \label{ex:2:49}
\ea\label{ex:2:49a}
\gll der Inhalt   folgend-es     Paragraph-en\\
the content following-\textsc{st} paragraph.\textsc{masc}-\textsc{gen}\\
\glt ‘the content of the following paragraph’
\ex\label{ex:2:49b}
\gll das Gesuch obig-es    Adressant-en\\
the  request above-\textsc{st} sender.\textsc{masc}-\textsc{gen}\\
\glt ‘the request of the above sender’
\z
\z

It is sometimes claimed that the weak ending -\textit{en} on the noun licenses the strong ending -\textit{es} on the adjective \citep[126]{Durrell2002}. However, there are several points worth making here. \citet[1307]{GunkelEtAl2017} observe that such cases often involve adjectives that are similar to determiners (in their words “Pronomina”). Notice now that both head nouns in \REF{ex:2:49} are singular count nouns that seem to occur without articles. Interestingly, both \textit{folgend-} ‘following’ and \textit{obig-} ‘above(-mentionted)’ are definite in interpretation. As such, these elements might be special types of adjectives, sometimes called definite adjectives or adjectival determiners (see \citealt{Roehrs2009a}: 167-68, \citealt{Velde2011}; also \chapref{sec:5}, \sectref{sec:5.5.2.4}). Additionally, \citet[1308]{GunkelEtAl2017} point out that cases like \REF{ex:2:49b} often sound archaic. This might imply that such instances are not regulated by the contemporary grammar. If these remarks are on the right track, then \REF{ex:2:49} does not fall under the purview of Impoverishment Rule 2.

  To summarize the last two sections, I have discussed three (or rather four) instances where adjectives preceded by \textit{ein} are strong (the remaining instances are weak) and two instances where adjectives without (overt) determiners are weak (the remaining instances are strong). The first set of cases was explained by exempting four instances of \textit{ein} from triggering Impoverishment; the second was accounted for by formulating another Impoverishment rule triggered by a certain featural context. Furthermore, I have documented that the presence of possessives, be they determiner-like elements like \textit{m(ein)} or Saxon Genitives like \textit{Marias}, does not have an impact on the inflection of a following adjective.

Overall, I conclude that adjectival inflections are regulated by \textit{der}-words, \textit{ein}, and a certain featural context. If so, then I also have an account of why the possessives themselves cannot have an influence on the adjectival inflections – possessive elements such as \textit{m(ein)} ‘my’ and \textit{Marias} ‘Mary’s’ are not determiners themselves, but they co-occur with such elements, \textit{ein} in the case of \textit{m}- and $\emptyset$\textsubscript{D} in the case of \textit{Marias}. As stated in \chapref{sec:1}, possessives are determiner-like elements: They are like definite determiners in that they occur before adjectives and bring about definiteness; they are unlike determiners in that some of them (i.e., Saxon Genitives) may also occur after the head noun (e.g., \textit{die Werke Goethes} ‘the works of Goethe’). This explains why possessives in German do not regulate adjectival inflections (neither in prenominal nor postnominal position). In the next section, I discuss more complex canonical DPs.

\subsection{Adjectives in extended-adjective constructions and after numerals}\label{sec:2.2.4}

If Impoverishment occurs in a local domain, then the question arises how adjectival inflections that appear to be deeply embedded can undergo this type of feature deletion. Consider the extended-adjective constructions in \REF{ex:2:50}, where an adjective takes an argument. Interestingly, this argument is separated from the adjective by the degree word \textit{sehr} ‘very’. Despite the presence of these elements, the adjective alternates between exhibiting a strong or a weak ending.

\ea%50
    \label{ex:2:50}
\ea\label{ex:2:50a}
\gll {ein} [auf seinen Sohn sehr stolz-er] {Vater}\\
an     {\db}of   his       son   very proud-\textsc{st} father.\textsc{masc}\\
\glt ‘a father very proud of his son’

\ex\label{ex:2:50b}
\gll der [auf seinen Sohn sehr stolz-e] {Vater}\\
the    {\db}of  his       son   very proud-\textsc{wk} father.\textsc{masc}\\
\glt ‘the father very proud of his son’
\z
\z

As discussed in \chapref{sec:1}, \sectref{sec:1.4.1.3}, I follow, among many others, \citet{Corver1991,Corver1997} in assuming that adjectives involve an extended projection, similar to verbs and nouns (\citealt{Grimshaw1991}, \citealt{vanRiemsdijk1998b}). In particular, I assume that the extended projection of the adjective includes a Degree Phrase (DegP). Furthermore, assuming that theta-role assignment occurs in a local fashion (i.e., within AP), the intervening degree word implies that the argument must have moved to the left, stranding the adjective. This in turn implies that the extended projection of the adjective has more structure on top of DegP. Given that the adjective appears to be so deeply embedded in the structure, it is important to make explicit how the analysis above deals with the strong/weak alternation in this type of data.

I proposed in \chapref{sec:1}, \sectref{sec:1.4.1.3} that adjectives and their inflections are base-generated in separate positions (e.g., \citealt{Corver2006}: 68, \citealt{Leu2015}): While the adjective stem forms the bottom part of the extended projection, the inflection closes this structure off. Moving top-down, I reiterate the claim here that the inflection projects InflP. The head Infl involves the abstract feature bundle eventually spelled out as a strong or weak ending. Furthermore, I assume that there is a functional phrase (FP), which can form the landing site for the argument of the adjective. Finally, DegP is on top of AP. The basic structure is provided in \figref{figex:2:51}. Now, with the PP-argument moved to Spec,FP, I propose that FP moves to Spec,InflP. This is illustrated for the part enclosed in square brackets in \REF{ex:2:50b} in a simplified fashion as in \figref{figex:2:51}.

\glltree[\label{figex:2:51}]{[\textsubscript{InflP} Infl [\textsubscript{FP} F [\textsubscript{DegP} Deg [\textsubscript{AP} Adj ]]]]}{
  [InflP 
    [FP$_k$
        [{PP\\{[\textit{auf seinen Sohn}]$_i$}}]
        [F$'$
            [F]
            [DegP 
                [\textit{sehr}]
                [AP
                    [\textit{stolz}]
                    [t$_i$]
                ]
            ]
        ]
    ]
    [Infl$'$
        [\textit{-e}]
        [{\dots}t$_k${\dots}]
    ]
  ]
}

After Linearization (and Vocabulary Insertion – recall that the tree representation in \figref{figex:2:51} contains all the vocabulary items for expository convenience), we obtain the (shortened) string in \figref{figex:2:52}. The inflection undergoes Local Dislocation onto the adjective yielding \textit{stolze} ‘proud’.

\begin{figure}
	\caption{Spell-out of the adjective \emph{stolze}}
    \label{figex:2:52}
    \begin{forest}
    [,phantom
      [~
      [\dots]
      [\textit{stolz}]
      [\textit{-e}, tier=bottom]
      ]
      [$\rightarrow$  \textit{stolze}, tier=bottom]
      ]
    \end{forest}
\end{figure}

This derivation has a number of advantages. First, Impoverishment can occur in a local fashion – the inflection is in Infl and InflP itself is in Spec,AgrP just like in the cases discussed above. Second, the argument of the adjective has moved to the left. As such, the inflected adjective is, on the surface, adjacent to the head noun, a restriction that has been widely noted and is traditionally called Head-Final Filter (e.g., \citealt{Williams1982}). If this is on the right track, then I can also account for some other cases.

Based on work by \citet[673]{vanRiemsdijk1998a}, \citet[222]{Roehrs2006a} discusses some instances where the adjectival inflection is not on the adjective itself but on an element that is in the right periphery of the extended projection of the adjective. Like above, the inflection alternates.

\ea%53
    \label{ex:2:53}
\ea\label{ex:2:53a}
\gll {ein} [so schnell wie möglich-es] Aufräumen\\
  a      {\db}so quick    as   possible-\textsc{st}   cleaning.\textsc{neut}\\
\glt ‘a cleaning as quick as possible’
\ex\label{ex:2:53b}
\gll {das} [so schnell wie möglich-e] Aufräumen\\
  the    {\db}so quick    as   possible-\textsc{wk} cleaning.\textsc{neut}\\
\glt ‘the cleaning as quick as possible’
\z
\z

As in the examples above, the question arises how the deeply embedded inflection can undergo Impoverishment on current assumptions. Furthermore, it is not clear how the inflection can occur on an element that is not the head of the AP in the first place.

  I propose that the discussion above can shed some light on these issues. Consider in more detail the portion in \REF{ex:2:53b} that is delineated by square brackets. The structure of this portion is illustrated in \figref{figex:2:54} below. I assume that the adjectival head cannot move out of the comparative structure, call it CompP. Abstracting away from the internal structure of CompP, the latter moves to Spec,InflP.

\begin{figure}
	\caption{Structure involving adjectives and complex comparatives}
    \label{figex:2:54}
    \begin{forest}
      [InflP
        [{CompP$_k$\\\textit{so schnell wie möglich}}]
        [Infl$'$
            [\textit{-e}]
            [{\dots}t\textsubscript{$k$}{\dots}]
        ]
      ]
    \end{forest}
\end{figure}

After Linearization (and Vocabulary Insertion), we obtain \figref{figex:2:55}. With \textit{möglich} ‘possible’ an adjective (at least by form), this provides an appropriate overt host for the inflection, and the latter can undergo Local Dislocation onto the adjective.

\begin{figure}
	\caption{Spell-out of \emph{mögliche}}
    \label{figex:2:55}
        \begin{forest}
    [,phantom
      [~
      [\dots]
      [\textit{möglich}]
      [\textit{-e}, tier=bottom]
      ]
      [$\rightarrow$  \textit{mögliche}, tier=bottom]
      ]
    \end{forest}
\end{figure}


This, then, allows Impoverishment not only to occur in a local fashion, just like above, but also explains the unexpected position of the adjectival inflection.\footnote{Another advantage of separating the inflection from the adjective stem is that it can account for the difference in German between attributive adjectives, which have an inflection, and predicative ones, which do not.
	\ea
	\gll Das Haus ist klein.\\
	the  house is small\\
	\glt ‘The house is small.’
	\z 
    
    This can be captured by assuming that InflP is present in the former cases but not in the latter.} I return to adjectives as extended projections in the context of non-restrictive adjectives in \chapref{sec:4}, \sectref{sec:4.4}.

  Above, I discussed adjectives that appear to be deeply embedded inside the specifier of AgrP. Next I turn to adjectives that are separated from the determiner by a numeral. In other words, the determiner and the adjective do not seem to be in a local relation either. Nevertheless, the adjective shows a weak ending.

\ea%56
    \label{ex:2:56}
\gll dies-e     zehn klein-en   Autos\\
  these-\textsc{st} ten   small-\textsc{wk} cars\\
  \glt ‘these ten small cars’
\z

Assuming successive-cyclic movement of determiners from ArtP to DP, I suggested above that (phrasal) demonstratives adjoin to AgrP on their way to Spec, DP. This is now illustrated in more detail with \figref{figex:2:57} below.

\glltree[\label{figex:2:57}\small]{
	\gll diese zehn kleinen Autos\\
        these ten small cars\\
    \glt ‘these ten small cars’
% \hspace*{-1cm}
% \fittable{
}{
  [DP
    [\textit{diese}$_i$, name=d]
    [D$'$
        [D]
        [CardP
            [\textit{\sout{diese}}$_i$, name=card]
            [CardP
                [QP\\\textit{zehn}]
                [Card$'$
                    [Card]
                    [AgrP
                        [\textit{\sout{diese}}$_i$, name=agr]
                        [AgrP
                            [InflP\\\textit{kleinen}]
                            [Agr$'$
                                [Agr]
                                [ArtP
                                    [\textit{\sout{diese}}$_i$, name=art]
                                    [Art$'$
                                        [Art]
                                        [{NumP\\\textit{Autos}}]
                                    ]
                                ]
                            ]
                        ]
                    ]
                ]
            ]
        ]
    ]
  ]
\draw[->](art)  to [bend left=45] (agr);
\draw[->](agr) to [bend left=45] (card);
\draw[->](card) to [bend left=45](d);
}


This adjunction of the demonstrative to AgrP provides the local context required for the application of Impoverishment (see again \sectref{sec:2.2.1.7}). Consequently, the adjective surfaces with a weak ending.

To sum up, this section discussed adjectives in the context of canonical noun phrases. It was proposed that determiners, or more precisely their categorial features [+D], trigger Impoverishment Rule 1 such that fully specified feature bundles on terminal heads undergo feature deletion and are realized as less-specified (weak) inflections. Impoverishment rule 2 works in a similar way except that it is triggered by genitive masculine/neuter contexts in AgrP. Making certain assumptions, all Impoverishment rules apply in local constellations. Strong adjectives occur when Impoverishment does not occur (e.g., features for definiteness or deixis are absent). In other words, these inflections spell out underlyingly fully specified features and present the elsewhere case. Next, I turn to the discussion of non-canonical structures to see how adjectival inflections fare in these contexts.

\section{Adjectival inflections in non-canonical DPs}\label{sec:2.3}

In the previous section, I focused on the simple (canonical) cases of the schematic form “determiner + adjective(s) + noun”, where all these types of elements agree in case, number, and gender. In addition, I also discussed Saxon Genitive constructions, extended-adjective constructions, and adjectives after numerals. I proposed that all these cases involve regular DPs, with adjectives in Spec,AgrP and the determiner in the DP-level. These are the contexts where weak adjectives occur. In this section, I contrast those noun phrases with non-canonical constructions. The latter are argued to involve structures different from regular DPs and different from one another. Some independent evidence is provided for the individual structures (for more detailed discussion, see the original sources referenced below).

If weak adjectives occur in regular, simple DPs only, as proposed in the previous section, then there are two distinct options for analyzing DPs containing strong adjectives and relevant determiners: It could be claimed that either the determiner is in a position where it cannot have an impact on the adjective or that the adjective itself is outside the determiner’s regular domain of influence. Note that in each of these scenarios, the structural relation between the determiner and the adjective differs from that in regular DPs, and Impoverishment Rule 1 does not apply. The following discussion features both options and provides evidence for Hypothesis 2a (i.e., inflections indicate abstract structure).\footnote{There are two more options. As a third option, we could suggest that both the determiner and the adjective are in non-canonical positions. To keep things simple, I do not consider this possibility. Fourth, as seen in the following discussion, some non-canonical DPs involve no relevant determiner to begin with. Consequently, Impoverishment cannot occur in these cases (independent of the structure).} For expository clarity, the non-canonical position of either the determiner or the adjective is marked by square brackets around the relevant (complex) element in the tree diagrams. Finally, note that with the exception of indefinite pronoun constructions (\sectref{sec:2.3.2}), all cases involve definite contexts. This is clear given that many of these instances contain definite determiners. As such, weak endings on the adjectives are expected. As we will see though, this is not borne out. Again, adjectival inflections in German cannot be a reflex of the definiteness of the containing noun phrase (cf. Hypothesis 1a).

\subsection{Regular DP vs. low right-adjunction: Close appositions}\label{sec:2.3.1}

I begin with a canonical noun phrase \REF{ex:2:58a} and compare it to a noun phrase where a name-like nominal such as \textit{Großer Bär} ‘Great Bear’ occurs to the right of another nominal \REF{ex:2:58b}. The second construction is an instance of close apposition, and it typically involves one intonation unit (e.g., \citealt{LekakouSzendrői2007}, \citealt{Löbel1991}). It is clear that the first case involves a weak inflection on the adjective but that the second case must have a strong ending on the adjective.

\ea%58
    \label{ex:2:58}
\ea\label{ex:2:58a}
\gll {der} groß-e(*r)  Bär\\
the big-\textsc{wk/}*\textsc{st} bear.\textsc{masc}\\
\glt ‘the big bear’
\ex  \label{ex:2:58b}
\gll das Sternbild      Groß-e*(r)  Bär\\
the constellation big-\textsc{st}/*\textsc{wk} bear.\textsc{masc}\\
\glt ‘the constellation Great Bear’
\z
\z

To repeat, regular, simple DPs as in \REF{ex:2:58a} involve one nominal; that is, they have one head noun. This is different for close appositions as in \REF{ex:2:58b}. Here, this structure contains two nominals where the nominal on the left – often called anchor – contains a categorizing count noun, and the nominal on the right is a proper name or a uniquely referring noun phrase. Given these inflectional and syntactic differences, the cases in \REF{ex:2:58} are argued to involve different structures.

  Turning first to the canonical case in \REF{ex:2:58a}, the relevant part of the structure is repeated in simplified form in \figref{figex:2:59}.

\begin{figure}
	\caption{Regular DP-structure (Simplified)}
    \label{figex:2:59}
\begin{forest}
  [DP
    [~]
    [D$'$
        [{D\\\textit{der}$_i$},name=d]
        [AgrP
            [{InflP\\\textit{große}}]
            [Agr$'$
                [{Agr\\\textit{\sout{der}}$_i$},name=agr]
                [ArtP
                    [~]
                    [Art$'$
                        [{Art\\\textit{\sout{der}}$_i$},name=art]
                        [{NumP\\\textit{Bär}}]
                    ]
                ]
            ]
        ]
    ]
  ]
\draw[->](art)  to [bend left=45] (agr);
\draw[->](agr) to [bend left=45](d);
\end{forest}
\end{figure}

As discussed in \sectref{sec:2}, I take this to be the structural constellation where weak adjectives come about. With the adjective strong in \REF{ex:2:58b}, something else must be assumed to explain this example.

  In certain northern dialects of German, proper names can take an optional proprial article as in \textit{(die) Anna} ‘Anna’ (\citealt{NüblingEtAl2015}: 123--28). As noted by \citet{Löbel1991}, this optional article is not possible in close appositions \REF{ex:2:60a}, and the name has to follow the noun it delimits. Compare \REF{ex:2:60a} to \REF{ex:2:60b}.\footnote{As a reviewer points out, (southern) dialects with an obligatory proprial article allow this type of apposition in \REF{ex:2:60a} (although the following genitive sounds somewhat formal). Note also that \REF{ex:2:60b} improves significantly if there is a long pause after \textit{Bruder} ‘brother’ (see the discussion of loose appositions in \sectref{sec:2.3.4}).}

\ea%60
    \label{ex:2:60}
\ea\label{ex:2:60a}
\gll die Tochter (*die) Anna meines Bruders\\
the daughter   {\db\db}the  Anna of.my  brother\\
\glt ‘Anna, my brother’s daughter’

\ex[*]{\label{ex:2:60b}
\gll die Tochter  meines Bruders (die) Anna\\
the daughter of.my  brother    {\db}the  Anna\\
}
\z
\z


Relating the close appositive in \REF{ex:2:60a} to the one in \REF{ex:2:58b} above, I propose for these cases a structure different from that of canonical DPs, namely the name-like element is adjoined to the NP of the larger DP. Abstracting away from movements internal to the matrix DP, this can schematically be represented as in \figref{figex:2:61} (for a symmetric structure, see also \citealt{LekakouSzendrői2012}; \citealt{Löbel1991}: 13).

\begin{figure}
	\caption{Close appositions}
    \label{figex:2:61}
\begin{forest}
  [DP
    [{D\\\textit{das}}]
    [NP
        [NP
            [{N\\\textit{Sternbild}}]
        ]
        [{[\textit{Großer Bär}]}]
    ]
  ]
\end{forest}
\end{figure}

As to the inflection on the adjective, it is clear that the determiner did not adjoin to the phrase containing the adjective. Consequently, Impoverishment Rule 1 does not apply, and this accounts for the strong adjective.

\subsection{Regular DP vs. mid right-adjunction: Indefinite pronoun constructions}\label{sec:2.3.2}
\largerpage[-2]
\citet{Roehrs2008} argues that indefinite pronoun constructions involve several different types. With current purposes in mind, there are two types relevant here:\pagebreak{} One type exhibits a weak adjective \REF{ex:2:62a}, and the other shows a strong adjective \REF[b-c]{ex:2:62}.\footnote{Note that both \textit{wer} ‘someone’ and \textit{jemand} ‘someone’ can also be followed by an adjective ending in -\textit{es} (e.g., \textit{jemand anderes} ‘someone different’). As discussed above, -\textit{es} cannot be a regular genitive inflection on the adjective – such an ending would be -\textit{en} (for discussion and analysis of these cases, see \citealt{Roehrs2008}). Also, \textit{wer anderer} 'someone different' in \REF{ex:2:62b} sounds rather marked in isolation (and there may also be some speaker variation). However, this string gets better in context (e.g., \textit{das macht wer anderer} ‘someone different will do that’, \textit{wer anderer als du} ‘someone/who else than you’; both examples were retrieved from the internet).}

\ea%62
    \label{ex:2:62}
    \ea[]{\label{ex:2:62a}
    \gll jed-er                   ander-e(\textsuperscript{\%}r)\\
    every(one)-\textsc{masc} different-\textsc{wk}/\textit{\textsuperscript{\%}}\textsc{st}\\
    \glt ‘everyone different’
    }
    \ex[\%]{\label{ex:2:62b}
    \gll w-er                  ander-e*(r)\\
    someone-\textsc{masc} different-\textsc{st}/*\textsc{wk}\\
    \glt ‘someone different’
    }
    \ex[]{\label{ex:2:62c}
    \gll {jemand}  {ander-e*(r)}\\
    someone.\textsc{masc} different-\textsc{st}/*\textsc{wk}\\
    \glt ‘someone different’
    }
    \z
\z

It is argued in that paper that although both types in \REF{ex:2:62} involve concord between the pronominal element and the adjective, they involve different structures. As also briefly mentioned in the context of uninflected \textit{ein} (\sectref{sec:2.2.2.1}), this means that concord can only be a necessary, but not sufficient, condition on the occurrence of weak adjectives.

Turning to the individual structures, I point out first that there are other important differences between these two types. Among others, the first type allows an overt noun, but the second crucially does not.

\ea%63
    \label{ex:2:63}
    \ea[]{\label{ex:2:63a}
    \gll jeder          ander-e         Mann\\
    every(one) different-\textsc{wk} man.\textsc{masc}\\
    \glt ‘every different man’
    }
    \ex[*]{\label{ex:2:63b}
    \gll wer         ander-e(r)          Mann\\
    someone different-\textsc{st}/\textsc{wk} man.\textsc{masc}\\
    }
    \ex[*]{\label{ex:2:63c}
    \gll jemand   ander-e(r)          Mann\\
    someone different-\textsc{st}/\textsc{wk} man.\textsc{masc}\\
    }
    \z
\z

I follow the aforementioned paper in that the first case involves a regular DP-structure with the option of overtly realizing the head noun but that the second and third example involve a completely different structure. \citet{Roehrs2008} argues that the adjectives here are part of a Modifier Phrase (ModP, see \citealt{Rubin1996}). The latter phrase is proposed to be right-adjoined to an Indefinite Pronoun Restrictor Phrase (IPRP). The structure can be illustrated as in \figref{figex:2:64} (e stands for a null element; see also \citealt{Leu2005}).\footnote{Given
    the indefiniteness of the construction, one objection might be that a strong ending is expected here. Importantly though, this construction also exhibits a strong ending in the dative (for detailed discussion, see \chapref{sec:3}, \sectref{sec:3.2.2}).

        \ea
        \gll mit  \{{jemand / \textsuperscript{\%}wem}\} ander-em\\
            with {\db}someone.\textsc{masc}  different-\textsc{st}\\
            \glt ‘with someone different’
        \z
        In other words, this indefinite pronoun construction is different from the noun phrases involving \textit{ein} ‘a’ (e.g., \textit{mit} \textit{einem ander-en Mann} ‘with a different-\textsc{wk} man’). As the latter are usually used to exemplify the inflectional alternation in indefinite environments, indefinite pronoun constructions deserve being mentioned in the current context – they show that not all indefinite constructions behave the same as regards inflections on adjectives. Again, it is clear that structure must be taken into account.
}

\begin{figure}
	\caption{Indefinite pronoun constructions}
    \label{figex:2:64}
\begin{forest}
  [DP
    [D
        [{\textit{je}\\\textit{wer}},tier=bottom]
    ]
    [IPRP
        [IPRP
            [IPR
                [{\textit{-mand}\\\textit{e}$_R$},tier=bottom]
            ]
            [NP
                [{\textit{e}$_N$\\\textit{e}$_N$}]
            ]
        ]
        [ModP
            [Mod]
            [AgrP
                [{[\textit{anderer e}$_N$]\\{}[\textit{anderer e}$_N$]}]
            ]
        ]
    ]
  ]
\end{forest}
\end{figure}

Some brief remarks about the structure are in order here. Note that I put the quantifying, determiner-like element under D. In conjunction with the DP-layer, IPRP accounts for the complex internal structure of indefinite pronouns (e.g., \textit{je-mand} ‘someone’). Finally, the presence of ModP is motivated by the corresponding constructions in some of the Romance languages. For instance, French makes use of \textit{de} ‘of’ in indefinite pronoun constructions (\textit{quelque chose}\textsubscript{MASC} \textit{de grand}\textsubscript{MASC} ‘something big’), and this element seems to mediate concord between the two nominals (cf. \textit{une bonne chose}\textsubscript{FEM} \textit{de faite}\textsubscript{FEM} ‘(a good thing of done =) at least, that is done’; for details, see \citealt{Roehrs2008}: 21-23). Returning to the discussion of the strong adjectives, note that there is no relevant determiner that can trigger Impoverishment, and a strong ending appears on the adjective.

\subsection{Regular DP vs. mid right-adjunction: Noun-adjective exclamatives}\label{sec:2.3.3}

Certain exclamatives involve a noun and a following definite article and inflected adjective \REF{ex:2:65a}. As can be seen in this example, these constructions license the absence of a determiner before the noun. In fact, the definite article can also be missing before the inflected adjective \REF{ex:2:65b}. When this article is missing, the adjective is strong. Compare \REF{ex:2:65a} to \REF{ex:2:65b}.\footnote{\citet[70]{Dürscheid2002} provides the following, related examples: \textit{Mist, verdammt-er!} ‘darn-\textsc{st} rubbish’; \textit{Biest, elend-es!} ‘miserable-\textsc{st} beast’; \textit{Idiot, verdammt-er!} ‘darn-\textsc{st} idiot’; \textit{Gott, allmächtig-er} ‘God almighty-\textsc{st}’.}

\ea%65
    \label{ex:2:65}
    \ea\label{ex:2:65a}
    \gll Schwein, das schwarz-e!\\
    pig.\textsc{neut} the  black-\textsc{wk}\\
    \glt ‘Stupid bastard!’

    \ex \label{ex:2:65b}
    \gll Schwein, schwarz-e*(s)!\\
    pig.\textsc{neut} black-\textsc{st/*wk}\\
    \glt ‘Stupid bastard!’
    \z
\z

While there are other versions of this type of exclamative, I limit myself here to pointing out that a pronominal determiner can also be present before the adjective \REF{ex:2:66a}. Crucially, though, an indefinite article cannot precede the adjective \REF{ex:2:66b}.

\ea%66
    \label{ex:2:66}
    \ea\label{ex:2:66a}
    \gll Schwein, du   schwarzes!\\
    pig.\textsc{neut} you black-\textsc{st}\\
    \glt ‘Bastard, you stupid one!’
    \ex[*]{\label{ex:2:66b}
    \gll Schwein, ein schwarz-es!\\
    pig.\textsc{neut} a    black-\textsc{st}\\
    }
    \z
\z

Like the indefinite pronoun construction in the previous section, these cases also involve concord in agreement features. Unlike the indefinite pronoun construction, the inflected adjective follows a noun here. Furthermore, it is clear that the exclamatives in \REF[a-b]{ex:2:65} and \REF{ex:2:66a} have similar, but not identical, referential properties. Specifically, while the examples in \REF[a-b]{ex:2:65} single out some individual in a specific context, \REF{ex:2:66a} addresses such an individual. I hypothesize that all these cases involve definite expressions. This fits well with the common assumption that vocatives of the type \textit{dummer Idiot} ‘stupid idiot’ involve definite contexts too (\sectref{sec:2.3.9}).

Turning to the structure, we can observe that the adjective in \REF{ex:2:65b} is strong in a definite context. Importantly, considering the ungrammaticality of \REF{ex:2:66b} and the definite reference of \REF{ex:2:65b}, it seems implausible that an indefinite determiner, including a null article, is or was present in \REF{ex:2:65b}. Furthermore, given that the inflected adjective occurs in postnominal position, I suggest that \REF{ex:2:65b} consists of two nominals. The matrix nominal involves (at least) NumP, where the noun \textit{Schwein} ‘pig’ has undergone raising to Num. It is possible that there is more structure on top of NumP (cf. the structure of vocatives in \sectref{sec:2.3.9}). As to the following adjective \textit{schwarz} ‘black’, I suggest that it is part of a second nominal. Given concord in agreement features between the overt noun and the inflected adjective, I argue that the adjective is followed by a null noun, which is coreferential with the overt noun in the matrix nominal. The adjective and the null noun make up their own nominal, and I assume that the latter is adjoined to the NP of the matrix nominal. Like in the indefinite pronoun construction, the adjunction is mediated by ModP. This derives \REF{ex:2:65b} as shown in \figref{figex:2:67}.\footnote{I discuss the strong adjective in cases like \REF{ex:2:66a} in detail in \chapref{sec:3}, \sectref{sec:3.5}.}

\begin{figure}
	\caption{Noun-adjective exclamatives}
    \label{figex:2:67}
\begin{forest}
  [NumP
    [~]
    [Num$'$
        [Num
            [\textit{Schwein}$_i$]
            [Num]
        ]
        [NP
            [{NP\\t$_i$}]
            [ModP
                [Mod]
                [AgrP
                    [{[\textit{schwarzes} \textit{e}$_N$]}]
                ]
            ]
        ]
    ]
  ]
\end{forest}
\end{figure}

\largerpage
Note that there is no determiner present in these cases. Consequently, Impoverishment is not triggered, and a strong ending surfaces on the adjective. Next, I turn to a third type of adjunction.

\subsection{Regular DP vs. high right-adjunction: Loose appositions}\label{sec:2.3.4}

A typical case of a – what is sometimes called – loose apposition is given in \REF{ex:2:68a} below. Like in the two types of adjunction cases from \sectref{sec:2.3.1} and \ref{sec:2.3.2}/\ref{sec:2.3.3}, the adjective can only be strong. This is in contrast to the loose apposition in \REF{ex:2:68b}, which involves a definite determiner.

\ea%68
    \label{ex:2:68}
    \ea\label{ex:2:68a}
    \gll Wir, begeistert-e(*n)       Linguisten, fordern  mehr Unterstützung.\\
    we   enthusiastic-\textsc{st}/*\textsc{wk} linguists     demand more support\\
    \glt ‘We, enthusiastic linguists, demand more support.’
    \ex\label{ex:2:68b}
    \gll Wir, die begeistert-e*(n)       Linguisten, wollten mehr Unterstützung.\\
    we   the enthusiastic-\textsc{wk}/*\textsc{st} linguists     wanted more support\\
    \glt ‘We, the enthusiastic linguists, wanted more support.’
    \z
\z

There are certain differences as regards the two types of cases in the previous sections. First, the nominal following \textit{wir} ‘we’ in \REF{ex:2:68a} is not a name-like element (unlike the close apposition in \REF{ex:2:58b}), and second, this nominal involves an overt head noun after the adjective (unlike the indefinite pronoun construction in \REF{ex:2:62c} and unlike the exclamative in \REF{ex:2:65b}). Third, these constructions have a distinctive intonation contour, which includes pauses before and after the appositive. Fourth, a genitive complement to the first head noun can only precede this type of apposition \REF{ex:2:69a} but cannot follow it \REF{ex:2:69b}.

\ea%69
    \label{ex:2:69}
\ea \label{ex:2:69a}
\gll die Professoren unserer Uni, sehr nett-e    Leute\\
the professors    of.our    univ, very nice-\textsc{st} people\\
\glt ‘the professors of our university, very nice people’

\ex[*]{\label{ex:2:69b}
\gll    die Professoren, sehr nett-e   Leute,   unserer Uni\\
  the professors,  very nice-\textsc{st} people, of.our    univ\\
  }
\z
\z

In order to capture the difference in \REF[a-b]{ex:2:69}, I propose that this is another case of right-adjunction. Unlike the two other types of instances, I suggest that this adjunction is higher in the structure, namely to the DP-level (cf. \citealt{DelormeDougherty1972}). \REF{ex:2:68a} and \REF{ex:2:69a} are derived in simplified form as shown in \figref{figex:2:70}.

\begin{figure}
	\caption{Loose appositions}
    \label{figex:2:70}
\begin{forest}
  [DP
    [DP
        [D
            [{\textit{wir}\\\textit{die}}]
        ]
        [NP
            [{\textit{e}$_N$\\\textit{Professoren unserer Uni}}, tier=bottom]
        ]
    ]
    [AgrP
        [{[\textit{begeisterte Linguisten}]\\{}[\textit{sehr nette Leute}]}, tier=bottom]
    ]
  ]
\end{forest}
\end{figure}

As in the cases above, a determiner did not adjoin to the phrase containing the adjective. With Impoverishment not occurring, a strong adjective surfaces. Note that the pronominal determiner \textit{wir} ‘we’ in \figref{figex:2:70} is taken to have a null noun as part of its complement. In the next section, I consider cases where this type of determiner occurs with an overt noun.

\subsection{Regular DP vs. complex specifier inside DP: Dis-agreement in pronominal DPs}\label{sec:2.3.5}

\citet{Postal1966} argues that pronouns are determiners in that they can combine with adjectives and nouns (also \citealt{Pesetsky1978}; for more recent discussion, see \citealt{DéchaineWiltschko2002}: 421-22, \citealt{Roehrs2005}, and the extensive work by Höhn, e.g., \citealt{Höhn2020}). A typical case of such a pronominal DP is given in \REF{ex:2:71a}. What is interesting about this case is that the adjective can, with some differences in preference, have a weak or a strong inflection. Compare \REF{ex:2:71a} to \REF{ex:2:71b}. Furthermore, \citet{Roehrs2006b} discusses some other pronominal DPs where plural pronouns combine with singular head nouns despite the fact that these two types of elements do not agree in morphological number \REF{ex:2:71c}. Importantly though, here only a strong adjective is possible (e.g., \citealt{Bhatt1990}: 154-55, \citealt{Darski1979}: 202, \citealt{GunkelEtAl2017}: 1308).\footnote{\citet{Roehrs2006b} provides some other, related examples: \textit{ihr verdammt-es Pack} ‘[intended:] you(\textsc{pl}) darn-\textsc{st} gang’, \textit{ihr blöd-e Bande} ‘[intended:] you(\textsc{pl}) stupid-\textsc{st} gang’, \textit{ihr dumm-es Gesindel} ‘[intended:] you(\textsc{pl}) dumb-\textsc{st} riff-raff’ (see also Footnote \ref{foot:2:47}).}

\judgewidth{\%}
\ea%71
    \label{ex:2:71}
    \ea[]{\label{ex:2:71a}
    \gll ihr        dumm-en   Idioten\\
    you(\textsc{pl}) stupid-\textsc{wk} idiots\\
    \glt ‘you stupid idiots’
    }

    \ex[\%]{\label{ex:2:71b}
    \gll ihr        dumm-e    Idioten\\
    you(\textsc{pl}) stupid-\textsc{st} idiots\\
    \glt ‘you stupid idiots’
    }

    \ex[]{\label{ex:2:71c}
    \gll ihr         jung-e*(s)               Gemüse\\
    you(\textsc{pl}) young-\textsc{st.sg/*wk} vegetable.\textsc{neut}\\
    \glt ‘you young folks’
    }
    \z
\z

\judgewidth{*}

Unlike the loose appositives in the previous section, none of the cases above involve comma intonation. In addition, pronominal DPs and loose appositives involve a difference in interpretation. While it is not impossible that the adjective in the pronominal DP may receive a restrictive interpretation (e.g., \textit{nur} \textit{ihr Großen} ‘only you tall ones’), this is not true of appositional structures. In the latter case, the adjective and noun simply provide additional information, which does not delimit the set denoted by \textit{wir} ‘we’ (e.g., \textit{ihr, begeisterte Studenten} ‘you, enthusiastic students’). Before moving on, note also that the unmarked example in \REF{ex:2:71a} runs counter to the traditional generalizations Weak After Strong and the Principle of Monoinflection.

It is pointed out in \citet{Roehrs2006b} that semantic agreement in number must hold between all the elements in \REF[a-c]{ex:2:71}. For instance, like the pronominal determiner \textit{ihr} ‘you(\textsc{pl})’, both the count noun \textit{Idioten} ‘idiots’ and the mass noun \textit{Gemüse} ‘(vegetable =) folks’ imply several individuals. This is similar to regular determiners such as \textit{diese} ‘these’ – they also require semantic agreement with the head noun. However, unlike pronominal determiners, regular determiners are more restrictive: They do not tolerate morphological dis-agreement at all (e.g., \textit{ihr blödes Pack} ‘[intended:] you(\textsc{pl}) stupid gang’ vs. *\textit{diese Pack} ‘[intended:] these gang’).

Furthermore, without a special intonation contour, agreeing nominals (e.g., \textit{dumme(n) Idioten} ‘stupid idiots’ as in \REF[a-b]{ex:2:71}) or dis-agreeing nominals (e.g., \textit{junges Gemüse} ‘young [vegetable =] folks’ as in \REF{ex:2:71c}) cannot be iterated in pronominal DPs, and both of these nominals cannot interact with one another. In other words, there can be only one overt noun and related adjective in the pronominal DP.

\ea%72
    \label{ex:2:72}
\ea\label{ex:2:72a}
    \gll ihr         Idioten\\
    you(\textsc{pl}) idiots\\
    \glt ‘you idiots’
\ex[*]{\label{ex:2:72b}
    \gll  ihr         Idioten jung-es      (Gemüse)\\
    you(\textsc{pl}) idiots   young-\textsc{st.sg} {\db}vegetable.\textsc{neut}\\
    }
\ex[*]{\label{ex:2:72c}
    \gll  ihr         jung-es         (Gemüse) Idioten\\
    you(\textsc{pl}) young-\textsc{st.sg} {\db}vegetable.\textsc{neut} idiots\\
    }
\z
\z

\ea%73
    \label{ex:2:73}
    \ea\label{ex:2:73a}
    \gll ihr        Gemüse\\
    you(\textsc{pl}) vegetable.\textsc{neut}\\
    \glt ‘you (young) folks’
    \ex[*]{\label{ex:2:73b}
        \gll ihr        Gemüse             dumm-en      (Idioten)\\
            you(\textsc{pl}) vegetable.\textsc{neut} stupid-\textsc{wk.pl} {\db}idiots\\
    }
    \ex[*]{\label{ex:2:73c}
    \gll ihr         dumm-en      (Idioten) Gemüse\\
    you(\textsc{pl}) stupid-\textsc{wk.pl} {\db}idiots     vegetable.\textsc{neut}\\
    }
    \z
\z

In order to capture these facts, \citet{Roehrs2006b} proposes that the pronominal determiner can optionally select either an AgrP resulting in the (regular) agreeing cases or a – what he calls for lack of a better name – Dis-agreement Phrase (DisP) bringing about the dis-agreeing cases.\footnote{With adjunction less constrained (e.g., \citealt{Pysz2006}: 66-67), above-mentioned restrictions are best stated in terms of selection.}

I assume that prononomal DPs with an agreeing adjective and noun as in \REF[a-b]{ex:2:71} involve a regular, canonical DP where the pronominal determiner surfaces in the DP-layer, the adjective is in Spec,AgrP, and the noun is in NP. This derives \REF{ex:2:71a} as \figref{figex:2:74} (recall that the AP in Spec,AgrP involves InflP at the top).

\begin{figure}
	\caption{Transitive pronouns (Regular DP -- simplified)}
    \label{figex:2:74}
\begin{forest}
  [DP
    [~]
    [D$'$
        [{D\\\textit{ihr}$_i$},name=d]
        [AgrP
            [{InflP\\\textit{dummen}}]
            [Agr$'$
                [{Agr\\\sout{ihr}$_i$},name=agr]
                [ArtP
                    [~]
                    [Art$'$
                        [{Art\\\textit{\sout{ihr}}$_i$},name=art]
                        [{NumP\\\textit{Idioten}}]
                    ]
                ]
            ]
        ]
    ]
  ]
\draw[->](art)  to [bend left=45] (agr);
\draw[->](agr) to [bend left=45] (d);
\end{forest}
\end{figure}

In \chapref{sec:3}, \sectref{sec:3.5}, I discuss the strong/weak alternation in pronominal DPs, including the strong inflection on the adjective in \REF{ex:2:71b}, in more detail.

\largerpage[-1]
As for cases like \REF{ex:2:71c}, pronominal DPs with a morphologically dis-agreeing nominal have a different structure. They involve non-canonical DPs where the entire dis-agreeing element is assumed to be in Spec,DisP, and the matrix nominal involves a null noun. Compared to \figref{figex:2:74}, the construction in \figref{figex:2:75} features an intransitive pronoun.\footnote{\label{foot:2:47} Note that both pronouns in \figref{figex:2:74} and \figref{figex:2:75} take complements, overt material in \figref{figex:2:74} but a null noun in \figref{figex:2:75}. As such, I use the terms transitive and intransitive as regards the overtness and covertness of the following material. Also, the structure in \figref{figex:2:75} has some wider application. In \chapref{sec:6}, it is pointed out that a (number-neutral) bare noun like \textit{Bauer} ‘peasant’ must be singular in interpretation in dis-agreement cases such as \textit{Sie Bauer} ‘you peasant’ (where \textit{Sie} ‘you’ is morphological plural and could, at least in principle, be plural in interpretation). I argue in more detail there that the nominal \textit{Bauer} is in Spec,DisP.}

\begin{figure}
	\caption{Intransitive pronouns (Dis-agreement)}
    \label{figex:2:75}
\begin{forest}
  [DP
    [~]
    [D$'$
        [{D\\\textit{ihr}$_i$},name=d]
        [DisP
            [{[\textsubscript{AgrP}\textit{junges Gemüse}]}]
            [Dis$'$
                [{Dis\\\textit{\sout{ihr}}$_i$},name=dis]
                [ArtP
                    [~]
                    [Art$'$
                        [{Art\\\textit{\sout{ihr}}$_i$},name=art]
                        [{NumP\\\textit{e}$_N$}]
                    ]
                ]
            ]
        ]
    ]
  ]
\draw[->](art)  to [bend left=45] (dis);
\draw[->](dis) to [bend left=45] (d);
\end{forest}
\end{figure}

Note now that there is a clear difference between the agreeing DP in \REF{ex:2:71a} and the dis-agreeing case in \REF{ex:2:71c}. In the former, the plural adjective and its projected InflP reside in Spec,AgrP, see again \figref{figex:2:74}; in the latter, the singular adjective also projects InflP, but there is more linguistic material that relates to this adjective – the singular noun. Since both agree in concord features, they make up their own nominal, presumably AgrP. Given the lack of morphological agreement with the plural pronoun, this AgrP has been located in Spec,DisP as in \figref{figex:2:75}. To be clear, unlike in \figref{figex:2:74}, the adjective in \figref{figex:2:75} is more deeply embedded (cf. InflP vs. AgrP, which contains InflP). To repeat, the adjective in the latter establishes its own agreement relation with the overt noun inside the embedded AgrP and is morphologically independent of the larger DP (for the discussion of lack of concord here, see \chapref{sec:8}, \sectref{sec:8.3.1}). Given this complex specifier, it is clear that Impoverishment cannot occur resulting in a strong adjective. More generally, this section has shown that adjectival inflections provide clues about various degrees of embeddings of adjectives (Hypothesis 2a).

\subsection{Regular DP vs. separate base-generation: Split topicalizations}\label{sec:2.3.6}

\citet{Fanselow1988} and \citet{vanRiemsdijk1989} discuss discontinuous noun phrases, often referred to as split topicalizations (also \citealt{Ott2011a}). Putting it in simple terms, the lower part of a noun phrase occurs in a higher position of the clause, and the higher part of the same noun phrase is in a lower position. Compare the continuous nominal in \REF{ex:2:76a} to its discontinuous counterpart in \REF{ex:2:76b}.

\ea%76
    \label{ex:2:76}
\ea\label{ex:2:76a}
\gll Er hat zehn Hemden getragen. \\
he has ten   shirts     worn\\
\glt ‘He wore ten shirts.’
\ex\label{ex:2:76b}
\gll Hemden hat er zehn getragen. \\
shirts     has he ten   worn\\
\glt ‘As for shirts, he wore ten.’
\z
\z

This type of construction has a number of interesting properties. As pointed out by \citet[249-50]{Bhatt1990} and \citet{FanselowĆavar2002}, some speakers allow the split of a definite noun phrase. Importantly, while an adjective in a non-split noun phrase has a weak ending \REF{ex:2:77a}, this adjective has a strong inflection when it is topicalized stranding the determiner \REF{ex:2:77b}.

\ea%77
    \label{ex:2:77}
\ea\label{ex:2:77a}
\gll Ich habe immer nur  diese bunt-e*(n)          Hemden da    getragen. \\
I    have always only these colored-\textsc{wk}/*\textsc{st} shirts     there worn\\
\glt ‘I have always worn only these multi-colored shirts there.’
\ex\label{ex:2:77b}
\gll Bunt-e(*n)         Hemden habe ich immer nur   diese da     getragen. \\
colored-\textsc{st}/*\textsc{wk} shirts      have I    always only these there worn\\
\glt ‘As for multi-colored shirts, I have always worn only these there.’
\z
\z

This split results in two nominals. \citet{Fanselow1988} proposes that these two nominals are base-generated separately in the VP and that one of them undergoes movement. This can be illustrated for \REF{ex:2:77b} in \figref{figex:2:78}. For concreteness, I put adverbial elements as adjuncts to a Perfective Phrase (PerfP), where the auxiliary \textit{haben} ‘to have’ originates (for more detailed discussion of split topicalization, see \chapref{sec:4}, \sectref{sec:4.3}).

\begin{figure}
	\caption{Split topicalizations}
    \label{figex:2:78}
\hspace*{-1cm}
\fittable{
\begin{forest}
  [CP
    [{AgrP$_k$\\{[\textit{bunte Hemden}]}}]
    [C$'$
        [{C\\\textit{habe}$_i$}]
        [TP
            [\textit{ich}]
            [T$'$
                [PerfP
                    [\textit{immer}]
                    [PerfP
                        [\textit{nur}]
                        [PerfP
                            [VP
                                [t$_k$]
                                [V$'$
                                    [{DP\\\textit{diese e\textsubscript{N} da}}]
                                    [{V\\\textit{getragen}}]
                                ]
                            ]
                            [{Perf\\t$_i$}]
                        ]
                    ]
                ]
                [{T\\t$_i$}]
            ]
        ]
    ]
  ]
\end{forest}
}
\end{figure}
\largerpage[-1]
Some independent evidence for this analysis comes from the fact that the discontinuous noun phrase can have two mismatching determiners, for instance, \textit{’n} ‘a’ and \textit{dieses} ‘this’ below (recall that \textit{’n} is the reduced form of the indefinite article \textit{ein}).

\ea%79
    \label{ex:2:79}
  \gll    ’N Hemd        habe ich immer  nur  dieses getragen.\\
 {\db}a  shirt.\textsc{neut} have I     always only this     worn\\
\glt ‘As for a shirt, I only wore this one.’
\z

With the two nominals assembled separately from one another, it should be clear that the demonstrative in \REF{ex:2:77b} did not adjoin to the AgrP that contains the adjective. Impoverishment does not occur yielding a strong ending on the adjective.

\subsection{Regular DP vs. complex compound modifier: Nominal compounds}\label{sec:2.3.7}

As illustrated before, adjectives in canonical DPs undergo the strong/weak alternation \REF{ex:2:80a}. This is also possible with phrasal proper names like \textit{die Deutsche Bank} ‘the German Bank’ \REF{ex:2:80b}. Interestingly, these proper names can form complex modifiers of nominal compounds (\citealt{Lieber1988}, \citealt{Wiese1996b}). Crucially, here the adjectives have to appear with strong endings \REF{ex:2:80c}.

\ea%80
    \label{ex:2:80}
\ea\label{ex:2:80a}
\gll mit   dem deutsch-en   Chef\\
with the   German-\textsc{wk} boss.\textsc{masc}\\
\glt ‘with the German boss’
\ex\label{ex:2:80b}
\gll mit   der Deutsch-en  Bank\\
with the German-\textsc{wk} Bank.\textsc{fem}\\
\glt ‘with the German Bank’
\ex\label{ex:2:80c}
\gll mit   dem [Deutsch-e(*n)    Bank] Chef\\
with the    {\db}German-\textsc{st/*wk} Bank  boss.\textsc{masc}\\
\glt ‘with the German Bank-boss’
\z
\z

The example in \REF{ex:2:80c} involves a complex nominal, and it can be represented as in \figref{figex:2:81}, where the complex modifier is adjoined to the left of the head noun.

\begin{figure}
	\caption{Nominal compounds}
    \label{figex:2:81}
\begin{forest}
  [DP
    [{D\\\textit{dem}}]
    [NP
        [N
            [{[\textit{Deutsche Bank}]}]
            [{N\\\textit{Chef}}]
        ]
    ]
  ]
\end{forest}
\end{figure}

As in the cases above, the determiner did not adjoin to the structure containing the adjective. With Impoverishment not occurring, a strong adjective appears.

So far, I have discussed a number of cases, where the adjective is outside the determiner’s regular domain of influence. Next, I turn to a case where the determiner itself is in a position where it cannot bring about a weak adjective.

\subsection{Regular DP vs. outside of DP proper: Predeterminers}\label{sec:2.3.8}

\citet[Chapter 4]{Roehrs2009a} discusses some cases where determiners and determi-\linebreak ner-like elements may co-occur (also \citealt{Wood2007}; \chapref{sec:1}, \sectref{sec:1.4.1.2}). To set the stage, consider first \REF{ex:2:82a}, where a demonstrative occurs with a weak adjective. In contrast, a possessive article occurs with a strong adjective \REF{ex:2:82b}. Now, the demonstrative and the possessive article can also be combined as in \REF{ex:2:82c}. Here, the adjective must be strong (\citealt{Duden1995}: 286, \citealt{GunkelEtAl2017}: 1308).

\ea%82
    \label{ex:2:82}
    \ea\label{ex:2:82a}
    \gll dieses groß-e(*s)     Glück\\
    this     great-\textsc{wk}/*\textsc{st} happiness.\textsc{neut}\\
    \glt ‘this great happiness’
    \ex\label{ex:2:82b}
    \gll mein groß-e*(s)      Glück\\
    my    great-\textsc{st}/*\textsc{wk} happiness.\textsc{neut}\\
    \glt ‘my great happiness’
    \ex\label{ex:2:82c}
    \gll dieses mein groß-e*(s)      Glück\\
    this     my    great-\textsc{st}/*\textsc{wk} happiness.\textsc{neut}\\
    \glt ‘this my great happiness’
    \z
\z

Note that, morphologically, the presence of the demonstrative in \REF{ex:2:82c} does not play a role for the inflection on the adjective in this example. This is surprising given that it does have an impact in \REF{ex:2:82a}. Before commenting on the syntax and semantics, note also that the strong adjective in \REF{ex:2:82c} is preceded by the strongly inflected determiner element \textit{dieses} ‘this’ in violation of the traditional generalizations about adjectival inflections.

  Syntactically, it is well known that the demonstrative can only occur to the left of the possessive (cf. *\textit{mein dieses} \textit{große(s) Glück} ‘[intended:] my this great happiness’) -- it is a predeterminer in such combinations. Semantically, this element seems to function as an intensifier for deixis \citep{Wood2007}. In order to account for this different behavior in the morphology, syntax, and semantics, it is proposed in \citet{Roehrs2009a} that the demonstrative is merged in a phrase above the DP-level. In \figref{figex:2:83}, I label this projection Left Periphery Phrase (LPP) (see \citealt{GiustiIovino2016}).

\begin{figure}
	\caption{Predeterminers}
    \label{figex:2:83}
\begin{forest}
  [LPP
    [{[\textit{dieses}]}]
    [DP
        [\textit{mein}]
        [AgrP
            [\textit{großes}]
            [{ArtP\\\textit{Glück}}]
        ]
    ]
  ]
\end{forest}
\end{figure}

As the intensifying demonstrative is base-generated above the DP-level, it did not adjoin to the phrase containing the adjective. Consequently, Impoverishment does not occur, and as a result, the adjective exhibits a strong ending.

\subsection{Regular DP vs. no determiner: Vocatives}\label{sec:2.3.9}

Vocatives are structures that occur in non-argument position. As illustrated below, these cases exhibit strong adjectives as well.

\ea%84
    \label{ex:2:84}
    \ea\label{ex:2:84a}
    \gll Blöd-e*(s)       Schwein!\\
    stupid-\textsc{st/*wk} pig.\textsc{neut}\\
    \glt ‘Stupid idiot!’
    \ex\label{ex:2:84b}
    \gll Dumm-e*(r)    Idiot!\\
    stupid-\textsc{st/*wk} idiot.\textsc{masc}\\
    \glt ‘Stupid idiot!’
    \z
\z

It can be pointed out that vocatives involve the presupposition that the individual addressed is present in the situational context. As such, vocatives are usually taken to be definite constructions (see also the exclamatives in \sectref{sec:2.3.3}). Following other work, \citet[86]{Julien2016} proposes that this definiteness follows from the presence of a second-person feature on a Vocative head. This Voc head selects a DP with an empty D head (except for an agreeing second-person feature). Given the presence of VocP, this means that vocatives as in \REF{ex:2:84} involve non-canonical structures, see \figref{figex:2:85}.

\begin{figure}
	\caption{Vocatives}
    \label{figex:2:85}
\begin{forest}
  [VocP
    [Voc]
    [DP
        [D]
        [AgrP
            [\textit{blödes}]
            [{NP\\\textit{Schwein}}]
        ]
    ]
  ]
\end{forest}
\end{figure}

As a determiner is absent, this accounts for the strong adjectives. Finally, note that these data confirm once again that strong adjectives are not tied to indefiniteness.

  To sum up the entire section, I discussed one context in \sectref{sec:2} where weak adjectives occur (regular, simple DP); in this section, I examined nine contexts where strong adjectives surface: The adjective may be part of different types of adjunction, it may be deeply embedded inside a specifier, the determiner and adjective are in their typical positions but both occur separately in a discontinuous noun phrase, the adjective may be part of a complex nominal compound, the determiner may be outside of the DP proper, and a determiner may be absent altogether. In all relevant cases, the adjective or the determiner is in a position different from that of a regular (continuous) DP. Some independent evidence was provided for the assumption of different structures. What all these nine types of constructions have in common is that Impoverishment Rule 1 cannot apply as a determiner does not adjoin to the phrase (immediately) containing the adjective. As a consequence, the inflectional feature bundle of the adjective does not undergo Impoverishment, remains unaltered, and is spelled out as a strong inflection.

  Thus far, I have focused on the discussion of the inflections on adjectives. As we have seen though, determiners also have strong inflections, and the same goes for (non-possessive) determiner-like elements – intensifying predeterminers like \textit{diese} ‘this’ and \textit{alle} ‘all’. In the next section, I turn to the explanation of the strong inflections on these elements.

\section{Adjectival inflections on determiners and predeterminers}\label{sec:2.4}

First, consider \REF{ex:2:86a} and \REF{ex:2:86b}, where both \textit{diese} ‘these’ and \textit{alle} ‘all’ are followed by a weak adjective.

\ea%86
    \label{ex:2:86}
    \ea\label{ex:2:86a}
    \gll dies-e     klein-en   Autos\\
    these-\textsc{st} small-\textsc{wk} cars\\
    \glt ‘these small cars’

    \ex\label{ex:2:86b}
    \gll all-e   klein-en   Autos\\
    all-\textsc{st} small-\textsc{wk} cars\\
    \glt ‘all small cars’
    \z
\z

I proposed in \sectref{sec:2} that determiners move to the DP-level triggering Impoverishment on adjectives. As to the determiners themselves, recall that the strong endings on the determiners are identical to those on the adjectives, and as such they should receive a similar account. Now, if we make the plausible assumption that similar to adjectives, the inflections on determiners can undergo Impoverishment (for such cases involving Impoverishment Rule 2, see \chapref{sec:3}, \sectref{sec:3.4}), then the latter mechanism cannot have occurred on either \textit{diese} ‘these’ or \textit{alle} ‘all’ in \REF{ex:2:86}. I propose that Impoverishment does not occur with the inflections on determiners as the determiners themselves are not adjoined to by a(nother) determiner – after all, there is only one determiner in each of the nominal structures in \REF{ex:2:86}. As the feature bundles for case, number, and gender are not reduced, this yields strong endings on the determiners (for the significance of this discussion in the context of the Scandinavian languages, see \citealt{KatzirSiloni2014}: 278).

  In \sectref{sec:2.3.8}, we saw that determiners can co-occur with intensifying predeterminers \REF{ex:2:87a}. Importantly, \textit{dieses} ‘this’ itself has a strong ending, but the possessive article is uninflected (for independent reasons, see again \sectref{sec:2.2.2.2}). However, there are also distributions where both determiner(-like) elements exhibit a strong ending at the same time \REF{ex:2:87b}. Comparing \REF{ex:2:87a} to \REF{ex:2:87b}, note also that the adjective is strong in the former but weak in the latter case.

\ea%87
    \label{ex:2:87}
    \ea\label{ex:2:87a}
    \gll dies-es mein groß-es  Glück\\
    this-\textsc{st} my    great-\textsc{st} happiness.\textsc{neut}\\
    \glt ‘this my great happiness’
    \ex\label{ex:2:87b}
    \gll all-e   dies-e(*n)      klein-en   Autos\\
    all-\textsc{st} these-\textsc{st/*wk} small-\textsc{wk} cars\\
    \glt ‘all these small cars’
    \z
\z

Importantly, the inflections on the predeterminers are not frozen. As can be seen in instances involving the dative case, they change just like those on the determiners themselves.

\ea%88
    \label{ex:2:88}
    \ea\label{ex:2:88a}
    \gll in dies-em mein-em groß-en  Glück\\
    in this-\textsc{st}  my-\textsc{st}     great-\textsc{wk} happiness.\textsc{neut}\\
    \glt ‘in this my great happiness’
    \ex\label{ex:2:88b}
    \gll mit all-en dies-en  klein-en   Autos\\
    with all-\textsc{st} these-\textsc{st} small-\textsc{wk} cars\\
    \glt ‘with all these small cars’
    \z
\z

Given this alternation, it is rather implausible to suggest that predeterminers and determiners form compounds. Note in this regard that \citet[217-18]{Bhatt1990} claims that \textit{alle} and \textit{diese} are heads that form a complex adjunction structure (see also \chapref{sec:3}, \sectref{sec:3.6}). In my view, a different analysis is called for. The structural account of the strong/weak alternation offered in this chapter extends to these cases without a problem.

Recall that I proposed in \sectref{sec:2.3.8} that \textit{dieses} in \REF{ex:2:87a} is a predeterminer and is taken to be base-generated in LPP; that is, it is higher up in the structure. This means that this element was neither adjoined to by another determiner, nor did it adjoin to another element – again, it originated in LPP. Consequently, it neither underwent nor triggered Impoverishment. I assume that \textit{alle} in \REF{ex:2:87b} is also base-generated in LPP. With Impoverishment not occurring, this yields strong inflections on the predeterminers and on the determiners. Briefly returning to the adjectives in \REF{ex:2:87}, \textit{mein} ‘my’ in \REF{ex:2:87a} does not trigger Impoverishment on the adjective as discussed in \sectref{sec:2.2.2.1}. This results in a strong ending. As to \REF{ex:2:87b}, \textit{diese} ‘these’ does trigger Impoverishment bringing about a weak ending on the adjective.

It is worth pointing out that these are not isolated cases. An example similar to \REF{ex:2:87a} can be provided in the plural \REF{ex:2:89a}, and both \textit{alle} and \textit{diese} can precede a possessive article in the same nominal \REF{ex:2:89b} (\citealt{Bhatt1990}: 217-18, \citealt{Vater1991}: 28-29). Again, in each case, the determiner and predeterminer(s) have a strong ending.

\ea%89
    \label{ex:2:89}
    \ea\label{ex:2:89a}
    \gll dies-e     mein-e(*n)  nett-en   Freunde\\
    these-\textsc{st} my-\textsc{st/*wk} nice-\textsc{wk} friends\\
    \glt ‘these my nice friends.’
    \ex\label{ex:2:89b}
    \gll all-e   dies-e(*n) mein-e(*n)  Freunde\\
    all-\textsc{st} these-\textsc{st/*wk} my-\textsc{st/*wk} friends\\
    \glt ‘all these friends of mine’
    \z
\z

I suggest that \textit{diese} in \REF{ex:2:89a} is also base-generated in LPP and that there is a second LPP in \REF{ex:2:89b} accommodating \textit{alle}. Note again that none of the determiners or predeterminers were adjoined to by another determiner. This explains why these elements do not have weak endings. Again, this is in line with the assumption that one determiner originates in ArtP and moves to the DP-level but that the second (or third) element that has the appearance of a determiner is in effect a predeterminer, a determiner-like element base-generated higher in the structure (LPP). As stated in \chapref{sec:1}, \sectref{sec:1.4.1.2}, determiner(-like) elements can be merged in the LPP provided they can function as intensifiers and are semantically compatible with the following determiner in the DP. More generally, this means that a sequence of two or three determiner(-like) elements with strong endings indicates more structure on top of the DP proper; that is, adjectival inflections are indicators of abstract structure (Hypothesis 2a).

To sum up the current proposal, I provided diachronic and synchronic evidence that adjectival inflections in (Modern) German cannot be a reflex of (in-)\linebreak definiteness. I proposed that these inflections are semantically vacuous (Hypothesis 1a). Furthermore, having shown that concord is only a necessary (but not sufficient) condition for the appearance of weak adjectives, I pointed out that these adjectives are only possible in regular, simple DPs, where the determiners adjoin to the phrases containing the adjectives. By contrast, strong inflections occur in very diverse contexts and do not seem to be subject to any special conditions. As such, they involve the elsewhere case. Indeed, we have seen distributions where the highest element or elements exhibit a strong inflection at the top of the structure although adjectives with weak inflections surface in the lower part of the structure. I argued that Impoverishment occurs locally and in a bottom-up fashion – it affects all adjectives but none of the determiners or predeterminers.

  More generally, if this is correct, then the strong/weak alternation of adjectival inflection appears to be a reflex of different structures. In fact, the different types of inflections may indicate the degree of embedding of adjectives (Spec,AgrP vs. Spec,DisP containing Spec,AgrP) and a certain amount of structure (DP vs. LPP on top of DP). The strong adjectives in the latter two types of contexts (Spec,DisP and LPP) provide evidence for Hypothesis 2a – more abstract structure is involved. Another result of the discussion is that the traditional terms weak and strong inflections have no significance other than different degrees of feature specificity. The weak endings form a subset of the strong endings in the sense of having the fewest features in their specifications. Finally, to highlight some of the points of the previous sections and to compare the current analysis to other work, I discuss three earlier proposals.

\section{A brief critique of some previous proposals}\label{sec:2.5}

\citet{Abney1987} stimulated much important research on the noun phrase. Interestingly, many of the (early) discussions of German have concentrated on the structure of the DP (e.g. \citealt{Felix1990}; \citealt{Haider1988}; \citealt{Löbel1990a,Löbel1990b}; \citealt{Vater1991}) and relatively fewer contributions have been devoted to the explanation of morpho-syntactic phenomena such as the distribution of adjectival inflections. It is perhaps telling that three monographs on the DP, \citet{Bhatt1990}, \citet{Demske2001}, and \citet{Wegener1995}, discuss only three different types of analyses at length (they all discuss \citealt{Olsen1991b}). In this section, I review three proposals in more detail. I start off by examining Olsen’s seminal work (\citeyear{Olsen1989a,Olsen1989b,Olsen1991a,Olsen1991b}), and then I turn to two more recent accounts: \citet{Schoorlemmer2009} and \citet{Murphy2018}.\footnote{For the discussion of other proposals, see some earlier remarks in this chapter but also \citet{KatzirSiloni2014} and \citet[175-91]{Roehrs2006a}.} Note that all these analyses discuss only the canonical cases but not the non-canonical structures, and it is often not clear how the latter can be accommodated in these accounts. As laid out above, I argue, however, that it is the non-canonical structures that reveal the true nature of adjectival inflections in German.

\subsection{\citet{Olsen1989a,Olsen1989b,Olsen1991a,Olsen1991b}}\label{sec:2.5.1}

In a series of papers (1989a, 1989b, 1991a, 1991b), Olsen discusses inflection within the German DP (I refer only to her last paper as that contains the main relevant insights). Assuming \citegen{Abney1987} DP-Hypothesis, Olsen proposes that phi-features are located under D. These features involve person, case, number, and gender. Abbreviated as AGR, they need to be made visible. Olsen assumes that agreement within the DP is brought about by D selecting NP as its complement and by Percolation of superscripts from NP down the tree. Here are the relevant definitions (my translations).\footnote{These are the original definitions (\citealt{Olsen1991b}: 40, 38).
	\ea
	\ea Prinzip der morphologischen Realisierung\\Grammatische Merkmale werden phonologisch sichtbar gemacht.
	\ex Kongruenzkette\\Eine Kongruenzkette besteht aus einer ununterbrochenen Folge identischer Indizes, die auf der Basis der funktionalen Selektion entsteht, die zwischen einer AGR-Kategorie und ihrem Komplement erfolgt.
	\z
	\z 
	}

\ea%90
    \label{ex:2:90}
    \ea\label{ex:2:90a}
    Principle of Morphological Realization\\
    Grammatical features are rendered phonologically visible.
    \ex\label{ex:2:90b}
    Agreement Chain\\
    An agreement chain consists of an uninterrupted sequence of identical
    indices which are brought about by functional selection, which holds
    between an AGR-category and its complement.
\z
\z

As an illustration, see the structure in \figref{figex:2:91}.

\glltree[\label{figex:2:91}]{
	\gll da-s    kalt-e     Wetter\\
	the\textsc{-st} cold-\textsc{wk} weather.\textsc{neut}\\
	\glt ‘the cold weather’
}{
	[DP
		[D\textsuperscript{i}\\\textit{das}]
		[NP\textsuperscript{i}
			[AP\textsuperscript{i} [A\textsuperscript{i}\\\textit{kalte}]]
			[N\textsuperscript{i}\\\textit{Wetter}]
		]
	]
}

AGR is made visible under D by the definite determiner \textit{das} ‘the’, which has a strong inflection. An identical superscript is on NP (by functional selection) and on N and A (by Percolation). To ensure unique realization of grammatical features, Olsen follows \citeapo[615]{Emonds1987} Invisible Category Principle.

\ea%92
    \label{ex:2:92}
Invisible Category Principle\\
A closed category B with positively specified features C\textsubscript{i} may remain empty
throughout a syntactic derivation if the features C\textsubscript{i} {\dots} are all alternatively realized
in a phrasal sister of B.
\z

To briefly illustrate with a construction in English, the Invisible Category Principle is taken to restrict the realization of the comparative to just one overt maker.

\ea%93
    \label{ex:2:93}
\ea\label{ex:2:93a} [\textsubscript{DEG} \textit{more}] bright$\emptyset$
\ex\label{ex:2:93b} [\textsubscript{DEG} $\emptyset$] bright-er
\z
\z

As regards adjectival inflections, this principle allows the realization of AGR on a sister node, the adjective in \REF{ex:2:94b}, and it rules out two strong endings in cases like \REF{ex:2:94c}.

\ea%94
    \label{ex:2:94}
\ea\label{ex:2:94a}
\gll da-s    kalt-e$\emptyset$ Wetter\\
the-\textsc{st} cold-\textsc{wk} weather.\textsc{neut}\\
\glt ‘the cold weather’
\ex\label{ex:2:94b}
\gll [\textsubscript{D} $\emptyset$] kalt-es  Wetter\\
     ~                  ~     cold-\textsc{st} weather.\textsc{neut}\\
\glt ‘cold weather’
\ex[*]{\label{ex:2:94c}
\gll da-s    kalt-es  Wetter\\
the-\textsc{st} cold-\textsc{st} weather.\textsc{neut}\\
\glt ‘the cold weather’
}
\z
\z

The weak ending in \REF{ex:2:94a} is assumed to be the “unmarked” inflection of the adjective \citep[44]{Olsen1991b}. Crucially, the strong ending can only be realized on the sister node, if D is empty \REF{ex:2:94b}; that is, the strong inflection can appear on the adjective (but not the noun, which cannot realize AGR). In order to rule out cases such as *\textit{ein-es kalte Wetter}, where \textit{ein} would have a strong ending in the three exceptional cases, \citet[47 fn. 14, 53]{Olsen1991b} points out that \textit{ein} ‘a’ is part of a group of simple determiner stems (“schlichte Determinans-Stämme”) that in the case of \textit{ein}, have no inflection in the nominative masculine and in the nominative/accusative neuter.

As far as I am aware, this is the first detailed proposal of the strong/weak alternation in German within the framework of the DP-Hypothesis. While this is an elegant account, it is not without problems. Besides cases where AGR is present but does not have to be overtly realized (\textit{ein Auto} ‘a car’; \textit{Karls Auto} ‘Karl’s car’), unmodified mass nouns do not involve a DP-level in Olsen’s account, and thus no AGR is present. \citet[191]{Bhatt1990} and others have pointed out that this leads to the problem that these noun phrases have to enter DP-external agreement relations without AGR. Furthermore, while \citet[44]{Olsen1991b} states that one strong inflection in \REF{ex:2:94} exemplifies a tendency for economical realization of features or an avoidance of redundancy, it raises questions about strong endings that appear on stacked adjectives: \textit{frisch-es kalt-es Wasser} ‘fresh-\textsc{st} cold-\textsc{st} water’.

Finally, note that instances such as \textit{dies-en Jahr-es} ‘of this-\textsc{wk} year-\textsc{gen}’ and non-canonical structures such as \textit{dies-es mein groß-es Glück} ‘this-\textsc{st} my great-\textsc{st} happiness’ raise issues. The first case involves a determiner with a weak inflection, which is unexpected in Olsen’s system. The second instance consists of two determiner elements where the leftmost element exhibits a strong ending, but the lower adjective also shows a strong inflection. Again, this is surprising under Olsen’s assumptions (for other problems, see \citealt{Wegener1995}: 159-63, \citealt{Bhatt1990}: 44).

\subsection{\citet{Schoorlemmer2009}}\label{sec:2.5.2}

Schoorlemmer proposes to account for the strong/weak alternation by employing certain aspects of the Agree relation \citep{Chomsky2000}. His novel claim is that the agreement relation between the adjective and the noun is not direct but rather indirect; that is, it is mediated by another element. In general terms, if this element is present and mediates the agreement between the adjective and the noun, the adjectival inflection is spelled out as strong. In contrast, if this agreement relation does not hold, then the features on the adjective are not valued/specified, and the inflection is spelled out as weak. The latter instance is interpreted as a default option. This account of agreement is labeled Indirect Agree.\footnote{Given the complexity of the system, I cannot do full justice to all the details of Schoorlemmer’s proposal. In what follows, I limit myself to the illustration of the basic account leaving many interesting features unexplored. However, it will become clear that there are also certain shortcomings in the basic parts of the proposal.}

In Chomsky’s Agree system, probes have uninterpretable features and appropriate goals have the corresponding interpretable features. If the probe c-com-\linebreak mands the goal, the former can value/specify its features on the basis of the latter. In Schoorlemmer’s terms, the probe and the goal share their features. Given that adjectival inflections are dependent on the phi-features of other elements, Schoorlemmer proposes that they are probes for nominal phi-features.

Specifically, adjectival inflections vary for gender and number. In addition, they may also exhibit properties of definiteness and case. The former (definiteness) is particularly clear in the Scandinavian languages and the latter (case) in German. Given that probes must c-command their goals, Schoorlemmer suggests that adjectives are higher than nouns, which are specified for gender and number, but that adjectives are also higher than determiners, which are specified for definiteness (note that case is a DP-external feature). In other words, adjectives c-command not only nouns but also determiners.

Adjectives can also function as restrictive modifiers; that is, they participate in delimiting the reference of the noun phrase. As such, adjectives on their restrictive interpretation are often assumed to be in the scope of determiners where scope is interpreted as the relevant c-command domain (for more discussion, see \chapref{sec:4}, \sectref{sec:4.4}). This leads Schoorlemmer to the following C-command Paradox.

\ea%95
    \label{ex:2:95}
    Attributive adjectives with weak adjectival inflection must be c-commanded by a definite   D for their interpretation, but they must c-command a definite D in order to license their inflection. \citep[12]{Schoorlemmer2009}
\z

It appears then as if there must be two positions for determiners, one above adjectives for interpretation purposes and one below for agreement reasons.\linebreak Schoorlemmer proposes to resolve this apparent paradox by assuming that the determiner moves from the lower position to the higher position.

  The movement of the determiner is triggered by the presence of an adjective. Adopting the general framework of \citet{HeimKratzer1998} (see \chapref{sec:1}, \sectref{sec:1.4.2.2}), consider the simplified structure in \figref{figex:2:96}. Note that the adjective (here assumed to be of type \textlangle\textlangle e,t\textrangle,\textlangle e,t\textrangle\textrangle) cannot combine with the lower DP (type \textlangle e\textrangle). The location of this type mismatch is indicated by underlining the incompatible elements. To resolve this mismatch, the determiner is moved to the left periphery.

\begin{figure}
	\caption{Movement of the determiner}
\label{figex:2:96}
\begin{forest}
  [DP\textsubscript{\textlangle e\textrangle}
    [det\textsubscript{\textlangle\textlangle e,t\textrangle, e\textrangle \textsubscript{$k$}}]
    [DP
        [\uline{AP\textsubscript{\textlangle\textlangle e,t\textrangle,\textlangle e,t\textrangle\textrangle}}
            [A]
        ]
        [\uline{DP\textsubscript{\textlangle e\textrangle}}
            [det\textsubscript{\textlangle\textlangle e,t\textrangle,e\textrangle \textsubscript{$k$}}]
            [NP\textsubscript{\textlangle e,t\textrangle}
                [N]
            ]
        ]
    ]
  ]
\end{forest}
\end{figure}

At PF, the lower copy of the determiner is deleted; at LF, the lower copy of the determiner and its projection are deleted \citep[27]{Schoorlemmer2009}. This is indicated by crossing out the relevant elements in \figref{figex:2:97}.

\begin{figure}
	\caption{Deletion of the lower determiner}
\label{figex:2:97}
\begin{forest}
  [DP\textsubscript{\textlangle e\textrangle}
    [det\textsubscript{\textlangle\textlangle e,t\textrangle e\textrangle}\textsubscript{$_k$}]
    [DP\textsubscript{\textlangle e,t\textrangle}
        [AP\textsubscript{\textlangle\textlangle e,t\textrangle,\textlangle e,t\textrangle\textrangle}
            [A]
        ]
        [\sout{DP}\textsubscript{\textlangle e\textrangle}
            [\sout{det}\textsubscript{\textlangle\textlangle e,t\textrangle,e\textrangle \textsubscript{$k$}} ]
            [NP\textsubscript{\textlangle e,t\textrangle}
                [N]
            ]
        ]
    ]
  ]
\end{forest}
\end{figure}

Observe that this leaves only one, the higher copy of the determiner, at PF. This gives the right surface order. As for LF, with the lower copy of the determiner and its projection deleted, the adjective can now combine with NP semantically. This yields an element of type \textlangle e,t\textrangle, which in turn can combine with the (higher) determiner. This brings about the right semantics of the noun phrase. Basically, all noun phrases with an attributive adjective have this much in common.

Turning to the strong/weak alternation, recall that Schoorlemmer claims that adjectives and nouns do not enter into an agreement relation directly. Rather, this relation is mediated by another element; that is, the agreement relation is indirect. Importantly, for the adjective not to c-command the noun and thus enter into a direct agreement relation with it, the agreement relation is taken to be based on dominance. Specifically, considering the structure in \figref{figex:2:97}, the adjective does not dominate any of the nodes of the noun. I first discuss the strong inflections and then the weak endings.

To repeat, Indirect Agree is a function of a mediating head. Mediating heads are proposed to be case-assigners such as little \textit{v}, T and presumably other elements. These heads have uninterpretable phi-features and thus function as probes too. For concreteness, suppose that \textit{guter Wein} ‘good wine’ is the subject of a clause and that the DP-external case-assigner is T. Assuming a null article (\textit{$\emptyset$\textsubscript{D}}), the DP is assembled as in \figref{figex:2:98}. After merging the DP in the clause, the mediating head T probes down the tree and finds the noun phrase \textit{guter Wein}. With the adjective the closest potential goal, T inspects the features on the adjective. However, the adjective does not have interpretable features, here marked as [--phi] (I comment on the strikethrough of [--phi] in \figref{figex:2:98} further below).

\glltree[\label{figex:2:98}]{
\gll gut-er    Wein\\
good-\textsc{st} wine.\textsc{masc}\\
\glt ‘good wine’
}{
[DP
    [$\emptyset_{D\textsubscript{$_k$}}$]
    [DP
        [AP
            [\textit{gut}\textsubscript{[--\sout{phi}]}]
        ]
        [DP
            [$\emptyset_{D\textsubscript{$_k$}}$]
            [NP
                [\textit{Wein}\textsubscript{[+phi]}]
            ]
        ]
    ]
]
}
\largerpage[-1]
As a consequence, the probe looks further down the structure and finds the noun. Since the noun does have the relevant interpretable features, both T and the noun enter into an Agree relation, and they share their features. Now, since T had entered into an agreement relation with the adjective before, T also shares its features with the adjective. With all its uninterpretable features licensed, the adjective is spelled out with a strong inflection. This is marked by crossing out the uninterpretable phi-features in \figref{figex:2:98}. To be clear, indirect agreement and feature sharing explain the strong endings (for more details, see \citealt{Schoorlemmer2009}: 145-46).

  Turning to the weak endings, Schoorlemmer assumes that unlike null articles \citep[156]{Schoorlemmer2009}, overt articles are probes – just like adjectives and case-assigners; that is, overt articles have uninterpretable features. This has consequences for the DP-internal agreement relations. Specifically, when the determiner probe is merged with the noun, it inspects the interpretable features of the noun. Both enter into an Agree relation and share their features. As all the features on the determiner have been specified, the determiner becomes a deactivated probe. Next, the adjective is merged and due to type mismatch, the determiner moves to the left periphery, as shown in \figref{figex:2:99}.

\glltree[\label{figex:2:99}]{
\gll der gut-e     Wein\\
the good-\textsc{wk} wine.\textsc{masc}\\
\glt ‘the good wine’
}{
  [DP
    [\textit{der}\textsubscript{[\sout{--phi}]\textsubscript{$k$}}]
    [DP
        [AP
            [\textit{gut}\textsubscript{[--phi]}]
        ]
        [DP
            [\textit{der}\textsubscript{[\sout{--phi}]\textsubscript{$k$}}]
            [NP
                [\textit{Wein}\textsubscript{[+phi]}]
            ]
        ]
    ]
  ]}
\largerpage[-1]
At this point, the DP is merged in the clause and T probes down the tree to find a goal. The closest goal is the determiner of the noun phrase. With its features already specified by the noun, the determiner deactivates T as probe. As both the determiner and T are deactivated, the features on the adjective remain unspecified. As a consequence, they are spelled out as a weak inflection by default. In a way, the determiner blocks the Agree relation between T and the adjective.

This is, in a nutshell, Schoorlemmer’s basic proposal. It presents a new and interesting view on the strong/weak alternation. To account for all the basic data in German, \citet[Chapter 5]{Schoorlemmer2009} has to make some other assumptions with regard to certain features. I do not review these finer points here. Rather, I point out again that this proposal makes some claims about the structure of the DP that are similar to the ones put forth in this book. Specifically, in Schoorlemmer’s account, the determiner is also merged in a low position and undergoes subsequent movement to the left periphery. Note though that there are some very general differences between Schoorlemmer’s and the current proposal.

For instance, while it is the strong inflections in Schoorlemmer’s analysis that are due to a specific operation (Indirect Agree), it is the weak endings in the current account that are brought about by a certain mechanism (Impoverishment). Now, as illustrated in \sectref{sec:2.3} and \ref{sec:2.4}, the strong endings occur in more diverse contexts than the weak ones. I interpreted the strong inflections as the elsewhere case. For Schoorlemmer, the presence of a mediating head is an essential ingredient to explain the strong adjectives. This raises the question as to what the mediating head in the non-canonical constructions, say vocatives in German, is.\footnote{If the Voc head discussed in \sectref{sec:2.3.9} is taken to be the relevant mediating head here, then this head is presumably absent in the Scandinavian languages, which have weak adjectives in vocatives.} In fact, some of the complex noun phrases, discussed in \sectref{sec:2.3}, involve more feature sharing relations than there seem to be mediating heads (e.g., \textit{das Sternbild Groß-er Bär} ‘the constellation Great-\textsc{st} Bear’). Again, it is not clear how strong adjectives in the different sub-structures of these nominals can be explained. In my view, these cases can be straightforwardly handled if we assume that the strong inflections involve the elsewhere case. After these general remarks, I point out some more specific issues with Schoorlemmer’s account.

To motivate the movement of the determiner, semantic types are interpreted as part of the syntactic feature bundles. It appears then as if semantic elements (types) and semantic operations (Functional Application) are either relocated into syntax or duplicated there (but see also \citealt{Schoorlemmer2012}). Moreover, it is not entirely obvious that little \textit{v}, T and other mediating heads have the full range of uninterpretable phi-features (e.g., gender). Also, while I cannot discuss this in detail here, it appears that Schoorlemmer’s system works best for the Scandinavian languages but is less straightforward for German. For instance, the mixed pattern involving \textit{ein}-words in German is left for future research. Finally, the cross-linguistic differences observed in canonical noun phrases involving possessives and demonstratives (\chapref{sec:1}, \sectref{sec:1.2.1.2}) do not follow from this uniform account.

\subsection{\citet{Murphy2018}}\label{sec:2.5.3}

\citet{Murphy2018} discusses the emergence of the strong ending on \textit{ein}-words in elliptical contexts; compare \REF{ex:2:100a} to \REF{ex:2:100b} (strikethrough marks ellipsis here).

\ea%100
    \label{ex:2:100}
\ea\label{ex:2:100a}
\gll ein (gut-es)  Buch\\
a     {\db}good-\textsc{st} book.\textsc{neut}\\
\glt ‘a good book’
\ex\label{ex:2:100b}
\gll ein-es \sout{Buch}\\
a-\textsc{st}    book\\
\glt ‘one’
\z
\z

He relates this change to the strong/weak alternation of adjectives. Rather than claiming that a strong ending licenses NP-ellipsis, here the strong ending is proposed to be a by-product of NP-ellipsis. Specifically, the latter creates a stranded affix that undergoes Local Dislocation onto a non-canonical element, the \textit{ein}-word. Murphy assumes the structure in \REF{ex:2:101}.\footnote{\label{foot:2:52} Besides \textit{$\varphi $} residing in a higher position than the adjective, the adjective can also be above \textit{$\varphi $}. These two options are important for the derivation of all the cases involving ellipsis. In this regard, Murphy assumes that \textit{$\varphi $} may induce ellipsis of its complement. Now, if AP is deleted (when \textit{$\varphi $} is above A as in \REF{ex:2:101} in the main text), \textit{eines \sout{gut Buch}} ‘one (good book)’ comes about; if \textit{n}P is deleted (when A is above \textit{$\varphi $} – the alternative ordering of \REF{ex:2:101}), \textit{ein gutes \sout{Buch}} ‘a good one’ is brought about (for \textit{eines \sout{Buch}} ‘one (book)’, see the main text below). Furthermore, note that the option of the adjective residing above \textit{$\varphi $} also accounts for some special inflectional features of adjectives like \textit{lila} ‘purple’ (see also \chapref{sec:3}, \sectref{sec:3.3}).} In his account, inflections and their hosts do not form constituents; that is, inflections are generated separately from their hosts: Determiners are assumed to be in Spec,DP and adjectives in A, but strong endings are taken to be located in D and weak inflections in \textit{$\varphi $}.

\ea%101
    \label{ex:2:101}

          [ [ DET ] D [ \textit{$\varphi $} [ A [ \textit{n} [ N ]]]]]

        \hspace*{1.5cm} |   \hspace*{.3cm}  |

                     \hspace*{1.38cm} \textsc{st}   \textsc{wk}
\z

In other words, inflections are part of the extended projection line of nouns (for similar ideas, see \citealt{Evans2019}: 76, \citealt{Rehn2019}: 194, \citealt{Wiltschko1998}). Murphy formulates a requirement that (basically) all adjectives must have an inflection in prenominal position. In order to combine the ending with the adjective or determiner, he follows assumptions from Distributed Morphology (see \chapref{sec:1}, \sectref{sec:1.4.2.1}), where Lowering precedes Local Dislocation. Specifically, he makes the suggestion that if (downward) Lowering is not possible, (leftward) Local Dislocation takes place (note that Local Dislocation is conceived of here as “leaning” to the left, \citealt{Murphy2018}: 346). I illustrate the proposal by way of discussing four derivations involving \textit{ein}-words.

I begin by discussing the most common case, where a determiner with a strong inflection precedes an adjective with a weak ending. This is exemplified by an instance in the dative neuter in \figref{figex:2:102}. The weak ending under \textit{$\varphi $} combines with the adjective by Lowering, marked by a downward pointing arrow in \figref{figex:2:102}. With the weak ending originating under \textit{$\varphi $}, the strong ending in D cannot undergo Lowering. Consequently, it combines with \textit{ein} by Local Dislocation. This is indicated by an upward pointing arrow (despite the fact that this mechanism operates on linear adjacency). This yields the structure in \figref{figex:2:102} below.

\glltree[\label{figex:2:102}]{
	\gll ein-em gut-en     Buch\\
	a-\textsc{st}     good-\textsc{wk} book.\textsc{neut}\\
	\glt ‘a good book’
}{
	[DP
		[\textit{ein}, name=ein]
		[D$^\prime$
			[D\\\textit{--em}, name=em]
			[$\varphi $P
				[$\varphi$\\\textit{--en}, name=en]
				[AP
					[A\\\textit{gut}, name=gut]
					[\textit{n}P
						[\textit{n}]
						[NP\\\textit{Buch}]
					]
				]
			]
		]
	]
\draw[->](em) to [bend left=45] (ein);
\draw[->](en) to [bend right=45] (gut);
}

To account for the idiosyncrasy of the three special cases of \textit{ein}-words, where the adjective is strong, it is stipulated that the strong ending is in \textit{$\varphi $} as in \figref{figex:2:103}, rather than in D \citep[356]{Murphy2018}. Recalling that Lowering precedes Local Dislocation, the strong ending is displaced onto the adjective deriving the structure in \figref{figex:2:103}.

\glltree[\label{figex:2:103}]{
	\gll ein gut-es    Buch\\
	a    good-\textsc{st} book.\textsc{neut}\\
	\glt ‘a good book’
}{
	[DP
		[\textit{ein}]
		[D$^\prime$
			[D]
			[\textit{$\varphi $}P
				[\textit{$\varphi $}\\\textit{-es}, name=es]
				[AP
					[A\\\textit{gut}, name=gut]
					[\textit{n}P
						[\textit{n}]
						[NP\\\textit{Buch}]
					]
				]
			]
		]
	]
\draw[->](es) to [bend right=45] (gut);
}

Removing the adjective from the noun phrase yields \figref{figex:2:104}. As just seen, the strong ending undergoes Lowering but this time onto \textit{n}. The result is spelled out as null (marked by parentheses below) as shown in \figref{figex:2:104}.

\glltree[\label{figex:2:104}]{
	\gll ein Buch\\
	a    book.\textsc{neut}\\
	\glt ‘a book’
}{
	[DP
		[\textit{ein}]
		[D$^\prime$
			[D]
			[\textit{$\varphi $}P
				[\textit{$\varphi $}\\(\textit{-es}), name=es]
				[\textit{n}P
					[\textit{n}, name=n]
					[NP\\\textit{Buch}]
				]
			]
		]
	]
\draw[->](es) to [bend right=45] (n);
}

Finally and most importantly, I turn to the cases where the noun is elided (for the ellipsis involving adjectives, see Footnote \ref{foot:2:52}). With \textit{n}P undergoing ellipsis, the strong ending cannot undergo Lowering. Rather, Local Dislocation applies as a kind of repair mechanism, and the strong ending surfaces on \textit{ein} as in \figref{figex:2:105} (the arch below sets off the elided material).

\glltree[\label{figex:2:105}]{
	\gll ein-es\\
	one-\textsc{st}\\
	\glt ‘one’
}{
	[DP
		[\textit{ein}, name=ein]
		[D$^\prime$
			[D]
			[\textit{$\varphi $}P
				[ \textit{$\varphi $}\\\textit{-es}, name=es]
				[\textit{n}P
					[\textit{n}]
					[NP\\\textit{Buch}]
				]
			]
		]
	]
\draw[->](es) to [bend left=45] (ein);
\draw(1.3,-4) to[bend left] (3,-2.2);
}

This is a novel proposal accounting for the emergence of the strong ending on \textit{ein}. As such, it is a welcome contribution.\footnote{There are other formal proposals seeking to explain the emergence of the strong inflection on \textit{ein} (for the discussion of \citealt{Evans2019}: Chapter 3, see \citealt{Roehrs2021}).} Note though that not all aspects of the proposal are entirely straightforward.

To obtain the correct surface forms, a number of adjustments have to take place. In particular, there are three instances where an inflectional suffix is spelled out as null: (i) As seen above, if a strong ending combines with \textit{n} by Lowering, it is spelled out as null yielding \textit{ein Buch} ‘a book’, (ii) if both D and \textit{$\varphi $} have an inflection but there is no adjective present in elliptical contexts, the weak ending in \textit{$\varphi $} is spelled out as null bringing about \textit{mein-em} ‘mine’ (Murphy’s page 358), and (iii) if Spec,DP is empty, the strong ending in D appears on the adjective, and the weak ending in \textit{$\varphi $} is spelled out as null resulting in \textit{heißer Kaffee} ‘hot coffee’ (his page 362).\footnote{The difference between the weak inflection being spelled out as null in (ii) and (iii) is that in (ii), the weak ending undergoes – what Murphy calls – Morphological Ellipsis, and that in (iii), the weak ending is contextually spelled out as null (as Spec,DP is empty). In my view, allowing all these adjustments opens the door for other issues. Assuming the structural option with A above \textit{$\varphi $} (Footnote \ref{foot:2:52}), it is not clear how to rule out the following ungrammatical example in the dative (where the strong ending is in D).
	\ea
	\gll * ein gut-em Buch\\
	{} a good-\textsc{st} book.\textsc{neut}\\
	\z 
	
   With Lowering taking precedence, the strong ending (in D) combines with the adjective. Given that the weak ending in \textit{$\varphi $} can, at least in principle, be spelled out as null, it is not clear to me how to rule this case out without stipulation. In my view, these issues are avoided if adjectives and inflections do not form individual heads in the extended projection line of the noun (but rather inflections are part of the structures of the determiners and adjectives themselves, as assumed here).}

Note that Murphy’s proposal is not compatible with the current analysis as all his determiners including \textit{ein}-words are in Spec,DP, and adjectives are in head positions. More importantly for current purposes, there are not many details provided as to what regulates the strong/weak alternation. While the Principle of Monoinflection is mentioned, it is not clear what derives (rather than stipulates) the fact that the strong ending must precede the weak one(s). From Murphy’s mechanism of partial copying, it is clear though that weak endings are taken as impoverished features that involve a subset of the features that spell out the strong endings. To be fair, while Murphy’s goal was not to provide those types of details, such information along with vocabulary insertion rules would be desirable.

It is also worth pointing out that there are two ways for adjectival endings to come about in Murphy’s account: Besides originating in D or \textit{$\varphi $}, strong and weak endings can also be the result of copying (accounting for multiple inflected adjectives). Furthermore, note that the stipulation to capture the idiosyncracy of the three special \textit{ein}-words (i.e., the strong ending is exceptionally located in \textit{$\varphi $}) may involve a non-local relation with D when the adjective intervenes (i.e., when A is above \textit{$\varphi $}, which is a possible ordering and required to generate \textit{ein gutes \sout{Buch}} ‘a good one’). This means that selection of \textit{$\varphi $} by D cannot be invoked here. Moreover, the three cases of \textit{ein} do not seem to have anything in common (unlike in the current account where they share features with a fourth instance of \textit{ein}).

Finally, I turn to two empirical issues that arise with non-canonical structures. First, split topicalizations are analyzed as either NP-ellipsis or NP-movement. However, it is not clear how to account for the two strong endings in \REF{ex:2:106a}. As plural forms, these types of \textit{ein}-words do not involve a strong ending in \textit{$\varphi $} (but only in D). It is clear that a second D must be present in the higher nominal. However, this is unexpected on NP-ellipsis or NP-movement. Second, pronominal DPs like \REF{ex:2:106b} present another challenge: Although D is present here, the strong inflection is absent. It appears as if the latter is, for some reason, not spelled out. Note that this strong ending did not combine with \textit{n} by Lowering as the adjective and \textit{$\varphi $} are present.

\ea%106
    \label{ex:2:106}
\ea\label{ex:2:106a}
\gll Nett-e   Leute   waren kein-e da. \\
nice-\textsc{st} people were   no-\textsc{st}  there\\
\glt ‘As for nice people, there were none there.’
\ex\label{ex:2:106b}
\gll wir nett-en  Studenten\\
we nice-\textsc{wk} students\\
\glt ‘us nice students’
\z
\z

To sum up, the three proposals reviewed in this section discuss the canonical patterns with different degrees of success and explanatory force. All accounts have an explanation of the strong inflections, but the status of the weak endings is less clear: This type of ending is taken to be the “unmarked” inflection (Olsen), the default inflection (Schoorlemmer), or it is base-generated or due to a certain type of partial copying (Murphy).\footnote{G. \citet[129]{Müller2002a} makes an explicit distinction between markers of case (strong inflection) and agreement (weak inflection).} Recall in this regard that there are actually two weak endings, -\textit{e} and -\textit{en}, and that their distributions do not relate to natural groups. In my view, it is important to account for the distribution of the weak inflections as well. In the current account, the strong and weak inflections have the same status, the difference being that the (traditional) weak inflections are less specified than the (unambiguous) strong endings.

More importantly, none of these proposals address the non-canonical cases, and as pointed out above, it is not clear how some of these instances can be accounted for by those analyses. As such, while these proposals are quite elegant, their empirical coverage is somewhat limited. As mentioned above, in my view, empirical coverage should be taken into account when evaluating the plausibility of a proposal. Indeed, one of the main claims of this book is that it is these non-canonical cases that reveal the true nature of adjectival endings in German. While this may turn out to be wrong, all analyses should eventually discuss these non-canonical structures to explore the consequences that these more complex structures might have for those analyses.

\section{Conclusion}\label{sec:2.6}

One goal of this book is to provide a detailed survey of the strong/weak alternation of adjectives in German and to determine the exact conditions for the emergence of the weak and strong endings. In the course of the discussion, I isolated one structure, the simple (canonical) DP, where weak endings occur. Illustrating that concord is a necessary (but not a sufficient) condition, it was proposed that the weak endings are underlyingly fully specified feature bundles that get reduced by Impoverishment. Impoverishment is triggered by determiners, which move across the specifier positions containing the adjectives.

I also investigated nine non-canonical contexts where strong endings surface. In each case, some independent evidence was provided that suggests that different structures are indeed involved. Importantly, arguing that Impoverishment proceeds locally and in a bottom-up fashion, I suggested that in these different structures, the relevant feature bundles remain unreduced and are spelled out as the strong endings. In that sense, the strong endings are the elsewhere case. This explains their diverse occurrences. The previous proposals discussed in the final section of this chapter do not discuss the whole range of data and as a consequence, reach very different conclusions. Another goal of this book is to draw some more general conclusions.

In this chapter, I showed that adjectival inflections in German do not correlate with (in-)definiteness. I proposed that these inflections are semantically vacuous (Hypothesis 1a). I argued that the strong/weak alternation is a reflex of different structures (Hypothesis 2a): On the one hand, weak endings can only occur in Spec,AgrP; on the other hand, strong inflections can be in Spec,AgrP, in DP, and in other, non-canonical positions (e.g., LPP, Spec,DisP, etc.). If this is on the right track, then we can utilize the strong/weak alternation as a diagnostic for the structures of other nominals. With this in mind, I turn to some consequences of the analysis above for the current account in \chapref{sec:3} and to some consequences for other proposals in \chapref{sec:4}. The discussion of these consequences will reveal other properties of adjectival inflections.
