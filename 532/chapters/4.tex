\chapter{Consequences for other analyses}\label{sec:4}

\section{Introduction}\label{sec:4.1}

In this chapter, I turn to some other consequences of the analysis developed in \chapref{sec:2}. Unlike the previous chapter, I now consider implications for other proposals. In so doing, I provide more evidence for the hypotheses proposed in \chapref{sec:1}. I consider three types of accounts where adjectival inflections are revealing as regards the proposed structure (Hypothesis 2a). Starting with analyses that seek to account for the presence of spurious indefinite articles, I show that weak adjectives present problems for structures involving Predicate Inversion and for analyses postulating null nouns. Second, discussing discontinuous noun phrases, I argue that strong endings on topicalized adjectives are only compatible with split topicalizations being analyzed as the base-generation of two separate nominals but not as movement involving one nominal. Third, returning to weak inflections, I show that adjectives involving a restrictive or a non-restrictive interpretation have the same basic structure. The discussion of these consequences documents some other properties of adjectival inflections. For instance, I demonstrate that strong endings are not “referential” but simply serve to make nominal features such as case, number, and gender visible (Hypothesis 2b). I begin by discussing the analysis involving Predicate Inversion and take up the discussion of the other points in the order mentioned above.

\section{Weak adjectives in the context of spurious indefinite articles}\label{sec:4.2}

In this section, I discuss two types of analysis that seek to account for the occurrence of spurious indefinite articles in Dutch. These proposals involve Predicate Inversion, on the one hand, and null nouns, on the other hand. Extending the discussion to German, it is shown that weak adjectives follow the spurious articles. Given that both proposals involve structures different from canonical DPs, the weak adjectives indicate that these types of analysis are not compatible with the system laid out in \chapref{sec:2}.

\subsection{Weak adjectives in structures involving Predicate Inversion}\label{sec:4.2.1}

Taking Dutch as their empirical basis, \citet{BennisEtAl1998} discuss – what they call – \textit{wat}-exclamative constructions \REF{ex:4:1a}. Like some other constructions they investigate \REF{ex:4:1b}, a singular indefinite article occurs with a plural noun. They call this element spurious article. This article is syntactically obligatory in the first construction but not in the second, where its presence or absence has semantic consequences (data are taken from \citealt{BennisEtAl1998}: 98, 101, see also \citealt{denDikken2006}).

\ea%1
Dutch
    \label{ex:4:1}
\ea\label{ex:4:1a}
\gll Wat *(een) jongens! \\
    what   {\db\db}a      boys\\
\glt ‘What boys!’
\ex\label{ex:4:1b}
\gll idioten van (een) mannen\\
idiots   of     {\db}a      men\\
\glt ‘idiots of men’
\z
\z

I focus on the exclamative construction in \REF{ex:4:1a} as that has a counterpart in German.

Adopting the general framework of  \citet{denDikken1995}, \citet{BennisEtAl1998} propose a small clause structure with some further functional positions on top. Considering \figref{figex:4:2} below, the small clause is represented by XP and the functional structure by DP. The nominal \textit{jongens} ‘boys’ is assumed to be the subject, \textit{wat} ‘what’ is the predicate, and \textit{een} ‘a’ is the head of the small clause. \citet{BennisEtAl1998} propose that D is an [+EXCL] operator, and this element needs to be lexicalized. As a consequence, \textit{een} raises from X to D. The predicate \textit{wat} undergoes Predicate Fronting to Spec,DP, a type of A’-movement. The authors interpret this as something similar to the Verb Second Constraint in the clause (their tree diagram on page 106 is slightly adapted here).

\glltree[\label{figex:4:2}]{
	\gll Wat een jongens!\\
        what a boys\\
    \glt ‘What boys!’
}{	
		[DP
			[\textit{Wat}\textsubscript{$k$}]
			[D$'$
				[\textit{een}\textsubscript{$i$}+{[+\textsc{excl}]}]
				[XP
					[\textit{jongens}]
					[X$'$
						[t\textsubscript{$i$}]
						[t\textsubscript{$k$}]
					]
				]
			]
		]
}

The counterpart of the \textit{wat}-exclamative in German involves similar (but not identical) properties. First, unlike in Dutch, the prepositional element \textit{für} ‘for’ must be present \REF{ex:4:3a}. Second, the spurious article is not obligatory in this construction. In fact, as pointed out in \chapref{sec:1}, \sectref{sec:1.2.2}, if it occurs, it typically appears in colloquial northern dialects (\% indicates dialectal variation). Note again that this \textit{ein} has a typical plural ending as seen in the attested example in \REF{ex:4:3b}, taken from the Appendix.

\ea%3
    \label{ex:4:3}
\ea[]{ \label{ex:4:3a}
\gll Was *(für) Idioten!\\
what   {\db\db}for   idiots\\
\glt ‘What idiots!’
}
\ex[\% ]{ \label{ex:4:3b}
\gll  ``Was   für eine Idioten" sagte Liam\\
 {\db}what for  a-\textsc{pl} idiots     said   Liam\\
\glt ‘“What idiots!”, said Liam.’
}
\z
\z

After these preliminary observations, I return to the main line of investigation. To make these cases relevant, it is important to determine how adjectives, especially their inflections, fare when they are added to the subject of the small clause.

Inserting an adjective yields an interesting asymmetry. With no determiner present, the adjective has a strong ending \REF{ex:4:4a}. This is expected and follows from the system developed in \chapref{sec:2}. When the spurious article is present, the adjective is weak. In fact, all the examples I have found involve weak adjectives only. Consequently, I mark the strong inflection in the attested example in \REF{ex:4:4b} as ungrammatical.

\ea%4
    \label{ex:4:4}
\ea[]{\label{ex:4:4a}
\gll Was  für geil-e(*n)             Bilder!\\
what for awesome-\textsc{st}/*\textsc{wk} pictures\\
\glt ‘What awesome pictures!’
}
\ex[\%]{  \label{ex:4:4b}
\gll   Was  für ein-e geil-e*(n)             Bilder    wer  hat die    bloß gemacht\\
what for a-\textsc{pl}  awesome-\textsc{wk}/*\textsc{st} pictures who has those \textsc{prt}  made\\
\glt ‘What awesome pictures! Who in the world took those?’
}
\z
\z

With \citegen{BennisEtAl1998} discussion in mind, the structure of \REF{ex:4:4b} would presumably look as follows.\footnote{In their discussion of \textit{N-of-a-N} constructions, \citet{BennisEtAl1998} allow a preposition and a spurious article to occupy the same head position. In other words, the structure in the main text should, in principle, be fine.}

\begin{figure}
	\caption{Predicate Inversion}
	\label{figex:4:5}
	\begin{forest}
		[DP
			[\textit{Was}\textsubscript{$k$}]
			[D$'$
				[\textit{für}+\textit{eine}\textsubscript{$i$}]
				[XP
					[{[\textit{geilen Bilder}]}]
					[X$'$
						[t\textsubscript{$i$}]
						[t\textsubscript{$k$}]
					]
				]
			]
		]
	\end{forest}
\end{figure}

The question that arises now is how to account for the weak ending in \REF{ex:4:4b}, given this structure.

Note that the Predicate Inversion structure above is configurationally identical to pronominal DPs involving dis-agreement. This type of case was discussed in \chapref{sec:2}, \sectref{sec:2.3.5} and is repeated below as \figref{figex:4:6}. Notice first that both the spurious article in \figref{figex:4:5} and the pronoun in \figref{figex:4:6} move from a lower position to the DP-level. Second and more importantly, in both cases, the adjective and noun form a complex specifier. In other words, unlike canonical DPs, the two cases in \figref{figex:4:5} and \figref{figex:4:6} have an additional layer of embedding. However, unlike \REF{ex:4:4b}, the nominal in \figref{figex:4:6} involves a strong adjective.

\glltree[\label{figex:4:6}]{
	\gll ihr          dumm-e*(s)    Pack\\
	you(\textsc{pl}) stupid-\textsc{st}/*\textsc{wk} gang.\textsc{neut}\\
	\glt ‘you stupid gang’
}{
	[DP
		[\textit{ihr}\textsubscript{$i$}]
		[DisP
			[{[\textsubscript{AgrP}\textit{dummes Pack}]}]
			[Dis$'$
				[Dis]
				[ArtP
					[t\textsubscript{$i$}]
					[NumP\\\textit{e}$_N$]
				]
			]
		]
	]
}

Before addressing the different inflections on the adjectives, it is instructive to consider a contrast between German and English.

As mentioned above, German \textit{ein} exhibits a typical plural ending in these cases. It appears then as if the spurious article morphologically agrees with the plural noun. Now, \citet[94, 97]{BennisEtAl1998} explicitly claim that the indefinite article does not form a constituent with the following noun. In their base positions, though, the noun and the indefinite article are in a Spec-head relation. This, for instance, accounts for the absence of spurious articles in English, as witnessed in the ungrammatical string *\textit{What} \textit{a men}.

In order to explain the contrast in the plural between English and German, we could assume that English \textit{a} is specified for singular morphological number but that German \textit{ein} is unspecified as regards morphological number: [$\alpha$PL morph] (see \chapref{sec:5}). Given the assumed Spec-head relation, English \textit{a} is only compatible with a singular noun but German \textit{ein} is with both a singular and plural noun. Note, however, that this Spec-head relation in and of itself does not explain the weak ending on the adjective in German.

Recall from \chapref{sec:2} that concord is not a sufficient condition for weak adjectives. What is required is an appropriate structure. As is clear from the above discussion, the Predicate Inversion structure is not a regular DP – the adjective is more deeply embedded. This leaves the occurrence of the weak adjective unexplained. However, it is possible that the structure in \figref{figex:4:5} is not correct. There are several options. For instance, one option might be to assume that a null article is present as part of the nominal in Spec,XP. As documented in \chapref{sec:2}, \sectref{sec:2.2.3} though, null articles occur with strong adjectives (except in the genitive masculine/neuter). In other words, a null article cannot account for the weak adjective either. To keep \citegen{BennisEtAl1998} structure, we would need to change the system of \chapref{sec:2} to account for the weak adjectives under the assumption of a null article.

As a second option, we could suggest that the adjective itself is in a position different from that assumed in \figref{figex:4:5}. Interestingly, for ordinary DPs such as \REF{ex:4:7a},  \citet[49]{denDikken2006} proposes the structure in \REF{ex:4:7b}, where the adjective is in the specifier position of a Relator Phrase (RP), and the noun is in the complement position of the Relator head (non-pronunciation is marked by capital letters).

\ea%7
    \label{ex:4:7}
\ea\label{ex:4:7a}  a big butterfly
\ex\label{ex:4:7b} [\textsubscript{DP/NumP} \textit{a} [\textsubscript{RP} [\textsubscript{AP} \textit{big} ] [ RELATOR [\textsubscript{NP} \textit{butterfly} ]]]]
\z
\z

Importantly, the adjective is also in the specifier below the determiner, just as in the canonical cases in \chapref{sec:2} (note that this holds independently of whether or not the adjective is base-generated or moved there). Updating the structure in \figref{figex:4:5}, notice though that the entire RP, which includes the noun in \REF{ex:4:7b}, would form the subject located in Spec,XP. This would essentially yield the same configuration as in \figref{figex:4:5}, and the weak adjectives remain unexplained.

To sum up this subsection, at best, we can observe that the indefinite article in these constructions is not entirely spurious in German. Accepting \citegen{BennisEtAl1998} structure (or adopting a similar structure), we would need to modify the analysis in \chapref{sec:2} to account for the weak adjectives. It is currently not clear to me how to do that without losing the account of the canonical and other non-canonical constructions. If a solution can be found, this modification should preferably not be a (construction-specific) mechanism that simply serves to “save” the Predicate Inversion analysis. At worst, the weak endings on the adjectives hint at the fact that \citegen{BennisEtAl1998} structure is not on the right track (for other issues, see \citealt{Matushansky2002}).

Note that \chapref{sec:2} and \ref{sec:3} documented a number of points that speak in favor of keeping the current analysis – it accounts not only for the canonical but also for many non-canonical cases. Indeed, if we accept the system laid out in \chapref{sec:2}, then this discussion reiterates the fact that the inflections of the adjectives are closely tied to the structure of the noun phrase as a whole (Hypothesis 2b). If this is on the right track, then we have a means to narrow down the choices of possible analyses of other constructions. In the next subsection, I consider another type of analysis that seeks to account for the occurrence of spurious articles. We will see that given current assumptions, this type of proposal is also ill-equipped to deal with weak adjectives.

\subsection{Weak adjectives in structures involving null nouns}\label{sec:4.2.2}

Extending earlier work by Leu, later published as \citet{Leu2008a,Leu2008b},  \citet{vanRiemsdijk2005} also discusses spurious articles in Dutch. Discussing various constructions,  \citet[165]{vanRiemsdijk2005} provides the example in \REF{ex:4:8} involving a non-\textit{wh}-exclamative (see also \citealt{BennisEtAl1998}: 92 fn. 7).

\ea%8
    \label{ex:4:8}
Dutch\\
\gll          Die auto heeft een deuken!     \\
  that car   has   a     dents\\
  \glt ‘That car has dents!’
\z

Unlike the Predicate Inversion analysis, van Riemsdijk argues for the presence of a semi-lexical null noun. He labels this element TYPE (again, capitalization here indicates non-pronunciation). Based on his structure on page 173, the datum in \REF{ex:4:8} is analyzed as in \REF{ex:4:9}, where !!! stands for an exclamative operator.

\ea%9
    \label{ex:4:9}

          !!!{\dots}{\dots}.[\textsubscript{DP} [ [\textsubscript{D} (\textit{een})] [ DEG [ [\textsubscript{n} TYPE] \textit{deuken} ]]]]
\z

With this in place, I return to the discussion of the spurious article and weak adjectives in German. Since I have not come across any examples like \REF{ex:4:8} in German that involve an adjective, I use a slightly different type of case as the basis for discussion – examples where \textit{so} ‘so’ precedes the indefinite article. As far as I can see, this additional element does not affect the point to be made.

  The presence of a null noun allows a different analysis of the spurious article. Consider \REF{ex:4:10a}, an example taken from the Appendix. Unlike Predicate Inversion, here a null noun is present by assumption, and the ending on \textit{ein} could be interpreted as feminine agreeing with that null noun (recall that the strong inflection -\textit{e} is ambiguous between feminine and plural in the nominative/accusative). As for the postulated null noun with feminine gender, an overt counterpart in German would be \textit{Art} ‘kind’ \REF{ex:4:10b}. In certain “affective” contexts, this noun would remain unpronounced \REF{ex:4:10c}.

\ea%10
    \label{ex:4:10}
\ea[\%]{\label{ex:4:10a}
\gll    ich hab  mal [so'n-e            ferien] in den alpen mit   ner jugendgruppe gemacht\\
I    have \textsc{prt}  {\db}{so a}-\textsc{pl/fem} holidays in the  Alps  with a     youth.group   made\\
\glt ‘Once, I had such a vacation with a youth group in the Alps.’
}
\ex\label{ex:4:10b}
\gll so eine Art   Ferien\\
so a      kind holidays\\
\glt ‘such a kind of holidays’
\ex  \label{ex:4:10c}
\gll so eine ART    Ferien\\
 so a      TYPE holidays\\
\z
\z

In other words, the bracketed string in \REF{ex:4:10a} is analyzed as in \REF{ex:4:10c}. Note that the presence of a null noun leads to a split of these structures into two subparts, each involving a noun (i.e., \textit{ART}, \textit{Ferien}). To be clear, then, the postulation of a null noun involves an analysis different from the Predicate Inversion structure.

Notice, however, that this type of account also faces problems as regards weak adjectives. Considering an attested example with an adjective \REF{ex:4:11a}, there are two ways to analyze this datum as regards the modifier. The adjective could precede the null noun \REF{ex:4:11b}, or it could follow the null noun \REF{ex:4:11c}.

\ea%11
    \label{ex:4:11}

\ea[\%]{\label{ex:4:11a}
\gll   ihr  macht so ne geilen   lieder\\
you make so a    awesome songs\\
\glt ‘You compose such awesome songs.’
}
\ex[(*)]{{so} {eine} {geilen} {ART Lieder}}\label{ex:4:11b}
\ex[(*)]{{so} {eine ART} {geilen} {Lieder}}\label{ex:4:11c}
\z
\z

There are strong indications that both analyses in \REF{ex:4:11b} and \REF{ex:4:11c} are not correct. Starting with \REF{ex:4:11b}, it is clear that the singular feminine form \textit{eine} can only occur in the nominative/accusative. I point out that -\textit{en} is not a possible adjective ending, strong or weak, in those instances (it would be -\textit{e}). As for \REF{ex:4:11c}, the adjective is located in the second subpart together with the plural noun. This type of structure is familiar from pseudo-partitive constructions. This seems an avenue worth taking. \citet{Löbel1989} observes in this regard that the second nominal in pseudo-partitives can be in all four morphological cases in German. In light of this, I entertain the possibility that the adjective inflection -\textit{en} could be interpreted as weak or strong.

Starting with the strong endings, I observe that -\textit{en} can only occur in the dative plural (the strong endings in the other cases are -\textit{e} or -\textit{er}). In the dative plural, the head noun usually takes -\textit{n} (e.g., \textit{mit geilen Lieder-n} ‘with awesome songs’). Crucially though, \textit{Liedern} is not at all possible in \REF{ex:4:11a}. This means that \textit{geilen} cannot be in the dative plural and thus cannot be strong. If so, then \textit{geilen} should be weak in these instances.

Note first that the weak ending -\textit{en} occurs in all morphological cases in the plural. As argued in \chapref{sec:2}, this implies the presence of a determiner triggering Impoverishment. However, \textit{geilen} in \REF{ex:4:11c} does not have its own relevant determiner (recall again that a null determiner, arguably present in the second nominal in \REF{ex:4:11c}, does not trigger Impoverishment). This leaves the option of the weak inflection unexplained. I conclude then that the presence of a singular feminine null noun cannot explain \REF{ex:4:11a}. In other words, weak adjectives also present a problem for this type of analysis (for other issues, see   \citealt{CorvervanKoppen2011a}: 61 fn. 6, 69 fn. 24).

In order to explain the weak adjective in the current system, it is clear that the indefinite article and the adjective must form a regular DP. While I cannot discuss all the above constructions in detail here, note that they are, in some sense, “affective” or “emotive”. As such, I would like to suggest that such contexts license the occurrence of the plural indefinite article in these constructions. In \chapref{sec:8}, \sectref{sec:8.2.2.3}, I propose that these contexts license the presence of certain null operators (cf. the operator !!! in \REF{ex:4:9}). These null operators are flagged by \textit{ein} yielding the instances of the plural article. The latter triggers Impoverishment on the adjectives in these types of constructions as discussed in \chapref{sec:2}.

Summarizing the last two subsections, I documented that overt plural indefinite articles are followed by weak adjectives in German. Given the current analysis, I pointed out that these weak adjectives pose a challenge for accounts involving Predicate Inversion or null nouns. There are two ways to proceed: Either the current analysis of adjectival inflections needs to be modified, or the above accounts of the spurious articles need to be changed. Regarding the first option, proponents of analyses involving Predicate Inversion or null nouns need to find a solution to explain the weak adjectives in these non-canonical constructions (NB: This proposal should also explain the canonical and other non-canonical cases). Since I am not aware of any such attempts, the alterntive is to suggest that constructions with spurious articles involve regular structures licensed in certain contexts.

More generally, given that the current analysis has broad empirical coverage as regards adjectival inflections, it shows good promise for a (more) comprehensive account. Indeed, the current system has revealed a number of interesting consequences. Most importantly, adjectival inflections in German are related to abstract structure (Hypothesis 2b). To reiterate, if this turns out to be correct, then this account narrows down the options of possible structural analyses of nominals. In the next section, I discuss strong adjectives that raise issues for certain structural analyses of split topicalization.

\section{Strong adjectives in structures involving split topicalization}\label{sec:4.3}

Discontinuous phrases have received much attention in the literature. German is interesting in that it allows the lower part of a noun phrase to be left dislocated. Compare \REF{ex:4:12a} to \REF{ex:4:12b}. The topicalized element in \REF{ex:4:12b} functions as a contrastive topic and the lower nominal forms a focus. These two parts are related by a bridge intonation contour indicated by slash signs below (capitalization indicates stress).

\ea%12
    \label{ex:4:12}
\ea\label{ex:4:12a}
\gll Ich habe keine Bücher gelesen.\\
  I     have no     books   read\\
\glt       ‘I have read no books.’

\ex\label{ex:4:12b}
\gll /\textit{BÜcher} habe ich \textit{KEI}{\textbackslash}\textit{ne} gelesen.\\
 { }books    have  I    none     read.\\
\glt ‘As for books, I have read none.’
\z
\z

To establish some terminology, I refer to this construction as split topicalization, to the left nominal as split-off, and to the right one as source.

As is well known, discontinuous DPs exhibit a number of paradoxical properties indicating both movement and separate base-generation. With current purposes in mind, I focus on the intriguing behavior of adjectives (for paradoxical properties not related to adjectives, see, for instance, \citealt{Ott2011a} and other references mentioned below). To account for these properties, different proposals have been made. In what follows, I discuss three types of analysis (see also  \citealt{vanHoof2006} and references cited therein). I show that movement out of one noun phrase faces problems in the account of strong adjectives. Similarly, a copy-and-delete approach to split topicalization cannot account for strong adjectives either. However, the base-generation of two independent noun phrases is completely compatible with the discussion in \chapref{sec:2}.

\subsection{Movement out of the source}\label{sec:4.3.1}

\Citet[122]{vanRiemsdijk1989} observes that the linear order of the adjectives in split topicalizations corresponds to the one without a split. Compare the sequences of adjectives in the unsplit examples in \REF{ex:4:13} to those in the split ones in \REF{ex:4:14}.\footnote{The judgments in \REF{ex:4:13b} and \REF{ex:4:14b} are not uncontroversial and probably too strong (see \citealt{FanselowĆavar2002}: 79-80, \citealt{Ott2011a}: 30). However, I continue providing the original judgments.}

\ea%13
    \label{ex:4:13}
\ea\label{ex:4:13a}
\gll ein neues amerikanisches Auto\\
a    new   American          car.\textsc{neut}\\
\glt ‘a new American car’
\ex[*]{\label{ex:4:13b}
\gll  ein amerikanisches neues Auto\\
       a    American      new   car.\textsc{neut}\\
}
\z
\z

\ea%14
    \label{ex:4:14}
\ea[]{\label{ex:4:14a}
\gll \textit{Ein} \textit{amerikanisches} \textit{Auto} kann ich mir \textit{kein} \textit{neues} leisten.\\
an   American           car\textsc{.neut} can   I     \textsc{refl} no    new     afford\\
\glt ‘As for an American car, I cannot afford a new one.’
}
\ex[*]{\label{ex:4:14b}
\gll  \textit{Ein} \textit{neues} \textit{Auto} kann ich mir \textit{kein} \textit{amerikanisches} leisten.\\
an   new    car\textsc{.neut} can   I     \textsc{refl} no    American           afford\\
}
\z
\z

These parallel differences in grammaticality can straightforwardly be explain\-ed under a movement analysis. In more detail,  \citet{vanRiemsdijk1989} and \citet[249-50]{Bhatt1990} argue for movement of the split-off out of the source. Adopting the DP-hypothesis, the above data can be analyzed as follows (see also \citealt{Pafel1995}, \citealt{Murphy2018}).

\ea%15
    \label{ex:4:15}
\ea \label{ex:4:15a}
\gll \textit{Ein} \textit{amerikanisches} \textit{Auto} kann ich mir \textit{kein} \textit{neues} leisten.\\
an   American           car\textsc{.neut} can   I     \textsc{refl} no    new     afford\\
\glt ‘As for an American car, I cannot afford a new one.’

\ex \label{ex:4:15b} [\textit{Ein} \textit{amerikanisches} \textit{Auto}]\textsubscript{i} kann ich mir [\textsubscript{DP} \textit{kein} \textit{neues} [t\textsubscript{i}]] {leisten.}
\z
\z

Note that there are two determiners in \REF{ex:4:15}, \textit{ein} ‘a’ and \textit{kein} ‘no’, where \textit{ein} in the split-off is apparently doubling the determiner element in the source. To explain the presence of \textit{ein} in the split-off, it is assumed that this element is inserted later in the derivation (in \sectref{sec:4.3.4}, I show that late insertion of \textit{ein} is unlikely to be correct).

Indefinite articles are usually absent in the split-off. Crucially, the distribution of adjectival inflections in such cases indicates an analysis different from simple movement. Note in this regard that adjectival inflections differ in regular and discontinuous noun phrases: The adjective is weak in a regular, unsplit DP \REF{ex:4:16a}; it is strong, if the adjective is topicalized \REF{ex:4:16b}.

\ea%16
    \label{ex:4:16}
\ea\label{ex:4:16a}
\gll Ich habe kein-e bunt-en                 Blumen gekauft.\\
I     have no-\textsc{st}  multi.colored-\textsc{wk} flowers bought\\
\glt ‘I have bought no multi-colored flowers.’
\ex\label{ex:4:16b}
\gll [\textit{Bunt-e}                \textit{Blumen}]\textsubscript{i} habe ich [\textsubscript{DP} \textit{kein-e} [t\textsubscript{i}]] gekauft.\\
      {\db}multi.colored-\textsc{st} flowers     have I   ~       none-\textsc{st}   ~    bought\\
\glt ‘As for multi-colored flowers, I have bought none.’
\z
\z

With \chapref{sec:2} in mind, I point out that the weak ending in \REF{ex:4:16a} shows that Impoverishment has taken place. The strong ending in \REF{ex:4:16b} indicates that Impoverishment has not occurred. This implies that either \REF{ex:4:16b} involves a non-canonical structure or that no relevant determiner is present in the split-off (or both).

Given a movement type of analysis, note that the entire noun phrase is assembled first. This is followed by movement of the split-off. Observe though that late separation, that is, building a regular DP first and then moving the lower part out would bring about a (wrong) weak ending in \REF{ex:4:16b}. In other words, assuming movement out of the DP, the change of the adjective ending from weak in \REF{ex:4:16a} to strong in \REF{ex:4:16b} becomes mysterious. Observe also that, as documented in \chapref{sec:2} and \ref{sec:3}, a simple surface-oriented account of the strong/weak alternation does not suffice to explain these data.\footnote{Note that proponents of this and other structures of split topicalization do usually not provide (m)any details of their account of the strong/weak alternation of adjectives.}

However, a strong ending on an unpreceded adjective is exactly what we expect if the two noun phrases in \REF{ex:4:16b} are base-generated independently of each other; that is, these elements do not seem to be related by movement. In the latter case though, the structure is different – there are two separate nominals, and the determiner that could trigger Impoverishment on the adjective in the split-off is in the other noun phrase, the source. As a consequence, Impoverishment does not occur, and the adjective in the split-off surfaces with a strong inflection as in \REF{ex:4:16b}.

To drive this point home, note that two related adjectives may show different inflections when split up such that the adjective in the source is weak, but the adjective in the split-off is strong (cf. \citealt{Haider1993}: 215 for similar data). Compare \REF{ex:4:17a} to \REF{ex:4:17b}.

\ea%17
    \label{ex:4:17}
\ea\label{ex:4:17a}
\gll Ich habe kein-e groß-en bunt-en                 Blumen gekauft.\\
I     have no-\textsc{st}  big-\textsc{wk}  multi.colored-\textsc{wk} flowers  bought\\
\glt ‘I have bought no big multi-colored flowers.’
\ex\label{ex:4:17b}
\gll \textit{Bunt-e}                 \textit{Blumen} habe ich \textit{kein-e} \textit{groß-en} gekauft.\\
multi.colored-\textsc{st} flowers  have I     no-\textsc{st}   big-\textsc{wk}  bought\\
\glt ‘As for multi-colored flowers, I have not bought any big ones.’
\z
\z

Again, a movement analysis cannot account for the different endings on the two adjectives. This is particularly clear here since the underlyingly higher adjective (\textit{groß} ‘big’) is weak, but the underlyingly lower adjective (\textit{bunt} ‘multi-colored’) is strong. This issue is brought out very clearly on current assumptions: With the determiner having moved to the DP-level in the source, it is the lower element(s) that should appear with a weak ending (usually adjectives) and the higher one(s) with a strong ending (usually determiner(-like) elements). Indeed, all adjectives should have the same inflection in a nominal, later split up by movement.\footnote{There is another type of analysis involving movement out of the source. \citet{Tappe1989} argues for a combination of different base-generations and movement. In particular, the split-off is base-generated in Spec,CP, and the source is base-generated in situ. The lower part of the source is proposed to move into the complement position of the split-off. The example in \REF{ex:4:4ia} is derived as in \REF{ex:4:4ib}.
	\ea
	\ea \label{ex:4:4ia}
	\gll So *(\textit{’nen}) \textit{Wagen} kann ich mir \textit{keinen} leisten\\
	such  {\db\db}a      car.\textsc{masc} can   I    \textsc{refl} none    afford\\
	\glt  ‘As for such a car, I cannot afford one.’
	\ex \label{ex:4:4ib} {[\textsubscript{DP} \textit{so’nen} [\textsubscript{NP} \textit{Wagen}]\textsubscript{i}] kann ich mir [\textsubscript{DP} \textit{keinen} t\textsubscript{i}] leisten}
	\z
	\z
	
	This proposal leads Tappe to revise standard assumptions about chains, which raises other issues.}

\subsection{Movement but not out of the source}\label{sec:4.3.2}

\citet{FanselowĆavar2002} hypothesize that split topicalizations involve movement but crucially \textit{not} out of the DP that will surface in two different parts. As a technical implementation, they argue for a different type of account adopting the copy-and-delete approach to movement \citep{Chomsky1995}. Moving the entire DP, they propose that deletion may affect \textit{both} copies. Glossing over some of the details here, they suggest that the determiner is deleted in the higher copy and the head noun in the lower one. This derives \REF{ex:4:18a} as in \REF{ex:4:18b}.

\ea%18
    \label{ex:4:18}
\ea\label{ex:4:18a}
\gll \textit{Wagen} hat  er sich  noch \textit{k-einen} leisten können.\\
car.\textsc{masc} has he \textsc{refl} yet    \textsc{neg}-one afford  could\\
\glt ‘As for a car, he has not been able to afford one.’

\ex\label{ex:4:18b}  \{\sout{einen} Wagen\} hat er sich noch k- \{einen \sout{Wagen}\} leisten können
\z
\z

Note that these authors treat \textit{keinen} ‘no/none’ as a composite form consisting of negative \textit{k}- and the indefinite article \textit{einen} (\chapref{sec:5}).

  At first glance, this analysis of distributed deletion seems to receive strong confirmation from the fact that the deletion of the higher copy of the determiner may be suspended yielding two instances of \textit{ein}.

\ea%19
    \label{ex:4:19}

\ea\label{ex:4:19a}
\gll \textit{Einen} \textit{Wagen} hat  er sich  noch \textit{k-einen} leisten können.\\
a         car.\textsc{masc} has he \textsc{refl} yet    \textsc{neg}-one afford  could\\
\glt ‘As for a car, he has not been able to afford one.’

\ex\label{ex:4:19b}  \{einen Wagen\} hat er sich noch k- \{einen \sout{Wagen}\} leisten können
\z
\z

However, upon closer inspection, it turns out that both determiners do not have to be the same \REF{ex:4:20a}. In fact, when the determiner in the source is definite, the one in the split-off cannot be definite \REF{ex:4:20b}.

\ea%20
    \label{ex:4:20}

\ea\label{ex:4:20a}
\gll \textit{Einen} \textit{Wagen} hat er  sich  nur   \textit{diesen} leisten können.\\
a         car.\textsc{masc} has he \textsc{refl} only this     afford  could\\
\glt ‘As for a car, he has only been able to afford this one.’

\ex[*]{\label{ex:4:20b}
\gll   \{\textit{Diesen} / \textit{Den}\} \textit{Wagen} hat  er sich  nur \textit{diesen} leisten können.\\
    {\db}this      / the     car.\textsc{masc} has he \textsc{refl} only this     afford  could\\
    }
\z
\z

Now, if a copy-and-delete type of analysis were correct, we would expect the grammaticality judgments in \REF[a-b]{ex:4:20} to be the reverse. It could be suggested though that this type of contrast is handled by repair rules. However, we see in \sectref{sec:4.3.4} that late insertion of \textit{ein} in \REF{ex:4:20a} is unlikely to be correct.

  Returning to the main line of argument, we can observe that this type of proposal is not compatible with the analysis of adjectival inflections in \chapref{sec:2} either. Specifically, constructing a regular DP first (\textit{keine bunt-en Blumen} ‘no multi-colored-\textsc{wk} flowers’), we would expect the weak ending -\textit{en} on the adjective in the split-off, contrary to fact. Note that this would hold independently of whether \textit{keine} is treated as a composite \REF{ex:4:21b} or not \REF{ex:4:21c}.

\judgewidth{(*)}
\ea%21
    \label{ex:4:21}
\ea[]{\label{ex:4:21a}  
\gll \textit{Bunt-e}                 \textit{Blumen} habe ich \textit{kein-e} gekauft.\\
multi.colored-\textsc{st} flowers  have I     none-\textsc{st} bought\\
\glt ‘As for multi-colored flowers, I have bought none.’
}
\ex[(*)]{\label{ex:4:21b}    \{\sout{eine} bunt-en Blumen\} habe ich k- \{ein-e \sout{bunten Blumen}\} gekauft.}
\ex[(*)]{\label{ex:4:21c}     \{\sout{keine} bunt-en Blumen\} habe ich \{kein-e \sout{bunten Blumen}\} gekauft.}
\z
\z

\judgewidth{\%}

As mentioned already above, a surface-oriented explanation of the strong ending is not sufficient to explain the whole range of facts. In view of these and some other issues (e.g., the licensing of Negative Polarity Items, see \citealt{Bosse2009}: 278), a different technical implementation of split topicalizations is called for.

\subsection{Separate base-generation}\label{sec:4.3.3}

Some alternative analyses take as their point of departure  \citegen{FanselowĆavar2002} generalization that split topicalizations involve movement but crucially \textit{not} out of the DP that surfaces in two separate parts. Specifically, these proposals involve the base-generation of two separate nominals in the VP and moving one of them (or both) to the left. I agree with this assessment, and as far as I am aware, a consensus seems to be emerging that this is indeed the correct characterization of the facts. There have been several proposals to make this idea concrete.

Specifically, \citet{Bosse2009} proposes an analysis where she links split topicalizations to Restrictive Elliptical Appositives \citep{vanRiemsdijk1998b}. \citet{Ott2011a} argues for an account that involves breaking up a symmetric bare-predication structure. Third, \citet{Roehrs2011} also argues for the base-generation of two separate nominals. All three analyses are compatible with the discussion in \chapref{sec:2}. Given that the third analysis involves fewer new assumptions, I illustrate this proposal here in some detail.

Basing the following account on \citet{Fanselow1988}, \citet{Roehrs2011} proposes that there is a division of labor between the syntax and the semantics. In particular, it is suggested in that paper that split topicalizations involve the separate base-generation of an argumental DP and a related predicative nominal in the same local domain, the VP. The argumental part contains an empty noun (\textit{e\-\textsubscript{N}}). In \figref{figex:4:22}, the initial stage of the derivation for the given example is illustrated.

\glltree[\label{figex:4:22}]{
	\gll \textit{Bunt-e} \textit{Blumen} habe ich \textit{kein-e} gekauft.\\
	multi.colored-\textsc{st} flowers  have I     none-\textsc{st} bought\\
	\glt {‘As for multi-colored flowers, I have bought none.’}
}{
	[VP
		[AgrP\\\textit{bunte Blumen}]
		[V$'$
			[DP\\\textit{keine e}$_N$]
			[V\\\textit{gekauft}]
		]
	]
}

Both the predicate split-off and the argumental source undergo movement to the left: AgrP undergoes topicalization to Spec,CP; DP moves for case. This yields the correct word order. Assuming that the overt nominal in the split-off and \textit{e\textsubscript{N}} in the source are of the same semantic type, the “unsaturated” predicate in Spec,CP is closed off by interpreting it in \textit{e\textsubscript{N}} of the argumental DP filling \textit{e\textsubscript{N}} with semantic content at the same time.

Returning to the issue of inflection, both nominals involve regular structures where the adjective in the split-off is in Spec,AgrP. It is clear though that the adjective is in a nominal different from that of the determiner, and as such Impoverishment does not occur. As a consequence, the CNG features on the adjective are not reduced, and the adjective surfaces with a strong ending. This gives the desired result (see also \chapref{sec:2}, \sectref{sec:2.3.6}). More generally, proposals involving separate base-generation are fully compatible with the analysis in \chapref{sec:2}. I briefly return to the discussion of \textit{ein}.

\subsection{\textit{Ein} in the split-off}\label{sec:4.3.4}

We saw in \sectref{sec:4.2} that \textit{ein} is not entirely spurious in that it triggers a weak ending on a following adjective. This means that \textit{ein}, or more precisely its feature bundles, cannot be inserted late in a high position. There is independent evidence for this conclusion. It derives from indefinite pronoun constructions occurring as split topicalizations.

  As illustrated in \chapref{sec:2} and \ref{sec:3}, adjectives can follow an indefinite pronoun \REF{ex:4:23a}. Interestingly, an adjective preceded by an indefinite article cannot \REF{ex:4:23b}.

\ea%23
    \label{ex:4:23}
\ea  \label{ex:4:23a}
\gll etwas                  Amerikanisches\\
something.\textsc{neut} American\\
\glt ‘something American’

\ex[*]{  \label{ex:4:23b}
\gll  etwas                  ein Amerikanisches\\
something.\textsc{neut} an  American\\
}
\z
\z


At first glance, we might claim that this is a phonetic restriction such that the pronoun and \textit{ein} cannot be adjacent. However, consistent with \REF{ex:4:23b}, split topicalizations formed on the indefinite pronoun construction cannot involve \textit{ein} in the split-off either. Compare \REF{ex:4:24a} to \REF{ex:4:24b}.

\ea%24
    \label{ex:4:24}
\ea\label{ex:4:24a}
\gll (\textit{Ein}) \textit{amerikanisches} hat er sich   nur \textit{eins} leisten können.\\
  {\db}an    American           has he \textsc{refl} only one  afford  could\\
\glt ‘As for American ones, he has been able to afford only one.’
\ex[]{\label{ex:4:24b}
\gll (*\textit{Ein}) \textit{Amerikanisches} hat er sich \textit{etwas} leisten können.\\
   {\db}{ }an     American            has he \textsc{refl} something.\textsc{neut} afford could\\
\glt ‘As for American stuff, he has been able to afford something.’
}
\z
\z

Considering \textit{nichts (*ein) Amerikanisches} ‘nothing American’, the same facts hold if the source involves the negative counterpart of \textit{etwas} ‘something’.

\ea%25
    \label{ex:4:25}
\ea \label{ex:4:25a}
\gll (\textit{Ein}) \textit{amerikanisches} hat er sich   noch \textit{keins} leisten können.\\
  {\db}an    American           has he \textsc{refl} yet    none  afford could\\
\glt ‘As for American ones, he has not been able to afford one yet.’
\ex[]{ \label{ex:4:25b}
\gll (*\textit{Ein}) \textit{Amerikanisches}  hat er sich noch \textit{nichts} leisten können.\\
   {\db}{ }an     American            has he \textsc{refl} yet    nothing.\textsc{neut} afford  could\\
\glt ‘As for American stuff, he has not been able to afford anyething.’
}
\z
\z

In view of the ungrammaticality induced by \textit{ein} in \REF{ex:4:24b} and \REF{ex:4:25b}, a phonetic restriction cannot explain all the ungrammatical cases: Unlike in \REF{ex:4:23b}, the pronoun and \textit{ein} are neither adjacent in \REF{ex:4:24b} nor in \REF{ex:4:25b}. Note in this respect that late insertion of \textit{ein} does not help explain the ungrammaticality in a surface-oriented account: If \textit{ein} were inserted late (i.e., in PF), the local context of \textit{ein} inside the split-off in the grammatical \REF{ex:4:24a} and \REF{ex:4:25a} and in the ungrammatical \REF{ex:4:24b} and \REF{ex:4:25b} would be the same; that is, \textit{ein} would precede the adjectives (and be separated from the relevant pronouns) in both the grammatical and ungrammatical cases of \REF{ex:4:24} and \REF{ex:4:25}. Consequently, this cannot explain the difference between the (a)-examples and the (b)-examples above. Rather, the difference between these cases seems to be located in the source: While \textit{eins} ‘one’ and \textit{keins} ‘none’ are pronominal forms related to \textit{ein} ‘a’ and \textit{kein} ‘no’, \textit{etwas} ‘something’ and \textit{nichts} ‘nothing’ are inherent pronouns.
\largerpage
Above, I suggested that the split-off is related to the source by interpreting the former inside the latter. Note that interpretation occurs at LF (and not PF). Now, the pronominal forms (\textit{eins}, \textit{keins}) presumably involve an internal makeup different from that of the inherent pronouns (\textit{etwas}, \textit{nichts}). While there are several options to implement this idea, it seems clear that interpreting the split-off containing \textit{ein} inside the source is compatible with the makeup of the pronominal forms but that this is not compatible with the makeup of the inherent pronouns.\footnote{One idea to make this more concrete is to assume that DPs and smaller elements can be interpreted in \textit{e\textsubscript{N}} of the pronominal forms but that DPs are too large to be interpreted inside the inherent pronouns – only smaller elements, the ones lacking \textit{ein}, can do so. Alternatively, in the second part of the book, I argue that \textit{ein} flags the presence of a null operator. If so, it might be claimed that it is actually this null operator that is incompatible with the inherent pronouns.} If this is on the right track, then it seems clear that the presence of \textit{ein} cannot be a late phenomenon in the ungrammatical \REF{ex:4:24b} and \REF{ex:4:25b} – its presence at LF helps explain the ungrammaticality. Note in this regard that the presence of an element at LF implies the presence of that element earlier in the syntactic derivation. If so, then the occurrence of \textit{ein} in the grammatical \REF{ex:4:24a} and \REF{ex:4:25a} cannot involve a PF phenomenon on parity of assumption. Returning briefly to \REF{ex:4:23b}, recall from \chapref{sec:2}, \sectref{sec:2.2.3} that indefinite pronoun constructions involve adjunction mediated by ModP. With this in mind, the ungrammaticality of this example could be explained by assuming that the head Mod selects AgrP, which excludes preceding \textit{ein}.

To sum up, I have demonstrated that split topicalizations are only compatible with the analysis of \chapref{sec:2} if they involve two separate base-generations (but not if they involve movement out of the DP or a copy-and-delete approach). Next, I briefly discuss adjectival inflections as regards certain semantic concepts. Specifically, I discuss inflections on adjectives as regards the restrictive and non-restrictive interpretation of adjectives, and I examine inflections on determiners as regards “referentiality”. In keeping with the current discussion, we will see that independently of their (non-)restrictive interpretation, inflected adjectives are in their regular positions (i.e., Spec,AgrP) and that the inflections themselves make no semantic contribution in German.

\section{Weak adjectives with non-restrictive interpretation}\label{sec:4.4}

As documented in detail in \chapref{sec:2}, adjectives are usually weak if they follow a determiner. Note that the adjective in these types of strings can actually have two interpretations, restrictive or non-restrictive (the latter is often called appositive). Considering the noun phrases in \REF[a-b]{ex:4:26}, the restrictive interpretation of the adjective is given in the translation line in \REF{ex:4:26a}, and the non-restrictive reading is provided in that of \REF{ex:4:26b}. For clarity, I use a relative clause to translate the adjective. The non-restrictive interpretation is clearly brought out by the addition of \textit{übrigens} ‘by the way’ and is sometimes called a by-the-way remark. To repeat, the adjectives are weak in both contexts.\footnote{In other languages, the strong/weak alternation of the adjective does indicate a difference in (non-)restrictive interpretation. In Icelandic, adjectives with a weak ending are restrictive in interpretation \REF{ex:4:6:ia}, but adjectives with a strong ending are not restrictive \REF{ex:4:6:ib} (see \citealt{Delsing1993}: 132 fn. 25; \citealt{Thráinsson1994}: 166, \citeyear{Thráinsson2007}: 3, 89; \citealt{Sigurðsson2006}: 200 fn. 3).
	\ea
	Icelandic
	\ea \label{ex:4:6:ia}
	\gll gul-i bill-inn\\
	yellow\textsc{-wk} car-\textsc{def}\\
	\glt ‘the car that is yellow’\\
	\ex \label{ex:4:6:ib}
	\gll gul-ur       bill-inn\\
	yellow-\textsc{st} car-\textsc{def}\\
	\glt ‘the car, which is yellow’\\
	\z
	\z
	
	 For recent discussion, see also \citet{Pfaff2017}. As regards inflection and position of prenominal adjectives, note also \citet{Evans2021}, who discusses a correlation of inflected and uninflected adjectives with different interpretations in Dutch. He proposes that inflected and uninflected adjectives are in different positions yielding the different interpretations.}

\ea%26
    \label{ex:4:26}
\ea\label{ex:4:26a}
\gll der alt-e Mann\\
the  old\textsc{-wk} man.\textsc{masc}\\
\glt ‘the man that is old’
\ex\label{ex:4:26b}
\gll der (übrigens) alt-e     Mann\\
the   {\db}incidentally old-\textsc{wk} man.\textsc{masc}\\
\glt ‘the man, who is (by the way) old’
\z
\z

With the above analysis of weak endings in mind, this implies that both restrictive and non-restrictive interpretations involve the same basic (regular) structure in which Impoverishment has occurred. In other words, inflected non-restrictive adjectives cannot be inserted late, that is, after Impoverishment has occurred. Rather, I suggest that the structure of non-restrictive adjectives involves an additional lexical element.

  In \citet[104]{Roehrs2009a}, I propose that the main difference between the restrictive and the non-restrictive interpretation of adjectives involves the absence or presence of a null \textit{pro} co-indexed with the hosting DP. This can be schematically represented as follows.

\ea%27
    \label{ex:4:27}
\gll [der [(pro\textsubscript{i}) alte] Mann]\textsubscript{(i)}\\
 {\db}the {} old man.\textsc{masc}\\
\z
In order to account for the uniform inflectional behavior of the adjectives, it is important to be more precise about the relevant structures and derivations. With the focus on the morpho-syntax, I only briefly discuss the semantics (for the denotations, see \citealt{Roehrs2009a}: Chapter 3). I start with the case involving a restrictive interpretation.

Recall from \chapref{sec:2} that the determiner moves from below the adjective to the DP-level as shown in \figref{figex:4:28}. Given this structure and the presence of a determiner, Impoverishment occurs bringing about the weak ending on the adjective. As to the semantics, I assume that kind nouns are of type \textlangle e\textrangle. Following  \citet{deSwartEtAl2007}, I argue in \chapref{sec:6} that a kind noun combines with the realization operator REL (type \textlangle e\textlangle e,t\textrangle\textrangle) to yield a predicate (type \textlangle e,t\textrangle). This predicate combines with the adjective by Predicate Modification. The resulting conjunction of the two predicates is combined with the determiner (type \textlangle\textlangle e,t\textrangle e\textrangle) by Functional Application yielding an entity (type \textlangle e\textrangle).

\begin{figure}
	\caption{Restrictive interpretation}
	\label{figex:4:28}
	\begin{forest}
		[DP\textsubscript{\textlangle e\textrangle}
			[\textit{der}\textsubscript{\textlangle\textlangle e,t\textrangle e\textrangle\textsubscript{$i$}}]
			[AgrP\textsubscript{\textlangle e,t\textrangle}, edge label={node[midway,above right]{(Functional Application)}}
				[\textit{alte}\textsubscript{\textlangle e,t\textrangle}]
				[ArtP\textsubscript{\textlangle e,t\textrangle}, edge label={node[midway, above right]{(Predicate Modification)}}
					[\sout{\textit{der}}\textsubscript{$i$}]
					[NumP\textsubscript{\textlangle e,t\textrangle}
						[Num\textsubscript{\textsc{rel}\textlangle e\textlangle e,t\textrangle\textrangle}]
						[NP\\\textit{Mann}\textsubscript{\textlangle e\textrangle}, edge label={node[midway, right]{~~~~~(Application of \textsc{rel})}}]
					]
				]
			]
		]
	\end{forest}
\end{figure}

The structure of the non-restrictive reading is basically the same. Unlike above, however, the determiner is interpreted in its base-position; that is, the adjective is outside the scope of the determiner, see \figref{figex:4:29} (see also \citealt{Partee1973}: 54, \citealt{Schoorlemmer2009}). Proceeding bottom-up and much like above, the kind noun combines with the realization operator REL to yield a predicate. Unlike above, the low copy of the determiner combines with the predicate noun yielding an element of type \textlangle e\textrangle. In order for this resultant nominal to combine with a clausal predicate (type \textlangle e,t\textrangle), I adopt a model of multiple semantic spell-out, whereby the higher by-the-way remark is sent off for interpretation separately (for details, see \citealt{Roehrs2009a}: 102-06). Structurally, this remark consists of the adjective and \textit{pro}, indicated by square brackets below. These two elements combine by Functional Application yielding a truth value (I continue the discussion of this subcomponent further below). After this element is sent off to Spell-out, the remaining nominal can combine with a clausal predicate (not shown here).

\begin{figure}
	\caption{Non-restrictive interpretation}
	\label{figex:4:29}
\scalebox{.96}{
	\begin{forest}
		[DP\textsubscript{\textlangle e\textrangle}
			[\sout{\textit{der}}\textsubscript{$i$}]
			[AgrP\textsubscript{\textlangle e\textrangle}
				[$\underbrace{\textnormal{{[\textit{pro}\textsubscript{\textlangle e\textrangle} \textit{alte}\textsubscript{\textlangle e,t\textrangle}]}}}_{\textnormal{sem. spell-out}}$\textsubscript{\textlangle t\textrangle}]
				[ArtP\textsubscript{\textlangle e\textrangle}
					[\textit{der}\textsubscript{\textlangle\textlangle e,t\textrangle e\textrangle\textsubscript{$i$}}]
					[NumP\textsubscript{\textlangle e,t\textrangle}, edge label={node[midway,above right]{(Functional Application)}}
						[Num\textsubscript{\textsc{rel}\textlangle e\textlangle e,t\textrangle\textrangle}]
						[NP\\\textit{Mann}\textsubscript{\textlangle e\textrangle}, edge label={node[midway, right]{~~~~~(Application of \textsc{rel})}}]
					]
				]
			]			
		]
	\end{forest}
}
\end{figure}

Recalling that \textit{pro} and \textit{der Mann} ‘the man’ are coindexed, the expression involving \textit{pro} and the adjective is interpreted as an additional, non-restrictive remark about \textit{der Mann}, something along the lines of ‘the man – he is old’. To be clear then, the determiner in both interpretations has moved to the DP-level in syntax triggering Impoverishment. However, the determiner is interpreted in different positions at LF. I consider the internal structure of the non-restrictive modifier in more detail.

In \chapref{sec:2}, \sectref{sec:2.2.4}, I argued that adjectives involve extended projections. In particular, I suggested there that they consist of an adjective stem at the bottom and an inflectional head at the top. Note that adjectives can take dependents, for instance, an assumed \textit{pro} (notice that the structure involving AP in \figref{figex:4:30} is presumably a simplification as \textit{pro} acts like a subject). In order for the stem to combine with its inflection, the stem along with its dependents moves to Spec,InflP. The minimal structure is as follows.

\begin{figure}
	\caption{Structure of non-restrictive adjectives}
	\label{figex:4:30}
	\begin{forest}
		[InflP
			[AP\textsubscript{\textlangle t\textrangle\textsubscript{$k$}}
				[\textit{pro}\textsubscript{\textlangle e\textrangle}]
				[\textit{alt}\textsubscript{\textlangle e,t\textrangle}]
			]
			[Infl$'$
				[\textit{-e}]
				[t\textsubscript{$k$}, edge label={node[midway, right]{~~(Functional Application)}}]
			]
		]
	\end{forest}
\end{figure}

Semantically, the pronominal element \textit{pro} (type \textlangle e\textrangle) combines with the adjective (type \textlangle e,t\textrangle) by Functional Application yielding a truth value (type \textlangle t\textrangle). This is the desired result for the by-the-way remark ‘he is old’.

Observe that independently of the presence of \textit{pro}, Impoverishment can occur in a local fashion, namely on the highest head of the extended projection of the adjective (Infl). In other words, given this structure, Impoverishment is compatible with different interpretations of the adjective. If this is accepted, then I can continue claiming that weak adjectival inflections indicate one and the same structural constellation and are independent of the semantics (Hypothesis 1a). Finally, I turn to a case where an adjectival inflection seems to be related to “referentiality”.

\section{Adjectival inflections make nominal features visible}\label{sec:4.5}

Adjectival inflections occur on determiners and adjectives but not, for instance, on verbs. It is clear that they are nominal in nature. I argued above that adjectival inflections are semantically vacuous. I demonstrated that adjectival inflections do not indicate, or are a reflex of, (in-)definiteness or (non-)restrictiveness in interpretation. Considering certain other data, there is another possible semantic concept they could be associated with.

\citet[31]{AlexiadouSchäfer2011} point out that inflected \textit{dieses} ‘this’ is an anaphor for noun phrases only, but uninflected \textit{dies} ‘this’ is an anaphor for both noun phrases and clauses. Compare \REF{ex:4:31a} to \REF{ex:4:31b} (data slightly adapted; see also \citealt{HelbigBuscha2001}: 229-30).

\ea%31
    \label{ex:4:31}
    \ea\label{ex:4:31a}
\gll Hans hat ein rotes Buch.         {\{Dies-es / Dies\}} war sehr teuer.\\
Hans has a    red    book.\textsc{neut}. {\dbrace}this-\textsc{st}              was very expensive\\
\glt ‘Hans has a red book. It was very expensive.’
\ex\label{ex:4:31b}
\gll Daß Maria bereits  angekommen ist, {\{*dieses / dies\}} weiß ich genau.\\
that  Mary   already arrived           is       {\dbrace}{ }this                know I     well\\
\glt ‘I am positive that Mary has already arrived.’
\z
\z

Given the data above, we might suggest that the two demonstrative forms are semantically different in that inflected \textit{dieses} must have a noun phrase as its antecedent, but uninflected \textit{dies} is less restricted and tolerates both a noun phrase and a clause as its antecedent. As these two elements differ in inflection, we might further suggest that adjectival endings have or are sensitive to the semantics after all. In particular, since inflected \textit{dieses} can only refer back to noun phrases, we could claim that inflectional endings are, in some sense, “referential”. I think this possible interpretation of the facts is not correct.

  In \chapref{sec:3}, \sectref{sec:3.3}, I discussed adjectives like \textit{lila} ‘purple’ and \textit{rosa} ‘pink’. I pointed out that the inflections on these adjectives are optional when a noun follows \REF{ex:4:32a} but that they are obligatory when such a noun is absent \REF{ex:4:32b}. Above, I argued that the source of split topicalization structures contains a null noun, which is licensed by the predicate split-off. With this in mind, I suggest that \REF{ex:4:32b} also involves a null noun, here licensed by the anaphoric relation with the noun in the previous sentence (\citealt{Fanselow1988}: 101, \citealt{Murphy2018}).

\ea%32
    \label{ex:4:32}
\ea\label{ex:4:32a}
\gll ein lila(-nes)     Kleid\\
a    purple-\textsc{infl} dress.\textsc{neut}\\
\glt ‘a purple dress’
\ex\label{ex:4:32b}
\gll Da    waren viele  bunte              Kleider. Ich kaufte  ein lila*(-nes).\\
there were   many multi.colored dresses   I    bought a    purple-\textsc{infl}\\
\glt ‘There were many multi-colored dresses. I bought a purple one.’
\z
\z

Notably, indeclinable elements never have endings including in anaphoric contexts. This can be illustrated with the numeral for ‘ten’.

\ea%33
    \label{ex:4:33}
\ea\label{ex:4:33a}
\gll zehn Kleider\\
 ten   dresses\\
\glt ‘ten dresses’
\ex\label{ex:4:33b}
\gll Da    waren viele Kleider. Ich kaufte  zehn.\\
 there were   many dresses  I     bought ten\\
\glt ‘There were many dresses. I bought ten.’
\z
\z

The distinction of declinability makes \textit{lila}- ‘purple’ in \REF{ex:4:32} and \textit{zehn} ‘ten’ in \REF{ex:4:33} different. The generalization that seems to be emerging is that adjectival inflections are obligatory when the hosting elements are declinable and occur in anaphoric contexts; indeclinable elements remain uninflected in anaphoric contexts. If this holds more generally, then there is an interesting consequence for the two forms of the demonstrative \textit{dies} and \textit{dieses}.

At first glance, this demonstrative appears to be similar to the adjectives \textit{lila} and \textit{rosa} – when overt material follows, the inflection is optional \REF{ex:4:34}. Note\linebreak though that uninflected \textit{dies} can only occur in nominative/accusative contexts, both as a determiner \REF{ex:4:34a} or as a predeterminer \REF{ex:4:34b}.\footnote{Recall that this restriction makes \textit{dies(es)} ‘this’ different from \textit{all(e)} ‘all’. The inflection of the latter is not subject to a restriction in morphological case (see again \chapref{sec:3}, \sectref{sec:3.6}).}

\ea%34
    \label{ex:4:34}
\ea \label{ex:4:34a}
\gll dies(-es)  schön-e    Kleid\\
this-\textsc{infl} pretty-\textsc{wk} dress.\textsc{neut}\\
\glt ‘this pretty dress’
\ex \label{ex:4:34b}
\gll dies(-es)  mein Glück.\\
this-\textsc{infl} my    happiness.\textsc{neut}\\
\glt ‘this happiness of mine’
\z
\z

There are two differences to the adjectives above. First, both the inflected and the uninflected forms of the demonstrative are possible in anaphoric contexts involving noun phrases as in \REF{ex:4:31a}. Note that the noun phrase in \REF{ex:4:31a} is in the neuter gender. Second, the demonstrative is also possible in anaphoric contexts involving clauses, but here only the uninflected form appears as in \REF{ex:4:31b}. If the generalization above is correct, then we cannot say that the demonstrative involves one element with an optional inflection. If this were so, then we would expect only the inflected form \textit{dieses} to occur in anaphoric contexts, contrary to fact. In order to explain the occurrence of both the inflected \textit{and} the uninflected form of the demonstrative in \REF{ex:4:31a}, we have to assume that there are actually two independent elements, a declinable \textit{dieses} and an indeclinable \textit{dies}. In other words, we cannot simply assume that \textit{dies} is based on \textit{dieses} such that the ending has been deleted in PF (for this idea, see \citealt{Gallmann2004}: 154 fn. 3, \citealt{Roehrs2009a}: 159 fn. 33, cf. also G. \citealt{Müller2002a}: 117 fn. 8). Note that the proposal of two separate elements is consistent with the aforementioned fact that unlike inflected \textit{dieses}, uninflected \textit{dies} is restricted to nominative/accusative neuter contexts.

The structure of inflected \textit{dieses} is repeated in \figref{figex:4:35}; the relevant vocabulary insertion rules are given in \REF[a-c]{ex:4:35}.

\begin{figure}
	\caption{Structure of inflected \emph{dieses}}
	\label{figex:4:35}
	\begin{forest}
		[InflP\textsubscript{\parbox{0mm}{\mbox{[+D; +DEF, +DEIX][F, N, O, S]}}}
			[{[F, N, O, S]}]
			[DemP
				[Dem\\{[+D; +DEF, +DEIX]}]
			]
		]
	\end{forest}
\end{figure}

\ea \label{ex:4:35}
\ea \label{ex:4:35a} {[+D; +DEF, +DEIX}] $\rightarrow$ \textit{dies-}
\ex \label{ex:4:35b} {[+F, --N, +O, $\alpha$S]} \hspace*{.4cm} $\rightarrow$ \textit{-er}
\ex \label{ex:4:35c} etc.
\z
\z

Recall that the stem \textit{dies}- spells out [+D; +DEF, +DEIX] by \REF{ex:4:35a} and that the varying inflections realize the CNG feature bundle in Infl, see the vocabulary insertion rules in \REF[b-c]{ex:4:35}. As for uninflected \textit{dies}, I propose that this element is like the first and second-person pronominals discussed in \chapref{sec:3}, \sectref{sec:3.5}. As such, I assume that \textit{dies} involves a reduced demonstrative structure, see \figref{figex:4:36}. The restriction of \textit{dies} to nominative/accusative neuter contexts can be captured by – what I call here – restriction features. These types of features are on [+DEIX], indicated below by the features following the colon sign after [+DEIX]. They are meant to have the following consequences: On the one hand, the features [--F, +N, --O] are only compatible with the structure in \figref{figex:4:36} being inserted in nominative/accusative neuter contexts; on the other hand, this type of deictic feature involving a restriction only projects a reduced structure.\footnote{These types of restriction features are discussed in more detail in \chapref{sec:8}, \sectref{sec:8.2.2.5} and \sectref{sec:8.2.2.6}. Among others, they account for the varying distribution of \textit{ein} in regular singular contexts (e.g., \textit{ein Auto} ‘a car’) vs. in plural contexts (e.g., \textit{m-eine Autos} ‘my cars’).}

\begin{figure}
	\caption{Structure of uninflected \emph{dies}}
	\label{figex:4:36}
	\begin{forest}
		[DemP\textsubscript{\parbox{0mm}{\mbox{[+D; +DEF, +DEIX: --F, +N, --O]}}}
			[Dem\\{[+D; +DEF, +DEIX: --F, +N, --O]}]
		]
	\end{forest}
\end{figure}

Note that the vocabulary insertion rule for \textit{dies}- in \REF{ex:4:35a} can apply to both \figref{figex:4:35} and \figref{figex:4:36}. If it spells out the features in \figref{figex:4:35}, then we obtain \textit{dies}-, where the CNG bundle is separately realized by the insertion rules in \REF[b-c]{ex:4:35}; if it applies to \figref{figex:4:36}, then we get an uninflected form of \textit{dies}, with the proviso that the structure in \figref{figex:4:36} can only occur in nominative/accusative contexts. To be clear, the possibility of projecting two different demonstrative structures accounts for the two different forms – the form of the demonstrative simply depends on which structure, \figref{figex:4:35} or \figref{figex:4:36}, is merged in Spec,ArtP of the syntactic representation. With this in mind, I return to the data in \REF{ex:4:31} from the beginning of this section.

Above, I proposed that anaphoric contexts involve null nouns. I extend this claim now to the two forms of the demonstrative. Specifically, I propose that the structure of the demonstrative for the nominal case in \REF{ex:4:31a} involves a DP that contains either \figref{figex:4:35} or \figref{figex:4:36} along with a null noun \REF[a-b]{ex:4:37}. As for the clausal instance in \REF{ex:4:31b}, I assume for concreteness that the two demonstrative structures combine with a propositional element, here illustrated with TP \REF[c-d]{ex:4:37}.

\ea%37
    \label{ex:4:37}
\ea\label{ex:4:37a} [\textsubscript{DP} \textit{dieses} [\textsubscript{NP} \textit{e\textsubscript{N}}]]
\ex\label{ex:4:37b} [\textsubscript{DP} \textit{dies} [\textsubscript{NP} \textit{e\textsubscript{N}}]]
\ex\label{ex:4:37c} [\textsubscript{DP} \textit{dies} [\textsubscript{TP} \textit{e\textsubscript{T}}]]
\ex[*]{\label{ex:4:37d}  [\textsubscript{DP} \textit{dieses} [\textsubscript{TP} \textit{e\textsubscript{T}}]]}
\z
\z

Starting with uninflected \textit{dies} in \REF[b-c]{ex:4:37}, I propose that this element is not subject to any conditions as regards agreement. Given the absence of inflection, \textit{dies} does not have to undergo concord in agreement features. Consequently, it is fine in both nominal and clausal contexts. Note again that the restriction to nominative/accusative neuter environments follows from the restriction features on [+DEIX] in \figref{figex:4:36}.

  Turning to \textit{dieses} in \REF[a,d]{ex:4:37}, observe now that -\textit{es} is a neuter inflection. Given the presence of a null noun in \REF{ex:4:37a}, the inflection on \textit{dieses} is licensed by concord in agreement features with this null noun (recall that this noun is part of the anaphor that is related to an antecedent in the neuter gender). Consequently, the demonstrative in \REF{ex:4:37a} can be part of a nominal anaphor. In order to explain the ungrammaticality of \textit{dieses} in the clausal context in \REF{ex:4:37d}, I suggest that the nominal features of the inflection cannot be licensed as there is no noun present. As such, \textit{dieses} is ungrammatical in this context and cannot serve as a propositional anaphor.\footnote{There are dialects that do not make this difference; that is, \textit{dieses} is also possible as a propositional anaphor (see, e.g., \citealt{Duden1995}: 336, \citealt{GriesbachSchulz1965}: 148). I assume that in these dialects, the licensing conditions of the neuter inflection are different; for instance, neuter gender could be licensed as a default option.}

  More generally, I propose that the varying inflectional morphology and the different semantic restrictions in \REF{ex:4:31} result from the different demonstrative structures and the element following the demonstrative, either a null nominal \REF[a-b]{ex:4:37} or a null clausal element \REF[c-d]{ex:4:37}. If so, I can maintain the claim that adjectival endings are semantically vacuous (Hypothesis 1a). Indeed, as this section has made clear again, adjectival inflections are licensed in certain contexts only. Furthermore, I propose that adjectival inflections serve to make nominal features such as case, number, and gender visible (Hypothesis 2b). This is particularly clear in contexts where overt nouns are absent. Before I continue the exploration of Hypothesis 2b, I briely need to return to the discussion of weak adjectives.

  It is important to comment on the inflection of the adjective in \REF{ex:4:34a} above, restating that data point below by separating it according to the inflected vs. the uninflected form of the demonstrative. Note again that the two forms of the demonstrative, \textit{dieses} and \textit{dies}, can only take a weak adjective.

\ea%38
    \label{ex:4:38}
\ea \label{ex:4:38a}
\gll dies-es schön-e    Kleid\\
this-\textsc{st} pretty-\textsc{wk} dress.\textsc{neut}\\
\glt ‘this pretty dress’
\ex\label{ex:4:38b}
\gll dies schön-e    Kleid\\
this pretty-\textsc{wk} dress.\textsc{neut}\\
\glt ‘this pretty dress’
\z
\z

If \textit{dieses} and \textit{dies} are indeed two separate elements, then the weak inflection on the following adjective in \REF{ex:4:38b} cannot be a function of a preceding strong ending but rather of the preceding determiner stem. Note in this regard that both \textit{dieses} and \textit{dies} involve the feature [+DEF], a context where [+D] triggers Impoverishment. Another interesting point about Hypothesis 2b can be made more explicit here.

  Recall that Impoverishment deletes [S] from the syntactic representation – the other features remain intact. Furthermore, as mentioned before, the vocabulary insertion rules are underspecified for some of the CNG features. This means that overt inflections only make a subset of the underlying features visible. In other words, Hypothesis 2b does not claim that adjectival inflections make all CNG features visible. Rather, it claims that adjectival inflections make features visible that are left in the syntactic representation after Impoverishment has occurred. In fact, adjectival inflections only make features visible that are specified in their vocabulary insertion rules.

Finally, I documented in \chapref{sec:2} that \textit{ein}-words can involve pronominal forms too. Exemplifying with \textit{ein}, this element is different from the cases discussed above in certain ways. Similar to the demonstrative, \textit{ein} can also occur in anaphoric contexts \REF{ex:4:39}. Unlike the demonstrative, \textit{ein} must have an ending in anaphoric contexts relating noun phrases.

\ea%39
    \label{ex:4:39}
\gll Zwei Männer standen auf der Straße. Ein*(-er) von ihnen kam zur     Tür.\\
  two   men       stand     in   the  street    one-\textsc{st}     of    them came to.the door\\
  \glt ‘Two men were standing in the street. One of them came to the door.’
\z

Assuming the presence of a null noun following \textit{einer} ‘one’, I argue that the obligatory inflection on \textit{ein} indicates that there can be only one type of \textit{ein}. As proposed in \chapref{sec:2}, \sectref{sec:2.2.2}, this element is a determiner. Second, like the inflection on \textit{dieses} and other elements, the ending on \textit{ein} in \REF{ex:4:39} makes nominal features visible. In the next chapter, I discuss \textit{ein} in more detail. We will see that there are actually two types of \textit{ein}. I hasten to add though that the other type of \textit{ein} is an adjective that follows definite determiners and is crucially not related to the article \textit{ein} in \REF{ex:4:39}.

\section{Conclusion}\label{sec:4.6}

In this chapter, I turned to four consequences of the analysis laid out in \chapref{sec:2}. I discussed two influential proposals of spurious articles – Predicate Inversion and null nouns. I argued that the weak endings on adjectives in these structures indicate that the plural indefinite article in German is not entirely spurious. The two proposals just mentioned involve non-canonical structures. As such, the weak adjectives do not follow from the system developed in \chapref{sec:2}. Awaiting an account that can explain the weak adjectives in these non-canonical structures without losing the account of the canonical and other non-canonical instances, I assume that spurious articles actually involve regular structures. While the discussion of these structures is postponed to \chapref{sec:8}, I continue assuming that the current account of the strong/weak alternation is on the right track.

  Next, I turned to some consequences for the discussion of discontinuous noun phrases. Illustrating some paradoxical properties, I showed that analyses postulating movement that involve one underlying noun phrase cannot account for the strong adjective in the split-off under current assumptions. In contrast, accounts involving the base-generation of two separate underlying nominals and subsequent movement are completely compatible with the analysis of \chapref{sec:2}. I also provided evidence that \textit{ein} (or its feature bundles) is unlikely to be inserted late.

  Finally, I suggested that non-restrictive adjectives have a similar structural analysis as restrictive ones, the main difference being the presence of a null co-indexed pronoun in the former. In other words, inflectional endings do not signal (non-)restrictiveness of the interpretation of adjectives. In a similar vein, I argued that adjectival inflections are not tied to concepts like “referentiality” but make nominal features like case, number, and gender visible (Hypothesis 2b).

More generally, I can maintain the claim that adjectival endings are a reflex of the structure (Hypothesis 1b) but not of the semantics (Hypothesis 1a). Furthermore, the discussion of Predicate Inversion has shown in more detail that the degree of the embedding of the adjective is important for the account of the strong/weak alternation (Hypothesis 2a). In the next chapter, I turn to the discussion of \textit{ein} in more detail. We will see that \textit{ein} shares some of the properties of adjectival inflections.
