\chapter{Introduction}\label{sec:1}

\section{Vacuous elements – an imperfection of language?}\label{sec:1.1}

In many languages, ordinary sentences contain nominative subjects. Nominative case is often taken to be structural; that is, it is assigned under certain syntactic conditions. This type of case has received a lot of attention in the linguistic literature. In the generative tradition, structural case – often spelled Case – is frequently represented by uninterpretable features. \citet[e.g.,][119]{Chomsky2000} pointed out that uninterpretable features may be an imperfection in the design of languages. Specifically, structural case features may receive no interpretation in either PF or LF. In terms of optimally efficient language design, it is surprising that these elements exist at all.

It is usually proposed that uninterpretable features have to be removed during the derivation to adhere to the Principle of Full Interpretation \citep{Chomsky1995}. This removal has to do with feature checking (or valuation). Specifically, feature checking (or valuation) motivates movement, including that of subjects. As the result of this movement, features including structural case features are removed from the derivation (e.g., \citealt{Chomsky1995,Chomsky2000}). In other words, no uninterpretable features are left at the end of the derivation, and the derivation succeeds.

Besides these abstract features, there are also word-level elements that raise similar issues about the design of language – expletives. Although these elements do get an interpretation in PF (i.e., they have a phonetic realization), they seem to have no semantics either. In the next section, I briefly illustrate two such cases that have been intensively discussed in the literature: expletive \textit{there} and the proprial article. In the second section, I discuss two other elements that have received less attention in this regard: German adjectival inflections and the indefinite article \textit{ein} ‘a’. In this book, I propose that the latter two elements also have no semantics at all; that is, they have neither semantics of their own nor do they make semantic features visible. I hypothesize that these elements are not an imperfection of language, but provide overt clues about the presence of abstract linguistic elements.

\subsection{The clause}\label{sec:1.1.1}

Sentence pairs such as \REF[a-b]{ex:1:1} have received much attention in the literature (\citealt{Milsark1974} and much subsequent work). Clauses like \REF{ex:1:1b} are usually referred to as \textit{there}-existentials and are used in presentational contexts.

\ea%1
    \label{ex:1:1}
\ea\label{ex:1:1a}   A man is in the garden.
\ex\label{ex:1:1b}   There is a man in the garden.
\z
\z

Both sentences in \REF{ex:1:1} have very similar meanings: They are about a man being in the garden. As \citet[396]{Hazout2004} points out, \textit{there} itself cannot have any semantics; for instance, if \textit{there} were a deictic element, its distal semantics should clash with the proximal meaning of \textit{here} in \REF{ex:1:2a}. Evidently, this is not the case. Second, the unaccusative predicate \textit{arrive} assigns one theta role in \REF{ex:1:2b}, namely to \textit{three men}. With the classical Theta-Criterion \citep[36]{Chomsky1981} satisfied, \textit{there} cannot be an argument or a predicate. If \textit{there} makes no semantic contribution, then we might expect that it can be left out without a (significant) loss in meaning.\footnote{Referencing work by Edwin Williams, \citeauthor{Chomsky1991} (1991 [reprinted in \citealt{Chomsky1995}: 157]) points out that scopal elements in constructions with \textit{there} vs. without the expletive may have different readings (also \citealt{LasnikEtAl2005}: 157).} This is borne out as can easily be verified in \REF{ex:1:2c}.

\ea%2
    \label{ex:1:2}
\ea\label{ex:1:2a}   There are too many people here.
\ex\label{ex:1:2b}   There arrived three men.
\ex\label{ex:1:2c}   Three men arrived.
\z
\z

The conclusion is that \textit{there} is indeed an expletive (i.e., pleonastic) element.

There are many proposals that aim to explain \textit{there}-existentials (for a brief survey of some of the relevant issues, see, e.g., \citealt{LasnikEtAl2005}: 153-69). I briefly illustrate the basic account in \citeauthor{Chomsky1991} (1991 [reprinted in \citealt{Chomsky1995}]), one of the first proposals of this construction in the Minimalist tradition. Note though that the purpose of this book is not to discuss this construction in detail, but rather to lay out some basic assumptions and to relate existential constructions to other similar instances.

\citet[154-57]{Chomsky1995} assumes the Principle of Full Interpretation, according to which all elements in a linguistic expression must have an interpretation in LF. Above, I showed that \textit{there} is semantically vacuous; that is, it has no semantics. As a consequence, \textit{there} can, by itself, not be interpreted in LF. To explain the grammaticality of \textit{there}-constructions, Chomsky proposes that \textit{there} is a LF-affix and that the overt noun phrase, the associate, undergoes movement to adjoin to \textit{there} (also \citeauthor{ChomskyLasnik1993} (1993 [reprinted in \citealt{Chomsky1995}: 65-66]). As the expletive is licensed by adjunction of another element in LF, Full Interpretation is not violated. I assume that the proposal involving the licensing of the expletive by moving the associate is basically correct.

Clauses and noun phrases are usually held to be parallel in meaning and structure (e.g., \citealt{Abney1987}, \citealt{Chomsky1970}, \citealt{Iordăchioaia2020}). This can most easily be seen in \REF{ex:1:3}, which juxtaposes a verb and its arguments in \REF{ex:1:3a} with its derived nominal counterpart in \REF{ex:1:3b}. It is important to point out that both the verb in the sentence and the noun in the noun phrase have an agentive subject (\textit{the Romans}) and an affected object (\textit{the city}).

\ea%3
    \label{ex:1:3}
\ea\label{ex:1:3a}   The Romans destroyed the city.
\ex\label{ex:1:3b}   the Romans’ destruction of the city
\z
\z

Based on these similarities in meaning, a certain structural parallelism has come to be established. Considering \REF{ex:1:4}, NP is taken to be the nominal counterpart of VP. Furthermore, NumP is similar to AgrP, DP to TP, and PP to CP.\footnote{This alignment based on \citet{Grimshaw1991} is not uncontroversial; for instance, other authors have proposed that DP is parallel to CP (e.g., \citealt{Szabolcsi1994}). Note that constructions such as \REF{ex:1:3} are often discussed in this context. This book will not consider cases involving theta nouns like \textit{destruction} as in \REF{ex:1:3b} in much detail. Rather, investigating many non-canonical constructions (i.e., structures more complex than simple DPs), we will come across cases where more structure is projected above the DP (and below the PP) in \REF{ex:1:4b}. Below, this structural level is labeled Left Periphery Phrase (see \sectref{sec:1.4.1.2}). Furthermore, there is debate in German as to whether subject-initial root clauses involve CPs (e.g., \citealt{SchwartzVikner1996}) or TPs (e.g., \citealt{Zwart1997} for Dutch, which extends to German). Given these complexities and controversies, I do not engage in the debate of the alignment in \REF{ex:1:4} here. Rather, I simply follow the alignment in \REF{ex:1:4} and assume that the DP is parallel to the TP (see also \chapref{sec:6} and \chapref{sec:7} below). For clauses, I assume that subject-initial structures are (head-initial) TPs and non-subject initial structures are CPs \citep{Zwart1997}, but nothing important hinges on this choice.}

\ea%4
    \label{ex:1:4}
\ea\label{ex:1:4a}   CP {} TP {} AgrP {} ~~VP
\ex\label{ex:1:4b}   PP {} DP {} NumP {} NP
\z
\z

These and other observations have led to the general hypothesis that noun phrases are similar to clauses. Part of this work has been the establishment of the DP-Hypothesis \citep{Abney1987}. This line of investigation has led to many empirical discoveries and theoretical innovations (for a survey, see \citealt{AlexiadouEtAl2007}). The current work espouses the DP-Hypothesis. While noun phrases in argument position are assumed to be at least DPs (e.g., \citealt{Longobardi1994}, \citealt{Stowell1989}), we see in the course of the discussion that noun phrases in non-argument position (e.g., nominal predicates) may be structures of smaller sizes.

Before proceeding, I briefly provide here details about some of the terminology used in this book. The terms noun phrase and nominal are used as general designations for structures involving a noun and possibly some modifiers or other dependents. Terms such as DP, NumP, or NP are used when the structure of the nominal (/noun phrase) is of particular interest. With this in place, I return to the main line of exposition. Given the existence of expletives in clauses and the assumed parallelism between clauses and noun phrases, it may not come as a surprise that expletive elements have also been proposed to occur in the nominal domain.

\subsection{The noun phrase}\label{sec:1.1.2}

\citet{Longobardi1994} develops a theory of noun movement in (overt) syntax and LF to account for a number of similarities and differences between the Romance and Germanic languages. \citet[651]{Longobardi1994} suggests that the following alternation can be related to the \textit{there}-existentials discussed in the previous section. Similar to the (associate) noun phrases in \REF{ex:1:1}, proper names may appear in different positions. Comparing \REF{ex:1:5a} to \REF{ex:1:5b}, we observe that the proper name in \REF{ex:1:5b} follows a definite article. This determiner is often referred to as proprial article. Longobardi relates the distribution in \REF{ex:1:5a} to \REF{ex:1:5b} by proposing that the proper name in \REF{ex:1:5a} has undergone movement to D within the larger subject DP. This is illustrated in \REF{ex:1:5c} (more on this below).\footnote{\citet[640]{Longobardi1994} actually assumes that N in \REF{ex:1:5c} raises to D by substitution. To keep the discussion parallel to the clause, I assume adjunction here.}

\ea%5
    \label{ex:1:5}
    Italian\\
\ea\label{ex:1:5a}
\gll Gianni mi  ha  telefonato.\\
    Gianni me has called\\
\glt ‘Gianni called me.’
\ex\label{ex:1:5b}
\gll Il   Gianni mi ha  telefonato.\\
the Gianni me has called\\
\glt ‘Gianni called me.’
\ex\label{ex:1:5c} [\textsubscript{DP} \textit{Gianni}\textsubscript{i}+D [\textsubscript{NP} t\textsubscript{i}]]
\z
\z

Importantly, both DPs in \REF{ex:1:5a} and \REF{ex:1:5b} are interpreted as referential; that is, the definite article in \REF{ex:1:5b} does not make a semantic contribution (see also \citealt{LekakouSzendrői2012}, \citealt{VergnaudZubizarreta1992}). In other words, the article is added in \REF{ex:1:5b} with no change in the meaning of the noun phrase as a whole. In fact, note that if the definite article in \REF{ex:1:5b} were to make a contribution, it would raise issues with regard to the redundancy of certain semantic components: Singular definite articles typically presuppose the existence of a unique entity; however, proper names, in their referential use, already denote unique individuals by themselves. To avoid this issue, \citet[648-50, 655]{Longobardi1994} argues that the proprial article functions as an expletive element similar to \textit{there} above.\footnote{For different views of the proprial article, see \citet{Matushansky2008} and \citet{Muñoz2019}.}

Longobardi proposes that \REF{ex:1:5a} involves movement of the noun to D. Among others, this N-to-D movement lexically licenses the null D position of a syntactic argument, and it obviates a default existential interpretation. With the discussion of \textit{there}-existentials in mind, I assume with \citet[651 fn. 48]{Longobardi1994} that this movement is overt in \REF{ex:1:5a} but covert in \REF{ex:1:5b}. In the latter case, the proper name – in Chomsky’s terms, the associate – licenses the expletive proprial article.

There is other distributional and morphological evidence that the proper name undergoes movement and that the corresponding article is different from other types of determiners. Simplifying somewhat, \citet[623]{Longobardi1994} points out that the proper name can precede or follow a possessive adjective. Compare \REF{ex:1:6a} and \REF{ex:1:6b}. If the proper name follows the possessive adjective, the definite article must appear. In this regard, observe the difference in grammaticality between \REF{ex:1:6b} and \REF{ex:1:6c}.

\ea%6
    \label{ex:1:6}
    Italian\\
\ea\label{ex:1:6a}
\gll Gianni mio ha finalmente telefonato.\\
    Gianni my  has finally        called\\
\glt ‘My Gianni finally called.’
\ex\label{ex:1:6b}
\gll Il   mio Gianni ha finalmente telefonato.\\
the my  Gianni has finally        called\\
\glt ‘My Gianni finally called.’
\ex[*]{\label{ex:1:6c}
    \gll Mio Gianni ha  finalmente telefonato.\\
    my  Gianni has finally        called\\
}
\z
\z

Similar to the paradigm in \REF{ex:1:5}, Longobardi proposes that the proper name moves from a lower position within the larger subject DP \REF{ex:1:6b} to the higher position D \REF{ex:1:6a}. The example in \REF{ex:1:6c} is ungrammatical because the empty D is not lexically licensed in syntax, something the Romance languages require.

In addition, \citet[656]{Longobardi1994} provides evidence that pleonastic articles can be morphologically different from substantive articles. Compare \REF{ex:1:7a} involving a proper name to \REF{ex:1:7b} containing a common noun, where the article is \textit{en} in the first case but \textit{el} in the second one (for more details, see Longobardi’s work; see also \citealt{Panagiotidis2000} for Northern Greek).

\ea%7
    \label{ex:1:7}
    Catalan\\
  \ea \label{ex:1:7a}
  \gll en  Pere\\
      the Peter\\
  \glt ‘Peter’

  \ex \label{ex:1:7b}
  \gll el   gos\\
      the dog\\
  \glt ‘the dog’
  \z
\z

Given the difference in articles in \REF{ex:1:7}, Longobardi presents a strong argument that articles come in several types with different semantics, expletive or contentful.

To sum up, I take it as established that there are expletive elements in the clause as well as in the noun phrase. The assumption that certain lexical elements are semantically vacuous allows us to avoid issues such as contradictions (e.g., as regards deixis in the clause) and redundancies (e.g., as regards uniqueness in the noun phrase). Furthermore, the assumption that such elements are expletives explains why these elements can be left out without a significant loss in meaning. Note that in each case discussed so far, the expletive element is in a position higher than the substantive element. Schematically, this can be illustrated as shown in \figref{figex:1:8}.

\begin{figure}
\begin{forest}
[$\alpha$
[EXPL
    [{\textit{there}\\proprial article}]
]
[$\beta$
    [SUBST
        [{associate\\proper name}]
    ]
    [~]
]
]
\end{forest}
\caption{\label{figex:1:8}Structural relation between expletives and substantive elements}
\end{figure}

In order not to violate Full Interpretation, the pleonastic element must be licensed. In both cases, this licensing process involves movement of a lower substantive element to the higher expletive one.

  Two questions arise from the discussion above: (i) Are there other semantically vacuous elements and (ii) Why do these elements exist at all? The first question is discussed in detail for the German noun phrase in the following pages. I argue that adjectival inflections and the indefinite article \textit{ein} ‘a’ are also semantically vacuous elements. The second question is much harder to answer and is only briefly addressed in this book. I suggest that expletives indicate the presence of abstract linguistic elements. In that sense, adjectival inflections and \textit{ein} are not an imperfection of language.

  Note that this book is not intended to contribute to the theory of expletive elements \textit{per se}. Rather, it tries to identify more elements that share some of the properties of the expletives that have been established in the literature. Having said that, I engage in a more detailed discussion of \textit{there} vs. adjectival inflections and \textit{ein} in the final section of this book (\chapref{sec:8}, \sectref{sec:8.3.2}). There I point out that both types of elements share a number of traits but that they also seem to exhibit some differences. Although I make some very brief remarks of what might explain some of those differences, I will leave a more detailed exploration for another time.

\section{German as the language under investigation}\label{sec:1.2}

In this section, I explain why the focus of this book is on one language – German. To this end, I provide some brief cross-linguistic discussion of adjectival inflections and singular indefinite articles in German, Yiddish, and Norwegian. It is shown that these languages exhibit a number of striking differences. This includes evidence for the presence of the indefinite articles in plural contexts. Unfortunately, many of these and other non-canonical data points have not received much attention in the theoretical literature. In this book, I add many data points to the discussion of German, the language I am most familiar with, and analyze these constructions in detail. However, it is beyond the scope of this book to do this kind of empirical work for the other languages, to discuss the relevant constructions in detail, and to provide a thorough cross-linguistic analysis of all these languages.

\subsection{Some cross-linguistic differences}\label{sec:1.2.1}

Before I discuss some cross-linguistic data, I start with some brief general remarks.

\subsubsection{Canonical and non-canonical constructions}\label{sec:1.2.1.1}

Focusing on German, this book may seem restricted in scope. However, it strives to be more comprehensive than previous accounts by discussing adjectival inflections and the indefinite article \textit{ein} ‘a’ in very diverse structural contexts in that language. On the one hand, reference grammars (e.g., the German Duden) provide a fairly comprehensive survey of the relevant empirical domains. However, their goal is not to lay out a detailed theoretical analysis. On the other hand, theoretically oriented works often make insightful and elegant proposals of the canonical cases. These are simple DPs that involve the schematic structure “determiner + adjective(s) + noun”, where the three elements agree in case, number, and gender \REF{ex:1:9a}.\footnote{Simple DPs involving prenominal Saxon Genitives (e.g., \textit{Maries groß-er Bär} ‘Mary’s big-\textsc{st} bear’) are counted as canonical constructions here despite the fact that the Saxon Genitive itself does not agree in features with the following adjective and/or noun.} This is in contrast to more complex cases like \REF[b-c]{ex:1:9}, constructions that are frequently relegated to footnotes or even completely left out (for specific references, see \chapref{sec:2} and \chapref{sec:5}). Specifically, the string in \REF{ex:1:9b} is a complex nominal involving two overt head nouns, and \REF{ex:1:9c} is a complex nominal where the pronoun does not agree in number with the following adjective and noun (e\textsubscript{N} stands for null noun).

\ea%9
    \label{ex:1:9}
 German\\
  \ea\label{ex:1:9a}
  \gll der groß-e Bär\\
      the big-\textsc{wk} bear.\textsc{masc}\\
  \glt ‘the big bear’

  \ex \label{ex:1:9b}
  \gll das Sternbild [Groß-er {Bär}]\\
  the constellation {\db}big-\textsc{st}    bear.\textsc{masc}\\
  \glt ‘the constellation Great Bear’

\ex\label{ex:1:9c}
\gll {ihr} [jung-es {Gemüse] e\textsubscript{N}}\\
     you(\textsc{pl}) {\db}young-\textsc{st.sg} vegetable.\textsc{neut}\\
  \glt ‘you young folks’
  \z
\z

Given the characterization of simple DPs above, the latter two examples involve non-canonical structures. This book puts these non-canonical constructions on center stage. In fact, I argue that it is these types of constructions that reveal the true nature of adjectival inflections and \textit{ein}.\footnote{This idea has been echoed by other authors; for instance,   \citet[22]{SigurðssonWood2020} state that the exceptional cases may throw light on the more general instances.}

Before moving on, I clarify some more terminology. To distinguish the different nominals in \REF[b-c]{ex:1:9}, I use the terms matrix nominal and embedded nominal. Embedded nominals are marked in \REF[b-c]{ex:1:9} by brackets. They may consist of adjuncts \REF{ex:1:9b}, specifiers \REF{ex:1:9c}, and various complements. The structures without brackets are labeled matrix nominals. Note that each head noun projects its own nominal; that is, both \textit{Sternbild} ‘constellation’ and \textit{Bär} ‘bear’ in \REF{ex:1:9b} involve extended projections (\sectref{sec:1.4.1.1}).\footnote{Notice that if a chapter designation is not provided, the section or footnote is part of the current chapter (i.e., the reference to \sectref{sec:1.4.1.1} in the main text is shorthand for \chapref{sec:1}, \sectref{sec:1.4.1.1}).} Head nouns may also include null elements \REF{ex:1:9c}.

  As illustrated in the next subsection, languages, even closely related ones, show empirical differences in the domains of adjectival inflections and indefinite articles. This becomes clear when we compare German, Yiddish, and Norwegian in canonical as well as non-canonical structures.

\subsubsection{Cross-linguistic differences with adjectival inflections}\label{sec:1.2.1.2}

I start with a canonical construction where German, Yiddish, and Norwegian show the same inflectional patterns. In simple DPs, adjectives have weak inflections (WK) when they follow definite articles (the data in this subsection are taken from \citealt{Roehrs2015}; note also that Norwegian involves a second, suffixal determiner; for discussion, see \citealt{Julien2005a} among many others).

\ea%10
    \label{ex:1:10}
\ea
    German\\
\gll das groß-e Haus\\
    the big-\textsc{wk} house.\textsc{neut}\\
\glt ‘the big house’
\ex
    Yiddish\\
\gll dos groys-e hoyz\\
    the big-\textsc{wk}  house.\textsc{neut}\\
\glt ‘the big house’
\ex
    Norwegian\\
\gll det stor-e    hus-et\\
    the big-\textsc{wk} house.\textsc{neut}-\textsc{def}\\
\glt ‘the big house’
\z
\z

Differences emerge when adjectives are preceded by Saxon Genitives. Here, German patterns with Yiddish in exhibiting a strong ending (ST) on the adjective, but Norwegian has a weak inflection ($\emptyset$ indicates an assumed null ending).

\ea%11
    \label{ex:1:11}
\ea
    German\\
\gll Peters  groß-es Haus\\
    Peter’s big-\textsc{st}   house.\textsc{neut}\\
\glt ‘Peter’s big house’

\ex
    Yiddish\\
\gll Berls  groys-$\emptyset$ hoyz\\
    Berl’s big-\textsc{st}    house.\textsc{neut}\\
\glt ‘Berl’s big house’
\ex
    Norwegian\\
\gll Pers  stor-e   hus\\
    Per’s big-\textsc{wk} house.\textsc{neut}\\
\glt ‘Per’s big house’
\z
\z

Another distinction appears when adjectives follow distal demonstratives. In these cases, German patterns with Norwegian in showing a weak inflection on the adjective, but Yiddish has a strong ending (the stressed demonstrative in Norwegian, which is similar to the definite article in form, is capitalized).

\ea%12
    \label{ex:1:12}
\ea
    German\\
\gll jenes groß-e  Haus\\
    that   big-\textsc{wk} house.\textsc{neut}\\
\glt ‘that big house’
\ex
    Yiddish\\
\gll yents groys-$\emptyset$ hoyz\\
    that   big-\textsc{st}    house.\textsc{neut}\\
\glt ‘that big house’
\ex
    Norwegian\\
\gll DET stor-e   hus-et\\
    that  big-\textsc{wk} house.\textsc{neut}-\textsc{def}\\
\glt ‘that big house’
\z
\z

  As to non-canonical constructions, suffice it to consider one structure. In \REF{ex:1:13}, complex names are juxtaposed with their categorizing nominals. Again, the three languages do not pattern the same. Here, German and Yiddish contrast with Norwegian again.

\pagebreak
\ea%13
    \label{ex:1:13}
\ea
	German\\
\gll das Sternbild  Groß-er Bär\\
the constellation big-\textsc{st}    bear.\textsc{masc}\\
\glt ‘the constellation Great Bear’
\ex
    Yiddish\\
\gll der zhurnal    Sovetish-$\emptyset$ Heymland\\
    the magazine Soviet-\textsc{st}   homeland.\textsc{neut}\\
\glt ‘the magazine Soviet Homeland’
\ex
    Norwegian\\
\gll indianer-en Stor-e   Bjørn\\
   Indian-\textsc{def}  big-\textsc{wk} bear.\textsc{cmn}\\
\glt ‘the Indian Big Bear’
\z
\z

To sum up, it is clear that the three languages differ in their distribution of adjectival inflections in canonical and non-canonical constructions.

  \citet{RoehrsJulien2014} show that German and Norwegian differ quite generally. Documenting with nine sets of data (possessives involving proper names and pronominals, embedded and unembedded proper names, “dis-agreeing” pronominal DPs, appositives, definite adjectives, vocatives, and discontinuous noun phrases), they show that German consistently has strong endings on the adjectives but that Norwegian has weak endings on the adjectives. They propose that adjectival inflections in German are a function of lexical and structural factors (see \chapref{sec:2}) but that Norwegian is regulated by semantic ones (also \citealt{Harbert2007}: 131; \citealt{Lohrmann2011}; \citealt{Schoorlemmer2009,Schoorlemmer2012}, and others). \citet{Roehrs2015} points out that Yiddish is similar to German but differs in the lexical elements that bring about the weak inflections on the adjectives.

\subsubsection{Cross-linguistic differences with singular indefinite articles}\label{sec:1.2.1.3}
\largerpage[-1]
Consider the three languages as regards the singular indefinite articles. In German, the indefinite article \textit{ein} ‘a’ is part of the so-called \textit{ein}-words (a convenient term often used in textbooks on German, for instance, \citealt{RankinWells2016}). The latter comprise the indefinite article \textit{ein} ‘a’, the (stressed) singularity numeral \textit{EIN} ‘one’, the negative article \textit{kein} ‘no’, and possessive articles like \textit{mein} ‘my’, \textit{dein} ‘your’, etc. Given certain similarities (\chapref{sec:5}), these words are taken to be related to one another and are usually discussed together. Comparing the three languages as regards these types of words, we make the initial observation that German, Yiddish, and Norwegian show the same basic patterns – \textit{ein}-words precede their related head nouns (stressed singularity numerals that have the same spelling as their corresponding indefinite articles are distinguished by capitalization).
\ea%14
    \label{ex:1:14}
\ea\label{ex:1:14a}
  German\\
\gll ein / EIN / kein / mein Bruder\\
    a    / one  / no   / my    brother.\textsc{masc}\\
\glt ‘a/one/no/my brother’
\ex\label{ex:1:14b}
    Yiddish\\
\gll a / eyn / keyn / mayn bruder\\
    a / one / no    / my     brother.\textsc{masc}\\
\glt ‘a/one/no/my brother’
\ex\label{ex:1:14c}
    Norwegian\\
\gll en / EN / ingen / min bror\\
    a   / one / no     / my  brother.\textsc{cmn}\\
\glt ‘a/one/no/my brother’
\z
\z

Considering \REF[a]{ex:1:14}, we note that the singularity numeral, the negative article, and the possessive article are related to the indefinite article in German in that all these forms involve \textit{ein}. This is less straightforward with Yiddish and Norwegian in \REF[b-c]{ex:1:14}, where the possessive element, for instance, does not seem to relate to the indefinite article in a straightforward way. Besides this first indication that the three languages differ, Yiddish and Norwegian show other, syntactic distinctions.

  Yiddish is different from German and Norwegian in at least three aspects. Yiddish allows an inflected possessive pronoun to precede an indefinite article \REF{ex:1:15a}, the negative article can co-occur with the numeral for ‘one’ \REF{ex:1:15b}, and the – what looks like – inflected singularity numeral may precede an indefinite article \REF{ex:1:15c}. The following data are taken from \citet[54, 66]{Lockwood1995} and \citet[195, 205]{Weinreich1999}.

\ea%15
    \label{ex:1:15}
      Yiddish\\
\ea\label{ex:1:15a}
\gll mayn-er    a bruder\\
    mine-\textsc{infl} a brother.\textsc{masc}\\
\glt ‘a brother of mine’
\ex\label{ex:1:15b}
\gll keyn eyn land\\
    no    one country.\textsc{neut}\\
\glt ‘not a single country’
\ex\label{ex:1:15c}
\gll eyn-er     a mentsh\\
    one-\textsc{infl} a person.\textsc{masc}\\
\glt ‘a certain person’
\z
\z

I argue in \citet{Roehrs2022} that the inflected possessive pronoun in Yiddish \REF{ex:1:15a} involves a second, separate nominal that is in a higher position than the possessive articles in Yiddish \REF{ex:1:14b}, in German \REF{ex:1:14a}, and in Norwegian \REF{ex:1:14c}. This means that \REF{ex:1:15a} presents a non-canonical construction. Turning to \REF{ex:1:15b}, \textit{eyn} is syntactically optional. Semantically, it seems to intensify the negation (if present). As for \REF{ex:1:15c} and similar to \REF{ex:1:15a}, Yiddish most likely involves a structure different from the \textit{ein}-words in the three distributions illustrated in \REF{ex:1:14}. Extending the discussion to \textit{epes a khaver} ‘(something a =) some friend’, it is proposed in the aforementioned paper that \textit{eyner} and \textit{epes} are in a higher position similar to the possessive pronoun in \REF{ex:1:15a}.

  Norwegian is also different from German and Yiddish; for instance, it tolerates the indefinite article between two adjectives and between adjectives and nouns (data are from \citealt{Julien2002}: 269).

\ea{%16
    \label{ex:1:16}}
[Nynorsk] Norwegian\\
\sn[?]{
\gll   eit stor-t  eit styg-t    eit hus\\
        a   big-\textsc{st} an ugly-\textsc{st} a   house.\textsc{neut}\\
\glt ‘a big ugly house’
}
\z

\citet[270]{Julien2002} argues that the articles in \REF{ex:1:16} cannot be interpreted as adjectival agreement – note that each adjective has a strong inflection. She proposes that the lower instances of the article are in lower head positions of the nominal structure (Julien’s $\alpha$) and that the adjectival inflections are part of complex specifiers that contain the adjectives (see \sectref{sec:1.4.1.3}).

To conclude, this section has shown that adjectival inflections and indefinite articles are cross-linguistically similar but crucially not identical. Illustrating with German, Yiddish, and Norwegian, this could be observed in both canonical and non-canonical constructions. This conclusion is strengthened when we look at indefinite articles occurring in plural contexts – again, we will see that languages do not pattern the same.

\subsection{The indefinite article in plural contexts}\label{sec:1.2.2}

One of the proposals of this book is that the indefinite article \textit{ein} ‘a’ is not a reflex of singular number. This is probably the most controversial claim of these pages. After some brief cross-linguistic remarks, I provide empirical evidence that shows that \textit{ein} can indeed occur in various plural contexts in many German dialects, both in the north and in the south of central Europe. In my view, this shows that \textit{ein} is not a reflex of singular number. In the second subsection, I discuss some of the linguistic properties of these non-singular instances.

\subsubsection{Occurrence}\label{sec:1.2.2.1}

The indefinite article in plural contexts is perhaps most prominently discussed by \citet{BennisEtAl1998} for Dutch, where this element occurs in various contexts, in \textit{wh}-exclamatives, non-\textit{wh}-exclamatives, and \textit{N-of-a-N} constructions (\REF[a,c]{ex:1:17} are taken from \citealt[98, 101]{BennisEtAl1998}; \REF{ex:1:17b} is from \citealt[165]{vanRiemsdijk2005}). This plural element is often referred to as spurious article.

\ea%17
    \label{ex:1:17}
    Dutch\\
\ea\label{ex:1:17a}
\gll Wat *(een) jongens!\\
    what   {\db\db}a      boys\\
\glt ‘What boys!’
\ex\label{ex:1:17b}
\gll Die auto heeft een deuken!\\
    that car   has   a     dents\\
\glt ‘That car has dents!’
\ex\label{ex:1:17c}
\gll idioten van (een) mannen\\
    idiots   of     {\db}a      men\\
\glt ‘idiots of men’
\z
\z

It is perhaps less well known that this article also occurs in other Germanic languages, specifically in the three languages discussed above. We see below that all three languages have indefinite articles in plural contexts in \textit{wh-}constructions. While the occurrence of the article is restricted to that context in Yiddish, the other languages have more options but differ from each other; for instance,\linebreak Swedish\footnote{Due to lack of data for Norwegian, I discuss closely related Swedish here. \citet[33]{Delsing1993} mentions though that the indefinite article is also possible in plural contexts in Norwegian dialects. Unfortunately, the referenced works in Delsing are currently not accessible to me. Note that Icelandic also allows the indefinite article to occur in certain plural contexts \citep[81]{Pétursson1992}.} has this article also in predicative contexts, northern dialects of German also in constructions involving \textit{so} ‘such’, and southern dialects of German have plural \textit{ein} also in argumental expressions without \textit{so} ‘such’. After some brief remarks about Yiddish and Swedish, I engage in a more detailed discussion of German.

Yiddish has indefinite articles in plural contexts in \textit{wh-}constructions (\REF{ex:1:18a} is taken from \citealt{Jacobs2005}: 188; \REF{ex:1:18b} is from \citealt{Lockwood1995}: 55).\footnote{In Olsvanger’s collection of stories, there is a second example, similar to \REF{ex:1:18c}, that also shows the verb in the singular (his page 172). In German, the verb would have to be in the plural to yield a grammatical example.}

\ea%18
    \label{ex:1:18}
    Yiddish\\
\ea\label{ex:1:18a}
\gll vos    far a bikher\\
     what for  a books\\
\glt ‘what kind of books’
\ex\label{ex:1:18b}
\gll voser      a shprakhn\\
    what.for a languages\\
\glt ‘what kind of languages’
\ex\label{ex:1:18c}
\gll vos   iz dos  far a verter.\\
  what is that for a words\\
\glt ‘what kind of words are these’

(from \citeauthor{Olsvanger1947}’s \textit{Röyte pomerantsen}, p. 98)
\z
\z

In addition, this element has also been identified in Swedish. \citet[33-35, 143-44]{Delsing1993} points out that it occurs in \textit{wh-}constructions \REF{ex:1:19a}. Delsing states that in predicative contexts, the article is “nearly obligatory” \REF{ex:1:19b}. Note that it can also surface between an adjective and a noun \REF{ex:1:19c}. Delsing points out that these are non-argumental articles.

\ea%19
    \label{ex:1:19}
\ea \label{ex:1:19a}
Colloquial Swedish\\
\gll Vad är  ni    för ena filurer?\\
    what are you for a     slyboots\\
\glt ‘What kind of slyboots are you?’

\ex \label{ex:1:19b}
\gll Pelle och Lisa är *(ena) idioter\\
    Pelle and Lisa are   {\db\db}a      idiots\\
\glt ‘Pelle and Lisa are idiots.’

\ex \label{ex:1:19c}
Northern Swedish\\
\gll Dänna      var    he     stor a husa.\\
    over.there were there big  a  houses\\
\glt ‘There were big houses over there.’
\z
\z

It is less clear if this article occurs in German. While its existence is sometimes played down or even denied (e.g., \citealt{Schoorlemmer2009}: 238 fn. 50), I argue that it is beyond doubt that \textit{ein} occurs in plural contexts in this language as well. Having said that, there seems to be a difference between northern and southern dialects of German. As far as I have been able to establish, there are three syntactic contexts in Northern German where plural \textit{ein} occurs with a following overt noun. Similar to Yiddish and Swedish, German has plural \textit{ein} in the \textit{wh-}construction \REF{ex:1:20a}. Furthermore, these dialects seem to have cases like \REF{ex:1:20b} (cf. \REF{ex:1:17b} in Dutch above). The latter type of example might be related to \REF{ex:1:20c}, at least in some subdialects.\footnote{There is another context in which \textit{ein} can appear in plural contexts.
	\ea
	\gll die einen, die anderen\\
	the ones the others\\
	\glt `these, those’
	\z

However, this \textit{ein} is of a different type – it is adjectival (see \chapref{sec:5}).}

\ea%20
    \label{ex:1:20}
    Northern German\\
\ea\label{ex:1:20a}
\gll Was  für’ne / Was   für eine Idioten!\\
    what {for a}    / what for a      idiots\\
\glt ‘What kind of idiots!’
\ex\label{ex:1:20b}
\gll Eine Idioten!\\
    a      idiots\\
\glt ‘Idiots!’
\ex\label{ex:1:20c}
\gll So’ne / So eine Idioten!\\
    {so a}    / so a     idiots\\
\glt ‘Such idiots!’
\z
\z

Note that all these forms involve exclamative contexts. As will become clear though, \REF{ex:1:20c} can also occur in argument positions in Northern German (see also the examples from Southern German below, where these strings occur in argument position without \textit{so} ‘such’). For the most part, I will focus on the exclamative constructions.

There is no doubt that \textit{’ne} in \REF{ex:1:20a} and \REF{ex:1:20c} stands for \textit{eine}. In an online search (for details, see below), I identified 152 examples involving plural \textit{ein}. These examples are collected in the Appendix, Sections A.1-9. Of these 152 tokens, 21 involved the full form \textit{eine}, 19 with \textit{was} \emph{für} and 2 with \textit{so} ‘such’ (both examples below are taken from the Appendix).

\ea%21
    \label{ex:1:21}
\ea\label{ex:1:21a}
\gll ``Was   für eine Idioten" sagte Liam\\
      {\db}what for  a     idiots      said  Liam\\
\glt `{``}Such idiots!'', said Liam.'

\ex\label{ex:1:21b}
\gll Das Personal war sehr unfreundlich. Ich hatte noch nie     so eine Ferien.\\
    the  staff         was very unfriendly     I     had    still  never so a      holidays\\
\glt `The staff was very unfriendly. I have never had such a (bad) vacation.'
\z
\z




Indeed, \textit{’ne} is a typical reduction of the indefinite article as can readily be seen in the nominative feminine: \textit{eine Frau} > \textit{’ne Frau} ‘a woman’. Note that \textit{so’ne} can also be spelled as \textit{so ne} or \textit{sone}.\footnote{While spelling can only be used cautiously as evidence, notice that the speakers of the examples listed in the Appendix, Sections A.1-9 spell \textit{so} and \textit{eine}, when the latter is reduced, as \textit{so ne} (61 times) or \textit{so’ne} (26 times). It is perhaps telling that not a single speaker provided the written form \textit{sone}. In my view, this hints at the fact that these speakers analyze the two elements as separate items, similar to \textit{so eine} (2 times).} I refer to \textit{so} and plural \textit{ein} by using the first variant (\textit{so’ne}).

In more detail, I conducted an online seach of the three cases in \REF{ex:1:20} above through Google, on Twitter, and on Facebook. All three cases in \REF{ex:1:20} occur with different frequencies: While \REF{ex:1:20b} is hard to search for, \REF{ex:1:20c} seems to be the most common construction out of the three in Northern German. The numeric results are provided in \tabref{tab:1:1a} (Adj-\textit{en} stands for a weak adjective ending in -\textit{en}). Note though that it would be easy to find more examples, especially for columns four and five of \tabref{tab:1:1a}, where I restricted my search to certain nouns, numerals, or adjectives (again, all these hits are listed in the Appendix, Sections A.1-9).\footnote{When
    searching for these types of examples, care must be taken; for instance, many cases involve false positives where a following plural noun actually forms the first part of a singular compound.
	\ea
	\gll Was für ne Männer Wg?\\
	what for a men living.community.\textsc{fem}\\
	\glt `What kind of a living community for men?'
	\z 
	
	The examples in the Appendix involve genuine plural nouns (which are not part of compounds).
}




In the Appendix, Sections A.1-9, the examples are provided in the order of the rows and columns of \tabref{tab:1:1a}, with the relevant strings in italics for ease of identification (no glosses or translations are given there). Unless indicated otherwise, all examples below involving \textit{ein} in plural contexts are authentic; that is, these data are taken from the Appendix, Sections A.1-9.
\largerpage
In addition to finding data in online media, I conducted a search of plural \textit{ein} in \textit{Datenbank für Gesprochenes Deutsch} (DGD; Database of Spoken German; available online at \url{https://dgd.ids-mannheim.de/}). The DGD is a convenient platform comprising a number of spoken corpora that can be searched individually or collectively. This search yielded a total of 60 instances of plural \textit{ein}, broken down according to corpus and construction in \tabref{tab:1:1b}.\footnote{It is unlikely that this search was exhaustive, given the size of the corpora and the fact that plural \textit{ein} is spelled in the DGD in different ways. Transcribers of the DGD, who presumably have training in linguistics, spell \textit{so} and \textit{ne} equally as two separate words like \textit{so ne} (8 times), \textit{so-ne} (18 times), \textit{so\_ne} (1 time; total count: 27) or as one word like \textit{sone} (27 times). However, there may be other variants. Note that when searching for examples, researchers should not enter \textstylew{\textit{für-ne}} or \textit{so-ne} (or \textstylew{\textit{für\_ne}} or \textit{so\_ne}, for that matter) in the search bar: While some data points are orthographically represented like this in some of the texts, they can only be found by replacing the hyphen (or underscore) with a space (i.e., \textstylew{\textit{für ne}}, \textit{so ne}). Finally, it should also be pointed out that a number of very similar hits and a few unclear ones were removed from the results.} To allow a direct comparison, \tabref{tab:1:1b} is organized in the same way as \tabref{tab:1:1a} above (further below, I comment on the strong adjectives marked by ST in parentheses in \tabref{tab:1:1b}).


\begin{table}[t]
\caption{\label{tab:1:1a} Numeric results of plural \textit{ein} identified in online media (September 20, 2020)}
\fittable{
\begin{tabular}{lrrrr}
\lsptoprule
 & \textit{was für (ei)ne} & \textit{was für (ei)ne} Adj-\textit{en} & \textit{so (ei)ne} & \textit{so (ei)ne} Adj-\textit{en}\\
\midrule
% Source &  &  &  &\\
Google & 19 & 12 & 4\footnote{with the  noun \textit{Ferien}  ‘holidays’.} & 33\footnote{23 (with the  adjective \textit{geil}  ‘awesome’) and 10 (with  s\emph{ü}\textit{ß} ‘cute’).}\\
Twitter & 28 & 4 & 3\footnote{with the  numeral \textit{zwei} ‘two’ following.} & 44\footnote{23 (with the  adjective \textit{geil}  ‘awesome’) and 21  (with \textit{dumm}  ‘stupid’).}\\
Facebook & hard to  search & hard to  search & hard to  search~ & 5~~~\\
\midrule
Total: 152 & 47 & 16 & 7~~~ & 82~\\
\lspbottomrule
\end{tabular}
}
\end{table}

% \begin{table}
% \caption{\label{tab:1:1a} Numeric results of plural \textit{ein} identified in online media (September 20, 2020)}
% \begin{tabularx}{\textwidth}{lYYQQ}
% \lsptoprule
% String & \textit{was für (ei)ne} & \textit{was für (ei)ne} Adj-\textit{en} & \textit{so (ei)ne} & \textit{so (ei)ne} Adj-\textit{en}\\
% \midrule
% Source &  &  &  &\\
% Google & 19 & 12 & 4 (with the  noun \textit{Ferien}  ‘holidays’) & 23 (with the  adjective \textit{geil}  ‘awesome’) and 10 (with  s\emph{ü}\textit{ß} ‘cute’)\\
% Twitter & 28 & 4 & 3 (with the  numeral \textit{zwei} ‘two’ following) & 23 (with the  adjective \textit{geil}  ‘awesome’) and 21  (with \textit{dumm}  ‘stupid’)\\
% Facebook & hard to  search & hard to  search & hard to  search & 5\\
% \midrule
% Total: 152 & 47 & 16 & 7 & 82\\
% \lspbottomrule
% \end{tabularx}
% \end{table}


\begin{table}
\caption{\label{tab:1:1b} Numeric results of plural \textit{ein} identified in DGD (October 6, 2024)}
\begin{tabularx}{\textwidth}{Xrrrr}
\lsptoprule
String & \textit{was für (ei)ne} & \textit{was für (ei)ne} \newline Adj-\textit{en} & \textit{so (ei)ne} & \textit{so (ei)ne} \newline Adj-\textit{en}\\
\midrule
Source\footnote{\small Corpora:\\
\begin{tabularx}{\textwidth}{@{}l@{\, =~}Q}
    AD    &  Australiendeutsch (Australian German)\\
    BW    &  Berliner Wendekorpus (Berlin Corpus of the Peaceful Revolution)\\
    BR    &  Biographische und Reiseerzählungen (Biographical and Travel Stories)\\
    DH    &  Deutsch heute (German Today)\\
    DNAM  &  Deutsch in Namibia (German in Namibia)\\
    FEGB  &  Flucht und Emigration nach Großbritannien (Escape and Emigation to Great Britain)\\
    FOLK  &  Forschungs- und Lehrkorpus Gesprochenes Deutsch (Research and Teaching Corpus of   Spoken German)\\
    GWSS  &  Gesprochene Wissenschaftssprache Kontrastiv (Spoken Scientific Language   Contrastively)\\
    MEND  &  Mennonitenplautdietsch in Nord- und Südamerika (Mennonite Low German in North   and South America)\\
    RUDI  &  Russlanddeutsche Dialekte (Russian German Dialects)\\
    ZW    &  Zwirner-Korpus (Corpus named [after] Zwirner)\\
\end{tabularx}
    } &  &  &  &\\
AD & 2 &  &  & 1(ST)\\
BW & 1 &  & 18 & 9\\
BR &  &  &  & 1\\
DH &  &  & 2 &\\
DNAM & 3 &  & 6 &\\
FEGB &  &  & 1 &\\
FOLK &  &  & 5 & 3\\
GWSS &  &  & 1 &\\
MEND &  &  & 2 &\\
RUDI &  &  & 1 & 1(ST) / \newline 1(ST \textit{so})\\
ZW &  &  & 1 & 1\\
\midrule
Total: 60 & 6 &  & 37 & 17\\
\lspbottomrule
\end{tabularx}
\end{table}

Similar to the search of online media, almost all relevant speakers in the DGD are from the northern part of Germany (to the extent available, this type of information is provided with all the retrieved examples in the Appendix, Section A.10).\text{} About 45 percent of the examples involving plural \textit{ein} are contained in the Berlin Corpus alone – all speakers there are from Berlin and its surrounding areas. In fact, some speakers from other corpora are also from this region. Note that in general, speakers are of different genders and ages (some born as early as 1879).

The examples in the Appendix, Section A.10 are sorted according to the name of the corpus they appear in, and the corpora are provided in alphabetical order. The examples are given in the order of the columns of \tabref{tab:1:1b}. As above, the relevant strings are provided in italics, and no glosses or translations are given. Some comments on the numeric results in \tabref{tab:1:1b} are in order.

Note that there are relatively few instances (six) involving plural \textit{ein} in \textstylew{\textit{was-für}} contexts, and some of these instances are special in other ways. While one example is from a speaker in the Berlin Corpus (no specific information about this speaker is available), all five other examples were recorded in areas outside Germany – one adult speaker was born in Germany but lives now in Australia, one speaker is probably an interviewer from Germany making the recordings in Australia, and three speakers were children born in Namibia where at least one parent speaks German. One may wonder now why there are not more examples of this type. This probably has to do with the fact that these constructions are less frequent than those involving \textit{so’ne} in general (see also \tabref{tab:1:1a}) and the fact that plural \textit{ein} in \textit{was-für} constructions typically involves strongly emotive language (e.g., swear words) used in exclamative contexts. None of the six examples are of this type, and the corpora in the DGD do not seem to involve many such instances in general (perhaps due to the fact that these are recorded conversations).

  As to the three strong adjectives, there are two types to distinguish: two instances marked by (ST) vs. one instance indicated by (ST \textit{so}) in \tabref{tab:1:1b}. Starting with the first type, one of the two examples was found in the corpus \textit{Australiendeutsch} uttered by a person born in Australia with ancestors from Silesia; the other was identified in the corpus \textit{Russlanddeutsche Dialekte} said by a person born in the Ukraine. Regarding the second type, this example was retrieved from the corpus \textit{Russlanddeutsche Dialekte} uttered by the same person just mentioned where, crucially, the strong adjective precedes \textit{so’ne}. While the strong adjective in the latter case is expected (there is no preceding determiner), the first two instances are in contrast to what I have generally encountered in my searches. Having said that, these two cases presumably involve speakers where German is not the dominant language, and I will move forward assuming that only weak adjectives occur in these contexts.

These two searches totaled over 210 examples – all listed in the Appendix. Furthermore, there are a few examples that were identified in other ways as indicated further below (they are not part of the example counts in the tables above). Besides that, I also consulted some non-linguist friends from the county of Oberhavel (north of Berlin, Germany). All allowed \REF{ex:1:20c}, one found \REF{ex:1:20b} to be fine, but most stated that \REF{ex:1:20a} was somewhat marked. The individual who allowed \REF{ex:1:20b} took it to be a short form of \REF{ex:1:20c}.

  Observe now that the occurrence of \textit{ein} in plural contexts has already been established independently.  \citet[389-394]{ElmentalerRosenberg2015} document cases of plural \textit{so’ne} ‘(so’a =) such’ in \textit{Norddeutscher Sprachatlas} (Northern German Language Atlas).\footnote{The existence of plural \textit{ein} is particularly clear in the central and eastern parts of this area (in the western part, the plural form \textit{so} as in \textit{so} \textit{Dinge} ‘(so =) such things’ is more prevalent).  \citet[383]{ElmentalerRosenberg2015} also point out that the full form (\textit{so eine Dinge}) is not possible. I disagree. As mentioned above, I found a few examples of this type. In fact, if the indefinite element is stressed, as is fairly common in exclamative contexts, then unreduced \textit{eine} is certainly possible (and indeed required; see some of the \textstylew{\textit{was-für}} constructions).} Below are two examples from \citet[392]{ElmentalerRosenberg2015}, who identified this element in both interviews and (dinner) table conversations.

\ea%22
    \label{ex:1:22}
    Northern German\\
\ea\label{ex:1:22a}
\gll so’ne Kurse\\
    {so a}   courses\\
\glt ‘such courses’
\ex\label{ex:1:22b}
\gll so’ne komisch-en Löcher\\
    {so a}   weird-\textsc{wk}         holes\\
\glt ‘such weird holes’
\z
\z

These authors agree with \citet[6-7]{Keller2004} that \textit{so’ne} fills a gap by forming an indefinite demonstrative. Note though that this idea cannot extend to the other constructions in \REF[a-b]{ex:1:20} (which lack \textit{so}). As alluded to above, I came across examples in my own search with clear geographical indications that these types of cases exist in Berlin, Hamburg, and Lübeck. In addition, linguistic features such as unshifted consonants (e.g., \textit{wat} for \textit{was} ‘what’) indicate the same geographical area.

  Besides northern dialects of German, plural \textit{ein} also occurs in southern dialects like Swabian and Bavarian \citep[106]{Glaser1993}. While its use and distribution are not the same across all subdialects, it is clear that plural \textit{ein} does occur \REF{ex:1:23a}. The latter is a Bavarian example taken from \citet[109]{Glaser1993}. Hubert Haider (p.c.) points out that these cases also exist in Southern Bavarian, for instance, in Carinthia, Austria \REF{ex:1:23b}. Another southern example was identified during my search of online media \REF{ex:1:23c}.\footnote{Note that second-person plural \textit{es} ‘you’ in \REF{ex:1:23b} derives from an old dual pronoun meaning ‘you (two)’ \citep[222]{PaulEtAl1989}. The -\textit{s} on the verb \textit{brauchen} ‘need’ is an agreement marker sharing features with \textit{es}. I glossed the -\textit{s} on the verb \textit{haben} ‘have’ in \REF{ex:1:23c} as an encliticized \textit{es}. I thank David Fertig and Mark Louden for discussion of these points. Notice also that Southern Bavarian \textit{ane} ‘a.\textsc{pl}’ has found its way into the media (\url{https://www.diepresse.com/443923/karntnerisch-fur-anfanger}) and even into advertisements (\href{https://pensionria.at/woerthersee-kaerntnerisch/}{{https://pensionria.at/woerthersee-kaerntnerisch/}}). I thank Hubert Haider for providing these sources to me. In addition, the online Austrian dictionary discusses some relevant examples (\url{https://www.ostarrichi.org/wort/15577/ane_oane}).}

\ea%23
    \label{ex:1:23}
    Bavarian German\\
\ea\label{ex:1:23a}
\gll Dǫ    sàn õa Epfe   drõ.\\
    there are a   apples there.on\\
\glt ‘There are apples on it.’
\ex\label{ex:1:23b}
\gll Brauchts  es~  ane Untatatzalan oda tans de  Schalalan alaan aa?\\
need.\textsc{agr} you a     saucers          or   do.it the cups          alone too\\
\glt ‘Do you need saucers or will the cups by themselves do?’
\ex\label{ex:1:23c}
\gll Was  für ne Vögel habts       denn da     aufgetrieben?\\
    what for a   idiots have.you \textsc{prt}   there found\\
\glt ‘What kind of idiots have you got there?’
\z
\z

Unlike many of the instances in Northern German, the cases in \REF[a-b]{ex:1:23} involve examples in argument positions (that also lack \textit{so} ‘such’).

Before moving on, it is worth pointing out that the occurrence of this \textit{ein} in Bavarian fits well with the Yiddish facts above. Notice in this respect that the plural indefinite article in Yiddish cannot be an innovation from Slavic (e.g., Polish or Russian) as these languages do not have articles to begin with. However, as is well known, Yiddish is historically related to southern dialects of German, specifically Bavarian (for some discussion, see \citealt{Jacobs2005}: Chapter 2).

Indeed, diachronically, plural \textit{ein} is not a recent development as it occurred already in older varieties of German with a specific indefinite interpretation (ENHG stands for Early New High German).\footnote{The indefinite article began to evolve from the singularity numeral and indefinite pronoun \textit{ein} in Old High German (\citealt{BrauneReiffenstein2004}: 234). Consequently, I gloss \textit{ein} as ‘one/a’ in Old High German and Middle High German. Also, singular \textit{ein} in the older varieties is often paraphrased using Modern German \textit{irgendein, ein gewisser} (e.g., \citealt{BrauneReiffenstein2004}: 234). This could be rendered into English as ‘a certain’ in the singular and as ‘certain’ in the plural. Finally, I thank Chris Sapp for providing the example from ENHG. He points out though that this sentence could also have a singular interpretation of \textit{eyne} (i.e., ‘…with respect to the 15 families, one with the assessors and those who have sat on the council.’), but it seems unlikely that all of the assessors and city council members came from one particular family.}

\ea%24
    \label{ex:1:24}
    Old High German\\
\ea \label{ex:1:24a}
\gll eino   ziti\\
    one/a times\\
\glt ‘(certain) times’
\ex \label{ex:1:24b}
\gll in einēn buachon\\
    in one/a books\\
\glt ‘in (certain) books’
(both from Otfrid, \citealt[234]{BrauneReiffenstein2004})
\z
\z

\ea%25
    \label{ex:1:25}
    Middle High German\\
\ea\label{ex:1:25a}
\gll in einen zîten\\
    in one/a times\\
\glt ‘during (certain) times’
      (Der Nibelunge Nôt 1143:1, \citealt[388]{PaulEtAl1989})
\ex\label{ex:1:25b}
\gll einu   liute\\
    one/a people\\
\glt ‘(certain) people’
      (Himmelreich Jerusalem 14:12, \citealt[388]{PaulEtAl1989})
\z
\z

\ea%26
    \label{ex:1:26}
ENHG\\
\gll Dyt sijnt alle alsulge sachen vnd geschichte, as sich  dese  nyeste. xxxvj. Iair her  enbynnen der Stat van Coelne   oeurmitz      die. xv. geslechte eyne mit   den Scheffen  vnd den ghenen, die   mit     yn raide   gesessen haint, ergangen haint.\\
  that are all such things and history as \textsc{refl} these recent  36  years  here in the city of   Cologne by.means.of the 15  families a with the assessors and the those who along in council sat have  happened have\\
\glt ‘Those are all such things and history, as has happened in the last 36 years here in the city of Cologne with respect to the 15 families, some with the assessors and those who have sat on the council.’ (1360 Hauwe Neues Buch Cologne, 33)
\z

If these data are indeed related, then plural \textit{ein} has been in the language for a long time.\footnote{It is not clear if the development of the spurious article in Modern German is a direct development from the plural forms in the older varieties. A reviewer states that \textit{so’ne} may not have originally come from \textit{so eine}, but may be a phonetically reduced form of \textit{solche}. Specifically, in the 19th century, Georg Wenker had 40 Standard German sentences (“Wenkersätze”) translated into local dialects by schoolteachers and their pupils (a total of 40,000 schools; this questionnaire and its replies are available online at: \href{https://apps.dsa.info/wenker/}{{Wenkerbogen-Transliterations-App (dsa.info)}}). Sentence 28 was \textit{Ihr sollt nicht solche} \textit{Kindereien treiben} ‘You should not engage in such childishness’. The determiner \textit{solche} is regularily translated as \textit{soon}, \textit{sone}, \textit{sonne}, \textit{sunne}, \textit{sön’n}, \textit{soon}, \textit{son’n} etc. in the Low German area and even in some of the Middle German areas. Sentence 36 was \textit{Was sitzen da für Vögelchen …} ‘What kind of little birds are sitting there …’. The replies to the \textit{was}-\textit{für} question may already show an enclitic indefinite plural article or an inflected preposition in some of the dialects of the Ruhr valley (town of Benrath: \textit{Wat setten door fün} \textit{Voskes…}; town of Bredeney: \textit{Wat setten do von} \textit{Vögelsches…}). Thus, the dialectal forms of \textit{solche} possibly have been reanalyzed as \textit{so} and \textit{eine}, and \textit{eine} – in some regions – has been reanalyzed as a spurious plural article and has spread to \textit{was-für} contexts. Given these two possible paths of development (direct or indirect via reanalysis), it is clear that more work is needed here to make a decision about the diachrony of plural \textit{ein}.}

  It seems clear that these cases occur in colloquial contexts, in contemporary speech and writing. This is even acknowledged in Duden, both the dictionary and the grammar. Without indication of the dialect, these works state that \textit{so’ne} ‘such’ occurs in colloquial German (\REF[a]{ex:1:27} is taken from \citealt{Duden1989}: 1414 and \REF{ex:1:27b} from \citealt{Duden1995}: 276).

\ea%27
    \label{ex:1:27}
\ea\label{ex:1:27a}
\gll es gibt   immer  sone und solch-e\\
    it  gives always so.a  and such-\textsc{pl}\\
\glt ‘There are all kinds.’
\ex\label{ex:1:27b}
\gll Ich kann sone Leute   nicht leiden.\\
I     can   so.a  people not    stand\\
\glt ‘I cannot stand such people.’
\z
\z

  As regards writing, these constructions occur mostly in unedited texts like Twitter messages. However, they can also be found in edited contexts like ebooks such as \textit{Hilfe ich bin’n Vampir} ‘Help, I am a Vampire’ \REF{ex:1:28a} or \textit{Wolfskinder} ‘Wolf Children’ \REF{ex:1:28b}.

\ea%28
    \label{ex:1:28}
\ea \label{ex:1:28a}
\gll So ne Mädchen gehn mir auf die Nerven.\\
    so  a   girls        go     me  on  the nerves\\
\glt ‘Such girls get on my nerves.’
(\url{https://www.bookrix.de/book.html?bookID=cronalein_1360707129.5978798866#3276,468,12258})
\ex \label{ex:1:28b}
\gll Smilla recherchiert und entdeckt,  das schon   so eine Frauen verschwanden.\\
Smilla researches    and discovers that already so a     women  disappeared\\
\glt ‘Smilla does some research and discovers that such women already disappeared.’
(\url{https://www.vorablesen.de/buecher/wolfskinder/rezensionen/duester-a6154566-6616-445c-967e-966d6578811c})
\z
\z

Recall that these examples are not counted in the tables above and that they are not listed in the Appendix. The following are some linguistic properties of these instances involving plural \textit{ein}.

\subsubsection{Linguistic properties}\label{sec:1.2.2.2}

These strings are clearly in the plural: \textit{Ein} itself has a plural inflection, the following adjective has the typical weak ending, and the noun has a plural inflection along with the corresponding meaning. Compare \REF{ex:1:29a} to \REF{ex:1:29b}.

\ea%29
    \label{ex:1:29}
\ea \label{ex:1:29a}
    \gll Was  für ein-e geil-en Bild-er\\
    what for a-\textsc{st}  awesome-\textsc{wk} picture-s\\
\glt ‘Such awesome pictures!’
\ex\label{ex:1:29b}
\gll dies-e     geil-en           Bild-er\\
    these-\textsc{st} awesome-\textsc{wk} picture-s\\
\glt ‘these awesome pictures’
\z
\z

Furthermore, verbs and pronouns agree with the plural noun phrase in number.

\ea%30
    \label{ex:1:30}
\ea\label{ex:1:30a}
\gll Was  für eine Idioten wohn-en nur in Sachsen-Anhalt?\\
    what for a      idiots   live-\textsc{pl}    \textsc{prt} in Saxony-Anhalt\\
\glt ‘What kind of idiots live in Saxony-Anhalt?’
\ex\label{ex:1:30b}
\gll was   für eine Idioten, die        so Aufmerksamkeit woll-en\\
    what for a      idiots     who.\textsc{pl} so attention             want-\textsc{pl}\\
\glt ‘what kind of idiots that want attention like this’
\z
\z

Observe that these cases do not involve syntactically frozen (or fossilized) templates: They can involve several adjectives in a row \REF{ex:1:31a}, coordinated adjectives \REF{ex:1:31b}, and nominalized adjectives \REF{ex:1:31c}.

\ea%31
    \label{ex:1:31}
\ea\label{ex:1:31a}
\gll Wieso gibt   es so ne dummen, bescheuerten Leute   die  sowas          machen? \\
    why    gives it  so a  stupid      dumb             people who so.something do\\
\glt ‘Why are there such stupid, dumb people that do something like that?’
\ex\label{ex:1:31b}
\gll es gibt  halt so ne dummen und abgehobenen leute\\
  it gives \textsc{prt} so a   stupid     and stuck-up         people\\
\glt ‘There are indeed such stupid and stuck-up people.’
\ex\label{ex:1:31c}
\gll {ach  so'ne dummen gibt   es öfter. }\\
    oh    {so a}   stupid     gives it  more.often\\
\glt ‘Oh, there are often such stupid ones.’
\z
\z

Note in this regard that plural \textit{ein} is not an adjectival article, an article triggered by the presence of an adjective. This is clear from the fact that the cases without an adjective outnumber those with an adjective at a ratio of about 2.5 to 1 (see \tabref{tab:1:1b} above).

  Additionally, these types of cases may include numerals \REF[a-b]{ex:1:32} or other preceding elements such as \textit{irgend} ‘any’ \REF{ex:1:32c}.

\ea%32
    \label{ex:1:32}
\ea\label{ex:1:32a}
\gll drei so ne geilen Mädels\\
    three so a   awesome girls\\
\glt ‘three such awesome girls’
\ex\label{ex:1:32b}
\gll Da    waren heute 2 so'ne dummen Weiber beim  Zumba-Kurs,   die\\
  there were   today 2 {so a}   stupid     women at.the Zumba-course who\\
\glt   ‘Today, there were two such stupid women at the Zumba course who’
\ex\label{ex:1:32c}
\gll Alter irgend so ne dummen Hunde haben direkt    neben  mir n Böller   gezündet\\
  man  any     so a   stupid     dogs    have   directly next.to me a firecracker lit\\
\glt   ‘Boy, some such stupid idiots lit a firecracker right next to me.’
\z
\z
\largerpage
Indeed, numerals can also follow \textit{so’n}e.\footnote{I have not found any examples involving numerals in \textit{was-für} constructions. Note though that while \REF{ex:1:18:1a} sounds possible to my ears, \REF{ex:1:18:1b} is completely out.
	\ea
	\ea \label{ex:1:18:1a}
	\gll was  für 'ne zwei Typen\\
	what for {  }a  two  guys\\
	\glt ‘what kind of two guys’
	\ex[*]{ \label{ex:1:18:1b}
	\gll was   für zwei (ei)ne Typen\\
	what for two   a        guys\\
	}
	\z
	\z
}

\ea%33
    \label{ex:1:33}
\ea\label{ex:1:33a}
\gll ach man wir sind schon so ne zwei wärmflaschen\\
    oh  man   we are   \textsc{prt}     so a   two  warm.bottles\\
\glt ‘Oh boy, we are such two hot-water bottles.’
\ex\label{ex:1:33b}
\gll und so ne zwei Kämpfer die   alles umpflügen,\\
    and so a   two  fighters   who all     under.plough\\
\glt ‘and such two guys, who create such chaos’
\z
\z

Finally, as we might expect, \textit{was-für} splits are also possible. Compare \REF{ex:1:34a} to \REF{ex:1:34b}.

\ea%34
    \label{ex:1:34}
\ea\label{ex:1:34a}
\gll was für ne krassen Menschen albanische Wurzeln haben\\
    what for a   weird     people      Albanian    roots      have\\
\glt ‘what kind of weird people have Albanian roots’
\ex\label{ex:1:34b}
\gll was  meine Unachtsamkeit für ne krassen Folgen hat\\
    what my     carelessness     for a   bad        consequences has\\
\glt ‘what kind of bad consequences my carelessness has’
\z
\z

To be clear, these are not frozen (or fossilized) templates but productive constructions in (Northern) German.

  To sum up, I take it as established that \textit{ein} occurs in plural contexts in German. Indeed, \textit{ein} exhibits the hallmarks of a plural element: Morphologically, \textit{ein} itself has a typical plural ending; syntactically, it may be followed by an adjective with a weak plural ending; and semantically, it occurs within a noun phrase with plural meaning. As to its geographical distribution, cases like \textit{so’ne Äpfel} ‘such apples’ are common in the northern dialects, and instances like \textit{eine Äpfel} ‘(an) apples’ occur in the southern dialects. Plural \textit{ein} also appears in \textit{wh-}constructions, but this seems less common.\footnote{While \citet[341-42]{Duden1995} discusses \textit{was-für} constructions with special forms for the plural in Northern German, this work explicitly states that there is no \textit{ein} in these cases. As is clear, these forms do occur, at least with some speakers.} While a more systematic investigation of the differences between the various dialects and constructions must be left for future research, I proceed on the assumption that most (possibly all) dialects of German have some evidence of plural \textit{ein}. Most of the following discussion focuses on the constructions of the northern dialects. More generally, we may observe that the plural indefinite articles in Dutch, Yiddish, and Swedish are in good company.

  To conclude the entire section, what this brief illustration of adjectival inflections and indefinite articles has made clear is that languages, even closely related ones like German, Yiddish, and Norwegian/Swedish, differ in these empirical domains. Consequently, they should not receive the same account. Analyses have been advanced for all these languages (e.g., \citealt{Olsen1991b}, \citealt{Vater2002} for German; \citealt{Roehrs2015, Roehrs2022}, for Yiddish; \citealt{Julien2005a}, \citealt{Delsing1993} for Norwegian). However, many languages lack detailed analyses of some or many of the non-canonical constructions (with the notable exception of English, see \citealt{Perlmutter1970}). More empirical and theoretical work is called for. Awaiting more comprehensive analyses of Yiddish and Norwegian, I focus on German, the language I know best, in the rest of this book, striving to make headway for this language. Specifically, I focus on (colloquial) Standard German. Having said that, I make a few cross-linguistic remarks throughout the discussion (e.g., when another language exhibits a revealing similarity or difference).

  As for empirical grounding of this study, most of the data involving adjectival inflections have been reported in previous work (e.g., in the well-known reference grammar Duden). They appear to be relatively straightforward and accepted. As already seen above, this statement is only partially true of \textit{ein}. The data here are taken from the literature and my own work. Importantly, the literature often does not make a distinction between the reduced, unstressed forms of the indefinite article (e.g., \textit{ein} > \textit{’n} ‘a’) and the unreduced, stressed forms of \textit{ein} (e.g., \textit{ein} > \textit{EIN} ‘one’). However, such a distinction is made here and plays a crucial part in the classification and analysis of these elements below.\footnote{For some phonetic discussion of indefinite (and definite) articles in spontaneous speech, see \citet{Wesener1999}. I thank Laura Smith for pointing out this reference to me.}

\section{Basic properties and main hypotheses}\label{sec:1.3}

In this section, I discuss some of the characteristics of adjectival inflections and the indefinite article \textit{ein} ‘a’. On the basis of that, I formulate the main hypotheses of this book.

\subsection{Adjectival inflections and the article \textit{ein} }\label{sec:1.3.1}

Taking German as the main language under investigation, this monograph focuses on two elements: adjectival inflections and \textit{ein} ‘a’. While these elements have an overt realization and thus receive an interpretation in PF, I argue that they are semantically vacuous. Adjectival inflections are affixes and \textit{ein} is, in some cases, a free morpheme and, in others, part of a word. I suggest later that adjectival inflections are licensed by movement of the adjective stem and \textit{ein} by that of certain operators. Some of these operators have an overt manifestation of their own and some do not.

First, I briefly illustrate why these elements are of interest. Similar to the cases in \sectref{sec:1.1}, the assumption that adjectival inflections and \textit{ein} make semantic contributions raises some issues. In general terms, if adjectival inflections and \textit{ein} had semantic import, that would lead to some contradictory conclusions. These issues disappear if we assume that these elements are semantically vacuous. This assumption is strengthened by the fact that while these elements have to be generally present for independent reasons, they can be left out in certain well-defined contexts without a loss of meaning. Indeed, these elements can also co-occur without a change in interpretation. I illustrate these points in more detail below.

\subsubsection{Basic properties of adjectival inflections}\label{sec:1.3.1.1}

Jacob \citet[718-56]{Grimm1870} pointed out that adjectives can alternate as regards their inflections: In his (now widely-accepted) terminology, adjectives can exhibit strong or weak endings.\footnote{In the literature, there are other terms for these inflections: pronominal (i.e., diachronically related to third-person pronouns and pronominals) vs. nominal, determining vs. determined, and primary vs. secondary. Here, I adopt the neutral terms strong and weak.}  The distribution of these endings is often referred to as the strong/weak alternation. Before turning to the distribution of these inflections, I start with some brief diachronic remarks.

It is generally assumed (for detailed discussion, see, e.g., \citealt{Evans2019}, \citealt{Lockwood1968}: 37-46, \citealt{Petrova2024}: 187-92, \citealt{Pfaff2020}, \citealt{Rehn2019}: Chapters 2 and 3, and references cited therein) that Proto-Indo-European did not have a distinct category of adjectives but rather seemed to have a more general category of nominal elements (comprising adjectives and nouns). The (current) strong inflections on adjectives are a continuation of the old nominal \textit{o}-/\textit{a}-stem inflections into Germanic, but these inflections also recruited several endings from the pronominal paradigms (e.g., demonstratives). Besides a set of strong adjective inflections, this yielded a difference between the nominal categories of noun and adjective. By contrast, weak inflections appear to be an innovation in Germanic. Specifically, a nominalizing suffix -\textit{n} was added that produced \textit{n}-stem nominals and eventually adjectives with weak inflections (for a dissenting view, see \citealt{Evans2019}: Chapter 2). Finally and returning to the strong inflections, \citet[152-53]{Evans2019} points out that regular sound changes in West Germanic led to a number of zero endings in the strong paradigm that, one the one hand, resulted in uninflected adjectives in predicative contexts but, on the other hand, were later replenished by the inflections of pronominals yielding generally inflected adjectives in attributive contexts in (Standard) German (but see also \citealt{Rehn2019}: 135-39).

Returning to the main focus of the discussion, Modern German, note first that noun phrases inflect for case, number, and gender.\footnote{For the most part, I do not consider person in this book.} While specifications for number and, to some degree, case can be inspected on suffixes to nouns, gender is usually expressed on other elements – by the inflections on determiners including articles and on adjectives. As mentioned above, adjectives can alternate between exhibiting strong or weak inflections. It is also worth pointing out already here that strong inflections have more different forms than weak ones, thus expressing more distinctions in case, number and gender (\chapref{sec:2}).

As for the distribution of these inflections, consider the canonical cases in \REF{ex:1:35}. We can observe that the strong/weak alternation involves an adjective with a weak inflection if a preceding determiner has a strong inflection \REF{ex:1:35a}, but it shows an adjective with a strong inflection if not \REF{ex:1:35b}. This distribution is concisely described in the generalization in \REF{ex:1:36} (for statements along these lines, see \citealt{Bierwisch1967}: 257, \citealt{Eisenberg1998}: 171-73, \citealt{Gallmann1998}: 144, \citealt{Giusti2015}: 207, G. \citealt{Müller2002a}: 129, \citealt{Petrova2024}: 184-85, \citealt{Pfaff2017}: 286, \citealt{Rehn2019}: 58, \citealt{Sauerland1996}: 34, \citealt{Schoorlemmer2009}: 53, and many others).\footnote{\citet[268]{KatzirSiloni2014} claim that this type of generalization holds for all Germanic languages. For the discussion of the Principle of Monoinflection, see \chapref{sec:2}.}

\ea%35
    \label{ex:1:35}
\ea\label{ex:1:35a}
\gll d-er    gut-e        Kaffee\\
    the-\textsc{st} good-\textsc{wk} coffee.\textsc{masc}\\
\glt ‘the good coffee’
\ex\label{ex:1:35b}
\gll ein gut-er    Kaffee\\
    a    good-\textsc{st} coffee.\textsc{masc}\\
\glt ‘a good coffee’
\z
\z

\ea%36
    \label{ex:1:36}
          Weak After Strong\\
          An adjective with a weak inflection is preceded by a determiner with a strong inflection.
\z

Notice that adjectives with strong inflections are not explicitly mentioned in the generalization. However, since adjectives in prenominal position only alternate between showing strong or weak inflections, the generalization above implies that strong adjectives occur in all instances not mentioned in \REF{ex:1:36}. Given this, the generalization in \REF{ex:1:36} captures the cases in \REF{ex:1:35}. For brevity, it is often referred to as Weak After Strong. Notice that when the term Weak After Strong is used, the reader should keep in mind that weak and strong refer to inflections on elements of different lexical categories (adjective vs. determiner). This is important as we will see later that two co-occuring adjectives have the same ending (and the same applies to two determiners to the extent these elements can co-occur). Note also that although it is the inflection of the adjective that alternates (i.e., the adjective \textit{gut}- ‘good’ in \REF{ex:1:35} above has either a weak or a strong inflection), I often refer to these cases simply as strong or weak adjectives. This is in contrast to verbs and nouns, which are lexically specified as strong or weak.

Observe that the generalization in \REF{ex:1:36} is surface-oriented, making reference to the notion of precedence. It works quite well for canonical cases such as \REF{ex:1:35} above. However, it fares less well with some other instances. There are two types of exceptions (see \chapref{sec:3}, \sectref{sec:3.7} for fuller discussion). First, strong adjectives can be preceded by determiners with a strong inflection. This holds when an inflected demonstrative precedes an uninflected possessive article \REF{ex:1:37a}. Second, weak adjectives can be preceded by determiners without strong inflections. This is the case with pronominal determiners (e.g., \citealt{DéchaineWiltschko2002}: 421-22, \citealt{Höhn2020}, \citealt{Pesetsky1978}, \citealt{Postal1966}, \citealt{Roehrs2005}), which do not have an inflection in the first and second person \REF{ex:1:37b}.

\ea%37
    \label{ex:1:37}
\ea\label{ex:1:37a}
\gll dies-es mein groß-es   Glück\\
    this-\textsc{st} my    great-\textsc{st}  happiness.\textsc{neut}\\
\glt ‘this great happiness of mine’
\ex\label{ex:1:37b}
\gll ihr nett-en   Studenten\\
    you nice-\textsc{wk} students\\
\glt ‘you nice students’
\z
\z

It seems clear that a simple notion of precedence referring to inflections cannot capture all the distributions. Rather, it is argued in \chapref{sec:2} and \chapref{sec:3} that structural and lexical factors need to be taken into account. Note though that other explanations have also been put forward.

It is sometimes suggested that this alternation involves a semantic distribution; that is, it correlates with the (in-)definiteness of the larger noun phrase. This kind of statement is usually made in the context of the Scandinavian languages (see again \sectref{sec:1.2.1.2} above) and the older varieties of German (see \chapref{sec:2}, \sectref{sec:2.1}). However, it can also be occasionally found in the older literature on Modern German (e.g., \citealt{Curme1910}, \citealt{Lockwood1968}, \citealt{Prokosch1939}; see \citealt{Esau1973} for critical discussion of these works) and even in the more recent literature (e.g., \citealt{KarnowskiPafel2004}: 177-78, \citealt{Pafel1994}: 257-64). Having said that, a consensus seems to have emerged that the strong/weak alternation is not regulated by the semantics in Modern German but rather by the morpho-syntax (see \citealt{Harbert2007}: 131 and the many references cited below). As I show in detail throughout this book, this characterization is indeed correct, with the qualification that there is also a lexical component to the explanation.

To provide some brief illustration here, there are indeed many cases where a strong inflection occurs in an indefinite context \REF{ex:1:38a} and a weak ending in a definite environment \REF{ex:1:38b} (note that German \textit{wir} typically appears as English \textit{us} here, see references mentioned for \REF{ex:1:37b} above).

\ea%38
    \label{ex:1:38}
\ea\label{ex:1:38a}
\gll lauter dumm-e Idioten\\
    many  stupid-\textsc{st} idiots\\
\glt ‘many stupid idiots’
\ex\label{ex:1:38b}
\gll wir dumm-en Idioten\\
    we  stupid-\textsc{wk} Idiots\\
\glt ‘us stupid idiots’
\z
\z

However, a weak inflection can also occur in an indefinite context, and a strong ending is possible in a definite environment as well. Consider \REF{ex:1:39a} and \REF{ex:1:39b}.

\ea%39
    \label{ex:1:39}
\ea\label{ex:1:39a}
\gll mancher klug-e       Freund\\
    some       smart-\textsc{wk} friend.\textsc{masc}\\
\glt ‘many a smart friend’
\ex\label{ex:1:39b}
\gll Peters  klug-er   Freund\\
    Peter’s smart-\textsc{st} friend.\textsc{masc}\\
\glt ‘Peter’s smart friend’
\z
\z

It is clear from the two paradigms in \REF{ex:1:38} and \REF{ex:1:39} that strong endings do not correlate with indefiniteness or weak endings with definiteness. Below, I demonstrate that this holds more generally. Faced with these contradictory patterns, I propose that adjectival inflections have neither semantics of their own nor do they make such features visible – they are not associated with (in-)definiteness. This is what I mean by stating that these elements are semantically vacuous.\footnote{In the context of Icelandic, \citet[293]{Pfaff2017} states that the “weak inflection – rather than expressing definiteness by itself – is a semantically vacuous reflection of definiteness marked elsewhere.” Crucially, unlike in German, the weak inflection in Icelandic is dependent on the presence of definiteness in the DP. \citet[316]{Pfaff2017} makes this dependency more formal by postulating an agreement relation between a definite D and an uninterpretable definiteness feature [uDEF] on the adjectival inflection. Adjectival inflections in German do not have such a feature – they do not reflect definiteness.} If so, the expectation is that adjectival inflections can be left out without a change in meaning.

Adjectival inflections do not appear under certain well-defined structural and lexical conditions in German. Structurally, adjectives must be inflected before the noun \REF[a-b]{ex:1:40}, but they remain uninflected when they follow it \REF[c-d]{ex:1:40}.

\ea%40
    \label{ex:1:40}
\ea\label{ex:1:40a}
\gll ein auf seine Frau stolz-er Mann\\
    an  of   his    wife  proud\textsc{-st} man.\textsc{masc}\\
\glt ‘a man proud of his wife’
\ex\label{ex:1:40b}
\gll der auf seine Frau stolz-e      Mann\\
    the  of   his    wife  proud-\textsc{wk} man.\textsc{masc}\\
\glt ‘the man proud of his wife’
\ex\label{ex:1:40c}
\gll ein Mann,        stolz   auf seine Frau\\
    a    man.\textsc{masc} proud of   his    wife\\
\glt ‘a man, proud of his wife’
\ex\label{ex:1:40d}
\gll der Mann,        stolz   auf seine Frau\\
    the  man.\textsc{masc} proud of   his    wife\\
\glt ‘the man, proud of his wife’
\z
\z

Comparing \REF{ex:1:40a} to \REF{ex:1:40c} and \REF{ex:1:40b} to \REF{ex:1:40d}, there is no change in the (in-)\linebreak definiteness of the noun phrase, and yet adjectival inflections are present in \REF[a-b]{ex:1:40} but absent in \REF[c-d]{ex:1:40}. This could be evidence that inflections make no semantic contribution. One objection might be that the adjectives in \REF[c-d]{ex:1:40} are not part of the noun phrase proper given that they belong to reduced relative clauses. While this may be a valid point, there is other evidence that adjectival inflections can be left out without a loss in meaning. Before I turn to these cases, note that the paradigm in \REF{ex:1:40} does indicate that structural conditions play a role in the occurrence of adjectival inflections – only prenominal adjectives have inflections.

Besides being sensitive to structure, adjectival inflections can or must be left out in certain lexical contexts; that is, adjectival inflections are also sensitive to lexical conditions. Among others, this can be seen with specific sets of adjectives that can occur prenominally and as such, they are clearly part of the noun phrase. As just mentioned, the inflections on these adjectives can or must be absent. It will become clear, though, that the presence or absence of these inflections does not lead to a change in the semantics.

First, there is a special set of color adjectives involving \textit{lila} ‘purple’ and \textit{rosa} ‘pink’ as well as \textit{orange} ‘orange’ and \textit{beige} ‘beige’ that show that adjectival endings can be optional. Note that the first two adjectives (\textit{lila}, \textit{rosa}) require -\textit{n}- to be inserted \REF[a-b]{ex:1:41} but that the second two adjectives (\textit{orange}, \textit{beige}) do not \REF{ex:1:41c}.\footnote{Notice that the pronunciation of \textit{orange(-s)} ‘orange’ in \REF{ex:1:41c} has some special properties. Without the inflection -\textit{s}, -\textit{ge} is pronounced as [ʃ] – the -\textit{e} is silent triggering Final Devoicing of the consonant. However, when the inflection is added yielding -\textit{ges}, this sound sequence is rendered as [ʒəs]. Observe that besides \REF{ex:1:41c}, \textit{das orange Kleid} ‘the orange dress’ is also possible. Here, however, the ending on \textit{orange} (i.e., -\textit{e}) is pronounced as schwa and can be interpreted as the regular weak ending. This interpretation is not possible with \textit{lila} and \textit{rosa}. The latter two end in -\textit{a}, which is not pronounced as schwa.} Note that the presence or absence of the inflection makes no semantic difference.

\ea%41
    \label{ex:1:41}
\ea\label{ex:1:41a}
\gll ein lila(-n-es)   Kleid\\
    a    purple-n-\textsc{st} dress.\textsc{neut}\\
\glt ‘a purple dress’
\ex\label{ex:1:41b}
\gll das lila(-n-e)      Kleid\\
    the purple-n-\textsc{wk} dress.\textsc{neut}\\
\glt ‘the purple dress’
\ex\label{ex:1:41c}
\gll ein orange(-s) Kleid\\
    an  orange-\textsc{st}  dress.\textsc{neut}\\
\glt ‘an orange dress’
\z
\z

Second, with a few types of adjectives, the inflection must be absent. This holds for adjectives derived from geographical names, so-called toponymic formations (-\textit{er} is an adjective-forming suffix).

\ea%42
    \label{ex:1:42}
\ea\label{ex:1:42a}
\gll ein Berlin-er(*-es)    Bier\\
a    Berlin-\textsc{adj}-*\textsc{st} beer.\textsc{neut}\\
\glt ‘a Berlin beer’
\ex\label{ex:1:42b}
\gll das Berlin-er(*-e)      Bier\\
the Berlin-\textsc{adj}-*\textsc{wk} beer.\textsc{neut}\\
\glt ‘the Berlin beer’
\z
\z

The same applies to adjectives like \textit{prima} ‘great’, \textit{sexy} ‘sexy’, \textit{super} ‘super’, etc. \citep[256-57]{Duden1995}.\footnote{A reviewer points out that the latter adjectives are loan words that may not be fully intergrated into the language yet. This also applies to the color adjectives discussed in the previous paragraph.} Note that since adjectival inflections are absent, they can neither indicate nor contribute to (in-)definiteness.

To be clear, while inflections on prenominal adjectives are usually obligatory, there are some special cases where they can or must be absent. Importantly, this presence or absence of inflection does not correlate with a change in the (in-)\linebreak definite interpretation of the noun phrase as a whole. Besides the semantic property of (in-)definiteness, I demonstrate in \chapref{sec:4} that adjectival inflections are also not a reflex of other semantic concepts such as (non-)restrictiveness of the interpretation of modifiers or referentiality. In view of that, the account of the strong/weak alternation must involve something else.

The following noun phrases in the (a)-examples are in the nominative and the ones in the (b)-examples are in the dative. In keeping with the facts above, the strong and weak endings occur both in indefinite and definite contexts.

\ea%43
    \label{ex:1:43}
\ea\label{ex:1:43a}
\gll ein rot-es Auto \\
    a    red-\textsc{st} car.\textsc{neut}\\
\glt ‘a red car’
\ex\label{ex:1:43b}
\gll mit einem rot-en   Auto \\
    with a     red-\textsc{wk} car.\textsc{neut}\\
\glt ‘with a red car’
\z
\z

\ea%44
    \label{ex:1:44}
\ea\label{ex:1:44a}
\gll sein rot-es Auto \\
    his   red-\textsc{st} car.\textsc{neut}\\
\glt ‘his red car’
\ex\label{ex:1:44b}
\gll mit seinem  rot-en   Auto \\
    with his        red-\textsc{wk} car.\textsc{neut}\\
\glt ‘with his red car’
\z
\z

Again, it is unlikely that semantic concepts like (in-)definiteness can explain the strong/weak alternation. Rather, it seems more promising to find an explanation in the morpho-syntax. Considering the similarities between \REF{ex:1:43} and \REF{ex:1:44}, I assume that possessive articles consist of a possessive component (\textit{s}-) and \textit{ein} (\chapref{sec:5}). If so, the types of adjectival endings seem to be a function of preceding \textit{ein}, independently of the presence or absence of the possessive component. In other words, \REF{ex:1:43} and \REF{ex:1:44} are directly relatable. Given that, we might claim that the strong/weak alternation is a reflex of the semantics of \textit{ein} behaving differently in the nominative vs. dative case. However, like adjectival inflections, \textit{ein} exhibits not only contradictory properties with regard to the semantics but also shows characteristics of having no meaning at all. These points can be made most straightforwardly as regards semantic number.\footnote{Making the following points with regard to the (in-)definiteness of \textit{ein} would take me too far afield here. I discuss the relation between adjectival inflections and \textit{ein} as regards (in-)\linebreak definiteness in detail in \chapref{sec:5}. I propose there that \textit{ein} has nothing to do with (in-)\linebreak definiteness, either in the nominative or in the dative. Thus, this semantic dimension cannot explain the distribution of adjectival inflections either.}

\subsubsection{Basic properties of \textit{ein}}\label{sec:1.3.1.2}

As briefly mentioned in \sectref{sec:1.2.2.1}, the indefinite article \textit{ein} emerged on the basis of the singularity numeral and indefinite pronoun \textit{ein} in Old High German (e.g., \citealt{BrauneReiffenstein2004}: 234, \citealt{Presslich2000}, \citealt{Rehn2019}: 88-90). It is often claimed that \textit{ein} in Modern German is semantically singular in meaning (e.g., \citealt{Vater2002}). Specifically, \REF{ex:1:45a} makes mention of the fact that a girl was seen, and \REF{ex:1:45b} emphasizes the point that only one girl has a certain property, namely having been at a party (parentheses around \textit{ei} yielding \textit{n} indicate the reduced form of \textit{ein}).

\ea%45
    \label{ex:1:45}
\ea\label{ex:1:45a}
\gll Ich habe gestern   (ei)n Mädchen getroffen.\\
I    have  yesterday {\db}a     girl.\textsc{neut} met\\
\glt ‘I met a girl yesterday.’

\ex\label{ex:1:45b}
\gll Nur EIN Mädchen war auf der Party! \\
    only one  girl.\textsc{neut} was at   the party\\
\glt ‘Only one girl was at the party.’
\z
\z

Crucially, \textit{ein} can also imply the existence of a second entity \REF{ex:1:46a}. In these cases, the first clause often, but not always, occurs with the second one put in parentheses below. Independently of the second clause, \textit{ein} preceded by a definite determiner yields a duality partitive; that is, it is presupposed that there is a second girl present (NB: This is different in the English counterpart).\footnote{In English, \textit{one} following a definite article often has an intensifying meaning close to ‘only’.
	\ea She was the (one) girl I did not recognize.
	\z } In fact, \textit{ein} can be part of a noun phrase denoting a more numerous plurality of entities. As already discussed in \sectref{sec:1.2.2}, \textit{ein} can clearly occur in semantically plural contexts. To repeat, \REF{ex:1:46b} is a googled example from the Appendix, and \REF{ex:1:46c} is an example from the ebook \textit{Wolfskinder} ‘Wolf Children’.

\ea%46
    \label{ex:1:46}
\ea\label{ex:1:46a}
\gll Das ein-e     Mädchen blieb        stehen (,das andere ging  weg).\\
the  one-\textsc{wk} girl.\textsc{neut} remained standing {\db}the  other    went away\\
\glt ‘One of the two girls stopped walking, the other went away.’
\ex\label{ex:1:46b}
\gll Was  für ein-e geil-en Bilder    wer hat  die    bloß gemacht\\
what for a-\textsc{pl}  awesome-\textsc{wk} pictures who has those just  made\\
\glt ‘Such awesome pictures! Just who took those?’
\ex\label{ex:1:46c}
\gll Smilla recherchiert und entdeckt,  das schon   so ein-e Frauen verschwanden.\\
Smilla researches    and discovers that already so a-\textsc{pl} women  disappeared\\
\glt ‘Smilla does some research and discovers that such women already disappeared.’        (\url{https://www.vorablesen.de/buecher/wolfskinder/rezensionen/duester-a6154566-6616-445c-967e-966d6578811c})
\z
\z

As with adjectival endings, we are faced with a contradictory state of affairs: In this case, \textit{ein} seems to imply not only singularity but also different pluralities. Given this, we might also expect that similar to adjectival inflections, \textit{ein} can sometimes be left out without a change in meaning.

To illustrate this point, consider nominals in predicate contexts. Note that \textit{ein} is obligatory in \REF{ex:1:47a} but that it is optional for many speakers in \REF{ex:1:47b}.\footnote{I comment on the optionality of \textit{ein} with [-figurative] role nouns as in \REF{ex:1:47b} further below.}

\ea%47
    \label{ex:1:47}
\ea \label{ex:1:47a}
\gll Er ist ein Mann.\\
he is  a    man.\textsc{masc}\\
\glt ‘He is a man.’
\ex \label{ex:1:47b}
\gll Er ist (ein) Lehrer.\\
he is    {\db}a     teacher.\textsc{masc}\\
\glt ‘He is a teacher.’
\z
\z

It is clear that both of these predicate nominals denote a property; that is, \textit{ein} does not make a semantic contribution with regard to singular number (or indefiniteness, for that matter). Following \citet{deSwartEtAl2007}, I argue in \chapref{sec:6} that the presence of \textit{ein} is a function of the different types of nouns, a kind noun in \REF{ex:1:47a} vs. a role noun in \REF{ex:1:47b}. Thus, similar to adjectival endings, I conclude that the presence or absence of \textit{ein} makes no difference with regard to the semantics, here illustrated with semantic number. In fact, I show below that this type of element is not a reflex of other semantic concepts such as indefiniteness or a certain type of emotiveness either. I propose that \textit{ein} neither has semantics nor does it make such features visible – it is semantically vacuous. This also means that the strong/weak alternation of adjectives briefly illustrated in \sectref{sec:1.3.1.1} cannot be explained by the semantics of \textit{ein} either.

If it is true that adjectival inflections and \textit{ein} are semantically vacuous, then we may wonder what their function is; that is, why they exist at all. This question is particulary interesting in view of the fact that similar to the expletives in \sectref{sec:1.1}, these elements can also co-occur with elements that seem similar or related in some way.\footnote{Recall that \textit{there} can occur with a locative adverbial and that the (formally definite) proprial article occurs with a proper name.} In fact, here adjectival inflections and \textit{ein} can each co-occur yielding multiple instances of the same elements; consider \REF{ex:1:48a} and \REF{ex:1:48b}. I assume for now (and argue in \chapref{sec:5}) that similar to possessives like \textit{sein} ‘his’, the negative article \textit{kein} ‘no’ consists of a negative element, NEG, and \textit{ein} (where NEG is later spelled out as \textit{k}-; \REF{ex:1:48b} is adapted from \citealt{GallmannLindauer1994}: 24).

\ea%48
    \label{ex:1:48}
\ea\label{ex:1:48a}
\gll frisch-er heiß-er Kaffee\\
    fresh-\textsc{st}  hot-\textsc{st}  coffee.\textsc{masc}\\
\glt ‘fresh hot coffee’
\ex\label{ex:1:48b}
\gll k-ein so’n Mann\\
    \textsc{neg}-a {so a} man.\textsc{masc}\\
\glt ‘no such man’
\z
\z
\largerpage
Neither the multiple occurrence of the adjectival inflections nor that of \textit{ein} seems to make a semantic difference -- the meanings stem from the adjectives themselves, the negation \textit{k-} and the type particle \textit{so} 'such'. Importantly, given the proposed pleonastic status of the adjectival inflections and \textit{ein}, note that not only one but both of these elements seem to be redundant. Observe though that neither of the two adjectives in \REF{ex:1:48a} can occur without an ending. Similarly, neither \textit{ein} nor \textit{’n} in \REF{ex:1:48b} can be left out. Also, if adjectival inflections and \textit{ein} were assumed to make specific semantic contributions, then it would remain unclear why several instances of the same elements can, and in these cases must, co-occur. Again, this curious fact finds a natural explanation if we assume that adjectival endings and \textit{ein} are semantically vacuous and their obligatory presence in \REF{ex:1:48a} and \REF{ex:1:48b} is due to a different, independent reason or reasons. Below, I propose that adjectival inflections make abstract morpho-syntactic features visible and that \textit{ein} supports certain semantic operators.

To recapitulate, I have illustrated some cases where adjectival inflections and \textit{ein} seem to lead to contradictory conclusions as regards the semantics: Adjectival inflections, strong or weak, occur in both definite and indefinite contexts, and \textit{ein} appears in singular and plural environments. At the same time, we have seen that these elements seem to have no semantics at all as they are absent in certain well-defined contexts without loss of meaning: Adjectival inflections do not have to or cannot occur with special sets of adjectives, and \textit{ein} does not have to occur with a specific set of nouns in predicative contexts. Furthermore, I provided evidence that these elements can co-occur without any semantic import. I proposed that adjectival inflections and \textit{ein} are semantically vacuous. If so and returning to the more general issues, what is interesting then about adjectival endings and \textit{ein} is that they are overt elements that receive no interpretation in LF – like \textit{there} and the proprial article discussed above.

\subsection{Main hypotheses and relatedness of adjectival inflections and \textit{ein}}\label{sec:1.3.2}
In this section, I briefly summarize the general characteristics of adjectival inflections and \textit{ein}, and I formulate the main hypotheses about these elements. On the basis of that, I provide further motivation why adjectival inflections and \textit{ein} can be fruitfully discussed in tandem.

\subsubsection{Main hypotheses}\label{sec:1.3.2.1}
Some of the main general properties of adjectival inflections and \textit{ein} have already been touched upon; others will be introduced later on in the book. They can be summarized as follows. Specifically, while these elements (usually) have to occur for independent reasons, adjectival inflections and \textit{ein} can each:

\begin{itemize}
\item co-occur
\item appear in contradictory semantic contexts
\item be left out in certain well-defined contexts without loss of meaning
\end{itemize}

Furthermore, adjectival inflections are \textit{not} a reflex of:

\begin{itemize}
  \item (in-)definiteness
\item (non-)restrictiveness of interpretation of modifiers
\item referentiality
\end{itemize}

Finally, \textit{ein} is \textit{not} a reflex of:

\begin{itemize}
  \item indefiniteness
\item emotiveness
\item  singular number/countability
\end{itemize}

Given this, I formulate the two main claims that adjectival endings and \textit{ein} have in common. While \REF{ex:1:49a} has already been mentioned above, here I add \REF{ex:1:49b}.

\ea%49
\label{ex:1:49}

Hypothesis 1\\
Adjectival inflections and \textit{ein}:
  \ea\label{ex:1:49a}     are expletive elements and
  \ex\label{ex:1:49b}     indicate abstract structure in the noun phrase.
  \z
\z

Hypothesis 1b is slightly different for adjectival inflections and \textit{ein}. This can be fleshed out in the two (a)-statements below. In addition, each of these two types of elements have another difference summarized in the (b)-statements.\footnote{Note though that the (b)-statements are also partially relatable: Both adjectival inflections and \textit{ein} make elements visible, morpho-syntactic features vs. operators (the latter in the context of flagging, cf. \REF{ex:1:51b}). On a different note, I argue in \chapref{sec:5} that there are two basic types of \textit{ein}: the so-called indefinite article characterized in Hypothesis 3 and adjectival \textit{eine}, which is not related to the article -- adjectival \textit{eine} is different in that it has semantics and is argued to be in a different position.}

\ea%50
    \label{ex:1:50}
    Hypothesis 2\\
  Adjectival inflections:\\
\ea\label{ex:1:50a}  indicate abstract structure in the higher layers of the noun phrase (DP vs. LPP);       they provide clues about structures involving various degrees of embedding of       adjectives (simple vs. complex DPs), and
\ex\label{ex:1:50b}  they make nominal features like case, number, and gender visible.
\z
\z

\ea%51
    \label{ex:1:51}
   Hypothesis 3\\
  \textit{Ein}:\\
\ea\label{ex:1:51a}  indicates abstract structure in the lower layers of the noun phrase (NP vs. ArtP),       and
\ex\label{ex:1:51b}  it supports overt semantic operators (e.g., NEG \textit{k}-) and flags the presence of       covert semantic operators (e.g., REL).
\z
\z

Some remarks are in order here. The structural levels LPP and ArtP stand for Left Periphery Phrase and Article Phrase. The first is above the DP-layer, and the second is below the position of adjectives (AgrP; for detailed discussion of my structural assumptions, see \sectref{sec:1.4}). Furthermore, supporting as mentioned in \REF{ex:1:51b} has been discussed in the context of the well-known phenomenon of \textit{do}-support in English (\citealt{Chomsky1957}, \citealt{Lasnik2000}). In the current discussion, supporting involves overt operators. More precisely, operators like negation (NEG) have a detectable manifestation when spelled out, and that element is supported by \textit{ein} (i.e., \textit{k-ein} ‘no’). As to flagging, \textit{ein} indicates the presence of operators like REL, which is a realization operator in cases like \textit{ein} \textit{Mann} ‘a man’ \citep{deSwartEtAl2007}. However, here the operator itself has no detectable manifestation and remains invisible.\footnote{In the typological literature, the term flagging was proposed by \citet{Haspelmath2019} for marking nominals with case markers and/or adpositions. Unlike in the main text, here flags (i.e., case markers, adpositions) involve role-identifiers that appear with \textit{overt} nominals.}

Some of the claims in \REF{ex:1:49} to \REF{ex:1:51} are not entirely new; for instance, Hypothesis 1b as instantiated by adjectives in Hypothesis 2a was established by \citet{Pfaff2017} for Icelandic adjectives. Hypothesis 2b is related to \citet[40]{Olsen1991b} and \citet[122-26]{Rehn2019}, who state that morphological features need to be made visible (cf. also \citealt{Esau1973}, who takes the strong endings as case markers). Hypothesis 3b was, in part, argued for by \citet{deSwartEtAl2007}. In the course of the following discussion, I discuss this and other related work.

Finally, I followed Chomsky in \sectref{sec:1.1} in assuming that an expletive needs to be licensed by a substantive element (the associate) to avoid a violation of the Principle of Full Interpretation. This licensing can be seen as a semantic licensing condition and was instantiated above by moving the substantive element to the expletive. In contrast, the hypotheses above state that expletives indicate the presence of linguistic elements. This, in turn, can be interpreted as a syntactic licensing condition.

If this is on the right track, then there is a division of labor: Syntactically, we may state that the expletive indicates the substantive element; semantically, we may say that the substantive element identifies the expletive.\footnote{These two conditions are reminiscent of (but different from) the licensing of \textit{pro} proposed by \citet{Rizzi1986}: \textit{Pro} is formally licensed by government, and the content of \textit{pro} is recovered through rich agreement specification (for ellipsis involving \textit{pro}, see also \citealt{Lobeck1995}).}

\eabox{%52
 \label{ex:1:52}
 \begin{tikzpicture}
 	\node[] (EXPL) {EXPL};
 	\node[] (SUBST) [right=of EXPL, xshift=3cm] {SUBST};
 	
 	\draw[->] ([yshift=2.5pt] EXPL.east) -- node[anchor=south] {Indicate (syntax)} ([yshift=2.5pt] SUBST.west);
 	\draw[<-] ([yshift=-2.5pt] EXPL.east) -- node[anchor=north] {Identify (semantics)} ([yshift=-2.5pt] SUBST.west);
 \end{tikzpicture}
}


Considering \REF{ex:1:52}, there is an interplay between the two elements, in that each element licenses the other. I return to this in \chapref{sec:8}.

\subsubsection{Relatedness of adjectival inflections and \textit{ein}}\label{sec:1.3.2.2}

It is reasonable to ask why adjectival inflections and the article \textit{ein} ‘a’ should be discussed together in one volume. Adjectival inflections are bound inflectional morphemes, and the indefinite article is, at least in some cases, a free morpheme. Given their different status, it is expected that they exhibit differences. However, as seen in the previous section, they also show many similarities. To the extent that the hypotheses above prove tenable, both of these elements are expletives, they indicate structure, and they make linguistic elements visible. In addition, both elements involve functional items inside the nominal spine of the DP. What is special about them is that both adjectival inflections and \textit{ein} can each appear in multiple syntactic positions inside the DP, and they interact morphologically.

In the absence of evidence to the contrary, I assume that the strong and weak inflections appearing on different elements are the same (e.g., \citealt{MilnerMilner1972}, also \chapref{sec:2}). Specifically, strong endings can appear on adjectives \REF{ex:1:53a}. In addition, the same (strong) endings surface on predeterminers (e.g., \textit{alle} ‘all’) and on determiners \REF{ex:1:53b}.

\ea%53
    \label{ex:1:53}
\ea\label{ex:1:53a}
\gll gut-e      Studenten\\
    good-\textsc{st} students\\
\glt ‘good students’
\ex\label{ex:1:53b}
\gll all-e   dies-e    gut-en      Studenten\\
    all-\textsc{st} these-\textsc{st} good-\textsc{wk} students\\
\glt ‘all these good students’
\z
\z

It is argued in later parts of the book that adjectives, determiners, and predeterminers are in different syntactic positions (i.e., AgrP, DP, and LPP). Indeed, weak endings can also occur on different elements. Not only do they occur on adjectives as seen in \REF{ex:1:53b} and \REF{ex:1:54a}, but they can also (optionally) occur on certain determiners in the genitive masculine/neuter as in \REF{ex:1:54b} (for more detailed discussion, see \chapref{sec:3}, \sectref{sec:3.4}).

\ea%54
    \label{ex:1:54}
\ea\label{ex:1:54a}
\gll der Verkauf heiß-en Kaffee-s\\
    the  sale      hot-\textsc{wk}  coffee.\textsc{masc}-\textsc{gen}\\
\glt ‘the sale of hot coffee’
\ex\label{ex:1:54b}
\gll im      Sommer dies-en  Jahr-es\\
    in.the summer this-\textsc{wk} year.\textsc{neut}-\textsc{gen}\\
\glt ‘in the summer of this year’
\z
\z

This means that while more restricted, weak endings may also occur on elements in different positions (i.e., AgrP and DP). Given these distributions, I refer to these endings, including the ones on determiners and predeterminers, as adjectival inflections (rather than adjective endings).

  Similar to adjectival inflections, \textit{ein} can appear in different syntactic positions. This can be shown with certain (complex) \textit{ein}-words. As already briefly mentioned in \sectref{sec:1.3.1.2}, the negative article \textit{kein} ‘no’ consists of \textit{ein} and the negative element NEG \REF{ex:1:55a}. Similarly, I argue in \chapref{sec:5} that the singularity numeral \textit{EIN} ‘one’ is made up of \textit{ein} and the null element $\emptyset$\textsubscript{[--}\textsc{\textsubscript{pl}}\textsubscript{]} bringing about singularity \REF{ex:1:55b}. Both \textit{kein} and \textit{EIN} can co-occur with a lower instance of \textit{ein} provided \textit{so} ‘such’ intervenes.

\ea%55
    \label{ex:1:55}
\ea \label{ex:1:55a}
\gll k-ein-e     so’n-e  Frauen\\
    \textsc{neg}-a-\textsc{st} {so a-\textsc{st}} women\\
\glt ‘no such women’
\ex \label{ex:1:55b}
\gll EIN-E        so’n-e   Frau\\
    $\emptyset$\textsubscript{[--}\textsc{\textsubscript{pl}}\textsubscript{]}+a-\textsc{st} {so  a-\textsc{st}} woman.\textsc{fem}\\
\glt ‘one such woman’
\z
\z

Recalling the string \textit{kein so’n Mann} ‘no such man’ from above, note that all instances of \textit{ein} have the same inflection or lack thereof. I argue in later chapters that \textit{ein} along with its inflection may surface in the head position of (at least) ArtP, CardP, and DP.

  To repeat, adjectival inflections and \textit{ein} can each appear in various syntactic positions: Strong inflections occur in AgrP, DP, and LPP; weak inflections surface in AgrP and DP; and \textit{ein} occurs in ArtP, CardP, and DP. This is different from inflections on nouns, which only surface on the nouns themselves, and different from other determiners, which only surface in the DP-layer.\footnote{It
    could be claimed that the dative plural ending -\textit{n} on nouns \REF{ex:1:34:1a} can appear on other elements when the noun itself is elided \REF{ex:1:34:1b} (see \citealt{CorvervanKoppen2010,CorvervanKoppen2011b} for discussion).

      \ea
        \ea \label{ex:1:34:1a}
        \gll mit   den zwei Männer-n\\
        with the  two  men-\textsc{dat}\\
        \glt ‘with the two men’
        \ex \label{ex:1:34:1b}
        \gll mit   den zwei-n\\
        with the two-\textsc{infl}\\
        \glt ‘with those two’
        \z
      \z
    However, this is not possible with the genitive masculine/neuter ending -\textit{s} or plural endings in general.
}
It is clear then that adjectival inflections and \textit{ein} share similar syntactic properties.

Besides these similarities, adjectival inflections and \textit{ein} interact directly as regards inflectional morphology. In the typical cases, the determiner has a strong ending, and the following adjective has a weak one \REF{ex:1:56a}. However, as is well known, there are certain instances where \textit{ein} has no ending, and the adjective exhibits the strong ending \REF{ex:1:56b}. Importantly, when the adjective (and noun) is absent, \textit{ein} has the strong ending \REF{ex:1:56c}.

\ea%56
    \label{ex:1:56}
\ea\label{ex:1:56a}
\gll d-er    gut-e       Student\\
    the-\textsc{st} good-\textsc{wk} student.\textsc{masc}\\
\glt ‘the good student’
\ex\label{ex:1:56b}
\gll ein gut-er    Student\\
    a    good-\textsc{st} student.\textsc{masc}\\
\glt ‘a good student’
\ex\label{ex:1:56c}
\gll ein-er\\
    one-\textsc{st}\\
\glt ‘one’
\z
\z

As discussed in detail in \chapref{sec:2}, there are three instances where \textit{ein} does not have an inflection as in \REF{ex:1:56b} – in the remaining cases, \textit{ein} works like the definite article in \REF{ex:1:56a}. These three instances have received an enormous amount of attention. Specifically, note that the presence of uninflected \textit{ein} has a special impact on the ending of the adjective (the latter can be strong as in \REF{ex:1:56b}) but that \textit{ein} itself can take the ending of the adjective (if \textit{ein} occurs by itself as in \REF{ex:1:56c}). In other words, inflections and \textit{ein} may interact directly: \textit{Ein} may regulate the distribution of the adjectival inflections, and adjectival inflections occur on \textit{ein} if the adjective (and noun) is absent. Crucially, this alternation in \REF[b-c]{ex:1:56} is not possible with other determiners.\footnote{\label{foot:1:35}Definite
  articles are different from \textit{ein} – they always have an inflection when they are followed by an adjective and/or a noun \REF{ex:1:35:1a}. However, they may get an additional ending when they are used pronominally \REF{ex:1:35:1b} (again, see \citealt{CorvervanKoppen2010,CorvervanKoppen2011b} for discussion, also \citealt{Roehrs2013a}: 398-402).
  \ea
    \ea \label{ex:1:35:1a}
    \gll mit   d-en    gut-en  Männer-n\\
        with the-\textsc{st} gut-\textsc{wk} men-\textsc{dat}\\
        \glt ‘with the good men’

    \ex \label{ex:1:35:1b}
    \gll mit   d-en-en\\
        with the-\textsc{st}-\textsc{infl}\\
        \glt ‘with those’
    \z
  \z
  This additional ending -\textit{en} is not a regular adjectival inflection. Furthermore, it is restricted to dative plural contexts as in \REF{ex:1:35:1b} and to genitive instances more generally (e.g., \textit{d-er-en} ‘the-\textsc{st}-\textsc{infl}’ in the feminine and plural). Note also that \textit{manch} ‘some’, \textit{solch} ‘such’, and \textit{welch} ‘which’ are also different from \textit{ein}: Besides similar distributions as in \REF[b-c]{ex:1:56} (e.g., \textit{solch gut-er Student} ‘such good student’; \textit{solch-er} ‘such a one’), these three elements also allow the inflection to occur on the first element similar to definite articles (e.g., \textit{solch-er gut-e Student} ‘such good student’). Furthermore, the three elements, when uninflected, are not restricted to the three instances where uninflected \textit{ein} occurs.
}

To sum up, discussing adjectival inflections and \textit{ein} in one volume allows us to recognize and appreciate the above-mentioned similarities, the related differences, and the interactions of the two elements. Indeed, the analysis of adjectival inflections in \chapref{sec:2} will feature prominently in the discussion of the different types of \textit{ein} in \chapref{sec:5}. In fact, as will become clear, the analysis of this interaction narrows down the options of plausible analyses of certain other phenomena and raises issues for specific types of accounts of those phenomena. In addition, the following detailed discussion of both of these elements side-by-side goes beyond the canonical cases providing a more comprehensive discussion needed for both of these elements. These points make it advantageous to discuss adjectival inflections and \textit{ein} in tandem.

\section{Basic assumptions}\label{sec:1.4}

In this section, I briefly lay out my assumptions of the syntactic structure of canonical noun phrases (non-canonical constructions are discussed in later chapters in detail). Then I detail my assumptions about determiners, adjectives, and numerals/quantifiers. Finally, I summarize the main points of Distributed Morphology and Type Theory, approaches to the morphology and semantics that play an important part in the analysis to follow.

\subsection{Basic assumptions about structure}\label{sec:1.4.1}

I start by discussing noun phrases as a whole. This is followed by specifying my assumptions about determiners, adjectives, and numerals/quantifiers.

\subsubsection{DPs as a whole}\label{sec:1.4.1.1}

This book is written in the general generative tradition (e.g., \citealt{Chomsky1995,Chomsky2000}). As mentioned above, I assume the DP-Hypothesis such that nominals in argument position are (at least) DPs but nominals in non-argument position may be of smaller size. Taking a cartographic approach, I assume that DPs like the one in \figref{figex:1:57} have the structure shown there. Proceeding bottom-up, I assume that nouns project NPs and that morphological number is specified in NumP \citep{Ritter1991}.\footnote{In this work, I take a (traditional) projectionist view. In contrast, neo-constructionist models postulate that certain properties of lexical elements are determined by higher functional ones. For instance, \citet{Marantz1997} proposed that lexical items such as nouns and adjectives are merged as acategorial or category-neutral roots that receive their lexical category during the derivation; that is, after they have merged with a category-defining head (for similar ideas, see \citealt{Borer2005, Borer2013}). Below, I assume that ellipsis involves a null noun. Since I am not entirely sure how a null noun (or \textit{pro} as in \citealt{Olsen1987} and related work) is compatible with acategorial/category-neutral roots, I do not adopt this type of (neo-constructionist) view here, with the qualification that the mass/count distinction of nouns is determined by the higher Num head (\chapref{sec:7}).} I claim here (and argue below) that determiners originate in a lower Article Phrase (ArtP). Following \citet{Cinque1994,Cinque2010}, I take adjectives to reside in the specifiers of a recursive AgrP. I follow \citet{Zamparelli2000} in arguing that (weak) quantifiers are housed in Spec,CardP. Finally, I assume that determiners move from ArtP to the DP-level, with definite and indefinite articles moving to D but demonstratives moving to Spec,DP (for more detailed background discussion, see among many others \citealt{Abney1987}; \citealt{AlexiadouEtAl2007}; \citealt{Bernstein2001a}; \citealt{Julien2005a}; \citealt{Longobardi2001}; \citealt{Roehrs2020b}; \citealt{Salzmann2020,Salzmann2022}).

\glltree[\label{figex:1:57}]{\normalsize
    \gll die zehn kleinen Autos\\
        the ten   small      cars\\
    \glt ‘the ten small cars’
}{[DP
    [~]
    [D{$'$}
        [{D\\\textit{die}\textsubscript{$i$}}]
        [CardP
            [{QP\\\textit{zehn}}]
            [Card{$'$}
                [Card]
                [AgrP
                    [{AP\\\textit{kleinen}}]
                    [Agr{$'$}
                        [Agr]
                        [ArtP
                            [~]
                            [Art{$'$}
                                [{Art\\t\textsubscript{$i$}}]
                                [NumP
                                    [~]
                                    [Num{$'$}
                                        [{Num\textsubscript{[+\textsc{pl}]}}]
                                        [{NP\\\textit{Autos}}]
                                    ]
                                ]
                            ]
                        ]
                    ]
                ]
            ]
        ]
    ]
]
}

I assume with \citet{Julien2005b} that nouns undergo partial N-raising to Num (not shown in \figref{figex:1:57}). Note also that with the noun at the bottom of the tree, the higher projections can be interpreted as the extended projection of the noun (\citealt{Grimshaw1991}, \citealt{vanRiemsdijk1998b}). We see below that there are other types of extended projection; for instance, QP and AP in \figref{figex:1:57} are more complex (and are fleshed out further below).

The structure in \figref{figex:1:57} involves a simple DP – there are no other nominals embedded as specifiers, adjuncts, or complements. Furthermore, as mentioned above, case, number, and gender are concord features in the German noun phrase. I assume that gender comes from the noun and that (morphological) number originates in NumP. Case is determined DP-externally. Note that all elements in simple DPs like the one in \figref{figex:1:57} exhibit concord in agreement features.\footnote{It is not clear what the mechanism is that yields concord in agreement features (see, e.g., \citealt{Sigurðsson1989}: 112-13 for an analysis based on feature percolation/spreading, \citealt{Schoorlemmer2009} for an Agree-based account, \citealt{Norris2014} for a proposal in terms of feature spreading and local feature copying, and \citealt{Carnie2021}: 293 and \citealt{Olsen1991b}: 38 for an idea involving selection; for insightful discussion of agreement in the clause, see \citealt{PesetskyTorrego2007}). Note in this regard that the successive-cyclic movement of the determiner, as discussed below, could do double duty and mitigate concord. I leave this interesting option open here.} There is one exception. I assume that Saxon Genitives are also part of simple DPs despite the fact that they do not agree in features with the rest of the nominal. To be clear then, I take simple DPs to involve canonical constructions. Since the low position of determiners plays an important role in the analysis to be developed, I review some previous work on it.

Cases of Double Definiteness have received much attention in the literature on the Scandinavian languages; for instance, Norwegian \textit{den gamle mannen} ‘the old man’ involves a free-standing determiner (\textit{den}) and a suffixal determiner on the noun (-\textit{en}).\footnote{For cases of double articulation of (in-)definiteness in other languages and other constructions, see \citet{Plank2003} and \citet{Alexiadou2014}.} \citet[428]{Taraldsen1990} was one of the first to propose that the suffixal determiner originates in a D-type position below prenominal adjectives and that the free-standing determiner appears in a second D-type position above adjectives. \citet{Julien2002, Julien2005a} elaborated on that providing arguments that the low determiner position is tied to the feature [\textsc{specificity}] in certain Scandinavian languages (see also \citealt{Schoorlemmer2012}; cf. \citealt{IhsanePuskás2001}). \citet{Roehrs2002, Roehrs2009a} proposed that both of these determiner positions are related by movement accounting for the different interpretations of adjectives as regards (non-)restrictiveness (see also \chapref{sec:4}, \sectref{sec:4.4}). In order to explain the Definiteness Cycle in the history of English, \citet{Nykiel2015} employed determiner movement from a low position as well (see also the discussion of Old Norse in  \citealt{vanGelderen2007}: 293--94). \citet[Chapter 4]{Borer2005} derives the mass/count distinction of nouns by assuming the absence or presence of a Classifier Phrase (cf. NumP below) and movement of determiners from this low position to higher positions in the nominal structure (see also \citealt{Rehn2019}: 172--73 and \citealt{HeycockZamparelli2005}).\footnote{Unlike the current account, which merges determiners, singular or plural, in ArtP, \citet[123]{Borer2005} has to assume that, given her assumption about plural morphology, singular and plural determiners are merged in different positions (singular determiners in ClP, but plural ones in the higher \#P; cf. CardP below).} \citet{Schoorlemmer2009} utilized the two determiner locations in his analysis of concord of the noun phrase (see \chapref{sec:2}, \sectref{sec:2.5.2}). \citet{HeckEtAl2008} proposed that a definiteness feature moves from N to D when a prenominal adjective is present. Finally, separating the definiteness feature into different components, \citet{Roehrs2015,Roehrs2019} provided evidence that the low position of some of these definiteness components explains adjective endings in the Scandinavian languages and certain interactions between demonstratives/possessives and suffixal determiners.\footnote{There are other proposals that assume that determiners originate in lower positions. In her analysis of constructions involving DP-internal degree Quantifier Raising (e.g., \textit{an idiot of a doctor}, \textit{too big (of) a house}), \citet[272-73]{Matushansky2002} locates the indefinite and definite articles in a head position below the DP-level (presumably Card in the current analysis). Similarly, \citet{WoodVikner2011,WoodVikner2013} argue that there are two positions in which the indefinite article can occur in doubling cases of the type \textit{a such a hotel} and \textit{a so bad a hotel}: D and Num (which is Card here). Furthermore, in order to account for the referential defectiveness of French articles (\citealt{VergnaudZubizarreta1992}),  \citet[428-31]{DéchaineWiltschko2002} propose that the definite article in French is introduced below the DP-level.}

  Adopting a similar position here, I assume that the determiner moves from ArtP to the DP-level in a successive-cyclic fashion. Leaving out the numeral in \figref{figex:1:57} above and adding the adjective \textit{rot} ‘red’, the derivation can be updated as shown in \figref{figex:1:58}.

\glltree[\label{figex:1:58}\small]{\normalsize
    \gll die kleinen roten Autos\\
        the small red cars\\
    \glt ‘the small red cars’
}{
% \scalebox{.96}
[DP
    [~]
    [D{$'$}
        [{D\\\textit{die}$_i$},name=d]
        [AgrP
            [{AP\\\textit{kleinen}}]
            [Agr$' $
                [{Agr\\\textit{\sout{die}}$_i$},name=agr2]
                [AgrP
                    [{AP\\\textit{roten}}]
                    [Agr$'$
                        [{Agr\\\textit{\sout{die}}$_i$},name=agr1]
                        [ArtP
                            [~]
                            [Art$'$
                                [{Art\\\textit{\sout{die}}$_i$},name=art]
                                [NumP
                                    [~]
                                    [Num$'$
                                        [Num
                                            [\textit{Autos}$_k$]
                                            [Num$_{[+\textsc{pl}]}$]
                                        ]
                                        [{NP\\t$_k$}]
                                    ]
                                ]
                            ]
                        ]
                    ]
                ]
            ]
        ]
    ]
]
\draw[->](art)  to [bend left=45] (agr1);
\draw[->](agr1) to [bend left=45] (agr2);
\draw[->](agr2) to [bend left=45](d);
}


I assume that definite articles move to D by head adjunction as shown in \figref{figex:1:58} but that (phrasal) demonstratives move to Spec,DP by adjunction to phrasal projections.\footnote{Head movement of the definite article in \figref{figex:1:58} is illustrated in a simplified way. Note that there are well-known problems with assuming that head movement is syntactic and proceeds by head adjunction. Among others, head adjunction constitutes a violation of the Extension Condition \citep{Chomsky1995}. Solutions to these issues have been proposed; for instance, \citet[45-46]{Borer2005} argues for head-pairs where one element of this pair is merged above the other (but crucially not by head adjunction). The higher element of that pair can move up to the next phrasal level merging above the head of this phrase forming another head-pair – there is no violation of the Extension Condition (note that there are other, related analyses: \citealt{Matushansky2006}: 91-93 proposes that head movement consists of displacement of a lower head to the specifier of a higher phrase followed by m-merger of this head with the head of the higher phrase; “excorporation” is possible if m-merger does not take place; \citealt{Giusti2015}: 84, 116-26 argues that articles are realizations of remerging N-heads).  In what follows, I simply assume that heads raise by adjunction to the next higher head. Head nouns raise to adjoin to Num, and these two types of elements are spelled out as the appropriate singular or plural form of the noun. As for determiners, I assume that they also raise by adjunction to the next head. However, I assume that after adjunction, they excorporate and move to the next higher head (keeping in mind that there are other solutions, as just mentioned). Note that unlike the movement of nouns, the movement of determiners is not structure-building – it involves excorporation.}

Unlike case, number, and gender, definiteness is not a concord feature. Rather, I assume that definite determiners like \textit{der} ‘the’ have the feature [+DEF] as part of their structure (see \sectref{sec:1.4.1.2}) and move to the DP-level to specify the definiteness feature on D such that [uDEF] becomes [+DEF]. The article \textit{ein} ‘a’ is different. Proposing that it is a semantically vacuous element, it does not have a feature for definiteness. This means that when \textit{ein} is present, D has no definiteness feature.\footnote{This implies that there are two types of D in argument DPs, one with a feature for definiteness and one without it. Alternatively, we could postulate just one type of D and assume that its definiteness feature can but does not have to be checked/valued. The question of whether there is one or two types of D extends to indefinite nominals more generally. Predicate DPs are indefinite by form (e.g., \textit{Er ist ein netter Mann.} ‘He is a nice man.’), but not by meaning (they denote properties). This is different for indefinite argument DPs, which may assert the existence of an entity (e.g., \textit{Ein netter Mann war da}. ‘A nice man was there.’). Considering that \textit{ein} occurs in both types of nominals, it is clear that \textit{ein} itself does not bring about semantic indefiniteness, as stated in the main text. Note in this regard that the current assumptions are in line with \citegen{Pfaff2017} analysis of the strong/weak alternation of adjectives in Icelandic, where definite DPs have a definiteness feature, but indefinite DPs, predicative and argumental, do not. Given that the question of the different types of D is a more general issue, I leave it for future research and continue assuming that \textit{ein} has no definiteness feature.} Given the lack of a definiteness feature on \textit{ein}, this element can surface in lower positions (cf. also \citealt{Borer2005}: 144-59, \citealt{Rehn2019}). Among others, this allows me to postulate structures smaller than DP, for instance, when nominals involving \textit{ein} occur in predicative contexts without a modifier (e.g., \textit{Er ist ein Mann} ‘He is a man’; see \chapref{sec:6}).

There is overt distributional evidence that determiners originate in low positions in German (for detailed cross-linguistic discussion, see \citealt{Roehrs2009a} and references cited therein). \citet[255]{Sternefeld2008} reports work by Horst Simon, who observes that the negative article \textit{kein} ‘no’ can be repeated in dialectal German \REF{ex:1:59a}.\footnote{This dialect is not further specified by Sternefeld. Note that \citet[187]{HaegemanLohndal2010} provide an example from West Flemish where the negative article appears to be in a low position similar to \REF{ex:1:59a}.} Sternefeld concludes that another position is needed for the second (low) instance. Above, I identified this position as Art. Furthermore, I noted above that colloquial German also tolerates two indefinite articles \REF[b-c]{ex:1:59}.

\ea%59
    \label{ex:1:59}
\ea\label{ex:1:59a}
Dialectal German\\
\gll Ich hab   k-ein  blau-es k-ein   Kleid         nicht.\\
    I     have \textsc{neg}-a blue\textsc{-st} \textsc{neg}-a dress.\textsc{neut} not\\
\glt ‘I don’t have a blue dress.’
\ex\label{ex:1:59b}
\gll k-ein   so’n Kleid\\
    \textsc{neg}-a {so a}  dress.\textsc{neut}\\
\glt ‘no such dress’
\ex\label{ex:1:59c}
\gll EIN       so’n Kleid\\
    $\emptyset$\textsubscript{[--}\textsc{\textsubscript{pl}}\textsubscript{]}+a {so a} dress.\textsc{neut}\\
\glt ‘one such dress’
\z
\z

Observe that unlike the recurring indefinite articles in Norwegian\linebreak (\sectref{sec:1.2.1.3}), in German \textit{ein} can only be repeated if both instances of \textit{ein} are part of another element, for instance, the negative article \textit{kein} ‘(not a =) no’ and the type particle \textit{so’n} ‘such a/of that kind’ (see also \chapref{sec:8}). Given the works cited above and the empirical evidence just provided, I take it as established that determiners originate low in the structure.

With the structure of canonical DPs in place, I should point out that the current proposal is based on morpho-syntactic subanalyses where words are separated into stems and inflections, each involving a position in the syntactic representation as explicated in greater detail in the next subsections.

\ea%60
    \label{ex:1:60}
\ea\label{ex:1:60a}
\gll gut-er\\
    good-\textsc{st}\\
\glt ‘good’
\ex\label{ex:1:60b}
\gll dies-er\\
    this-\textsc{st}\\
\glt ‘this’
\z
\z

\largerpage[-1]
Note that these subanalyses are part of a more general approach to morpho-syntax involving the segmentation of word and morpheme forms into smaller units (e.g., \citealt{Anderson1992}; \citealt{DéchaineWiltschko2002}: 422; \citealt{Fischer2006}; \citealt{GunkelEtAl2017}: 1297-98; \citealt{Leu2015}; G. \citealt{Müller2002a}: 125; \citealt{Olsen1991b}: 38; \citealt{Pike1963,Pike1965}; \citealt{Rehn2019}; \citealt{Roehrs2013a}; \citealt{Stump2001}; B. \citealt{Wiese1999}; \citealt{Wiltschko1998}, \citealt{Wunderlich1997}; but see also \citealt{JandaJoseph1992}). Furthermore, similar to Distributed Morphology, which is utilized here (see \sectref{sec:1.4.2.1}), the recent approach of nanosyntax also involves analyses at the submorphemic level (see \citealt{Hachem2015} and \citealt{Starke2009}). However, the latter approach explicitly rejects Impoverishment rules (\citealt{ArregiNevins2012}: 341), a type of operation essential to the account to be developed below. I return to some of these works in later parts of the book. With this in mind, I discuss determiners and related elements in the next subsection, and adjectives and similar items in the subsection after that.

\subsubsection{Determiners, articles, and determiner-like elements}\label{sec:1.4.1.2}
\largerpage[-1]
In German, determiners are categorized into three groups: \textit{der}-words, \textit{ein}-words, and null articles (e.g., \citealt{RankinWells2016}).\footnote{For complex determiners like \textit{ein jeder} ‘(an every =) each’, see, e.g., \citealt{KarnowskiPafel2004}: 172-76, \citealt{Pafel1994}, and below.} The first set includes \textit{der} ‘the’, (stressed) \textit{DER} ‘that’, \textit{dieser} ‘this’, \textit{jener} ‘that’, \textit{jeder} ‘every’, \textit{mancher} ‘some’, \textit{solcher} ‘such’, \textit{welcher} ‘which’, and \textit{alle} ‘all’ in the plural (and occasionally with mass and abstract nouns; for more details on the use of these elements, see \citealt{ZifonunEtAl1997}: 1930--51). Second, as already stated above, \textit{ein}-words are comprised of the indefinite article \textit{ein} ‘a’ including its reduced form \textit{’n}, the (stressed) singularity numeral \textit{EIN} ‘one’, the negative article \textit{kein} ‘no’, and possessive articles like \textit{mein} ‘my’, \textit{dein} ‘your’, etc. The third category involves null articles that occur with mass and plural nouns in indefinite contexts. They are indicated here by $\emptyset$\textsubscript{D}.

  Articles are a specific subgroup of determiners. They comprise the definite article \textit{der} ‘the’, all \textit{ein}-words, and the null articles (I argue in detail in \chapref{sec:5} that complex \textit{ein}-words are composites consisting of the vacuous article \textit{ein} and another, semantic component). Finally, possessives (including the possessive components of the possessive articles) and predeterminers are labeled determiner-like elements here. I assume that all determiners originate in ArtP. This is different for determiner-like elements as explained further below.

  I argue in \chapref{sec:2} that articles are heads consisting of a categorial feature [+D]. The indefinite null articles are illustrated in \figref{figex:1:61a}, which includes a definiteness feature with a negative value. Unlike the null articles, the overt articles in \figref{figex:1:61}(b-c) have a separate feature bundle for case, number, and gender ([CNG]), later to be spelled out as the inflection.\footnote{I assume that agreement features are part of syntax (see also \citealt{Murphy2018}; for the advantages and disadvantages of such an assumption in the framework of Distributed Morphology, see \citealt{EmbickNoyer2007}: 305-10).} Above, I proposed that \textit{ein} is a semantically vacuous element. As such, it does not have a feature for definiteness, see \figref{figex:1:61b}. By contrast, the definite article \textit{der} ‘the’ has a feature for definiteness. However, unlike the null indefinite articles, the value of the definiteness feature is positive, see \figref{figex:1:61c}.\footnote{\label{foot:1:46}I finalize the discussion of Vocabulary Insertion of the determiners in \chapref{sec:8}, Sections \ref{sec:8.2.2.5} and \ref{sec:8.2.2.6}. Also, note that the existence of null articles is a fairly standard assumption (e.g., \citealt{Carnie2021}: 271, \citealt{Schoorlemmer2009}). Among others, it allows us to claim that all argument DPs have determiners. In the context of the current analysis, their existence also helps explain the distribution of the article \textit{ein} vs. adjectival \textit{eine} (\chapref{sec:5}, \sectref{sec:5.5.2.3}). Furthermore, assuming that the null articles here have the feature [-DEF] has a number of advantages: (i) It makes $\emptyset$\textsubscript{D} parallel to \textit{der} ‘the’ in terms of (in-)definiteness, (ii) it provides a straightforward solution to the distribution of articles with certain operators (\chapref{sec:8}, Sections \ref{sec:8.2.2.5} and \ref{sec:8.2.2.6}), and (iii) it makes \textit{ein} the least specified determiner (stem).}

\begin{figure}
    \subfigure[Indefinite articles $\emptyset$\textsubscript{D}\label{figex:1:61a}]{
    \hspace*{1cm}\begin{forest}
    [Art
        [{[+D; --DEF]}]
    ]
    \end{forest}\hspace*{1cm}
}

\subfigure[Indefinite article \textit{ein}]{
    \label{figex:1:61b}
    \hspace*{1cm}\begin{forest}
    [Art
        [{[+D]}]
        [{[CNG]}]
    ]
    \end{forest}\hspace*{1cm}
}

\subfigure[Definite article \textit{der}]{
    \label{figex:1:61c}
	    \begin{forest}
        [Art
        [{[+D; +DEF]}]
        [{[CNG]}]
        ]
    \end{forest}
}
\caption{Structures of articles}
\label{figex:1:61}
\end{figure}

\largerpage
Abstracting away from the adjectival inflection, note already here that \textit{ein} is the least specified article in German – it only has the categorial feature [+D].

Demonstratives such as \textit{dieser} ‘this’ are phrasal in structure (e.g., \citealt{Bernstein1997,Bernstein2001b}; \citealt{Brugè1996,Brugè2002}; \citealt{Giusti1997,Giusti2002}; \citealt{RauthSpeyer2021}). For simplicity’s sake, I assume that there are two terminal heads, see \figref{figex:1:62}. Dem involves the features [+D; +DEF, +DEIX] and builds its own extended projection with an Inflectional Phrase (InflP) at the top (see \citealt{Leu2007,Leu2015}; \citealt{Roehrs2010,Roehrs2013a}).\footnote{The head Dem involving the features [+D; +DEF, +DEIX] could be analyzed as two heads, Art with the features [+D; +DEF] and a deictic head Deic with the feature [+DEIX] (for discussion, see the above-mentioned works). Since these finer distinctions do not play a role here, I continue utilizing the simpler analysis in the main text.} The features for case, number, and gender are in Infl. Dem moves to adjoin to Infl (not shown).\clearpage

\begin{figure}
	\caption{Demonstrative \textit{dieser}}
    \label{figex:1:62}
    \begin{forest}
      [InflP
        [{[CNG]}]
        [DemP
            [{Dem\\{}[+D;+DEF,+DEIX]}]
        ]
      ]
    \end{forest}
\end{figure}

As will become clear in \chapref{sec:3}, \sectref{sec:3.4}, phrasal determiners have some special properties; for instance, they can optionally have a weak ending in genitive masculine/neuter contexts. Besides \textit{dieser} ‘this’, this holds for \textit{jener} ‘that’, \textit{jeder} ‘every’, \textit{mancher} ‘some’, \textit{solcher} ‘such’, \textit{welcher} ‘which’, and \textit{alle} ‘all’. I assume that all these elements have a structure similar to \figref{figex:1:62}.\footnote{The demonstrative \textit{DER} ‘that’ seems to have an intermediate status. On the one hand, this element is expected to project a phrasal structure (similar to the demonstrative \textit{dieser} ‘this’); on the other hand, it behaves like the definite article \textit{der} ‘the’ in not exhibiting the optionality of strong or weak inflection in the genitive masculine/neuter.}

  Possessives will not feature prominently in the following discussion. They involve complex structures that deserve more discussion than I can provide here (but see \citealt{Roehrs2013b,Roehrs2020a}). However, they need to be briefly addressed as they are mentioned in various contexts below. Focusing on prenominal possessives, there are different types: possessive articles involving the possessive component \textit{s}- and \textit{ein} \REF{ex:1:63a}, Saxon Genitives consisting of a possessor and the possessive marker -\textit{s} \REF{ex:1:63b}, and \textit{von}-possessives comprising a possessor and the possessive preposition \textit{von} ‘of’ \REF{ex:1:63c} (see also \citealt{Bhatt1990}: 223-24, \citealt{Fortmann1996}, \citealt{Haider1992}: 315).

\ea%63
    \label{ex:1:63}
\ea\label{ex:1:63a}
\gll sein Auto\\
    his  car.\textsc{neut}\\
\glt ‘his car’
\ex\label{ex:1:63b}
\gll Peters  Auto\\
    Peter’s car.\textsc{neut}\\
\glt ‘Peter’s car’
\ex\label{ex:1:63c}
\gll von Peter das Auto\\
    of   Peter  the car.\textsc{neut}\\
\glt ‘the car of Peter’
\z
\z

Following \citet{Roehrs2013b,Roehrs2020a}, I assume that possessives involve multi-com-ponent constituents consisting of a possessive functor and a possessor, see \figref{figex:1:64}. Possessors may involve overt or null elements (more on \textit{pro} below). The possessive functor forms the head of a Possessive Phrase (PossP) taking the possessor as its complement. A functional phrase (XP) is at the top. This yields the following underlying structure of the possessives in \REF{ex:1:63}.

\begin{figure}
	\caption{Structure of possessives}
    \label{figex:1:64}
    \begin{forest}
      [XP
        [~]
        [X$'$
            [X]
            [PossP
                [~]
                [Poss$'$
                    [{Poss\\
                        \textit{s-}\\
                        \textit{-s}\\
                        \textit{von}}]
                    [{DP\\
                    \textit{pro}\\
                    \textit{Peter}\\
                    \textit{Peter}}]
                ]
            ]
        ]
      ]
    \end{forest}
\end{figure}

Note that similar to demonstratives, possessives involve extended projections. I assume that all Poss heads, that is, the possessive component \textit{s}-, the possessive marker -\textit{s}, and the possessive preposition \textit{von} ‘of’, move to X in \figref{figex:1:64}. If \textit{s}- or -\textit{s} move, this brings about a nominal extension of Poss; if \textit{von} moves, it results in a prepositional extension of Poss. Furthermore, the possessor DP undergoes movement to Spec,XP in the first two cases but stays in situ with \textit{von}-possessives (for more details, see \citealt{Roehrs2020a}: 123). This yields the correct surface strings for Saxon Genitives (\textit{Peters}) and \textit{von}-possessives (\textit{von Peter}).

As to the possessive articles, consider briefly the periphrastic possessive construction \textit{Peter sein Auto} ‘(Peter his =) Peter’s car’ \citep{Fiva1985}, where a possessor is added to the left. This distribution is sometimes referred to as Possessor Doubling. If we assume with \citet[315]{Haider1992},  \citet[53]{GrohmannHaegeman2003}, and \citet[611]{AlexiadouEtAl2007} that \textit{Peter} can be replaced by a null argument (\textit{pro}), then the periphrastic possessive construction in \REF{ex:1:65a} can be directly related to the simple possessive article in \REF{ex:1:65b}.

\newpage
\ea%65
    \label{ex:1:65}
\ea\label{ex:1:65a}   [\textsubscript{XP} \textit{Peter s}-]ein Auto
\ex\label{ex:1:65b}   [\textsubscript{XP} \textit{pro s}-]ein Auto
\z
\z

The possessive structure XP in \figref{figex:1:64} can surface in different positions inside the larger DP. I assume that XP is base-generated in Spec,NP, where it may receive a theta role (depending on the type of head noun). Recall that the head noun moves to Num. If XP stays in situ, postnominal possessives are derived; if XP moves higher, prenominal possessives come about. Focusing on the latter again and abstracting away from the movement of the article, the possessive component \textit{s}- and its possessor (\textit{Peter} or \textit{pro}) move to Spec,DP \REF{ex:1:66a}. The Saxon Genitive consisting of the possessive marker -\textit{s} and its possessor also moves to Spec,DP \REF{ex:1:66b}.\footnote{Note again that possessive -\textit{s} is part of the specifier (and not in D). This is \citegen[81--85]{Abney1987} “preferred” analysis (see also \citealt{Borer2005}: 40-41, \citealt{Lowe2016}: 172, 174, and others).} Unlike those two cases, the \textit{von}-possessive moves to a higher specifier. \citet{GiustiIovino2016} propose that there is a Left Periphery Phrase (LPP) above the DP-layer. I assume that the \textit{von}-possessive moves to the specifier of that phrase.\footnote{Rather than LPP, we could postulate K(ase)P (see \citealt{Pfaff2017}: 294 for discussion and references). However, I prefer the functionally neutral name Left Periphery Phrase here.} This derives the data in \REF[a-c]{ex:1:63} above as follows.

\ea%66
    \label{ex:1:66}
\ea\label{ex:1:66a}   [\textsubscript{DP} [\textsubscript{XP} \textit{Peter}/\textit{pro} \textit{s}-]\textsubscript{k} \textit{ein} [\textsubscript{NumP} \textit{Auto}\textsubscript{i}+Num [\textsubscript{NP} t\textsubscript{k} t\textsubscript{i}]]]
\ex\label{ex:1:66b}   [\textsubscript{DP} [\textsubscript{XP} \textit{Peter}s]\textsubscript{k} $\emptyset$\textsubscript{D} [\textsubscript{NumP} \textit{Auto}\textsubscript{i}+Num [\textsubscript{NP} t\textsubscript{k} t\textsubscript{i}]]]
\ex\label{ex:1:66c}   [\textsubscript{LPP} [\textsubscript{XP} \textit{von} \textit{Peter}]\textsubscript{k} LPP [\textsubscript{DP} \textit{das} [\textsubscript{NumP} \textit{Auto}\textsubscript{i}+Num [\textsubscript{NP} t\textsubscript{k} t\textsubscript{i}]]]]
\z
\z
 
Comparing \REF{ex:1:66a} to \REF{ex:1:66b}, I follow \citet{Krause1999} and \citet{Roehrs2013b,Roehrs2020a} in that the periphrastic possessives (including the possessive articles involving \textit{pro}) and the Saxon Genitives have essentially the same structure.

While I cannot argue for these structures and derivations in more detail here (see references mentioned above), one immediate advantage is that all prenominal possessives in German have the same basic internal structure (XP) and that they all have the same basic derivation undergoing movement from a low position to a high position within the larger noun phrase. Indeed, each noun phrase involves a determiner in D: \textit{ein} in \REF{ex:1:66a}, $\emptyset$\textsubscript{D} in \REF{ex:1:66b}, and the definite article in \REF{ex:1:66c}.\footnote{\label{foot:1:51} At this point, an issue arises. As is well known, possessors in prenominal position may spread their (in-)definiteness to the larger noun phrase (e.g., \citealt{Alexiadou2005}, \citealt{Roehrs2022}). If this definiteness spread is instantiated by Spec-head agreement between Spec,DP and D, then definite Saxon Genitives and null articles specified as [-DEF] are incompatible (cf. \REF{ex:1:66b}). In Footnote \ref{foot:1:46}, I pointed out some advantages of assuming that null articles exist and that they have a negative feature for definiteness. Given these points, I proceed with the assumption of a negative feature for definiteness, but I offer two solutions to this issue in \chapref{sec:8}, \sectref{sec:8.2.2.6} (once further discussion is in place).}

Finally, predeterminers are intensifiers like \textit{alle} ‘all’, \textit{diese} ‘these’, and \textit{ein} ‘a’ emphasizing exhaustiveness, deixis, and distributivity, respectively.

\ea%67
    \label{ex:1:67}
\ea\label{ex:1:67a}
\gll alle meine Freunde\\
    all   my     friends\\
\glt ‘all my friends’
\ex\label{ex:1:67b}
\gll diese meine Freunde\\
    these my     friends\\
\glt ‘these friends of mine’
\ex\label{ex:1:67c}
\gll ein jeder meiner Freunde\\
    an  every of.my  friends\\
\glt ‘each of my friends’
\z
\z

Note that predeterminers form a subset of determiners. In other words, all predeterminers also do double duty as determiners (but not vice versa). I propose below that all these elements have the categorial feature [+D]. As determiners, they are base-generated in ArtP and move to the DP-level. However, if ArtP already contains a determiner, a second determiner element can only be merged as a predeterminer provided it can supply an intensifying meaning compatible with the rest of the nominal. Given that there is only one determiner in the DP-layer, I assume that predeterminers are located in LPP, as shown by \textit{alle} below.

\ea%68
    \label{ex:1:68}
          [\textsubscript{LPP} \textit{alle} [\textsubscript{DP} \textit{meine Freunde}]]
\z

As such, I assume that LPP can be occupied by prenominal \textit{von}-possessives and predeterminers. To distinguish predeterminers and possessives from determiners, I often refer to the former set of elements as determiner-like elements.

\subsubsection{Adjectives, numerals, and quantifiers}\label{sec:1.4.1.3}

Elements in specifiers project their own phrases. Above, I illustrated this with demonstratives and possessives. Each of these elements project their own complex structure. In this subsection, I turn to adjectives and numerals/quantifiers. We see that AP and QP in \figref{figex:1:57} above are also more complex.

Starting with adjectives in \figref{figex:1:69}, I assume that they also involve extended projections (e.g., \citealt{Corver1991,Corver1997}; \citealt{NeelemanDoetjes2004}; \citealt{Zamparelli2000}: Chapter 7). Minimally, the adjective projects an AP, and InflP is at the top (e.g., \citealt{Corver2006}: 68, \citealt{Leu2015}, \citealt{SappRoehrs2016}). The adjective stem is located in A and the inflection in Infl. Both combine by movement of the adjective to Spec,InflP (not shown, but see \chapref{sec:2}, \sectref{sec:2.2.4}).

\begin{figure}
	\caption{Structure of adjectives}
    \label{figex:1:69}
\begin{forest}
  [InflP
    [~]
    [Infl$'$
        [{[CNG]}]
        [AP
            [~]
            [A$'$
                [{A\\\textit{klein}}]
                [~]
            ]
        ]
    ]
  ]
\end{forest}
\end{figure}

Finally, numerals (e.g., \textit{zwei} ‘two’) and quantifiers (e.g., \textit{viele} ‘many’) are assumed to have the same representation as \figref{figex:1:69}, where Q is at the bottom projecting more structure. The internal makeup of the singularity numeral \textit{EIN} ‘one’ is discussed in detail in \chapref{sec:5}.

These are the basic structural assumptions of the noun phrase that I take to be fairly uncontroversial. Against this backdrop, a number of other, non-canonical constructions involving adjectival inflections and/or \textit{ein} are discussed in later sections. It will become clear that not all nominal constructions can have the same structure. More importantly, as already mentioned above, I argue that it is these types of different constructions that reveal the true nature of adjectival inflections and \textit{ein}.

\subsection{Distributed Morphology and Type Theory}\label{sec:1.4.2}

Besides the general syntactic assumptions in the previous section, I make use of Distributed Morphology and semantic Type Theory to account for adjectival inflections and \textit{ein}.

\subsubsection{Distributed Morphology}\label{sec:1.4.2.1}

I usually provide syntactic representations of the form in \figref{figex:1:57} and \figref{figex:1:58}, where vocabulary items are directly merged in the syntactic tree, that is, during the syntactic derivation. This is a convenient shortcut. While not relevant in many cases, I assume that such syntactic representations are actually built on terminal nodes containing morpho-syntactic features (rather than vocabulary items). This is a basic tenet of Distributed Morphology (DM), a realizational approach to morphology where the derivation of complex words is separate from the spelling out (or realization) of those words (see, e.g., \citealt{ArregiNevins2012}: 3-11; \citealt{EmbickNoyer2007}; \citealt{Halle1997}; \citealt{HalleMarantz1993,HalleMarantz1994}; \citealt{HarleyNoyer1999}; \citealt{Sauerland1996}).

In DM, terminal nodes of a syntactic tree involve abstract features. These features can be manipulated in certain ways; for instance, features can be rearranged (Lowering) or deleted (Impoverishment). After Linearization, the resultant features are spelled out by vocabulary items. In other words, the insertion of these items occurs late in the derivation (after syntax) and depends on the specific morpho-syntactic features present on the terminal nodes. As just stated, these features may have undergone some changes.

Vocabulary Insertion is regulated by the Subset Principle. Provided that all features of the vocabulary item match those on the terminal node, the vocabulary item with the most specifications, that is, the most matching features, is inserted over less-specified items. When a more specific vocabulary item cannot be inserted (i.e., at least one of its features does not match those on the terminal head), the vocabulary item with fewer or no features is inserted as the elsewhere case. In other words, it is assumed in DM that vocabulary items may be underspecified as regards their morpho-syntactic features. This allows competition between related items in certain contexts.

After insertion, vocabulary items can undergo Local Dislocation combining with other vocabulary items. Unlike Lowering (which operates on hierarchical structures), Local Dislocation involves linear adjacency. Note that while not all these operations are equally important in this book, they are assumed to occur in the general order below (see also \citealt{Murphy2018}: 359). The operations whose orderings are not relevant here are enclosed in curly brackets.\footnote{(Late) Vocabulary Insertion usually concerns functional elements but not lexical ones. The latter are merged as terminal nodes and involve phonological features (\citealt{EmbickNoyer2007}: 295).}

\ea%70
    \label{ex:1:70}

          \{Lowering, Impoverishment\} >> Linearization >> Vocabulary Insertion >> Local   Dislocation
\z

It is clear that Impoverishment precedes Vocabulary Insertion, and we observe in \chapref{sec:2} that this operation seems to have access to structural information. As such, Impoverishment has been ordered with Lowering. This is compatible with   \citet[342]{ArregiNevins2012}, who order Impoverishment before Linearization and the latter before Vocabulary Insertion.

  Impoverishment plays an important part in the analysis to be developed. To illustrate briefly, inflections are taken to be the overt spell-out of abstract feature bundles on terminal nodes in syntax.\footnote{I assume that terminal nodes may contain feature bundles (rather than just individual features).}  These feature bundles can be manipulated in certain ways. \citet{Sauerland1996} proposes for Norwegian (see \chapref{sec:2} for German) that the weak inflections on adjectives are the result of the deletion of a feature in contexts where strong inflections occur otherwise. To see this, consider the inflectional paradigms of the strong and weak endings in \tabref{tab:1:2} and \tabref{tab:1:3}.

\begin{table}
\caption{Strong adjective inflections in Norwegian}
\label{tab:1:2}

\begin{tabular}{lll}
\lsptoprule
          & --neuter & +neuter\\
\midrule
--plural & -Ø & -t\\
+plural & -e & -e\\
\lspbottomrule
\end{tabular}
\end{table}

\begin{table}
\caption{Weak adjective inflections in Norwegian}
\label{tab:1:3}
\begin{tabular}{lll}
\lsptoprule
      & --neuter & +neuter\\
\midrule
--plural & -e & -e\\
+plural & -e & -e\\
\lspbottomrule
\end{tabular}
\end{table}

Sauerland establishes the generalization that the weak inflections (-\textit{e}) form a proper subset of the strong inflections (-\textit{e}, -$\emptyset$, -\textit{t}). He observes that the weak endings occur in the same featural contexts as their homophonous strong endings (i.e., in [+plural] contexts in \tabref{tab:1:2} and \tabref{tab:1:3}) and, additionally, in other environments (i.e., [--plural] in \tabref{tab:1:3}). Sauerland proposes that the weak inflections are the least marked endings from the set of strong endings. In other words, there is no difference between strong and weak inflections other than the degree of their specificity. He argues that the weak endings appear when the feature bundles of the strong inflections undergo the deletion of a feature. This feature deletion is instantiated by Impoverishment.

\largerpage[2]
Sauerland formulates the following vocabulary insertion rules.
  
\TabPositions{3cm}
\ea%71
    \label{ex:1:71}
\ea\label{ex:1:71a}   [--plural, +neuter] \tab $\rightarrow$   -\textit{t}
\ex\label{ex:1:71b}   [--plural, --neuter]  \tab $\rightarrow$   -$\emptyset$
\ex\label{ex:1:71c}   []  \tab $\rightarrow$   -\textit{e}
\z
\z\clearpage

He proposes for Norwegian that Impoverishment deletes the gender feature in the syntactic representation. If the feature [neuter] is deleted from the relevant terminal node, then \REF{ex:1:71a} and \REF{ex:1:71b} can no longer be inserted in that terminal node. As a consequence, only \REF{ex:1:71c} can be inserted resulting in weak inflections on adjectives.

  Note that the exact conditions as to when Impoverishment is triggered in Norwegian are not provided (but Sauerland states that those conditions are morpho-syntactic, at least in German). For some discussion of the contexts and conditions where Norwegian weak endings appear, see \citet{Julien2005a}; \citet{KatzirSiloni2014}; \citet{RoehrsJulien2014}; \citet{Schoorlemmer2009,Schoorlemmer2012}, and many others (for the discussion of related Icelandic, see \citealt{Pfaff2017}). The morpho-syntactic conditions postulated by Sauerland for German are reviewed and discussed in more detail in Chapter 2. With this brief illustration of DM in place, I return to the syntax for a moment.

I assume with \citet{Nunes2001} that Move, more recently termed Internal Merge, is not a primitive operation but rather the output of Copy, Merge, Form Chain, and Copy Reduction. I assume that the first three operations apply to feature bundles which are later spelled out by the corresponding vocabulary items. We see later that the fourth operation, Copy Reduction, is sensitive to the overt form of vocabulary items (free vs. bound). As such, I assume that this operation applies after Vocabulary Insertion. This is compatible with \citet{Schoorlemmer2012}, who explains cases of Double Definiteness in Swedish DPs by ordering Copy Reduction after Local Dislocation.\footnote{\citet{Nunes2001} argues that Copy Reduction takes place before Linearization. For discussion in favor of the ordering Copy Reduction after Linearization, see \citet[135-37]{Schoorlemmer2012}.} For convenience and expository simplicity, especially when the underlying details of the derivation are not relevant, I indicate movement by indexed traces and provide the more familiar syntactic representations given in \sectref{sec:1.4.1.1}.

\subsubsection{Type Theory}\label{sec:1.4.2.2}

This book is not about the semantics of the DP \textit{per se} (or the semantics of the TP, for that matter). Nevertheless, I make some statements about the semantics involved. Rather than providing detailed denotations (see \citealt{Roehrs2009a}), I use Type Theory to show that the relevant elements are semantically compatible with one another. In other words, the goal of employing Type Theory is to find plausible structures by narrowing down the analytical options.

To illustrate the main assumptions of Type Theory, consider the following three sentences.

\ea%72
    \label{ex:1:72}
\ea\label{ex:1:72a}
\gll Peter ist blond.\\
Peter is  blond\\
\glt ‘Peter is blond.’

\ex\label{ex:1:72b}
\gll Er ist blond.\\
he is  blond\\
\glt ‘He is blond.’

\ex\label{ex:1:72c}
\gll Der Mann ist blond.\\
 the  man    is  blond\\
\glt ‘The man is blond.’
\z
\z

What all these sentences have in common is that somebody is blond. In fact, assuming that \textit{er} ‘he’ and \textit{der Mann} ‘the man’ are the same person as \textit{Peter}, all these sentences are true if a certain man called Peter has the property of being blond.

  I assume that the copular verb \textit{sein} ‘to be’ is semantically vacuous (\citealt{CoppockBeaver2015}: 399, \citealt{HeimKratzer1998}: 61--62). A sentence has a truth value of 1 if its proposition is true in a given context or 0 if it is false. Illustrating with \REF{ex:1:72a}, if a certain Peter is indeed blond in the given context, then this statement has a truth value of 1; that is, it is true. Truth values are of type \textlangle t\textrangle {} \REF{ex:1:73a}. It is usually assumed that individuals like \textit{Peter} are entities. They are of type \textlangle e\textrangle {} \REF{ex:1:73b}. Like \textit{Peter}, \textit{er} ‘he’ and \textit{der Mann} ‘the man’ are definite expressions and can replace \textit{Peter}. Thus, I assume that they are also of type \textlangle e\textrangle. Turning to (intersective) adjectives like \textit{blond} ‘blond’, they involve predicates. Put simply, predication involves membership of an element in a certain set of ordinary/individual entities. Specifically, the adjective \textit{blond} denotes a set of entities that are all blond. Predicates are of type \textlangle e,t\textrangle {} \REF{ex:1:73c} – they are functors combining with entities and return truth values. Finally, definite determiners like \textit{der} ‘the’ are functors as well presupposing uniqueness. They combine with predicates yielding entities \REF{ex:1:73d}.

  \TabPositions{4.5cm}
\ea%73
    \label{ex:1:73}
  \ea\label{ex:1:73a}   truth values (type \textlangle t\textrangle):\tab   \textit{Peter ist} \textit{blond}.
  \ex\label{ex:1:73b}   entities (type \textlangle e\textrangle):\tab     \textit{Peter}, \textit{er}, \textit{der Mann}
  \ex\label{ex:1:73c}   predicates (type \textlangle e,t\textrangle):\tab   \textit{blond}
  \ex\label{ex:1:73d}   determiners (type \textlangle\textlangle e,t\textrangle e\textrangle):\tab   \textit{der}
  \z
\z

Before commenting on \textit{der Mann} ‘the man’ in \REF{ex:1:73b}, note that Type Theory involves a combinatorial system whereby elements are related to one another, two at a time. There are two main operations. First, Functional Application combines a functor and an argument by plugging the latter into the former \REF{ex:1:74a}. For instance, the predicate \textit{blond} can be a functor, and it takes an entity, say \textit{Peter}, as its argument yielding a truth value, schematically: X\textsubscript{\textlangle e,t\textrangle}(Y\textsubscript{\textlangle e\textrangle}) = Z\textsubscript{\textlangle t\textrangle}. This yields the sentence in \REF{ex:1:72a}. Second, Predicate Modification combines two predicates; that is, it forms an intersection of two sets of certain entities \REF{ex:1:74b}, schematically: X\textsubscript{\textlangle e,t\textrangle}, Y\textsubscript{\textlangle e,t\textrangle} = X\textsubscript{\textlangle e,t\textrangle} \& Y\textsubscript{\textlangle e,t\textrangle}. For example, predicates like \textit{unbekannt}  ‘unknown’ and \textit{blond} ‘blond’ can be combined as in \textit{der unbekannte blonde Mann} ‘the unknown blond man’ such that a certain man is a member of the set of unknown entities and a member of the set of blond entities; that is, he has the properties of being both unknown and blond.

\ea%74
    \label{ex:1:74}
  \ea\label{ex:1:74a}   Functional Application:  functor(argument)
  \ex\label{ex:1:74b}   Predicate Modification:   predicate \& predicate
  \z
\z

Returning to \textit{der Mann} in \REF{ex:1:73b}, I assumed above that this string is of type \textlangle e\textrangle {} and that the determiner \textit{der} itself is of type \textlangle \textlangle e,t\textrangle e\textrangle. Given Functional Application, the noun \textit{Mann} is often taken to be a predicate, just like the adjective \textit{blond}. Note that here the predicate \textit{Mann} itself is the argument, and the determiner is the functor.

To illustrate the workings of Type Theory in a syntactic representation, consider the more complex nominal in \figref{figex:1:75}. Proceeding bottom-up and leaving out NumP and ArtP, \textit{Mann} passes up its semantic type to NP. The predicate NP and the intersective adjective \textit{blond} are both of the same type (\textlangle e,t\textrangle). They combine by Predicate Modification resulting in a complex element of the same type. Finally, the determiner \textit{der} is a functor that takes the complex predicate in AgrP as its argument returning an entity (\textlangle e\textrangle).


\glltree[\label{figex:1:75}]{\gll der blonde Mann\\
      the blond   man.\textsc{masc}\\
  \glt ‘the blond man’}{
  [DP\textsubscript{\textlangle e\textrangle}
    [\textit{der}\textsubscript{\textlangle \textlangle e,t\textrangle e\textrangle}]
    [AgrP\textsubscript{\textlangle e,t\textrangle}, edge label={node[midway,above right]{(Functional Application)}}
        [\textit{blonde}\textsubscript{\textlangle e,t\textrangle}]
        [{NP\textsubscript{\textlangle e,t\textrangle}\\
        \textit{Mann}\textsubscript{\textlangle e,t\textrangle}}, edge label={node[midway,right]{~~~~(Predicate Modification)}}]
    ]
  ]
%\node[above right = 5pt of der]{(Functional Application)};
%\node[above right = 5pt of Mann]{(Predicate Modification)};
}


These are the basic assumptions of Type Theory (as discussed in \citealt{HeimKratzer1998}), and they are sufficient to account for most cases. In \chapref{sec:6} and \chapref{sec:7}, I refine my assumptions about nouns and extend the discussion to pronominal determiners like \textit{du} ‘you’.

  To sum up, this section gave an overview of my basic assumptions as regards the syntax, morphology, and semantics. More details are provided as they become relevant.

\section{Overview of the chapters}\label{sec:1.5}

This book pursues two goals. First, it seeks to provide an overview of certain synchronic data in the nominal domain in German. In particular, \chapref{sec:2} to \chapref{sec:7} discuss in detail adjectival inflections, \textit{ein} along with its related words, and consequences of the proposed analyses. The second goal is to find commonalities between these different phenomena; that is, to tie these analyses together to offer some remarks about the larger issues involved. While touched upon throughout the book, Chapter 8, the final chapter, focuses on these more general points. On the basis of the similarities and differences of adjectival inflections and \textit{ein}, I engage there in a summary discussion of the main hypotheses stated in \REF{ex:1:49} through \REF{ex:1:51}, I point out further consequences, and I indicate avenues for future research.

It is important to point out that both of these goals go in partially different directions. Specifically, an overview aims to be fairly exhaustive, but discussing the larger issues involved is an attempt to see what different sub-domains, empirical or theoretic, have in common. As such, discussing issues to a comprehensive degree tends to move the focus away from traits shared by all the different domains. However, as already briefly illustrated above, the discussion of one phenomenon often forms the background for the analysis of another. It is on the basis of the shared properties and the interwoven argumentation that I pointed out above that adjectival inflections and \textit{ein} can and should be discussed in tandem.

The book is organized into two primary chapters (\ref{sec:2},\ref{sec:5}), four supporting chapters (\ref{sec:3},\ref{sec:4},\ref{sec:6},\ref{sec:7}), and a conclusion (\ref{sec:8}). Here is a brief overview:

\textsc{Chapter} \ref{sec:2} (\textit{The Structural Nature of Adjectival Inflections}) investigates adjectival inflections in a wide range of nominal constructions. It is observed that weak inflections only occur in canonical constructions. These involve simple DPs of the form “determiner + adjective(s) + noun”, where all these elements involve concord in agreement features. It is proposed that weak inflections are underlyingly fully specified feature bundles that undergo Impoverishment \citep{Sauerland1996}. This feature deletion is triggered by determiners (Impoverishment Rule 1) or by a certain featural context (Impoverishment Rule 2). In contrast, feature bundles surface as strong endings if Impoverishment does not occur, either in canonical or non-canonical structures. As such, the strong inflections present the elsewhere case. The final section discusses three previous proposals in detail pointing out that none of these accounts can account for all the data.

\textsc{Chapter} \ref{sec:3} (\textit{Variation and} \textit{Secondary Mechanisms}) focuses on variation. Given the analysis in \chapref{sec:2}, a number of unexpected inflections are discussed. It is pointed out that these cases are very restricted in that they occur only in specific, well-defined contexts. It is argued that unexpected strong adjectives follow from the assumption that pronominal determiners do not trigger Impoverishment. In contrast, unexpected weak adjectives are accounted for by a phonetic rule, which is a reflex of markedness reduction. Unexpected weak inflections on determiners follow from the extension of Impoverishment Rule 2. Overall, it is proposed that adjectival inflections are due to several mechanisms. Finally, inflectional variation on certain sets of adjectives and predeterminers is addressed, issues with the traditional generalizations Weak After Strong and Principle of Monoinflection are discussed, and the strong/weak alternation of adjectives in the dialect of Mannheim is analyzed.

\textsc{Chapter} \ref{sec:4} (\textit{Consequences for Other Analyses}) discusses some other consequences detailed in the first part of the book. Specifically, it is shown that weak adjectives in the context of plural \textit{ein} raise issues for analyses involving Predicate Inversion (\citealt{BennisEtAl1998}) and for accounts postulating the presence of certain null nouns (\citealt{vanRiemsdijk2005}). Furthermore, it is argued that the present analysis is not compatible with split topicalization constructions if simply analyzed as movement (\citealt{vanRiemsdijk1989}) but only if analyzed as involving the separate base-generation of two nominals, where one or both of these nominals undergo movement later \citep{Fanselow1988}. Finally, it is suggested that non-restrictive adjectives must have the same basic structure as restrictive ones and that strong inflections are not “referential” in nature but serve to make nominal features like case, number, and gender visible.

\textsc{Chapter} \ref{sec:5} (Ein\textit{-words and Adjectival} eine) investigates the different kinds of \textit{ein}. It is argued that there are basically two main types: semantically vacuous \textit{ein} and adjectival \textit{eine}. The former can occur by itself as the indefinite article but also as part of the singularity numeral \textit{EIN} ‘one’, possessive articles like \textit{mein} ‘my’, etc., and the negative article \textit{kein} ‘no’. These four types of elements are labeled \textit{ein}-words. It is proposed that \textit{EIN}, \textit{mein}, and \textit{kein} are composite forms consisting of vacuous \textit{ein} and another component. This derives the morpho-syntactic similarities of these elements. The distinctions follow from the second component, which has different feature specifications from \textit{ein} and occupies different positions in the syntactic representation. The second main type involves adjectival \textit{eine}. This element is a lexically different element: It is an adjective that only appears in definite contexts and induces a duality presupposition. This adjectival element is proposed to be located in a high AgrP.

\textsc{Chapter} \ref{sec:6} (Ein \textit{and Emotiveness}) presents the first consequence of the second major component of this book. Contrasting pronominal DPs and copular TPs, it is shown that singular contexts exhibit the most restrictions. Three different interpretations are distinguished. First, the ordinary reading is emotive in pronominal DPs but both neutral and emotive in copular TPs where TPs show \textit{ein} in both neutral and emotive contexts. Second, the comparative reading is emotive in both DPs and TPs, where the latter involve \textit{ein}. Finally, the capacity reading is neutral in DPs involving \textit{als} ‘as’ and emotive in TPs involving \textit{ein}. It is proposed that \textit{ein} is not responsible for the emotiveness in the various interpretations. Rather, following \citet{Rauh2004} and \citet{deSwartEtAl2007}, I propose that the emotive readings are due to certain pragmatic restrictions and the realization operator REL. The presence of the latter is flagged by \textit{ein} accounting for the presence of \textit{ein} in these contexts. More generally, this and the next chapter discuss the relatedness of noun phrases and clauses of the form “pronoun + (copular verb +) noun".

\textsc{Chapter} \ref{sec:7} (Ein \textit{and Number}) presents a second consequence of \chapref{sec:5}. Comparing again the nominal and clausal domains, I discuss four different constructions: pronominal DPs, copular TPs, pronominals followed by \textit{als}-nominals (\textit{als} ‘as’), and non-copular TPs involving \textit{als}-nominals. I show that pronominal DPs are very restricted with regard to morphological and semantic number; that is, singular DPs are singular in interpretation, and plural DPs are plural in interpretation. This is different for the other three constructions. While nouns preceded by \textit{ein} can only be singular in interpretation and nouns in the plural can only be plural in interpretation, bare nouns can be both singular and plural in interpretation and combine with singular or plural pronominals. I propose that pronominal DPs always project NumP but that predicate nominals in the other constructions may lack NumP, provided NumP is the highest phrase of the predicate nominal. It is suggested that number derives from an interaction between the noun and the Number head and that \textit{ein} flags the presence of an operator.

\textsc{Chapter} \ref{sec:8} (\textit{Concluding Remarks}) ties the previous parts together by discussing in more detail the hypotheses proposed for adjectival inflections and \textit{ein}. Reviewing important empirical and theoretical points, it strengthens the conclusions that both adjectival inflections and \textit{ein} are semantically vacuous elements and that both elements provide clues about the structure of the noun phrase. In addition, this chapter tentatively extends the discussion of adjectival inflections to another dialectal variant, Alemannic German, and it considers \textit{ein} in a number of other semantic contexts. Suggestions are made as to how these extensions can be accommodated in the current system. Finally, this chapter addresses some further consequences as regards concord in agreement features in non-canonical noun phrases, and it considers the question of what all semantically vacuous elements may have in common.
