\chapter {\textit{Ein}  and emotiveness}\label{sec:6}

\section{Introduction}\label{sec:6.1}

In \chapref{sec:5}, I argued that \textit{ein} is semantically vacuous (Hypothesis 1a). I now turn to two consequences. In this and the next chapter, I discuss \textit{ein} with regard to emotiveness and number. These next two chapters are also meant to be broader in their empirical and theoretical range; that is, they discuss \textit{ein} in the context of both the noun phrase (DP) and the clause (TP). Specifically, I provide a more detailed survey of two (partially known) restrictions. Focusing on the contribution of \textit{ein}, I identify three types of readings in this chapter where the relatedness of \textit{ein} to emotiveness is of importance in singular, out-of-the-blue contexts. In the next chapter, I isolate restrictions with regard to both morphological and semantic number such that singular contexts involving \textit{ein} are more restricted than singular contexts without \textit{ein}. Despite initial appearances and consistent with the previous discussion, I reach the conclusion that \textit{ein} does not make a semantic contribution. To contextualize the following discussion, I begin with some general remarks.

\subsection{General structural similarities between the DP and TP}\label{sec:6.1.1}

It was pointed out in \chapref{sec:1} that it is generally assumed that clauses and noun phrases are parallel in meaning and structure (e.g., \citealt{Abney1987}, \citealt{Chomsky1970}, \citealt{Iordăchioaia2020}). To repeat the relevant data, consider the following juxtaposition.

\ea%1
    \label{ex:6:1}
\ea \label{ex:6:1a}  The Romans destroyed the city.
\ex \label{ex:6:1b}  the Romans’ destruction of the city
\z
\z

\newpage
In view of the similar interpretations of the arguments in \REF{ex:6:1}, the structural parallelism in \REF{ex:6:2} has come to be established.\footnote{As mentioned in \chapref{sec:1}, this alignment is not uncontroversial. Recall that I take DP to be parallel to TP. In this and the next chapter, I use Type Theory (\chapref{sec:1}, \sectref{sec:1.4.2.2}) to highlight certain similarities and differences in the DP and TP. Note in this regard that \citet{HeimKratzer1998} take (root) clauses to be of category S. As is clear from their discussion of relative clauses (their Chapter 5), S is under CP. Given the alignment in \REF{ex:6:2}, I take S to be TP here.}

\ea%2
    \label{ex:6:2}
\ea \label{ex:6:2a}  CP {} TP {} AgrP {} ~~VP
\ex \label{ex:6:2b}  PP {} DP {} NumP {} NP
\z
\z

The investigation of this parallel has been very fruitful in its empirical discoveries and theoretical innovations (for a survey, see \citealt{AlexiadouEtAl2007}). This and the next chapter intend to add to this body of work by focusing on certain combinations of “pronoun + (copular verb +) noun” in simple DPs and copular TPs. Unlike the cases in \REF{ex:6:1} above, I investigate noun phrases and copular sentences that involve non-theta nouns; for instance, the noun Idiot ‘idiot’ in \REF{ex:6:3} and \REF{ex:6:4} below.

  Observe first that the DPs and TPs under discussion here all involve a pronoun and a noun \REF[a-b]{ex:6:3}. The difference is that the TPs also contain an auxiliary.

\ea%3
    \label{ex:6:3}
\ea \label{ex:6:3a}
\gll wir Idioten\\
we idiots\\
\glt ‘us idiots’
\ex \label{ex:6:3b}
\gll Wir sind Idioten.\\
we are   idiots\\
\glt ‘We are idiots.’
\z
\z

Unlike in the plural above, \textit{ein} is added in TPs involving the singular.\footnote{Recall that I put the nominal complement of singular pronominal determiners in parentheses in the English translations.}

\ea%4
    \label{ex:6:4}
\ea \label{ex:6:4a}
\gll du   Idiot\\
you idiot.\textsc{masc}\\
\glt ‘you (idiot)’
\ex \label{ex:6:4b}
\gll Du  bist (ei)n Idiot.\\
you are   {\db}an   idiot.\textsc{masc}\\
\glt ‘You are an idiot.’
\z
\z

Considering \REF{ex:6:3} and \REF{ex:6:4}, we can observe that these constructions are fairly similar to each other and involve comparatively little complexity. Specifically, taking the auxiliary to be semantically vacuous (\citealt{CoppockBeaver2015}: 429; \citealt{HeimKratzer1998}: 61-62), I assume that this verbal element simply indicates the presence of more structure, for instance, the TP-layer. As we will see below, \textit{ein} is similar to the auxiliary in this regard. Focusing on these types of constructions affords us a fairly simple and direct comparison of the workings of \textit{ein} in the nominal and clausal domains. Note also that some of the cases discussed below involve non-canonical constructions.

  Overall, I reach the conclusion that \textit{ein} makes no semantic contribution in these constructions either and that the nominal and clausal combinations of “pronoun + (copular verb +) noun” are quite similar albeit not entirely identical. The few differences between the two domains follow from certain assumptions about the different pragmatics involved \citep{Rauh2004}, lexical differences between certain overt and covert elements (\textit{als} ‘as’ vs. ALS), and the obligatoriness of NumP in the DP but its syntactic optionality in the predicate nominal of the TP. If this discussion is on the right track, then I provide additional evidence that the nominal and clausal domains are essentially parallel. With these general remarks in mind, I begin the investigation. Postponing the discussion of number to the next chapter, I start the examination of \textit{ein} in emotive contexts with some general remarks.

\subsection{General interpretation differences between the DP and TP}\label{sec:6.1.2}
\largerpage[-3]

In this chapter, I compare the differences in structure and interpretation of constructions where pronouns combine with bare nouns \REF{ex:6:5} or with nouns preceded by overt elements like \textit{als} ‘as’ or \textit{ein} ‘a’ \REF{ex:6:6}. To avoid certain lexical and structural ambiguities of \textit{ein} (\chapref{sec:5}), the indefinite article is provided in its reduced and unstressable form \textit{’n}. Interestingly, nouns such as \textit{Bauer} are ambiguous in that they have a neutral/literal meaning (‘farmer’) and an emotive/figurative one (‘peasant’) (e.g., \citealt{Harbert2007}: 147, \citealt{deSwartEtAl2007}, and references below). However, this ambiguity disappears in these different types of DP and TP in interesting ways. Specifically, bare nouns have an emotive/figurative meaning in the DP \REF{ex:6:5a} but a neutral/literal meaning in the TP \REF{ex:6:5b}. In contrast, nouns preceded by \textit{als} have a neutral/literal meaning in the DP \REF{ex:6:6a}, but nouns preceded by \textit{ein} have a (predominantly) emotive/figurative meaning in the TP \REF{ex:6:6b} (\# indicates that the interpretation is not available; \% means that the interpretation is possible but somewhat less prevalent or for some speakers not available at all).\footnote{A brief note on the data in \REF{ex:6:6} is in order here. The string in \REF{ex:6:6a} can felicitously be used in a sentence like \textit{Was würdest du als Bauer dazu sagen?} ‘What would you as a farmer say about this?’. As to \REF{ex:6:6b}, \citet[147-48]{Harbert2007} states that the indefinite article is not possible with role nouns in predicative contexts in German, a statement also often found in textbooks of German. As we see here and below, this statement is too strong. Indeed, the use of \textit{ein} with terms of professions is already attested in ENHG (\citealt{EbertEtAl1993}: 316). Cross-linguistically, predicate nouns in English typically have an indefinite article in cases like \REF{ex:6:5b} yielding \REF{ex:6:6b}, but their Romance counterparts do not. \citegen{Zamparelli2008} explanation that this is due to lack of morphological gender in English is probably not the whole story as Yiddish, which does have morphological gender, patterns with English (see \citealt{Lockwood1995}: 112, also \citealt{Harbert2007}: 148). For the discussion of morphological case on predicate nominals, see \citet{MalingSprouse1995}.}

\ea%5
    \label{ex:6:5}
\ea \label{ex:6:5a}
\gll du   Bauer\\
you peasant.\textsc{masc}\\
\glt ‘you (peasant)’

    \#‘you (farmer)’
\ex \label{ex:6:5b}
\gll Du  bist Bauer.\\
you are  farmer.\textsc{masc}\\
\glt ‘You are a farmer.’

    \#‘You are a peasant.’
\z
\z

\ea%6
    \label{ex:6:6}
\ea  \label{ex:6:6a}
\gll du   als Bauer\\
you as  farmer.\textsc{masc}\\
\glt ‘you as a farmer’

    \#‘you as a peasant’
\ex  \label{ex:6:6b}
\gll Du  bist ’n Bauer.\\
you are   {\db}a peasant/farmer.\textsc{masc}\\
\glt ‘You are a peasant.’

    \%‘You are a farmer.’
\z
\z

Comparing the (a)-examples to the (b)-examples, we find a near complementary distribution of the two readings of the noun. Indeed, comparing the (a)-examples to each other and the (b)-examples to each other, this difference in interpretation correlates with the presence or absence of \textit{als} ‘as’ and \textit{ein} ‘a’. Focusing on the latter, this means that \textit{ein} seems to have, at least in some contexts, semantic import. Given the proposal that \textit{ein} is semantically vacuous, this is unexpected and in need of an explanation.

  To anticipate the discussion, I follow \citet{Rauh2004} for \REF{ex:6:5a} and \REF{ex:6:6a} in arguing that the pragmatics involved here need to be taken into consideration to explain this difference. As to \REF{ex:6:5b} and \REF{ex:6:6b}, it could be claimed that \textit{ein} brings about the semantic difference. However, I suggest that the added emotiveness in \REF{ex:6:6b} stems from the interaction between an operator, specifically the realization operator REL (\citealt{deSwartEtAl2007}), and nouns lexically specified as [+figurative]. I propose that on the one hand, \textit{ein} indicates the presence of syntactic structure above NP (Hypothesis 3a) and that on the other hand, it flags the presence of the realization operator REL (Hypothesis 3b). As such, similar to adjectival inflections, I maintain that \textit{ein} is a reflex of the structure but not of the semantics.

  The chapter is organized as follows. \sectref{sec:6.2} provides the data discussing three different readings in the DP and TP, and the role emotiveness plays in them. In \sectref{sec:6.3}, I first lay out my main assumptions, and then I provide the proposal and detailed derivations of these different readings. \sectref{sec:6.4} summarizes the main findings of this chapter.

\section{Data}\label{sec:6.2}

This section investigates the parallels between the nominal and clausal domains with regard to certain readings. Of particular interest here are interpretations that seem to relate the presence of \textit{ein} to emotiveness. In order to understand and classify the data better, I start with some preliminary remarks, where I dicuss the different types of nouns involved in pronominal DPs and copular TPs and how their various meanings relate to different readings. In the second section, I turn to the data in more detail, and I provide a summary of the different readings in the nominal and clausal domains.

\subsection{Preliminary remarks}\label{sec:6.2.1}

In this section, I illustrate some of the properties of the different types of nouns. Specifically, I discuss how role and kind nouns fare in pronominal DPs and copular TPs. In the second subsection, I expand on concepts like emotiveness and figurative extension. In the final subsection, I establish three readings that help classify and analyze the data.

\subsubsection{Basic differences between role and kind nouns in the DP and TP}\label{sec:6.2.1.1}

Nouns can be classified along different dimensions: role vs. kind, neutral vs. emotive, and related to neutral/emotive, literal vs. figurative (for the latter, see next subsection). It is important to point out that the restrictions to be discussed reveal themselves with certain types of nouns only.

  Starting with role nouns,   \citet[453]{deSwartEtAl2005} provide a convenient summary of the properties of bare predicate nominals such as \textit{Landwirt} ‘farmer’:\footnote{For relevant discussion, see also \citet{Kupferman1991}, \citet{Stowell1989,Stowell1991}, and more recently \citet[814-17]{Alexiadou2005}, \citet{Déprez2005}, \citet{MatushanskySpector2005},  \citet[843-49]{MunnSchmitt2005}, \citet[779-83]{Winter2005}, and \citet{Zamparelli2008}. The latter paper also contains detailed critical discussion of earlier work on this topic.} (i) They form a restricted class of nouns (typically, names for professions, nationalities, and religions), (ii) they only exhibit a restricted range of interpretations (what the authors call capacities), (iii) they allow capacity qualifiers such as \textit{by profession}, (iv) they exhibit number neutrality, and (v) they combine with certain adjectives, which remain uninflected (at least, in Dutch). In the discussion to follow, I revisit (i), (ii), and (iv). Furthermore, related to (ii), I add a sixth characteristic to this: In their literal meaning, bare role nouns are never emotive in pragmatically neutral contexts.

Kind nouns comprise a more diverse group; for instance, they involve names for humans, animals, and inanimates. For current purposes, animate nouns can be classified as regards neutral (\textit{Mann} ‘man’) and emotive (\textit{Idiot} ‘idiot’). With this in place, I turn to the combinations of “pronoun + (copular verb +) noun” to identify some initial properties and restrictions. I focus for now on pragmatically neutral, out-of-the-blue contexts (but see \sectref{sec:6.3.1.3}).

  It is well known that pronominal DPs in the singular only allow emotive kind nouns like \textit{Idiot} ‘idiot’; (neutral) role nouns like \textit{Landwirt} ‘farmer’ and neutral kind nouns like \textit{Mann} ‘man’ are marked \REF{ex:6:7}.

\judgewidth{??}
\ea%7
    \label{ex:6:7}
\ea[]{ \label{ex:6:7a}
\gll du  Idiot\\
you idiot.\textsc{masc}\\
\glt ‘you (idiot)’
}
\ex[??]{ \label{ex:6:7b}
\gll du Landwirt\\
you farmer.\textsc{masc}\\
\glt ‘you (farmer)’
}
\ex[??]{ \label{ex:6:7c}
\gll du Mann\\
you man.\textsc{masc}\\
\glt ‘you (man)’
}
\z
\z
\judgewidth{\%}

In contrast, copular constructions can involve bare predicate nouns only if the relevant nominals consist of role nouns like \textit{Landwirt}. Note again though that the indefinite article is optional here for many speakers \REF{ex:6:8a}. Unlike role nouns, kind nouns like emotive \textit{Idiot} and non-emotive \textit{Mann} require an indefinite article \REF[b-c]{ex:6:8}.

\ea%8
    \label{ex:6:8}
\ea\label{ex:6:8a}
\gll Du  bist (’n) Landwirt.\\
you are    {\db\db}a   farmer.\textsc{masc}\\
\glt ‘You are a farmer.’
\ex\label{ex:6:8b}
\gll Du  bist *(’n) Idiot.\\
you are     {\db\db\db}an idiot.\textsc{masc}\\
\glt ‘You are an idiot.’
\ex\label{ex:6:8c}
\gll Du  bist *(’n) Mann.\\
you are     {\db\db\db}a   man.\textsc{masc}\\
\glt ‘You are a man.’
\z
\z

To be clear, abstracting away from the determiners in \REF{ex:6:7a} and \REF{ex:6:8a}, bare nouns involve emotive kind nouns in the DP but (neutral) role nouns in the TP. Furthermore, role nouns in the DP \REF{ex:6:7b} and emotive kind nouns in the TP \REF{ex:6:8b} pattern like neutral kind nouns such as \textit{Mann} in the (c)-examples.

  Examining these cases, it becomes clear that true minimal pairs between singular DPs and related TPs are not possible. In other words, in pragmatically neutral, out-of-the-blue contexts, there do not seem to be lexical items that share properties of being both emotive nouns and role nouns; that is, there are no nouns that allow licit patterns such as \textit{du Noun\textsubscript{1}} ‘you Noun’ as well as \textit{Du bist Noun\textsubscript{1}} ‘You are a Noun’. The generalization to be explained, then, is that emotive nouns cannot involve role nouns and that role nouns (in their literal meaning) are not emotive. Expressed in more structural terms, emotive nouns can occur in DPs but require \textit{ein} in TPs, and role nouns cannot occur in DPs but can be bare in TPs. As shown in more detail below, these restrictions only hold in singular, pragmatically neutral contexts.

  Before moving on, note that adding an evaluative adjective creates minimal pairs. However, this covers up the differences in \REF{ex:6:7} and \REF{ex:6:8} above, lexico-semanti-\linebreak cally (all nouns are equally fine now) and syntactically (all copulative cases must have \textit{ein} now).

\ea%9
    \label{ex:6:9}
\ea  \label{ex:6:9a}
\gll du   blöder Idiot\\
you stupid   idiot.\textsc{masc}\\
\glt ‘you (stupid idiot)’
\ex  \label{ex:6:9b}
\gll du   blöder Landwirt\\
you stupid  farmer.\textsc{masc}\\
\glt ‘you (stupid farmer)’
\ex  \label{ex:6:9c}
\gll du   blöder Mann\\
you stupid  man.\textsc{masc}\\
\glt ‘you (stupid man)’
\z
\z

\ea%10
    \label{ex:6:10}
\ea   \label{ex:6:10a}
\gll Du  bist *(’n) blöder Landwirt.\\
you are       {\db\db\db}a   stupid farmer.\textsc{masc}\\
\glt ‘You are a stupid farmer.’
\ex   \label{ex:6:10b}
\gll Du  bist *(’n) blöder Idiot.\\
you are       {\db\db\db}a   stupid idiot.\textsc{masc}\\
\glt ‘You are a stupid idiot.’
\ex   \label{ex:6:10c}
\gll Du  bist *(’n) blöder Mann.\\
you are       {\db\db\db}a   stupid man.\textsc{masc}\\
\glt ‘You are a stupid man.’
\z
\z

As can be seen in \REF{ex:6:10a}, modified role nouns must also take \textit{ein} in predicative contexts. This is explained in \sectref{sec:6.3.4.1}. Thus, in order to probe into the restrictions, I use unmodified nouns.\footnote{I concentrate on countable nouns (for the discussion of the concept of countability, see \citealt{Allan1980}, also \chapref{sec:7} and \ref{sec:8}), and I basically leave aside group nouns and \textit{pluralia tantum} nouns. Also, I abstract away from special cases such as “royal” \textit{ihr} ‘you’ and “nursely” \textit{wir} ‘we’.} In the introduction, I made a distinction of nouns as regards their meanings: neutral/literal vs. emotive/figurative. It is important to be more precise about this. We will see that emotiveness and figurativeness are not the same.

\subsubsection{Emotiveness and figurative extension}\label{sec:6.2.1.2}

Role and kind nouns can involve two subtypes as regards their meaning. Focusing on the main cases discussed below, role nouns like \textit{Landwirt} have a neutral-only meaning (‘farmer’); others like \textit{Bauer} can take on an additional figurative meaning: Besides ‘farmer’, \textit{Bauer} can also mean ‘peasant’ in certain contexts (\sectref{sec:6.2.2}).\footnote{\label{foot:6:6}Besides figurative, there are other terms found in the literature: metaphorical, extended, approximative, subjective, descriptive, gradable, and [+scalar]. Also, note that this difference in interpretative possibilities also holds true of inanimate kind nouns and proper names; for instance, while \textit{Obelisk} ‘obelisk’ and \textit{Joachim} ‘Joachim’ pattern with \textit{Landwirt} ‘farmer’, \textit{Säule} ‘column/pillar’ and \textit{Willi} ‘Willy/idiot’ align with \textit{Bauer} ‘farmer/peasant’.} Briefly, the figurative meaning of a role noun like \textit{Bauer} denotes certain stereotypical properties of a farmer (but there is no implication that this person is a farmer; more on this below). This figurative meaning is emotive. This means that while role nouns are not emotive in their literal meaning, they are in their figurative meaning. Kind nouns like \textit{Schwein} have a neutral-only meaning (‘pig’), but if applied to contexts involving humans, they can also take on a figurative meaning (‘swine/pig-like person’); other kind nouns like \textit{Idiot} have an emotive-only meaning (‘idiot’).

  As regards emotiveness then, some nouns become emotive when “figuratively extended” in certain pragmatic contexts (\textit{Schwein}, \textit{Bauer}), but others are inherently emotive (\textit{Idiot}). This means that not all emotive readings are due to figurative extension. That said, it is important to elaborate on the figurative extension of \textit{Schwein} and \textit{Bauer} some more. In their neutral/literal meaning, these nouns denote properties like ‘pig’ or ‘farmer’; in their emotive/figurative meaning, these nouns denote certain features of a stereotypical representative of the relevant kind. In other words, while the individuals under discussion are ascribed certain features of a pig or farmer, there is no implication here that the individuals are actually pigs or farmers. Below, this type of reading is labeled comparative.

   In sum, while certain role nouns only have a neutral/literal meaning (\textit{Landwirt}), other such nouns may have an additional emotive/figurative one (\textit{Bauer}). Emotive kind nouns have an emotive meaning only (\textit{Idiot}), but neutral kind nouns may have an additional emotive/figurative meaning (\textit{Schwein}). Impor-\linebreak tantly, while a figurative meaning is emotive (\textit{Bauer}/\textit{Schwein}), an unambiguous emotive noun does not have a figurative meaning (\textit{Idiot}). In the summary \tabref{tab:6:1}, I distinguish the different types of nouns by the terms unambiguous and ambiguous (NB: Since the neutral meaning of \textit{Schwein} ‘pig’ cannot apply to a human, I put it in parentheses; the negative value in parentheses indicates that this is not a possible reading in pragmatically neutral contexts).\footnote{As we will see below, role nouns in their literal meaning (e.g., \textit{Landwirt/Bauer} ‘farmer’) can take on an emotive component (‘very good/bad farmer’). The same goes for kind nouns (but this is not relevant for current purposes as humans cannot be pigs even in emotive contexts). Furthermore, emotive nouns like \textit{Idiot} ‘idiot’ cannot have a neutral reading and, as far as I can tell, they cannot have a figurative reading. As to the latter, note that \textit{Du bist ’n Idiot!} does not seem to mean ‘You are an idiot-like person!’, which states that the person addressed only has certain features of an idiot (whatever those may be).}

\begin{table}
\caption{Different types of count nouns}
\label{tab:6:1}
\small
% \begin{tabularx}{\textwidth}{llcccc}
% \lsptoprule
% \multicolumn{2}{l}{Type of noun} & \multicolumn{2}{c}{Role} & \multicolumn{2}{c}{Kind}\\
% \midrule
% &  & {\itshape Landwirt} & {\itshape Bauer} & {\itshape Idiot}  & {\itshape Schwein}
% \\
% \multicolumn{2}{l}{(Non-)ambiguity} & Unamb. & Ambiguous & Unamb. & Ambiguous\\
% of the noun & & & & \\
% Meaning & Neutral/literal & ‘farmer’ & ‘farmer’ & - & (‘pig’)\\
% & Emotive & (-) & (-) & ‘idiot’ & (-)\\
% & {Figur. (emot.)} & - & { ‘peasant’} & - & { ‘swine’}\\
% & [Comparative] & & [“farmer-like person”] &  & [“pig-like person”]\\
% \lspbottomrule
% \end{tabularx}
\begin{tabular}{llccc}
\lsptoprule
Type of noun & (Non-)ambiguity & \multicolumn{3}{c}{Meaning} \\
\cmidrule{3-5}
& & Neutral /  & Emotive & Figurative (emotive) /\\
 & & literal & & [Comparative]\\
\midrule
Role & \textit{Landwirt} (unamb.)  & ‘farmer’ & (-) & -\\
& \textit{Bauer} (amb.) & ‘farmer’ & (-)  & ‘peasant’ /\\
& & & & [“farmer-like person”] \\
Kind & \textit{Idiot}    (unamb.) & - & ‘idiot’ & -\\
 & \textit{Schwein} (amb.) & (‘pig’) & (-) & ‘swine’ /\\
  & & & & [“pig-like person”] \\
\lspbottomrule
\end{tabular}
\end{table}

Note that the word pairs \textit{Landwirt/Bauer} and \textit{Idiot/Schwein} in \tabref{tab:6:1} are intriguing in that one element is unambiguous and the other is ambiguous (again, with the qualification that the literal meaning of \textit{Schwein} ‘pig’ will not be relevant below). To account for this difference, some amount of lexical specification must be assumed (if so, we might also expect some dialectal and cross-linguistic variation in this regard, which I think is true; I return to the issue of lexical specification in \sectref{sec:6.3.1.2}). In combination with pronominal determiners, these different types of nouns yield three different readings, as explicated in the next subsection. These readings help organize the data in \sectref{sec:6.2.2} and the subsequent analysis in \sectref{sec:6.3}.

\subsubsection{Three different readings}\label{sec:6.2.1.3}

Basically following \citegen[Chapter 5]{denDikken2006} discussion of binominal structures of the type \textit{an idiot of a doctor}, I point out that there are three relevant interpretations in the cases under discussion here. Focusing for now on pronominal DPs \REF{ex:6:11}, I label the first case as ordinary reading. The second and the third case are called comparative and capacity readings.\footnote{Den Dikken labels the capacity reading as attributive, a term I would like to avoid.}

\ea%11
    \label{ex:6:11}
\ea     \label{ex:6:11a} Ordinary reading:

\gll  du   Idiot\\
you idiot.\textsc{masc}\\
\glt ‘you (idiot)’
\ex    \label{ex:6:11b} Comparative reading:

\gll  du   Schwein\\
you pig.\textsc{neut}\\
\glt ‘you (idiot)’

\newpage
\ex    \label{ex:6:11c} Capacity reading:

\gll du   als Landwirt\\
you as  farmer.\textsc{masc}\\
\glt ‘you as a farmer’
\z
\z

I paraphrase each of these three readings as follows in \REF{ex:6:12}.


\ea%12
    \label{ex:6:12}

           the unique x (informally addressed) such that
\ea    \label{ex:6:12a}  x is an idiot
\ex     \label{ex:6:12b} x is, in a certain way, similar to a pig
\ex     \label{ex:6:12c} x is being singled out in the capacity of a farmer
\z
\z


As we see in the data section below, similar readings can also be identified in the TP. Before I turn to the data, some further remarks about the readings are in order here.

I point out that the ordinary reading in the DP is most often negatively emotive \citep{Vater2000}, but a positive connotation is also possible. In this regard, \citet{Rauh2004} provides the following example: \textit{du Glückspilz} ‘(you luck.mushroom =) lucky you’. Assuming that the pronoun and noun stand in a predication relation with one another (see \sectref{sec:6.3.2}), I suggest that this type involves direct predication.

The comparative reading is also emotive but involves nouns with emotive\slash{}figurative meanings. Unlike the first case, it involves indirect predication; that is, the predication seems to be mediated by a predicate such as \textit{like} or \textit{similar to}.\footnote{This is what I assume for now. Below, I suggest that the comparative reading follows from the realization operator REL.} Note again that epithetically used nouns like \textit{Schwein} ‘(pig =) idiot’ in \REF{ex:6:11b} characterize the individual in their entirety, but there is no implication that the individual being addressed is actually a pig, as reflected in the translation.\footnote{Recall that I translate the figurative meaning of animal names simply as ‘idiot’ (although such names may imply different attributes).}

Finally, in contrast to the first two cases, the capacity reading is neutral and features nouns with neutral/literal meanings. Like the second reading, it also involves indirect predication, which in this case is mediated by the element \textit{als} ‘as; in the capacity of’. As noted by \citet[85-86, 94]{Rauh2004}, these \textit{als-}nominals clearly presuppose the presence of other properties, but only the overt one following \textit{als} is taken to be relevant. In contrast to the comparative reading, which characterizes individuals in their entirety, the capacity reading zooms in on a certain aspect of the individual.

With this in place, I turn to the data in more detail to see how role nouns and emotive or figurative kind nouns pattern as regards the three readings in various morpho-syntactic contexts. I compare the relevant readings in the DP to those of the TP.

\subsection{Data}\label{sec:6.2.2}

I start with the data involving nouns in the plural. We will see that with the exception of the capacity reading, there are no restrictions in these cases. This is followed by unambiguous nouns in the singular, which reveal more restrictions. After providing an interim summary, I discuss ambiguous nouns in the singular, which exhibit a complementary loss of their potential meanings.

\subsubsection{Plural}\label{sec:6.2.2.1}

I begin with the two unambiguous nouns from \tabref{tab:6:1}. As can be seen in \REF{ex:6:13} and \REF{ex:6:14}, role as well as emotive kind nouns are equally possible in ordinary readings.

\ea%13
    \label{ex:6:13}
\ea\label{ex:6:13a}
\gll ihr   Landwirte\\
you farmers\\
\glt ‘you farmers’
\ex\label{ex:6:13b}
\gll Ihr  seid Landwirte.\\
you are  farmers\\
\glt ‘You are farmers.’
\z
\z

\ea%14
    \label{ex:6:14}
\ea \label{ex:6:14a}
\gll ihr Idioten\\
you idiots\\
\glt ‘you idiots’
\ex \label{ex:6:14b}
\gll Ihr  seid Idioten.\\
you are  idiots\\
\glt ‘You are idiots.’
\z
\z

  Continuing with the ambiguous kind noun in \tabref{tab:6:1}, such nouns are figuratively extended in the comparative reading.

\ea%15
    \label{ex:6:15}
\ea \label{ex:6:15a}
\gll ihr  Schweine\\
you pigs\\
\glt ‘you idiots’
\ex \label{ex:6:15b}
\gll Ihr  seid Schweine.\\
you are  pigs\\
\glt ‘You are idiots.’
\z
\z

  As for the capacity reading, I note that only certain predicate nouns are possible. In particular, role nouns (and certain kind nouns, see below) are fine \REF{ex:6:16a}. However, as pointed out in \citet[93]{Lawrenz1993} and \citet[86]{Rauh2004}, emotive and figuratively extended kind nouns are marked \REF[b-c]{ex:6:16}. It seems to me that the latter also holds true of kind nouns that have a more general, hypernym-like denotation \REF{ex:6:16d}.

\ea%16
    \label{ex:6:16}
\ea   \label{ex:6:16a}
\gll ihr  als Landwirte\\
you as  farmers\\
\glt ‘you as farmers’
\ex[??]{   \label{ex:6:16b}
\gll ihr  als Idioten\\
you as    idiots\\
}
\ex[??]{   \label{ex:6:16c}
\gll ihr  als Schweine\\
you as  pigs\\
}
\ex[??]{   \label{ex:6:16d}
\gll ihr  als Personen\\
you as  persons\\
}
\z
\z

The restriction shown in \REF[b-d]{ex:6:16} is probably due to the fact that \textit{als} ‘as’ singles out a person with regard to a specific capacity or skill. Importantly, this quality must be restrictive enough such that only some people typically have it in a given context. In other words, while it is clear that not each and every person is a farmer \REF{ex:6:16a}, every person can presumably be in someone else’s bad books \REF[b-c]{ex:6:16}. As to \REF{ex:6:16d}, every person is a human making \textit{Personen} ‘persons’ too general in meaning to be restrictive in the relevant way. To be clear, only \textit{Landwirte} ‘farmers’ in \REF{ex:6:16a} singles out the addressed individuals in the relevant way but the other nouns in \REF{ex:6:16} do not. I turn to the capacity reading in the clause after I discuss ambiguous role nouns.

  Ambiguous role nouns such as \textit{Bauer} ‘farmer/peasant’ (\tabref{tab:6:1}) are possible in their neutral/literal and emotive/figurative meanings in simple DPs and copular cases \REF[a-b]{ex:6:17}, where both an ordinary and a comparative reading are available. In contrast, the restriction discussed for \REF{ex:6:16} emerges again in \textit{als-}nominals \REF{ex:6:17c}. In other words, the capacity reading in \REF{ex:6:17c} is the sum of \textit{als} ‘as’ and the neutral/literal reading of the noun.

\ea%17
    \label{ex:6:17}
\ea \label{ex:6:17a}
\gll ihr  Bauern\\
you farmers/peasants\\
\glt ‘you farmers/peasants’
\ex \label{ex:6:17b}
\gll Ihr  seid Bauern.\\
you are  farmers/peasants\\
\glt ‘You are farmers/peasants.’
\ex \label{ex:6:17c}
\gll ihr  als Bauern\\
you as  farmers\\
\glt ‘you as farmers’

    \#‘you as peasants’
\z
\z

  Unlike the DP cases as in \REF{ex:6:17c}, clauses do not involve \textit{als} ‘as’ in the capacity reading \REF{ex:6:18a}. However, this reading can, as an approximation, be brought out when the modal particle \textit{vielleicht} ‘really’ is added \REF{ex:6:18b}. Note that this example also gets an intensified comparative reading with \textit{Bauern} meaning ‘peasants’. What is interesting to note is that even an unambiguous role noun like \textit{Landwirt} ‘farmer’ can take on an emotive component in such a context, with a negative connotation being somewhat more prevalent \REF{ex:6:18c}.\footnote{Again, translations are not straightforward. Note that the element \textit{some} in the English translations is intended to convey the emotive flavor. Also, these constructions can take – what \citet[36]{Delsing1993} calls – an implicit argument. To illustrate with German, the dative pronoun \textit{mir} ‘me’ can be added to \REF{ex:6:18b}.

  	\ea \label{ex:6:11:i}
  	\gll Ihr  seid mir vielleicht ’n paar     Bauern!\\
  	you are  me  \textsc{prt}  {\db}a couple farmers/peasants\\
  	\glt ‘(To me,) you are some farmers/peasants!’
  	\z
  	
  	Note that this oft-called ethical dative seems to bring out more clearly the capacity reading of \textit{Bauern} meaning ‘farmers’ such that the presence of other properties is presupposed, but only the overt predicate is taken to be relevant. I stick to the simpler cases in the main text.}

\ea%18
    \label{ex:6:18}
\ea[*]{ \label{ex:6:18a}
\gll Ihr  seid als Bauern.\\
you are  as  farmers/peasants\\
}
\ex \label{ex:6:18b}
\gll Ihr  seid vielleicht ’n paar     Bauern!\\
you are  \textsc{prt}            {\db}a couple farmers/peasants\\
\glt ‘You are some farmers/peasants!’
\ex \label{ex:6:18c}
\gll Ihr  seid vielleicht ’n paar    Landwirte!\\
you are  \textsc{prt}            {\db}a couple farmers\\
\glt ‘You are some farmers!’
\z
\z

Observe that the emotive reading in \REF{ex:6:18c} is not due to the properties of the noun itself or figurative extension. Rather, the modal particle \textit{vielleicht} plays a role here. Specifically, this particle invokes a [+scalar] capacity reading; that is, the persons being addressed are described as being good or bad in the capacity of farmers.

\subsubsection{Singular}\label{sec:6.2.2.2}

The cases in the singular are more restricted. While the same readings as above are in principle possible, some well-known restrictions with regard to the predicate noun emerge. As with the plural cases above, I start with unambiguous nouns in the ordinary reading.

\hspace*{-2pt} Recalling that I focus for now on pragmatically neutral contexts, I already noted in the introduction that role nouns are not possible in the DP \REF{ex:6:19a} but that they pattern more freely in the TP. As also mentioned above, many speakers allow these nouns not only to occur without \textit{ein} but also with the indefinite article \REF{ex:6:19b}. Unlike role nouns, emotive kind nouns are possible in the DP \REF{ex:6:20a} but must co-occur with an article in the TP \REF{ex:6:20b}.

\ea%19
    \label{ex:6:19}
\ea[??]{ \label{ex:6:19a}
\gll du   Landwirt\\
you farmer.\textsc{masc}\\
}
\ex   \label{ex:6:19b}
\gll Du  bist (’n) Landwirt.\\
you are     {\db\db}a  farmer.\textsc{masc}\\
\glt ‘You are a farmer.’
\z
\z

\ea%20
    \label{ex:6:20}
\ea \label{ex:6:20a}
\gll du   Idiot\\
you idiot.\textsc{masc}\\
\glt ‘you (idiot)’
\ex \label{ex:6:20b}
\gll Du  bist *(’n)  Idiot.\\
you are       {\db\db\db}an idiot.\textsc{masc}\\
\glt ‘You are an idiot.’
\z
\z

\hspace*{-2pt} In the comparative reading in \REF{ex:6:21}, the distribution is identical to the cases involving emotive kind nouns with ordinary readings (as just seen).

\ea%21
    \label{ex:6:21}
\ea \label{ex:6:21a}
\gll du   Schwein\\
you pig.\textsc{neut}\\
\glt ‘you (idiot)’
\ex \label{ex:6:21b}
\gll Du  bist *(’n) Schwein\\
you are       {\db\db\db}a  pig.\textsc{neut}\\
\glt ‘You are an idiot.’
\z
\z

\hspace*{-2pt} In other words, kind nouns, be they inherently emotive or figuratively extended, pattern alike. Finally, I turn to the capacity reading.
\hspace*{-2pt} Similar to the cases in the plural, only certain nouns are possible in \textit{als-}nominals.

\ea%22
    \label{ex:6:22}
\ea \label{ex:6:22a}
\gll du   als Landwirt\\
you as  farmer.\textsc{masc}\\
\glt ‘you as a farmer’
\ex[??]{ \label{ex:6:22b}
\gll du   als Idiot\\
you as  idiot.\textsc{masc}\\
}
\ex[??]{ \label{ex:6:22c}
\gll du   als Schwein\\
you as   pig.\textsc{neut}\\
}
\ex[??]{ \label{ex:6:22d}
\gll du   als Person\\
you as  person.\textsc{fem}\\
}
\z
\z

I assume that the restriction involved in \REF[b-d]{ex:6:22} is the same as that seen in the plural above.

  In the clausal counterpart, the indefinite article is obligatory with role nouns, and these cases are emotive \REF{ex:6:23a}. Notice also that when emotive or figuratively extended kind nouns combine with the modal particle, the ordinary and comparative readings get intensified \REF[b-c]{ex:6:23}.

\ea%23
    \label{ex:6:23}
\ea\label{ex:6:23a}
\gll Du  bist vielleicht *(’n) Landwirt!\\
you are  \textsc{prt}              {\db\db\db}a   farmer.\textsc{masc}\\
\glt ‘You are some farmer!’
\ex \label{ex:6:23b}
\gll Du  bist vielleicht ’n   Idiot!\\
you are  \textsc{prt}           {\db}an idiot.\textsc{masc}\\
\glt ‘You are really an idiot!’
\ex\label{ex:6:23c}
\gll Du  bist vielleicht ’n Schwein!\\
you are  \textsc{prt}           {\db}a pig.\textsc{neut}\\
\glt ‘You are really an idiot!’
\z
\z

Finally, for the sake of completeness, I discuss neutral kind nouns that involve words for humans.

  As already seen in \sectref{sec:6.2.1.1}, kind nouns such as \textit{Mann} ‘man’ are awkward in the DP \REF{ex:6:24a}. Importantly, although non-emotive, they nonetheless require \textit{ein} in the TP \REF{ex:6:24b}.

\ea%24
    \label{ex:6:24}
\ea[??]{\label{ex:6:24a}
\gll du   Mann\\
  you man.\textsc{masc}\\
  }
\ex\label{ex:6:24b}
\gll Du  bist *(’n) Mann.\\
  you are       {\db\db\db}a   man.\textsc{masc}\\
\glt ‘You are a man.’
\z
\z


These types of nouns do not occur with a comparative reading, but they do have a capacity reading.

\ea%25
    \label{ex:6:25}
\ea \label{ex:6:25a}
\gll du   als Mann\\
you as  man.\textsc{masc}\\
\glt ‘you as a man’
\ex \label{ex:6:25b}
\gll Du  bist vielleicht ’n Mann!\\
you are  \textsc{prt}           {\db}a  man.\textsc{masc}\\
\glt ‘You are some man!’
\z
\z

Before I discuss ambiguous role nouns like \textit{Bauer} ‘farmer/peasant’ in the singular, I briefly summarize the discussion thus far.

\subsubsection{Interim summary}\label{sec:6.2.2.3}

Starting with some remarks on emotiveness, the previous two subsections have shown that unambiguous role nouns can take on an emotive component in the clausal capacity reading. The latter involves the modal particle \textit{vielleicht} ‘really’. In other words, emotiveness can be an inherent property (\textit{Idiot}), it can be due to figurative extension of a lexical meaning (\textit{Schwein} ‘pig’ > ‘swine’, \textit{Bauer} ‘farmer’ > ‘peasant’), and it can be brought about in certain contexts where in conjunction with a modal particle, a neutral-only role noun may describe an individual performing their profession in an amazing or appalling way (\textit{Landwirt} ‘farmer’ > ‘some farmer’). The latter becomes particularly clear when evaluative adjectives like \textit{toller} ‘great’ or \textit{miserabler} ‘poor’ are added. In other words, the context where a neutral noun appears may change the reading of the noun. With this in place, I summarize the nominal and clausal cases according to morphological number and the three readings.

  As can be seen in \tabref{tab:6:2}, DPs and TPs in the plural have basically the same readings – only the capacity reading differs in emotiveness.

\begin{table}
\caption{Summary of the readings in the plural}
\label{tab:6:2}
\begin{tabular}{llll}
\lsptoprule
 & Ordinary reading & Comparative reading & Capacity reading\\
\midrule
DP & Emotive/Neutral & Emotive & Neutral [\textit{als}]\\
TP & Emotive/Neutral & Emotive & Emotive [\textit{vielleicht}]\\
\lspbottomrule
\end{tabular}
\end{table}

The singular cases are more restricted, especially in the ordinary reading. In more detail, the ordinary reading in the DP only allows (inherent) emotive kind nouns, but the two other readings in the DP are as in the plural. As for the TP, role nouns take an optional article in the ordinary reading, but kind nouns must have \textit{ein}. In the two other readings, nouns are emotive and require a determiner. Considering all three readings, \textit{ein} appears in both neutral and emotive contexts. This is summarized in \tabref{tab:6:3} (the subscript K stands for kind noun and the subscript R for role noun).

\begin{table}
\caption{Summary of the readings in the singular}
\label{tab:6:3}

\begin{tabularx}{\textwidth}{lQQQ}
\lsptoprule
 & Ordinary reading & \mbox{Comparative reading} & Capacity reading\\
\midrule
DP & Emotive \newline [\textit{du Idiot}\textsubscript{K}] & Emotive \newline [\textit{du Schwein}\textsubscript{K}] & Neutral \newline [\textit{du als Landwirt}\textsubscript{R} \newline \textit{/ Mann}\textsubscript{K}]\\
TP & Neutral \newline [(\textit{ein}) \textit{Landwirt}\textsubscript{R}] \newline
Emotive \newline [\textit{ein Idiot}\textsubscript{K}] \newline
Neutral \newline [\textit{ein Mann}\textsubscript{K}] & Emotive \newline [\textit{ein Schwein}\textsubscript{K}] & Emotive \newline [\textit{vielleicht ein \newline Landwirt}\textsubscript{R} \newline \textit{/ Mann}\textsubscript{K}]\\
\lspbottomrule
\end{tabularx}
\end{table}

Note that all comparative readings involve figurative extension and consequently are emotive. I make some more remarks.

Starting with the singular (\tabref{tab:6:3}), I can point out again that the ordinary reading and the capacity reading are complementary as regards emotiveness in the DP. The absence vs. presence of \textit{als} ‘as’ seems to play a role here. Furthermore, emotiveness has less structure in the nominal domain (\textit{als} is not present) but more structure in the clausal domain (\textit{ein} must be present). In fact, the bidirectional entailment “non-emotive $\leftrightarrow $ \textit{als}” holds in the DP, and assuming that pronouns are determiners (\chapref{sec:3}, \sectref{sec:3.5}), the unidirectional entailment “emotive $\rightarrow$ determiner” holds in both domains. Finally, as mentioned above, the unidirectional entailment “no determiner $\rightarrow$ role noun” holds in the predicate nominal of the TP.

  Interestingly, these entailments allow the combination of a role noun and \textit{ein} in the TP. This state-of-affairs is neatly summarized for the clause by  \citet[451]{deSwartEtAl2005}, who point out that the reading of a bare (role) noun is basically a subset of the readings of the corresponding “\textit{ein} + noun” combination (also \citealt{Zamparelli2008}: 114). In other words, while a bare (role) noun can only be neutral/literal in meaning, the same noun preceded by \textit{ein} can be both neutral/literal (for many speakers) and emotive/figurative (for all speakers).\footnote{Note that \citet{deSwartEtAl2007} seem to have retreated from this position. However, I believe their former generalization is correct. In this regard, see, for instance, the discussion of John F. Kennedy’s famous sentence \textit{Ich bin ein Berliner} in \citet{Eichhoff1993}, which for many speakers does not only have the emotive/figurative reading but also the neutral/literal one. I thank Veronika Ehrich for this reference and some discussion of this topic.}  Below, we will see that the occurrence of \textit{ein} is simply a side effect and does not cause this figurative extension in interpretation.

  Returning briefly to the plural (\tabref{tab:6:2}), the facts are somewhat different. First, \textit{als} ‘as’ does not have to be present for a neutral ordinary reading in the DP (e.g., \textit{ihr Linguisten} ‘you linguists’), and second, \textit{ein} is not present in the plural.\footnote{Note that non-canonical predicative contexts may involve plural \textit{ein}. This can be seen in the constructed example below. 
    \judgewidth{(?)}
  	\ea[(?)]{ \label{ex:6:13:i}
  	\gll  Ihr  seid vielleicht ein-e Bauern!\\
  	  you are   \textsc{prt}            a-\textsc{pl}  peasants/farmers\\
  	\glt ‘You are some peasants!’
        \glt \%‘You are some farmers!’
        }
  	\z
    \judgewidth{\%}

  	 } With this in place, I finally turn to ambiguous role nouns in singular contexts.

\subsubsection{Ambiguous role nouns in the singular}\label{sec:6.2.2.4}

With the above discussion in mind, I consider examples involving \textit{Bauer} ‘farmer / peasant’. The expectation is that only some readings are possible in certain constructions. In fact, in the relevant cases, there should be a complementary loss of ambiguity. This is indeed borne out. To compare the distribution of the readings, I organize the data here according to syntactic domain.

  Starting with the nominal domain, I showed in \sectref{sec:6.2.2.2} that unambiguous role nouns (e.g., \textit{Landwirt} ‘farmer’) exhibit degraded grammaticality in the ordinary reading but are fine in the capacity reading. If so, we expect that ambiguous nouns have a complementary loss of meaning. Specifically, we predict that ambiguous nouns only involve an (emotive) comparative reading in the DP and that they have a neutral capacity reading in \textit{als-}nominals. This is exactly what we find in \REF{ex:6:26a} and \REF{ex:6:26b}, respectively.

\ea%26
    \label{ex:6:26}
\ea\label{ex:6:26a}
\gll du   Bauer\\
you peasant.\textsc{masc}\\
\glt ‘you (peasant)’

    \#‘you (farmer)’
\ex\label{ex:6:26b}
\gll du   als Bauer\\
you as  farmer.\textsc{masc}\\
\glt ‘you as a farmer’

    \#‘you as a peasant’
\z
\z

  Turning to the clausal cases, a bare role noun should only have a neutral ordinary reading \REF{ex:6:27a}, a role noun with an indefinite article should be ambiguous between a comparative and a neutral ordinary reading \REF{ex:6:27b}, and a role noun in the context of the modal particle \textit{vielleicht} ‘really’ should be ambiguous between an intensified comparative and an emotive capacity reading \REF{ex:6:27c}. Again, this is exactly what the data bear out.

\ea%27
    \label{ex:6:27}
\ea \label{ex:6:27a}
\gll Du  bist Bauer.\\
you are  farmer.\textsc{masc}\\
\glt ‘You are a farmer.’

    \#‘You are a peasant.’
\ex \label{ex:6:27b}
\gll Du  bist ’n Bauer.\\
you are   {\db}a peasant/farmer.\textsc{masc}\\
\glt ‘You are a peasant.’

    \%‘You are a farmer.’
\ex \label{ex:6:27c}
\gll Du  bist vielleicht ’n Bauer!\\
you are  \textsc{prt}           {\db}a peasant/farmer.\textsc{masc}\\
\glt ‘You are some peasant/farmer!’
\z
\z

The discussion can be summarized in \tabref{tab:6:4}.

\begin{table}
\caption{Summary of the readings of ambiguous role nouns in the singular}
\label{tab:6:4}
\begin{tabularx}{\textwidth}{lXXX}
\lsptoprule
 & Ordinary reading & \mbox{Comparative reading} & Capacity reading\\
\midrule
DP &  & Emotive \newline [\textit{du Bauer}] & Neutral \newline [\textit{du als Bauer}]\\
TP & Neutral \newline [(\textit{ein}) \textit{Bauer}] & Emotive \newline [\textit{ein Bauer}] & Emotive \newline [\textit{vielleicht ein \newline Bauer}]\\
\lspbottomrule
\end{tabularx}
\end{table}

\tabref{tab:6:4} is similar to \tabref{tab:6:3}, except for the ordinary reading. Unlike the kind noun \textit{Idiot} ‘idiot’, the role noun \textit{Bauer} ‘farmer/peasant’ cannot have an ordinary reading – the latter is only available in the TP, and it must be neutral there. Furthermore, note that \textit{Bauer} is only ambiguous in the presence of \textit{ein}. However, the article by itself does not have an “emotivizing” function as it does allow a neutral ordinary reading of \textit{Bauer} in the TP.

  Above, we saw that role nouns in clauses have optional indefinite articles but that kind nouns, independently of emotiveness, have obligatory indefinite articles. This difference between role and kind nouns is taken up in the next section. We will see that the appearance of \textit{ein} is a mere side effect of the presence of a certain semantic operator such that \textit{ein} flags the presence of that operator (Hypothesis 3b). Thus, I continue to claim that \textit{ein} is semantically vacuous (Hypothesis 1a). I also provide an explanation of the other highlighted properties of the aforementioned constructions. The proposal is based on two important works, \citet{deSwartEtAl2007} and \citet{Rauh2004}, that manifest an interesting division of labor, at least in the DP.

\section{Proposal}\label{sec:6.3}

In this section, I account for the similarities and differences between pronominal DPs and their clausal counterparts. First, I lay out my assumptions. This includes the discussion of two previous proposals. Then I turn to the three readings in detail.

\subsection{Combining two previous proposals}\label{sec:6.3.1}

In order to account for the readings in the nominal and clausal domains, I basically follow \citegen{deSwartEtAl2007} discussion of the copular cases and extend it to pronominal DPs. Before I get into the specifics, I lay the groundwork.

\subsubsection{\Citet{DeSwartEtAl2007}}\label{sec:6.3.1.1}

As mentioned before, this book is not about the semantics of the DP \textit{per se} (or the semantics of the TP, for that matter). Rather than providing detailed denotations, I use Type Theory to show that the relevant elements are semantically compatible with one another (for general background, see \chapref{sec:1}, \sectref{sec:1.4.2.2}; \citealt{HeimKratzer1998}). As discussed in \chapref{sec:3}, \sectref{sec:3.5}, I take pronominal elements like \textit{du} ‘you’ to be similar to definite articles like \textit{der} ‘the’; that is, pronouns are determiners.

  As discussed in \sectref{sec:6.2.1.1}, common nouns come in two types: kind nouns and (a specific set of) role nouns.\footnote{\Citet{deSwartEtAl2007} call the latter capacity nouns. There are two notions of capacity now: an interpretative one (a certain type of reading) and a lexical one (certain groups of nouns). These notions are not identical. For instance, certain kind nouns may also occur in capacity readings (\sectref{sec:6.2}). In order to avoid confusion, I continue calling the relevant lexical items role nouns.} As is well known, kind nouns can be used to generically refer to a species (but not to individuals of that species) in cases like \REF{ex:6:28a}. \citet{Carlson1980} proposed that kind nouns are like names; that is, they are of type \textlangle e\textrangle. Assuming that generic determiners are expletives (cf. \citealt{Longobardi1994}: 655-59), kind nouns and verbal predicates (type \textlangle e,t\textrangle) can combine to yield a truth value (type \textlangle t\textrangle). Turning to \REF{ex:6:28b}, these types of cases are different. Here, the nominal refers to an individual of that species. \Citet{deSwartEtAl2007} follow \citet{Carlson1980} in arguing that the realization operator REL takes a kind noun (type \textlangle e\textrangle) and maps it to a set of individuals (type \textlangle e,t\textrangle). The (substantive) determiner (type \textlangle\textlangle e,t\textrangle, e\textrangle) combines with this nominal predicate returning an entity (type \textlangle e\textrangle). The latter can combine with a verbal predicate to yield a truth value.

\ea%28
    \label{ex:6:28}
\ea\label{ex:6:28a}   The dinosaur is a big animal / is extinct.
\ex\label{ex:6:28b}   The dinosaur ate some fruit.
\z
\z

The novel claim of \citet{deSwartEtAl2007} is that role nouns are also of type \textlangle e\textrangle. To map them into sets of individuals, they have to combine with the capacity operator CAP, a realization operator restricted to role nouns. To be clear then, both sorts of nouns are of type \textlangle e\textrangle\ and have to be mapped to type \textlangle e,t\textrangle\ if they are to function as predicate nouns.

  \Citet{deSwartEtAl2007} relate the different realization operators to different syntactic structures. Kind nouns combine with REL, where the latter is assumed to be in NumP \REF{ex:6:29a}. In contrast, \citet{deSwartEtAl2007} propose that there are actually two options for role nouns. The first option involves a direct path, whereby role nouns combine with CAP. Note that CAP is in NP \REF{ex:6:29b}. The second option constitutes an indirect path, whereby role nouns undergo kind coercion resulting in kind nouns. \Citet[213]{deSwartEtAl2007} state that \textit{kind} coercion involves an operator as well (but I will not go into the specifics here to keep the discussion simple). The latter elements then combine with REL in NumP \REF[b\parbox{0mm}{$^{'}$}\,]{ex:6:29}.

\TabPositions{2cm}
\ea%29
    \label{ex:6:29}
\ea\label{ex:6:29a}   kind nouns:\tab  N\textsubscript{\textlangle e\textrangle} + REL       $\rightarrow$ NumP\textsubscript{\textlangle e,t\textrangle}
\ex\label{ex:6:29b}   role nouns:\tab   N\textsubscript{\textlangle e\textrangle} + CAP       $\rightarrow$ NP\textsubscript{\textlangle e,t\textrangle}

\exi{b\parbox{0mm}{$^{'}$}.} \label{ex:6:29b2}  role nouns:\tab  N\textsubscript{\textlangle e\textrangle} $\rightarrow$ kind coercion + REL  $\rightarrow$ NumP\textsubscript{\textlangle e,t\textrangle}
\z
\z

Note that REL and CAP are both of type \textlangle e\textlangle e,t\textrangle\textrangle {} – these operators take an entity as their argument and return a predicate.\footnote{\Citet[217]{deSwartEtAl2007} argue that CAP is actually more complicated than \textlangle e\textlangle e,t\textrangle\textrangle. However, the latter simplication is sufficient for the current account (keeping in mind that CAP only combines with role nouns).} Furthermore, these operators are mutually exclusive; that is, when one applies, the other cannot. Crucially for the current discussion, REL is part of NumP, which triggers the presence of the indefinite article \textit{ein} ‘a’ (note that this is not further elaborated on in \citeauthor{deSwartEtAl2007}). For the most part, I follow \citegen{deSwartEtAl2007} proposal. However, I refine it in certain ways below.

The two different paths to derive a predicate nominal in \REF[b-b']{ex:6:29} are argued to explain the fact that bare role nouns are only neutral/literal in meaning but that role nouns preceded by \textit{ein} ‘a’ are both neutral/literal and emotive/figurative (\citegen{deSwartEtAl2005} generalization, see \sectref{sec:6.2.2.3} again).\footnote{A different proposal is made by \citet[125]{Zamparelli2008}, who claims that role nouns are ambiguous between nominals that denote certain classes of human beings and nominals that denote abstract, well-established activities that identify those classes of human beings. Unlike the former, the latter nominals lack an inherent value for abstract gender, which, in turn, is taken to explain the absence of the indefinite article in predicative contexts. To the extent that I understand this correctly, this basically implies the existence of two lexical items for every role noun. In contrast, \citegen{deSwartEtAl2007} analysis takes one relevant lexical item as its point of departure and proposes two operations for the role noun to be able to function as a predicate. The latter approach appears to be more desirable.} As such, \citeauthor{deSwartEtAl2007} relate overt morphological clues directly to the relevant interpretative differences. Specifically, as predicate nominals, bare (role) nouns imply the presence of an NP, which in turn implies CAP \REF{ex:6:30a}. In contrast, plural suffixes on predicate nouns, exemplified in \REF{ex:6:30b} by -\textit{s}, are taken to be in NumP, and \textit{ein} is assumed to be in D in \citeauthor{deSwartEtAl2007}. Recall from above though that I proposed that singular \textit{ein} is in ArtP, and this is what I assume for unmodified predicative nominals \REF{ex:6:30c}. Note that both \REF{ex:6:30b} and \REF{ex:6:30c} involve REL. Thus, the minimal structures of predicate nominals are as follows (also \citealt{Borer2005}: 67 fn. 4, \citealt{MunnSchmitt2005}: 827, cf. \citealt{Déprez2005}: 866).\footnote{\Citet{deSwartEtAl2007} make another important claim about these data. They propose that morphological number is actually not specified on nouns. In particular, bare nouns \REF{ex:6:30a} are number neutral (this is one of the properties mentioned in \sectref{sec:6.2.1.1}, discussed in more detail in \chapref{sec:7}), but nouns with plural morphology \REF{ex:6:30b} and nouns preceded by \textit{ein} \REF{ex:6:30c} are specified for plural and singular, respectively (the latter two are called Marked Nominals by \citeauthor{deSwartEtAl2007}.). Given the presence of NumP in the latter two cases, \citet[214, 217]{deSwartEtAl2007} take morphological and semantic number to originate in NumP. This is similar to my assumptions in \chapref{sec:7}.}

\TabPositions{1cm,5.5cm}
\ea%30
    \label{ex:6:30}
\ea  \label{ex:6:30a}  N  	      \tab$\rightarrow$ [ NP ]          \tab$\rightarrow$ CAP
\ex \label{ex:6:30b}   N-\textit{s}   \tab$\rightarrow$ [ NumP [ NP ]]  \tab$\rightarrow$ REL
\ex \label{ex:6:30c}
{ein} N  			\tab$\rightarrow$ [ ArtP [ NumP [ NP ]]]  \tab$\rightarrow$ REL
\z
\z

Given \REF[a-c]{ex:6:30}, it is clear that unmodified predicate nominals do not involve DPs. Rather, such nominals are of a smaller size; that is, in the absence of adjectival modifiers, predicate nominals preceded by \textit{ein} do not involve DP but ArtP (recall in this regard that \textit{ein} has no feature for definiteness and does not have to move to the DP-level).\footnote{In fact, even in the presence of adjectival modifiers, these nominals can be smaller than DP – \textit{ein} could be in Card, which is above AgrP, the phrase hosting adjectives. If this turns out to be tenable, then we could claim that DPs involve arguments, but not predicates (e.g., \citealt{Borer2005}: 65-66, \citealt{DéchaineWiltschko2002}: 419, \citealt{Stowell1989}; cf. also \citealt{Longobardi1994}: 628).} If so, then this allows me to state that \textit{ein} and REL are in a local relation (i.e., in adjacent phrases in underlying structure).\footnote{If REL were to turn out to be in ArtP (at least in the singular cases), both \textit{ein} and REL would be in the same phrase.} To be clear, considering \REF{ex:6:30}, predicate nominals can be of different sizes: NP, NumP, ArtP, and larger structures (if adjectival modifiers are present).

  Comparing \REF{ex:6:30a} to \REF{ex:6:30c}, I note already here that \textit{ein} indicates the presence of structure beyond NP (Hypothesis 3a) and that \textit{ein} flags the presence of REL (Hypothesis 3b). The latter claim provides a reason for the appearance of \textit{ein}.\footnote{Notice that this comes close to one of \citegen[15]{Ackles1996} tentative suggestions that English \textit{a(n)} might be a dummy that makes the presence of NumP in an indefinite singular count noun phrase identifiable. Assuming the structural sequence of Q D Num N, she proposes that \textit{a(n)} is the minimal marker that is inserted in NumP when the latter is the leftmost empty phrase of a noun phrase.} Thus far, the proposal seems straightforward.

  \largerpage
However, it is not entirely clear which part of \citegen{deSwartEtAl2007} account derives the emotive/figurative reading. Although intuitively attractive at first glance, I show that the emotive/figurative reading does, most likely, not derive from kind coercion but from the presence of REL. Furthermore, as already briefly pointed out in \sectref{sec:6.2.1.2}, not all nouns undergo figurative extension (e.g., \textit{Landwirt} ‘farmer’). Thus, the figurative extension of the literal meaning (e.g., \textit{Bauer} ‘farmer’ $>$ ‘peasant’) must take a certain lexical specification of the head noun into account. Finally, I extend \citegen{deSwartEtAl2007} proposal to pronominal DPs (note in this regard that \citealt{deSwartEtAl2007}: 215 assume that ordinary DPs like \textit{the child} also involve REL). I consider these points in more detail.

\largerpage
\subsubsection{Lexical features on nouns}\label{sec:6.3.1.2}

As seen above, certain nouns can only be neutral/literal in meaning (e.g., \textit{Landwirt} ‘farmer’), but others can, under certain conditions, be ambiguous (e.g., \textit{Bauer} ‘farmer/peasant’). I start with the latter. It is clear from \citegen{deSwartEtAl2007} proposal that the emotive/figurative reading is not derived from the direct path involving CAP. If this were the case, a bare role noun in the TP could have an emotive/figurative reading, contrary to fact. This leaves the indirect path where kind coercion and REL are at work. With REL present, NumP will be projected, and the indefinite article will be present in ArtP too. Unlike the bare noun counterpart, \textit{ein Bauer} can have both a neutral/literal and a emotive/figurative reading. The question then arises whether kind coercion or REL is responsible for the figurative reading. As we see below, the answer is not entirely straightforward and involves several considerations. I wind up suggesting that in combination with a lexical feature, REL brings about the interpretative extension (\textit{pace} \citealt{deSwartEtAl2007}).

  First, considering \textit{ein Landwirt} ‘a farmer’, we observe that an article is present with a role noun. This means that ArtP, NumP, and REL are present. With a structure larger than NP, kind coercion must have occurred. However, this nominal has a neutral/literal meaning only. This means that certain nouns do not undergo extension of their neutral/literal meaning. As such, nouns must be lexically specified as to whether or not they can undergo this interpretative extension. In other words, neither kind coercion nor REL automatically invokes the figurative extension.

  Second, when applied to a person, names for animals (e.g., \textit{Schwein} ‘pig’) can only have an emotive/figurative meaning. Since these nouns are kind nouns to begin with, they do not undergo kind coercion. As such and making minimal assumptions, REL must be responsible for this extension in meaning (‘pig-like person/swine’).\footnote{Alternatively, it is possible to suggest here that a null predicate, call it LIKE, is responsible for the extension in meaning (cf. \citegen[178]{denDikken2006} predicate SIMILAR in the comparative binominals). However, I proceed on minimal assumptions; that is, I assume that REL brings about the figurative extension.} However, REL performing this task cannot be the whole story. Certain other kind nouns (e.g., \textit{Katze} ‘cat’) do not seem to be able to undergo this extension easily or at all. This makes these kind nouns similar to the role noun \textit{Landwirt} (see also the word pairs in Footnote \ref{foot:6:6}).

  I propose that certain vocabulary items (\textit{Bauer}, \textit{Schwein}) involve a lexical feature that allows figurative extension; other nouns (\textit{Landwirt}, \textit{Katze}) do not have this feature and cannot undergo this extension. With this in mind, I make the plausible assumption that \textit{Bauer} and \textit{Schwein} are similar in the relevant way (recall that the literal meaning of \textit{Schwein} ‘pig’ is excluded for pragmatic reasons when the relevant entity is anatomically not a pig). For concreteness, I assume that the relevant lexical items are specified as [+figurative] if they are able to undergo the extension. Thus, while both \textit{Bauer} and \textit{Schwein} are specified with this feature, \textit{Landwirt} and \textit{Katze} are not. Given that \textit{ein Bauer} can, for many speakers, also have a neutral/literal meaning, I assume that figurative extension is not obligatory (unless forced by the pragmatics).

  To be clear then, I basically follow \citet{deSwartEtAl2007} with the refinement that figurative extension is not obligatory, that it is due to REL (and not kind coercion), and that it is constrained by the feature [+figurative] present on certain nouns.\footnote{Notice that kind coercion is still needed to map a role noun to a kind noun, so that the noun can combine with REL.} Note again that kind coercion only applies to role nouns but figurative extension to both role and kind nouns. In other words, figurative extension does not entail kind coercion. However, figurative extension does yield a special case of emotiveness. More generally, if REL is indeed responsible for the (optional) figurative extension, then I can continue to claim that \textit{ein} makes no semantic, but only a morpho-syntactic, contribution. I reiterate the proposal that the main functions of \textit{ein} are indicating the size of the structure (Hypothesis 3a) and flagging the presence of an operator (Hypothesis 3b). In other words, flagging REL is the reason why \textit{ein} occurs in these contexts. Before I extend \citegen{deSwartEtAl2007} discussion to pronominal DPs, it is important to point out that the full explanation of the nominal cases involves a(nother) pragmatic aspect as part of the account.

\subsubsection{\citet{Rauh2004}} \label{sec:6.3.1.3}

Thus far, I have only considered pragmatically neutral, out-of-the-blue examples for the ordinary reading in the DP \REF{ex:6:31a}. However, as \citet{Rauh2004} discusses in detail, these cases are perfectly fine given an appropriate context. Two of her examples may suffice here \REF[b-c]{ex:6:31} (stress is indicated here by capital letters).

\ea%31
    \label{ex:6:31}
\ea[??]{ \label{ex:6:31a}
\gll  \{ich / du\} Linguist\\
{\db}I    / you linguist.\textsc{masc}\\
  }
\ex \label{ex:6:31b}
 {Ihr Literaturwissenschaftler mögt den jetzigen Zustand für angemessen halten,       aber ich LINGUIST halte die Linguistik für weit unterrepräsentiert}.\\
\glt ‘You literature scholars may consider the current status quo as adequate but I       (linguist) consider linguistics as quite underrepresented.’
\ex\label{ex:6:31c}
{Wenn noch nicht einmal du LINGUIST die neue Rechtschreibung beherrschst,       wer sollte es dann}?\\
\glt ‘If not even you (linguist) have mastered the new spelling rules, who else could       do that?’
\z
\z


While I cannot do full justice to all the facets of her account here (e.g., the different stress patterns involved), Rauh’s basic proposal is based on the different deictic qualities of the personal pronouns involved and two Gricean maxims (\citealt{Grice1975,Grice1978}).

  As a rule, \textit{ich} ‘I’ and \textit{du} ‘you(\textsc{sg})’ are disambiguously specified in their deixis such that the identity of the person concerned is clear in a given context. Additional restrictive information provided by an NP complement (e.g., \textit{Linguist}) is not needed. In fact, by Grice’s maxim of quantity (informally: “Be as informative as required but not more”), it is redundant and thus yields a marked string. Given a different, more involved context as in \REF[b-c]{ex:6:31} above, such an NP complement may make a relevant contribution, and as such it is allowed by Grice’s maxim of relation (informally: “Be relevant”). Specifically, this contribution involves a contrastive or emotive component; compare \REF{ex:6:31b} to \REF{ex:6:31c}. With this in mind, emotive NP complements such as \textit{Idiot} ‘idiot’ are always relevant and thus always possible. Finally, plural pronouns are less deictically specified and thus allow restrictive material in the complement position. I relate \citegen{Rauh2004} proposal for the DP to \citegen{deSwartEtAl2007} account of the TP.

  As I discuss in detail in \chapref{sec:7}, all DPs including pronominal DPs must involve NumP, and as such REL is always present there. With \citegen{deSwartEtAl2007} discussion of the clause in mind, this makes both a neutral and an emotive noun, at least in principle, possible inside the complement position of a pronominal determiner. Now, recalling \citegen{Rauh2004} pragmatic account of singular DPs, only an emotive complement remains possible in pragmatically neutral, out-of-the-blue contexts. In contrast, singular DPs in a different, appropriate context or plural DPs in general can have neutral or emotive NP complements. To be clear, then, the explanation of pronominal DPs involves a straightforward extension of \citet{deSwartEtAl2007}, once we assume that NumP (and thus REL) is always present there and that the explanation must also involve pragmatic considerations. Conversely, predicate nominals in TPs make an assertion, and as such, they provide informative and relevant information. Consequently, they are in agreement with \citegen{Rauh2004} assumptions, here extended to TPs. Finally, combining these two proposals also explains the fact that in pragmatically neutral, out-of-the-blue contexts, there are no true minimal pairs between pronominal DPs and copular TPs when both involve bare predicate nouns (for more details, see \sectref{sec:6.3.3}).

  In the next section, I turn to the specifics of my proposal. I focus on the cases in the singular where I have identified most restrictions and where \textit{ein} is present in emotive contexts. Before doing so, I summarize in \tabref{tab:6:5} the cases in the singular (without \textit{als} ‘as’), discussed in this and the previous section. Note that the first column indicates the inherent (non-)emotiveness of the head noun but that the second column shows the readings that noun may take on in DPs and TPs. These readings are exemplified in the third and fourth columns.

\begin{table}
\caption{Nouns and their readings in the DP and TP}
\label{tab:6:5}
\small
\begin{tabularx}{\textwidth}{lXQQ}
\lsptoprule
Noun & Reading & DP & TP\\
\midrule
Neutral & Literal & {\itshape du LANDWIRT} \newline
		    {‘you farmer’}
 (stressed -  contrastive) & { \textit{Du bist (’n) Landwirt}.} \newline
				‘You are a  farmer.’\\
Emotive & Emotive & {\itshape du Idiot}\newline
			    ‘you idiot’ & {\itshape Du bist ’n Idiot!}\newline
						    ‘You are an Idiot!’\\
Neutral & Emotive/figurative [comparative] & {\itshape du Schwein  / Bauer}  \newline
						    ‘you idiot  / peasant’ & {\itshape Du bist ’n  Schwein / Bauer!}\newline
						    ‘You are an idiot  / peasant!’\\
Neutral & Emotive & {\itshape du LANDWIRT}\newline
			    { ‘you farmer’} (stressed -  emotive) & {\itshape Du bist vielleicht ’n Landwirt!} \newline
			    ‘You are some  farmer!’\\
\lspbottomrule
\end{tabularx}
\end{table}

Note again that all the emotive readings have a determiner including \textit{ein} in singular predicate nominals. However, it is clear from the optionality of \textit{ein} in clausal cases like \textit{Du bist (’n) Landwirt} that the presence of \textit{ein} does not necessarily lead to emotiveness. This fact is consistent with the proposal that \textit{ein} is semantically vacuous. This means that in instances where \textit{ein} is present in emotive contexts, \textit{ein} must be due to a different reason. I proposed above that \textit{ein} flags the presence of a null operator. With this in place, I turn to the explanation of the individual readings in the DP and TP starting with the – what I have called – ordinary reading.

\subsection{Ordinary reading}\label{sec:6.3.2}
\largerpage
I propose that the ordinary reading involves direct predication. What I mean by that is that the pronoun and noun combine with one another without invoking a figurative extension of the literal meaning of the head noun. This is different from the comparative reading discussed in the next section. First, I address the nominal domain and then the clausal one.

\subsubsection{Ordinary reading in the DP}\label{sec:6.3.2.1}

As discussed in \chapref{sec:3}, I assume that pronominal DPs consist of a head noun and a pronominal determiner. To keep the exposition simple, I abstract away from the lower base position of the determiner and its movement to the DP-level. I assume here and argue below that NumP is sandwiched between the head noun and the pronominal determiner. As such, pronominal DPs exhibit a regular DP structure (see \chapref{sec:1}, \sectref{sec:1.4}). Consider the example in \figref{figex:6:32} and its structural analysis (for ease of exposition, the operator REL is put on Num).

\glltree[\label{figex:6:32}]{\small
	Ordinary reading in DP\\
	\gll du Idiot\\
	you idiot.\textsc{masc}\\
	\glt \small{‘you (idiot)’}
}{
	[DP
		[\textit{du}\textsubscript{\parbox{0mm}{\mbox{\textlangle\textlangle e,t\textrangle e\textrangle}}}]
		[NumP\textsubscript{\parbox{0mm}{\mbox{\textlangle e,t\textrangle}}}, edge label={node[midway,above right]{(Functional Application)}}, s sep=20mm
			[Num\textsubscript{\parbox{0mm}{\mbox{\textsc{rel}\textlangle e\textlangle e,t\textrangle\textrangle}}}]
			[NP\\\textit{Idiot}\textsubscript{\textlangle e\textrangle}, edge label={node[above right]{(Functional Application)}}]
		]
	]
}

I turn to the semantics. With \sectref{sec:6.3.1.1} in mind, I restrict myself to giving the semantic types of the relevant elements and how they are combined. This is shown in \figref{figex:6:32}. Putting it into words and starting at the bottom of the tree, the kind noun \textit{Idiot} ‘idiot’ combines with the REL operator in NumP to return a predicate nominal. I take \citegen{Postal1966} claim that pronouns are determiners at face value. In other words, I assume that \textit{du} ‘you(\textsc{sg})’ is a pronominal determiner that is semantically similar to an ordinary definite determiner.\footnote{In \chapref{sec:7}, I discuss the ungrammaticality of cases like \textit{*du ’n Idiot} ‘(lit.) you an idiot’. I assume that pronominal determiners can flag the presence of REL preventing the occurrence of \textit{ein} (also \sectref{sec:6.3.4.1}). This makes these types of determiners similar to ordinary determiners like definite articles (see \citealt{deSwartEtAl2007}: 215).} Given that, Functional Application can apply to the predicate nominal and the pronominal determiner resulting in an entity with the following reading: the unique x (informally addressed) such that x is an idiot. The derivation is the same for \textit{Mann} ‘man’ but requires a special pragmatic context \citep{Rauh2004}. Similar assumptions hold for \textit{Landwirt} ‘farmer’, but the latter undergoes kind coercion. Since \textit{Landwirt} is not marked by [+figurative], figurative extension does not occur.

  Before I discuss the clausal cases, I need to be more explicit about pronouns that have no noun as their complement. Similar to what we have just seen, I assume that they are not simply “intransitive” but more complex. In particular, consistent with \chapref{sec:3}, \sectref{sec:3.5}, I assume that there is actually a head noun. I suggest that this head is an unpronounced kind noun, indicated below as \textit{e\textsubscript{N}} (cf. PERSON in \citealt{Rauh2004}: 89). The latter combines with REL to give a predicate nominal. This complex null element, semantically REL(\textit{e\textsubscript{N}}), can now function as an argument to the pronominal functor resulting in an entity. Note that \figref{figex:6:33} has the same abstract structure and derivation as in \figref{figex:6:32} above.

\begin{figure}
	\caption{Pronouns without overt complement}
	\label{figex:6:33}
	\begin{forest}
		[DP\textsubscript{\parbox{0mm}{\mbox{\textlangle e\textrangle}}}
			[\textit{du}\textsubscript{\parbox{0mm}{\mbox{\textlangle\textlangle e,t\textrangle e\textrangle}}}]
			[NumP\textsubscript{\parbox{0mm}{\mbox{\textlangle e,t\textrangle}}}, s sep=20mm
				[Num\textsubscript{\parbox{0mm}{\mbox{\textsc{rel}\textlangle e\textlangle e,t\textrangle\textrangle}}}]
				[NP\\\textit{e\textsubscript{N}}\textsubscript{\textlangle e\textrangle}]
			]
		]
	\end{forest}
\end{figure}

\subsubsection{Ordinary reading in the TP}\label{sec:6.3.2.2}
\largerpage[-1]

We saw in \sectref{sec:6.3.1.1} that predicate nominals can be of different sizes: NP, NumP, ArtP, or larger (if an adjectival modifier is present). As proposed in the previous subsection, pronominal DPs involve regular structures that consist of NPs, NumPs, ArtPs, and other phrases. I propose that predicate nominals in copular TPs are different. Unlike DPs, I claim that predicate nominals in copular clauses do not necessarily involve NumPs, ArtPs, or other phrases – they can just be NPs. Now, similar to the DP, the ordinary reading in the clause does not involve figurative extension either. Adopting fairly traditional tree representations, I assume that the clause involves a Predication Phrase (PrP, \citealt{Bowers1993}) where the head Pr selects NP.\footnote{There are other possibilities; for instance, the auxiliary could select a Small Clause (e.g., \citealt{Hoekstra1984}: 231). Note also that \citet{Bowers1993} employs PrP in the DP. However, I follow here the more familiar structural view of DPs already discussed above. The resulting differences in names for functional phrases should not be taken as an indication that the DP and TP are not parallel.} The underlying structure of the example of \figref{figex:6:34} is given in the figure above it (for simplicity, I abstract away here from further movements to the left, see \chapref{sec:7}).

\glltree[\label{figex:6:34}]{\small
	Ordinary reading in TP with role nouns\\
	\gll Du  bist Bauer.\\
	you are  farmer.\textsc{masc}\\
	\glt \small{‘You are a farmer.’}
}{
	[PrP\textsubscript{\parbox{0mm}{\mbox{\textlangle t\textrangle}}}
		[{[\textit{du} REL \textit{e\textsubscript{N}}]}\textsubscript{\textlangle e\textrangle}]
		[ Pr$'$\textsubscript{\parbox{0mm}{\mbox{\textlangle e,t\textrangle}}}
			[\textit{bist}]	
            [NP\textsubscript{\parbox{0mm}{\mbox{\textsc{cap}\textlangle e\textlangle e,t\textrangle\textrangle}}}\\\textit{Bauer}\textsubscript{\parbox{0mm}{\mbox{\textlangle e\textrangle}}}]
		]
	]
}

As to the semantics, \textit{Bauer} is a role noun, and unlike kind nouns it can combine with CAP. The latter is located in NP and brings about a predicate nominal. Above, we saw that “intransitive” pronouns like \textit{du} ‘you(\textsc{sg})’ actually involve pronominal DPs; that is, they involve entities enclosed by square brackets in \figref{figex:6:34}. Recalling that the copula is semantically vacuous, the pronominal DP and the predicate nominal combine by Functional Application to yield a truth value. Note now that \figref{figex:6:34} cannot have the emotive/figurative reading ‘peasant’. This is so because CAP (but not REL) brought about the predicate nominal involving \textit{Bauer}. Finally, in order to account for the full grammaticality of these types of examples, observe that the predicate noun makes an assertion about the subject, and as such it is informative and relevant from a pragmatic point of view. The same holds if we exchange \textit{Bauer} with \textit{Landwirt} ‘farmer’.

Finally, it is worth mentioning again that the combination of “\textit{ein} + N” can also result in ordinary readings. With \textit{Idiot} ‘idiot’ an emotive kind noun, it combines with REL in NumP bringing about \textit{ein} in ArtP. Here, the head Pr selects ArtP (see also further below). Other than that, the remainder of the derivation proceeds as above and is illustrated in \figref{figex:6:35}.

\glltree[\label{figex:6:35}]{\small
	Ordinary reading in TP with kind nouns\\
	\gll Du  bist ’n  Idiot.\\
	you are   {\db}an idiot.\textsc{masc}\\
	\glt \small{‘You are an idiot.’}
}{
	[PrP\textsubscript{\parbox{0mm}{\mbox{\textlangle t\textrangle}}}
		[{[\textit{du} REL \textit{e\textsubscript{N}}]}\textsubscript{\textlangle e\textrangle}]
		[Pr$'$
			[\textit{bist}]
			[ArtP
				[\textit{'n}]
				[NumP\textsubscript{\parbox{0mm}{\mbox{\textlangle e,t\textrangle}}}, s sep=20mm
					[Num\textsubscript{\parbox{0mm}{\mbox{\textsc{rel}\textlangle e\textlangle e,t\textrangle\textrangle}}}]
					[NP\\\textit{Idiot}\textsubscript{\textlangle e\textrangle}]
				]
			]
		]
	]
}

The same holds if the noun involves \textit{Mann} ‘man’. As to \textit{ein}, note again that this element does not have an emotivizing function (e.g., \textit{ein Mann} ‘a man’); that is, \textit{ein} is an expletive element and does not participate in the semantic derivation. The ambiguous noun \textit{Bauer} ‘farmer/peasant’ preceded by \textit{ein} will be discussed in the next section (comparative readings). This is also where I comment on the optional presence of \textit{ein} in the ordinary reading in the TP (e.g., \textit{Du bist (’n) Bauer}. ‘You are a farmer.’).

  To summarize, the ordinary reading is emotive in the nominal domain but neutral or emotive in the clausal domain. Both cases involve direct predication; that is, there is no figurative extension in the DP (i.e., the head noun is inherently emotive), and the TP involves a role noun in combination with CAP or an emotive or neutral kind noun combining with REL. Next, I discuss cases with figurative extension.

\subsection{Comparative reading}\label{sec:6.3.3}

Before I provide the structures of the relevant DPs and TPs, I start with some general considerations.

\subsubsection{Preliminaries}\label{sec:6.3.3.1}

As a point of departure, I begin with more complex comparative structures. It is well documented that an attributive adjective and its head noun must agree in phi-features yielding concord. This means that the adjectives in \REF{ex:6:36} below are in the extended projection of their following noun (\chapref{sec:1}, \sectref{sec:1.4}). 

\ea%36
    \label{ex:6:36}
\ea  \label{ex:6:36a}
\gll ihr               dumm-en   Schweine\\
you\textsc{.pl.nom} stupid-\textsc{wk} pigs\\
\glt ‘you stupid idiots’
\ex  \label{ex:6:36b}
\gll mir               groß-en   Gans\\
me\textsc{.sg.dat} great-\textsc{wk} goose.\textsc{fem}\\
\glt ‘me (stupid idiot)’
\z
\z

In \chapref{sec:3}, \sectref{sec:3.5}, I proposed that pronominal DPs involve canonical structures. In other words, these comparative constructions have the pronominal determiner, the adjective, and the noun in a regular DP structure, just like with the ordinary reading. I proceed with examples involving unmodified nouns.

\subsubsection{Comparative reading in the DP}\label{sec:6.3.3.2}

Recall that I proposed that the comparative reading involves a figurative extension of the neutral/literal meaning of the head noun; for example, the neutral/literal meaning of \textit{Bauer} ‘farmer’ is extended to ‘farmer-like person’ or ‘peasant’. I proposed that this is due to REL. This fits well with the assumption that NumP is present in the DP. The example below is derived as in \figref{figex:6:37}.

\glltree[\label{figex:6:37}]{\small
	Comparative reading in DP\\
	\gll du   Bauer\\
	 you peasant.\textsc{masc}\\
	\glt \small{‘you (peasant)’\\
	\#‘you (farmer)’}
}{
	[DP\textsubscript{\parbox{0mm}{\mbox{\textlangle e\textrangle}}}
		[\textit{du}\textsubscript{\parbox{0mm}{\mbox{\textlangle\textlangle e,t\textrangle e\textrangle}}}]
		[ NumP\textsubscript{\parbox{0mm}{\mbox{\textlangle e,t\textrangle}}}, s sep=20mm
			[Num\textsubscript{\parbox{0mm}{\mbox{\textsc{rel}\textlangle e\textlangle e,t\textrangle\textrangle}}}]
			[NP\\\textit{Bauer}\textsubscript{\textlangle e\textrangle}]
		]
	]
}

As for the interpretation, the role noun \textit{Bauer} undergoes kind coercion and combines with REL in NumP. As the noun is lexically marked as [+figurative], it can undergo figurative extension. Unlike the neutral/literal reading, this extension yields emotiveness and is felicitous in pragmantically neutral, out-of-the-blue contexts \citep{Rauh2004}. Next, NumP combines with the pronoun resulting in an entity with the following reading: the unique x (informally addressed) such that x is, in some way, similar to a farmer. Similar considerations apply to \textit{Schwein} ‘pig’, with the proviso that the interpretation can only be ‘pig-like person’ or ‘swine’ for pragmatic reasons.

\subsubsection{Comparative reading in the TP}\label{sec:6.3.3.3}

If the above discussion of the DP is tenable, then the comparative reading in the TP should involve a similar account. See for example \figref{figex:6:38}.

\glltree[\label{figex:6:38}]{\small
	Comparative reading in TP\\
	\gll Du  bist ’n Bauer.\\
	you are   a peasant/farmer.\textsc{masc}\\
	\glt \small{‘You are a peasant.’\\
	\%‘You are a farmer.’}
}{
	[PrP\textsubscript{\textlangle t\textrangle}
		[{ [\textit{du} REL \textit{e\textsubscript{N}}]}\textsubscript{\textlangle e\textrangle}]
		[Pr$'$
			[\textit{bist}]
			[ArtP
				[\textit{'n}]
				[ NumP\textsubscript{\parbox{0mm}{\mbox{\textlangle e,t\textrangle}}}, s sep=20mm
					[ Num\textsubscript{\parbox{0mm}{\mbox{\textsc{rel}\textlangle e\textlangle e,t\textrangle\textrangle}}}]
					[NP\\\textit{Bauer}\textsubscript{\textlangle e\textrangle}]
				]
			]
		]
	]
}

Like above, \textit{Bauer} undergoes kind coercion and then combines with REL in NumP. Structurally, this brings about \textit{ein} ‘a’ in ArtP. Semantically, this complex element, namely REL(\textit{Bauer}), combines with the pronominal DP to return a truth value. Note that \textit{Bauer} has the option of keeping its neutral/literal meaning or undergoing figurative extension. This is so because this noun is specified [+figurative], but the figurative extension by REL is not obligatory. This derives the two readings in \figref{figex:6:38}. The same basic derivation as in \figref{figex:6:38} applies to \textit{Schwein} ‘pig’, with the same proviso mentioned for the DP above.

Returning to an issue left open in the previous section, notice that the ordinary reading of the role noun involves the indirect path in \figref{figex:6:38}, and as such it entails more structure and more operations than the direct path (\sectref{sec:6.3.1.1}). This might explain why the combination of \textit{ein} and \textit{Bauer} in its neutral/literal meaning is somewhat less easily available or even absent for some speakers.

  Finally, recall also that the modal particle \textit{vielleicht} ‘really’ can intensify the comparative reading resulting in an interpretation close to ‘really like’ (see the example under \figref{figex:6:39}). Consonant with the above discussion and adding an adjective to the predicate nominal, the noun and its modifier are in their regular position, see \figref{figex:6:39}. Note that the adjectival predicate and the nominal predicate combine by Predicate Modification.

\glltree[\label{figex:6:39}]{\small
	Intensified comparative reading in TP\\
        \gll Du  bist vielleicht ’n echter Bauer!\\
	you are  \textsc{prt}           {\db}a  real     peasant.\textsc{masc}\\
	\glt \small{‘You are really like a peasant!’}
}{
	[CardP
		[\textit{'n}]
		[AgrP\textsubscript{\parbox{0mm}{\mbox{\textlangle e,t\textrangle}}}
			[\textit{echter}\textsubscript{\textlangle e,t\textrangle}]
			[NumP\textsubscript{\parbox{0mm}{\mbox{\textlangle e,t\textrangle}}}, s sep=20mm, edge label={node[midway, above right]{~~~~~(Predicate Modification)}}
				[ Num\textsubscript{\parbox{0mm}{\mbox{\textsc{rel}\textlangle e\textlangle e,t\textrangle\textrangle}}}]
				[NP\\\textit{Bauer}\textsubscript{\textlangle e\textrangle}]
			]
		]
	]
}

Taking stock, the comparative reading in the nominal and clausal domain involves the operator REL in NumP. In combination with a [+figurative] role noun, this results in an emotive/figurative reading in the DP and in the TP (although a neutral/literal reading is not excluded in the latter domain). Note now that the interaction between \citet{deSwartEtAl2007} and \citet{Rauh2004} also affords us an explanation of why there are no true minimal pairs of the type \textit{du Noun\textsubscript{1}} ‘you Noun’ and \textit{Du bist Noun\textsubscript{1}} ‘You are a Noun’ in pragmatically neutral contexts.

Specifically, all pronominal DPs involve NumP and thus REL. The latter may bring about the emotive/figurative extension of the meaning of the head noun. Now, \citegen{Rauh2004} pragmatic proposal explains why only inherently emotive kind nouns (\textit{Idiot}), [+figurative] kind nouns (\textit{Schwein}), and [+figurative] role nouns (\textit{Bauer}) are possible in the DP. In contrast, \citet{deSwartEtAl2007} argue that bare predicate nouns in the TP only involve NP and thus CAP. With REL absent here, this only allows the occurrence of role nouns, with the feature [+figurative] (\textit{Bauer}) or without (\textit{Landwirt}). Due to the absence of REL, \textit{Bauer} can only have the neutral/literal meaning like \textit{Landwirt}. To be clear, then, \citegen{Rauh2004} proposal forces the bare noun to be emotive in the DP, and \citegen{deSwartEtAl2007} analysis explains why the bare noun can only be neutral in the TP. If so, the absence of minimal pairs should not be taken as an argument that DPs and TPs are fundamentally different after all – the difference is explained by the interplay of the two proposals above, which involve pragmatic considerations. I turn to the last reading.

\subsection{Capacity reading}\label{sec:6.3.4}
\largerpage
Before I discuss the detailed analyses of the DP and TP, I need to provide some background information motivating the relevant structures.

\subsubsection{Preliminaries}\label{sec:6.3.4.1}

We have seen that DPs involving \textit{als} ‘as’ have a (neutral) capacity reading \REF{ex:6:40a}. The latter can be paraphrased as ‘in the capacity of / with regard to being’. In the TP \REF{ex:6:40b}, the modal particle \textit{vielleicht} ‘really’ invokes an emotive capacity reading similar to ‘bad/good in the capacity of / with regard to being’, and it intensifies the comparative reading (for the structure of the latter, see previous section).\footnote{\Citet[218]{deSwartEtAl2007} assume that the combination of “\textit{als} + NP” is a verb modifier. Note though that this does not work for \textit{als}-nominals that are part of pronominal DPs: Given the Verb Second Constraint in German, the latter clearly involve constituents: \textit{Du als Arzt solltest das nicht machen} ‘You as a doctor should not do that’.}

\ea%40
    \label{ex:6:40}
\ea  \label{ex:6:40a}
\gll du   als Bauer\\
you as  farmer.\textsc{masc}\\
\glt ‘you as a farmer’

    \#‘you as a peasant’
\ex  \label{ex:6:40b}
\gll Du  bist vielleicht ’n Bauer!\\
you are  \textsc{prt}           {\db}a  farmer/peasant.\textsc{masc}\\
\glt ‘You are some farmer/peasant!’
\z
\z

\largerpage
Starting with the DP, I make the fairly straightforward proposal that \textit{als} ‘as’ brings about the capacity reading. As to the TP, I would like to suggest that in combination with the modal particle, the null equivalent of \textit{als} (i.e., ALS) derives the corresponding clausal reading. Like all null elements, ALS has to be licensed. I assume that the modal particle is responsible for that. Schematically, we arrive at the following, where the relevant noun phrases of the data in \REF[a-b]{ex:6:40} are put in brackets in \REF[a-b]{ex:6:41}.

\ea%41
    \label{ex:6:41}
\ea \label{ex:6:41a} [du als Bauer]
\ex \label{ex:6:41b}
 {vielleicht} [ein ALS Bauer]
\z
\z

The bracketed strings are of particular relevance here, specifically, the syntactic properties of the \textit{als}/ALS-nominals. First, I consider \textit{als}-nominals in combination with third-person pronouns. As is well known, these pronouns are syntactically very restrictive and therefore help us find a plausible structural analysis.

  Unlike pronouns of the first and second person, pronouns of the third person cannot directly combine with adjectives and/or overt nouns (see \citealt{Höhn2020} and \chapref{sec:3}, \sectref{sec:3.5}). Compare \REF{ex:6:42a} to \REF[b-d]{ex:6:42}.

\ea%42
    \label{ex:6:42}
\ea  \label{ex:6:42a}
\gll er als Bauer\\
he as  farmer.\textsc{masc}\\
\glt ‘he as a farmer’
\ex[*]{  \label{ex:6:42b}
\gll er Gute(r)\\
he  good\\
}
\ex[*]{  \label{ex:6:42c}
\gll er Bauer\\
he farmer.\textsc{masc}\\
}
\ex[*]{
\gll er gute(r) Bauer\\
he good    farmer.\textsc{masc}\\
}
\z
\z

However, they can be modified by relative clauses \REF{ex:6:43a} (see \citealt{Rauh2003}, \citealt{Vater1985}). In other words, I take \REF{ex:6:42a} to be on a par with \REF{ex:6:43a}. Assuming the traditional adjunction analysis for relative clauses, I propose that \textit{als Bauer} is also adjoined to the pronoun (cf. \citealt{Rauh2004}: 94), schematically illustrated in \REF{ex:6:43b}.

\ea%43
    \label{ex:6:43}
\ea  \label{ex:6:43a}
\gll er, der  gerade durch    die Tür  gekommen ist\\
he who just      through the door come         is\\
\glt ‘he, who just came through the door’
\ex  \label{ex:6:43b}   Adjunction-type Analysis\\
du [als Bauer]
\z
\z

Turning to the clausal counterpart \REF{ex:6:41b}, that is, \textit{ein} ALS \textit{Bauer}, recall from \chapref{sec:2} that \textit{ein} ‘a’ cannot have an ending in certain morpho-syntactic contexts when an adjective and/or a noun follows. This holds independently of the presence or absence of the modal particle. Compare \REF{ex:6:44a} to \REF[b-d]{ex:6:44}.

\ea%44
    \label{ex:6:44}
\ea   \label{ex:6:44a}
\gll Er ist vielleicht ein    (guter) Bauer!\\
he is  \textsc{prt}           a/one {\db}good   farmer.\textsc{masc}\\
\glt ‘He is some great farmer!’
\ex[*]{   \label{ex:6:44b}
\gll Er ist (vielleicht) ein-er     Gute(r)\\
he is    {\db}\textsc{prt}           a/one\textsc{-st} good\\
}
\ex[*]{\label{ex:6:44c}
\gll Er ist (vielleicht) ein-er     Bauer\\
he is    \textsc{prt}           a/one-\textsc{st} farmer.\textsc{masc}\\
}
\ex[*]{\label{ex:6:44d}
\gll Er ist (vielleicht) ein-er     gute(r) Bauer\\
he is    {\db}\textsc{prt}           a/one-\textsc{st} good    farmer.\textsc{masc}\\
}
\z
\z

Rather, inflected \textit{ein} is only possible here when the modifier is a relative clause (a genitive DP, or a PP) or when no relevant element follows at all \REF{ex:6:45a}. Thus, in order to explain the absence of the inflection on \textit{ein} in \REF{ex:6:41b}, I propose that similar to regular adjectives, ALS \textit{Bauer} is in a specifier. This is shown in simplified form in \REF{ex:6:45b}.

\ea%45
    \label{ex:6:45}
\ea \label{ex:6:45a}
\gll Er ist ein-er (, der  viel     arbeitet).\\
he is  one-\textsc{st}  {} who much works\\
\glt ‘He is one that works a lot.’
\ex   \label{ex:6:45b}
Specifier-type Analysis\\
    ein [ALS Bauer] e\textsubscript{N}
\z
\z

  To sum up thus far, we have seen some evidence that \textit{als Bauer} is most likely in an adjoined position and that ALS \textit{Bauer} is in a specifier position. Importantly, \citet[200]{deSwartEtAl2007} and others have noted that modifiers such as relative clauses and adjectives make optional determiners obligatory. This is particularly clear with role nouns \REF[a-b]{ex:6:46}, even if such a noun is not pronounced \REF{ex:6:46c}.

\ea%46
    \label{ex:6:46}
\ea \label{ex:6:46a}
\gll Er ist *(’n) Bauer,           der  viel    arbeitet.\\
he is       {\db\db\db}a   farmer.\textsc{masc} that much works\\
\glt ‘He is a farmer that works a lot.’
\ex \label{ex:6:46b}
\gll Er ist *(’n) guter Bauer.\\
he is       {\db\db\db}a   good farmer.\textsc{masc}\\
\glt ‘He is a good farmer.’
\ex \label{ex:6:46c}
\gll Peter ist ’n schlechter Bauer,          aber Hans ist *(’n) guter.\\
Peter is    {\db}a bad           farmer.\textsc{masc} but   Hans is       {\db\db\db}a   good\\
\glt ‘Peter is a bad farmer, but Hans is a good one.’
\z
\z

It is clear then that modifiers in both adjoined and specifier positions require the presene of \textit{ein}. I propose that \textit{ein} is present because the modifiers in \REF{ex:6:46} combine with their nouns by Predicate Modification. Notice that this operation conjoins elements of type \textlangle e,t\textrangle.  Now, for the noun to be of this type, it must have combined with REL first. As we know from above, this operator implies the presence of NumP and thus \textit{ein}.\footnote{For some discussion of why CAP cannot be involved here, see \chapref{sec:7}.} Now, assuming that nominals involving a noun and ALS (or \textit{als} ‘as’) are also predicate modifiers, I point out that REL is also present, and this explains the presence of \textit{ein} in \REF{ex:6:45b}.

I briefly return to the analysis of the first case, \textit{du als Bauer} in \REF{ex:6:43b}. Observe that this string involves adjunction just like \REF{ex:6:46a} above. However, unlike \REF{ex:6:46a}, there is no \textit{ein} in \REF{ex:6:43b}. In order to address this difference, recall the discussion from \sectref{sec:6.3.2.1}, where \textit{du} itself was proposed to involve a complex structure. Specifically, \textit{du} is a determiner with NumP and a null noun in its complement structure. I update \REF{ex:6:43b} as in \REF{ex:6:47a}. Notice now that this makes \REF{ex:6:47a} and \REF{ex:6:46a} completely parallel in that the adjoined material (in brackets) is preceded by a structurally complex nominal.\footnote{To the extent that \textit{ein} can be followed by \textit{als Bauer}, it must have an inflection: \textit{ein*(-er) als Bauer}. This fits well with the adjunction-type analysis of \textit{als Bauer} in \REF{ex:6:47a}.}

\ea%47
    \label{ex:6:47}
  \ea\label{ex:6:47a}
  {du} REL e\textsubscript{N} [als Bauer]
  \ex\label{ex:6:47b}
  {’n} REL Bauer [der viel arbeitet]
  \z
\z

To explain the absence of \textit{ein} in the first nominal in \REF{ex:6:47a}, I suggest that elements like pronominal determiners can also flag REL. This is compatible with \citegen{deSwartEtAl2007} assumptions about regular DPs like \textit{the child}. In other words, besides \textit{ein} and the definite article, other determiners can also indicate the presence of REL.

To sum up, \textit{du als Bauer} involves adjunction of \textit{als Bauer}, and \textit{ein} ALS \textit{Bauer} contains a complex specifier occupied by ALS \textit{Bauer}. The determiners \textit{du} and \textit{ein} are present due to REL. The absence of the inflection on \textit{ein} is explained by the presence of an overt element in the specifier just below \textit{ein}. With this much in place, I consider the relevant analyses in more detail. I start with the (neutral) capacity reading in the DP.

\subsubsection{Capacity reading in the DP}\label{sec:6.3.4.2}

I have proposed that \textit{du} itself involves a complex structure and that \textit{als Bauer} is adjoined to that structure as in \REF{ex:6:47a} above. I assume that this type of adjunction is instantiated by a Modifier Phrase (ModP; see \chapref{sec:2}, \sectref{sec:2.3.2} and \ref{sec:2.3.3}). I interpret \textit{als} as the head of ModP, and this head takes \textit{Bauer} as a complement. The head \textit{als} brings about the capacity reading of the DP. I derive the example under \figref{figex:6:48} as the figure above it.

\glltree[\label{figex:6:48}]{\small
	(Neutral) capacity reading licensed DP-internally\\
	\gll du   als Bauer\\
	you as  farmer.\textsc{masc}\\
	\glt \small{‘you as a farmer’\\
	\#‘you as a peasant’}
}{
	[ DP\textsubscript{\parbox{0mm}{\mbox{\textlangle e\textrangle}}}
		[\textit{du}\textsubscript{\parbox{0mm}{\mbox{\textlangle\textlangle e,t\textrangle e\textrangle}}}]
		[NumP\textsubscript{\parbox{0mm}{\mbox{\textlangle e,t\textrangle}}}
			[NumP\textsubscript{\parbox{0mm}{\mbox{\textlangle e,t\textrangle}}}, s sep=20mm
				[Num\textsubscript{\parbox{0mm}{\mbox{\textsc{rel}\textlangle e\textlangle e,t\textrangle\textrangle}}}]
				[NP\\\textit{e\textsubscript{N}}\textsubscript{\textlangle e\textrangle}]
			]
			[ModP\textsubscript{\parbox{0mm}{\mbox{\textlangle e,t\textrangle}}}, edge label={node[midway, above right]{~~~~(Predicate Modification)}}
				[\textit{als}]
				[NP\\\textit{Bauer}\textsubscript{\textlangle e\textrangle}]
			]
		]
	]
}

Observe that the lower NumP and ModP are both of type \textlangle e,t\textrangle\ (I comment on the role of \textit{als} below). I assume that they combine by Predicate Modification.

I point out that \textit{als-}nominals are special in that \textit{als} also combines with bare kind nouns such as \textit{Mann} ‘man’ in appropriate contexts \REF{ex:6:49a}. In fact, the presence of \textit{ein} leads to awkwardness \REF{ex:6:49b}.

\ea%49
    \label{ex:6:49}
  \ea \label{ex:6:49a}
  \gll Was  würdest du   als Mann        dazu sagen?\\
  what would    you as  man.\textsc{masc} it.to  say\\
  \glt ‘What would you as a male say about this?’
  \ex[?]{ \label{ex:6:49b}
  \gll Was  würdest du   als ’n Mann         dazu sagen?\\
  what would    you as   {\db}a  man.\textsc{masc} it.to  say\\
  \glt ‘What would you as a male say about this?’
  }
  \z
\z

Since a kind noun is involved here, CAP cannot be at work in the adjoined structure. Furthermore, if REL were at work here, we would expect that the presence of \textit{ein} is fully grammatical, contrary to fact. I assume that \textit{als} is a general(ized) capacity operator that maps role and kind nouns alike into predicate nominals; that is, \textit{als} is of type \textlangle e\textlangle e,t\textrangle\textrangle. I discuss this in more detail in \chapref{sec:7}. I turn to the data from the clausal domain.

\subsubsection{Capacity reading in the TP}\label{sec:6.3.4.3}

As illustrated many times, adjectives exhibit strong endings after uninflected \textit{ein}. In \chapref{sec:2}, I discussed two options to analyze this: Either the adjective is in the regular specifier position, and the strong ending follows from the special morpho-syntactic properties of \textit{ein}, or alternatively, the adjective is in a non-canonical position where its ending cannot undergo Impoverishment in the first place. As discussed in that chapter, one such non-canonical position involves the adjective to be deeply embedded in a complex specifier.

Recall now that the combination of “\textit{vielleicht} + \textit{ein} N” has two readings. I made use of the first option mentioned just above (the adjective is in its regular position) with the intensified comparative reading in \sectref{sec:6.3.3.3}. The second option (the adjective is more deeply embedded) was proposed for ALS-nominals in the preliminary \sectref{sec:6.3.4.1} (see \REF{ex:6:45b} above). I now turn to the second option in more detail.

  Specifically, I propose that cases with the emotive capacity reading have the noun (and added adjective) embedded in a specifier position as in \figref{figex:6:50}. As just mentioned, this complex specifier contains the null element ALS. I propose that similar to the DP, the embedded nominal involves ModP, and I assume that ALS is its head. Again, like the DP, Predicate Modification combines the two nominals, here ModP and Agr’.

\glltree[\label{figex:6:50}]{\small
	Emotive capacity reading licensed DP-externally\\
	\gll Du  bist vielleicht ’n  erstaunlicher Bauer!\\
	you are  \textsc{prt}           {\db}an amazing         farmer.\textsc{masc}\\
	\glt \small{‘You are really an amazing farmer!’}
}{
	[CardP
		[\textit{’n}]
		[AgrP\textsubscript{\parbox{0mm}{\mbox{\textlangle e,t\textrangle}}}
			[ModP\textsubscript{\parbox{0mm}{\mbox{\textlangle e,t\textrangle}}}
				[ALS]
				[AgrP\textsubscript{\parbox{0mm}{\mbox{\textlangle e,t\textrangle}}}\\{[\textit{erstaunlicher} REL \textit{Bauer}]}]
			]
			[Agr$'$\textsubscript{\parbox{0mm}{\mbox{\textlangle e,t\textrangle}}}, edge label={node[midway, above right]{~~(Predicate Modification)}}
				[ NumP\textsubscript{\parbox{0mm}{\mbox{\textsc{rel}}}} [NP\\\textit{e\textsubscript{N}}\textsubscript{\textlangle e\textrangle}]]
			]
		]
	]
}

Recall that ALS is licensed by the modal particle \textit{vielleicht}. Furthermore, \textit{ein} is present due to REL in the matrix nominal, and \textit{ein} is uninflected due to the presence of an overt element in the specifier of the higher AgrP.\footnote{Note that the instances with an adjective involve REL in the embedded nominal as in \figref{figex:6:50}, but I assume that cases without an adjective also involve REL. In other words, ALS always involves REL and thus NumP as part of its complement. Observe in this regard that these cases involve concord in agreement features between \textit{ein} in the matrix nominal and the (adjective and) noun embedded in Spec,AgrP. I take it that the head Mod (ALS) mediates the agreement between these two nominals (also \chapref{sec:8}, \sectref{sec:8.3.1}).} Finally, I assume that the emotive flavor originates with the modal particle.

  For clarity, I review the capacity reading in the nominal and clausal domains adding a few more details in the process. There are two differences between the DP and the TP. First, the DP has a neutral interpretation, but the TP has an emotive one. This difference in emotiveness follows from the need to be restrictive enough in the former case (namely when the speaker singles someone out in a certain capacity, see \sectref{sec:6.2.2.1}) and the presence of the modal particle in the latter case.

Second, the indefinite determiner is absent in the DP (\textit{du als Bauer}) but present in the TP (\textit{ein} ALS \textit{erstaunlicher Bauer e\textsubscript{N}}). I consider two structural domains here, delineated by square brackets in the following examples where the embedded nominal is enclosed by the brackets. In the matrix nominal, \textit{ein} is absent in \textit{du} [\textit{als Bauer}] as \textit{du} itself flags REL of the matrix nominal; \textit{ein} is present in \textit{ein} [\textit{erstaunlicher Bauer}] \textit{e\textsubscript{N}} as it flags REL. In the embedded nominal, \textit{ein} is absent in [\textit{als Bauer}] since \textit{als} is a general(ized) capacity operator that takes role and kind nouns as arguments. As to the absence of \textit{ein} in [\textit{erstaunlicher Bauer}], I assume that this is due to haplology with \textit{ein} in the matrix nominal.

To sum up this section, I derived the three readings established at the beginning of this chapter. In the DP, the ordinary and comparative readings involve canonical, simple DPs, but the capacity reading manifests an \textit{als-}nominal in an adjoined position yielding a non-canonical nominal. As for the TP, the first two readings exhibit regular copular structures, but the capacity reading involves a modal particle licensing an unpronounced ALS. The latter is part of a complex specifier, and this specifier also brings about a non-canonical nominal.

\section{Conclusion}\label{sec:6.4}

This chapter discussed the first consequence of the proposal laid out \chapref{sec:5} – \textit{ein} in the context of emotive constructions. I started with the observation that the DP and the TP are parallel in structure and interpretation. Making this my general heuristic methodology, I extended the discussion to pronominal DPs and copular TPs containing non-theta nouns. Establishing three basic readings (cf.  \citealt{denDikken2006}), I provided a more detailed investigation of how \textit{ein} combines with certain role and kind nouns and how these combinations fare with regard to emotiveness in the nominal and clausal domains.

In order to explain the commonalities and slight differences, I employed, with a few refinements, \citegen{deSwartEtAl2007} account of the clause and extended it to pronominal DPs. In order to explain the whole range of data, the explanation of the DP had to include some pragmatic considerations \citep{Rauh2004}. In fact, the interaction between these two proposals also explained the absence of true minimal pairs of pronominal DPs and copular TPs involving bare predicate nouns. More importantly, I showed that \textit{ein} is not responsible for the emotiveness property. Rather, this property is due to REL in combination with a [+figurative] noun. It was argued that \textit{ein} indicates the presence of structure on top of NP (Hypothesis 3a) and the presence of REL (Hypothesis 3b). In other words, I can maintain the claim that \textit{ein} is semantically vacuous (Hypothesis 1a).

So far, I only discussed cases where we find morphological agreement between the pronoun and the predicate noun. Next I turn to cases of – what looks like – dis-agreement in number.
