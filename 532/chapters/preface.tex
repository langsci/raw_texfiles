\addchap{\lsPrefaceTitle}
 
Anyone teaching German as a foreign language knows that mastering the adjectival endings and \textit{ein}{}-words is particularly challenging for students. I became interested in these grammatical topics over 25 years ago when I started working as a Teaching Assistant in the United States. I have been fascinated by these phenomena ever since. A couple of years ago, this interest was strongly piqued again by the phrase \textit{Eine Störche!} uttered by my dad during a hike. With my family interested in language, a discussion followed where we agreed that this phrase could be loosely rendered as ‘Wow! So many storks!’ However, my family is, to this day, surprised that a singular article like \textit{ein} ‘a’ can combine with a plural noun. Wow, indeed! This book attempts to shed some light on this and other issues. Beside the cases frequently discussed in the literature, here I also engage in the discussion of many non-canonical instances. In particular, I focus on various constructions involving \textit{ein}{}-words and adjectival inflections.

Thinking back, the chapter on \textit{ein}{}-words was the first to be written for my dissertation \citep{Roehrs2006a}. However, it has taken the longest to present it in a more polished form. Although the basic idea is still the same, many details, empirical and analytical, have been modified or added. In the meantime, adjectival inflections have become a second major point of interest for me. Again, while the basic system was already laid out in \citet[Chapter 4]{Roehrs2009a}, more “exotic” cases are discussed in the following pages. I believe it is these new details about \textit{ein}{}-words and adjectival inflections that reveal the true nature of these elements allowing us to formulate some new and interesting hypotheses.

It is my hope that due to its richness of data and analyses, many of which are new, this book will be of interest to anyone working on semantically vacuous elements, in general, and on the syntax and semantics of the DP, in particular. With the main focus on \textit{ein}{}-words and adjectival inflections in German, the book will not be the final word on these topics as the investigation of other languages will surely reveal more interesting facts and lead to other theoretical insights. However, I hope that this book will make a contribution toward the description and explanation of these parts of the grammar, topics I believe have not received due attention.

  The material in this book was presented at many linguistic colloquia and conferences, too many to detail here. I thank the respective audiences for questions and comments, especially, Marcel den Dikken, David Fertig, Volker Gast, Hubert Haider, Tracy Hall, Tom Leu, Mark Louden, Joe Salmons, Chris Sapp, Erik Schoorlemmer, and Laura Smith. A very special thanks goes to Björn Köhnlein, who contributed a number of very interesting points. I would also like to express gratitude to Jochen Trommer and Bernd Wiese for helping me find some of their work. Furthermore, I am indebted to many reviewers whose comments over the years have helped me shape the ideas presented here. In particular, I am very grateful to the two anonymous reviewers of this book, who provided many constructive comments, data points, and references, and to my editor Mike Putnam for many helpful suggestions. Finally, note that parts of \chapref{sec:5} are based on \citet[Chapter 5, Part II]{Roehrs2006a}. \chapref{sec:7} was supported by a Faculty Research Grant (G34217) from the University of North Texas, which I hereby gratefully acknowledge. The first version of this book was thoroughly revised during my Faculty Development Leave in Fall 2020, and many other changes were made after that.

I dedicate this book to my late dad, who encouraged me to look for commonalities, to my mom, who emphasized I should write clearly, to my brother, who reminded me to stop and smell the roses, and to my partner in crime, who makes life just beautiful. \textit{Danke, Siggi, Rosen, Keule and Xixi!}

\vspace{1cm}

Denton,  {November 2024}

Dorian Roehrs
