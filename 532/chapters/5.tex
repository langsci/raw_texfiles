\chapter{{\textit{Ein}}{-words and adjectival} {\textit{eine}}}\label{sec:5}

\section{Introduction}\label{sec:5.1}

The previous three chapters detailed the current account of the various properties of adjectival inflections. I now turn to \textit{ein}-words, which exhibit related properties. Note though that besides this general relatedness of the two types of elements, adjectival endings will continue to play an important part in the following discussion as they help narrow down the choices when proposing plausible structures. With that in mind, I turn to the main hypotheses about \textit{ein}, formulated in \chapref{sec:1} and repeated here for convenience. Recall that \REF{ex:5:1b} can be fleshed out as in \REF{ex:5:2a}.

\ea%1
    \label{ex:5:1}

          Hypothesis 1

Adjectival inflections and \textit{ein}:

\ea \label{ex:5:1a}    are expletive elements and

\ex\label{ex:5:1b}     indicate abstract structure in the noun phrase.
\z
\z


\ea%2
    \label{ex:5:2}

        Hypothesis 3

  \textit{Ein}:
\ea \label{ex:5:2a}  indicates abstract structure in the lower layers of the noun phrase (NP vs. ArtP),       and
\ex \label{ex:5:2b}  it supports overt semantic operators (e.g., NEG \textit{k}-) and flags the presence of       covert semantic operators (e.g., REL).
\z
\z

As already stated in \chapref{sec:1}, the difference between supporting and flagging the operator by \textit{ein} is that the presence of the operator itself is in some way detectable in the first case but not in the second. The hypotheses in \REF{ex:5:1} and \REF{ex:5:2} are addressed in detail in this and the next two chapters. I start with some general considerations and some basic data involving \textit{ein}.

\subsection{Preliminaries and basic data} \label{sec:5.1.1}

As amply illustrated above, German noun phrases have determiners. For instance, depending on the context, a singular noun may occur with the appropriate form of the indefinite article. The latter is often reduced in speech \REF{ex:5:3a}. This reduction of the stem from \textit{ein-} to \textit{n-} is marked by parentheses or apostrophe below.\footnote{I distinguish different kinds of \textit{ein} below. Parentheses indicate the optionality of the vocalic part of the stem provided that the element is not stressed (which is the case with the indefinite article \textit{ein}, but not with the singularity numeral \textit{EIN} or adjectival \textit{eine}). In some cases, I use an apostrophe to clearly indicate the (reduced) indefinite article. In the latter cases, the unreduced form of the article would lead to a difference in meaning and/or grammaticality.} Alternatively, such a noun can surface with the definite article or its almost-homophonous demonstrative counterpart \REF{ex:5:3b} (the ambiguous status of \textit{die} as the article ‘the’ or the demonstrative ‘that’ is indicated as DET(erminer) in the gloss).

\ea%3
    \label{ex:5:3}
\ea \label{ex:5:3a}
\gll (ei)ne Freundin\\
 {\db}a       girlfriend.\textsc{fem}\\
\glt ‘a girlfriend’
\ex \label{ex:5:3b}
\gll die  Freundin\\
\textsc{det} girlfriend.\textsc{fem}\\
\glt ‘the/that girlfriend’
\z
\z

As is well known, indefinite and definite determiners cannot co-occur. This applies to both reduced and unreduced indefinite articles as well as to definite articles and demonstratives (for \REF{ex:5:5b} below, see \citealt{Pafel1994}: 251; I discuss the unreduced form of \textit{eine} in contexts like \REF{ex:5:5b} below).

\ea%4
    \label{ex:5:4}
\ea[*]{ \label{ex:5:4a}
\gll  (ei)ne (ei)ne Freundin\\
 {\db}an      a        girlfriend.\textsc{fem}\\
 }
\ex[*]{\label{ex:5:4b}
\gll (ei)ne die  Freundin\\
 {\db}an     \textsc{det} girlfriend.\textsc{fem}\\
 }
\z
\z

\ea%5
    \label{ex:5:5}
\ea[*]{ \label{ex:5:5a}
\gll die  die   Freundin\\
\textsc{det} \textsc{det} girlfriend.\textsc{fem}\\
}
\ex[*]{\label{ex:5:5b}
\gll die ’ne Freundin\\
\textsc{det} {\db}a   girlfriend.\textsc{fem}\\
}
\z
\z

The same distributional restrictions hold if \textit{die} is replaced by the proximal demonstrative \textit{diese} ‘this’.

Casting the empirical net wider, it is not surprising that the negation particle \textit{nicht} ‘not’ can precede determiners \REF{ex:5:6}. However, I point out that possessives like \textit{mein} ‘my’ and the negator \textit{kein} ‘no’ are in complementary distribution with the indefinite article, the definite article, and the related demonstrative; consider \REF{ex:5:7} and \REF{ex:5:8}.

\ea%6
    \label{ex:5:6}
\ea\label{ex:5:6a}
\gll nicht ’ne Freundin\\
  not     {\db}a   girlfriend.\textsc{fem}\\
\glt ‘not a  girlfriend’
\ex\label{ex:5:6b}
\gll nicht die  Freundin\\
  not    \textsc{det} girlfriend.\textsc{fem}\\
\glt ‘not the/that girlfriend’
\z
\z

\ea%7
    \label{ex:5:7}
\ea[*]{\label{ex:5:7a}
\gll  meine ’ne Freundin\\
  my       {\db}a   girlfriend.\textsc{fem}\\
  }
\ex[*]{\label{ex:5:7b}
\gll meine die  Freundin\\
  my     \textsc{det} girlfriend.\textsc{fem}\\
  }
\z
\z

\ea%8
    \label{ex:5:8}
\ea[*]{  \label{ex:5:8a}
\gll   keine ’ne Freundin\\
  no       {\db}a   girlfriend.\textsc{fem}\\
  }
\ex[*]{  \label{ex:5:8b}
\gll  keine die  Freundin\\
  no     \textsc{det} girlfriend.\textsc{fem}\\
  }
\z
\z

The same holds if the determiner precedes the possessive or negator in \REF{ex:5:7} and \REF{ex:5:8}. In view of this mutually exclusive distribution, I point out again that the possessive pronominal and the negator are labeled possessive article and negative article, respectively. These names foreshadow the analysis to be developed below.

Interestingly, the distribution is partially different when \textit{ein} is stressed and must occur in its unreduced form (stress is indicated here with capital letters). Similar to \REF{ex:5:6a}, the negation particle \textit{nicht} is compatible with \textit{eine} \REF{ex:5:9a}. Unlike \REF{ex:5:7a}, \textit{ein} can, when stressed and unreduced, co-occur with the possessive article \REF{ex:5:9b}. Importantly, this element is still impossible with the negative article or the indefinite article \REF[c-d]{ex:5:9} (\REF[c]{ex:5:9} is adopted from \citealt{Fanselow1988}: 111 fn. 29).\footnote{Note already here that the two instances of \textit{ein} in \REF{ex:5:9a} and \REF{ex:5:9b} are not the same: \REF{ex:5:9a} involves the singularity numeral \textit{EIN}, but \REF{ex:5:9b} contains adjectival \textit{eine}. One of the differences is that \REF{ex:5:9b} has an additional interpretation that involves a duality presupposition roughly translated as ‘one of my two girlfriends’. The differences including the various interpretations are discussed below.}

\ea%9
    \label{ex:5:9}
\ea\label{ex:5:9a}
\gll nicht EINE Freundin\\
  not    one     girlfriend.\textsc{fem}\\
\glt ‘not one girlfriend’

\ex\label{ex:5:9b}
\gll meine EINE Freundin\\
  my      one    girlfriend.\textsc{fem}\\
\glt ‘my one girlfriend’

\ex[*]{\label{ex:5:9c}
\gll  keine EINE Freundin\\
  no      one    girlfriend.\textsc{fem}\\
  }
\ex[*]{\label{ex:5:9d}
\gll (ei)ne EINE Freundin\\
   {\db}an     one    girlfriend.\textsc{fem}\\
   }
\z
\z

It seems clear that the difference between the determiner elements in \REF{ex:5:9b} and \REF[c-d]{ex:5:9} relates to definiteness: The possessive in \REF{ex:5:9b} is definite, but the negator and the indefinite article in \REF[c-d]{ex:5:9} are not. Thus, besides stress and phonetic non-reducibility, definiteness seems to be a relevant factor in the distribution of \textit{ein}. In other words, any account that seeks to be on the right track needs to take into account the different stress and reduction patterns of \textit{ein} and the syntactic-semantic context this article occurs in.

  It is apparent from the previous literature (see \sectref{sec:5.7}) that there is relatively little discussion about the morpho-syntax and semantics of all the different kinds of \textit{ein}.\footnote{To anticipate the discussion in \sectref{sec:5.7}, it will become clear there that when we consider some of the previous accounts, we cannot help but notice that opinions diverge considerably with regard to the nature of this kind of indefinite element. Furthermore, none of these proposals discusses all the different kinds of \textit{ein}.}  As with adjectival inflections, only the canonical cases are usually discussed. In what follows, I seek to provide a more comprehensive overview of this type of element in German (for some brief cross-linguistic discussion, see again \chapref{sec:1}, \sectref{sec:1.2.1.3}). Similar to adjectival inflections, I propose that the discussion of the non-canonical cases reveals the true nature of \textit{ein}.

\subsection{Initial taxonomy of \textit{ein}}\label{sec:5.1.2}

I assume the following initial classification. This taxonomy contains three main types of \textit{ein} (\citealt{KarnowskiPafel2004}: 174-75) and some subtypes. Below, I propose in detail that certain elements are composite forms consisting of the article and another component.
\newpage
\ea%10
    \label{ex:5:10}
\ea\label{ex:5:10a}  \textit{ein} as an article:
\begin{itemize}
	\item indefinite article
	\item vacuous article:
	\begin{itemize}
		\item \textit{ein} as part of a composite:
		\begin{itemize}
			\item possessive
			\item negation
		\end{itemize}
		\item \textit{ein} in predicate noun phrases
	\end{itemize}
	\item \textit{ein} as part of a complex determiner
\end{itemize}

\ex\label{ex:5:10b} \textit{ein} as a numeral
\ex\label{ex:5:10c}\textit{ein} as an adjective
\z
\z

I refer to the types of \textit{ein} in \REF{ex:5:10a} and \REF{ex:5:10b} collectively as \textit{ein}-words; the type in \REF{ex:5:10c} is called adjectival \textit{eine}. In fact, reducing the numeral to the article, I propose below that there are just two types of \textit{ein}: the article and the adjective. Thus, the two corresponding designations, \textit{ein}-word and adjectival \textit{eine}, reflect the analysis to be developed below. Before I go into any specifics, consider first some illustrative examples of the above taxonomy. Note already here that while \textit{ein}-words have no ending in \REF[a-f]{ex:5:11}, adjectival \textit{eine} has a weak ending in \REF{ex:5:11g}.\footnote{The German noun \textit{Brot} ‘bread’ is countable. This count reading is translated into English using the classifier \textit{loaf} (e.g., \textit{zwei Brote} ‘(two breads =) two loaves of bread’). Also, I often use the [-figurative] noun \textit{Lehrer} ‘teacher’ in predicative contexts. As a term for a profession, it takes an optional indefinite article in such contexts (\chapref{sec:6}). This optionality is crucial for formulating hypotheses about the true nature of \textit{ein}.}

\ea%11
    \label{ex:5:11}
\ea \label{ex:5:11a}  Indefinite Article

\gll Ich habe immer  nur (ei)n frisch-es Brot mitgebracht.\\
I     have always only {\db}a     fresh-\textsc{st}  bread.\textsc{neut} brought\\
\glt ‘I have always brought only a fresh loaf of bread.’
\ex \label{ex:5:11b}  Possessive

\gll Ich habe immer  nur  mein frisch-es Brot mitgebracht.\\
I     have always only my   fresh-\textsc{st}  bread.\textsc{neut} brought\\
\glt ‘I have always brought only my fresh loaf of bread.’
\ex \label{ex:5:11c}  Negation

\gll Ich habe kein frisch-es Brot mitgebracht.\\
I     have no   fresh-\textsc{st}  bread.\textsc{neut} brought\\
\glt ‘I have brought no fresh loaf of bread.’
\ex \label{ex:5:11d}   Predicative

\gll Mein Vater  ist (ein) Lehrer.\\
my     father is   {\db}a      teacher.\textsc{masc}\\
\glt ‘My father is a teacher.’

\ex \label{ex:5:11e} Complex Determiner

\gll     Ich habe ein jedes  frisch-es Brot            mitgebracht.\\
   I    have  an  every fresh-\textsc{st}  bread.\textsc{neut} brought\\
\glt ‘I have brought each fresh loaf of bread.’
\ex  \label{ex:5:11f} Numeral

\gll Ich habe immer  nur  EIN frisch-es Brot mitgebracht.\\
I     have always only one fresh-\textsc{st}  bread.\textsc{neut} brought\\
\glt ‘I have always brought only one fresh loaf of bread.’
\ex  \label{ex:5:11g} Adjective

\gll Ich habe nur  das ein-e     frisch-e   Brot             mitgebracht.\\
I     have only the one-\textsc{wk} fresh-\textsc{wk} bread.\textsc{neut} brought\\
\glt ‘I have only brought the one fresh loaf of bread.’
\z
\z

These different kinds of contexts \textit{ein} appears in will be discussed at length in what follows. The discussion of the current and the next two chapters can be anticipated as follows.

\subsection{Outlook}\label{sec:5.1.3}

One goal of this chapter is to provide a more comprehensive survey of the different types of \textit{ein} with the intention of ultimately reducing them in number. I propose that there are two lexical types of \textit{ein}: the article and the adjective (cf. \citealt{Pafel2005}: 179). The second goal is to determine the properties of the different types of \textit{ein}. I argue that the article is semantically vacuous (Hypothesis 1a) but that adjectival \textit{eine} is not. In other words, the article is, in a number of ways, similar to adjectival inflections discussed in \chapref{sec:2}.

In more detail, I propose that the possessive articles, the negative article, and the singularity numeral are composite forms consisting of the indefinite article and another component. Providing some brief remarks on the possessive and negative composites, I dedicate most of the discussion to the article, the numeral, and adjectival \textit{eine}. Some distinctions between the last three elements are argued to follow from their different featural specifications and others from their different positions in the syntactic tree. Specifically, deriving the numeral from the combination of the indefinite article and another component, I account for the identical morphology but the different semantics of the numeral and article. Thus, I propose that the numeral is related to the article in a way that adjectival \textit{eine} is not – the latter is a separate lexical element. All three elements are argued to be in different positions.

As to the remaining types in \REF{ex:5:11} above, predicative \textit{ein} is illustrated below but discussed in greater detail in \chapref{sec:6}. I do not address the complex determiner \textit{ein jeder} ‘(an every =) each’ in much detail here (but see \citealt{Roehrs2012}, \citealt{Zimmermann2011}).\footnote{\label{foot:5:5} Note that \textit{ein} here occurs in the context of the semantically plural and definite element \textit{jeder} ‘every’. It is proposed in \citet{Roehrs2012} that \textit{ein} intensifies the distributivity of \textit{jeder}. I tentatively suggest in \chapref{sec:8} that \textit{ein} may flag the presence of a null distributivity operator (cf. \citealt{deSwartEtAl2007}: 218). This is consistent with the proposal that \textit{ein} is semantically vacuous, including in the string \textit{ein jeder}.} More generally, the current chapter discusses the morpho-syntax of \textit{ein} and its relation to indefiniteness, and the next two chapters focus on \textit{ein} in relation to semantic notions like emotiveness (\chapref{sec:6}) and number (\chapref{sec:7}).

The current chapter is organized as follows. In order to motivate the approach that some types of \textit{ein} should be treated in the same way, I first illustrate certain morphological similarities between the different kinds of \textit{ein}. In \sectref{sec:5.3}, some phonetic and semantic differences are pointed out. These differences are summarized in \tabref{tab:5:1} below. A bipartite proposal involving different feature specifications and different syntactic positions is discussed in \sectref{sec:5.4} and \ref{sec:5.5}, respectively. \sectref{sec:5.6} provides some diachronic and cross-linguistic evidence that \textit{ein}-words are indeed composite forms in (Modern) German. In \sectref{sec:5.7}, I consider some previous accounts briefly critiquing one of them in more detail, and \sectref{sec:5.8} forms the conclusion.

\section{Similarities}\label{sec:5.2}
\largerpage[-2]
In \sectref{sec:5.1.2}, I provided my taxonomy along with an illustration of the basic cases. With these reference points in mind, I discuss the occurrence of the different inflections on \textit{ein} in split topicalizations and with elided nouns. Note that these constructions require special contexts to be felicitous (see again \chapref{sec:4}, \sectref{sec:4.3}).

\subsection{Split topicalization}\label{sec:5.2.1}

Comparing \REF{ex:5:11} to \REF{ex:5:12}, split topicalizations with in-situ adjectives have the same morphology as non-split noun phrases; that is, \textit{ein}-words have no endings in \REF[a-e]{ex:5:12}, but adjectival \textit{eine} has a weak inflection in \REF{ex:5:12f}.\footnote{To investigate the relevant aspects of the morpho-syntax of predicative \textit{ein} in split topicalizations, I chose a non-canonical construction involving the modal particle \textit{vielleicht} ‘really’. This element makes the split of the predicative construction more felicitous. Note also that the translations of the following examples into English are not always straightforward – they are often approximations of the German originals.}

\ea%12
    \label{ex:5:12}
\ea \label{ex:5:12a}  Indefinite Article

\gll Brot             habe ich immer  nur  (ei)n  frisch-es mitgebracht.\\
bread.\textsc{neut} have I     always only  {\db}a      fresh-\textsc{st}  brought\\
\glt ‘As for bread, I have always brought only a fresh loaf.’
\ex  \label{ex:5:12b} Possessive

\gll Brot             habe ich immer nur   mein  frisch-es mitgebracht.\\
bread.\textsc{neut} have I     always only my    fresh-\textsc{st}  brought\\
\glt ‘As for bread, I have always brought only my fresh loaf.’
\ex  \label{ex:5:12c} Negation

\gll Brot            habe ich kein frisch-es mitgebracht.\\
bread.\textsc{neut} have I     no   fresh-\textsc{st}  brought\\
\glt ‘As for bread, I have brought no fresh loaf.’
\ex   \label{ex:5:12d}Numeral

\gll Brot             habe ich immer  nur  EIN frisch-es mitgebracht.\\
bread.\textsc{neut} have I     always only one fresh-\textsc{st}  brought\\
\glt ‘As for bread, I have always brought only one fresh loaf.’

\ex \label{ex:5:12e} Predicative

\gll  Brot             ist das vielleicht (ei)n frisch-es!\\
bread.\textsc{neut} is  that \textsc{prt}   {\db}a     fresh-\textsc{st}\\
\glt ‘As for bread, this is really a fresh loaf.’
\ex  \label{ex:5:12f} Adjective

\gll Brot             habe ich nur   das ein-e     frisch-e   mitgebracht.\\
bread.\textsc{neut} have I     only the  one-\textsc{wk} fresh-\textsc{wk} brought\\
\glt ‘As for bread, I have only brought the one fresh loaf.’
\z
\z

\subsection{Split topicalization with fronted adjective}\label{sec:5.2.2}

If the adjective is part of the split-off, the \textit{ein}-word exhibits a strong ending \REF[a-e]{ex:5:13}. Interestingly, the stranded \textit{ein}-words have an optional schwa (see also \citealt{Roehrs2009a}: 156, \citealt{Sternefeld2008}: 152), where the option with schwa seems to be, at least to my ears, of elevated style. With a determiner preceding, there is no change for \textit{eine} in \REF{ex:5:13f}.\footnote{When
  \textit{ein} is stranded by itself \REF{ex:5:13a}, it is actually stressed. This makes the indefinite article ambiguous with the numeral in \REF{ex:5:13d}. Note that even if another stressed element such as \textit{so} ‘such’ is added, the unreduced form of \textit{ein} is still much better here.

  \ea \label{ex:5:7:i}
  \gll Brot habe ich immer  nur so   {\{ein(e)s / *?’ns / *’nes\}} mitgebracht.\\
  bread.\textsc{neut} have I     always only such {\db}a                                   brought\\
  \glt ‘As for bread, I have always brought only this kind of loaf.’
  \z
  
  Notice that \textit{ein} in \REF{ex:5:7:i}  is used pronominally and as such, it has some stress on it. This presumably explains the degradedness of the reduced forms. It is not clear to me though if this instance of \textit{ein} is the unreduced article or the numeral (which consists of the article and an additional component).
}

\ea%13
    \label{ex:5:13}
\ea \label{ex:5:13a}  Indefinite Article

\gll (Frisch-es) Brot            habe ich immer  nur  ein-(e)s mitgebracht.\\
 {\db}fresh-\textsc{st}    bread.\textsc{neut} have I     always only a-\textsc{st}      brought\\
\glt ‘As for (fresh) bread, I have always brought only a loaf.’
\ex  \label{ex:5:13b} Possessive

\gll (Frisch-es) Brot             habe ich immer  nur  mein-(e)s mitgebracht.\\
 {\db}fresh-\textsc{st}     bread.\textsc{neut} have I     always only my\textsc{-st}     brought\\
\glt ‘As for (fresh) bread, I have always brought only my loaf.’
\ex  \label{ex:5:13c} Negation

\gll (Frisch-es) Brot             habe ich kein-(e)s mitgebracht.\\
 {\db}fresh-\textsc{st}     bread.\textsc{neut} have I     no-\textsc{st}      brought\\
\glt ‘As for (fresh) bread, I have brought no loaf.’
\ex  \label{ex:5:13d} Numeral

\gll (Frisch-es) Brot             habe ich immer nur  EIN-(E)S mitgebracht.\\
 {\db}fresh-\textsc{st}     bread.\textsc{neut} have I    always only one-\textsc{st}     brought\\
\glt ‘As for (fresh) bread, I have always brought only one loaf.’

\ex \label{ex:5:13e} Predicative

\gll (Frisch-es) Brot             ist das vielleicht ein-(e)s!\\
     {\db}fresh-\textsc{st}     bread.\textsc{neut} is  that \textsc{prt}          one-\textsc{st}\\
\glt ‘As for (fresh) bread, this is really one [fresh] loaf.’
\ex \label{ex:5:13f}  Adjective

\gll (Frisch-es) Brot             habe ich nur  das ein-e     mitgebracht.\\
 {\db}fresh-\textsc{st}     bread.\textsc{neut} have I    only the  one-\textsc{wk} brought\\
\glt ‘As for (fresh) bread, I have only brought the one loaf.’
\z
\z

Comparing the inflections on the adjectives in \sectref{sec:5.2.1} to the ones on the \textit{ein}-words here, we can observe again that they are the same. In keeping with \chapref{sec:4}, \sectref{sec:4.5}, I assume that these inflections make nominal features like case, number, and gender visible (Hypothesis 2b). Recall also that the emergence of the strong ending on the \textit{ein}-words is due to a certain vocabulary insertion rule (\chapref{sec:2}, \sectref{sec:2.2.2}). Parallel facts hold when the noun is elided.

\subsection{Adjective followed by elided noun}\label{sec:5.2.3}

When adjectives are present, noun phrases with elided nouns have the same morphology as those with non-elided nouns (\chapref{sec:2}); that is, the \textit{ein}-words have no endings in \REF[a-e]{ex:5:14}, but adjectival \textit{eine} has a weak inflection in \REF{ex:5:14f}.

\ea%14
    \label{ex:5:14}
\ea  \label{ex:5:14a}  Indefinite Article

\gll Ich habe immer  nur (ei)n frisch-es mitgebracht.\\
I     have always only {\db}a     fresh-\textsc{st}  brought\\
\glt ‘I have always brought only a fresh loaf.’
\ex   \label{ex:5:14b} Possessive

\gll Ich habe immer  nur  mein frisch-es mitgebracht.\\
I     have always only my   fresh-\textsc{st}  brought\\
\glt ‘I have always brought only my fresh loaf.’
\ex   \label{ex:5:14c} Negation

\gll Ich habe kein frisch-es mitgebracht.\\
I     have no    fresh-\textsc{st}  brought\\
\glt ‘I have brought no fresh loaf.’
\ex   \label{ex:5:14d} Numeral

\gll Ich habe immer  nur  EIN frisch-es mitgebracht.\\
I     have always only one fresh-\textsc{st}  brought\\
\glt ‘I have always brought only one fresh loaf.’

\ex  \label{ex:5:14e} Predicative

\gll Das ist vielleicht (ei)n frisch-es!\\
that is  \textsc{prt}           {\db}a     fresh-\textsc{st}\\
\glt ‘This is really a fresh loaf.’
\ex  \label{ex:5:14f}  Adjective

\gll Ich habe nur   das ein-e      frisch-e   mitgebracht.\\
I     have only the  one-\textsc{wk} fresh-\textsc{wk} brought\\
\glt ‘I have only brought the one fresh loaf.’
\z
\z

Finally, consider topicalization where the split-off contains an adjective and an elided noun.

\subsection{Split topicalization with fronted adjective and elided noun}\label{sec:5.2.4}

If the noun phrases are split and the adjectives and elided nouns are part of the split-offs, then similar to \sectref{sec:5.2.2}, the \textit{ein}-words have strong endings \REF[a-e]{ex:5:15}. Again, there is no change for \textit{eine} in \REF{ex:5:15f}.

\ea%15
    \label{ex:5:15}
\ea  \label{ex:5:15a} Indefinite Article

\gll Frisch-es habe ich immer  nur  ein-(e)s mitgebracht.\\
fresh-\textsc{st}   have I     always only a-\textsc{st}     brought\\
\glt ‘As for a fresh one, I have always brought only a loaf.’
\ex \label{ex:5:15b}  Possessive

\gll Frisch-es habe ich immer  nur  mein-(e)s mitgebracht.\\
fresh-\textsc{st}   have I     always only my-\textsc{st}     brought\\
\glt ‘As for a fresh one, I have always brought only my loaf.’
\ex  \label{ex:5:15c} Negation

\gll Frisch-es habe ich kein-(e)s mitgebracht.\\
fresh-\textsc{st}   have I     no-\textsc{st}      brought\\
\glt ‘As for a fresh one, I have brought no loaf.’
\ex  \label{ex:5:15d} Numeral

\gll Frisch-es habe ich immer  nur  EIN-(E)S mitgebracht.\\
fresh-\textsc{st}   have I     always only one-\textsc{st}     brought\\
\glt ‘As for a fresh one, I have always brought only one loaf.’

\ex \label{ex:5:15e} Predicative

\gll  Frisch-es ist das vielleicht ein-(e)s!\\
fresh-\textsc{st}   is  that \textsc{prt}          one-\textsc{st}\\
\glt ‘As for a fresh one, this is really one [fresh] loaf.’
\ex  \label{ex:5:15f} Adjective

\gll Frisch-es habe ich nur  das ein-e     mitgebracht.\\
fresh-\textsc{st}   have I     only the one-\textsc{wk} brought\\
\glt ‘As for a fresh one, I have only brought the one loaf.’
\z
\z

To summarize these sections, all \textit{ein}-words are marked by the emergence of strong endings when they are “stranded” by themselves in split topicalizations. Adjectival \textit{eine} is different – it remains unchanged. Similar facts of the \textit{ein}-words were observed when the latter occur in elliptical contexts; that is, when the adjective and the noun are not present at all (\chapref{sec:2}, \sectref{sec:2.2.2.2}). The emergence of the strong inflections on the \textit{ein}-words follows from the same vocabulary insertion rule stated in \sectref{sec:2.2.2.2}: It applies to \textit{ein}-words, independently of whether \textit{ein} is “stranded” by split topicalization or occurs in elliptical contexts – in each case, there is no overt material following \textit{ein}.\footnote{Recall from \chapref{sec:4}, \sectref{sec:4.3} that \textit{ein} in split topicalizations is not actutally stranded by the adjective and/or noun moving out of the source DP. Rather, the adjective and/or noun are part of a separate nominal, the split-off. As such, on my assumptions, both split topicalization and ellipsis involve null nouns in the nominal containing \textit{ein}.}

\section{Differences}\label{sec:5.3}

In this section, I focus on three phonetic and semantic differences: encliticization, stressability, and semantic singularity (see also \citealt{KarnowskiPafel2004}: 174-75, \citealt{Pafel2005}: 180). These differences are summarized in tabular form at the end of this section.

\subsection{Encliticization}\label{sec:5.3.1}

As seen above, reduced forms of the indefinite article are independently possible. Thus, I take encliticization of \textit{ein} to be instantiated when the indefinite article undergoes further phonetic changes due to its preceding element, its host. For instance, coronal \textit{’n} as the reduced form of \textit{ein} becomes \textit{’m} when it is encliticized onto a word ending in a labial sound (\citealt{Wiese1996a}: 166). For this purpose, I use, for the most part, verbal hosts with their endings apocopated (e.g., \textit{habe} > \textit{hab’} ‘have’). With this in mind, note that cliticization to a preceding word is only possible with an indefinite article \REF{ex:5:16a} and predicative \textit{ein} \REF[e]{ex:5:16} (encliticization is indicated in the gloss by a period; I do not use phonetic brackets to indicate the phonetic change).\footnote{In the predicative context, the copular \textit{sein} ‘to be’ was changed to \textit{bleiben} ‘to remain’. This was done to have a verbal stem ending in a non-coronal sound.}

\ea%16
    \label{ex:5:16}
\ea[]{ \label{ex:5:16a}  Indefinite Article\\
\gll Ich hab’m (frisch-es) Brot mitgebracht.\\
 I    have.a   {\db}fresh-\textsc{st} bread.\textsc{neut} brought\\
\glt ‘I have brought a (fresh) loaf of bread.’
}
\ex[]{Possessive\label{ex:5:16b}}
\sn[*]{
\gll  Ich hab’ m’m  (frisch-es) Brot mitgebracht.\\
  I    have \textsc{poss}.a {\db}fresh-\textsc{st} bread.\textsc{neut} brought\\
\glt ‘I have brought my (fresh) loaf of bread.’
}
\ex[]{Negation}
\sn[*]{ \label{ex:5:16c}
\gll  Ich hab’ k’ŋ (frisch-es) Brot mitgebracht.\\
  I    have \textsc{neg}.a {\db}fresh-\textsc{st} bread.\textsc{neut} brought\\
\glt ‘I have brought no (fresh) loaf of bread.’
}
\ex[]{Numeral}
\sn[*]{ \label{ex:5:16d}
\gll  Ich hab’M  (frisch-es) Brot mitgebracht.\\
  I have.one  {\db}fresh-\textsc{st} bread.\textsc{neut} brought\\
\glt ‘I have brought one (fresh) loaf of bread.’
}
\ex[]{Predicative}
\sn[]{ \label{ex:5:16e}
\gll  Ich bleib’m  (jung-er)   Lehrer.\\
 I remain.a  {\db}young-\textsc{st} teacher.\textsc{masc}\\
\glt ‘I remain a (young) teacher.’
}
\ex[]{Adjective}
\sn[*]{ \label{ex:5:16f}
\gll  Ich habe gestern     nur  dem’m-en  (frisch-en) Brot zugesprochen.\\
  I  have  yesterday only the.one-\textsc{wk} {\db}fresh-\textsc{wk}  bread.\textsc{neut} eaten\\
\glt ‘Yesterday, I only ate the one (fresh) loaf of bread.’
}
\z
\z

To be clear, encliticization of \textit{ein} is not possible when \textit{ein} is part of a composite \REF[b-c]{ex:5:16} or stressed \REF[d,f]{ex:5:16}. I will not have much more to say here about the regularities that govern encliticization.

\subsection{Stressability}\label{sec:5.3.2}

With regard to the possibility of bearing stress, some of the judgments from the previous section reverse. The types of \textit{ein} fall into three groups: First, the indefinite article may not be stressed \REF{ex:5:17a}; second, the possessive article, the negative article, and \textit{ein} in predicate noun phrases may be stressed \REF[b-c,e]{ex:5:17}; and third, the numeral and adjectival \textit{eine} must be stressed \REF[d,f,f’]{ex:5:17}.
\newpage
\ea%17
    \label{ex:5:17}
\ea[]{\label{ex:5:17a}   Indefinite article}
\sn[*]{
\gll  Ich habe ’N (frisch-es) Brot            mitgebracht.\\
       I     have  { }a   {\db}fresh-\textsc{st}   bread.\textsc{neut} brought\\
\glt ‘I have brought a (fresh) loaf of bread.’
}
\ex[]{ \label{ex:5:17b}   Possessive}
\sn[]{\gll Ich habe MEIN (frisch-es) Brot            mitgebracht.\\
I     have my       {\db}fresh-\textsc{st}   bread.\textsc{neut} brought\\
\glt ‘I have brought my (fresh) loaf of bread.’
}
\ex[]{ \label{ex:5:17c}   Negation}
\sn[]{
\gll Ich habe KEIN (frisch-es) Brot            mitgebracht.\\
I     have no        {\db}fresh-\textsc{st}   bread.\textsc{neut} brought\\
\glt ‘I have brought no (fresh) loaf of bread.’
}
\ex[]{ \label{ex:5:17d}  Numeral}
\sn[*]{
\gll  Ich habe ’n (frisch-es) Brot            mitgebracht.\\
      I     have  { }one {\db}fresh-\textsc{st}   bread.\textsc{neut} brought\\
\glt ‘I have brought one (fresh) loaf of bread.’
}
\ex[]{  \label{ex:5:17e} Predicative}
\sn[]{
\gll Mein Vater ist EIN Lehrer.\\
 my    father is one  teacher.\textsc{masc}\\
\glt ‘My father is one teacher.’
}
\ex[]{ \label{ex:5:17f}  Adjective}
\sn[*]{
\gll   Ich habe nur   das ’n-e       (frisch-e)  Brot            mitgebracht.\\
      I    have  only the   { }one-\textsc{wk} {\db}fresh-\textsc{wk} bread.\textsc{neut} brought\\
\glt ‘I have brought only the one (fresh) loaf of bread.’
}
\exi{f’.}  \hspace{3.5mm}Adjective\\
\sn[]{
\gll  Ich habe nur   das EIN-E  (frisch-e)  Brot            mitgebracht.\\
I    have  only the  one-\textsc{wk} {\db}fresh-\textsc{wk} bread.\textsc{neut} brought\\
\glt ‘I have brought only the one (fresh) loaf of bread.’
}
\z
\z

Although stressed, I do not provide adjectival \textit{eine} in capital letters below in order to distinguish it better from the singularity numeral. More importantly, note that with the exception of the singularity numeral and adjectival \textit{eine} (which are both independently stressed), it seems clear that stress has a semantic effect. For instance, \REF{ex:5:17e} implies that my father is not just a teacher but one teacher among others (cf. \citegen{Higginbotham1987}: 68 fn. 4 discussion of English).\footnote{In fact, stressed \textit{EIN} in \REF{ex:5:17e} might be a numeral in a predicative context. Note that with the exception of the singularity numeral and adjectival \textit{eine}, I do not discuss the effects of stress on the semantics.}

\subsection{Semantic singularity}\label{sec:5.3.3}

I turn to some semantic differences. While the indefinite article usually implies singularity of the entity \REF{ex:5:18a}, the numeral emphasizes singularity as opposed to plurality \REF{ex:5:18b}.

\ea%18
    \label{ex:5:18}
\ea\label{ex:5:18a}
\gll Ich habe (ei)n Mädchen geküßt.\\
I     have  {\db}a      girl.\textsc{neut} kissed\\
\glt ‘I have kissed a girl.’
\ex\label{ex:5:18b}
\gll Ich habe EIN Mädchen geküßt (nicht ZWEI).\\
I     have one girl.\textsc{neut} kissed   {\db}not    two\\
\glt ‘I have kissed one girl (not two).’
\z
\z

Below, I argue that singularity in \REF{ex:5:18a} does not come from \textit{ein} itself – the latter is proposed to be a semantically vacuous element. Note again that this is consistent with the indefinite article occuring in plural contexts as illustrated in the previous chapters and discussed in more detail in \sectref{sec:5.4} below. In contrast, the singularity numeral has semantics, and it is proposed below that it consists of (vacuous) \textit{ein} and a contentful element.

  There are other cases where \textit{ein} is not associated with singularity (also \citealt{Schoorlemmer2009}: 197 fn. 17). It is clear that \textit{ein} as part of the possessive articles or the negative article has no relevance with regard to semantic singularity either. In fact, these two \textit{ein}-words can take head nouns with plural morphology \REF[a-b]{ex:5:19}. Note also that a predicate noun phrase does not denote an entity but a property \REF{ex:5:19c}. As such, semantic singularity is not a relevant characteristic here either (see \chapref{sec:7}).

\ea%19
    \label{ex:5:19}
\ea\label{ex:5:19a}
\gll Ich fahre m-eine Autos.\\
I    drive  \textsc{poss}-a cars\\
\glt ‘I drive my cars.’
\ex\label{ex:5:19b}
\gll Ich fahre k-eine Autos.\\
I    drive  \textsc{neg}-a cars\\
\glt ‘I drive no cars.’
\ex\label{ex:5:19c}
\gll BMW  ist (ei)n Auto.\\
BMW is    {\db}a     car.\textsc{neut}\\
\glt ‘BMW is a car.’
\z
\z

Note in passing that the cases above also show that \textit{ein} cannot be related to indefiniteness.

Finally, nominalized infinitives and generic noun phrases make no claim about singularity of the relevant event or entity either (the data are adapted from \citealt{Bisle-Müller1991}: 115, 151).\footnote{While a nominalized infinitive is not compatible with the singularity numeral \REF{ex:5:20b}, it is fine with adjectival \textit{einmalig} ‘one-time’ preceded by the indefinite article.   
	\ea
	\label{ex:5:11:i}
	\gll (Ei)n einmaliges Abweichen vom Kurs ist verzeihbar.\\
	 {\db}a      one-time    departing.\textsc{neut} from.the course is  forgivable\\
	 \glt ‘Departing from one’s course one time is forgivable.’
	 \z
	 
	 Presumably, this has to do with the event structure of the nominal. Also, note that \REF{ex:5:20a}, \REF{ex:5:20c}, and \REF{ex:5:11:i} allow \textit{ein} to be replaced by a definite article (triggering a weak inflection on the following adjective in \REF{ex:5:11:i}). Finally, notice that while it is true that \REF{ex:5:20c} does not denote a single individual, this generic noun phrase refers to one species. \citet{Carlson1980} proposes that kind nouns are like names – they are entities (type <e>; see also Chapter 6, \sectref{sec:6.3.1.1}). As such, these noun phrases could be taken to be singular in a certain sense. Interestingly, \citet[655--659]{Longobardi1994} argues that (definite) articles before kind nouns in generic noun phrases are expletives. If this holds for all articles occurring in front of such kind nouns, then the presumed singularity in \REF{ex:5:20c} is not due to \textit{ein}.}

\ea%20
    \label{ex:5:20}
\ea[]{ \label{ex:5:20a}
\gll (Ei)n Abweichen        vom        Kurs ist nicht gut.\\
  { }a     departing.\textsc{neut} from.the course is  not    good\\
\glt ‘Departing from one’s course is not good.’
}
\ex[??]{ \label{ex:5:20b}
\gll EIN Abweichen         vom        Kurs    ist nicht gut.\\
one  departing.\textsc{neut} from.the course is  not    good\\
\glt ‘Departing from one’s course is not good.’
}
\ex[]{\label{ex:5:20c}
\gll (Ei)n Wal              ist ein Säugetier.\\
  {\db}a     whale.\textsc{masc} is   a    mammal\\
\glt ‘A whale is a mammal.’
}
\ex[*]{\label{ex:5:20d}
\gll  EIN Wal               ist ein Säugetier.\\
one  whale.\textsc{masc} is   a    mammal\\
\glt ‘A whale is a mammal.’
}
\z
\z

As to adjectival \textit{eine}, note again that like the numeral \textit{EIN} in \REF{ex:5:21a}, this type of \textit{ein} is stressed \REF{ex:5:21b} (recall though that I do not mark it as such). Crucially, unlike the numeral, adjectival \textit{eine} usually presupposes the existence of a second entity and thus implies a certain plurality of the members of the relevant kind. In fact, as observed by M. \citet[43]{Müller1986}, \textit{eine} in \REF{ex:5:21b} has a partitive sense, presupposing a set of typically two entities in the relevant world of discourse (cf. also \citealt{Vater1982}: 71). I indicated this in the English translation.

\ea%21
    \label{ex:5:21}
\ea\label{ex:5:21a}
\gll EIN Mann\\
one  man.\textsc{masc}\\
\glt ‘one man’
\ex\label{ex:5:21b}
\gll der eine Mann\\
the  one  man.\textsc{masc}\\
\glt ‘one of the two men’
\z
\z

Note that this duality presupposition cannot come from the definite article – the latter typically presupposes uniqueness in singular contexts. Importantly, \textit{eine} must be preceded by a definite element, and it is often contrasted with a second DP containing \textit{andere} ‘other’.\footnote{\label{foot:5:12} This
  construction already existed in OHG \REF{ex:5:12:ia}. Furthermore, this distribution is also possible in related Yiddish \REF{ex:5:12:ib} (from \citealt{Reershemius1997}: 362).
    \ea
      \ea \label{ex:5:12:ia}
      OHG\\
      \gll ther eino – ther ander\\
      the  one    ~  the  other\\
      \glt ‘one (of the two) – the other’  (Otfrid, \citealt{BrauneReiffenstein2004}: 234)
      \ex \label{ex:5:12:ib}
      Yiddish\\
      \gll ...tsvey brider …Der eyner hot zikh  ungerufn Elon un der tsveyter Aladan.  \\
      {\db\db}two   brothers {\db\db}the   one    has \textsc{refl} called      Elon and the second   Aladan\\
      \glt ‘…two brothers… One (of the two) was called Elon and the other Aladan.’
      \z
    \z
}

\ea%22
    \label{ex:5:22}
\gll  Der eine Mann         kam, der andere nicht.\\
  the  one  man.\textsc{masc} came the other    not\\
\glt ‘One of the (two) men came, the other did not.’
\z

Finally, like \textit{ein} in the possessive and negative composites, adjectival \textit{eine} can also be morphologically plural. In this case, adjectival \textit{eine} presupposes two sets of elements. Compare \REF{ex:5:22} to \REF{ex:5:23a}.\text{} In fact, as pointed out by a reviewer, it is possible to combine a set involving one member with a set involving multiple members \REF{ex:5:23b} (for Swedish in this regard, see \citealt{Börjars1998}: 18 fn. 7).

\ea%23
    \label{ex:5:23}
\ea\label{ex:5:23a}
\gll die      einen, die      anderen\\
the.\textsc{pl} one     the.\textsc{pl} other\\
\glt ‘these, those’
\ex\label{ex:5:23b}
\gll Der eine Sohn arbeitet, und die drei  anderen machen Pause.\\
the  one  son   works    and the three others    take       break\\
\glt ‘One of the sons is working, and the three others are taking a break.’
\z
\z

This means that the duality has to do with two sets that may differ from one another in size. Finally, we see in \sectref{sec:5.4.3.1} below that the duality presupposition associated with adjectival \textit{eine} can be cancelled.

The differences discussed above are summarized in \tabref{tab:5:1} (the properties are coded as follows: OK indicates an optional property; +/- signifies an inherent characteristic; N.A. means that this criterion is not applicable; I comment on the use of the parentheses in \tabref{tab:5:1} in the next paragraph).

\begin{table}\small
\caption{\normalsize Summary of the differences between the types of \textit{ein}}
\label{tab:5:1}
\begin{tabular}{lllllll}
\lsptoprule
\multicolumn{3}{l}{Kinds of \textit{ein}} & Enclitic & Stress & Semantic & Morphol.\\
& & & & & singularity & plural \\
\midrule
Article & \multicolumn{2}{l}{Indefinite} & OK & - & (+) & (-)\\
& \multicolumn{1}{l}{Vacuous} & Possessive & - & OK & N.A. & OK\\
&  & Negative & - & OK & N.A. & OK\\
&  & Predicative & OK & OK & N.A. & (-)\\
\multicolumn{3}{l}{Numeral} & - & + & + & -\\
\multicolumn{3}{l}{Adjective} & - & + & OK (with & OK\\
& & & & & cancelled & \\
& & & & & presupposition) & \\
\lspbottomrule
\end{tabular}
\end{table}

These are the most typical properties. Having set out the basic similarities and differences, I turn to accounting for them. In the course of the following discussion, I refine the statements about the indefinite article, especially with regard to semantic and morphological number (for predicative \textit{ein}, see \chapref{sec:6} and \ref{sec:7}). Specifically, I propose that the article \textit{ein} is semantically vacuous (Hypothesis 1a), and I illustrate again that it can occur in morphologically plural contexts. The upcoming refinement of the statements about number is indicated in \tabref{tab:5:1} by parentheses (that is, it is argued below that the indefinite article is not specified for number).

\section{Step 1 of the proposal: Morphology and semantics}\label{sec:5.4}

Recall the generalization from \sectref{sec:5.1} that determiners may, independently of word order, not co-occur. Focusing on \textit{ein}-words, the definite article, and the demonstratives, this is illustrated again below (recall that \textit{die}, glossed below as DET, comprises the definite article \textit{die} and its related distal demonstrative \textit{DIE}).

\ea%24
    \label{ex:5:24}
\gll  *  keine / meine / ’ne / die  / diese Freundin\\
 {}  no      / my     /   {\db}a  / \textsc{det} / this   girlfriend.\textsc{fem}\\
\z

There were basically two potential exceptions to this generalization: (i) Definite determiners may occur with stressed \textit{eine} \REF[a-c]{ex:5:25}, and (ii) \textit{diese} ‘this’ can occur with a possessive article \REF{ex:5:25d}. In \sectref{sec:5.4.3.1}, I discuss the different interpretations of the strings in \REF[a-b]{ex:5:25} in detail.\footnote{\label{foot:5:13} In poetic or elevated German, a possessive element can also be combined with a definite article \REF{ex:5:13:ia}. Given the presence of the article and the weak inflection on the possessive element, this lower possessive is presumably an adjective. Note that this syntactic distribution is familiar from older varieties of German \REF{ex:5:13:ib} (\REF{ex:5:13:ib} is taken from \citealt{Harbert2007}: 155).
	\ea
	\ea \label{ex:5:13:ia}
	\gll Du bist d-er mein-e.\\
	 you are  the-\textsc{st} my-\textsc{wk}\\
	 \glt ‘You are mine.’
	 \ex \label{ex:5:13:ib} OHG\\
	 \gll in dheru sineru heilegun chiburdi\\
	 in the      his       holy        birth\\
	 \glt ‘in his holy birth’
	 \z
	 \z

	Given these points, I assume that \REF{ex:5:13:ia} is part of a different grammar.}

\ea%25
    \label{ex:5:25}
\ea\label{ex:5:25a}
\gll meine eine Freundin\\
my     one  girlfriend.\textsc{fem}\\
\glt ‘one of the two of my girlfriends’\\
\glt ‘my one girlfriend’
\ex\label{ex:5:25b}
\gll die  eine Freundin\\
\textsc{det} one  girlfriend.\textsc{fem}\\
\glt ‘one of the two girlfriends’\\
\glt ‘that one girlfriend’
\ex\label{ex:5:25c}
\gll diese eine Freundin\\
this   one  girlfriend.\textsc{fem}\\
\glt ‘this one girlfriend’
\ex\label{ex:5:25d}
\gll diese meine Freundin\\
this   my      girlfriend.\textsc{fem}\\
\glt ‘this girlfriend of mine’
\z
\z

Furthermore, I showed in \sectref{sec:5.2} that the indefinite article (including predicative \textit{ein}), the possessive articles, the negative article, and the singularity numeral exhibit the same inflectional behavior. In other words, semantically quite diverse elements behave morphologically the same. In this and the next section, I account for these and some other facts. I provide a brief preview of the account of \textit{ein}.

Starting with \REF{ex:5:24}, I follow much discussion in the literature and assume that indefinite and definite articles are in D and that demonstratives are in Spec,DP (e.g., \citealt{AlexiadouEtAl2007}: 105-20; \citealt{Bernstein1997,Bernstein2001b}; \citealt{Giusti1997,Giusti2002}; \citealt{Leu2007,Leu2015}; \citealt{Roehrs2010};  \citealt{vanGelderen2007}; see also \chapref{sec:1}, \sectref{sec:1.4.1.2}). As mentioned above, only one such element can be in the DP-level in German. I propose that only one element can be merged in ArtP and move to the DP-level.\footnote{Note that for the DP-level, the Doubly-filled DP Filter is usually brought into play here (for some discussion, see \citealt{Abney1987}: 271; \citealt{Giusti1997}: 109, \citeyear{Giusti2002}: 70). In \citet{Roehrs2019}, I propose that unlike the North Germanic languages, the West Germanic languages have only one (complex) feature bundle for definiteness (merged in ArtP) explaining why there is only one element showing definiteness in the noun phrase in the latter type of language. Note that this account makes no claim about the co-occurrence of a definite and a semantically vacuous element (see the discussion of \textit{m+ein} ‘my’ below).} If so, this accounts for the non-co-occurrence of these three elements in the same DP. Turning to the two remaining elements in \REF{ex:5:24}, that is, to the negative and the possessive articles, I develop a composite analysis of \textit{keine} and \textit{meine} (and other elements) below arguing that these elements consist of vacuous \textit{ein} and an abstract component denoting negation or possession. Among others, this proposal explains, on the one hand, the non-co-occurrence of the latter two elements with articles and demonstratives (only one element is merged in ArtP) and, on the other hand, the identical morphology of the various composites and \textit{ein} (only \textit{ein} has an inflection).

  If this is on the right track, then \textit{eine} in \REF[a-c]{ex:5:25} cannot be derived from vacuous \textit{ein} and must be a different element. I propose that this type of \textit{ein} is an adjective in a high specifier position.\footnote{That this type of \textit{ein} is indeed special is quite clear in Norwegian. As is well known, this language has a singularity numeral that has the neuter form \textit{ett} ‘one’ \REF{ex:5:15:ia}. However, in the construction under discussion, the expected form \textit{ette} is impossible, and only a – what looks like – non-neuter form can be used \REF{ex:5:15:ib} (Marit Julien, p.c.).

  \ea Norwegian
  \ea \label{ex:5:15:ia}
  \gll ett   stor-t  hus\\
        one big-\textsc{st} house.\textsc{neut}\\
        \glt ‘one big house’
    \ex \label{ex:5:15:ib}
    \gll det en-e       stor-e   hus-et\\
    the one-\textsc{wk} big-\textsc{wk} house.\textsc{neut}-\textsc{def}\\
    \glt ‘one of the two big houses’

    \z
    \z
For Icelandic in this regard, see \citet[41 fn. 20]{Pfaff2015}.
}
If so, the distributions in \REF[a-c]{ex:5:25} do not present exceptions to the generalization about the non-co-occurrence of determiners. Turning to \REF{ex:5:25d}, this datum has greater potential of being an exception to this generalization. However, as already discussed in \chapref{sec:2}, \sectref{sec:2.4}, this type of \textit{diese} is not in Spec,DP. Rather, it is a predeterminer, a semantic intensifier, in LPP (also \sectref{sec:5.5.2.2}). Thus, there is only one element in the DP-level here as well.

These initial remarks can be summarized and further detailed in the following simplified structure in \figref{figex:5:26} where all elements just discussed are put in their surface positions. Disregarding restrictions on the co-occurrence of these elements, this one structure illustrates the left part of the noun phrase in German. For the sake of completeness, I have added the semantic component of the singularity numeral \textit{EIN} ‘one’, illustrated below as $\emptyset$\textsubscript{[--PL]} (see \sectref{sec:5.5.1}). Note that NEG is the abstract element later to be spelled out as \textit{nicht} ‘not’ or \textit{k(-ein)} ‘no’ (see \sectref{sec:5.4.1.2}).

\begin{figure}
	\caption{Structural position of determiners and certain other elements}
	\label{figex:5:26}
	\begin{forest}
		[LPP
			[NEG]
			[LPP
				[\textit{diese}]
				[DP
					[\textit{diese}\\\textit{DIE}\\\textit{m-}]
					[D$'$
						[D\\\textit{die}\\\textit{(ei)ne}]
						[CardP
							[$\emptyset$\textsubscript{[--PL]}]
							[Card$'$
								[Card]
								[AgrP
									[\textit{eine}\textsubscript{ADJ}]
									[Agr$'$
										[Agr]
										[$\dots$]
									]
								]
							]
						]
					]
				]
			]
		]
	\end{forest}
\end{figure}

To anticipate the structural discussion below, I assume that only one element with the categorial feature [+D] can be merged in ArtP. Given that determiners move to the DP-level, this explains why only one determiner can surface in this layer: \textit{diese}, \textit{DIE}, \textit{die}, \textit{(ei)ne}. Note now that NEG, the predeterminer \textit{diese}, possessives, the singularity component $\emptyset$\textsubscript{[--PL]}, and adjectival \textit{eine} are not merged in ArtP and can, at least in principle, co-occur with the determiners. Specifically, NEG is adjoined to LPP, the predeterminer \textit{diese} is base-generated in Spec,LPP, possessives move from Spec,NP to Spec,DP, and the singularity component is base-generated in Spec,CardP. Focusing on the different types of \textit{ein}, it will become clear in the next section that the negative article, the possessive articles, and the singularity numeral are restricted to the occurrence of \textit{ein} – these elements form composites. Finally, adjectival \textit{eine} is base-generated in the specifier position of a high AgrP and is restricted to definite contexts. These general ideas are fleshed out in what follows.

\subsection{Composite elements: Article \textit{ein} as a supporting and flagging element}\label{sec:5.4.1}

I start by providing my basic proposal. This is followed by a detailed discussion of the derivations of the different \textit{ein}-words. Finally, I point out some advantages and initial consequences of this proposal.

\subsubsection{Basic proposal}\label{sec:5.4.1.1}

Starting with \textit{ein} itself, I provided a fair amount of empirical evidence above that this element is special. Among others, it can occur not only in singular but also plural contexts. Furthermore, \textit{ein} can surface not only in indefinite but also definite environments. This means that \textit{ein} is not a singular indefinite article. Rather, I propose that it is a semantically vacuous element (Hypothesis 1a). This is in agreement with some of the semantic literature. For instance,  \citet[62]{HeimKratzer1998} state that the indefinite article can be defined as follows: $\lsem$a$\rsem$ = $\lambda$f $\in$ D\textsubscript{\textlangle e,t\textrangle}. f, which maps every function in D\textsubscript{\textlangle e,t\textrangle} to itself. In other words, \textit{ein} makes no new semantic contribution. Similarly, \citet[400--01]{CoppockBeaver2015} assume that the indefinite article is a vacuous identity function on predicates: \textit{a(n)} = $\lambda$\textit{P.P} (where like in \citeauthor{HeimKratzer1998} the input and the output are the same predicate).\footnote{The indefinite article is often related to its definite counterpart. For instance, \citet[177-81]{Giusti2015} argues that definite articles have nothing to do with definiteness or the iota operator. She proposes that definite (but also indefinite) articles are not determiners but high segments of an N-projection; in other words, they spell out inflectional features of a scattered, reprojected N-head (and in some cases those of an adjective). \citet[208]{Giusti2015} analyzes the definite article in German as \textit{d-er} ‘the’, where \textit{d}- is a dummy (see also \citealt{Roehrs2013a}) and -\textit{er} involves features of N. Extending this to the indefinite article, as Guisti does, this is compatible with the current claim that \textit{ein} is semantically vacuous (note though that in \citealt{Giusti2015}: 208, the definite article and the indefinite article in German are in different positions – D vs. Spec,DP, respectively). Finally, note that the indefinite article is sometimes also assumed to be semantically vacuous in other languages. For instance,  \citet[245]{MatushanskySpector2005} state that the indefinite article in post-copular position in French is semantically vacuous and a reflex of a syntactic operation.}

Turning to the other \textit{ein}-words, it was suggested in \citet[Chapter 4]{Roehrs2009a} that \textit{ein} is part of the negative article \textit{kein} ‘no’, possessive articles like \textit{mein} ‘my’, and the singularity numeral \textit{EIN} ‘one’ \REF[a-c]{ex:5:27}. Below, this is implemented by proposing that \textit{ein} supports certain operators that are detectable, segmentally as in the case of NEG \textit{k}- and POSS \textit{m}- or suprasegmentally as in the case of $\emptyset$\textsubscript{[--PL]}, which induces stress.\footnote{I define operator broadly here. They involve scopal elements (negators, quantifiers), indexical elements (possessives), and other semantically active, functional elements (see REL below and certain other elements in subsequent chapters).} We see in \chapref{sec:6} that \textit{ein} also indicates the presence of the realization operator REL \REF{ex:5:27d}. Unlike \REF[a-c]{ex:5:27}, here \textit{ein} flags the presence of an operator that has no independent manifestation. I indicate the co-occurrence relation of \textit{ein} and REL by an underscore. Completing the picture, I propose that unlike these four cases, adjectival \textit{eine} is an independent element \REF{ex:5:27e}.

\TabPositions{5cm}
\ea%27
    \label{ex:5:27}
\ea\label{ex:5:27a}   (vacuous) \textit{ein} + NEG \tab$\rightarrow$   \textit{kein}

\ex\label{ex:5:27b}   (vacuous) \textit{ein} + POSS\textsubscript{[1ST}\textsc{}\textsubscript{; --PL]}  \tab $\rightarrow$   \textit{mein}

\ex\label{ex:5:27c}   (vacuous) \textit{ein} + $\emptyset$\textsubscript{[--PL]}  \tab $\rightarrow$   \textit{EIN}

\ex\label{ex:5:27d}  (vacuous) \textit{ein} \_ REL   \tab $\rightarrow$   \textit{ein}
\ex\label{ex:5:27e}   (non-composite) \textit{eine}\textsubscript{ADJ}
\z
\z

The claim that \REF[a-b]{ex:5:27} are composite forms is by no means new; for instance, \citet{FanselowCavar2002} and  \citet[247]{KobeleZimmermann2012} employ \REF{ex:5:27a}, and \citet[4]{Corver2003} and  \citet[114]{CorvervanKoppen2010} assume \REF{ex:5:27b} in their discussion of Dutch (cf. also \citealt{Julien2016}: 94-95 for certain possessives in Scandinavian). \citet[331]{Murphy2018} and \citet[197 fn. 17]{Schoorlemmer2009} accept both \REF{ex:5:27a} and \REF{ex:5:27b} for German. The analysis in \REF{ex:5:27c} might be relatable to \citet{Ackles1996}, who proposes that English \textit{a(n)} licenses NumP, an element without overt realization, in the context of singular count nouns. \REF{ex:5:27d} is also in good company; for instance, \citet{BennisEtAl1998} claim that the operator [+EXCL] needs to be lexicalized by \textit{een} ‘a’ in Dutch (\chapref{sec:4}, \sectref{sec:4.2.1}). Finally, adjectival \textit{eine} is proposed by \citet[155]{Gallmann2004}, \citet[160-61]{Lindauer1995}, M. \citet[45]{Müller1986}, and \citet[179]{Pafel2005}. Note that these related works lend some initial credence to the current proposal (more references are provided in the course of the following discussion). Consider the derivation of the composite elements in \REF[a-c]{ex:5:27} in the framework of DM.

\subsubsection{Derivation of composite forms}\label{sec:5.4.1.2}

Beginning with \textit{ein} itself, I proposed in \chapref{sec:2}, \sectref{sec:2.2.1.6} that this element consists of the categorial feature [+D] and features for case, number, and gender (I leave the values unspecified below). Unlike the definite article and the demonstratives, \textit{ein} has no features for definiteness or deixis. As such, this is the least specified determiner. This fits well with the proposal above that \textit{ein} is a semantically vacuous element.

\begin{figure}
	\caption{Structure of the indefinite article \emph{ein}}
	\label{figex:5:28}
	\begin{forest}
		[ Art\textsubscript{\parbox{0mm}{\mbox{[+D][F, N, O, S]}}}
			[{[+D]}]
			[{[F, N, O, S]}]
		]
	\end{forest}
\end{figure}

I turn to the operators in \REF[a-c]{ex:5:27}. Starting with the possessive articles, recall from \chapref{sec:2}, \sectref{sec:2.2.2.3} that I analyze these elements as bound or free morphemes \REF[a-b]{ex:5:29}.\footnote{A reviewer remarks that it is unusual that determiner stems of the same kind seem to fall into bound and free morphemes (but see also personal pronouns mentioned in Footnote \ref{foot:5:19} below). Note that bound in this case means a bipartite stem (e.g., \textit{m-ein}) and free signifies a monopartite stem (e.g., \textit{ihr}). While I flesh out this oft-suggested analysis below, there are alternatives. Note in this regard that unlike accusative/dative personal pronouns, all monopartite possessive articles consist of \textit{(e)r} at the end: \textit{ih-r} ‘her; their’ vs. \textit{ih-n/ih-m} ‘him’; \textit{uns-er} ‘our’ vs. \textit{uns} ‘us’; \textit{eu-er} ‘your’ vs. \textit{eu-ch} ‘you’. A different account than the one proposed below (where I suggest that the pronunciation of \textit{ein} is suppressed in the context of free/monopartite possessives) would be to claim that \textit{ein} is actually spelled out as -\textit{r} in the relevant possessive contexts – another case of contextually conditioned allomorphy. This would mean that all possessive articles involve bound/bipartite forms. I proceed on more traditional assumptions.} I point out now that these forms are more general (also \citealt{Fischer2006}). For instance, the bound morphemes in \REF{ex:5:29a} also occur as part of reflexive pronominal forms \REF{ex:5:29c}. In fact, this decomposition is quite pervasive with personal pronouns \REF{ex:5:29d} (I leave out the genitive forms, which are diachronically related to the possessive forms in \REF{ex:5:29a}).\footnote{\label{foot:5:19}Note that third-person \textit{s}- is somewhat special: It only involves masculine with possessives \REF{ex:5:29a} but masculine and/or other specifications with reflexives \REF{ex:5:29c} and personal pronouns \REF{ex:5:29d}. Notice also that \textit{S-ie} ‘you(\textsc{formal})’, not provided in \REF{ex:5:29}, is morphologically third but semantically second person (see \chapref{sec:7}). Furthermore, recall that the third-person pronoun \textit{sie} ‘she, her; they, them’ was claimed to take an adjectival inflection that is later deleted due to the avoidance of a hiatus (\chapref{sec:3}, \sectref{sec:3.5.3.2}). In other words, given current assumptions, this pronoun has the underlying form of \textit{s-ie-e}. Compared to third-person singular \textit{e-r} ‘he’, we find another case of bound/bipartite vs. free/monopartite determiner stems.}

\ea%29
    \label{ex:5:29}
\ea \label{ex:5:29a}
\gll m-,   d-,               s-\\
    my-, your(\textsc{sg})-, his-\\
\ex \label{ex:5:29b}
\gll ihr,                                  unser, euer\\
her/their/your(\textsc{formal}), our,     your(\textsc{pl})\\
\ex \label{ex:5:29c}
\gll m-ich,   d-ich,      s-ich\\
myself, yourself, himself/herself/themselves\\
\glt ‘myself, yourself, himself/herself/themselves’
\ex \label{ex:5:29d}
\gll m-ich, m-ir;   d-u,     d-ich,       d-ir;     s-ie\\
me.\textsc{acc,} me.\textsc{dat}; you.\textsc{nom}(\textsc{sg}), you.\textsc{acc}(\textsc{sg}), you.\textsc{dat}(\textsc{sg}); she.\textsc{nom}/her.\textsc{acc}/they.\textsc{nom}/them.\textsc{acc}\\
\glt ‘me; you; she/her/they/them’
\z
\z

There is another subregularity here in that all elements in \REF{ex:5:29b} involve feminine and plural forms, which tend to pattern together in the nominal paradigms (\chapref{sec:2}, \sectref{sec:2.2.1.5}). Taken together, this provides some empirical motivation for separating possessive articles into bound and free morphemes in \REF{ex:5:29a} and \REF{ex:5:29b}, respectively.

  \hspace*{-1pt} In \citet{Roehrs2020a}, I propose that possessives involve a Possessive Phrase (PossP) and that they are base-generated low in the nominal structure. Following standard assumptions, I assume that prenominal possessives move from Spec,NP to Spec,DP (\chapref{sec:1}, \sectref{sec:1.4.1.2}). After movement of the article and PossP, the relevant part of the structure, the DP-level, can be illustrated as follows in \figref{figex:5:30}.

\begin{figure}
	\caption{Movement of possessives and articles}
	\label{figex:5:30}
	\begin{forest}
		[DP
			[PossP\textsubscript{$k$} [Poss]]
			[D$'$
				[D
					[Art\textsubscript{$i$}
						[{[+D]}]
						[{[F, N, O, S]}]
					]
					[D]
				]
				[{[… t\textsubscript{$i$} … t\textsubscript{$k$} …]}]
			]
		]
	\end{forest}
\end{figure}

After Linearization of \figref{figex:5:30}, Vocabulary Insertion occurs. There are two types of cases involving possessive articles.

  First, after bound possessive morphemes like \textit{m}- in \REF{ex:5:29a} are inserted in the left periphery of the nominal, the string in \figref{figex:5:31} is obtained. In \chapref{sec:2}, \sectref{sec:2.2.2.2}, I formulated specific vocabulary insertion rules for \textit{ein}. If an overt element like an adjective and/or noun follows, then \textit{ein} in nominative masculine and nominative/accusative neuter contexts spells out both the categorial feature [+D] and the CNG feature bundle (cf. \figref{figex:5:30} above) yielding uninflected \textit{ein}; in all other cases, \textit{ein} spells out [+D], and an inflectional suffix spells out [F, N, O, S]. The latter scenario yields inflected forms of \textit{ein}. Both options are illustrated below by putting the inflection in parentheses.

\begin{figure}
	\caption{Spell-out of \emph{mein}}
	\label{figex:5:31}
	\begin{forest}
		[
			[\textit{m-}]
			[\textit{ein}]
			[(-INFL)]
		]
	\end{forest}
\end{figure}

I propose that \textit{ein} supports both the bound morpheme \textit{m}- and the inflection (unless \textit{ein} spells out both [+D] and the CNG feature bundle at the same time). Note now that all relevant elements are adjacent to each other. Given these local, linear relations, I suggest that the operation of support is instantiated by Local Dislocation such that \textit{ein} undergoes this operation onto \textit{m}- and that the inflection, if present, does so onto \textit{m-ein} (cf. \citealt{Murphy2018}).

Second, the free possessive morphemes in \REF{ex:5:29b} involve the same derivation as in \figref{figex:5:30} above. After Linearization and Vocabulary Insertion, the following string obtains exemplifying with \textit{ihr} ‘her’.

\begin{figure}
	\caption{Spell-out of \emph{ihr} (Initial stage)}
	\label{figex:5:32}
	\begin{forest}
		[
		[\textit{ihr}]
		[\textit{ein}]
		[(-INFL)]
		]
	\end{forest}
\end{figure}

This is not the correct surface string yet. Note that there are two free morphemes in \figref{figex:5:32}: \textit{ihr} and \textit{ein}. I suggest that only one free form can surface in this local domain. Given that the possessive has semantics but \textit{ein} does not, I assume that the latter does not appear, see \figref{figex:5:33}.\footnote{In a traditional tree structure, the Doubly-filled DP Filter could be brought into play here. However, the framework adopted here (DM) inserts vocabulary items after Linearization. If the tree structure is no longer accessible after Linearization, then a different way needs to be found to rule out the co-occurrence of the free possessive element and \textit{ein}. Note in this regard that we cannot claim that \textit{ein} is only merged as a Last Resort option (NB: This is consistent with \chapref{sec:4}, \sectref{sec:4.3.4}, where I argued that \textit{ein} is unlikely to be inserted late); for instance, we cannot claim that \textit{ein} is inserted unless there is already another element present that can perform \textit{ein}’s two main functions (supporting and flagging). This is so because, if \textit{ein} were not inserted in the context of \figref{figex:5:32}, then the CNG bundle would remain untouched in the three (/four) exceptional cases. As a consequence, the inflection would be spelled out and would appear on (i.e., be supported by) the possessive in every instance yielding ungrammatical forms such as *\textit{ihr-er} in, say, a nominative masculine context. Given that, a more promising way to deal with the absence of \textit{ein} would be to delete it in the (simplified) context of [Poss, pron, +F], which covers feminine and plural possessive pronouns. Alternatively, we could follow \citet[86-87]{Matushansky2006} in suggesting that \textit{ein} and the possessive element undergo Fusion (into one element) in feminine and plural contexts. Either option would lead to the absence of \textit{ein}.}

\begin{figure}
	\caption{Spell-out of \emph{ihr} (Final stage)}
	\label{figex:5:33}
	\begin{forest}
		[
		[\textit{ihr}]
		[\textit{\sout{ein}}]
		[(-INFL)]
		]
	\end{forest}
\end{figure}

If an inflection is present (i.e., the CNG bundle of the article structure of \textit{ein} remains untouched by the vocabulary insertion rules of \textit{ein}), it will undergo Local Dislocation onto the possessive element.

As to the two remaining composite elements in \REF[b-c]{ex:5:27}, I assume that \textit{kein} ‘no’ in \REF[b]{ex:5:27} is similar to \textit{mein} ‘my’, the difference being that NEG is adjoined to the noun phrase in \figref{figex:5:30} and that it can be realized as either \textit{k}- or \textit{nicht}. The basic rule is provided in \REF{ex:5:34}.

\ea%34
    \label{ex:5:34}

          NEG   $\rightarrow$   \textit{k}-   / \_\_ (unstressed) \textit{ein}

  \hspace*{.78cm}  $\rightarrow$   \textit{nicht}   (elsewhere)
\z

In words, NEG is spelled out as \textit{k}- in the context of unstressed \textit{ein} but as \textit{nicht} in the remaining instances (but see also \sectref{sec:5.4.2.1} for some complications arising from the apparent optionality of \textit{kein} vs. \textit{nicht} in certain instances). As with the bound possessive morphemes, \textit{ein} undergoes Local Dislocation onto \textit{k}-. Finally, the singularity numeral $\emptyset$\textsubscript{[--PL]} in \REF[c]{ex:5:27} is also supported by \textit{ein} instantiated by Local Dislocation (for detailed discussion including a tree structure, see \sectref{sec:5.5.1}). To be clear, the elements on the right of the arrows in \REF[a-c]{ex:5:27} are taken to be the spell-out forms of the combination of the elements on the left of them. With regard to the workings of \textit{ein}, I continue speaking of support; that is, \textit{ein} supports other elements (instantiated by Local Dislocation).

\subsubsection{Some advantages and initial consequences}\label{sec:5.4.1.3}

There are two immediate advantages of this proposal. With these composite elements made up of \textit{ein} and another component, it is easy to see how semantically quite diverse elements can share identical morphology. Furthermore, with the article \textit{ein} present, it follows that a second determiner cannot occur, and the complementary distribution with other determiners follows. Interestingly, noun phrases involving indefinite articles may have different interpretations with regard to specificity. As shown below, these interpretations are often claimed to be tied to the indefinite article being in different positions, for instance, D and Card. It could be suggested that vacuous \textit{ein} also supports D\textsubscript{[+}\textsc{\textsubscript{spec}}\textsubscript{]} and Card\textsubscript{[--}\textsc{\textsubscript{spec}}\textsubscript{]}, with the higher position associated with specificity and the lower one with non-specificity. However, these are structural elements, and it is not likely that they are supported in the sense above.

  In more detail, it is well known that indefinite articles are weak determiners \citep{Milsark1974}. Furthermore, indefinite noun phrases may have different readings (see, among many others, \citealt{FodorSag1982}, \citealt{Harbert2007}: 140, \citealt{Heusinger2011}, \citealt{Zamparelli2005}: 760; also \citealt{Diesing1992} and \citealt{Hallman2004}). For instance, they can have a non-specific \REF{ex:5:35a} or a presuppositional, specific \REF{ex:5:35b} interpretation.

\ea%35
    \label{ex:5:35}
\ea   \label{ex:5:35a}
\gll Ich würde nur (ei)n Auto        mit   lila      Punkten kaufen.\\
I     would only {\db}a     car.\textsc{neut} with purple dots       buy\\
\glt ‘I would only buy a car with purple dots.’
\ex \label{ex:5:35b}
\gll Ich habe gestern   (ei)n Auto        mit   lila      Punkten gekauft.\\
I     have yesterday {\db}a      car.\textsc{neut} with purple dots        bought\\
\glt ‘I bought a car with purple dots yesterday.’
\z
\z

Without going into much detail here, these readings are licensed by noun phrase-external factors, for instance, the different moods in the hosting clause (subjunctive vs. indicative). I basically follow \citet[105]{Jackendoff1977} and \citet{Bowers1988} in that weak determiners may be in different positions inside the noun phrase (licensed by these noun phrase-external factors).

To make the discussion concrete, I assume that there are two positions inside the DP relevant for the different readings: D, the head of DP, and Card, the head of CardP, which is located just below the DP (\chapref{sec:1}). In particular, on the weak, non-specific reading in \REF{ex:5:35a}, I assume that the indefinite article surfaces in Card, and on the strong, specific reading in \REF{ex:5:35b}, it appears in D (also \citealt{AlexiadouEtAl2007}: 225, \citealt{Borer2005}: 144-45, \citealt{Chomsky1995}: 342, \citealt{Zamparelli2000}: 264-65). As proposed in \chapref{sec:1}, \textit{ein} originates in ArtP. With \textit{ein} lacking a feature for definiteness, it does not have to undergo movement to D. Thus, I propose that \textit{ein} moves to Card if the nominal has a weak reading and to D if the nominal has a strong reading. As such, I assume that vacuous \textit{ein} in Card\textsubscript{[--}\textsc{\textsubscript{spec}}\textsubscript{]} or in D\textsubscript{[+}\textsc{\textsubscript{spec}}\textsubscript{]} does not involve composite forms – the feature bundles later to be spelled out as \textit{ein} simply move to these positions.

Finally, as regards \REF{ex:5:27d}, I claim in \chapref{sec:6} that vacuous \textit{ein} (or a definite determiner) flags the presence of the realization operator REL. Recall that I suggested above that the negative, possessive, and singularity components of the \textit{ein}-words also involve operators. Given that, I reiterate the proposal that there are two ways to make operators visible by \textit{ein}: supporting where the operator itself is in some way detectable, as claimed for \REF[a-c]{ex:5:27}, and flagging where the operator itself remains invisible as in \REF{ex:5:27d}. If this is tenable, then the term indefinite article seems inappropriate, but I continue with the traditional terminology. As to \REF{ex:5:27e}, adjectival \textit{eine} does not involve supporting or flagging but is a separate lexical item (for more details, see \sectref{sec:5.5.2}).

\subsection{Some evidence for the composite analyses}\label{sec:5.4.2}

In the previous section, I provided the derivations that bring about the three composite forms. What they all have in common is that the two relevant elements are adjacent to one another. I consider this in more detail now providing some evidence for the composite analyses. I start with the most straightforward case.

\subsubsection{The negative article \textit{kein}}\label{sec:5.4.2.1}

It is a fairly standard proposal that \textit{kein} ‘no’ consists of NEG + \textit{ein} (e.g., \citealt{Bech1955}, \citealt{Jacobs1980}, \citealt{Kratzer1995}: 144-47). I provide two morpho-syntactic arguments for the composite analysis. First, considering the data below, it seems clear that the contraction of NEG and \textit{ein} is obligatory in certain cases \REF{ex:5:36}, optional in others \REF{ex:5:37}, but cannot occur in yet other contexts \REF{ex:5:38}.

\ea%36
    \label{ex:5:36}
\ea[*]{\label{ex:5:36a}
\gll Ich habe nicht (ei)n  Buch          gekauft.\\
I     have not      {\db}a      book.\textsc{neut} bought\\
\glt ‘I did not buy a book.’
}
\ex\label{ex:5:36b}
\gll Ich habe kein Buch           gekauft.\\
I     have no    book.\textsc{neut} bought\\
\glt ‘I bought no book.’
\z
\z

\ea%37
    \label{ex:5:37}
\ea\label{ex:5:37a}
\gll Ich habe nicht (ei)n BUCH        gekauft, sondern (ei)n HEFT.\\
I     have not      {\db}a     book.\textsc{neut} bought   but          {\db}a      booklet.\textsc{neut}\\
\glt ‘I did not buy a book but a booklet.’
\ex\label{ex:5:37b}
\gll Ich habe kein BUCH        gekauft, sondern (ei)n HEFT.\\
I     have no    book.\textsc{neut} bought   but          {\db}a      booklet.\textsc{neut}\\
\glt ‘I did not buy a book but a booklet.’
\z
\z

\ea%38
    \label{ex:5:38}
\ea \label{ex:5:38a}
\gll (Ei)n Buch          habe ich nicht gekauft.\\
  {\db}a      book.\textsc{neut} have I     not    bought\\
\glt ‘I bought no book/I did not buy a book.’

\ex[*]{\label{ex:5:38b}
\gll (Ei)n Buch          habe ich kein gekauft\\
 {\db}a      book.\textsc{neut} have I     no    bought\\
\glt ‘I bought no book/I did not buy a book.’
}
\z
\z

Specifically, with ordinary stress, NEG and \textit{(ei)n} must form the negative article \REF{ex:5:36}. This is different with contrastive stress on the noun where NEG and \textit{ein} can optionally be spelled out separately as \textit{nicht ein} or as the composite form \textit{kein} \REF{ex:5:37}. Finally, when NEG and \textit{(ei)n} are not adjacent \REF{ex:5:38}, both elements are spelled out separately, with NEG being realized as \textit{nicht}.\footnote{Note again that \textit{ein} and \textit{kein} can co-occur in split topicalizations, see \REF{ex:5:21:i}.

\ea \label{ex:5:21:i}
\gll (Ei)n Buch          habe ich kein-s    gekauft.\\
  {\db}a      book.\textsc{neut} have I     none-\textsc{st} bought\\
  \glt ‘As for books, I bought none.’
\z

  Considering the strong ending on \textit{kein} in \REF{ex:5:21:i}, I assume that the negative article is followed by a null noun (see \chapref{sec:4}, \sectref{sec:4.3.3}). In other words, there are two separately base-generated nominals in \REF{ex:5:21:i}, each with its own \textit{ein}. As such, the grammaticality of this example is not an argument against \textit{kein} requiring adjacency of NEG and \textit{(ei)n}, which is illustrated in \REF{ex:5:38}.} Now, the fact that NEG can be realized in two different ways depending on the stress pattern and the adjacency with \textit{ein}, supports the claim that \textit{kein} is a composite form.

  There is a second piece of evidence for this composite analysis. The examples in \REF{ex:5:39a} and \REF{ex:5:39b} establish that negation is higher than the type particle \textit{so} ‘such’ and that the latter can intervene between the negation and the indefinite article.\footnote{Note that \textit{nicht} ‘not’ can form a constituent with the nominal (with stress on \textit{so} ‘such’). 
  	\ea \label{ex:5:22:i}
  	\gll Nicht so    ein Mann kam, sondern ein anderer.\\
  	not    such a    man   came but         a   different\\
  	\glt ‘Not such a man came, but a different one.’
  	\z 
  	}
  However, with an intervening \textit{so}, negation is not adjacent to \textit{ein}, and consequently, these two elements cannot be spelled out as \textit{kein} by the morphology as evidenced by \REF{ex:5:39c} and \REF{ex:5:39d}.

\ea%39
    \label{ex:5:39}
\ea\label{ex:5:39a}
\gll nicht so    (ei)n Idiot\\
not    such {\db}an    idiot.\textsc{masc}\\
\glt ‘not such an idiot’
\ex[*]{\label{ex:5:39b}
\gll so nicht (ei)n Idiot\\
so not     {\db}an   idiot.\textsc{masc}\\
}
\ex[?*]{\label{ex:5:39c}
\gll kein so     Idiot\\
no    such idiot.\textsc{masc}\\
}
\ex[*]{\label{ex:5:39d}
\gll so kein Idiot\\
so no    idiot.\textsc{masc}\\
}
\z
\z

This provides a second argument that \textit{kein} ‘no’ is a composite form subject to the adjacency of its two components. For other syntactic and semantic arguments for a composite analysis, see \citet[186-87]{Pafel2005}, \citet[467-68]{vonFintelIatridou2007}, and \citet{Zeijlstra2011} (for English, see \citealt{Klima1964}: 273-76); for the discussion of \textit{kein so’n Idiot} ‘(NEG+a so an =) no such idiot’, see \chapref{sec:8}, \sectref{sec:8.2.2.2}.\footnote{There are different proposals to implement the claim that \textit{kein} ‘no’ is a composite form. For instance, like the current analysis, \citet[119, 131]{Zeijlstra2011} assumes that \textit{kein} is a complex element. Unlike the current analysis, he argues that \textit{kein} is a complex head consisting of a top node that dominates a negation operator and an existential operator. These varying implementations have different consequences, something I will not discuss here.} If \textit{kein} involves a composite analysis, then it is not implausible to propose this for the other complex \textit{ein}-words as well.

\subsubsection{Possessive articles such as \textit{mein}}\label{sec:5.4.2.2}

There is also evidence that possessive articles are composite forms. While Saxon Genitives can both precede and follow their head nouns \REF[a-b]{ex:5:40} (e.g., \citealt{Duden2007}: 366, \citealt{EisenbergSmith2002}: 125, \citealt{Fuß2011}), possessive articles can only precede their head nouns \REF[c-d]{ex:5:40}.

\ea%40
    \label{ex:5:40}
\ea \label{ex:5:40a}
\gll Magdalenas  Buch          ist schön.\\
Magdalena’s book.\textsc{neut} is  beautiful\\
\glt ‘Magdalena’s book is beautiful.’
\ex  \label{ex:5:40b}
\gll Das Buch          Magdalenas  ist schön.\\
the  book.\textsc{neut} Magdalena’s is  beautiful\\
\glt ‘The book of Magdalena’s is beautiful.’
\ex \label{ex:5:40c}
\gll Mein Buch          ist schön.\\
my    book.\textsc{neut} is  beautiful\\
\glt ‘My book is beautiful.’
\ex[*]{ \label{ex:5:40d}
\gll Das Buch          mein(s) ist schön.\\
the  book.\textsc{neut} mine      is  beautiful\\
\glt ‘The book of mine is beautiful.’
}
\z
\z

These facts follow from the assumption that possessive articles consist of a possessive component and \textit{ein}. Specifically, \textit{ein} can, as an article, only precede its head noun explaining the restriction in \REF{ex:5:40d}. There is other evidence for a composite analysis.

It is well known that the possessive component of the possessive article agrees in gender with its antecedent but that the remaining part of the possessive article agrees in case, number, and gender with the head noun \REF{ex:5:41a}. Furthermore, besides gender, the possessive component also agrees in person and number with its antecedent, as can easily be seen in \REF{ex:5:41b}. These agreement relations are illustrated with different indices below.\footnote{\citet[2077]{GeorgiSalzmann2011} observe the data points in \REF{ex:5:24:i}, where a neuter possessor can occur with both a neuter and a feminine possessive article.
	\ea \label{ex:5:24:i}
	\ea \label{ex:5:24:ia}
	\gll Das Mädchen hat seine / ihre Schuhe verloren\\
	the  girl.\textsc{neut} has its     / her   shoes   lost\\
	\glt ‘The girl has lost her shoes.’
	\ex \label{ex:5:24:ib}
	\gll dem       Mädchen seine / ihre Schuhe\\
	the.\textsc{dat} girl.\textsc{neut} its     / her  shoes \\
	\glt ‘the girl’s shoes’
	\z
	\z
They argue that the agreement relation between the possessor and the possessive article involves an anaphoric binding relation.}

\ea%41
    \label{ex:5:41}
\ea \label{ex:5:41a}
\gll Peter\textsubscript{i} hat s\textsubscript{i}-einer\textsubscript{k} Freundin\textsubscript{k} geholfen.\\
Peter  has \textsc{poss}-a   girlfriend.\textsc{fem} helped\\
\glt ‘Peter helped his girlfriend.’
\ex \label{ex:5:41b}
\gll Ich\textsubscript{i} habe m\textsubscript{i}-einen\textsubscript{k} Freunden\textsubscript{k} geholfen.\\
I      have \textsc{poss}-a     friends       helped\\
\glt ‘I helped my friends.’
\z
\z

These different agreement relations accessing the same word are straightforwardly explained by the composite analysis.\footnote{\citeauthor{Olsen1989b} (\citeyear{Olsen1989b}: 139, \citeyear{Olsen1991b}: 53) also assumes decomposition of possessive articles, but it is different from the current analysis. Exemplifying with \textit{meinen} ‘my’, \textit{mein} is in Spec,DP, and -\textit{en} is in D. There are three issues here. First, Olsen also locates the possessive marker -\textit{s} in D. Note that D has the same features when the inflection -\textit{en} or the marker -\textit{s} is located in D (\citealt{Olsen1991b}: 48, 53). This raises the question of what rules out the possessive marker surfacing on the possessive pronoun (cf. *\textit{meins neues Auto} ‘my-\textsc{poss} new-\textsc{st} car’ vs. \textit{Peters neues Auto} ‘Peter’s new-\textsc{st} car’). Second and more importantly, the lack of decomposition of the stem into \textit{m}- and \textit{ein} loses the generalization that all \textit{ein}-words behave the same morpho-syntactically. Third, we pointed out above that \textit{ein} and adjectival inflections have a status different from the possessive component: The first two elements are semantically vacuous, and they make linguistic items visible -- \textit{ein} supports or flags operators and adjectival inflections provide overt exponents for abstract features. As such, \textit{ein} should be grouped with the inflection in D as in the current account (and not with the possessive component in Spec,DP as in Olsen’s work). Recall that here \textit{ein} involves a complex head consisting of [+D] and a CNG feature bundle.}

\subsubsection{The singularity numeral \textit{EIN}} \label{sec:5.4.2.3}

I discuss the advantages of a composite analysis of \textit{EIN} ‘one’ in detail in \sectref{sec:5.5}. Note already here that $\emptyset$\textsubscript{[--PL]} is located in Spec,CardP in the weak reading of this indefinite element and that the supporting element \textit{ein} is in Card (recall that \textit{ein} lacks a definiteness feature and as such, it does not have to move to the DP-level). Something similar holds for the strong reading where $\emptyset$\textsubscript{[--PL]} has moved to Spec,DP, and \textit{ein} is in D. Crucially, these are all Spec-head constellations, where adjacency holds inherently. Consequently, \textit{ein} can support $\emptyset$\textsubscript{[--PL]}, and both elements are spelled out by the morphology as the numeral (note also that \textit{ein} supporting $\emptyset$\textsubscript{[--PL]} licenses this null element at the same time). When $\emptyset$\textsubscript{[--PL]} is not present, the vacuous article is spelled out as \textit{ein} under Card or D as discussed in \sectref{sec:5.4.1.3}.

  To sum up, I take two points as established: First, the negative article \textit{kein} ‘no’, possessive articles such as \textit{mein} ‘my’, and the singularity numeral \textit{EIN} ‘one’ are all composites; second, these elements involve post-syntactic spell-out forms. As we have seen and will see throughout this chapter, this analysis has a number of advantages allowing for a fairly simple account of the different kinds of \textit{ein}. Before I turn to some syntactic differences, I consider some of the semantic and morphological distinctions of these elements in more detail.

\subsection{Feature specifications} \label{sec:5.4.3}

First, I document that the presupposition induced by adjectival \textit{eine} can be cancelled under certain conditions. This lays the foundation for the second subsection, where I provide the feature specifications of the different types of \textit{ein}.

\subsubsection{Cancelling the presupposition of adjectival \textit{eine}}\label{sec:5.4.3.1}

Recall that adjectival \textit{eine} has to occur in definite contexts and that it involves a duality presupposition. This presupposition seems to be strongest with the definite article \REF{ex:5:42a}, it is possible with the possessive article \REF{ex:5:42b}, but it is absent with the demonstrative \textit{diese} ‘this’ \REF{ex:5:42c}.

\ea%42
    \label{ex:5:42}
\ea \label{ex:5:42a}
\gll die  ein-e    Freundin\\
the one-\textsc{wk} girlfriend.\textsc{fem}\\
\glt ‘one of the two girlfriends’
\ex \label{ex:5:42b}
\gll meine ein-e     Freundin\\
my     one-\textsc{wk} girlfriend.\textsc{fem}\\
\glt ‘one of the two of my girlfriends’\\
\glt ‘my one girlfriend’
\ex \label{ex:5:42c}
\gll diese ein-e     Freundin\\
this   one-\textsc{wk} girlfriend.\textsc{fem}\\
\glt ‘this one girlfriend’
\z
\z

Recall also that the duality presupposition cannot stem from the singular definite article. On the one hand, this determiner carries a uniqueness presupposition; on the other hand, other definite elements like possessive articles can also license adjectival \textit{eine}. This means that the duality cannot come from the definite article itself. Rather, I pointed out above that in conjunction with a definite element, the presence of \textit{eine} brings about partitive semantics in that it presupposes the existence of two sets of elements where the sets may involve one or several members.

  Besides the definite and the possessive articles, there are other definite elements that occur with adjectival \textit{eine}. Specifically, the periphrastic possessive in \REF{ex:5:43a} and the Saxon Genitive in \REF{ex:5:43b} can each involve a duality presupposition (for \REF{ex:5:43b}, see \citealt{GunkelEtAl2017}: 1300).

\ea%43
    \label{ex:5:43}
\ea\label{ex:5:43a}
\gll Peter sein ein-er  Sohn\\
Peter his   one-\textsc{st} son.\textsc{masc}\\
\glt ‘one of the two of Peter’s sons’\\
\glt ‘Peter’s one son’
\ex\label{ex:5:43b}
\gll Peters  ein-er  Sohn\\
Peter’s one-\textsc{st} son.\textsc{masc}\\
\glt ‘one of the two of Peter’s sons’\\
\glt ‘Peter’s one son’
\z
\z

Before moving on, note already here that the weak inflection on \textit{eine} in \REF{ex:5:42} and the strong inflection on \textit{eine} in \REF{ex:5:43} indicate that \textit{eine} is indeed adjectival (see \sectref{sec:5.5.2} for detailed morpho-syntactic discussion). Returning to the semantics, the question arises if there are other cases where the presupposition is absent and why the duality presupposition may be absent.

  That the duality presupposition is missing with demonstratives is quite clear in \REF{ex:5:44a}. In this case, there is no presupposition that there is a second boy with the relevant property. As might be expected, if the definite article is stressed and functions as the distal demonstrative, there is no presupposition either \REF{ex:5:44b}.\footnote{Note that \textit{eine} in \REF{ex:5:44b} has fairly strong stress – \textit{eine} is inherently stressed and, additionally, it appears in phrase-final position. As a possible consequence of that, adjacent \textit{die} is not (strongly) stressed.}  In a similar vein, \REF{ex:5:44c} consists of a singular subject and a predicate nominal modified by a relative clause. Indeed, there is no presupposition that there is a second man with the property denoted by the relative clause. Furthermore, Orrin Robinson (p.c.) raises the question as to whether the duality presupposition can disappear when one possessive followed by adjectival \textit{eine} is coordinated with a different possessive followed by a numeral \REF{ex:5:44d}. This is indeed the case (\REF[b]{ex:5:44} is taken from M. \citealt{Müller1986}: 45).\footnote{In Footnote \ref{foot:5:12}, we saw that \textit{ein} with duality presupposition could also follow a definite determiner in OHG. Apparently, this presupposition could also be cancelled with demonstratives \REF{ex:5:27:ib} and in contexts involving a unique entity \REF{ex:5:27:ib} in this older variety of German. 
  	\ea OHG
  	\ea \label{ex:5:27:ia}
  	\gll thiz eina jār\\
  	this one  year.\textsc{neut}\\
  	\glt ‘this year’ (Tatian, \citealt{BrauneReiffenstein2004}: 234)
  	\ex \label{ex:5:27:ib}
  	\gll der eino almahtico cot\\
  	the one  almighty   god.\textsc{masc}\\
  	\glt ‘the almighty God’ (Wessobrunner Gebet 7, \citealt{Schrodt2004}: 21)
  	\z
  	\z
  }

\ea%44
    \label{ex:5:44}
\ea  \label{ex:5:44a}
\gll Dieser eine Junge        hat viele  Wunder  vollbracht.\\
this      one  boy.\textsc{masc} has many miracles accomplished\\
\glt ‘This one boy has performed many miracles.’
\ex  \label{ex:5:44b}
\gll alle Frauen wollen plötzlich   nur   das (/dies(es)) eine: Zärtlichkeit\\
all   women want    suddently only that {\db}/this one tenderness.\textsc{neut}\\
\glt ‘Suddenly, all women only want one thing: tenderness.’
\ex  \label{ex:5:44c}
\gll Du  bist der eine Mann,        der mich geliebt hat.\\
you are the  one  man.\textsc{masc} that me    loved   has\\
\glt ‘You are the one man that has loved me.’
\ex\label{ex:5:44d}
\gll Meine eine Freundin          und seine zwei sind ausgegangen.\\
my      one girl-friend.\textsc{fem} and his     two  are   {gone out}\\
\glt ‘My one girl-friend and his two went out.’
\z
\z

To be clear, then, although \textit{eine} has adjectival morphology in each case and is presumably the same element, it may lack the presupposition property under certain conditions (also M. \citealt{Müller1986}: 45).

I tentatively suggest for \REF[a-b]{ex:5:44} that demonstrative determiners, with their strong deictic force, cancel the presupposition of adjectival \textit{eine}. Furthermore, note that \textit{eine} does not seem to be (strongly) stressed in \REF{ex:5:44a}, where in contrast to \REF{ex:5:44b}, the head noun is present. This presumably is also a reflex of the presence of word stress on the preceding demonstrative. Possessives as in \REF{ex:5:42b} and \REF{ex:5:43} have been related to demonstratives in that possessives are also deictic elements. For instance, \textit{meine} ‘my’ can be compared to \textit{diese} ‘this’ such that both are associated with the speaker (see \citealt{Lyons1999}: 18-19, \citealt{Roehrs2019}: 378). This presumably explains why the duality presupposition can be absent with possessives (it is less clear to me though why this presupposition can sometimes be present with possessives).\footnote{The possible presence of a duality presupposition with possessives might be related to the fact that possessives do not presuppose singleton sets (\citealt{CoppockBeaver2015}).}  As to \REF{ex:5:44c}, I assume that the duality presupposition of adjectival \textit{eine} is also cancelled when the nominal containing \textit{eine} forms the predicate of a singular subject. Finally turning to \REF{ex:5:44d}, I assume that the coordination of the two nominals leads to a list-type effect where the different numbers of the people involved are contrasted. This resultant contrast also allows a singularity reading of adjective \textit{eine}.

To sum up, I proposed that \textit{eine} is an adjective that can only occur in definite contexts. I assume that this distributional restriction has to do with \textit{eine}’s own (contribution to the) duality presupposition such that presuppositions in general often arise only in definite contexts. With these remarks in mind, I turn to the feature specifications.

\subsubsection{Feature specifications of the different types of \textit{ein}}\label{sec:5.4.3.2}

As documented in \chapref{sec:1}, \sectref{sec:1.2.2}, \textit{ein} as an article can occur not only in singular but also in plural contexts \REF{ex:5:45}. Note that in each of the two cases below, there is no duality presupposition. This is expected given that these are indefinite contexts.

\ea%45
    \label{ex:5:45}
\ea[\%]{ \label{ex:5:45a}
\gll ``Was   für eine Idioten'' sagte Liam\\
  {\db}what for  a     idiots      said  Liam\\
\glt ‘“Such idiots!”, said Liam.’
}
\ex[\%]{ \label{ex:5:45b}
\gll Das Personal war sehr unfreundlich. Ich hatte noch nie     so eine Ferien.\\
    the  staff         was very unfriendly     I     had    still  never so a      holidays\\
\glt ‘The staff was very unfriendly. I have never had such a (bad) vacation.’
}
\z
\z


Again, it is clear that \textit{ein} can appear in plural contexts. Given the proposal that this element is a semantically vacuous element, I assume that \textit{ein} can be plural as regards its morphology.

In contrast, adjectival \textit{eine} has semantics regarding number as just discussed. Thus, I make a distinction between morphological and semantic number (also \chapref{sec:7}), morph vs. sem for short. Consequently, I assume two types of plural features [PL] in the feature makeup where $\alpha$ may range over a negative or positive value.\footnote{With \chapref{sec:2} in mind, plural actually consists of the features [+F, +N]. The difference between plural and singular is that in the plural, both $\alpha$ and $\beta$ in [$\alpha$F, $\beta$N] are valued positively but in the singular, $\alpha$ and/or $\beta$ is valued negatively (the latter was captured by a category variable: [--$\gamma$]). For simplicity, I use the feature [±PL] moving forward.}

\ea%46
    \label{ex:5:46}

          Number:  [$\alpha$PL morph; $\alpha$PL sem]
\z

  Starting with the feature specification of the indefinite article, I proposed that \textit{ein} is a semantically vacuous element. To review briefly, this element can occur in singular and plural contexts as well as in indefinite and definite environments. As such, this element cannot have a specification for semantic number or definiteness.\footnote{A reviewer points out that \textit{ein} occurring in plural contexts is not a strong argument that \textit{ein} is not associated with singularity, given that in some languages the plural seems to operate on the singular. As discussed in S. \citet[Chapter 2]{Grimm2012}, there are two cases as regards nouns: Count nouns can be pluralized as in English, and collective nouns can be singularized and then pluralized as in Welsh (incidentally, note also that plural can be subtractive in Murle, see \citealt{HaspelmathSims2010}: 37, or in Hessian German). Schematically, these two cases could be represented as follows: Stem+PL and Stem+SG+PL. It could be claimed then that -\textit{e} in plural \textit{ein} spells out the plural part and that \textit{ein} realizes the Stem (and singular).  There are certain points that seem to militate against this. First, the observation above was made on the basis of nouns. While singular \textit{ein} has traditionally been related to the feature [COUNT], it has not been associated with the feature [COLLECTIVE] (definite articles in German seem to be neutral in that they can occur in both of these singular contexts). Second, note that the plural form of the indefinite article is not morphologically more complex than its singular counterpart. In fact, both exhibit the same complexity and can even be identical (e.g., in the nominative feminine and plural: \textit{eine}). Third, if one or both of the two schematic structures above, Stem+PL and Stem+SG+PL, is or are generally available, then we need to find a way to rule out spell-out forms such as *\textit{de-e} or *\textit{de-r-e} in the plural (nominative masculine) and *\textit{da-e} or *\textit{da-s-e} in the plural (nominative neuter). As far as I can see, these schematic structures do not capture the fact that in German, genders in the singular are not relevant in the plural. Finally, it remains unclear why plural \textit{ein} cannot occur in all plural contexts in Standard German.} Furthermore, given the composite analysis of the possessive and negative articles, which can each occur with a mass noun (e.g., \textit{meine / keine Milch} ‘my / no milk'), \textit{ein} does not bring about countability either.\footnote{In \chapref{sec:7}, I discuss in more detail that countability in the current analysis is due to a side-effect, which is related to number agreement mitigated by the head of NumP.} Thus, this element has no feature for countability. However, we have seen that \textit{ein} morphologically agrees with its singular or plural head noun. I conclude again that this element must have morpho-syntactic features. Above, I proposed that \textit{ein} involves the feature [+D] and a feature bundle for case, number, and gender.

\begin{figure}
	\caption{Structure and spell-out of the indefinite overt article \emph{ein}}
	\label{figex:5:47}
	\begin{forest}
		[,phantom
		[Art\textsubscript{\parbox{0mm}{\mbox{[+D][F, N, O, S]}}}
			[{[+D]}]
			[{[F, N, O, S]}, tier=A]
		]
		[$\rightarrow$  \textit{ein-}, no edge, tier=A]
		]
	\end{forest}
\end{figure}

%\ea%47
%    \label{ex:5:47}
%
%          \textit{Indefinite Overt Article}
%
%    
%%%[Warning: Draw object ignored]
%%%[Warning: Draw object ignored]
%        $\rightarrow$  \textit{ein-}
%\z

The CNG features in \figref{figex:5:47} are valued/checked during the derivation. Abstracting away from case and gender, morphological number has to do with [F, N], here more simply restated as [$\alpha$PL morph]. Thus, besides [+D], [$\alpha$PL morph] is part of the feature specification of \textit{ein}, specifically, the feature bundle CNG.\footnote{Note that the usual restriction of \textit{ein} to singular count contexts follows from a – what I call – restriction feature on the categorial feature yielding [+D: --PL] (cf. also uninflected \textit{dies} ‘this’ in \chapref{sec:4}, \sectref{sec:4.5}). This morpho-syntactic restriction feature can be deleted by the operators POSS, NEG, and certain others that are licensed in emotive/affective contexts (cf. the exclamative operators [+EXCL] and !!! dicussed in \chapref{sec:4}, \sectref{sec:4.2}). This deletion will widen the distribution of \textit{ein} allowing it to appear in certain non-singular contexts (for detailed discussion, see \chapref{sec:8}, \sectref{sec:8.2.2.5} and \ref{sec:8.2.2.6}.}

Turning to the singularity numeral \textit{EIN}, I proposed above that this element is a composite form consisting of vacuous \textit{ein} and the numeral component $\emptyset$\textsubscript{[--PL]}. I suggested that $\emptyset$\textsubscript{[--PL]} is the contentful part, a type of operator, and I assume that it is this component that is responsible for singularity and stress (cf. \citealt{Barbiers2005}: 168). I propose that $\emptyset$\textsubscript{[--PL]} is inherently specified for [--PL] semantically. Given that $\emptyset$\textsubscript{[--PL]} and vacuous \textit{ein} form a composite, I take it that morphological number comes from \textit{ein} as specified above. Now, since the semantic number and the morphological number of \textit{EIN} have to match, I assume that there is a condition such that semantically singular $\emptyset$\textsubscript{[--PL]} is only compatible with morphological singular contexts. The number specification for $\emptyset$\textsubscript{[--PL]} can be stated as [--PL sem, where $\alpha$PL morph = $\alpha$PL sem]. I discuss the syntactic structure of \textit{EIN} in \sectref{sec:5.5.1.2}.

  Adjectival \textit{eine} is different. Like \textit{ein}, I suggest that \textit{eine}\textsubscript{ADJ} has an unvalued/un-\linebreak checked feature for morphological number. Unlike \textit{ein}, it is an adjective, and it has a specification for semantic number as regards the cardinality of sets (n) presupposed. This can be stated as follows: [$\alpha$PL morph; n = 2 sem, where n is the number of presupposed sets of entities]. Morphological number is encoded in the CNG feature bundle, and the duality presupposition is located with the adjectival stem \textit{ein}, see \figref{figex:5:48}.

\begin{figure}
	\caption{Structure of adjectival \emph{eine}}
	\label{figex:5:48}
	\begin{forest}
		[InflP
			[{[F, N, O, S]}]
			[AP [A\\\textit{ein}]]
		]
	\end{forest}
\end{figure}

Abstracting away from the categorial features (and case and gender), I summarize the feature specifications of the three types of \textit{ein}.

\TabPositions{1.5cm}
\ea%49
    \label{ex:5:49}
\ea\label{ex:5:49a} {ein}:\tab   [$\alpha$PL morph]
\ex\label{ex:5:49b} $\emptyset$\textsubscript{[--PL]}:\tab  [--PL sem, where $\alpha$PL morph = $\alpha$PL sem]
\ex\label{ex:5:49c} {eine}\textsubscript{ADJ}:\tab [$\alpha$PL morph; n = 2 sem unless in the context of a\\
                                              \tab demonstrative, etc.]
\z
\z

To be clear, given the complex internal structure of determiners and adjectives (i.e., stem + inflection), the semantic specifications in \REF{ex:5:49} are assumed to be on the stems of the relevant elements, but the morphological ones are encoded in the inflections as unvalued/unchecked feature bundles. With this in mind, I briefly address the statement in \REF{ex:5:49b} again.

There is an interesting interaction between \REF{ex:5:49a} and \REF{ex:5:49b}. As proposed in more detail below, the numeral \textit{EIN} derives from the combination of \REF{ex:5:49a} and \REF{ex:5:49b}, where the stem of \textit{ein} adds the categorial feature [+D], the inflection on \textit{ein} supplies unvalued/unchecked morphological features for case, number, and gender, and $\emptyset$\textsubscript{[--PL]} imposes the restriction that only singular number on other elements in the larger noun phrase will be compatible with $\emptyset$\textsubscript{[--PL]} itself. This means that in order to bring about a good derivation, the number features of all those elements must be valued/checked for singular by the more general operation that brings about concord in agreement features. If those number features are valued/checked for plural, incompability with the number specification of $\emptyset$\textsubscript{[--PL]} arises.

  Finally, I suggested above that the wider linguistic context (e.g., different\linebreak moods in the hosting clause) licenses the different readings of indefinites with regard to specificity. I proposed that this involves different positions of \textit{ein} inside the DP. If these claims are on the right track, then we can maintain the claim that \textit{ein} is semantically vacuous (Hypothesis 1a).

\section{Step 2 of the proposal: Syntax}\label{sec:5.5}

In the first subsection, I propose in more syntactic detail that synchronically, the numeral \textit{EIN} is, in certain respects, based on the indefinite article even though diachronically, the article derived from the numeral. With the different specifications from above in mind, the remaining differences between \textit{EIN} and \textit{ein} are argued to follow from the two different positions that $\emptyset$\textsubscript{[--PL]} and \textit{ein} occupy in the syntactic tree. In the second subsection, adjectival \textit{eine} is shown to be independent of the indefinite article and numeral, and occupies a third position.

\subsection{Article vs. numeral}\label{sec:5.5.1}

In this subsection, I first provide two pieces of evidence that the numeral \textit{EIN}, specifically the contentful component of it, is in a different position than the indefinite article. Then I proceed to derive certain aspects of the numeral from the indefinite article.

\subsubsection{Uniform position(s) of all numerals}\label{sec:5.5.1.1}

Recall from \chapref{sec:1} that numerals are in a position different from determiners or determiner-like elements. Straightforward evidence for this comes from the following cases.

\ea%50
    \label{ex:5:50}
\ea\label{ex:5:50a}
\gll die  zwei Freunde\\
\textsc{det} two  friends\\
\glt ‘the/those two friends’
\ex\label{ex:5:50b}
\gll Peters  zwei Freunde\\
Peter’s two  friends\\
\glt ‘Peter’s two friends’
\z
\z

I assumed that numerals are in Spec,CardP. Following previous work though, I also suggested that if there is no determiner, then numerals may be in Spec,DP yielding a specific interpretation. The picture is slightly more complicated with \textit{ein}-words, specifically the numeral \textit{EIN} and the article \textit{ein}.

In the previous section, I discussed the different interpretations of nominals involving \textit{ein} with regard to specificity. In particular, I proposed that when the interpretation is non-specific, \textit{ein} is in Card, but when it is specific, \textit{ein} is in D. In other words, \textit{ein} in \REF{ex:5:51a} may be in two different positions. Furthermore, I proposed that \textit{EIN} consists of the singularity part $\emptyset$\textsubscript{[--PL]} and vacuous \textit{ein}. As regards the different specificity interpretations, I pointed out that when the interpretation is non-specific, $\emptyset$\textsubscript{[--PL]} is in Spec,CardP and supporting \textit{ein} is in Card, but when it is specific, $\emptyset$\textsubscript{[--PL]} is in Spec,DP and \textit{ein} is in D. In other words, just like other numerals, $\emptyset$\textsubscript{[--PL]} may be located in two phrasal positions in \REF{ex:5:51b}.\footnote{Despite certain differences between the singularity numeral and the other numerals, \citet[171]{Barbiers2005} also assumes the same position for all numerals (see also his Footnote 35). For typological discussion of numerals, see \citet{Corbett2000} and \citet{Hurford2003}.}

\ea%51
    \label{ex:5:51}
\ea \label{ex:5:51a}
\gll (ei)n Freund\\
  {\db}a     friend.\textsc{masc}\\
\glt ‘a friend’
\ex \label{ex:5:51b}
\gll EIN Freund\\
one  friend.\textsc{masc}\\
\glt ‘one friend’
\z
\z

Regarding \textit{ein}-words then, the relevant elements are in two positions involving phrases (Spec,CardP and Spec,DP for $\emptyset$\textsubscript{[--PL]}) and in two positions involving heads (Card and D for \textit{ein}). There are two issues that need to be addressed now.

First, if the article and the numeral (or a part of it) are in different positions, then we need to explain why the two cannot co-occur \REF[a-b]{ex:5:52}. This is presumably not due to semantic reasons as other (adjectival) elements emphasizing singularity are possible \REF[c-d]{ex:5:52}. In fact, under the right (emotive) conditions, \textit{ein} can occur with a non-singularity numeral \REF{ex:5:52e}, repeating an example from \chapref{sec:1}, \sectref{sec:1.2.2} here (for the latter distribution in Dutch, see \citealt{BennisEtAl1998}: 112).\footnote{As
    is expected from the above discussion, adjectival \textit{eine} is also ungrammatical following \textit{ein} \REF{ex:5:34:ia} as the former is only licensed in a definite context \REF{ex:5:34:ib}. 
    \ea
    \ea[*]{ \label{ex:5:34:ia}
    \gll  ein einer Bruder\\
    an  one    brother.\textsc{masc}\\
    }
    \ex[]{ \label{ex:5:34:ib}
    \gll Peters  einer Bruder\\
    Peter’s one    brother.\textsc{masc}\\
    \glt ‘Peter’s one brother’\\
    ‘one of the two of Peter’s brothers’
    }
    \z
    \z
    
    Furthermore, an indication that adjectival \textit{eine} is different from \textit{einziger} ‘only’ can be seen in Negative Polarity contexts. Note first that neither a definite nor an indefinite article can license \textit{je} ‘ever’ \REF{ex:5:34:iia}. However, adding \textit{einziger}, but not \textit{eine}, after the definite article results in a completely fine example \REF{ex:5:34:iib}.

    \ea
      \ea  \label{ex:5:34:iia}
      \gll * Du  bist \{der / ein\} Mann,       der mich je     geliebt hat.\\
     {} you are   {\db}the  / a     man.\textsc{masc} that me   ever loved   has  \\

      \ex \label{ex:5:34:iib}
      \gll Du  bist der \{einzige / ??eine\} Mann,         der mich je     geliebt hat.\\
      you are  the   {\db}only     /     {\db\db}one    man.\textsc{masc} that me   ever loved   has  \\
      \glt ‘You are the only man that has ever loved me.’
      \z
      \z

      Finally, another indication that \textit{eine} is different from \textit{einziger} ‘only’ is that both elements can cooccur \REF{ex:5:34:iii} (\REF{ex:5:34:iii} is from \citealt{ZifonunStrecker1997}: 1931; to my ears, this example sounds slightly marked without a pause between \textit{eine} and \textit{einzige}).  
      \ea \label{ex:5:34:iii}
      \gll der eine einzige Gegensatz\\
      the one  unique  contrast.\textsc{masc}\\
      \glt ‘the only contrast’
      \z
}

\ea%52
    \label{ex:5:52}
  \ea[*]{  \label{ex:5:52a}
  \gll (ei)n EIN Freund\\
    {\db}a     one  friend.\textsc{masc}\\
    }
  \ex[*]{  \label{ex:5:52b}
  \gll EIN (ei)n Freund\\
  one   { }a     friend.\textsc{masc}\\
  }
  \ex[]{  \label{ex:5:52c}
  \gll (ei)n einziger Freund\\
    {\db}a     sole        friend.\textsc{masc}\\
  \glt ‘a sole friend’
  }
  \ex[]{  \label{ex:5:52d}
  \gll (ei)n einzelner  Mann\\
    {\db}an   individual man.\textsc{masc}\\
  \glt ‘a individual man’
  }
  \ex[\%]{  \label{ex:5:52e}
  \gll ach man wir sind schon so     ne zwei wärmflaschen\\
  oh  boy   we are   \textsc{prt}     such a   two  warm.bottles\\
  \glt ‘Oh boy, we are such two hot-water bottles.’
  }
  \z
\z

The ungrammaticality of \REF[a-b]{ex:5:52} follows from the assumptions above. Recall that \textit{ein} originates in Art and undergoes movement to Card or D. Proposing that $\emptyset$\textsubscript{[--PL]} and \textit{ein} are spelled out together as \textit{EIN} under adjacency explains why an indefinite article cannot co-occur with the singularity numeral but only with other (non-composite) numerals \REF{ex:5:52e}. I return to the discussion of \REF{ex:5:52e} at the end of this section. Adjectives like \textit{einziger} ‘sole’ \REF{ex:5:52c} and \textit{einzelner} ‘individual’ \REF{ex:5:52d} can occur with \textit{ein} as they are elements morpho-syntactically independent of \textit{ein}.

  Turning to the second issue, I documented above that the indefinite article and the singularity numeral have the same morphology. What is interesting to note is that under certain conditions, the German numerals for ‘two’ and ‘three’ can take endings in the genitive. In this instance, these two numerals and the following adjective have identical inflections \REF{ex:5:53a}. This is in stark contrast to \textit{EIN}, which does not have the same ending as the following adjective \REF{ex:5:53b}.

\ea%53
    \label{ex:5:53}
\ea \label{ex:5:53a}
\gll das Auto zwei-er       nett-er  Freunde\\
the car    two-\textsc{st.gen} nice-\textsc{st} friends\\
\glt ‘the car of two nice friends’
\ex \label{ex:5:53b}
\gll das Auto EIN-ES       nett-en   Freundes\\
the car    one-\textsc{st.gen} nice-\textsc{wk} friend.\textsc{neut}\\
\glt ‘the car of one nice friend’
\z
\z

Assuming that all numerals are in the same positions (Spec,CardP or Spec,DP), we need to explain why the numerals for ‘two’ and ‘three’ have a different morphological impact on the following adjective than \textit{EIN}.

  Historically, the indefinite article derives from the numeral for ‘one’. Importantly, the inflectional properties of the singularity numeral have changed over time. In particular, OHG \textit{éin} behaves like Modern German \textit{zwei} ‘two’ and \textit{drei} ‘three’ but crucially not like Modern German \textit{EIN}. Specifically, \textit{éin} takes a strong adjective like \textit{zwei} and \textit{drei}, but \textit{EIN} takes a weak one. Compare the dative strings in \REF{ex:5:54a} and \REF{ex:5:54b} (the (a)-example is taken from \citealt{Demske2001}: 76).

\ea%54
OHG
    \label{ex:5:54}
\ea \label{ex:5:54a}
\gll mít   éin-emo rôt-emo tûoche    \\
  with one-\textsc{st}   red-\textsc{st}   scarf.\textsc{neut}\\
\glt ‘with one red scarf’
\ex \label{ex:5:54b}
\gll mit   EIN-EM rot-en  Tuch\\
  with one-\textsc{st}   red-\textsc{wk} scarf.\textsc{neut}\\
\glt ‘with one red scarf’
\z
\z

It is clear, then, that the singularity numeral in these two varieties of German is, in certain ways, different and should not receive the same account.

Recall that diachronically, the indefinite article derives from the numeral for ‘one’. On a somewhat speculative note, we could suggest that in the development from OHG to Modern German, \textit{éin} split into two parts: (vacuous) \textit{ein} + $\emptyset$\textsubscript{[--PL]}. Now, once these two separate elements are available, the indefinite article can occur with $\emptyset$\textsubscript{[--PL]} (yielding the singularity numeral), but it can also start occurring by itself. In other words, the null part $\emptyset$\textsubscript{[--PL]} can also be left out resulting in the emergence of the indefinite article. If we make this assumption, then it is not implausible to suggest that synchronically, Modern German \textit{EIN} is based on the indefinite article, at least in certain respects. This, in turn, provides an explanation of why Modern German \textit{EIN} has the morphological properties of a determiner – it consists in part of vacuous \textit{ein}. The latter element may trigger Impoverishment, and this accounts for the weak adjectives in \REF{ex:5:53b} and \REF{ex:5:54b}. \textit{EIN} has the semantic properties of a numeral because it involves $\emptyset$\textsubscript{[--PL]}. In \sectref{sec:5.6}, I return to the diachronic split of \textit{éin} into two components suggesting that the same occurred with possessive and negative articles – the other two complex \textit{ein}-words.

More generally, these morpho-syntactic and semantic differences indicate different positions of the two instances of \textit{ein}, the article and the numeral. Morpho-syntactically, adjectives following \textit{EIN} behave differently from those following other numerals. Assuming a composite analysis of \textit{EIN}, it is \textit{ein}, but not $\emptyset$\textsubscript{[--PL]} or other numerals, that trigger Impoverishment. Note in this regard that \textit{ein} undergoes head movement from Art up the tree. Crucially though, it cannot move into Spec,CardP, which is occupied by $\emptyset$\textsubscript{[--PL]} (see also next section). Semantically, \textit{EIN} involves singularity unlike \textit{ein}. If we assume that numerals are in Spec,CardP and specify the cardinality of a nominal, then that implies that \textit{ein} is in a different position. To be clear, like other articles, \textit{ein} is in a head position; like other numerals, $\emptyset$\textsubscript{[--PL]} is in a phrasal position.

\subsubsection{Different scope of \textit{mehr als}}\label{sec:5.5.1.2}

Above, I proposed that articles are in head positions but that numerals are in specifier positions. The second piece of evidence that the article and the numeral are in different positions derives from scopal facts. To begin, \textit{mehr als} ‘more than’ can take scope over the entire noun phrase where the nuclear stress is on the noun \REF{ex:5:55}. In more detail, \textit{mehr als} in \REF[a-b]{ex:5:55} implies that the relevant person is more than just a student (perhaps he is also the speaker’s friend), and \textit{mehr als} in \REF{ex:5:55c} implies that not only exactly the one hundred students came but perhaps other students showed up or even other people that are not students.

\ea%55
    \label{ex:5:55}
\ea\label{ex:5:55a}
\gll Er ist [mehr als (ei)n Student].\\
    he is    {\db}more than {\db}a      student.\textsc{masc}\\
\glt ‘He is more than a student.’
\ex\label{ex:5:55b}
\gll Er ist [mehr als mein Student].\\
    he is    {\db}more than my    student.\textsc{masc}\\
\glt ‘He is more than my student.’
\ex\label{ex:5:55c}
\gll Es     kamen [mehr als   die hundert Studenten].\\
    there came     {\db}more than the one.hundred students\\
\glt ‘More than the one hundred students came.’
\z
\z

These scopal effects can be derived as follows.

Illustrating with \REF{ex:5:55b}, the possessive element has undergone movement to Spec,DP, and the article has moved to D. Consider the relevant portion of the syntactic tree in \figref{figex:5:56}. Since Spec,DP is occupied, the scopal element must be outside the DP proper. For concreteness, I assume that it is adjoined to the DP.

\glltree[\label{figex:5:56}]{
	\gll mehr als mein Student\\
            more than my student\\
    \glt ‘more than my student’
}{
		[DP
			[\textit{mehr als}]
			[DP
				[PossP \\\textit{m}-\textsubscript{$k$}]
				[D$'$
					[D\\\textit{ein}\textsubscript{$i$}]
					[ArtP\\{[t\textsubscript{$i$} t\textsubscript{$k$} \textit{Student}]}]
				]
			]
		]
}

To be clear, \textit{mehr als} c-commands the entire DP. It takes scope over the possessive and the head noun allowing the various interpretations.

  In contrast, \textit{mehr als} may also take scope over numerals only \REF{ex:5:57}. For the following sentences to be true, it must hold for \REF{ex:5:57a} that at least two students came and for \REF{ex:5:57b} that at least one hundred and one did.

\ea%57
    \label{ex:5:57}
\ea \label{ex:5:57a}
\gll Es     kam [mehr als  EIN] Student.\\
there came {\db}more than one   student.\textsc{masc}\\
\glt ‘More than one student came.’
\ex \label{ex:5:57b}
\gll Es     kamen die [mehr als   HUNDERT] Studenten.\\
there came   the   {\db}more than one.hundred  students\\
\glt ‘The more than one hundred students came.’
\z
\z

Returning briefly to the first set of data from this section, note that \textit{mehr als} in \REF{ex:5:55} occurs in the left periphery of the DP: It immediately precedes \textit{ein}-words as well as the definite article. In contrast, \textit{mehr als} in \REF{ex:5:57} immediately precedes the numerals – this is particularly clear in \REF{ex:5:57b}, where \textit{mehr als} follows the definite article. In order to derive the different scopal readings in \REF{ex:5:55} vs. \REF{ex:5:57}, I propose that \textit{mehr als} is in a different position when it precedes the numerals in \REF{ex:5:57}. This, in turn, will provide an argument that the two instances of \textit{ein} (article vs. numeral) are in different positions. I consider this in more detail.

  To repeat, the scopal element \textit{mehr als} immediately precedes the numerals in \REF{ex:5:57}. Given the readings above, it only affects the interpretation of the numerals. In order to find a plausible structure, I follow \citegen[445-46]{Svenonius1993} argumentation for adjectives in that the modifier and its modifiee must be embedded inside a specifier position. This prevents the modifier from taking scope over the entire noun phrase. If we assume that the same holds for modifiers and their numeral modifiees, then numerals are also in specifier positions in this case (rather than in a head positions in the extended projection of the noun), and their modifiers (including scopal elements) are part of the same specifiers. Again, I assume that this position is Spec,CardP (or after movement Spec,DP). I continue focusing on the singularity numeral.

With this in mind, recall that the numeral \textit{EIN} consists of vacuous \textit{ein} and contentful $\emptyset$\textsubscript{[--PL]}. The article \textit{ein} undergoes head movement from Art to Card, see \figref{figex:5:58}. As to $\emptyset$\textsubscript{[--PL]}, this component is responsible for singularity and stress (but has no independent segmental phonetic realization). I propose that $\emptyset$\textsubscript{[--PL]} and the scopal element \textit{mehr als} are both part of a specifier, below illustrated with the specifier of CardP. Embedded in a specifier, \textit{mehr als} cannot take scope over the entire noun phrase. Some further remarks are in order.

\glltree[\label{figex:5:58}]{
	\gll mehr als EIN Student\\
            more than one student\\
    \glt ‘more than one student’
}{
		[DP
			[~]
			[D$'$
				[D]
				[CardP
					[QP
						[\textit{mehr als}]
						[Q\\$\emptyset$\textsubscript{[--PL]}]
					]
					[Card$'$
						[Card\\\textit{ein}\textsubscript{$i$}]
						[ArtP
							[~]
							[Art$'$
								[Art\\\sout{ein}\textsubscript{$i$}]
								[NumP\\\textit{Student}]
							]
						]
					]
				]
			]
		]
}

Note again that the indefinite article has a feature for [+D], but not for definiteness. Consequently, \textit{ein} does not have to move to the DP-level but can surface in Card. Indeed, on the current analysis, $\emptyset$\textsubscript{[--PL]} and \textit{ein} are separate elements in different positions: Given standard assumptions about movement, \textit{ein} as a head moves to Card, but it cannot move into a specifier position, presumably to combine with $\emptyset$\textsubscript{[--PL]}. However, there is another way to form a composite of these two elements. 

Linearization, Vocabulary Insertion, and Copy Reduction of the relevant portion of \figref{figex:5:58} yields \figref{figex:5:59}. 

\begin{figure}
	\caption{Spell-out of the singularity numeral \textit{EIN}}
	\label{figex:5:59}
	\begin{forest}
		[
			[$\emptyset$\textsubscript{[--PL]}]
			[\textit{ein}]
			[(-INFL)]
		]
	\end{forest}
\end{figure}

Now, null elements have to be licensed in some way (\citealt{Kester1996a,Kester1996b}; \citealt{Lobeck1995}; \citealt{Rizzi1986}; but also \citealt{Murphy2018}). This also applies to $\emptyset$\textsubscript{[--PL]}. I proposed above that \textit{ein} licenses $\emptyset$\textsubscript{[--PL]} by supporting it – both elements are adjacent to each other in \figref{figex:5:59}. The operation of support can be instantiated by Local Dislocation (\sectref{sec:5.4.1.2}). Thus, as with the possessive and negative articles, the two relevant elements form a composite. This derives, in certain respects, the numeral from the indefinite article or, put differently, the numeral is based in part on the indefinite article.\footnote{Several authors (e.g., \citealt{Bernstein1993}: 128, \citealt{Julien2002}: 274) share the intuition that the indefinite article is merged lower and then raises to the DP-level. In contrast to the text proposal, these authors basically derive the indefinite article from the numeral. As far as I can see, there are a number of issues with this type of account, at least for Modern German (for some issues in English, see \citealt{Perlmutter1970}: 239 fn. 10, 244 fn. 13): for instance, it is unclear to me how to account for the different morphological properties of \textit{ein} and \textit{EIN} vs. non-singularity, non-composite numerals as regards the inflections on the following adjectives (cf. \REF{ex:5:53} above) – if \textit{EIN} were a non-composite numeral, then the adjectives following \textit{EIN} and the (other) non-composite numerals should pattern the same in German, contrary to fact. Furthermore, deriving \textit{ein} from the singularity numeal does not explain why \textit{ein} can co-occur with other numerals under certain conditions (assuming that only one numeral may occupy Spec,CardP).}

I summarize the discussion of scope. Unlike in \figref{figex:5:56}, in \figref{figex:5:58} \textit{mehr als} is part of a specifier inside the DP c-commanding the numeral \textit{EIN}, or more precisely, the semantically active part $\emptyset$\textsubscript{[--PL]}. In contrast, \textit{ein} was argued to be in Card. As such, similar to \sectref{sec:5.5.1.1}, these scopal data along with standard assumptions about movement present clear evidence that $\emptyset$\textsubscript{[--PL]} is in a different location from \textit{ein}. Finally, recall that, in order to derive a strong reading of the numeral in \figref{figex:5:58}, I assume that the article and $\emptyset$\textsubscript{[--PL]} (or rather QP) move to D and Spec,DP, respectively; morphological spell-out occurs as in \figref{figex:5:59}.

To take stock thus far, deriving the Modern German singularity numeral from the combination of $\emptyset$\textsubscript{[--PL]} and the vacuous article has a number of advantages. On the one hand, we can put all numerals in the same phrasal position(s), and we can explain the different scopal effects of \textit{mehr als} ‘more than’ on the various \textit{ein}-words and their related larger noun phrases. On the other hand, we can account for the facts that \textit{EIN} inflects like the indefinite article, that both of these elements cannot co-occur, and that both elements can take a weak adjective. In other words, the splitting of the numeral into two underlying parts and their subsequent composite spell-out accounts for the hybrid properties of \textit{EIN}.\footnote{A composite analysis could also explain the alternation between the indefinite article and the numeral for ‘one’ in some other languages closely related to Standard German.
	\ea
	\ea a(n) vs. one \hspace*{.8cm}(English)
	\ex a(n) vs. eyn \hspace*{.8cm}(Yiddish)
	\ex een \hspace*{.07cm}vs. één \hspace*{.8cm}(Dutch)
	\ex a \hspace*{.4cm}vs. õa \hspace*{.93cm}(Bavarian German)
	\z
	\z
   
   In some of these cases, the spell-out of the relevant component(s) results in quite different surface forms.} To highlight the special status of \textit{EIN} further, consider the interaction between \textit{ein} and other numerals in more detail.

Returning briefly to the co-occurrence of \textit{ein} and non-singularity numerals, consider the examples in \REF{ex:5:60}, where \REF{ex:5:60a} contains plural \textit{ein} and \REF{ex:5:60b} involves the negative article in a plural context.

\ea%60
    \label{ex:5:60}
\ea[\%]{ \label{ex:5:60a}
\gll   und so ne zwei Kämpfer die   alles umpflügen,\\
 and so a  two  fighters   who all     under.plough\\
\glt ‘and such two guys, who create such chaos’
}
\ex[]{ \label{ex:5:60b}
\gll Es waren k-eine zehn Leute  da.\\
it   were   \textsc{neg}-a  ten   people there\\
\glt ‘There were not even ten people.’
}
\z
\z

These distributions follow from the fact that unlike \textit{EIN}, other numerals are not composite forms – they are not morphologically related to \textit{ein}. Consequently, the non-composite numerals and the article \textit{ein} can co-occur. As to the structure, I suggest that the numerals in \REF{ex:5:60} are in Spec,CardP and that \textit{ein} is in a higher head position (D).

\subsection{Article / numeral vs. adjective} \label{sec:5.5.2}

Above, I derived some of the morpho-syntactic aspects of the numeral from the indefinite article. Among others, this accounted for the fact that the numeral and the indefinite article may not (separately) co-occur although they are in different positions (specifier vs. head). This now makes the prediction that if a form of \textit{ein} were to occur with a determiner, then this \textit{ein} could not be the indefinite article (or the numeral). We will see again that such distributions are indeed possible, but I argue that this instance of \textit{ein} is adjectival. In what follows I provide more evidence for this categorically different \textit{ein} suggesting that it is in a different position.

\subsubsection{Different morphology}\label{sec:5.5.2.1}

Determiners do not have weak endings. There is just one exception. As discussed in \chapref{sec:3}, \sectref{sec:3.4}, some determiners may optionally have a weak ending in genitive masculine/neuter contexts. However, as seen there, this does not apply to \textit{ein}-words. Now, we saw in \sectref{sec:5.3} above that adjectival \textit{eine} can have a weak ending in singular contexts (e.g., \textit{der ein-e}) or in plural ones (e.g., \textit{die ein-en}). Thus, this element behaves morphologically like a regular adjective (and not like an \textit{ein}-word). This provides some first motivation for the current classification of \textit{ein} as regards adjective vs. article/numeral. In addition to the difference in weak inflection, there are other inflectional distinctions. I discuss this in more detail.

  As amply documented in \sectref{sec:5.2}, the indefinite article and the singularity numeral cannot have an ending in nominative masculine and nominative/accusative neuter contexts when an overt noun follows \REF[a-b]{ex:5:61}. An account of this was provided in the context of adjectival inflections in \chapref{sec:2}, \sectref{sec:2.2.2.2}. We have also seen that adjectival \textit{eine} has an inflection when a noun follows. Recall though that is has to occur in a definite context – an indefinite environment yields ungrammaticality \REF{ex:5:61c}.

\ea%61
    \label{ex:5:61}
\ea  \label{ex:5:61a}
\gll (ei)n(*-er) Sohn\\
 {\db}a*-\textsc{st}          son.\textsc{masc}\\
\glt ‘a son’
\ex  \label{ex:5:61b}
\gll EIN(*-ER) Sohn\\
one*-\textsc{st}        son.\textsc{masc}\\
\glt ‘one son’
\ex[*]{\label{ex:5:61c}
\gll  ein-er  Sohn\\
one-\textsc{st} son.\textsc{masc}\\
}
\z
\z

It is less well known that PPs can be added to the left periphery of the noun phrase (\citealt{Bhatt1990}, \citealt{Fortmann1996}, \citealt{Haider1992}, \citealt{Roehrs2020a}). This includes \textit{von}-possessives. Interestingly, this context yields the same facts as regards the morphology on the different kinds of \textit{ein}. Compare the examples in \REF{ex:5:61} to their respective counterparts below.

\ea%62
    \label{ex:5:62}
  \ea
  \gll von Peter (ei)n Sohn\\
  of    Peter   {\db}a     son.\textsc{masc}\\
  \glt ‘a son of Peter’

  \ex
  \gll von Peter EIN Sohn\\
  of    Peter one  son.\textsc{masc}\\
  \glt ‘one son of Peter’

  \ex[*]{
  \gll von Peter ein-er  Sohn\\
  of    Peter one-\textsc{st} son.\textsc{masc}\\
  }
  \z
\z

The judgments reverse with a Saxon Genitive, a context involving definiteness. Compare \REF[a-c]{ex:5:62} to \REF[a-c]{ex:5:63}. Note also that adjectival \textit{eine} has a strong ending in this context \REF{ex:5:63c}, again exhibiting similarities with a regular adjective \REF{ex:5:63d}.

\ea%63
    \label{ex:5:63}
\ea[*]{ \label{ex:5:63a}
\gll Peters (ei)n Sohn\\
Peter’s  {\db}a     son.\textsc{masc}\\
}
\ex[*]{ \label{ex:5:63b}
\gll Peters EIN Sohn\\
Peter’s one son.\textsc{masc}\\
}
\ex \label{ex:5:63c}
\gll Peters  ein-er  Sohn\\
Peter’s one-\textsc{st} son.\textsc{masc}\\
\glt ‘one of the two of Peter’s sons’\\
\glt ‘Peter’s one son’
\ex \label{ex:5:63d}
\gll Peters  groß-er Sohn\\
Peter’s big-\textsc{st}   son.\textsc{masc}\\
\glt ‘Peter’s big son’
\z
\z

As is expected, concomitant with the different distributions in (\ref{ex:5:61}-\ref{ex:5:62}) vs. \REF{ex:5:63}, there is a difference in the semantics: While there is no duality presupposition in the first two sets of data, it is present in \REF{ex:5:63c} of the third paradigm. This is confirmed by the fact that this nominal can straightforwardly be followed by \textit{Peters anderer Sohn} ‘Peter’s other son’.

The different inflections on \textit{ein} and the different semantics just discussed correlate with another morpho-syntactic distinction: The definite article can replace \textit{ein} in \REF[a-b]{ex:5:61} and \REF[a-b]{ex:5:62} but not \textit{einer} in \REF{ex:5:63c}. Consider \REF[a-b]{ex:5:64} vs. \REF{ex:5:64c}. With a definite article present in \REF[a-b]{ex:5:64}, adjectival \textit{eine} can be added to these strings yielding \REF{ex:5:64d}.

\ea%64
    \label{ex:5:64}
\ea  \label{ex:5:64a}
\gll der Sohn\\
the son.\textsc{masc}\\
\glt ‘the son’
\ex  \label{ex:5:64b}
\gll von Peter der Sohn\\
of   Peter  the son.\textsc{masc}\\
\glt ‘Peter’s son’
\ex[*]{  \label{ex:5:64c}
\gll Peters  der Sohn\\
Peter’s the son.\textsc{masc}\\
}
\ex  \label{ex:5:64d}
\gll (von Peter) der eine Sohn\\
  {\db}of   Peter   the one  son.\textsc{masc}\\
\glt ‘one of the two sons (of Peter’s)’
\z
\z

Thus far, we have seen that the indefinite article and the singularity numeral exhibit different inflections than adjectival \textit{eine} – the latter inflects like an adjective (and must appear in definite contexts). Before I continue the discussion of the inflections on the different kinds of \textit{ein}, I provide a interim summary of the data and account for the different distributions of the possessives and their following elements.

  To take stock thus far, \textit{von}-possessives can occur with the definite article, the indefinite article, the singularity numeral, but not with adjectival \textit{eine} (unless a definite article is also present). In contrast, Saxon Genitives can occur with adjectival \textit{eine}, but not with the definite article, the indefinite article, and the singularity numeral. In \citet{Roehrs2020a}, I propose that \textit{von}-possessives are outside the DP proper but that Saxon Genitives are in Spec,DP. Given this analysis, it follows that the former allows different determiners including the singularity numeral but that the latter does not (recall also from \chapref{sec:2}, \sectref{sec:2.2.3} that Saxon Genitives involve null articles). The reason why \textit{von}-possessives do not tolerate adjectival \textit{eine}, but Saxon Genitives do, is that the former do not involve a definite context but that the latter do.

  Returning to the inflections, the morphological parallelism of adjectival \textit{eine} and regular adjectives can be strengthened further. As already seen above, the indefinite article and the singularity numeral have endings different from the following adjective. This is illustrated here again in the nominative and dative neuter.

\ea%65
    \label{ex:5:65}
\ea\label{ex:5:65a}
\gll (ei)n frisch-es        Brot\\
 {\db}a     fresh-\textsc{nom.st} bread.\textsc{neut}\\
\glt ‘a loaf of fresh bread’
\ex\label{ex:5:65b}
\gll EIN frisch-es        Brot\\
one fresh-\textsc{nom.st} bread.\textsc{neut}\\
\glt ‘one loaf of fresh bread’
\ex\label{ex:5:65c}
\gll (ei)n-em   frisch-en Brot\\
 {\db}a-\textsc{dat.st} fresh-\textsc{wk} bread.\textsc{neut}\\
\glt ‘a loaf of fresh bread’
\ex\label{ex:5:65d}
\gll EIN-EM      frisch-en Brot\\
one-\textsc{dat.st} fresh-\textsc{wk} bread.\textsc{neut}\\
\glt ‘one loaf of fresh bread’
\z
\z

Above, we saw that adjectival \textit{eine} and a regular adjective can each have a strong inflection in isolation. We expect then that when both of these elements co-occur, they both have the same endings. This is indeed borne out: Both elements are weak in \REF[a-b]{ex:5:66} but strong in \REF{ex:5:66c}.

\ea%66
    \label{ex:5:66}
\ea  \label{ex:5:66a}
\gll da-s             ein-e     frisch-e   Brot\\
the-\textsc{nom.st} one-\textsc{wk} fresh-\textsc{wk} bread.\textsc{neut}\\
\glt ‘one of the two loaves of fresh bread’
\ex  \label{ex:5:66b}
\gll de-m          ein-en    frisch-en Brot\\
the-\textsc{dat.st} one-\textsc{wk} fresh-\textsc{wk} bread.\textsc{neut}\\
\glt ‘one of the two loaves of fresh bread’
\ex  \label{ex:5:66c}
\gll Peters  ein-er  lieb-er  Sohn        und Peters  ander-er lieb-er  Sohn verstehen   sich   gut.\\
Peter’s one-\textsc{st} nice-\textsc{st} son.\textsc{masc} and Peter’s other-\textsc{st}  nice-\textsc{st} son.\textsc{masc} understand \textsc{refl} well\\
\glt ‘Peter’s first nice son and Peter’s second nice son get along well.’
\z
\z

This juxtaposition of \textit{ein} and regular adjectives presents more evidence that \textit{ein} as an article or numeral in \REF{ex:5:65} is different from \textit{ein} as an adjective in \REF{ex:5:66}. I return to the parallelism of adjectival \textit{eine} and \textit{andere} ‘other’ seen in \REF{ex:5:66c} in \sectref{sec:5.5.2.4} below.

Recall from the discussion of feature specifications in \sectref{sec:5.4.3.2} that adjectival \textit{eine} involves an adjectival structure (i.e., it has InflP on the top of AP). Now, in order to derive the adjectival endings in a uniform way (compatible with \chapref{sec:2}), I propose that the structure of adjectival \textit{eine} is merged in a position similar to that of other adjectives. Presumably, this is the highest Spec,AgrP in the nominal structure (see also \citealt{Gallmann2004}: 155, \citealt{Pafel2005}: 179). The tree representation for adjectival \textit{eine} is provided in the next subsection.

\subsubsection{Co-occurrence with possessive \textit{ein}-words} \label{sec:5.5.2.2}

Above, \textit{ein} was discussed in the context of prenominal possessives like \textit{von}-\linebreak phrases and Saxon Genitives. Turning to possessive articles, it is worth pointing out again that unlike the indefinite and negative articles in \REF[a-b]{ex:5:67}, the possessive article in \REF{ex:5:67c} may occur with adjectival \textit{eine}. This includes cases where the possessive article is part of a periphrastic possessive construction as in \REF{ex:5:67d}.
\largerpage
\ea%67
    \label{ex:5:67}
\ea[*]{ \label{ex:5:67a}
\gll  (ei)ne eine Freundin\\
     {\db}an      one  girlfriend.\textsc{fem}\\
     }
\ex[*]{\label{ex:5:67b}
\gll  keine eine Freundin\\
    no      one  girlfriend.\textsc{fem}\\
    }
\ex[]{\label{ex:5:67c}
\gll meine eine Freundin\\
  my      one  girlfriend.\textsc{fem}\\
\glt ‘one of the two of my girlfriends’\\
\glt ‘my one girlfriend’
}
\ex[]{\label{ex:5:67d}
\gll Peter seine eine Freundin\\
Peter his     one  girlfriend.\textsc{fem}\\
\glt ‘one of the two of Peter’s girlfriends’\\
\glt ‘Peter’s girlfriend’
}
\z
\z

The sequence of two instances of \textit{ein} in \REF{ex:5:67a} and \REF{ex:5:67b} cannot simply be ruled out by haplology. Roughly, this describes the reduction of identical sequences of sounds (for the discussion of \REF{ex:5:67a} in this regard, see \citealt{Bhatt1990}: 201; more generally, see \citealt{NeelemanvandeKoot2006}). If this were the case, we would expect \REF[c-d]{ex:5:67} to be ungrammatical as well, similar to \REF[a-b]{ex:5:67}, but contrary to fact. The difference between \REF[a-b]{ex:5:67} and \REF[c-d]{ex:5:67} has to do with definiteness. Again, I conclude that adjectival \textit{eine} must occur in definite contexts.

  Note though that \textit{ein} preceding the noun in \REF{ex:5:67} is actually morphologically ambiguous in the feminine between the singularity numeral and the adjective (indeed, both of these elements are also stressed). Thus, it could be objected that \textit{eine} here is not the adjective. To confirm the analysis of \textit{ein} as an adjective in \REF{ex:5:67}, I consider each of the two analytical options of the \textit{ein} before the noun (i.e., numeral vs. adjective) more carefully.

  I start with the first option, where lower \textit{ein} could be taken to be the numeral \textit{EIN}. Consider cases in the nominative masculine, where the morphology clearly distinguishes between \textit{ein}-words and adjectives. As can be seen in \REF{ex:5:68}, these instances with the numeral \textit{EIN} are ungrammatical.

\ea%68
    \label{ex:5:68}
\ea[*]{  \label{ex:5:68a}
\gll   (ei)n EIN Freund\\
  {\db}a     one  friend.\textsc{masc}\\
  }
\ex[*]{  \label{ex:5:68b}
\gll   kein EIN Freund\\
 no    one  friend.\textsc{masc}\\
 }
\ex[*]{  \label{ex:5:68c}
\gll   mein EIN Freund\\
 my   one  friend.\textsc{masc}\\
 }
\ex[*]{  \label{ex:5:68d}
\gll   Peter sein EIN Freund\\
 Peter  his  one  friend.\textsc{masc}\\
 }
 \z
 \z


Given the ungrammaticality of \REF[a-b]{ex:5:67}, the status of \REF[a-b]{ex:5:68} may not be unexpected. However, note that unlike the grammatical cases in \REF[c-d]{ex:5:67}, their counterparts in \REF[c-d]{ex:5:68} are ungrammatical here. This means that the lower \textit{ein} in \REF[c-d]{ex:5:67} cannot involve the numeral \textit{EIN}.

In order to account for \REF{ex:5:68}, observe that there are two copies of \textit{ein} in these instances, one before \textit{EIN} and one that has – apparently – formed a composite with $\emptyset$\textsubscript{[--PL]} yielding the singularity numeral \textit{EIN} itself. In order to account for the ungrammaticality, I suggest that after Copy Reduction, only the higher copy of \textit{ein} is present. This means that (lower) $\emptyset$\textsubscript{[--PL]} cannot form a composite with \textit{ein}; that is, the distributions in \REF{ex:5:68} cannot be generated. This explains the ungrammaticality of \REF{ex:5:68}.\footnote{In \chapref{sec:8}, \sectref{sec:8.2.2.2}, we see that there are instances where a second copy of \textit{ein} can occur \REF{ex:5:37:ia}. Crucially, this second copy must be a reduced form of \textit{ein} as can be verified in \REF{ex:5:37:ib}.
	\ea
	\ea[]{ \label{ex:5:37:ia}
	\gll EINE so’ne Tochter\\
	one    {so a}   daughter.\textsc{fem}\\
	\glt ‘one such daughter’
	}
	\ex[*]{ \label{ex:5:37:ib}
	\gll  EINE so     eine Tochter\\
	one    such a      daughter.\textsc{fem}\\
	}
	\z
	\z
     
     Following \citet{Nunes2001}, I argue in \chapref{sec:8} that encliticized elements can escape Copy Reduction. Note that a reduced copy of \textit{ein} is not possible with $\emptyset$\textsubscript{[--PL]} as \textit{’n} cannot encliticize to a null element (e.g., *\textit{mein'n}). This can, perhaps most clearly, be seen in the feminine (which involves adjectival inflections, thus separating the two instances of \textit{ein} by an inflection): *\textit{meine} [$\emptyset$\textsubscript{[--PL]}]\textit{’ne Tochter} ‘my one daughter’.}

This is different for adjectival \textit{eine}, the second option for lower \textit{ein} in \REF{ex:5:67}. While this element cannot occur with an indefinite or a negative article in the nominative masculine either \REF[a-b]{ex:5:69}, it can appear with the possessives under discussion here \REF[c-d]{ex:5:69}, much like the Saxon Genitives discussed in \sectref{sec:5.5.2.1}.

\ea%69
    \label{ex:5:69}
\ea[*]{  \label{ex:5:69a}
\gll  (ei)n ein-er  Freund\\
 {\db}a     one-\textsc{st} friend.\textsc{masc}\\
 }
\ex[*]{  \label{ex:5:69b}
\gll   kein ein-er   Freund\\
no    one-\textsc{st} friend.\textsc{masc}\\
}
\ex  \label{ex:5:69c}
\gll mein ein-er Freund\\
my   one-\textsc{st} friend.\textsc{masc}\\
\glt ‘one of the two of my friends’\\
\glt ‘my one friend’
\ex  \label{ex:5:69d}
\gll Peter sein ein-er Freund\\
Peter  his  one-\textsc{st} friend.\textsc{masc}\\
\glt ‘one of the two of Peter’s friends’\\
\glt ‘Peter’s one friend’
\z
\z

This essentially yields the distribution in \REF{ex:5:67} above confirming the claim that \REF{ex:5:67} involves adjectival \textit{eine}. Note again that \REF[a-b]{ex:5:69} are ungrammatical as adjectival \textit{eine} is not in definite contexts here. No such semantic problems arise with the possessives in \REF[c-d]{ex:5:69}, which are definite in interpretation. I turn to the syntactic derivation of cases involving adjectival \textit{eine}.

  I have argued that \textit{ein} as part of the possessive article and \textit{eine} as an adjective are of different lexical categories. I can point out then that they do not stand in a relevant morpho-syntactic relation with one another. If so, it is expected that these types of \textit{ein} can co-occur. Recalling that adjectival \textit{eine} involves InfP at the top of its AP, the syntactic distribution can be illustrated as seen in \figref{figex:5:70}. This derivation also applies to more complex cases.

\glltree[\label{figex:5:70}]{
	\gll meine eine Freundin\\
        my one girlfriend\\
    \glt ‘my one girlfriend’
}{
		[DP
			[PossP\\\textit{m-}]
			[D$'$
				[D\\\textit{eine}\textsubscript{$i$}]
				[AgrP
					[InflP\\\textit{eine}\textsubscript{ADJ}]
					[Agr$'$
						[Agr\\\sout{\textit{eine}}\textsubscript{$i$}]
						[ArtP
							[~]
							[Art$'$
								[Art\\\sout{\textit{eine}}\textsubscript{$i$}]
								[NumP\\\textit{Freundin}]
							]
						]
					]
				]
			]
		]
}

Unsurprisingly, like possessive articles, \textit{diese} ‘this’ can also co-occur with adjectival \textit{eine} \REF{ex:5:71a}. As discussed in \chapref{sec:2}, \textit{diese} and \textit{meine} can co-occur as well \REF{ex:5:71b}. In fact, adjectival \textit{eine} can appear with both of these elements at the same time \REF{ex:5:71c}, and an adjective can also be added \REF{ex:5:71d} (recall that there is no duality presupposition with demonstratives).

\ea%71
    \label{ex:5:71}
\ea    \label{ex:5:71a}
\gll diese eine Freundin\\
this    one  girlfriend.\textsc{fem}\\
\glt ‘this one girlfriend’
\ex    \label{ex:5:71b}
\gll diese meine Freundin\\
this   my      girlfriend.\textsc{fem}\\
\glt ‘this girlfriend of mine’
\ex    \label{ex:5:71c}
\gll diese meine eine Freundin\\
this   my      one  girlfriend.\textsc{fem}\\
\glt ‘this one girlfriend of mine’
\ex    \label{ex:5:71d}
\gll und diese meine eine große Liebe       zu sehen\\
and this   my      one  great  love.\textsc{fem} to see\\
\glt ‘and to see this one great love of mine’ (\url{https://www.kirmesforum.de/threads/faszination-break-\%09dancer.1520/page-2})
\z
\z

On current assumptions, \textit{diese} is in Spec,DP in \REF{ex:5:71a} but in Spec,LPP in \REF[b-d]{ex:5:71}; \textit{meine} is in Spec,DP in \REF[b-d]{ex:5:71}. Thus, these distributions are expected to be possible. Conversely, if we were to assume just one type of \textit{ein}; that is, if we were to try and derive both the numeral \textit{EIN} and adjectival \textit{eine} from (vacuous) \textit{ein}, then a number of distributions would be hard to account for. This is what I focus on next.

\subsubsection{Co-occurrence of \textit{ein} and determiners revisited}\label{sec:5.5.2.3}

In this subsection, I make the point that adjectival \textit{eine} is a different element from another perspective. Recall that the indefinite article and the numeral have no ending when they appear in front of a noun in certain morpho-syntactic contexts \REF[a-b]{ex:5:72}. As stated before, a definite determiner cannot precede either of them \REF[c-d]{ex:5:72}.

\ea%72
    \label{ex:5:72}
\ea  \label{ex:5:72a}
\gll (ei)n Mann\\
 {\db}a      man.\textsc{masc}\\
\glt ‘a man’
\ex  \label{ex:5:72b}
\gll EIN Mann\\
one  man.\textsc{masc}\\
\glt ‘one man’
\ex[*]{  \label{ex:5:72c}
\gll der (ei)n Mann\\
the  {\db}one  man.\textsc{masc}\\
}
\ex[*]{  \label{ex:5:72d}
\gll der EIN Mann\\
the one  man.\textsc{masc}\\
}
\z
\z

Note that \textit{ein} with a weak inflection is possible here \REF{ex:5:73a}. Finally, recall again that Saxon Genitives can occur with \textit{ein} provided the latter has a strong inflection. Compare \REF{ex:5:73b} to \REF[c-d]{ex:5:73}.

\ea%73
    \label{ex:5:73}
\ea[*]{ \label{ex:5:73a}
  \gll (der) ein-e     Mann\\
   {\db}the   one-\textsc{wk} man.\textsc{masc}\\
\glt ‘one of the two men’
}
\ex \label{ex:5:73b}
\gll Peters  ein-er Sohn\\
Peter’s one-\textsc{st} son.\textsc{masc}\\
\glt ‘one of the two of Peter’s sons’\\
\glt ‘Peter’s one son’
\ex[*]{ \label{ex:5:73c}
\gll Peters  (ei)n Sohn\\
Peter’s  {\db}a      son.\textsc{masc}\\
}
\ex[*]{ \label{ex:5:73d}
\gll Peters  EIN Sohn\\
Peter’s one  son.\textsc{masc}\\
}
\z
\z

Consider \REF{ex:5:72} and \REF{ex:5:73} in the context of the current system.

Under current assumptions, the indefinite article in \REF{ex:5:72c} is ruled out because only one article can appear in a noun phrase. If we assume again that \textit{EIN} consists, in part, of (vacuous) \textit{ein}, then we can rule out the option of \REF{ex:5:72d} under the same assumption. In other words, these cases are not ruled out because \textit{ein} does not have a weak adjectival ending. Rather, there are two articles in the DP, \textit{der} and (vacuous) \textit{ein}, but only one of them can originate in ArtP and move to the left. Turning to \REF{ex:5:73}, if we accept the assumption that adjectival \textit{eine} is a non-composite form, then this element is predicted to occur with another article in \REF{ex:5:73a}. Similarly, if we assume that Saxon Genitives as in \REF{ex:5:73b} involve null articles (\chapref{sec:2}), then this distribution follows from the same assumptions. This means that \REF[c-d]{ex:5:73} are out as there are two articles in each case, vacuous \textit{ein} and a null article.

To be clear, then, assuming two basic types of \textit{ein}, we can explain why sometimes a form of \textit{ein} cannot occur with another article but sometimes it can, provided that form of \textit{ein} appears in a definite context. These distributions are hard to account for if we assume just one type of \textit{ein}. Indeed, if \textit{EIN} and adjectival \textit{eine} were both derived from vacuous \textit{ein}, then \REF[a-b]{ex:5:73} should be, under these assumptions, ungrammatical – there would be two articles just like \REF[c-d]{ex:5:73}. Given the different grammaticalities, this, however, does not seem to be the case.

Finally, if a different analysis than the current one is to be pursued, then assumptions about the base-generated position of \textit{ein} and its different possible landing sites need to be layed out clearly. Additionally, the assumptions about the inflections on \textit{ein} need to be spelled out in detail. As demonstrated in previous chapters, a simple surface-oriented explanation does not suffice. This means that if the inflection on \textit{ein} is invoked as an explanation for some of the patterns above, it has to be part of a (more) comprehensive account of \textit{ein} itself and adjectival inflections in general. In addition to making the different syntactic and morphological assumptions explicit, such an account also needs to capture the different semantics of the various types of \textit{ein}. As of now, I am not aware of the existence of such an alternative account. I return to the parallelism between adjectival \textit{eine} and \textit{andere} ‘other’.

\subsubsection{More evidence for adjectival \textit{eine}}\label{sec:5.5.2.4}

The claim that \textit{ein} can be an adjective is further strengthened if \textit{eine} is treated as categorially parallel to \textit{andere} ‘other’. Consider \REF{ex:5:74a}, where \textit{eine} and \textit{andere} are in different, coordinated DPs showing similar morpho-syntactic behavior. Now, as briefly discussed in \chapref{sec:2}, \sectref{sec:2.2.3}, there are certain types of adjectives – often called definite adjectives – that can license singular count nouns \REF[b-b']{ex:5:74}. The adjective \textit{andere} is different and requires the presence of a determiner \REF[c-c']{ex:5:74}.\footnote{Another indication of categorial parallelism comes from the fact that adjectival \textit{eine} can also be co-ordinated with \textit{andere}, as in the following idiom.
	\ea
	\gll der [eine oder andere] Mann\\
	the    {\db}one  or     other      man.\textsc{masc}\\
	\glt ‘some men’
	\z
	
	 This argument must be used with caution though as the relevant interpretation of \textit{eine} is different here; that is, the idiom does not denote just two people.}

\ea%74
    \label{ex:5:74}
\ea\label{ex:5:74a}
\gll Meine ein-e     Tochter  kam,   mein-e andere    nicht.\\
my      one-\textsc{wk} daughter.\textsc{fem} came, my      other-\textsc{wk} not\\
\glt ‘One of my daughters came, the other did not.’

\ex\label{ex:5:74b}
\gll Folgendes Beispiel           illustriert das.\\
following  example.\textsc{neut} illustrate  this\\
\glt ‘The following example illustrates this.’

\exi{b’.}
\gll  Das folgende  Beispiel            illustriert das.\\
the  following example.\textsc{neut} illustrate  this\\
\glt ‘The following example illustrates this.’
\ex\label{ex:5:74c}
\gll *  Anderes Beispiel           illustriert das.\\
{} other      example.\textsc{neut} illustrate  this\\

\exi{c’.}
\gll Das andere Beispiel            illustriert das.\\
the  other     example.\textsc{neut} illustrate  this\\
\glt ‘The other example illustrates this.’
\z
\z

The need for the presence of a determiner in \REF[c-c']{ex:5:74} is exactly what we saw with adjectival \textit{eine} above. In other words, both adjectival \textit{eine} and \textit{andere} ‘other’ require a determiner. Thus, both of these elements are not definite adjectives. Interestingly, the adjective \textit{andere} can also have the meaning ‘different’. This meaning is subject to a certain context, and this reveals another parallelism of \textit{eine} and \textit{andere} in the meaning of ‘other’.

\subsubsection{Different semantics: \textit{EIN} vs. \textit{eine} and \textit{andere} meaning ‘different’ vs. ‘other’}\label{sec:5.5.2.5}

In this final subsection, I make more remarks on the different semantics of \textit{EIN} and adjectival \textit{eine} and on the different readings of \textit{andere} meaning ‘different’ or ‘other’. As seen above, adjectival \textit{eine} and \textit{andere} ‘other’ are related in that they often co-occur. To set the stage for the discussion of \textit{andere} in its different meanings, I start with \textit{EIN} and adjectival \textit{eine}.

  As discussed above, the numeral \textit{EIN} involves singularity, but adjectival \textit{eine} presupposes duality. Besides this semantic difference, numerals can be modified by \textit{mehr als} ‘more than’ \REF[a-b,d]{ex:5:75} whereas adjectival \textit{eine} cannot \REF{ex:5:75c}.

\ea%75
    \label{ex:5:75}
\ea\label{ex:5:75a}
\gll [Mehr als  EIN Student] kam   zur     Party.\\
  {\db}more than one student.\textsc{masc} came to.the party\\
\glt ‘More than one student came to the party.’
\ex\label{ex:5:75b}
\gll [Mehr als   HUNDERT  Studenten] kamen zur     Party.\\
     {\db}more than one.hundred students      came   to.the party\\
\glt ‘More than one hundred students came to the party.’
\ex[*]{\label{ex:5:75c}
\gll Der [mehr als   eine] Student          kam   zur     Party.\\
  the    {\db}more than one    student.\textsc{masc} came to.the party\\
  }

\ex\label{ex:5:75d}
\gll Die [mehr als  HUNDERT] Studenten kamen zur      Party.\\
the      {\db}more than one.hundred students   came    to.the party\\
\glt ‘The more than one hundred students came to the party.’
\z
\z


Again, adjectival \textit{eine} is an element semantically different from numerals including \textit{EIN}. With this in place, I turn to the different meanings of \textit{andere}.

  As just observed, adjectival \textit{eine} cannot be modified by \textit{mehr als} ‘more than’. Something similar holds for adjectival \textit{eine} when modified by the type/degree particle \textit{so} ‘such’. Compare \REF{ex:5:75a} to \REF{ex:5:76a} and \REF{ex:5:75c} to \REF{ex:5:76b}.\footnote{Note that \REF{ex:5:75a} involves the singularity numeral \textit{EIN} but that \REF{ex:5:76a} contains the article \textit{ein}. This is due to the fact that the scopal element \textit{mehr als} ‘more than’ modifies a numeral in \REF{ex:5:75a} but that the type/degree particle \textit{so} ‘such’ is supported by \textit{ein} in \REF{ex:5:76a} (see also \chapref{sec:8}, \sectref{sec:8.2.2.2}).} Interestingly, when \textit{eine} in \REF{ex:5:76b} is replaced by \textit{andere}, the example becomes grammatical, but \textit{andere} can only be interpreted as ‘different’ (but not as ‘other’) in this context \REF{ex:5:76c}.

\ea%76
    \label{ex:5:76}
\ea  \label{ex:5:76a}
\gll so (ei)n Mann\\
  so   {\db}a     man.\textsc{masc}\\
  \glt ‘such a man/a man like that’
\ex[*]{\label{ex:5:76b}
\gll der so eine Mann\\
  the so one  man.\textsc{masc}\\
  }
\ex\label{ex:5:76c}
\gll der so andere   Mann\\
  the so different man.\textsc{masc}\\
  \glt ‘the so different man’\\
  \#‘the so other man’
\z
\z


Thus, the ungrammaticality of \REF{ex:5:76b} and the interpretative restriction in \REF{ex:5:76c} fit well with the discussion above, where I showed that adjectival \textit{eine} often co-occurs with \textit{andere} in the meaning of ‘other’. More generally, this provides additional evidence that this type of \textit{ein}, just like \textit{andere} ‘other’, is a special kind of adjective.

  To sum up this section, I discussed more evidence that there are two main types of \textit{ein}: the indefinite article and adjectival \textit{eine}. The singularity numeral \textit{EIN} is based, in part, on the indefinite article. Each of these three elements is in a different position: The indefinite article is in Card or D, the singularity numeral, that is, its semantic component, is in Spec,CardP or Spec,DP, and adjectival \textit{eine} is in a high Spec,AgrP position. The varying adjectival inflections on these different elements support the classification into the two main types. The different inflections follow from the system developed in \chapref{sec:2}, again highlighting the fact that adjectival inflections and \textit{ein} can (and should) be discussed in tandem. Before I review some previous work on \textit{ein}, I briefly return to the discussion of \textit{ein}-words as composite forms.

\section{Diachronic and cross-linguistic evidence for \textit{ein}-words as composites}\label{sec:5.6}

As mentioned in \sectref{sec:5.4.2.2}, there is an interesting asymmetry in the syntactic distribution of Saxon Genitives and possessive articles. Whereas the former can follow their head nouns, the latter cannot.

\ea%77
    \label{ex:5:77}
\ea\label{ex:5:77a}
\gll das Buch          Maximilians\\
the book.\textsc{neut} Maximilian’s\\
\glt ‘Maximilian’s book’
\ex[*]{\label{ex:5:77b}
\gll das Buch          sein\\
the book.\textsc{neut} his\\
}
\z
\z

This distribution was used to motivate the composite analysis of possessive articles – since \textit{ein} must precede the head noun, the possessive article must precede it as well. This accounts for the ungrammaticality in \REF{ex:5:77b}. Interestingly, other languages seem to be revealing in this respect documenting that the lower position of the possessive pronominal in \REF{ex:5:77b} is not ungrammatical in principle. This means that \REF{ex:5:77b} requires an explanation in German. I consider this in more detail.

Unlike Modern Standard German, OHG allows possessive pronominals to follow their head nouns \REF{ex:5:78a}. Furthermore, note that Norwegian can also have possessive pronominals in postnominal position \REF{ex:5:78b} (\REF[a]{ex:5:78} is from \citealt{Demske2001}: 173; \REF{ex:5:78b} is from \citealt{Julien2005a}: 140).

\ea%78
    \label{ex:5:78}
\ea \label{ex:5:78a}
OHG\\
\gll (ther) fater            min     \\
     {\db}the   father.\textsc{masc} my\\
\glt ‘my father’
\ex \label{ex:5:78b}
Norwegian\\
\gll katt-a           mi    \\
    cat.\textsc{fem}-\textsc{def} my\\
\glt ‘my cat’
\z
\z

I reiterate the proposal that the properties of the possessive articles in Modern Standard German follow from the decompositional analysis discussed above. Specifically, since \textit{ein} itself cannot follow the head noun \REF{ex:5:79a}, other \textit{ein}-words cannot either. This was illustrated for the possessive article in \REF{ex:5:77b} above and is illustrated for the singularity numeral and the negative article in \REF[b-c]{ex:5:79} below.

\ea%79
    \label{ex:5:79}
\ea[*]{\label{ex:5:79a}
\gll {\{das / ein / $\emptyset$\textsubscript{D}\}} Buch          ein\\
 {{\db}the  / a } book.\textsc{neut} a\\
  }
\ex[*]{\label{ex:5:79b}
\gll {\{das\ /  ein / $\emptyset$\textsubscript{D}\}}  Buch          EIN\\
  {{\db}the  / a}  book.\textsc{neut} one\\
}
\ex[*]{\label{ex:5:79c}
\gll {\{das / ein / $\emptyset$\textsubscript{D}\}}  Buch           kein\\
  {{\db}the  / a}  book.\textsc{neut} no\\
  }
\z
\z

Now, while \REF[b-c]{ex:5:79} are also out for independent reasons (i.e., the high base positions of $\emptyset$\textsubscript{[--PL]} and NEG), the ungrammaticality of \REF{ex:5:79a} suggests a straightforward account of the absence of postnominal possessive pronominals in Modern Standard German as in \REF{ex:5:77b}.

In more detail, as seen in \REF{ex:5:77a} and \REF{ex:5:78} above, possessives can be base-\linebreak generated in a low position. As such, possessive pronominals in Modern Standard German could also be expected to be grammatical in such a low position. This, however, is not the case. Given the composite analysis, note that the possessive component of the article (e.g., \textit{s}- in \REF{ex:5:77b}) cannot be supported by \textit{ein} in Modern German if the possessive stays in situ (Spec,NP) – as indicated by \REF{ex:5:79a}, \textit{ein} itself cannot occur in such a low position since \textit{ein}, like all determiners, is base-generated in ArtP, a position higher than the (surface) position of the noun (Num). Assuming that the possessive pronominals in OHG and Norwegian are not composite forms involving an indefinite article, they can surface in postnominal position.\footnote{Interestingly, prenominal possessives can also occur lower in the structure in OHG, namely between the determiner and the head noun (see also Footnote \ref{foot:5:13}; \REF{ex:5:40:i} is taken from \citealt{Demske2001}: 227).
	\ea OHG\\
	\label{ex:5:40:i}
	\gll in dhemu heilegin daniheles chiscribe\\
	in the       holy       Daniel’s   scripture.\textsc{neut}\\
	\glt ‘in Daniel’s holy scripture’
	\z
	
	Again, this indicates that possessives have a different analysis in the older varieties of German.} This diachronic and cross-linguistic difference lends more credence to the proposal that possessive articles in (Modern) Standard German are composite forms.\footnote{Note that dialectal forms of the possessive article (e.g., Northern German \textit{min} ‘my’) may also involve a decompositional analysis. Their spell-out forms, however, may involve more than simple Local Dislocation to support the possessive element.} I briefly sketch the diachronic path of the three \textit{ein}-words involving composites.

In \sectref{sec:5.5.1.1}, I tentatively suggested that OHG \textit{éin} ‘one’ was split into two components in the development of the language \REF{ex:5:80a} leading to the emergence of the indefinite article. With this in mind, I suggest that the other \textit{ein}-words were also split into two parts over time. Specifically, possessive pronominals were split into a possessive component and vacuous \textit{ein} \REF{ex:5:80b}.\footnote{This process is possibly tied to ENHG Diphthongization, which changed [i:] to [a\textsc{i}]. Furthermore, the split of the possessive pronoun could have come about by analogy with the splits of the singularity numeral and/or the negative article.} As for the negative article, judging from the diachronic development of \textit{kein} ‘no’ described in \citet[235]{PaulEtAl1989}, it appears as if the negative article consisted of two components early on \REF{ex:5:80c}. Importantly though, the inner makeup seems to have changed from the combination of a negative element with the singularity numeral to a negative element with the indefinite article.

\ea%80
    \label{ex:5:80}
\ea \label{ex:5:80a}
\gll éin  $\rightarrow$  $\emptyset$\textsubscript{[--PL]}+\textit{ein}\\
one   {} $\emptyset$\textsubscript{[--PL]}+a\\
\ex \label{ex:5:80b}
\gll mīn  $\rightarrow$  \textit{m}     +\textit{ein}\\
my  {}  \textsc{poss} +a\\
\ex \label{ex:5:80c}
\gll ne(c)h +ein     $\rightarrow$     \textit{nekein} $\rightarrow$     \textit{k} +\textit{ein}\\
not +one      {} {} {}          \textsc{neg} +a\\
\z
\z

To sum up, unlike vacuous \textit{ein}, the other Modern German \textit{ein}-words are composite forms. In contrast, the counterparts of the latter elements in OHG (and Norwegian) are not (or not in the same way). Note that this difference in decompositionality is also compatible with the change in inflection that adjectives went through in the history of German; for instance, recall from \chapref{sec:2}, \sectref{sec:2.1.2} that possessive pronominals in OHG (and Norwegian) take adjectives with weak inflections but that possessive articles in Modern German take adjectives with mixed inflections. Thus, while the old possessive pronominals pattern with the definite article, the Modern German counterparts are related to \textit{ein}. It is clear that German has undergone some changes in this empirical domain. In other words, OHG and Modern German must have different accounts of these phenomena.

In the next section, I turn to some previous analyses examining one proposal in more detail. We will see that unlike adjectival endings discussed in \chapref{sec:2}, there are much fewer contributions investigating the different kinds of \textit{ein}. Like the previous analyses of adjectival inflections, the accounts here also cannot capture all the different cases.

\section{A brief critique of a previous proposal}\label{sec:5.7}

In the descriptive literature, the indefinite and the definite article are usually discussed together, with the former taken to be the indefinite counterpart of the latter (for German, see \citealt{Duden1995}: 303-21). Recognizing certain shortcomings of this juxtaposition, a number of other views have emerged. For example, \citet{Perlmutter1970} derives English \textit{a(n)} as an unstressed version of the numeral \textit{one}, \citet[47]{Higginbotham1987} argues that the indefinite article in predicate nominals is an adjective meaning ‘one’, and \citet{Ackles1996} argues that \textit{a(n)} marks the presence of NumP with singular count nouns.\footnote{For the discussion of English \textit{a(n)} in quantifying expressions (e.g., \textit{a lot}, \textit{a few}), see \citet{Klockmann2020}. The author tentatively extends the discussion to ordinary noun phrases like \textit{a book} claiming that \textit{a(n)} is not an indefinite element – it is featureless but involves countability. This is partially different from the current discussion of German. Here, \textit{ein} is also claimed not to be related to indefiniteness, but \textit{ein} is proposed to have a feature (i.e., [+D]), and it is held not to be responsible for countabilily.} Elaborating on work by \citet{Oomen1977}, \citet{Vater1982,Vater1984,Vater2002} proposes that there is no indefinite article at all but only a numeral/quantifier (“Quantor”). He mainly discusses German but his empirical coverage is meant to be wider.

It is interesting to point out that all proposals just mentioned make different claims. No consensus seems to have emerged. Recall now that I showed in \chapref{sec:1}, \sectref{sec:1.2.1.3} that German, Nowegian, and Yiddish seem to differ in some of the empirical details. Similar to adjectival inflections in \chapref{sec:2}, I have proposed that it is these – on the surface minor – details that show the true nature of the indefinite article in German. In what follows, I review only Vater’s proposal, which explicitly discusses German. As far as I am aware, this proposal seems to have received fairly wide acceptance, at least for German. To give just one example, although providing a critique of some points of \citet{Vater1982,Vater1984}, \citet[100-16]{Bisle-Müller1991} reaches a similar conclusion.

  In a series of papers (\citeyear{Vater1982,Vater1984,Vater2002}), Vater argues against the traditional opposition of the indefinite vis-à-vis the definite article (I refer only to \citealt{Vater1982} as that contains the main relevant insights). He proposes that \textit{ein} is not the indefinite counterpart of definite \textit{der} ‘the’. According to Vater, this element is not indefinite as \textit{ein} can lead not only to a non-specific but also to a specific interpretation of the containing noun phrase (for examples, see \sectref{sec:5.4.1.3} above). In addition, it denotes a specific amount, namely singularity. Furthermore, this element is not an article as \textit{ein} is not necessarily “localizing” in function in the sense of \citet{Hawkins1978}.

Rather, \citet{Vater1982} proposes that \textit{ein} is a cardinal numeral, that is, a type of quantifier that denotes a specific number of entities (also \citealt{Oomen1977}). Generalizing his discussion, he claims that determiners or articles only involve definite elements. In contrast, the other determiner elements belong to a different part of speech – numerals/quantifiers. He provides some syntactic and semantic arguments for this claim. After illustrating some of these arguments, I return to them showcasing some shortcomings.

  Starting with some syntactic arguments, \citet[71]{Vater1982} points out that like other quantifiers, \textit{ein} can undergo quantifier float (note that the translations of the examples here and below are my own).

\ea%81
    \label{ex:5:81}
\ea \label{ex:5:81a}
\gll Antrag  habe ich keinen gestellt.\\
application.\textsc{masc} have I     no        submitted\\
\glt ‘As for applications, I have submitted none.’
\ex \label{ex:5:81b}
\gll Antrag habe ich einen gestellt.\\
application.\textsc{masc} have I     one    submitted\\
\glt ‘As for applications, I have submitted one.’
\z
\z

Second, like other numerals \REF{ex:5:82a}, \textit{ein} can co-occur with a determiner \REF{ex:5:82b}.

\ea%82
    \label{ex:5:82}
\ea  \label{ex:5:82a}
\gll die zwei Bücher\\
the two  books\\
\glt ‘the two books’
\ex  \label{ex:5:82b}
\gll das eine Buch\\
the  one  book.\textsc{neut}\\
\glt ‘one of the two books’
\z
\z

Turning to some semantic arguments, \citet[71-72]{Vater1982} argues that like other numerals, \textit{ein} can individuate mass nouns. Glossing over some of the details, the interpretation in \REF{ex:5:83a} can involve certain types of bread or certain amounts of bread. Modulo the singular, the example in \REF{ex:5:83b} has a similar range of readings. Importantly, the definite article by itself does not have this individuating function as regards types of bread. Note though that the example in \REF{ex:5:83c} is still ambiguous, here between a mass and a count reading.

\ea%83
    \label{ex:5:83}
\ea \label{ex:5:83a}
\gll zwei Brote\\
two  breads\\
\glt ‘two types of bread’\\
\glt ‘two loaves of bread’
\ex \label{ex:5:83b}
\gll ein  Brot\\
one bread.\textsc{neut}\\
\glt ‘one type of bread’\\
\glt ‘one loaf of bread’
\ex \label{ex:5:83c}
\gll das Brot\\
 the bread.\textsc{neut}\\
\glt ‘the (substance of) bread’\\
\glt ‘the loaf of bread’
\z
\z

In order to derive the count reading in \REF{ex:5:83c}, Vater follows certain aspects of \citet{Perlmutter1970}. Specifically, \citeauthor{Vater1982} (\citeyear{Vater1982}: 71-72, \citeyear{Vater1984}: 39) proposes that the mass interpretation in \REF{ex:5:83c} only involves a definite article \REF{ex:5:84a}. In contrast, the definite singular count reading in \REF{ex:5:83c} involves \textit{ein}, which is deleted \REF{ex:5:84b}. Interestingly, this \textit{ein} can also surface. In this scenario, the interpretation is, according to Vater, partitive. This is indicated by the translation in \REF{ex:5:84c} (below, I return to the word put in square brackets).

\ea%84
    \label{ex:5:84}
\ea  \label{ex:5:84a}
\gll das Brot\\
the bread.\textsc{neut}\\
\glt ‘the (substance of) bread’
\ex  \label{ex:5:84b}
\gll das \sout{eine} Brot\\
the one  bread.\textsc{neut}\\
\glt ‘the loaf of bread’
\ex  \label{ex:5:84c}
\gll das eine Brot\\
the  one  bread.\textsc{neut}\\
\glt ‘one of the [two] breads’
\z
\z

After this brief illustration, I return to the above arguments pointing out some shortcomings.

  Note that the first syntactic argument does not involve quantifier float but rather a different type of discontinuous noun phrase. In \chapref{sec:4}, \sectref{sec:4.3}, I labeled this construction split topicalization. One argument against Vater’s view is that under certain conditions, this type of split does allow (other) determiners to be stranded.

\ea%85
    \label{ex:5:85}
\gll Hemden habe ich immer  nur diese da     getragen\\
  shirts      have I     always only these there worn\\
  \glt ‘As for shirts, I have always worn only these there.’
\z

Given that either \textit{ein} or \textit{diese} ‘these’ can occur in a source DP, we can maintain the claim that \textit{ein} is a determiner.

Turning to the second syntactic argument, it is true that determiners and quantifiers can co-occur. However, there are also cases where two determiner elements can be combined \REF{ex:5:86a}. The same holds for two quantifiers on Vater’s assumptions or, alternatively, two determiner elements on my assumptions \REF{ex:5:86b}.

\ea%86
    \label{ex:5:86}
\ea \label{ex:5:86a}
\gll diese meine Freunde\\
these my     friends\\
\glt ‘these friends of mine’
\ex \label{ex:5:86b}
\gll ein jeder von uns\\
  an  every of   us\\
\glt ‘each of us’
\z
\z

Again, given the possibility of co-occurrence, I continue claiming that \textit{ein} is a determiner(-like) element when it occurs with another determiner as in \REF{ex:5:86b} (specifically, this type of \textit{ein} is a predeterminer, see also Footnote \ref{foot:5:5}). As for the semantic arguments, I agree that \textit{ein} seems to be individuating in nature. However, I suggest in \chapref{sec:8}, \sectref{sec:8.2.2.3} that \textit{ein} is not responsible for this effect, but rather it flags the presence of an operator. Finally, I discuss the partitive interpretation in more detail.

  \citet[71]{Vater1982} claims that \textit{ein} preceded by a definite article can be paraphrased as \textit{ein} followed by a genitive noun phrase in the plural; that is, \textit{ein} is taken to be the singularity numeral. Consider \REF[a-b]{ex:5:87}. \citet[72]{Vater1982} explicitly states that \textit{ein} here presupposes a larger set of elements. However, this claim is not quite accurate. Rather, while the paraphrase does indeed imply a larger set, \textit{ein} preceded by the definite article as in \REF{ex:5:87a} typically presupposes just two elements. Furthermore, as mentioned above, this duality presupposition can be cancelled when a demonstrative replaces the definite article \REF{ex:5:87b}. Importantly, the partitive reading in the paraphrase is not cancelled when the demonstrative replaces the definite article there (note again that the translations are mine).

\ea%87
    \label{ex:5:87}
\ea\label{ex:5:87a}
\gll das eine Buch          – eins der     Bücher\\
the  one  book.\textsc{neut} – one  of.the books\\
\glt ‘one of the two books – one of the books’
\ex\label{ex:5:87b}
\gll dieses eine Buch          – eins dieser    Bücher\\
this     one  book.\textsc{neut} – one  of.these books\\
\glt ‘this one book – one of these books’
\z
\z

I take this to mean that the paraphrases do not capture the semantics of this \textit{ein} accurately. In other words, \textit{ein} is not the singularity numeral here. Furthermore, it appears that other numerals do not invoke this partitivity either; that is, \textit{die zwei Bücher} ‘the two books’ does not mean ‘two of the books’. I conclude that the \textit{ein} in \REF{ex:5:87} is of a different type. Above, I proposed that this \textit{ein} is adjectival, an element only licensed in definite contexts. Besides these issues, it is also worth pointing out that a number of other, important types of \textit{ein} are not discussed in Vater. This shows the limited empirical coverage of that proposal.

Predicate nominals and cases in the plural are not mentioned in Vater’s discussion at all.

\ea%88
    \label{ex:5:88}
\ea[]{ \label{ex:5:88a}
\gll Er ist (ein) Lehrer\\
he is    {\db}a     teacher.\textsc{masc}\\
\glt ‘He is a teacher.’
}
\ex[\%]{ \label{ex:5:88b}
\gll  So    ein-e Idioten!\\
 such a-\textsc{pl}  idiots\\
\glt ‘Such Idiots!’
}
\z
\z

It is quite clear that the noun phrases in \REF{ex:5:88a} and \REF{ex:5:88b} have nothing to do with singularity: \REF{ex:5:88a} denotes a property, and \REF{ex:5:88b} involves a plurality.

Furthermore, note also that Vater does not discuss reduced forms. However, as can be seen in \REF{ex:5:89}, the form of \textit{ein}, reduced or unreduced, \textit{can} make an important difference with regard to the grammaticality judgments and with regard to the interpretations (for \REF{ex:5:89c}, see \citealt{Pafel1994}: 251).

\ea%89
    \label{ex:5:89}
\ea  \label{ex:5:89a}
\gll Geben Sie mir (ei)nen Apfel!\\
give    you me   {\db}an       apple.\textsc{masc}\\
\glt ‘Give me an apple!’
\ex  \label{ex:5:89b}
\gll Geben Sie  mir *(EI)NEN Apfel!\\
give    you me    {\db\db}one        apple.\textsc{masc}\\
\glt ‘Give me one apple!’
\ex  \label{ex:5:89c}
\gll Geben Sie  mir *(ei)nen!\\
give    you me    {\db\db}one\\
\glt ‘Give me one!’
\z
\z

The distinction of reduced and unreduced forms has played an important role in categorizing the different kinds of \textit{ein} in the current account. \REF{ex:5:89a} involves the indefinite article \textit{ein}, and \REF{ex:5:89b} shows the singularity numeral \textit{EIN}. As for \REF{ex:5:89c}, note that \textit{ein} is used pronominally here. I reiterate the idea that the reduced form of \textit{ein} is ungrammatical as pronominal forms of \textit{ein} involve some stress, which prevents the reduced forms from occurring.

Finally, Vater treats the negator \textit{kein} ‘no’ as an unanalyzed form, and possessives such as \textit{mein} ‘my’ are not discussed at all. As documented above, \textit{ein} and the other \textit{ein}-words exhibit a number of morpho-syntactic similarities; for instances, as regards their inflections. Since inflections are not addressed by Vater, an important morpho-syntactic generalization over the different \textit{ein}-words is being missed here. In my view, adjectival inflections and \textit{ein}-words should be discussed in tandem. In this respect, I point out again that adjectival inflections provide important clues about the lexical categories of items in the noun phrase – they only surface on determiners, certain quantifiers, and adjectives – and they provide indications of the various structures of the noun phrase as a whole.

  To sum up, I agree with Vater (and others) that \textit{ein} is not an indefinite article (although I keep the name for convenience). I disagree with him (and others) that \textit{ein} is the singularity numeral. Rather, I claim that there are two types of \textit{ein}: One is a semantically vacuous element, and the other is adjectival \textit{eine}, which has semantics.

\section{Conclusion} \label{sec:5.8}

\begin{table}[b]
\caption{Summary of the properties of the types of \textit{ein}}
\label{tab:5:2}
\begin{tabularx}{\textwidth}{XXXXX}
\lsptoprule
\multicolumn{2}{l}{Kinds of \textit{ein}} & Morphology & Semantics & Position\\
\midrule
\multicolumn{1}{l}{Article \textit{ein}} & Vacuous (indefinite, possessive, negative, predicative) & $\alpha$PL morph &  & Card or D \newline (depending on the read- \newline ing, weak vs. \newline strong)\footnote{In \chapref{sec:6}, I propose that predicative \textit{ein} can also occur in Art (the head of ArtP).}\\
& Numeral $\emptyset$\textsubscript{[--PL]} & ($\alpha$PL morph \newline = $\alpha$PL sem) & --PL sem & Spec,CardP or Spec,DP\\
\multicolumn{2}{l}{Adjectival \textit{eine}} & $\alpha$PL morph & n = 2 (unless in certain \newline contexts) & (high) Spec,AgrP\\
\lspbottomrule
\end{tabularx}
\end{table}


One goal of this chapter was to provide a more comprehensive survey of the different types of \textit{ein}. Arguing that \textit{ein} cannot involve the singularity numeral only, I examined three types: the article, the numeral, and the adjective. In order to capture the morphological similarities and the non-co-occurrence of the article and the numeral, I proposed that the numeral is a composite form – it involves the null element $\emptyset$\textsubscript{[--PL]}, and the latter is supported by the article. Something similar holds for the possessive and negative articles. Adjectival \textit{eine} was proposed to be an independent lexical element. In order to account for the differences, I suggested that the three elements have different feature specifications and are in different positions in the syntactic tree. \tabref{tab:5:2} summarizes these differences.


It is clear that the conditions under which adjectival \textit{eine} induces and loses its duality presupposition require more investigation, something I will no pursue here.

The second goal was to identify more general properties of \textit{ein}. Similar to adjectival inflections in \chapref{sec:2}, I concluded here that \textit{ein} is also semantically vacuous (Hypothesis 1a). In the next two chapters, I turn to more properties of \textit{ein} by discussing some consequences of the current analysis. In particular, I discuss \textit{ein} in relation to semantic concepts such as emotiveness and number. This includes the discussion of \textit{ein} in predicative contexts. We will find confirmation of the conclusions reached thus far. Furthermore, we will see more evidence that \textit{ein} can also indicate a certain amount of structure in the noun phrase and that it can flag the presence of covert operators (Hypothesis 3).

