% This file was converted to LaTeX by Writer2LaTeX ver. 1.4
% see http://writer2latex.sourceforge.net for more info
\documentclass[12pt]{article}
\usepackage[utf8]{inputenc}
\usepackage[T1]{fontenc}
\usepackage[english]{babel}
\usepackage{amsmath}
\usepackage{amssymb,amsfonts,textcomp}
\usepackage{array}
\usepackage{hhline}
\usepackage{hyperref}
\hypersetup{colorlinks=true, linkcolor=blue, citecolor=blue, filecolor=blue, urlcolor=blue}
\usepackage{graphicx}
% footnotes configuration
\makeatletter
\renewcommand\thefootnote{\arabic{footnote}}
\@addtoreset{footnote}{section}
\renewcommand\@makefnmark{\mbox{\textstyleFootnoteanchor{\@thefnmark}}}
\makeatother
\newcommand\textsubscript[1]{\ensuremath{{}_{\text{#1}}}}
% Text styles
\newcommand\textstylew[1]{#1}
\newcommand\textstyleFootnoteSymbol[1]{\textsuperscript{#1}}
\newcommand\textstyleInternetlink[1]{#1}
\newcommand\textstyleWWviiiNumxviiiz[1]{\textrm{#1}}
\newcommand\textstyleWWviiiNumxviiz[1]{\textrm{#1}}
\newcommand\textstyleWWviiiNumxixz[1]{\textrm{#1}}
\newcommand\textstyleWWviiiNumiiiz[1]{\textup{#1}}
\newcommand\textstyleWWviiiNumviz[1]{#1}
\newcommand\textstyleWWviiiNumviiiz[1]{#1}
\newcommand\textstyleWWviiiNumxz[1]{#1}
\newcommand\textstyleWWviiiNumxxz[1]{#1}
\newcommand\textstyleWWviiiNumiiz[1]{#1}
\newcommand\textstyleWWviiiNumxviz[1]{#1}
\newcommand\textstyleWWviiiNumivz[1]{#1}
\newcommand\textstyleWWviiiNumvz[1]{#1}
\newcommand\textstyleWWviiiNumxvz[1]{#1}
\newcommand\textstyleWWviiiNumxiz[1]{\textrm{#1}}
\newcommand\textstyleWWviiiNumxiiz[1]{\textrm{#1}}
\newcommand\textstyleWWviiiNumixz[1]{\textrm{#1}}
\newcommand\textstyleWWviiiNumviiz[1]{\textrm{#1}}
\newcommand\textstyleWWviiiNumxiiiz[1]{#1}
\newcommand\textstyleWWviiiNumxivz[1]{#1}
\newcommand\textstyleunicodei[1]{\textsf{#1}}
\newcommand\textstylenv[1]{#1}
\newcommand\textstyleolmarker[1]{#1}
\newcommand\textstyleptbrandiii[1]{#1}
\newcommand\textstyledisplayname[1]{#1}
\newcommand\textstylecontributor[1]{#1}
\newcommand\textstylemetadataandcontributorsfont[1]{#1}
\newcommand\textstyleFootnoteanchor[1]{\textsuperscript{#1}}
% Headings and outline numbering
\makeatletter
\renewcommand\section{\@startsection{section}{1}{0.0cm}{0cm}{0.1mm}{\normalfont\normalsize\fontsize{12pt}{14.4pt}\selectfont\rmfamily\bfseries\centering}}
\renewcommand\subsubsection{\@startsection{subsubsection}{3}{0.0cm}{0.1665in}{0.0417in}{\normalfont\normalsize\fontsize{13pt}{15.6pt}\selectfont\rmfamily\bfseries}}
\renewcommand\subparagraph{\@startsection{subparagraph}{5}{0.0cm}{0.1665in}{0.0417in}{\normalfont\normalsize\fontsize{13pt}{15.6pt}\selectfont\sffamily\bfseries\itshape}}
\renewcommand\@seccntformat[1]{\csname @textstyle#1\endcsname{\csname the#1\endcsname}\csname @distance#1\endcsname}
\setcounter{secnumdepth}{0}
\newcommand\@distancesection{}
\newcommand\@textstylesection[1]{#1}
\newcommand\@distancesubsection{}
\newcommand\@textstylesubsection[1]{#1}
\newcommand\@distancesubsubsection{}
\newcommand\@textstylesubsubsection[1]{#1}
\newcommand\@distanceparagraph{}
\newcommand\@textstyleparagraph[1]{#1}
\newcommand\@distancesubparagraph{}
\newcommand\@textstylesubparagraph[1]{#1}
\makeatother
\makeatletter
\newcommand\arraybslash{\let\\\@arraycr}
\makeatother
\raggedbottom
% Paragraph styles
\renewcommand\familydefault{\rmdefault}
\newenvironment{styleStandard}{\setlength\leftskip{0cm}\setlength\rightskip{0cm}\setlength\parindent{0cm}\setlength\parfillskip{0pt plus 1fil}\setlength\parskip{0cm plus 1pt}\writerlistparindent\writerlistleftskip\leavevmode\normalfont\normalsize\writerlistlabel\ignorespaces}{\unskip\vspace{0cm plus 1pt}\par}
\newenvironment{styleFootnote}{\setlength\leftskip{0cm}\setlength\rightskip{0cm}\setlength\parindent{0cm}\setlength\parfillskip{0pt plus 1fil}\setlength\parskip{0cm plus 1pt}\writerlistparindent\writerlistleftskip\leavevmode\normalfont\normalsize\fontsize{10pt}{12.0pt}\selectfont\writerlistlabel\ignorespaces}{\unskip\vspace{0cm plus 1pt}\par}
\newenvironment{styleFooter}{\setlength\leftskip{0cm}\setlength\rightskip{0cm}\setlength\parindent{0cm}\setlength\parfillskip{0pt plus 1fil}\setlength\parskip{0cm plus 1pt}\writerlistparindent\writerlistleftskip\leavevmode\normalfont\normalsize\writerlistlabel\ignorespaces}{\unskip\vspace{0cm plus 1pt}\par}
\newenvironment{styleJBHeadingii}{\setlength\leftskip{0.3937in}\setlength\rightskip{0in}\setlength\parindent{-0.3937in}\setlength\parfillskip{0pt plus 1fil}\setlength\parskip{0cm plus 1pt}\writerlistparindent\writerlistleftskip\leavevmode\normalfont\normalsize\itshape\writerlistlabel\ignorespaces}{\unskip\vspace{0cm plus 1pt}\par}
\newenvironment{styleCommentText}{\setlength\leftskip{0cm}\setlength\rightskip{0cm}\setlength\parindent{0cm}\setlength\parfillskip{0pt plus 1fil}\setlength\parskip{0cm plus 1pt}\writerlistparindent\writerlistleftskip\leavevmode\normalfont\normalsize\fontsize{10pt}{12.0pt}\selectfont\writerlistlabel\ignorespaces}{\unskip\vspace{0cm plus 1pt}\par}
\newenvironment{styleJBExample}{\setlength\leftskip{0in}\setlength\rightskip{0in}\setlength\parindent{0.3937in}\setlength\parfillskip{0pt plus 1fil}\setlength\parskip{0cm plus 1pt}\writerlistparindent\writerlistleftskip\leavevmode\normalfont\normalsize\writerlistlabel\ignorespaces}{\unskip\vspace{0cm plus 1pt}\par}
\newenvironment{styleCaption}{\setlength\leftskip{0cm}\setlength\rightskip{0cm}\setlength\parindent{0cm}\setlength\parfillskip{0pt plus 1fil}\setlength\parskip{0.0835in plus 0.00835in}\writerlistparindent\writerlistleftskip\leavevmode\normalfont\normalsize\fontsize{10pt}{12.0pt}\selectfont\bfseries\writerlistlabel\ignorespaces}{\unskip\vspace{0.0835in plus 0.00835in}\par}
\newenvironment{styleTextbody}{\setlength\leftskip{0cm}\setlength\rightskip{0cm}\setlength\parindent{0cm}\setlength\parfillskip{0pt plus 1fil}\setlength\parskip{0in plus 1pt}\writerlistparindent\writerlistleftskip\leavevmode\normalfont\normalsize\writerlistlabel\ignorespaces}{\unskip\vspace{0.0835in plus 0.00835in}\par}
\newenvironment{styleTextbodyindent}{\setlength\leftskip{0.25in}\setlength\rightskip{0in}\setlength\parindent{0in}\setlength\parfillskip{0pt plus 1fil}\setlength\parskip{0in plus 1pt}\writerlistparindent\writerlistleftskip\leavevmode\normalfont\normalsize\writerlistlabel\ignorespaces}{\unskip\vspace{0.0835in plus 0.00835in}\par}
\newenvironment{styleNewTimesRoman}{\setlength\leftskip{0cm plus 1fil}\setlength\rightskip{0cm plus 1fil}\setlength\parindent{0cm}\setlength\parfillskip{0pt}\setlength\parskip{0in plus 1pt}\writerlistparindent\writerlistleftskip\leavevmode\normalfont\normalsize\fontsize{14pt}{16.8pt}\selectfont\mdseries\writerlistlabel\ignorespaces}{\unskip\vspace{0in plus 1pt}\par}
\newenvironment{styleHTMLPreformatted}{\setlength\leftskip{0cm}\setlength\rightskip{0cm}\setlength\parindent{0cm}\setlength\parfillskip{0pt plus 1fil}\setlength\parskip{0cm plus 1pt}\writerlistparindent\writerlistleftskip\leavevmode\normalfont\normalsize\fontsize{10pt}{12.0pt}\selectfont\ttfamily\writerlistlabel\ignorespaces}{\unskip\vspace{0cm plus 1pt}\par}
\newenvironment{styleSubtitle}{\setlength\leftskip{0cm plus 1fil}\setlength\rightskip{0cm plus 1fil}\setlength\parindent{0cm}\setlength\parfillskip{0pt}\setlength\parskip{0cm plus 1pt}\writerlistparindent\writerlistleftskip\leavevmode\normalfont\normalsize\fontsize{11pt}{13.2pt}\selectfont\bfseries\itshape\writerlistlabel\ignorespaces}{\unskip\vspace{0cm plus 1pt}\par}
\newenvironment{styleBodyTextIndentii}{\setlength\leftskip{0.1972in}\setlength\rightskip{0in}\setlength\parindent{-0.1972in}\setlength\parfillskip{0pt plus 1fil}\setlength\parskip{0cm plus 1pt}\writerlistparindent\writerlistleftskip\leavevmode\normalfont\normalsize\fontsize{10pt}{12.0pt}\selectfont\writerlistlabel\ignorespaces}{\unskip\vspace{0cm plus 1pt}\par}
\newenvironment{styleref}{\setlength\leftskip{0.1972in}\setlength\rightskip{0in}\setlength\parindent{-0.1972in}\setlength\parfillskip{0pt plus 1fil}\setlength\parskip{0cm plus 1pt}\writerlistparindent\writerlistleftskip\leavevmode\normalfont\normalsize\fontsize{9pt}{10.8pt}\selectfont\writerlistlabel\ignorespaces}{\unskip\vspace{0cm plus 1pt}\par}
\newenvironment{styleJBReferenceList}{\setlength\leftskip{0.1972in}\setlength\rightskip{0in}\setlength\parindent{-0.1972in}\setlength\parfillskip{0pt plus 1fil}\setlength\parskip{0cm plus 1pt}\writerlistparindent\writerlistleftskip\leavevmode\normalfont\normalsize\writerlistlabel\ignorespaces}{\unskip\vspace{0cm plus 1pt}\par}
% List styles
\newcommand\writerlistleftskip{}
\newcommand\writerlistparindent{}
\newcommand\writerlistlabel{}
\newcommand\writerlistremovelabel{\aftergroup\let\aftergroup\writerlistparindent\aftergroup\relax\aftergroup\let\aftergroup\writerlistlabel\aftergroup\relax}
\newcounter{listWWviiiNumxviiilevelii}
\newcounter{listWWviiiNumxviiileveliii}[listWWviiiNumxviiilevelii]
\newcounter{listWWviiiNumxviiileveliv}[listWWviiiNumxviiileveliii]
\renewcommand\thelistWWviiiNumxviiilevelii{\arabic{listWWviiiNumxviiilevelii}}
\renewcommand\thelistWWviiiNumxviiileveliii{\arabic{listWWviiiNumxviiileveliii}}
\renewcommand\thelistWWviiiNumxviiileveliv{\arabic{listWWviiiNumxviiileveliv}}
\newcommand\labellistWWviiiNumxviiileveli{\textstyleWWviiiNumxviiiz{{\textbullet}}}
\newcommand\labellistWWviiiNumxviiilevelii{\thelistWWviiiNumxviiilevelii.}
\newcommand\labellistWWviiiNumxviiileveliii{\thelistWWviiiNumxviiileveliii.}
\newcommand\labellistWWviiiNumxviiileveliv{\thelistWWviiiNumxviiileveliv.}
\newenvironment{listWWviiiNumxviiileveli}{\def\writerlistleftskip{\setlength\leftskip{0.5in}}\def\writerlistparindent{}\def\writerlistlabel{}\def\item{\def\writerlistparindent{\setlength\parindent{-0.25in}}\def\writerlistlabel{\makebox[0cm][l]{\labellistWWviiiNumxviiileveli}\hspace{-0.635cm}\writerlistremovelabel}}}{}
\newenvironment{listWWviiiNumxviiilevelii}{\def\writerlistleftskip{\setlength\leftskip{0.75in}}\def\writerlistparindent{}\def\writerlistlabel{}\def\item{\def\writerlistparindent{\setlength\parindent{-0.25in}}\def\writerlistlabel{\stepcounter{listWWviiiNumxviiilevelii}\makebox[0cm][l]{\labellistWWviiiNumxviiilevelii}\hspace{0.25in}\writerlistremovelabel}}}{}
\newenvironment{listWWviiiNumxviiileveliii}{\def\writerlistleftskip{\setlength\leftskip{1in}}\def\writerlistparindent{}\def\writerlistlabel{}\def\item{\def\writerlistparindent{\setlength\parindent{-0.25in}}\def\writerlistlabel{\stepcounter{listWWviiiNumxviiileveliii}\makebox[0cm][l]{\labellistWWviiiNumxviiileveliii}\hspace{0.25in}\writerlistremovelabel}}}{}
\newenvironment{listWWviiiNumxviiileveliv}{\def\writerlistleftskip{\setlength\leftskip{1.25in}}\def\writerlistparindent{}\def\writerlistlabel{}\def\item{\def\writerlistparindent{\setlength\parindent{-0.25in}}\def\writerlistlabel{\stepcounter{listWWviiiNumxviiileveliv}\makebox[0cm][l]{\labellistWWviiiNumxviiileveliv}\hspace{0.25in}\writerlistremovelabel}}}{}
\newcounter{listWWviiiNumxviilevelii}
\newcounter{listWWviiiNumxviileveliii}[listWWviiiNumxviilevelii]
\newcounter{listWWviiiNumxviileveliv}[listWWviiiNumxviileveliii]
\renewcommand\thelistWWviiiNumxviilevelii{\arabic{listWWviiiNumxviilevelii}}
\renewcommand\thelistWWviiiNumxviileveliii{\arabic{listWWviiiNumxviileveliii}}
\renewcommand\thelistWWviiiNumxviileveliv{\arabic{listWWviiiNumxviileveliv}}
\newcommand\labellistWWviiiNumxviileveli{\textstyleWWviiiNumxviiz{{\textbullet}}}
\newcommand\labellistWWviiiNumxviilevelii{\thelistWWviiiNumxviilevelii.}
\newcommand\labellistWWviiiNumxviileveliii{\thelistWWviiiNumxviileveliii.}
\newcommand\labellistWWviiiNumxviileveliv{\thelistWWviiiNumxviileveliv.}
\newenvironment{listWWviiiNumxviileveli}{\def\writerlistleftskip{\setlength\leftskip{0.5in}}\def\writerlistparindent{}\def\writerlistlabel{}\def\item{\def\writerlistparindent{\setlength\parindent{-0.25in}}\def\writerlistlabel{\makebox[0cm][l]{\labellistWWviiiNumxviileveli}\hspace{-0.635cm}\writerlistremovelabel}}}{}
\newenvironment{listWWviiiNumxviilevelii}{\def\writerlistleftskip{\setlength\leftskip{0.75in}}\def\writerlistparindent{}\def\writerlistlabel{}\def\item{\def\writerlistparindent{\setlength\parindent{-0.25in}}\def\writerlistlabel{\stepcounter{listWWviiiNumxviilevelii}\makebox[0cm][l]{\labellistWWviiiNumxviilevelii}\hspace{0.25in}\writerlistremovelabel}}}{}
\newenvironment{listWWviiiNumxviileveliii}{\def\writerlistleftskip{\setlength\leftskip{1in}}\def\writerlistparindent{}\def\writerlistlabel{}\def\item{\def\writerlistparindent{\setlength\parindent{-0.25in}}\def\writerlistlabel{\stepcounter{listWWviiiNumxviileveliii}\makebox[0cm][l]{\labellistWWviiiNumxviileveliii}\hspace{0.25in}\writerlistremovelabel}}}{}
\newenvironment{listWWviiiNumxviileveliv}{\def\writerlistleftskip{\setlength\leftskip{1.25in}}\def\writerlistparindent{}\def\writerlistlabel{}\def\item{\def\writerlistparindent{\setlength\parindent{-0.25in}}\def\writerlistlabel{\stepcounter{listWWviiiNumxviileveliv}\makebox[0cm][l]{\labellistWWviiiNumxviileveliv}\hspace{0.25in}\writerlistremovelabel}}}{}
\newcounter{listWWviiiNumxixlevelii}
\newcounter{listWWviiiNumxixleveliii}[listWWviiiNumxixlevelii]
\newcounter{listWWviiiNumxixleveliv}[listWWviiiNumxixleveliii]
\renewcommand\thelistWWviiiNumxixlevelii{\arabic{listWWviiiNumxixlevelii}}
\renewcommand\thelistWWviiiNumxixleveliii{\arabic{listWWviiiNumxixleveliii}}
\renewcommand\thelistWWviiiNumxixleveliv{\arabic{listWWviiiNumxixleveliv}}
\newcommand\labellistWWviiiNumxixleveli{\textstyleWWviiiNumxixz{{\textbullet}}}
\newcommand\labellistWWviiiNumxixlevelii{\thelistWWviiiNumxixlevelii.}
\newcommand\labellistWWviiiNumxixleveliii{\thelistWWviiiNumxixleveliii.}
\newcommand\labellistWWviiiNumxixleveliv{\thelistWWviiiNumxixleveliv.}
\newenvironment{listWWviiiNumxixleveli}{\def\writerlistleftskip{\setlength\leftskip{0.5in}}\def\writerlistparindent{}\def\writerlistlabel{}\def\item{\def\writerlistparindent{\setlength\parindent{-0.25in}}\def\writerlistlabel{\makebox[0cm][l]{\labellistWWviiiNumxixleveli}\hspace{-0.635cm}\writerlistremovelabel}}}{}
\newenvironment{listWWviiiNumxixlevelii}{\def\writerlistleftskip{\setlength\leftskip{0.75in}}\def\writerlistparindent{}\def\writerlistlabel{}\def\item{\def\writerlistparindent{\setlength\parindent{-0.25in}}\def\writerlistlabel{\stepcounter{listWWviiiNumxixlevelii}\makebox[0cm][l]{\labellistWWviiiNumxixlevelii}\hspace{0.25in}\writerlistremovelabel}}}{}
\newenvironment{listWWviiiNumxixleveliii}{\def\writerlistleftskip{\setlength\leftskip{1in}}\def\writerlistparindent{}\def\writerlistlabel{}\def\item{\def\writerlistparindent{\setlength\parindent{-0.25in}}\def\writerlistlabel{\stepcounter{listWWviiiNumxixleveliii}\makebox[0cm][l]{\labellistWWviiiNumxixleveliii}\hspace{0.25in}\writerlistremovelabel}}}{}
\newenvironment{listWWviiiNumxixleveliv}{\def\writerlistleftskip{\setlength\leftskip{1.25in}}\def\writerlistparindent{}\def\writerlistlabel{}\def\item{\def\writerlistparindent{\setlength\parindent{-0.25in}}\def\writerlistlabel{\stepcounter{listWWviiiNumxixleveliv}\makebox[0cm][l]{\labellistWWviiiNumxixleveliv}\hspace{0.25in}\writerlistremovelabel}}}{}
\newcounter{listWWviiiNumiiileveli}
\newcounter{listWWviiiNumiiilevelii}[listWWviiiNumiiileveli]
\newcounter{listWWviiiNumiiileveliii}[listWWviiiNumiiilevelii]
\newcounter{listWWviiiNumiiileveliv}[listWWviiiNumiiileveliii]
\renewcommand\thelistWWviiiNumiiileveli{\roman{listWWviiiNumiiileveli}}
\renewcommand\thelistWWviiiNumiiilevelii{\arabic{listWWviiiNumiiilevelii}}
\renewcommand\thelistWWviiiNumiiileveliii{\arabic{listWWviiiNumiiileveliii}}
\renewcommand\thelistWWviiiNumiiileveliv{\arabic{listWWviiiNumiiileveliv}}
\newcommand\labellistWWviiiNumiiileveli{\textstyleWWviiiNumiiiz{(\thelistWWviiiNumiiileveli)}}
\newcommand\labellistWWviiiNumiiilevelii{\thelistWWviiiNumiiilevelii.}
\newcommand\labellistWWviiiNumiiileveliii{\thelistWWviiiNumiiileveliii.}
\newcommand\labellistWWviiiNumiiileveliv{\thelistWWviiiNumiiileveliv.}
\newenvironment{listWWviiiNumiiileveli}{\def\writerlistleftskip{\setlength\leftskip{1in}}\def\writerlistparindent{}\def\writerlistlabel{}\def\item{\def\writerlistparindent{\setlength\parindent{-0.5in}}\def\writerlistlabel{\stepcounter{listWWviiiNumiiileveli}\makebox[0cm][l]{\labellistWWviiiNumiiileveli}\hspace{-1.27cm}\writerlistremovelabel}}}{}
\newenvironment{listWWviiiNumiiilevelii}{\def\writerlistleftskip{\setlength\leftskip{0.75in}}\def\writerlistparindent{}\def\writerlistlabel{}\def\item{\def\writerlistparindent{\setlength\parindent{-0.25in}}\def\writerlistlabel{\stepcounter{listWWviiiNumiiilevelii}\makebox[0cm][l]{\labellistWWviiiNumiiilevelii}\hspace{0.25in}\writerlistremovelabel}}}{}
\newenvironment{listWWviiiNumiiileveliii}{\def\writerlistleftskip{\setlength\leftskip{1in}}\def\writerlistparindent{}\def\writerlistlabel{}\def\item{\def\writerlistparindent{\setlength\parindent{-0.25in}}\def\writerlistlabel{\stepcounter{listWWviiiNumiiileveliii}\makebox[0cm][l]{\labellistWWviiiNumiiileveliii}\hspace{0.25in}\writerlistremovelabel}}}{}
\newenvironment{listWWviiiNumiiileveliv}{\def\writerlistleftskip{\setlength\leftskip{1.25in}}\def\writerlistparindent{}\def\writerlistlabel{}\def\item{\def\writerlistparindent{\setlength\parindent{-0.25in}}\def\writerlistlabel{\stepcounter{listWWviiiNumiiileveliv}\makebox[0cm][l]{\labellistWWviiiNumiiileveliv}\hspace{0.25in}\writerlistremovelabel}}}{}
\newcounter{listWWviiiNumvileveli}
\newcounter{listWWviiiNumvilevelii}[listWWviiiNumvileveli]
\newcounter{listWWviiiNumvileveliii}[listWWviiiNumvilevelii]
\newcounter{listWWviiiNumvileveliv}[listWWviiiNumvileveliii]
\renewcommand\thelistWWviiiNumvileveli{\roman{listWWviiiNumvileveli}}
\renewcommand\thelistWWviiiNumvilevelii{\arabic{listWWviiiNumvilevelii}}
\renewcommand\thelistWWviiiNumvileveliii{\arabic{listWWviiiNumvileveliii}}
\renewcommand\thelistWWviiiNumvileveliv{\arabic{listWWviiiNumvileveliv}}
\newcommand\labellistWWviiiNumvileveli{\textstyleWWviiiNumviz{(\thelistWWviiiNumvileveli)}}
\newcommand\labellistWWviiiNumvilevelii{\thelistWWviiiNumvilevelii.}
\newcommand\labellistWWviiiNumvileveliii{\thelistWWviiiNumvileveliii.}
\newcommand\labellistWWviiiNumvileveliv{\thelistWWviiiNumvileveliv.}
\newenvironment{listWWviiiNumvileveli}{\def\writerlistleftskip{\setlength\leftskip{1in}}\def\writerlistparindent{}\def\writerlistlabel{}\def\item{\def\writerlistparindent{\setlength\parindent{-0.5in}}\def\writerlistlabel{\stepcounter{listWWviiiNumvileveli}\makebox[0cm][l]{\labellistWWviiiNumvileveli}\hspace{-1.27cm}\writerlistremovelabel}}}{}
\newenvironment{listWWviiiNumvilevelii}{\def\writerlistleftskip{\setlength\leftskip{0.75in}}\def\writerlistparindent{}\def\writerlistlabel{}\def\item{\def\writerlistparindent{\setlength\parindent{-0.25in}}\def\writerlistlabel{\stepcounter{listWWviiiNumvilevelii}\makebox[0cm][l]{\labellistWWviiiNumvilevelii}\hspace{0.25in}\writerlistremovelabel}}}{}
\newenvironment{listWWviiiNumvileveliii}{\def\writerlistleftskip{\setlength\leftskip{1in}}\def\writerlistparindent{}\def\writerlistlabel{}\def\item{\def\writerlistparindent{\setlength\parindent{-0.25in}}\def\writerlistlabel{\stepcounter{listWWviiiNumvileveliii}\makebox[0cm][l]{\labellistWWviiiNumvileveliii}\hspace{0.25in}\writerlistremovelabel}}}{}
\newenvironment{listWWviiiNumvileveliv}{\def\writerlistleftskip{\setlength\leftskip{1.25in}}\def\writerlistparindent{}\def\writerlistlabel{}\def\item{\def\writerlistparindent{\setlength\parindent{-0.25in}}\def\writerlistlabel{\stepcounter{listWWviiiNumvileveliv}\makebox[0cm][l]{\labellistWWviiiNumvileveliv}\hspace{0.25in}\writerlistremovelabel}}}{}
\newcounter{listWWviiiNumviiileveli}
\newcounter{listWWviiiNumviiilevelii}[listWWviiiNumviiileveli]
\newcounter{listWWviiiNumviiileveliii}[listWWviiiNumviiilevelii]
\newcounter{listWWviiiNumviiileveliv}[listWWviiiNumviiileveliii]
\renewcommand\thelistWWviiiNumviiileveli{\alph{listWWviiiNumviiileveli}}
\renewcommand\thelistWWviiiNumviiilevelii{\arabic{listWWviiiNumviiilevelii}}
\renewcommand\thelistWWviiiNumviiileveliii{\arabic{listWWviiiNumviiileveliii}}
\renewcommand\thelistWWviiiNumviiileveliv{\arabic{listWWviiiNumviiileveliv}}
\newcommand\labellistWWviiiNumviiileveli{\textstyleWWviiiNumviiiz{(\thelistWWviiiNumviiileveli)}}
\newcommand\labellistWWviiiNumviiilevelii{\thelistWWviiiNumviiilevelii.}
\newcommand\labellistWWviiiNumviiileveliii{\thelistWWviiiNumviiileveliii.}
\newcommand\labellistWWviiiNumviiileveliv{\thelistWWviiiNumviiileveliv.}
\newenvironment{listWWviiiNumviiileveli}{\def\writerlistleftskip{\setlength\leftskip{1in}}\def\writerlistparindent{}\def\writerlistlabel{}\def\item{\def\writerlistparindent{\setlength\parindent{-0.25in}}\def\writerlistlabel{\stepcounter{listWWviiiNumviiileveli}\makebox[0cm][l]{\labellistWWviiiNumviiileveli}\hspace{-1.905cm}\writerlistremovelabel}}}{}
\newenvironment{listWWviiiNumviiilevelii}{\def\writerlistleftskip{\setlength\leftskip{0.75in}}\def\writerlistparindent{}\def\writerlistlabel{}\def\item{\def\writerlistparindent{\setlength\parindent{-0.25in}}\def\writerlistlabel{\stepcounter{listWWviiiNumviiilevelii}\makebox[0cm][l]{\labellistWWviiiNumviiilevelii}\hspace{0.25in}\writerlistremovelabel}}}{}
\newenvironment{listWWviiiNumviiileveliii}{\def\writerlistleftskip{\setlength\leftskip{1in}}\def\writerlistparindent{}\def\writerlistlabel{}\def\item{\def\writerlistparindent{\setlength\parindent{-0.25in}}\def\writerlistlabel{\stepcounter{listWWviiiNumviiileveliii}\makebox[0cm][l]{\labellistWWviiiNumviiileveliii}\hspace{0.25in}\writerlistremovelabel}}}{}
\newenvironment{listWWviiiNumviiileveliv}{\def\writerlistleftskip{\setlength\leftskip{1.25in}}\def\writerlistparindent{}\def\writerlistlabel{}\def\item{\def\writerlistparindent{\setlength\parindent{-0.25in}}\def\writerlistlabel{\stepcounter{listWWviiiNumviiileveliv}\makebox[0cm][l]{\labellistWWviiiNumviiileveliv}\hspace{0.25in}\writerlistremovelabel}}}{}
\newcounter{listWWviiiNumxleveli}
\newcounter{listWWviiiNumxlevelii}[listWWviiiNumxleveli]
\newcounter{listWWviiiNumxleveliii}[listWWviiiNumxlevelii]
\newcounter{listWWviiiNumxleveliv}[listWWviiiNumxleveliii]
\renewcommand\thelistWWviiiNumxleveli{\alph{listWWviiiNumxleveli}}
\renewcommand\thelistWWviiiNumxlevelii{\arabic{listWWviiiNumxlevelii}}
\renewcommand\thelistWWviiiNumxleveliii{\arabic{listWWviiiNumxleveliii}}
\renewcommand\thelistWWviiiNumxleveliv{\arabic{listWWviiiNumxleveliv}}
\newcommand\labellistWWviiiNumxleveli{\textstyleWWviiiNumxz{\thelistWWviiiNumxleveli.}}
\newcommand\labellistWWviiiNumxlevelii{\thelistWWviiiNumxlevelii.}
\newcommand\labellistWWviiiNumxleveliii{\thelistWWviiiNumxleveliii.}
\newcommand\labellistWWviiiNumxleveliv{\thelistWWviiiNumxleveliv.}
\newenvironment{listWWviiiNumxleveli}{\def\writerlistleftskip{\setlength\leftskip{0.75in}}\def\writerlistparindent{}\def\writerlistlabel{}\def\item{\def\writerlistparindent{\setlength\parindent{-0.25in}}\def\writerlistlabel{\stepcounter{listWWviiiNumxleveli}\makebox[0cm][l]{\labellistWWviiiNumxleveli}\hspace{-1.27cm}\writerlistremovelabel}}}{}
\newenvironment{listWWviiiNumxlevelii}{\def\writerlistleftskip{\setlength\leftskip{0.75in}}\def\writerlistparindent{}\def\writerlistlabel{}\def\item{\def\writerlistparindent{\setlength\parindent{-0.25in}}\def\writerlistlabel{\stepcounter{listWWviiiNumxlevelii}\makebox[0cm][l]{\labellistWWviiiNumxlevelii}\hspace{0.25in}\writerlistremovelabel}}}{}
\newenvironment{listWWviiiNumxleveliii}{\def\writerlistleftskip{\setlength\leftskip{1in}}\def\writerlistparindent{}\def\writerlistlabel{}\def\item{\def\writerlistparindent{\setlength\parindent{-0.25in}}\def\writerlistlabel{\stepcounter{listWWviiiNumxleveliii}\makebox[0cm][l]{\labellistWWviiiNumxleveliii}\hspace{0.25in}\writerlistremovelabel}}}{}
\newenvironment{listWWviiiNumxleveliv}{\def\writerlistleftskip{\setlength\leftskip{1.25in}}\def\writerlistparindent{}\def\writerlistlabel{}\def\item{\def\writerlistparindent{\setlength\parindent{-0.25in}}\def\writerlistlabel{\stepcounter{listWWviiiNumxleveliv}\makebox[0cm][l]{\labellistWWviiiNumxleveliv}\hspace{0.25in}\writerlistremovelabel}}}{}
\newcounter{listWWviiiNumxxleveli}
\newcounter{listWWviiiNumxxlevelii}[listWWviiiNumxxleveli]
\newcounter{listWWviiiNumxxleveliii}[listWWviiiNumxxlevelii]
\newcounter{listWWviiiNumxxleveliv}[listWWviiiNumxxleveliii]
\renewcommand\thelistWWviiiNumxxleveli{\arabic{listWWviiiNumxxleveli}}
\renewcommand\thelistWWviiiNumxxlevelii{\arabic{listWWviiiNumxxleveli}.\arabic{listWWviiiNumxxlevelii}}
\renewcommand\thelistWWviiiNumxxleveliii{\arabic{listWWviiiNumxxleveli}.\arabic{listWWviiiNumxxlevelii}.\arabic{listWWviiiNumxxleveliii}}
\renewcommand\thelistWWviiiNumxxleveliv{\arabic{listWWviiiNumxxleveli}.\arabic{listWWviiiNumxxlevelii}.\arabic{listWWviiiNumxxleveliii}.\arabic{listWWviiiNumxxleveliv}}
\newcommand\labellistWWviiiNumxxleveli{\textstyleWWviiiNumxxz{\thelistWWviiiNumxxleveli.}}
\newcommand\labellistWWviiiNumxxlevelii{\textstyleWWviiiNumxxz{\thelistWWviiiNumxxlevelii.}}
\newcommand\labellistWWviiiNumxxleveliii{\textstyleWWviiiNumxxz{\thelistWWviiiNumxxleveliii.}}
\newcommand\labellistWWviiiNumxxleveliv{\textstyleWWviiiNumxxz{\thelistWWviiiNumxxleveliv.}}
\newenvironment{listWWviiiNumxxleveli}{\def\writerlistleftskip{\setlength\leftskip{0.25in}}\def\writerlistparindent{}\def\writerlistlabel{}\def\item{\def\writerlistparindent{\setlength\parindent{-0.25in}}\def\writerlistlabel{\stepcounter{listWWviiiNumxxleveli}\makebox[0cm][l]{\labellistWWviiiNumxxleveli}\hspace{0.0cm}\writerlistremovelabel}}}{}
\newenvironment{listWWviiiNumxxlevelii}{\def\writerlistleftskip{\setlength\leftskip{0.25in}}\def\writerlistparindent{}\def\writerlistlabel{}\def\item{\def\writerlistparindent{\setlength\parindent{-0.25in}}\def\writerlistlabel{\stepcounter{listWWviiiNumxxlevelii}\makebox[0cm][l]{\labellistWWviiiNumxxlevelii}\hspace{0.0cm}\writerlistremovelabel}}}{}
\newenvironment{listWWviiiNumxxleveliii}{\def\writerlistleftskip{\setlength\leftskip{0.5in}}\def\writerlistparindent{}\def\writerlistlabel{}\def\item{\def\writerlistparindent{\setlength\parindent{-0.5in}}\def\writerlistlabel{\stepcounter{listWWviiiNumxxleveliii}\makebox[0cm][l]{\labellistWWviiiNumxxleveliii}\hspace{0.0cm}\writerlistremovelabel}}}{}
\newenvironment{listWWviiiNumxxleveliv}{\def\writerlistleftskip{\setlength\leftskip{0.5in}}\def\writerlistparindent{}\def\writerlistlabel{}\def\item{\def\writerlistparindent{\setlength\parindent{-0.5in}}\def\writerlistlabel{\stepcounter{listWWviiiNumxxleveliv}\makebox[0cm][l]{\labellistWWviiiNumxxleveliv}\hspace{0.0cm}\writerlistremovelabel}}}{}
\newcounter{listWWviiiNumiileveli}
\newcounter{listWWviiiNumiilevelii}[listWWviiiNumiileveli]
\newcounter{listWWviiiNumiileveliii}[listWWviiiNumiilevelii]
\newcounter{listWWviiiNumiileveliv}[listWWviiiNumiileveliii]
\renewcommand\thelistWWviiiNumiileveli{\roman{listWWviiiNumiileveli}}
\renewcommand\thelistWWviiiNumiilevelii{\arabic{listWWviiiNumiilevelii}}
\renewcommand\thelistWWviiiNumiileveliii{\arabic{listWWviiiNumiileveliii}}
\renewcommand\thelistWWviiiNumiileveliv{\arabic{listWWviiiNumiileveliv}}
\newcommand\labellistWWviiiNumiileveli{\textstyleWWviiiNumiiz{(\thelistWWviiiNumiileveli)}}
\newcommand\labellistWWviiiNumiilevelii{\thelistWWviiiNumiilevelii.}
\newcommand\labellistWWviiiNumiileveliii{\thelistWWviiiNumiileveliii.}
\newcommand\labellistWWviiiNumiileveliv{\thelistWWviiiNumiileveliv.}
\newenvironment{listWWviiiNumiileveli}{\def\writerlistleftskip{\setlength\leftskip{1in}}\def\writerlistparindent{}\def\writerlistlabel{}\def\item{\def\writerlistparindent{\setlength\parindent{-0.5in}}\def\writerlistlabel{\stepcounter{listWWviiiNumiileveli}\makebox[0cm][l]{\labellistWWviiiNumiileveli}\hspace{-1.27cm}\writerlistremovelabel}}}{}
\newenvironment{listWWviiiNumiilevelii}{\def\writerlistleftskip{\setlength\leftskip{0.75in}}\def\writerlistparindent{}\def\writerlistlabel{}\def\item{\def\writerlistparindent{\setlength\parindent{-0.25in}}\def\writerlistlabel{\stepcounter{listWWviiiNumiilevelii}\makebox[0cm][l]{\labellistWWviiiNumiilevelii}\hspace{0.25in}\writerlistremovelabel}}}{}
\newenvironment{listWWviiiNumiileveliii}{\def\writerlistleftskip{\setlength\leftskip{1in}}\def\writerlistparindent{}\def\writerlistlabel{}\def\item{\def\writerlistparindent{\setlength\parindent{-0.25in}}\def\writerlistlabel{\stepcounter{listWWviiiNumiileveliii}\makebox[0cm][l]{\labellistWWviiiNumiileveliii}\hspace{0.25in}\writerlistremovelabel}}}{}
\newenvironment{listWWviiiNumiileveliv}{\def\writerlistleftskip{\setlength\leftskip{1.25in}}\def\writerlistparindent{}\def\writerlistlabel{}\def\item{\def\writerlistparindent{\setlength\parindent{-0.25in}}\def\writerlistlabel{\stepcounter{listWWviiiNumiileveliv}\makebox[0cm][l]{\labellistWWviiiNumiileveliv}\hspace{0.25in}\writerlistremovelabel}}}{}
\newcounter{listWWviiiNumxvileveli}
\newcounter{listWWviiiNumxvilevelii}[listWWviiiNumxvileveli]
\newcounter{listWWviiiNumxvileveliii}[listWWviiiNumxvilevelii]
\newcounter{listWWviiiNumxvileveliv}[listWWviiiNumxvileveliii]
\renewcommand\thelistWWviiiNumxvileveli{\roman{listWWviiiNumxvileveli}}
\renewcommand\thelistWWviiiNumxvilevelii{\arabic{listWWviiiNumxvilevelii}}
\renewcommand\thelistWWviiiNumxvileveliii{\arabic{listWWviiiNumxvileveliii}}
\renewcommand\thelistWWviiiNumxvileveliv{\arabic{listWWviiiNumxvileveliv}}
\newcommand\labellistWWviiiNumxvileveli{\textstyleWWviiiNumxviz{(\thelistWWviiiNumxvileveli)}}
\newcommand\labellistWWviiiNumxvilevelii{\thelistWWviiiNumxvilevelii.}
\newcommand\labellistWWviiiNumxvileveliii{\thelistWWviiiNumxvileveliii.}
\newcommand\labellistWWviiiNumxvileveliv{\thelistWWviiiNumxvileveliv.}
\newenvironment{listWWviiiNumxvileveli}{\def\writerlistleftskip{\setlength\leftskip{1in}}\def\writerlistparindent{}\def\writerlistlabel{}\def\item{\def\writerlistparindent{\setlength\parindent{-0.5in}}\def\writerlistlabel{\stepcounter{listWWviiiNumxvileveli}\makebox[0cm][l]{\labellistWWviiiNumxvileveli}\hspace{-1.27cm}\writerlistremovelabel}}}{}
\newenvironment{listWWviiiNumxvilevelii}{\def\writerlistleftskip{\setlength\leftskip{0.75in}}\def\writerlistparindent{}\def\writerlistlabel{}\def\item{\def\writerlistparindent{\setlength\parindent{-0.25in}}\def\writerlistlabel{\stepcounter{listWWviiiNumxvilevelii}\makebox[0cm][l]{\labellistWWviiiNumxvilevelii}\hspace{0.25in}\writerlistremovelabel}}}{}
\newenvironment{listWWviiiNumxvileveliii}{\def\writerlistleftskip{\setlength\leftskip{1in}}\def\writerlistparindent{}\def\writerlistlabel{}\def\item{\def\writerlistparindent{\setlength\parindent{-0.25in}}\def\writerlistlabel{\stepcounter{listWWviiiNumxvileveliii}\makebox[0cm][l]{\labellistWWviiiNumxvileveliii}\hspace{0.25in}\writerlistremovelabel}}}{}
\newenvironment{listWWviiiNumxvileveliv}{\def\writerlistleftskip{\setlength\leftskip{1.25in}}\def\writerlistparindent{}\def\writerlistlabel{}\def\item{\def\writerlistparindent{\setlength\parindent{-0.25in}}\def\writerlistlabel{\stepcounter{listWWviiiNumxvileveliv}\makebox[0cm][l]{\labellistWWviiiNumxvileveliv}\hspace{0.25in}\writerlistremovelabel}}}{}
\newcounter{listWWviiiNumivleveli}
\newcounter{listWWviiiNumivlevelii}[listWWviiiNumivleveli]
\newcounter{listWWviiiNumivleveliii}[listWWviiiNumivlevelii]
\newcounter{listWWviiiNumivleveliv}[listWWviiiNumivleveliii]
\renewcommand\thelistWWviiiNumivleveli{\roman{listWWviiiNumivleveli}}
\renewcommand\thelistWWviiiNumivlevelii{\arabic{listWWviiiNumivlevelii}}
\renewcommand\thelistWWviiiNumivleveliii{\arabic{listWWviiiNumivleveliii}}
\renewcommand\thelistWWviiiNumivleveliv{\arabic{listWWviiiNumivleveliv}}
\newcommand\labellistWWviiiNumivleveli{\textstyleWWviiiNumivz{(\thelistWWviiiNumivleveli)}}
\newcommand\labellistWWviiiNumivlevelii{\thelistWWviiiNumivlevelii.}
\newcommand\labellistWWviiiNumivleveliii{\thelistWWviiiNumivleveliii.}
\newcommand\labellistWWviiiNumivleveliv{\thelistWWviiiNumivleveliv.}
\newenvironment{listWWviiiNumivleveli}{\def\writerlistleftskip{\setlength\leftskip{1in}}\def\writerlistparindent{}\def\writerlistlabel{}\def\item{\def\writerlistparindent{\setlength\parindent{-0.5in}}\def\writerlistlabel{\stepcounter{listWWviiiNumivleveli}\makebox[0cm][l]{\labellistWWviiiNumivleveli}\hspace{-1.27cm}\writerlistremovelabel}}}{}
\newenvironment{listWWviiiNumivlevelii}{\def\writerlistleftskip{\setlength\leftskip{0.75in}}\def\writerlistparindent{}\def\writerlistlabel{}\def\item{\def\writerlistparindent{\setlength\parindent{-0.25in}}\def\writerlistlabel{\stepcounter{listWWviiiNumivlevelii}\makebox[0cm][l]{\labellistWWviiiNumivlevelii}\hspace{0.25in}\writerlistremovelabel}}}{}
\newenvironment{listWWviiiNumivleveliii}{\def\writerlistleftskip{\setlength\leftskip{1in}}\def\writerlistparindent{}\def\writerlistlabel{}\def\item{\def\writerlistparindent{\setlength\parindent{-0.25in}}\def\writerlistlabel{\stepcounter{listWWviiiNumivleveliii}\makebox[0cm][l]{\labellistWWviiiNumivleveliii}\hspace{0.25in}\writerlistremovelabel}}}{}
\newenvironment{listWWviiiNumivleveliv}{\def\writerlistleftskip{\setlength\leftskip{1.25in}}\def\writerlistparindent{}\def\writerlistlabel{}\def\item{\def\writerlistparindent{\setlength\parindent{-0.25in}}\def\writerlistlabel{\stepcounter{listWWviiiNumivleveliv}\makebox[0cm][l]{\labellistWWviiiNumivleveliv}\hspace{0.25in}\writerlistremovelabel}}}{}
\newcounter{listWWviiiNumvleveli}
\newcounter{listWWviiiNumvlevelii}[listWWviiiNumvleveli]
\newcounter{listWWviiiNumvleveliii}[listWWviiiNumvlevelii]
\newcounter{listWWviiiNumvleveliv}[listWWviiiNumvleveliii]
\renewcommand\thelistWWviiiNumvleveli{\roman{listWWviiiNumvleveli}}
\renewcommand\thelistWWviiiNumvlevelii{\arabic{listWWviiiNumvlevelii}}
\renewcommand\thelistWWviiiNumvleveliii{\arabic{listWWviiiNumvleveliii}}
\renewcommand\thelistWWviiiNumvleveliv{\arabic{listWWviiiNumvleveliv}}
\newcommand\labellistWWviiiNumvleveli{\textstyleWWviiiNumvz{(\thelistWWviiiNumvleveli)}}
\newcommand\labellistWWviiiNumvlevelii{\thelistWWviiiNumvlevelii.}
\newcommand\labellistWWviiiNumvleveliii{\thelistWWviiiNumvleveliii.}
\newcommand\labellistWWviiiNumvleveliv{\thelistWWviiiNumvleveliv.}
\newenvironment{listWWviiiNumvleveli}{\def\writerlistleftskip{\setlength\leftskip{1in}}\def\writerlistparindent{}\def\writerlistlabel{}\def\item{\def\writerlistparindent{\setlength\parindent{-0.5in}}\def\writerlistlabel{\stepcounter{listWWviiiNumvleveli}\makebox[0cm][l]{\labellistWWviiiNumvleveli}\hspace{-1.27cm}\writerlistremovelabel}}}{}
\newenvironment{listWWviiiNumvlevelii}{\def\writerlistleftskip{\setlength\leftskip{0.75in}}\def\writerlistparindent{}\def\writerlistlabel{}\def\item{\def\writerlistparindent{\setlength\parindent{-0.25in}}\def\writerlistlabel{\stepcounter{listWWviiiNumvlevelii}\makebox[0cm][l]{\labellistWWviiiNumvlevelii}\hspace{0.25in}\writerlistremovelabel}}}{}
\newenvironment{listWWviiiNumvleveliii}{\def\writerlistleftskip{\setlength\leftskip{1in}}\def\writerlistparindent{}\def\writerlistlabel{}\def\item{\def\writerlistparindent{\setlength\parindent{-0.25in}}\def\writerlistlabel{\stepcounter{listWWviiiNumvleveliii}\makebox[0cm][l]{\labellistWWviiiNumvleveliii}\hspace{0.25in}\writerlistremovelabel}}}{}
\newenvironment{listWWviiiNumvleveliv}{\def\writerlistleftskip{\setlength\leftskip{1.25in}}\def\writerlistparindent{}\def\writerlistlabel{}\def\item{\def\writerlistparindent{\setlength\parindent{-0.25in}}\def\writerlistlabel{\stepcounter{listWWviiiNumvleveliv}\makebox[0cm][l]{\labellistWWviiiNumvleveliv}\hspace{0.25in}\writerlistremovelabel}}}{}
\newcounter{listWWviiiNumxvleveli}
\newcounter{listWWviiiNumxvlevelii}[listWWviiiNumxvleveli]
\newcounter{listWWviiiNumxvleveliii}[listWWviiiNumxvlevelii]
\newcounter{listWWviiiNumxvleveliv}[listWWviiiNumxvleveliii]
\renewcommand\thelistWWviiiNumxvleveli{\roman{listWWviiiNumxvleveli}}
\renewcommand\thelistWWviiiNumxvlevelii{\arabic{listWWviiiNumxvlevelii}}
\renewcommand\thelistWWviiiNumxvleveliii{\arabic{listWWviiiNumxvleveliii}}
\renewcommand\thelistWWviiiNumxvleveliv{\arabic{listWWviiiNumxvleveliv}}
\newcommand\labellistWWviiiNumxvleveli{\textstyleWWviiiNumxvz{(\thelistWWviiiNumxvleveli)}}
\newcommand\labellistWWviiiNumxvlevelii{\thelistWWviiiNumxvlevelii.}
\newcommand\labellistWWviiiNumxvleveliii{\thelistWWviiiNumxvleveliii.}
\newcommand\labellistWWviiiNumxvleveliv{\thelistWWviiiNumxvleveliv.}
\newenvironment{listWWviiiNumxvleveli}{\def\writerlistleftskip{\setlength\leftskip{1in}}\def\writerlistparindent{}\def\writerlistlabel{}\def\item{\def\writerlistparindent{\setlength\parindent{-0.5in}}\def\writerlistlabel{\stepcounter{listWWviiiNumxvleveli}\makebox[0cm][l]{\labellistWWviiiNumxvleveli}\hspace{-1.27cm}\writerlistremovelabel}}}{}
\newenvironment{listWWviiiNumxvlevelii}{\def\writerlistleftskip{\setlength\leftskip{0.75in}}\def\writerlistparindent{}\def\writerlistlabel{}\def\item{\def\writerlistparindent{\setlength\parindent{-0.25in}}\def\writerlistlabel{\stepcounter{listWWviiiNumxvlevelii}\makebox[0cm][l]{\labellistWWviiiNumxvlevelii}\hspace{0.25in}\writerlistremovelabel}}}{}
\newenvironment{listWWviiiNumxvleveliii}{\def\writerlistleftskip{\setlength\leftskip{1in}}\def\writerlistparindent{}\def\writerlistlabel{}\def\item{\def\writerlistparindent{\setlength\parindent{-0.25in}}\def\writerlistlabel{\stepcounter{listWWviiiNumxvleveliii}\makebox[0cm][l]{\labellistWWviiiNumxvleveliii}\hspace{0.25in}\writerlistremovelabel}}}{}
\newenvironment{listWWviiiNumxvleveliv}{\def\writerlistleftskip{\setlength\leftskip{1.25in}}\def\writerlistparindent{}\def\writerlistlabel{}\def\item{\def\writerlistparindent{\setlength\parindent{-0.25in}}\def\writerlistlabel{\stepcounter{listWWviiiNumxvleveliv}\makebox[0cm][l]{\labellistWWviiiNumxvleveliv}\hspace{0.25in}\writerlistremovelabel}}}{}
\newcounter{listWWviiiNumxilevelii}
\newcounter{listWWviiiNumxileveliii}[listWWviiiNumxilevelii]
\newcounter{listWWviiiNumxileveliv}[listWWviiiNumxileveliii]
\renewcommand\thelistWWviiiNumxilevelii{\arabic{listWWviiiNumxilevelii}}
\renewcommand\thelistWWviiiNumxileveliii{\arabic{listWWviiiNumxileveliii}}
\renewcommand\thelistWWviiiNumxileveliv{\arabic{listWWviiiNumxileveliv}}
\newcommand\labellistWWviiiNumxileveli{\textstyleWWviiiNumxiz{{\textbullet}}}
\newcommand\labellistWWviiiNumxilevelii{\thelistWWviiiNumxilevelii.}
\newcommand\labellistWWviiiNumxileveliii{\thelistWWviiiNumxileveliii.}
\newcommand\labellistWWviiiNumxileveliv{\thelistWWviiiNumxileveliv.}
\newenvironment{listWWviiiNumxileveli}{\def\writerlistleftskip{\setlength\leftskip{0.5in}}\def\writerlistparindent{}\def\writerlistlabel{}\def\item{\def\writerlistparindent{\setlength\parindent{-0.25in}}\def\writerlistlabel{\makebox[0cm][l]{\labellistWWviiiNumxileveli}\hspace{-0.635cm}\writerlistremovelabel}}}{}
\newenvironment{listWWviiiNumxilevelii}{\def\writerlistleftskip{\setlength\leftskip{0.75in}}\def\writerlistparindent{}\def\writerlistlabel{}\def\item{\def\writerlistparindent{\setlength\parindent{-0.25in}}\def\writerlistlabel{\stepcounter{listWWviiiNumxilevelii}\makebox[0cm][l]{\labellistWWviiiNumxilevelii}\hspace{0.25in}\writerlistremovelabel}}}{}
\newenvironment{listWWviiiNumxileveliii}{\def\writerlistleftskip{\setlength\leftskip{1in}}\def\writerlistparindent{}\def\writerlistlabel{}\def\item{\def\writerlistparindent{\setlength\parindent{-0.25in}}\def\writerlistlabel{\stepcounter{listWWviiiNumxileveliii}\makebox[0cm][l]{\labellistWWviiiNumxileveliii}\hspace{0.25in}\writerlistremovelabel}}}{}
\newenvironment{listWWviiiNumxileveliv}{\def\writerlistleftskip{\setlength\leftskip{1.25in}}\def\writerlistparindent{}\def\writerlistlabel{}\def\item{\def\writerlistparindent{\setlength\parindent{-0.25in}}\def\writerlistlabel{\stepcounter{listWWviiiNumxileveliv}\makebox[0cm][l]{\labellistWWviiiNumxileveliv}\hspace{0.25in}\writerlistremovelabel}}}{}
\newcounter{listWWviiiNumxiilevelii}
\newcounter{listWWviiiNumxiileveliii}[listWWviiiNumxiilevelii]
\newcounter{listWWviiiNumxiileveliv}[listWWviiiNumxiileveliii]
\renewcommand\thelistWWviiiNumxiilevelii{\arabic{listWWviiiNumxiilevelii}}
\renewcommand\thelistWWviiiNumxiileveliii{\arabic{listWWviiiNumxiileveliii}}
\renewcommand\thelistWWviiiNumxiileveliv{\arabic{listWWviiiNumxiileveliv}}
\newcommand\labellistWWviiiNumxiileveli{\textstyleWWviiiNumxiiz{{\textbullet}}}
\newcommand\labellistWWviiiNumxiilevelii{\thelistWWviiiNumxiilevelii.}
\newcommand\labellistWWviiiNumxiileveliii{\thelistWWviiiNumxiileveliii.}
\newcommand\labellistWWviiiNumxiileveliv{\thelistWWviiiNumxiileveliv.}
\newenvironment{listWWviiiNumxiileveli}{\def\writerlistleftskip{\setlength\leftskip{0.5in}}\def\writerlistparindent{}\def\writerlistlabel{}\def\item{\def\writerlistparindent{\setlength\parindent{-0.25in}}\def\writerlistlabel{\makebox[0cm][l]{\labellistWWviiiNumxiileveli}\hspace{-0.635cm}\writerlistremovelabel}}}{}
\newenvironment{listWWviiiNumxiilevelii}{\def\writerlistleftskip{\setlength\leftskip{0.75in}}\def\writerlistparindent{}\def\writerlistlabel{}\def\item{\def\writerlistparindent{\setlength\parindent{-0.25in}}\def\writerlistlabel{\stepcounter{listWWviiiNumxiilevelii}\makebox[0cm][l]{\labellistWWviiiNumxiilevelii}\hspace{0.25in}\writerlistremovelabel}}}{}
\newenvironment{listWWviiiNumxiileveliii}{\def\writerlistleftskip{\setlength\leftskip{1in}}\def\writerlistparindent{}\def\writerlistlabel{}\def\item{\def\writerlistparindent{\setlength\parindent{-0.25in}}\def\writerlistlabel{\stepcounter{listWWviiiNumxiileveliii}\makebox[0cm][l]{\labellistWWviiiNumxiileveliii}\hspace{0.25in}\writerlistremovelabel}}}{}
\newenvironment{listWWviiiNumxiileveliv}{\def\writerlistleftskip{\setlength\leftskip{1.25in}}\def\writerlistparindent{}\def\writerlistlabel{}\def\item{\def\writerlistparindent{\setlength\parindent{-0.25in}}\def\writerlistlabel{\stepcounter{listWWviiiNumxiileveliv}\makebox[0cm][l]{\labellistWWviiiNumxiileveliv}\hspace{0.25in}\writerlistremovelabel}}}{}
\newcounter{listWWviiiNumixlevelii}
\newcounter{listWWviiiNumixleveliii}[listWWviiiNumixlevelii]
\newcounter{listWWviiiNumixleveliv}[listWWviiiNumixleveliii]
\renewcommand\thelistWWviiiNumixlevelii{\arabic{listWWviiiNumixlevelii}}
\renewcommand\thelistWWviiiNumixleveliii{\arabic{listWWviiiNumixleveliii}}
\renewcommand\thelistWWviiiNumixleveliv{\arabic{listWWviiiNumixleveliv}}
\newcommand\labellistWWviiiNumixleveli{\textstyleWWviiiNumixz{{\textbullet}}}
\newcommand\labellistWWviiiNumixlevelii{\thelistWWviiiNumixlevelii.}
\newcommand\labellistWWviiiNumixleveliii{\thelistWWviiiNumixleveliii.}
\newcommand\labellistWWviiiNumixleveliv{\thelistWWviiiNumixleveliv.}
\newenvironment{listWWviiiNumixleveli}{\def\writerlistleftskip{\setlength\leftskip{0.5in}}\def\writerlistparindent{}\def\writerlistlabel{}\def\item{\def\writerlistparindent{\setlength\parindent{-0.25in}}\def\writerlistlabel{\makebox[0cm][l]{\labellistWWviiiNumixleveli}\hspace{-0.635cm}\writerlistremovelabel}}}{}
\newenvironment{listWWviiiNumixlevelii}{\def\writerlistleftskip{\setlength\leftskip{0.75in}}\def\writerlistparindent{}\def\writerlistlabel{}\def\item{\def\writerlistparindent{\setlength\parindent{-0.25in}}\def\writerlistlabel{\stepcounter{listWWviiiNumixlevelii}\makebox[0cm][l]{\labellistWWviiiNumixlevelii}\hspace{0.25in}\writerlistremovelabel}}}{}
\newenvironment{listWWviiiNumixleveliii}{\def\writerlistleftskip{\setlength\leftskip{1in}}\def\writerlistparindent{}\def\writerlistlabel{}\def\item{\def\writerlistparindent{\setlength\parindent{-0.25in}}\def\writerlistlabel{\stepcounter{listWWviiiNumixleveliii}\makebox[0cm][l]{\labellistWWviiiNumixleveliii}\hspace{0.25in}\writerlistremovelabel}}}{}
\newenvironment{listWWviiiNumixleveliv}{\def\writerlistleftskip{\setlength\leftskip{1.25in}}\def\writerlistparindent{}\def\writerlistlabel{}\def\item{\def\writerlistparindent{\setlength\parindent{-0.25in}}\def\writerlistlabel{\stepcounter{listWWviiiNumixleveliv}\makebox[0cm][l]{\labellistWWviiiNumixleveliv}\hspace{0.25in}\writerlistremovelabel}}}{}
\newcounter{listWWviiiNumviilevelii}
\newcounter{listWWviiiNumviileveliii}[listWWviiiNumviilevelii]
\newcounter{listWWviiiNumviileveliv}[listWWviiiNumviileveliii]
\renewcommand\thelistWWviiiNumviilevelii{\arabic{listWWviiiNumviilevelii}}
\renewcommand\thelistWWviiiNumviileveliii{\arabic{listWWviiiNumviileveliii}}
\renewcommand\thelistWWviiiNumviileveliv{\arabic{listWWviiiNumviileveliv}}
\newcommand\labellistWWviiiNumviileveli{\textstyleWWviiiNumviiz{{\textbullet}}}
\newcommand\labellistWWviiiNumviilevelii{\thelistWWviiiNumviilevelii.}
\newcommand\labellistWWviiiNumviileveliii{\thelistWWviiiNumviileveliii.}
\newcommand\labellistWWviiiNumviileveliv{\thelistWWviiiNumviileveliv.}
\newenvironment{listWWviiiNumviileveli}{\def\writerlistleftskip{\setlength\leftskip{0.5in}}\def\writerlistparindent{}\def\writerlistlabel{}\def\item{\def\writerlistparindent{\setlength\parindent{-0.25in}}\def\writerlistlabel{\makebox[0cm][l]{\labellistWWviiiNumviileveli}\hspace{-0.635cm}\writerlistremovelabel}}}{}
\newenvironment{listWWviiiNumviilevelii}{\def\writerlistleftskip{\setlength\leftskip{0.75in}}\def\writerlistparindent{}\def\writerlistlabel{}\def\item{\def\writerlistparindent{\setlength\parindent{-0.25in}}\def\writerlistlabel{\stepcounter{listWWviiiNumviilevelii}\makebox[0cm][l]{\labellistWWviiiNumviilevelii}\hspace{0.25in}\writerlistremovelabel}}}{}
\newenvironment{listWWviiiNumviileveliii}{\def\writerlistleftskip{\setlength\leftskip{1in}}\def\writerlistparindent{}\def\writerlistlabel{}\def\item{\def\writerlistparindent{\setlength\parindent{-0.25in}}\def\writerlistlabel{\stepcounter{listWWviiiNumviileveliii}\makebox[0cm][l]{\labellistWWviiiNumviileveliii}\hspace{0.25in}\writerlistremovelabel}}}{}
\newenvironment{listWWviiiNumviileveliv}{\def\writerlistleftskip{\setlength\leftskip{1.25in}}\def\writerlistparindent{}\def\writerlistlabel{}\def\item{\def\writerlistparindent{\setlength\parindent{-0.25in}}\def\writerlistlabel{\stepcounter{listWWviiiNumviileveliv}\makebox[0cm][l]{\labellistWWviiiNumviileveliv}\hspace{0.25in}\writerlistremovelabel}}}{}
\newcounter{listWWviiiNumxiiileveli}
\newcounter{listWWviiiNumxiiilevelii}[listWWviiiNumxiiileveli]
\newcounter{listWWviiiNumxiiileveliii}[listWWviiiNumxiiilevelii]
\newcounter{listWWviiiNumxiiileveliv}[listWWviiiNumxiiileveliii]
\renewcommand\thelistWWviiiNumxiiileveli{\alph{listWWviiiNumxiiileveli}}
\renewcommand\thelistWWviiiNumxiiilevelii{\arabic{listWWviiiNumxiiilevelii}}
\renewcommand\thelistWWviiiNumxiiileveliii{\arabic{listWWviiiNumxiiileveliii}}
\renewcommand\thelistWWviiiNumxiiileveliv{\arabic{listWWviiiNumxiiileveliv}}
\newcommand\labellistWWviiiNumxiiileveli{\textstyleWWviiiNumxiiiz{\thelistWWviiiNumxiiileveli)}}
\newcommand\labellistWWviiiNumxiiilevelii{\thelistWWviiiNumxiiilevelii.}
\newcommand\labellistWWviiiNumxiiileveliii{\thelistWWviiiNumxiiileveliii.}
\newcommand\labellistWWviiiNumxiiileveliv{\thelistWWviiiNumxiiileveliv.}
\newenvironment{listWWviiiNumxiiileveli}{\def\writerlistleftskip{\setlength\leftskip{0.75in}}\def\writerlistparindent{}\def\writerlistlabel{}\def\item{\def\writerlistparindent{\setlength\parindent{-0.25in}}\def\writerlistlabel{\stepcounter{listWWviiiNumxiiileveli}\makebox[0cm][l]{\labellistWWviiiNumxiiileveli}\hspace{-1.27cm}\writerlistremovelabel}}}{}
\newenvironment{listWWviiiNumxiiilevelii}{\def\writerlistleftskip{\setlength\leftskip{0.75in}}\def\writerlistparindent{}\def\writerlistlabel{}\def\item{\def\writerlistparindent{\setlength\parindent{-0.25in}}\def\writerlistlabel{\stepcounter{listWWviiiNumxiiilevelii}\makebox[0cm][l]{\labellistWWviiiNumxiiilevelii}\hspace{0.25in}\writerlistremovelabel}}}{}
\newenvironment{listWWviiiNumxiiileveliii}{\def\writerlistleftskip{\setlength\leftskip{1in}}\def\writerlistparindent{}\def\writerlistlabel{}\def\item{\def\writerlistparindent{\setlength\parindent{-0.25in}}\def\writerlistlabel{\stepcounter{listWWviiiNumxiiileveliii}\makebox[0cm][l]{\labellistWWviiiNumxiiileveliii}\hspace{0.25in}\writerlistremovelabel}}}{}
\newenvironment{listWWviiiNumxiiileveliv}{\def\writerlistleftskip{\setlength\leftskip{1.25in}}\def\writerlistparindent{}\def\writerlistlabel{}\def\item{\def\writerlistparindent{\setlength\parindent{-0.25in}}\def\writerlistlabel{\stepcounter{listWWviiiNumxiiileveliv}\makebox[0cm][l]{\labellistWWviiiNumxiiileveliv}\hspace{0.25in}\writerlistremovelabel}}}{}
\newcounter{listWWviiiNumxivleveli}
\newcounter{listWWviiiNumxivlevelii}[listWWviiiNumxivleveli]
\newcounter{listWWviiiNumxivleveliii}[listWWviiiNumxivlevelii]
\newcounter{listWWviiiNumxivleveliv}[listWWviiiNumxivleveliii]
\renewcommand\thelistWWviiiNumxivleveli{\roman{listWWviiiNumxivleveli}}
\renewcommand\thelistWWviiiNumxivlevelii{\arabic{listWWviiiNumxivlevelii}}
\renewcommand\thelistWWviiiNumxivleveliii{\arabic{listWWviiiNumxivleveliii}}
\renewcommand\thelistWWviiiNumxivleveliv{\arabic{listWWviiiNumxivleveliv}}
\newcommand\labellistWWviiiNumxivleveli{\textstyleWWviiiNumxivz{(\thelistWWviiiNumxivleveli)}}
\newcommand\labellistWWviiiNumxivlevelii{\thelistWWviiiNumxivlevelii.}
\newcommand\labellistWWviiiNumxivleveliii{\thelistWWviiiNumxivleveliii.}
\newcommand\labellistWWviiiNumxivleveliv{\thelistWWviiiNumxivleveliv.}
\newenvironment{listWWviiiNumxivleveli}{\def\writerlistleftskip{\setlength\leftskip{1in}}\def\writerlistparindent{}\def\writerlistlabel{}\def\item{\def\writerlistparindent{\setlength\parindent{-0.5in}}\def\writerlistlabel{\stepcounter{listWWviiiNumxivleveli}\makebox[0cm][l]{\labellistWWviiiNumxivleveli}\hspace{-1.27cm}\writerlistremovelabel}}}{}
\newenvironment{listWWviiiNumxivlevelii}{\def\writerlistleftskip{\setlength\leftskip{0.75in}}\def\writerlistparindent{}\def\writerlistlabel{}\def\item{\def\writerlistparindent{\setlength\parindent{-0.25in}}\def\writerlistlabel{\stepcounter{listWWviiiNumxivlevelii}\makebox[0cm][l]{\labellistWWviiiNumxivlevelii}\hspace{0.25in}\writerlistremovelabel}}}{}
\newenvironment{listWWviiiNumxivleveliii}{\def\writerlistleftskip{\setlength\leftskip{1in}}\def\writerlistparindent{}\def\writerlistlabel{}\def\item{\def\writerlistparindent{\setlength\parindent{-0.25in}}\def\writerlistlabel{\stepcounter{listWWviiiNumxivleveliii}\makebox[0cm][l]{\labellistWWviiiNumxivleveliii}\hspace{0.25in}\writerlistremovelabel}}}{}
\newenvironment{listWWviiiNumxivleveliv}{\def\writerlistleftskip{\setlength\leftskip{1.25in}}\def\writerlistparindent{}\def\writerlistlabel{}\def\item{\def\writerlistparindent{\setlength\parindent{-0.25in}}\def\writerlistlabel{\stepcounter{listWWviiiNumxivleveliv}\makebox[0cm][l]{\labellistWWviiiNumxivleveliv}\hspace{0.25in}\writerlistremovelabel}}}{}
\setlength\tabcolsep{1mm}
\renewcommand\arraystretch{1.3}
\title{Pre-nominal Inflections in German as a Result of Different Mechanisms}
\author{}
\date{2024-12-05}
\begin{document}
\clearpage\setcounter{page}{1}\begin{styleStandard}
Semantically Vacuous Elements in German: 
\end{styleStandard}

\begin{styleStandard}
Adjectival Inflections and the Article \textit{ein}\newline
 
\end{styleStandard}

\begin{styleStandard}
\newline
Dorian Roehrs
\end{styleStandard}

\clearpage\begin{styleStandard}\bfseries
Table of Contents
\end{styleStandard}

\begin{styleStandard}
Preface\ \ \ \ \ \ \ \ \ \ \ \ \ \ \ \ \ \ viii
\end{styleStandard}

\begin{styleStandard}
Abbreviations \ \ \ \ \ \ \ \ \ \ \ \ \ \ \ \ x
\end{styleStandard}

\begin{styleStandard}
Chapter 1: Introduction\ \ \ \ \ \ \ \ \ \ \ \ \ \ 1
\end{styleStandard}

\begin{styleStandard}\bfseries
1.\ \ Vacuous Elements – an Imperfection of Language?
\end{styleStandard}

\begin{styleStandard}\itshape
\ \ 1.1.\ \ The Clause
\end{styleStandard}

\begin{styleStandard}\itshape
\ \ 1.2.\ \ The Noun Phrase
\end{styleStandard}

\begin{styleStandard}\bfseries
2.\ \ German as the Language under Investigation
\end{styleStandard}

\begin{styleStandard}\itshape
\ \ 2.1.\ \ Some Cross-linguistic Differences
\end{styleStandard}

\begin{styleStandard}
\ \ \ \ 2.1.1. Canonical and Non-canonical Constructions
\end{styleStandard}

\begin{styleStandard}
2.1.2. Cross-linguistic Differences with Adjectival Inflections
\end{styleStandard}

\begin{styleStandard}
\ \ \ \ 2.1.3. Cross-linguistic Differences with Singular Indefinite Articles
\end{styleStandard}

\begin{styleStandard}\itshape
\ \ 2.2.\ \ The Indefinite Article in Plural Contexts
\end{styleStandard}

\begin{styleStandard}
\ \ \ \ 2.2.1. Occurrence
\end{styleStandard}

\begin{styleStandard}
\ \ \ \ 2.2.2. Linguistic Properties
\end{styleStandard}

\begin{styleStandard}\bfseries
3.\ \ Basic Properties and Main Hypotheses
\end{styleStandard}

\begin{styleStandard}
\textit{\ \ 3.1.\ \ Adjectival Inflections and the Article }ein
\end{styleStandard}

\begin{styleStandard}
\ \ \ \ 3.1.1. Basic Properties of Adjectival Inflections
\end{styleStandard}

\begin{styleStandard}
\ \ \ \ 3.1.2. Basic Properties of \textit{ein}
\end{styleStandard}

\begin{styleStandard}
\textit{\ \ \ 3.2.\ \ Main Hypotheses and Relatedness of Adjectival Inflections and }ein
\end{styleStandard}

\begin{styleStandard}
\ \ \ \ 3.2.1. Main Hypotheses
\end{styleStandard}

\begin{styleStandard}
\ \ \ \ 3.2.2. Relatedness of Adjectival Inflections and \textit{ein}
\end{styleStandard}

\begin{styleStandard}\bfseries
4.\ \ Basic Assumptions 
\end{styleStandard}

\begin{styleStandard}\itshape
\ \ 4.1.\ \ Basic Assumptions about Structure
\end{styleStandard}

\begin{styleStandard}
\ \ \ \ 4.1.1. DPs as a Whole
\end{styleStandard}

\begin{styleStandard}
\ \ \ \ 4.1.2. Determiners, Articles, and Determiner-like Elements
\end{styleStandard}

\begin{styleStandard}
\ \ \ \ 4.1.3. Adjectives, Numerals, and Quantifiers
\end{styleStandard}

\begin{styleStandard}\itshape
\ \ 4.2.\ \ Distributed Morphology and Type Theory
\end{styleStandard}

\begin{styleStandard}
\ \ \ \ 4.2.1. Distributed Morphology
\end{styleStandard}

\begin{styleStandard}
\ \ \ \ 4.2.2. Type Theory
\end{styleStandard}

\begin{styleStandard}\bfseries
5.\ \ Overview of the Chapters
\end{styleStandard}

\begin{styleStandard}
Chapter 2: The Structural Nature of Adjectival Inflections\ \ 53\newline
\textbf{1.\ \ Introduction}
\end{styleStandard}

\begin{styleStandard}\itshape
1.1.\ \ The Strong/Weak Alternation of Adjectives
\end{styleStandard}

\begin{styleStandard}\itshape
1.2.\ \ Adjectival Inflections and (In-)definiteness
\end{styleStandard}

\begin{styleStandard}
\textit{1.3.\ \ Outlook\newline
}\textbf{2.\ \ Adjectival Inflections in Canonical DPs\newline
}\textit{ \ \ 2.1.\ \ Weak Adjectives: Impoverishment\newline
} \ \ \ \ 2.1.1. Basic Alternation and Traditional Generalizations\newline
\ \ \ \ 2.1.2. Inventory of Determiners and Adjectival Inflections
\end{styleStandard}

\begin{styleStandard}
\ \  \ \ \ \ 2.1.3. Sauerland (1996)
\end{styleStandard}

\begin{styleStandard}
2.1.4. Basic Proposal
\end{styleStandard}

\begin{styleStandard}
2.1.5. The Proposal in More Detail: Vocabulary Insertion Rules
\end{styleStandard}

\begin{styleStandard}
2.1.6. The Proposal in More Detail: Structural Positions of Inflections
\end{styleStandard}

\begin{styleStandard}
2.1.7. The Proposal in More Detail: Impoverishment\newline
\textit{2.2.\ \ Strong or Weak Adjectives in Canonical DPs: }Ein\textit{{}-words}
\end{styleStandard}

\begin{styleStandard}
2.2.1. Adjectives after \textit{ein}{}-words: Strong Inflections
\end{styleStandard}

\begin{styleStandard}
2.2.2. Adjectives after \textit{ein}{}-words: Uninflected \textit{ein}
\end{styleStandard}

\begin{styleStandard}
2.2.3. Adjectives after \textit{ein}{}-words: Some Extensions 
\end{styleStandard}

\begin{styleStandard}\itshape
2.3.\ \ Strong or Weak Adjectives in Canonical DPs: Null Articles and Saxon Genitives
\end{styleStandard}

\begin{styleStandard}\itshape
2.4.\ \ Adjectives in Extended-Adjective Constructions and After Numerals
\end{styleStandard}

\begin{styleStandard}
\textbf{3.\ \ Adjectival Inflections in Non-canonical DPs\newline
}\textit{ \ \ 3.1.\ \ Regular DP vs. Low Right-Adjunction: Close Appositions\newline
 \ \ 3.2.\ \ Regular DP vs. Mid Right-Adjunction: Indefinite Pronoun Constructions\newline
\ \ 3.3.\ \ Regular DP vs. Mid Right-Adjunction: Noun-Adjective Exclamatives}
\end{styleStandard}

\begin{styleStandard}
\textit{\ \ \ 3.4.\ \ Regular DP vs. High Right-Adjunction: Loose Appositions\newline
 \ \ 3.5.\ \ Regular DP vs. Complex Specifier inside DP: Dis-agreement in Pronominal DPs\newline
\ \ 3.6.\ \ Regular DP vs. Separate Base-generation: Split Topicalizations}
\end{styleStandard}

\begin{styleStandard}\itshape
\ \ 3.7.\ \ Regular DP vs. Complex Compound Modifier: Nominal Compounds 
\end{styleStandard}

\begin{styleStandard}
\textit{\ \ \ 3.8.\ \ Regular DP vs. Outside of DP Proper: Predeterminers\newline
\ \ 3.9.\ \ Regular DP vs. No Determiner: Vocatives}
\end{styleStandard}

\begin{styleStandard}\bfseries
4.\ \ Adjectival Inflections on Determiners and Predeterminers
\end{styleStandard}

\begin{styleStandard}\bfseries
5.\ \ A Brief Critique of some Previous Proposals
\end{styleStandard}

\begin{styleStandard}
\textit{\ \ \ 5.1.\ \ Olsen (1989a, 1989b, 1991a, 1991b)}
\end{styleStandard}

\begin{styleFootnote}\itshape
\ \ 5.2.\ \ Schoorlemmer (2009)
\end{styleFootnote}

\begin{styleFootnote}\itshape
\ \ 5.3.\ \ Murphy (2018)
\end{styleFootnote}

\begin{styleStandard}\bfseries
6.\ \ Conclusion\newline

\end{styleStandard}

\begin{styleFootnote}
Chapter 3: Variation and Secondary Mechanisms\ \ \ \ \ \ 124
\end{styleFootnote}

\begin{styleFootnote}\bfseries
1.\ \ Introduction
\end{styleFootnote}

\begin{styleFootnote}\bfseries
2.\ \ Canonical DPs with Unexpected Weak Adjectives: Phonetic Rule
\end{styleFootnote}

\begin{styleFootnote}\itshape
\ \ 2.1.\ \ Two Adjectives without a Determiner
\end{styleFootnote}

\begin{styleFootnote}\itshape
\ \ 2.2.\ \ Indefinite Pronoun Constructions Revisited
\end{styleFootnote}

\begin{styleFootnote}\bfseries
3.\ \ Canonical DPs Involving Adjectives with Optional or no Inflections and Definite \ \ Determiners with no Inflections
\end{styleFootnote}

\begin{styleFootnote}\bfseries
4. \ \ Canonical DPs with Unexpected Weak Determiners: Impoverishment Rule 2
\end{styleFootnote}

\begin{styleFootnote}\bfseries
5.\ \ Canonical DPs with Unexpected Strong Adjectives: Pronominal Determiners
\end{styleFootnote}

\begin{styleFootnote}\itshape
\ \ 5.1.\ \ Data
\end{styleFootnote}

\begin{styleFootnote}\itshape
\ \ 5.2.\ \ Three Types of Proposals
\end{styleFootnote}

\begin{styleFootnote}
\ \ \ \ 5.2.1. One Structure: Agreeing vs. Non-agreeing Pronominal Elements
\end{styleFootnote}

\begin{styleFootnote}
\ \ \ \ 5.2.2. One Structure: Pronominal Elements in Different Positions
\end{styleFootnote}

\begin{styleFootnote}
\ \ \ \ 5.2.3. Two Different Structures: Complementation and Adjunction
\end{styleFootnote}

\begin{styleFootnote}
\ \ \ \ 5.2.4. Absence of Weak Adjectives in Certain Instances
\end{styleFootnote}

\begin{styleFootnote}\itshape
\ \ 5.3.\ \ Pronominal Determiners and Complementation
\end{styleFootnote}

\begin{styleFootnote}
\ \ \ \ 5.3.1. Structures and Feature Specifications
\end{styleFootnote}

\begin{styleFootnote}
\ \ \ \ 5.3.2. Vocabulary Insertion Rules for Pronominal Elements
\end{styleFootnote}

\begin{styleFootnote}
\ \ \ \ 5.3.3. Preferences for Strong or Weak Adjectives
\end{styleFootnote}

\begin{styleFootnote}\itshape
\ \ 5.4.\ \ Summary of the Discussion
\end{styleFootnote}

\begin{styleFootnote}
\textbf{6.\ \ Non-canonical Structures with Optional Inflections on Predeterminer }\textbf{\textit{alle}}
\end{styleFootnote}

\begin{styleFootnote}\bfseries
7.\ \ Revisiting Two Traditional Generalizations of Adjectival Inflections
\end{styleFootnote}

\begin{styleFootnote}\bfseries
8.\ \ Dialectal Variation: Mannheim German
\end{styleFootnote}

\begin{styleFootnote}\bfseries
9.\ \ Conclusion
\end{styleFootnote}

\begin{styleFootnote}
Chapter 4: Consequences for Other Analyses\ \ \ \ \ \ 179
\end{styleFootnote}

\begin{styleFootnote}\bfseries
1.\ \ Introduction
\end{styleFootnote}

\begin{styleStandard}\bfseries
2.\ \ Weak Adjectives in the Context of Spurious Indefinite Articles
\end{styleStandard}

\begin{styleStandard}\itshape
\ \ 2.1.\ \ Weak Adjectives in Structures Involving Predicate Inversion
\end{styleStandard}

\begin{styleStandard}\itshape
\ \ 2.2.\ \ Weak Adjectives in Structures Involving Null Nouns
\end{styleStandard}

\begin{styleStandard}\bfseries
3.\ \ Strong Adjectives in Structures Involving Split Topicalization
\end{styleStandard}

\begin{styleFootnote}
\textit{\ \ \ 3.1.\ \ Movement out of the Source}
\end{styleFootnote}

\begin{styleFooter}
\textit{\ \ \ 3.2. \ \ Movement but not out of the Source}
\end{styleFooter}

\begin{styleStandard}
\textit{\ \ \ 3.3.\ \ Separate Base-generation}
\end{styleStandard}

\begin{styleStandard}
\textit{\ \ \ 3.4.\ \ }Ein\textit{ in the Split-off}
\end{styleStandard}

\begin{styleFootnote}\bfseries
4. \ \ Weak Adjectives with Non-restrictive Interpretation
\end{styleFootnote}

\begin{styleStandard}\bfseries
5.\ \ Adjectival Endings Make Nominal Features Visible
\end{styleStandard}

\begin{styleFootnote}\bfseries
6. \ \ Conclusion
\end{styleFootnote}

\section[Chapter 5: Ein{}-words and Adjectival eine\ \ \ \ \ \ \ \ 201]{\textmd{Chapter 5: }\textmd{\textit{Ein}}\textmd{{}-words and Adjectival }\textmd{\textit{eine}}\textmd{\ \ \ \ \ \ \ \ 201}}
\begin{styleStandard}
\textbf{1. \ \ Introduction\newline
\ \ }\textit{1.1.\ \ Preliminaries and Basic Data\newline
\ \ 1.2. \ \ Initial Taxonomy of }ein
\end{styleStandard}

\begin{styleStandard}
\textit{1.3.\ \ Outlook}\textbf{\newline
2. \ \ Similarities\newline
}\textit{ \ \ 2.1. \ \ Split Topicalization\newline
\ \ 2.2. \ \ Split Topicalization with Fronted Adjective\newline
\ \ 2.3. \ \ Adjective Followed by Elided Noun\newline
\ \ 2.4. \ \ Split Topicalization with Fronted Adjective and Elided Noun\newline
}\textbf{3. \ \ Differences\newline
 \ \ }\textit{3.1. \ \ Encliticization\newline
 \ \ 3.2. \ \ Stressability\newline
 \ \ 3.3. \ \ Semantic Singularity\newline
}\textbf{4. \ \ Step 1 of the Proposal: Morphology and Semantics\newline
 \ \ }\textit{4.1.\ \ Composite Elements: Article }ein\textit{ as a Supporting and Flagging Element}
\end{styleStandard}

\begin{styleStandard}
4.1.1. Basic Proposal
\end{styleStandard}

\begin{styleStandard}
4.1.2. Derivation of Composite Forms
\end{styleStandard}

\begin{styleStandard}
4.1.3. Some Advantages and Initial Consequences\newline
\textit{ 4.2.\ \ Some Evidence for the Composite Analyses\newline
\ \ }4.2.1. The Negative Article \textit{kein} \newline
\ \ 4.2.2. Possessive Articles such as \textit{mein}\newline
\ \ 4.2.3. The Singularity Numeral \textit{EIN} \newline
\textit{4.3.\ \ Feature Specifications }
\end{styleStandard}

\begin{styleStandard}
\textit{\ \ \ \ }4.3.1. Cancelling the Presupposition of Adjectival \textit{eine}
\end{styleStandard}

\begin{styleStandard}
\ \ \ \ 4.3.2. Feature Specifications of the Different Types of \textit{ein}
\end{styleStandard}

\begin{styleStandard}
\textbf{5. \ \ Step 2 of the Proposal: Syntax\newline
\ \ }\textit{5.1. \ \ Article vs. Numeral\newline
 \ \ \ \ }5.1.1. Uniform Position(s) of all Numerals\newline
 \ \ \ \ 5.1.2. Different Scope of \textit{mehr als\newline
 \ \ 5.2. \ \ Article / Numeral vs. Adjective \newline
 \ \ \ \ }5.2.1. Different Morphology\newline
 \ \ \ \ 5.2.2. Co-occurrence with Possessive\textit{ ein}{}-words 
\end{styleStandard}

\begin{styleStandard}
\ \ 5.2.3. Co-occurrence of \textit{ein} and Determiners Revisited\newline
\ \ \ \ 5.2.4. More Evidence for Adjectival \textit{eine}
\end{styleStandard}

\begin{styleFooter}
5.2.5. Different Semantics: \textit{EIN} vs. \textit{eine} and \textit{andere }meaning ‘different’ vs. \ \ \ \ \ \ ‘other’
\end{styleFooter}

\begin{styleFooter}
\textbf{6.\ \ Diachronic and Cross-linguistic Evidence for }\textbf{\textit{ein}}\textbf{{}-words as Composites}
\end{styleFooter}

\begin{styleStandard}\bfseries
7.\ \ A Brief Critique of a Previous Proposal\newline
8.\ \ Conclusion \newline

\end{styleStandard}

\begin{styleJBHeadingii}
\textup{Chapter 6: }Ein\textup{ and Emotiveness\ \ \ \ \ \ \ \ \ \ 258}
\end{styleJBHeadingii}

\begin{styleStandard}\bfseries
1. \ \ Introduction
\end{styleStandard}

\begin{styleStandard}\itshape
\ \ 1.1.\ \ General Structural Similarities between the DP and TP
\end{styleStandard}

\begin{styleStandard}\itshape
\ \ 1.2.\ \ General Interpretatory Differences between the DP and TP
\end{styleStandard}

\begin{styleStandard}\bfseries
2.\ \ Data
\end{styleStandard}

\begin{styleStandard}
\textit{\ \ \ 2.1.\ \ Preliminary Remarks}
\end{styleStandard}

\begin{styleStandard}
\ \ \ \ 2.1.1. Basic Differences Between Role and Kind Nouns in the DP and TP
\end{styleStandard}

\begin{styleStandard}
\ \ \ \ 2.1.2. Emotiveness and Figurative Extension
\end{styleStandard}

\begin{styleStandard}
\ \ \ \ 2.1.3. Three Different Readings
\end{styleStandard}

\begin{styleStandard}\itshape
\ \ 2.2.\ \ Data
\end{styleStandard}

\begin{styleStandard}
\ \ \ \ 2.2.1. Plural
\end{styleStandard}

\begin{styleStandard}
\textit{\ \ \ \ }2.2.2. Singular
\end{styleStandard}

\begin{styleStandard}
\ \ \ \ 2.2.3. Interim Summary
\end{styleStandard}

\begin{styleStandard}
\ \ \ \ 2.2.4. Ambiguous Role Nouns in the Singular
\end{styleStandard}

\begin{styleStandard}\bfseries
3.\ \ Proposal
\end{styleStandard}

\begin{styleStandard}
\textit{\ \ \ 3.1. \ \ Combining Two Previous Proposals}
\end{styleStandard}

\begin{styleStandard}
\ \ \ \ 3.1.1. De Swart, Winter \& Zwarts (2007)
\end{styleStandard}

\begin{styleStandard}
\ \ \ \ 3.1.2. Lexical Features on Nouns
\end{styleStandard}

\begin{styleStandard}
\ \ \ \ 3.1.3. Rauh (2004)
\end{styleStandard}

\begin{styleStandard}
\textit{\ \ \ 3.2.\ \ Ordinary Reading}
\end{styleStandard}

\begin{styleStandard}
\ \ \ \ 3.2.1. Ordinary Reading in the DP
\end{styleStandard}

\begin{styleStandard}
\ \ \ \ 3.2.2. Ordinary Reading in the TP
\end{styleStandard}

\begin{styleStandard}
\textit{\ \ \ 3.3.\ \ Comparative Reading}
\end{styleStandard}

\begin{styleStandard}
\ \ \ \ 3.3.1. Preliminaries
\end{styleStandard}

\begin{styleStandard}
\ \ \ \ 3.3.2. Comparative Reading in the DP
\end{styleStandard}

\begin{styleStandard}
\ \ \ \ 3.3.3. Comparative Reading in the TP
\end{styleStandard}

\begin{styleStandard}
\textit{\ \ \ 3.4.\ \ Capacity Reading}
\end{styleStandard}

\begin{styleStandard}
\ \ \ \ 3.4.1. Preliminaries
\end{styleStandard}

\begin{styleStandard}
\ \ \ \ 3.4.2. Capacity Reading in the DP
\end{styleStandard}

\begin{styleStandard}
\ \ \ \ 3.4.3. Capacity Reading in the TP
\end{styleStandard}

\begin{styleStandard}\bfseries
4.\ \ Conclusion
\end{styleStandard}

\begin{styleStandard}
Chapter 7: \textit{Ein} and Number \ \ \ \ \ \ \ \ \ \ \ \ 292
\end{styleStandard}

\begin{styleStandard}\bfseries
1.\ \ Introduction
\end{styleStandard}

\begin{styleStandard}\itshape
\ \ 1.1.\ \ A Brief Review
\end{styleStandard}

\begin{styleStandard}\itshape
\ \ 1.2.\ \ Number in the DP and TP
\end{styleStandard}

\begin{styleStandard}\bfseries
2.\ \ Data
\end{styleStandard}

\begin{styleStandard}\itshape
\ \ 2.1.\ \ Cases of Agreement
\end{styleStandard}

\begin{styleStandard}\itshape
\ \ 2.2.\ \ Cases of Dis-agreement
\end{styleStandard}

\begin{styleStandard}
\textit{\ \ 2.3.\ \ More Cases Of Dis-agreement: Instances Involving }als\textit{{}-nominals}
\end{styleStandard}

\begin{styleStandard}\itshape
\ \ 2.4.\ \ Summary of the Data
\end{styleStandard}

\begin{styleStandard}\bfseries
3.\ \ Proposal
\end{styleStandard}

\begin{styleStandard}
\textit{\ \ \ 3.1.\ \ Basic Assumptions}
\end{styleStandard}

\begin{styleStandard}
\textit{\ \ 3.2.\ \ Agreement in Constructions without }als
\end{styleStandard}

\begin{styleStandard}
\ \ \ \ \ 3.2.1. Agreement and Dis-agreement
\end{styleStandard}

\begin{styleStandard}
\ \ \ \ 3.2.2. Agreement and NumP
\end{styleStandard}

\begin{styleStandard}
3.2.3. Plural Nouns
\end{styleStandard}

\begin{styleStandard}
\ \ \ \ \ 3.2.4. Non-plural Nouns
\end{styleStandard}

\begin{styleStandard}
\ \ \ \ \ 3.2.5. Singular Nouns
\end{styleStandard}

\begin{styleStandard}
\textit{\ \ \ 3.3.\ \ Morphological and Semantic Number}
\end{styleStandard}

\begin{styleStandard}
\ \ \ \ 3.3.1. Relating Morphological and Semantic Number
\end{styleStandard}

\begin{styleStandard}
\ \ \ \ 3.3.2. Dis-agreement Revisited
\end{styleStandard}

\begin{styleStandard}
\ \ \ \ 3.3.3. Pronominal DPs Involving \textit{Sie} in More Detail
\end{styleStandard}

\begin{styleStandard}
\textit{\ \ \ 3.4.\ \ Agreement in Constructions Involving }als-\textit{nominals}
\end{styleStandard}

\begin{styleStandard}
\ \ \ \ 3.4.1. Structure and Semantics
\end{styleStandard}

\begin{styleStandard}
\ \ \ \ 3.4.2. Agreement 
\end{styleStandard}

\begin{styleStandard}\bfseries
4.\ \ Conclusion
\end{styleStandard}

\begin{styleStandard}
Chapter 8: Discussion and Concluding Remarks\ \ \ \ \ \ 323\newline
\textbf{1. \ \ \ The Bigger Picture}\newline
\textit{ \ \ 1.1. \ \ Commonalities of Adjectival Inflections and }ein
\end{styleStandard}

\begin{styleStandard}
\ \ \ \ 1.1.1. Some Evidence for Hypothesis 1a
\end{styleStandard}

\begin{styleStandard}
\ \ \ \ 1.1.2. Some Evidence for Hypothesis 1b: Hypothesis 2a vs. 3a
\end{styleStandard}

\begin{styleStandard}
\textit{\ \ \ 1.2. \ \ Differences between Adjectival Inflections and }ein
\end{styleStandard}

\begin{styleStandard}
\ \ \ \ 1.2.1. Some Evidence for Hypothesis 2b
\end{styleStandard}

\begin{styleStandard}
\ \ \ \ 1.2.2. Some Evidence for Hypothesis 3b
\end{styleStandard}

\begin{styleStandard}\itshape
\ \ 1.3.\ \ Summary of the Main Claims
\end{styleStandard}

\begin{styleStandard}\bfseries
2. \ \ Some Extensions and Further Consequences 
\end{styleStandard}

\begin{styleStandard}
\textit{\ \ \ 2.1.\ \ Extensions and Consequences of Adjectival Inflections: German Dialects}
\end{styleStandard}

\begin{styleStandard}
\textit{\ \ 2.2.\ \ Extensions and Consequences Involving }ein
\end{styleStandard}

\begin{styleStandard}
2.2.1. Reviewing Cases of Supporting and Flagging
\end{styleStandard}

\begin{styleStandard}
\ \ \ \ 2.2.2. Nominals with Two Overt Operators
\end{styleStandard}

\begin{styleStandard}
\ \ \ \ 2.2.3. Flagging of Other Operators
\end{styleStandard}

\begin{styleStandard}
\ \ \ \ 2.2.4. Contexts of Vocabulary Insertion for \textit{ein}: Covert Operators
\end{styleStandard}

\begin{styleStandard}
\ \ \ \ 2.2.5. Contexts of Vocabulary Insertion for \textit{ein}: Covert and Overt Operators
\end{styleStandard}

\begin{styleStandard}
2.2.6. Contexts of Vocabulary Insertion for \textit{ein}: Feature Deletions
\end{styleStandard}

\begin{styleStandard}\bfseries
3.\ \ More General Considerations
\end{styleStandard}

\begin{styleStandard}\itshape
\ \ 3.1.\ \ Nominal Structure and Concord
\end{styleStandard}

\begin{styleStandard}
\textit{\ \ \ 3.2.\ \ Expletive Elements more Generally}
\end{styleStandard}

\begin{styleStandard}
\newline
Appendix: Examples of Plural \textit{ein\ \ \ \ \ \ \ \ \ \ }361
\end{styleStandard}

\begin{styleStandard}
\textit{1.1.\ \ Googled: }was für (ei)ne\textit{ without adjective}
\end{styleStandard}

\begin{styleStandard}
\textit{1.2.\ \ Googled: }was für (ei)ne\textit{ with adjective}
\end{styleStandard}

\begin{styleStandard}
\textit{1.3. \ \ Googled: }so (ei)ne\textit{ without adjective}
\end{styleStandard}

\begin{styleStandard}
\textit{1.4. \ \ Googled: }so (ei)ne\textit{ with adjective}
\end{styleStandard}

\begin{styleStandard}
\textit{2.1.\ \ Twitter: }was für (ei)ne\textit{ without adjective}
\end{styleStandard}

\begin{styleStandard}
\textit{2.2.\ \ Twitter: }was für (ei)ne\textit{ with adjective}
\end{styleStandard}

\begin{styleStandard}
\textit{2.3.\ \ Twitter: }so (ei)ne\textit{ withtout adjective (but with numeral)}
\end{styleStandard}

\begin{styleStandard}
\textit{2.4.\ \ Twitter: }so (ei)ne\textit{ with adjective}
\end{styleStandard}

\begin{styleStandard}
\textit{3.1.\ \ Facebook: }so (ei)ne\textit{ with adjective}
\end{styleStandard}

\begin{styleStandard}\itshape
4. \ \ Datenbank für Gesprochenes Deutsch
\end{styleStandard}

\begin{styleStandard}
References\ \ \ \ \ \ \ \ \ \ \ \ \ \ \ \ \ \ 374
\end{styleStandard}

\begin{styleStandard}
Language Index\ \ \ \ \ \ \ \ \ \ \ \ \ \ \ \ N/A
\end{styleStandard}

\begin{styleStandard}
Subject Index\ \ \ \ \ \ \ \ \ \ \ \ \ \ \ \ N/A
\end{styleStandard}

\clearpage\begin{styleStandard}
Preface
\end{styleStandard}

\begin{styleStandard}
Anyone teaching German as a foreign language knows that mastering the adjectival endings and \textit{ein}{}-words is particularly challenging for students. I became interested in these grammatical topics over 25 years ago when I started working as a Teaching Assistant in the United States. I have been fascinated by these phenomena ever since. A couple of years ago, this interest was strongly piqued again by the phrase \textit{Eine Störche! }uttered by my dad during a hike. With my family interested in language, a discussion followed where we agreed that this phrase could be loosely rendered as ‘Wow! So many storks!’ However, my family is, to this day, surprised that a singular article like \textit{ein} ‘a’ can combine with a plural noun. Wow, indeed! This book attempts to shed some light on this and other issues. Beside the cases frequently discussed in the literature, here I also engage in the discussion of many non-canonical instances. In particular, I focus on various constructions involving \textit{ein}{}-words and adjectival inflections. 
\end{styleStandard}

\begin{styleStandard}
Thinking back, the chapter on \textit{ein}{}-words was the first to be written for my dissertation. However, it has taken the longest to present it in a more polished form. Although the basic idea is still the same, many details, empirical and analytical, have been modified or added. In the meantime, adjectival inflections have become a second major point of interest for me. Again, while the basic system was already laid out in Roehrs (2009a: Chapter 4), more “exotic” cases are discussed in the following pages. I believe it is these new details about \textit{ein}{}-words and adjectival inflections that reveal the true nature of these elements allowing us to formulate some new and interesting hypotheses. 
\end{styleStandard}

\begin{styleStandard}
It is my hope that due to its richness of data and analyses, many of which are new, this book will be of interest to anyone working on semantically vacuous elements, in general, and on the syntax and semantics of the DP, in particular. With the main focus on German, the book will not be the final word on these topics as the investigation of other languages will surely reveal more interesting facts and lead to other theoretical insights. However, I hope that this book will make a contribution toward the description and explanation of these parts of the grammar, topics I believe have not received due attention.
\end{styleStandard}

\begin{styleStandard}
\ \ The material in this book was presented at many linguistic colloquia and conferences, too many to detail here. I thank the respective audiences for questions and comments, especially, Marcel den Dikken, David Fertig, Volker Gast, Hubert Haider, Tracy Hall, Tom Leu, Mark Louden, Joe Salmons, Chris Sapp, Erik Schoorlemmer, and Laura Smith. A very special thanks goes to Björn Köhnlein, who contributed a number of very interesting points. I would also like to express gratitude to Jochen Trommer and Bernd Wiese for helping me find some of their work. Furthermore, I am indebted to many reviewers whose comments over the years have helped me shape the ideas presented here. In particular, I am very grateful to the two anonymous reviewers of this book, who provided many constructive suggestions, data points, and references, and to my editor Mike Putnam for many helpful suggestions. Finally, note that parts of Chapter 5 are based on Roehrs (2006a: Chapter 5, Part II). Chapter 7 was supported by a Faculty Research Grant (G34217) from the University of North Texas, which I hereby gratefully acknowledge. The first version of this book was thoroughly revised during my Faculty Development Leave in Fall 2020, and many other changes were made after that.
\end{styleStandard}

\begin{styleStandard}
I dedicate this book to my late dad, who encouraged me to look for commonalities, to my mom, who emphasized I should write clearly, to my brother, who reminded me to stop and smell the roses, and to my partner in crime, who makes life just beautiful. \textit{Danke, Siggi, Rosen, Keule and Xixi!}
\end{styleStandard}

\begin{styleStandard}
Denton, November 2024
\end{styleStandard}

\begin{styleStandard}
Dorian Roehrs
\end{styleStandard}

\begin{styleStandard}
Abbreviations
\end{styleStandard}

\begin{styleStandard}
ACC \ \ – accusative case
\end{styleStandard}

\begin{styleStandard}
CMN\ \ – common gender 
\end{styleStandard}

\begin{styleStandard}
DAT \ \ – dative case
\end{styleStandard}

\begin{styleStandard}
FEM \ \ – feminine gender
\end{styleStandard}

\begin{styleStandard}
GEN \ \ – genitive case
\end{styleStandard}

\begin{styleStandard}
MASC\ \ – masculine gender
\end{styleStandard}

\begin{styleStandard}
NEUT \ \ – neuter gender
\end{styleStandard}

\begin{styleStandard}
NOM \ \ – nominative case
\end{styleStandard}

\begin{styleStandard}
PL \ \ – plural number
\end{styleStandard}

\begin{styleStandard}
PRT \ \ – particle 
\end{styleStandard}

\begin{styleStandard}
REFL\ \ – reflexive particle
\end{styleStandard}

\begin{styleStandard}
SGL \ \ – singular number
\end{styleStandard}

\begin{styleStandard}
ST \ \ – strong adjectival ending
\end{styleStandard}

\begin{styleStandard}
WK \ \ – weak adjectival ending
\end{styleStandard}

\clearpage\setcounter{page}{1}\begin{styleStandard}
Chapter 1: Introduction
\end{styleStandard}

\begin{styleStandard}\bfseries
1.\ \ Vacuous Elements – an Imperfection of Language?
\end{styleStandard}

\begin{styleStandard}
In many languages, ordinary sentences contain nominative subjects. Nominative case is often taken to be structural; that is, it is assigned under certain syntactic conditions. This type of case has received a lot of attention in the linguistic literature. In the generative tradition, structural case – often spelled Case – is frequently represented by uninterpretable features. Chomsky (e.g., 2000: 119) pointed out that uninterpretable features may be an imperfection in the design of languages. Specifically, structural case features may receive no interpretation in either PF or LF. In terms of optimally efficient language design, it is surprising that these elements exist at all. 
\end{styleStandard}

\begin{styleStandard}
It is usually proposed that uninterpretable features have to be removed during the derivation to adhere to the Principle of Full Interpretation (Chomsky 1995). This removal has to do with feature checking (or valuation). Specifically, feature checking (or valuation) motivates movement including that of subjects. As the result of this movement, features including structural case features are removed from the derivation (e.g., Chomsky 1995, 2000). In other words, no uninterpretable features are left at the end of the derivation, and the derivation succeeds.
\end{styleStandard}

\begin{styleStandard}
Besides these abstract features, there are also word-level elements that raise similar issues about the design of language – expletives. Although these elements do get an interpretation in PF (i.e., they have a phonetic realization), they seem to have no semantics either. In the next section, I briefly illustrate two such cases that have been intensively discussed in the literature: expletive \textit{there} and the proprial article. In the second section, I discuss two other elements that have received less attention in this regard: German adjectival inflections and the indefinite article \textit{ein} ‘a’. In this book, I propose that the latter two elements also have no semantics at all; that is, they have neither semantics of their own nor do they make semantic features visible. I hypothesize that these elements are not an imperfection of language but provide overt clues about the presence of abstract linguistic elements. 
\end{styleStandard}

\begin{styleStandard}\itshape
1.1.\ \ The Clause
\end{styleStandard}

\begin{styleStandard}
Sentence pairs such as (1a-b) have received much attention in the literature (Milsark 1974 and much subsequent work). Clauses like (1b) are usually referred to as \textit{there}{}-existentials and are used in presentational contexts:
\end{styleStandard}

\begin{styleStandard}
(1)\ \ a.\ \ A man is in the garden.
\end{styleStandard}

\begin{styleStandard}
\ \ b.\ \ There is a man in the garden.
\end{styleStandard}

\begin{styleStandard}
Both sentences in (1) have very similar meanings: they are about a man being in the garden. As Hazout (2004: 396) points out, \textit{there} itself cannot have any semantics; for instance, if \textit{there} were a deictic element, its distal semantics should clash with the proximal meaning of \textit{here} in (2a). Evidently, this is not the case. Second, the unaccusative predicate \textit{arrive} assigns one theta role in (2b), namely to \textit{three men}. With the classical Theta-Criterion (Chomsky 1981: 36) satisfied, \textit{there} cannot be an argument or a predicate. If \textit{there} makes no semantic contribution, then we might expect that it can be left out without a (significant) loss in meaning.\footnote{\ Referencing work by Edwin Williams, Chomsky (1991 [reprinted in Chomsky 1995: 157]) points out that scopal elements in constructions with \textit{there} vs. without the expletive may have different readings (also Lasnik \& Uriagereka with Boeckx 2005: 157).} This is borne out as can easily be verified in (2c):
\end{styleStandard}

\begin{styleStandard}
(2)\ \ a.\ \ There are too many people here.
\end{styleStandard}

\begin{styleStandard}
\ \ b.\ \ There arrived three men.
\end{styleStandard}

\begin{styleStandard}
\ \ c.\ \ Three men arrived.
\end{styleStandard}

\begin{styleStandard}
The conclusion is that \textit{there} is indeed an expletive (i.e., pleonastic) element. 
\end{styleStandard}

\begin{styleStandard}
There are many proposals that aim to explain \textit{there}{}-existentials (for a brief survey of some of the relevant issues, see, e.g., Lasnik \& Uriagereka with Boeckx 2005: 153-169). I briefly illustrate the basic account in Chomsky (1991 [reprinted in Chomsky 1995]), one of the first proposals of this construction in the Minimalist tradition. Note though that the purpose of this book is not to discuss this construction in detail but rather to lay out some basic assumptions and to relate existential constructions to other, similar instances.
\end{styleStandard}

\begin{styleStandard}
Chomsky (1995: 154-57) assumes the Principle of Full Interpretation, according to which all elements in a linguistic expression must have an interpretation in LF. Above, I showed that \textit{there} is semantically vacuous; that is, it has no semantics. As a consequence, \textit{there} can, by itself, not be interpreted in LF. To explain the grammaticality of \textit{there}{}-constructions, Chomsky proposes that \textit{there} is a LF-affix and that the overt noun phrase, the associate, undergoes movement to adjoin to \textit{there} (also Chomsky \& Lasnik (1993 [reprinted in Chomsky 1995: 65-66]). As the expletive is licensed by adjunction of another element in LF, Full Interpretation is not violated. I assume that the proposal involving the licensing of the expletive by moving the associative is basically correct. 
\end{styleStandard}

\begin{styleStandard}
Clauses and noun phrases are usually held to be parallel in meaning and structure (e.g., Abney 1987, Chomsky 1970, Iordăchioaia 2020). This can most easily be seen in (3), which juxtaposes a verb and its arguments in (3a) with its derived nominal counterpart in (3b). It is important to point out that both the verb in the sentence and the noun in the noun phrase have an agentive subject (\textit{the Romans}) and an affected object (\textit{the city}):
\end{styleStandard}

\begin{styleStandard}
(3) \ \ a.\ \ The Romans destroyed the city.
\end{styleStandard}

\begin{styleStandard}
\ \ b.\ \ the Romans’ destruction of the city 
\end{styleStandard}

\begin{styleStandard}
Based on these similarities in meaning, a certain structural parallelism has come to be established. Considering (4), NP is taken to be the nominal counterpart of VP. Furthermore, NumP is similar to AgrP, DP to TP, and PP to CP:\footnote{\ This alignment based on Grimshaw (1991) is not uncontroversial; for instance, other authors have proposed that DP is parallel to CP (e.g., Szabolcsi 1994). Note that constructions such as (3) are often discussed in this context. This book will not consider cases involving theta nouns as in (3b) in much detail. Rather, investigating many non-canonical constructions (i.e., structures more complex than simple DPs), we will come across cases where more structure is projected above the DP (and below the PP) in (4b). Below, this structural level is labeled Left Periphery Phrase (see Section 4.1.2). Furthermore, there is debate in German as to whether subject-initial root clauses involve CPs (e.g., Schwartz \& Vikner 1996) or TPs (e.g., Zwart 1997 for Dutch, which extends to German). Given these complexities and controversies, I do not engage in the debate of the alignment in (4) here. Rather, I simply follow the alignment in (4) and assume that the DP is parallel to the TP (see also Chapters 6 and 7 below). For clauses, I assume that subject-initial structures are TPs and non-subject initial structures are CPs (Zwart 1997), but nothing important hinges on this choice.}
\end{styleStandard}

\begin{styleStandard}
(4) \ \ a.\ \ CP\ \ TP\ \ AgrP\ \  \ \ \ \ VP
\end{styleStandard}

\begin{styleStandard}
\ \ b.\ \ PP\ \ DP\ \ NumP\ \  \ \ \ \ NP
\end{styleStandard}

\begin{styleStandard}
These and other observations have led to the general hypothesis that noun phrases are similar to clauses. Part of this work has been the establishment of the DP-Hypothesis (Abney 1987). This line of investigation has led to many empirical discoveries and theoretical innovations (for a survey, see Alexiadou \textit{et al}. 2007). The current work espouses the DP-Hypothesis. While noun phrases in argument position are assumed to be at least DPs (e.g., Longobardi 1994, Stowell 1989), we see in the course of the discussion that noun phrases in non-argument position (e.g., nominal predicates) may be structures of smaller sizes. 
\end{styleStandard}

\begin{styleStandard}
Before proceeding, I briefly provide here details about some of the terminology used in this book. The terms noun phrase and nominal are used as general designations for structures involving a noun and possibly some modifiers or other dependents. Terms such as DP, NumP, or NP are used when the structure of the nominal (/noun phrase) is of particular interest. With this in place, I return to the main line of exposition. Given the existence of expletives in clauses and the assumed parallelism between clauses and noun phrases, it may not come as a surprise that expletive elements have also been proposed to occur in the nominal domain.
\end{styleStandard}

\begin{styleStandard}\itshape
1.2.\ \ The Noun Phrase
\end{styleStandard}

\begin{styleStandard}
Longobardi (1994) develops a theory of noun movement in (overt) syntax and LF to account for a number of similarities and differences between the Romance and Germanic languages. Longobardi (1994: 651) suggests that the following alternation can be related to the\textit{ there}{}-existentials discussed in the previous section. Similar to the (associative) noun phrases in (1), proper names may appear in different positions. Comparing (5a) to (5b), we observe that the proper name in (5b) follows a definite article. This determiner is often referred to as proprial article. Longobardi relates the distribution in (5a) to (5b) by proposing that the proper name in (5a) has undergone movement to D within the larger subject DP. This is illustrated in (5c) (more on this below):\footnote{\ Longobardi (1994: 640) actually assumes that N in (5c) raises to D by substitution. To keep the discussion parallel to the clause, I assume adjunction here. }
\end{styleStandard}

\begin{styleStandard}
(5)\ \ a.\ \ \textit{Gianni mi \ ha \ telefonato}.\ \ \ \ (Italian)
\end{styleStandard}

\begin{styleStandard}
\ \ \ \ Gianni me has called
\end{styleStandard}

\begin{styleStandard}
‘Gianni called me.’
\end{styleStandard}

\begin{styleStandard}
\ \ b.\ \ \textit{Il} \textit{\ \ }\textit{\textsubscript{\ }}\textit{Gianni mi }\textit{\textsubscript{\ }}\textit{ha \ telefonato}.
\end{styleStandard}

\begin{styleStandard}
\ \ \ \ the Gianni me has called
\end{styleStandard}

\begin{styleStandard}
‘Gianni called me.’
\end{styleStandard}

\begin{styleStandard}
\ \ c.\ \ [\textsubscript{DP} \textit{Gianni}\textsubscript{i}+D [\textsubscript{NP} t\textsubscript{i} ]]
\end{styleStandard}

\begin{styleStandard}
Importantly, both DPs in (5a) and (5b) are interpreted as referential; that is, the definite article in (5b) does not make a semantic contribution (see also Lekakou \& Szendrői 2012, Vergnaud \& Zubizaretta 1992). In other words, the article is added in (5b) with no change in the meaning of the noun phrase as a whole. In fact, note that if the definite article in (5b) were to make a contribution, it would raise issues with regard to the redundancy of certain semantic components: singular definite articles typically presuppose the existence of a unique entity; however, proper names, in their referential use, already denote unique individuals by themselves. To avoid this issue, Longobardi (1994: 648-50, 655) proposes that the proprial article functions as an expletive element similar to \textit{there} above.\footnote{\ For different views of the proprial article, see Matushansky (2008) and Muñoz (2019).} 
\end{styleStandard}

\begin{styleStandard}
Longobardi proposes that (5a) involves movement of the noun to D. Among others, this N-to-D movement lexically licenses the null D position of a syntactic argument, and it obviates a default existential interpretation. With the discussion of \textit{there}{}-existentials in mind, I assume with Longobardi (1994: 651 fn. 48) that this movement is overt in (5a) but covert in (5b). In the latter case, the proper name – in Chomsky’s terms, the associate – licenses the expletive proprial article. 
\end{styleStandard}

\begin{styleStandard}
There is other distributional and morphological evidence that the proper name undergoes movement and that the corresponding article is different from other types of determiners. Simplifying somewhat, Longobardi (1994: 623) points out that the proper name can precede or follow a possessive adjective. Compare (6a) and (6b). If the proper name follows the possessive adjective, the definite article must appear. In this regard, observe the difference in grammaticality between (6b) and (6c):
\end{styleStandard}

\begin{styleStandard}
(6)\ \ a.\ \ \textit{Gianni mio ha \ }\textit{\textsubscript{\ }}\textit{finalmente telefonato}.\ \ (Italian)
\end{styleStandard}

\begin{styleStandard}
\ \ \ \ Gianni my \ has finally \ \ \ \ \ \ \ called
\end{styleStandard}

\begin{styleStandard}
\ \ \ \ ‘My Gianni finally called.’
\end{styleStandard}

\begin{styleStandard}
\ \ b.\ \ \textit{Il \ \ }\textit{\textsubscript{\ }}\textit{mio Gianni ha \ }\textit{\textsubscript{\ }}\textit{finalmente telefonato}.
\end{styleStandard}

\begin{styleStandard}
\ \ \ \ the my \ Gianni has finally \ \ \ \ \ \ \ called
\end{styleStandard}

\begin{styleStandard}
\ \ \ \ ‘My Gianni finally called.’
\end{styleStandard}

\begin{styleStandard}
\ \ c. \ *\ \ \textit{Mio Gianni ha \ finalmente telefonato}.
\end{styleStandard}

\begin{styleStandard}
\ \ \ \ my \ Gianni has finally \ \ \ \ \ \ \textsubscript{\ }called
\end{styleStandard}

\begin{styleStandard}
Similar to the paradigm in (5), Longobardi proposes that the proper name moves from a lower position within the larger subject DP (6b) to the higher position D (6a). The example in (6c) is ungrammatical because the empty D is not lexically licensed in syntax, something the Romance languages require. 
\end{styleStandard}

\begin{styleStandard}
Second, Longobardi (1994: 656) provides evidence that pleonastic articles can be morphologically different from substantive articles. Compare (7a) involving a proper name to (7b) containing a common noun, where the article is \textit{en} in the first case but \textit{el} in the second one (for more details, see Longobardi’s work; see also Panagiotidis 2000 for Northern Greek):
\end{styleStandard}

\begin{styleStandard}
(7)\ \ a.\ \ \textit{en \ Pere}\ \ \ \ \ \ (Catalan)
\end{styleStandard}

\begin{styleStandard}
\ \ \ \ the Peter
\end{styleStandard}

\begin{styleStandard}
\ \ \ \ ‘Peter’
\end{styleStandard}

\begin{styleStandard}
\ \ b.\ \ \textit{el \ \ gos}
\end{styleStandard}

\begin{styleStandard}
\ \ \ \ the dog
\end{styleStandard}

\begin{styleStandard}
\ \ \ \ ‘the dog’\ \ 
\end{styleStandard}

\begin{styleStandard}
Given the difference in articles in (7), Longobardi presents a strong argument that articles come in several types with different semantics, expletive or contentful.
\end{styleStandard}

\begin{styleStandard}
To sum up, I take it as established that there are expletive elements in the clause as well as in the noun phrase. The assumption that certain lexical elements are semantically vacuous allows us to avoid issues such as contradictions (e.g., as regards deixis in the clause) and redundancies (e.g., as regards uniqueness in the noun phrase). Furthermore, the assumption that such elements are expletives explains why these elements can be left out without a significant loss in meaning. Note that in each case discussed so far, the expletive element is in a position higher than the substantive element. Schematically, this can be illustrated as follows:
\end{styleStandard}

\begin{styleStandard}
[Warning: Draw object ignored][Warning: Draw object ignored](8)\ \ \ \  \ \ \ \ \ \ \ \ \ \ $\alpha $
\end{styleStandard}

\begin{styleStandard}
\ \ EXPL\ \ \ \ \ \  \ \ \ \ \ $\beta $
\end{styleStandard}

\begin{styleStandard}
[Warning: Draw object ignored][Warning: Draw object ignored][Warning: Draw object ignored]
\end{styleStandard}

\begin{styleStandard}
[Warning: Draw object ignored]\ SUBST\ \ 
\end{styleStandard}

\begin{styleStandard}
\ \ \textit{there}\ \ \ \ associate
\end{styleStandard}

\begin{styleStandard}
\ \ prop. art.\ \ proper name
\end{styleStandard}

\begin{styleStandard}
In order not to violate Full Interpretation, the pleonastic element must be licensed. In both cases, this licensing process involves movement of a lower substantive element to the higher expletive one.
\end{styleStandard}

\begin{styleStandard}
\ \ Two questions arise from the discussion above: (i) are there other semantically vacuous elements and (ii) why do these elements exist at all? The first question is discussed in detail for the German noun phrase in the following pages. I argue that adjectival inflections and the indefinite article \textit{ein} ‘a’ are also semantically vacuous elements. The second question is much harder to answer and is only briefly addressed in this book. I suggest that expletives indicate the presence of abstract linguistic elements. In that sense, adjectival inflections and \textit{ein} are not an imperfection of language.
\end{styleStandard}

\begin{styleStandard}
\ \ Note that this book is not intended to contribute to the theory of expletive elements \textit{per se}. Rather, it tries to identify more elements that share some of the properties of the expletives that have been established in the literature. Having said that, I engage in a more detailed discussion of \textit{there} vs. adjectival inflections and \textit{ein} in the final section of this book (Chapter 8, Section 3.2). There I point out that both types of elements share a number of traits but that they also seem to exhibit some differences. Although I make some very brief remarks of what might explain some of those differences, I will leave a more detailed exploration for another time. 
\end{styleStandard}

\begin{styleStandard}\bfseries
2.\ \ German as the Language under Investigation
\end{styleStandard}

\begin{styleStandard}
In this section, I explain why the focus of this book is on one language – German. To this end, I provide some brief cross-linguistic discussion of adjectival inflections and singular indefinite articles in German, Yiddish, and Norwegian. It is shown that these languages exhibit a number of striking differences. This includes evidence for the presence of the indefinite articles in plural contexts. Unfortunately, many of these and other non-canonical data points have not received much attention in the theoretical literature. In this book, I add many data points to the discussion of German, the language I am most familiar with, and analyze these constructions in detail. However, it is beyond the scope of this book to do this kind of empirical work for the other languages, to discuss the relevant constructions in detail, and to provide a thorough cross-linguistic analysis of all these languages. 
\end{styleStandard}

\begin{styleStandard}\itshape
2.1.\ \ Some Cross-linguistic Differences
\end{styleStandard}

\begin{styleStandard}
Before I discuss some cross-linguistic data, I start with some brief general remarks.
\end{styleStandard}

\begin{styleStandard}
2.1.1. Canonical and Non-canonical Constructions
\end{styleStandard}

\begin{styleStandard}
Focusing on German, this book may seem restricted in scope. However, it strives to be more comprehensive than previous accounts by discussing adjectival inflections and the indefinite article \textit{ein} ‘a’ in very diverse structural contexts in that language. On the one hand, reference grammars (e.g., the German Duden) provide a fairly comprehensive survey of the relevant empirical domains. However, their goal is not to lay out a detailed theoretical analysis. On the other hand, theoretically oriented works often make insightful and elegant proposals of the canonical cases. These are simple DPs that involve the schematic structure “determiner + adjective(s) + noun”, where the three elements agree in case, number, and gender (9a).\footnote{\ Simple DPs involving prenominal Saxon Genitives (e.g., \textit{Maries groß-er Bär} ‘Mary’s big-\textsc{st} bear’) are counted as canonical constructions here despite the fact that the Saxon Genitive itself does not agree in features with the following adjective and/or noun.} This is in contrast to more complex cases like (9b-c), constructions that are frequently relegated to footnotes or even completely left out (for specific references, see Chapters 2 and 5). Specifically, the string in (9b) is a complex nominal involving two overt head nouns, and (9c) is a complex nominal where the pronoun does not agree in number with the following adjective and noun:
\end{styleStandard}

\begin{styleStandard}
(9)\ \ a.\ \ \textit{der} \textit{groß-e \ Bär\ \ \ \ \ \ }(German)
\end{styleStandard}

\begin{styleStandard}
the big-\textsc{wk} bear.\textsc{masc}
\end{styleStandard}

\begin{styleStandard}
‘the big bear’
\end{styleStandard}

\begin{styleStandard}
\ \ b.\textit{\ \ das Sternbild \ \ \ \ }[\textit{Groß-er Bär}]
\end{styleStandard}

\begin{styleStandard}
the constellation big-\textsc{st} \ \ \ bear.\textsc{masc}
\end{styleStandard}

\begin{styleStandard}
‘the constellation Great Bear’
\end{styleStandard}

\begin{styleStandard}
c. \ \ \textit{ihr \ \ \ \ \ \ \ }[\textit{jung-es \ \ \ \ \ \ \ \ \ \ Gemüse}] e\textsubscript{N}
\end{styleStandard}

\begin{styleStandard}
\ \ \ \ you(\textsc{pl}) young-\textsc{st.sgl} vegetable.\textsc{neut}
\end{styleStandard}

\begin{styleStandard}
\ \ \ \ ‘you young folks’
\end{styleStandard}

\begin{styleStandard}
Given the characterization of simple DPs above, the latter two examples involve non-canonical structures. This book puts these non-canonical constructions on center stage. In fact, I argue that it is these types of constructions that reveal the true nature of adjectival inflections and \textit{ein}.\footnote{\ This idea has been echoed by other authors; for instance, Sigurðsson \& Wood (2020: 22) state that the exceptional cases may throw light on the more general instances.}
\end{styleStandard}

\begin{styleStandard}
Before moving on, I clarify some more terminology. To distinguish the different nominals in (9b-c), I use the terms matrix nominal and embedded nominal. Embedded nominals are marked in (9b-c) by brackets. They may consist of adjuncts (9b), specifiers (9c), and various complements. The structures without brackets are labeled matrix nominals. Note that each head noun projects its own nominal; that is, both \textit{Sternbild} ‘constellation’ and \textit{Bär} ‘bear’ in (9b) involve extended projections (Section 4.1.1).\footnote{\ Notice that if a chapter designation is not provided, the section or footnote is part of the current chapter (i.e., the reference to Section 4.1.1 in the main text is shorthand for Chapter 1, Section 4.1.1).} Head nouns may also include null elements (9c).
\end{styleStandard}

\begin{styleStandard}
\ \ As illustrated in the next subsection, languages, even closely related ones, show empirical differences in the domains of adjectival inflections and indefinite articles. This becomes clear when we compare German, Yiddish, and Norwegian in canonical as well as non-canonical structures. 
\end{styleStandard}

\begin{styleStandard}
2.1.2. Cross-linguistic Differences with Adjectival Inflections
\end{styleStandard}

\begin{styleStandard}
I start with a canonical construction where German, Yiddish, and Norwegian show the same inflectional patterns. In simple DPs, adjectives have weak inflections (WK) when they follow definite articles (the data in this subsection are taken from Roehrs 2015; note also that Norwegian involves a second, suffixal determiner; for discussion, see Julien 2005a among many others):
\end{styleStandard}

\begin{styleStandard}
(10)\ \ a.\ \ \textit{das groß-e Haus}\ \ \ \ (German)
\end{styleStandard}

\begin{styleStandard}
\ \ \ \ the big-\textsc{wk} house.\textsc{neut}
\end{styleStandard}

\begin{styleStandard}
\ \ \ \ ‘the big house’
\end{styleStandard}

\begin{styleStandard}
\ \ b.\ \ \textit{dos groys-e hoyz}\ \ \ \ (Yiddish)
\end{styleStandard}

\begin{styleStandard}
\ \ \ \ the big-\textsc{wk} \ house.\textsc{neut}
\end{styleStandard}

\begin{styleStandard}
\ \ \ \ ‘the big house’
\end{styleStandard}

\begin{styleStandard}
\ \ c.\ \ \textit{det stor-e \ \ \ hus-et}\ \ \ \ (Norwegian)
\end{styleStandard}

\begin{styleStandard}
\ \ \ \ the big-\textsc{wk} house.\textsc{neut}{}-\textsc{def}
\end{styleStandard}

\begin{styleStandard}
\ \ \ \ ‘the big house’
\end{styleStandard}

\begin{styleStandard}
Differences emerge when adjectives are preceded by Saxon Genitives. Here, German patterns with Yiddish in exhibiting a strong ending (ST) on the adjective, but Norwegian has a weak inflection (\textit{Ø} indicates an assumed null ending):
\end{styleStandard}

\begin{styleStandard}
(11)\ \ a.\ \ \textit{Peters \ groß-es Haus}\ \ \ \ (German)
\end{styleStandard}

\begin{styleStandard}
\ \ \ \ Peter’s big-\textsc{st} \ \ house.\textsc{neut}
\end{styleStandard}

\begin{styleStandard}
\ \ \ \ ‘Peter’s big house’
\end{styleStandard}

\begin{styleStandard}
\ \ b.\ \ \textit{Berls \ groys-Ø hoyz}\ \ \ \ (Yiddish)
\end{styleStandard}

\begin{styleStandard}
\ \ \ \ Berl’s big-\textsc{st} \ \ \ house.\textsc{neut}
\end{styleStandard}

\begin{styleStandard}
\ \ \ \ ‘Berl’s big house’
\end{styleStandard}

\begin{styleStandard}
\ \ c.\ \ \textit{Pers \ stor-e \ \ hus}\ \ \ \ (Norwegian)
\end{styleStandard}

\begin{styleStandard}
\ \ \ \ Per’s big-\textsc{wk} house.\textsc{neut}
\end{styleStandard}

\begin{styleStandard}
\ \ \ \ ‘Per’s big house’
\end{styleStandard}

\begin{styleStandard}
Another distinction appears when adjectives follow distal demonstratives. In these cases, German patterns with Norwegian in showing a weak inflection on the adjective, but Yiddish has a strong ending (the stressed demonstrative in Norwegian, which is similar to the definite article in form, is capitalized):
\end{styleStandard}

\begin{styleStandard}
(12)\ \ a.\ \ \textit{jenes groß-e \ Haus}\ \ \ \ (German)
\end{styleStandard}

\begin{styleStandard}
\ \ \ \ that \ \ big-\textsc{wk} house.\textsc{neut}
\end{styleStandard}

\begin{styleStandard}
\ \ \ \ ‘that big house’
\end{styleStandard}

\begin{styleStandard}
\ \ b.\ \ \textit{yents groys-Ø hoyz}\ \ \ \ (Yiddish)
\end{styleStandard}

\begin{styleStandard}
\ \ \ \ that \ \ big-\textsc{st} \ \ \ house.\textsc{neut}
\end{styleStandard}

\begin{styleStandard}
\ \ \ \ ‘that big house’
\end{styleStandard}

\begin{styleStandard}
\ \ c.\ \ \textit{DET stor-e \ \ hus-et}\ \ \ \ (Norwegian)
\end{styleStandard}

\begin{styleStandard}
\ \ \ \ that \ big-\textsc{wk} house.\textsc{neut}{}-\textsc{def}
\end{styleStandard}

\begin{styleStandard}
\ \ \ \ ‘that big house’
\end{styleStandard}

\begin{styleStandard}
\ \ As to non-canonical constructions, suffice it to consider one structure. In (13), complex names are juxtaposed with their categorizing nominals. Again, the three languages do not pattern the same. Here, German and Yiddish contrast with Norwegian again:
\end{styleStandard}

\begin{styleStandard}
(13)\ \ a.\textit{\ \ das Sternbild \ \ \ \ }\ \textit{Groß-er Bär}\ \ \ \ \ \ (German)
\end{styleStandard}

\begin{styleStandard}
the constellation big-\textsc{st} \ \ \ bear.\textsc{masc}
\end{styleStandard}

\begin{styleStandard}
‘the constellation Great Bear’
\end{styleStandard}

\begin{styleStandard}
b.\ \ \textit{der zhurnal \ \ \ Sovetish-Ø Heymland}\ \ \ \ (Yiddish)
\end{styleStandard}

\begin{styleStandard}
\ \ \ \ the magazine Soviet-\textsc{st} \ \ homeland.\textsc{neut}
\end{styleStandard}

\begin{styleStandard}
\ \ \ \ ‘the magazine Soviet Homeland’
\end{styleStandard}

\begin{styleStandard}
c.\ \ \textit{indianer-en Stor-e \ \ Bjørn} \ \ \ \ \ \ (Norwegian)
\end{styleStandard}

\begin{styleStandard}
\ \ \ Indian-\textsc{def} \ big-\textsc{wk} bear.\textsc{cmn}
\end{styleStandard}

\begin{styleStandard}
\ \ \ \ ‘the Indian (called) Big Bear’
\end{styleStandard}

\begin{styleStandard}
To sum up, it is clear that the three languages differ in their distribution of adjectival inflections in canonical and non-canonical constructions.
\end{styleStandard}

\begin{styleStandard}
\ \ Roehrs \& Julien (2014) show that German and Norwegian differ quite generally. Documenting with nine sets of data (possessives involving proper names and pronominals, embedded and unembedded proper names, “dis-agreeing” pronominal DPs, appositives, definite adjectives, vocatives, and discontinuous noun phrases), they show that German consistently has strong endings but that Norwegian has weak endings. They propose that adjectival inflections in German are a function of lexical and structural factors (see Chapter 2) but that Norwegian is regulated by semantic ones (also Harbert 2007: 131; Lohrmann 2011; Schoorlemmer 2009, 2012, and others). Roehrs (2015) points out that Yiddish is similar to German but differs in the lexical elements that bring about the weak inflections on the adjectives.
\end{styleStandard}

\begin{styleStandard}
2.1.3. Cross-linguistic Differences with Singular Indefinite Articles
\end{styleStandard}

\begin{styleStandard}
Consider the three languages as regards the singular indefinite articles. In German, the indefinite article \textit{ein} ‘a’ is part of the so-called \textit{ein}{}-words (a convenient term often used in textbooks on German, for instance, Rankin \& Wells 2016). The latter comprise the indefinite article \textit{ein} ‘a’, the (stressed) singularity numeral \textit{EIN }‘one’, the negative article \textit{kein} ‘no’, and possessive articles like \textit{mein} ‘my’, \textit{dein} ‘your’, etc. Given certain similarities (Chapter 5), these words are taken to be related to one another and are usually discussed together. Comparing the three languages as regards these types of words, we make the initial observation that German, Yiddish, and Norwegian show the same basic patterns – \textit{ein}{}-words precede their related head nouns (stressed singularity numerals that have the same spelling as their corresponding indefinite articles are distinguished by capitalization):
\end{styleStandard}

\begin{styleFootnote}
(14)\ \ a.\ \ \textit{ein / EIN / kein / mein Bruder}\ \ \ \ (German)
\end{styleFootnote}

\begin{styleFootnote}
\ \ \ \ a \ \ \ / one \ / no \ \ / my \ \ \ brother.\textsc{masc}
\end{styleFootnote}

\begin{styleFootnote}
\ \ \ \ ‘a/one/no/my brother’
\end{styleFootnote}

\begin{styleFootnote}
b.\ \ \textit{a / eyn / keyn / mayn bruder}\ \ \ \ (Yiddish)
\end{styleFootnote}

\begin{styleFootnote}
\ \ \ \ a / one / no \ \ \ / my \ \ \ \ brother.\textsc{masc}
\end{styleFootnote}

\begin{styleFootnote}
\ \ \ \ ‘a/one/no/my brother’
\end{styleFootnote}

\begin{styleFootnote}
c.\ \ \textit{en / EN / ingen / min bror}\ \ \ \ (Norwegian)
\end{styleFootnote}

\begin{styleFootnote}
\ \ \ \ a \ \ / one / no \ \ \ \ / my \ brother.\textsc{cmn}
\end{styleFootnote}

\begin{styleFootnote}
\ \ \ \ ‘a/one/no/my brother’
\end{styleFootnote}

\begin{styleFootnote}
Considering (14a-c), we note that the singularity numeral, the negative article, and the possessive article are related to the indefinite article in German in that all these forms involve \textit{ein}. This is less straightforward with Yiddish and Norwegian, where the possessive element, for instance, does not seem to relate to the indefinite article in a straightforward way. Besides this first indication that the three languages differ, Yiddish and Norwegian show other, syntactic distinctions.
\end{styleFootnote}

\begin{styleFootnote}
\ \ Yiddish is different from German and Norwegian in at least three aspects. Yiddish allows an inflected possessive pronoun to precede an indefinite article (15a), the negative article can co-occur with the numeral for ‘one’ (15b), and the – what looks like – inflected singularity numeral may precede an indefinite article (15c). The following data are taken from Lockwood (1995: 54, 66) and Weinreich (1999: 195, 205):
\end{styleFootnote}

\begin{styleFootnote}
(15)\ \ a.\ \ \textit{mayn-er \ \ \ a bruder} \ \ \ \ \ \ (Yiddish)
\end{styleFootnote}

\begin{styleFootnote}
\ \ \ \ mine-\textsc{infl} a brother.\textsc{masc}
\end{styleFootnote}

\begin{styleFootnote}
\ \ \ \ \ ‘a brother of mine’
\end{styleFootnote}

\begin{styleFootnote}
\ \ b.\ \ \textit{keyn eyn land} 
\end{styleFootnote}

\begin{styleFootnote}
\ \ \ \ no \ \ \ one country.\textsc{neut}
\end{styleFootnote}

\begin{styleFootnote}
‘not a single country’
\end{styleFootnote}

\begin{styleFootnote}
\ \ c.\ \ \textit{eyn-er \ \ \ \ a mentsh} 
\end{styleFootnote}

\begin{styleFootnote}
\ \ \ \ one-\textsc{infl} a person.\textsc{masc}
\end{styleFootnote}

\begin{styleFootnote}
\ \ \ \ \ ‘a certain person’
\end{styleFootnote}

\begin{styleFootnote}
I argue in Roehrs (2022) that the inflected possessive pronoun in (15a) involves a second, separate nominal that is in a higher position than the possessive articles in Yiddish (14b), in German (14a), and in Norwegian (14c). This means that (15a) presents a non-canonical construction. Turning to (15b), \textit{eyn} is syntactically optional. Semantically, it seems to intensify the negation (if present). As for (15c) and similar to (15a), Yiddish most likely involves a structure different from the \textit{ein}{}-words in the three distributions illustrated in (14). Extending the discussion to \textit{epes a khaver} ‘(something a =) some friend’, it is proposed in the aforementioned paper that \textit{eyner} and \textit{epes} are in a higher position similar to the possessive pronoun in (15a). 
\end{styleFootnote}

\begin{styleFootnote}
\ \ Norwegian is also different from German and Yiddish in that it tolerates the indefinite article between two adjectives and between adjectives and nouns (data are from Julien 2002: 269):
\end{styleFootnote}

\begin{styleStandard}
(16) \ ?\ \ \textit{eit stor-t \ eit styg-t \ \ \ eit hus}\ \ \ \ \ \ ([Nynorsk] Norwegian)
\end{styleStandard}

\begin{styleStandard}
\ \ a \ \ big-\textsc{st} an ugly-\textsc{st} a \ \ house.\textsc{neut}
\end{styleStandard}

\begin{styleStandard}
\ \ ‘a big ugly house’
\end{styleStandard}

\begin{styleFootnote}
Julien (2002: 270) argues that the articles in (16) cannot be interpreted as adjectival agreement – note that each adjective has a strong inflection. She proposes that the lower instances of the article are in lower head positions of the nominal structure (Julien’s $\alpha $) and that the adjectival inflections are part of complex specifiers that contain the adjectives (see Section 4.1.3).
\end{styleFootnote}

\begin{styleStandard}
To conclude, this section has shown that adjectival inflections and indefinite articles are cross-linguistically similar but crucially not the same. Illustrating with German, Yiddish, and Norwegian, this could be observed in both canonical and non-canonical constructions. This conclusion is strengthened when we look at indefinite articles occurring in plural contexts – again, we will see that languages do not pattern the same.
\end{styleStandard}

\begin{styleStandard}
\textit{2.2.\ \ The Indefinite Article in Plural Contexts}
\end{styleStandard}

\begin{styleStandard}
One of the proposals of this book is that the indefinite article \textit{ein} ‘a’ is not a reflex of singular number. This is probably the most controversial claim of these pages. After some brief cross-linguistic remarks, I provide empirical evidence that shows that \textit{ein} can indeed occur in various plural contexts in many German dialects, both in the north and in the south of central Europe. In my view, this shows that \textit{ein} is not a reflex of singular number. In the second subsection, I discuss some of the linguistic properties of these non-singular instances.
\end{styleStandard}

\begin{styleStandard}
2.2.1. Occurrence
\end{styleStandard}

\begin{styleStandard}
The indefinite article in plural contexts is perhaps most prominently discussed by Bennis \textit{et al}. (1998) for Dutch, where this element occurs in various contexts, in \textit{wh}{}-exclamatives, non-\textit{wh}{}-exclamatives, and \textit{N-of-a-N} constructions ((17a,c) are taken from Bennis \textit{et al}. 1998: 98, 101; (17b) is from van Riemsdijk 2005: 165):
\end{styleStandard}

\begin{styleStandard}
(17) \ \ a.\ \ \textit{Wat *(een) jongens!\ \ \ \ \ \ }(Dutch)
\end{styleStandard}

\begin{styleStandard}
\ \ \ \ what \ \ a \ \ \ \ \ boys
\end{styleStandard}

\begin{styleStandard}
\ \ \ \ ‘What boys!’
\end{styleStandard}

\begin{styleFootnote}
b.\ \ \textit{Die auto heeft een deuken!}\ \ 
\end{styleFootnote}

\begin{styleFootnote}
\ \ \ \ that car \ \ has \ \ a \ \ \ \ dents
\end{styleFootnote}

\begin{styleFootnote}
\ \ \ \ ‘That car has dents!’
\end{styleFootnote}

\begin{styleStandard}
\ \ c.\ \ \textit{idioten van (een) mannen}
\end{styleStandard}

\begin{styleStandard}
\ \ \ \ idiots \ \ of \ \ \ \ a \ \ \ \ \ men
\end{styleStandard}

\begin{styleStandard}
\ \ \ \ ‘idiots of men’
\end{styleStandard}

\begin{styleStandard}
This plural element is often referred to as spurious article. 
\end{styleStandard}

\begin{styleStandard}
It is perhaps less well known that it also occurs in other Germanic languages, specifically in the three languages discussed above. We see below that all three languages have indefinite articles in plural contexts in \textit{wh-}constructions. While the occurrence of the article is restricted to that context in Yiddish, the other languages have more options but differ from each other; for instance, Swedish\footnote{\ Due to lack of data for Norwegian, I discuss closely related Swedish here. Delsing (1993: 33) mentions though that the indefinite article is also possible in plural contexts in Norwegian dialects. Unfortunately, the referenced works in Delsing are currently not accessible to me. Note that Icelandic also allows the indefinite article to occur in certain plural contexts (Pétursson 1992: 81).} has this article also in predicative contexts, northern dialects of German also in constructions involving \textit{so} ‘such’, and southern dialects of German have plural \textit{ein} also in argumental expressions without \textit{so} ‘such’. After some brief remarks about Yiddish and Swedish, I engage in a more detailed discussion of German.
\end{styleStandard}

\begin{styleStandard}
Yiddish has indefinite articles in plural contexts in \textit{wh-}constructions ((18a) is taken from Jacobs 2005: 188; (18b) is from Lockwood 1995: 55):\footnote{\ In Olsvanger’s collection of stories, there is a second example, similar to (18c), that also shows the verb in the singular (his page 172). In German, the verb would have to be in the plural to yield a grammatical example.}
\end{styleStandard}

\begin{styleStandard}
(18) \ \ a.\ \ \textit{vos \ \ \ far a bikher\ \ \ \ \ \ }(Yiddish)\newline
 \ \ \ \ what for \textsubscript{\ }a books
\end{styleStandard}

\begin{styleStandard}
\ \ \ \ \ ‘what kind of books’
\end{styleStandard}

\begin{styleStandard}
\ \ b.\ \ \textit{voser \ \ \ \ \ a shprakhn}
\end{styleStandard}

\begin{styleStandard}
\ \ \ \ what.for a languages
\end{styleStandard}

\begin{styleStandard}
\ \ \ \ ‘what kind of languages’
\end{styleStandard}

\begin{styleStandard}
\ \ c.\ \ \textit{, vos \ \ iz dos \ far a verter}. 
\end{styleStandard}

\begin{styleStandard}
\ \ \ \  \ what is that for a words
\end{styleStandard}

\begin{styleStandard}
\ \ \ \ ‘what kind of words are these’
\end{styleStandard}

\begin{styleStandard}
(from Olsvanger’s \textit{Röyte pomerantsen}, p. 98)
\end{styleStandard}

\begin{styleStandard}
In addition, this element has also been identified in Swedish. Delsing (1993: 33-35, 143-44) points out that it occurs in \textit{wh-}constructions (19a). Delsing states that in predicative contexts, the article is “nearly obligatory” (19b). Note that it can also surface between an adjective and a noun (19c):
\end{styleStandard}

\begin{styleStandard}
(19) \ \ a.\ \ \textit{Vad \ }\textit{\textsubscript{\ }}\textit{är \ ni \ \ \ för ena filurer?\ \ \ \ \ \ }(colloquial Swedish)
\end{styleStandard}

\begin{styleStandard}
\ \ \ \ what are you for a \ \ \ \ slyboots
\end{styleStandard}

\begin{styleStandard}
\ \ \ \ ‘What kind of slyboots are you?’
\end{styleStandard}

\begin{styleStandard}
b.\ \ \textit{Pelle och Lisa är *(ena) idioter\ \ \ \ }
\end{styleStandard}

\begin{styleStandard}
\ \ \ \ Pelle and Lisa are \ \ a \ \ \ \ \ idiots 
\end{styleStandard}

\begin{styleStandard}
\ \ \ \ ‘Pelle and Lisa are idiots.’
\end{styleStandard}

\begin{styleStandard}
\ \ c.\ \ \textit{Dänna \ \ \ \ \ var \ \ \ he \ \ \ \ stor a husa}.\ \ \ \ (Northern Swedish)
\end{styleStandard}

\begin{styleStandard}
\ \ \ \ over.there were there big \ a \ houses
\end{styleStandard}

\begin{styleStandard}
\ \ \ \ ‘There were big houses over there.’
\end{styleStandard}

\begin{styleStandard}
Delsing points out that these are non-argumental articles.
\end{styleStandard}

\begin{styleStandard}
\ \ It is less clear if this article occurs in German. While its existence is sometimes played down or even denied (e.g., Schoorlemmer 2009: 238 fn. 50), I argue that it is beyond doubt that \textit{ein} occurs in plural contexts in this language as well. Having said that, there seems to be a difference between northern and southern dialects of German. As far as I have been able to establish, there are three syntactic contexts in Northern German where plural \textit{ein} occurs with a following overt noun. Similar to Yiddish and Swedish, German has plural \textit{ein} in the \textit{wh-}construction (20a). Furthermore, these dialects seem to have (20b) (cf. (17b) in Dutch above). The latter type of example might be related to (20c), at least in some dialects:\footnote{\ There is another context in which \textit{ein} can appear in plural contexts:\par (i)\ \ \textit{die einen, die anderen}\par \ \ \ \ the ones \ \ the others\par \ \ \ \ ‘these, those’\par However, this \textit{ein} is of a different type – it is adjectival (see Chapter 5).}
\end{styleStandard}

\begin{styleStandard}
(20) \ \ a.\ \ \textit{Was \ für’ne / Was \ \ für eine Idioten!\ \ \ \ }(Northern German)
\end{styleStandard}

\begin{styleStandard}
\ \ \ \ what for.a \ \ \ / what for a \ \ \ \ \ idiots
\end{styleStandard}

\begin{styleStandard}
\ \ \ \ ‘What kind of idiots!’
\end{styleStandard}

\begin{styleStandard}
\ \ b.\ \ \textit{Eine Idioten!}
\end{styleStandard}

\begin{styleStandard}
\ \ \ \ a \ \ \ \ \ idiots
\end{styleStandard}

\begin{styleStandard}
\ \ \ \ ‘Idiots!’
\end{styleStandard}

\begin{styleStandard}
\ \ c.\ \ \textit{So’ne / So eine Idioten!}
\end{styleStandard}

\begin{styleStandard}
\ \ \ \ so.a \ \ \ / so a \ \ \ \ idiots
\end{styleStandard}

\begin{styleStandard}
\ \ \ \ ‘Such idiots!’
\end{styleStandard}

\begin{styleStandard}
Note that all these forms involve exclamative contexts. As will become clear though, (20c) can also occur in argument positions in Northern German (see also the examples from Southern German below, where these strings occur in argument position without \textit{so} ‘such’). For the most part, I will focus on the exclamative constructions.
\end{styleStandard}

\begin{styleStandard}
There is no doubt that \textit{’ne} in (20a) and (20c) stands for \textit{eine}. In an online search (for details, see momentarily), I identified 152 examples involving plural \textit{ein}. These examples are collected in the Appendix, Sections 1-3. Of these 152 tokens, 21 involved the full form \textit{eine}, 19 with \textit{was} \emph{für \textup{and }}2 with \textit{so} ‘such’ (both examples below are taken from the Appendix):
\end{styleStandard}

\begin{styleCommentText}
(21)\ \ a.\ \ {\textquotedbl}\emph{Was \ \ für eine Idioten}{\textquotedbl}\textit{ sagte Liam}
\end{styleCommentText}

\begin{styleCommentText}
\ \ \ \  \ what for \ a \ \ \ \ idiots \ \ \ \ \ said \ Liam
\end{styleCommentText}

\begin{styleCommentText}
\ \ \ \ ‘“Such idiots!” said Liam.’
\end{styleCommentText}

\begin{styleStandard}
\ \ b.\ \ \emph{Das Personal war sehr unfreundlich. Ich hatte noch nie \ \ \ \ so eine Ferien\textup{.}}~
\end{styleStandard}

\begin{styleStandard}
\ \ \ \ the \ staff \ \ \ \ \ \ \ \ was very unfriendly \ \ \ \ I \ \ \ \ had \ \ \ still \ never so a \ \ \ \ \ holidays\ \ 
\end{styleStandard}

\begin{styleStandard}
\ \ \ \ ‘The staff was very unfriendly. I have never had such a (bad) vacation.’
\end{styleStandard}

\begin{styleStandard}
Indeed, \textit{’ne} is a typical reduction of the indefinite article as can readily be seen in the nominative feminine: \textit{eine Frau} {\textgreater} \textit{’ne Frau} ‘a woman’. Note that \textit{so’ne} can also be spelled as \textit{so ne }or\textit{ sone}.\footnote{\ While spelling can only be used cautiously as evidence, notice that the speakers of the examples listed in the Appendix, Sections 1-3 spell \textit{so} and \textit{eine}, when the latter is reduced, as \textit{so ne} (61 times) or \textit{so’ne} (26 times). It is perhaps telling that not a single speaker provided the written form \textit{sone}. In my view, this hints at the fact that these speakers analyze the two elements as separate items, similar to \textit{so eine }(2 times).} I refer to \textit{so} and plural \textit{ein} by using the first variant (\textit{so’ne}).
\end{styleStandard}

\begin{styleStandard}
\ \ In more detail, I conducted an online seach of the three cases in (20) above through Google, on Twitter, and on Facebook. All three cases in (20) occur with different frequencies: while (20b) is hard to search for, (20c) seems to be the most common construction in Northern German. The numeric results are provided in Table 1a. Note though that it would be easy to find more examples, especially for columns four and five of Table 1a, where I restricted my search to certain nouns, numerals, or adjectives (again, all these hits are listed in the Appendix, Sections 1-3):\footnote{\ When searching for these types of examples, care must be taken; for instance, many cases involve false positives where a following plural noun actually forms the first part of a singular compound:\par (i) \ \ \textit{Was \ für ne Männer Wg?} \par what for a \ \ men \ \ \ \ \ \ living.community.\textsc{fem}\par ‘What kind of a living community for men?’\par The examples in the Appendix involve genuine plural nouns (which are not part of compounds).}
\end{styleStandard}

\begin{styleStandard}
Table 1a: Numeric Results of Plural \textit{ein} Identified in Online Media (September 20, 2020)
\end{styleStandard}

\begin{flushleft}
\begin{tabular}{|m{0.73025984in}|m{1.0622599in}|m{1.5462599in}|m{1.2337599in}|m{1.6087599in}|}

\hline
String &
\centering \textit{was f}\emph{ü}\textit{r (ei)ne} &
\centering \textit{was f}\emph{ü}\textit{r (ei)ne} Adj-\textit{en} &
\centering{\itshape so (ei)ne} &
\centering\arraybslash \textit{so (ei)ne} Adj-\textit{en}\\\hline
Source &
 &
 &
 &
\\\hline
Google &
\centering 19 &
\centering 12 &
\centering 4 (with the noun \textit{Ferien }‘holidays’) &
\centering\arraybslash 23 (with the adjective \textit{geil} ‘awesome’) and 10 (with s\emph{ü}\textit{ß }‘cute’)\\\hline
Twitter &
\centering 28 &
\centering 4 &
\centering 3 (with the numeral \textit{zwei }‘two’ following) &
\centering\arraybslash 23 (with the adjective \textit{geil} ‘awesome’) and 21 (with \textit{dumm }‘stupid’)\\\hline
Facebook &
\centering hard to search &
\centering hard to search &
\centering hard to search &
\centering\arraybslash 5\\\hline
Total: 152 &
\centering 47 &
\centering 16 &
\centering 7 &
\centering\arraybslash 82\\\hline
\end{tabular}
\end{flushleft}
\begin{styleStandard}
In the Appendix, Sections 1-3, the examples are provided in the order of the rows and columns of Table 1a, with the relevant strings in bold print for ease of identification (no glosses or translations are given there). Unless indicated otherwise, all examples below involving \textit{ein} in plural contexts are authentic; that is, these data are taken from the Appendix, Sections 1-3. 
\end{styleStandard}

\begin{styleStandard}
In addition to finding data in online media, I conducted a search of plural \textit{ein} in \textit{Datenbank für Gesprochenes Deutsch} (DGD; Database of Spoken German; available online at \url{https://dgd.ids-mannheim.de/}). The DGD involves a convenient platform comprising a number of spoken corpora that can be searched individually or collectively. This search yielded a total of 60 instances of plural \textit{ein}, broken down according to corpus and construction in Table 1b.\footnote{\ It is unlikely that this search was exhaustive, given the size of the corpora and the fact that plural \textit{ein} is spelled in the DGD in different ways. Transcribers of the DGD, who presumably have training in linguistics, spell \textit{so }and \textit{ne} about equally as two separate words like \textit{so ne }(8 times), \textit{so-ne }(18 times), \textit{so\_ne} (1 once; total count: 27) or as one word like \textit{sone }(27 times). However, there may be other variants. Note that when searching for examples, researchers should not enter \textstylew{\textit{für-ne}} or \textit{so-ne} (or \textstylew{\textit{für\_ne}} or \textit{so\_ne}, for that matter) in the search bar: while some data points are orthographically represented like this in some of the texts, they can only be found by replacing the hyphen (or underscore) with a space (i.e., \textit{f}\textstylew{\textit{ü}}\textit{r ne}, \textit{so ne}). Finally, it should also be pointed out that a number of very similar and a few unclear hits were removed from the results.\par } To allow a direct comparison, Table 1b is organized in the same way as Table 1a above (further below, I comment on the strong adjectives marked by ST in parentheses in Table 1b). 
\end{styleStandard}

\begin{styleStandard}
Table 1b: Numeric Results of Plural \textit{ein} Identified in DGD (October 6, 2024)
\end{styleStandard}

\begin{flushleft}
\begin{tabular}{|m{0.7268598in}|m{1.0490599in}|m{1.5226599in}|m{1.2170599in}|m{1.5837599in}|}

\hline
String &
\centering \textit{was f}\emph{ü}\textit{r (ei)ne} &
\centering \textit{was f}\emph{ü}\textit{r (ei)ne} Adj-\textit{en} &
\centering{\itshape so (ei)ne} &
\centering\arraybslash \textit{so (ei)ne} Adj-\textit{en}\\\hline
Source\textsuperscript{a} &
 &
 &
 &
\\\hline
AD &
\centering 2 &
 &
 &
\centering\arraybslash 1(ST)\\\hline
BW &
\centering 1 &
 &
\centering 18 &
\centering\arraybslash 9\\\hline
BR &
 &
 &
 &
\centering\arraybslash 1\\\hline
DH &
 &
 &
\centering 2 &
\\\hline
DNAM &
\centering 3 &
 &
\centering 6 &
\\\hline
FEGB &
 &
 &
\centering 1 &
\\\hline
FOLK &
 &
 &
\centering 5 &
\centering\arraybslash 3\\\hline
GWSS &
 &
 &
\centering 1 &
\\\hline
MEND &
 &
 &
\centering 2 &
\\\hline
RUDI &
 &
 &
\centering 1 &
\centering\arraybslash 1(ST) / 1(ST \textit{so})\\\hline
ZW &
 &
 &
\centering 1 &
\centering\arraybslash 1\\\hline
Total: 60 &
\centering 6 &
 &
\centering 37 &
\centering\arraybslash 17\\\hline
\end{tabular}
\end{flushleft}
\begin{styleStandard}
\textsuperscript{a}Corpora:
\end{styleStandard}

\begin{styleStandard}
AD \ \ = Australiendeutsch (Australian German)
\end{styleStandard}

\begin{styleStandard}
BW \ \ = Berliner Wendekorpus (Berlin Corpus of the Peaceful Revolution)
\end{styleStandard}

\begin{styleStandard}
BR \ \ = Biographische und Reiseerzählungen (Biographical and Travel Stories)
\end{styleStandard}

\begin{styleStandard}
DH \ \ = Deutsch heute (German Today)
\end{styleStandard}

\begin{styleStandard}
DNAM= Deutsch in Namibia (German in Namibia)
\end{styleStandard}

\begin{styleStandard}
FEGB \ \ = Flucht und Emigration nach Großbritannien (Escape and Emigation to Great Britain)
\end{styleStandard}

\begin{styleStandard}
FOLK \ \ = Forschungs- und Lehrkorpus Gesprochenes Deutsch (Research and Teaching Corpus of \ \ Spoken German)
\end{styleStandard}

\begin{styleStandard}
GWSS\ \ = Gesprochene Wissenschaftssprache Kontrastiv (Spoken Scientific Language \ \ Contrastively)
\end{styleStandard}

\begin{styleStandard}
MEND\ \ = Mennonitenplautdietsch in Nord- und Südamerika (Mennonite Low German in North \ \ and South America)
\end{styleStandard}

\begin{styleStandard}
RUDI \ \ = Russlanddeutsche Dialekte (Russian German Dialects)
\end{styleStandard}

\begin{styleStandard}
ZW \ \ = Zwirner-Korpus (Corpus named [after] Zwirner)
\end{styleStandard}

\begin{styleStandard}
Similar to the search of online media, almost all relevant speakers in the DGD are from the northern part of Germany (to the extent available, this type of information is provided with all the retrieved examples in the Appendix, Section 4).\textstyleFootnoteSymbol{ }About 45 percent of the examples involving plural \textit{ein} are contained in the Berlin Corpus alone – all speakers there are from Berlin and its surrounding area. In fact, some speakers from other corpora are also from this region. Note that in general, speakers are of different genders and ages (some born as early as 1879). 
\end{styleStandard}

\begin{styleStandard}
The examples in the Appendix, Section 4 are sorted according to the name of the corpus they appear in, and the corpora are provided in alphabetical order. The examples are given in the order of the columns of Table 1b. As above, the relevant strings are provided in bold print, and no glosses or translations are given. Some comments on the numeric results in Table 1b are in order.
\end{styleStandard}

\begin{styleStandard}
Note that there are relatively few instances (six) involving plural \textit{ein} in \textit{was-f}\emph{ü}\textit{r} contexts, and some of these instances are special in other ways. While one example is from a speaker in the Berlin Corpus (no specific information about this speaker is available), all five other examples were recorded in areas outside Germany – one adult speaker was born in Germany but lives now in Australia, one speaker is probably an interviewer from Germany making the recordings in Australia, and three speakers were pupils born in Namibia where at least one parent speaks German. One may wonder now why there are not more examples of this type. This probably has to do with the fact that these constructions are less frequent than those involving \textit{so’ne} in general (see also Table 1a) and the fact that plural \textit{ein} in \textit{was-f}\emph{ü}\textit{r} constructions typically involves strongly emotive language (e.g., swear words) used in exclamative contexts. None of the six examples are of this type, and the corpora in the DGD do not seem to involve many such instances in general (perhaps due to the fact that these are recorded conversations).
\end{styleStandard}

\begin{styleStandard}
\ \ As to the three strong adjectives, there are two types to distinguish: two instances marked by (ST) vs. one instance indicated by (ST \textit{so}) in Table 1b. Starting with the first type, one of the two examples was found in the corpus \textit{Australiendeutsch} uttered by a person born in Australia with ancestors from Silesia; the other was identified in the corpus \textit{Russlanddeutsche Dialekte} said by a person born in the Ukraine. Regarding the second type, this example was retrieved from the corpus \textit{Russlanddeutsche Dialekte} uttered by the same person just mentioned where, crucially, the strong adjective precedes \textit{so’ne}. While the strong adjective in the latter case is expected (there is no preceding determiner), the first two instances are in contrast to what I have generally encountered in my searches. Having said that, these two cases presumably involve speakers where German is not the dominant language, and I will move forward assuming that only weak adjectives occur in these contexts. 
\end{styleStandard}

\begin{styleStandard}
These two searches totaled over 210 examples – all listed in the Appendix. Furthermore, there are a few examples that were identified in other ways as indicated further below (they are not part of the example counts in the tables above). Besides that, I also consulted some non-linguist friends from the county of Oberhavel (north of Berlin, Germany). All allowed (20c), one found (20b) to be fine, but most stated that (20a) was somewhat marked. The individual who allowed (20b) took it to be a short form of (20c). 
\end{styleStandard}

\begin{styleStandard}
\ \ Observe now that the occurrence of \textit{ein} in plural contexts has already been established independently. Elmentaler \& Rosenberg (2015: 389-394) document cases of plural \textit{so’ne} ‘(so’a =) such’ in \textit{Norddeutscher Sprachatlas }(Northern German Language Atlas).\footnote{\ The existence of plural \textit{ein} is particularly clear in the central and eastern parts of this area (in the western part, the plural form \textit{so }as in \textit{so} \textit{Dinge} ‘(so =) such things’ is more prevalent). Elmentaler \& Rosenberg (2015: 383) also point out that the full form (\textit{so eine Dinge}) is not possible. I disagree. As mentioned above, I found a few examples of this type. In fact, if the indefinite element is stressed – as is fairly common in exclamative contexts, unreduced \textit{eine} is certainly possible (and indeed required; see some of the \textit{was-f}\emph{ü}\textit{r} constructions).} Below are two examples from Elmentaler \& Rosenberg (2015: 392), who identified this element in both interviews and (dinner) table conversations:
\end{styleStandard}

\begin{styleStandard}
(22) \ \ a.\ \ \textit{so’ne Kurse\ \ \ \ \ \ \ \ }(Northern German)
\end{styleStandard}

\begin{styleStandard}
\ \ \ \ so a \ \ courses
\end{styleStandard}

\begin{styleStandard}
\ \ \ \ ‘such courses’
\end{styleStandard}

\begin{styleStandard}
\ \ b.\ \ \textit{so’ne komischen Löcher}
\end{styleStandard}

\begin{styleStandard}
\ \ \ \ so a \ \ weird \ \ \ \ \ \ \ \ holes
\end{styleStandard}

\begin{styleStandard}
\ \ \ \ ‘such weird holes’
\end{styleStandard}

\begin{styleStandard}
These authors agree with Keller (2004: 6-7) that \textit{so’ne} fills a gap by forming an indefinite demonstrative. Note though that this idea cannot extend to the other constructions in (20a-b) (which lack \textit{so}). In my own search, I came across examples with clear geographical indications that these types of cases exist in Berlin, Hamburg, and Lübeck. In addition, linguistic features such as unshifted consonants (e.g., \textit{wat} for \textit{was }‘what’) indicate the same geographical area.
\end{styleStandard}

\begin{styleFootnote}
\ \ Besides northern dialects of German, this element also occurs in southern dialects like Swabian and Bavarian (Glaser 1993: 106). While its use and distribution are not the same across all subdialects, it is clear that plural \textit{ein} does occur (23a). The latter is a Bavarian example taken from Glaser (1993: 109). Hubert Haider (p.c.) points out that these cases also exist in Southern Bavarian, for instance, in Carinthia, Austria (23b). Another southern example was identified during my search of onlike media (23c):\footnote{\ Note that second-person plural \textit{es} ‘you’ in (23b) derives from an old dual pronoun meaning ‘you (two)’ (Paul \textit{et al}. 1989: 222). The -\textit{s} on the verb \textit{brauchen} ‘need’ is an agreement marker sharing features with \textit{es}. I glossed the -\textit{s} on the verb \textit{haben} ‘have’ in (23c) as an encliticized \textit{es}. I thank David Fertig and Mark Louden for discussion of these points. Notice also that more cases of Southern Bavarian can be found in \textit{Kärntner Wörterbuch} (Carinthian dictionary) available at~\href{https://nam04.safelinks.protection.outlook.com/?url=http\%3A\%2F\%2Fwww.stelzel.at\%2Fkaernten-woerterbuach\%2Fwort-suachn&data=02\%7C01\%7CDorian.Roehrs@unt.edu\%7C898c9f868c1844be616208d85eba3e2f\%7C70de199207c6480fa318a1afcba03983\%7C0\%7C0\%7C637363501307224768&sdata=IQdk9TJvjkG\%2BO\%2Fu5wBkFLnUx4ibawqb3ZG35j6UxqKU\%3D&reserved=0}{\textstyleInternetlink{http://www.stelzel.at/kaernten-woerterbuach/wort-suachn}}. In fact, these types of examples have even found their way into the media (\url{https://www.diepresse.com/443923/karntnerisch-fur-anfanger}) and advertisements (\href{https://nam04.safelinks.protection.outlook.com/?url=https\%3A\%2F\%2Fpensionria.at\%2Fwoerthersee-kaerntnerisch\%2F&data=02\%7C01\%7CDorian.Roehrs@unt.edu\%7Ca92114d23c0e4d0283d208d85f8406f3\%7C70de199207c6480fa318a1afcba03983\%7C0\%7C0\%7C637364367960730551&sdata=gKpHkjpnbKev9SXmEq6vjuXAmE8ubb7AZC0YmQz3ThY\%3D&reserved=0}{\textstyleInternetlink{https://pensionria.at/woerthersee-kaerntnerisch/}}). I thank Hubert Haider for providing these sources to me.}
\end{styleFootnote}

\begin{styleStandard}
(23) \ \ a.\ \ \textit{D[1EB?] \ \ \ sàn õa Epfe \ \ drõ}.\ \ \ \ (Bavarian German)\ \ 
\end{styleStandard}

\begin{styleStandard}
\ \ \ \ there are a \ \ apples there.on
\end{styleStandard}

\begin{styleStandard}
\ \ \ \ ‘There are apples on it.’
\end{styleStandard}

\begin{styleStandard}
\ \ b.\ \ \textit{Brauchts \ es~ \ ane Untatatzalan~oda tans de \ Schalalan alaan aa?}
\end{styleStandard}

\begin{styleStandard}
need.\textsc{agr} you a \ \ \ \ saucers \ \ \ \ \ \ \ \ \ or \ \ do.it the cups \ \ \ \ \ \ \ \ \ alone too
\end{styleStandard}

\begin{styleStandard}
‘Do you need saucers or will the cups by themselves do?’
\end{styleStandard}

\begin{styleStandard}
c.\ \ \textit{Was \ für ne Vögel habts \ \ \ \ \ \ denn da \ \ \ \ aufgetrieben?}
\end{styleStandard}

\begin{styleFootnote}
\ \ \ \ what for a \ \ idiots have.you \textsc{prt} \ \ there found
\end{styleFootnote}

\begin{styleFootnote}
\ \ \ \ ‘What kind of idiots have you got there?’
\end{styleFootnote}

\begin{styleFootnote}
Unlike many of the instances in Northern German, the cases in (23a-b) involve examples in argument positions (that also lack \textit{so} ‘such’).
\end{styleFootnote}

\begin{styleFootnote}
Before moving on, it is worth pointing out that the occurrence of this \textit{ein} in Bavarian fits well with the Yiddish facts above. Notice in this respect that the plural indefinite article in Yiddish cannot be an innovation from Slavic (e.g., Polish or Russian) as these languages do not have articles to begin with. However, as is well known, Yiddish is historically related to southern dialects of German, specifically Bavarian (for some discussion, see Jacobs 2005: chapter 2).
\end{styleFootnote}

\begin{styleStandard}
\ \ Indeed, diachronically, plural \textit{ein} is not a recent development as it occurred already in older varieties of German with a specific indefinite interpretation (ENHG stands for Early New High German):\footnote{\ The indefinite article began to evolve from the singularity numeral and indefinite pronoun \textit{ein} in Old High German (Braune \& Reiffenstein 2004: 234). Consequently, I gloss \textit{ein} as ‘one/a’ in Old High German and Middle High German. Also, singular \textit{ein} in the older varieties is often paraphrased using Modern German \textit{irgendein, ein gewisser} (e.g., Braune \& Reiffenstein 2004: 234). This could be rendered into English as ‘a certain’ in the singular and as ‘certain’ in the plural. Finally, I thank Chris Sapp for providing the example from ENHG. He points out though that this sentence could also have a singular interpretation of \textit{eyne} (i.e., ‘…with respect to the 15 families, one with the assessors and those who have sat on the council.’), but it seems unlikely that all of the assessors and city council members came from one\textit{ }particular family.}
\end{styleStandard}

\begin{styleStandard}
(24)\ \ a.\ \ \textit{eino \ \ ziti}\ \ \ \ \ \ \ \ \ \ \ \ (Old High German)
\end{styleStandard}

\begin{styleStandard}
\ \ \ \ one/a times
\end{styleStandard}

\begin{styleStandard}
\ \ \ \ ‘(certain) times’\ \ 
\end{styleStandard}

\begin{styleStandard}
\ \ b.\ \ \textit{in ein\=en buachon}
\end{styleStandard}

\begin{styleStandard}
\ \ \ \ in one/a books
\end{styleStandard}

\begin{styleStandard}
\ \ \ \ ‘in (certain) books’
\end{styleStandard}

\begin{styleStandard}
(both from Otfrid, Braune \& Reiffenstein 2004: 234)
\end{styleStandard}

\begin{styleStandard}
(25)\ \ a.\ \ \textit{in einen zîten}\ \ \ \ \ \ \ \ \ \ \ \ (Middle High German)
\end{styleStandard}

\begin{styleStandard}
\ \ \ \ in one/a times
\end{styleStandard}

\begin{styleStandard}
\ \ \ \ ‘during (certain) times’
\end{styleStandard}

\begin{styleStandard}
\ \ \ \ \ \ (Der Nibelunge Nôt 1143:1, Paul \textit{et al}. 1989: 388)
\end{styleStandard}

\begin{styleStandard}
\ \ b.\ \ \textit{einu \ \ liute}
\end{styleStandard}

\begin{styleStandard}
\ \ \ \ one/a people
\end{styleStandard}

\begin{styleStandard}
\ \ \ \ ‘(certain) people’
\end{styleStandard}

\begin{styleStandard}
\ \ \ \ \ \ (Himmelreich Jerusalem 14:12, Paul \textit{et al}. 1989: 388)
\end{styleStandard}

\begin{styleStandard}
(26)\ \ \textit{Dyt sijnt alle alsulge sachen vnd geschichte, as sich \ dese \ nyeste. xxxvj. Iair} \ \ \ \ \ \ (ENHG)
\end{styleStandard}

\begin{styleStandard}
\ \ that are ~ all ~ such ~ ~ things ~and history ~ \ \ \ \ \ as \textsc{refl} these recent \ 36 ~ \ \ \ years\newline

\end{styleStandard}

\begin{styleStandard}\itshape
her \ enbynnen der Stat van Coelne \ \ oeurmitz \ \ \ \ \ die. xv. geslechte eyne mit \ \ den
\end{styleStandard}

\begin{styleStandard}
~ ~\ \ here in ~ ~ ~ ~ ~ ~the ~city of \ \ Cologne by.means.of the 15 \ families \ \ a \ \ \ \ \ \ with the 
\end{styleStandard}

\begin{styleStandard}
\textit{Scheffen \ vnd den ghenen, die \ \ mit \ \ \ \ yn raide \ \ gesessen haint, ergangen haint}. 
\end{styleStandard}

\begin{styleStandard}
~ \ \ assessors and the those ~ ~ ~who along in council sat \ \ \ \ \ \ \ \ \ have \ happened have
\end{styleStandard}

\begin{styleStandard}
‘Those are all such things and history, as has happened in the last 36 years here in the city of Cologne with respect to the 15 families, some with the assessors and those who have sat on the council.’ (1360 Hauwe Neues Buch Cologne, 33)
\end{styleStandard}

\begin{styleStandard}
If these data are indeed related, then plural \textit{ein} has been in the language for a long time.\footnote{\ It is not clear if the development of the spurious article in Modern German is a direct development from the plural forms in the older varieties. A reviewer states that \textit{so’ne }may\textit{ }originally not come from \textit{so eine}, but may be a phonetically reduced form of \textit{solche}: In the 19th century, Georg Wenker had 40 Standard German sentences (“Wenkersätze”) translated into local dialects by schoolteachers and their pupils (a total of 40,000 schools; this questionnaire and its replies are available online at: \href{https://apps.dsa.info/wenker/}{\textstyleInternetlink{Wenkerbogen-Transliterations-App (dsa.info)}}). Sentence 28 was \textit{Ihr sollt nicht solche}\textbf{\textit{ }}\textit{Kindereien treiben }‘You should not engage in such childishness’. The determiner \textit{solche }is regularily translated as \textit{soon}, \textit{sone}, \textit{sonne}, \textit{sunne}, \textit{sön’n}, \textit{soon}, \textit{son’n }etc. in the Low German area and even in some of the Middle German areas. Sentence 36 was \textit{Was sitzen da für Vögelchen … }‘What kind of little birds are sitting there …’. The replies to the \textit{was}{}-\textit{für} question may already show an enclitic indefinite plural article or an inflected preposition in some of the dialects of the Ruhr valley (town of Benrath: \textit{Wat setten door fün}\textbf{\textit{ }}\textit{Voskes…}; town of Bredeney: \textit{Wat setten do von}\textbf{\textit{ }}\textit{Vögelsches…}). Thus, the dialectal forms of \textit{solche }possibly have been reanalyzed as \textit{so }and \textit{eine}, and \textit{eine }– in some regions – has been reanalyzed as a spurious plural article and has spread to \textit{was-für }contexts. Given these two possible paths of development (direct or indirect via reanalysis), it is clear that more work is needed here to make a decision about the diachrony of plural \textit{ein}. }
\end{styleStandard}

\begin{styleFootnote}
\ \ It seems clear that these cases occur in colloquial contexts, in contemporary speech and writing. This is even acknowledged in Duden, both the dictionary and the grammar. Without indication of the dialect, these works state that \textit{so’ne} ‘such’ occurs in colloquial German ((27a) is taken from Duden 1989: 1414 and (27b) from Duden 1995: 276). 
\end{styleFootnote}

\begin{styleStandard}
(27) \ \ a.\ \ \textit{es gibt \ \ immer \ sone und solch-e}
\end{styleStandard}

\begin{styleStandard}
\ \ \ \ it \ gives always so.a \ and such-\textsc{pl} \ \ 
\end{styleStandard}

\begin{styleStandard}
\ \ \ \ ‘There are all kinds.’
\end{styleStandard}

\begin{styleStandard}
\ \ b.\ \ \textit{Ich kann sone Leute \ \ nicht leiden}.
\end{styleStandard}

\begin{styleStandard}
\ \ \ \ I \ \ \ \ can \ \ so.a \ people not \ \ \ stand
\end{styleStandard}

\begin{styleStandard}
\ \ \ \ ‘I cannot stand such people.’
\end{styleStandard}

\begin{styleFootnote}
\ \ As regards writing, these constructions occur mostly in unedited texts like Twitter messages. However, they can also be found in edited contexts like ebooks such as \textit{Hilfe ich bin’n Vampir} ‘Help, I am a Vampire’ (28a) or \textit{Wolfskinder} ‘Wolf Children’ (28b):
\end{styleFootnote}

\begin{styleStandard}
(28) \ \ a.\ \ \textit{So ne} \textit{Mädchen gehn mir auf die Nerven.}
\end{styleStandard}

\begin{styleStandard}
\ \ \ \ so \ a \ \ girls \ \ \ \ \ \ \ go \ \ \ \ me \ on \ the nerves
\end{styleStandard}

\begin{styleStandard}
‘Such girls get on my nerves.’ 
\end{styleStandard}

\begin{styleStandard}
(\url{https://www.bookrix.de/book.html?bookID=cronalein_1360707129.5978798866#3276,468,12258})
\end{styleStandard}

\begin{styleStandard}
\ \ b.\ \ \textit{Smilla recherchiert und entdeckt, \ das schon \ \ so eine Frauen verschwanden}.
\end{styleStandard}

\begin{styleStandard}
\ \ \ \ Smilla researches \ \ \ and discovers that already so a \ \ \ \ women \ disappeared
\end{styleStandard}

\begin{styleStandard}
\ \ \ \ ‘Smilla did some research and discovered that such women already disappeared.’\ \ \ \ \ \ \ \ (\href{https://www.vorablesen.de/buecher/wolfskinder/rezensionen/duester-%09%09%09%09a6154566-6616-445c-967e-966d6578811c}{\textstyleInternetlink{https://www.vorablesen.de/buecher/wolfskinder/rezensionen/duester-\ \ \ \ \ \ \ \ a6154566-6616-445c-967e-966d6578811c}})
\end{styleStandard}

\begin{styleStandard}
Recall that these examples are not counted in the tables above and that they are not listed in the Appendix. Consider some linguistic properties of these instances involving plural \textit{ein}.
\end{styleStandard}

\begin{styleStandard}
2.2.2. Linguistic Properties
\end{styleStandard}

\begin{styleStandard}
These strings are clearly in the plural: \textit{ein} itself has a plural inflection, the following adjective has the typical weak ending, and the noun has a plural inflection along with the corresponding meaning. Compare (29a) to (29b):
\end{styleStandard}

\begin{styleStandard}
(29) \ \ a.\ \ \emph{Was \ für ein-e geil-en}~ \ \ \ \ \ \ \ \ \ \textit{Bild-er}
\end{styleStandard}

\begin{styleStandard}
\ \ \ \ what for a-\textsc{st} \ awesome-\textsc{wk} picture-s
\end{styleStandard}

\begin{styleStandard}
\ \ \ \ ‘Such awesome pictures!’
\end{styleStandard}

\begin{styleStandard}
\ \ b.\ \ \textit{dies-e \ \ \ \ geil-en \ \ \ \ \ \ \ \ \ \ Bild-er}
\end{styleStandard}

\begin{styleStandard}
\ \ \ \ these-\textsc{st} awesome-\textsc{wk} picture-s
\end{styleStandard}

\begin{styleStandard}
\ \ \ \ ‘these awesome pictures’
\end{styleStandard}

\begin{styleStandard}
Furthermore, verbs and pronouns agree with the plural noun phrase in number:
\end{styleStandard}

\begin{styleStandard}
(30) \ \ a.\ \ \textit{Was \ für eine Idioten wohn-en nur in Sachsen-Anhalt?} 
\end{styleStandard}

\begin{styleStandard}
\ \ \ \ what for a \ \ \ \ \ idiots \ \ live-\textsc{pl} \ \ \ \textsc{prt} in Sachsen-Anhalt
\end{styleStandard}

\begin{styleStandard}
\ \ \ \ ‘What kind of idiots live in Saxony-Anhalt?’
\end{styleStandard}

\begin{styleStandard}
\ \ b.\ \ \textit{was \ \ für eine Idioten, die \ \ \ \ \ \ \ so Aufmerksamkeit woll-en}
\end{styleStandard}

\begin{styleStandard}
\ \ \ \ what for a \ \ \ \ \ idiots \ \ \ \ who.\textsc{pl} so attention \ \ \ \ \ \ \ \ \ \ \ \ want-\textsc{pl}
\end{styleStandard}

\begin{styleStandard}
\ \ \ \ ‘what kind of idiots that want attention like this’
\end{styleStandard}

\begin{styleStandard}
Observe that these cases do not involve syntactically frozen (or fossilized) templates: they can involve several adjectives in a row (31a), coordinated adjectives (31b), and nominalized adjectives (31c):
\end{styleStandard}

\begin{styleStandard}
(31) \ \ a.\ \ \textit{Wieso gibt \ \ es so ne dummen, bescheuerten Leute \ \ die \ sowas \ \ \ \ \ \ \ \ \ machen?? D:}
\end{styleStandard}

\begin{styleStandard}
\ \ \ \ why \ \ \ gives it \ so a \ stupid \ \ \ \ \ dumb \ \ \ \ \ \ \ \ \ \ \ \ people who so.something do
\end{styleStandard}

\begin{styleStandard}
\ \ \ \ ‘Why are there such stupid, dumb people that do something like that?’
\end{styleStandard}

\begin{styleStandard}
b.\ \ \textit{es gibt \ halt so ne dummen und abgehobenen leute}
\end{styleStandard}

\begin{styleStandard}
\ \ it gives \textsc{prt} so a \ \ stupid \ \ \ \ and stuck-up \ \ \ \ \ \ \ \ people
\end{styleStandard}

\begin{styleStandard}
\ \ ‘There are indeed such stupid and stuck-up people.’
\end{styleStandard}

\begin{styleStandard}
\ \ c.\ \ \textit{ach \ so {}'ne dummen gibt \ \ es öfter. xD}
\end{styleStandard}

\begin{styleStandard}
\ \ \ \ oh \ \ \ so a \ \ stupid \ \ \ \ gives it \ more.often
\end{styleStandard}

\begin{styleStandard}
\ \ \ \ ‘Oh, there are often such stupid ones.’
\end{styleStandard}

\begin{styleStandard}
Note in this regard that plural \textit{ein} is not an adjectival article, an article triggered by the presence of an adjective. This is clear from the fact that the cases without an adjective outnumber those with an adjective at a ratio of about 2.5 to 1 (see Table 1b above). 
\end{styleStandard}

\begin{styleStandard}
\ \ Additionally, these types of cases may include numerals (32a-b) or other preceding elements such as \textit{irgend} ‘any’ (32c):
\end{styleStandard}

\begin{styleStandard}
(32) \ \ a.\ \ \textit{drei}~ \ \emph{so ne geilen}~ \ \ \ \ \textit{Mädels}
\end{styleStandard}

\begin{styleStandard}
\ \ \ \ three so a \ \ awesome girls
\end{styleStandard}

\begin{styleStandard}
\ \ \ \ ‘three such awesome girls’
\end{styleStandard}

\begin{styleStandard}
b.\ \ \textit{Da \ \ \ waren heute 2 so'ne dummen Weiber beim \ Zumba-Kurs, \ \ die}
\end{styleStandard}

\begin{styleStandard}
\ \ there were \ \ today 2 so.a \ \ stupid \ \ \ \ women at.the Zumba-course who
\end{styleStandard}

\begin{styleStandard}
\ \ ‘Today, there were two such stupid women at the Zumba course who’
\end{styleStandard}

\begin{styleStandard}
c.\ \ \textit{Alter irgend so ne dummen Hunde haben direkt \ \ \ neben \ mir n Böller \ \ gezündet}
\end{styleStandard}

\begin{styleStandard}
\ \ man \ any \ \ \ \ so a \ \ stupid \ \ \ \ dogs \ \ \ have \ \ directly next.to me a firecracker lit
\end{styleStandard}

\begin{styleStandard}
\ \ ‘Man, some such stupid idiots lit a firecracker right next to me.’
\end{styleStandard}

\begin{styleStandard}
Indeed, numerals can also follow \textit{so’n}e:\footnote{\ I have not found any examples involving numerals in \textit{was-f}\emph{ü}\textit{r} constructions. Note though that while (ia) sounds possible to my ears, (ib) is completely out:\par (i)\ \ a.\ \ \textit{was \ f}\emph{ü}\textit{r ne zwei Typen}\par \ \ \ \ \ \ what for a \ two \ guys\par \ \ \ \ \ \ ‘what kind of two guys’\par \ \ \ \ b. *\ \ \textit{was \ \ f}\emph{ü}\textit{r zwei (ei)ne Typen}\par \ \ \ \ \ \ what for two \ \ a \ \ \ \ \ \ \ guys}
\end{styleStandard}

\begin{styleStandard}
(33) \ \ a.\ \ \textit{ach man wir sind schon so ne zwei wärmflaschen}
\end{styleStandard}

\begin{styleStandard}
\ \ \ \ oh \ boy \ \ we are \ \ \textsc{prt} \ \ \ \ so a \ \ two \ warm.bottles
\end{styleStandard}

\begin{styleStandard}
\ \ \ \ ‘Oh boy, we are such two hot-water bottles.’
\end{styleStandard}

\begin{styleStandard}
\ \ b.\ \ \textit{und so ne zwei Kämpfer die \ \ alles umpflügen,}
\end{styleStandard}

\begin{styleStandard}
\ \ \ \ and so a \ \ two \ fighters \ \ who all \ \ \ \ under.plough
\end{styleStandard}

\begin{styleStandard}
\ \ \ \ ‘and such two guys, who create such chaos’
\end{styleStandard}

\begin{styleStandard}
Finally, as we might expect, \textit{was-f}\emph{ü}\textit{r} splits are also possible. Compare (34a) to (34b):
\end{styleStandard}

\begin{styleStandard}
(34) \ \ a.\ \ \textit{was~ \ }\emph{für ne krassen}\textit{~Menschen albanische Wurzeln haben}
\end{styleStandard}

\begin{styleStandard}
\ \ \ \ what for a \ \ weird \ \ \ \ people \ \ \ \ \ Albanian \ \ \ roots \ \ \ \ \ have
\end{styleStandard}

\begin{styleStandard}
\ \ \ \ ‘what kind of weird people have Albanian roots’
\end{styleStandard}

\begin{styleStandard}
\ \ b.\ \ \textit{was \ meine Unachtsamkeit~}\emph{für ne krassen}\textit{~Folgen \ \ \ \ \ \ \ \ \ \ \ hat}
\end{styleStandard}

\begin{styleStandard}
\ \ \ \ what my \ \ \ \ carelessness \ \ \ \ for a \ \ bad \ \ \ \ \ \ \ consequences has
\end{styleStandard}

\begin{styleStandard}
\ \ \ \ ‘what kind of bad consequences my carelessness has’
\end{styleStandard}

\begin{styleStandard}
To be clear, these are not frozen (or fossilized) templates but productive constructions in (Northern) German. 
\end{styleStandard}

\begin{styleStandard}
\ \ To sum up, I take it as established that \textit{ein} occurs in plural contexts in German. Indeed, \textit{ein} exhibits the hallmarks of a plural element: morphologically, \textit{ein} itself has a typical plural ending; syntactically, it may be followed by an adjective with a weak plural ending; and semantically, it occurs within a noun phrase with plural meaning. As to its geographical distribution, cases like \textit{so’ne Äpfel} ‘such apples’ are common in the northern dialects, and instances like \textit{eine Äpfel} ‘(an) apples’ occur in the southern dialects. Plural \textit{ein} also appears in \textit{wh-}constructions, but this seems less common.\footnote{\ While Duden (1995: 341-42) discusses \textit{was-für} constructions with special forms for the plural in Northern German, this work explicitly states that there is no \textit{ein} in these cases. As is clear, these forms do occur, at least with some speakers.} While a more systematic investigation of the differences between the various dialects and constructions must be left for future research, I proceed on the assumption that most (possibly all) dialects of German have some evidence of plural \textit{ein}. Most of the following discussion focuses on the constructions of the northern dialects. More generally, we may observe that the plural indefinite articles in Dutch, Yiddish, and Swedish are in good company. 
\end{styleStandard}

\begin{styleStandard}
\ \ To conclude the entire section, what this brief illustration of adjectival inflections and indefinite articles has made clear is that languages, even closely related ones like German, Yiddish, and Norwegian/Swedish, differ in these empirical domains. Consequently, they should not receive the same account. Analyses have been advanced for all these languages (e.g., Olsen 1991b, Vater 2002 for German; Roehrs 2015, 2022 for Yiddish; Julien 2005a, Delsing 1993 for Norwegian). However, many languages lack detailed analyses of some or many of the non-canonical constructions (with the notable exception of English, see Perlmutter 1970). More empirical and theoretical work is called for. Awaiting more comprehensive analyses of Yiddish and Norwegian, I focus on German, the language I know best, in the rest of this book striving to make headway for this language. Specifically, I focus on (colloquial) Standard German. Having said that, I make a few cross-linguistic remarks throughout the discussion (e.g., when another language exhibits a revealing similarity or difference).
\end{styleStandard}

\begin{styleStandard}
\ \ As for empirical grounding of this study, most of the data involving adjectival inflections have been reported in previous work (e.g., in the well-known reference grammar Duden). They appear to be relatively straightforward and accepted. As already seen above, this statement is only partially true of \textit{ein}. Furthermore, a distinction between the reduced, unstressed forms of the indefinite article (e.g., \textit{ein} {\textgreater} \textit{’n }‘a’) and the unreduced, stressed forms of \textit{ein} (e.g., \textit{ein} {\textgreater} \textit{EIN} ‘one’) is often not made in the literature. However, such a distinction is made here and plays a crucial part in the classification and analysis of these elements below.\footnote{\ For some phonetic discussion of indefinite (and definite) articles in spontaneous speech, see Wesener (1999). I think Laura Smith for pointing out this reference to me.}
\end{styleStandard}

\begin{styleStandard}\bfseries
3.\ \ Basic Properties and Main Hypotheses
\end{styleStandard}

\begin{styleStandard}
In this section, I discuss some of the characteristics of adjectival inflections and the indefinite article \textit{ein} ‘a’. On the basis of that, I formulate the main hypotheses of this book. 
\end{styleStandard}

\begin{styleStandard}
\textit{3.1.\ \ Adjectival Inflections and the Article }ein
\end{styleStandard}

\begin{styleStandard}
Taking German as the main language under investigation, this monograph focuses on two elements: adjectival inflections and \textit{ein} ‘a’. While these elements have an overt realization and thus receive an interpretation in PF, I argue that they are semantically vacuous. Adjectival inflections are affixes and \textit{ein} is, at least in some cases, a free morpheme. I suggest later that adjectival inflections are licensed by movement of the adjective stem and \textit{ein} by that of certain operators. Some of these operators have an overt manifestation of their own and some do not. 
\end{styleStandard}

\begin{styleStandard}
First, I briefly illustrate why these elements are of interest. Similar to the cases in Section 1, the assumption that adjectival inflections and \textit{ein} make semantic contributions raises some issues. In general terms, if adjectival inflections and \textit{ein} had semantic import, that would lead to some contradictory conclusions. These issues disappear if we assume that these elements are semantically vacuous. This assumption is strengthened by the fact that while these elements have to be generally present for independent reasons, they can be left out in certain, well-defined contexts without a loss of meaning. Indeed, these elements can also co-occur without a change in interpretation. I illustrate these points in more detail below.
\end{styleStandard}

\begin{styleStandard}
3.1.1. Basic Properties of Adjectival Inflections
\end{styleStandard}

\begin{styleStandard}
Jacob Grimm (1870: 718-756) pointed out that adjectives can alternate as regards their inflections: in his (now widely-accepted) terminology, adjectives can exhibit strong or weak endings.\footnote{\ In the literature, there are other terms for these inflections: pronominal (i.e., diachronically related to third-person pronouns and pronominals) vs. nominal, determining vs. determined, and primary vs. secondary. Here, I adopt the neutral terms strong and weak. } The distribution of these endings is often referred to as the strong/weak alternation. Before turning to the distribution of these inflections, I start with some brief diachronic remarks.
\end{styleStandard}

\begin{styleStandard}
\ \ It is generally assumed (for detailed discussion, see, e.g., Evans 2019, Lockwood 1968: 37-46, Petrova 2024: 187-92, Pfaff 2020, Rehn 2019: chapters 2 and 3, and references cited therein) that Proto-Indo-European did not have a distinct category of adjectives but rather seemed to have a more general category of nominal elements (comprising adjectives and nouns). The strong inflections on adjectives are a continuation of the old nominal \textit{o}{}-/\textit{a}{}-stem inflections into Germanic, but these inflections also recruited several endings from the pronominal paradigms (e.g., demonstratives). Besides a set of strong adjective inflections, this yielded a difference between the nominal categories of noun and adjective. By contrast, weak inflections appear to be an innovation in Germanic. Specifically, a nominalizing suffix -\textit{n}{}- was added that produced \textit{n}{}-stem nominals and eventually adjectives with weak inflections (for a dissenting view, see Evans 2019: chapter 2). Finally and returning to the strong inflections, Evans (2019: 152-53) points out that regular sound changes in West Germanic led to a number of zero endings in the strong paradigm that, one the one hand, resulted in uninflected adjectives in predicative contexts but, on the other hand, were later replenished by the inflections of pronominals yielding uniformly inflected adjectives in attributive contexts in (Standard) German (but see also Rehn 2019: 135-39).
\end{styleStandard}

\begin{styleStandard}
\ \ Returning to the main focus of the discussion, Modern German, note first that noun phrases inflect for case, number, and gender.\footnote{\ For the most part, I do not consider person in this book.} While specifications for number and, to some degree, case can be inspected on suffixes to nouns, gender is usually expressed on other elements – by the inflections on determiners including articles and on adjectives. As mentioned above, adjectives can alternate between exhibiting strong or weak inflections. It is also worth pointing out already here that strong inflections have more different forms than weak ones thus expressing more distinctions in case, number and gender (Chapter 2). 
\end{styleStandard}

\begin{styleFootnote}
\ \ As for the distribution of these inflections, consider the canonical cases in (35). We can observe that the strong/weak alternation involves an adjective with a weak inflection if a preceding determiner has a strong inflection (35a), but it shows an adjective with a strong inflection if not (35b). This distribution is concisely described in the generalization in (36) (for statements along these lines, see Bierwisch 1967: 257, Eisenberg 1998: 171-73, Gallmann 1998: 144, Giusti 2015: 207, G. Müller 2002a: 129, Petrova 2024: 184-85, Pfaff 2017: 286, Rehn 2019: 58, Sauerland 1996: 34, Schoorlemmer 2009: 53, and many others):\footnote{\ Katzir \& Siloni (2014: 268) claim that this type of generalization holds for all Germanic languages. For the discussion of the Principle of Monoinflection, see Chapter 2. }
\end{styleFootnote}

\begin{styleFootnote}
(35)\ \ a.\ \ \textit{d-er \ \ \ gut-e \ \ \ \ \ \ \ Kaffee}
\end{styleFootnote}

\begin{styleFootnote}
\ \ \ \ the-\textsc{st} good-\textsc{wk} coffee.\textsc{masc}
\end{styleFootnote}

\begin{styleFootnote}
\ \ \ \ ‘the good coffee’
\end{styleFootnote}

\begin{styleFootnote}
\ \ b.\ \ \textit{ein gut-er \ \ \ Kaffee}
\end{styleFootnote}

\begin{styleFootnote}
\ \ \ \ a \ \ \ good-\textsc{st} coffee.\textsc{masc}
\end{styleFootnote}

\begin{styleFootnote}
\ \ \ \ ‘a good coffee’
\end{styleFootnote}

\begin{styleFootnote}
(36)\ \ \textit{Weak After Strong}
\end{styleFootnote}

\begin{styleFootnote}
An adjective with a weak inflection is preceded by a determiner with a strong inflection. 
\end{styleFootnote}

\begin{styleFootnote}
Notice that adjectives with strong inflections are not explicitly mentioned in the generalization. However, since adjectives in prenominal position only alternate between showing strong or weak inflections, the generalization above implies that the strong inflections occur in all instances not mentioned in (36). Given this, the generalization in (36) captures the cases in (35). For brevity, it is often referred to as Weak After Strong. Notice that when the term Weak After Strong is used, the reader should keep in mind that weak and strong refer to inflections on elements of different lexical categories (adjective vs. determiner). This is important as we will see later that two co-occuring adjectives have the same ending (and the same applies to two determiners to the extent these elements can co-occur). Note also that although it is the inflection of the adjective that alternates (i.e., the adjective \textit{gut}{}- ‘good’ in (35) above has either a weak or a strong inflection), I often refer to these cases simply as strong or weak adjectives. This is in contrast to verbs and nouns, which are lexically specified as strong or weak.
\end{styleFootnote}

\begin{styleFootnote}
\ \ Observe that the generalization in (36) is surface-oriented making reference to the notion of precedence. It works quite well for canonical cases such as (35) above. However, it fares less well with some other instances. There are two types of exceptions (see Chapter 3, Section 7 for fuller discussion). First, strong adjectives can be preceded by determiners with a strong inflection. This holds when an inflected demonstrative precedes an uninflected possessive article (37a). Second, weak adjectives can be preceded by determiners without strong inflections. This is the case with pronominal determiners (e.g., Déchaine \& Wiltschko 2002: 421-22, Höhn 2020, Pesetsky 1978, Postal 1966, Roehrs 2005), which do not have an inflection in the first and second person (37b):
\end{styleFootnote}

\begin{styleStandard}
(37)\ \ a. \ \ \ \textit{dies-es mein groß-es \ \ Glück}\ \ 
\end{styleStandard}

\begin{styleStandard}
\ \ \ \ this-\textsc{st} my \ \ \ great-\textsc{st} \ happiness.\textsc{neut}
\end{styleStandard}

\begin{styleStandard}
\ \ \ \ ‘this great happiness of mine’
\end{styleStandard}

\begin{styleFootnote}
\ \ b.\ \ \textit{ihr \ }\textit{\textsubscript{\ }}\textit{nett-en \ \ Studenten}
\end{styleFootnote}

\begin{styleFootnote}
\ \ \ \ you nice-\textsc{wk} students
\end{styleFootnote}

\begin{styleFootnote}
\ \ \ \ ‘you nice students’
\end{styleFootnote}

\begin{styleFootnote}
It seems clear that a simple notion of precedence referring to inflections cannot capture all the distributions. Rather, it is argued in Chapters 2 and 3 that structural and lexical factors need to be taken into account. Note though that other explanations have also been put forward.
\end{styleFootnote}

\begin{styleStandard}
It is sometimes suggested that this alternation involves a semantic distribution; that is, it correlates with the (in-)definiteness of the larger noun phrase. This kind of statement is usally made in the context of the Scandinavian languages (see again Section 2.1.2 above) and the older varieties of German (see Chapter 2, Section 1). However, it can, occasionally, also be found in the older literature on Modern German (e.g., Curme 1910, Lockwood 1968, Prokosch 1939; see Esau 1973 for critical discussion of these works) and even in the more recent literature (e.g., Karnowski \& Pafel 2004: 177-78, Pafel 1994: 257-64). Having said that, a consensus seems to have emerged that the strong/weak alternation is not regulated by the semantics in Modern German but rather by the morpho-syntax (see Harbert 2007: 131 and the many references cited below). As I show in detail throughout this book, this characterization is indeed correct, with the qualification that there is also a lexical component to the explanation. 
\end{styleStandard}

\begin{styleStandard}
To provide some brief illustration here, there are indeed many cases where a strong inflection occurs in an indefinite context (38a) and a weak ending in a definite environment (38b):
\end{styleStandard}

\begin{styleStandard}
(38)\ \ a.\ \ \textit{lauter dumm-e \ }\textit{\textsubscript{\ }}\textit{Idioten}
\end{styleStandard}

\begin{styleStandard}
\ \ \ \ many \textsubscript{\ }stupid-\textsc{st} idiots
\end{styleStandard}

\begin{styleStandard}
\ \ \ \ ‘many stupid idiots’
\end{styleStandard}

\begin{styleStandard}
\ \ b.\ \ \textit{wir dumm-en \ }\textit{\textsubscript{\ }}\textit{Idioten}
\end{styleStandard}

\begin{styleStandard}
\ \ \ \ we \textsubscript{\ }stupid-\textsc{wk} Idiots
\end{styleStandard}

\begin{styleStandard}
\ \ \ \ ‘us stupid idiots’
\end{styleStandard}

\begin{styleStandard}
However, a weak inflection can also occur in an indefinite context, and a strong ending is possible in a definite environment as well. Consider (39a) and (39b):
\end{styleStandard}

\begin{styleStandard}
(39)\ \ a.\ \ \textit{mancher klug-e \ \ \ \ \ \ Freund}
\end{styleStandard}

\begin{styleStandard}
\ \ \ \ some \ \ \ \ \ \textsubscript{\ }smart-\textsc{wk} friend.\textsc{masc}
\end{styleStandard}

\begin{styleStandard}
\ \ \ \ ‘many a smart friend’
\end{styleStandard}

\begin{styleStandard}
\ \ b.\ \ \textit{Peters \ klug-er \ \ Freund}
\end{styleStandard}

\begin{styleStandard}
\ \ \ \ Peter’s smart-\textsc{st} friend.\textsc{masc}
\end{styleStandard}

\begin{styleStandard}
\ \ \ \ ‘Peter’s smart friend’
\end{styleStandard}

\begin{styleStandard}
It is clear from the two paradigms in (38) and (39) that strong endings do not correlate with indefiniteness or weak endings with definiteness. Below, I demonstrate that this holds more generally. Faced with these contradictory patterns, I propose that adjectival inflections have neither semantics of their own nor do they make such features visible – they are not associated with (in-)definiteness. This is what I mean by stating that these elements are semantically vacuous.\footnote{\ In the context of Icelandic, Pfaff (2017: 293) states that the “weak inflection – rather than expressing definiteness by itself – is a semantically vacuous reflection of definiteness marked elsewhere.” Crucially, unlike in German, the weak inflection in Icelandic is dependent on the presence of definiteness in the DP. Pfaff (2017: 316) makes this dependency more formal by postulating an agreement relation between a definite D and an uninterpretable definiteness feature [uDEF) on the adjectival inflection. Adjectival inflections in German do not have such a feature – they do not reflect definiteness.} If so, the expectation is that adjectival inflections can be left out without a change in meaning.
\end{styleStandard}

\begin{styleStandard}
\ \ Adjectival inflections do not appear under certain, well-defined structural and lexical conditions in German. Structurally, adjectives must be inflected before the noun (40a-b), but they remain uninflected when they follow it (40c-d):
\end{styleStandard}

\begin{styleStandard}
(40)\ \ a.\ \ \textit{ein auf seine Frau stolz-er \ }\textit{\textsubscript{\ }}\textit{Mann}
\end{styleStandard}

\begin{styleStandard}
\ \ \ \ an \ of \ \textsubscript{\ }his \ \ \ wife \ proud\textsc{{}-st} man.\textsc{masc}
\end{styleStandard}

\begin{styleStandard}
\ \ \ \ ‘a man proud of his wife’
\end{styleStandard}

\begin{styleStandard}
\ \ b.\ \ \textit{der auf seine Frau stolz-e \ \ \ \ \ Mann}
\end{styleStandard}

\begin{styleStandard}
\ \ \ \ the \textsubscript{\ }of \ \textsubscript{\ }his \ \ \ wife \ proud-\textsc{wk} man.\textsc{masc}
\end{styleStandard}

\begin{styleStandard}
\ \ \ \ ‘the man proud of his wife’
\end{styleStandard}

\begin{styleStandard}
\ \ c.\ \ \textit{ein Mann, \ \ \ \ \ \ \ stolz \ \ auf seine Frau}
\end{styleStandard}

\begin{styleStandard}
\ \ \ \ a \ \ \ man.\textsc{masc} proud of \ \ his \ \ \ wife
\end{styleStandard}

\begin{styleStandard}
\ \ \ \ ‘a man, proud of his wife’
\end{styleStandard}

\begin{styleStandard}
\ \ d.\ \ \textit{der Mann, \ \ \ \ \ \ \ stolz \ \ auf seine Frau}
\end{styleStandard}

\begin{styleStandard}
\ \ \ \ the \textsubscript{\ }man.\textsc{masc} proud of \ \ his \ \ \ wife
\end{styleStandard}

\begin{styleStandard}
\ \ \ \ ‘the man, proud of his wife’
\end{styleStandard}

\begin{styleStandard}
Comparing (40a) to (40c) and (40b) to (40d), there is no change in the (in-)definiteness of the noun phrase, and yet adjectival inflections are present in (40a-b) but absent in (40c-d). This could be evidence that inflections make no semantic contribution. One objection might be that the adjectives in (40c-d) are not part of the noun phrase proper given that they belong to reduced relative clauses. While this may be a valid point, there is other evidence that adjectival inflections can be left out without a loss in meaning. Before I turn to these cases, note that the paradigm in (40) does indicate that structural conditions play a role in the occurrence of adjectival inflections – only prenominal adjectives have inflections.
\end{styleStandard}

\begin{styleStandard}
Besides being sensitive to structure, adjectival inflections can or must be left out in certain lexical contexts; that is, adjectival inflections are also sensitive to lexical conditions. Among others, this can be seen with specific sets of adjectives that can occur prenominally and as such, they are clearly part of the noun phrase. As just mentioned, the inflections on these adjectives can or must be absent. It will become clear though that the presence or absence of these inflections does not lead to a change in the semantics. 
\end{styleStandard}

\begin{styleStandard}
First, there is a special set of color adjectives involving \textit{lila} ‘purple’ and \textit{rosa} ‘pink’ as well as \textit{orange} ‘orange’ and \textit{beige} ‘beige’ that show that adjectival endings can be optional. Note that the first two adjectives (\textit{lila}, \textit{rosa}) require -\textit{n}{}- to be inserted (41a-b) but that the second two adjectives (\textit{orange}, \textit{beige}) do not (41c):\footnote{\ Notice that the pronunciation of \textit{orange(-s)} ‘orange’ in (41c) has some special properties. Without the inflection -\textit{s}, -\textit{ge} is pronounced as [[283?]] – the -\textit{e} is silent triggering Final Devoicing of the consonant. However, when the inflection is added yielding -\textit{ges}, this sound sequence is rendered as [[292?][259?]s]. Observe that besides (41c), \textit{das orange Kleid} ‘the orange dress’ is also possible. Here, however, the ending on \textit{orange} (i.e., -\textit{e}) is pronounced as schwa and can be interpreted as the regular weak ending. This interpretation is not possible with \textit{lila} and \textit{rosa}. The latter two end in -\textit{a}, which is not pronounced as schwa.}
\end{styleStandard}

\begin{styleStandard}
(41)\ \ a.\ \ \textit{ein lila(-n-es) \ \ Kleid} 
\end{styleStandard}

\begin{styleStandard}
\ \ \ \ a \ \ \ purple-n-\textsc{st }dress.\textsc{neut}
\end{styleStandard}

\begin{styleStandard}
\ \ \ \ ‘a purple dress’
\end{styleStandard}

\begin{styleStandard}
\ \ b.\ \ \textit{das lila(-n-e) \ \ \ \ \ Kleid}
\end{styleStandard}

\begin{styleStandard}
\ \ \ \ the purple-n-\textsc{wk }dress.\textsc{neut}
\end{styleStandard}

\begin{styleStandard}
\ \ \ \ ‘the purple dress’
\end{styleStandard}

\begin{styleStandard}
\ \ c.\ \ \textit{ein orange(-s) Kleid}
\end{styleStandard}

\begin{styleStandard}
\ \ \ \ an \ orange-\textsc{st} \ dress.\textsc{neut}
\end{styleStandard}

\begin{styleStandard}
\ \ \ \ ‘an orange dress’
\end{styleStandard}

\begin{styleStandard}
Note that the presence or absence of the inflection makes no semantic difference.
\end{styleStandard}

\begin{styleStandard}
Second, with a few types of adjectives, the inflection must be absent. This holds for adjectives derived from geographical names, so-called toponymic formations (-\textit{er} is an adjective-forming suffix):
\end{styleStandard}

\begin{styleFootnote}
(42)\ \ a.\ \ \textit{ein Berlin-er(*-es) \ \ \ Bier}
\end{styleFootnote}

\begin{styleFootnote}
a \ \ \ Berlin-\textsc{adj}(-*\textsc{st)} beer.\textsc{neut}
\end{styleFootnote}

\begin{styleStandard}
‘a Berlin beer’
\end{styleStandard}

\begin{styleFootnote}
\ \ b.\ \ \textit{das Berlin-er(*-e) \ \ \ \ \ Bier}
\end{styleFootnote}

\begin{styleFootnote}
the Berlin-\textsc{adj}(-*\textsc{wk)} beer.\textsc{neut}
\end{styleFootnote}

\begin{styleStandard}
‘the Berlin beer’
\end{styleStandard}

\begin{styleStandard}
The same applies to adjectives like \textit{prima} ‘great’, \textit{sexy} ‘sexy’, \textit{super} ‘super’, etc. (Duden 1995: 256-57).\footnote{\ A reviewer points out that the latter adjectives are loan words that may not be fully intergrated into the language yet. This also applies to the color adjectives discussed in the previous paragraph.} Note that since adjectival inflections are absent, they cannot indicate or contribute to (in-)definiteness.
\end{styleStandard}

\begin{styleStandard}
\ \ To be clear, while inflections on prenominal adjectives are usually obligatory, there are some special cases where they can or must be absent. Importantly, this presence or absence of inflection does not correlate with a change in the (in-)definite interpretation of the noun phrase as a whole. Besides the semantic property of (in-)definiteness, I demonstrate in Chapter 4 that adjectival inflections are also not a reflex of other semantic concepts such as (non-)restrictiveness of the interpretation of modifiers or referentiality. In view of that, the account of the strong/weak alternation must involve something else. Consider the following.
\end{styleStandard}

\begin{styleStandard}
The following noun phrases in the (a)-examples are in the nominative and the ones in the (b)-examples are in the dative. In keeping with the facts above, the strong and weak endings occur both in indefinite and definite contexts:
\end{styleStandard}

\begin{styleStandard}
(43)\ \ a.\ \ \textit{ein rot-es }\textit{\textsubscript{\ }}\textit{Auto}
\end{styleStandard}

\begin{styleStandard}
\ \ \ \ a \ \ \ red-\textsc{st} car.\textsc{neut}
\end{styleStandard}

\begin{styleStandard}
\ \ \ \ ‘a red car’
\end{styleStandard}

\begin{styleStandard}
\ \ b.\ \ \textit{mit}\textit{\textsubscript{ \ \ }}\textit{\ einem rot-en \ \ Auto}
\end{styleStandard}

\begin{styleStandard}
\ \ \ \ with a \ \ \ \textsubscript{\ \ \ \ \ \ \ }red-\textsc{wk} car.\textsc{neut}
\end{styleStandard}

\begin{styleStandard}
\ \ \ \ ‘with a red car’
\end{styleStandard}

\begin{styleStandard}
(44)\ \ a.\ \ \textit{sein rot-es }\textit{\textsubscript{\ }}\textit{Auto}
\end{styleStandard}

\begin{styleStandard}
\ \ \ \ his \ \textsubscript{\ }red-\textsc{st} car.\textsc{neut}
\end{styleStandard}

\begin{styleStandard}
\ \ \ \ ‘his red car’
\end{styleStandard}

\begin{styleStandard}
\ \ b.\ \ \textit{mit}\textit{\textsubscript{ }}\textit{\ \ \ seinem \ rot-en \ \ Auto}
\end{styleStandard}

\begin{styleStandard}
\ \ \ \ with his \ \ \ \ \ \ \textsubscript{\ \ \ }red-\textsc{wk} car.\textsc{neut}
\end{styleStandard}

\begin{styleStandard}
\ \ \ \ ‘with his red car’
\end{styleStandard}

\begin{styleStandard}
Again, it is unlikely that semantic concepts like (in-)definiteness can explain the strong/weak alternation. Rather, it seems more promising to find an explanation in the morpho-syntax. Considering the similarities between (43) and (44), I assume that possessive articles consist of a possessive component (\textit{s}{}-) and \textit{ein }(Chapter 5). If so, the types of adjectival endings seem to be a function of preceding \textit{ein}, independently of the presence or absence of the possessive component. In other words, (43) and (44) are directly relatable. Given that, we might claim that the strong/weak alternation is a reflex of the semantics of \textit{ein} behaving differently in the nominative vs. dative case. However, like adjectival inflections, \textit{ein} exhibits not only contradictory properties with regard to the semantics but also shows characteristics of having no meaning at all. These points can be made most straightforwardly as regards semantic number.\footnote{\ Making the following points with regard to the (in-)definiteness of \textit{ein} would take me too far afield here. I discuss the relation between adjectival inflections and \textit{ein} as regards (in-)definiteness in detail in Chapter 5. I propose there that \textit{ein} has nothing to do with (in-)definiteness, neither in the nominative nor in the dative. Thus, this semantic dimension cannot explain the distribution of adjectival inflections either.}
\end{styleStandard}

\begin{styleStandard}
3.1.2. Basic Properties of \textit{ein}
\end{styleStandard}

\begin{styleStandard}
As briefly mentioned in Section 2.2.1, the indefinite article \textit{ein} emerged on the basis of the singularity numeral and indefinite pronoun \textit{ein} in Old High German (e.g., Braune \& Reiffenstein 2004: 234, Presslich 2000, Rehn 2019: 88-90). It is often claimed that \textit{ein} in Modern German is semantically singular in meaning (e.g., Vater 2002). Specifically, (45a) makes mention of the fact that a girl was seen, and (45b) emphasizes the point that only one girl has a certain property, namely having been at a party (parentheses around \textit{ei} yielding \textit{n} indicate the reduced form of \textit{ein}):
\end{styleStandard}

\begin{styleStandard}
(45)\ \ a.\ \ \textit{Ich habe gestern \ \ (ei)n Mädchen getroffen}.
\end{styleStandard}

\begin{styleStandard}
\ \ \ \ I \ \ \ have \ yesterday a \ \ \ \ girl.\textsc{neut} met
\end{styleStandard}

\begin{styleStandard}
\ \ \ \ ‘I met a girl yesterday.’
\end{styleStandard}

\begin{styleStandard}
\ \ b.\ \ \textit{Nur} \ \textit{EIN Mädchen war auf der Party!}
\end{styleStandard}

\begin{styleStandard}
\ \ \ \ only one \textsubscript{\ }girl.\textsc{neut}\textsubscript{ \ }was at \ \ the party
\end{styleStandard}

\begin{styleStandard}
\ \ \ \ ‘Only one girl was at the party.’
\end{styleStandard}

\begin{styleStandard}
Crucially, \textit{ein} can also imply the existence of a second entity (46a). In these cases, the first clause often, but not always, occurs with the second one put in parentheses below. Independently of the second clause, \textit{ein} preceded by a definite determiner yields a duality partitive; that is, it is presupposed that there is a second girl present (NB: this is different in the English counterpart).\footnote{\ In English, \textit{one} following a definite article often has an intensifying meaning close to ‘only’:\par (i)\ \ She was the (one) girl I did not recognize.} In fact, \textit{ein} can be part of a noun phrase denoting a more numerous plurality of entities. As already discussed in Section 2.2, \textit{ein} can clearly occur in semantically plural contexts. To repeat, (46b) is a googled example from the Appendix, and (46c) is an example from the ebook \textit{Wolfskinder} ‘Wolf Children’:
\end{styleStandard}

\begin{styleStandard}
(46)\ \ a.\ \ \textit{Das ein-e \ \ \ \ Mädchen blieb \ \ \ \ \ \ \ stehen (, das andere ging \ weg)}.
\end{styleStandard}

\begin{styleStandard}
\ \ \ \ the \ one-\textsc{wk} girl.\textsc{neut} remained standing, the \textsubscript{\ }other \ \ \textsubscript{\ }went away
\end{styleStandard}

\begin{styleStandard}
\ \ \ \ ‘One of the two girls stopped walking, the other went away.’
\end{styleStandard}

\begin{styleStandard}
\ \ b.\ \ \emph{Was \ für ein-e geil-en}\textit{~ \ \ \ \ \ \ \ \ \ Bilder \ \ \ wer hat \ die \ \ \ bloß gemacht}
\end{styleStandard}

\begin{styleStandard}
what for a-\textsc{pl} \ awesome-\textsc{wk} pictures who has those just \ made
\end{styleStandard}

\begin{styleStandard}
‘Such awesome pictures! Just who took those?’
\end{styleStandard}

\begin{styleStandard}
\ \ c.\ \ \textit{Smilla recherchiert und entdeckt, \ das schon \ \ so ein-e Frauen verschwanden}.
\end{styleStandard}

\begin{styleStandard}
\ \ \ \ Smilla researches \ \ \ and discovers that already so a-\textsc{pl} women \ disappeared
\end{styleStandard}

\begin{styleStandard}
\ \ \ \ ‘Smilla did some research and discovered that such women already disappeared.’\ \ \ \ \ \ \ \ (\href{https://www.vorablesen.de/buecher/wolfskinder/rezensionen/duester-%09%09%09%09a6154566-6616-445c-967e-966d6578811c}{\textstyleInternetlink{https://www.vorablesen.de/buecher/wolfskinder/rezensionen/duester-\ \ \ \ \ \ \ \ a6154566-6616-445c-967e-966d6578811c}})
\end{styleStandard}

\begin{styleStandard}
As with adjectival endings, we are faced with a contradictory state of affairs: in this case, \textit{ein} seems to imply not only singularity but also different pluralities. Given this, we might also expect that similar to adjectival inflections, \textit{ein} can sometimes be left out without a change in meaning.
\end{styleStandard}

\begin{styleStandard}
\ \ To illustrate this point, consider nominals in predicate contexts. Note that \textit{ein} is obligatory in (47a) but that it is optional for many speakers in (47b):\footnote{\ I comment on the optionality of \textit{ein} with [-figurative] roles nouns as in (47b) further below.}
\end{styleStandard}

\begin{styleStandard}
(47)\ \ a.\ \ \textit{Er ist ein Mann}.
\end{styleStandard}

\begin{styleStandard}
\ \ \ \ he is \ a \ \ \ man.\textsc{masc}
\end{styleStandard}

\begin{styleStandard}
\ \ \ \ ‘He is a man.’
\end{styleStandard}

\begin{styleStandard}
\ \ b.\ \ \textit{Er ist (ein) Lehrer}.
\end{styleStandard}

\begin{styleStandard}
\ \ \ \ he is \ \ \textsubscript{\ }a \ \ \ \ teacher.\textsc{masc}
\end{styleStandard}

\begin{styleStandard}
\ \ \ \ ‘He is a teacher.’
\end{styleStandard}

\begin{styleStandard}
It is clear that both of these predicate nominals denote a property; that is, \textit{ein} does not make a semantic contribution with regard to singular number (or indefiniteness, for that matter). Following de Swart \textit{et al}. (2007), I argue in Chapter 6 that the presence of \textit{ein} is a function of the different types of nouns, a kind noun in (47a) vs. a role noun in (47b). Thus, similar to adjectival endings, I conclude that the presence or absence of \textit{ein} makes no difference with regard to the semantics, here illustrated with semantic number. In fact, I show below that this type of element is not a reflex of other semantic concepts such as indefiniteness or a certain type of emotiveness either. I propose that \textit{ein} neither has semantics nor does it make such features visible – it is semantically vacuous. This also means that the strong/weak alternation of adjectives briefly illustrated in Section 3.1.1 cannot be explained by the semantics of \textit{ein} either.
\end{styleStandard}

\begin{styleStandard}
\ \ If it is true that adjectival inflections and \textit{ein} are semantically vacuous, then we may wonder what their function is; that is, why they exist at all. This question is particulary interesting in view of the fact that similar to the expletives in Section 1, these elements can also co-occur with elements that seem similar or related in some way.\footnote{\ Recall that \textit{there} can occur with a locative adverbial and that the (formally definite) proprial article occurs with a proper name.} In fact, here adjectival inflections and \textit{ein} can each co-occur yielding multiple instances of the same elements; consider (48a) and (48b). I assume for now (and argue in Chapter 5) that similar to possessives like \textit{sein} ‘his’, the negative article \textit{kein} ‘no’ consists of a negative element, NEG, and \textit{ein }(where NEG is later spelled out as \textit{k}{}-; (48b) is adapted from Gallmann \& Lindauer 1994: 24):
\end{styleStandard}

\begin{styleStandard}
(48)\ \ a.\ \ \textit{frisch-er heiß-er Kaffee}
\end{styleStandard}

\begin{styleStandard}
\ \ \ \ fresh-\textsc{st} \ hot-\textsc{st} \ coffee.\textsc{masc}
\end{styleStandard}

\begin{styleStandard}
\ \ \ \ ‘fresh hot coffee’
\end{styleStandard}

\begin{styleStandard}
\ \ b.\ \ \textit{k-ein \ }\textit{\textsubscript{\ }}\textit{so’n Mann}
\end{styleStandard}

\begin{styleStandard}
\ \ \ \ \textsc{neg}{}-a so a man.\textsc{masc}
\end{styleStandard}

\begin{styleStandard}
\ \ \ \ ‘no such man’
\end{styleStandard}

\begin{styleStandard}
Importantly, given the proposed pleonastic status of the adjectival inflections and \textit{ein}, note that not only one but both of these elements seem to be redundant. Observe though that neither of the two adjectives in (48a) can occur with no ending at all. Similarly, neither \textit{ein} nor \textit{’n} in (48b) can be left out. If adjectival inflections and \textit{ein} were to make specific semantic contributions, then it would remain unclear why several instances of the same elements can, and in these cases must, co-occur. Again, this curious fact finds a natural explanation if we assume that adjectival endings and \textit{ein} are semantically vacuous and their obligatory presence in (48a) and (48b) is due to a different, independent reason or reasons. Below, I propose that adjectival inflections make abstract morpho-syntactic features visible and that \textit{ein} supports certain semantic operators.
\end{styleStandard}

\begin{styleStandard}
To recapitulate, I have illustrated some cases where adjectival inflections and \textit{ein} seem to lead to contradictory conclusions about their relation to the semantics: adjectival inflections, strong or weak, occur in both definite and indefinite contexts, and \textit{ein} appears in singular and plural environments. At the same time, we have seen that these elements seem to have no semantics at all as they are absent in certain, well-defined contexts without loss of meaning: adjectival inflections do not have to or cannot occur with special sets of adjectives, and \textit{ein} does not have to occur with a specific set of nouns in predicative contexts. Furthermore, I provided evidence that these elements can co-occur without any semantic import. I proposed that adjectival inflections and \textit{ein} are semantically vacuous. If so and returning to the more general issues, what is interesting then about adjectival endings and \textit{ein} is that they are overt elements that receive no interpretation in LF – like \textit{there} and the proprial article discussed above.
\end{styleStandard}

\begin{styleStandard}
\textit{3.2.\ \ Main Hypotheses and Relatedness of Adjectival Inflections and }ein
\end{styleStandard}

\begin{styleStandard}
In this section, I briefly summarize the general characteristics of adjectival inflections and \textit{ein}, and I formulate the main hypotheses about these elements. On the basis of that, I provide further motivation why adjectival inflections and \textit{ein} can be fruitfully discussed in tandem.
\end{styleStandard}

\begin{styleStandard}
3.2.1. Main Hypotheses
\end{styleStandard}

\begin{styleStandard}
Some of the main general properties of adjectival inflections and \textit{ein }have already been touched upon; others will be introduced later on in the book. They can be summarized as follows. Specifically, while these elements (usually) have to occur for independent reasons, adjectival inflections and \textit{ein }can each:
\end{styleStandard}

\begin{listWWviiiNumxviiileveli}
\item 
\begin{styleStandard}
co-occur
\end{styleStandard}
\item 
\begin{styleStandard}
appear in contradictory semantic contexts
\end{styleStandard}
\item 
\begin{styleStandard}
be left out in certain, well-defined contexts without loss of meaning
\end{styleStandard}
\end{listWWviiiNumxviiileveli}
\begin{styleStandard}
Furthermore, adjectival inflections are \textit{not} a reflex of:
\end{styleStandard}

\begin{listWWviiiNumxviileveli}
\item 
\begin{styleStandard}
(in-)definiteness
\end{styleStandard}
\item 
\begin{styleStandard}
(non-)restrictiveness of interpretation of modifiers
\end{styleStandard}
\item 
\begin{styleStandard}
referentiality
\end{styleStandard}
\end{listWWviiiNumxviileveli}
\begin{styleStandard}
Finally, \textit{ein} is \textit{not} a reflex of:
\end{styleStandard}

\begin{listWWviiiNumxixleveli}
\item 
\begin{styleStandard}
indefiniteness
\end{styleStandard}
\item 
\begin{styleStandard}
emotiveness
\end{styleStandard}
\item 
\begin{styleStandard}
singular number/countability
\end{styleStandard}
\end{listWWviiiNumxixleveli}
\begin{styleStandard}
Given this, I formulate the two main claims that adjectival endings and \textit{ein} have in common. While (49a) has already been mentioned above, I add (49b) now:
\end{styleStandard}

\begin{styleStandard}
(49) \ \ \textit{Hypothesis 1}
\end{styleStandard}

\begin{styleStandard}
Adjectival inflections and \textit{ein}:
\end{styleStandard}

\begin{styleStandard}
\ \ \  \ \ a. \ \ \ \ are expletive elements and
\end{styleStandard}

\begin{styleStandard}
\ \ \  \ \ b. \ \ \ \ indicate abstract structure in the noun phrase.
\end{styleStandard}

\begin{styleStandard}
Hypothesis 1b is slightly different for adjectives and \textit{ein}.\textit{ }This can be fleshed out in the two (a)-statements below. In addition, each of these two types of elements have another difference summarized in the (b)-statements:\footnote{\ Note though that the (b)-statements are also partially relatable: both adjectival inflections and \textit{ein} make elements visible, morpho-syntactic features vs. operators (the latter in the context of flagging, cf. (51b)). On a different note, I argue in Chapter 5 that there are two basic types of \textit{ein}: the so-called indefinite article characterized in Hypothesis 3 and adjectival \textit{eine}, which is not related to the article: adjectival \textit{eine} is different in that it has semantics and is argued to be in a different position. }
\end{styleStandard}

\begin{styleStandard}
(50)\ \ \textit{Hypothesis 2}
\end{styleStandard}

\begin{styleStandard}
\ \ Adjectival inflections:
\end{styleStandard}

\begin{styleStandard}
\ \ \ a.\ \ indicate abstract structure in the higher layers of the noun phrase (DP vs. LPP); \ \ \ \ \ \ they provide clues about structures involving various degrees of embedding of \ \ \ \ \ \ adjectives (simple vs. complex DPs), and
\end{styleStandard}

\begin{styleStandard}
\ \ \ b.\ \ they make nominal features like case, number, and gender visible.
\end{styleStandard}

\begin{styleStandard}
(51)\ \ \textit{Hypothesis 3}
\end{styleStandard}

\begin{styleStandard}
\ \ \textit{Ein}:
\end{styleStandard}

\begin{styleStandard}
\ \ \ a.\ \ indicates abstract structure in the lower layers of the noun phrase (NP vs. ArtP), \ \ \ \ \ \ and
\end{styleStandard}

\begin{styleStandard}
\ \ \ b.\ \ it supports overt semantic operators (e.g., NEG \textit{k}{}-) and flags the presence of \ \ \ \ \ \ covert semantic operators (e.g., REL).
\end{styleStandard}

\begin{styleStandard}
Some remarks are in order here. The structural levels LPP and ArtP stand for Left Periphery Phrase and Article Phrase. The first is above the DP-layer, and the second is below the position of adjectives (AgrP; for detailed discussion of my structural assumptions, see Section 4). Furthermore, supporting as mentioned in (51b) has been discussed in the context of the well-known phenomenon of \textit{do}{}-support in English (Chomsky 1957, Lasnik 2000). In the current discussion, supporting involves overt operators. More precicely, operators like negation (NEG) have a detectible manifestation when spelled out, and that element is supported by \textit{ein} (i.e., \textit{k-ein} ‘no’). As to flagging, \textit{ein} indicates the presence of operators like REL, which is a realization operator in cases like \textit{ein} \textit{Mann} ‘a man’ (de Swart \textit{et al}. 2007). However, here the operator itself has no detectable manifestation and remains invisible.\footnote{\ In the typological literature, the term flagging has been proposed by Haspelmath (2019) for marking nominals with case markers and/or adpositions. Unlike in the main text, here flags (i.e., case markers, adpositions) involve role-identifiers that appear with \textit{overt} nominals.} 
\end{styleStandard}

\begin{styleStandard}
Some of the claims in (49) to (51) are not entirely new; for instance, Hypothesis 1b as instantiated by adjectives in Hypothesis 2a has been established by Pfaff (2017) for Icelandic adjectives. Hypothesis 2b is related to Olsen (1991b: 40) and Rehn (2019: 122-26), who state that morphological features need to be made visible (cf. also Esau 1973, who takes the strong endings as case markers). Hypothesis 3b has, in part, been argued for by de Swart \textit{et al}. (2007). In the course of the following discussion, I discuss this and other related work. 
\end{styleStandard}

\begin{styleStandard}
Finally, I followed Chomsky in Section 1 in assuming that an expletive needs to be licensed by a substantive element (the associate) to avoid a violation of the Principle of Full Interpretation. This licensing can be seen as a semantic licensing condition and was instantiated above by moving the substantive element to the expletive. In contrast, the hypotheses above state that expletives indicate the presence of linguistic elements. This, in turn, can be interpreted as a syntactic licensing condition. 
\end{styleStandard}

\begin{styleStandard}
If this is on the right track, then there is a division of labor: syntactically, we may state that the expletive indicates the substantive element; semantically, we may say that the substantive element identifies the expletive:\footnote{\ These two conditions are reminiscent of the licensing of \textit{pro} proposed by Rizzi (1986): \textit{pro} is formally licensed by government, and the content of \textit{pro} is recovered through rich agreement specification (for ellipsis involving \textit{pro}, see also Lobeck 1995).}
\end{styleStandard}

\begin{styleStandard}
\ \ \ \  \ \ \ \ \ \ \ \ Indicate (syntax)
\end{styleStandard}

\begin{styleStandard}
[Warning: Draw object ignored][Warning: Draw object ignored](52)\ \ EXPL\ \ \ \ \ \ \ \ \ \ SUBST
\end{styleStandard}

\begin{styleStandard}
\ \ \ \  \ \ \ \ \ \ Identify (semantics)
\end{styleStandard}

\begin{styleStandard}
Considering (52), there is an interplay between the two elements in that each element licenses the other. I return to this in Chapter 8.
\end{styleStandard}

\begin{styleStandard}
3.2.2. \textit{Relatedness of Adjectival Inflections and }ein
\end{styleStandard}

\begin{styleStandard}
It is reasonable to ask why adjectival inflections and the article \textit{ein} ‘a’ should be discussed together in one volume. Adjectival inflections are bound inflectional morphemes, and the indefinite article is, at least in some cases, a free morpheme. Given their different status, it is expected that they exhibit differences. However, as seen in the previous section, they also show many similarities. To the extent that the hypotheses above prove tenable, both of these elements are expletives, they indicate structure, and they make linguistic elements visible. In addition, both elements involve functional items inside the nominal spine of the DP. What is special about them is that both adjectival inflections and \textit{ein} can each appear in multiple syntactic positions inside the DP, and they interact morphologically.
\end{styleStandard}

\begin{styleStandard}
In the absence of evidence to the contrary, I assume that the strong and weak inflections appearing on different elements are the same (e.g., Milner \& Milner 1972, also Chapter 2). Specifically, strong endings can appear on adjectives (53a). In addition, the same (strong) endings surface on predeterminers (e.g., \textit{alle} ‘all’) and on determiners (53b):
\end{styleStandard}

\begin{styleStandard}
(53)\ \ a.\ \ \textit{gut-e \ \ \ \ \ Studenten}
\end{styleStandard}

\begin{styleStandard}
\ \ \ \ good-\textsc{st} students
\end{styleStandard}

\begin{styleStandard}
\ \ \ \ ‘good students’
\end{styleStandard}

\begin{styleStandard}
\ \ b.\ \ \textit{all-e \ \ dies-e \ \ \ gut-en \ \ \ \ \ Studenten}
\end{styleStandard}

\begin{styleStandard}
\ \ \ \ all-\textsc{st} these-\textsc{st} good-\textsc{wk} students
\end{styleStandard}

\begin{styleStandard}
\ \ \ \ ‘all these good students’
\end{styleStandard}

\begin{styleStandard}
It is argued in later parts of the book that adjectives, determiners, and predeterminers are in different syntactic positions (i.e., AgrP, DP, and LPP). Indeed, weak endings can also occur on different elements. Not only do they occur on adjectives as seen in (53b) and (54a), but they can also (optionally) occur on certain determiners in the genitive masculine/neuter (for more detailed discussion, see Chapter 3, Section 4):
\end{styleStandard}

\begin{styleFootnote}
(54)\ \ a.\ \ \textit{der Verkauf heiß-en Kaffee-s}
\end{styleFootnote}

\begin{styleFootnote}
\ \ \ \ the \ sale \ \ \ \ \ hot-\textsc{wk} \ coffee.\textsc{masc}{}-\textsc{gen}
\end{styleFootnote}

\begin{styleFootnote}
\ \ \ \ ‘the sale of hot coffee’
\end{styleFootnote}

\begin{styleStandard}
\ \ b.\ \ \textit{im \ \ \ \ \ Sommer dies-en \ Jahr-es}
\end{styleStandard}

\begin{styleStandard}
\ \ \ \ in.the summer this-\textsc{wk} year.\textsc{neut}{}-\textsc{gen}
\end{styleStandard}

\begin{styleStandard}
\ \ \ \ ‘in the summer of this year’
\end{styleStandard}

\begin{styleStandard}
This means that while more restricted, weak endings may also occur on elements in different positions (i.e., AgrP and DP). Given these distributions, I refer to these endings, including the ones on determiners and predeterminers, as adjectival inflections (rather than adjective endings).
\end{styleStandard}

\begin{styleStandard}
\ \ Similar to adjectival inflections, \textit{ein} can appear in different syntactic positions. This can be shown with certain (complex) \textit{ein}{}-words. As already briefly mentioned in Section 3.1.2, the negative article \textit{kein} ‘no’ consists of \textit{ein} and the negative element NEG (55a). Similarly, I argue in Chapter 5 that the singularity numeral \textit{EIN} ‘one’ is made up of \textit{ein} and the null element \textit{Ø}\textsubscript{[-}\textsc{\textsubscript{pl}}\textsubscript{] }bringing about singularity (55b). Both \textit{kein} and \textit{EIN} can co-occur with a lower instance of \textit{ein} provided \textit{so} ‘such’ intervenes:
\end{styleStandard}

\begin{styleFootnote}
(55)\ \ a.\ \ \textit{k-ein-e \ \ \ \ so’n-e \ Frauen}\ \ \ \ 
\end{styleFootnote}

\begin{styleFootnote}
\ \ \ \ \textsc{neg}{}-a-\textsc{st} so a-\textsc{st} women
\end{styleFootnote}

\begin{styleFootnote}
\ \ \ \ ‘no such women’
\end{styleFootnote}

\begin{styleFootnote}
\ \ b.\ \ \textit{EIN-E \ \ \ \ \ \ \ so’n-e \ \ Frau}
\end{styleFootnote}

\begin{styleFootnote}
\ \ \ \ \textit{Ø}\textsubscript{[-}\textsc{\textsubscript{pl}}\textsubscript{]}+a-\textsc{st} so \textsubscript{\ }a-\textsc{st} woman.\textsc{fem}
\end{styleFootnote}

\begin{styleFootnote}
\ \ \ \ ‘one such woman’
\end{styleFootnote}

\begin{styleStandard}
Recalling the string \textit{kein so’n Mann} ‘no such man’ from above, note that all instances of \textit{ein} have the same inflection or lack thereof. I argue in later chapters that \textit{ein} along with its inflection may surface in the head positions of (at least) ArtP, CardP, and DP.
\end{styleStandard}

\begin{styleStandard}
\ \ To repeat, adjectival inflections and \textit{ein} can each appear in various syntactic positions: strong inflections occur in AgrP, DP, and LPP; weak inflections surface in AgrP and DP; and \textit{ein} occurs in ArtP, CardP, and DP. This is different from inflections on nouns, which only surface on the nouns themselves, and different from other determiners, which only surface in the DP-layer.\footnote{\ It could be claimed that the dative plural ending -\textit{n} on nouns (ia) can appear on other elements when the noun itself is elided (ib) (see Corver \& van Koppen 2010, 2011b for discussion):\par 
\setcounter{listWWviiiNumiiileveli}{0}
\begin{listWWviiiNumiiileveli}
\item 
\begin{styleFootnote}
a.\ \ \textit{mit \ \ den zwei Männer-n}
\end{styleFootnote}
\end{listWWviiiNumiiileveli}
with the \ two \ men-\textsc{dat}\par ‘with the two men’\par b.\ \ \textit{mit \ \ den zwei-n}\par with the two-\textsc{infl}\par ‘with those two’\par However, this is not possible with the genitive masculine/neuter ending -\textit{s} or other plural endings.} It is clear then that adjectival inflections and \textit{ein} share similar syntactic properties.
\end{styleStandard}

\begin{styleStandard}
\ \ Besides these similarities, adjectival inflections and \textit{ein} interact directly as regards inflectional morphology. In the typical cases, the determiner has a strong ending, and the following adjective has a weak one (56a). However, as is well known, there are certain instances where \textit{ein} has no ending, and the adjective exhibits the strong ending (56b). Importantly, when the adjective (and noun) is absent, \textit{ein} has the strong ending (56c):
\end{styleStandard}

\begin{styleStandard}
(56)\ \ a.\ \ \textit{d-er \ \ \ gut-e \ \ \ \ \ \ Student }
\end{styleStandard}

\begin{styleStandard}
\ \ \ \ the-\textsc{st} good-\textsc{wk} student.\textsc{masc}
\end{styleStandard}

\begin{styleStandard}
\ \ \ \ ‘the good student’
\end{styleStandard}

\begin{styleStandard}
\ \ b.\ \ \textit{ein gut-er \ \ \ Student}
\end{styleStandard}

\begin{styleStandard}
\ \ \ \ a \ \ \ good-\textsc{st} student.\textsc{masc}
\end{styleStandard}

\begin{styleStandard}
\ \ \ \ ‘a good student’
\end{styleStandard}

\begin{styleStandard}
\ \ c.\ \ \textit{ein-er }
\end{styleStandard}

\begin{styleStandard}
\ \ \ \ one-\textsc{st }
\end{styleStandard}

\begin{styleStandard}
\ \ \ \ ‘one’ 
\end{styleStandard}

\begin{styleStandard}
As discussed in detail in Chapter 2, there are three instances where \textit{ein} does not have an inflection in (56b) – in the remaining cases, \textit{ein} works like the definite article in (56a). These three instances have received an enormous amount of attention. Specifically, note that the presence of uninflected \textit{ein} has a special impact on the ending of the adjective (the latter can be strong as in (56b)) but that \textit{ein} itself can take the ending of the adjective (if \textit{ein} occurs by itself as in (56c)). In other words, inflections and \textit{ein} may intersect directly: \textit{ein} may regulate the distribution of the adjectival inflections, and adjectival inflections occur on \textit{ein }if the adjective (and noun) is absent. Crucially, this alternation in (56b-c) is not possible with other determiners.\footnote{\ Definite articles are different from \textit{ein} – they always have an inflection when they are followed by an adjective and/or a noun (ia). However, they may get an additional ending when they are used pronominally (ib) (again, see Corver \& van Koppen 2010, 2011b for discussion, also Roehrs 2013a: 398-402):\par 
\setcounter{listWWviiiNumvileveli}{0}
\begin{listWWviiiNumvileveli}
\item 
\begin{styleFootnote}
a.\ \ \textit{mit \ \ d-en \ \ \ gut-en \ Männer-n}
\end{styleFootnote}
\end{listWWviiiNumvileveli}
with the-\textsc{st} gut-\textsc{wk} men-\textsc{dat}\par ‘with the good men’\par b.\ \ \textit{mit \ \ d-en-en}\par \ \ with the-\textsc{st}{}-\textsc{infl}\par \ \ ‘with those’\par This additional ending -\textit{en} is not a regular adjectival inflection. Furthermore, it is restricted to dative plural contexts as in (ib) and to genitive instances more generally (e.g., \textit{d-er-en} ‘the-\textsc{st}{}-\textsc{infl}’ in the feminine and plural). Note also that \textit{manch} ‘some’, \textit{solch} ‘such’, and \textit{welch} ‘which’ are also different from \textit{ein}: besides similar distributions as in (56b-c) (e.g., \textit{solch gut-er Student} ‘such good student’; \textit{solch-er} ‘such a one’), these three elements also allow the inflection to occur on the first element like definite articles (e.g., \textit{solch-er gut-e Student} ‘such good student’). Furthermore, uninflected \textit{ein} is restricted to three instances, but the other three elements are not.} 
\end{styleStandard}

\begin{styleStandard}
To sum up, discussing adjectival inflections and \textit{ein} in one volume allows us to recognize and appreciate the above-mentioned similarities, the related differences, and the interactions of the two elements. The analysis of adjectival inflections in Chapter 2 will feature prominently in the discussion of the different types of \textit{ein} in Chapter 5. In fact, the analysis of this interaction narrows down the options of plausible analyses of certain other phenomena and raises issues for specific types of accounts of those phenomena. In addition, the following detailed discussion of both of these elements side-by-side goes beyond the canonical cases providing a more comprehensive discussion needed for both of these elements. These points make it advantageous to discuss adjectival inflections and \textit{ein} in tandem.
\end{styleStandard}

\begin{styleStandard}\bfseries
4.\ \ Basic Assumptions
\end{styleStandard}

\begin{styleStandard}
In this section, I briefly lay out my assumptions of the syntactic structure of canonical noun phrases (non-canonical constructions are discussed in later chapters in detail). Then I detail my assumptions about determiners, adjectives, and numerals/quantifiers. Finally, I summarize the main points of Distributed Morphology and Type Theory, approaches to the morphology and semantics that play an important part in the analysis to follow. 
\end{styleStandard}

\begin{styleStandard}\itshape
4.1.\ \ Basic Assumptions about Structure
\end{styleStandard}

\begin{styleStandard}
I start by discussing noun phrases as a whole. This is followed by specifying my assumptions about determiners, adjectives, and numerals/quantifiers.
\end{styleStandard}

\begin{styleStandard}
4.1.1. DPs as a Whole
\end{styleStandard}

\begin{styleStandard}
This book is written in the general generative tradition (e.g., Chomsky 1995, 2000). As mentioned above, I assume the DP-Hypothesis such that nominals in argument position are (at least) DPs but nominals in non-argument position may be of smaller size. Taking a cartographic approach, I assume that DPs like (57a) have the structure shown in (57b). Proceeding bottom-up, I assume that nouns project NPs and that morphological number is specified in NumP (Ritter 1991).\footnote{\ In this work, I take a (traditional) projectionist view. In contrast, neo-constructionist models postulate that certain properties of lexical elements are determined by higher functional ones. For instance, Marantz (1997) proposed that lexical items such as nouns and adjectives are merged as acategorial or category-neutral roots that receive their lexical category during the derivation; that is, after they have merged with a category-defining head (for similar ideas, see Borer 2005, 2013). Below, I assume that ellipsis involves a null noun. Since I am not entirely sure how a null noun (or \textit{pro} as in Olsen 1987 and related work) are compatible with acategorial/category-neutral roots, I do not adopt this type of (neo-constructionist) view here, with the qualification that the mass/count distinction of nouns is determined by the higher Num head (Chapter 7).} I claim here (and argue momentarily) that determiners originate in a lower Article Phrase (ArtP). Following Cinque (1994, 2010), I take adjectives to reside in the specifiers of a recursive AgrP. I follow Zamparelli (2000) in arguing that (weak) quantifiers are housed in Spec,CardP. Finally, I assume that determiners move from ArtP to the DP-level, with definite and indefinite articles moving to D but demonstratives moving to Spec,DP (for more detailed background discussion, see among many others Abney 1987; Alexiadou \textit{et al.} 2007; Bernstein 2001a; Julien 2005a; Longobardi 2001; Roehrs 2020b; Salzmann 2020, 2022):
\end{styleStandard}

\begin{styleFootnote}
(57)\ \ a.\ \ \textit{die zehn kleinen Autos}
\end{styleFootnote}

\begin{styleFootnote}
\ \ \ \ the ten \ \ small\textsc{ \ \ \ \ \ }cars
\end{styleFootnote}

\begin{styleFootnote}
\ \ \ \ ‘the ten small cars’
\end{styleFootnote}

\begin{styleFootnote}
\ \ b.\ \  DP
\end{styleFootnote}

\begin{styleFootnote}
[Warning: Draw object ignored][Warning: Draw object ignored]
\end{styleFootnote}

\begin{styleFootnote}
\ \ \ \  \ \ D’
\end{styleFootnote}

\begin{styleFootnote}\itshape
[Warning: Draw object ignored][Warning: Draw object ignored]
\end{styleFootnote}

\begin{styleStandard}
\ \ \ \  \ \ D\ \ \ \  CardP
\end{styleStandard}

\begin{styleStandard}
\textit{\ }[Warning: Draw object ignored][Warning: Draw object ignored]\textit{die}\textsubscript{i}
\end{styleStandard}

\begin{styleStandard}
\ \ \ \ \ \ QP\ \  \ \ \ Card’
\end{styleStandard}

\begin{styleStandard}
[Warning: Draw object ignored][Warning: Draw object ignored]\textit{\ \ \ \ \ \ \ \ \ \ \ zehn}
\end{styleStandard}

\begin{styleStandard}
\ \ \ \ \ \ \ \ Card\ \ \ \ AgrP
\end{styleStandard}

\begin{styleStandard}
\ \ \ \ \ \ \ \ \ \  \ \ [Warning: Draw object ignored][Warning: Draw object ignored]
\end{styleStandard}

\begin{styleStandard}
\ \ \ \ \ \ \ \ \ \  AP\ \ \ \ Agr’
\end{styleStandard}

\begin{styleStandard}
\textit{\ \ \ \ \ \ \ \ \ kleinen }[Warning: Draw object ignored][Warning: Draw object ignored]
\end{styleStandard}

\begin{styleStandard}
\ \ \ \ \ \ \ \ \ \ \ \ Agr\ \ \ \  ArtP
\end{styleStandard}

\begin{styleStandard}
[Warning: Draw object ignored][Warning: Draw object ignored]\ \ \ \ \ \ \ \ \ \ \ \  \ \ \  \ \ \ \ \ \ \ \ \ \ 
\end{styleStandard}

\begin{styleStandard}
\ \ \ \ \ \ \ \ \ \ \ \ \ \  \ \ \ \ \ \  \ Art’
\end{styleStandard}

\begin{styleStandard}
[Warning: Draw object ignored][Warning: Draw object ignored]\ \ \ \ \ \ \ \ \ \ \ \ \ \ \textit{ }
\end{styleStandard}

\begin{styleStandard}
\ \ \ \ \ \ \ \ \ \ \ \ \ \ \ \  Art\ \ \ \ NumP
\end{styleStandard}

\begin{styleStandard}
[Warning: Draw object ignored][Warning: Draw object ignored]\ \ \ \ \ \ \ \ \ \ \ \ \ \ \ \  \ t\textsubscript{i}
\end{styleStandard}

\begin{styleStandard}
\ \ \ \ \ \ \ \ \ \ \ \ \ \ \ \ \ \ \ \ \ \ Num’
\end{styleStandard}

\begin{styleStandard}
[Warning: Draw object ignored][Warning: Draw object ignored]
\end{styleStandard}

\begin{styleStandard}
\ \ \ \ \ \ \ \ \ \ \ \ \ \ \ \ \ \ \ \ Num\ \ \ \  \ NP
\end{styleStandard}

\begin{styleStandard}
\ \ \ \ \ \ \ \ \ \ \ \ \ \ \ \ \ \ \ \ [+\textsc{pl}]\ \ \ \ \textit{Autos}
\end{styleStandard}

\begin{styleStandard}
I assume with Julien (2005b) that nouns undergo partial N-raising to Num (not shown). Note also that with the noun at the bottom of the tree, the higher projections can be interpreted as the extended projection of the noun (Grimshaw 1991, van Riemsdijk 1998b). We see below that there are other types of extended projection; for instance, QP and AP in (57b) are more complex (and are fleshed out further below). 
\end{styleStandard}

\begin{styleStandard}
\ \ The structure in (57b) involves a simple DP – there are no other nominals embedded as specifiers, as adjuncts, or as complements. Furthermore, as mentioned above, case, number, and gender are concord features in the German noun phrase. I assume that gender comes from the noun and that (morphological) number originates in NumP. Case is determined DP-externally. Note that all elements in simple DPs like (57b) exhibit concord in agreement features.\footnote{\ It is not clear what the mechanism is that yields concord in agreement features (see, e.g., Sigurðsson 1989: 112-113 for an analysis based on feature percolation/spreading, Schoorlemmer 2009 for an Agree-based account, Norris 2014 for a proposal in terms of feature spreading and local feature copying, and Carnie 2021: 293 and Olsen 1991b: 38 for an idea involving selection; for insightful discussion of agreement in the clause, see Pesetsky \& Torrego 2007). Note in this regard that the successive-cyclic movement of the determiner, as discussed below, could do double duty and mitigate concord. I leave this interesting option open here.} There is one exception. I assume that Saxon Genitives are also part of simple DPs despite the fact that they do not agree in features with the rest of the nominal. To be clear then, I take simple DPs to involve canonical constructions. Since the low position of determiners plays an important role in the analysis to be developed, I review some previous work on it.
\end{styleStandard}

\begin{styleStandard}
\ \ Cases of Double Definiteness have received much attention in the literature on the Scandinavian languages; for instance, Norwegian \textit{den gamle mannen} ‘the old man’ involves a free-standing determiner (\textit{den}) and a suffixal determiner on the noun (-\textit{en}).\footnote{\ For cases of double articulation of (in-)definiteness in other languages and other constructions, see Plank (2003) and Alexiadou (2014).} Taraldsen (1990: 428) was one of the first to propose that the suffixal determiner originates in a D-type position below prenominal adjectives and that the free-standing determiner appears in a second D-type position above adjectives. Julien (2002, 2005a) elaborated on that providing arguments that the low determiner position is tied to the feature [\textsc{specificity}] in certain Scandinavian languages (see also Schoorlemmer 2012; cf. Ihsane \& Puskás 2001). Roehrs (2002, 2009a) proposed that both of these determiner positions are related by movement accounting for the different interpretations of adjectives as regards (non-)restrictiveness (see also Chapter 4, Section 4). In order to explain the Definiteness Cycle in the history of English, Nykiel (2015) employed determiner movement from a low position as well (see also the discussion of Old Norse in van Gelderen 2007: 293-94). Borer (2005: chapter 4) derives the mass/count distinction of nouns by assuming the absence or presence of a Classifier Phrase (cf. NumP below) and movement of determiners from this low position to higher positions in the nominal structure (see also Rehn 2019: 172-73 and Heycock \& Zamparelli 2005).\footnote{\ Unlike the current account, which merges determiners, singular or plural, in ArtP, Borer (2005: 123) has to assume that given her assumption about plural morphology, singular and plural determiners are merged in different positions (singular determiners in ClP, but plural ones in the higher \#P; cf. CardP below).} Schoorlemmer (2009) utilized the two determiner locations in his analysis of concord of the noun phrase (see Chapter 2, Section 4.2). Heck \textit{et al.} (2008) proposed that a definiteness feature moves from N to D when a prenominal adjective is present. Finally, separating the definiteness feature into different components, Roehrs (2015, 2019) provided evidence that the low position of some of these definiteness components explains adjective endings in the Scandinavian languages and certain interactions between demonstratives/possessives and suffixal determiners.\footnote{\ There are other proposals that assume that determiners originate in lower positions. In her analysis of constructions involving DP-internal degree Quantifier Raising (e.g., \textit{an idiot of a doctor}, \textit{too big (of) a house}), Matushanksy (2002: 272-73) locates the indefinite and definite articles in a head position below the DP-level (presumably Card in the current analysis). Similarly, Wood \& Vikner (2011, 2013) argue that there are two positions in which the indefinite article can occur in doubling cases of the type \textit{a such a hotel} and \textit{a so bad a hotel}: D and Num (which is Card here). Furthermore, in order to account for the referential defectiveness of French articles (Vergnaud \& Zubizaretta 1992), Déchaine \& Wiltschko (2002: 428-31) propose that the definite article in French is introduced below the DP-level. }
\end{styleStandard}

\begin{styleStandard}
\ \ Adopting a similar position here, I assume that the determiner moves from ArtP to the DP-level in a successive-cyclic fashion. Leaving out the numeral in (57a) above and adding the adjective \textit{rot} ‘red’, the derivation can be updated as follows:
\end{styleStandard}

\begin{styleStandard}
(58)\ \  DP
\end{styleStandard}

\begin{styleFootnote}
[Warning: Draw object ignored][Warning: Draw object ignored]
\end{styleFootnote}

\begin{styleFootnote}
[Warning: Draw object ignored]\textit{\ \ \ }\ \  \ D’
\end{styleFootnote}

\begin{styleFootnote}
[Warning: Draw object ignored][Warning: Draw object ignored]
\end{styleFootnote}

\begin{styleFootnote}
\ \  \ \ D\ \ \ \ AgrP
\end{styleFootnote}

\begin{styleFootnote}
[Warning: Draw object ignored][Warning: Draw object ignored]\ \  \textit{die}\textsubscript{i}
\end{styleFootnote}

\begin{styleFootnote}
[Warning: Draw object ignored]\ \  \ \ \ \ \ \ \ \ \ \ \  AP\ \ \ \  Agr’
\end{styleFootnote}

\begin{styleFootnote}
\textit{\ \ \ \ \ \ \ \ \ kleinen}[Warning: Draw object ignored][Warning: Draw object ignored]
\end{styleFootnote}

\begin{styleFootnote}
\ \ \ \ \ \ Agr\ \ \ \  AgrP
\end{styleFootnote}

\begin{styleFootnote}
[Warning: Draw object ignored][Warning: Draw object ignored]\ \ \ \ \ \ die\textsubscript{i}
\end{styleFootnote}

\begin{styleFootnote}
[Warning: Draw object ignored]\ \ \ \ \ \  \ \ \ \ \ \ \ \ \ \ \  AP\ \ \ \  Agr’
\end{styleFootnote}

\begin{styleFootnote}
\textit{\ \ \ \ \ \ \ \ \ \ \ roten}[Warning: Draw object ignored][Warning: Draw object ignored]
\end{styleFootnote}

\begin{styleFootnote}
[Warning: Draw object ignored][Warning: Draw object ignored]\ \ \ \ \ \ \ \ \ \ Agr\ \ \ \ ArtP
\end{styleFootnote}

\begin{styleFootnote}
\ \ \ \ \ \ \ \ \ \ die\textsubscript{i}
\end{styleFootnote}

\begin{styleFootnote}
Art’
\end{styleFootnote}

\begin{styleFootnote}
[Warning: Draw object ignored][Warning: Draw object ignored]
\end{styleFootnote}

\begin{styleFootnote}
\ \ \ \ \ \ \ \ \ \ \ \  \ \ \ \ \ \ \ \ \ \ \ \  Art \ \ \ \ \ \ \ \ \ \ \ \ \ \ \ \ NumP
\end{styleFootnote}

\begin{styleFootnote}
[Warning: Draw object ignored][Warning: Draw object ignored]\ \ \ \ \ \ \ \ \ \ \ \ \ \  die\textsubscript{i}
\end{styleFootnote}

\begin{styleFootnote}
\ \ \ \ \ \ \ \ \ \ \ \ \ \ \ \ \ \ \ \ Num’
\end{styleFootnote}

\begin{styleStandard}
[Warning: Draw object ignored][Warning: Draw object ignored]
\end{styleStandard}

\begin{styleStandard}
\ \ \ \ \ \ \ \ \ \ \ \ \ \ \ \  \ \ \ \ \ \ \ \ \ Num\ \ \ \  \ NP
\end{styleStandard}

\begin{styleStandard}
[Warning: Draw object ignored][Warning: Draw object ignored]\ \ \ \ \ \ \ \ \ \ \ \ \ \ \ \ \ \ \ \ \ \  \ \ t\textsubscript{k}
\end{styleStandard}

\begin{styleStandard}
\ \ \ \ \ \ \ \ \ \ \ \ \ \  \ \ \ \ \ \ \ \textit{Autos}\textsubscript{k} \ \ \ \ \ \ \ \ \ \ \ \ \ Num\textsubscript{[+}\textsc{\textsubscript{pl}}\textsubscript{]}
\end{styleStandard}

\begin{styleStandard}
I assume that definite articles move to D by head adjunction as shown in (58) but that (phrasal) demonstrative move to Spec,DP by adjunction to phrasal projections.\footnote{\ Head movement of the definite article in (58) is illustrated in a simplified way. Note that there are well-known problems with assuming that head movement is syntactic and proceeds by head adjunction. Among others, head adjunction constitutes a violation of the Extension Condition (Chomsky 1995). Solutions to these issues have been proposed; for instance, Borer (2005: 45-46) argues for head-pairs where one element of this pair is merged above the other (but crucially not by head adjunction). The higher element of that pair can move up to the next phrasal level merging above the head of this phrase forming another head-pair – there is no violation of the Extension Condition (note that there are other, related analyses: Matushansky 2006: 91-93 proposes that head movement consists of displacement of a lower head to the specifier of a higher phrase followed by m-merger of this head with the head of the higher phrase; “excorporation” is possible if m-merger does not take place; Giusti 2015: 84, 116-26 argues that articles are realizations of remerging N-heads).\par \ \ In what follows, I simply assume that heads raise by adjunction to the next higher head. The head noun raises to adjoin to Num, and both elements are spelled out as the appropriate singular or plural form of the noun. As for determiners, I assume that they also raise by adjunction to the next head. However, I assume that after adjunction, they excorporate and move to the next higher head (keeping in mind that there are other solutions, as just mentioned). Note that unlike the movement of the noun, the movement of determiners is not structure-building – it involves excorporation.}
\end{styleStandard}

\begin{styleFootnote}
\ \ Unlike case, number, and gender, definiteness is not a concord feature. Rather, I assume that definite determiners like \textit{der} ‘the’ have the feature [+DEF] as part of their structure (see Section 4.1.2) and move to the DP-level to specify the definitenss feature on D such that [uDEF] becomes [+DEF]. The article \textit{ein} ‘a’ is different. Proposing that it is a semantically vacuous element, it does not have a feature for definiteness. This means that when \textit{ein} is present, D has no definiteness feature.\footnote{\ This implies that there are two types of D in argument DPs, one with a feature for definiteness and one without it. Alternatively, we could postulate just one type of D and assume that its definiteness feature can but does not have to be checked/valued. The question of whether there is one or two types of D extends to indefinite nominals more generally. Predicate DPs are indefinite by form (e.g., \textit{Er ist ein netter Mann.} ‘He is a nice man.’), but not by meaning (they denote properties). This is different for indefinite argument DPs, which may assert the existence of an entity (e.g., \textit{Ein netter Mann war da}. ‘A nice man was there.’). Considering that \textit{ein} occurs in both types of nominals, it is clear that \textit{ein} itself does not bring about semantic indefiniteness, as claimed in the main text. Note in this regard that the current assumptions are in line with Pfaff’s (2017) analysis of the strong/weak alternation of adjectives in Icelandic, where definite DPs have a definiteness feature, but indefinite DPs, predicative and argumental, do not. Given that the question of the different types of D is a more general issue, I leave it for future research and continue assuming that \textit{ein} has no definiteness feature.} Given the lack of a definiteness feature on \textit{ein}, this element can surface in lower positions (cf. also Borer 2005: 144-59, Rehn 2019). Among others, this allows me to postulate structures smaller than DP, for instance, when nominals involving \textit{ein} occur in predicative contexts without a modifier (e.g., \textit{Er ist ein Mann}. ‘He is a man.’; see Chapter 6).
\end{styleFootnote}

\begin{styleFooter}
\ \ There is overt distributional evidence that determiners originate in low positions in German (for detailed cross-linguistic discussion, see Roehrs 2009a and references cited therein). Sternefeld (2008: 255) reports work by\textbf{ }Horst Simon, who observes that the negative article \textit{kein} ‘no’ can be repeated in dialectal German (59a).\footnote{\ This dialect is not further specified by Sternefeld. An example where the negative article appears to be in a low position similar to (59a) is also provided by Haegeman \& Lohndal (2010: 187) from West Flemish.} Sternefeld concludes that another position is needed for the second (low) instance. Above, I identified this position as Art. Furthermore, I noted above that colloquial German also tolerates two indefinite articles (59b-c):
\end{styleFooter}

\begin{styleFooter}
(59)\textit{ \ \ }a.\ \ \textit{Ich hab \ \ k-ein \ blau-es }\textit{\textsubscript{\ }}\textit{k-ein \ \ Kleid \ \ \ \ \ \ \ \ nicht}.\ \ (dialectal German)
\end{styleFooter}

\begin{styleFooter}
\ \ \ \ I \ \ \ \ have \textsc{neg}{}-a blue\textsc{{}-st} \textsc{neg}{}-a dress.\textsc{neut} not
\end{styleFooter}

\begin{styleFooter}
\ \ \ \ ‘I don’t have a blue dress.’
\end{styleFooter}

\begin{styleFootnote}
\ \ b.\ \ \textit{k-ein \ \ so’n Kleid}\ \ \ \ 
\end{styleFootnote}

\begin{styleFootnote}
\ \ \ \ \textsc{neg}{}-a so a \ dress.\textsc{neut}
\end{styleFootnote}

\begin{styleFootnote}
\ \ \ \ ‘no such dress’
\end{styleFootnote}

\begin{styleFootnote}
\ \ c.\ \ \textit{EIN \ \ \ \ \ \ so’n Kleid}
\end{styleFootnote}

\begin{styleFootnote}
\ \ \ \ \textit{Ø}\textsubscript{[-}\textsc{\textsubscript{pl}}\textsubscript{]}+a so a dress.\textsc{neut}
\end{styleFootnote}

\begin{styleFootnote}
\ \ \ \ ‘one such dress’
\end{styleFootnote}

\begin{styleFootnote}
Observe that unlike the recurring indefinite articles in Norwegian (Section 2.1.3), in German \textit{ein} can only be repeated if both instances of \textit{ein} are part of another element, for instance, the negative article \textit{kein} ‘(not a =) no’ and the type particle \textit{so’n} ‘such a/of that kind’ (see also Chapter 8). Given the works cited above and the empirical evidence just provided, I take it as established that determiners originate low in the structure.
\end{styleFootnote}

\begin{styleFootnote}
With the structure of canonical DPs in place, I should point out that the current proposal is based on morpho-syntactic subanalyses where words are separated into stems and inflections, each involving a position in the syntactic representation as explicated in greater detail in the next subsections:
\end{styleFootnote}

\begin{styleStandard}
(60)\ \ a.\ \ \textit{gut-er}
\end{styleStandard}

\begin{styleStandard}
\ \ \ \ good-\textsc{st}
\end{styleStandard}

\begin{styleStandard}
\ \ \ \ ‘good’
\end{styleStandard}

\begin{styleStandard}
\ \ b.\ \ \textit{d-er}
\end{styleStandard}

\begin{styleStandard}
\ \ \ \ the-\textsc{st}
\end{styleStandard}

\begin{styleStandard}
\ \ \ \ ‘the’
\end{styleStandard}

\begin{styleStandard}
Note that these subanalyses are part of a more general approach to morpho-syntax involving the segmentation of word and morpheme forms into smaller units (e.g., Anderson 1992; Déchaine \& Wiltschko 2002: 422; Fischer 2006; Gunkel \textit{et al}. 2017: 1297-98; Leu 2015; G. Müller 2002a: 125; Olsen 1991b: 38; Pike 1963, 1965; Rehn 2019; Roehrs 2013a; Stump 2001; B. Wiese 1999; Wiltschko 1998, Wunderlich 1997; but see also Janda \& Joseph 1992). Furthermore, similar to Distributed Morphology, which is utilized here (see Section 4.2.1), the recent approach of nanosyntax also involves analyses at the submorphemic level (see Hachem 2015 and Starke 2009). However, the latter approach explicitly rejects Impoverishment rules (Arregi \& Nevins 2012: 341), a type of operation essential to the account to be developed below. I return to some of these works in later parts of the book. With this in mind, I discuss determiners and related elements in the next subsection, and adjectives and similar items in the subsection after that.
\end{styleStandard}

\begin{styleStandard}
4.1.2. Determiners, Articles, and Determiner-like Elements
\end{styleStandard}

\begin{styleStandard}
In German, determiners are categorized into three groups: \textit{der}{}-words, \textit{ein}{}-words, and null articles (e.g., Rankin \& Wells 2016).\footnote{\ For complex determiners like \textit{ein jeder} ‘(an every =) each’, see, e.g., Karnowski \& Pafel 2004: 172-76, Pafel 1994, and below.} The first set includes \textit{der} ‘the’, (stressed) \textit{DER }‘that’, \textit{dieser }‘this’,\textit{ jener }‘that’,\textit{ jeder }‘every’,\textit{ mancher }‘some’,\textit{ solcher }‘such’,\textit{ welcher }‘which’, and\textit{ alle }‘all’ in the plural (and occasionally with mass and abstract nouns; for more details on the use of these elements, see Zifonun \textit{et al.} 1997: 1930-51). Second, as already stated above, \textit{ein}{}-words are comprised of the indefinite article \textit{ein} ‘a’ including its reduced form \textit{’n}, the (stressed) singularity numeral \textit{EIN} ‘one’, the negative article \textit{kein} ‘no’, and possessive articles like \textit{mein} ‘my’, \textit{dein} ‘you’, etc. The third category involves null articles that occur with mass and plural nouns in indefinite contexts. They are indicated here by \textit{Ø}\textit{\textsubscript{D}}.
\end{styleStandard}

\begin{styleStandard}
\ \ Articles are a specific subgroup of determiners. They comprise the definite article \textit{der} ‘the’, all \textit{ein}{}-words, and the null articles (I argue in detail in Chapter 5 that complex \textit{ein}{}-words are composites consisting of the vacuous article \textit{ein} and another, semantic component). Finally, possessives (including the possessive components of the possessive articles) and predeterminers are labeled determiner-like elements here. I assume that all determiners originate in ArtP. This is different for determiner-like elements as explained further below.
\end{styleStandard}

\begin{styleStandard}
\ \ I argue in Chapter 2 that articles are heads consisting of a categorial feature [+D]. The indefinite null articles are illustrated in (61a), which includes a definiteness feature with a negative value. Unlike the null articles, the overt articles in (61b-c) have a separate feature bundle for case, number, and gender ([CNG]), later to be spelled out as the inflection.\footnote{\ I assume that agreement features are part of syntax (see also Murphy 2018; for the advantages and disadvantages of such an assumption in the framework of Distributed Morphology, see Embick \& Noyer 2007: 305-10).} Above, I proposed that \textit{ein} is a semantically vacuous element. As such, it does not have a feature for definiteness (61b). By contrast, the definite article \textit{der} ‘the’ has a feature for definiteness. However, unlike the null indefinite articles, the value of the definiteness feature is positive (61c):\footnote{\ I finalize the discussion of Vocabulary Insertion of the determiners in Chapter 8, Sections 2.2.5 and 2.2.6. Also, note that the existence of null articles is a fairly standard assumption (e.g., Carnie 2021: 271, Schoorlemmer 2009). Among others, it allows us to claim that all argument DPs have determiners. In the context of the current analysis, their existence also helps explain the distribution of the article \textit{ein} vs. adjectival \textit{eine} (Chapter 5, Section 5.2.3). Furthermore, assuming that the null articles have the feature [-DEF] has a number of advantages: (i) it makes \textit{Ø}\textit{\textsubscript{D}} parallel to \textit{der} ‘the’ in terms of (in-)definiteness, (ii) it provides a straightforward solution to the distribution of articles with certain operators (Chapter 8, Sections 2.2.5 and 2.2.6), and (iii) it makes \textit{ein} the least specified determiner (stem).}
\end{styleStandard}

\begin{styleStandard}
(61)\ \ a.\ \ \textit{Indefinite Articles }Ø\textsubscript{D}
\end{styleStandard}

\begin{styleStandard}
\ \  \ \ \ \ \  \ \ \ \ \ \ Art
\end{styleStandard}

\begin{styleStandard}
[Warning: Draw object ignored]
\end{styleStandard}

\begin{styleStandard}
\ \  \ \ \ \ \ \ \ \ \ \ \ [+D; -DEF]\ \ \ \ 
\end{styleStandard}

\begin{styleStandard}
\ \ b.\ \ \textit{Indefinite Article }ein\ \ 
\end{styleStandard}

\begin{styleStandard}
\ \ \ \ \ \ Art
\end{styleStandard}

\begin{styleStandard}
[Warning: Draw object ignored][Warning: Draw object ignored]
\end{styleStandard}

\begin{styleStandard}
\ \ \ \ [+D]\ \ \ \ [CNG]
\end{styleStandard}

\begin{styleStandard}
\ \ c.\ \ \textit{Definite Article }der
\end{styleStandard}

\begin{styleStandard}
\ \ \ \ \ \ Art
\end{styleStandard}

\begin{styleStandard}
[Warning: Draw object ignored][Warning: Draw object ignored]
\end{styleStandard}

\begin{styleStandard}
\ \  \ \ \ [+D; +DEF]\ \ \ \ [CNG]\ \ 
\end{styleStandard}

\begin{styleStandard}
Abstracting away from the adjectival inflection, note already here that \textit{ein} is the least specified article in German – it only has the categorial feature [+D].
\end{styleStandard}

\begin{styleStandard}
Demonstrative such as \textit{dieser} ‘this’ are phrasal in structure (e.g., Bernstein 1997, 2001b; Brugè 1996, 2002; Giusti 1997, 2002; Rauth \& Speyer 2021). For simplicity’s sake, I assume that there are two terminal heads (62). Dem involves the features [+D; +DEF, +DEIX] and builds its own extended projection with an Inflectional Phrase (InflP) at the top (see Leu 2007, 2015; Roehrs 2010, 2013a).\footnote{\ The head Dem involving the features [+D; +DEF, +DEIX] could be analyzed as two heads, Art with the features [+D; +DEF] and a deictic head Deic with the feature [+DEIX] (for discussion, see the above-mentioned works). Since these finer distinctions do not play a role here, I continue utilizing the simpler analysis in the main text.} The features for case, number, and gender are in Infl. Dem moves to adjoin to Infl (not shown):
\end{styleStandard}

\begin{styleStandard}
(62)\ \ \textit{Demonstrative }dieser
\end{styleStandard}

\begin{styleJBExample}
\ \ \ \ \ \ \ \ \ \ \ \  \ \ \ \ InflP
\end{styleJBExample}

\begin{styleStandard}\bfseries
[Warning: Draw object ignored][Warning: Draw object ignored]
\end{styleStandard}

\begin{styleStandard}
\ \ \ \ \ \  \ \ [\textsc{cng}]\ \ \ \ DemP
\end{styleStandard}

\begin{styleStandard}
\ \ \ \ \ \ \ \  \ \ \ {\textbar} \ \ \ \ \ \ 
\end{styleStandard}

\begin{styleStandard}
\ \ \ \ \ \ \ \ Dem
\end{styleStandard}

\begin{styleStandard}
\ \ \ \ \ \  \ \ [+D; +DEF, +DEIX]\ \ 
\end{styleStandard}

\begin{styleStandard}
As will become clear in Chapter 3, Section 4, phrasal determiners have some special properties; for instance, they can optionally have a weak ending in genitive masculine/neuter contexts. Besides \textit{dieser} ‘this’, this includes \textit{jener} ‘that’, \textit{jeder} ‘every’, \textit{mancher} ‘some’, \textit{solcher} ‘such’, \textit{welcher} ‘which’, and \textit{alle} ‘all’. I assume that all these elements have a structure similar to (62).\footnote{\ The demonstrative \textit{DER} ‘that’ seems to have an intermediate status. On the one hand, this element is expected to project a phrasal structure (similar to the demonstrative \textit{dieser} ‘this’); on the other hand, it behaves like the definite article \textit{der} ‘the’ in not exhibiting the optionality of strong or weak inflection in the genitive masculine/neuter.}
\end{styleStandard}

\begin{styleStandard}
\ \ Possessives will not feature prominently in the following discussion. They involve complex structures that deserve more discussion than I can provide here (but see Roehrs 2013b, 2020a). However, they need to be briefly addressed as they are mentioned in various contexts below. Focusing on prenominal possessives, there are different types: possessive articles involving the possessive component \textit{s}{}- and \textit{ein} (63a), Saxon Genitives consisting of a possessor and the possessive marker -\textit{s} (63b), and \textit{von}{}-possessives comprising a possessor and the possessive preposition \textit{von} ‘of’ (63c) (see also Bhatt 1990: 223-24, Fortmann 1996, Haider 1992: 315): 
\end{styleStandard}

\begin{styleStandard}
(63)\ \ a.\ \ \textit{sein Auto}
\end{styleStandard}

\begin{styleStandard}
\ \ \ \ his \ car.\textsc{neut}
\end{styleStandard}

\begin{styleStandard}
\ \ \ \ ‘his car’
\end{styleStandard}

\begin{styleStandard}
\ \ b.\ \ \textit{Peters \ Auto}
\end{styleStandard}

\begin{styleStandard}
\ \ \ \ Peter’s car.\textsc{neut}
\end{styleStandard}

\begin{styleStandard}
\ \ \ \ ‘Peter’s car’
\end{styleStandard}

\begin{styleStandard}
\ \ c.\ \ \textit{von Peter das Auto}
\end{styleStandard}

\begin{styleStandard}
\ \ \ \ of \ \ Peter \ the car.\textsc{neut}
\end{styleStandard}

\begin{styleStandard}
\ \ \ \ ‘the car of Peter’
\end{styleStandard}

\begin{styleStandard}
Following Roehrs (2013b, 2020a), I assume that possessives involve multi-component constituents consisting of a possessive functor and a possessor (64). Possessors may involve overt or null elements (more on \textit{pro} below). The possessive functor forms the head of a Possessive Phrase (PossP) taking the possessor as its complement. A functional phrase (XP) is at the top. This yields the following underlying structure of the possessives in (63):
\end{styleStandard}

\begin{styleFootnote}
(64)\ \ \ \  XP
\end{styleFootnote}

\begin{styleFootnote}
[Warning: Draw object ignored][Warning: Draw object ignored]
\end{styleFootnote}

\begin{styleFootnote}
\ \  \ \ \  \ X’
\end{styleFootnote}

\begin{styleFootnote}\itshape
[Warning: Draw object ignored][Warning: Draw object ignored]
\end{styleFootnote}

\begin{styleStandard}
\ \  \ \ \ \ \ \ \ \ \ \ \ \  \ X\ \  \ \ \ \ \ \ \ \ \ \ PossP
\end{styleStandard}

\begin{styleStandard}\itshape
\ [Warning: Draw object ignored][Warning: Draw object ignored]
\end{styleStandard}

\begin{styleStandard}
\ \ \ \ \ \ \ \  \ \ \ Poss’
\end{styleStandard}

\begin{styleStandard}\itshape
[Warning: Draw object ignored][Warning: Draw object ignored]\ \ \ \ \ \ \ \ \ \ \ \ 
\end{styleStandard}

\begin{styleStandard}
\ \ \ \ \ \ \ \ Poss\ \ \ \  \ DP
\end{styleStandard}

\begin{styleStandard}
\ \ \ \ \ \  \ \ \ \ \ \ \ \ \ \textit{\ \ s-}\ \ \ \ \textit{pro}
\end{styleStandard}

\begin{styleStandard}\itshape
\ \ \ \ \ \ \ \ {}-s\ \ \ \ Peter
\end{styleStandard}

\begin{styleStandard}\itshape
\ \ \ \ \ \ \ \ von\ \ \ \ Peter
\end{styleStandard}

\begin{styleStandard}
Note that similar to demonstratives, possessives involve extended projections. I assume that all Poss heads, that is, the possessive component \textit{s}{}-, the possessive marker -\textit{s}, and the possessive preposition \textit{von} ‘of’, move to X in (64). If \textit{s}{}- or -\textit{s} move, this brings about a nominal extension of Poss; if \textit{von} moves, it results in a prepositional extension of Poss. Furthermore, the possessor DP undergoes movement to Spec,XP in the first two cases but stays in situ with \textit{von}{}-possessives (for more details, see Roehrs 2020a: 123). This yields the correct surface strings for Saxon Genitives (\textit{Peters}) and \textit{von}{}-possessives (\textit{von Peter}). 
\end{styleStandard}

\begin{styleStandard}
As to the possessive articles, consider briefly the periphrastic possessive construction \textit{Peter sein Auto} ‘(Peter his =) Peter’s car’ (Fiva 1985), where a possessor is added to the left. This distribution is sometimes referred to as Possessor Doubling. If we assume with Haider (1992: 315), Grohmann \& Haegeman (2003: 53), and Alexiadou \textit{et al}. (2007: 611) that \textit{Peter} can be replaced by a null argument (\textit{pro}), then the periphrastic possessive construction in (65a) can be directly related to the simple possessive article in (65b):
\end{styleStandard}

\begin{styleStandard}
(65)\ \ a.\ \ [\textsubscript{XP} \textit{Peter s}{}-]\textit{ein Auto}
\end{styleStandard}

\begin{styleStandard}
\ \ b.\ \ [\textsubscript{XP} \textit{pro s}{}-]\textit{ein Auto}
\end{styleStandard}

\begin{styleStandard}
\ \ The possessive structure XP in (64) can surface in different positions inside the larger DP. I assume that XP is base-generated in Spec,NP, where it may receive a theta role (depending on the type of head noun). Recall that the head noun moves to Num. If XP stays in situ, postnominal possessives are derived; if XP moves higher, prenominal possessives come about. Focusing on the latter again and abstracting away from the movement of the article, the possessive component \textit{s}{}- and its possessor (\textit{Peter} or \textit{pro}) move to Spec,DP (66a). The Saxon Genitive consisting of the possessive marker -\textit{s} and its possessor also moves to Spec,DP (66b).\footnote{\ Note again that possessive -\textit{s} is part of the specifier (and not in D). This is Abney’s (1987: 81-85) “preferred” analysis (see also Borer 2005: 40-41; Lowe 2016: 172, 174, and others).} Unlike those two cases, the \textit{von}{}-possessive moves to a higher specifier. Giusti \& Iovino (2016) propose that there is a Left Periphery Phrase (LPP) above the DP-layer. I assume that the \textit{von}{}-possessive moves to the specifier of that phrase.\footnote{\ Rather than LPP, we could postulate K(ase)P (see Pfaff 2017: 294 for discussion and references). However, I prefer the functionally neutral name Left Periphery Phrase here.} This derives the data in (63a-c) above as follows:
\end{styleStandard}

\begin{styleStandard}
(66)\ \ a.\ \ [\textsubscript{DP} [\textsubscript{XP} \textit{Peter}/\textit{pro} \textit{s}{}-]\textsubscript{k} \textit{ein} [\textsubscript{NumP} \textit{Auto}\textsubscript{i} [\textsubscript{NP} t\textsubscript{k} t\textsubscript{i} ]]]
\end{styleStandard}

\begin{styleStandard}
\ \ b.\ \ [\textsubscript{DP} [\textsubscript{XP} \textit{Peter}s]\textsubscript{k} \textit{Ø}\textit{\textsubscript{D}} [\textsubscript{NumP} \textit{Auto}\textsubscript{i} [\textsubscript{NP} t\textsubscript{k} t\textsubscript{i} ]]]
\end{styleStandard}

\begin{styleStandard}
\ \ c.\ \ [\textsubscript{LPP} [\textsubscript{XP} \textit{von} \textit{Peter}]\textsubscript{k} LPP [\textsubscript{DP} \textit{das} [\textsubscript{NumP} \textit{Auto}\textsubscript{i} [\textsubscript{NP} t\textsubscript{k} t\textsubscript{i} ]]]]
\end{styleStandard}

\begin{styleStandard}
Comparing (66a) to (66b), I follow Krause (1999) and Roehrs (2013b, 2020a) in that the periphrastic possessives (including the possessive articles involving \textit{pro}) and the Saxon Genitives have essentially the same structure. 
\end{styleStandard}

\begin{styleStandard}
While I cannot argue for these structures and derivations in more detail here (see references mentioned above), one immediate advantage is that all prenominal possessives in German have the same basic internal structure (XP) and that they all have the same basic derivation undergoing movement from a low position to a high position within the larger noun phrase. Indeed, each noun phrase involves a determiner in D: \textit{ein} in (66a), \textit{Ø}\textit{\textsubscript{D}} in (66b), and the definite article in (66c).\footnote{\ At this point, an issue arises. As is well known, possessors in prenominal position may spread their (in-)definiteness to the larger noun phrase (e.g., Alexiadou 2005, Roehrs 2022). If this definiteness spread is instantiated by Spec-head agreement between Spec,DP and D, then definite Saxon Genitives and null articles specified as [-DEF] are incompatible (cf. (66b)). In Footnote Error: Reference source not found, I pointed out some advantages of assuming that null articles exist and that they have a negative feature for definiteness. Given these points, I proceed with the assumption of a negative feature for definiteness, but I offer two solutions to this issue in Chapter 8, Section 2.2.6 (once further discussion is in place).}
\end{styleStandard}

\begin{styleStandard}
Finally, predeterminers are intensifiers like \textit{alle} ‘all’, \textit{diese} ‘these’, and \textit{ein} ‘a’ emphasizing exhaustiveness, deixis, and distributivity, respectively: 
\end{styleStandard}

\begin{styleStandard}
(67)\ \ a.\ \ \textit{alle meine Freunde}
\end{styleStandard}

\begin{styleStandard}
\ \ \ \ all \ \ my \ \ \ \ friends
\end{styleStandard}

\begin{styleStandard}
\ \ \ \ ‘all my friends’
\end{styleStandard}

\begin{styleStandard}
\ \ b.\ \ \textit{diese meine Freunde}
\end{styleStandard}

\begin{styleStandard}
\ \ \ \ these my \ \ \ \ friends
\end{styleStandard}

\begin{styleStandard}
\ \ \ \ ‘these friends of mine’
\end{styleStandard}

\begin{styleStandard}
\ \ c.\ \ \textit{ein jeder meiner Freunde}
\end{styleStandard}

\begin{styleStandard}
\ \ \ \ an \ every of.my \ friends
\end{styleStandard}

\begin{styleStandard}
\ \ \ \ ‘each of my friends’
\end{styleStandard}

\begin{styleStandard}
Note that predeterminers form a subset of determiners. In other words, all predeterminers also do double duty as determiners (but not vice versa). I propose below that all these elements have the categorial feature [+D]. As determiners, they are base-generated in ArtP and move to the DP-level. However, if ArtP already contains a determiner, a second determiner element can only be merged as a predeterminer provided it can supply an intensifying meaning compatible with the rest of the nominal. Given that there is only one determiner in the DP-layer, I assume that predeterminers are located in LPP illustrating this with \textit{alle }here:
\end{styleStandard}

\begin{styleStandard}
(68)\ \ [\textsubscript{LPP} \textit{alle} [\textsubscript{DP} \textit{meine Freunde}]]
\end{styleStandard}

\begin{styleStandard}
As such, I assume that LPP can be occupied by prenominal \textit{von}{}-possessives and predeterminers. To distinguish predeterminers and possessives from determiners, I often refer to the former set of elements as determiner-like elements.
\end{styleStandard}

\begin{styleStandard}
4.1.3. Adjectives, Numerals, and Quantifiers
\end{styleStandard}

\begin{styleStandard}
Elements in specifiers project their own phrases. Above, I illustrated this with demonstratives and possessives. Each of these elements project their own complex structure. In this subsection, I turn to adjectives and numerals/quantifiers. We see that AP and QP in (57b) above are also more complex.
\end{styleStandard}

\begin{styleStandard}
\ \ Starting with adjectives, I assume that they also involve extended projections (e.g., Corver 1991, 1997; Neeleman \textit{et al}. 2004; Zamparelli 2000: chapter 7). Minimally, the adjective projects an AP, and InflP is at the top (e.g., Corver 2006: 68, Leu 2015, Sapp \& Roehrs 2016). The adjective stem is located in A and the inflection in Infl. Both combine by movement of the adjective to Spec,InflP (not shown, but see Chapter 2, Section 2.4):
\end{styleStandard}

\begin{styleFootnote}
(69)\ \ \ \  InflP
\end{styleFootnote}

\begin{styleFootnote}
[Warning: Draw object ignored][Warning: Draw object ignored]
\end{styleFootnote}

\begin{styleFootnote}
\ \  \ \ \  Infl’
\end{styleFootnote}

\begin{styleFootnote}\itshape
[Warning: Draw object ignored][Warning: Draw object ignored]
\end{styleFootnote}

\begin{styleStandard}
\ \  \ \ \ \ \ \ \ \ \ \ [CNG]\ \ \ \  \ AP
\end{styleStandard}

\begin{styleStandard}\itshape
\ [Warning: Draw object ignored][Warning: Draw object ignored]
\end{styleStandard}

\begin{styleStandard}
\ \ \ \ \ \ \ \  \ \ \  \ \ A’
\end{styleStandard}

\begin{styleStandard}\itshape
[Warning: Draw object ignored][Warning: Draw object ignored]\ \ \ \ \ \ \ \ \ \ \ \ 
\end{styleStandard}

\begin{styleStandard}
\ \ \ \ \ \ \ \  \ A\ \ \ \ 
\end{styleStandard}

\begin{styleStandard}
\ \ \ \ \ \  \ \ \ \ \ \ \ \ \ \ \textit{klein}
\end{styleStandard}

\begin{styleStandard}
Finally, numerals (e.g., \textit{zwei} ‘two’) and quantifiers (e.g., \textit{viele} ‘many’) are assumed to have the same representation as (69), where Q is at the bottom projecting more structure. The internal makeup of the singularity numeral \textit{EIN} ‘one’ is discussed in detail in Chapter 5.
\end{styleStandard}

\begin{styleStandard}
These are the basic structural assumptions of the noun phrase that I take to be fairly uncontroversial. Against this backdrop, a number of other, non-canonical constructions involving adjectival inflections and/or \textit{ein} are discussed in later sections. It will become clear that not all nominal constructions can have the same structure. More importantly, as already mentioned above, I argue that it is these types of different constructions that reveal the true nature of adjectival inflections and \textit{ein}.
\end{styleStandard}

\begin{styleStandard}\itshape
4.2.\ \ Distributed Morphology and Type Theory
\end{styleStandard}

\begin{styleStandard}
Besides the general syntactic assumptions in the previous section, I make use of Distributed Morphology and semantic Type Theory to account for adjectival inflections and \textit{ein}.
\end{styleStandard}

\begin{styleStandard}
4.2.1. Distributed Morphology
\end{styleStandard}

\begin{styleStandard}
I usually provide syntactic representations of the form in (57b) and (58), where vocabulary items are directly merged in the syntactic tree, that is, during the syntactic derivation. This is a convenient shortcut. While not relevant in many cases, I assume that such syntactic representations are actually built on terminal nodes containing morpho-syntactic features (rather than lexical elements). This is a basic tenet of Distributed Morphology (DM) (see, e.g., Arregi \& Nevins 2012: 3-11; Embick \& Noyer 2007; Halle 1997; Halle \& Marantz 1993, 1994; Harley \& Noyer 1999; Sauerland 1996), a realizational approach to morphology where the derivation of complex words is separate from the spelling out (or realization) of those words. 
\end{styleStandard}

\begin{styleStandard}
In DM, terminal nodes of a syntactic tree involve abstract features. These features can be manipulated in certain ways; for instance, features can be rearranged (Lowering) or deleted (Impoverishment). After Linearization, the resultant features are spelled out by overt vocabulary items. In other words, the insertion of these items occurs late in the derivation (after syntax) and depends on the specific morpho-syntactic features present on the terminal nodes. As just stated, these features may have undergone some changes.
\end{styleStandard}

\begin{styleStandard}
Vocabulary Insertion is regulated by the Subset Principle. Provided that all features of the vocabulary item match those on the terminal node, the vocabulary item with the most specifications, that is, the most matching features, is inserted over less specified items. When a more specific vocabulary item cannot be inserted (i.e., at least one of its features does not match those on the terminal head), the vocabulary item with fewer or no features is inserted as the elsewhere case. In other words, it is assumed in DM that vocabulary items may be underspecified as regards their morpho-synactic features. This allows competition between related items in certain contexts. 
\end{styleStandard}

\begin{styleStandard}
After insertion, vocabulary items can undergo Local Dislocation combining with other vocabulary items. Unlike Lowering (which operates on hierarchical structures), Local Dislocation involves linear adjacency. Note that while not all these operations are equally important in this book, they are assumed to occur in the general order below (see also Murphy 2018: 359). The operations whose orderings are not relevant here are enclosed in curly brackets:\footnote{\ (Late) Vocabulary Insertion usually concerns functional elements but not lexical ones. The latter are merged as terminal nodes and involve phonological features (Embick \& Noyer 2007: 295).}
\end{styleStandard}

\begin{styleStandard}
(70)\ \ \{Lowering, Impoverishment\} {\textgreater}{\textgreater} Linearization {\textgreater}{\textgreater} Vocabulary Insertion {\textgreater}{\textgreater} Local \ \ Dislocation
\end{styleStandard}

\begin{styleStandard}
It is clear that Impoverishment precedes Vocabulary Insertion, and we observe in Chapter 2 that this operation seems to have access to structural information. As such, Impoverishment has been ordered with Lowering. This is compatible with Arregi \& Nevins (2012: 342), who order Impoverishment before Linearization and the latter before Vocabulary Insertion. 
\end{styleStandard}

\begin{styleStandard}
\ \ Impoverishment plays an important part in the analysis to be developed. To illustrate briefly, inflections are taken to be the overt spell-out of abstract feature bundles on terminal nodes in syntax.\footnote{\ I assume that terminal nodes may contain feature bundles (rather than just individual features). } These feature bundles can be manipulated in certain ways. Sauerland (1996) proposes for Norwegian (see Chapter 2 for German) that the weak inflections on adjectives are the result of the deletion of a feature in contexts where strong inflections occur otherwise. To see this, consider the inflectional paradigms of the strong and weak endings in Tables 2 and 3:
\end{styleStandard}

\begin{styleStandard}
Table 2: Strong Adjective Inflections in Norwegian
\end{styleStandard}

\begin{flushleft}
\begin{tabular}{|m{1.2511599in}|m{1.2511599in}|m{1.2511599in}|}

\hline
 &
{}-neuter &
+neuter\\\hline
{}-plural &
\centering {}-Ø &
\centering\arraybslash {}-t\\\hline
+plural &
\centering {}-e &
\centering\arraybslash {}-e\\\hline
\end{tabular}
\end{flushleft}
\begin{styleStandard}
Table 3: Weak Adjective Inflections in Norwegian
\end{styleStandard}

\begin{flushleft}
\begin{tabular}{|m{1.2511599in}|m{1.2511599in}|m{1.2511599in}|}

\hline
 &
{}-neuter &
+neuter\\\hline
{}-plural &
\centering {}-e &
\centering\arraybslash {}-e\\\hline
+plural &
\centering {}-e &
\centering\arraybslash {}-e\\\hline
\end{tabular}
\end{flushleft}
\begin{styleStandard}
Sauerland establishes the generalization that the weak inflections (-\textit{e}) form a proper subset of the strong inflections (-\textit{e}, -\textit{Ø}, -\textit{t}). He observes that the weak endings occur in the same featural contexts as their homophonous strong endings (i.e., in [+plural] contexts in Tables 2 and 3) and, additionally, in other environments (i.e., [-plural] in Table 3). Sauerland proposes that the weak inflections are the least marked endings from the set of strong endings. In other words, there is no difference between strong and weak inflections other than the degree of their specificity. He argues that the weak endings appear when the feature bundles of the strong inflections undergo the deletion of a feature. This feature deletion is instantiated by Impoverishment.
\end{styleStandard}

\begin{styleStandard}
\ \ Sauerland formulates the following vocabulary insertion rules:
\end{styleStandard}

\begin{styleStandard}
(71)\ \ a.\ \ [-plural, +neuter]\ \ → \ \ {}-\textit{t}
\end{styleStandard}

\begin{styleStandard}
\ \ b.\ \ [-plural, -neuter]\ \ → \ \ {}-\textit{Ø}
\end{styleStandard}

\begin{styleStandard}
\ \ c.\ \ []\ \ \ \ \ \ → \ \ {}-\textit{e}
\end{styleStandard}

\begin{styleStandard}
He proposes for Norwegian that Impoverishment deletes the gender feature in the syntactic representation. If the feature [neuter] is deleted from the relevant terminal node, then (71a) and (71b) can no longer be inserted in that terminal node. As a consequence, only (71c) can be inserted resulting in weak inflections on adjectives. 
\end{styleStandard}

\begin{styleStandard}
\ \ Note that the exact conditions as to when Impoverishment is triggered in Norwegian are not provided (but Sauerland states that those conditions are morpho-syntactic, at least in German). For some discussion of the contexts and conditions where Norwegian weak endings appear, see Julien (2005a); Katzir \& Siloni (2014); Roehrs \& Julien (2014); Schoorlemmer 2009, 2012, and many others (for the discussion of related Icelandic, see Pfaff 2017). The morpho-syntactic conditions postulated by Sauerland for German are reviewed and discussed in more detail in Chapter 2. With this brief illustration of DM in place, I return to the syntax for a moment. 
\end{styleStandard}

\begin{styleStandard}
I assume with Nunes (2001) that Move, more recently termed Internal Merge, is not a primitive operation but rather the output of Copy, Merge, Form Chain, and Copy Reduction. I assume that the first three operations apply to feature bundles which are later spelled out by the corresponding overt vocabulary items. We see later that the fourth operation, Copy Reduction, is sensitive to the overt form of vocabulary items (free vs. bound). As such, I assume that this operation applies after Vocabulary Insertion. This is compatible with Schoorlemmer (2012), who explains cases of Double Definiteness in Swedish DPs by ordering Copy Reduction after Local Dislocation.\footnote{\ Nunes (2001) argues that Copy Reduction takes place before Linearization. For discussion in favor of the ordering Copy Reduction after Linearization, see Schoorlemmer (2012: 135-37).} For convenience and expository simplicity, especially when the underlying details of the derivation are not relevant, I indicate movement by indexed traces and provide the more familiar syntactic representations given in Section 4.1.1. 
\end{styleStandard}

\begin{styleStandard}
4.2.2. Type Theory
\end{styleStandard}

\begin{styleStandard}
This book is not about the semantics of the DP \textit{per se} (or the semantics of the TP, for that matter). Nevertheless, I make some statements about the semantics involved. Rather than providing detailed denotations (see Roehrs 2009a), I use Type Theory to show that the relevant elements are semantically compatible with one another. In other words, the goal of employing Type Theory is to find plausible structures by narrowing down the analytical options. 
\end{styleStandard}

\begin{styleStandard}
To illustrate the main assumptions of Type Theory, consider the following three sentences:
\end{styleStandard}

\begin{styleStandard}
(72)\ \ a.\ \ \textit{Peter ist blond}.
\end{styleStandard}

\begin{styleStandard}
\ \ \ \ Peter is \ blond
\end{styleStandard}

\begin{styleStandard}
\ \ \ \ ‘Peter is blond.’
\end{styleStandard}

\begin{styleStandard}
\ \ b.\ \ \textit{Er ist blond}.
\end{styleStandard}

\begin{styleStandard}
\ \ \ \ he is \ blond
\end{styleStandard}

\begin{styleStandard}
\ \ \ \ ‘He is blond.’
\end{styleStandard}

\begin{styleStandard}
\ \ c.\ \ \textit{Der Mann ist blond}.
\end{styleStandard}

\begin{styleStandard}
\ \ \ \ \ the \ man \ \ \ is \ blond
\end{styleStandard}

\begin{styleStandard}
\ \ \ \ ‘The man is blond.’
\end{styleStandard}

\begin{styleStandard}
What all these sentences have in common is that somebody is blond. In fact, assuming that \textit{er} ‘he’ and \textit{der Mann} ‘the man’ are the same person as \textit{Peter}, all these sentences are true if a certain man called Peter has the property of being blond.
\end{styleStandard}

\begin{styleStandard}
\ \ I assume that the copular verb \textit{sein} ‘to be’ is semantically vacuous (Coppock \& Beaver 2015: 399, Heim \& Kratzer 1998: 61-62). A sentence has a truth value of 1 if its proposition is true in a given context or 0 if it is false. Illustrating with (72a), if a certain Peter is indeed blond in the given context, then this statement has a truth value of 1; that is, it is true. Truth values are of type {\textless}t{\textgreater} (73a). It is usually assumed that individuals like \textit{Peter} are entities. They are of type {\textless}e{\textgreater} (73b). Like \textit{Peter}, \textit{er} ‘he’ and \textit{der Mann} ‘the man’ are definite expressions and can replace \textit{Peter}. Thus, I assume that they are also of type {\textless}e{\textgreater}. Turning to (intersective) adjectives like \textit{blond} ‘blond’, they involve predicates. Put simply, predication involves membership of an element in a certain set of ordinary/individual entities. Specifically, the adjective \textit{blond} denotes a set of entities that are all blond. Predicates are of type {\textless}e,t{\textgreater} (73c) – they are functors combining with entities and return truth values. Finally, definite determiners like \textit{der} ‘the’ are functors as well presupposing uniqueness. They combine with predicates yielding entities (73d):
\end{styleStandard}

\begin{styleStandard}
(73)\ \ a.\ \ truth values (type {\textless}t{\textgreater}): \ \ \textit{Peter ist} \textit{blond}.
\end{styleStandard}

\begin{styleStandard}
\ \ b.\ \ entities (type {\textless}e{\textgreater}): \ \ \ \ \textit{Peter}, \textit{er}, \textit{der Mann}
\end{styleStandard}

\begin{styleStandard}
\ \ c.\ \ predicates (type {\textless}e,t{\textgreater}): \ \ \textit{blond}
\end{styleStandard}

\begin{styleStandard}
\ \ d.\ \ determiners (type {\textless}{\textless}e,t{\textgreater}e{\textgreater}): \ \ \textit{der}
\end{styleStandard}

\begin{styleStandard}
Before commenting on \textit{der Mann} ‘the man’ in (73b), note that Type Theory involves a combinatorial system whereby elements are related to one another, two at a time. There are two main operations. First, Functional Application combines a functor and an argument by plugging the latter into the former (74a). For instance, the predicate \textit{blond} can be a functor, and it takes an entity, say \textit{Peter}, as its argument yielding a truth value, schematically: X\textsubscript{{\textless}e,t{\textgreater}}(Y\textsubscript{{\textless}e{\textgreater}}) = Z\textsubscript{{\textless}t{\textgreater}}. This yields the sentence in (72a). Second, Predicate Modification combines two predicates; that is, it forms an intersection of two sets of certain entities (74b), schematically: X\textsubscript{{\textless}e,t{\textgreater}}, Y\textsubscript{{\textless}e,t{\textgreater}} = X\textsubscript{{\textless}e,t{\textgreater}} \& Y\textsubscript{{\textless}e,t{\textgreater}}. For example, predicates like \textit{unbekannt }\ ‘unknown’ and \textit{blond} ‘blond’ can be combined as in \textit{der unbekannte blonde Mann} ‘the unknown blond man’ such that a certain man is a member of the set of unknown entities and a member of the set of blond entities; that is, he has the properties of being both unknown and blond:
\end{styleStandard}

\begin{styleStandard}
(74)\ \ a.\ \ Functional Application:\ \ functor(argument)
\end{styleStandard}

\begin{styleStandard}
\ \ b.\ \ Predicate Modification: \ \ predicate \& predicate
\end{styleStandard}

\begin{styleStandard}
Returning to \textit{der Mann} in (73b), I assumed above that this string is of type {\textless}e{\textgreater} and that the determiner \textit{der} itself is of type {\textless}{\textless}e,t{\textgreater}e{\textgreater}. Given Functional Application, the noun \textit{Mann} is often taken to be a predicate, just like the adjective \textit{blond}. Note that here the predicate \textit{Mann} itself is the argument, and the determiner is the functor. 
\end{styleStandard}

\begin{styleStandard}
To illustrate the workings of Type Theory in a syntactic representation, consider the more complex nominal in (75a), derived as in (75b). Proceeding bottom-up and leaving out NumP and ArtP, \textit{Mann} passes up its semantic type to NP. The predicate NP and the intersective adjective \textit{blond} are both of the same type ({\textless}e,t{\textgreater}). They combine by Predicate Modification resulting in a complex element of the same type. Finally, the determiner \textit{der} is a functor that takes the complex predicate in AgrP as its argument returning an entity ({\textless}e{\textgreater}):
\end{styleStandard}

\begin{styleStandard}
(75)\ \ a.\ \ \textit{der blonde Mann}
\end{styleStandard}

\begin{styleStandard}
\ \ \ \ the blond \ \ man.\textsc{masc}
\end{styleStandard}

\begin{styleStandard}
\ \ \ \ ‘the blond man’
\end{styleStandard}

\begin{styleFootnote}
\ \ b.\ \  DP\textsubscript{{\textless}e{\textgreater}}
\end{styleFootnote}

\begin{styleFootnote}
[Warning: Draw object ignored][Warning: Draw object ignored]\ \ \ \ \ \ \ \  \ \ \ (Functional Application)
\end{styleFootnote}

\begin{styleFootnote}
\textit{der}\textsubscript{{\textless}{\textless}e,t{\textgreater}e{\textgreater}} \ \ \ \ \ \ \ \ AgrP\textsubscript{{\textless}e,t{\textgreater}}
\end{styleFootnote}

\begin{styleFootnote}
[Warning: Draw object ignored][Warning: Draw object ignored]\textit{\ \ \ \ \ \ \ \  \ \ }(Predicate Modification)
\end{styleFootnote}

\begin{styleStandard}
\ \  \ \ \ \ \ \ \ \ \ \ \textit{blonde}\textsubscript{{\textless}e,t{\textgreater}}\ \  \ NP\textsubscript{{\textless}e,t{\textgreater}}
\end{styleStandard}

\begin{styleStandard}
\textit{\ \ \  \ \ \ \ \ \ \ \ \ \ Mann}\textsubscript{{\textless}e,t{\textgreater}}
\end{styleStandard}

\begin{styleStandard}
These are the basic assumptions (as discussed in Heim \& Kratzer 1998), and they are sufficient to account for most cases. In Chapter 6 and 7, I refine my assumptions about nouns and extend the discussion to pronominal determiners like \textit{du} ‘you’.
\end{styleStandard}

\begin{styleStandard}
\ \ To sum up, this section gave an overview of my basic assumptions as regards the syntax, morphology, and semantics. More details are provided as they become relevant.
\end{styleStandard}

\begin{styleStandard}\bfseries
5.\ \ Overview of the Chapters
\end{styleStandard}

\begin{styleStandard}
This book pursues two goals. First, it seeks to provide an overview of certain synchronic data in the nominal domain in German. In particular, Chapters 2-7 discuss in detail adjectival inflections, \textit{ein} along with its related words, and consequences of the proposed analyses. The second goal is to find commonalities between these different phenomena; that is, to tie these analyses together to offer some remarks about the larger issues involved. While touched on throughout the book, Chapter 8, the final chapter, focuses on these more general points. On the basis of the similarities and differences of adjectival inflections and \textit{ein}, I engage there in a summary discussion of the main hypotheses stated in (49) through (51), point out further consequences, and indicate avenues for future research. 
\end{styleStandard}

\begin{styleStandard}
It is important to point out that both of these goals go in partially different directions. Specifically, an overview aims to be fairly exhaustive, but discussing the larger issues involved is an attempt to see what different sub-domains, empirical or theoretic, have in common. As such, discussing issues to a comprehensive degree tends to move the focus away from traits shared by all the different domains. However, as already briefly illustrated above, the discussion of one phenomenon often forms the background for the analysis of another. It is on the basis of the shared properties and the interwoven argumentation that I pointed out above that adjectival inflections and \textit{ein} can and should be discussed in tandem. 
\end{styleStandard}

\begin{styleStandard}
The book is organized into two primary chapters (2, 5), four supporting chapters (3, 4, 6, 7), and a conclusion (8). Here is a brief overview:
\end{styleStandard}

\begin{styleStandard}
\ \ \textbf{Chapter 2} (\textit{The Structural Nature of Adjectival Inflections}) This chapter investigates adjectival inflections in a wide range of nominal constructions. It is observed that weak inflections only occur in canonical constructions. These involve simple DPs of the form “determiner + adjective(s) + noun”, where all these elements involve concord in agreement features. It is proposed that weak inflections are underlyingly fully specified feature bundles that undergo Impoverishment (Sauerland 1996). This feature deletion is triggered by determiners (Impoverishment Rule 1) or by a certain featural context (Impoverishment Rule 2). In contrast, feature bundles surface as strong endings if Impoverishment does not occur, either in canonical or non-canonical structures. As such, the strong inflections present the elsewhere case. The final section discusses three previous proposals in detail pointing out that none of these accounts can account for all the data.
\end{styleStandard}

\begin{styleFootnote}
\textbf{Chapter 3} (\textit{Variation and} \textit{Secondary Mechanisms}) This chapter focuses on variation. Given the analysis in Chapter 2, a number of unexpected inflections are discussed. It is pointed out that these cases are very restricted in that they occur only in specific, well-defined contexts. It is argued that unexpected strong adjectives follow from the assumption that pronominal determiners do not trigger Impoverishment. In contrast, unexpected weak adjectives are accounted for by a phonetic rule, which is a reflex of markedness reduction. Unexpected weak inflections on determiners follow from the extension of Impoverishment Rule 2. Overall, it is proposed that adjectival inflections are due to several mechanisms. Finally, inflectional variation on certain sets of adjectives and predeterminers is addressed, issues with the traditional generalizations Weak After Strong and Principle of Monoinflection are discussed, and the strong/weak alternation of the dialect of Mannheim is analyzed.
\end{styleFootnote}

\begin{styleStandard}
\ \ \textbf{Chapter 4} (\textit{Consequences for Other Analyses}) This chapter discusses some other consequences of the first part of the book. Specifically, it is shown that weak adjectives in the context of plural \textit{ein} raise issues for analyses involving Predicate Inversion (Bennis \textit{et al}. 1998) and for accounts postulating the presence of certain null nouns (van Riemsdijk 2005). Furthermore, it is argued that the present analysis is not compatible with split topicalization constructions if simply analyzed as movement (van Riemskijk 1989) but only if analyzed as involving the separate base-generation of two nominals where one or both of these nominals undergo movement later (Fanselow 1988). Finally, it is suggested that non-restrictive adjectives must have the same basic structure as restrictive ones and that strong inflections are not “referential” in nature but serve to make nominal features like case, number, and gender visible.
\end{styleStandard}

\begin{styleStandard}
\ \ \textbf{Chapter 5} (Ein\textit{{}-words and Adjectival }eine) This chapter investigates the different kinds of \textit{ein}. It is argued that there are basically two main types: semantically vacuous \textit{ein} and adjectival \textit{eine}. The former can occur by itself as the indefinite article but also as part of the singularity numeral \textit{EIN} ‘one’, possessive articles like \textit{mein} ‘my’, etc., and the negative article \textit{kein} ‘no’. These four types of elements are labeled \textit{ein}{}-words. It is proposed that \textit{EIN}, \textit{mein}, and \textit{kein} are composite forms consisting of vacuous \textit{ein} and another component. This derives the morpho-syntactic similarities of these elements. The distinctions follow from the second component, which has different feature specifications from \textit{ein} and occupies different positions in the syntactic representation. The second main type involves adjectival \textit{eine}. This element is a lexically different element: it is an adjective that only appears in definite contexts and induces a duality presupposition. This adjectival element is proposed to be located in a high AgrP.
\end{styleStandard}

\begin{styleStandard}
\ \ \textbf{Chapter 6} (Ein\textit{ and Emotiveness}) This chapter presents the first consequence of the second major component of this book. Contrasting pronominal DPs and copular TPs, it is shown that singular contexts exhibit the most restrictions. Three different interpretations are distinguished. First, the ordinary reading is emotive in pronominal DPs but both neutral and emotive in copular TPs where TPs show \textit{ein} in both neutral and emotive contexts. Second, the comparative reading is emotive in both DPs and TPs, where the latter involve \textit{ein}. Finally, the capacity reading is neutral in DPs involving \textit{als} ‘as’ and emotive in TPs involving \textit{ein}. It is proposed that \textit{ein} is not responsible for the emotiveness in the various interpretations. Rather, following Rauh (2004) and de Swart \textit{et al}. (2007), I propose that the emotive readings are due to certain pragmatic restrictions and the realization operator REL. The presence of the latter is flagged by \textit{ein }accounting for the presence of \textit{ein} in these contexts. More generally, this and the next chapter discuss the relatedness of noun phrases and clauses.
\end{styleStandard}

\begin{styleStandard}
\textbf{Chapter 7} (Ein\textit{ and Number}) This chapter presents a second consequence of Chapter 5. Comparing again the nominal and clausal domains, I discuss four different constructions: pronominal DPs, copular TPs, pronominals followed by \textit{als}{}-nominals (\textit{als} ‘as’), and non-copular TPs involving \textit{als}{}-nominals. I show that pronominal DPs are very restricted with regard to morphological and semantic number; that is, singular DPs are singular in interpretation, and plural DPs are plural in interpretation. This is different for the other three constructions. While nouns preceded by \textit{ein} can only be singular in interpretation and nouns in the plural can only be plural in interpretation, bare nouns can be both singular and plural in interpretation and combine with singular or plural pronominals. I propose that pronominal DPs always project NumP but that predicate nominals in the other constructions may lack NumP, provided NumP is the highest phrase of the predicate nominal. It is suggested that number derives from an interaction between the noun and the Number head and that \textit{ein} flags the presence of an operator.
\end{styleStandard}

\begin{styleStandard}
\ \ \textbf{Chapter 8} (\textit{Concluding Remarks}) This chapter ties the previous parts together by discussing in more detail the hypotheses proposed for adjectival inflections and \textit{ein}. Reviewing important empirical and theoretical points, it strengthens the conclusions that both adjectival inflections and \textit{ein} are semantically vacuous elements and that both elements provide clues about the structure of the noun phrase. In addition, this chapter tentatively extends the discussion of adjectival inflections to another dialectal variant – Alemannic German, and it considers \textit{ein} in a number of other semantic contexts. Suggestions are made as to how these extensions can be accommodated in the current system. Finally, this chapter addresses some further consequenes as regards concord in agreement features in non-canonical noun phrases, and it considers the question of what all semantically vacuous elements may have in common.
\end{styleStandard}

\clearpage\setcounter{page}{53}\begin{styleStandard}
Chapter 2: The Structural Nature of Adjectival Inflections
\end{styleStandard}

\begin{styleStandard}\bfseries
1.\ \ Introduction
\end{styleStandard}

\begin{styleStandard}
This chapter provides detailed evidence in support of Hypothesis 1, namely that adjectival inflections are semantically vacuous and indicate abstract structure. As regards the latter, the discussion supports Hypothesis 2a: adjectival inflections indicate structure in the higher layers of the noun phrase, and they provide clues about various degrees of embeddings of adjectives.
\end{styleStandard}

\begin{styleStandard}\itshape
1.1.\ \ The Strong/Weak Alternation of Adjectives
\end{styleStandard}

\begin{styleStandard}
As briefly discussed in Chapter 1, adjectives can take strong or weak endings. Compare the following typical inflectional alternation on adjectives where the ending is strong when no determiner is present, and it is weak when there is a determiner. Below, I show these patterns in the nominative case using a singular masculine mass noun and a plural count noun (for a complete inventory of all the combinations of case, number, and gender, see Section 2.1.2):
\end{styleStandard}

\begin{styleStandard}
(1)\textit{ \ \ }a.\ \ \textit{frisch-er Kaffee}
\end{styleStandard}

\begin{styleStandard}
\ \ \ \ fresh-\textsc{st} \ coffee.\textsc{masc}
\end{styleStandard}

\begin{styleStandard}
‘fresh coffee’
\end{styleStandard}

\begin{styleStandard}
b.\ \ \textit{d-er \ \ \ \ frisch-e \ \ Kaffee}
\end{styleStandard}

\begin{styleStandard}
\ \ \ \ the-\textsc{st} fresh-\textsc{wk} coffee.\textsc{masc}
\end{styleStandard}

\begin{styleStandard}
‘the fresh coffee’
\end{styleStandard}

\begin{styleStandard}
(2)\textit{ \ \ }a.\ \ \textit{nett-e \ \ Frauen}
\end{styleStandard}

\begin{styleStandard}
\ \ \ \ \ nice-\textsc{st} women
\end{styleStandard}

\begin{styleStandard}
\ \ \ \ ‘nice women’
\end{styleStandard}

\begin{styleStandard}
\ \ b.\ \ \textit{di-e \ \ \ \ nett-en \ \ Frauen}
\end{styleStandard}

\begin{styleStandard}
\ \ \ \ the-\textsc{st} nice-\textsc{wk} women
\end{styleStandard}

\begin{styleStandard}
\ \ \ \ ‘the nice women’
\end{styleStandard}

\begin{styleStandard}
These are the basic patterns. Recall that the phenomenon involving the two related forms of the adjective is labeled the strong/weak alternation. 
\end{styleStandard}

\begin{styleStandard}
These inflectional patterns have received a good amount of attention in the literature (e.g., Murphy 2018, Olsen 1991b, Schoorlemmer 2009; for a discussion of these works, see Section 5).\footnote{\ To anticipate the discussion in Section 5, previous accounts of adjectival inflections in German have focused on the canonical cases, that is, the alternation of the adjective in simple noun phrases of the schematic type “determiner + adjective(s) + noun”, where the three elements agree in case, number, and gender. As argued in that section, it is not clear how these proposals can explain some of the non-canonical instances. In my view, considerations of elegance cannot be the sole or decisive measure to evaluate a proposal – the empirical coverage must be taken into account to evaluate the plausibility of an analysis.} As pointed out in Chapter 1, it could be claimed that the strong and the weak endings correlate with an indefinite or definite interpretation of the noun phrase. In the (a)-examples above, the nominals with the strong adjectives are indefinite, but in the (b)-examples, the nominals with the weak adjectives are definite. This is often referred to as the semantic distribution of adjectival inflections. At first glance then, these different inflections seem to be a reflex of the varying semantics of the noun phrase. 
\end{styleStandard}

\begin{styleStandard}
Demske (2001), Lohrmann (2010: 60-62), and Rehn (2019) show that this is indeed the case in the older Germanic languages including older varieties of German.\footnote{\ Focusing on the development from Proto-Indo-European to Old Norse, Pfaff (2020) notes that weak adjectives occur indeed in definite contexts but that strong adjectives have nothing to do with indefiniteness. Evans (2019: 119-47) makes an even stronger claim for the older Germanic languages such that weak adjectives only occur in the context of determiners (for the lack of a clear semantic distribution in OHG, see Petrova 2024). } Interestingly, the older literature on Modern German (e.g., Curme 1910, Lockwood 1968, Prokosch 1939) also claims that adjective inflections are regulated by the semantics. Indeed, this claim can occasionally be encountered in the more recent literature as well (e.g., Karnowski \& Pafel 2004: 177-78, Pafel 1994: 257-64).\footnote{\ To explain certain exceptions, these authors claim that \textit{ein} ‘a’, \textit{kein} ‘no’, and possessives involve idiosyncracies.} To briefly illustrate one such proposal, Abraham (2013: 259) seems to claim that different adjectival inflections mark anaphoric/thematic vs. non-anaphoric/rhematic relations, especially in dialectal German. It should be pointed out though that not many details are provided.
\end{styleStandard}

\begin{styleStandard}
Harbert (2007: 130-137) points out that the strong/weak alternation is different in some of the contemporary languages (also Evans 2019, Gallmann 1996: 302-03, Roehrs \& Julien 2014, Schoorlemmer 2009: 53 fn. 55). In fact, he claims that diachronically, the two sets of endings developed from a semi-regular distribution toward regularization and functionalization (Harbert 2007: 131). This development occurred in different ways: in the Scandinavian languages, the alternation signals (in-)definiteness (e.g., Julien 2005a, Lohrmann 2011); in German, it developed into an economy principle such that the strong ending is limited to one per noun phrase (e.g., Esau 1973). Looking at the literature, it is probably fair to state that a consensus seems to have emerged that the strong/weak alternation in Modern German is regulated by the morpho-syntax (and not the semantics).
\end{styleStandard}

\begin{styleStandard}
While I basically concur with Harbert and others, I argue that the German facts are more complicated than often presented (I briefly discuss earlier proposals and the related traditional generalizations in Section 2.1.1). Before I turn to a detailed discussion of the strong/weak alternation in German, I provide more evidence that adjectival inflections in German do not correlate with (in-)definiteness. This lack of correlation can be shown on the basis of diachronic and synchronic considerations.
\end{styleStandard}

\begin{styleStandard}\itshape
1.2.\ \ Adjectival Inflections and (In-)definiteness
\end{styleStandard}

\begin{styleStandard}
Prenominal adjectives in the Germanic languages vary as regards a correlation between the different types of inflections and (in-)definiteness. As briefly discussed in Chapter 1, Section 2.1.2, German, Yiddish, and Norwegian exhibit a number of distinctions in this regard. Specifically, adjectival inflections in German clearly lack a correlation regarding (in-)definiteness, and this point can be strengthened by a side-by-side diachronic comparison of Modern German and Old High German (OHG). I provide two types of examples (the OHG instances are taken from Demske 2001: 67). First, adjectives behave differently in vocatives: in Modern German, adjectives are strong; in OHG, adjectives are weak:\footnote{\ As pointed out by Petrova (2024: 203-04, 211-12), vocatives could also involve strong adjectives in OHG, and the same goes for possessive pronominals (see below). Note that this variation is not possible in Modern German.}
\end{styleStandard}

\begin{styleFootnote}
(3)\ \ a.\textit{ \ \ Dumm-er Idiot!\ \ \ \ \ \ }(Modern German)
\end{styleFootnote}

\begin{styleFootnote}
\ \ \ \ stupid-\textsc{st} \ idiot.\textsc{masc}
\end{styleFootnote}

\begin{styleStandard}
\ \ ‘Stupid idiot!’
\end{styleStandard}

\begin{styleFootnote}
\ \ b.\ \ \textit{líob-o \ \ \ \ man}\ \ \ \ \ \ \ \ (OHG)
\end{styleFootnote}

\begin{styleFootnote}
\ \ \ \ dear-\textsc{wk} man.\textsc{masc}
\end{styleFootnote}

\begin{styleFootnote}
\ \ \ \ ‘dear man’
\end{styleFootnote}

\begin{styleStandard}
Second, possessive elements also occur with adjectives showing different inflections. As above, Modern German exhibits strong adjectives but OHG weak ones:
\end{styleStandard}

\begin{styleFootnote}
(4)\ \ a.\ \ \textit{mein lieb-er \ Sohn\ \ \ \ \ \ }(Modern German)
\end{styleFootnote}

\begin{styleFootnote}
\ \ \ \ my \ \ \textsubscript{\ }dear-\textsc{st} son.\textsc{masc}
\end{styleFootnote}

\begin{styleFootnote}
\ \ \ \ ‘my dear son’
\end{styleFootnote}

\begin{styleFootnote}
\ \ b.\ \ \textit{mîn liob-o \ \ \ \ sun}\ \ \ \ \ \ (OHG)
\end{styleFootnote}

\begin{styleFootnote}
\ \ \ \ my \textsubscript{\ }dear-\textsc{wk} son.\textsc{masc}
\end{styleFootnote}

\begin{styleFootnote}
\ \ \ \ ‘my dear son’
\end{styleFootnote}

\begin{styleFootnote}
I know of no evidence that indicates that the definiteness involved in these vocatives and possessives is different in the two language varieties. As such, Modern German and OHG show different endings on adjectives in the same definite contexts. Recalling the cross-linguistic discussion from Chapter 1, Section 2.1.2, I conclude that adjectival endings do not uniformly indicate the definiteness of noun phrases across the different Germanic languages, synchronically (German vs. Norwegian) and diachronically (Modern German vs. OHG).\footnote{\ There is no agreement in the literature on when adjectival inflections changed from a semantic to a morpho-syntactic distribution. Demske (2001) claims that this change occurred during ENHG. In contrast, Rehn (2019: 79-88) argues that this happened between OHG and MHG (for discussion, see also Evans 2019: 147-57, Klein 2007, Petrova 2024). Observing an increase in the occurrence of the indefinite article \textit{ein}, Rehn (2019: 88-90) proposes that the marking of indefiniteness by the article triggered the change in the way adjectival inflections are regulated in German. } This lack of correlation in Modern German between inflection and interpretation can be made even more forcefully when we consider certain synchronic data of that language.
\end{styleFootnote}

\begin{styleStandard}
\ \ I also briefly showed in Chapter 1, Section 3.1.1 that noun phrases involving indefinite articles or possessive articles take either a strong adjective or a weak adjective in (Modern) German. This depends on the morphological case the noun phrase appears in. Compare the nominals in the nominative in the (a)-examples to those in the dative in the (b)-examples, where the former have strong adjectives but the latter weak ones:
\end{styleStandard}

\begin{styleFootnote}
(5)\textit{ \ \ }a.\ \ \textit{ein} \textit{\ groß-es Auto}
\end{styleFootnote}

\begin{styleFootnote}
\ \ \ \ a \ \ \ \ big\textsc{{}-st} \ \ \ car.\textsc{neut}
\end{styleFootnote}

\begin{styleFootnote}
\ \ \ \ ‘a big car’
\end{styleFootnote}

\begin{styleFootnote}
b.\ \ \textit{mit} \textsubscript{\ \ \ }\textit{ein-em} \textit{\ groß-en Auto}
\end{styleFootnote}

\begin{styleFootnote}
\ \ \ \ with a-\textsc{st} \ \ \ \ \ big-\textsc{wk} \ car.\textsc{neut}
\end{styleFootnote}

\begin{styleFootnote}
\ \ \ \ ‘with a big car’
\end{styleFootnote}

\begin{styleFootnote}
(6)\textit{ \ \ }a.\ \ \textit{mein groß-es Auto}
\end{styleFootnote}

\begin{styleFootnote}
\ \ \ \ my \ \ \ big-\textsc{st} \ \ car.\textsc{neut}
\end{styleFootnote}

\begin{styleFootnote}
\ \ \ \ ‘my big car’
\end{styleFootnote}

\begin{styleFootnote}
b.\ \ \textit{mit} \textsubscript{\ \ \ }\textit{mein-em groß-en Auto}
\end{styleFootnote}

\begin{styleFootnote}
\ \ \ \ with my-\textsc{st} \ \ \ \ big-\textsc{wk} \ car.\textsc{neut}
\end{styleFootnote}

\begin{styleFootnote}
\ \ \ \ ‘with my big car’
\end{styleFootnote}

\begin{styleStandard}
As shown in (5) and (6), adjectives pattern the same after indefinite articles and possessive articles in German; that is, adjectives can take a strong or weak ending in both indefinite and definite contexts. I argue in detail in Chapter 5 that possessive articles in Modern German consist of a possessive component and vacuous \textit{ein} (e.g., \textit{m-ein} ‘my’). This composite analysis of possessive articles directly relates the latter to \textit{ein}. If so, it is \textit{ein} that seems to be responsible for the regulation of the inflectional endings in (5) and (6). In fact, the distinguishing factor appears to be case. However, it is undisputed that the different morphological cases in (5) and (6) have no impact on the (in-)definiteness of those instances. Indeed, Saxon Genitives also involve strong and weak adjectives, similar to (6) above, but in different morphological cases (for detailed discussion, see Section 2.3). 
\end{styleStandard}

\begin{styleStandard}
\ \ There is more evidence that morphological case alone cannot explain the strong/weak alternation in this language. This can be exemplified with the same lexical item where the morphological case remains unchanged. For instance, the quantifying element \textit{manch}{}- ‘some’ is indefinite. If it is inflectionless, the adjective is strong (7a); if it has a strong ending, the adjective is weak (7b). Similarly, the pronominal determiner \textit{wir} ‘we’ is definite but can occur with both types of endings (8a-b). Note that \% below marks a less preferred but certainly possible form (see Bhatt 1990: 154-55; Darski 1979: 198; Duden 1995: 280, 2007: 39):\footnote{\ The strong ending in (8b) becomes more acceptable if the overt noun is missing (Bhatt 1990: 179, Duden 2007: 39). Presumably, a strong ending is better able to license a null noun.}
\end{styleStandard}

\begin{styleStandard}
(7)\ \ a.\ \ \textit{manch nett-e \ \ Studenten}
\end{styleStandard}

\begin{styleStandard}
\ \ \ \ some \ \ nice-\textsc{st} students
\end{styleStandard}

\begin{styleStandard}
\ \ \ \ ‘some nice students’
\end{styleStandard}

\begin{styleStandard}
\ \ b.\ \ \textit{manch-e nett-en \ \ Studenten}
\end{styleStandard}

\begin{styleStandard}
\ \ \ \ some-\textsc{st} nice-\textsc{wk} students
\end{styleStandard}

\begin{styleStandard}
\ \ \ \ ‘some nice students’
\end{styleStandard}

\begin{styleStandard}
(8)\ \ a.\ \ \textit{wir nett-en \ \ \ Studenten}
\end{styleStandard}

\begin{styleStandard}
we \ nice-\textsc{wk} \textsubscript{\ }students
\end{styleStandard}

\begin{styleStandard}
‘we nice students’
\end{styleStandard}

\begin{styleStandard}
\ \ b. \ \%\ \ \textit{wir nett-e \ \ \ Studenten}
\end{styleStandard}

\begin{styleStandard}
we \ nice-\textsc{st} students
\end{styleStandard}

\begin{styleStandard}
‘we nice students’
\end{styleStandard}

\begin{styleStandard}
As will become clear below, these are not isolated cases: they also apply to \textit{solch}{}- ‘such’, \textit{welch}{}- ‘which’, \textit{mir} ‘me(\textsc{dat})’, \textit{dir} ‘you(\textsc{dat.sgl})’, and \textit{ihr} ‘you(\textsc{nom.pl})’. 
\end{styleStandard}

\begin{styleStandard}
To summarize, it is, in my view, beyond doubt that at least in (Modern) German, different adjectival endings do not correlate with differences in definiteness. Again, the majority of the recent literature on German concludes the same. However, despite this emerged consensus, the fact that the strong/weak alternation is regulated by the semantics in the related Scandinavian languages and given the occuasional claims that this is also the case in German, I will continue arguing in favor of a morpho-syntactic distribution making this point in different ways throughout this book. This fits well with my overall hypothesis that adjectival inflections are semantically vacuous: they neither have semantics of their own nor do they make semantic features visible.
\end{styleStandard}

\begin{styleStandard}
Returning briefly to the diachronic data, it was pointed out by Demske (2001: 82) that adjectives in OHG are weak after the possessive element but strong after the indefinite article. As illustrated above, this is different in Modern German, where adjectives pattern the same after the possessive element and the indefinite article. If so, this implies that the possessive element and the indefinite article were not related to each other in the older varieties of German as compared to Modern German. It seems clear that German underwent a number of changes in this regard during its diachronic development (for the discussion of some diachronic issues, see Chapter 5; also Alexiadou 2004, Demske 2001, Evans 2019, Rehn 2019, and Wood 2007).
\end{styleStandard}

\begin{styleStandard}\itshape
1.3.\ \ Outlook
\end{styleStandard}

\begin{styleStandard}
The following discussion focuses on (Modern) German with a few cross-linguistic remarks if the phenomenon under discussion shows an interesting parallel or contrast between German and another language. Restricting the discussion to prenominal adjectives, I reach the conclusion that, at least in German, adjectival inflections are semantically vacuous (Hypothesis 1a) and that the strong/weak alternation is not simply a surface phenomenon. Rather, it is, in part, a reflex of different structures of the noun phrase (Hypothesis 2a). In fact, this alternation is argued to provide a good diagnostic of the structure, but not the semantics, of the nominal domain. 
\end{styleStandard}

\begin{styleStandard}
To anticipate the discussion, I propose that the derivation of the weak endings on adjectives must take a certain structure into account. I argue that adjectives are only weak if they occur in canonical DPs and undergo a certain reduction process. This process is argued to be feature deletion (Impoverishment), and it is triggered by certain determiners (Impoverishment Rule 1) or in a specific feature context (Impoverishment Rule 2). The diverse distribution of the strong adjectives in other canonical DPs and non-canonical DPs more generally is explained by the fact that Impoverishment does not occur, either because there is no determiner, no relevant determiner, or because the determiner or the adjective is in a position that is different from that in canonical DPs. In a sense, the strong endings form the elsewhere case. 
\end{styleStandard}

\begin{styleStandard}
If this is on the right track, then a number of other cases involving canonical DPs appear to be exceptional. I argue in Chapter 3 that the inflectional patterns of adjectives (and determiners) in German are due to several mechanisms. It is important to point out though that the other mechanisms only occur in specific, well-defined contexts; that is, unlike Impoverishment Rule 1, they are restricted to certain combinations of case and gender, or case and number. Chapter 4 discusses some consequences of the current proposal for other analyses. I utilize the system developed for adjectival inflections as a diagnostic of the plausibility of the structures put forth in those works.
\end{styleStandard}

\begin{styleStandard}
\ \ This chapter is organized as follows. In Section 2, I provide the basic proposal deriving the infections on adjectives in the canonical cases. I propose that the operation of Impoverishment explains these cases. Section 3 turns to the discussion of adjectives in non-canonical DPs, and it is proposed that these cases involve different structures where Impoverishment does not apply. In Section 4, I address the inflections on determiners and determiner-like elements arguing that Impoverishment does not apply to these elements either. Section 5 discusses three previous proposals of the strong/weak alternation, and Section 6 concludes the chapter.
\end{styleStandard}

\begin{styleStandard}\bfseries
2.\ \ Adjectival Inflections in Canonical DPs
\end{styleStandard}

\begin{styleStandard}
The basic proposal is as follows. Adopting the framework of Distributed Morphology (Chapter 1, Section 4.2.1), I assume that adjectival inflections are the result of the spell-out of the features for case, number, and gender on the terminal heads in the syntactic representation. I propose that the weak endings follow from the operation of Impoverishment. Arguing that this mechanism only occurs in canonical DPs, it is taken to reduce the fully specified feature bundles in the tree representation. These reduced feature bundles are then spelled out as the weak endings. Given the diverse distribution of the strong endings, I propose that the occurrence of the latter instances is not subject to any specific conditions. Rather, the strong endings surface in the absence of Impoverishment as the fully specified (underlying) feature bundles. In the current account, the strong inflections involve the elsewhere case. Below, I develop this proposal in more detail.
\end{styleStandard}

\begin{styleStandard}
\ \ This section is organized as follows. First I discuss adjectives in the context of \textit{der}{}-words. This is followed by noun phrases involving \textit{ein}{}-words, null articles, and Saxon Genitives. At the end of this section, I turn to adjectives occurring in more complex canonical DPs, canonical DPs that contain extended-adjective constructions and numerals.
\end{styleStandard}

\begin{styleStandard}\itshape
2.1.\ \ Weak Adjectives: Impoverishment
\end{styleStandard}

\begin{styleStandard}
I start by repeating the basic alternation of adjectival inflections involving \textit{der}{}-words, briefly reviewing some traditional generalizations that have been put forth to describe these patterns. This is followed by providing an inventory of the adjectival inflections and discussing a previous proposal. After these preliminaries, I lay the foundation of my own analysis of weak adjectives and spell out some of the details of my account.
\end{styleStandard}

\begin{styleStandard}
2.1.1. Basic Alternation and Traditional Generalizations
\end{styleStandard}

\begin{styleStandard}
Recall the basic alternation as regards adjectival inflections: if a determiner is not present, then adjectives surface with strong endings (9a). If a determiner is present, then the determiner has a strong ending, and the adjectives appear with weak inflections (9b):
\end{styleStandard}

\begin{styleFootnote}
(9)\textit{ }\ \ a.\textit{\ \ frisch-e*(r) \ \ \ schwarz-e*(r) Kaffee}
\end{styleFootnote}

\begin{styleFootnote}
\ \ \ \ fresh-\textsc{st}/*\textsc{wk} black-\textsc{st}/*\textsc{wk} coffee.\textsc{masc}
\end{styleFootnote}

\begin{styleFootnote}
\ \ \ \ ‘fresh black coffee’
\end{styleFootnote}

\begin{styleFootnote}
b.\ \ \textit{d-er \ \ \ frisch-e(*r)}\textit{\textsubscript{ \ }}\textit{schwarz-e(*r) Kaffee}
\end{styleFootnote}

\begin{styleFootnote}
\ \ \ \ the-\textsc{st} hot-\textsc{wk}/*\textsc{st} black-\textsc{wk}/*\textsc{st} \ coffee.\textsc{masc}
\end{styleFootnote}

\begin{styleStandard}
\ \ \ \ ‘the fresh black coffee’
\end{styleStandard}

\begin{styleStandard}
Note again that the definite article in (9b) has the same ending as the adjectives in (9a). In fact, following Milner \& Milner (1972), Leu (2015) and many others have pointed out that the endings on determiners (D) are the same as the strong endings on adjectives (for detailed discussion, see the next subsection). Assuming a null article for (9a), I summarize this determiner-adjective interaction in the following schematic patterns ({\textgreater} stands for precede; * means multiple occurrences):
\end{styleStandard}

\begin{styleStandard}
(10)\textit{ \ \ }a.\ \ \textit{Ø}\textit{\textsubscript{D}} {\textgreater} STRONG\textsubscript{ADJ*}
\end{styleStandard}

\begin{styleStandard}
\ \ b.\ \ STRONG\textsubscript{D} {\textgreater} WEAK\textsubscript{ADJ*}
\end{styleStandard}

\begin{styleFootnote}
Observe that this abstract distribution is instantiated by the canonical cases, simple noun phrases of the schematic type “determiner + adjective(s) + noun”, where the three elements agree in case, number, and gender.
\end{styleFootnote}

\begin{styleFootnote}
Different generalizations have been offered to describe the distribution in (10). They basically fall into two types. One type of generalization focuses on the inflectional behavior of the adjectives. I provide one such statement in (11), which is often conveniently referred to as Weak After Strong (for similar statements, see Bierwisch 1967: 257, Eisenberg 1998: 171-73, Gallmann 1998: 144, Giusti 2015: 207, G. Müller 2002a: 129, Petrova 2024: 184-85, Pfaff 2017: 286, Rehn 2019: 58, Sauerland 1996: 34, Schoorlemmer 2009: 53, and many others):
\end{styleFootnote}

\begin{styleFootnote}
(11)\ \ \textit{Weak After Strong}
\end{styleFootnote}

\begin{styleFootnote}
An adjective with a weak inflection is preceded by a determiner with a strong inflection. 
\end{styleFootnote}

\begin{styleFootnote}
Note that this particular formulation does not explicitly mention adjectives that are not preceded by (inflected) determiners. However, since prenominal adjectives are only subject to a two-way alternation, the distribution of strong adjectives can be inferred from (11).
\end{styleFootnote}

\begin{styleFootnote}
The other type of generalization is broader in that it focuses on inflections more generally, that is, on the inflections of both the adjective and the preceding determiner. It highlights the facts that the strong inflection occurs only once in the noun phrase and that it appears in first position. Roehrs (2009a: 135) provides the following formulation of the so-called Principle of Monoinflection (see also, e.g., Darski 1979, Esau 1973: 140, Helbig \& Buscha 2001: 274-76, Murphy 2018: 344, Wegener 1995: 105, 153, cf. Nübling 2011: 178):\footnote{\ There is a variant of this generalization that includes the inflection of the head noun. Evans (2019: 58) provides the following statement of the so-called “Once-per-DP” Principle: In the maximal projection of a noun (i.e., in a DP), the strong ending occurs on at most one lexical type (determiner, adjective, or noun). Evans (2019: chapter 3) proposes that the order of strong endings occurring before weak endings is derived from c-command. However, this leads to some issues. One problem is the relevance of the head noun in the calculation of the adjectival inflections (see Roehrs 2021).}
\end{styleFootnote}

\begin{styleStandard}
(12)\textit{ \ \ Principle of Monoinflection}
\end{styleStandard}

\begin{styleFootnote}
The first inflected element within a noun phrase carries the strong and the second one the \ \ weak ending.
\end{styleFootnote}

\begin{styleFootnote}
These are the two main types of traditional generalizations for adjectival inflections in German. Unlike the semantic distribution, they describe a morpho-syntactic pattern. Both types seek to capture the fact that some combinations of the strong and weak endings do not occur; for instance, both exclude patterns where a weak determiner in -\textit{e }or\textit{ -en} is followed by a strong adjective in -\textit{er},\textit{ -es, }or\textit{ -em} (e.g., *\textit{d-en gut-em Noun} ‘the-\textsc{wk} good-\textsc{st} Noun’). 
\end{styleFootnote}

\begin{styleFootnote}
While the more general statement in (12) should, in my view, be preferred, both generalizations face two types of exceptions. As briefly discussed in Chapter 1, Section 3.1.1, these issues have to do (i) with weak elements occurring without the presence of strong elements, and (ii) with strong elements surfacing despite the presence of other strong elements. To my mind, these issues are the result of stating the generalizations in terms of surface precedence and the inflections involved (rather than in terms of the structure and the lexical and featural triggers involved, as made more precise below). The relevant problematic cases are addressed in detail in the course of the following discussion and are summarized in Chapter 3, Section 7. Having said that, the generalizations above are good approximations of the canonical facts in German and provide a useful heuristic of the investigation. 
\end{styleFootnote}

\begin{styleFootnote}
The generalization that I will work towards is the following:
\end{styleFootnote}

\begin{styleStandard}
(13)\ \ \textit{Strong/Weak Alternation in German}\ \ 
\end{styleStandard}

\begin{styleStandard}
Disregarding a few (mostly) lexical exceptions, both adjectives and determiners have strong or weak inflections. Weak inflections only occur in canonical DPs, either 
\end{styleStandard}


\setcounter{listWWviiiNumviiileveli}{0}
\begin{listWWviiiNumviiileveli}
\item 
\begin{styleStandard}
in the context of certain lexical items (e.g., definite articles), or
\end{styleStandard}
\item 
\begin{styleStandard}
in certain featural contexts (e.g., genitive masculine/neuter), or
\end{styleStandard}
\item 
\begin{styleStandard}
in a combination of (a) and (b) (e.g., null articles and stacked adjectives in dative masculine/neuter).
\end{styleStandard}
\end{listWWviiiNumviiileveli}
\begin{styleStandard}
Strong inflections occur in all other environments.
\end{styleStandard}

\begin{styleFootnote}
While this generalization is, admittedly, more complex than its predecessors, it will be shown that it is more in line with the empirical facts.
\end{styleFootnote}

\begin{styleFootnote}
\ \ Returning to the data in (9), it is important to point out that the grammaticality judgments of these instances are very sharp. I propose that the morphological mechanism of Impoverishment explains these data. I closely follow the discussion of Roehrs \& Julien (2014), who base their proposal on Sauerland (1996). Before I turn to my analysis, I provide an inventory of the relevant elements, and I briefly review Sauerland (1996).
\end{styleFootnote}

\begin{styleStandard}
2.1.2. Inventory of Determiners and Adjectival Inflections
\end{styleStandard}

\begin{styleStandard}
In this subsection, I discuss the inventory of determiners and adjectival inflections. This allows me to formulate specific vocabulary insertion rules. Starting with determiners, they can be categorized into three groups: \textit{der}{}-words, \textit{ein}{}-words, and null articles:
\end{styleStandard}

\begin{styleStandard}
(14)\ \ a. \ \ \textit{der}{}-words: 
\end{styleStandard}

\begin{styleStandard}
\textit{der} ‘the’, (stressed) \textit{DER }‘that’, \textit{dieser }‘this’,\textit{ jener }‘that’,\textit{ jeder }‘every’,\textit{ mancher }‘some’,\textit{ solcher }‘such’,\textit{ welcher }‘which’, and\textit{ alle }‘all’\footnote{\ These are the most frequent \textit{der}{}-words (usually provided in reference works).}
\end{styleStandard}

\begin{styleStandard}
\ \ b.\ \ \textit{ein}{}-words:
\end{styleStandard}

\begin{styleStandard}
\textit{ein} ‘a’, (stressed) \textit{EIN} ‘one’, \textit{kein} ‘no’, and possessive articles like \textit{mein} ‘my’, \ \ \textit{dein} ‘you’, etc.
\end{styleStandard}

\begin{styleStandard}
\ \ c. \ \ null articles: 
\end{styleStandard}

\begin{styleStandard}
\textit{Ø}\textit{\textsubscript{D}}
\end{styleStandard}

\begin{styleStandard}
The stem forms of the definite article \textit{der} ‘the’ may vary between \textit{de}{}-, \textit{da}{}-, and \textit{di}{}- depending on case and gender. Furthermore, the form \textit{die} is pronounced [di] as the article or [di:] as the distal demonstrative; that is, this form has no inflection. I dedicate a separate section to these and related points in Chapter 3, Section 3. Given that, I illustrate the \textit{der}{}-words with the proximal demonstrative \textit{dieser} ‘this’ setting off its endings by a hyphen (I discuss \textit{ein}{}-words and null articles in Sections 2.2 and 2.3). The arrangement of case, number, and gender features in Table 1 follows Bierwisch (1967). This alignment allows me to state syncretisms based on natural classes (more on this below). I also follow Bierwisch (1967: 250) and others in assuming that the schwa before a consonantal inflection is due to schwa-epenthesis (R. Wiese 1996a: 109-10, 243). This is not indicated separately below:
\end{styleStandard}

\begin{styleStandard}
Table 1: Determiner Inflections
\end{styleStandard}

\begin{flushleft}
\begin{tabular}{|m{1.2511599in}|m{1.2511599in}|m{1.2511599in}|m{1.2511599in}|m{1.2518599in}|}

\hline
 &
\centering Masculine &
\centering Neuter &
\centering Feminine &
\centering\arraybslash Plural\\\hline
Nominative &
\centering dies-er &
\centering dies-es &
\centering dies-e &
\centering\arraybslash dies-e\\\hline
Accusative &
\centering dies-en &
\centering dies-es &
\centering dies-e &
\centering\arraybslash dies-e\\\hline
Dative &
\centering dies-em &
\centering dies-em &
\centering dies-er &
\centering\arraybslash dies-en\\\hline
Genitive &
\centering dies-es &
\centering dies-es &
\centering dies-er &
\centering\arraybslash dies-er\\\hline
\end{tabular}
\end{flushleft}
\begin{styleStandard}
As already noted above, the endings on the determiners are the same as the strong endings on the adjectives. This is exemplified with the adjective \textit{gut}{}- ‘good’ in Table 2. Comparing Tables 1 and 2, there are only two instances (in the genitive masculine/neuter) where the endings on the determiners are different from the endings on the adjectives (-\textit{es} vs. -\textit{en}):
\end{styleStandard}

\begin{styleStandard}
Table 2: Strong Adjective Inflections
\end{styleStandard}

\begin{flushleft}
\begin{tabular}{|m{1.2511599in}|m{1.2511599in}|m{1.2511599in}|m{1.2511599in}|m{1.2518599in}|}

\hline
 &
\centering Masculine &
\centering Neuter &
\centering Feminine &
\centering\arraybslash Plural\\\hline
Nominative &
\centering gut-er &
\centering gut-es &
\centering gut-e &
\centering\arraybslash gut-e\\\hline
Accusative &
\centering gut-en &
\centering gut-es &
\centering gut-e &
\centering\arraybslash gut-e\\\hline
Dative &
\centering gut-em &
\centering gut-em &
\centering gut-er &
\centering\arraybslash gut-en\\\hline
Genitive &
\centering gut-en &
\centering gut-en &
\centering gut-er &
\centering\arraybslash gut-er\\\hline
\end{tabular}
\end{flushleft}
\begin{styleStandard}
Studying Tables 1 and 2 further, we also notice that in many instances, masculine and neuter pattern together (e.g., in the dative and genitive) and feminine and plural do too (e.g., in the nominative, accusative, and genitive). The weak inflections are provided in Table 3:
\end{styleStandard}

\begin{styleStandard}
Table 3: Weak Adjective Inflections
\end{styleStandard}

\begin{flushleft}
\begin{tabular}{|m{1.2511599in}|m{1.2511599in}|m{1.2511599in}|m{1.2511599in}|m{1.2518599in}|}

\hline
 &
\centering Masculine &
\centering Neuter &
\centering Feminine &
\centering\arraybslash Plural\\\hline
Nominative &
\centering gut-e &
\centering gut-e &
\centering gut-e &
\centering\arraybslash gut-en\\\hline
Accusative &
\centering gut-en &
\centering gut-e &
\centering gut-e &
\centering\arraybslash gut-en\\\hline
Dative &
\centering gut-en &
\centering gut-en &
\centering gut-en &
\centering\arraybslash gut-en\\\hline
Genitive &
\centering gut-en &
\centering gut-en &
\centering gut-en &
\centering\arraybslash gut-en\\\hline
\end{tabular}
\end{flushleft}
\begin{styleStandard}
In view of Table 2 and Table 3, German is traditionally taken to have an inventory of five strong and two weak endings (Duden 1995). The latter (-\textit{e}, -\textit{en}) form a proper subset of the former (-\textit{e}, -\textit{en}, -\textit{er}, -\textit{es}, -\textit{em}). Considering all three tables, we notice that the nominative is the same as the accusative within each gender and plural, with one well-known exception (e.g., Blevins 1995: 145-46, Eisenberg 1998: 173-74): the masculine accusative has a different form (-\textit{en}) from the masculine nominative (-\textit{er}, -\textit{e}), something I return to below. Given the overall similarity of the inflections on determiners and adjectives, I treat them the same and refer to them as adjectival endings. 
\end{styleStandard}

\begin{styleFootnote}
To reiterate, strong endings have more different forms than weak ones. In fact, comparing the tables above, they express more distinctions in case, number and gender. One of the main insights of Sauerland (1996) is that strong endings involve more features than weak ones. As such, there is no real difference between the traditional strong and weak endings (although I retain this distinction in name for expository purposes). The vocabulary insertion rules formulated in Section 2.1.5 reflect this. Indeed, taking the number of features as the relevant metrics, we will see that the (less specified) weak endings form a proper subset of the (more specified) strong endings.\footnote{\ Rather than an account based on feature specificity as developed below, G. Müller (2002a) provides an optimality-theoretic proposal (also Gallmann 2004). In order to formulate simpler constraints, Müller analyzes noun phrases as NPs although DPs (e.g., \textit{dieser} ‘this’) exist in the account. Furthermore, and also unlike the current account, Müller’s proposal is that inflections have no independent status as morphemes in the lexicon.} Sauerland (1996) proposes to derive the weak inflections from the strong endings by a deletion process. Consider his proposal in more detail.
\end{styleFootnote}

\begin{styleFootnote}
2.1.3. Sauerland (1996)
\end{styleFootnote}

\begin{styleFootnote}
Adopting the general framework of Distributed Morphology (see Chapter 1, Section 4.2.1), Sauerland (1996) proposes to derive the basic alternation illustrated in Section 2.1.1 by a feature deletion process called Impoverishment (focusing here on German only, I make some minor adjustments to the presentation of his account). He assumes that inflections involve fully specified, abstract feature bundles underlyingly. The feature bundles make up the terminal nodes of the syntactic representation. If these feature bundles are reduced by Impoverishment, they are spelled out as weak endings; if not, they surface as strong endings. Impoverishment is triggered morpho-syntactically: the inflectional feature bundle of the adjective undergoes feature deletion if the preceding determiner has a strong inflection. Sauerland combines Impoverishment and Feature Hierarchy in his account. 
\end{styleFootnote}

\begin{styleFootnote}
\ \ In more detail, the weak endings occur, unsurprisingly, in all featural contexts. Sauerland observes though that like in Norwegian, in German they surface in the same featural contexts as their homophonous strong endings; for instance, -\textit{e} is not only the corresponding weak inflection of strong -\textit{er} in the nominative masculine but also of strong -\textit{e} in the nominative feminine (see Tables 2 and 3 above). Similar considerations hold for the other weak inflection; for instance, -\textit{en} is not only the corresponding weak inflection of strong -\textit{em} in the dative masculine but also of strong -\textit{en} in the accusative masculine. In other words, the two weak endings -\textit{e} and -\textit{en} also occur as strong endings. Now, since they occur in more contexts (weak and strong), -\textit{e} and -\textit{en} are taken to be the least marked strong endings. This means that there are actually no weak endings \textit{per se} – only (strong) inflections that differ in their specificity.
\end{styleFootnote}

\begin{styleFootnote}
\ \ As regards -\textit{e} and -\textit{en} themselves, they are also clearly distinguished as regards their indiviudal specificity. Considering the relevant feature combinations, strong -\textit{e} can appear as weak -\textit{e} (e.g., in the nominative feminine), and strong -\textit{en} can surface as weak -\textit{en} (e.g., in the dative plural). However, while strong -\textit{e} is changed to weak -\textit{en} (e.g., in the nominative plural), strong -\textit{en} is never altered to weak -\textit{e}. Sauerland concludes that this indicates that -\textit{e} is more specified than -\textit{en}. 
\end{styleFootnote}

\begin{styleFootnote}
\ \ In order to account for the strong/weak alternation, Sauerland adopts the Hierarchy of Case features from Harris (1994):
\end{styleFootnote}

\begin{styleFootnote}
(15)\ \ \ \  Case
\end{styleFootnote}

\begin{styleFootnote}
[Warning: Draw object ignored][Warning: Draw object ignored]
\end{styleFootnote}

\begin{styleFootnote}
\ \ direct\ \ \ \ oblique
\end{styleFootnote}

\begin{styleFootnote}
[Warning: Draw object ignored][Warning: Draw object ignored]
\end{styleFootnote}

\begin{styleFootnote}
\ \ accusative\ \ genitive
\end{styleFootnote}

\begin{styleFootnote}
Evidence in German that the direct cases (i.e., nominative and accusative) pattern together in opposition to the oblique cases (i.e., dative and genitive) comes from the weak inflections in the neuter and feminine where -\textit{e} occurs in the direct cases but -\textit{en} in the oblique ones (see Table 3).
\end{styleFootnote}

\begin{styleFootnote}
Sauerland formulates the following seven vocabulary insertion rules:
\end{styleFootnote}

\begin{styleFootnote}
(16)\ \ a.\ \ [+plural, +oblique, -genitive]\ \ \ \ →\ \ {}-\textit{en}
\end{styleFootnote}

\begin{styleFootnote}
\ \ b.\ \ [+masc, +direct, -accusative] \ \ \ \ →\ \ {}-\textit{er}
\end{styleFootnote}

\begin{styleFootnote}
\ \ c.\ \ [-fem, +oblique, -genitive] \ \ \ \ →\ \ {}-\textit{em}
\end{styleFootnote}

\begin{styleFootnote}
\ \ d.\ \ [-fem] \ \ \ \ \ \ \ \ \ \ →\ \ {}-\textit{es}
\end{styleFootnote}

\begin{styleFootnote}
\ \ e.\ \ [+oblique] \ \ \ \ \ \ \ \ →\ \ {}-\textit{er}
\end{styleFootnote}

\begin{styleFootnote}
\ \ f.\ \ [+direct] \ \ \ \ \ \ \ \ → \ \ {}-\textit{e}
\end{styleFootnote}

\begin{styleFootnote}
\ \ g.\ \ [] \ \ \ \ \ \ \ \ \ \ → \ \ {}-\textit{en}
\end{styleFootnote}

\begin{styleFootnote}
He proposes that feature deletion widens the distribution of -\textit{e} in (16f) and -\textit{en }in (16g). In his account, there are three types of deletion rules yielding a total of six:\footnote{\ I added the feature [+genitive] to (17b) to allow -\textit{em} to surface on strong adjectives in dative contexts.}
\end{styleFootnote}

\begin{styleFootnote}
(17)\ \ a. Impoverishment rule applying to inflections on determiners and adjectives:\ \ 
\end{styleFootnote}

\begin{styleFootnote}
(i) Delete Agr with [+masc, +direct, +accusative]
\end{styleFootnote}

\begin{styleFootnote}
\ \ b. Impoverishment rule applying to inflections on adjectives:
\end{styleFootnote}

\begin{styleFootnote}
\ \ \ \ (ii) Delete Agr with [-fem, +oblique, +genitive]
\end{styleFootnote}

\begin{styleFootnote}
\ \ c. Impoverishment rules applying to adjectives in weak positions:
\end{styleFootnote}

\begin{styleFootnote}
\ \ \ \ (iii) Delete [Gender]
\end{styleFootnote}

\begin{styleFootnote}
\ \ \ \ (iv) Delete [accusative]
\end{styleFootnote}

\begin{styleFootnote}
\ \ \ \ (v) \ Delete [oblique]
\end{styleFootnote}

\begin{styleFootnote}
\ \ \ \ (vi) Delete [Case] in the environment of [+plural]\ \ 
\end{styleFootnote}

\begin{styleFootnote}
After briefly commenting on the role of the various deletion rules, I discuss some shortcomings and point out how the analysis to be developed below deals with these issues.
\end{styleFootnote}

\begin{styleFootnote}
\ \ As mentioned above, accusative masculine -\textit{en} is special with both determiners and adjectives: it is the only instance where strong and weak inflections differ in the nominative and accusative cases. This exception is captured by formulating the deletion rule in (17a). As a consequence, only (16g) can apply.\footnote{\ Schoorlemmer (2009: 235) formulates an Impoverishment rule to account for the weak ending -\textit{en} (on adjectives) in the accusative masculine. } Similarly, adjectives in the genitive masculine/neuter are exceptional in that their ending is -\textit{en} (rather than -\textit{es }as compared to the determiners). This follows from the deletion rule in (17b) such that only (16g) can apply. 
\end{styleFootnote}

\begin{styleFootnote}
\ \ Finally, the set of rules in (17c) accounts for the (traditional) weak inflections. The weak positions of adjectives are defined by adjectives occurring after determiners with strong inflections (no further details are provided). Adopting additional Feature Hierarchies, Sauerland assumes that features with a capital letter subsume all instances of that category. This means that the deletion of Gender removes masculine, neuter, and feminine features and that the deletion of Case removes direct (which includes nominative and accusative) and oblique (which consists of dative and genitive) cases. After the deletion of the features in (17c) from the syntactic representation, the vocabulary items in (16a-e) can no longer be inserted. Rather, only the rule in (16f), restricted now to singular contexts (due to (vi) in (17c)), and the rule in (16g) can apply. 
\end{styleFootnote}

\begin{styleFootnote}
Given certain interpretations (e.g., which exact features are present on terminal nodes), the system works, and I think it is basically on the right track. However, there are also some issues. Pointing out these shortcomings is not meant to be a criticism of Sauerland’s proposal. Rather, the following points are intended to be understood as issues that any proposal that works on a similar set of assumptions needs to address. There are three general and three technical issues.
\end{styleFootnote}

\begin{styleFootnote}
Starting with the general points, there are three types of deletion rules (feature deletion for both determiners and adjectives (17a), feature deletion for adjectives only (17b), and feature deletion to bring about weak inflections (17c)). In the system to be developed below, the effect of (17a) does not follow from an Impoverishment rule but rather from the way the vocabulary insertion rule for accusative masculine -\textit{en} is stated: Impoverishment deletes a feature, but that feature is not part of the vocabulary insertion rule for this -\textit{en}. The effects of (17b) and (17c) do involve Impoverishment rules in the system to be developed (but in modified form): the effect of (17b) is capured by Impoverishment Rule 2 and that of (17c) by Impoverishment Rule 1.
\end{styleFootnote}

\begin{styleFootnote}
Second, there are six deletion rules in total. Below, only two Impoverishment rules are proposed. In fact, it is suggested that the application of (17b) is more general: Impoverishment Rule 2 also accounts for the weak inflections on certain determiners in genitive masculine/neuter contexts (Chapter 3, Section 4). Third, since the weak inflections on adjectives depend on the strong inflections of determiners, the latter must be inserted first. It is left unspecified by Sauerland why this must be so (and cannot be the other way around). Below, no such ordering is crucial. It is proposed that it is not the inflection of the determiner but rather its stem (specifically, its categorial feature [+D] present in the syntax) that triggers Impoverishment.
\end{styleFootnote}

\begin{styleFootnote}
Turning to some technical questions, it is not clear (except by stipulation) why (16d) must be ordered before both (16e) and (16f) – all these vocabulary insertion rules are equally specified (note that if (16e) were ordered before (16d), this would yield the incorrect forms \textit{der }‘the’ or\textit{ guter} ‘good’ in genitive masculine/neuter contexts; if (16f) were ordered before (16d), this would result in the incorrect forms \textit{die }‘the’ or\textit{ gute} ‘good’ in nominative/accusative neuter contexts).\footnote{\ Halle \& Marantz (1994: 281) assume “a reasonable feature hierarchy that treats Case features as more specific than gender features”. If so, then both (16e) and (16f) should indeed be ordered before (16d) yielding incorrect forms as discussed in the main text. This means that Sauerland must stipulate the ordering of the vocabulary items (see Embick \& Noyer 2007: 298 fn. 14).} Second, as far as I can tell, the deletion rule (iv) in (17c) is not needed. Finally, if Agr, presumably the head in the syntactic representation that involves the inflection, is deleted by (17a) or (17b), it is not clear where or how the default ending -\textit{en} will be attached.
\end{styleFootnote}

\begin{styleFootnote}
The following proposal draws heavily on the insights of Sauerland (1996) but tries to avoid the general and technical issues raised above. In addition, it aims at spelling out in detail the morpho-syntactic (and lexical) conditions the strong/weak alternation is subject to.
\end{styleFootnote}

\begin{styleFootnote}
2.1.4. Basic Proposal
\end{styleFootnote}

\begin{styleStandard}
I propose that determiners, \textit{der}{}-words and \textit{ein}{}-words, form triggers for Impovishment. I assume that Impoverishment operates in a local fashion. More precisely, I propose that the feature bundles that are to be realized as the overt inflections on the adjective undergo Impoverishment inside the phrase that hosts them (i.e., AgrP). This means that the trigger for Impoverishment, the determiner, and its target, the inflection on the adjective, must be in a local relation at some point in the derivation. With the determiners surfacing in the DP-level, this implies that determiners must have originated in a lower position. 
\end{styleStandard}

\begin{styleStandard}
\ \ In Chapter 1, Section 4.1.1, I argued that there are two determiner positions in the noun phrase, ArtP and DP (e.g., Julien 2002, 2005a; Taraldsen 1990), and that both are connected by movement (e.g., see again Borer 2005; Nykiel 2015; Rehn 2019; Roehrs 2009a, 2015, 2019; Schoorlemmer 2009, 2012; van Gelderen 2007). I provided some distributional evidence for a lower position of determiners in that section. Adopting the following structure and derivation, I reiterate the proposal that the determiner moves from ArtP to DP in a successive-cyclic fashion. This is illustrated below with an article (for the DP structure with a demonstrative undergoing movement, see Section 2.4):
\end{styleStandard}

\begin{styleFootnote}
(18)\ \ a.\ \ \textit{d-er \ \ \ frisch-e}\textit{\textsubscript{ \ }}\textit{schwarz-e Kaffee}
\end{styleFootnote}

\begin{styleFootnote}
\ \ \ \ the-\textsc{st} hot-\textsc{wk} black-\textsc{wk} \ coffee.\textsc{masc}
\end{styleFootnote}

\begin{styleStandard}
\ \ \ \ ‘the fresh black coffee’
\end{styleStandard}

\begin{styleStandard}
\ \ b.\ \  DP
\end{styleStandard}

\begin{styleFootnote}
[Warning: Draw object ignored][Warning: Draw object ignored]
\end{styleFootnote}

\begin{styleFootnote}
[Warning: Draw object ignored]\textit{\ \ \ \ \ }\ \  \ D’
\end{styleFootnote}

\begin{styleFootnote}
[Warning: Draw object ignored][Warning: Draw object ignored]
\end{styleFootnote}

\begin{styleFootnote}
\ \ \ \  \ \ D\ \ \ \ AgrP
\end{styleFootnote}

\begin{styleFootnote}
[Warning: Draw object ignored][Warning: Draw object ignored]\ \ \ \  \textit{der}\textsubscript{i}
\end{styleFootnote}

\begin{styleFootnote}
[Warning: Draw object ignored]\ \ \ \  \ \ \ \ \ \ \ \ \ \ \ InflP\ \  \ \ Agr’
\end{styleFootnote}

\begin{styleFootnote}
[Warning: Draw object ignored][Warning: Draw object ignored]\textit{\ \ \ \ \ \ \ \ \ \ \ frische}
\end{styleFootnote}

\begin{styleFootnote}
\ \ \ \ \ \ \ \ Agr\ \  \ \ \ \ \ \ \ \ \ \ AgrP
\end{styleFootnote}

\begin{styleFootnote}
[Warning: Draw object ignored][Warning: Draw object ignored]\ \ \ \ \ \ \ \ der\textsubscript{i}
\end{styleFootnote}

\begin{styleFootnote}
[Warning: Draw object ignored]\ \ \ \ \ \ \ \  \ \ \ \ \ \ \ \ \ InflP\ \ \ \ Agr’
\end{styleFootnote}

\begin{styleFootnote}
[Warning: Draw object ignored][Warning: Draw object ignored]\textit{\ \ \ \ \ \ \ \ schwarze}
\end{styleFootnote}

\begin{styleFootnote}
\ \ \ \ \ \ \ \ \ \ \ \ Agr\ \  \ \ \ \ \ \ \ \ \ \ ArtP
\end{styleFootnote}

\begin{styleFootnote}
[Warning: Draw object ignored][Warning: Draw object ignored]\ \ \ \ \ \ \ \ \ \ \ \ der\textsubscript{i}
\end{styleFootnote}

\begin{styleFootnote}
Art’
\end{styleFootnote}

\begin{styleFootnote}
[Warning: Draw object ignored][Warning: Draw object ignored]\ \ 
\end{styleFootnote}

\begin{styleFootnote}
\ \ \ \ \ \ \ \ \ \ \ \ \ \  \ \ \ \ \ \ \ \ \ \ \ \ Art \ \ \ \ \ \ \ \ \ \ \ \ \ \ \ NumP
\end{styleFootnote}

\begin{styleFootnote}
[Warning: Draw object ignored][Warning: Draw object ignored]\ \ \ \ \ \ \ \ \ \ \ \ \ \ \ \ der\textsubscript{i}
\end{styleFootnote}

\begin{styleFootnote}
\ \ \ \ \ \ \ \ \ \ \ \ \ \ \ \ \ \ \ \  \ \ \ \ \ \ \ \ \ Num’
\end{styleFootnote}

\begin{styleStandard}
[Warning: Draw object ignored][Warning: Draw object ignored]
\end{styleStandard}

\begin{styleStandard}
\ \ \ \ \ \ \ \ \ \ \ \ \ \ \ \ \ \  \ \ \ \ \ \ \ \ Num\ \ \ \ NP
\end{styleStandard}

\begin{styleStandard}
[Warning: Draw object ignored][Warning: Draw object ignored]\ \ \ \ \ \ \ \ \ \ \ \ \ \ \ \ \ \ \ \ \ \ \ \  t\textsubscript{k}
\end{styleStandard}

\begin{styleStandard}
\ \ \ \ \ \ \ \ \ \ \ \ \ \ \ \  \ \ \ \ \ \ \textit{Kaffee}\textsubscript{k}\textit{\ \  \ \ \ \ \ \ \ \ }Num\ \ \ \ 
\end{styleStandard}

\begin{styleFooter}
Recall also that definite determiners have the feature [+DEF] and move to the DP-level to specify the definiteness feature on D. In contrast, \textit{ein} is semantically vacuous and does not have a feature for definiteness. Consequently, it can surface in different positions (i.e., Art, Card, or D). This, in turn, makes it possible to claim that \textit{ein} supports overt operators or flags the presence of covert operators (Chapter 5). To be clear, I assume that \textit{ein} remains in ArtP unless it has to move up for a reason. As shown below, weak inflections only occur in regular, simple DPs like (18b).
\end{styleFooter}

\begin{styleFooter}
I propose that determiners move via adjunction to the DP-level. Articles (Art) are heads, and they move by head adjunction to D; demonstratives (Dem) are phrases, and they move by phrasal adjunction to Spec,DP.\footnote{\ Recall from Chapter 1, Section 4.1.2 that demonstratives involve complex structures with InflP at the top. To avoid confusion, I use Dem (for demonstrative) in the tree representation in (19b) below.} I assume that all elements including structural elements are fully specified for case, number, and gender yielding concord. Impoverishment involves the deletion of a certain feature (to be made more precise below). Taking determiners to be triggers for Impovishment, I illustrate this feature reduction by [+I(mpoverishment)] for now. I propose that adjunction by the determiner reduces the feature bundle of the element adjoined to. Assuming Percolation within the same phrase, this feature reduction spreads from the head to the phrase in case of an article (19a), or from the phrase to the head in case of a demonstrative (19b).\footnote{\ Olsen (1991b) assumes Percolation from the phrase to the head; Norris (2014: 136-37) points out that Percolation from the head to the phrase immediately follows from Bare Phrase Structure (Chomsky 1995).} Finally, assuming Spec-head agreement and Percolation, [+I] spreads to InflP in Spec,AgrP and to its head Infl. The latter contains the feature bundle for case, number, and gender ([CNG]) and will be spelled out as the ending on the adjective stem in AP in (19a-b). The relevant portions of the trees, which include the adjoined determiner, are given for these two scenarios below:
\end{styleFooter}

\begin{styleFooter}
(19)\ \ a.\ \ \textit{Adjunction of Article}
\end{styleFooter}

\begin{styleFooter}
\ \ \ \ \ \  \ \ \ \ \ AgrP\textsubscript{[+I]}
\end{styleFooter}

\begin{styleFooter}
[Warning: Draw object ignored][Warning: Draw object ignored]
\end{styleFooter}

\begin{styleFooter}
\ \ \ \ InflP\textsubscript{[+I]}\ \ \ \ Agr’\textsubscript{[+I]}
\end{styleFooter}

\begin{styleFooter}
[Warning: Draw object ignored][Warning: Draw object ignored][Warning: Draw object ignored][Warning: Draw object ignored]
\end{styleFooter}

\begin{styleFooter}
\ \ \ \ \ \ \ \ \ [CNG]\textsubscript{[+I]}\ \  \ AP\ \ Agr\textsubscript{[+I]}\ \ \ \ … t\textsubscript{k}…
\end{styleFooter}

\begin{styleFooter}
[Warning: Draw object ignored][Warning: Draw object ignored]\ \ \ \ 
\end{styleFooter}

\begin{styleFooter}
\ \ \ \ \ \ Art\textsubscript{k}\ \ \ \ Agr\textsubscript{[+I]}
\end{styleFooter}

\begin{styleFooter}
\ \ b.\ \ \textit{Adjunction of Demonstrative}
\end{styleFooter}

\begin{styleFooter}
\ \ \ \  \ \ \ \ AgrP\textsubscript{[+I]}
\end{styleFooter}

\begin{styleFooter}
[Warning: Draw object ignored][Warning: Draw object ignored]
\end{styleFooter}

\begin{styleFooter}
\ \  \ Dem\textsubscript{k}\ \  \ \  \ \ \ \ AgrP\textsubscript{[+I]}
\end{styleFooter}

\begin{styleFooter}
[Warning: Draw object ignored][Warning: Draw object ignored]
\end{styleFooter}

\begin{styleFooter}
\ \ \ \ InflP\textsubscript{[+I]}\ \ \ \ Agr’\textsubscript{[+I]}
\end{styleFooter}

\begin{styleFooter}
[Warning: Draw object ignored][Warning: Draw object ignored][Warning: Draw object ignored][Warning: Draw object ignored]
\end{styleFooter}

\begin{styleFooter}
\ \ [CNG]\textsubscript{[+I]}\ \  \ AP\ \ Agr\textsubscript{[+I]}\ \ \ \ … t\textsubscript{k}…\ \ 
\end{styleFooter}

\begin{styleFooter}
Note that the adjunction to heads and phrases, feature Percolation within the same phrase, and Spec-head agreement are all local relations; that is, Impoverishment occurs in a locally restricted domain. Given a recursive AgrP in (18b) above and successive-cyclic movement of the determiner from ArtP to DP, this analysis provides an immediate explanation of why either all adjectives are strong or all are weak. Next, I flesh out this basic proposal. The following discussion is based on Roehrs \& Julien’s (2014) analysis of German with one modification (Footnote Error: Reference source not found). In the course of the discussion, I add many details to that proposal. As mentioned above, Roehrs \& Julien (2014) was heavily influenced by Sauerland (1996).
\end{styleFooter}

\begin{styleStandard}
2.1.5. The Proposal in More Detail: Vocabulary Insertion Rules
\end{styleStandard}

\begin{styleStandard}
Following Bierwisch (1967) and others, I argue that case, number, and gender are not primitive features. Rather, I assume that case involves a two-category system and can be represented by the features [O(blique)] and [S(tructural)].\footnote{\ The feature [S(tructural)] was chosen over [D(irect)] to avoid confusion. Below, I propose that the feature [S] is deleted by Impoverishment, which is triggered by the presence of the categorial feature of determiners [+D]. This choice allows me to clearly distinguish between the feature that undergoes deletion and the one that involves the trigger of that deletion.} Each of these features can have a negative or a positive value (i.e., [-, +]) yielding the following decomposition:
\end{styleStandard}

\begin{styleFooter}
(20)\ \ a.\ \ nominative:\ \ [ -O, \ {}-S]
\end{styleFooter}

\begin{styleFooter}
\ \ b.\ \ accusative:\ \ [ -O, +S]
\end{styleFooter}

\begin{styleFooter}
\ \ c.\ \ dative:\ \ \ \ [+O, \ {}-S]
\end{styleFooter}

\begin{styleFooter}
\ \ d.\ \ genitive:\ \ [+O, +S]
\end{styleFooter}

\begin{styleFooter}
Similarly, gender consists of the features [F(eminine)] and [N(euter)] and can be broken down as in (21a-c). Plural is the neutralization of gender (21d), something Sternefeld (2008: 80) refers to as the “fourth gender”:\footnote{\ Krifka (2009: 163-65) points out that there are advantages in assuming that plural involves a separate gender (for the relatedness of feminine and plural, see also Leiss 1994: 291-94).}
\end{styleFooter}

\begin{styleFooter}
(21)\ \ a.\ \ masculine:\ \ [ -F, \ {}-N]
\end{styleFooter}

\begin{styleFooter}
b.\ \ neuter: \ \ [ -F, +N]
\end{styleFooter}

\begin{styleFooter}
c.\ \ feminine: \ \ [+F, \ {}-N]
\end{styleFooter}

\begin{styleFooter}
\ \ d.\ \ plural:\ \ \ \ [+F, +N]
\end{styleFooter}

\begin{styleFooter}
Unlike plural, all singular forms involve at least one negative value for [F] or [N].
\end{styleFooter}

\begin{styleFooter}
\ \ Roehrs \& Julien (2014: 250 fn. 5) provide motivation for the system above. Most importantly, following Harbert’s (2007: 468) Case Hierarchy, they assume that nominative is the least and genitive the most oblique case. Based on Steinmetz (2001), they assume that masculine is the default gender in German, and following Sternefeld (2008: 80), they take plural to be the neutralization of gender. Note that taking the masculine gender and the nominative case as the least marked categories – they each have two negative values in their feature decompositions – is consonant with Bierwisch’s (1967: 253) statement that there is more syncretism within marked categories (observe that with the strong inflections, masculine has the most different forms in Table 4; conversely, genitive has the fewest different forms).\footnote{\ There are many different systems to decompose case, number, and gender – too many to review here in detail. For other, partially similar decompositions, see Bierwisch (1967: 246-50), Evans (2019: 51), Gallmann (2004: 124-25), G. Müller (2002a: 119), Trommer (2005), B. Wiese (1996, 1999), Wunderlich (1997: 48-49), and many others. Blevins (1995) captures syncretism as part of Feature Hierarchy, with primary (e.g., feminine vs. non-feminine) and subsidiary (e.g., masculine vs. neuter) features. This hierarchy as well as Feature Geometry more generally will not play a role here (for some brief remarks on hierarchies as regards DM, see Arregi \& Nevins 2012: 204-05, Harley \& Noyer 1999: 6).}
\end{styleFooter}

\begin{styleFooter}
\ \ Before moving on, I point out that the decomposition of gender and number in (21) allows us to state certain well-known syncretisms that we find in the masculine/neuter vs. in the feminine/plural, that is, in the opposition [-F] vs. [+F]: (i) strong adjectival inflections in the nominative/accusative (C = consonant): -\textit{eC} vs. -\textit{e}, (ii) strong adjectival inflections in the dative: -\textit{em} vs. -\textit{er/-en}, (iii) strong adjectival inflections in the genitive: -\textit{es} vs. -\textit{er}, (iv) third-person pronouns in the nominative/accusative (abstracting away from \textit{ihn }‘him’): \textit{er}/\textit{es} ‘he/it’ vs. \textit{sie} ‘she, her; they, them’, (v) \textit{ein}{}-words in the nominative/accusative (abstracting away from \textit{einen }‘a’): uninflected \textit{ein} vs. inflected \textit{eine}, (vi) definite article stems in the nominative/accusative: \textit{de}{}-/\textit{da}{}- ‘the’ vs. \textit{di}{}- ‘the’, (vii) possessive articles consisting of bound and free morphemes: \textit{sein} ‘(POSS+\textit{ein} =) his, its’ vs. \textit{ihr} ‘her, their’ (Chapter 5, Section 4.1.2), and perhaps less widely known (viii) pronominal DPs in the dative masculine/neuter prefer strong adjectives vs. pronominal DPs in the dative feminine and nominative plural prefer weak adjectives (Chapter 3, Section 5).
\end{styleFooter}

\begin{styleFooter}
\ \ As to case, syncretism is much less prevalent here. As should be clear from the discussion of syncretism in gender and number provided just above, there is syncretism in the nominative/accusative vs. the oblique cases. Specifically, while nominative and accusative have the same inflections in the neuter, feminine, and plural (e.g., plural -\textit{e}), the oblique cases dative and genitive have the same inflections only in the feminine (i.e., -\textit{er}). It is also worth pointing out that there are very few instances of syncretism between the structural and oblique cases; for instance, the forms of the first and second person plural of the personal pronoun are the same in the accusative and dative cases (i.e., \textit{uns} ‘us’, \textit{euch} ‘you’). Note though that the latter instances are the exception, rather than the rule. This may have to do with the fact that these items are non-third person elements.\footnote{\ This may also be related to the fact that the structural cases behave differently from the oblique cases in syntactic terms (see Bayer \textit{et al}. 2001). To account for these differences, these authors propose that the oblique cases project a K(ase) P on top of the DP, an idea that I will not follow here.}
\end{styleFooter}

\begin{styleFooter}
\ \ In order to state the syncretisms above and to reduce the number of vocabulary insertion rules, I utilize value variables ($\alpha $, $\beta $), whereby $\alpha $ or $\beta $ in a vocabulary insertion rule is compatible with a positive or negative value in the syntactic representation (note that the variable does not get specified for such values – it is a compatibility feature). To give an example, the specification [$\alpha $O] in a vocabulary insertion rule is compatible with a positive or a negative value in the syntactic representation (but not both at the same time). Among others, this captures the syncretism of nominative and accusative case (cf. (20a-b) above). Similarly, I employ a category variable ($\gamma $), which ranges over [F] or [N]; that is, over one of these two features (but not both at the same time). To illustrate, [-$\gamma $] in a vocabulary insertion rule is compatible with either [-F] or [-N] in the syntactic represenation. This means that [-$\gamma $] corresponds to singular (as opposed to plural, which has a positive value for both [F] and [N]; for the discussion of the explicit use of disjunctions, see G. Müller 2002a: 120 fn. 14).\footnote{\ The category variable [$\gamma $], with a negative value, will find more application with first and second-person pronouns, which lack gender (Chapter 3, Section 5).} 
\end{styleFooter}

\begin{styleFooter}
Following Sauerland (1996), I propose that the two weak endings have the same feature specifications as certain strong endings (cf. also Gallmann 2004: 140). I assume that these instances are related and have the same vocabulary insertion rules (see below). For clarity, these related instances are respectively marked by round and curly brackets in Table 4. Using the feature decomposition above, all strong and weak inflections can be analyzed as follows (for convenience, I added the traditional labels as subscripts after the featural decomposition):
\end{styleFooter}

\begin{styleStandard}
Table 4:\textit{ }Strong and Weak Inflections in German
\end{styleStandard}

\begin{flushleft}
\begin{tabular}{|m{0.9837598in}|m{0.7337598in}|m{0.7962598in}|m{0.7337598in}|m{0.7962598in}|m{0.5462598in}|m{0.5462598in}|m{0.7337598in}|}

\hline
\centering{\bfseries STRONG} &
\centering [-F, -N]\textsubscript{M} &
\centering [-F, +N]\textsubscript{N} &
\centering [+F, -N]\textsubscript{F} &
\centering [+F, +N]\textsubscript{PL} &
\centering{\bfseries WEAK} &
\centering [-$\gamma $]\textsubscript{SGL} &
\centering\arraybslash [+F, +N]\textsubscript{PL}\\\hline
[ -O, \ {}-S]\textsubscript{NOM} &
\centering {}-er &
\centering {}-es &
\centering (-e) &
\centering {}-e &
\centering [-O] &
\centering (-e) &
\\\hline
[ -O, +S]\textsubscript{ACC} &
\centering {}-en &
\centering {}-es &
\centering (-e) &
\centering {}-e &
 &
 &
\\\hhline{-------~}
[+O, \ {}-S]\textsubscript{DAT} &
\centering {}-em &
\centering {}-em &
\centering {}-er &
\centering \{-en\} &
\centering [+O] &
\multicolumn{2}{m{1.3587599in}|}{\centering \{-en\}}\\\hline
[+O, +S]\textsubscript{GEN} &
\centering {}-es &
\centering {}-es &
\centering {}-er &
\centering {}-er &
 &
 &
\\\hhline{-------~}
\end{tabular}
\end{flushleft}
\begin{styleStandard}
Comparing Table 4 to Table 3 from above, we notice that the accusative masculine ending -\textit{en} is missing from the weak endings here. The special status of this ending follows from the way the strong ending for accusative masculine is stated in (22c).\footnote{\ In Roehrs \& Julien (2014), the exceptional accusative masculine form -\textit{en} contained the specification [+S]. Removing this feature from the vocabulary entry as in (22c) still allows me to specify this ending disambiguously (as nominative masculine -\textit{er} is more specific). In fact, this is now the only strong ending without a specification for [S]. This will become significant during the discussion of Impoverishment.} Assuming that features can be underspecified, the rules for vocabulary insertion of the endings can be stated as follows:\footnote{\ In his critique of some previous – what he calls – constructive analyses (Bierwisch 1967, Blevins 1995, B. Wiese 1999, and Wunderlich 1997), G. Müller (2002b) points out that all these types of accounts have rules that show different contexts for the insertion of (some of) the same strong inflections, for instance, -\textit{er} in (22a) and (22b). This leads to systematic as well as accidental syncretism. According to G. Müller (2002a, 2002b), destructive type of analyses involving rules that ban the insertion of certain inflections fare better.\par \ \ This criticism seems valid and can also be leveled against the current proposal. However, I think such multiple vocabulary insertion rules for one and the same inflectional form(s) cannot completely be avoided (also Sauerland 1996: 24). Considering that certain endings occur in other, unrelated domains as well (e.g., plural inflections on nouns: -\textit{e}, -\textit{en}, -\textit{er}, -\textit{s}; personal endings on verbs: -\textit{e}, -\textit{en}), it is not clear how these inflections can be related yielding just one rule of vocabulary insertion (or ban) for each inflection. Note in this regard that unspecified -\textit{en} in (23b) could potentially do multiple duty as a plural inflection for nouns and verbs. \par An advantage of the current proposal is that unlike some of the previous analyses, here the weak inflections are explicitly discussed; in fact, they are integrated with the strong inflections, and there is no inherent difference between these types of inflections (for the discussion of quite different types of analyses of weak inflections, see the end of Section 5.3).}
\end{styleStandard}

\begin{styleStandard}
(22)\ \ Strong (except for feminine \textit{{}-e} and plural \textit{{}-en}):
\end{styleStandard}

\begin{styleStandard}
\ \ a.\ \ [+F, -N, +O, $\alpha $S]\ \ →\ \ \textit{{}-er}
\end{styleStandard}

\begin{styleStandard}
\ \ \ \ [+F, +N, -O, $\alpha $S]\ \ →\ \ \textit{{}-e}
\end{styleStandard}

\begin{styleStandard}
\ \ b.\ \ [$\alpha $F, $\alpha $N, $\alpha $O, $\alpha $S]\ \ →\ \ \textit{{}-er}
\end{styleStandard}

\begin{styleStandard}
\ \ c.\ \ [-F, -N, -O]\ \ \ \ →\ \ \textit{{}-en} 
\end{styleStandard}

\begin{styleStandard}
\ \ \ \ [-F, +O, -S]\ \ \ \ →\ \ \textit{{}-em}
\end{styleStandard}

\begin{styleStandard}
\ \ d.\ \ [-F, $\alpha $O, $\beta $S]\ \ \ \ →\ \ \textit{{}-es}
\end{styleStandard}

\begin{styleStandard}
(23) \ \ \ \ Weak (including strong feminine \textit{{}-e} and plural \textit{{}-en}):
\end{styleStandard}

\begin{styleStandard}
\ \ a.\ \ [-$\gamma $, -O]\ \ \ \ \ \ →\ \ \textit{{}-e}
\end{styleStandard}

\begin{styleStandard}
\ \ b.\ \ []\ \ \ \ \ \ →\ \ \textit{{}-en}
\end{styleStandard}

\begin{styleStandard}
Comparing (22) to (23), note first that the (unambiguously) strong endings are stated in more specific terms than the weak endings. In other words, weak endings are underspecified to a greater degree. This means that there is no real difference between (22) and (23). This is confirmed by the fact that some of the strong endings (i.e., certain instances of -\textit{e} and -\textit{en}) share the same vocabulary insertion rules as the weak inflections -\textit{e} and -\textit{en} (23). This was one of Sauerland’s (1996) main insights. Second, I assume that a vocabulary item involving the same number of features with a variable is less specific than a corresponding vocabu\-lary item without such a variable. Accordingly, I have listed the vocabulary items with descending degrees of specificity in (22) and (23). Note that equally specified items differ from one another (in both (22a) and (22c), the two respective items vary in the feature [O]). Finally, as is often stated, note that the two paradigms, strong and weak, in Table 4 are simply generalizations. They have been replaced by the vocabulary insertion rules above; that is, paradigms are an epiphenomenon without independent status. Before moving on, I comment on the relationship between the vocabulary insertion rules and the terminal nodes in the syntactic representation.
\end{styleStandard}

\begin{styleFootnote}
Starting with the weak inflections, note that the weak ending -\textit{e} never occurs in the plural. Thus, it distinguishes number as it only occurs in the singular. However, it does not distinguish individual cases or genders. This is particularly clear in the current account where -\textit{e} is specified [-$\gamma $, -O]. The weak ending -\textit{en} is not specified for any CNG features. In the current account, it is the elsewhere case. What both of these inflections have in common is that they do not distinguish case and gender. In contrast, the strong endings are specified for more features. Note first that strong endings are more specific in their occurrence (e.g., -\textit{em} only occurs in the dative masculine/neuter; -\textit{s} only in the masculine/neuter). This is reflected by the contexts in which their vocabulary insertion rules apply; for instance, the inflection -\textit{em} is specified as [-F, +O, -S]. Compared to weak inflections, strong inflections distinguish case, number, and gender better.
\end{styleFootnote}

\begin{styleStandard}
The feature specifications on the terminal nodes in the syntactic representation are different. If present, they always involve specifications for the four features [F, N, O, S]. In other words, all features are present in the abstract syntactic representation, but only a subset of those features are specified in the vocabulary insertion rules – the latter involve underspecification allowing competition between some of the vocabulary insertion rules. Note again that these are basic tenets of DM.
\end{styleStandard}

\begin{styleFooter}
2.1.6. The Proposal in More Detail: Structural Position of Inflections
\end{styleFooter}

\begin{styleStandard}
In DM, vocabulary items are inserted late. Matching the maximum number of features on the terminal head, the availability of a more specific vocabulary item precludes the insertion of a less specific one. With the system laid out above in mind, the abstract feature bundle in Infl ([CNG]) of the larger adjective structure can now be restated (for convenience, I continue referring to the feature bundle as CNG if the decomposition is not relevant). Taking dative masculine as an example, the terminal head Infl is specified as in (24). Turning to Vocabulary Insertion, note that [-F, +O, -S] in (22c) is the most specific, matching vocabulary item. It is inserted and spelled out as the strong ending -\textit{em }on \textit{klein}{}- ‘small’ resulting in the form after the arrow:
\end{styleStandard}

\begin{styleFootnote}
(24)\ \ \ \  InflP
\end{styleFootnote}

\begin{styleFootnote}
[Warning: Draw object ignored][Warning: Draw object ignored]
\end{styleFootnote}

\begin{styleFootnote}
\ \  \ \ \  Infl’
\end{styleFootnote}

\begin{styleFootnote}\itshape
[Warning: Draw object ignored][Warning: Draw object ignored]
\end{styleFootnote}

\begin{styleStandard}
\ \ [-F, -N, +O, -S]\ \  \ AP
\end{styleStandard}

\begin{styleStandard}\itshape
\ [Warning: Draw object ignored][Warning: Draw object ignored]
\end{styleStandard}

\begin{styleStandard}
\ \ \ \ \ \ \ \  \ \ \  \ \ A’
\end{styleStandard}

\begin{styleStandard}\itshape
[Warning: Draw object ignored][Warning: Draw object ignored]\ \ \ \ \ \ \ \ \ \ \ \ 
\end{styleStandard}

\begin{styleStandard}
\ \ \ \ \ \ \ \  \ \ A\ \ \ \ 
\end{styleStandard}

\begin{styleStandard}
\ \ \ \ \ \  \ \ \ \ \ \ \ \ \ \ \ \textit{klein\ \ \ \ \ \ }→\ \ \textit{klein}{}-\textit{em}
\end{styleStandard}

\begin{styleStandard}
The structure above is a convenient shorthand for a more detailed derivation. More specifically, after movement of the adjective stem \textit{klein} to Spec,InflP, Linearization, and Vocabulary Insertion, we obtain the string in (25). Following Murphy (2018), I assume that the adjectival suffix undergoes Local Dislocation (instantiated as a type of leaning) combining with the adjective stem to yield the spell-out form after the arrow (see also Section 5.3):
\end{styleStandard}

\begin{styleStandard}
[Warning: Draw object ignored][Warning: Draw object ignored](25)\ \ 
\end{styleStandard}

\begin{styleStandard}
\textit{\ \ klein\ \ \ \ {}-em\ \ \ \ \ \ \ \ }→\ \ \textit{kleinem}
\end{styleStandard}

\begin{styleStandard}
Unless these details are of signifance, I usually illustrate the analysis by a syntactic representation as in (24). As briefly discussed in Chapter 1, Section 4.1.2, determiners also have internal structure. Since they are proposed to be the trigger of Impoverishment, I discuss their inner makeup in more detail now.
\end{styleStandard}

\begin{styleStandard}
I focus here on four common types of determiners: the indefinite null articles (in the contexts of mass and plural nouns), the indefinite article \textit{ein}, the definite article, and the demonstrative. Again illustrating with the dative masculine, these elements have the following internal structure. The terminal head Art where the indefinite null articles later surface has a categorial feature [+D(eterminer)] and a negative feature for definiteness. As null elements, they have no features for case, number, and gender that are later spelled out as inflections (26a). The indefinite article \textit{ein} only has the categorial feature [+D] and features for case, number, and gender. Given these different types of features (category vs. CNG), I assume that these make up two individual feature bundles. These two separate bundles undergo feature union yielding Art as in (26b).\footnote{\ There is evidence that inflections are generated with their determiner heads. As mentioned in Chapter 1 and discussed in more detail in Chapter 8, Section 2.2.2, there are cases where \textit{ein} occurs in two positions. Crucically, \textit{ein} in the lower position has the same, varying inflection as related \textit{kein} ‘no’. Compare (ia) to (ib):\par (i)\ \ a.\ \ \textit{k-ein-e \ \ \ \ so’n-e \ Leute}\par \ \ \ \ \ \ \textsc{neg}{}-a-\textsc{st} so.a-\textsc{st} people\par \ \ \ \ \ \ ‘no such people’\par \ \ b.\ \ \textit{mit} \textit{\ \ k-ein-en }\textit{\textsubscript{\ }}\textit{\ so’n-en Leuten}\par \ \ \ \ \ \ with \textsc{neg}{}-a-\textsc{st} so.a-\textsc{st} people\par \ \ \ \ \ \ ‘with no such people’\par This clearly shows that inflected \textit{ein} has a bipartite structure. As indefinite articles are usually assumed to be heads, this is captured in the main text by two separate feature bundles under Art.} Like the indefinite overt article, the definite article has a feature bundle for case, number, and gender; like the indefinite null articles, the definite article has a definiteness feature, but it is specified as positive (26c):
\end{styleStandard}

\begin{styleStandard}
(26)\ \ a.\ \ \textit{Indefinite Null Articles }
\end{styleStandard}

\begin{styleStandard}
\ \  \ \ \ \ \  \ \ \ \ \ \ Art\textsubscript{[+D; -DEF]}
\end{styleStandard}

\begin{styleStandard}
[Warning: Draw object ignored]
\end{styleStandard}

\begin{styleStandard}
\ \  \ \ \ \ \ \ \ \ \ \ \ [+D; -DEF]\ \ \ \ \ \ \ \ →\ \ \textit{Ø}\textit{\textsubscript{D}}
\end{styleStandard}

\begin{styleStandard}
\ \ b.\ \ \textit{Indefinite Overt Article}\ \ 
\end{styleStandard}

\begin{styleStandard}
\ \ \ \ \ \ Art\textsubscript{[+D][-F, -N, +O, -S]}
\end{styleStandard}

\begin{styleStandard}
[Warning: Draw object ignored][Warning: Draw object ignored]
\end{styleStandard}

\begin{styleStandard}
\ \ \ \ [+D]\ \ \ \ [-F, -N, +O, -S]\ \ →\ \ \textit{ein-em}
\end{styleStandard}

\begin{styleStandard}
\ \ c.\ \ \textit{Definite Article}
\end{styleStandard}

\begin{styleStandard}
\ \ \ \ \ \ Art\textsubscript{[+D; +DEF][-F, -N, +O, -S]}
\end{styleStandard}

\begin{styleStandard}
[Warning: Draw object ignored][Warning: Draw object ignored]
\end{styleStandard}

\begin{styleStandard}
\ \  \ \ \ [+D; +DEF]\ \ \ \ [-F, -N, +O, -S]\ \ →\ \ \textit{d-em}
\end{styleStandard}

\begin{styleStandard}
The spell-out forms are as follows: [+D; -DEF] is realized as \textit{Ø}\textit{\textsubscript{D}}, [+D] as \textit{ein}{}-, and [+D; +DEF] as \textit{d}{}-. As for the inflection, only [-F, +O, -S] matches the CNG bundle yielding -\textit{em}. Taken together, this spells out the forms of the determiners provided after the arrow.
\end{styleStandard}

\begin{styleStandard}
Unlike the first three elements, the demonstrative is phrasal. I assume there are two terminal heads (27). Dem involves the features [+D; +DEF, +DEIX] and projects its own extended projection with InflP at the top (see Leu 2007, 2015; Roehrs 2010, 2013a). The features for case, number, and gender are in Infl. Dem moves to adjoin to Infl (not shown). Given feature union and Percolation, all features spread to InflP:
\end{styleStandard}

\begin{styleStandard}
(27)\ \ \textit{Demonstrative}
\end{styleStandard}

\begin{styleJBExample}
\ \ \ \ \ \ \ \ \ \ \ \  \ \ \ \ InflP\textsubscript{[+D; +DEF, +DEIX][-F, -N, +O, -S]}
\end{styleJBExample}

\begin{styleStandard}\bfseries
[Warning: Draw object ignored][Warning: Draw object ignored]
\end{styleStandard}

\begin{styleStandard}
\ \ \ \ \ [-F, -N, +O, -S]\ \ \ \ DemP
\end{styleStandard}

\begin{styleStandard}
\ \ \ \ \ \ \ \  \ \ \ {\textbar} \ \ \ \ \ \ 
\end{styleStandard}

\begin{styleStandard}
\ \ \ \ \ \ \ \ Dem
\end{styleStandard}

\begin{styleStandard}
\ \ \ \ \ \  \ \ [+D; +DEF, +DEIX]\ \ →\ \ \textit{dies-em}
\end{styleStandard}

\begin{styleStandard}
The features [+D; +DEF, +DEIX] and [-F, -N, +O, -S] are spelled out as \textit{dies}{}- and -\textit{em}, respectively, yielding the form after the arrow.
\end{styleStandard}

\begin{styleStandard}
Besides the demonstrative \textit{dieser} ‘this’, other phrasal \textit{der}{}-words have this general structure: (stressed) \textit{DER }‘that’, \textit{jener }‘that’,\textit{ jeder }‘every’,\textit{ mancher }‘some’,\textit{ solcher }‘such’,\textit{ welcher }‘which’, and\textit{ alle }‘all’ (see also Chapter 3, Section 4). While I cannot specify (and motivate) all the relevant features of these individual words, I assume that all these elements have the categorial feature [+D]. Furthermore, like \textit{dieser} ‘this’, the demonstratives \textit{DER} ‘that’ and \textit{jener} ‘that’ involve the features [+DEF, +DEIX]. The quantifiers \textit{jeder} ‘every’ and \textit{alle} ‘all’ and the (presuppositional) interrogative \textit{welcher }‘which’ have the feature [+DEF] but lack the deixis feature. In contrast, \textit{mancher }‘some’ and\textit{ solcher }‘such’ lack the definiteness feature but have the feature [+DEIX] in their feature makeup.\footnote{\ The presence of the deixis feature is most straightforward with \textit{solch} ‘such’. For simplicity, I also assume this for \textit{manch} ‘some’ (although this element may turn out to have a different positively valued (relevant) feature in its makeup). Notice that these positively valued features play a role later in the discussion (Section 2.2.1). On a different note, recall that \textit{mancher }‘some’,\textit{ solcher }‘such’, and\textit{ welcher }‘which’ can also occur without an inflection (i.e., \textit{manch}, \textit{solch}, \textit{welch}). Given the lack of inflection, I assume that they have a different structure and analysis than in (27). Observe in this regard that they cannot occur directly before nouns (in non-formal contexts) but rather require \textit{ein} and/or an adjective to be present: \textit{manch *(ein/guter) Student} ‘some (good) student’. Given this dependency, I assume that they are not determiners but modificational elements, which do not involve the categorial feature [+D]. }
\end{styleStandard}

\begin{styleStandard}
Note that all determiners have the categorial feature [+D]. I propose that this is the trigger for Impoverishment. In addition, some determiners have features for definiteness and/or deixis, which also play a role in the account below. The indefinite article \textit{ein} is different – it has no features for definiteness and deixis. Note already here that considering the features of the various determiner stems, \textit{ein} is the least specified element. In Chapter 5, I propose in more detail that \textit{ein} is a semantically vacuous element.
\end{styleStandard}

\begin{styleFooter}
2.1.7. The Proposal in More Detail: Impoverishment
\end{styleFooter}

\begin{styleStandard}
Impoverishment involves a rule deleting a specific feature. Note that in the current context, Impoverishment cannot be stated in terms of the features [F] or [N] as these features are still relevant for the weak ending -\textit{e} given the category variable $\gamma $ in (23a). Furthermore, it cannot be stated in terms of the feature [O] as that feature is also mentioned in (23a). I propose that Impoverishment deletes the feature [S] if a determiner adjoins to a head or a phrase in the nominal structure:
\end{styleStandard}

\begin{styleStandard}
(28)\ \ \textit{Impoverishment Rule 1:}
\end{styleStandard}

\begin{styleStandard}
\ \ [\textsubscript{$\delta $} Determiner [\textsubscript{$\delta $} [S] ]], where $\delta $ = X, XP
\end{styleStandard}

\begin{styleStandard}
Note that this is a language-specific rule that applies to German only (for the discussion of Impoverishment in Yiddish, see Roehrs 2015). Furthermore, recall the partial ordering of \{Lowering, Impoverishment\} {\textgreater}{\textgreater} Vocabulary Insertion from Chapter 1, Section 4.2.1. First, as observed by Sauerland (1996), Impoverishment widens the distribution of the least specified (i.e., weak) inflections. This is consistent with the ordering of Impoverishment preceding Vocabulary Insertion. Second, Impoverishment Rule 1 involves the structural relation of adjunction. Bearing in mind that Lowering also has access to structural relations and precedes Vocabulary Insertion, this also fits with the claim that Impoverishment precedes Vocabulary Insertion. In other words, current assumptions are in agreement with the general layout of DM provided earlier.\footnote{\ Notice that Impoverishment Rule 1 and Rule 2 (see below) bear some resemblance to Arregi \& Nevins’ (2012) syntagmatic neutralization rules, Impoverishment rules that involve two distinct nodes where a rule affecting one node makes reference to the (external) morpho-syntactic environment of a second node. Note though that both nodes in their system are inside the same M-word (defined as a X\textsuperscript{0} that is not immediately dominated by another X\textsuperscript{0}; note that M-words are typically complex heads assembled by head movement). While the Impoverishment rules in the main text also apply in a local context, they do not occur inside M-words. If it turns out that the context of application of Impoverishment only involves M-words, then there are several options to update the proposal above; for instance, we could assume that all determiners and adjectives are heads and that determiners move by head adjunction and excorporation to D.}
\end{styleStandard}

\begin{styleStandard}
\ \ To illustrate the workings of Impoverishment Rule 1, I begin by discussing the determiners, the triggers of Impoverishment, and how they relate to the rest of the nominal structure. With (28) in place, the property [+I(mpoverishment)] from Section 2.1.4 can now be restated. I propose that the categorial feature [+D] triggers Impoverishment. Recall that the definite article moves to adjoin to Agr. Leaving out the positive/negative values of the CNG features, the general constellation is as in (29). Impoverishment Rule 1 deletes [S], marked by strikethrough, from Agr, the adjunction site. In fact, due to Percolation and Spec-head agreement, this feature is deleted from the feature bundles on all elements except the terminal head Art of the determiner itself. The relevant portion of the nominal structure is as follows (for the indefinite articles \textit{ein} and \textit{Ø}\textit{\textsubscript{D}}, see Sections 2.2 and 2.3): 
\end{styleStandard}

\begin{styleFooter}
(29)\ \ \textit{Impoverishment by Definite Article}
\end{styleFooter}

\begin{styleFooter}
\ \ \ \ \ \  \ \ \ \ \ AgrP\textsubscript{[F, N, O, S]}
\end{styleFooter}

\begin{styleFooter}
[Warning: Draw object ignored][Warning: Draw object ignored]
\end{styleFooter}

\begin{styleFooter}
\ \ \ \ InflP\textsubscript{[F, N, O, S]}\ \ \ \ Agr’\textsubscript{[F, N, O, S]}
\end{styleFooter}

\begin{styleFooter}
[Warning: Draw object ignored][Warning: Draw object ignored][Warning: Draw object ignored][Warning: Draw object ignored]
\end{styleFooter}

\begin{styleFooter}
\ \ \ \ \ \ \ [F, N, O, S]\ \  \ AP\ \ Agr\textsubscript{[F, N, O, S]}\ \ … t\textsubscript{k}…
\end{styleFooter}

\begin{styleFooter}
[Warning: Draw object ignored][Warning: Draw object ignored]\ \ \ \ 
\end{styleFooter}

\begin{styleFooter}
\ \ \ \  \ \ \ \ Art\textsubscript{[+D; +DEF][ F, N, O, S]k \ \ \ \ \ }Agr\textsubscript{[F, N, O, S]}
\end{styleFooter}

\begin{styleStandard}
[Warning: Draw object ignored][Warning: Draw object ignored]\ \ \ \ \ \ \ \ \ \ \ \ \ 
\end{styleStandard}

\begin{styleStandard}
\ \ \ \ \ \ \ \ \ [+D; +DEF] \ \ \ \ \ \ \ [F, N, O, S]
\end{styleStandard}

\begin{styleStandard}
Operations in DM occur after syntax. Given the successive-cyclic movement of the determiner from ArtP to DP, there are several copies of this element in the nominal structure. Each of these copies triggers Impoverishment deleting the feature [S]. This explains why multiple adjectives in a noun phrase all undergo Impoverishment (provided there is a determiner). Note also that after Impoverishment and Copy Reduction, there is still (at least) one fully specified feature bundle in the structure – it is on the terminal head Art later to be spelled out as the appropriate form of the determiner. Surfacing in the DP-level, this element is accessible to DP-external operations (e.g., case checking/valuing).
\end{styleStandard}

\begin{styleStandard}
\ \ The updated analysis of the demonstrative is similar, but this element undergoes adjunction to AgrP triggering Impoverishment (the demonstrative structure in (27) above is simplified here for reasons of space):
\end{styleStandard}

\begin{styleFooter}
(30)\ \ \textit{Impoverishment by Demonstrative}
\end{styleFooter}

\begin{styleFooter}
\ \ \ \ \ \  \ \ \ \ AgrP\textsubscript{[F, N, O, S]}
\end{styleFooter}

\begin{styleFooter}
[Warning: Draw object ignored][Warning: Draw object ignored]
\end{styleFooter}

\begin{styleFooter}
\ \  \ Dem\textsubscript{[+D; +DEIX][F, N, O, S]k }\ \  \ \ \ \ AgrP\textsubscript{[F, N, O, S]}
\end{styleFooter}

\begin{styleFooter}
[Warning: Draw object ignored][Warning: Draw object ignored][Warning: Draw object ignored][Warning: Draw object ignored]
\end{styleFooter}

\begin{styleFooter}
[+D; +DEIX] \ [F, N, O, S]\ \ InflP\textsubscript{[F, N, O, S]}\ \ \ \ Agr’\textsubscript{[F, N, O, S]}
\end{styleFooter}

\begin{styleFooter}
[Warning: Draw object ignored][Warning: Draw object ignored][Warning: Draw object ignored][Warning: Draw object ignored]
\end{styleFooter}

\begin{styleFooter}
\ \ \ \ \ \ [F, N, O, S]\ \  \ AP\ \ Agr\textsubscript{[F, N, O, S]}\ \ … t\textsubscript{k}…\ \ 
\end{styleFooter}

\begin{styleStandard}
Finally, I turn to the discussion of the inflection on the adjective, the target of Impoverishment.
\end{styleStandard}

\begin{styleStandard}
\ \ If Impoverishment applies, and the feature [S] is deleted in the nominal structure, then this impacts the inflection on the adjective with visible effect – InflP in (29) and (30) undergoes feature deletion. For instance, the removal of the feature [S] yields (31) in the dative masculine. With [S] absent, only [] in (23b) above is a match here spelling out the weak ending -\textit{en}:
\end{styleStandard}

\begin{styleFootnote}
(31)\ \ \ \  InflP
\end{styleFootnote}

\begin{styleFootnote}
[Warning: Draw object ignored][Warning: Draw object ignored]
\end{styleFootnote}

\begin{styleFootnote}
\ \  \ \ \  Infl’
\end{styleFootnote}

\begin{styleFootnote}\itshape
[Warning: Draw object ignored][Warning: Draw object ignored]
\end{styleFootnote}

\begin{styleStandard}
\ \  \ \ \ \ [-F, -N, +O]\ \  \ AP
\end{styleStandard}

\begin{styleStandard}\itshape
\ [Warning: Draw object ignored][Warning: Draw object ignored]
\end{styleStandard}

\begin{styleStandard}
\ \ \ \ \ \ \ \  \ \ \  \ \ A’
\end{styleStandard}

\begin{styleStandard}\itshape
[Warning: Draw object ignored][Warning: Draw object ignored]\ \ \ \ \ \ \ \ \ \ \ \ 
\end{styleStandard}

\begin{styleStandard}
\ \ \ \ \ \ \ \  \ \ A\ \ \ \ 
\end{styleStandard}

\begin{styleStandard}
\ \ \ \ \ \  \ \ \ \ \ \ \ \ \ \ \ \textit{klein\ \ \ \ \ \ }→\ \ \textit{klein}{}-\textit{en}
\end{styleStandard}

\begin{styleStandard}
I discuss the inflections on the determiners themselves in Section 4. Before moving on, note again that all strong endings in (22) have the feature [S], except for one: -\textit{en}. In other words, the accusative masculine ending is left untouched by Impoverishment and surfaces as the (so-called) exceptional case in strong and weak contexts alike. In the current account, this inflection is not exceptional: like the vocabulary entries in (23), -\textit{en} in (22c) lacks the feature [S]; unlike the entries in (23), -\textit{en} in (22c) is (simply) more specified as regards other features. 
\end{styleStandard}

\begin{styleStandard}
The above discussion explains the weak inflections in canonical constructions. Importantly, note that the trigger of Impoverishment, the categorial feature [+D], is on the stem of the determiner (rather than on the inflection). In other words, the current account of the strong/weak alternation is independent of the presence of adjectival inflections on determiners. With this in place, I turn to the discussion of strong adjectives in canonical structures. In the next section, I address adjectives in the context of \textit{ein}{}-words. In Section 2.3, I discuss adjectives following null articles and Saxon Genitives. Section 2.4 is dedicated to the discussion of more complex canonical DPs – those involving extended-adjective constructions and numerals. It is shown that these cases also follow from the system developed above. In the course of the discussion, I refine the current proposal further.
\end{styleStandard}

\begin{styleFooter}
\textit{2.2.\ \ Strong or Weak Adjectives in Canonical DPs: }Ein\textit{{}-words}
\end{styleFooter}

\begin{styleFooter}
In this section, I provide my analysis of strong adjectives after certain \textit{ein}{}-words, and I discuss the inflectional behavior of the \textit{ein}{}-words themselves.
\end{styleFooter}

\begin{styleFooter}
2.2.1. Adjectives after \textit{ein}{}-words: Strong Inflections
\end{styleFooter}

\begin{styleFooter}
In this subsection, I consider canonical DPs that involve adjectives preceded by \textit{ein}{}-words. Recall that \textit{ein}{}-words consist of the indefinite article \textit{ein} ‘a’, the singularity numeral \textit{EIN} ‘one’, the negative article \textit{kein} ‘no’, and possessive articles like \textit{mein} ‘my’, \textit{dein} ‘your’, etc. As is well known, these elements occur with strong adjectives in three instances: in the nominative masculine and in the nominative/accusative neuter. This is exemplified in the neuter with an indefinite article and a possessive article in (32a). In all the other instances, the adjective must be weak as shown with the dative in (32b):
\end{styleFooter}

\begin{styleFootnote}
(32)\textit{ \ \ }a.\ \ \textit{(m-)ein groß-es Auto}
\end{styleFootnote}

\begin{styleFootnote}
\ \ \ \ (my) a \ big-\textsc{st} \ \ car.\textsc{neut}
\end{styleFootnote}

\begin{styleFootnote}
\ \ \ \ ‘a / my big car’
\end{styleFootnote}

\begin{styleFootnote}
b.\ \ \textit{mit} \ \ \textit{(m-)einem groß-en Auto}
\end{styleFootnote}

\begin{styleFootnote}
\ \ \ \ with (my) a \ \ \ \ \ \ big-\textsc{wk} car.\textsc{neut}
\end{styleFootnote}

\begin{styleFootnote}
\ \ \ \ ‘with a / my big car’
\end{styleFootnote}

\begin{styleFootnote}
This inflectional distribution is often referred to as mixed pattern. Considering (32), it is clear that the presence of possessive \textit{m}{}- does not make a difference for the inflection on the adjective. Furthermore, given this pattern, I need to say something about the determiners that in some instances, do not trigger Impoverishment (32a) but in others, do (32b). 
\end{styleFootnote}

\begin{styleFootnote}
Note first that it is not possible to propose that (32a) involves no concord in features, and consequently, there is a strong ending on the adjective. This is so as uninflected \textit{ein} is restricted to specific featural combinations, and the following adjective has a regular strong ending. In other words, the agreement features need to be present underlyingly so that the correct forms of the article and the adjectival inflection can be inserted.\footnote{\ Note in this regard that the generalization Weak After Strong makes crucial reference to the agreement morpheme of a determiner. In other words, determiners are not unanalyzed word forms that involve some abstract features independently of the overt inflection. As discussed above, I also assume that determiners consist of morphemes. However, I argue that the underlying features are crucial, not the overt inflection.} More generally, this means that concord is not a sufficient condition for the occurrence of weak endings (see also Section 3). In order to explain the distribution in (32), I do not modify the general system laid out above but derive these patterns by a certain property of the relevant \textit{ein}{}-words themselves. I focus on \textit{ein} and later briefly comment on possessive articles, the negative article \textit{kein} ‘no’, and the singularity numeral \textit{EIN} ‘one’.
\end{styleFootnote}

\begin{styleFootnote}
\ \ Some scholars claim that uninflected \textit{ein} is, in a sense, invariable: for some reason, it does not have an ending in the three above-mentioned cases (Demske 2001: 33, Eisenberg 1999: 233, Olsen 1991b: 47 fn. 14). Consequently, the following adjective must be strong to spell out the relevant features for case, number, and gender. These authors seem to suggest that there are two types of lexical items: a determiner involving uninflected \textit{ein}, and a pronoun involving inflected \textit{einer} (more on this in the next subsection).\footnote{\ Note that it is not possible to simply claim that the adjectival inflection on \textit{ein} emerges to make case, number, and gender features visible. This is so because there are cases where the adjectival inflection is optional (e.g., \textit{ein lila(nes) Kleid} ‘a purple dress’) and instances where no CNG features are visible (e.g., \textit{zehn (Kleider)} ‘ten (dresses)’).} Other authors propose a timing mechanism to explain the three exceptional instances of \textit{ein}. For instance, Roehrs (2009a: chapter 4) suggests that certain cases of \textit{ein} move to the DP-level later in the derivation. As a consequence, they do not trigger Impoverishment on the adjective, and the ending on \textit{ein} itself is not licensed. In order to avoid such late syntactic movement, I make a different proposal here.
\end{styleFootnote}

\begin{styleFooter}
\ \ I break down the behavior of \textit{ein} into two issues: (i) \textit{ein} itself has no inflection, and (ii) the following adjective is strong (i.e., \textit{ein} does not trigger Impoverishment). At first glance, these points seem to be related (e.g., Murphy 2018 and references cited therein). Indeed, they have given rise to the traditional generalizations discussed in Section 2.1.1: Weak After Strong and the Principle of Monoinflection. Note though that we have already indicated that these generalizations do not cover all the facts (e.g., \textit{wir nett-en Studenten} ‘we nice-\textsc{wk} students’; for a more detailed discussion, see Chapter 3, Section 7). Second, generalizations are descriptions of the facts, not explanations.
\end{styleFooter}

\begin{styleFooter}
\ \ Starting with the second issue (i.e., the strong inflection on the adjective after \textit{ein}), I proposed in the previous section that the categorial feature [+D] of the determiner is the trigger for Impoverishment. To explain the special properties of \textit{ein}, I refine this proposal by claiming that the cateroial feature [+D] of determiners may be a trigger for Impoverishment but only under certain (featural) conditions. In order to establish a natural group, I propose that four (not three) instances of \textit{ein} are special. Besides the three traditional cases (i.e., nominative masculine and nominative/accusative neuter), I add accusative masculine to the group.\footnote{\ There are two points that deserve mentioning here. First, grouping the accusative masculine of \textit{ein} with the (traditional) three exceptions means that -\textit{en} on the adjective is a strong (rather than weak) ending, an interpretation that is possible given that -\textit{en} is both a strong and weak ending in this feature combination. Indeed, I formulated a vocabulary insertion rule in (22c) above that captures this identity directly. Second, Paul \textit{et al.} (1989: 234) provide the following forms of \textit{ein} in the nominative/accusative singular in MHG: \par (i)\ \ \ \ MASC\ \ NEUT\ \ FEM\par \ \ \ \ NOM\ \ ein\ \ ein\ \ ein\ \ \ \ \ \ (MHG)\par \ \ \ \ ACC\ \ ein-en\ \ ein\ \ ein(e)\par It appears as if over time, the feminine form of \textit{ein} has taken on -\textit{e}, possibly to mark feminine gender more consistently (cf. \textit{dies-e Lamp-e} ‘this lamp’; for discussion involving markedness and syncretism formation, see Bittner 2006 and Krifka 2009). This change in the feminine may have led to the establishment of a natural group containing the four other instances of \textit{ein} in (i), as suggested in the main text.} In the system laid out above, nominative/accusative masculine and nominative/accusative neuter are the least marked items in terms of the features [O] and [F]: both have a negative value in these four instances (recall from Section 2.1.5 that I take negative values to represent less marked instances). 
\end{styleFooter}

\begin{styleFooter}
I propose that [+D] is only a trigger for Impoverishment if it occurs in the context of positively valued [O], [F], [DEF], or [DEIX]. Note now that the terminal heads that are later to be spelled out as the four relevant instances of \textit{ein} have [+D] but lack [DEF], [DEIX]. Furthermore, they have no positive value on [O] and [F]. Consequently, these cases of \textit{ein} will not trigger Impoverishment. This is illustrated with \textit{ein} in the nominative masculine:
\end{styleFooter}

\begin{styleFooter}
(33)\ \ \textit{No} \textit{Impoverishment by }ein
\end{styleFooter}

\begin{styleFooter}
\ \ \ \ \ \  \ \ \ \ \ AgrP\textsubscript{[-F, -N, -O, -S]}
\end{styleFooter}

\begin{styleFooter}
[Warning: Draw object ignored][Warning: Draw object ignored]
\end{styleFooter}

\begin{styleFooter}
\ \ \ \ InflP\textsubscript{[-F, -N, -O, -S]}\ \ Agr’\textsubscript{[-F, -N, -O, -S]}
\end{styleFooter}

\begin{styleFooter}
[Warning: Draw object ignored][Warning: Draw object ignored][Warning: Draw object ignored][Warning: Draw object ignored]
\end{styleFooter}

\begin{styleFooter}
\ \ \ \ \ \ [-F, -N, -O, -S]\ \  \ AP\ \ Agr\textsubscript{[-F, -N, -O, -S]}\ \ … t\textsubscript{k}…
\end{styleFooter}

\begin{styleFooter}
[Warning: Draw object ignored][Warning: Draw object ignored]\ \ \ \ 
\end{styleFooter}

\begin{styleFooter}
\ \ \ \  \ \ \ \ Art\textsubscript{[+D][-F, -N, -O, -S]k \ \ \ \ \ \ \ \ \ \ \ \ \ \ }Agr\textsubscript{[-F, -N, -O, -S]}
\end{styleFooter}

\begin{styleStandard}
[Warning: Draw object ignored][Warning: Draw object ignored]\ \ \ \ \ \ \ \ \ \ \ \ \ 
\end{styleStandard}

\begin{styleStandard}
\ \  \ \ \ [+D]\ \  \ \ \ \ \ \ \ [-F, -N, -O, -S]
\end{styleStandard}

\begin{styleFooter}
To be clear, like the other determiners, the four instances of \textit{ein} undergo movement up the syntactic representation by way of adjunction. Unlike the other determiners, they do not trigger Impoverishment given the feature specification under Art. In keeping with the tradition, I usually refer to the special cases as the three (not four) instances of \textit{ein}.
\end{styleFooter}

\begin{styleFooter}
2.2.2. Adjectives After \textit{ein}{}-words: Uninflected \textit{ein}
\end{styleFooter}

\begin{styleFooter}
Turning to the first issue (i.e., \textit{ein} may have no inflection), note that the endings on \textit{ein} do appear in elliptical contexts when no noun or adjective is present. Compare (34a-b) to (34c):\footnote{\ Observe that ellipsis is irrelevant for Impoverishment; that is, the ellipsis of a noun in a regular DP does not impact the strong/weak alternation on the adjective itself:\par (i)\ \ \textit{der gut-e}\par \ \ \ \ the good-\textsc{wk}\par \ \ \ \ ‘the good one’\par Also, recall from Chapter 1, Footnote Error: Reference source not found that \textit{der}{}-words used pronominally are different from \textit{ein} in (34c) in that they have two inflections in certain feature combinations, the usual strong inflection and an additional -\textit{en} (e.g., \textit{d-en-en} ‘the-\textsc{st}{}-\textsc{infl} in the dative plural). This means that the additional strong inflection on \textit{ein} in (34c) has its own explanation.}
\end{styleFooter}

\begin{styleFooter}
(34)\ \ a.\ \ \textit{Das ist ein-(*er) Wagen}.
\end{styleFooter}

\begin{styleFooter}
\ \ \ \ this is \ a-*\textsc{st} \ \ \ \ \ \ car.\textsc{masc}
\end{styleFooter}

\begin{styleFootnote}
\ \ \ \ ‘This is a car.’
\end{styleFootnote}

\begin{styleFooter}
\ \ b.\ \ \textit{Das ist ein-(*er) gut-er}.
\end{styleFooter}

\begin{styleFooter}
\ \ \ \ this is \ a-*\textsc{st} \ \ \ \ \ \ good-\textsc{st}
\end{styleFooter}

\begin{styleFootnote}
\ \ \ \ ‘This is a good one.’
\end{styleFootnote}

\begin{styleFooter}
\ \ c.\ \ \textit{Das ist ein-*(er)}.
\end{styleFooter}

\begin{styleFooter}
\ \ \ \ this is \ \ one-\textsc{st} \ \ \ \ \ \ 
\end{styleFooter}

\begin{styleFootnote}
\ \ \ \ ‘This is one.’
\end{styleFootnote}

\begin{styleFootnote}
With Panagiotidis (2002, 2003) and others, I assume that (34b-c) involve null nouns (cf. also \textit{pro} as in Kester 1996a,b; Lobeck 1995; Olsen 1987; Rehn 2019: 209-12). This makes these two nominals structurally parallel to (34a), which involves an overt noun.
\end{styleFootnote}

\begin{styleFootnote}
\ \ Inflected \textit{einer} as in (34c) can be followed by overt elements provided these elements are outside the noun phrase that contains \textit{ein} and the (null) matrix noun. Such elements include genitive DPs, PPs, and relative clauses:
\end{styleFootnote}

\begin{styleFootnote}
(35)\ \ a.\ \ \textit{ein-er \ mein-er \ Freunde}
\end{styleFootnote}

\begin{styleFootnote}
\ \ \ \ one-\textsc{st} my-\textsc{gen} friends
\end{styleFootnote}

\begin{styleFootnote}
\ \ \ \ ‘one of my friends’
\end{styleFootnote}

\begin{styleFootnote}
\ \ b.\ \ \textit{ein-er \ von mein-en Freunden}
\end{styleFootnote}

\begin{styleFootnote}
\ \ \ \ one-\textsc{st} of \ \ my-\textsc{dat} friends
\end{styleFootnote}

\begin{styleFootnote}
\ \ \ \ ‘one of my friends’
\end{styleFootnote}

\begin{styleFootnote}
\ \ c.\ \ \textit{ein-er, den ich gesehen habe}
\end{styleFootnote}

\begin{styleFootnote}
\ \ \ \ one-\textsc{st} that I \ \ \ seen \ \ \ \ \ \ have
\end{styleFootnote}

\begin{styleFootnote}
\ \ \ \ ‘one that I have seen’
\end{styleFootnote}

\begin{styleFooter}
I summarize the complete set of \textit{ein}{}-words in Table 5, where the inflections in brackets indicate the forms occurring in the absence of the relevant overt material (e.g., adjectives and nouns in the matrix nominals). Given the restrictions on the occurrence of \textit{ein} in the plural (Chapter 1, Section 2.2), I illustrate this with the possessive article \textit{mein}{}- ‘my’:\footnote{\ The two neuter forms of \textit{meines} are often realized without a schwa as \textit{meins} (the same goes for the other \textit{ein}{}-word forms in these instances: \textit{eins} ‘a’, \textit{EINS} ‘one’, and \textit{keins} ‘no’). }
\end{styleFooter}

\begin{styleStandard}
Table 5: Endings on \textit{ein}{}-words
\end{styleStandard}

\begin{flushleft}
\begin{tabular}{|m{1.2511599in}|m{1.2511599in}|m{1.2511599in}|m{1.2511599in}|m{1.2518599in}|}

\hline
 &
\centering Masculine &
\centering Neuter &
\centering Feminine &
\centering\arraybslash Plural\\\hline
Nominative &
\centering mein-[er] &
\centering mein-[es] &
\centering mein-e &
\centering\arraybslash mein-e\\\hline
Accusative &
\centering mein-en &
\centering mein-[es] &
\centering mein-e &
\centering\arraybslash mein-e\\\hline
Dative &
\centering mein-em &
\centering mein-em &
\centering mein-er &
\centering\arraybslash mein-en\\\hline
Genitive &
\centering mein-es &
\centering mein-es &
\centering mein-er &
\centering\arraybslash mein-er\\\hline
\end{tabular}
\end{flushleft}
\begin{styleFooter}
Abstracting away from the brackets for a moment, the inflections in Table 5 are the same as those on the \textit{der}{}-words discussed earlier. Having said that, recall that the inflection on \textit{ein} in the three special instances is a function of the presence or absence of following overt material such as an adjective and/or a noun. Given the large overlap of forms, it is, in my view, undesirable to state two types of vocabulary items, one involving \textit{ein} as a determiner and one with \textit{einer} as a pronominal form. Rather, since the presence of the inflection depends on overt material, I propose that the alternation between uninflected and inflected \textit{ein} is brought about by certain vocabulary insertion rules. These are late operations that occur after lexical elements such as adjectives and nouns have been inserted.
\end{styleFooter}

\begin{styleFootnote}
Inspired by Höhn’s (2020: 13-16) analysis of the third-person gap in pronominal DPs (e.g., *\textit{they linguists}), I provide a – what he calls – PF-analysis, an account of uninflected \textit{ein} that occurs after syntax. I propose that the different forms of \textit{ein} present another case of contextually conditioned allomorphy. Höhn states that the relevant overt material must be linearly adjacent and in the same spell-out domain. As to the latter, I assume that the DP as a whole involves an independent domain – a phase in Chomsky (2000 and subsequent work). Rather than formulating three separate, exceptional instances of \textit{ein }in the nominative masculine and nominative/accusative neuter, I use the feature system from Section 2.1.5 and propose the following vocabulary insertion rules.
\end{styleFootnote}

\begin{styleFootnote}
The insertion rules are provided in (36): (36a-c) state the vocabulary entries for \textit{ein} and (36d-e) for the adjectival inflections on \textit{ein} (note that (36d) shows the entry for the nominative masculine inflection -\textit{er}, and (36e) stands for the remaining vocabulary entries of the endings from Section 2.1.5). As mentioned several times before, instances in the accusative masculine are exceptional in the nominal system in German. As such, I single out this feature combination as the most specific one (36a). In order to make sure that this element is inserted in the right feature combination, the context of the application of the rule is specified by the relevant features after the forward slash sign. Note also that (36a) only spells out the stem feature – the CNG features are spelled out by the vocabulary insertion rules for the adjectival inflections. In contrast, (36b) spells out the stem feature and the CNG features by one element. This explains the lack of inflection on \textit{ein} in these cases. Furthermore, (36b) is sensitive to the presence of overt material, the following adjective and/or noun, at the right edge of the same phase. This adheres to Höhn’s condition of linear adjacency. Given the relevance of word order, this is consistent with the assumption from Chapter 1 that Vocabulary Insertion follows Linearization. Notice that taken together, (36a) and (36b) form the four special cases of \textit{ein}. Finally, the vocabulary entry in (36c) presents the elsewhere case for \textit{ein} (the closing square bracket followed by $\Phi $ in (36b) indicates the right edge of the DP phase, more on this momentarily):\footnote{\ For the use of a forward slash sign to define the contextual condition of vocabulary items, see Embick \& Noyer (2007: 299). Also, the setup in (36a-c) is inspired by Blevins’ (1995: 146) treatment of weak adjectives in German where the form \textit{kleinen} ‘small’ is both the most and least specific, and \textit{kleine} is inbetween (also Evans 2019: 56). Note that in an earlier version of this work, I formulated (36a) as [+D][-F, -N, -O, +S] → \textit{einen}, where the latter spells out both the categorial feature and the CNG features by one element. Observe though that \textit{einen} would now be an unanalyzed form; that is, the ending -\textit{en} would no longer be a regular adjectival inflection. It is worth pointing out that the specification of a feature context after the forward slash sign as in (36a) will find wider application with inflected third-person pronouns (see next subsection), inflected definite articles (see Chapter 3, Section 3), and first and second-person pronouns, which do not have CNG features as part of their structure (Chapter 3, Section 5). Finally, note again that (36b) involves two separate feature bundles spelled out by one morpheme. This has to do with the earlier observation that articles, despite being heads, have internal structure consisting of stems and inflections. In some cases, these individual bundles are spelled out by one morpheme (36b); in others, they are realized by two morphemes where (36a,c) and (36d-e) are later combined by Local Dislocation (for sample derivations, see the main text below). That two feature bundles are spelled out by one morpheme is independently needed for (uninflected) third-person pronouns in the genitive (see next subsection). More generally, the different types of insertion rules in (36) will be used for other cases as well.}
\end{styleFootnote}

\begin{styleFootnote}
(36)\ \ a.\ \ [+D]\ \ \ \ \ \ → \ \ \textit{ein- }/ [-F, -N, -O, +S]
\end{styleFootnote}

\begin{styleFootnote}
\ \ b.\ \ [+D][-F, -O]\ \ \ \ → \ \ \textit{ein} / \_\_ word ]\textsubscript{$\Phi $}
\end{styleFootnote}

\begin{styleFootnote}
\ \ c.\ \ [+D]\ \ \ \ \ \ → \ \ \textit{ein}{}-
\end{styleFootnote}

\begin{styleFootnote}
d.\ \ [+F, -N, +O, $\alpha $S]\ \ →\ \ \textit{{}-er}
\end{styleFootnote}

\begin{styleFootnote}
\ \ e.\ \ etc.
\end{styleFootnote}

\begin{styleStandard}
Some more comments are in order. \ 
\end{styleStandard}

\begin{styleStandard}
I assume that a rule with specifications for both the categorial feature [+D] and the CNG features is more specific than a rule with either the categorial feature [+D] or the CNG features alone; that is, (36a-b) are more specific than (36c-e). Note again that (36a) is more specific in its feature specification than (36b): [-F, -N, -O, +S] vs. [-F, -O]. This means that (36a) applies first, provided its feature context matches that of the noun phrase. In conjunction with (36e), this brings about the inflected form \textit{einen}. If (36a) does not apply, (36b) may, again provided its features and context match the syntactic representation. Specifically, (36b) spells out uninflected \textit{ein} in the context of a following overt word inside the phase $\Phi $. However, if both (36a) and (36b) do not apply, (36c) does. In conjunction with (36d) or (36e), this brings about the remaining inflected forms of \textit{ein} including the ones in elliptical contexts. Note that the latter cases also concern the three exceptional instances where \textit{ein} surfaces with a strong inflection in elliptical contexts. Finally, observe that the insertion rules in (36a-c) involve three related vocabulary items and that there is no inherent difference as regards determiner vs. pronominal forms. Next, I provide more details as regards the morpho-syntax involved.
\end{styleStandard}

\begin{styleStandard}
\ \ After movement of \textit{ein} to the DP-level (provided the nominal is an argument), the simplified structure of the noun phrase can be represented as below:
\end{styleStandard}

\begin{styleFootnote}
(37)\ \ \ \ \ \  \ \ \ DP
\end{styleFootnote}

\begin{styleFootnote}
[Warning: Draw object ignored][Warning: Draw object ignored]
\end{styleFootnote}

\begin{styleFootnote}
[Warning: Draw object ignored][Warning: Draw object ignored]\ \ \ \  \ \ \  \ \ \ \ \ D’
\end{styleFootnote}

\begin{styleStandard}
\ \ \ \ \ \ D\ \  \ \ \  \ \ \ \ \ \ \ \ \ ArtP
\end{styleStandard}

\begin{styleStandard}\itshape
[Warning: Draw object ignored][Warning: Draw object ignored][Warning: Draw object ignored][Warning: Draw object ignored]\ 
\end{styleStandard}

\begin{styleStandard}
\ \  \ \ \ \ \ \ \ \ \ \ Art\textsubscript{k}\ \ \ \ D\ \ \ \  \ \ \ \ \ \ \ \ \ Art’
\end{styleStandard}

\begin{styleStandard}\itshape
[Warning: Draw object ignored][Warning: Draw object ignored][Warning: Draw object ignored][Warning: Draw object ignored]\ \ \ \ \ \ \ \ \ \ \ \ 
\end{styleStandard}

\begin{styleStandard}
\ \ \ \ \ \ \ \ \ \ [+D]\ \  \ \ \ \ [F, N, O, S]\ \  \ \ \ \ \ \ t\textsubscript{k}\ \  \ \ \ \ \ \ \ NumP
\end{styleStandard}

\begin{styleStandard}
[Warning: Draw object ignored][Warning: Draw object ignored]
\end{styleStandard}

\begin{styleStandard}
\ \ \ \ \ \ \ \ \ \ \ \ \ \ \ \  \ \ \ \ \ \ \ \ Num’
\end{styleStandard}

\begin{styleStandard}
[Warning: Draw object ignored][Warning: Draw object ignored]
\end{styleStandard}

\begin{styleStandard}
\ \ \ \ \ \ \ \ \ \ \ \ \ \  \ \ \ \ \ \ \ Num\ \ \ \ NP
\end{styleStandard}

\begin{styleStandard}
[Warning: Draw object ignored][Warning: Draw object ignored]\ \ \ \ \ \ \ \ \ \ \ \ \ \ \ \ \ \ \ \  \ t\textsubscript{i}
\end{styleStandard}

\begin{styleStandard}
\ \ \ \ \ \ \ \ \ \ \ \  \ \ \ \ \ \ \ \ \ N\textsubscript{i}\ \  \ \ \ \ \ \ \ Num\ \ \ \ 
\end{styleStandard}

\begin{styleStandard}
Like Höhn (2020), I assume that lexical elements like adjective and nouns are inserted before functional elements like determiners and inflections. This is consistent with Embick \& Noyer’s (2007: 295) assumption that lexical elements are merged as terminal nodes that involve phonological features.\footnote{\ This early presence of lexical elements in the derivation is presumably needed independently as nouns in German have inherent grammatical gender. Mediated by a syntactic mechanism that brings about concord in agreement features, this gender determines the form of the inflected determiner (e.g., \textit{der}\textsubscript{MASC} vs. \textit{das}\textsubscript{NEUT} vs. \textit{die}\textsubscript{FEM}).} Now, after Linearization, functional elements and lexical elements like, for instance, \textit{Wagen} ‘car’ or a null noun, yield strings such as in (38) below. To finalize the derivation, I highlight the interaction of the rules in (36) with the presence of overt material, specifically, the noun. I use examples in the nominative and dative masculine for illustration (I comment on (36a), which involves accusative masculine, at the end).
\end{styleStandard}

\begin{styleStandard}
First, when the features later to be spelled out as \textit{ein} are in the nominative masculine and appear in the context of an overt noun, insertion rule (36b) applies yielding the spell-out form after the arrow in (38a). Specifically, (36b) spells out the adjacent feature bundles for category and case, number, and gender by one element. Second, when the same features occur with a null noun, insertion rules (36c) and (36d) apply bringing about the spell-out in (38b). In contrast to the first instance, here both feature bundles of the article are realized separately, and the inflection undergoes Local Dislocation onto the determiner stem. Third, similar assumptions hold for the dative masculine. Here, (36c) and (36e) apply. The presence of a relevant overt element makes no difference in this feature combination (38c):
\end{styleStandard}

\begin{styleFooter}
[Warning: Draw object ignored][Warning: Draw object ignored][Warning: Draw object ignored](38)\ \ a.\ \ 
\end{styleFooter}

\begin{styleStandard}
\ \ \ \ [+D]\ \  \ \ \ [-F, -N, -O, -S]\ \ \textit{Wagen}\ \ \ \ → \ \ \textit{ein Wagen}
\end{styleStandard}

\begin{styleFooter}
[Warning: Draw object ignored][Warning: Draw object ignored][Warning: Draw object ignored]\ \ b.
\end{styleFooter}

\begin{styleStandard}
\ \ \ \ [+D]\ \  \ \ \ [-F, -N, -O, -S]\ \  \ \ \ e\textsubscript{N}\ \ \ \ → \ \ \textit{einer}
\end{styleStandard}

\begin{styleFooter}
[Warning: Draw object ignored][Warning: Draw object ignored][Warning: Draw object ignored]\ \ c.
\end{styleFooter}

\begin{styleStandard}
\ \ \ \ [+D]\ \  \ \ \ [-F, -N, +O, -S] \ \ \ \ \textit{Wagen}/e\textsubscript{N}\ \ → \ \ \textit{einem }(\textit{Wagen})
\end{styleStandard}

\begin{styleStandard}
Briefly commenting on (36a), it works as in (38c), only in a different featural context.
\end{styleStandard}

\begin{styleStandard}
To be clear, the vocabulary insertion rules only supply the phonetic form of \textit{ein} and that of the inflection – the abstract features have been present during the entire syntactic derivation. Furthermore, unlike adjectives and nouns as in (36b), elements outside the DP (genitive DPs, PPs, or relative clauses) are not part of the DP phase, delinated above by $\Phi $, and do not have an impact on the calculation of the right edge. This means that (36b) does not apply. Hence, genitive DPs, PPs, and relative clauses occur with inflected \textit{ein }if an adjective and/or noun is absent in the matrix DP. Before turning to the other \textit{ein}{}-words, I briefly discuss two extensions of the insertion rules in (36a-b).
\end{styleStandard}

\begin{styleFooter}
2.2.3. Adjectives after \textit{ein}{}-words: Some Extensions
\end{styleFooter}

\begin{styleStandard}
First, note that these insertion rules may also be relevant to another, very recent development in German. Vogel (2006) observes that the reduced form of the indefinite article \textit{‘n} is replaced in the nominative masculine and nominative/accusative neuter – the three exceptional cass – by \textit{nen}. This is particularly clear in chat contexts (notice that the example below makes it clear that \textit{nen} is not an accusative form):
\end{styleStandard}

\begin{styleFootnote}
(39)\ \ \textit{das \ waere \ \ \ \ nen gut-er \ \ \ preis}
\end{styleFootnote}

\begin{styleFootnote}
\ \ that would.be a \ \ \ good-\textsc{st} price.\textsc{masc}
\end{styleFootnote}

\begin{styleFootnote}
\ \ ‘That would be a good price.’
\end{styleFootnote}

\begin{styleFootnote}
Vogel proposes that speakers might take \textit{‘n} to be too short filling it with more phonetic material. With an added initial \textit{n} and schwa, this makes \textit{nen} more similar to other reduced forms of the article (e.g., accusative masculine \textit{nen}).
\end{styleFootnote}

\begin{styleFootnote}
\ \ I interpret \textit{nen} as the reduced form of \textit{einen} (see Chapter 5). With this in mind, notice that this form \textit{nen} occurs in the featural context of (36b), that is, in the masculine/neuter in the nominative and accusative cases. Thus, as an alternative to Vogel’s explanation, we could suggest that the featural context of the insertion rule (36a) has become less specific yielding the context in (36b): [+D] → \textit{(ei)n- }/ [-F, -O]. 
\end{styleFootnote}

\begin{styleFootnote}
Vogel also states that unreduced \textit{einen} is sometimes replaced by reduced \textit{ein} (rather than \textit{nen}) in the accusative masculine. This makes the accusative identical to the nominative here. This holds for Mannheim German, more generally. I argue in detail in Chapter 3, Section 8 that in this dialect, (36a) has been deleted. What seems to be clear then is that the featural contexts of (36a-b) – masculine/neuter in the nominative/accusative – are special. I take this as confirmation that these four instances of \textit{ein} form a natural group, in Standard German and in other varieties. We see in the next paragraphs that the forms of third person pronouns are also delineated by this feature combination.
\end{styleFootnote}

\begin{styleStandard}
\ \ The second extension of the analysis of \textit{ein} involves pronominal determiners (for fuller discussion, see Chapter 3, Section 5). As is well known, third-person personal pronouns have the same inflections as definite determiners. Consider Table 6. As mentioned in Section 2.1.5 above, there is syncretism in the nominative and accusative such that masculine and neuter pattern together, and feminine and plural do too: \textit{e-/ih}{}- ‘he, him; it’ vs. \textit{sie} ‘she, her; they, them’. Note that like with determiners (and adjectives), the accusative masculine form (\textit{ihn}) is exceptional. Also, all dative forms start in \textit{ih}{}-, and there is an additional ending (-\textit{en}) on the dative plural pronoun \textit{ihnen} (which I abstract away from here; for the historical development of this -\textit{en}, see Lühr 1991). Finally, notice that unlike the other forms, the genitive pronouns do not have regular inflections:
\end{styleStandard}

\begin{styleStandard}
Table 6: Third-Person Personal Pronouns in German
\end{styleStandard}

\begin{flushleft}
\begin{tabular}{|m{0.6920598in}|m{0.6601598in}|m{0.6601598in}|m{0.6601598in}|m{0.6601598in}|}

\hline
Gen/num &
\centering Masc &
\centering Neut  &
\centering Fem &
\centering\arraybslash PL\\\hline
Nom &
\centering e-r &
\centering e-s &
\centering sie &
\centering\arraybslash sie\\\hline
Acc &
\centering ih-n &
\centering e-s &
\centering sie &
\centering\arraybslash sie\\\hline
Dat &
\centering ih-m &
\centering ih-m &
\centering ih-r &
\centering\arraybslash ih-n-en\\\hline
Gen &
\centering seiner &
\centering seiner &
\centering ihrer &
\centering\arraybslash ihrer\\\hline
\end{tabular}
\end{flushleft}
\begin{styleStandard}
There are some interesting parallels to the distribution of \textit{ein}. 
\end{styleStandard}

\begin{styleStandard}
Recall that the three exceptional cases of \textit{ein }involve nominative masculine and nominative/accusative neuter, where \textit{ein} does not have an inflection if an adjective and/or noun follows. The corresponding third-person personal pronouns are also special: these three cases are the only ones that start in \textit{e}{}-, and as is well known, none of the third-person personal pronouns can be followed by an adjective and/or noun (e.g., \textit{er} \textit{(*Idiot)} ‘he (*idiot)’; Höhn 2020). In other words, both uninflected \textit{ein} and the stem \textit{e}{}- occur in exactly the same contexts, and both are sensitive to the overt material that follows them (although in opposite ways).
\end{styleStandard}

\begin{styleStandard}
\ \  Similar to the vocabulary insertion rules of \textit{ein}, I propose that the exceptional accusative masculine form \textit{ihn} ‘him’ is singled out as the most specific element (40a).\footnote{\ In Chapter 3, Section 5, I propose in detail that personal pronouns do not involve features for definiteness and deixis but rather for author (AUTH) and participant (PART) (for a full list of the vocabulary insertion rules for personal pronouns, see Chapter 3, Section 5.3.2).} Note that it only spells out the stem form (but not the CNG features; see also Gunkel \textit{et al}. 2017: 1297-98). In order to make sure that this form (and others like it) is inserted in the right feature combinations, an indication to that effect is provided after the forward slash sign. The genitive pronouns are the second most specified elements (40b-c). Unlike \textit{ihn} ‘him’ (and the other forms), they spell out the stem features and the CNG features by one element. This explains their irregular endings. The pronouns starting in \textit{e}{}- are the third most specified forms (40d). Note now that [-F, -O] is the same feature combination as that of uninflected \textit{ein}. As to the remaining rules, (40e) provides the insertion rule for the feminine/plural form in the nominative/accusative, and (40f) involves the elsewhere case. Observe that all vocabulary insertion rules specify that there cannot be an overt word in the right edge of the DP adopting an idea from Höhn (2020: 16). Finally, (40g) and (40h) specify the vocabulary insertion rules for the adjectival inflections (see (22) and (23) above; recall that the schwa in the insertion rules in (40g-h) is due to epenthesis, not applying here):
\end{styleStandard}

\begin{styleStandard}
(40)\ \ a.\ \ [+D; -AUTH, -PART] \ \ \ \ → \ \ \textit{ih-}\textsubscript{ }\ \ \ \ \ \ \ \ / \_\_]\textsubscript{$\Phi $} [-F, -N, -O, +S]
\end{styleStandard}

\begin{styleFootnote}
b.\ \ [+D; -AUTH, -PART]\ \ [ -F, +O, +S]\ \ → \ \ \textit{seiner \ }\textit{\textsubscript{\ }}\ / \_\_]\textsubscript{$\Phi $} 
\end{styleFootnote}

\begin{styleFootnote}
c.\ \ [+D; -AUTH, -PART][+O, +S]\ \ → \ \ \textit{ihrer \ }\textit{\textsubscript{\ \ \ \ }}\ / \_\_]\textsubscript{$\Phi $} 
\end{styleFootnote}

\begin{styleFootnote}
d.\ \ [+D; -AUTH, -PART]\ \ \ \ \ \ → \ \ \textit{e- \ }\textit{\textsubscript{\ \ \ \ \ \ \ \ \ \ \ }}\ / \_\_]\textsubscript{$\Phi $} [ -F, \ {}-O]
\end{styleFootnote}

\begin{styleStandard}
e.\ \ [+D; -AUTH, -PART]\ \ \ \ \ \ → \ \ \textit{sie-} \ \ \ \ \ \ \ / \_\_]\textsubscript{$\Phi $} [-O]
\end{styleStandard}

\begin{styleStandard}
f.\ \ [+D; -AUTH, -PART]\ \ \ \ \ \ → \ \ \textit{ih-}\textsubscript{ \ \ \ \ \ \ \ \ \ \ \ }\ / \_\_]\textsubscript{$\Phi $} \ \ 
\end{styleStandard}

\begin{styleStandard}
g.\ \ [+F, -N, +O, $\alpha $S] \ \ \ \ \ \ →\ \ \textit{{}-r}
\end{styleStandard}

\begin{styleStandard}
\ \ h.\ \ etc.
\end{styleStandard}

\begin{styleStandard}
Note that the extension of the discussion of \textit{ein} to third-person pronouns highlights several points. First, one morpheme can spell out two feature bundles at the same time (40b-c).\footnote{\ For another advantage of \textit{ein} spelling out two feature bundles, see Chapter 8, Section 2.1, where I briefly discuss Rehn’s (2019) analysis of adjectival inflections in Alemannic German.} Second, the exceptional accusative masculine form is part of the four special cases (masculine/neuter in the nominative/accusative). In order to capture this, the set-up of the vocabulary insertion rules in (36) and (40) is basically the same: (40a) corresponds to (36a), (40d) relates to (36b), and (40f) is the elsewhere case just like (36c). Finally, I comment on the other types of \textit{ein}{}-words.
\end{styleStandard}

\begin{styleFootnote}
Recall that all \textit{ein}{}-words behave the same as regards their inflection (also Chapter 5). It would be undesirable though to formulate the same type of vocabulary insertion rules in (36a-c) for possessive articles, the negative article, and the singularity numeral, respectively. Rather, the composite analysis of these \textit{ein}{}-words (consisting of \textit{ein} and another component) allows us to use the vocabulary items in (36a-c) here as well. While I provide a more detailed discussion in Chapter 5, I point out briefly here that possessive elements undergo some late interaction with \textit{ein} (i.e., after the\textit{ }insertion of \textit{ein}). Note in this regard that possessive articles fall into two groups: elements that are phonetically similar to \textit{ein}, and those that are not. Decomposing \textit{mein} ‘my’, \textit{dein} ‘your(\textsc{sgl})’, and \textit{sein} ‘his’ into \textit{ein} and possessive components, the latter parts surface as in (41a). The second type of the possessive articles involves free forms as in (41b):
\end{styleFootnote}

\begin{styleFootnote}
(41)\ \ a.\ \ \textit{m}{}-, \ \textit{d}{}-, \ \ \ \ \ \ \ \ \ \ \ \ \ \textit{s}{}-
\end{styleFootnote}

\begin{styleFootnote}
\ \ \ \ my, your(\textsc{sgl}), his
\end{styleFootnote}

\begin{styleFootnote}
\ \ b.\ \ \textit{ihr}, \ \ \ \ \ \ \ \ \ \ \ \ \ \ \ \ \ \ \ \ \ \ \ \ \ \ \ \ \ \ \ \ \ \textit{unser}, \textit{euer}
\end{styleFootnote}

\begin{styleFootnote}
\ \ \ \ her/their/your(\textsc{formal}), our, \ \ \ \ your(\textsc{pl})
\end{styleFootnote}

\begin{styleFootnote}
Recall also that \textit{ein} is proposed to be a semantically vacuous element. After Vocabulary Insertion, the elements in (41a) combine with \textit{ein}; that is, the possessive components are supported by \textit{ein}. The same holds for the negative article \textit{kein} ‘(NEG+a =) no’ and the singularity numeral \textit{EIN} ‘(\textit{Ø}\textsubscript{[-}\textsc{\textsubscript{pl}}\textsubscript{]}+a =) one’. As for (41b), I assume that these forms suppress the pronunciation of \textit{ein }(for more details, see Chapter 5). In the next section, I turn to another set of cases where strong or weak adjectives are possible. 
\end{styleFootnote}

\begin{styleFootnote}
\textit{2.3. \ \ Strong or Weak Adjectives in Canonical DPs: Null Articles and Saxon Genitives}
\end{styleFootnote}

\begin{styleFootnote}
I briefly argued in Chapter 1, Section 4.1.2 that prenominal possessives are in Spec,DP. This means that possessives can structurally occur with articles. In the previous section, this is what I tacitly assumed for possessive articles where the possessive component (e.g., \textit{m}{}-) is in Spec,DP and \textit{ein} is in D yielding \textit{mein} ‘my’ (more on this in Chapter 5). Similarly, I suggested in Chapter 1 that Saxon Genitives are also in Spec,DP. Unlike possessive articles, they do not occur with an overt determiner, and I assumed that Saxon Genitives are followed by a null article. 
\end{styleFootnote}

\begin{styleFootnote}
There are several advantages of this idea: on the one hand, all possessives occur with articles; on the other hand, we can capture the fact that adjectives that occur after Saxon Genitives behave in the same way as adjectives that are not preceded by an (overt) determiner or determiner-like element. This is illustrated with the nominative in (42a) and with the dative in (42b):
\end{styleFootnote}

\begin{styleFootnote}
(42)\textit{ \ \ }a.\ \ \textit{(Marias) kalt-es \ Bier}
\end{styleFootnote}

\begin{styleFootnote}
\ \ \ \ \textsubscript{ }Mary’s \ \ cold-\textsc{st} beer.\textsc{neut}
\end{styleFootnote}

\begin{styleFootnote}
\ \ \ \ ‘(Mary’s) cold beer’
\end{styleFootnote}

\begin{styleFootnote}
\ \ b.\ \ \textit{mit} \ \textit{(Marias) kalt-em Bier}
\end{styleFootnote}

\begin{styleFootnote}
\ \ \ \ with Mary’s \ \ cold-\textsc{st} beer.\textsc{neut}
\end{styleFootnote}

\begin{styleFootnote}
\ \ \ \ ‘with (Mary’s) cold beer’
\end{styleFootnote}

\begin{styleFootnote}
On my assumptions then, both (42a) and (42b) involve null articles. We can observe that like in the previous section, the presence of the possessive does not have an impact on the inflection of the adjective. However, unlike above, the adjective here is also strong in the dative. As such, Saxon Genitives are in stark contrast to possessive articles such as \textit{mein} ‘my’ (and the other \textit{ein}{}-words). Before proceeding, note also that \textit{wessen} ‘whose’ and \textit{dessen} ‘his’ behave like Saxon Genitives as seen in the dative:
\end{styleFootnote}

\begin{styleFootnote}
(43)\ \ a.\ \ \textit{mit} \ \textit{(wessen) kalt-em Bier}
\end{styleFootnote}

\begin{styleFootnote}
\ \ \ \ with whose \ \ \ cold-\textsc{st} beer.\textsc{neut}
\end{styleFootnote}

\begin{styleFootnote}
\ \ \ \ ‘with (whose) cold beer’
\end{styleFootnote}

\begin{styleFootnote}
\ \ b.\ \ \textit{mit \ (dessen) kalt-em Bier}
\end{styleFootnote}

\begin{styleFootnote}
\ \ \ \ with his \ \ \ \ \ \ \ cold-\textsc{st} beer.\textsc{neut}
\end{styleFootnote}

\begin{styleFootnote}
\ \ \ \ ‘with (his) cold beer’
\end{styleFootnote}

\begin{styleFootnote}
This means that \textit{wessen} ‘whose’ is morpho-syntactically related to Saxon Genitives and \textit{dessen} ‘his’ (but not to possessive articles). Note that this might be relatable to the proposal that unlike possessive articles, Saxon Genitives and \textit{dessen} do not involve composite forms (which consist of a possessive component and the article \textit{ein}). Be that as it may, given the different inflections on the following adjectives, Saxon Genitives and possessive articles must have a different analysis. Interestingly, the inflectional behavior of adjectives in the genitive is revealing here.
\end{styleFootnote}

\begin{styleFootnote}
\ \ As mentioned in Section 2.1.2, (overtly) unpreceded adjectives have strong inflections, with the exception of two instances: in genitive masculine/neuter contexts, these adjectives are weak. Note that the noun has an inflection in these very contexts. Compare (44a) to (44b). Once again, the presence of a Saxon Genitive does not make a difference:
\end{styleFootnote}

\begin{styleFootnote}
(44)\textit{ \ \ }a. \ *\ \ \textit{statt \ \ \ \ \ \ }\textit{\textsubscript{\ }}\textit{\ }\textit{\textsubscript{\ \ \ }}\textit{kalt-es \ Bier-es}
\end{styleFootnote}

\begin{styleFootnote}
\ \ \ \ instead.of \ cold-\textsc{st} bier.\textsc{neut}{}-\textsc{gen}
\end{styleFootnote}

\begin{styleFootnote}
\ \ \ \ ‘instead of cold beer’
\end{styleFootnote}

\begin{styleFootnote}
\textit{\ \ \ }b.\ \ \textit{statt \ \ \ \ \ \ }\textit{\textsubscript{\ }}\textit{\ (Marias) kalt-en \ \ Bier-es}
\end{styleFootnote}

\begin{styleFootnote}
\ \ \ \ instead.of Mary’s \ cold-\textsc{wk} bier.\textsc{neut}{}-\textsc{gen}
\end{styleFootnote}

\begin{styleFootnote}
\ \ \ \ ‘instead of Mary’s cold beer’
\end{styleFootnote}

\begin{styleFootnote}
Recall that the inflections on the determiners themselves do not undergo Impoverishment (i.e., they are strong). This means that the ending -\textit{es} on, for instance, the determiners \textit{des} ‘the’ or \textit{eines} ‘a’ in the genitive masculine/neuter, is directly related to the underlying features. Now, if the endings on the determiners and those on the adjectives are indeed the same, then -\textit{en} on the adjectives in the genitive masculine/neuter must be the result of Impoverishment.\footnote{\ Historically, adjectives in this context had the strong ending -\textit{es}. According to Demske (2001: 84), the weak ending -\textit{en} started to spread in the 15\textsuperscript{th} century, and according to Sahel (2021: 29), this change was basically completed during the 18\textsuperscript{th} century.}
\end{styleFootnote}

\begin{styleFootnote}
\ \ On the current analysis, a weak adjective in these contexts could imply the presence of a determiner triggering Impoverishment. One could suggest that this determiner is an(other) null article \textit{Ø}\textit{\textsubscript{D}} with the feature specification [+D][-F, +O, +S] yielding \textit{Ø-es}. Following Roehrs (2009a), another timing mechanism could be suggested. It could be claimed that the null article in the genitive masculine/neuter moves to the DP-level in the regular fashion triggering Impoverishment (the remaining instances of the null article would move later in the derivation). Since null articles cannot support overt suffixes, a repair mechanism could be formulated. It could be suggested that the lower copy of the null article combines with the head noun by partial N-raising, and the higher copy of the null article is deleted under Recoverability of Deletion: 
\end{styleFootnote}

\begin{styleFootnote}
(45)\ \ [\textsubscript{DP} \textit{Ø}{}-\textit{es} [\textsubscript{AgrP} \textit{kalten} [\textsubscript{ArtP} \textit{Bier}\textsubscript{k}\textit{{}-Ø}{}-\textit{es} [\textsubscript{NP} t\textsubscript{k} ]]]\ \ 
\end{styleFootnote}

\begin{styleFootnote}
Considering that nouns are usually not inflected for case in German, this derivation would explain the weak ending of the adjective and the presence of a suffix on the noun. However, I would like to avoid a timing mechanism involving late syntactic movement (cf. the discussion of \textit{ein} in the previous section). 
\end{styleFootnote}

\begin{styleFootnote}
Note that null articles cannot be triggers of Impoverishment in the current analysis. Like the other determiners, null articles have the categorial feature [+D] and, additionally, they have the feature [-DEF]. Unlike the other determiners, they do not have feature bundles for CNG. While the absence of CNG features avoids the issue of stranded affixes – after all, null elements cannot provide overt hosts for affixes, note that there are also no positively valued features for definiteness or deixis on the null article that make [+D] a trigger for Impoverishment. 
\end{styleFootnote}

\begin{styleFootnote}
\ \ I propose that Impoverishment in the current cases occurs independently of the presence of determiners. I formulate a second Impoverishment rule that has to do with the features on the terminal head Agr and its projections; that is, Impoverishment operates in a certain featural context. This rule says: When InflP is a daughter of AgrP, delete [S] on InflP in genitive masculine/neuter contexts:
\end{styleFootnote}

\begin{styleFootnote}
(46)\ \ \textit{Impoverishment Rule 2:}
\end{styleFootnote}

\begin{styleFootnote}
\ \ \ \  \ \ \ \ \ \ \ \ \ \ AgrP
\end{styleFootnote}

\begin{styleFootnote}
[Warning: Draw object ignored][Warning: Draw object ignored]
\end{styleFootnote}

\begin{styleFootnote}
\ \  \ \ \ \ \ \ \ \ \ InflP\ \ \ \ Agr’
\end{styleFootnote}

\begin{styleFootnote}
\ \ \ \ \ \ \ [-F, $\alpha $N, +O, +S]\ \  \ \ \ \ [-F, $\alpha $N, +O, +S]
\end{styleFootnote}

\begin{styleFootnote}
Notice that the feature specification in (46) excludes [+D]; that is, this rule applies independently of the presence of a determiner. In fact, it is the specific featural contexts and the structural relation of motherhood between AgrP and InflP that trigger the rule. Observe also that the same feature (i.e., [S]) is deleted here as in Impoverishment Rule 1.\footnote{\ In Chapter 3, Section 4, we see that this rule is more general applying to elements not only in AgrP but also to elements in other phrases. Note also that this rule has the looks of a dissimilation operation where one of two identical features (here [S]) is deleted (cf. Arregi \& Nevins’ 2012: 213 Participant Dissimilation). Interestingly, Impoverishment Rule 1 also deletes [S]. Assuming that the two Impoverishment rules are on the right track, it is currently not clear to me why [S] has this special status in German.}
\end{styleFootnote}

\begin{styleFootnote}
To illustrate the application of this rule, note first that the null article moves to adjoin to Agr but does not trigger Impoverishment. Rather, the features on Agr percolate to AgrP and induce Impoverishment on InflP by the rule in (46). This reduced feature set on InflP percolates to Infl:
\end{styleFootnote}

\begin{styleFooter}
(47)\ \ \ \ \ \  \ \ \ \ \ AgrP\textsubscript{[-F, $\alpha $N, +O, +S]}
\end{styleFooter}

\begin{styleFooter}
[Warning: Draw object ignored][Warning: Draw object ignored]
\end{styleFooter}

\begin{styleFooter}
\ \ \ \ InflP\textsubscript{[-F, $\alpha $N,+O, +S]}\ \ Agr’\textsubscript{[-F, $\alpha $N, +O, +S]}
\end{styleFooter}

\begin{styleFooter}
[Warning: Draw object ignored][Warning: Draw object ignored][Warning: Draw object ignored][Warning: Draw object ignored]
\end{styleFooter}

\begin{styleFooter}
\ \ [-F, $\alpha $N, +O, +S]\ \  \ AP \ Agr\textsubscript{[-F, $\alpha $N, +O, +S] \ \ \ }… t\textsubscript{k}…
\end{styleFooter}

\begin{styleFooter}
[Warning: Draw object ignored][Warning: Draw object ignored]\ \ \ \ 
\end{styleFooter}

\begin{styleFooter}
\ \ \ \ \ \ Art\textsubscript{k}\ \ \ \ Agr\textsubscript{[-F, $\alpha $N, +O, +S]}
\end{styleFooter}

\begin{styleStandard}
\ \ \ \ \ \ \ \ \ \ \ \ \ \ \ \ \ \ [+D; -DEF]
\end{styleStandard}

\begin{styleFootnote}
There are only three vocabulary items in (22) and (23) that do not involve [S]: two have a negative specification for [O], and one is the elsewhere case in (23b). The latter is the only vocabulary item that matches [-F, $\alpha $N, +O] in Infl yielding the desired ending -\textit{en} on the adjective. Some other remarks are in order.
\end{styleFootnote}

\begin{styleFootnote}
Null articles are in complementary distribution with other determiners. If a null article is present, then there is no overt determiner. Consequently, there is no (additional) Impoverishment (triggered by another determiner). Conversely, if an overt determiner is present, then there is no null article. However, in this scenario, there is an overlap in the application of the two Impoverishment rules. Given that feature deletion occurs postsyntactically, if Rule 1 applies first, Rule 2 does not (as [S] has already been removed in the entire structure); if Rule 2 applies first, Rule 1 still applies to the structural elements in AgrP (without visible effect), but not to the adjective in Spec,AgrP (where the feature [S] has already been removed by Rule 2). In either scenario, only the least specified element -\textit{en} can be inserted under Infl.
\end{styleFootnote}

\begin{styleFootnote}
Impoverishment Rule 2 is stated independently of the suffix on the noun. Two more remarks are in order here. First, there are nouns that do not take the genitive suffix -\textit{es}; for instance, \textit{des Faschismus(*-es)} ‘of fascism’ or \textit{des Jazz(*es)} ‘of Jazz’ (see also Olsen 1991b: 41). Furthermore, there is another type of genitive suffix on singular nouns: -\textit{en}. It is sometimes claimed (e.g., Rehn 2019: 116) that weak adjectival inflections also occur on nouns, specifically on so-called weak masculine nouns. Note that -\textit{en} occurs in all instances on these nouns except in the nominative singular – just like with weak adjectives in the masculine singular. However, there are some important differences. While the endings on these nouns are obligatory in the genitive singular, they are optional in the accusative/dative singular, at least for some speakers (note that Eisenberg 1998: 153 claims that this optionality holds more generally; Gallmann 1996: 289, 1998: 143 points out some interesting restrictions on this optionality).\footnote{\ Krischke (2012) notes that besides -\textit{en}, the endings -\textit{s}, -\textit{ens}, and quite rarely no ending can appear on weak masculine nouns in the genitive in non-standard contexts. He concludes that these forms are not established rival variants yet and are far from becoming part of the standard language (his page 66). Rather, they are due to analogy formations with similar forms in local contexts (\textit{locale Analogien}, his page 72). The relation of these genitives to accusative and dative forms is only very briefly discussed.} In contrast, weak endings on adjectives are not optional at all (and there is no variation). I conclude that these two cases should not be collapsed (although they are diachronically related, Chapter 1, Section 3.1.1). In other words, adjectival endings appear on determiners and adjectives (but not on nouns). Note that the Impoverishment rule in (46) is independent of these types of considerations.
\end{styleFootnote}

\begin{styleFootnote}
As to the second remark, note that although the genitive suffix -\textit{es} does not play a role in the current account of the strong/weak alternation, this does not mean that this suffix is synchronically not relevant. The so-called Genitive Rule states that a noun phrase in the genitive must contain at least one element that is sufficiently specific as regards case (i.e., it must contain -\textit{es} in the masculine/neuter or -\textit{er }in the feminine/plural). In conjunction with the analysis above, this explains the ungrammaticality of the following data where both weak and strong adjectives are ungrammatical with nouns that have -\textit{en} as their genitive suffix (48a-b) (data from Gunkel \textit{et al}. 2017: 1308). Indeed, this also accounts for the ungrammaticality of nouns that do not have a genitive ending at all (48c-d): 
\end{styleFootnote}

\begin{styleFootnote}
(48) \ \ \ \ a. \ \ * \ \ \textit{der Geruch gebraten-en Ochse-n}
\end{styleFootnote}

\begin{styleFootnote}
\ \ \ \ the smell \ \ \ \ fried-\textsc{wk} \ \ \ \ \ ox.meat.\textsc{masc}{}-\textsc{gen}
\end{styleFootnote}

\begin{styleFootnote}
\ \ \ \ ‘the smell of roast ox’
\end{styleFootnote}

\begin{styleFootnote}
b. \ \ * \ \ \textit{der Geruch gebraten-es Ochse-n}
\end{styleFootnote}

\begin{styleFootnote}
\ \ \ \ the smell \ \ \ \ fried-\textsc{st} \ \ \ \ \ \ ox.meat.\textsc{masc}{}-\textsc{gen}
\end{styleFootnote}

\begin{styleFootnote}
\ \ \ \ ‘the smell of roast ox’
\end{styleFootnote}

\begin{styleFootnote}
\ \ c. ?(?)\ \ \textit{eine Stunde toll-en \ \ \ \ Jazz}
\end{styleFootnote}

\begin{styleFootnote}
\ \ \ \ an \ \ \ hour \ \ \ great-\textsc{wk} jazz
\end{styleFootnote}

\begin{styleFootnote}
\ \ \ \ ‘an hour of great jazz’
\end{styleFootnote}

\begin{styleFootnote}
\ \ d. \ \ *\ \ \textit{eine Stunde toll-es \ \ \ Jazz}
\end{styleFootnote}

\begin{styleFootnote}
\ \ \ \ an \ \ \ hour \ \ \ great-\textsc{st} jazz
\end{styleFootnote}

\begin{styleFootnote}
\ \ \ \ ‘an hour of great jazz’
\end{styleFootnote}

\begin{styleFootnote}
Specifically, the weak adjectives in (48a,c) are ungrammatical as the Genitive Rule is violated (for more discussion of the Genitive Rule, see Chapter 3, Section 4);\footnote{\ Note that (48c) has a fairly good status. This presumably has to do with the fact that the second nominal in pseudo-partitive constructions can, for many speakers, be in all four morphological cases in German (Löbel 1989). Note in this regard that the second nominal in (48c) is morphologically ambiguous between genitive and accusative. } the strong adjectives in (48b,d) are ungrammatical as Impoverishment Rule 2 should have applied in this context.
\end{styleFootnote}

\begin{styleFootnote}
\ \ Before moving on, note that there are some special, rare cases where adjectives appear to have a strong inflection in genitive masculine/neuter contexts ((49a) is from Gunkel \textit{et al}. 2017: 1308, (49b) is from Durrell 2002: 126):
\end{styleFootnote}

\begin{styleFootnote}
(49)\ \ a.\ \ \textit{der Inhalt \ \ folgend-es \ \ \ \ Paragraph-en}
\end{styleFootnote}

\begin{styleFootnote}
\ \ \ \ the content following-\textsc{st} paragraph.\textsc{masc}{}-\textsc{gen}
\end{styleFootnote}

\begin{styleFootnote}
\ \ \ \ ‘the content of the following paragraph’
\end{styleFootnote}

\begin{styleFootnote}
b.\ \ \textit{das Gesuch obig-es \ \ \ Adressant-en} 
\end{styleFootnote}

\begin{styleFootnote}
\ \ \ \ the \ request above-\textsc{st} sender.\textsc{masc}{}-\textsc{gen}
\end{styleFootnote}

\begin{styleFootnote}
‘the request of the above sender’ 
\end{styleFootnote}

\begin{styleFootnote}
It is sometimes claimed that the weak ending -\textit{en} on the noun licenses the strong ending -\textit{es} on the adjective (Durrell 2002: 126). However, there are several points worth making here. Gunkel \textit{et al}. (2017: 1307) observe that such cases often involve adjectives that are similar to determiners (in their words “Pronomina”). Notice now that both head nouns in (49) are\textit{ }singular count nouns that seem to occur without articles. Interestingly, both \textit{folgend-} ‘following’ and \textit{obig-} ‘above(-mentionted)’ are definite in interpretation. As such, these elements might be special types of adjectives, sometimes called definite adjectives or adjectival determiners (see Roehrs 2009a: 167-68, van de Velde 2011; also Chapter 5, Section 5.2.4). Additionally, Gunkel \textit{et al}. (2017: 1308) point out that cases like (49b) often sound archaic. This might imply that such instances are not regulated by the contemporary grammar. If these remarks are on the right track, then (49) does not fall under the purview of Impoverishment Rule 2.
\end{styleFootnote}

\begin{styleFootnote}
\ \ To summarize the last two sections, I have discussed three (or rather four) instances where adjectives preceded by \textit{ein} have a strong ending (the remaining instances are weak) and two instances where adjectives without (overt) determiners are weak (the remaining instances are strong). The first set of cases was explained by exempting four instances of \textit{ein} from triggering Impoverishment; the second was accounted for by formulating another Impoverishment rule triggered by a certain featural context. Furthermore, I have documented that the presence of possessives, be they determiner-like elements like \textit{m(ein)} or Saxon Genitives like \textit{Marias}, does not have an impact on the inflection of a following adjective.
\end{styleFootnote}

\begin{styleFootnote}
Overall, I conclude that adjectival inflections are regulated by \textit{der}{}-words, \textit{ein}, and a certain featural context. If so, then I also have an account of why the possessives themselves cannot have an influence on the adjectival inflections – possessive elements such as \textit{m(ein)} ‘my’ and \textit{Marias} ‘Mary’s’ are not determiners themselves, but they co-occur with such elements, \textit{ein} in the case of \textit{m}{}- and \textit{Ø}\textit{\textsubscript{D}} in the case of \textit{Marias}. As stated in Chapter 1, possessives are determiner-like elements: they are like definite determiners in that they occur before adjectives and bring about definiteness; they are unlike determiners in that some of them (i.e., Saxon Genitives) may also occur after the head noun (e.g., \textit{die Werke Goethes} ‘the works of Goethe’). This explains why possessives in German do not regulate adjectival inflections (neither in prenominal nor postnominal position). In the next section, I discuss more complex canonical DPs.
\end{styleFootnote}

\begin{styleFooter}\itshape
2.4. \ \ Adjectives in Extended-Adjective Constructions and After Numerals
\end{styleFooter}

\begin{styleFooter}
If Impoverishment occurs in a local domain, then the question arises how adjectival inflections that appear to be deeply embedded can undergo this type of feature deletion. Consider the extended-adjective constructions in (50), where an adjective takes an argument. Interestingly, this argument is separated from the adjective by the degree word \textit{sehr} ‘very’. Despite the presence of these elements, the adjective alternates between exhibiting a strong or a weak ending: 
\end{styleFooter}

\begin{styleStandard}
(50)\textit{ \ \ }a.\ \ \textit{ein }[\textit{ auf seinen Sohn sehr stolz-er }]\textit{ Vater}
\end{styleStandard}

\begin{styleStandard}
\ \ \ \ an \ \ \ \textsubscript{\ }of \ \textsubscript{\ }his \ \ \ \ \ \ son \ \ very proud-\textsc{st} father.\textsc{masc}
\end{styleStandard}

\begin{styleStandard}
\ \ \ \ ‘a father very proud of his son’
\end{styleStandard}

\begin{styleStandard}
\textit{\ \ }b.\ \ \textit{der }[\textit{ auf seinen Sohn sehr stolz-e }]\textit{ \ \ \ Vater}
\end{styleStandard}

\begin{styleStandard}
\ \ \ \ the \ \ \ of \ his \ \ \ \ \ \ son \ \ very proud-\textsc{wk} father.\textsc{masc}
\end{styleStandard}

\begin{styleStandard}
\ \ \ \ ‘the father very proud of his son’
\end{styleStandard}

\begin{styleFooter}
As discussed in Chapter 1, Section 4.1.3, I follow, among many others, Corver (1991, 1997) in assuming that adjectives involve an extended projection, similar to verbs and nouns (Grimshaw 1991, van Riemsdijk 1998b). In particular, I assume that the extended projection of the adjective includes a Degree Phrase (DegP). Furthermore, assuming that theta-role assignment occurs in a local fashion (i.e., within AP), the intervening degree word implies that the argument must have moved to the left, stranding the adjective. This in turn implies that the extended projection of the adjective has more structure on top of DegP. Given that the adjective appears to be so deeply embedded in the structure, it is important to make explicit how the analysis above deals with the strong/weak alternation in this type of data.
\end{styleFooter}

\begin{styleFooter}
\ \ I proposed in Chapter 1, Section 4.1.3 that adjectives and their inflections are base-generated in separate positions (e.g., Corver 2006: 68, Leu 2015): while the adjective stem forms the bottom part of the extended projection, the inflection closes this structure off. Moving top-down, I reiterate the claim here that the inflection projects InflP. The head Infl involves the abstract feature bundle eventually spelled out as a strong or weak ending. Furthermore, I assume that there is a functional phrase (FP), which can form the landing site for the argument of the adjective. Finally, DegP is on top of AP. The basic structure is provided in (51a). Now, with the PP-argument moved to Spec,FP, I propose that FP moves to Spec,InflP. This is illustrated for the part enclosed in square brackets in (50b) in a simplified fashion as in (51b): 
\end{styleFooter}

\begin{styleStandard}
(51)\textit{ \ \ }a.\ \ [\textsubscript{InflP} Infl [\textsubscript{FP} F [\textsubscript{DegP} Deg [\textsubscript{AP} Adj ]]]] 
\end{styleStandard}

\begin{styleFooter}
\ \ b.\ \ \ \ \ \  \ \ \ \ \ InflP
\end{styleFooter}

\begin{styleFooter}
[Warning: Draw object ignored][Warning: Draw object ignored]
\end{styleFooter}

\begin{styleFooter}
\ \ \ \ \ \ FP\textsubscript{k}\ \ \ \ \ \  \ Infl’
\end{styleFooter}

\begin{styleFooter}
[Warning: Draw object ignored][Warning: Draw object ignored][Warning: Draw object ignored][Warning: Draw object ignored]
\end{styleFooter}

\begin{styleFooter}
\ \ \ \ PP\ \ \ \  \ \ F’\ \  \ \ {}-\textit{e}\ \  \ \ ... t\textsubscript{k} ...
\end{styleFooter}

\begin{styleFooter}
[Warning: Draw object ignored][Warning: Draw object ignored]\ \ \ \ \ [\textit{auf seinen Sohn}]\textsubscript{i}\ \ 
\end{styleFooter}

\begin{styleFooter}
\ \ \ \ \ \  \ \ F\ \ \ \ DegP
\end{styleFooter}

\begin{styleFooter}
[Warning: Draw object ignored][Warning: Draw object ignored]
\end{styleFooter}

\begin{styleFooter}
\ \ \ \ \ \ \ \ \textit{sehr}\ \ \ \ AP
\end{styleFooter}

\begin{styleFooter}
[Warning: Draw object ignored][Warning: Draw object ignored]
\end{styleFooter}

\begin{styleFooter}
\ \ \ \ \ \ \ \ \ \ \textit{stolz}\ \ \ \  \ t\textsubscript{i}
\end{styleFooter}

\begin{styleFooter}
After Linearization (and Vocabulary Insertion – recall that the tree representation in (51b) contains all the vocabulary items for expository convenience), we obtain the (abbreviated) string in (52). The inflection undergoes Local Dislocation onto the adjective yielding \textit{stolze} ‘proud’:
\end{styleFooter}

\begin{styleFooter}
[Warning: Draw object ignored][Warning: Draw object ignored][Warning: Draw object ignored](52)\ \ 
\end{styleFooter}

\begin{styleFooter}
\ \ … \textit{\ \ \ \ \ stolz} \ \ \ \ \ \ {}-\textit{e \ \ \ \ }→\ \ \textit{stolze}
\end{styleFooter}

\begin{styleFooter}
This derivation has a number of advantages. First, Impoverishment can occur in a local fashion – the inflection is in Infl and InflP itself is in Spec,AgrP just like in the cases discussed above. Second, the argument of the adjective has moved to the left. As such, the inflected adjective is, on the surface, adjacent to the head noun, a restriction that has been widely noted and is traditionally called Head-Final Filter (e.g., Williams 1982). If this is on the right track, then I can also account for some other cases.
\end{styleFooter}

\begin{styleFooter}
\ \  Based on work by van Riemsdijk (1998a: 673), Roehrs (2006a: 222) discusses some instances where the adjectival inflection is not on the adjective itself but on an element that is in the right periphery of the extended projection of the adjective. Like above, the inflection alternates:
\end{styleFooter}

\begin{styleStandard}
(53)\textit{ \ \ }a.\ \ \textit{ein }[\textit{ so schnell wie möglich-es }]\textit{ Aufräumen}
\end{styleStandard}

\begin{styleStandard}
\ \ a \ \ \ \ \ so quick \ \ \ as \ \ possible-\textsc{st} \ \ cleaning.\textsc{neut}
\end{styleStandard}

\begin{styleStandard}
\ \ \ \ ‘a cleaning as quick as possible’
\end{styleStandard}

\begin{styleStandard}
b.\ \ \textit{das }[\textit{ so schnell wie möglich-e }]\textit{ \ Aufräumen}
\end{styleStandard}

\begin{styleStandard}
\ \ the \ \ \ so quick \ \ \ as \ \ possible-\textsc{wk} cleaning.\textsc{neut}
\end{styleStandard}

\begin{styleStandard}
\ \ \ \ ‘the cleaning as quick as possible’
\end{styleStandard}

\begin{styleStandard}
As in the examples above, the question arises how the deeply embedded inflection can undergo Impoverishment on current assumptions. Furthermore, it is not clear how the inflection can occur on an element that is not the head of the AP in the first place.
\end{styleStandard}

\begin{styleStandard}
\ \ I propose that the discussion above can shed some light on these issues. Consider in more detail the portion in (53b) that is delineated by square brackets. The structure of this portion is illustrated in (54) below. I assume that the adjectival head cannot move out of the comparative structure, call it CompP. Abstracting away from the internal structure of CompP, the latter moves to Spec,InflP: 
\end{styleStandard}

\begin{styleStandard}
(54)\textit{ \ \ } \ \ \ \  \ \ \ \ \ InflP
\end{styleStandard}

\begin{styleFooter}
[Warning: Draw object ignored][Warning: Draw object ignored]
\end{styleFooter}

\begin{styleFooter}
\ \  \ \ \ \ \ \ \ CompP\textsubscript{k}\ \ \ \ \ \  \ Infl’
\end{styleFooter}

\begin{styleFooter}
\ \ \ \ \ \ \ \textit{so schnell wie möglich}[Warning: Draw object ignored][Warning: Draw object ignored]
\end{styleFooter}

\begin{styleFooter}
\ \ \ \ \ \ \ \  \ \ {}-\textit{e}\ \  \ \ ... t\textsubscript{k} ...
\end{styleFooter}

\begin{styleFooter}
After Linearization (and Vocabulary Insertion), we obtain (55). With \textit{möglich} ‘possible’ an adjective (at least by form), this provides an appropriate overt host for the inflection, and the latter can undergo Local Dislocation onto the adjective :
\end{styleFooter}

\begin{styleFooter}
[Warning: Draw object ignored][Warning: Draw object ignored][Warning: Draw object ignored](55)\ \ 
\end{styleFooter}

\begin{styleFooter}
\ \ … \textit{\ \ \ möglich} \ \ \ {}-\textit{e \ \ \ \ }→\ \ \textit{mögliche}
\end{styleFooter}

\begin{styleFooter}
This, then, allows Impoverishment not only to occur in a local fashion, just like above, but also explains the unexpected position of the adjectival inflection.\footnote{\ Another advantage of separating the inflection from the adjective stem is that it can account for the difference in German between attributive adjectives, which have an inflection, and predicative ones, which do not:\par (i)\ \ \textit{Das Haus ist klein}.\par \ \ \ \ the \ house is small\par \ \ \ \ ‘The house is small.’\par This can be captured by assuming that InflP is present in the former cases but not in the latter.} I return to adjectives as extended projections in the context of non-restrictive adjectives in Chapter 4, Section 4. 
\end{styleFooter}

\begin{styleFooter}
\ \ Above, I discussed adjectives that appear to be deeply embedded inside the specifier of AgrP. Next I turn to adjectives that are separated from the determiner by a numeral. In other words, the determiner and the adjective do not seem to be in a local relation either. Nevertheless, the adjective shows a weak ending:
\end{styleFooter}

\begin{styleFooter}
(56)\ \ \textit{dies-e \ \ \ \ zehn klein-en \ \ Autos}
\end{styleFooter}

\begin{styleFooter}
\ \ these-\textsc{st} ten \ \ small-\textsc{wk} cars
\end{styleFooter}

\begin{styleFooter}
\ \ ‘these ten small cars’
\end{styleFooter}

\begin{styleFooter}
Assuming successive-cyclic movement of determiners from ArtP to DP, I suggested above that (phrasal) demonstratives adjoin to AgrP on their way to Spec,DP. This is now illustrated in more detail with (57) below:
\end{styleFooter}

\begin{styleFooter}
[Warning: Draw object ignored]
\end{styleFooter}

\begin{styleFooter}
(57)\ \  DP
\end{styleFooter}

\begin{styleFootnote}
[Warning: Draw object ignored][Warning: Draw object ignored][Warning: Draw object ignored]
\end{styleFootnote}

\begin{styleFootnote}
\textit{diese\ \ }\ \  \ D’
\end{styleFootnote}

\begin{styleFootnote}
[Warning: Draw object ignored][Warning: Draw object ignored]
\end{styleFootnote}

\begin{styleFootnote}
\ \  \ \ D\ \ \ \ CardP
\end{styleFootnote}

\begin{styleFootnote}
[Warning: Draw object ignored][Warning: Draw object ignored]\ \  
\end{styleFootnote}

\begin{styleFootnote}
\ \  \ \ \ \ \ \ \ \ \ diese\textsubscript{i}\ \  \ \ CardP
\end{styleFootnote}

\begin{styleFootnote}\itshape
[Warning: Draw object ignored]\ \ \ \ \ \ \ \ \ \ [Warning: Draw object ignored][Warning: Draw object ignored]
\end{styleFootnote}

\begin{styleFootnote}
\ \ \ \ \ \ InflP\textsubscript{Q}\ \ \ \  Card‘
\end{styleFootnote}

\begin{styleFootnote}
[Warning: Draw object ignored][Warning: Draw object ignored]\ \ \ \ \ \ \textit{zehn}
\end{styleFootnote}

\begin{styleFootnote}
\ \ \ \ \ \  \ \ \ \ \ \ \ \ \ Card\ \ \ \  AgrP
\end{styleFootnote}

\begin{styleFootnote}\itshape
\ \ \ \ \ \ \ \ \ \ [Warning: Draw object ignored][Warning: Draw object ignored]
\end{styleFootnote}

\begin{styleFootnote}
[Warning: Draw object ignored][Warning: Draw object ignored]\ \ \ \ \ \ \ \  \ \ \ \ \ \ \ \ \ \ diese\textsubscript{i}\ \ \ \ AgrP
\end{styleFootnote}

\begin{styleFootnote}
InflP \ \ \ \ Agr’
\end{styleFootnote}

\begin{styleFootnote}
\textit{\ \ \ \ \ \ \ \ \ \ \  \ \ \ \ \ \ \ \ \ \ kleinen}[Warning: Draw object ignored][Warning: Draw object ignored]
\end{styleFootnote}

\begin{styleFootnote}
\ \ \ \ \ \ \ \ \ \ \ \  \ \ \ \ \ \ \ \ \ \ \ \  Agr \ \ \ \ \ \ \ \ \ \ \ \ \ \ \ \ ArtP
\end{styleFootnote}

\begin{styleFootnote}
[Warning: Draw object ignored][Warning: Draw object ignored]\ \ \ \ \ \ \ \ \ \ \ \ \ \  
\end{styleFootnote}

\begin{styleFootnote}
\ \ \ \ \ \ \ \ \ \ \ \ \ \  \ \ \ \ \ \ \ \ \ \ diese\textsubscript{i}\ \ \ \ Art’
\end{styleFootnote}

\begin{styleStandard}
[Warning: Draw object ignored][Warning: Draw object ignored]
\end{styleStandard}

\begin{styleStandard}
\ \ \ \ \ \ \ \ \ \ \ \ \ \ \ \ \ \ Art\ \  \ \ \ \ \ \ \ \ \ \ NumP
\end{styleStandard}

\begin{styleStandard}
\ \ \ \ \ \ \ \ \ \ \ \ \ \ \ \ \ \ \ \ \ \ \textit{Autos \ \ }
\end{styleStandard}

\begin{styleFooter}
This adjunction of the demonstrative to AgrP provides the local context required for the application of Impoverishment (see again Section 2.1.6). Consequently, the adjective surfaces with a weak ending. 
\end{styleFooter}

\begin{styleFooter}
\ \ To sum up, this section discussed adjectives in the context of canonical noun phrases. It was proposed that determiners, or more precisely their categorial features [+D], trigger Impoverishment Rule 1 such that fully specified feature bundles on terminal heads undergo feature deletion and are realized as less specified (weak) inflections. Impoverishment rule 2 works in a similar way except that it is triggered by genitive masculine/neuter contexts in AgrP. Making certain assumptions, all Impoverishment rules apply in local constellations. Strong adjectives occur when Impoverishment does not occur (e.g., features for definiteness or deixis are absent). In other words, these inflections spell out underlyingly fully specified features and present the elsewhere case. Next, I turn to the discussion of non-canonical structures to see how adjectival inflections fare in these contexts.
\end{styleFooter}

\begin{styleStandard}\bfseries
3.\ \ Adjectival Inflections in Non-canonical DPs
\end{styleStandard}

\begin{styleStandard}
In the previous section, I focused on the simple (canonical) cases of the schematic form “determiner + adjective(s) + noun”, where all these types of elements agree in case, number, and gender. In addition, I also discussed Saxon Genitive constructions, extended-adjective constructions, and adjectives after numerals. I proposed that all these cases involve regular DPs, with adjectives in Spec,AgrP and the determiner in the DP-level. These are the contexts where weak adjectives occur. In this section, I contrast those noun phrases with non-canonical constructions. The latter are argued to involve structures different from regular DPs and different from one another. Some independent evidence is provided for the individual structures (for more detailed discussion, see the original sources referenced below). 
\end{styleStandard}

\begin{styleStandard}
If weak adjectives occur in regular, simple DPs only, as proposed in the previous section, then there are two distinct options for analyzing DPs containing strong adjectives and relevant determiners: it could be claimed that either the determiner is in a position where it cannot have an impact on the adjective or that the adjective itself is outside the determiner’s regular domain of influence. Note that in each of these scenarios, the structural relation between the determiner and the adjective differs from that in regular DPs, and Impoverishment Rule 1 does not apply. The following discussion features both options and provides evidence for Hypothesis 2a (i.e., inflections indicate abstract structure).\footnote{\ There are two more options. As a third option, we could suggest that both the determiner and the adjective are in non-canonical positions. To keep things simple, I do not consider this possibility. Fourth, as seen in the following discussion, some non-canonical DPs involve no relevant determiner to begin with. Consequently, Impoverishment cannot occur in these cases (independent of the structure).} For expository clarity, the non-canonical position of either the determiner or the adjective is marked by square brackets around the relevant (complex) element in the tree diagrams. Finally, note that with the exception of indefinite pronoun constructions (Section 3.2), all cases involve definite contexts. This is clear given that many of these instances contain definite determiners. As such, weak endings on the adjectives are expected. As we will see though, this is not borne out. Again, adjectival inflections in German cannot be a reflex of the definiteness of the containing noun phrase (cf. Hypothesis 1a).
\end{styleStandard}

\begin{styleStandard}\itshape
3.1.\ \ Regular DP vs. Low Right-Adjunction: Close Appositions
\end{styleStandard}

\begin{styleStandard}
I begin with a canonical noun phrase (58a) and compare it to a noun phrase where a name-like nominal such as \textit{Großer Bär} ‘Great Bear’ occurs to the right of another nominal (58b). The second construction is an instance of close apposition, and it typically involves one intonation unit (e.g., Lekakou \& Szendrői 2007, Löbel 1991). It is clear that the first case involves a weak inflection on the adjective but that the second case must have a strong ending on the adjective:
\end{styleStandard}

\begin{styleStandard}
(58)\textit{ \ \ }a.\ \ \textit{der} \textit{groß-e(*r) \ Bär}
\end{styleStandard}

\begin{styleStandard}
\ \ \ \ the big-\textsc{wk/}*\textsc{st} bear.\textsc{masc}
\end{styleStandard}

\begin{styleStandard}
‘the big bear’
\end{styleStandard}

\begin{styleStandard}
\ \ b.\textit{\ \ das Sternbild \ \ \ \ \ Groß-e*(r) \ Bär}
\end{styleStandard}

\begin{styleStandard}
the constellation big-\textsc{st}/*\textsc{wk} bear.\textsc{masc}
\end{styleStandard}

\begin{styleStandard}
‘the constellation Great Bear’
\end{styleStandard}

\begin{styleStandard}
To repeat, regular, simple DPs as in (58a) involve one nominal; that is, they have one head noun. This is different for close appositions as in (58b). Here, this structure contains two nominals where the nominal on the left – often called anchor – contains a categorizing count noun, and the nominal on the right is a proper name or a uniquely referring noun phrase. Given these inflectional and syntactic differences, the cases in (58) are argued to involve different structures.
\end{styleStandard}

\begin{styleStandard}
\ \ Turning first to the canonical case in (58a), the relevant part of the structure is repeated in simplified form below: 
\end{styleStandard}

\begin{styleFootnote}
(59)\textit{ \ \ Regular DP-Structure (Simplified)}
\end{styleFootnote}

\begin{styleStandard}
\ \  DP
\end{styleStandard}

\begin{styleFootnote}
[Warning: Draw object ignored][Warning: Draw object ignored]
\end{styleFootnote}

\begin{styleFootnote}
[Warning: Draw object ignored]\textit{\ \ \ }\ \  \ D’
\end{styleFootnote}

\begin{styleFootnote}
[Warning: Draw object ignored][Warning: Draw object ignored]
\end{styleFootnote}

\begin{styleFootnote}
\ \  \ \ D\ \ \ \ AgrP
\end{styleFootnote}

\begin{styleFootnote}
[Warning: Draw object ignored][Warning: Draw object ignored]\ \  \textit{der}\textsubscript{i}
\end{styleFootnote}

\begin{styleFootnote}
[Warning: Draw object ignored]\ \  \ \ \ \ \ \ \ \ \ \ \ InflP\ \  \ \ Agr’
\end{styleFootnote}

\begin{styleFootnote}
\textit{\ \ \ \ \ \ \ \ \ \ }[Warning: Draw object ignored][Warning: Draw object ignored]\textit{\ \ große}
\end{styleFootnote}

\begin{styleFootnote}
\ \ \ \ \ \ Agr\ \ \ \  ArtP
\end{styleFootnote}

\begin{styleFootnote}
[Warning: Draw object ignored][Warning: Draw object ignored]\ \ \ \ \ \ der\textsubscript{i}
\end{styleFootnote}

\begin{styleFootnote}
\ \ \ \ \ \  \ \ \ \ \ \ \ \ \ \ \ \ \ \ \  Art’
\end{styleFootnote}

\begin{styleFootnote}\itshape
\ \ \ \ \ \ \ \ [Warning: Draw object ignored][Warning: Draw object ignored]
\end{styleFootnote}

\begin{styleFootnote}
\ \ \ \ \ \ \ \ \ \ Art\ \ \ \  \ NP
\end{styleFootnote}

\begin{styleFootnote}
\ \ \ \ \ \ \ \ \ \ der\textsubscript{i}\ \ \textit{\ \  Bär}
\end{styleFootnote}

\begin{styleFootnote}
As discussed in Section 2, I take this to be the structural constellation where weak adjectives come about. With the adjective strong in (58b), something else must be assumed to explain this example. 
\end{styleFootnote}

\begin{styleFootnote}
\ \ In certain northern dialects of German, proper names can take an optional proprial article as in \textit{(die) Anna} ‘Anna’ (Nübling \textit{et al} 2015: 123-28). As noted by Löbel (1991), this optional article is not possible in close appositions (60a), and the name has to follow the noun it delimits. Compare (60a) to (60b):\footnote{\ As a reviewer points out, (southern) dialects with an obligatory proprial article allow this type of apposition in (60a) (although the following genitive sounds somewhat formal). Note also that (60b) improves significantly if there is a long pause after \textit{Bruder} ‘brother’ (see the discussion of loose appositions in Section 3.4).}
\end{styleFootnote}

\begin{styleFootnote}
(60) \ \ a.\ \ \textit{die Tochter (*die) Anna meines Bruders}
\end{styleFootnote}

\begin{styleFootnote}
the daughter \ \ the \ Anna of.my \ brother
\end{styleFootnote}

\begin{styleFootnote}
‘Anna, my brother’s daughter’
\end{styleFootnote}

\begin{styleFootnote}
b. \ *\ \ \textit{die Tochter \ meines Bruders (die) Anna}
\end{styleFootnote}

\begin{styleFootnote}
the daughter of.my \ brother \ \ \textsubscript{\ }the \ Anna 
\end{styleFootnote}

\begin{styleFootnote}
Relating the close appositive in (60a) to the one in (58b) above, I propose for these cases a structure different from that of canonical DPs, namely the name-like element is adjoined to the NP of the larger DP. Abstracting away from movements internal to the matrix DP, this can schematically be represented as follows (for a symmetric structure, see also Lekakou \& Szendrői 2012; cf. Löbel 1991: 13):
\end{styleFootnote}

\begin{styleStandard}
(61)\textit{ \ \ Close Appositions}
\end{styleStandard}

\begin{styleStandard}
\ \ \ \ DP
\end{styleStandard}

\begin{styleStandard}
[Warning: Draw object ignored][Warning: Draw object ignored]
\end{styleStandard}

\begin{styleStandard}
\ \ D\ \ \ \ NP
\end{styleStandard}

\begin{styleStandard}
[Warning: Draw object ignored][Warning: Draw object ignored]\ \ \ \ \ \ \ \ \ \ \ \textit{das\ \ \ \  \ }
\end{styleStandard}

\begin{styleStandard}
\ \ \ \ NP\ \  \ \ \ \ \ \ [\textit{Großer Bär}]
\end{styleStandard}

\begin{styleStandard}
\ \ \ \  {\textbar}
\end{styleStandard}

\begin{styleStandard}
\ \ \ \ N\ \  \ \ \ 
\end{styleStandard}

\begin{styleStandard}
\ \  \ \ \ \ \ \textit{Sternbild\ \ \ \ }
\end{styleStandard}

\begin{styleStandard}
As to the inflection on the adjective, it is clear that the determiner did not adjoin to the phrase containing the adjective. Consequently, Impoverishment Rule 1 does not apply, and this accounts for the strong adjective.
\end{styleStandard}

\begin{styleStandard}\itshape
3.2.\ \ Regular DP vs. Mid Right-Adjunction: Indefinite Pronoun Constructions
\end{styleStandard}

\begin{styleStandard}
Roehrs (2008) argues that indefinite pronoun constructions involve several different types. With current purposes in mind, there are two types relevant here: one type exhibits a weak adjective (62a), and the other shows a strong adjective (62b-c):\footnote{\ Note that both \textit{wer} ‘someone’ and \textit{jemand} ‘someone’ can also be followed by an adjective ending in -\textit{es} (e.g., \textit{jemand anderes} ‘someone different’). As discussed above, -\textit{es} cannot be a regular genitive inflection on the adjective – such an ending would be -\textit{en} (for discussion and analysis of these cases, see Roehrs 2008). Also, \textit{wer anderer} in (62b) sounds rather marked in isolation (and there may also be some speaker variation). However, this string gets better in context (e.g., \textit{das macht wer anderer} ‘someone different will do that’, \textit{wer anderer als du} ‘someone/who else than you’; both examples were retrieved from the internet).} 
\end{styleStandard}

\begin{styleStandard}
(62)\textit{ \ \ }a.\ \ \textit{jed-er \ \ \ \ \ \ \ \ \ \ \ \ \ \ \ \ \ \ ander-e(}\textit{\textsuperscript{\%}}\textit{r)}
\end{styleStandard}

\begin{styleStandard}
\ \ \ \ every(one)-\textsc{masc} different-\textsc{wk}/\textit{\textsuperscript{\%}}\textsc{st}
\end{styleStandard}

\begin{styleStandard}
\ \ \ \ ‘everyone different’
\end{styleStandard}

\begin{styleStandard}
\ \ b. \ \%\ \ \textit{w-er \ \ \ \ \ \ \ \ \ \ \ \ \ \ \ \ \ ander-e*(r)}
\end{styleStandard}

\begin{styleStandard}
\ \ \ \ someone-\textsc{masc} different-\textsc{st}/*\textsc{wk}
\end{styleStandard}

\begin{styleStandard}
\ \ \ \ ‘someone different’
\end{styleStandard}

\begin{styleStandard}
\ \ c.\ \ \textit{jemand} \textit{\ \ \ \ \ \ \ \ \ \ \ \ \ ander-e*(r)}
\end{styleStandard}

\begin{styleStandard}
\ \ \ \ someone.\textsc{masc} different-\textsc{st}/*\textsc{wk}
\end{styleStandard}

\begin{styleStandard}
\ \ \ \ ‘someone different’
\end{styleStandard}

\begin{styleStandard}
It is argued in that paper that although both types in (62) involve concord between the pronominal element and the adjective, they involve different structures. As also briefly mentioned in the context of uninflected \textit{ein }(Section 2.2.1), this means that concord can only be a necessary, but not sufficient, condition on the occurrence of weak adjectives. 
\end{styleStandard}

\begin{styleStandard}
Turning to the individual structures, I point out first that there are other important differences between these two types. Among others, the first type allows an overt noun, but the second crucially does not:
\end{styleStandard}

\begin{styleStandard}
(63)\textit{ \ \ }a.\ \ \textit{jeder \ \ \ \ \ \ \ \ \ ander-e \ \ \ \ \ \ \ \ Mann}
\end{styleStandard}

\begin{styleStandard}
\ \ \ \ every(one) different-\textsc{wk} man.\textsc{masc}
\end{styleStandard}

\begin{styleStandard}
\ \ \ \ ‘every different man’
\end{styleStandard}

\begin{styleStandard}
\ \ b. *\ \ \textit{wer \ \ \ \ \ \ \ \ ander-e(r) \ \ \ \ \ \ \ \ \ Mann}
\end{styleStandard}

\begin{styleStandard}
\ \ \ \ someone different-\textsc{st}/\textsc{wk} man.\textsc{masc}
\end{styleStandard}

\begin{styleStandard}
\ \ c. *\ \ \textit{jemand \ \ ander-e(r) \ \ \ \ \ \ \ \ \ Mann}
\end{styleStandard}

\begin{styleStandard}
\ \ \ \ someone different-\textsc{st}/\textsc{wk} man.\textsc{masc}
\end{styleStandard}

\begin{styleStandard}
I follow the aforementioned paper in that the first case involves a regular DP-structure with the option of overtly realizing the head noun but that the second and third example involve a completely different structure. Roehrs (2008) argues that the adjectives here are part of a Modifier Phrase (ModP, see Rubin 1996). The latter phrase is proposed to be right-adjoined to an Indefinite Pronoun Restrictor Phrase (IPRP). The structure can be illustrated as follows (e stands for a null element; see also Leu 2005):\footnote{\ Given the indefiniteness of the construction, one objection might be that a strong ending is expected here. Importantly though, this construction also exhibits a strong ending in the dative (for detailed discussion, see Chapter 3, Section 2.2):\par (i)\textit{ \ \ mit \ \{jemand / }\textit{\textsuperscript{\%}}\textit{wem\} ander-em} \ \ \ \ \ \ \ \ \ \ \ \ \ \par \ \ with someone.\textsc{masc} \ \ different-\textsc{st}\par \ \ \ \ ‘with someone different’\par In other words, this indefinite pronoun construction is different from the noun phrases involving \textit{ein} ‘a’ (e.g., \textit{mit} \textit{einem ander-en Mann} ‘with a different-\textsc{wk} man’). As the latter are usually used to exemplify the inflectional alternation in indefinite environments, indefinite pronoun constructions deserve being mentioned in the current context – they show that not all indefinite constructions behave the same as regards inflections on adjectives. Again, it is clear that structure must be taken into account.}
\end{styleStandard}

\begin{styleStandard}
(64)\textit{ \ \ Indefinite Pronoun Constructions}
\end{styleStandard}

\begin{styleStandard}
\ \ \ \  \ \ \ \ \ \ DP
\end{styleStandard}

\begin{styleStandard}
[Warning: Draw object ignored][Warning: Draw object ignored]
\end{styleStandard}

\begin{styleStandard}
\ \  \ D\ \ \ \ \ \ IPRP
\end{styleStandard}

\begin{styleStandard}\itshape
[Warning: Draw object ignored][Warning: Draw object ignored][Warning: Draw object ignored]\ \  
\end{styleStandard}

\begin{styleStandard}
\ \ \ \ \ \ IPRP\ \  \ \ \ \ \ \ \ \ \ \ ModP
\end{styleStandard}

\begin{styleStandard}
[Warning: Draw object ignored][Warning: Draw object ignored][Warning: Draw object ignored][Warning: Draw object ignored]\ \ \ \ \ \ \ \ \ \  \ \ \ 
\end{styleStandard}

\begin{styleStandard}
\ \ \ \ IPR\ \ \ \ NP\ \ Mod\ \  \ \ \ \ \ AgrP
\end{styleStandard}

\begin{styleStandard}
\ \ \ \  \ {\textbar}\ \ \ \  \ {\textbar}\ \ \ \  \ \ \ \ \ \ \ \ {\textbar}
\end{styleStandard}

\begin{styleStandard}
\ \ \ \ \ \ \ \ \ \ \  \textit{je} \ \ \ \ \ \textit{{}-mand}\ \  \ \ \textit{e}\textit{\textsubscript{N}}\textsubscript{\ \ \ \ }[\textit{anderer} \textit{e}\textit{\textsubscript{N}}]
\end{styleStandard}

\begin{styleStandard}
\ \ \ \ \ \ \ \ \ \ \ \textit{wer} \ \ \ \ \ \ \textit{e}\textit{\textsubscript{R}}\ \  \ \ \textit{e}\textit{\textsubscript{N}}\textsubscript{\ \ \ \ }[\textit{anderer} \textit{e}\textit{\textsubscript{N}}]
\end{styleStandard}

\begin{styleStandard}
Some brief remarks about the structure are in order here. Note that I put the quantifying, determiner-like element under D. In conjunction with the DP-layer, IPRP accounts for the complex internal structure of indefinite pronouns (e.g., \textit{je-mand} ‘someone’). Finally, the presence of ModP is motivated by the corresponding constructions in some of the Romance languages. For instance, French makes use of \textit{de} ‘of’ in indefinite pronoun constructions (\textit{quelque chose}\textsubscript{MASC}\textit{ de grand}\textsubscript{MASC} ‘something big’), and this element seems to mediate concord between the two nominals (cf. \textit{une bonne chose}\textsubscript{FEM} \textit{de faite}\textsubscript{FEM} ‘(a good thing of done =) at least, that is done’; for details, see Roehrs 2008: 21-23). Returning to the discussion of the strong adjectives, note that there is no relevant determiner that can trigger Impoverishment, and a strong ending appears on the adjective. 
\end{styleStandard}

\begin{styleStandard}\itshape
3.3.\ \ Regular DP vs. Mid Right-Adjunction: Noun-Adjective Exclamatives
\end{styleStandard}

\begin{styleFootnote}
Certain exclamatives involve a noun and a following definite article and inflected adjective (65a). As can be seen in this example, these constructions license the absence of a determiner before the noun. In fact, the definite article can also be missing before the inflected adjective (65b). When this article is missing, the adjective is strong. Compare (65a) to (65b):\footnote{\ Dürscheid (2002: 70) provides the following, related examples: \textit{Mist, verdammt-er!} ‘darn-\textsc{st }rubbish’; \textit{Biest, elend-es!} ‘miserable-\textsc{st }beast’; \textit{Idiot, verdammt-er!} ‘darn-\textsc{st }idiot’; \textit{Gott, allmächtig-er} ‘God almighty-\textsc{st}’.} 
\end{styleFootnote}

\begin{styleStandard}
(65)\ \ a.\ \ \textit{Schwein, das schwarz-e!}
\end{styleStandard}

\begin{styleFooter}
\ \ \ \ pig.\textsc{neut} the \ black-\textsc{wk}
\end{styleFooter}

\begin{styleFooter}
\ \ \ \ ‘Stupid bastard!’
\end{styleFooter}

\begin{styleStandard}
\textit{\ \ \ }b.\ \ \textit{Schwein, schwarz-e*(s)!\ \ }
\end{styleStandard}

\begin{styleFooter}
\ \ \ \ pig.\textsc{neut} black-\textsc{st/*wk}
\end{styleFooter}

\begin{styleFooter}
\ \ \ \ ‘Stupid bastard!’
\end{styleFooter}

\begin{styleFooter}
While there are other versions of this type of exclamative, I limit myself here to pointing out that a pronominal determiner can also be present before the adjective (66a). Crucially, though, an indefinite article cannot precede the adjective (66b):
\end{styleFooter}

\begin{styleFooter}
(66)\ \ a.\ \ \textit{Schwein, du \ \ schwarzes!} 
\end{styleFooter}

\begin{styleFooter}
pig.\textsc{neut} you black-\textsc{st}
\end{styleFooter}

\begin{styleFooter}
\ \ \ \ \ ‘Bastard, you stupid one!’
\end{styleFooter}

\begin{styleFooter}
\ \ b. \ *\ \ \textit{Schwein, ein schwarz-es!}
\end{styleFooter}

\begin{styleFooter}
\ \ \ \ pig.\textsc{neut} a \ \ \ black-\textsc{st}
\end{styleFooter}

\begin{styleFooter}
Like the indefinite pronoun construction in the last section, these cases also involve concord in agreement features. Unlike the indefinite pronoun construction, the inflected adjective follows a noun here. Furthermore, it is clear that the exclamatives in (65a-b) and (66a) have similar, but not identical, referential properties. Specifically, while the examples in (65a-b) single out some individual in a specific context, (66a) addresses such an individual. I hypothesize that all these cases involve definite expressions. This fits well with the common assumption that vocatives of the type \textit{dummer Idiot} ‘stupid idiot’ involve definite contexts too (Section 3.9).
\end{styleFooter}

\begin{styleFooter}
Turning to the structure, we can observe that the adjective in (65b) is strong in a definite context. Importantly, considering the ungrammaticality of (66b) and the definite reference of (65b), it seems implausible that an indefinite determiner, including a null article, is or was present in (65b). Furthermore, given that the inflected adjective occurs in postnominal position, I suggest that (65b) consists of two nominals. The matrix nominal involves (at least) NumP, where the noun \textit{Schwein} ‘pig’ has undergone raising to Num. It is possible that there is more structure on top of NumP (cf. the structure of vocatives in Section 3.9). As to the following adjective \textit{schwarz} ‘black’, I suggest that it is part of a second nominal. Given concord in agreement features between the overt noun and the inflected adjective, I argue that the adjective is followed by a null noun, which is coreferential with the overt noun in the matrix nominal. The adjective and the null noun make up their own nominal, and I assume that the latter is adjoined to the NP of the matrix nominal. Like in the indefinite pronoun construction, the adjunction is mediated by ModP. This derives (65b) as follows:\footnote{\ I discuss the strong adjective in cases like (66a) in detail in Chapter 3, Section 5.}
\end{styleFooter}

\begin{styleStandard}
(67)\ \ \textit{Noun-Adjective} \textit{Exclamatives}
\end{styleStandard}

\begin{styleStandard}
\ \ \ \  \ \ \ \ NumP
\end{styleStandard}

\begin{styleStandard}
[Warning: Draw object ignored][Warning: Draw object ignored]
\end{styleStandard}

\begin{styleStandard}
\ \ \ \ \ \  \ \ \ \ Num’
\end{styleStandard}

\begin{styleStandard}
[Warning: Draw object ignored][Warning: Draw object ignored]
\end{styleStandard}

\begin{styleStandard}
\ \ \ \  Num\ \ \ \ \ \  \ NP
\end{styleStandard}

\begin{styleStandard}
[Warning: Draw object ignored][Warning: Draw object ignored][Warning: Draw object ignored][Warning: Draw object ignored]
\end{styleStandard}

\begin{styleStandard}
\ \ \textit{Schwein}\textsubscript{i}\ \ Num\ \  \ NP\ \ \ \ ModP
\end{styleStandard}

\begin{styleStandard}
[Warning: Draw object ignored][Warning: Draw object ignored]\ \ \ \ \ \ \ \ \ \ \ \ \ \ \ t\textsubscript{i}
\end{styleStandard}

\begin{styleStandard}
\ \ \ \ \ \ \ \ \ \ \ \ Mod\ \  \ \ \ \ \ AgrP
\end{styleStandard}

\begin{styleStandard}
\ \ \ \ \ \ \ \ \ \ \ \ \ \  \ \ \ \ \ \ \ \ {\textbar}
\end{styleStandard}

\begin{styleStandard}
\ \ \ \ \ \ \ \ \ \ \ \  \ \ \ \ \ \ \ \ \ [\textit{schwarzes} \textit{e}\textit{\textsubscript{N}}]
\end{styleStandard}

\begin{styleStandard}
Note that there is no determiner present in these cases. Consequently, Impoverishment is not triggered, and a strong ending surfaces on the adjective. Next, I turn to a third type of adjunction.
\end{styleStandard}

\begin{styleStandard}\itshape
3.4.\ \ Regular DP vs. High Right-Adjunction: Loose Appositions
\end{styleStandard}

\begin{styleStandard}
A typical case of a – what is sometimes called – loose apposition is given in (68a) below. Like in the two types of adjunction cases from Sections 3.1 and 3.2/3.3, the adjective can only be strong. This is in contrast to the loose apposition in (68b), which involves a definite determier:
\end{styleStandard}

\begin{styleStandard}
(68)\textit{ \ \ }a.\ \ \textit{Wir, begeistert-e(*n) \ \ \ \ \ \ Linguisten, fordern \ mehr Unterstützung}.
\end{styleStandard}

\begin{styleStandard}
we \ \ enthusiastic-\textsc{st}/*\textsc{wk} linguists \ \ \ \ demand more support
\end{styleStandard}

\begin{styleStandard}
‘We, enthusiastic linguists, demand more support.’
\end{styleStandard}

\begin{styleStandard}
\ \ b.\ \ \textit{Wir, die begeistert-e*(n) \ \ \ \ \ \ Linguisten, wollten mehr Unterstützung}.
\end{styleStandard}

\begin{styleStandard}
we \ \ the enthusiastic-\textsc{wk}/*\textsc{st} linguists \ \ \ \ wanted more support
\end{styleStandard}

\begin{styleStandard}
‘We, the enthusiastic linguists, wanted more support.’
\end{styleStandard}

\begin{styleStandard}
There are other differences as regards the two types of cases in the previous sections. First, the nominal following \textit{wir} ‘we’ in (68a) is not a name-like element (unlike the close apposition in (58b)), and second, this nominal involves an overt head noun after the adjective (unlike the indefinite pronoun construction in (62c) and unlike the exclamative in (65b)). Third, these constructions have a distinctive intonation contour, which includes pauses before and after the appositive. Fourth, a genitive complement to the first head noun can only precede this type of apposition (69a) but cannot follow it (69b):
\end{styleStandard}

\begin{styleStandard}
(69)\textit{ \ \ }a.\textit{\ \ die Professoren unserer Uni, \ }\textit{\textsubscript{\ }}\textit{sehr nett-e \ \ \ Leute}
\end{styleStandard}

\begin{styleStandard}
\ \ \ \ the professors \ \ \textsubscript{\ }of.our \ \ \ univ, very nice-\textsc{st} people
\end{styleStandard}

\begin{styleStandard}
\ \ \ \ ‘the professors of our university, very nice people’
\end{styleStandard}

\begin{styleStandard}
b. \ * \ \ \ \textit{die Professoren, sehr nett-e \ \ Leute, \ \ unserer Uni}
\end{styleStandard}

\begin{styleStandard}
\ \ the professors, \textsubscript{\ \ \ \ }very nice-\textsc{st} people, of.our \ \ \ univ
\end{styleStandard}

\begin{styleStandard}
In order to capture the difference in (69a-b), I propose that this is another case of right-adjunction. Unlike the two other types of instances, I suggest that this adjunction is higher in the structure, namely to the DP-level (cf. Delorme \& Dougherty 1972). (68a) and (69a) are derived in simplified form as follows:
\end{styleStandard}

\begin{styleStandard}
(70)\textit{ \ \ Loose Appositions}
\end{styleStandard}

\begin{styleStandard}
\ \ \ \ \ \  \ \ \ \ \ \ \ \ DP
\end{styleStandard}

\begin{styleStandard}
[Warning: Draw object ignored][Warning: Draw object ignored]
\end{styleStandard}

\begin{styleStandard}
\ \ \ \ DP\ \  \ \ \ \ \ \  \ \ \ AgrP
\end{styleStandard}

\begin{styleStandard}
[Warning: Draw object ignored][Warning: Draw object ignored][Warning: Draw object ignored]
\end{styleStandard}

\begin{styleStandard}
D\ \ \ \ NP
\end{styleStandard}

\begin{styleStandard}
\ \  {\textbar}\ \ \ \  \ {\textbar}
\end{styleStandard}

\begin{styleStandard}
\ \ \ \ \ \ \ \ \ \ \ \textit{wir}\ \ \ \  \textit{e}\textit{\textsubscript{N\ \ \ \  \ \ }}[\textit{begeisterte Linguisten}]
\end{styleStandard}

\begin{styleStandard}
\ \ \ \ \ \ \ \ \ \ \ \textit{die \ \ \ Professoren unserer Uni\ \  \ }[\textit{sehr nette Leute}]
\end{styleStandard}

\begin{styleStandard}
As in the cases above, a determiner did not adjoin to the phrase containing the adjective. With Impoverishment not occurring, a strong adjective surfaces. Note that the pronominal determiner \textit{wir} ‘we’ in (70) is taken to have a null noun as part of its complement. In the next section, I consider cases where this type of determiner occurs with an overt noun.
\end{styleStandard}

\begin{styleStandard}\itshape
3.5.\ \ Regular DP vs. Complex Specifier inside DP: Dis-agreement in Pronominal DPs
\end{styleStandard}

\begin{styleStandard}
Postal (1966) argues that pronouns are determiners in that they can combine with adjectives and nouns (also Pesetsky 1978; for more recent discussion, see Déchaine \& Wiltschko 2002: 421-22, Roehrs 2005, and the extensive work by Höhn, e.g., Höhn 2020). A typical case of such a pronominal DP is given in (71a). What is interesting about this case is that the adjective can, with some differences in preference, have a weak or a strong inflection. Compare (71a) to (71b). Furthermore, Roehrs (2006b) discusses some other pronominal DPs where plural pronouns combine with singular head nouns despite the fact that these two types of elements do not agree in morphological number (71c). Importantly though, here only a strong adjective is possible (e.g., Bhatt 1990: 154-55, Darski 1979: 202, Gunkel \textit{et al}. 2017: 1308):\footnote{\ Roehrs (2006b) provides some other, related examples: \textit{ihr verdammt-es Pack} ‘[intended:] you(PL) darn-\textsc{st} gang’, \textit{ihr bl\=od-e Bande} ‘[intended:] you(PL) stupid-\textsc{st} gang’, \textit{ihr dumm-es Gesindel} ‘[intended:] you(PL) dumb-\textsc{st} riff-raff’ (see also Footnote Error: Reference source not found).} 
\end{styleStandard}

\begin{styleStandard}
(71)\textit{ \ \ }a.\ \ \textit{ihr \ \ \ \ \ \ \ dumm-en \ \ Idioten}
\end{styleStandard}

\begin{styleStandard}
\ \ \ \ you(\textsc{pl}) stupid-\textsc{wk} idiots
\end{styleStandard}

\begin{styleStandard}
\ \ \ \ ‘you stupid idiots’
\end{styleStandard}

\begin{styleStandard}
\textit{\ \ \ }b. \%\ \ \textit{ihr \ \ \ \ \ \ \ dumm-e \ \ \ Idioten}
\end{styleStandard}

\begin{styleStandard}
\ \ \ \ you(\textsc{pl}) stupid-\textsc{st} idiots
\end{styleStandard}

\begin{styleStandard}
\ \ \ \ ‘you stupid idiots’
\end{styleStandard}

\begin{styleStandard}
c. \ \ \textit{ihr \ \ \ \ \ \ \ \ jung-e*(s) \ \ \ \ \ \ \ \ \ \ \ \ \ \ Gemüse}
\end{styleStandard}

\begin{styleStandard}
\ \ \ \ you(\textsc{pl}) young-\textsc{st.sgl/*wk} vegetable.\textsc{neut}
\end{styleStandard}

\begin{styleStandard}
\ \ \ \ ‘you young folks’
\end{styleStandard}

\begin{styleStandard}
Unlike the loose appositives in the last section, none of the cases above involve comma intonation. In addition, pronominal DPs and loose appositives involve a difference in interpretation. While it is not impossible that the adjective in the pronominal DP may receive a restrictive interpretation (e.g., \textit{nur} \textit{ihr Großen }‘only you tall ones’), this is not true of appositional structures. In the latter case, the adjective and noun simply provide additional information, which does not delimit the set denoted by \textit{wir }‘we’ (e.g., \textit{ihr, begeisterte Studenten} ‘you, enthusiastic students’). Before moving on, note also that the unmarked example in (71a) runs counter to the traditional generalizations Weak After Strong and the Principle of Monoinflection.
\end{styleStandard}

\begin{styleStandard}
It is pointed out in Roehrs (2006b) that semantic agreement in number must hold between all the elements in (71a-c). For instance, like the pronominal determiner \textit{ihr} ‘you(\textsc{pl})’, both the count noun \textit{Idioten} ‘idiots’ and the mass noun \textit{Gemüse} ‘(vegetable =) folks’ imply several individuals. This is similar to regular determiners such as \textit{diese} ‘these’ – they also require semantic agreement with the head noun. However, unlike pronominal determiners, regular determiners are more restrictive: they do not tolerate morphological dis-agreement at all (e.g., \textit{ihr blödes Pack} ‘[intended:] you(PL) stupid gang’ vs. *\textit{diese Pack} ‘[intended:] these gang’). 
\end{styleStandard}

\begin{styleStandard}
Furthermore, without a special intonation contour, agreeing nominals (e.g., \textit{dumme(n) Idioten} ‘stupid idiots’ as in (71a-b)) or dis-agreeing nominals (e.g., \textit{junges Gemüse} ‘young [vegetable =] folks’ as in (71c)) cannot be iterated in pronominal DPs, and both of these nominals cannot interact with one another. In other words, there can be only one overt noun and related adjective in the pronominal DP:
\end{styleStandard}

\begin{styleStandard}
(72)\ \ a.\ \ \textit{ihr \ \ \ \ \ \ \ \ Idioten}
\end{styleStandard}

\begin{styleStandard}
\ \ \ \ you(\textsc{pl}) idiots
\end{styleStandard}

\begin{styleStandard}
\ \ \ \ ‘you idiots’\ \ \ \ 
\end{styleStandard}

\begin{styleStandard}
\ \ b. *\ \ \textit{ihr \ \ \ \ \ \ \ \ Idioten jung-es} \ \ \ \ \ \ \ \ (\textit{Gemüse})
\end{styleStandard}

\begin{styleStandard}
\ \ \ \ you(\textsc{pl}) idiots \ \ young-\textsc{st.sgl} vegetable.\textsc{neut} 
\end{styleStandard}

\begin{styleStandard}
\ \ c. *\ \ \textit{ihr \ \ \ \ \ \ \ \ jung-es} \ \ \ \ \ \ \ \ (\textit{Gemüse}) \ \ \ \ \ \ \ \ \ \ \ \textit{Idioten}
\end{styleStandard}

\begin{styleStandard}
\ \ \ \ you(\textsc{pl}) young-\textsc{st.sgl} vegetable.\textsc{neut} idiots
\end{styleStandard}

\begin{styleStandard}
(73)\ \ a.\ \ \textit{ihr \ \ \ \ \ \ \ Gemüse}
\end{styleStandard}

\begin{styleStandard}
\ \ \ \ you(\textsc{pl}) vegetable.\textsc{neut} 
\end{styleStandard}

\begin{styleStandard}
\ \ \ \ ‘you (young) folks’
\end{styleStandard}

\begin{styleStandard}
\ \ b. \ *\ \ \textit{ihr \ \ \ \ \ \ \ Gemüse \ \ \ \ \ \ \ \ \ \ \ \ dumm-en} \ \ \ \ \ (\textit{Idioten})
\end{styleStandard}

\begin{styleStandard}
\ \ \ \ you(\textsc{pl}) vegetable.\textsc{neut} stupid-\textsc{wk.pl} idiots
\end{styleStandard}

\begin{styleStandard}
\ \ c. \ *\ \ \textit{ihr \ \ \ \ \ \ \ \ dumm-en} \ \ \ \ \ (\textit{Idioten}) \textit{Gemüse}
\end{styleStandard}

\begin{styleStandard}
\ \ \ \ you(\textsc{pl}) stupid-\textsc{wk.pl} idiots \ \ \ \ vegetable.\textsc{neut}
\end{styleStandard}

\begin{styleStandard}
In order to capture these facts, Roehrs (2006b) proposes that the pronominal determiner can optionally select either an AgrP resulting in the (regular) agreeing cases or a – what he calls for lack of a better name – Dis-agreement Phrase (DisP) bringing about the dis-agreeing cases.\footnote{\ With adjunction less constrained (e.g., Pysz 2006: 66-67), above-mentioned restrictions are best stated in terms of selection.} 
\end{styleStandard}

\begin{styleStandard}
I assume that prononomal DPs with an agreeing adjective and noun as in (71a-b) involve a regular, canonical DP where the pronominal determiner surfaces in the DP-layer, the adjective is in Spec,AgrP, and the noun is in NP. This derives (71a) as follows (recall that the AP in Spec,AgrP involves InflP at the top; I abstract away from NumP here): 
\end{styleStandard}

\begin{styleFootnote}
(74)\textit{ \ \ Transitive Pronouns (Regular DP - simplified)}
\end{styleFootnote}

\begin{styleStandard}
\ \  DP
\end{styleStandard}

\begin{styleFootnote}
[Warning: Draw object ignored][Warning: Draw object ignored]
\end{styleFootnote}

\begin{styleFootnote}
[Warning: Draw object ignored]\textit{\ \ \ }\ \  \ D’
\end{styleFootnote}

\begin{styleFootnote}
[Warning: Draw object ignored][Warning: Draw object ignored]
\end{styleFootnote}

\begin{styleFootnote}
\ \  \ \ D\ \ \ \ AgrP
\end{styleFootnote}

\begin{styleFootnote}
[Warning: Draw object ignored][Warning: Draw object ignored]\ \  \textit{ihr}\textsubscript{i}
\end{styleFootnote}

\begin{styleFootnote}
[Warning: Draw object ignored]\ \  \ \ \ \ \ \ \ \ \ \ \ InflP\ \  \ \ Agr’
\end{styleFootnote}

\begin{styleFootnote}
\textit{\ \ \ \ \ \ \ \ \ }[Warning: Draw object ignored][Warning: Draw object ignored]\textit{\ dummen}
\end{styleFootnote}

\begin{styleFootnote}
\ \ \ \ \ \ Agr\ \ \ \  ArtP
\end{styleFootnote}

\begin{styleFootnote}
[Warning: Draw object ignored][Warning: Draw object ignored]\ \ \ \ \ \ ihr\textsubscript{i}
\end{styleFootnote}

\begin{styleFootnote}
\ \ \ \ \ \  \ \ \ \ \ \ \ \ \ \ \ \ \ \ \  Art’
\end{styleFootnote}

\begin{styleFootnote}\itshape
\ \ \ \ \ \ \ \ [Warning: Draw object ignored][Warning: Draw object ignored]
\end{styleFootnote}

\begin{styleFootnote}
\ \ \ \ \ \ \ \ \ \ Art\ \ \ \  \ NP
\end{styleFootnote}

\begin{styleFootnote}
\ \ \ \ \ \ \ \ \ \ ihr\textsubscript{i}\ \ \textit{ \ \ \ \ \ \ \ \ \ Idioten}
\end{styleFootnote}

\begin{styleFootnote}
In Chapter 3, Section 5, I discuss the strong/weak alternation in pronominal DPs, including the strong inflection on the adjective in (71b), in more detail. 
\end{styleFootnote}

\begin{styleFootnote}
As for cases like (71c), pronominal DPs with a morphologically dis-agreeing nominal have a different structure. They involve non-canonical DPs where the entire dis-agreeing element is assumed to be in Spec,DisP, and the matrix nominal involves a null noun. Compared to (74), this construction features an intransitive pronoun:\footnote{\ Note that both pronouns in (74) and (75) take complements, overt material in (74) but a null noun in (75). As such, I use the terms transitive and intransitive as regards the overtness and covertness of the following material. Also, the structure in (75) has wider application. In Chapter 6, it is pointed out that a (number-neutral) bare noun like \textit{Bauer} ‘peasant’ must be singular in interpretation in dis-agreement cases such as \textit{Sie Bauer} ‘you peasant’ (where \textit{Sie} ‘you’ is morphological plural and could, at least in principle, be plural in interpretation). I argue in more detail there that the nominal \textit{Bauer} is in Spec,DisP.}
\end{styleFootnote}

\begin{styleStandard}
(75)\textit{ \ \ Intransitive Pronouns (Dis-agreement)}
\end{styleStandard}

\begin{styleStandard}
\ \  DP
\end{styleStandard}

\begin{styleFootnote}
[Warning: Draw object ignored][Warning: Draw object ignored]
\end{styleFootnote}

\begin{styleFootnote}
[Warning: Draw object ignored]\textit{\ \ \ }\ \  \ D’
\end{styleFootnote}

\begin{styleFootnote}
[Warning: Draw object ignored][Warning: Draw object ignored]
\end{styleFootnote}

\begin{styleFootnote}
\ \  \ \ D\ \ \ \ DisP
\end{styleFootnote}

\begin{styleFootnote}
[Warning: Draw object ignored][Warning: Draw object ignored]\ \  \textit{ihr}\textsubscript{i}
\end{styleFootnote}

\begin{styleFootnote}
[Warning: Draw object ignored]\ \  \ \ \ \ \ [\textsubscript{AgrP} \textit{junges Gemüse}] \ \ Dis’
\end{styleFootnote}

\begin{styleFootnote}\itshape
\ [Warning: Draw object ignored][Warning: Draw object ignored]
\end{styleFootnote}

\begin{styleFootnote}
\ \ \ \ \ \ Dis\ \ \ \  ArtP
\end{styleFootnote}

\begin{styleFootnote}
[Warning: Draw object ignored][Warning: Draw object ignored]\ \ \ \ \ \ ihr\textsubscript{i}
\end{styleFootnote}

\begin{styleFootnote}
\ \ \ \ \ \  \ \ \ \ \ \ \ \ \ \ \ \ \ \ \  Art’
\end{styleFootnote}

\begin{styleFootnote}\itshape
\ \ \ \ \ \ \ \ [Warning: Draw object ignored][Warning: Draw object ignored]
\end{styleFootnote}

\begin{styleFootnote}
\ \ \ \ \ \ \ \ \ \ Art\ \ \ \  \ NP
\end{styleFootnote}

\begin{styleFootnote}
\ \ \ \ \ \ \ \ \ \ ihr\textsubscript{i}\ \ \textit{ \ \ \ \ \ \ \ \ \ \ \ \ \ e}\textit{\textsubscript{N}}
\end{styleFootnote}

\begin{styleStandard}
Note now that there is a clear difference between the agreeing DP in (71a) and the dis-agreeing case in (71c). In the former, the plural adjective and its projected InflP reside in Spec,AgrP (74); in the latter, the singular adjective also projects InflP, but there is more linguistic material that relates to this adjective – the singular noun. Since both agree in concord features, they make up their own nominal, presumably AgrP. Given the lack of morphological agreement with the plural pronoun, this AgrP has been located in Spec,DisP (75). To be clear, unlike (74), the adjective in (75) is more deeply embedded (cf. InflP vs. AgrP, which contains InflP). To repeat, the adjective in the latter establishes its own agreement relation with the overt noun inside the embedded AgrP and is morphologically independent of the larger DP (for the discussion of lack of concord here, see Chapter 8, Section 3.1). Given this complex specifier, it is clear that Impoverishment cannot occur resulting in a strong adjective. More generally, this section has shown that adjectival inflections provide clues about various degrees of embeddings of adjectives (Hypothesis 2a).
\end{styleStandard}

\begin{styleStandard}\itshape
3.6.\ \ Regular DP vs. Separate Base-generation: Split Topicalizations
\end{styleStandard}

\begin{styleStandard}
Fanselow (1988) and van Riemsdijk (1989) discuss discontinuous noun phrases, often referred to as split topicalizations (also Ott 2011a). Putting it in simple terms, the lower part of a noun phrase occurs in a higher position of the clause, and the higher part of the same noun phrase is in a lower position. Compare the continuous nominal in (76a) to its discontinuous counterpart in (76b): 
\end{styleStandard}

\begin{styleStandard}
(76)\ \ a.\ \ \textit{Er hat zehn Hemden getragen}.
\end{styleStandard}

\begin{styleStandard}
\ \ \ \ he has ten \ \ shirts \ \ \ \ worn
\end{styleStandard}

\begin{styleStandard}
\ \ \ \ ‘He wore ten shirts.’
\end{styleStandard}

\begin{styleStandard}
\ \ b.\ \ \textit{Hemden hat er zehn getragen}.
\end{styleStandard}

\begin{styleStandard}
\ \ \ \ shirts \ \ \ \ has he ten \ \ worn
\end{styleStandard}

\begin{styleStandard}
\ \ \ \ ‘As for shirts, he wore ten.’
\end{styleStandard}

\begin{styleStandard}
This type of construction has a number of interesting properties. As pointed out by Bhatt (1990: 249-50) and Fanselow \& Ćavar (2002), some speakers allow the split of a definite noun phrase. Importantly, while an adjective in a non-split noun phrase has a weak ending (77a), this adjective has a strong inflection when it is topicalized stranding the determiner (77b):
\end{styleStandard}

\begin{styleStandard}
(77)\textit{ \ \ }a.\ \ \textit{Ich habe immer nur \ diese bunt-e*(n) \ \ \ \ \ \ \ \ \ Hemden da \ \ \ getragen}.
\end{styleStandard}

\begin{styleStandard}
\ \ \ \ I \ \ \ have always only these colored-\textsc{wk}/*\textsc{st} shirts \ \ \ \ there worn
\end{styleStandard}

\begin{styleStandard}
‘I have always worn only these colored shirts there.’
\end{styleStandard}

\begin{styleStandard}
\ \ b.\ \ \textit{Bunt-e(*n) \ \ \ \ \ \ \ \ Hemden habe ich immer nur \ \ diese da \ \ \ \ getragen}.
\end{styleStandard}

\begin{styleStandard}
\ \ \ \ colored-\textsc{st}/*\textsc{wk} shirts \ \ \ \ \ have I \ \ \ always only these there worn
\end{styleStandard}

\begin{styleStandard}
‘As for colored shirts, I have always worn only these there.’
\end{styleStandard}

\begin{styleStandard}
This split results in two nominals. Fanselow (1988) proposes that these two nominals are base-generated separately in the VP and that one of them undergoes movement. This can be illustrated for (77b) in (78). For concreteness, I put adverbial elements as adjuncts to a Perfective Phrase (PerfP), where the auxiliary \textit{haben} ‘to have’ originates (for more detailed discussion of split topicalization, see Chapter 4, Section 3): 
\end{styleStandard}

\begin{styleStandard}
(78)\ \ \textit{Split Topicalizations}
\end{styleStandard}

\begin{styleStandard}
\ \ \ \  \ \ \ \ \ CP
\end{styleStandard}

\begin{styleStandard}
[Warning: Draw object ignored][Warning: Draw object ignored]
\end{styleStandard}

\begin{styleStandard}
\ \ \ \ AgrP\textsubscript{k}\ \ \ \ \ \  C’
\end{styleStandard}

\begin{styleStandard}
\textit{\ }[Warning: Draw object ignored][Warning: Draw object ignored]\textit{\ }[\textit{bunte Hemden}]
\end{styleStandard}

\begin{styleStandard}
\ \ \ \ \ \ \ \  C\ \ \ \ TP
\end{styleStandard}

\begin{styleStandard}
[Warning: Draw object ignored][Warning: Draw object ignored]\ \ \ \ \ \  \ \ \ \ \ \ \ \ \ \textit{habe}\textsubscript{i}
\end{styleStandard}

\begin{styleStandard}
\ \ \ \ \ \ \ \ \ \ \textit{ich}\ \ \ \  T’
\end{styleStandard}

\begin{styleStandard}
[Warning: Draw object ignored][Warning: Draw object ignored]
\end{styleStandard}

\begin{styleStandard}
\ \ \ \ \ \ \ \ \ \  \ \ \ \ \ \ \ \ \ PerfP\ \ \ \  T
\end{styleStandard}

\begin{styleStandard}
[Warning: Draw object ignored][Warning: Draw object ignored]\ \ \ \ \ \ \ \ \ \ \ \ \ \ \ \  \ t\textsubscript{i}
\end{styleStandard}

\begin{styleStandard}
\ \ \ \ \ \ \ \  \ \ \ \ \ \ \ \ \textit{immer}\ \  \ \ \ \ \ \ \ \ \ PerfP
\end{styleStandard}

\begin{styleStandard}
[Warning: Draw object ignored][Warning: Draw object ignored]
\end{styleStandard}

\begin{styleStandard}
\ \ \ \ \ \ \ \ \ \  \ \ \ \ \ \ \ \ \ \textit{nur}\ \  \ \ \ \ \ \ \ \ \ PerfP
\end{styleStandard}

\begin{styleStandard}
[Warning: Draw object ignored][Warning: Draw object ignored]
\end{styleStandard}

\begin{styleStandard}
\ \ \ \ \ \ \ \ \ \ \ \  \ \ \ \ \ \ \ \ \ \ VP\ \  \ \ \ \ \ \ \ \ \ Perf
\end{styleStandard}

\begin{styleStandard}
[Warning: Draw object ignored][Warning: Draw object ignored]\ \ \ \ \ \ \ \ \ \ \ \ \ \ \ \  \ \ \ \ \ \ \ \ \ \ \ t\textsubscript{i}
\end{styleStandard}

\begin{styleStandard}
\ \ \ \ \ \ \ \ \ \  \ \ \ \ \ \ \ \ \ \ \ t\textsubscript{k}\ \  \ \ \ \ \ \ \ \ \ \ V’
\end{styleStandard}

\begin{styleStandard}
[Warning: Draw object ignored][Warning: Draw object ignored]\ \ \ \ \ \ \ \ \ \ \ \ 
\end{styleStandard}

\begin{styleStandard}
\ \ \ \ \ \ \ \ \ \ \ \  \ \ \ \ \ \ \ \ \ DP\ \  \ \ \ \ \ \ \ \ \ \ V
\end{styleStandard}

\begin{styleStandard}
\textit{\ \ \ diese e}\textit{\textsubscript{N}}\textit{ da\ \  \ \ \ \ \ getragen}
\end{styleStandard}

\begin{styleStandard}
Some independent evidence for this analysis comes from the fact that the discontinuous noun phrase can have two mismatching determiners, for instance, \textit{‘n} ‘a’ and \textit{dieses} ‘this’ below (recall that \textit{‘n} is the reduced form of the indefinite article \textit{ein}): 
\end{styleStandard}

\begin{styleStandard}
(79)\textit{ \ \ }‘\textit{N Hemd \ \ \ \ \ \ \ habe ich immer \ nur \ dieses getragen.}
\end{styleStandard}

\begin{styleStandard}
\textit{\ \  }a \ shirt.\textsc{neut}\textsubscript{ }have I \ \ \ \ always only this \ \ \ \ worn
\end{styleStandard}

\begin{styleStandard}
‘As for a shirt, I only wore this one.’
\end{styleStandard}

\begin{styleStandard}
With the two nominals assembled separately from one another, it should be clear that the demonstrative in (77b) did not adjoin to the AgrP that contains the adjective. Impoverishment does not occur yielding a strong ending on the adjective.
\end{styleStandard}

\begin{styleStandard}\itshape
3.7.\ \ Regular DP vs. Complex Compound Modifier: Nominal Compounds
\end{styleStandard}

\begin{styleStandard}
As illustrated before, adjectives in canonical DPs undergo the strong/weak alternation (80a). This is also possible with phrasal proper names like \textit{die Deutsche Bank} ‘the German Bank’ (80b). Interestingly, these proper names can form complex modifiers of nominal compounds (Lieber 1988, R. Wiese 1996b). Crucially, here the adjectives have to appear with strong endings (80c):
\end{styleStandard}

\begin{styleStandard}
(80)\ \ a.\ \ \textit{mit \ \ dem deutsch-en \ \ Chef}
\end{styleStandard}

\begin{styleStandard}
\ \ \ \ with the \ \ German-\textsc{wk} boss.\textsc{masc}
\end{styleStandard}

\begin{styleStandard}
\ \ \ \ ‘with the German boss’
\end{styleStandard}

\begin{styleStandard}
\ \ b.\ \ \textit{mit \ \ der Deutsch-en \ Bank}
\end{styleStandard}

\begin{styleStandard}
with the German-\textsc{wk} Bank.\textsc{fem}
\end{styleStandard}

\begin{styleStandard}
‘with the German Bank’
\end{styleStandard}

\begin{styleStandard}
\ \ c.\ \ \textit{mit \ \ dem }[\textit{Deutsch-e(*n) \ \ \ Bank}]\textit{ Chef}
\end{styleStandard}

\begin{styleStandard}
\ \ \ \ with the \ \ \ German-\textsc{st/*wk} Bank \textsubscript{\ }boss.\textsc{masc}
\end{styleStandard}

\begin{styleStandard}
\ \ \ \ ‘with the German Bank-boss’
\end{styleStandard}

\begin{styleStandard}
The example in (80c) involves a complex nominal, and it can be represented as in (81), where the complex modifier is adjoined to the left of the head noun:
\end{styleStandard}

\begin{styleStandard}
(81)\ \ \textit{Nominal Compounds}
\end{styleStandard}

\begin{styleStandard}
\ \ \ \ DP
\end{styleStandard}

\begin{styleStandard}
[Warning: Draw object ignored][Warning: Draw object ignored]
\end{styleStandard}

\begin{styleStandard}
\ \ D\ \ \ \ NP
\end{styleStandard}

\begin{styleStandard}
\ \ \ \ \ \ \ \ \ \ \textit{dem\ \ }\ \  \ {\textbar}
\end{styleStandard}

\begin{styleStandard}
\ \ \ \ \ \  N
\end{styleStandard}

\begin{styleStandard}
[Warning: Draw object ignored][Warning: Draw object ignored]
\end{styleStandard}

\begin{styleStandard}
\ \ [\textit{Deutsche Bank}]\ \  N
\end{styleStandard}

\begin{styleStandard}
\ \ \ \ \ \  \ \ \ \ \ \ \ \ \ \ \textit{Chef}
\end{styleStandard}

\begin{styleFooter}
As in the cases above, the determiner did not adjoined to the structure containing the adjective. With Impoverishment not occurring, a strong adjective appears.
\end{styleFooter}

\begin{styleStandard}
So far, I have discussed a number of cases, where the adjective is outside the determiner’s regular domain of influence. Next, I turn to a case where the determiner itself is in a position where it cannot bring about a weak adjective.
\end{styleStandard}

\begin{styleStandard}\itshape
3.8.\ \ Regular DP vs. Outside of DP Proper: Predeterminers
\end{styleStandard}

\begin{styleStandard}
Roehrs (2009a: Chapter 4) discusses some cases where determiners and determiner-like elements may co-occur (also Wood 2007; Chapter 1, Section 4.1.2). To set the stage, consider first (82a), where a demonstrative occurs with a weak adjective. In contrast, a possessive article occurs with a strong adjective (82b). Now, the demonstrative and the possessive article can also be combined as in (82c). Here, the adjective must be strong (Duden 1995: 286, Gunkel \textit{et al}. 2017: 1308):
\end{styleStandard}

\begin{styleStandard}
(82)\textit{ \ \ }a.\ \ \textit{dieses groß-e(*s) \ \ \ \ Glück}
\end{styleStandard}

\begin{styleStandard}
\ \ \ \ this \ \ \ \ great-\textsc{wk}/*\textsc{st} happiness.\textsc{neut}
\end{styleStandard}

\begin{styleStandard}
‘this great happiness’
\end{styleStandard}

\begin{styleStandard}
\ \ b.\ \ \textit{mein groß-e*(s) \ \ \ \ \ Glück}
\end{styleStandard}

\begin{styleStandard}
\ \ \ \ my \ \ \textsubscript{\ }great-\textsc{st}/*\textsc{wk} happiness.\textsc{neut}
\end{styleStandard}

\begin{styleStandard}
‘my great happiness’
\end{styleStandard}

\begin{styleStandard}
\ \ c.\ \ \textit{dieses mein groß-e*(s) \ \ \ \ \ Glück}
\end{styleStandard}

\begin{styleStandard}
\ \ \ \ this \ \ \ \ my \ \ \textsubscript{\ }great-\textsc{st}/*\textsc{wk} happiness.\textsc{neut}
\end{styleStandard}

\begin{styleStandard}
‘this my great happiness’
\end{styleStandard}

\begin{styleStandard}
Note that, morphologically, the presence of the demonstrative in (82c) does not play a role for the inflection on the adjective in this example. This is surprising given that it does have an impact in (82a). Before commenting on the syntax and semantics, note also that the strong adjective in (82c) is preceded by the strongly inflected determiner element \textit{dieses} ‘this’ in violation of the traditional generalizations about adjectival inflections. 
\end{styleStandard}

\begin{styleStandard}
\ \ Syntactically, it is well known that the demonstrative can only occur to the left of the possessive (cf. *\textit{mein dieses} \textit{große(s) Glück} ‘[intended:] my this great happiness’) – it is a predeterminer in such combinations. Semantically, this element seems to function as an intensifier for deixis (Wood 2007). In order to account for this different behavior in the morphology, syntax, and semantics, it is proposed in Roehrs (2009a) that the demonstrative is merged in a phrase above the DP-level. Here, I label this projection Left Periphery Phrase (LPP) (see Giusti \& Iovino 2016):
\end{styleStandard}

\begin{styleStandard}
(83)\textit{ \ \ Predeterminers}
\end{styleStandard}

\begin{styleJBExample}
\ \ \ \ \ \ \ \ \ LPP
\end{styleJBExample}

\begin{styleStandard}\bfseries
[Warning: Draw object ignored][Warning: Draw object ignored]
\end{styleStandard}

\begin{styleStandard}
\ \ \ [\textit{dieses}]\ \  \ \ \ \ \ DP
\end{styleStandard}

\begin{styleStandard}
[Warning: Draw object ignored][Warning: Draw object ignored]\ \ \ \ \ \ \ 
\end{styleStandard}

\begin{styleStandard}
\ \ \ \ \ \ \ \ \ \ \ \  \ \ \ \ \textit{mein}\ \  \ \ \ \ AgrP
\end{styleStandard}

\begin{styleStandard}
[Warning: Draw object ignored][Warning: Draw object ignored]\ \  \ \ \ \ 
\end{styleStandard}

\begin{styleStandard}
\ \ \ \  \ \textit{großes}\ \  \ \ \ \ \ NP
\end{styleStandard}

\begin{styleStandard}
\ \ \ \ \ \ \ \  \ \ \ \textit{Glück}
\end{styleStandard}

\begin{styleStandard}
As the intensifying demonstrative is base-generated above the DP-level, it did not adjoin to the phrase containing the adjective. Consequently, Impoverishment does not occur, and as a result, the adjective exhibits a strong ending.
\end{styleStandard}

\begin{styleStandard}\itshape
3.9.\ \ Regular DP vs. No Determiner: Vocatives
\end{styleStandard}

\begin{styleFootnote}
Vocatives are structures that occur in non-argument position. As illustrated below, these cases exhibit strong adjectives as well:
\end{styleFootnote}

\begin{styleFootnote}
(84)\ \ a.\ \ \textit{Blöd-e*(s) \ \ \ \ \ \ Schwein!}
\end{styleFootnote}

\begin{styleFootnote}
\ \ \ \ stupid-\textsc{st/*wk} pig.\textsc{neut}
\end{styleFootnote}

\begin{styleFootnote}
\ \ \ \ ‘Stupid idiot!’
\end{styleFootnote}

\begin{styleFootnote}
b.\ \ \textit{Dumm-e*(r) \ \ \ Idiot!}
\end{styleFootnote}

\begin{styleFootnote}
\ \ \ \ stupid-\textsc{st/*wk} idiot.\textsc{masc}
\end{styleFootnote}

\begin{styleStandard}
‘Stupid idiot!’
\end{styleStandard}

\begin{styleFootnote}
It can be pointed out that vocatives involve the presupposition that the individual addressed is present in the situational context. As such, vocatives are usually taken to be definite constructions (see also the exclamatives in Section 3.3). Following other work, Julien (2016: 86) proposes that this definiteness follows from the presence of a second-person feature on a Vocative head. This Voc head selects a DP with an empty D head (except for an agreeing second-person feature). Given the presence of VocP, this means that vocatives as in (84) involve non-canonical structures: 
\end{styleFootnote}

\begin{styleStandard}
(85)\textit{ \ \ Vocatives}
\end{styleStandard}

\begin{styleJBExample}
\ \ \ \ \ \ \ VocP
\end{styleJBExample}

\begin{styleStandard}\bfseries
[Warning: Draw object ignored][Warning: Draw object ignored]
\end{styleStandard}

\begin{styleStandard}
\ \ \ \ \ Voc\ \  \ \ \ \ \ \  \ \ \ \ \ \ DP
\end{styleStandard}

\begin{styleStandard}
[Warning: Draw object ignored][Warning: Draw object ignored]\ \ \ \ \ \ \ 
\end{styleStandard}

\begin{styleStandard}
\ \ \ \ \ \ \ \ \ \ \ \  \ \ \ \ \ \ D\ \  \ \ \ \ \  \ \ \ \ AgrP
\end{styleStandard}

\begin{styleStandard}
[Warning: Draw object ignored][Warning: Draw object ignored]\ \  \ \ \ \ 
\end{styleStandard}

\begin{styleStandard}
\ \ \ \  \ \textit{blödes \ \ \ }\ \  \ \ \ \ \ NP
\end{styleStandard}

\begin{styleStandard}
\ \ \ \ \ \ \ \  \ \textit{Schwein}
\end{styleStandard}

\begin{styleFootnote}
As a determiner is absent, this accounts for the strong adjectives. Finally, note that these data confirm once again that strong adjectives are not tied to indefiniteness.
\end{styleFootnote}

\begin{styleStandard}
\ \ To sum up the entire section, I discussed one context in Section 2 where weak adjectives occur (regular, simple DP); in this section, I examined nine contexts where strong adjectives surface: the adjective may be part of different types of adjunction, it may be deeply embedded inside a specifier, the determiner and adjective are in their typical positions but both occur separately in a discontinuous noun phrase, the adjective may be part of a complex nominal compound, the determiner may be outside of the DP proper, and a determiner may be absent altogether. In all relevant cases, the adjective or the determiner is in a position different from that of a regular (continuous) DP. Some independent evidence was provided for the assumption of different structures. What all these nine types of constructions have in common is that Impoverishment Rule 1 cannot apply as a determiner does not adjoin to the phrase (immediately) containing the adjective. As a consequence, the inflectional feature bundle of the adjective does not undergo Impoverishment, remains unaltered, and is spelled out as a strong inflection.
\end{styleStandard}

\begin{styleStandard}
\ \ Thus far, I have focused on the discussion of the inflections on adjectives. As we have seen though, determiners also have strong inflections, and the same goes for (non-possessive) determiner-like elements – intensifying predeterminers like \textit{diese} ‘this’ and \textit{alle} ‘all’. In the next section, I turn to the explanation of the strong inflections on these elements.
\end{styleStandard}

\begin{styleFooter}\bfseries
4.\ \ Adjectival Inflections on Determiners and Predeterminers
\end{styleFooter}

\begin{styleStandard}
First, consider (86a) and (86b), where both \textit{diese} ‘these’ and \textit{alle} ‘all’ are followed by a weak adjective: 
\end{styleStandard}

\begin{styleStandard}
(86)\ \ a.\ \ \textit{dies-e \ \ \ \ klein-en \ \ Autos}
\end{styleStandard}

\begin{styleStandard}
\ \ \ \ these-\textsc{st} small-\textsc{wk} cars
\end{styleStandard}

\begin{styleFootnote}
\ \ \ \ ‘these small cars’
\end{styleFootnote}

\begin{styleStandard}
\textit{\ \ }b.\ \ \textit{all-e \ \ klein-en \ \ Autos}
\end{styleStandard}

\begin{styleStandard}
\ \ \ \ all-\textsc{st} small-\textsc{wk} cars
\end{styleStandard}

\begin{styleFootnote}
\ \ \ \ ‘all small cars’
\end{styleFootnote}

\begin{styleStandard}
I proposed in Section 2 that determiners move to the DP-level triggering Impoverishment on adjectives. As to the determiners themselves, recall that the strong endings on the determiners are identical to those on the adjectives, and as such they should receive a similar account. Now, if we make the plausible assumption that similar to adjectives, the inflections on determiners can undergo Impoverishment (for such cases involving Impoverishment Rule 2, see Chapter 3, Section 4), then the latter mechanism cannot have occurred on either \textit{diese} ‘these’ or \textit{alle} ‘all’ in (86). I propose that Impoverishment does not apply to the inflections on determiners as the determiners themselves are not adjoined to by a(nother) determiner – after all, there is only one determiner in each of the nominal structures in (86). As the feature bundles for case, number, and gender are not reduced, this yields strong endings on the determiners (for the significance of this discussion in the context of the Scandinavian languages, see Katzir \& Siloni 2014: 278).
\end{styleStandard}

\begin{styleStandard}
\ \ In Section 3.8, we saw that determiners can co-occur with intensifying predeterminers (87a). Importantly, \textit{dieses} ‘this’ itself has a strong ending, but the possessive article is uninflected (for independent reasons, see again Section 2.2.2). However, there are also distributions where both determiner(-like) elements exhibit a strong ending at the same time (87b). Comparing (87a) to (87b), note also that the adjective is strong in the former but weak in the latter case:
\end{styleStandard}

\begin{styleStandard}
(87)\ \ a.\ \ \textit{dies-es mein groß-es \ Glück}
\end{styleStandard}

\begin{styleStandard}
\ \ \ \ this-\textsc{st} my \ \ \ great-\textsc{st} happiness.\textsc{neut}
\end{styleStandard}

\begin{styleStandard}
‘this my great happiness’
\end{styleStandard}

\begin{styleStandard}
\ \ b.\ \ \textit{all-e \ \ dies-e(*n) \ \ \ \ \ klein-en \ \ Autos}
\end{styleStandard}

\begin{styleStandard}
\ \ \ \ all-\textsc{st} these-\textsc{st/*wk} small-\textsc{wk} cars
\end{styleStandard}

\begin{styleFootnote}
\ \ \ \ ‘all these small cars’
\end{styleFootnote}

\begin{styleFooter}
Importantly, the inflections on the predeterminers are not frozen. As can be seen in instances involving the dative case, they change just like those on the determiners themselves:
\end{styleFooter}

\begin{styleStandard}
(88)\ \ a.\ \ \textit{in} \textit{dies-em mein-em groß-en \ Glück}
\end{styleStandard}

\begin{styleStandard}
\ \ \ \ in this-\textsc{st} \ my-\textsc{st} \ \ \ \ great-\textsc{wk} happiness.\textsc{neut}
\end{styleStandard}

\begin{styleStandard}
‘in this my great happiness’
\end{styleStandard}

\begin{styleFooter}
\ \ b.\ \ \textit{mit} \textit{\ \ all-en dies-en \ klein-en \ \ Autos}
\end{styleFooter}

\begin{styleStandard}
\ \ \ \ with all-\textsc{st} these-\textsc{st} small-\textsc{wk} cars
\end{styleStandard}

\begin{styleFootnote}
\ \ \ \ ‘with all these small cars’
\end{styleFootnote}

\begin{styleFooter}
Given this alternation, it is rather implausible to suggest that predeterminers and determiners form compounds. Note in this regard that Bhatt (1990: 217-218) claims that \textit{alle }and \textit{diese} are heads that form a complex adjunction structure (see also Chapter 3, Section 6). In my view, a different analysis is called for. The structural account of the strong/weak alternation offered in this chapter extends to these cases without a problem. 
\end{styleFooter}

\begin{styleStandard}
Recall that I proposed in Section 3.8 that \textit{dieses} in (87a) is predeterminer and is taken to be base-generated in LPP; that is, it is higher up in the structure. This means that this element was neither adjoined to by another determiner, nor did it adjoin to another element – again, it originated in LPP. Consequently, it neither underwent nor triggered Impoverishment. I assume that \textit{alle} in (87b) is also base-generated in LPP. With Impoverishment not occurring, this yields strong inflections on the predeterminers and on the determiners. Briefly returning to the adjectives in (87), \textit{mein} ‘my’ in (87a) does not trigger Impoverishment on the adjective as discussed in Section 2.2.1. This results in a strong ending. As to (87b), \textit{diese} ‘these’ does trigger Impoverishment bringing about a weak ending on the adjective. 
\end{styleStandard}

\begin{styleStandard}
It is worth pointing out that these are not isolated cases. An example similar to (87a) can be provided in the plural (89a), and both \textit{alle} and \textit{diese} can precede a possessive article in the same nominal (89b) (Bhatt 1990: 217-18, Vater 1991: 28-29). Again, in each case, the determiner and predeterminer(s) have a strong ending:
\end{styleStandard}

\begin{styleFooter}
(89)\ \ a.\ \ \textit{dies-e \ \ \ \ mein-e(*n) \ nett-en \ \ Freunde} 
\end{styleFooter}

\begin{styleFooter}
\ \ these-\textsc{st} my-\textsc{st/*wk} nice-\textsc{wk} friends
\end{styleFooter}

\begin{styleFooter}
‘these my nice friends.’ 
\end{styleFooter}

\begin{styleFooter}
\ \ b.\ \ \textit{all-e \ \ dies-e(*n) \ \ \ \ \ }\textit{\textsubscript{\ }}\textit{mein-e(*n) \ Freunde}
\end{styleFooter}

\begin{styleStandard}
\ \ \ \ all-\textsc{st} these-\textsc{st/*wk} my-\textsc{st/*wk} friends
\end{styleStandard}

\begin{styleFootnote}
\ \ \ \ ‘all these friends of mine’
\end{styleFootnote}

\begin{styleFooter}
I suggest that \textit{diese} in (89a) is also base-generated in LPP and that there is a second LPP in (89b) accommodating \textit{alle}. Note again that none of the determiners or predeterminers were adjoined to by another determiner. This explains why these elements do not have weak endings. Again, this is in line with the assumption that one determiner originates in ArtP and moves to the DP-level but that the second (or third) element that has the appearance of a determiner is in effect a predeterminer, a determiner-like element base-generated higher in the structure (LPP). As stated in Chapter 1, Section 4.1.2, determiner(-like) elements can be merged in the LPP provided they can function as intensifiers and are semantically compatible with the following determiner in the DP. More generally, this means that a sequence of two or three determiner(-like) elements with strong endings indicates more structure on top of the DP proper; that is, adjectival inflections are indicators of abstract structure (Hypothesis 2a).
\end{styleFooter}

\begin{styleStandard}
To sum up the current proposal, I provided diachronic and synchronic evidence that adjectival inflections in (Modern) German cannot be a reflex of (in-)definiteness. I proposed that these inflections are semantically vacuous (Hypothesis 1a). Furthermore, having shown that concord is only a necessary (but not sufficient) condition for the appearance of weak adjectives, I pointed out that these adjectives are only possible in regular, simple DPs, where the determiners adjoin to the phrases containing the adjectives. By contrast, strong inflections occur in very diverse contexts and do not seem to be subject to any special conditions. As such, they involve the elsewhere case. Indeed, we have seen distributions where the highest element or elements exhibit a strong inflection at the top of the structure although adjectives with weak inflections surface in the lower part of the structure. I argued that Impoverishment occurs locally and in a bottom-up fashion – it affects all adjectives but none of the determiners or predeterminers.
\end{styleStandard}

\begin{styleStandard}
\ \ More generally, if this is correct, then the strong/weak alternation of adjectival inflection appears to be a reflex of different structures. In fact, the different types of inflections may indicate the degree of embedding of adjectives (Spec,AgrP vs. Spec,DisP containing Spec,AgrP) and a certain amount of structure (DP vs. LPP on top of DP). The strong adjectives in the latter two types of contexts (Spec,DisP and LPP) provide evidence for Hypothesis 2a – more abstract structure is involved. Another result of the discussion is that the traditional terms weak and strong inflections have no significance other than different degrees of feature specificity. The weak endings form a subset of the strong endings in the sense of having the fewest features in their specifications. Finally, to highlight some of the points of the previous sections and to compare the current analysis to other work, I discuss three earlier proposals.
\end{styleStandard}

\begin{styleFootnote}\bfseries
5.\ \ A Brief Critique of some Previous Proposals\newline

\end{styleFootnote}

\begin{styleStandard}
Abney (1987) stimulated much important research on the noun phrase. However, many of the discussions of German have concentrated on the structure of the DP (e.g. Felix 1990; Haider 1988; Löbel 1990a,b; Vater 1991) and relatively fewer contributions have been devoted to the explanation of morpho-syntactic phenomena such as the distribution of adjectival inflections. It is perhaps telling that three monographs on the DP, Bhatt (1990), Demske (2001), and Wegener (1995), discuss only three different types of analyses at length (they all discuss Olsen 1991b). In this section, I review three proposals in more detail. I start off by examining Olsen’s seminal work (1989a, 1989b, 1991a, 1991b), and then I turn to two more recent accounts: Schoorlemmer (2009) and Murphy (2018).\footnote{\ For the discussion of other proposals, see some earlier remarks in this chapter but also Katzir \& Siloni (2014) and Roehrs (2006a: 175-191).} Note that all these analyses discuss only the canonical cases but not the non-canonical structures, and it is often not clear how the latter can be accommodated in these accounts. As laid out above, I argue, however, that it is the non-canonical structures that reveal the true nature of adjectival inflections in German.
\end{styleStandard}

\begin{styleStandard}\itshape
5.1.\ \ Olsen (1989a, 1989b, 1991a, 1991b)
\end{styleStandard}

\begin{styleStandard}
In a series of papers (1989a, 1989b, 1991a, 1991b), Olsen discusses inflection within the German DP (I refer only to her last paper as that contains the main relevant insights). Assuming Abney’s (1987) DP-Hypothesis, Olsen proposes that phi-features are located under D. These features involve person, case, number, and gender. Abbreviated as AGR, they need to be made visible. Olsen assumes that agreement within the DP is brought about by D selecting NP as its complement and by Percolation of superscripts from NP down the tree. Here are the relevant definitions (my translations):\footnote{\ These are the original definitions (Olsen 1991b: 40, 38):\par (i)\ \ a.\ \ \textit{Prinzip der morphologischen Realisierung}\par \ \ \ \ Grammatische Merkmale werden phonologisch sichtbar gemacht.\par \ \ b.\ \ \textit{Kongruenzkette}:\par Eine Kongruenzkette besteht aus einer ununterbrochenen Folge identischer \par Indizes, die auf der Basis der funktionalen Selektion entsteht, die zwischen einer \par AGR-Kategorie und ihrem Komplement erfolgt.}
\end{styleStandard}

\begin{styleStandard}
(90)\ \ a.\ \ \textit{Principle of Morphological Realization}
\end{styleStandard}

\begin{styleStandard}
\ \ \ \ Grammatical features are rendered phonologically visible.\ \ \ \ \ \ 
\end{styleStandard}

\begin{styleStandard}
\ \ b.\ \ \textit{Agreement Chain}:
\end{styleStandard}

\begin{styleStandard}
\ \ \ \ An agreement chain consists of an uninterrupted sequence of identical 
\end{styleStandard}

\begin{styleStandard}
indices which are brought about by functional selection, which holds 
\end{styleStandard}

\begin{styleStandard}
between an AGR-category and its complement.
\end{styleStandard}

\begin{styleStandard}
As an illustration, the example in (91a) is derived as in (91b):
\end{styleStandard}

\begin{styleStandard}
(91)\ \ a.\ \ \textit{da-s \ \ \ kalt-e \ \ \ \ Wetter}
\end{styleStandard}

\begin{styleStandard}
\ \ the\textsc{{}-st} cold-\textsc{wk} weather.\textsc{neut}
\end{styleStandard}

\begin{styleStandard}
\ \ ‘the cold weather’
\end{styleStandard}

\begin{styleStandard}
\ \ b.\ \  \ \ \ \ \ \ \ \ \ \ DP
\end{styleStandard}

\begin{styleStandard}
[Warning: Draw object ignored][Warning: Draw object ignored]
\end{styleStandard}

\begin{styleStandard}
\ \ \ \ D\textsuperscript{i}\ \  \ \ \ \ \ \ \ \ \ \ NP\textsuperscript{i}
\end{styleStandard}

\begin{styleStandard}
[Warning: Draw object ignored][Warning: Draw object ignored]\ \  \ \ \ \ \ \ \ \ \ \ \textit{das}
\end{styleStandard}

\begin{styleStandard}
\ \ \ \  \ \ \ \ \ \ \ \ \ \ AP\textsuperscript{i}\ \ \ \ N\textsuperscript{i}
\end{styleStandard}

\begin{styleStandard}
[Warning: Draw object ignored]\ \ \ \ \ \ \ \  \ \ \ \ \ \ \ \textit{Wetter}
\end{styleStandard}

\begin{styleStandard}
\ \ \ \ \ \ A\textsuperscript{i}
\end{styleStandard}

\begin{styleStandard}
\ \ \ \  \ \ \ \ \ \ \ \ \textit{kalte\ \ }
\end{styleStandard}

\begin{styleStandard}
AGR is made visible under D by the definite determiner \textit{das }‘the’, which has a strong inflection. An identical superscript is on NP (by functional selection) and on N and A (by Percolation). To ensure unique realization of grammatical features, Olsen follows Emonds’ (1987: 615) Invisible Category Principle:
\end{styleStandard}

\begin{styleStandard}
(92)\ \ \textit{Invisible Category Principle}
\end{styleStandard}

\begin{styleStandard}
\ \ A closed category B with positively specified features C\textsubscript{i} may remain empty 
\end{styleStandard}

\begin{styleStandard}
throughout a syntactic derivation if the features C\textsubscript{i} ... are all alternatively realized 
\end{styleStandard}

\begin{styleStandard}
in a phrasal sister of B.
\end{styleStandard}

\begin{styleStandard}
To briefly illustrate with a construction in English, the Invisible Category Principle is taken to restrict the realization of the comparative to just one overt maker:
\end{styleStandard}

\begin{styleStandard}
(93)\ \ a.\ \ [\textsubscript{DEG} more ] brightØ
\end{styleStandard}

\begin{styleStandard}
\ \ b.\ \ [\textsubscript{DEG} Ø ] bright-er
\end{styleStandard}

\begin{styleStandard}
As regards adjectival inflections, this principle allows the realization of AGR on a sister node, the adjective in (94b), and it rules out two strong endings in cases like (94c):
\end{styleStandard}

\begin{styleStandard}
(94)\ \ a.\ \ \textit{da-s \ \ \ kalt-e}Ø\textit{ \ Wetter}\ \ \ \ 
\end{styleStandard}

\begin{styleStandard}
\ \ \ \ the-\textsc{st} cold-\textsc{wk} weather.\textsc{neut}
\end{styleStandard}

\begin{styleFooter}
\ \ \ \ ‘the cold weather’
\end{styleFooter}

\begin{styleStandard}
\ \ b.\ \ [\textsubscript{D} Ø ] \textit{kalt-es \ Wetter}
\end{styleStandard}

\begin{styleStandard}
\ \ \ \  \ \ \ \ \ \ \ \ \ cold-\textsc{st} weather.\textsc{neut}
\end{styleStandard}

\begin{styleStandard}
\ \ \ \ ‘cold weather’
\end{styleStandard}

\begin{styleStandard}
\ \ c. *\ \ \textit{da-s \ \ \ kalt-es \ Wetter}
\end{styleStandard}

\begin{styleStandard}
\ \ \ \ the-\textsc{st} cold-\textsc{st} weather.\textsc{neut}
\end{styleStandard}

\begin{styleStandard}
\ \ \ \ ‘the cold weather’
\end{styleStandard}

\begin{styleStandard}
The weak ending in (94a) is assumed to be the “unmarked” inflection of the adjective (Olsen 1991b: 44). Crucially, the strong ending can only be realized on the sister node, if D is empty (94b); that is, the strong inflection can appear on the adjective (but not the noun, which cannot realize AGR). In order to rule out cases such as *\textit{ein-es kalte Wetter}, where \textit{ein} would have a strong ending in the three exceptional cases, Olsen (1991b: 47 fn. 14, 53) points out that \textit{ein} ‘a’ is part of a group of simple determiner stems (“schlichte Determinans-Stämme”) that in the case of \textit{ein}, have no inflection in the nominative masculine and in the nominative/accusative neuter. 
\end{styleStandard}

\begin{styleStandard}
As far as I am aware, this is the first detailed proposal of the strong/weak alternation in German within the framework of the DP-Hypothesis. While this is an elegant account, it is not without problems. Besides cases where AGR is present but does not have to be overtly realized (\textit{ein Auto} ‘a car’; \textit{Karls Auto} ‘Karl’s car’), unmodified mass nouns do not involve a DP-level in Olsen’s account, and thus no AGR is present. Bhatt (1990: 191) and others have pointed out that this leads to the problem that these noun phrases have to enter DP-external agreement relations without AGR. Furthermore, while Olsen (1991b: 44) states that one strong inflection in (94) exemplifies a tendency for economical realization of features or an avoidance of redundancy, it raises questions about strong endings that appear on stacked adjectives: \textit{frisch-es kalt-es Wasser} ‘fresh-\textsc{st} cold-\textsc{st} water’. 
\end{styleStandard}

\begin{styleStandard}
Finally, note that instances such as \textit{dies-en Jahr-es} ‘of this-\textsc{wk} year-\textsc{gen}’ and non-canonical structures such as \textit{dies-es mein groß-es Glück} ‘this-\textsc{st} my great-\textsc{st} happiness’ raise issues. The first case involves a determiner with a weak inflection, which is unexpected in Olsen’s system. The second instance consists of two determiner elements where the leftmost element exhibits a strong ending, but the lower adjective also shows a strong inflection. Again, this is surprising under Olsen’s assumptions (for other problems, see Wegener 1995: 159-63, Bhatt 1990: 44). 
\end{styleStandard}

\begin{styleFootnote}\itshape
5.2.\ \ Schoorlemmer (2009)
\end{styleFootnote}

\begin{styleFootnote}
Schoorlemmer proposes to account for the strong/weak alternation by employing certain aspects of the Agree relation (Chomsky 2000). His novel claim is that the agreement relation between the adjective and the noun is not direct but rather indirect; that is, it is mediated by another element. In general terms, if this element is present and mediates the agreement between the adjective and the noun, the adjectival inflection is spelled out as strong. In contrast, if this agreement relation does not hold, then the features on the adjective are not valued/specified, and the inflection is spelled out as weak. The latter instance is interpreted as a default option. This account of agreement is labeled Indirect Agree.\footnote{\ Given the complexity of the system, I cannot do full justice to all the details of Schoorlemmer’s proposal. In what follows, I limit myself to the illustration of the basic account leaving many interesting features unexplored. However, it will become clear that there are also certain shortcomings in the basic parts of the proposal.}
\end{styleFootnote}

\begin{styleFootnote}
\ \ In Chomsky’s Agree system, probes have uninterpretable features and appropriate goals have the corresponding interpretable features. If the probe c-commands the goal, the former can value/specify its features on the basis of the latter. In Schoorlemmer’s terms, the probe and the goal share their features. Given that adjectival inflections are dependent on the phi-features of other elements, Schoorlemmer proposes that they are probes for nominal phi-features.
\end{styleFootnote}

\begin{styleFootnote}
Specifically, adjectival inflections vary for gender and number. In addition, they may also exhibit properties of definiteness and case. The former (definiteness) is particularly clear in the Scandinavian languages and the latter (case) in German. Given that probes must c-command their goals, he suggests that adjectives are higher than nouns, which are specified for gender and number, but that adjectives are also higher than determiners, which are specified for definiteness (note that case is a DP-external feature). In other words, adjectives c-command not only nouns but also determiners. 
\end{styleFootnote}

\begin{styleFootnote}
Adjectives can also function as restrictive modifiers; that is, they participate in delimiting the reference of the noun phrase. As such, adjectives on their restrictive interpretation are often assumed to be in the scope of the determiner where scope is interpreted as the relevant c-command domain (for more discussion, see Chapter 4, Section 4). This leads Schoorlemmer to the following C-command Paradox:
\end{styleFootnote}

\begin{styleFootnote}
(95)\ \ Attributive adjectives with weak adjectival inflection must be c-commanded by a definite \ \ D for their interpretation, but they must c-command a definite D in order to license their \ \ inflection (Schoorlemmer 2009: 12).
\end{styleFootnote}

\begin{styleFootnote}
It appears then as if there must be two positions for determiners, one above adjectives for interpretatory purposes and one below for agreement reasons. Schoorlemmer proposes to resolve this apparent paradox by assuming that the determiner moves from the lower position to the higher position.
\end{styleFootnote}

\begin{styleFootnote}
\ \ The movement of the determiner is triggered by the presence of an adjective. Adopting the general framework of Heim \& Kratzer (1998) (see Chapter 1, Section 4.2.2), consider the simplified structure in (96). Note now that the adjective (here assumed to be of type {\textless}{\textless}e,t{\textgreater},{\textless}e,t{\textgreater}{\textgreater}) cannot combine with the lower DP (type {\textless}e{\textgreater}). The location of this type mismatch is indicated by underlining the incompatible elements. To resolve this mismatch, the determiner is moved to the left periphery:
\end{styleFootnote}

\begin{styleCaption}
\textmd{(96)\ \ \ \ DP}\textsubscript{{\textless}}\textmd{\textsubscript{e}}\textsubscript{{\textgreater}}\textmd{\ \ }
\end{styleCaption}

\begin{styleStandard}
[Warning: Draw object ignored][Warning: Draw object ignored]\ \ 
\end{styleStandard}

\begin{styleStandard}
\ \ \ \ \ \ \ \ \ \ \ det\textsubscript{{\textless}{\textless}e,t{\textgreater},e{\textgreater}k}\ \ DP
\end{styleStandard}

\begin{styleStandard}
[Warning: Draw object ignored][Warning: Draw object ignored]\ \ \ \ \ \ \ \ \ 
\end{styleStandard}

\begin{styleStandard}
\ \  \ \ \ \ \ \ \ \ \ \ \textsubscript{\ }AP\textsubscript{{\textless}{\textless}e,t{\textgreater},{\textless}e,t{\textgreater}{\textgreater}} \ \ \ DP\textsubscript{{\textless}e{\textgreater}}
\end{styleStandard}

\begin{styleStandard}
[Warning: Draw object ignored][Warning: Draw object ignored]\ \ \ \  \ {\textbar}
\end{styleStandard}

\begin{styleStandard}
\ \ \ \  A \ \ \ \ \ \ det\textsubscript{{\textless}{\textless}e,t{\textgreater},e{\textgreater}k} \ \ \ \ \ \ \ \ NP\textsubscript{{\textless}e,t{\textgreater}}
\end{styleStandard}

\begin{styleStandard}
\ \ \ \  \ \ \ \ \ \ \ \ \ \ \ \  \ {\textbar}
\end{styleStandard}

\begin{styleStandard}
\ \ \ \ \ \ \ \ \ \  N
\end{styleStandard}

\begin{styleFootnote}
At PF, the lower copy of the determiner is deleted; at LF, the lower copy of the determiner and its projection are deleted (Schoorlemmer 2009: 27). This is indicated by crossing out the relevant elements in (97): 
\end{styleFootnote}

\begin{styleCaption}
\textmd{(97)\ \ \ \ DP}\textmd{\textsubscript{{\textless}e{\textgreater}}}\textmd{\ \ }
\end{styleCaption}

\begin{styleStandard}
[Warning: Draw object ignored][Warning: Draw object ignored]\ \ 
\end{styleStandard}

\begin{styleStandard}
\ \ \ \ \ \ \ \ \ \ \ det\textsubscript{{\textless}{\textless}e,t{\textgreater},e{\textgreater}k}\ \ DP
\end{styleStandard}

\begin{styleStandard}
[Warning: Draw object ignored][Warning: Draw object ignored]\ \ \ \ \ \ \ \ \ 
\end{styleStandard}

\begin{styleStandard}
\ \  \ \ \ \ \ \ \ \ \ \ \textsubscript{\ }AP\textsubscript{{\textless}{\textless}e,t{\textgreater},{\textless}e,t{\textgreater}{\textgreater}} \ \ \ DP\textsubscript{{\textless}e{\textgreater}}
\end{styleStandard}

\begin{styleStandard}
[Warning: Draw object ignored][Warning: Draw object ignored]\ \ \ \  \ {\textbar}
\end{styleStandard}

\begin{styleStandard}
\ \ \ \  A \ \ \ \ \ \ det\textsubscript{{\textless}{\textless}e,t{\textgreater},e{\textgreater}k} \ \ \ \ \ \ \ \ NP\textsubscript{{\textless}e,t{\textgreater}}
\end{styleStandard}

\begin{styleStandard}
\ \ \ \  \ \ \ \ \ \ \ \ \ \ \ \  \ {\textbar}
\end{styleStandard}

\begin{styleStandard}
\ \ \ \ \ \ \ \ \ \  N
\end{styleStandard}

\begin{styleFootnote}
Observe that this leaves only one, the higher copy of the determiner at PF. This gives the right surface order. As for LF, with the lower copy of the determiner and its projection deleted, the adjective can now combine with NP semantically. This yields an element of type {\textless}e,t{\textgreater}, which in turn can combine with the (higher) determiner. This brings about the right semantics of the noun phrase. Basically, all noun phrases with an attributive adjective have this much in common.
\end{styleFootnote}

\begin{styleFootnote}
\ \ Turning to the strong/weak alternation, recall that Schoorlemmer claims that adjectives and nouns do not enter into an agreement relation directly. Rather, this relation is mediated by another element; that is, the agreement relation is indirect. Importantly, for the adjective not to c-command the noun and thus enter into a direct agreement relation with it, the agreement relation is taken to be based on dominance. Specifically, considering the structure in (97), the adjective does not dominate any of the nodes of the noun. I first discuss the strong inflections and then the weak endings.
\end{styleFootnote}

\begin{styleFootnote}
\ \ Note again that Indirect Agree is a function of a mediating head. Mediating heads are proposed to be case-assigners such as little \textit{v}, T and presumably other elements. These heads have uninterpretable phi-features and thus function as probes too. For concreteness, suppose that \textit{guter Wein} ‘good wine’ is the subject of a clause and that the DP-external case-assigner is T. Assuming a null article (\textit{Ø}\textit{\textsubscript{D}}), the DP in (98a) is assembled as in (98b). After merging the DP in the clause, the mediating head T probes down the tree and finds the noun phrase \textit{guter Wein}. With the adjective the closest potential goal, T inspects the features on the adjective. However, the adjective does not have interpretable features, here marked as [-phi] (I comment on the strikethrough of [-phi] in (98) further below):
\end{styleFootnote}

\begin{styleFootnote}
(98)\ \ a.\ \ \textit{gut-er \ \ \ Wein}
\end{styleFootnote}

\begin{styleFootnote}
\ \ \ \ good-\textsc{st} wine.\textsc{masc}
\end{styleFootnote}

\begin{styleFootnote}
\ \ \ \ ‘good wine’
\end{styleFootnote}

\begin{styleCaption}\mdseries
\ \ b.\ \ DP\ \ 
\end{styleCaption}

\begin{styleStandard}
[Warning: Draw object ignored][Warning: Draw object ignored]\ \ 
\end{styleStandard}

\begin{styleStandard}
\ \ \ \ \ \ \ \ \ \ \ \ \textsubscript{\ }\textit{Ø}\textit{\textsubscript{D}}\textsubscript{k}\ \ \ \ DP
\end{styleStandard}

\begin{styleStandard}
[Warning: Draw object ignored][Warning: Draw object ignored]\ \ \ \ \ \ \ \ \ 
\end{styleStandard}

\begin{styleStandard}
\ \  \ \ \ \ \ \ \ \ \ \ \textsubscript{\ }AP\ \  \ \ \ \ \ \ \ \ \ \ \ DP
\end{styleStandard}

\begin{styleStandard}
[Warning: Draw object ignored][Warning: Draw object ignored]\ \ \ \  \ {\textbar}
\end{styleStandard}

\begin{styleStandard}
\ \  \ \ \ \ \ \ \ \ \ \textit{gut}\textit{\textsubscript{[- phi]}} \ \textit{Ø}\textit{\textsubscript{D}}\textsubscript{k} \ \ \ \ \ \  \ \ \ \ \ \ \ \ \ \ \ NP
\end{styleStandard}

\begin{styleStandard}
\ \ \ \  \ \ \ \ \ \ \ \ \ \ \ \  \ {\textbar}
\end{styleStandard}

\begin{styleStandard}
\ \ \ \ \ \ \ \  \ \ \ \ \ \ \ \ \textit{Wein}\textit{\textsubscript{[+phi]}}
\end{styleStandard}

\begin{styleFootnote}
As a consequence, the probe looks further down the structure and finds the noun. Since the noun does have the relevant interpretable features, both T and the noun enter into an Agree relation, and they share their features. Now, since T had entered into an agreement relation with the adjective before, T also shares its features with the adjective. With all its uninterpretable features licensed, the adjective is spelled out with a strong inflection. This is marked by crossing out the uninterpretable phi-features in (98). To be clear, indirect agreement and feature sharing explain the strong endings (for more details, see Schoorlemmer 2009: 145-46).
\end{styleFootnote}

\begin{styleFootnote}
\ \ Turning to the weak endings, Schoorlemmer assumes that unlike null articles (Schoorlemmer 2009: 156), overt articles are probes – just like adjectives and case-assigners; that is, overt articles have uninterpretable features. This has consequences for the DP-internal agreement relations. Specifically, when the determiner probe is merged with the noun, it inspects the interpretable features of the noun. Both enter into an Agree relation and share their features. As all the features on the determiner have been specified, the determiner becomes a deactivated probe. Next, the adjective is merged and due to type mismatch, the determiner moves to the left periphery:
\end{styleFootnote}

\begin{styleFootnote}
(99)\ \ a.\ \ \textit{der gut-e \ \ \ \ Wein}
\end{styleFootnote}

\begin{styleFootnote}
\ \ \ \ the good-\textsc{st} wine.\textsc{masc}
\end{styleFootnote}

\begin{styleFootnote}
\ \ \ \ ‘the good wine’
\end{styleFootnote}

\begin{styleCaption}\mdseries
\ \ b.\ \ DP\ \ 
\end{styleCaption}

\begin{styleStandard}
[Warning: Draw object ignored][Warning: Draw object ignored]\ \ 
\end{styleStandard}

\begin{styleStandard}
\ \ \ \ \ \ \ \ \ \ \textsubscript{\ }\textit{der}\textit{\textsubscript{[- phi]}}\textsubscript{k}\ \ DP
\end{styleStandard}

\begin{styleStandard}
[Warning: Draw object ignored][Warning: Draw object ignored]\ \ \ \ \ \ \ \ \ 
\end{styleStandard}

\begin{styleStandard}
\ \  \ \ \ \ \ \ \ \ \ \ \textsubscript{\ }AP\ \  \ \ \ \ \ \ \ \ \ \ \ DP
\end{styleStandard}

\begin{styleStandard}
[Warning: Draw object ignored][Warning: Draw object ignored]\ \ \ \  \ {\textbar}
\end{styleStandard}

\begin{styleStandard}
\ \  \ \ \ \ \ \ \ \ \ \textit{gut}\textit{\textsubscript{[- phi]}} \textit{der}\textit{\textsubscript{[- phi]}}\textsubscript{k} \ \ \ \ \ \ \ \ \ \ NP
\end{styleStandard}

\begin{styleStandard}
\ \ \ \  \ \ \ \ \ \ \ \ \ \ \ \  \ {\textbar}
\end{styleStandard}

\begin{styleStandard}
\ \ \ \ \ \ \ \  \ \ \ \ \ \ \ \ \textit{Wein}\textit{\textsubscript{[+phi]}}
\end{styleStandard}

\begin{styleFootnote}
At this point, the DP is merged in the clause and T probes down the tree to find a goal. The closest goal is the determiner of the noun phrase. With its features already specified by the noun, the determiner deactivates T as probe. As both the determiner and T are deactivated, the features on the adjective remain unspecified. As a consequence, they are spelled out as a weak inflection by default. In a way, the determiner blocks the Agree relation between T and the adjective.
\end{styleFootnote}

\begin{styleFootnote}
\ \ This is, in a nutshell, Schoorlemmer’s basic proposal. It presents a new and interesting view on the strong/weak alternation. To account for all the basic data in German, Schoorlemmer (2009: chapter 5) has to make some other assumptions with regard to certain features. I do not review these finer points here. Rather, I point out again that this proposal makes some claims about the structure of the DP that are similar to the ones put forth in this book. Specifically, in Schoorlemmer’s account, the determiner is also merged in a low position and undergoes subsequent movement to the left periphery. Note though that there are some very general differences between Schoorlemmer’s and the current proposal. 
\end{styleFootnote}

\begin{styleFootnote}
For instance, while it is the strong inflections in Schoorlemmer’s analysis that are due to a specific operation (Indirect Agree), it is the weak endings in the current account that are brought about by a certain mechanism (Impoverishment). Now, as illustrated in Sections 3 and 4, the strong endings occur in more diverse contexts than the weak ones. I interpreted the strong inflections as the elsewhere case. For Schoorlemmer, the presence of a mediating head is an essential ingredient to explain the strong adjectives. This raises the question as to what the mediating head in the non-canonical constructions, say vocatives in German, is.\footnote{\ If the Voc head discussed in Section 3.9 is taken to be the relevant mediating head here, then this head is presumably absent in the Scandinavian languages, which have weak adjectives in vocatives.} In fact, some of the complex noun phrases, discussed in Section 3, involve more feature sharing relations than there seem to be mediating heads (e.g., \textit{das Sternbild Groß-er Bär} ‘the constellation Great-\textsc{st} Bear’). Again, it is not clear how strong adjectives in the different sub-structures of these nominals can be explained. In my view, these cases can be straightforwardly handled if we assume that the strong inflections involve the elsewhere case. After these general remarks, I point out some more specific issues with Schoorlemmer’s account.
\end{styleFootnote}

\begin{styleFootnote}
\ \ To motivate the movement of the determiner, semantic types are interpreted as part of the syntactic feature bundles. It appears then as if semantic elements (types) and semantic operations (Functional Application) are either relocated into syntax or duplicated there (but see also Schoorlemmer 2012). Moreover, it is not entirely obvious that little \textit{v}, T and other mediating heads have the full range of uninterpretable phi-features (e.g., gender). Finally, while I cannot discuss this in detail here, it appears that Schoorlemmer’s system works best for the Scandinavian languages but is less straightforward for German. For instance, the mixed pattern involving \textit{ein}{}-words in German is left for future research, and the cross-linguistic differences observed in canonical noun phrases involving possessives and demonstratives (Chapter 1, Section 2.1.2) do not follow from this uniform account.
\end{styleFootnote}

\begin{styleFootnote}\itshape
5.3.\ \ Murphy (2018)
\end{styleFootnote}

\begin{styleFootnote}
Murphy (2018) discusses the emergence of the strong ending on \textit{ein}{}-words in elliptical contexts; compare (100a) to (100b) (strikethrough marks ellipsis here):
\end{styleFootnote}

\begin{styleFootnote}
(100)\ \ a.\ \ \textit{ein (gut-es) \ Buch}
\end{styleFootnote}

\begin{styleFootnote}
a \ \ \ \ good-\textsc{st} book.\textsc{neut}
\end{styleFootnote}

\begin{styleFootnote}
\ \ \ \ ‘a good book’
\end{styleFootnote}

\begin{styleFootnote}
\ \ b.\ \ \textit{ein-es Buch}
\end{styleFootnote}

\begin{styleFootnote}
a-\textsc{st} \ \ \ book
\end{styleFootnote}

\begin{styleFootnote}
\ \ \ \ ‘one’
\end{styleFootnote}

\begin{styleFootnote}
He relates this change to the strong/weak alternation of adjectives. Rather than claiming that a strong ending licenses NP-ellipsis, here the strong ending is proposed to be a by-product of NP-ellipsis. Specifically, the latter creates a stranded affix that undergoes Local Dislocation onto a non-canonical element, the \textit{ein}{}-word. Murphy assumes the structure in (101).\footnote{\ Besides \textit{$\varphi $} residing in a higher position than the adjective, the adjective can also be above \textit{$\varphi $}. These two options are important for the derivation of all the cases involving ellipsis. In this regard, Murphy assumes that \textit{$\varphi $} may induce ellipsis of its complement. Now, if AP is deleted (when \textit{$\varphi $} is above A as in (101) in the main text), \textit{eines gut Buch} ‘one (good book)’ comes about; if \textit{n}P is deleted (when A is above \textit{$\varphi $} – the alternative ordering of (101)), \textit{ein gutes Buch} ‘a good one’ is brought about (for \textit{eines Buch} ‘one (book)’, see the main text below). Furthermore, note that the option of the adjective residing above \textit{$\varphi $} also accounts for some special inflectional features of adjectives like \textit{lila} ‘purple’ (see also Chapter 3, Section 3).} In his account, inflections and their hosts do not form constituents; that is, inflections are generated separately from their hosts: determiners are assumed to be in Spec,DP and adjectives in A, but strong endings are taken to be located in D and weak inflections in \textit{$\varphi $}:
\end{styleFootnote}

\begin{styleFootnote}
(101)\ \ [ [ DET ] D [ \textit{$\varphi $} [ A [ \textit{n} [ N ]]]]]
\end{styleFootnote}

\begin{styleFootnote}
\ \ \ \  \ \ \ \ {\textbar} \ \ \ \ {\textbar}
\end{styleFootnote}

\begin{styleFootnote}
\textsc{\ \ \  \ \ \ \ \ \ \ \ \ \ \ \ \ \ \ \ \ \ st \ wk}
\end{styleFootnote}

\begin{styleFootnote}
In other words, inflections are part of the extended projection line of nouns (for similar ideas, see Evans 2019: 76, Rehn 2019: 194, Wiltschko 1998). Murphy formulates a requirement that (basically) all adjectives must have an inflection in prenominal position. In order to combine the ending with the adjective or determiner, he follows assumptions from Distributed Morphology (see Chapter 1, Section 4.2.1), where Lowering precedes Local Dislocation. Specifically, he makes the suggestion that if (downward) Lowering is not possible, (leftward) Local Dislocation takes place (note that Local Dislocation is conceived of here as “leaning” to the left, Murphy 2018: 346). I illustrate the proposal by way of discussing four derivations involving \textit{ein}{}-words.
\end{styleFootnote}

\begin{styleFootnote}
\ \ I begin by discussing the most common case, where a determiner with a strong inflection precedes an adjective with a weak ending. This is exemplified by an instance in the dative neuter in (102a). The weak ending under \textit{$\varphi $} combines with the adjective by Lowering, marked by a downward pointing arrow in (102b). With the weak ending originating under \textit{$\varphi $}, the strong ending in D cannot undergo Lowering. Consequently, it combines with \textit{ein} by Local Dislocation. This is indicated by an upward pointing arrow (despite the fact that this mechanism operates on lineary adjacency). This derives (102a) as in (102b) below:
\end{styleFootnote}

\begin{styleFootnote}
(102)\ \ a.\ \ \textit{ein-em gut-en \ \ \ \ Buch}
\end{styleFootnote}

\begin{styleFootnote}
a-\textsc{st} \ \ \ \ good-\textsc{wk} book.\textsc{neut}
\end{styleFootnote}

\begin{styleFootnote}
\ \ \ \ ‘a good book’
\end{styleFootnote}

\begin{styleFootnote}
\ \ b.\ \  DP
\end{styleFootnote}

\begin{styleFootnote}
[Warning: Draw object ignored][Warning: Draw object ignored][Warning: Draw object ignored]
\end{styleFootnote}

\begin{styleFootnote}
\textit{ein}\ \  \ \ \  \ \ D’
\end{styleFootnote}

\begin{styleFootnote}\itshape
[Warning: Draw object ignored][Warning: Draw object ignored]
\end{styleFootnote}

\begin{styleStandard}
\ \ \ \  \ \ D\ \ \ \  \ \textit{$\varphi $}P
\end{styleStandard}

\begin{styleStandard}\itshape
[Warning: Draw object ignored]{}-em [Warning: Draw object ignored][Warning: Draw object ignored]
\end{styleStandard}

\begin{styleStandard}
\ \ \ \ \ \  \ \textit{$\varphi $}\ \  \ \ \  \ AP
\end{styleStandard}

\begin{styleStandard}
\textit{\ \ \ \ \ \ \ \ \ \ }[Warning: Draw object ignored][Warning: Draw object ignored]\textit{\ \ {}-en}
\end{styleStandard}

\begin{styleStandard}
\ \ \ \ \ \ \ \  \ A\ \ \ \  \ \textit{n}P
\end{styleStandard}

\begin{styleStandard}
[Warning: Draw object ignored][Warning: Draw object ignored]\ \ \ \ \ \ \ \ \textit{gut}
\end{styleStandard}

\begin{styleStandard}
\ \ \ \ \ \ \ \ \ \  \ \ \textit{n}\ \ \ \  \ NP
\end{styleStandard}

\begin{styleStandard}
\textit{\ \ \ \ \ \ \ \ \ \ \ \ \ \ \ Buch}
\end{styleStandard}

\begin{styleFootnote}
To account for the idiosyncracy of the three special cases of \textit{ein}{}-words, where the adjective is strong (103a), it is stipulated that the strong ending is in \textit{$\varphi $ }(103b), rather than in D (Murphy 2018: 356). Recalling that Lowering precedes Local Dislocation, the strong ending is displaced onto the adjective deriving (103a) as in (103b):
\end{styleFootnote}

\begin{styleFootnote}
(103)\ \ a.\ \ \textit{ein gut-es \ \ \ Buch}
\end{styleFootnote}

\begin{styleFootnote}
a \ \ \ good-\textsc{st} book.\textsc{neut}
\end{styleFootnote}

\begin{styleFootnote}
\ \ \ \ ‘a good book’
\end{styleFootnote}

\begin{styleFootnote}
\ \ b.\ \  DP
\end{styleFootnote}

\begin{styleFootnote}
[Warning: Draw object ignored][Warning: Draw object ignored]
\end{styleFootnote}

\begin{styleFootnote}
\textit{ein}\ \  \ \ \  \ \ D’
\end{styleFootnote}

\begin{styleFootnote}\itshape
[Warning: Draw object ignored][Warning: Draw object ignored]
\end{styleFootnote}

\begin{styleStandard}
\ \ \ \  \ \ D\ \ \ \  \ \textit{$\varphi $}P
\end{styleStandard}

\begin{styleStandard}\itshape
[Warning: Draw object ignored]\ [Warning: Draw object ignored][Warning: Draw object ignored]
\end{styleStandard}

\begin{styleStandard}
\ \ \ \ \ \  \ \textit{$\varphi $}\ \  \ \ \  AP
\end{styleStandard}

\begin{styleStandard}
\textit{\ \ \ \ \ \ \ \ \ \ }[Warning: Draw object ignored][Warning: Draw object ignored]\textit{\ \ {}-es}
\end{styleStandard}

\begin{styleStandard}
\ \ \ \ \ \ \ \  \ A\ \ \ \  \ \textit{n}P
\end{styleStandard}

\begin{styleStandard}
[Warning: Draw object ignored][Warning: Draw object ignored]\ \ \ \ \ \ \ \ \textit{gut}
\end{styleStandard}

\begin{styleStandard}
\ \ \ \ \ \ \ \ \ \  \ \ \textit{n}\ \ \ \  \ NP
\end{styleStandard}

\begin{styleStandard}
\textit{\ \ \ \ \ \ \ \ \ \ \ \ \ \ \ Buch}
\end{styleStandard}

\begin{styleFootnote}
Removing the adjective from the noun phrase yields (104a). As just seen, the strong ending undergoes Lowering but this time onto \textit{n }(104b). The result is spelled out as null (marked by parentheses below):
\end{styleFootnote}

\begin{styleFootnote}
(104)\ \ a.\ \ \textit{ein Buch}
\end{styleFootnote}

\begin{styleFootnote}
a \ \ \ book.\textsc{neut}
\end{styleFootnote}

\begin{styleFootnote}
\ \ \ \ ‘a book’
\end{styleFootnote}

\begin{styleFootnote}
\ \ b.\ \  DP
\end{styleFootnote}

\begin{styleFootnote}
[Warning: Draw object ignored][Warning: Draw object ignored]
\end{styleFootnote}

\begin{styleFootnote}
\textit{ein}\ \  \ \ \  \ \ D’
\end{styleFootnote}

\begin{styleFootnote}\itshape
[Warning: Draw object ignored][Warning: Draw object ignored]
\end{styleFootnote}

\begin{styleStandard}
\ \ \ \  \ \ D\ \ \ \  \ \textit{$\varphi $}P
\end{styleStandard}

\begin{styleStandard}\itshape
[Warning: Draw object ignored]\ [Warning: Draw object ignored][Warning: Draw object ignored]
\end{styleStandard}

\begin{styleStandard}
\ \ \ \ \ \  \ \textit{$\varphi $}\ \  \ \ \  \textit{n}P
\end{styleStandard}

\begin{styleStandard}
\textit{\ \ \ \ \ \ \ \ \ \ }[Warning: Draw object ignored][Warning: Draw object ignored]\textit{\ }(\textit{{}-es})
\end{styleStandard}

\begin{styleStandard}
\ \ \ \ \ \ \ \  \ \textit{n}\ \ \ \  \ NP
\end{styleStandard}

\begin{styleStandard}
\textit{\ \ \ \ \ \ \ \ \ \ \ Buch}
\end{styleStandard}

\begin{styleFootnote}
Finally and most importantly, I turn to the cases where the noun is elided (105a) (for the ellipsis involving adjectives, see Footnote Error: Reference source not found). With \textit{n}P undergoing ellipsis, the strong ending cannot undergo Lowering. Rather, Local Dislocation applies as a kind of repair mechanism, and the strong ending surfaces on \textit{ein} (105b) (the arch below sets off the elided material):
\end{styleFootnote}

\begin{styleFootnote}
(105)\ \ a.\ \ \textit{ein-es}
\end{styleFootnote}

\begin{styleFootnote}
one-\textsc{st} \ \ \ \ 
\end{styleFootnote}

\begin{styleFootnote}
[Warning: Draw object ignored]\ \ \ \ ‘one’
\end{styleFootnote}

\begin{styleFootnote}
\ \ b.\ \  DP
\end{styleFootnote}

\begin{styleFootnote}
[Warning: Draw object ignored][Warning: Draw object ignored]
\end{styleFootnote}

\begin{styleFootnote}
\textit{ein}\ \  \ \ \  \ \ D’
\end{styleFootnote}

\begin{styleFootnote}\itshape
[Warning: Draw object ignored][Warning: Draw object ignored][Warning: Draw object ignored]
\end{styleFootnote}

\begin{styleStandard}
\ \ \ \  \ \ D\ \ \ \  \ \textit{$\varphi $}P
\end{styleStandard}

\begin{styleStandard}\itshape
\ [Warning: Draw object ignored][Warning: Draw object ignored]
\end{styleStandard}

\begin{styleStandard}
\ \ \ \ \ \  \ \textit{$\varphi $}\ \  \ \ \  \textit{n}P
\end{styleStandard}

\begin{styleStandard}
\textit{\ \ \ \ \ \ \ \ \ \ }[Warning: Draw object ignored][Warning: Draw object ignored]\textit{\ \ {}-es}
\end{styleStandard}

\begin{styleStandard}
\ \ \ \ \ \ \ \  \ \textit{n}\ \ \ \  \ NP
\end{styleStandard}

\begin{styleStandard}
\ \ \ \ \ \ \ \ \ \ \ \ \textit{Buch}
\end{styleStandard}

\begin{styleFootnote}
This is a novel proposal accounting for the emergence of the strong ending on \textit{ein}. As such, it is a welcome contribution.\footnote{\ There are other formal proposals seeking to explain the emergence of the strong inflection on \textit{ein} (for the discussion of Evans 2019: chapter 3, see Roehrs 2021).} Note though that not all aspects of the proposal are entirely straightforward.
\end{styleFootnote}

\begin{styleFootnote}
\ \ To obtain the correct surface forms, a number of adjustments have to take place. In particular, there are three instances where an inflectional suffix is spelled out as null: (i) as seen above, if a strong ending combines with \textit{n }by Lowering, it is spelled out as null yielding \textit{ein Buch} ‘a book’, (ii) if both D and \textit{$\varphi $ }have an inflection but there is no adjective present in elliptical contexts, the weak ending in \textit{$\varphi $ }is spelled out as null bringing about \textit{mein-em} ‘mine’ (Murphy’s page 358), and (iii) if Spec,DP is empty, the strong ending in D appears on the adjective, and the weak ending in \textit{$\varphi $ }is spelled out as null resulting in \textit{heißer Kaffee} ‘hot coffee’ (his page 362).\footnote{\ The difference between the weak inflection being spelled out as null in (ii) and (iii) is that in (ii), the weak ending undergoes – what Murphy calls – Morphological Ellipsis, and that in (iii), the weak ending is contextually spelled out as null (as Spec,DP is empty). In my view, allowing these adjustments opens the door for other issues. Assuming the structural option with A above \textit{$\varphi $ }(Footnote Error: Reference source not found), it is not clear how to rule out the following ungrammatical example in the dative (where the strong ending is in D):\par (i) \ *\ \ \textit{ein gut-em \ \ Buch}\par \ \ \ \ a \ \ \ good-\textsc{st} book.\textsc{neut}\par With Lowering taking precedence, the strong ending (in D) combines with the adjective. Given that the weak ending in \textit{$\varphi $} can, at least in principle, be spelled out as null, it is not clear to me how to rule this case out without stipulation. In my view, these issues are avoided if adjectives and inflections do not form individual heads in the extended projection line of the noun (but rather inflections are part of the structures of the determiners and adjectives themselves, as assumed here). }
\end{styleFootnote}

\begin{styleFootnote}
Note that Murphy’s proposal is not compatible with the current analysis as all his determiners including \textit{ein}{}-words are in Spec,DP, and adjectives are in head positions. More importantly for current purposes, there are not many details provided as to what regulates the strong/weak alternation. While the Principle of Monoinflection is mentioned, it is not clear what derives (rather than stipulates) the fact that the strong ending must precede the weak one(s). From Murphy’s mechanism of partial copying, it is clear though that weak endings are taken as impoverished features that involve a subset of the features that spell out the strong endings. To be fair, while Murphy’s goal was not to provide those types of details, such information along with vocabulary insertion rules would be desirable. 
\end{styleFootnote}

\begin{styleFootnote}
It is also worth pointing out that there are two ways for adjectival endings to come about in Murphy’s account: besides originating in D or \textit{$\varphi $}, strong and weak endings can also be the result of copying (accounting for multiple inflected adjectives). Furthermore, note that the stipulation to capture the idiosyncracy of the three special \textit{ein}{}-words (i.e., the strong ending is exceptionally located in \textit{$\varphi $}) may involve a non-local relation when the adjective intervenes (i.e., when A is above \textit{$\varphi $}, which is the required ordering to generate \textit{ein gutes Buch} ‘a good one’). This means that selection cannot be invoked here. Moreover, the three cases of \textit{ein} do not seem to have anything in common (unlike in the current account where they share features with a fourth instance of \textit{ein}).
\end{styleFootnote}

\begin{styleFootnote}
\ \ Finally, I turn to two empirical issues that arise with non-canonical structures. First, split topicalizations are analyzed as either NP-ellipsis or NP-movement. However, it is not clear how to account for the two strong endings in (106a). As plural forms, these types of \textit{ein}{}-words do not involve a strong ending in \textit{$\varphi $} (but only in D). It is clear that a second D must be present in the higher nominal. However, this is unexpected on NP-ellipsis or NP-movement. Second, pronominal DPs like (106b) present another challenge: although D is present here, the strong inflection is absent. It appears as if the latter is, for some reason, not spelled out. Note that this strong ending did not combine with \textit{n} by Lowering as the adjective and \textit{$\varphi $} are present:
\end{styleFootnote}

\begin{styleFootnote}
(106)\ \ a.\ \ \textit{Nett-e \ \ Leute \ \ waren kein-e da}.
\end{styleFootnote}

\begin{styleFootnote}
\ \ \ \ nice-\textsc{st} people were \ \ no-\textsc{st} \ there
\end{styleFootnote}

\begin{styleFootnote}
\ \ \ \ ‘As for nice people, there were none there.’
\end{styleFootnote}

\begin{styleFootnote}
\ \ \ b. \ \ \ \textit{wir nett-en \ Studenten} 
\end{styleFootnote}

\begin{styleFootnote}
we nice-\textsc{wk} students
\end{styleFootnote}

\begin{styleFootnote}
‘we nice students’
\end{styleFootnote}

\begin{styleFootnote}
\ \ To sum up, the three proposals reviewed in this section discuss the canonical patterns with different degrees of success and explanatory force. All accounts have an explanation of the strong inflections, but the status of the weak endings is less clear: this type of ending is taken to be the “unmarked” inflection (Olsen), the default inflection (Schoorlemmer), or it is base-generated or due to a certain type of partial copying (Murphy).\footnote{\ G. Müller (2002a: 129) makes an explicit distinction between markers of case (strong inflection) and agreement (weak inflection).} Recall in this regard that there are actually two weak endings, -\textit{e} and -\textit{en}, and that their distributions do not relate to natural groups. In my view, it is important to account for the distribution of the weak inflections as well. In the current account, the strong and weak inflections have the same status, the difference being that the (traditional) weak inflections are less specified than the (unambiguous) strong endings. 
\end{styleFootnote}

\begin{styleFootnote}
More importantly, none of these proposals address the non-canonical cases, and as pointed out above, it is not clear how some of these instances can be accounted for by those analyses. As such, while these proposals are quite elegant, their empirical coverage is somewhat limited. As mentioned above, in my view, empirical coverage should be taken into account when evaluating the plausibility of a proposal. Indeed, one of the main claims of this book is that it is these non-canonical cases that reveal the true nature of adjectival endings in German. While this may turn out to be wrong, all analyses should eventually discuss these non-canonical structures to explore the consequences of those analyses for these more complex structures.
\end{styleFootnote}

\begin{styleStandard}\bfseries
6.\ \ Conclusion
\end{styleStandard}

\begin{styleStandard}
One goal of this book is to provide a detailed survey of the strong/weak alternation of adjectives in German and to determine the exact conditions for the emergence of the weak and strong endings. In the course of the discussion, I isolated one structure, the simple (canonical) DP, where weak endings occur. Illustrating that concord is a necessary (but not a sufficient) condition, it was proposed that the weak endings are underlyingly fully specified feature bundles that get reduced by Impoverishment. Impoverishment is triggered by determiners, which move across the specifier positions containing the adjectives. \newline
\ \ I also investigated nine contexts where strong endings surface. In each case, some independent evidence was provided that suggests that different structures are indeed involved. Importantly, arguing that Impoverishment proceeds locally and in a bottom-up fashion, I suggested that in these different structures, the relevant feature bundles remain unreduced and are spelled out as the strong endings. In that sense, the strong endings are the elsewhere case. This explains their diverse occurrences. The previous proposals discussed in the final section of this chapter do not discuss the whole range of data and as a consequence, reach very different conclusions. Another goal of this book is to draw some more general conclusions. 
\end{styleStandard}

\begin{styleStandard}
\ \ In this chapter, I showed that adjectival inflections in German do not correlate with (in-)definiteness. I proposed that these inflections are semantically vacuous (Hypothesis 1a). I argued that the strong/weak alternation is a reflex of different structures (Hypothesis 2a): on the one hand, weak endings can only occur in Spec,AgrP; on the other hand, strong inflections can be in Spec,AgrP, in DP, and in other, non-canonical positions (e.g., LPP, Spec,DisP, etc.). If this is on the right track, then we can utilize the strong/weak alternation as a diagnostic for the structures of other nominals. With this in mind, I turn to some consequences of the analysis above for the current account in Chapter 3 and to some consequences for other proposals in Chapter 4. The discussion of these consequences will reveal other properties of adjectival inflections.
\end{styleStandard}

\clearpage\setcounter{page}{124}\begin{styleFootnote}
Chapter 3: Variation and Secondary Mechanisms
\end{styleFootnote}

\begin{styleFootnote}\bfseries
1. \ \ Introduction
\end{styleFootnote}

\begin{styleFootnote}
This chapter and the next continue the discussion of adjectival inflections begun in the previous chapter. While the current chapter focuses on consequences for and extensions of the present proposal, the next chapter discusses implications for other analyses. In more detail, this chapter addresses data involving variation and how this variation can be accounted for in the current system. I examine two types: (i) variation within colloquial Standard German and (ii) variation as regards different regional dialects with special focus on Mannheim German. I provide a brief overview of the relevant cases before I engage in a more detailed discussion.\footnote{\ It is not an easy task to explain variation, especially optionality, in a formal system. Thus, while certain parts of the discussion below are admittedly somewhat tentative, I discuss new data and offer novel (and I hope, interesting) generalizations and proposals.}
\end{styleFootnote}

\begin{styleFootnote}
\ \ The first type of variation involves both canonical and non-canonical constructions. In canonical DPs, strong, weak, or no inflections are possible on adjectives and determiners. This variation can only occur in certain structures. Strong or weak inflections are possible on the second adjective of two stacked modifiers when a determiner is missing, on adjectives in pronominal DPs, and on certain \textit{der}{}-words. Strong or no inflections occur on certain other \textit{der}{}-words (e.g., \textit{d-er} vs. \textit{die} both meaning ‘the/that’). 
\end{styleFootnote}

\begin{styleFootnote}
Importantly, this type of variation only surfaces in specific, well-defined featural contexts in these structures; that is, this variation within colloquial Standard German is restricted to certain combinations of “case + gender” or “case + number”. I argue that there is another mechanism besides Impoverishment: a phonetic rule. Furthermore, the application of Impoverishment Rule 2 is extended, a phonological constraint (avoidance of a hiatus) is held to explain certain unexpected determiner forms, and Impoverishment is proposed not to be triggered by pronominal determiners. In addition, I address the optional inflections on certain adjectives (e.g., \textit{lila} ‘purple’), and I discuss the absence of inflections on some other adjectives (e.g., \textit{prima} ‘great’).
\end{styleFootnote}

\begin{styleFootnote}
\ \ There is also variation within colloquial Standard German in non-canonical structures. In this respect, I discuss the inflectional options on the predeterminer \textit{all-} ‘all’, where in the plural, the ending can be present or absent in all four morphological cases. All these constructions, their restricted variation, and their analyses are briefly summarized in Table 1 (\textsc{infl} indicates where the variation occurs):
\end{styleFootnote}

\begin{styleFootnote}
Table 1: Restricted Variation in Colloquial Standard German
\end{styleFootnote}

\begin{flushleft}
\begin{tabular}{|m{1.6212599in}|m{0.9837598in}|m{2.1712599in}|m{1.5587599in}|}

\hline
\centering Construction &
\centering Ending &
\centering Restrictions &
\centering\arraybslash Analysis \\\hline
{\fontsize{10pt}{12.0pt}\selectfont adj + adj-\textsc{infl }+ noun} &
weak (unexpected) &
only in dative masc/neut (in sgl) &
phonetic rule

\\\hline
{\fontsize{10pt}{12.0pt}\selectfont det-\textsc{infl} + noun} &
weak (unexpected) &
{\fontsize{10pt}{12.0pt}\selectfont only in genitive masc/neut (in sgl); certain \textit{der}{}-words} &
Impoverishment Rule 2\\\hline
{\fontsize{10pt}{12.0pt}\selectfont det-\textsc{infl} + noun} &
absent &
{\fontsize{10pt}{12.0pt}\selectfont only in nom/acc in fem/pl; \textit{die} ‘the/that/those’ } &
deletion of inflection

(avoid hiatus)\\\hline
{\fontsize{10pt}{12.0pt}\selectfont det + adj-\textsc{infl }+ noun } &
(optional/) absent &
{\fontsize{10pt}{12.0pt}\selectfont lexically restricted to a few adjectives (e.g., \textit{lila} ‘purlple’)} &
(optional) deletion of inflection\\\hline
{\fontsize{10pt}{12.0pt}\selectfont pron + adj-\textsc{infl }+ noun} &
strong (unexpected) &
(expected) weak endings only occur in dat sgl and nom pl  &
ST: pron. det.; WK: phonetic rule, analogy \\\hline
{\fontsize{10pt}{12.0pt}\selectfont predet-\textsc{infl} noun} &
optional &
{\fontsize{10pt}{12.0pt}\selectfont \textit{all}{}- ‘all’} &
Deletion of inflection in LPP\\\hline
\end{tabular}
\end{flushleft}
\begin{styleFootnote}
As part of the discussion of the absence of inflections on \textit{die} ‘the/that/those’, I also address the different stem forms of the definite article and the almost homophonous distal demonstrative (i.e., \textit{de}{}-\textit{r}, \textit{da}{}-\textit{s}, \textit{di}{}-), and I relate the distal and proximal demonstratives to the definite article. 
\end{styleFootnote}

\begin{styleFootnote}
\ \ Studying Table 1, we note that some variation is found in the oblique cases in masculine and neuter contexts and that other variation occurs in the structural cases in the feminine and plural. Indeed, the workings of the phonetic rule and Impoverishment Rule 2 are restricted to certain constructions and specific featural combinations making them different from Impoverishment Rule 1. Consequently, I conclude below that Impoverishment Rule 1 is the primary mechanism and that the other two are secondary ones. Based on the discussion of this variation, I review the traditional generalizations of the strong/weak alternation.
\end{styleFootnote}

\begin{styleFootnote}
The second type of variation concerns regional varieties, here focusing on one dialect – Mannheim German (for some brief remarks on Alemannic German, see Chapter 8, Section 2.1). This variation can be captured in the current system: while it involves Impoverishment Rule 1 as formulated in Chapter 2, some of the vocabulary insertion rules provided there are deleted here or have a different featural context of application and, as a consequence, are reordered. These adjustments to the vocabulary insertion rules account for the inflectional distribution in this dialect. 
\end{styleFootnote}

\begin{styleFootnote}
\ \ More generally, I propose in this chapter that adjectival inflections in German do not receive a homogenous account but rather are the result of (at least) three different mechanisms, one primary (Impoverishment Rule 1) and two secondary ones (phonetic rule, Impoverishment Rule 2). Furthermore, this chapter provides more evidence that adjectival inflections do not mark (in-)definiteness; that is, they are semantically vacuous elements (Hypothesis 1a). 
\end{styleFootnote}

\begin{styleFootnote}
\ \ The chapter is organized as follows. First, I turn to instances involving different inflections on two unpreceded adjectives. Section 3 is dedicated to the inflectional properties of certain adjectives and certain \textit{der}{}-words, and to the stem forms of the definite article and the demonstratives. In Section 4, I take up the variation of inflections on certain other \textit{der}{}-words. Section 5 discusses the variation in pronominal DPs. In Section 6, I address the presence or absence of the strong inflection on the predeterminer \textit{all-} ‘all’. Section 7 returns to the discussion of the traditional generalizations Weak After Strong and the Principle of Monoinflection. In Section 8, I provide a comparison between Standard German and the German dialect of Mannheim, and Section 9 closes the chapter with a summary of the main inflectional patterns in colloquial Standard German and their respective analyses. 
\end{styleFootnote}

\begin{styleStandard}\bfseries
2.\ \ Canonical DPs with Unexpected Weak Adjectives: Phonetic Rule
\end{styleStandard}

\begin{styleStandard}
I start the investigation with the variation that has seen the most interest in the theoretical literature. As such, I begin by discussing noun phrases that lack an overt determiner but involve two adjectives in a row where the first adjective exhibits a strong ending, but the second shows a weak one. This inflectional alternation concerns nasal sounds, and I provide a phonetic rule suggesting that this involves phonetic markedness reduction. In the second subsection, this discussion is extended to indefinite pronoun constructions.
\end{styleStandard}

\begin{styleStandard}\itshape
2.1.\ \ Two Adjectives Without a Determiner
\end{styleStandard}

\begin{styleStandard}
In the previous chapter, we saw examples where two co-occurring adjectives have the same ending, with either both endings being strong (1a) or both endings being weak (1b): 
\end{styleStandard}

\begin{styleFootnote}
(1)\textit{ }\ \ a.\textit{\ \ frisch-er schwarz-er Kaffee}
\end{styleFootnote}

\begin{styleFootnote}
\ \ \ \ fresh-\textsc{st} \ black-\textsc{st} \ \ \ \ coffee.\textsc{masc}
\end{styleFootnote}

\begin{styleFootnote}
\ \ \ \ ‘fresh black coffee’
\end{styleFootnote}

\begin{styleFootnote}
b.\ \ \textit{d-er \ \ \ \ frisch-e schwarz-e Kaffee}
\end{styleFootnote}

\begin{styleFootnote}
\ \ \ \ the-\textsc{st} hot-\textsc{wk} \ black-\textsc{wk} \ coffee.\textsc{masc}
\end{styleFootnote}

\begin{styleStandard}
\ \ \ \ ‘the fresh black coffee’
\end{styleStandard}

\begin{styleStandard}
Recall that \textit{der} ‘the’ in (1b) involves [+DEF], a context where [+D] triggers Impoverishment. With the determiner moving from ArtP to DP in a successive-cyclic fashion, the CNG features of the two adjectives undergo Impoverishment and are spelled out as weak. As for (1a), Impoverishment does not occur, and the underlying features are spelled out as the strong endings.
\end{styleStandard}

\begin{styleStandard}
\ \ There is one much-discussed exception to the pattern in (1a) above: in the dative masculine (and neuter), there is the option of the second adjective exhibiting a weak ending. Comparing (2a-b), we observe that (2a) exhibits the expected pattern but that (2b) shows two adjectives in a row that have different endings. The latter is often referred to as non-parallel distribution. This is in stark contrast to all the other patterns I discussed above. As is clear from (2c-d), patterns with a weak ending on the first adjective are not possible (also Nübling 2011, Sahel 2021), independently of whether the second adjective is strong or weak. I call the inflectional option of -\textit{em} vs. -\textit{en} nasal alternation. Note also that contrasting the examples involving the different nasal inflections yields less sharp grammaticality judgments (\% indicates variation with speakers; see Table 4 below):\footnote{\ Nübling (2011: 182) states that the coexistence of the parallel and non-parallel adjective strings has a long history, possibly several hundred years. Note that this co-occurrence seems to have existed already in ENHG (see Ebert \textit{et al}. 1993: 201-02). However, it is hard to tell when this phenomenon started as there is some independent variation with adjectives in the earlier stages of German (e.g., Peter 2013, Sahel 2021). Sahel (2021) observes that (2b) is the only non-parallel string that continues occurring after the 18\textsuperscript{th} century.\par \ \ Also, Nübling (2011: 183) finds in her corpus study that the parallel string occurs slightly more often (57\%) and that all classes of adjectives have this variation (albeit adjectives in coordinations much less so; more on this in Footnote Error: Reference source not found). Interestingly, there is a slight correlation such that adjective classes that occur closer to the noun (in the nominal extended projection) tend to be inflected weak more often than those further away (her page 187), provided they are preceded by a (strong) adjective. The corpus studies by Peter (2013) and Sahel (2021) offer partially different results, but both Peter and Sahel point out that non-parallel strings seem to be on the increase.}
\end{styleStandard}

\begin{styleStandard}
\ (2)\textit{ \ \ }a. \ \%\ \ \textit{mit \ \ frisch-em schwarz-em Kaffee}\ \ 
\end{styleStandard}

\begin{styleFootnote}
\ \ \ \ with fresh-\textsc{st} \ \ black-\textsc{st} \ \ \ \ \ coffee.\textsc{masc}
\end{styleFootnote}

\begin{styleFootnote}
\ \ \ \ ‘with fresh black coffee’
\end{styleFootnote}

\begin{styleFootnote}
\ \ b. \ \%\ \ \textit{mit \ \ frisch-em schwarz-en Kaffee}
\end{styleFootnote}

\begin{styleFootnote}
\ \ \ \ with fresh-\textsc{st} \ \ black-\textsc{wk} \ \ \ coffee.\textsc{masc}
\end{styleFootnote}

\begin{styleFootnote}
\ \ \ \ ‘with fresh black coffee’
\end{styleFootnote}

\begin{styleFootnote}
\ \ c. \ *?\ \ \textit{mit \ \ frisch-en schwarz-em Kaffee}
\end{styleFootnote}

\begin{styleFootnote}
\ \ \ \ with fresh-\textsc{wk} black-\textsc{st} \ \ \ \ \ coffee.\textsc{masc}
\end{styleFootnote}

\begin{styleFootnote}
\ \ d. \ *?\ \ \textit{mit \ \ frisch-en schwarz-en Kaffee}
\end{styleFootnote}

\begin{styleFootnote}
\ \ \ \ with fresh-\textsc{wk} black-\textsc{wk} \ \ \ coffee.\textsc{masc}
\end{styleFootnote}

\begin{styleFootnote}
Sternefeld (2004: 290) points out that the alternation in (2a-b) is also possible with Saxon Genitives:
\end{styleFootnote}

\begin{styleFootnote}
(3)\ \ a.\ \ \textit{Johanns \ gut-em \ \ alt-em Wein}
\end{styleFootnote}

\begin{styleFootnote}
\ \ \ \ Johann’s good-\textsc{st} old-\textsc{st} wine.\textsc{masc}
\end{styleFootnote}

\begin{styleStandard}
\ \ \ \ ‘Johann’s good old wine’
\end{styleStandard}

\begin{styleFootnote}
b.\ \ \textit{Johanns \ gut-em \ \ alt-en \ \ Wein}
\end{styleFootnote}

\begin{styleFootnote}
\ \ \ \ Johann’s good-\textsc{st} old-\textsc{wk} wine.\textsc{masc}
\end{styleFootnote}

\begin{styleStandard}
\ \ \ \ ‘Johann’s good old wine’
\end{styleStandard}

\begin{styleStandard}
As in the canonical cases (e.g., \textit{d-em frisch-en Kaffee} ‘the-\textsc{st} fresh-\textsc{wk} coffee’), the strong ending precedes the weak one yielding a left-to-right asymmetry in (2b) and (3b). Notice also that the weak ending in (2b) and similar instances below cannot mark definiteness.
\end{styleStandard}

\begin{styleStandard}
Duden (2007: 38) claims that the weak adjective occurs most frequently when the second adjective and noun form a whole concept (“Gesamtbegriff”). Nübling (2011) suggests that the weak adjective strengthens the tendency to mark the left bracket in the noun phrase (cf. the sentence bracket in German). However, Sahel (2021: 9-14) argues convincingly that there is no semantic or functional difference between the two variants. Having said that, these patterns do have a number of peculiar properties. First, as just mentioned, the grammaticality judgments are less sharp than in the cases discussed in the previous chapter. Second, these inflectional distributions are restricted to the two instances above (i.e., dative masculine/neuter). For instance, a weak ending on a second adjective is not possible in the feminine gender:
\end{styleStandard}

\begin{styleStandard}
(4)\textit{ \ \ }a.\ \ \textit{mit \ \ gut-er \ \ \ rot-er \ Sauce}
\end{styleStandard}

\begin{styleStandard}
\ \ \ \ with good-\textsc{st} red-\textsc{st} sauce.\textsc{fem}
\end{styleStandard}

\begin{styleStandard}
‘with good red sauce’
\end{styleStandard}

\begin{styleStandard}
\ \ b. *\ \ \textit{mit \ \ gut-er \ \ \ rot-en \ \ Sauce}
\end{styleStandard}

\begin{styleStandard}
\ \ \ \ with good-\textsc{st} red-\textsc{wk} sauce.\textsc{fem}
\end{styleStandard}

\begin{styleStandard}
Third, as noted in Roehrs (2009b), different authors have reported different possibilities for (2a-b). I refer to this inter-speaker variation as dialects. Considering the two inflectional options in Table 2 (column 1), Gallmann (1996: 296, 2004: 156), Schlenker (1999: 119), and Demske (2001: 53) describe dialect 1 where a strong or a weak ending is equally possible on the second adjective. G. Müller (2002a: 139) discusses dialect 2 where the strong ending on the second adjective is preferred. Finally, Schlenker (1999) reports that some speakers have dialect 3 where the weak ending on the second adjective is strongly preferred. Dialect 4 does not exist. In other words, all speakers have at least one of the two inflectional distributions in their language:
\end{styleStandard}

\begin{styleStandard}
Table 2: Different Dialects for the Dative Masculine/Neuter 
\end{styleStandard}

\begin{flushleft}
\begin{tabular}{|m{1.2462599in}|m{1.1087599in}|m{1.0740598in}|m{1.0809599in}|m{1.1087599in}|}

\hline
String &
\centering Dialect 1 &
\centering Dialect 2 &
\centering Dialect 3 &
\centering\arraybslash Dialect 4\\\hline
Adj+\textit{em} Adj+\textit{em} &
\centering ${\surd}$ &
\centering ${\surd}$ &
\centering ?? &
\centering\arraybslash {}-\\\hline
Adj+\textit{em} Adj+\textit{en} &
\centering ${\surd}$ &
\centering ? &
\centering ${\surd}$ &
\centering\arraybslash {}-\\\hline
\end{tabular}
\end{flushleft}
\begin{styleStandard}
The weak ending -\textit{en} cannot be the result of Impoverishment. Recall that Impoverishment is a local process that occurs within a phrase (AgrP). Non-local effects are brought about by determiner movement (Impoverishment Rule 1) or a specific featural context (Impoverishment Rule 2). Both options affect all adjectives in the same way. In the cases discussed here, the weak ending occurs only on the second adjective and seems to be parasitic on the presence of the first adjective. I return to this statement in Section 2.2. 
\end{styleStandard}

\begin{styleStandard}
\ \ An account of this alternation should answer the questions as to why this is only possible with adjectives ending in -\textit{em} but not, for instance, with adjectives in -\textit{es} instead or why this alternation is not possible with adjectives in -\textit{em} and -\textit{es}. In this regard, Sahel (2021: 19-20) points out that -\textit{m} and -\textit{n} have the most similarities as compared to all other strong/weak inflectional pairs (e.g., -\textit{er} vs. -\textit{en }as in (4a-b)). In order to explain the above properties, I proposed in Roehrs (2009b) that the alternation in (2a-b) and (3a-b) follows from a phonetic rule, according to which the inflection on the second adjective is changed from -\textit{m} to -\textit{n} (+ indicates a morpheme boundary; \# signifies a word boundary):
\end{styleStandard}

\begin{styleStandard}
(5)\textit{ \ \ Rule (preliminary version):}\ \ 
\end{styleStandard}

\begin{styleStandard}
m\ \ →\ \ n\ \ / […]\textsubscript{A} […]\textsubscript{A} +\_\_\_\_\#\ \ 
\end{styleStandard}

\begin{styleStandard}
This rule turns a labial sound into its coronal (/alveolar) counterpart; that is, the place of articulation of this inflection is altered. Note in this regard that R. Wiese (1996a: 165) states that “alveolar is taken to be the default place of articulation for consonants”. Utilizing this, R. Wiese formulates rules where coronal is the default value (his page 219). In a similar vein, de Lacy (2006: 35-42) argues that coronal is a less marked place of articularion than labial or dorsal. Specifically, -\textit{n} is considered the unmarked nasal sound as compared to -\textit{m}. I suggest that the rule above is a reflex of markedness reduction, where a more marked sound is changed into a less marked sound (e.g., de Lacy 2006: 23, 75-76). For similar explanations of the data in (2a-b), see also Evans (2019: 78-79) and Sahel (2021).\footnote{\ In an earlier version of this work, I took this rule to involve phonetic simplification, presumably a reflex of ease of articulation. Joe Salmons (p.c.) pointed out to me that there are issues with such an idea (see Salmons 2021: 39-41, also references cited therein). He and Tracy Hall suggested to me that reduction in markedness might be more appropriate. In this vein, note that Gallmann (2004: 156) formulates a phonological markedness constraint in his optimality-theoretic proposal that bans German words that end in schwa + /m/ (cf. also Gunkel \textit{et al}. 2017: 1308).\par \ \ Note that -\textit{m} and -\textit{n} are diachronically and dialectally related. Braune \& Reiffenstein (2004: 120-22) point out for OHG that diachronically -\textit{m} changes to -\textit{n} in suffixal endings (if \textit{n} changes, it tends to change to \textit{l}). Paul \textit{et al}. (1989: 146) show for MHG that final -\textit{m} also changes to -\textit{n}, except with strong adjectival endings. In certain dialects, though, specifically in Middle Franconian, adjectival -\textit{em(e)} is changed to -\textit{en} in dative masculine/neuter contexts (Paul \textit{et al}. 1989: 211-12). ENHG also shows the relatedness of these two elements with certain nouns changing their final sound (e.g., \textit{besem} {\textgreater} \textit{besen} ‘broom’), but there is also general variation with articles, pronominal elements, and adjectives (Ebert \textit{et al}. 1993: 140, 192-93). With this in mind, we might expect that these two sounds are also dialectically related. This is borne out in that contemporary Southern Hessian dialects retain \textit{Besem} for \textit{Besen} ‘broom’ (Schirmunski 2010: 433). In my view, this diachronic and dialectal relatedness of -\textit{m} and -\textit{n} strengthens the proposal that the nasal alternation is a phonetic/phonological phenomenon (also Peter 2013, Sahel 2021: foontoe 6). I thank Tracy Hall, Joe Salmons, and Laura Smith for help with this section including providing references.}
\end{styleStandard}

\begin{styleStandard}
\ \ Returning to Table 2, the application of the rule is optional in dialect 1, “costly” in dialect 2, and obligatory in dialect 3.\footnote{\ The grammaticality judgments in Table 2 are taken from the original sources. It is not clear to me if the degree of the markedness/ungrammaticality of the respective bad patterns in dialect 2 or 3 is the same. If the relevant case in dialect 2 were marked more than “?”, we could suggest that the rule in (5) simply does not apply in that dialect at all.} Thus, this rule accounts for the unexpected weak adjectives in canonical DPs of dialects 1 and 3. Given this rule, non-nasal inflections are immediately excluded explaining why this pattern is so restricted in occurrence. The left-to-right asymmetry follows from the way the rule above is stated – only the second adjectives may undergo the phonetic change.\footnote{\ Besides Nübling (2011), Peter (2013: 194) points out that parallel inflections are much more frequent in coordinations, which includes two adjectives separated by a comma. Specifically, parallel inflections occur 39\% in non-coordinations vs. 79\% in coordinations. Björn Köhnlein (p.c.) makes an interesting observation in this regard. In non-coordinated noun phrases, main phrasal stress usually goes to the head noun, but there is a secondary stress on the initial element (this is probably determined rhythmically). In the current cases, the initial element is the first adjective. This means that the second adjective~has the lowest degree of prominence. This low degree of prominence may facilitate the reduction from -\textit{m} to -\textit{n}. In contrast, if there is a pause between the two adjectives as in coordinations, then the second adjective is not in a weak prosodic position but tends to have its own stress. This may make the retention of -\textit{m} more likely.} 
\end{styleStandard}

\begin{styleStandard}
There are two possible alternatives: First, the left-to-right asymmetry might also have to do with easier parsing because a disambiguating, strong ending occurs first in the nominal string. Matthias Schlesewsky (p.c.) points out to me that this claim about more effective processing is in agreement with work by John Hawkins. In particular, Hawkins (1994: 404-05; 2004: 49, 92-93) explains the left-to-right asymmetry seen with the examples above by general processing advantages (for an early intuition in this regard, see Esau 1973: 139). As a second alternative, we could follow G. Müller (2002a: 140) claiming that this asymmetry has to do with analogy with the canonical pattern \textit{d-em nett-en Mann} ‘the-\textsc{st} nice-\textsc{wk} man’. 
\end{styleStandard}

\begin{styleStandard}
The cases in the next subsection present a challenge to both of these alternatives. Indeed, what often goes unmentioned is that this nasal alternation is more general; for instance, it can also be found with indefinite pronoun constructions (and pronominal DPs, see Section 5). In my view, these are related phenomena that should be discussed in the same context and receive the same account (note again that all other authors only discuss the contrast in (2), with the qualification that Sternefeld 2004 also discusses (3)). Unlike the two alternatives, the phonetic rule above is able to capture the above-mentioned cases too once it is slightly modified. 
\end{styleStandard}

\begin{styleStandard}\itshape
2.2.\ \ Indefinite Pronoun Constructions Revisited
\end{styleStandard}

\begin{styleStandard}
In Chapter 2, Section 3.2, I discussed indefinite pronoun constructions. To extend this discussion, consider cases involving the pronoun \textit{etwas} ‘something’ and a following adjective. In either the nominative/accusative (6a-b) or the dative (6c-d), there is no variation – only a strong adjective is possible:
\end{styleStandard}

\begin{styleStandard}
(6)\textit{ \ \ }a.\ \ \textit{etwas \ \ \ \ \ \ \ \ \ \ \ \ \ \ \ \ \ ander-es}
\end{styleStandard}

\begin{styleStandard}
\ \ \ \ something.\textsc{neut} different-\textsc{st}
\end{styleStandard}

\begin{styleStandard}
‘something different’
\end{styleStandard}

\begin{styleStandard}
\ \ b. *\ \ \textit{etwas \ \ \ \ \ \ \ \ \ \ \ \ \ \ \ \ \ ander-e}
\end{styleStandard}

\begin{styleStandard}
\ \ \ \ something.\textsc{neut} different-\textsc{wk}
\end{styleStandard}

\begin{styleStandard}
\ \ c.\ \ \textit{mit \ \ etwas \ \ \ \ \ \ \ \ \ \ \ \ \ \ \ \ \ ander-em}\ \ 
\end{styleStandard}

\begin{styleStandard}
\ \ \ \ with something.\textsc{neut} different-\textsc{st}
\end{styleStandard}

\begin{styleStandard}
‘with something different’
\end{styleStandard}

\begin{styleStandard}
\ \ d. \ ?*\ \ \textit{mit \ \ etwas \ \ \ \ \ \ \ \ \ \ \ \ \ \ \ \ \ ander-en}\ \ 
\end{styleStandard}

\begin{styleStandard}
\ \ \ \ with something.\textsc{neut} different-\textsc{wk}
\end{styleStandard}

\begin{styleStandard}
It was proposed in the previous chapter that the nominative string \textit{jemand anderer} ‘someone different’ is analyzed as in (7). I now add (6a) to the analysis (recall that IPRP stands for Indefinite Pronoun Restrictor Phrase; for the internal makeup of indefinite pronouns, see also Footnote Error: Reference source not found):
\end{styleStandard}

\begin{styleStandard}
(7)\textit{ \ \ Indefinite Pronoun Construction}
\end{styleStandard}

\begin{styleStandard}
\ \ \ \  \ \ \ \ \ \ DP
\end{styleStandard}

\begin{styleStandard}
[Warning: Draw object ignored][Warning: Draw object ignored]
\end{styleStandard}

\begin{styleStandard}
\ \  \ D\ \ \ \ \ \ IPRP
\end{styleStandard}

\begin{styleStandard}\itshape
[Warning: Draw object ignored][Warning: Draw object ignored][Warning: Draw object ignored]\ \  
\end{styleStandard}

\begin{styleStandard}
\ \ \ \ \ \ IPRP\ \  \ \ \ \ \ \ \ \ \ \ ModP
\end{styleStandard}

\begin{styleStandard}
[Warning: Draw object ignored][Warning: Draw object ignored][Warning: Draw object ignored][Warning: Draw object ignored]\ \ \ \ \ \ \ \ \ \  \ \ \ 
\end{styleStandard}

\begin{styleStandard}
\ \ \ \ IPR\ \ \ \ NP\ \ Mod\ \  \ \ \ \ \ AgrP
\end{styleStandard}

\begin{styleStandard}
\ \ \ \  \ {\textbar}\ \ \ \  \ {\textbar}\ \ \ \  \ \ \ \ \ \ \ \ {\textbar}
\end{styleStandard}

\begin{styleStandard}
\ \ \ \ \ \ \ \ \ \ \  \textit{je} \ \ \ \ \ \textit{{}-mand}\ \  \ \ \textit{e}\textit{\textsubscript{N}}\textsubscript{\ \ \ \ }[\textit{anderer} \textit{e}\textit{\textsubscript{N}}]
\end{styleStandard}

\begin{styleStandard}
\ \ \ \ \ \ \ \ \ \ \  \textit{et} \ \ \ \ \ \textit{{}-was}\ \  \ \ \textit{e}\textit{\textsubscript{N}}\textsubscript{\ \ \ \ }[\textit{anderes} \textit{e}\textit{\textsubscript{N}}]
\end{styleStandard}

\begin{styleStandard}
Given right adjunction to the pronoun, the adjective is only expected to have a strong ending. However, Duden (2007: 491) points out that there is variation of the adjective in the dative when the adjectives follows \textit{jemand} ‘someone’ and \textit{niemand} ‘no one’, and the same goes for the two pronouns themselves.\footnote{\ There is also variation with indefinite pronouns in the accusative (e.g., \textit{jemanden} vs. \textit{jemand}), but the following adjective always ends in the expected inflection -\textit{en}. Furthermore, as far as I know, there is no variation with the dative pronoun \textit{ihm} ‘him’; in other words, this pronoun cannot appear as \textit{ihn} in dative contexts. In addition, this element is not part of the discussion of inflections on adjectives, as third-person pronominal elements cannot be followed by adjectives (Section 5.1). Given these remarks, the cases mentioned in this footnote are not relevant to the discussion of the nasal alternation.} I start by discussing the pronoun \textit{jemand }in more detail. This is followed by addressing the inflections on the adjective.
\end{styleStandard}

\begin{styleStandard}
\ \ Although \textit{jemand} ‘someone’ is the first element in the noun phrase, this pronoun shows the nasal alternation (8a-b). In addition, it can also occur without an ending (8c):
\end{styleStandard}

\begin{styleStandard}
(8)\ \ a.\ \ \textit{mit} \ \ \textit{jemand-em}
\end{styleStandard}

\begin{styleStandard}
with someone.\textsc{masc}{}-\textsc{st}\ \ \ \ 
\end{styleStandard}

\begin{styleStandard}
‘with someone’
\end{styleStandard}

\begin{styleStandard}
\ \ b.\ \ \textit{mit} \ \ \textit{jemand-en}
\end{styleStandard}

\begin{styleStandard}
with someone.\textsc{masc}{}-\textsc{wk} 
\end{styleStandard}

\begin{styleStandard}
‘with someone’
\end{styleStandard}

\begin{styleStandard}
\ \ c.\ \ \textit{mit} \ \ \textit{jemand}
\end{styleStandard}

\begin{styleStandard}
with someone.\textsc{masc}
\end{styleStandard}

\begin{styleStandard}
‘with someone’
\end{styleStandard}

\begin{styleStandard}
The same options can be found with \textit{niemand} ‘no one’. It is clear that this inflectional alternation is not due to the phonetic rule proposed above. On the one hand, this element occurs in first position; on the other hand, it is not an adjective: syntactically, \textit{jemand} can appear in argument position; morphologically, it can have a strong, a weak, or no ending at all. Rather, Roehrs (2009b) argues that \textit{jemand} belongs to three different morphological paradigms. Besides the older declensions that are similar to those of strong nouns and determiners, there seems to be a third paradigm in the progress of developing. In the latter case, \textit{jemand} has features of a weak noun where the ending -\textit{en} is generalized throughout the non-nominative cases. The three paradigms are summarized in Table 3:\footnote{\ Note that Duden (2007: 491) claims that dative pronouns in -\textit{en} are not part of the standard language. It is clear though that such pronouns occur in the colloquial standard language. Notice also that the different inflections in Table 3 presumably indicate different internal makeups of the pronoun. For instance, \textit{jemand} may involve a determiner component (ia), but it may also involve a determiner component with an indefinite pronoun restrictor where the latter has features of a weak noun (ib) or strong noun (ic): \par (i)\ \ a.\ \ [\textsubscript{D} \textit{jemand}]-\textit{em}\par \ \ \ \ b.\ \ [\textsubscript{D} \textit{je} ] + [\textsubscript{IPR} \textit{mand} ]-\textit{en}\par \ \ \ \ c.\ \ [\textsubscript{D} \textit{je} ] + [\textsubscript{IPR} \textit{mand} ]\par The analyses in (ib-c) indicate how I interpret the terms weak noun and strong noun in Table 3 – the latter are grammaticalized nouns under IPR in (7).} 
\end{styleStandard}

\begin{styleStandard}
Table 3: Different Paradigms of the Indefinite Pronoun \textit{jemand}
\end{styleStandard}

\begin{flushleft}
\begin{tabular}{|m{1.5837599in}|m{1.5837599in}|m{1.5837599in}|m{1.5837599in}|}

\hline
 &
\centering Strong Noun &
\centering Determiner &
\centering\arraybslash Weak Noun\\\hline
Nominative &
\centering jemand &
\centering jemand (/w-er) &
\centering\arraybslash jemand\\\hline
Accusative &
\centering jemand &
\centering jemand-en &
\centering\arraybslash jemand-en\\\hline
Dative &
\centering jemand &
\centering jemand-em &
\centering\arraybslash jemand-en\\\hline
Genitive &
\centering jemand-s &
\centering jemand-(e)s &
\centering\arraybslash jemand-en\\\hline
\end{tabular}
\end{flushleft}
\begin{styleStandard}
\ \ Adding an adjective after the pronoun retains the variation on the pronoun itself. In addition, the adjective also alternates between a strong and weak ending yielding six different options. Below, I provide my own grammaticality judgments:
\end{styleStandard}

\begin{styleStandard}
(9)\textit{ \ \ }a. (?)\ \ \textit{mit \ \ jemand-em \ \ \ \ \ \ \ \ \ \ \ \ ander-em} \ \ \ \ \ \ \ \ \ \ \ \ \ \ \ \ \ 
\end{styleStandard}

\begin{styleStandard}
\ \ \ \ with someone.\textsc{masc}{}-\textsc{st} different-\textsc{st}
\end{styleStandard}

\begin{styleStandard}
‘with someone different’
\end{styleStandard}

\begin{styleStandard}
\ \ b.\ \ \textit{mit \ \ jemand-em \ \ \ \ \ \ \ \ \ \ \ \ ander-en}
\end{styleStandard}

\begin{styleStandard}
\ \ \ \ with someone.\textsc{masc}{}-\textsc{st} different-\textsc{wk}
\end{styleStandard}

\begin{styleStandard}
‘with someone different’
\end{styleStandard}

\begin{styleStandard}
(10)\textit{ \ \ }a. (?)\ \ \textit{mit \ \ jemand-en \ \ \ \ \ \ \ \ \ \ \ \ \ \ ander-em} \ \ \ \ \ \ \ \ \ \ \ \ \ 
\end{styleStandard}

\begin{styleStandard}
\ \ \ \ with someone.\textsc{masc}{}-\textsc{wk} different-\textsc{st}
\end{styleStandard}

\begin{styleStandard}
‘with someone different’
\end{styleStandard}

\begin{styleStandard}
\ \ b. (?)\ \ \textit{mit \ \ jemand-en \ \ \ \ \ \ \ \ \ \ \ \ \ \ ander-en}
\end{styleStandard}

\begin{styleStandard}
\ \ \ \ with someone.\textsc{masc}{}-\textsc{wk} different-\textsc{wk}
\end{styleStandard}

\begin{styleStandard}
‘with someone different’
\end{styleStandard}

\begin{styleStandard}
(11)\ \ a. \ \ \textit{mit \ \ jemand \ \ \ \ \ \ \ \ \ \ \ \ \ ander-em} \ \ \ \ \ \ \ \ \ \ \ \ \ 
\end{styleStandard}

\begin{styleStandard}
\ \ \ \ with someone.\textsc{masc} different-\textsc{st}
\end{styleStandard}

\begin{styleStandard}
‘with someone different’
\end{styleStandard}

\begin{styleStandard}
\ \ b. (?)\ \ \textit{mit \ \ jemand \ \ \ \ \ \ \ \ \ \ \ \ \ ander-en}
\end{styleStandard}

\begin{styleStandard}
\ \ \ \ with someone.\textsc{masc} different-\textsc{wk}
\end{styleStandard}

\begin{styleStandard}
‘with someone different’
\end{styleStandard}

\begin{styleStandard}
All these patterns occur naturally but with different frequencies. Table 4a presents the numeric results of an informal \textit{google}{}-search involving three dative prepositions: \textit{mit} ‘with’, \textit{von} ‘from’, and \textit{bei} ‘at’ (note that while I checked quite a few examples for relevance, the totals are too large to examine all the individual instances). While these numbers are, admittedly, not precise, the tendencies are clear. Whereas all patterns above occur, a weak adjective is more frequent with a pronoun involving an inflection, be it -\textit{em} or -\textit{en }(cf. (9b), (10b)). In contrast, a strong adjective is more frequent with a pronoun involving no inflection (cf. (11a)). Notice also that the latter is the most frequent pattern overall (more on these points below). Finally, it is worth pointing out that the positive pronoun, separated by a forward slash sign in Table 4a, is more frequent than its negative counterpart.
\end{styleStandard}

\begin{styleStandard}
Table 4a: Numeric Results of IPC in the Dative Identified by \textit{Google} (October 7, 2020)
\end{styleStandard}

\begin{flushleft}
\begin{tabular}{|m{2.13796in}|m{2.13796in}|m{2.13796in}|}

\hline
Preposition and Pronoun &
\centering{\itshape anderem} &
\centering\arraybslash{\itshape anderen}\\\hline
{\itshape mit jemandem/niemandem} &
\centering 35K/425 &
\centering\arraybslash 95K/5K\\\hline
{\itshape von jemandem/niemandem} &
\centering 8K/711 &
\centering\arraybslash 63K/50K\\\hline
{\itshape bei jemandem/niemandem} &
\centering 3K/9 &
\centering\arraybslash 10K/347\\\hline
{\itshape mit jemanden/niemanden} &
\centering 9K/238 &
\centering\arraybslash 75K/3K\\\hline
{\itshape von jemanden/niemanden} &
\centering 7K/350 &
\centering\arraybslash 32K/16K\\\hline
{\itshape bei jemanden/niemanden} &
\centering 1K/109 &
\centering\arraybslash 14K/503\\\hline
{\itshape mit jemand/niemand} &
\centering 755K/129K &
\centering\arraybslash 408K/69K\\\hline
{\itshape von jemand/niemand} &
\centering 762K/195K &
\centering\arraybslash 362K/19K\\\hline
{\itshape bei jemand/niemand} &
\centering 141K/19K &
\centering\arraybslash 108K/816\\\hline
\end{tabular}
\end{flushleft}
\begin{styleStandard}
In addition to the \textit{google}{}-search, I examined these constructions in \textit{Datenbank für Gesprochenes Deutsch} (DGD; Database of Spoken German). The results are provided in Table 4b. Note that Table 4b is similar to Table 4a above. However, the prepositions were removed from the first column and added to the relevant cells. The number of occurrences of these prepositions in these constructions is provided behind the preposition (the one instance of R after the preposition \textit{von} ‘of’ indicates a speaker in the corpus \textit{Australiendeutsch}, who is probably the interviewer, but no information is available here; IO stands for indirect object). Since the number of hits involving the two indefinite pronouns and the adjective \textit{ander}{}- ‘different’ was fairly low, I extended the search to other adjectives. These cases are provided in the second row of each form of the two pronouns (as above, the forward slash sign separates the results of the positive pronoun from those of the negative pronoun). Consider the results in more detail.
\end{styleStandard}

\begin{styleStandard}
While all three different forms of the positive and negative indefinite pronouns (strong, weak, uninflected) can be found in dative contexts when the adjective is absent, this differs when the adjective is present. For instance, in this search, I did not find any examples of a pronoun with a weak ending followed by an adjective with a strong ending. With that in mind, note though that the numeric distributions in Table 4b are similar to those in Table 4a: inflected pronouns occur, with one exception, with weak adjectives, but uninflected pronouns occur most frequently with strong adjectives (18 strong vs. 5 weak). Also similar to above, examples involving uninflected pronouns are more frequent than those with inflected pronouns. 
\end{styleStandard}

\begin{styleStandard}
Table 4b: Numeric Results of IPC in the Dative Identified in DGD (October 10, 2024)
\end{styleStandard}

\begin{flushleft}
\begin{tabular}{|m{2.1129599in}|m{2.07266in}|m{2.07126in}|}

\hline
Pronoun &
\centering{\itshape anderem} &
\centering\arraybslash{\itshape anderen}\\\hline
{\itshape jemandem/niemandem} &
 &
{\centering 2 / 1\par}

\centering\arraybslash (\textit{bei }2) / (\textit{aus})\\\hline
 &
{\itshape mit jemandem Deutschem} &
{\itshape vor jemandem Fremden}\\\hline
{\itshape jemanden/niemanden} &
 &
{\centering 1 /-\par}

\centering\arraybslash (\textit{von }R) /-\\\hline
 &
 &
{\itshape mit jemanden Fremden}\\\hline
{\itshape jemand/niemand} &
{\centering 12 /-\par}

\centering (\textit{bei }2; \textit{mit }4; \textit{von }4; \textit{zu}; IO) /- &
{\centering 1 /-\par}

\centering\arraybslash (\textit{mit}) /-\\\hline
 &
\textit{mit jemand neuem/ Deutschem/ Fremdem} (2);

{\itshape von jemand Fremdem/ fremdem} &
\textit{mit jemand andern} (3);

{\itshape von jemand fremden}

\\\hline
\end{tabular}
\end{flushleft}
\begin{styleStandard}
Note that all constructions involving weak adjectives were checked.\footnote{\ There were two additional examples in the dative with a weak adjective (\textit{bei jemand anderen}, \textit{jemand anderen}). These were uttered by speakers whose first language is not German, and they were not counted in Table 4b above.} There is one example where the pronoun and the adjective both have a strong inflection. This instance was uttered by a speaker of Swiss German from Zurich. Finally, the weak form \textit{andern} ‘different’ (where schwa is missing in the inflection) was also found in another weak context (with at least one speaker). Now, while the numbers are not large, there are two types of instances where the distribution of the strong and weak adjectives can be directly compared. 
\end{styleStandard}

\begin{styleStandard}
First, the weak adjective \textit{fremd-en} ‘strange-\textsc{wk}’ occurs one time each with strong \textit{jemandem}, weak \textit{jemanden}, and uninflected \textit{jemand}. In other words, the weak adjective occurs independently of the inflection on \textit{jemand}. In contrast, the strong adjective \textit{fremd-em} ‘strange-\textsc{st}’ occurs four times with uninflected \textit{jemand}. In fact, there is a minimal pair where \textit{jemand} occurs either with weak \textit{fremd}{}-\textit{en} or strong \textit{fremd}{}-\textit{em}. Note that the latter scenario is more general with the second type of instances, which involve \textit{jemand} and \textit{ander}{}- ‘different’. In these minimal pairs, (uninflected) \textit{jemand} occurs several times each with weak \textit{ander-(e)n} (4 times) and with strong \textit{ander}{}-\textit{em }(12 times). 
\end{styleStandard}

\begin{styleStandard}
Analyzing the results of Tables 4a and 4b, it is clear that the adjectives exhibit the nasal alternation basically independent of whether the pronoun has a strong, a weak, or no inflection. For the cases discussed in Section 2.1, this means that the second adjective in -\textit{en} should not be taken to be parasitic on a preceding adjective ending in -\textit{em} (for such an idea, see Murphy’s 2018 proposal of partial copying). Rather, the weak adjective depends on a preceding element more generally. Furthermore, indefinite pronoun constructions have a structure different from canonical DPs. This suggests that the nasal alternation is independent of a specific structure. If these cases are indeed related by the nasal alternation, then no strong morpho-syntactic claims should be made on the basis of the data in Section 2.1. Note in this regard that Schlenker (1999) argues for top-down derivations and Sternefeld (2004: 288-90) for recursive DP-levels. 
\end{styleStandard}

\begin{styleStandard}
\ \ As seen above, the word preceding the (lower) adjective may not only be an adjective but also a pronoun. What these elements have in common is that they are nominal in nature. I modify the phonetic rule above slightly (the feature [+N] stands for nominal): 
\end{styleStandard}

\begin{styleStandard}
(12)\textit{ \ \ Rule (final version):}\ \ 
\end{styleStandard}

\begin{styleStandard}
m\ \ →\ \ n\ \ / […]\textsubscript{[+N]} […]\textsubscript{A} +\_\_\_\_\#\ \ 
\end{styleStandard}

\begin{styleStandard}
In words, an adjective preceded by another nominal element (in the same DP) may occur with the inflection -\textit{en}. Notice that the rule does not apply to adjectives that have already undergone Impoverishment due to a preceding determiner. The latter has changed the ending on the adjective from -\textit{em} to -\textit{en} independently. The varying inflections on the pronouns are due to the three paradigms above.\footnote{\ As my focus is on adjectives, I do not reformulate the three paradigms of the indefinite pronouns as vocabulary insertion rules.} 
\end{styleStandard}

\begin{styleStandard}
\ \ At first glance, the different frequencies in Tables 4a and 4b seem to confirm the left-to-right asymmetry noted in Section 2.1 (i.e., a strong inflection precedes a weak one). While (13a) is generally preferred over (13b), it is worth repeating that the strong ending on the adjective is preferred over the weak one in (13a) but vice versa in (13b). In other words, recalling the three paradigms of \textit{jemand} ‘someone’ in Table 3, a strong inflection is preferred on the first element that can take an inflection (the percentage sign indicates a less frequent option, which I interpret as a less preferred string):
\end{styleStandard}

\begin{styleStandard}
(13)\ \ a.\ \ \textit{jemand} \ \ \ \ \ \ \ \ \ \ \ \ \ \textit{ander}{}-\textit{em}/\textsuperscript{\%}{}-\textit{en}
\end{styleStandard}

\begin{styleStandard}
\ \ \ \ someone.\textsc{masc} different-\textsc{st}/-\textsc{wk}\ \ 
\end{styleStandard}

\begin{styleStandard}
\ \ \ \ ‘someone different’
\end{styleStandard}

\begin{styleStandard}
b. \ \%\ \ \textit{jemand-em/-en} \ \ \ \ \ \ \ \ \ \ \ \ \ \textit{ander}{}-\textit{en}/\textsuperscript{\%}{}-\textit{em}
\end{styleStandard}

\begin{styleStandard}
\ \ \ \ someone.\textsc{masc}{}-\textsc{st}/-\textsc{wk} different-\textsc{wk}/-\textsc{st}
\end{styleStandard}

\begin{styleStandard}
\ \ \ \ ‘someone different’
\end{styleStandard}

\begin{styleStandard}
However, unlike in Section 2.1, the different frequencies seen with the indefinite pronoun construction are only tendencies, rather than clearcut left-to-right asymmetries. In fact, an element with a weak ending can precede one with a strong ending (e.g., \textit{jemand-en ander-em} ‘someone-\textsc{wk} different-\textsc{st}’), and a weak adjective can occur despite the fact that there is no element with a strong inflection preceding it (e.g., \textit{jemand(-en) ander-en} ‘someone(-\textsc{wk})\textit{ }different-\textsc{wk}’). Recall that these inflectional distributions are not possible with two adjectives in a row. This means that explanations involving parsing or analogy as mentioned in Section 2.1 cannot be the whole story. I submit that the rule in (12) is better able to capture these inflectional distributions. Indeed, this rule is extended in Section 5 to pronominal DPs like \textit{mir nett-em/}\textit{\textsuperscript{\%}}\textit{{}-en Studenten} ‘me (nice-\textsc{st/}\textsc{\textsuperscript{\%}}\textsc{{}-wk} student)’.
\end{styleStandard}

\begin{styleStandard}
To sum up, we arrive then at a first secondary mechanism – a phonetic rule – to explain inflectional variation, here accounting for unexpected weak adjectives in canonical DPs. Note that the weak endings here occur in a combination of certain featural and lexical contexts (i.e., an adjective in the genitive masculine/neuter is preceded by another nominal element). Next, I turn to some special cases that share a phonological constraint but are restricted to certain lexical items.\ \ 
\end{styleStandard}

\begin{styleFootnote}\bfseries
3.\ \ Canonical DPs Involving Adjectives with Optional or no Inflections and Definite \ \ Determiners with no Inflections
\end{styleFootnote}

\begin{styleFootnote}
In this section, I discuss two types of variation. I address the special inflectional behavior of certain adjectives, and I analyze some uninflected forms of the definite article and its related distal demonstrative. On the basis of the second discussion, I relate the distal and proximal demonstratives to the definite article in terms of their decompositions. More generally, I propose that all these cases have something in common: they exhibit the avoidance of a hiatus – the occurrence of two adjacent vowels in a prosodic word.
\end{styleFootnote}

\begin{styleFootnote}
\ \ In Chapter 1, Section 3.1.1, I briefly illustrated some adjectives exhibiting special inflectional behaviors. Recall that there are three types of cases. First, the adjectives \textit{lila} ‘purple’ and \textit{rosa} ‘pink’ participate in the strong/weak alternation, but the inflection is optional: it can be absent, and it can be present (14a-b). In order to account for the presence of -\textit{n}{}-, I assume for convenience that the latter consonant has been epenthesized.\footnote{\ Björn Köhnlein (p.c.) points out that -\textit{n}{}- is not a typical epenthetic consonant (the glottal stop being the usual element).} Inflected forms without \textit{n}{}-epenthesis are not possible (14c):
\end{styleFootnote}

\begin{styleStandard}
(14)\ \ a.\ \ \textit{das lila(-n-e) \ \ \ \ \ Kleid}
\end{styleStandard}

\begin{styleStandard}
\ \ \ \ the purple-n-\textsc{wk }dress.\textsc{neut}
\end{styleStandard}

\begin{styleStandard}
\ \ \ \ ‘the purple dress’
\end{styleStandard}

\begin{styleStandard}
\ \ b.\ \ \textit{ein lila(-n-es) \ \ Kleid} 
\end{styleStandard}

\begin{styleStandard}
\ \ \ \ a \ \ \ purple-n-\textsc{st }dress.\textsc{neut}
\end{styleStandard}

\begin{styleStandard}
\ \ \ \ ‘a purple dress’
\end{styleStandard}

\begin{styleFootnote}
\ \ c. \ *\ \ \textit{ein} \textit{lila-es \ \ \ \ \ Kleid}
\end{styleFootnote}

\begin{styleFootnote}
a \ \ \ purple-\textsc{st }dress.\textsc{neut}
\end{styleFootnote}

\begin{styleFootnote}
There are two ways to analyze this optionality. On the one hand, we could assume that InflP of the extended projection of the adjective is built optionally: if InflP is not projected, we obtain the uninflected forms; if InflP is projected, we get the inflected forms, provided \textit{n}{}-epenthesis occurs. On the other hand, we could suggest that InflP is built in every instance. In this scenario, the inflection is deleted yielding the uninflect forms, or the inflection is not deleted, provided \textit{n}{}-epenthesis occurs.\footnote{\ This deletion could be made more formal by assuming that Impoverishment deletes either all the features of the Infl head, or it deletes a feature that all inflections have in common (perhaps a nominal feature like [+N]). } Note that either option involves \textit{n}{}-epenthesis and results in the avoidance of a hiatus. 
\end{styleFootnote}

\begin{styleFootnote}
Turning to the second case, certain adjectives never take inflections. This holds for \textit{prima} ‘great’, \textit{sexy} ‘sexy’, \textit{super} ‘super’, and others (Duden 1995: 256-57). Note that \textit{n}{}-epenthesis cannot rescue these cases (e.g., *\textit{prima-n-es}, *\textit{sexy-n-es}). Observe that many of these cases seem to be borrowings from other languages. Following the idea of a reviewer, I assume that these cases are not fully intergrated into the language yet. To make this idea more formal, I suggest that InflP is not projected in these instances (note in this regard that the emergence of InflP on top of an adjective takes time; see Sapp \& Roehrs 2016 for the discussion of \textit{viele} ‘many’ changing from the uninflected noun \textit{vil} in OHG to the inflected quantificational adjective emerging in the 16\textsuperscript{th} century).
\end{styleFootnote}

\begin{styleFootnote}
\ \ Third, toponymic formations where adjectives are derived from place names by adding -\textit{er} never take inflections either, for instance \textit{Berlin-er-(*es) }‘(from) Berlin’. Again, there are two options to analyze this. We could suggest that either this derivational suffix blocks the projection of InflP, or alternatively, we could assume that the derivational ending -\textit{er} does double duty as an invariant inflectional element under Infl. Note in this regard that this -\textit{er} can license a genitive noun phrase, for instance, \textit{der Verkauf Berlin-er Bier-es} ‘the sale of Berlin-\textsc{infl} beer-\textsc{gen}’ (Fuhrhop 2003) (for more discussion of the Genitive Rule, see Section 4).
\end{styleFootnote}

\begin{styleFootnote}
Notice that these three sets of cases have other intriguing properties that may help us decide between the different analytical options.\footnote{\ I have not investigated adjectives in ellipsis contexts in detail. As far as I am aware, all ordinary adjectives have the same inflections in ellipsis and non-ellipsis contexts. However, the three types of adjectives discussed in the main text are also special in ellipsis contexts. First, Fanselow (1988: 101) observes that the inflectional ending with \textit{lila} ‘purple’ and \textit{rosa} ‘pink’ is obligatory in ellipsis contexts:\par (i)\ \ \textit{ein lila*(-n-es) \ \ }\par \ \ \ \ a \ \ \ purple-n-\textsc{st }\par \ \ \ \ ‘a purple one’\par Second, uninflected adjectives like \textit{prima }‘great’, \textit{sexy} ‘sexy’, and \textit{super} ‘super’ cannot appear with an elided noun. Third, derived adjectives like \textit{Berliner} ‘(from) Berlin’ can occur in ellipsis contexts (as also noted by Rehn 2019: 221-24 for Alemannic German). Given that ordinary adjectives have the same forms in ellipsis contexts and that only the special cases behave differently, I leave the analysis of ellipsis for future research. Note though that a detailed analysis of these facts may help narrow down the analytical options mentioned in the main text, which were offered to account for the optionality or absence of the inflections in non-ellipsis contexts.} While I cannot investigate these three special cases of adjectives in more detail here, it seems clear that the avoidance of a hiatus plays a role here and in the cases I turn to next.
\end{styleFootnote}

\begin{styleFootnote}
\ \ Definite articles also have some special forms. Consider Table 5, where the inflections are set apart by hyphens. The parsing follows the one proposed in R. Wiese (1988: 33) closely:\footnote{\ Note that the presentation of the definite article might be a simplification as reduced forms like \textit{s} as in \textit{s Haus} ‘the house’ can be found in speech. For Swiss German, Studler (2011) shows that there are two paradigms of definite articles – the reduced and the full forms. Besides their different morphological manifestations, they also involve different semantics. Since the current investigation focuses on adjectival inflections and \textit{ein}, I will not investigate this for (colloquial) Standard German.}
\end{styleFootnote}

\begin{styleStandard}
Table 5: Definite Articles 
\end{styleStandard}

\begin{flushleft}
\begin{tabular}{|m{1.2511599in}|m{1.2511599in}|m{1.2511599in}|m{1.2511599in}|m{1.2518599in}|}

\hline
 &
\centering Masculine &
\centering Neuter &
\centering Feminine &
\centering\arraybslash Plural\\\hline
Nominative &
\centering de-r &
\centering da-s &
\centering die &
\centering\arraybslash die\\\hline
Accusative &
\centering de-n &
\centering da-s &
\centering die &
\centering\arraybslash die\\\hline
Dative &
\centering de-m &
\centering de-m &
\centering de-r &
\centering\arraybslash de-n\\\hline
Genitive &
\centering de-s &
\centering de-s &
\centering de-r &
\centering\arraybslash de-r\\\hline
\end{tabular}
\end{flushleft}
\begin{styleFootnote}
As can be observed in Table 5, there are two points of interest (see also Gunkel \textit{et al}. 2017: 1297-98). First, there is no inflection on \textit{die }‘the (/that)’. In other words, this form is pronounced [di] as an article and [di:] as its related distal demonstrative (where stress and the slightly different vowel quality of the demonstrative are often indicated by capitalization as in \textit{DIE}).\footnote{\ A reviewer asks why \textit{die} cannot be analyzed as \textit{d-i}. One advantage would be that now all articles would have an inflection. Note though that one of the determiner stems would have no vowel and that this new inflection would be the only full-vowel inflection in Standard German. Furthermore, it would increase the inflectional inventory considering that *\textit{dies-i} ‘this/these’ or *\textit{gut-i} ‘good’ are not possible in Standard German. Having said that, there may be a phonetic explanation of the absence of a full vowel in second (unstressed) syllables. Be that as it may, I proceed on the basis of the traditional assumptions regarding the inflectional inventory. Notice that assuming no inflection on \textit{die} allows me to relate \textit{wir} ‘we’ and \textit{ihr} ‘you(PL)’ to this element in the discussion of adjectival inflections in pronominal DPs (Section 5.3.1).} Second, the stem forms vary between \textit{de}{}- in most instances vs. \textit{da}{}- in the nominative/accusative neuter vs. \textit{di}{}- in the nominative/accusative in feminine/plural contexts.
\end{styleFootnote}

\begin{styleFootnote}
\ \ Starting with the first issue, it is clear that schwa cannot occur with the form \textit{die }‘the (/that)’. On the one hand, it cannot be added to the stem directly; compare (15a) to (15b). On the other hand, it cannot be added, mitigated by \textit{n}{}-epenthesis (15c):
\end{styleFootnote}

\begin{styleFootnote}
(15)\ \ a.\ \ \textit{die} \ \ [di(:)]
\end{styleFootnote}

\begin{styleFootnote}
\ \ \ \ the
\end{styleFootnote}

\begin{styleFootnote}
\ \ \ \ ‘the (/that)’
\end{styleFootnote}

\begin{styleFootnote}
b. \ *\ \ \textit{di-e} \ \ [di(:)[259?]]
\end{styleFootnote}

\begin{styleFootnote}
\ \ the-\textsc{st}
\end{styleFootnote}

\begin{styleFootnote}
c. \ *\ \ \textit{di-n-e}
\end{styleFootnote}

\begin{styleStandard}
\ \ \ \ the-n-\textsc{st}
\end{styleStandard}

\begin{styleStandard}
We can suggest that (15b) follows from the avoidance of a hiatus, and that (15c) indicates that \textit{n}{}-epenthesis is not possible. As seen with the proximal demonstrative below, the avoidance of a hiatus and the non-availability of \textit{n}{}-epenthesis also play a role in the account of other determiners. Indeed, this type of analysis is even more general accounting also for the third-person pronoun \textit{sie} ‘she, her; they, them’, which is pronounced as [zi:] (for the discussion of pronouns, see Section 5). If so, then the final -\textit{e} on \textit{die} and \textit{sie} can be taken as an orthographic element marking the quality of the stem vowel (cf. also \textit{nie} ‘never’, \textit{wie} ‘how’, \textit{Knie} ‘knee’, \textit{Chemie} ‘chemistry’, etc.).
\end{styleStandard}

\begin{styleStandard}
\ \ As to the second issue, the different stem forms, I assumed in Chapter 2, Section 2.1.2 that the schwas in adjectival inflections are due to epenthesis. This is presumably different for -\textit{e}{}- in the relevant stem forms of the definite article: \textit{der}, \textit{den}, \textit{dem}, \textit{des}. R. Wiese (1988: 34) observes that the stem vowels are short and lax if an obstruent follows (\textit{des}, also \textit{das}) but long and tense otherwise (\textit{der}, \textit{den}, \textit{dem}, also \textit{die}). Again, this is more general and applies to the different vowel qualities in \textit{es} ‘it’ vs. \textit{er} ‘he’, \textit{ihn} ‘him(\textsc{acc})’, \textit{ihm} ‘him(\textsc{dat})’, \textit{sie} ‘she, her; they, them’, \textit{ihr} ‘her(\textsc{dat})’, and \textit{ihnen} ‘them(\textsc{dat})’.
\end{styleStandard}

\begin{styleStandard}
\ \ Given this regularity and to make the vocabulary entries of the different definite article forms parallel, I include the stem vowel -\textit{e}{}- in the vocabulary insertion rule. I analyze the different stem forms as another instance of contextually conditioned allomorphy. The vocabulary insertion rule for the nominative/accusative form in the neuter is given in (16a), the one for the nominative/accusative in the feminine and plural is in (16b), and the elsewhere case is provided in (16c) (adapted here from Roehrs 2009a: 132):\footnote{\ Referencing unpublished work by B. Wiese, G. Müller (2002a: 125) derives both issues (absence of schwa on \textit{die}, different stem forms) by assuming one stem form (\textit{de}{}-), where the stem vowel /e/ is realized as [a] in nominative/accusative neuter contexts and as [i:] when the inflection schwa is added to \textit{de}{}- yielding a sequence of /e/-/e/. As far as I know, these are not regular phonetic realizations of an underlying /e/.}
\end{styleStandard}

\begin{styleStandard}
(16)\ \ a.\ \ [+D; +DEF] \ \ → \ \ \textit{da}{}- / \_\_ [-O, -F, +N]
\end{styleStandard}

\begin{styleStandard}
\ \ b.\ \ [+D; +DEF] \ \ → \ \ \textit{di}{}- / \_\_ [-O, +F]
\end{styleStandard}

\begin{styleStandard}
\ \ c.\ \ [+D; +DEF] \ \ → \ \ \textit{de}{}-
\end{styleStandard}

\begin{styleStandard}
Note that the stem form spells out the features on the left of the arrow; the features on the right specify the context of application of the vocabulary insertion rules. As just discussed, I assume that the phonological environment (i.e., elements such as obstruents) will determine the specific properties of the stem vowels in (16). To bring about the inflected forms of the definite article, recall that overt articles involve complex heads consisting of two separate feature bundles. As such, inflections are added to the stems in (16) by the vocabulary inflection rules, discussed in Chapter 2, Section 2.1.5. These rules spell out the CNG features of the bipartite article structure. With these insertion rules in mind, consider demonstratives.
\end{styleStandard}

\begin{styleStandard}
It is well known that demonstratives and definite articles are diachronically related in that the latter has evolved from the former (e.g., Greenberg 1978, van Gelderen 2007). As briefly mentioned in Chapter 1, Section 4.1.2, it is also often argued that they are synchronically related such that the demonstratives consist of the definite article and an additional deictic component (e.g., Leu 2007; Roehrs 2010, 2013a; and references cited therein). Starting with the proximal demonstrative, its decomposition is exemplified in (17a). This set of examples illustrates the masculine, neuter, and feminine forms in the nominative, where \textit{d}{}- spells out [+D; +DEF], -\textit{ies}{}- spells out [+DEIX], and the varying inflections spell out the different CNG feature bundles. Note though that the vocabulary insertion rules proposed in (16) should actually yield the strings in (17b) or (17c), contrary to fact:
\end{styleStandard}

\begin{styleFootnote}
(17)\ \ a.\ \ \textit{d-ies-er}, \ \ \ \ \ \ \ \textit{d-ies-es},\textit{ \ \ \ \ \ \ d-ies-e \ \ \ \ \ \ \ \ }
\end{styleFootnote}

\begin{styleFootnote}
\ \ \ \ the-\textsc{deix}{}-\textsc{st}, the-\textsc{deix}{}-\textsc{st}, the-\textsc{deix}{}-\textsc{st}
\end{styleFootnote}

\begin{styleFootnote}
\ \ \ \ ‘this’
\end{styleFootnote}

\begin{styleFootnote}
\ \ b. \ *\ \ \textit{de-ies-er},\textit{ \ \ \ \ \ da-ies-es}, \textit{\ \ \ \ di-ies-e \ \ \ \ \ \ }
\end{styleFootnote}

\begin{styleFootnote}
the-\textsc{deix}{}-\textsc{st}, the-\textsc{deix}{}-\textsc{st}, the-\textsc{deix}{}-\textsc{st} 
\end{styleFootnote}

\begin{styleFootnote}
\ \ c. \ *\ \ \textit{de-n-ies-er},\textit{ \ \ \ \ \ da-n-ies-es},\textit{ \ \ \ \ di-n-ies-e \ \ \ \ \ \ \ }
\end{styleFootnote}

\begin{styleFootnote}
the-n-\textsc{deix}{}-\textsc{st}, the-n-\textsc{deix}{}-\textsc{st}, the-n-\textsc{deix}{}-\textsc{st}
\end{styleFootnote}

\begin{styleStandard}
If demonstratives and definite articles are indeed related, then certain adjustments have to occur. To generate the correct forms in (17a), I follow Roehrs (2013a) in suggesting that (17b) shows that a hiatus is avoided. Furthermore, (17c) indicates that similar to definite articles, \textit{n}{}-epenthesis cannot rescue the cases in (17b). As to the distal demonstrative \textit{DER} ‘that’, I follow Roehrs (2013a) in assuming that the deitic component involves a null element here (i.e. \textit{de-Ø-r}). Given this null element, no adjustments have to be made. Moving forward, I abstract away from these finer points continuing with \textit{d}{}- as the stem for the definite article and its related distal demonstrative and with \textit{dies}{}- as the stem for the proximal demonstrative.
\end{styleStandard}

\begin{styleStandard}
\ \ To sum up, this section accounted for some unexpected forms of certain adjectives, definite articles and their related demonstratives. All these cases show that hiatuses do not occur. While it seems clear that a hiatus is generally avoided in the determiner system, the exact conditions that induce the avoidance of a hiatus with adjectives are less clear: a hiatus is avoided with certain disyllabic adjectives ending in -\textit{a}, but other instances tolerate a hiatus (e.g., \textit{blau-e} ‘blue’, \textit{roh-e} ‘raw’). I leave the detailed investigation of these differences for future research. Next, I discuss a type of variation that has received very little attention in the theoretical literature.
\end{styleStandard}

\begin{styleStandard}\bfseries
4.\ \ Canonical DPs with Unexpected Weak Determiners: Impoverishment Rule 2
\end{styleStandard}

\begin{styleFootnote}
In this section, I continue discussing variation that concerns the inflections on the determiners themselves. Like in the previous section, this variation is restricted, but here to genitive masculine/neuter environments. Furthermore, like in the previous section, this variation involves inflections on \textit{der}{}-words, but here on \textit{der}{}-words other than the definite article and its related distal demonstrative. Unlike in the previous section, the inflection on these determiners is present, but it alternates between strong and weak. Now, given the discussion of Chapter 2, it is surprising that determiners may have a weak ending at all. I propose that this variation is the result of Impoverishment; in particular, it is due to a more general application of Impoverishment Rule 2. First, I present the data, and then I provide my account along with an additional restriction.
\end{styleFootnote}

\begin{styleFootnote}
\ \ To begin, certain determiners show inflectional variation in genitive masculine/neuter contexts. Considering the data below, the (a)-examples exhibit the (expected) strong endings on the determiners \textit{dieser }‘this’, \textit{jeder }‘every’, and \textit{aller }‘all’; the (b)-examples show these elements with the corresponding weak endings:
\end{styleFootnote}

\begin{styleStandard}
(18)\ \ a.\ \ \textit{im \ \ \ \ \ Sommer dies-es Jahr-es}
\end{styleStandard}

\begin{styleStandard}
\ \ \ \ in.the summer this-\textsc{st} year.\textsc{neut}{}-\textsc{gen}
\end{styleStandard}

\begin{styleStandard}
\ \ \ \ ‘in the summer of this year’
\end{styleStandard}

\begin{styleStandard}
\ \ b.\ \ \textit{im \ \ \ \ \ Sommer dies-en Jahr-es}
\end{styleStandard}

\begin{styleStandard}
\ \ \ \ in.the summer this-\textsc{wk} year.\textsc{neut}{}-\textsc{gen}
\end{styleStandard}

\begin{styleStandard}
\ \ \ \ ‘in the summer of this year’
\end{styleStandard}

\begin{styleStandard}
(19)\ \ a.\ \ \textit{die Verarbeitung jed-es \ \ \ \ \ Holz-es}
\end{styleStandard}

\begin{styleStandard}
\ \ \ \ the processing \ \ \ \ every-\textsc{st} wood.\textsc{neut}{}-\textsc{gen}
\end{styleStandard}

\begin{styleStandard}
‘the processing of every (type of) wood’ 
\end{styleStandard}

\begin{styleStandard}
\ \ b.\ \ \textit{die Verarbeitung jed-en \ \ \ \ \ Holz-es}
\end{styleStandard}

\begin{styleStandard}
\ \ \ \ the processing \ \ \ \ every-\textsc{wk} wood.\textsc{neut}{}-\textsc{gen}
\end{styleStandard}

\begin{styleStandard}
‘the processing of every (type of) wood’ 
\end{styleStandard}

\begin{styleStandard}
(20)\ \ a.\ \ \textit{der Beginn \ \ \ \ all-es Schön-en}
\end{styleStandard}

\begin{styleStandard}
\ \ \ \ the beginning all-\textsc{st} beautiful.\textsc{neut}{}-\textsc{wk}
\end{styleStandard}

\begin{styleStandard}
\ \ \ \ ‘the beginning of everything beautiful’
\end{styleStandard}

\begin{styleStandard}
\ \ b.\ \ \textit{die Wege all-en Übel-s} 
\end{styleStandard}

\begin{styleStandard}
\ \ \ \ the ways all-\textsc{wk} evil.\textsc{neut}{}-\textsc{gen}
\end{styleStandard}

\begin{styleStandard}
\ \ \ \ ‘the ways of all evil’
\end{styleStandard}

\begin{styleStandard}
As noted by other scholars, this variation is possible with the following \textit{der}{}-words: \textit{dieser} ‘this’ (G. Müller 2002a: 137 fn. 38), \textit{jener} ‘that’ (Gallmann 2004: 154), \textit{jeder} ‘every’ (also \textit{jedweder} ‘every’, \textit{jeglicher} ‘every’), \textit{aller} ‘all’, \textit{mancher} ‘some’, \textit{solcher} ‘such’, \textit{welcher} ‘which’ (the latter are all mentioned in Zifonun \textit{et al}. 1997: 1936-48). In contrast, there is no inflectional variation with the definite article \textit{der} ‘the’, its related demonstrative \textit{DER }‘that’, and \textit{ein}{}-words in general – the latter must all have strong inflections:
\end{styleStandard}

\begin{styleStandard}
(21)\ \ a.\ \ \textit{der Verkauf d-es \ \ Wagen-s}
\end{styleStandard}

\begin{styleStandard}
\ \ \ \ the sale \ \ \ \ \ \ the-\textsc{st} car.\textsc{masc}{}-\textsc{gen}
\end{styleStandard}

\begin{styleStandard}
\ \ \ \ ‘the sale of the car’
\end{styleStandard}

\begin{styleStandard}
\ \ b. \ *\ \ \textit{der Verkauf d-en \ \ \ Wagen-s}
\end{styleStandard}

\begin{styleStandard}
\ \ \ \ the sale \ \ \ \ \ \ the-\textsc{wk} car.\textsc{masc}{}-\textsc{gen}
\end{styleStandard}

\begin{styleStandard}
\ \ c.\ \ \textit{der Verkauf ein-es Wagen-s}
\end{styleStandard}

\begin{styleStandard}
\ \ \ \ the sale \ \ \ \ \ \ \ a-\textsc{st} \ \ car.\textsc{masc}{}-\textsc{gen}
\end{styleStandard}

\begin{styleStandard}
\ \ \ \ ‘the sale of a car’
\end{styleStandard}

\begin{styleStandard}
\ \ d. \ *\ \ \textit{der Verkauf ein-en Wagen-s}
\end{styleStandard}

\begin{styleStandard}
\ \ \ \ the sale \ \ \ \ \ \ \ a-\textsc{wk} \ car.\textsc{masc}{}-\textsc{gen}
\end{styleStandard}

\begin{styleStandard}
B. Wiese (2009: 184-85) states that there is no variation in this context with regular adjectives either. However, unlike the determiners in (21), adjectives are always weak in this context. Rather than explaining the variation on the relevant \textit{der}{}-words by analogy with adjectives, I make the stronger and more interesting claim that the weak endings on these determiners are also due to Impoverishment.
\end{styleStandard}

\begin{styleStandard}
\ \ I argued in Chapter 2, Section 2.3 that adjectives in genitive masculine/neuter contexts have weak inflections due to Impoverishment Rule 2. Impoverishment affects elements that have their InflP in the specifier of AgrP:
\end{styleStandard}

\begin{styleFootnote}
(22)\ \ \textit{Impoverishment Rule 2:}
\end{styleFootnote}

\begin{styleFootnote}
\ \ \ \  \ \ \ \ \ \ \ \ \ \ AgrP
\end{styleFootnote}

\begin{styleFootnote}
[Warning: Draw object ignored][Warning: Draw object ignored]
\end{styleFootnote}

\begin{styleFootnote}
\ \  \ \ \ \ \ \ \ \ \ InflP\ \ \ \ Agr’
\end{styleFootnote}

\begin{styleFootnote}
\ \ \ \ \ \ \ [-F, $\alpha $N, +O, +S]\ \  \ \ \ \ [-F, $\alpha $N, +O, +S]
\end{styleFootnote}

\begin{styleFootnote}
Observe again that genitive masculine/neuter contexts are the exact same featural environments where the relevant \textit{der}{}-words from above can have weak inflections too. By contrast, note again that \textit{der} ‘the’, \textit{DER }‘that’, and \textit{ein}{}-words do not exhibit this variation. The same goes for the head noun itself – its ending is not changed from -\textit{(e)s} to -\textit{(e)n}. The difference between these two types of elements seems to involve structural size: the first group (\textit{der}{}-words other than \textit{der} and \textit{DER}) involves phrasal elements but the second group (\textit{der}, \textit{DER}, \textit{ein}{}-words, nouns) consists of heads. I propose that Impoverishment only applies to the inflections on the phrasal elements.\footnote{\ The demonstrative \textit{DER} ‘that’ presumably patterns with the article \textit{der} ‘the’ due to their obvious relatedness.} Consider this in more detail.
\end{styleFootnote}

\begin{styleFootnote}
\ \ I proposed in Chapter 2, Section 2.1.6 that demonstratives have a more complex structure than articles in that they have InflP at the top of their structure:
\end{styleFootnote}

\begin{styleStandard}
(23)\ \ \textit{Demonstrative}
\end{styleStandard}

\begin{styleJBExample}
\ \ \ \ \ \ \ \ \ \ \ \  \ \ \ \ InflP\textsubscript{[+D; +DEF, +DEIX][F, N, O, S]}
\end{styleJBExample}

\begin{styleStandard}\bfseries
[Warning: Draw object ignored][Warning: Draw object ignored]
\end{styleStandard}

\begin{styleStandard}
\ \ \ \ \ \ \ \ \ \ [F, N, O, S]\ \ \ \ DemP
\end{styleStandard}

\begin{styleStandard}
\ \ \ \ \ \ \ \  \ \ \ {\textbar} \ \ \ \ \ \ 
\end{styleStandard}

\begin{styleStandard}
\ \ \ \ \ \  \ \ [+D; +DEF, +DEIX]\ \ 
\end{styleStandard}

\begin{styleFootnote}
This yields a phrasal determiner, which occurs in phrasal (e.g., specifier) positions. Extending this discussion, I propose that besides \textit{dieser} ‘this’, \textit{jener} ‘that’, \textit{jeder} ‘every’ (also \textit{jedweder} ‘every’, \textit{jeglicher} ‘every’), \textit{aller} ‘all’, \textit{mancher} ‘some’, \textit{solcher} ‘such’, and \textit{welcher} ‘which’ also have structures similar to (23).
\end{styleFootnote}

\begin{styleFootnote}
\ \ With that in mind, I argue that Impoverishment Rule 2 has a more general context of application, updated as Impoverishment Rule 2’. Rather than just applying in AgrP, this rule operates in all phrases indicated by XP below: 
\end{styleFootnote}

\begin{styleFootnote}
(24)\ \ \textit{Impoverishment Rule 2’:}
\end{styleFootnote}

\begin{styleFootnote}
\ \ \ \  \ \ \ \ \ \ \ \ \ \ \ XP\ \ 
\end{styleFootnote}

\begin{styleFootnote}
[Warning: Draw object ignored][Warning: Draw object ignored]
\end{styleFootnote}

\begin{styleFootnote}
\ \  \ \ \ \ \ \ \ \ \ \ InflP\ \ \ \ X’
\end{styleFootnote}

\begin{styleFootnote}
\ \ \ \ \ \ \ [-F, $\alpha $N, +O, +S]\ \  \ \ \ \ [-F, $\alpha $N, +O, +S]
\end{styleFootnote}

\begin{styleFootnote}
Given this, the more general Impoverishment rule in (24) now applies to all determiners in Spec,ArtP, adjectives in Spec,AgrP, quantifiers/numerals in Spec,CardP, and determiners in Spec,DP.\footnote{\ Note that Impoverishment Rule 2’ applies to CardP with some restrictions. While it is applicable to inflected singular quantifiers (e.g., \textit{vieler} ‘much’), it is not relevant to the singularity numeral \textit{EIN} ‘one’, which consists of a null element in Spec,CardP and \textit{ein} in Card (Chapter 5).} More generally, if Impoverishment applies in this featural context, then only the least specified ending -\textit{en} can be inserted as discussed with adjectives in Chapter 2. This accounts for the weak ending on the relevant determiners. Importantly, note that this Impoverishment rule presents a second, secondary mechanism that accounts for the strong/weak alternation.
\end{styleFootnote}

\begin{styleFootnote}
\ \ It is not entirely clear to me how to derive the optionality of the strong/weak alternation on these determiners. Currently, I see two analytical options. First, it would be possible to stipulate that there are actually two independent Impoverishment rules, Rule 2 and Rule 2’, where the former applies obligatorily, but the latter does not. Note though that both rules have the same basic context of application. Assuming the existence of two separate rules applying in the same context runs the risk of losing an important generalization. Alternatively, it would be possible to suggest that – what I have proposed to be – phrasal determiners can actually vacillate between involving phrases or heads (cf. Borer 2005: 169-74). If the relevant determiners merged as heads, Impoverishment Rule 2’ would not apply to them (but only to their phrasal counterparts and adjectives more generally). Whichever is the right solution to pursue, note that determiners, be they phrases or heads, still trigger Impoverishment on the adjective (by Impoverishment Rule 1).\footnote{\ This makes the predication that a determiner with a weak inflection can be followed by an adjective with a weak inflection. While I have not found any attested examples yet, such cases seem possible to me:\par (i)\ \ a. \ \ \textit{im \ \ \ \ \ Verlauf dies-en schön-en \ \ \ \ \ \ \ Sommer-s}\par \ \ \ \ in.the course \ this-\textsc{wk} beautiful-\textsc{wk} summer.\textsc{masc}{}-\textsc{gen}\par \ \ \ \ \ \ ‘in the course of this beautiful summer’\par \ \ \ \ b.\ \ \textit{im \ \ \ \ \ Sommer jed-en \ \ \ \ \ heiß-en Jahr-es}\par \ \ \ \ in.the summer every-\textsc{wk} hot-\textsc{wk} year.\textsc{neut}{}-\textsc{gen}\par \ \ \ \ \ \ ‘in the summer of every hot year’}
\end{styleFootnote}

\begin{styleStandard}
I close this section by discussing a final restriction. The occurrence of weak endings on the determiners is constrained by the ending on the following noun: if the ending on the noun is -\textit{(e)s}, both strong or weak endings on the determiners are usually possible (25a-b); if the ending on the noun is -\textit{(e)n}, only strong endings on the determiners are possible; compare (25c) to (25d) (data are from Gallmann 1996: 293, also Gallmann 1990: 268-75):
\end{styleStandard}

\begin{styleStandard}
(25)\ \ a.\ \ \textit{der Traum manch-es Schüler-s}
\end{styleStandard}

\begin{styleStandard}
\ \ \ \ the dream \ some-\textsc{st} \ \ pupil.\textsc{masc}{}-\textsc{gen}
\end{styleStandard}

\begin{styleStandard}
\ \ \ \ ‘the dream of some pupil’
\end{styleStandard}

\begin{styleStandard}
\ \ b.\ \ \textit{der Traum manch-en Schüler-s}
\end{styleStandard}

\begin{styleStandard}
\ \ \ \ the dream \ some-\textsc{wk} \ pupil.\textsc{masc}{}-\textsc{gen}
\end{styleStandard}

\begin{styleStandard}
\ \ \ \ ‘the dream of some pupil’
\end{styleStandard}

\begin{styleStandard}
\ \ c.\ \ \textit{der Traum manch-es Student-en}
\end{styleStandard}

\begin{styleStandard}
\ \ \ \ the dream \ some-\textsc{st} \ \ student.\textsc{masc}{}-\textsc{gen}
\end{styleStandard}

\begin{styleStandard}
\ \ \ \ ‘the dream of some student’
\end{styleStandard}

\begin{styleStandard}
\ \ d. *\ \ \textit{der Traum manch-en Student-en}
\end{styleStandard}

\begin{styleStandard}
\ \ \ \ the dream \ some-\textsc{wk} \ student.\textsc{masc}{}-\textsc{gen}
\end{styleStandard}

\begin{styleStandard}
Note that this restriction has nothing to do with the strong/weak alternation per se. Rather, it involves a dependency between the inflection on the determiner (or adjective, not discussed here separately; for some remarks, see Chapter 2, Section 2.3) and the inflection on the noun.
\end{styleStandard}

\begin{styleStandard}
In Chapter 2, I provided a specificity-based account of the inventory and distribution of adjectival inflections. Gallmann (1996, 1998, 2018) utilizes specificity of inflections to account for the cases above. Given the interplay between the inflection on the determiner element and that on the noun, Gallmann (1996) formulates the Genitive Rule: a noun phrase in the genitive must contain at least one element that is sufficiently specific as regards case. Gallmann states that -\textit{es} and -\textit{er} are more specific than -\textit{en}: the first two are [+Genitive], which consists of [+Oblique] and other features, but the third is only [+Oblique] (see Gallmann 1998: 146, 153). This is consistent with the feature specifications in Chapter 2, where the strong inflections -\textit{es} and -\textit{er} are also more specific than the weak inflection -\textit{en} (the latter is interpreted as the elsewhere case and has no feature specification at all).\footnote{\ Although case suffixes on nouns and adjectival inflections on determiners and adjectives are not the same elements (Chapter 2, Section 2.3), the case suffix -\textit{es} presumably has a similar specification as the adjectival inflection -\textit{es}.} Since there is no sufficiently specific element in (25d), this example violates the Genitive Rule, and its ungrammaticality is accounted for.
\end{styleStandard}

\begin{styleFootnote}
\ \ To sum up, this section illustrated some inflectional variation on certain \textit{der}{}-words in genitive masculine/neuter contexts. I suggested that Impoverishment Rule 2 from Chapter 2 appears to have a more general context of application. I made two tentative suggestions to account for the (restricted) optionality of this variation. Given its restricted application, Impoverishment Rule 2 is interpreted as a second, secondary mechanism. More generally, besides Impoverishment Rule 1, adjectival inflections in German are, in certain contexts, regulated by a phonetic rule and Impoverishment Rule 2.
\end{styleFootnote}

\begin{styleStandard}\bfseries
5. \ \ Canonical DPs with Unexpected Strong Adjectives: Pronominal Determiners
\end{styleStandard}

\begin{styleStandard}
In this section, I return to the discussion of pronominal DPs. Recalling that these are definite constructions, I show that there is variation as regards the inflections on adjectives: in some cases, both (expected) weak and (unexpected) strong adjectives are possible; in others, only (unexpected) strong adjectives occur. I propose that pronominal DPs involve regular structures. However, pronominal determiners lack features that trigger Impoverishment. This accounts for the general option of strong adjectives. Weak adjectives are due to the phonetic rule discussed in Section 2.2 and to (phonetically constrained) analogy with ordinary definite DPs. 
\end{styleStandard}

\begin{styleStandard}
I also discuss briefly preferences for strong adjectives over weak adjectives and vice versa. I suggest that, if possible, strong adjectives are preferred as they are more specified as regards CNG features than their corresponding pronominal determiners. Note that I often refer to pronominal determiners (or pronouns) as pronominal elements or short as pronominals. Given the complexity of the data and the fact that this variation has not received much attention in the theoretical literature, I discuss pronominal DPs in quite some detail. At the end of this section, I summarize the data and analysis.
\end{styleStandard}

\begin{styleStandard}\itshape
5.1.\ \ Data
\end{styleStandard}

\begin{styleStandard}
First, I present the data under discussion (see also Bhatt 1990: 154-55; Darski 1979: 198; Duden 1995: 280, 2007: 39; Gunkel \textit{et al}. 2017: 1308). I focus on the first-person pronominal elements. Second-person pronominals show the same distribution and are only briefly discussed in what follows. Starting with the singular, there is no variation in the nominative and accusative in that only strong adjectives are possible:\footnote{\ Recall that adjectival inflections in the nominative/accusative feminine involve -\textit{e} (which is ambiguous between a strong and a weak inflection). Such a case is not provided in (26). Also, strings like (26a-b) are ungrammatical in English. Given that there are no straightforward translations of such German examples, I put the adjective and noun in parentheses in the English translations. For simplicity, I translate the figurative meaning of animal names as ‘idiot’. Note though that different animal names have different implications; for instance, \textit{Schwein} ‘pig’ implies that the person is, in some sense, dirty whereas \textit{Esel} ‘donkey’ implies that the person is not very bright.} 
\end{styleStandard}

\begin{styleStandard}
(26)\ \ a.\ \ \textit{ich \ \ \ \ \ dumm-er Idiot}
\end{styleStandard}

\begin{styleStandard}
I.\textsc{nom} stupid-\textsc{st} idiot.\textsc{masc}
\end{styleStandard}

\begin{styleStandard}
‘I (stupid idiot)’
\end{styleStandard}

\begin{styleStandard}
\ \ b.\ \ \textit{mich \ \ \ \ \ dumm-es Schwein}
\end{styleStandard}

\begin{styleStandard}
me.\textsc{acc }stupid-\textsc{st} pig.\textsc{neut}
\end{styleStandard}

\begin{styleStandard}
‘me (stupid idiot)’
\end{styleStandard}

\begin{styleStandard}
The second-person pronominal element \textit{du} ‘you(\textsc{nom.sgl})’ shows the pattern in (26a), and \textit{dich} ‘you(\textsc{acc.sgl})’ has the same distribution as in (26b).
\end{styleStandard}

\begin{styleStandard}
\ \ There is variation in the dative in all three genders. Strong adjectives are preferred over weak ones in the masculine and neuter but vice versa in the feminine (the less preferred option is indicated by \%):
\end{styleStandard}

\begin{styleFootnote}
(27)\ \ a. \ \%\ \ \textit{mir \ \ \ \ \ \ \ groß-en \ \ Esel}
\end{styleFootnote}

\begin{styleFootnote}
\ \ \ \ me.\textsc{dat} great-\textsc{wk} donkey.\textsc{masc}
\end{styleFootnote}

\begin{styleFootnote}
\ \ \ \ ‘me (stupid idiot)’
\end{styleFootnote}

\begin{styleFootnote}
\ \ b.\ \ \textit{mir \ \ \ \ \ \ \ groß-em Esel}
\end{styleFootnote}

\begin{styleFootnote}
\ \ \ \ me.\textsc{dat} great-\textsc{st} donkey.\textsc{masc}
\end{styleFootnote}

\begin{styleFootnote}
\ \ \ \ ‘me (stupid idiot)’
\end{styleFootnote}

\begin{styleFootnote}
(28)\ \ a. \ \%\ \ \textit{mir \ \ \ \ \ \ \ groß-en \ \ Schwein}
\end{styleFootnote}

\begin{styleFootnote}
\ \ \ \ me.\textsc{dat} great-\textsc{wk} pig.\textsc{neut}
\end{styleFootnote}

\begin{styleFootnote}
\ \ \ \ ‘me (stupid idiot)’
\end{styleFootnote}

\begin{styleFootnote}
\ \ b.\ \ \textit{mir \ \ \ \ \ \ \ groß-em Schwein}
\end{styleFootnote}

\begin{styleFootnote}
\ \ \ \ me.\textsc{dat} great-\textsc{st} pig.\textsc{neut}
\end{styleFootnote}

\begin{styleFootnote}
\ \ \ \ ‘me (stupid idiot)’
\end{styleFootnote}

\begin{styleFootnote}
(29)\ \ a.\ \ \textit{mir \ \ \ \ \ \ \ groß-en \ \ Gans}
\end{styleFootnote}

\begin{styleFootnote}
\ \ \ \ me.\textsc{dat} great-\textsc{wk} goose.\textsc{fem}
\end{styleFootnote}

\begin{styleFootnote}
\ \ \ \ ‘me (stupid idiot)’
\end{styleFootnote}

\begin{styleFootnote}
\ \ b. \ \%\ \ \textit{mir \ \ \ \ \ \ \ groß-er \ Gans}
\end{styleFootnote}

\begin{styleFootnote}
\ \ \ \ me.\textsc{dat} great-\textsc{st} goose.\textsc{fem}
\end{styleFootnote}

\begin{styleFootnote}
\ \ \ \ ‘me (stupid idiot)’
\end{styleFootnote}

\begin{styleFootnote}
The same patterns hold for the second-person pronominal element \textit{dir} ‘you(\textsc{dat.sgl})’. The cases in the singular are summarized in Table 6, where, if important, [-F] indicates pronominal DPs in the masculine/neuter and [+F] does so as regards feminine/plural:
\end{styleFootnote}

\begin{styleFootnote}
Table 6: Summary of Adjectival Inflections in Singular Pronominal DPs
\end{styleFootnote}

\begin{flushleft}
\begin{tabular}{|m{2.30876in}|m{1.4212599in}|m{1.2962599in}|m{1.3087599in}|}

\hline
Pronominal Element &
\centering Gender of the noun &
\centering Weak &
\centering\arraybslash Strong\\\hline
{\fontsize{10pt}{12.0pt}\selectfont \textit{ich} ‘I(\textsc{nom})’/\textit{du} ‘you(\textsc{nom})’} &
 &
\centering {}- &
\centering\arraybslash ${\surd}$\\\hline
{\fontsize{10pt}{12.0pt}\selectfont \textit{mich} ‘me(\textsc{acc})’/\textit{dich} ‘you(\textsc{acc})’} &
 &
\centering {}- &
\centering\arraybslash ${\surd}$\\\hline
{\fontsize{10pt}{12.0pt}\selectfont \textit{mir} ‘me(\textsc{dat})’/\textit{dir} ‘you(\textsc{dat})’} &
{\centering [ -F]\par}

\centering [+F] &
{\centering \%\par}

\centering ${\surd}$ &
{\centering ${\surd}$\par}

\centering\arraybslash \%\\\hline
\end{tabular}
\end{flushleft}
\begin{styleStandard}
In the plural, there is variation in the nominative where, like in the dative feminine singular, weak adjectives are preferred over strong ones (30a-b). There is no variation in the accusative and dative (30c-d), with the qualification that the inflections on the adjectives in the dative are ambiguous between weak and strong (30d):
\end{styleStandard}

\begin{styleStandard}
(30)\textit{ \ \ }a.\textit{\ \ wir \ \ \ \ \ \ \ \ nett-en \ \ Studenten\ \ \ \ \ \ }
\end{styleStandard}

\begin{styleStandard}
we.\textsc{nom} nice-\textsc{wk} students
\end{styleStandard}

\begin{styleStandard}
‘we nice students’
\end{styleStandard}

\begin{styleStandard}
\textit{\ \ \ }b. \ \%\textit{\ \ wir \ \ \ \ \ \ \ \ nett-e \ \ \ Studenten\ \ \ \ \ \ }
\end{styleStandard}

\begin{styleStandard}
we.\textsc{nom} nice-\textsc{st} students
\end{styleStandard}

\begin{styleStandard}
‘we nice students’
\end{styleStandard}

\begin{styleStandard}
\textit{\ \ }c.\ \ \textit{für uns \ \ \ \ \ \ nett-e \ \ \ Studenten}
\end{styleStandard}

\begin{styleFootnote}
\ \ \ \ for us.\textsc{acc} nice-\textsc{st} students
\end{styleFootnote}

\begin{styleFootnote}
\ \ \ \ ‘for us nice students’
\end{styleFootnote}

\begin{styleStandard}
\textit{\ \ \ }d.\textit{\ \ von \ \ uns \ \ \ \ \ \ nett-en \ \ \ \ \ \ \ Studenten}
\end{styleStandard}

\begin{styleFootnote}
\ \ \ \ from us.\textsc{dat} nice-\textsc{wk/st} students
\end{styleFootnote}

\begin{styleFootnote}
\ \ \ \ ‘from us nice students’
\end{styleFootnote}

\begin{styleFootnote}
The nominative plural pronominal element \textit{ihr} ‘you(\textsc{nom.pl})’ has the same distribution as in (30a-b), and the dative/accusative pronominal element \textit{euch} ‘you(\textsc{acc/dat.pl})’ has the patterns as in (30c-d). The plural cases are summarized in Table 7, where the specified features for [O] and [S] indicate the different morphological cases of the pronominal DPs involved: 
\end{styleFootnote}

\begin{styleFootnote}
Table 7: Summary of Adjectival Inflections in Plural Pronominal DPs
\end{styleFootnote}

\begin{flushleft}
\begin{tabular}{|m{2.18376in}|m{1.3587599in}|m{1.2962599in}|m{1.4962599in}|}

\hline
Pronominal element &
\centering Case of the DP &
\centering Weak &
\centering\arraybslash Strong\\\hline
{\fontsize{10pt}{12.0pt}\selectfont \textit{wir} ‘we(\textsc{nom})’/\textit{ihr} ‘you(\textsc{nom})’} &
\centering [ -O, \ {}-S] &
\centering ${\surd}$ &
\centering\arraybslash \%\\\hline
{\fontsize{10pt}{12.0pt}\selectfont \textit{uns} ‘us(\textsc{acc})’/\textit{euch} ‘you(\textsc{acc})’} &
\centering [ -O, +S] &
\centering {}- &
\centering\arraybslash ${\surd}$\\\hline
{\fontsize{10pt}{12.0pt}\selectfont \textit{uns} ‘us(\textsc{dat})’/\textit{euch} ‘you(\textsc{dat})’} &
\centering [+O, \ {}-S] &
\centering ambiguous &
\centering\arraybslash ambiguous\\\hline
\end{tabular}
\end{flushleft}
\begin{styleStandard}
Finally, as is well known, third-person pronominals do not take overt complements in the singular or plural (for recent discussion, see Höhn 2020):\footnote{\ A brief note on older varieties of German is of interest here. MHG appears to have tolerated forms of the third person, for instance, \textit{er süezer man vil guoter} ‘he (nice man), very good’ (see Paul \textit{et al}. 1989: 359). However, these strings may actually involve appositives (as suggested by the continuation with \textit{vil guoter}). Be that as it may, it is clear that MHG is different in that pronominal elements in the nominative singular could be followed by a weak adjective, for instance, \textit{ich arme tore} ‘I (poor-\textsc{wk} fool)’. For similar facts in ENHG, see Ebert \textit{et al}. (1993: 197).}
\end{styleStandard}

\begin{styleStandard}
(31)\ \ a. \ *\ \ \textit{er Idiot}
\end{styleStandard}

\begin{styleStandard}
\ \ \ \ he idiot.\textsc{masc}
\end{styleStandard}

\begin{styleStandard}
\ \ b. \ *\ \ \textit{sie \ \ \ Linguisten}
\end{styleStandard}

\begin{styleStandard}
\ \ \ \ they linguists
\end{styleStandard}

\begin{styleFootnote}
Note also that pronominal elements in the genitive are not commonly used in German and do not do double duty as (transitive) pronominal determiners – similar to third-person pronominals.\footnote{\ Recall that transitive determiners have overt complements but that intransitive determiners have covert complements.}
\end{styleFootnote}

\begin{styleStandard}
Taking stock thus far, pronominal determiners are definite elements. Given their similarity to ordinary definite determiners, we would expect all adjectives to have weak inflections. However, this is not the case. In fact, all instances allow strong adjectives. Crucially, weak adjectives only occur in the dative singular and in the nominative plural (but not in the nominative/accusative singular and not in the accusative plural). Adjectives in the dative plural are inflectionally ambiguous. Finally, third-person and genitive pronominals do not take adjectives and nouns.
\end{styleStandard}

\begin{styleStandard}
Observe again that the strong endings cannot mark indefiniteness. Furthermore, similar to the distribution of strong adjectival inflections where, for instance, -\textit{em} only occurs in the masculine/neuter, and -\textit{e} only surfaces in the feminine/plural, here masculine also patterns with neuter and feminine also with plural. In the current contexts though, the similarity concerns the preference of the relevant endings: strong endings are preferred in the dative masculine/neuter but weak endings in the dative feminine and in the nominative plural. 
\end{styleStandard}

\begin{styleStandard}
\ \ Before moving on, consider the complete set of pronominal elements in Table 8 (for detailed discussion of the internal structure, see Fischer 2006). Starting with the first and second-person pronominals, notice that these elements do not involve inflections. Furthermore, they do not distinguish gender in the singular, and they do not have different forms in the accusative and dative case in the plural (\textit{uns}, \textit{euch}):
\end{styleStandard}

\begin{styleStandard}
Table 8: Pronominal Elements in German
\end{styleStandard}

\begin{flushleft}
\begin{tabular}{|m{0.6601598in}|m{0.6601598in}|m{0.6601598in}|m{0.6601598in}|m{0.6601598in}|m{0.6601598in}|m{0.6601598in}|m{0.6601598in}|m{0.6601598in}|}

\hline
Number &
\multicolumn{5}{m{3.6157598in}|}{\centering Singular} &
\multicolumn{3}{m{2.13796in}|}{\centering Plural}\\\hline
Person  &
\centering 1\textsuperscript{st} Pers. &
\centering 2\textsuperscript{nd} Pers. &
\centering 3\textsuperscript{rd} Masc &
\centering 3\textsuperscript{rd} Neut &
\centering 3\textsuperscript{rd} Fem &
\centering 1\textsuperscript{st} Pers. &
\centering 2\textsuperscript{nd} Pers. &
\centering\arraybslash 3\textsuperscript{rd} Pers.\\\hline
Nom &
\centering ich &
\centering du &
\centering e-r &
\centering e-s &
\centering sie &
\centering wir &
\centering ihr &
\centering\arraybslash sie\\\hline
Acc &
\centering mich &
\centering dich &
\centering ih-n &
\centering e-s &
\centering sie &
\centering uns &
\centering euch &
\centering\arraybslash sie\\\hline
Dat &
\centering mir &
\centering dir &
\centering ih-m &
\centering ih-m &
\centering ih-r &
\centering uns &
\centering euch &
\centering\arraybslash ih-n-en\\\hline
Gen &
\centering meiner &
\centering deiner &
\centering seiner &
\centering seiner &
\centering ihrer &
\centering unserer &
\centering eurer &
\centering\arraybslash ihrer\\\hline
\end{tabular}
\end{flushleft}
\begin{styleStandard}
Third-person pronominal elements are different in a number of ways. First, they have the typical determiner endings, and the singular instances of these pronominals also distinguish gender. Second, note that the form \textit{sie} is special: on the one hand, it is pronounced as [zi:]; that is, it does not have an inflection (see Section 3); on the other hand, it is the only pronominal element unspecified for number: it translates as ‘she/her’ and ‘they/them’. Third, as mentioned above, third-person elements, like genitive pronominals more generally, do not take overt complements.\footnote{\ There is another type of \textit{sie}: when capitalized in writing, \textit{Sie} ‘you(\textsc{sgl/pl.formal})’ is morphologically third-person plural, but semantically, it is second-person singular/plural. Also, unlike lower case \textit{sie}, capitalized \textit{Sie} can take overt complements (for detailed discussion, see Chapter 7).} Finally, note that dative plural \textit{ihnen} has the additional inflection -\textit{en} (similar to the dative plural pronominal \textit{d-en-en} ‘those (ones)’) and that genitive pronominal elements do not involve (regular) adjectival inflections. This is particularly clear in masculine/neuter contexts where the strong inflection would be -\textit{es} or the weak one would be -\textit{en}, both contrasting with -\textit{er}. I provide the vocabulary insertion rules for all these elements in Section 5.3.2.
\end{styleStandard}

\begin{styleStandard}
The lack of gender specification in the first and second person singular and the fact that case is not uniquely differentiated by all these pronominal elements in the plural can be linked to the general preference of strong adjectives (provided there is variation). I propose in Section 5.3.3 that \textit{wir} ‘we’ and \textit{ihr} ‘you(\textsc{nom.pl})’ are the only first and second-person pronominals that are specified for all CNG features. Consequently, they are preferably followed by weak adjectives. By contrast, all other relevant pronominal determiners are not fully specified for CNG features (note again that one pronominal form spells out multiple feature combinations), and the following adjective is recruited to indicate those features with a strong inflection. 
\end{styleStandard}

\begin{styleStandard}
\ \ In the next section, I briefly discuss three types of proposals that might be offered to account for the inflectional variation on the adjectives inside the pronominal DPs. Given certain assumptions, I point out some shortcomings for all of them (if the reader is not interested in the detailed discussion of other proposals, they are invited to proceed to Section 5.2.4). In Section 5.3, I propose a different analysis that avoids these issues. Section 5.4 summarizes the discussion.
\end{styleStandard}

\begin{styleStandard}\itshape
5.2.\ \ Three Types of Proposals
\end{styleStandard}

\begin{styleFootnote}
As discussed in Chapter 2, Section 3.5, personal pronouns are determiners in that they take the adjective and noun as their complement (e.g., Déchaine \& Wiltschko 2002: 421-22, Höhn 2020, Pesetsky 1978, Postal 1966, Roehrs 2005); consider (30a) repeated below as (32a). This pronominal DP can be represented in the following, simplified structure in (32b) (AgrP abbreviates the rest of the canonical structure discussed at length in Chapters 1 and 2): 
\end{styleFootnote}

\begin{styleStandard}
(32)\ \ a.\ \ \textit{wir nett-en \ Studenten}
\end{styleStandard}

\begin{styleStandard}
\ \ \ \ we nice-\textsc{wk} students
\end{styleStandard}

\begin{styleStandard}
\ \ \ \ ‘us nice students’
\end{styleStandard}

\begin{styleFootnote}
\textit{\ \ \ }b.\ \ \textit{(Pro)nominal DP – Complementation }
\end{styleFootnote}

\begin{styleFootnote}
\textit{\ \ }\ \ (LPP)
\end{styleFootnote}

\begin{styleFootnote}
[Warning: Draw object ignored][Warning: Draw object ignored]
\end{styleFootnote}

\begin{styleFootnote}
\ \ \ \ \ \ DP
\end{styleFootnote}

\begin{styleFootnote}
[Warning: Draw object ignored][Warning: Draw object ignored]
\end{styleFootnote}

\begin{styleFootnote}
\ \  \ \ \ \ \ \ \ \ \ \ \ \textit{wir}\ \ \ \ CardP
\end{styleFootnote}

\begin{styleFootnote}
[Warning: Draw object ignored][Warning: Draw object ignored]
\end{styleFootnote}

\begin{styleFootnote}
\ \ \ \ \ \ \ \ \ \  \ AgrP
\end{styleFootnote}

\begin{styleFootnote}
\ \ \ \ \ \ \ \  \ \ \ \ \ \textit{netten} \textit{Studenten\ \ }
\end{styleFootnote}

\begin{styleStandard}
If personal pronouns are indeed determiners, then they should trigger Impoverishment. From this perspective, the weak ending on the following adjective in (32a) is expected (in Section 5.2.4, I discuss the absence of weak adjectives in certain instances; this will lead to the ultimate proposal that pronominal determiners do not trigger Impoverishment). In order to account for the variation, that is, the unexpected strong ending on the following adjective as in \textit{wir nette }\textit{Studenten }(cf. (30b)), I entertain three basic types of proposals. All these proposals involve the structural analysis in (32b) above and some additional assumptions.
\end{styleStandard}

\begin{styleStandard}
5.2.1. One Structure: Agreeing vs. Non-agreeing Pronominal Elements
\end{styleStandard}

\begin{styleStandard}
The first type of proposal involves one structure where the pronominal element is in the same position in (32b) above. This position could be Spec,DP or D.\footnote{\ Note that in the present account, determiners in either of these positions trigger Impoverishment. Having said that, I point out in Section 5.3.1 that pronominals of the first and second person seem to be associated with a certain kind of deixis (in fact, this kind of deixis will be crucial for the discussion of Impoverishment). Given the deixis, these pronominals are similar to demonstratives and are assumed to project phrasal structures. If so, these pronominal determiners are presumably in Spec,DP.} This is what I assumed in Chapter 2, Section 3.5 to account for the weak ending on the following adjective. To explain the strong ending, Roehrs (2005: 259) makes a distinction between “agreeing” and “non-agreeing” determiners. Note first that these pronominal elements share concord features with the following adjective and noun; that is, a pronominal in the nominative combines with an adjective and noun in the nominative (and not, say, in the dative). This means that non-agreeing cannot mean non-participation of these pronominal elements as regards concord in agreement features. Rather, we could assume that (non-)agreeing has to do with the presence or absence of agreement morphology on the pronominal element itself. This idea could be made more concrete by assuming a null inflection for the agreeing type (33a) and the lack of such an inflection for the non-agreeing type (33b):
\end{styleStandard}

\begin{styleStandard}
(33)\ \ a.\ \ \textit{wir-Ø} 
\end{styleStandard}

\begin{styleStandard}
\ \ we-\textsc{infl}
\end{styleStandard}

\begin{styleStandard}
\ \ ‘we’
\end{styleStandard}

\begin{styleStandard}
b.\ \ \textit{wir}
\end{styleStandard}

\begin{styleStandard}
\ \ \ \ we
\end{styleStandard}

\begin{styleStandard}
\ \ \ \ ‘we’
\end{styleStandard}

\begin{styleStandard}
Roehrs proposes that an agreeing determiner takes a weak adjective; a non-agreeing determiner takes a strong adjective (see again (30a-b)). Elements like this exist elsewhere in German, with the qualification that they show an overt inflection in the agreeing cases; compare (34a) to (34b) (the same distribution holds with \textit{solch(e)} ‘such’ and \textit{welch(e)} ‘which):\footnote{\ Recall from Chapter 2, Footnote Error: Reference source not found that I analyze uninflected \textit{manch} in (34b) as a modificational element.}
\end{styleStandard}

\begin{styleStandard}
(34)\ \ a.\ \ \textit{manch-e gut-en \ \ \ \ Freunde}
\end{styleStandard}

\begin{styleStandard}
\ \ \ \ some-\textsc{st} good-\textsc{wk} friends
\end{styleStandard}

\begin{styleStandard}
\ \ \ \ ‘some good friends’
\end{styleStandard}

\begin{styleStandard}
\ \ b.\ \ \textit{manch gut-e \ \ \ \ \ Freunde}
\end{styleStandard}

\begin{styleStandard}
\ \ \ \ some \ \ good-\textsc{st} friends
\end{styleStandard}

\begin{styleStandard}
\ \ \ \ ‘some good friends’
\end{styleStandard}

\begin{styleStandard}
Note though that the notion of (non-)agreeing determiner is not entirely straightforeward. Unlike with \textit{manch}{}- ‘some’, \textit{solch}{}- ‘such’, and \textit{welch}{}- ‘which’, agreeing pronominals involve an (assumed) null inflection. However, given that first and second-person pronominal elements never have an overt inflection in German, this raises the question if the postulation of a null inflection is indeed warranted. 
\end{styleStandard}

\begin{styleStandard}
\ \ To sum up, given the issue of a true distinction between agreeing vs. non-agreeing pronominal elements, this proposal seems to lack proper motivation. Furthermore, this distinction would not work in the current system at all as inflections on adjectives are a function of the determiner stem – its categorial feature [+D] – but not its inflection.
\end{styleStandard}

\begin{styleStandard}
5.2.2. One Structure: Pronominal Elements in Different Positions
\end{styleStandard}

\begin{styleStandard}
Similar to the previous subsection, this type of proposal also involves one structure – the one in (32b). However, the pronominal element could be proposed to be in different phrasal levels in that structure. Like in the previous subsection, the pronominal could be in the DP-level. In this case, the pronominal would be a determiner trigging Impoverishment and, consequently, it would occur with a weak adjective. In addition, the pronominal could be in a different structural level. There are two options: either it could be a quantifier in CardP, or it could be a predeterminer in LPP. As explicated in more detail below, Impoverishment would not be triggered in either case, and the following adjective would surface with a strong ending.
\end{styleStandard}

\begin{styleStandard}
\ \ As a first option to explain a strong adjective, the pronominal element could do double duty as a quantifier in CardP. This option was entertained in a previous version of this work. As a quantifier, the pronominal element would not have the feature [+D], and as such, it would not trigger Impoverishment (for details, see Roehrs 2021, also Roehrs 2009a: 166-67). However, there does not seem to be evidence that pronominal elements can also appear as quantifiers. In fact, they can co-occur with numerals, a type of quantifier located in CardP:
\end{styleStandard}

\begin{styleStandard}
(35)\ \ \textit{wir drei \ Freunde}
\end{styleStandard}

\begin{styleStandard}
\ \ we three friends
\end{styleStandard}

\begin{styleStandard}
\ \ ‘us three friends’
\end{styleStandard}

\begin{styleStandard}
As such, this possibility seems unlikely to be on the right track.
\end{styleStandard}

\begin{styleStandard}
\ \ The second option to explain a strong adjective involves the claim that pronominal elements can do double duty as predeterminers in LPP. Unlike quantifiers, predeterminers are determiner-like elements with the feature [+D]. Unlike determiners, these elements are base-generated in LPP and neither trigger nor undergo Impoverishment (Chapter 2, Section 4). Ambiguous elements like this exist elsewhere in German. For instance, \textit{all-} ‘all’ and \textit{dies-} ‘this’ can be in two positions. When in the DP-level, these elements are determiners triggering Impoverishment (36):
\end{styleStandard}

\begin{styleStandard}
(36)\ \ a.\ \ \textit{alle gut-en} \ \ \ \ \ \textit{Freunde}
\end{styleStandard}

\begin{styleStandard}
\ \ \ \ all \ \ good-\textsc{wk} friends
\end{styleStandard}

\begin{styleStandard}
\ \ \ \ ‘all good friends’
\end{styleStandard}

\begin{styleStandard}
\ \ b.\ \ \textit{diese gut-en \ \ \ \ \ Freunde}
\end{styleStandard}

\begin{styleStandard}
\ \ \ \ these good-\textsc{wk} friends
\end{styleStandard}

\begin{styleStandard}
\ \ \ \ ‘these good friends’
\end{styleStandard}

\begin{styleStandard}
When in the LPP-level, they are predeterminers, which do not bring about Impoverishment, neither on a following determiner (37) nor on a lower adjective (38):
\end{styleStandard}

\begin{styleStandard}
(37)\ \ a.\ \ \textit{alle mein-e} \textit{gut-en} \ \ \ \ \ \textit{Freunde} 
\end{styleStandard}

\begin{styleStandard}
\ \ \ \ all \ \ my-\textsc{st} \ good-\textsc{wk} friends
\end{styleStandard}

\begin{styleStandard}
\ \ \ \ ‘all my good friends’
\end{styleStandard}

\begin{styleStandard}
\ \ \ b.\ \ \textit{diese mein-e gut-en \ \ \ \ Freunde}
\end{styleStandard}

\begin{styleStandard}
\ \ \ \ these my-\textsc{st} good-\textsc{wk} friends
\end{styleStandard}

\begin{styleStandard}
\ \ \ \ ‘these good friends of mine’
\end{styleStandard}

\begin{styleStandard}
(38)\ \ a.\ \ \textit{all mein} \textit{groß-es \ Glück}
\end{styleStandard}

\begin{styleStandard}
\ \ \ \ all my \ \ \ great-\textsc{st} happiness.\textsc{neut}
\end{styleStandard}

\begin{styleStandard}
‘all my great happiness’
\end{styleStandard}

\begin{styleStandard}
b.\ \ \textit{dieses mein groß-es \ Glück}
\end{styleStandard}

\begin{styleStandard}
\ \ \ \ this \ \ \ \ my \ \ \ great-\textsc{st} happiness.\textsc{neut}
\end{styleStandard}

\begin{styleStandard}
‘this my great happiness’
\end{styleStandard}

\begin{styleStandard}
Before moving on, note that this conjecture is more promising as pronominals of the first and second person seem to be associated with deixis; that is, these pronominals are similar to the demonstrative \textit{dies-} ‘this’ above, which can appear as either a determiner or a predeterminer.
\end{styleStandard}

\begin{styleStandard}
Again, it is hard to provide direct independent evidence for the claim that pronominal elements can be in two different positions – in this case, the DP-layer for pronominals as determiners and the LPP-layer for pronominals as predeterminers. On the one hand, these pronominal elements are distributionally restricted (they must occur before numerals, adjectives, and nouns); on the other hand, first and second-person pronominals are inflectionally restricted (they exhibit no endings). Thus far, I have come across only one distributional restriction: these pronominals cannot co-occur with ordinary predeterminers in German (39a-b):
\end{styleStandard}

\begin{styleStandard}
(39)\ \ a. \ *\ \ \textit{all(e) \ \ wir Studenten}\ \ \ \ 
\end{styleStandard}

\begin{styleStandard}
\ \ all(\textsc{st}) we \ students
\end{styleStandard}

\begin{styleStandard}
\ \ ‘all us students’
\end{styleStandard}

\begin{styleStandard}
\ \ b. \ *\ \ \textit{diese wir Studenten\ \ \ \ }
\end{styleStandard}

\begin{styleStandard}
\ \ these we \ students
\end{styleStandard}

\begin{styleStandard}
\ \ ‘(these) us students’
\end{styleStandard}

\begin{styleStandard}
Let us consider how the data in (39) bear on the question of whether pronominals can be determiners or predeterminers. As we see momentarily, the explanation of the ungrammaticality of (39) under the option of pronominals as determiners has an independent (and indeed language-specific) explanation. In contrast, the explanation of the ungrammaticality of (39) under the option of pronominals as predeterminers receives a more general explanation. We will wind up concluding that pronominals are indeed determiners but do not do double duty as predeterminers.
\end{styleStandard}

\begin{styleStandard}
Starting with the option of pronominal elements as determiners, Haider (1988: 53) discusses the contrast between the possessive article in (40a) and the Saxon Genitive in (40b) (see also Haider 1992: 314-15, Bayer 2015). Both of these elements are usually assumed to be in the DP-level:
\end{styleStandard}

\begin{styleStandard}
(40)\ \ a.\ \ \textit{all(e) \ \ ihr-e \ \ \ Autos}
\end{styleStandard}

\begin{styleStandard}
\ \ \ \ all(\textsc{st}) her-\textsc{st} cars
\end{styleStandard}

\begin{styleStandard}
\ \ \ \ ‘all her cars’
\end{styleStandard}

\begin{styleStandard}
\ \ b. \ *\ \ \textit{all(e) \ \ Marias Autos}
\end{styleStandard}

\begin{styleStandard}
\ \ \ \ all(\textsc{st}) Mary’s cars
\end{styleStandard}

\begin{styleStandard}
\ \ \ \ ‘all Mary’s cars’
\end{styleStandard}

\begin{styleStandard}
To the constrast in (40), we can add a similar distinction involving the predeterminer \textit{dies-} ‘this’ (41). While the (a)-example may sound slightly marked, there is a clear contrast between (41a) and (41b), just as in (40) above: 
\end{styleStandard}

\begin{styleStandard}
(41)\ \ a. \ ?\ \ \textit{dies-e \ \ \ \ ihr-e \ \ Autos}
\end{styleStandard}

\begin{styleStandard}
\ \ \ \ these-\textsc{st} her-\textsc{st} cars
\end{styleStandard}

\begin{styleStandard}
\ \ \ \ ‘these cars of hers’
\end{styleStandard}

\begin{styleStandard}
b. \ *\ \ \textit{dies-e \ \ \ \ Marias Autos}
\end{styleStandard}

\begin{styleStandard}
\ \ \ \ these-\textsc{st} Mary’s cars
\end{styleStandard}

\begin{styleStandard}
\ \ \ \ ‘(these) Mary’s cars’
\end{styleStandard}

\begin{styleStandard}
The difference between \textit{ihre} ‘her’ in (40a) and (41a), on the one hand, and \textit{wir} ‘we’ in (39a-b) and \textit{Marias} ‘Mary’s’ in (40b) and (41b), on the other hand, is that the former has an overt inflection but latter do not. Indeed, \textit{wir} and \textit{Marias} never take an overt adjectival inflection (unlike \textit{mein} in (38), which, for instance, may appear as \textit{mein-em }in dative masculine/neuter contexts). The failure to (ever) take an inflection may account for this constrast (for fuller discussion, see Roehrs 2021). If this is on the right track, then we can account for the ungrammaticality of (39a-b) under the option of pronominals as determiners on par with the ungrammaticality of the determiner-like element \textit{Marias} in (40b) and (41b). Given that the ungrammaticality receives an independent explanation, we can continue maintaining that pronominals are determiners.\footnote{\ This is confirmed by the fact that this is a language-specific restriction. Nominals similar to (39a) above are possible in English, as pointed out by Pesetsky (1978: 353)\par \ \ (i)\ \ all us linguists\ \ } Indeed, this explanation is based on the very assumption that pronominal elements are in fact determiners in the DP-level, just like possessive articles and Saxon Genitives. To repeat, what is crucial here is that determiner(-like) elements in German have to be inflected if they follow a predeterminer. 
\end{styleStandard}

\begin{styleStandard}
\ \ As to the option of pronominal elements as predeterminers, it could be suggested that pronominals can also be in LPP. As seen in (42), ordinary predeterminers can be stacked: 
\end{styleStandard}

\begin{styleFooter}
(42)\ \ \textit{all-e \ \ dies-e \ \ \ mein-e Freunde}
\end{styleFooter}

\begin{styleStandard}
\ \ all-\textsc{st} these-\textsc{st} my-\textsc{st} friends
\end{styleStandard}

\begin{styleFootnote}
\ \ ‘all these friends of mine’
\end{styleFootnote}

\begin{styleStandard}
As such, there is no structural competition between predeterminers – each can be housed in its own LPP. In other words, the occurrence of multiple predeterminers should be fine. However, as we see momentarily, while analyzing the strings in (39a-b) in such a way provides an explanation of the ungrammaticality of these cases, it does not for the simpler, grammatical cases as in (30b). 
\end{styleStandard}

\begin{styleStandard}
Recall that pronominal DPs are definite in interpretation. If pronominal elements as predeterminer are in LPP, then the DP-level does not involve an overt determiner (e.g., a definite article) or determiner-like element (e.g., a Saxon Genitive). This, however, is something that is required for definite noun phrases. Note in this regard that \textit{Studenten} ‘students’ by itself cannot have a definite interpretation – the DP-level must be filled with a relevant overt element (see Alexiadou 2005, Longobardi 1994). Thus, the option of pronominals as predeterminers can also explain the ungrammaticality of (39), but, unlike the option of pronominals as determiners, this option receives a more general explanation. To repeat, the DP-layer must have an overt determiner(-like) element when the nominal is definite in interpretation.
\end{styleStandard}

\begin{styleStandard}
With this in mind, I return to the simple cases like \textit{wir nette Studenten} ‘we nice-\textsc{st} students’ as in (30b). If the discussion above is on the right track, then this means that such grammatical strings cannot be analyzed as \textit{wir} involving a predeterminer, as the DP-layer would not contain an overt element although the nominal is definite in interpretation. Such a structure would lead to ungrammaticality. However, since this simple pronominal DP is fine, we can argue that the pronominal must be a determiner in the DP-level here. 
\end{styleStandard}

\begin{styleStandard}
\ \ To conclude, both options of pronominal as quantifier and pronominal as predeterminer are unlikely to be correct. As such, the variation of the inflection on the adjective remains unclear on those assumptions.
\end{styleStandard}

\begin{styleStandard}
5.2.3. Two Different Structures: Complementation and Adjunction \ 
\end{styleStandard}

\begin{styleStandard}
Besides the structure in (32b), a second structure could be proposed to account for the possibility of a strong ending on the adjective. Beside being the complement of the pronominal determiner, we could assume that the adjective and noun is right-adjoined to the pronominal determiner. There are two basic options: the adjective and noun could be adjoined either high or in the middle of the structure. High right-adjunction was utilized in the discussion of appositional DPs (Chapter 2, Section 3.4). The structure of the pronominal DP would look as follows:
\end{styleStandard}

\begin{styleStandard}
(43)\ \ a.\ \ \textit{wir nett-e \ \ Studenten}
\end{styleStandard}

\begin{styleStandard}
\ \ \ \ we nice-\textsc{st} students
\end{styleStandard}

\begin{styleStandard}
\ \ \ \ ‘us nice students’
\end{styleStandard}

\begin{styleFootnote}
\textit{\ \ \ }b.\ \ \textit{Pronominal DP – High Right-Adjunction}
\end{styleFootnote}

\begin{styleFootnote}
\textit{\ \ }\ \  \ DP
\end{styleFootnote}

\begin{styleFootnote}
[Warning: Draw object ignored][Warning: Draw object ignored]
\end{styleFootnote}

\begin{styleFootnote}
\ \ DP\ \ \ \ AgrP
\end{styleFootnote}

\begin{styleStandard}\itshape
\ \ wir\ \  \ \ \ nette Studenten
\end{styleStandard}

\begin{styleStandard}
Importantly, unlike appositional DPs, pronominal DPs do not involve comma intonation, and their adjectives may, given the right conditions, receive a restrictive interpretation (in the plural cases). Alternatively, the adjunction could be located in the middle (44), where XP might range over CardP, AgrP, ArtP, NumP, or NP (recall that e\textsubscript{N} is the null head noun of a pronominal DP with an intransitive determiner):
\end{styleStandard}

\begin{styleStandard}
(44)\textit{ \ \ Pronominal DP – Mid Right-Adjunction}
\end{styleStandard}

\begin{styleFootnote}
\textit{\ \ }\ \  \ DP
\end{styleFootnote}

\begin{styleFootnote}
[Warning: Draw object ignored][Warning: Draw object ignored]
\end{styleFootnote}

\begin{styleFootnote}
\ \ \textit{wir}\ \ \ \  XP
\end{styleFootnote}

\begin{styleStandard}\itshape
[Warning: Draw object ignored][Warning: Draw object ignored]\ \ \ \ 
\end{styleStandard}

\begin{styleStandard}
\ \ \ \  XP\ \ \ \ AgrP
\end{styleStandard}

\begin{styleStandard}
[Warning: Draw object ignored]\textit{\ \ \ \ \ \ \ \ \ \ \ \ \ \ \ \ nette Studenten}
\end{styleStandard}

\begin{styleStandard}
\ \ \ \  \textit{e}\textit{\textsubscript{N}}
\end{styleStandard}

\begin{styleStandard}
Mid right-adjunction was employed in the discussion of indefinite pronoun constructions. The latter construction does not involve comma intonation and its adjective may have a restrictive interpretation. These properties fit better with pronominal DPs. Thus, unlike high adjunction, this structure presents a more promising candidate. Note now that this is a non-canonical structure. As such, Impoverishment would not occur, and a strong ending would surface on the adjective. Observe that assuming two structures (complementation, adjunction) would essentially result in the more general claim that pronominals are equally transitive and intransitive (cf. Bhatt 1990: 155), where complementation of the adjective and noun would instantiate a transitive pronominal element, but adjunction of the adjective and noun would manifest an intransitive pronominal element. This claim is interesting and requires further investigation, something I will not pursue here.
\end{styleStandard}

\begin{styleStandard}
\ \ To take stock thus far, note that despite the issues raised for the first two types of accounts above, all three proposals can potentially account for the strong adjectives, which are possible in all contexts. However, given these proposals, it is not clear why weak adjectives can occur in only some contexts. I make this issue of the three proposals more explicit in the next subsection.
\end{styleStandard}

\begin{styleStandard}
5.2.4. Absence of Weak Adjectives in Certain Instances
\end{styleStandard}

\begin{styleStandard}
Recall that the different structural options above were entertained to account for the variation of the inflections on the adjective following the pronominal determiner. As mentioned above, pronominal determiners are definite elements. As such, we expect weak adjectives to occur with them. However, as documented above, all featural combinations pronominal DPs appear in exhibit strong adjectives but only some combinations allow weak adjectives. Weak adjectives are missing in three instances: nominative and accusative singular (which is very clear in the masculine/neuter) and accusative plural. All three types of proposals discussed above have to come to terms with the variation in some contexts but the lack of it in others – the three instances just mentioned. 
\end{styleStandard}

\begin{styleStandard}
\ \ As I show below, all three proposals can account for the absence of weak adjectives in the nominative/accusative singular. Given certain assumptions, they all can account for the lack of variation here. It will turn out though that including the dative singular in the discussion will bring out issues for all three proposals. The absence of the weak adjectives in the accusative plural is special. As far as I can see, none of the three proposals can rule them out without ad-hoc stipulations. At the end of this subsection, I provide a functional explanation for the absence of weak adjectives here.
\end{styleStandard}

\begin{styleStandard}
\ \ In order to account for the absence of weak adjectives in the nominative/accusative singular, we could follow Roehrs (2005) in observing that the strong adjectives in these contexts are mirrored by certain strings involving \textit{ein}{}-words (also Bhatt 1990: 155). The first proposal could suggest that the pronominal elements involve non-agreeing determiners. Like \textit{ein}, these instances of the pronominal determiner would have no inflections at all, and consequently the adjectives would be strong. The second proposal could also assume that these instances of the pronominal determiner are like \textit{ein}{}-words. However, unlike the first proposal, this account could follow the assumptions in Chapter 2 claiming that these pronominal determiners do not induce Impoverishment. Consequently, the adjectives would be strong. The third proposal could take a different route. It could claim that all instances in the singular, including the dative, only involve adjunction (and not complementation) of the adjective and noun. 
\end{styleStandard}

\begin{styleStandard}
If so, the question arises why the dative singular does involve variation. The first two proposals have a straightforward answer: \textit{ein}{}-words are only special in the nominative and accusative singular. The third proposal has to make additional assumptions to allow weak adjectives in the dative. Now, while this may indicate an advantage of the first two proposals, an account involving \textit{ein}{}-words is not straightforward. As discussed in Chapter 2, Section 2.2.3 and Chapter 5, \textit{ein}{}-words are composite forms consisting of semantically vacuous \textit{ein} and another component. As such, it is \textit{ein} itself that has special properties. Crucially, \textit{ein} contrasts with pronominal elements as regards definiteness. This makes these two types of elements rather different from each other, and an extension of the properties of \textit{ein} to these pronominal elements does not seem straightforward. To be clear, given these assumptions, all three proposals face some challenges, either the parallel to \textit{ein}{}-words is not straightforward or additional assumptions have to be made to explain all the data. It appears then as if a different solution needs to be found (see next sextion).
\end{styleStandard}

\begin{styleStandard}
\ \ Turning to the instances in the accusative plural, recall that a weak adjective cannot occur in this context (45a). Furthermore, observe again that the dative counterpart in (45b) exhibits the ending -\textit{en} on the adjective, which is ambiguous between strong and weak:
\end{styleStandard}

\begin{styleStandard}
(45)\textit{\ \ }a.\ \ \textit{für uns nett-e(*n) \ \ \ \ Schüler}
\end{styleStandard}

\begin{styleFootnote}
\ \ \ \ for us \ \ nice-\textsc{st}/*\textsc{wk} pupils
\end{styleFootnote}

\begin{styleFootnote}
\ \ \ \ ‘for us nice pupils’
\end{styleFootnote}

\begin{styleStandard}
\textit{\ \ \ }b.\textit{\ \ von \ \ uns nett-en \ \ \ \ \ \ \ Schüler-n}
\end{styleStandard}

\begin{styleFootnote}
\ \ \ \ from us \ \ nice-\textsc{st/wk} pupils-\textsc{dat}
\end{styleFootnote}

\begin{styleFootnote}
\ \ \ \ ‘from us nice pupils’
\end{styleFootnote}

\begin{styleFootnote}
Note now that case is not overtly manifested by a unique form of the pronominal element – it is \textit{uns} ‘us’ (or \textit{euch} ‘you(\textsc{pl})’) in both the accusative and dative cases. In addition, notice also that the weak ending in the accusative plural would be -\textit{en}, the same as the (ambiguous) ending in the dative plural. As such, it is the different inflections on the adjectives in (45a-b) that distinguish the accusative from the dative.\footnote{\ Note that the head noun has the case inflection -\textit{n} in the dative in (45b), but not in the accusative in (45a) This inflection may potentially indicate case in the dative. There are two points that need to be made here. First, case in this instance would be manifested on the noun (rather than on the pronominal element). Second, Wegener (1995: 154-63) and Gallmann (1996: 289) point out that this case inflection on the noun has become somewhat unstable. Furthermore, this case marking cannot occur at all if the plural of the noun is formed in -\textit{n} or -\textit{s}, for instance, \textit{Junge-n-(*en)} ‘boy-s’ or \textit{Jung-s-(*en)} ‘(coll.) guy-s’. Given these points, I put the case marking on the noun aside.} 
\end{styleFootnote}

\begin{styleFootnote}
\ \ It is not clear how the three proposals can account for the lack of variation in the accusative plural: \textit{ein}{}-words are only special in the singular instances, and complementation (which allows weak inflections) should not be restricted to specific morphological cases like nominative and dative. As such, the accusative plural instances seem to be special. Given this state-of-affairs, I follow Duden (2007: 39) in providing a functional explanation of the absence of the weak inflection here. 
\end{styleFootnote}

\begin{styleFootnote}
I assume that morphological case needs to be distinguished (if possible). Since the cases at hand involve dative and accusative, this may be related to their typical grammatical functions, indirect object and direct object, respectively. Now, as is well known, German has relatively free word order due to the availability of scrambling its objects. Roehrs (2005: 261-62) provides the following examples where the pronominal DPs are in the order of indirect object {\textgreater} direct object in (46a) but in the sequence of direct object {\textgreater} indirect object in (46b): \ 
\end{styleFootnote}

\begin{styleFootnote}
(46)\ \ a.\ \ \textit{Sie \ \ haben uns nett-en \ \ \ \ \ \ \ \ \ \ \ \ \ \ \ Jungen euch klug-e \ \ \ \ \ \ \ \ \ \ \ Mädchen vorgestellt}.
\end{styleFootnote}

\begin{styleFootnote}
\ \ \ \ they have \ \ us \ \ nice-\textsc{dat.st/wk} boys \ \ \ \ \ you \ smart-\textsc{acc.st} girls \ \ \ \ \ \ \ \ introduced
\end{styleFootnote}

\begin{styleFootnote}
\ \ \ \ ‘They introduced you smart girls to us nice boys.’
\end{styleFootnote}

\begin{styleFootnote}
\ \ b.\ \ \textit{Sie \ \ haben uns nett-e \ \ \ \ \ \ \ \ \ \ Jungen euch klug-en \ \ \ \ \ \ \ \ \ \ \ \ \ \ \ \ Mädchen vorgestellt}.
\end{styleFootnote}

\begin{styleFootnote}
\ \ \ \ they have \ \ us \ \ nice-\textsc{acc.st} boys \ \ \ \ you \ \ smart-\textsc{dat.st/wk} girls \ \ \ \ \ \ \ \ introduced
\end{styleFootnote}

\begin{styleFootnote}
\ \ \ \ ‘They introduced us nice boys to you smart girls.’
\end{styleFootnote}

\begin{styleFootnote}
Note that despite the two orderings of the objects in (46a) vs. (46b), the different inflections on the adjectives inside the objects make it clear who was introduced to whom in each case. This functional explanation of the strong inflection in the accusative plural – the strong adjective clearly distinguishes accusative objects from dative objects – can apply to all three (and other) proposals. 
\end{styleFootnote}

\begin{styleFootnote}
Returning to the first part of this section, given the issues with the postulation of (non-)agreeing pronominals in the first proposal, with the suggestion that pronominals can do double duty as determiners and quantifiers or predeterminers in the second account, and with the claim of two co-existing types of structures in the third analysis, I make a different suggestion. 
\end{styleFootnote}

\begin{styleStandard}\itshape
5.3.\ \ Pronominal Determiners and Complementation 
\end{styleStandard}

\begin{styleStandard}
In this section, I offer my proposal of the structure of pronominal DPs. I also discuss the structure and feature specifications of the pronominal determiners themselves, I provide the vocabulary insertion rules for pronominal elements, and I account for the preferences of strong or weak adjectives in certain pronominal DPs. 
\end{styleStandard}

\begin{styleStandard}
5.3.1. Structures and Feature Specifications
\end{styleStandard}

\begin{styleStandard}
I propose that pronominal DPs involve complementation and that pronominal elements are determiners that lack features that make [+D] a trigger of Impoverishment. In other words, this proposal postulates one structure and one type of pronominal element. The fact that Impoverishment does not occur accounts for the general option of strong adjectives. The restricted occurrence of weak adjectives follows from certain phonetic considerations – the phonetic rule discussed in Section 2.2 and phonetically conditioned analogy with ordinary definite DPs. 
\end{styleStandard}

\begin{styleStandard}
\ \ As far as I am aware, the proposal sketched just above is the simplest account with the fewest additional assumptions. Furthermore, proposing that pronominal DPs involve complemenation is compatible with the discussion in later chapters. Specifically, Chapters 6 and 7 investigate similarities and differences between noun phrases (DPs) and clauses (TPs) of the form “pronoun + (copular verb) + noun”. TPs involve complementation and if DPs also constitute cases of complementation, where the string “pronoun + noun” is analyzed as the determiner taking the noun as its complement, then this allows a straightforward comparison between the nominal and clausal domains.
\end{styleStandard}

\begin{styleStandard}
\ \ The proposal that pronominal elements are determiners is compabitle with work by Postal (1966) and others, briefly discussed in Chapter 2, Section 3.5. As determiners, pronominals have the categorial feature [+D]. They are merged in ArtP and take overt complements. I propose that the difference in inflections on adjectives in ordinary vs. pronominal DPs lies in the different structure and feature specifications of the pronominal determiners themselves. At first glance, there seem to be two options. 
\end{styleStandard}

\begin{styleStandard}
Pronominal elements could involve structures of demonstratives as in (47a). Recalling that the relevant pronominal determiners are of the first and second person, they are associated with deixis, similar (but not identical) to that of third-person demonstratives. I follow Halle (1997: 429) and Nevins (2007: 288) in that grammatical person can be broken down into the features [±AUTH(or)] and [±PART(icipant)]: first person consists of [+AUTH, +PART], second person involves [-AUTH, +PART], and third person has [-AUTH, -PART].\footnote{\ It is possible that third person only includes [-PART].} As an alternative to (47a), pronominal determiners could project reduced structures where InflP is missing at the top (47b). This yields the following feature specifications for the two structural options of, say, the first-person pronominal \textit{wir} ‘we’:
\end{styleStandard}

\begin{styleStandard}
(47)\ \ a.\ \ \textit{Pronominal Determiner (Variant 1)}
\end{styleStandard}

\begin{styleJBExample}
\ \ \ \ \ \ \ \ \ \ \ \  \ \ \ \ InflP\textsubscript{[+D; +AUTH, +PART][F, N, O, S]}
\end{styleJBExample}

\begin{styleStandard}\bfseries
[Warning: Draw object ignored][Warning: Draw object ignored]
\end{styleStandard}

\begin{styleStandard}
\ \ \ \ \ \ \ \ \ \ [F, N, O, S]\ \ \ \ DemP
\end{styleStandard}

\begin{styleStandard}
\ \ \ \ \ \ \ \  \ \ \ {\textbar} \ \ \ \ \ \ 
\end{styleStandard}

\begin{styleStandard}
\ \ \ \ \ \ \ \ Dem
\end{styleStandard}

\begin{styleStandard}
\ \ \ \  \ \ \ \ \ \ \ \ \ [+D; +AUTH, +PART]\ \ \ \ →\ \ \textit{wir}
\end{styleStandard}

\begin{styleStandard}
\ \ b.\ \ \textit{Pronominal Determiner (Variant 2)}
\end{styleStandard}

\begin{styleStandard}
\ \ \ \ \ \ \ \ \ \ \ \ \ \ \ \ DemP\textsubscript{[+D; +AUTH, +PART]}
\end{styleStandard}

\begin{styleStandard}
\ \ \ \ \ \  \ \ \ {\textbar} \ \ \ \ \ \ 
\end{styleStandard}

\begin{styleStandard}
\ \ \ \ \ \ Dem
\end{styleStandard}

\begin{styleStandard}
\ \  \ \ \ \ \ \ \ \ [+D; +AUTH, +PART]\ \ \ \ →\ \ \textit{wir}
\end{styleStandard}

\begin{styleStandard}
The immediate advantage of the variant in (47a) is that all phrasal determiners have the same basic structure. As for (47b), this reduced demonstrative structure provides a straightforward account of the absence of inflections on first and second-person elements. Note that neither of the two structures involves features for definiteness or deixis. However, first and second person (i.e., [+PART]) presumably entail definiteness, and the category person is associated and indeed relatable to the category deixis (see, e.g., Lyons 1999: 18-19, Roehrs 2019: 378-79). Crucially, with [+DEF] and [+DEIX] absent, the presence of the CNG feature bundle in (47a) but its absence in (47b) have different consequences for the occurrence of Impoverishment. Thus, depending on which of the two structural variants in (47) is adopted, we need to find different ways to account for the general variation and the varying preferences as regards the inflectional distribution on the adjectives. 
\end{styleStandard}

\begin{styleStandard}
\ \ If we adopt (47a), we expect the adjectives after the pronominal determiners to pattern with those after \textit{ein}{}-words – the featural specification in (47a) is similar to the article \textit{ein} in that Impoverishment is not triggered in [-F, -O] contexts. In other words, the masculine/neuter instances in the nominative/accusative are expected to surface with strong adjectives. By contrast, Impoverishment is triggered in all other featural contexts yielding weak adjectives in the remaining cases (including the nominative/accusative feminine, where both the strong and the weak inflections are -\textit{e}). However, it is not clear how to account for the strong inflections in the oblique and/or plural instances. By assumption, there is no other structural option available as regards the larger DP (as all these pronominal DPs involve canonical structures). Furthermore, note that an explanation involving analogy with other, related DPs is not straightforward as definite DPs of the form “determiner + adjective + noun” do not have strong inflections in these instances either. This means that the strong inflections in the dative singular and in the plural instances remain unaccounted for. The variant in (47b) fares better.
\end{styleStandard}

\begin{styleStandard}
If (47b) is adopted, all adjectives should be strong. With no feature present that makes [+D] a trigger of Impoverishment, feature deletion does not occur at all. This accounts for the general option of strong adjectives. The restricted occurrence of weak adjectives requires a different account. There are three cases to consider. Note that all three can be related by certain phonetic considerations. 
\end{styleStandard}

\begin{styleStandard}
First, recall that instances in the dative masculine/neuter involve strong adjectives ending in -\textit{em} or weak adjectives ending in -\textit{en}. Notice that the weak inflection has a phonetic resemblance to the strong inflection – both involve nasal sounds. Extending the discussion in Section 2.2, I assume that this presents another case of nasal alternation such that -\textit{em} can be changed into -\textit{en}, provided a nominal element (here: the pronominal) precedes the adjective.
\end{styleStandard}

\begin{styleStandard}
\ \ Second, the dative feminine instances involve phonetic resemblance between – what sounds like – strong inflections on the pronominal determiners \textit{mi-r }‘me(\textsc{dat})’ and \textit{di-r} ‘you(\textsc{dat.sgl})’ and the strong inflection -\textit{er} on following adjectives. This brings about a special distribution such that two strong inflections occur on two different lexical categories at the same time (i.e., the pronominal determiner and adjective). Sequences like \textit{mi-r} \textit{groß-er Gans }‘me (stupid idiot)’ do not exist elsewhere in the system. Following Duden (1995: 281, 2007: 39), I assume that speakers tend to avoid such strings. Note again that the final sound on \textit{mir} and \textit{dir} is the same as the inflection on the dative feminine article \textit{der} ‘the’. I suggest that the weak adjectives are the result of analogy with other definite DPs (e.g., \textit{d-er nett-en Frau} ‘the-\textsc{st} nice-\textsc{wk} woman’). 
\end{styleStandard}

\begin{styleStandard}
Before moving on to the third case, the plural instances, notice that there is no phonetic rememblance in the nominative/accusative singular, neither as regards the nasal alternation on adjectives nor as regards the inflections on adjectives and the final sound on the pronominal determiners: the strong inflections -\textit{er}, -\textit{en}, -\textit{es}, -\textit{e} are very different from (or identical to) the weak inflection -\textit{e}, and all these inflections are different from the final sound on the pronominal determiners \textit{ich} ‘I’, \textit{mich} ‘me(\textsc{acc})’, \textit{du} ‘you(\textsc{nom.sgl})’, \textit{dich} ‘you(\textsc{acc.sgl})’. This means that the phonetic rule and analogy do not play a role in the nominative/accusative singular intances – as mentioned above, there is no variation here at all.
\end{styleStandard}

\begin{styleStandard}
\ \ The third case involves the instances in the plural. Starting with the dative, recall that the inflection is -\textit{en}, which is ambiguous between strong and weak. Given my analysis, I interpret this inflection as strong. Continuing with the nominative, note that we cannot easily state that the weak endings here are also due to the phonetic considerations mentioned above: the strong inflection -\textit{e} is different from the weak inflection -\textit{en}, and both of these inflections are different from the final sound on the pronominals \textit{wir} ‘we’ and \textit{ihr} ‘you(\textsc{nom.pl})’. However, like in the dative feminine, the forms of the ordinary and pronominal determiners are related. Unlike in the dative feminine, here the relatedness concerns the fact that \textit{die} ‘the/those’ has no inflection, just like \textit{wir }and \textit{ihr}. I suggest that the weak adjectives in the nominative plural are also due to analogy with other definite DPs (e.g., \textit{die nett-en Frauen} ‘the nice-\textsc{wk} women’). Finally, observe that uninflected \textit{die} also occurs in the accusative plural. However, as discussed in Section 5.2.4, pronominal DPs in the accusative plural are special. I assume that the analogy is blocked here – the absence of the weak adjectives has a functional explanation.
\end{styleStandard}

\begin{styleStandard}
\ \ Above, I made use of analogy with ordinary definite DPs in two instances, in the dative feminine and in the nominative plural. In order to constrain the application of analogy, I have assumed that analogy is only possible if there are special phonetic circumstances as regards the determiners: related endings on \textit{mi-r/di-r} and \textit{d-er}, and no inflections on \textit{wir/ihr} and \textit{die}. Note in passing that the presence or absence of endings on determiners seems to play a role here for the inflections on adjectives. This is reminiscent of the traditional generalization of Weak After Strong. However, both cases bring about weak adjectives by analogy, making these cases different from \textit{ein}{}-words (which have strong endings in the exceptional cases).
\end{styleStandard}

\begin{styleStandard}
\ \ To sum up, all pronominal DPs may involve strong adjectives. Assuming complementation, this is explained by the reduced demonstrative structure in (47b) and the feature specifications of the first and second-person pronominals. As a consequence, Impoverishment does not occur resulting in strong adjectives. Weak adjectives in the dative masculine/neuter are due to the fairly general phonetic rule that accounts for the nasal alternation on adjectives. Weak adjectives in the dative feminine and nominative plural are due to analogy (constrained by the similarity of the inflections of the ordinary determiners and the final sounds of the pronominal determiners). In other words, weak adjectives in all these cases only occur under specific phonetic conditions. Also, note again that masculine/neuter pattern in opposition to feminine/plural, which, as seen before, is a more general property of the nominal system in German. Before turning to the discussion of the different preferences as regards strong or weak adjectives, I provide the vocabulary insertion rules for pronominal elements.
\end{styleStandard}

\begin{styleStandard}
5.3.2. Vocabulary Insertion Rules for Pronominal Elements
\end{styleStandard}

\begin{styleStandard}
I illustrate the vocabulary insertion rules with first-person elements (second-person items are like the first-person ones but have a negative value for [AUTH]). Third-person pronominals are provided for completeness’ sake (for some discussion, see also Höhn 2020). Starting with the first-person pronominals, I take genitive pronouns to be the most specific elements. Recall though that they do not take overt complements. This is indicated in (48a-b) such that no overt element can be in the right edge of the noun phrase. Note also that the CNG features after the forward slash sign in (48) specify the featural context of insertion. As to the pronominals that do take overt complements, I pointed out in Section 5.1 that \textit{wir} ‘we’ is the only (transitive) pronominal that has a unique form for the relevant CNG features. Indeed, singular pronominals do not distinguish different genders, and \textit{uns} ‘us(\textsc{acc/dat})’ is the only form that does not distinguish (accusative/dative) case. Importantly, note again that accusative and dative case do not form a natural class in German (i.e., they are not syncretic except here). Given these points, I take \textit{wir} as the most specific transitive element (48c), and \textit{uns} as the least specific transitive item (48g). The latter presents the elsewhere case. The remaining, singular pronouns are provided in (48d-f). The vocabulary insertion rules below are ordered by decreasing specificity and are grouped according to case. Similar to the weak adjectival ending -\textit{e }(Chapter 2, Section 2.1.5), I utilize the category variable [$\gamma $], where a negative value indicates singular (48e-f):
\end{styleStandard}

\begin{styleStandard}
(48)\ \ a.\ \ [+D; +AUTH, +PART]\ \ → \ \ \textit{unserer} / \_\_]\textsubscript{$\Phi $} [+F, +N, +O, +S]
\end{styleStandard}

\begin{styleStandard}
\ \ b.\ \ [+D; +AUTH, +PART]\ \ →\ \ \textit{meiner} \ / \_\_]\textsubscript{$\Phi $} [+O, +S]
\end{styleStandard}

\begin{styleFootnote}
\ \ c.\ \ [+D; +AUTH, +PART]\ \ →\ \ \textit{wir} \ \ \textsubscript{\ \ \ \ \ \ \ }/ \ \ \ \ \ \ \ \ \textsubscript{\ }[+F, +N, -O,\textsubscript{ }\ {}-S]
\end{styleFootnote}

\begin{styleStandard}
\ \ d.\ \ [+D; +AUTH, +PART]\ \ →\ \ \textit{ich} \ \ \ \ \ \ \ / \ \ \ \ \ \ \ \ \textsubscript{\ }[ -O, \ {}-S]
\end{styleStandard}

\begin{styleStandard}
\ \ e.\ \ [+D; +AUTH, +PART]\ \ →\ \ \textit{mich} \ \ \ \ / \ \ \ \ \ \ \ \ \ [-$\gamma $, \ {}-O]
\end{styleStandard}

\begin{styleStandard}
\ \ f.\ \ [+D; +AUTH, +PART]\ \ →\ \ \textit{mir}\ \  / \ \ \ \ \ \ \ \ \ [-$\gamma $]
\end{styleStandard}

\begin{styleStandard}
\ \ g.\ \ [+D; +AUTH, +PART]\ \ →\ \ \textit{uns} \textsubscript{\ \ }\ \ \ \ \ \ \ \ \ \ \ \textsubscript{\ }\ \ 
\end{styleStandard}

\begin{styleStandard}
To round off the discussion, I also provide the vocabulary insertion rules for the third-person pronominals. 
\end{styleStandard}

\begin{styleStandard}
Without going into much detail here, it is clear that pronominals like \textit{er} ‘he’ have to be distinguished from ordinary determiners like \textit{der-}words\textit{, ein-}words, or the null articles. Note that pronominals are inherently definite but that ordinary determiners are specified for definiteness (see again the different determiner structures in Chapter 2, Section 2.1.6). As such, I assume that the features [-AUTH] and/or [-PART] also entail definiteness. This allows me to leave out the feature for definiteness in the vocabulary insertion rules, provided in (49).\footnote{\ If this turns out to be untenable, then the feature [+DEF] could be added to these pronominals. Note that this does not affect the discussion of Impoverishment as third-person pronominals do not take adjectives (and nouns) as their complement.} As above, the rules are ordered with decreasing specificity and are grouped according to case (if possible). Note that unlike first and second-person pronominals, their third-person counterparts involve adjectival inflections. Consequently, they have separate CNG feature bundles. As briefly discussed in Chapter 2, Section 2.2.3, I single out exceptional \textit{ihn} ‘him’ as the most specific case (49a). Recall also that I pointed out above that the genitive pronominals do not have regular inflections. As such, I assume that like uninflected \textit{ein}, they spell out the stem features and the CNG features by one morpheme (49b-c). Furthermore, (49c) and (49d) are equally specified but differ in the value of [O]. The remaining pronominals are given in (49e-f), with (49f) forming the elsewhere case. The inflections for (49a) and (49d-f) are given in (49g-h) (for some discussion, see also Gunkel \textit{et al}. 2017: 1298):\footnote{\ The elsewhere case in (49f) involves an additional -\textit{en} in the dative plural. This makes \textit{ih-n-en }‘them’ similar to the dative plural pronominal \textit{d-en-en} ‘those (ones)’ (for the historical development of this additional -\textit{en}, see Lühr 1991). Interestingly, this additional -\textit{en} is not possible with \textit{ein}{}-words (e.g., *\textit{mein-en-en} ‘mine’, *\textit{kein-en-en} ‘none’). Again, what the latter elements have in common is that they consist of vacuous \textit{ein} and another component. The obvious difference between the first type of elements and \textit{ein} is that the first involves definiteness. There are different ways to account for this additional -\textit{en} on the definite pronominals. We could follow, for instance, work by Corver \& van Koppen (2010, 2011b), who propose that -\textit{en} in Dutch is generated in a lower position, or we could formulate a late insertion rule. I leave the decision between these options for future research. }
\end{styleStandard}

\begin{styleStandard}
(49)\ \ a.\ \ [+D; -AUTH, -PART] \ \ \ \ → \ \ \textit{ih-}\textsubscript{ }\ \ \ \ \ \ \ \ / \_\_]\textsubscript{$\Phi $} [-F, -N, -O, +S]
\end{styleStandard}

\begin{styleFootnote}
b.\ \ [+D; -AUTH, -PART]\ \ [ -F, +O, +S]\ \ → \ \ \textit{seiner \ }\textit{\textsubscript{\ }}\ / \_\_]\textsubscript{$\Phi $} 
\end{styleFootnote}

\begin{styleFootnote}
c.\ \ [+D; -AUTH, -PART][+O, +S]\ \ → \ \ \textit{ihrer \ }\textit{\textsubscript{\ \ \ \ }}\ / \_\_]\textsubscript{$\Phi $} 
\end{styleFootnote}

\begin{styleFootnote}
d.\ \ [+D; -AUTH, -PART]\ \ \ \ \ \ → \ \ \textit{e- \ }\textit{\textsubscript{\ \ \ \ \ \ \ \ \ \ \ }}\ / \_\_]\textsubscript{$\Phi $} [ -F, \ {}-O]
\end{styleFootnote}

\begin{styleStandard}
e.\ \ [+D; -AUTH, -PART]\ \ \ \ \ \ → \ \ \textit{sie-} \ \ \ \ \ \ \ / \_\_]\textsubscript{$\Phi $} [-O]
\end{styleStandard}

\begin{styleStandard}
f.\ \ [+D; -AUTH, -PART]\ \ \ \ \ \ → \ \ \textit{ih-}\textsubscript{ \ \ \ \ \ \ \ \ \ \ \ }\ / \_\_]\textsubscript{$\Phi $} \ \ 
\end{styleStandard}

\begin{styleStandard}
g.\ \ [+F, -N, +O, $\alpha $S] \ \ \ \ \ \ →\ \ \textit{{}-r}
\end{styleStandard}

\begin{styleStandard}
\ \ h.\ \ etc.
\end{styleStandard}

\begin{styleStandard}
Recall from Section 3 that the inflection on \textit{sie} ‘she, her; they, them’ is later deleted to avoid a hiatus.
\end{styleStandard}

\begin{styleStandard}
5.3.3. Preferences for Strong or Weak Adjectives
\end{styleStandard}

\begin{styleStandard}
As discussed above, there is no variation as regards the inflections on the adjectives after \textit{ich} ‘I’, \textit{mich} ‘me(\textsc{acc})’, \textit{du} ‘you(\textsc{nom.sgl})’, \textit{dich} ‘you(\textsc{acc.sgl})’, \textit{uns} ‘us(\textsc{acc/dat})’, \textit{euch} ‘you(\textsc{acc/dat.pl})’. All these adjectives are strong – there are no specific phonetic conditions that induce the phonetic rule from Section 2.2 or facilitate analogy with other definite DPs. In contrast, dative singular \textit{mir} ‘me(\textsc{dat})’ and \textit{dir} ‘you(\textsc{dat.sgl})’, and nominative plural \textit{wir} ‘we’ and \textit{ihr} ‘you(\textsc{nom.pl})’ do involve variation where both strong and weak adjectives are possible. The account of this variation was based on certain phonological conditions. Interestingly, the preference for either strong or weak adjectives is not the same. I propose that this has to do with the underlying features of the larger noun phrase and the specificity of the vocabulary insertion rules of the pronominal determiners.
\end{styleStandard}

\begin{styleStandard}
Recall that elements in DPs including pronominal DPs share concord features in case, number, and gender. As is well known, ordinary DPs involving count nouns require a determiner. In fact, as has been observed for German, it is the determiner that typically spells out the features for case, number, and gender on its inflection.\footnote{\ This is particularly clear with gender, which is inherently specified on the noun but has no overt manifestation on it.} The inflection on the determiner is typically strong. This is different for the cases under discussion here. While first and second-person pronominals also combine with count nouns, they do not have inflections. In fact, they do not have spell-out forms specified for the different genders in the singular, and with the exception of \textit{wir} ‘we’ and \textit{ihr} ‘you(\textsc{nom.pl})’, they do not have spell-out forms specified for the different cases in the plural (Section 5.3.1). I propose that if an adjective is present, it is the latter that preferably surfaces with a strong ending thereby providing a more specified spell-out form of the underlying features of the noun phrase. 
\end{styleStandard}

\begin{styleStandard}
\ \ Table 9 lists the four pronouns that involve varying inflections on their following adjectives. It shows which case, number, or gender the pronominal forms distinguish, and it contains the featural contexts in which their corresponding vocabulary insertion rules apply (Section 5.3.2). Finally, the table repeats the general variation and the varying preferences, pointing out again that strong adjectives are preferred in dative masculine/neuter contexts and weak adjectives in dative feminine and nominative pural environments (the varying specification of [F] in the last column distinguishes the cases in the masculine/neuter and feminine):
\end{styleStandard}

\begin{styleStandard}
Table 9: Specifications of Pronominal Elements and their Relation to the Inflection on the Adjective
\end{styleStandard}

\begin{flushleft}
\begin{tabular}{|m{0.6413598in}|m{0.8212598in}|m{1.1886599in}|m{1.6712599in}|m{1.6712599in}|}

\hline
Pronom.

element &
\multicolumn{2}{m{2.0886598in}|}{\centering Unique form for} &
\centering Featural contexts of vocabulary insert. rules &
\centering\arraybslash Ending on the following adjective\\\hline
 &
\centering Case &
\centering Number/gender &
 &
\\\hline
\textit{mir}, \textit{dir} &
dative &
unspecified (gender) &
[{}-$\gamma $] &
[ -F]: \ \ \ \ \ \ strong/\textsuperscript{\%}weak\\\hline
 &
 &
 &
 &
[+F]: \ \ \ \ \ \ weak/\textsuperscript{\%}strong\\\hline
\textit{wir}, \textit{ihr} &
nominative &
plural &
[+F, +N, +O, +S] &
\ \ \ \ \ \ \ \ \ \ \ \ \ \ \ \textsubscript{\ }weak/\textsuperscript{\%}strong\\\hline
\end{tabular}
\end{flushleft}
\begin{styleStandard}
It is important to state that, while adjectival inflections spell out CNG features (Chapter 2, Section 2.1.5), pronominals of the first and second person do not. As seen in the previous subsection, these pronominals involve a reduced demonstrative structure and are inserted in the context of CNG features. Note though that the vocabulary insertion rules of adjectival inflections and pronominal determiners do have something in common: they are ordered as regards the specificity of insertion such that strong inflections take precedence over weak inflections, and \textit{wir} ‘we’ and \textit{ihr} ‘you(\textsc{nom.pl})’ take precedence over the other relevant pronominal determiners.
\end{styleStandard}

\begin{styleStandard}
Abstracting away from the dative feminine instances for a moment, Table 9 suggests a certain correlation provided in (50a) and (50b). This correlation can be collapsed into the statement in (50c):
\end{styleStandard}

\begin{styleStandard}
(50)\ \ a.\ \ less specified pronominal \~{} preference for more specified (/strong) adjective
\end{styleStandard}

\begin{styleStandard}
\ \ b.\ \ more specified pronominal \~{} preference for less specified (/weak) adjective
\end{styleStandard}

\begin{styleStandard}
c.\ \ DPs prefer one element to be more specified (where the element is of the lexical \ \ \ \ \ \ category determiner or adjective).
\end{styleStandard}

\begin{styleStandard}
Some comments are in order here. In both (50a) and (50b), a more specified and a less specified element combine (cf. again the last two columns of Table 9). Put differently, it appears as if more specified pronominals and strong adjectives have something in common, and less specified pronominals and weak adjectives have something in common – the two groups differ in their degree of specificity and stand in opposition to each other. Note again that this correlation is not about the spell-out of features, but rather it is about the specificity of the vocabulary insertion rules involved. 
\end{styleStandard}

\begin{styleStandard}
Furthermore, the unified, more general correlation in (50c) states a preference, not a requirement. Since the form of the pronominal is determined by the morpho-syntactic and semantic context, the actual choice has to do with the vocabulary insertion rule that yields the preferred inflection on the adjective (NB: the use of the term preference is also compatible with the fact that adjectives are generally optional). Observe that this is different from, for instance, the generalization Weak After Strong, which describes an either/or distribution of the different inflectional forms. Finally, note also that (50c) is not in conflict with Weak After Strong as (50c) involves a weaker statement. With these preliminary remarks in place, consider the three individual cases distinguished in the last column of Table 9.
\end{styleStandard}

\begin{styleStandard}
I start with \textit{wir} ‘we’ and \textit{ihr} ‘you(\textsc{nom.pl})’. These elements are the only relevant first and second-person pronominals whose vocabulary insertion rules involve a fully specified context as regards CNG features (i.e., [+F, +N, +O, +S]). I propose that since all the underlying features of the larger noun phrase are part of the vocabulary insertion rules for \textit{wir} and \textit{ihr}, the less specified weak inflection is preferred. The latter involves -\textit{en}, the vocabulary insertion rule of which is the elsewhere case.
\end{styleStandard}

\begin{styleStandard}
\ \ Singular pronominals of the first and second person are not specified for gender. The vocabulary insertion rules of the dative singular pronominals \textit{mir} ‘me(\textsc{dat})’ and \textit{dir} ‘you(\textsc{dat.sgl})’ are specified as [-$\gamma $]. Note that this involves a low degree of specificity. Starting with the dative masculine/neuter instances, I propose that the inflectable adjective is recruited to indicate the gender of the larger noun phrase. Specifically, since gender is not specified in the vocabulary insertion rules of the relevant pronominals, a more specified inflection on the adjective is preferred. Note in this regard that the vocabulary insertion rule for the strong inflection -\textit{em} is specified as [-F, +O, -S] (see Chapter 2, Section 2.1.5). This explains the preference for the strong adjective in the dative masculine/neuter instances. 
\end{styleStandard}

\begin{styleFootnote}
\ \ Finally, I turn to the feminine singular cases. Given that the dative pronominal in feminine contexts is also specified as [-$\gamma $], the question arises why these instances prefer weak adjectives. I suggest that this involves the same type of explanation that was offered in Section 5.3.1 to account for the very existence of weak adjectives in this context. Again, I follow Duden (1995: 281, 2007: 39) in observing that the feminine instances lead to – what sounds like – the sequence of two strong endings (e.g., \textit{mi-r} \textit{groß-er Gans}). It seems as if speakers tend to avoid similar endings on co-occurring but lexically different elements (i.e., the pronominal element and adjective). Note though, if -\textit{r} on the pronominal element were to be interpreted as a strong inflection, then it would be possible to claim that this element provides the more specified form (observe in this regard that the vocabulary insertion rule of -\textit{er} is specified as [+F, -N, +O, $\alpha $S]). Unlike in the dative masculine/neuter instances, here the “inflected” pronominal element itself would provide the more specified form. This would make these instances similar to \textit{wir} ‘we’ and \textit{ihr} ‘you(\textsc{nom.pl})’ as regards a higher degree of specificity. Indeed, this parallelism would be in line with the general observation that feminine and plural often behave alike.
\end{styleFootnote}

\begin{styleStandard}
\ \ Note that the preference of strong adjectives discussed above is consistent with the instances that do not involve variation. Recall that the pronominal determiners \textit{ich} ‘I’, \textit{mich} ‘me(\textsc{acc})’, \textit{du} ‘you(\textsc{nom.sgl})’, \textit{dich} ‘you(\textsc{acc.sgl})’, \textit{uns} ‘us(\textsc{acc/dat})’, \textit{euch} ‘you(\textsc{acc/dat.pl})’ only take strong adjectives. Importantly, none of these elements involve forms specified for gender in the singular or case in the plural. With that in place, I briefly comment on the different degrees of specificity of the vocabulary insertion rules for adjectival inflections and pronominal elements in the last part of this subsection. This includes a brief discussion of how these vocabulary insertion rules relate to the terminals nodes in the syntactic representation.
\end{styleStandard}

\begin{styleFootnote}
\ \ The different adjectival inflections and pronominal elements are not specified for all the features present in the underlying syntactic structure. Having adopted DM, vocabulary insertion rules involve underspecification, which regulates the order of the application of the rules. In other words, the absence of features in the vocabulary insertion rules has to do with the very assumptions of how Vocabulary Insertion works in DM. Indeed, the varying numbers of features in rules lead to different degrees of specificity of those rules. For instance, the strong inflection -\textit{er} has more features and is thus more specified than the strong inflection -\textit{em}, and both of these are more specified than the weak inflection -\textit{en}, the elsewhere case. Similar considerations hold for the contexts of vocabulary insertion of pronominal determiners. 
\end{styleFootnote}

\begin{styleFootnote}
\ \ Returning to the correlations in (50), note that the more specified elements involve three features (e.g., -\textit{em}: [-F, +O, -S]) or four features (e.g., \textit{wir}: [+F, +N, +O, +S]). By contrast, the less specified elements contain no feature (e.g., -\textit{en}: []) or one feature (e.g., \textit{mir}: [-$\gamma $]). Comparing adjectival inflections to each other and pronominal forms to each other, they each involve a difference of three CNG features. This constitutes different degrees of specificity that find overt manifestation in the varying preferences for the different inflections on the adjectives. 
\end{styleFootnote}

\begin{styleFootnote}
\ \ Finally, note that the specification of the vocabulary insertion rules differs from that of the terminal nodes in the syntactic representation. Provided that the relevant terminal nodes are present, they involve all CNG features. In other words, all CNG features are present in the abstract representation, but only a subset of those features are specified in the vocabulary insertion rules to be applied. Given the discussion above, it appears then as if German prefers one more and one less specified element to co-occur, provided such a choice is available to begin with (i.e., Impoverishment does not occur). I summarize the entire section.
\end{styleFootnote}

\begin{styleStandard}\itshape
5.4.\ \ Summary of the Discussion
\end{styleStandard}

\begin{styleFootnote}
All pronominal DPs involve canonical DPs; that is, pronominal determiners take adjectives and nouns as part of their complement structure. If modifiers are present, all pronominal DPs may exhibit strong adjectives. This is explained by the reduced demonstrative structure and the feature specifications involving [±AUTH] and [±PART]. With CNG, definiteness, and deixis features absent on the pronominal determiners, Impoverishment cannot occur resulting in the strong adjectives. Besides strong adjectives, some instances may also involve weak adjectives. The weak adjectives in the dative masculine/neuter are part of the nasal alternation and can be accounted for by the phonetic rule from Section 2.2. The weak adjectives in the dative feminine and nominative plural follow from analogy with ordinary definite DPs (constrained by the similarity of the inflections on the ordinary determiners and the final sounds of the pronominal determiners). In other words, all the weak adjectives surface under specific phonetic conditions only. 
\end{styleFootnote}

\begin{styleFootnote}
\ \ The variation in the dative singular and in the nominative plural shows varying preferences as regards strong or weak inflections on adjectives. This preference is a reflex of the different degrees of specification of two co-occurring elements, the pronominal determiner and the inflection on the following adjective. The preference for strong adjectives in the dative masculine/neuter is due to the preference for more specified forms provided by the adjectival inflection. In contrast, the preference for weak adjectives in the dative feminine and in the nominative plural is also due to the preference for more specified forms, but here provided by the pronominal determiners themselves. As the above discussion has made clear again, masculine and neuter pattern in opposition to feminine and plural. In the next section, I turn to variation in a non-canonical structure. Like in Sections 3 and 4, it is lexically restricted. 
\end{styleFootnote}

\begin{styleStandard}
\textbf{6. \ \ Non-canonical Structures with Optional Inflections on Predeterminer }\textbf{\textit{alle}}
\end{styleStandard}

\begin{styleStandard}
In this section, I discuss the presence and absence of the strong endings on the predeterminer \textit{alle} ‘all’ in the plural. Recall from Chapter 1, Section 4.1.2 that a predeterminer is a type of determiner-like element that intensifies the meaning of the following DP in a certain way. 
\end{styleStandard}

\begin{styleStandard}
As is well known, both uninflected \textit{all} and inflected \textit{alle} can appear before elements such as possessive determiners (51) or demonstratives (52). Interestingly, there is a restriction such that the co-occurrence of inflected \textit{alle} and a definite article results in slight degradedness (53b). Note that in this section, I gloss determiners like \textit{die} ‘the/those’ differently. Recall that they are pronounced as [di(:)]; that is, they are monosyllabic. This will become important below.
\end{styleStandard}

\begin{styleStandard}
(51)\ \ a.\ \ \textit{all mein-e Freunde}
\end{styleStandard}

\begin{styleStandard}
\ \ \ \ all my-\textsc{st} \ friends
\end{styleStandard}

\begin{styleStandard}
\ \ \ \ ‘all my friends’
\end{styleStandard}

\begin{styleStandard}
\ \ b.\ \ \textit{all-e \ \ mein-e Freunde}
\end{styleStandard}

\begin{styleStandard}
\ \ \ \ all-\textsc{st} my-\textsc{st} \ friends
\end{styleStandard}

\begin{styleStandard}
\ \ \ \ ‘all my friends’
\end{styleStandard}

\begin{styleStandard}
(52)\ \ a.\ \ \textit{all dies-e \ \ \ Leute}
\end{styleStandard}

\begin{styleStandard}
\ \ \ \ all these-\textsc{st} people
\end{styleStandard}

\begin{styleStandard}
\ \ \ \ ‘all these people’
\end{styleStandard}

\begin{styleStandard}
\ \ b.\ \ \textit{all-e \ \ dies-e \ \ \ \ Leute}
\end{styleStandard}

\begin{styleStandard}
\ \ \ \ all-\textsc{st} these-\textsc{st} people
\end{styleStandard}

\begin{styleStandard}
\ \ \ \ ‘all these people’
\end{styleStandard}

\begin{styleStandard}
(53)\ \ a.\ \ \textit{all die \ \ \ \ \ \ \ \ Leute}
\end{styleStandard}

\begin{styleStandard}
\ \ \ \ all the.\textsc{nom} people
\end{styleStandard}

\begin{styleStandard}
\ \ \ \ ‘all the people’
\end{styleStandard}

\begin{styleStandard}
\ \ b. \ ?\ \ \textit{all-e \ \ die \ \ \ \ \ \ \ \ \ Leute}
\end{styleStandard}

\begin{styleStandard}
\ \ \ \ all-\textsc{st} the.\textsc{nom} people
\end{styleStandard}

\begin{styleStandard}
\ \ \ \ ‘all the people’
\end{styleStandard}

\begin{styleStandard}
Before proceeding, a remark on these data is necessary.
\end{styleStandard}

\begin{styleFootnote}
Cirillo (2016: 197-200) states that inflected \textit{alle} in the (b)-examples above is marked or even ungrammatical for some speakers; that is, uninflected \textit{all} is the dominant or correct form. While I agree that (53b) is indeed marked, (51b) and (52b) are completely fine for me. An informal \textit{google}{}-search has revealed that uninflected \textit{all} is actually less frequent than inflected \textit{alle}. The results are provided in Table 10a (M stands for million).
\end{styleFootnote}

\begin{styleStandard}
Table 10a: Numeric Results of \textit{all(e)} Identified by \textit{Google} (ca. 2020)
\end{styleStandard}

\begin{flushleft}
\begin{tabular}{|m{2.13796in}|m{2.13796in}|m{2.13796in}|}

\hline
 &
\centering{\itshape all} &
\centering\arraybslash{\itshape alle}\\\hline
{\itshape meine} &
\centering 1.4M &
\centering\arraybslash 8.8M\\\hline
{\itshape diese} &
\centering 16.5M &
\centering\arraybslash 21.2M\\\hline
\end{tabular}
\end{flushleft}
\begin{styleStandard}
While I checked many instances for relevance, these numbers are too large to check every single example. Note though that while there are some false positives, it is clear that inflected \textit{alle} is fine for many speakers. A reviewer points out that the inflected forms may be more frequent in written German. 
\end{styleStandard}

\begin{styleStandard}
\ \ I also conducted a search in the DGD, specifically, the corpus of spoken German called \textit{Forschungs- und Lehrkorpus Gesprochenes Deutsch} (FOLK) (Research and Teaching Corpus of Spoken German), and found similar results. Consider Table 10b.
\end{styleStandard}

\begin{styleStandard}
Table 10b: Numeric Results of \textit{all(e) }Identified in FOLK (October 31, 2024)
\end{styleStandard}

\begin{flushleft}
\begin{tabular}{|m{2.13796in}|m{2.13796in}|m{2.13796in}|}

\hline
 &
\centering{\itshape all} &
\centering\arraybslash{\itshape alle}\\\hline
{\itshape meine} &
\centering 1 &
\centering\arraybslash 9\\\hline
{\itshape diese} &
\centering 9 &
\centering\arraybslash 12\\\hline
\end{tabular}
\end{flushleft}
\begin{styleStandard}
These numbers are much smaller. Note that every example was checked for relevance, and a few examples were removed – either they were not relevant or their status was somewhat unclear. Observe though that the relative frequencies of the uninflected and inflected forms of \textit{all}{}- here are the same as in Table 10a above. In other words, inflected \textit{alle} seems to be more frequent than uninflected \textit{all}, in both written and spoken contexts. As such, I point out that Cirillo’s (2016) description of the data is not entirely accurate. Given the data above, there are two questions that arise here. First, why is the inflection on \textit{all}{}- optional in (51) and (52)? Second, why is the inflection in (53b) dispreferred?
\end{styleStandard}

\begin{styleStandard}
\ \ Starting with the first question, I proposed in Chapter 2, Section 4 that \textit{alle} ‘all’ in the cases above is located in LPP. Not being part of the DP proper at any point, the predeterminer \textit{alle} can neither trigger nor undergo Impoverishment explaining the strong ending on \textit{alle} itself and on the following determiner:
\end{styleStandard}

\begin{styleStandard}
(54)\ \ \ \ LPP
\end{styleStandard}

\begin{styleStandard}
[Warning: Draw object ignored][Warning: Draw object ignored]
\end{styleStandard}

\begin{styleStandard}
\textit{alle}\ \ \ \ DP
\end{styleStandard}

\begin{styleStandard}
[Warning: Draw object ignored][Warning: Draw object ignored]\ \ \ \ \ \ \ \ \ \ \ 
\end{styleStandard}

\begin{styleStandard}
\ \ \ \ D\ \ \ \ NP
\end{styleStandard}

\begin{styleStandard}\itshape
\ \  \ \ \ \ \ \ \ meine\ \  \ \ \ \ \ \ \ \ \ Leute
\end{styleStandard}

\begin{styleStandard}
Pafel (1994: 264-66) makes a similar syntactic proposal for inflected \textit{alle} such that this predeterminer selects a DP complement.\footnote{\ Bhatt (1990: 217-18) claims that \textit{alle}, \textit{diese}, and \textit{meine} in \textit{alle diese meine Freunde} ‘all these friends of mine’ are heads. She proposes that \textit{alle} is adjoined to \textit{diese} and the resulting complex is adjoined to \textit{meine}. The entire complex is inside Spec,DP. Note that it is not a standard assumption that heads reside in phrasal positions.} Several options have been proposed to account for the absence of the inflection on \textit{all}{}-. There are two basic types. 
\end{styleStandard}

\begin{styleStandard}
\ \ First, a number of authors have proposed that uninflected \textit{all} is in a different position. For instance, Pafel (1994) proposes that \textit{all} is adjoined to the determiner in D. Bošković (2004: 701) argues that while inflected \textit{alle} is either left-adjoined to DP or in a higher position, uninflected \textit{all} is in that higher position (left unspecified in Bošković, see his footnote 23). Given the discussion above, this higher position can be related to the LPP-layer. 
\end{styleStandard}

\begin{styleStandard}
The second type of proposal involves the claim that inflected and uninflected \textit{all}{}- start out in the same position but that uninflected \textit{all} undergoes some further operation. To name just two ideas, we could build on Pafel suggesting that uninflected \textit{all} is a deficient element that starts out in LPP but undergoes adjunction to D late in the derivation. Alternatively, we could formulate an Impoverishment rule that deletes the inflection optionally (cf. Footnote Error: Reference source not found). Since the account of the inflection on \textit{all}{}- has repurcussions for the analysis of floating quantifiers (where \textit{all}{}- must have an inflection), I will not pursue this question further here and turn to the second query.\footnote{\ Notice also that it is sometimes claimed that uninflected \textit{all} and inflected \textit{alle} involve different semantics (see Merchant 1996: 183 and Kobele \& Zimmermann 2012: 249). For some brief discussion, see Chapter 8.}
\end{styleStandard}

\begin{styleStandard}
\ \ As far as I am aware, there is no explanation of the contrast in (53), neither in Pafel’s nor in any other work. Following an idea from Björn Köhnlein (p.c.), I make a tentative suggestion at the end of this section. To begin, note that the contrast above is found in the other morphological cases as well, here illustrated with the dative:\footnote{\ Like \textit{all}{}-, the demonstrative \textit{dies}{}- ‘this/these’ can also do double duty as a determiner and as a predeterminer. Unlike \textit{all}{}-, the inflection of \textit{dies}{}- can only be left out in the nominative/accusative neuter yielding \textit{dieses} vs. \textit{dies} (see Chapter 4, Section 5).}
\end{styleStandard}

\begin{styleStandard}
(55)\ \ a.\ \ \textit{mit \ \ all den \ \ \ \ \ \ Leuten}
\end{styleStandard}

\begin{styleStandard}
\ \ \ \ with all the.\textsc{dat} people
\end{styleStandard}

\begin{styleStandard}
\ \ \ \ ‘with all the people’
\end{styleStandard}

\begin{styleStandard}
\ \ b. \ ?\ \ \textit{mit \ \ all-en den \ \ \ \ \ \ \ Leuten}
\end{styleStandard}

\begin{styleStandard}
\ \ \ \ with all-\textsc{st} the.\textsc{dat} people
\end{styleStandard}

\begin{styleStandard}
\ \ \ \ ‘with all the people’
\end{styleStandard}

\begin{styleStandard}
If the (lower) DP is replaced by a disyllabic pronoun, the counterparts of (55a) and (55b) are both fine:
\end{styleStandard}

\begin{styleStandard}
(56)\ \ a.\ \ \textit{mit \ \ all denen}
\end{styleStandard}

\begin{styleStandard}
\ \ \ \ with all those
\end{styleStandard}

\begin{styleStandard}
\ \ \ \ ‘with all those’
\end{styleStandard}

\begin{styleStandard}
\ \ b.\ \ \textit{mit \ \ all-en denen}
\end{styleStandard}

\begin{styleStandard}
\ \ \ \ with all-\textsc{st} those
\end{styleStandard}

\begin{styleStandard}
\ \ \ \ ‘with all those’
\end{styleStandard}

\begin{styleStandard}
The obvious difference between these plural elements is that possessive determiners and demonstratives are disyllabic but that definite articles are monosyllabic. Assuming that LPP constitutes the left periphery of the noun phrase, I formulate a preliminary version of the relevant generalization:
\end{styleStandard}

\begin{styleStandard}
(57)\ \ \textit{Generalization (preliminary version):}
\end{styleStandard}

\begin{styleStandard}
A disyllabic determiner-like element in the left periphery cannot be followed by a monosyllabic determiner.
\end{styleStandard}

\begin{styleStandard}
Note that the generalization is silent about sequences starting with a monosyllabic element; that is, it says nothing about the grammatical (a)-examples above. 
\end{styleStandard}

\begin{styleStandard}
At first glance, it might be suggested that German prefers trochaic feet, which yield a certain phonotactic rhythm in the pronunciation. This suggestion seems to get confirmation when three determiner elements are present (58a). However, we have already seen above that two monosyllabic elements do co-occur (\textit{all die }‘all the’). Furthermore, as pointed out to me by David Fertig (p.c.), the string in (53b) above becomes better with stress on \textit{die }(58b); that is, when \textit{die} functions as a demonstrative. Note that cases as in (58b) are often followed by a relative clause (also Pafel 1994: 238):
\end{styleStandard}

\begin{styleStandard}
(58)\ \ a.\ \ \textit{all-e \ \ dies-e \ \ \ \ mein-e Freunde}
\end{styleStandard}

\begin{styleStandard}
\ \ \ \ all-\textsc{st} these-\textsc{st} my-\textsc{st} \ friends
\end{styleStandard}

\begin{styleStandard}
\ \ \ \ ‘all these friends of mine’
\end{styleStandard}

\begin{styleStandard}
b.\ \ \textit{all-e \ \ DIE \ \ \ \ \ \ \ \ \ \ Leute (, die \ \ in einer Stadt wohnen)}
\end{styleStandard}

\begin{styleStandard}
\ \ \ \ all-\textsc{st} those.\textsc{nom} people \ who in a \ \ \ \ \ \ \ city \ \ live
\end{styleStandard}

\begin{styleStandard}
‘all those people (who live in a city)’
\end{styleStandard}

\begin{styleStandard}
While \textit{die} is stressed now, it is still a monosyllabic element. I interpret multiple syllabicity and stress as related measures of heaviness. In other words, I assume that multiple syllables in an element, on the one hand, or stress on an element, on the other hand, leads to heaviness of a word. This groups disyllabic and stressed monosyllabic words together in opposition to unstressed monosyllabic words. With this in mind, I finalize the generalization as follows:
\end{styleStandard}

\begin{styleStandard}
(59)\ \ \textit{Generalization (final version):}
\end{styleStandard}

\begin{styleStandard}
A multisyllabic determiner-like element in the left periphery cannot be followed by a lighter determiner.
\end{styleStandard}

\begin{styleStandard}
Put differently, the relevant elements must be of the same weight or the relevant elements must increase (but not decrease) in weight from left to right. The question arises as to how to deal with this generalization.
\end{styleStandard}

\begin{styleStandard}
\ \ Björn Köhnlein (p.c.) makes the interesting suggestion that the generalization above actually involves an epiphenomenon of the fact that articles are clitics. In more detail, phonology tends to avoid lapses, that is, sequences of two adjacent unstressed syllables in the same phonological word. Assuming that definite articles are clitics, the phonological word [\textit{all-die}] consists of a sequence of a stressed and one unstressed syllable. This is fine in the phonology. In contrast, the phonological word [\textit{alle-die}] consists of a sequence of a stressed and two adjacent unstressed syllables, something that is marked in the phonology (yielding the slight degree of grammatical markedness). Unlike articles, demonstratives and possessive determiners are not clitics, and they do not form a phonological word with \textit{all(e)}. Consequently, those sequences are fine, with or without the inflection on \textit{all}{}-. Similarly, when \textit{die} is stressed, it is no longer a clitic and forms a different phonological word.
\end{styleStandard}

\begin{styleStandard}
\ \ In the last couple of sections, I discussed variation of adjectival inflections in noun phrases in Standard German. With this in mind, I return to the traditional generalizations Weak After Strong and the Principle of Monoinflection.
\end{styleStandard}

\begin{styleFootnote}\bfseries
7.\ \ Revisiting Two Traditional Generalizations of Adjectival Inflections
\end{styleFootnote}

\begin{styleFootnote}
In this section, I return to the two traditional generalizations that describe the strong/weak alternation in German. To recapitulate, compare the canonical cases in (60), where the adjective is weak provided it is preceded by a determiner with a strong inflection (60a), but the adjective is strong in all other cases (60b-c): 
\end{styleFootnote}

\begin{styleFootnote}
(60)\ \ a.\ \ \textit{d-er \ \ \ gut-e \ \ \ \ \ \ \ Kaffee}
\end{styleFootnote}

\begin{styleFootnote}
\ \ \ \ the-\textsc{st} good-\textsc{wk} coffee.\textsc{masc}
\end{styleFootnote}

\begin{styleFootnote}
\ \ \ \ ‘the good coffee’
\end{styleFootnote}

\begin{styleFootnote}
\ \ b.\ \ \textit{ein gut-er \ \ \ Kaffee}
\end{styleFootnote}

\begin{styleFootnote}
\ \ \ \ a \ \ \ good-\textsc{st} coffee.\textsc{masc}
\end{styleFootnote}

\begin{styleFootnote}
\ \ \ \ ‘a good coffee’
\end{styleFootnote}

\begin{styleFootnote}
\ \ c.\ \ \textit{gut-er \ \ \ Kaffee}
\end{styleFootnote}

\begin{styleFootnote}
\ \ \ \ good-\textsc{st} coffee.\textsc{masc}
\end{styleFootnote}

\begin{styleFootnote}
\ \ \ \ ‘good coffee’
\end{styleFootnote}

\begin{styleFootnote}
This inflectional distribution can be summarized in the generalization in (61), repeated from Chapter 2, Section 2.1.1 (again, for similar statements, see Bierwisch 1967: 257, Eisenberg 1998: 171-73, Gallmann 1998: 144, Giusti 2015: 207, G. Müller 2002a: 129, Petrova 2024: 184-85, Pfaff 2017: 286, Rehn 2019: 58, Sauerland 1996: 34, Schoorlemmer 2009: 53, and many others):
\end{styleFootnote}

\begin{styleFootnote}
(61)\ \ \textit{Weak After Strong}
\end{styleFootnote}

\begin{styleFootnote}
An adjective with a weak inflection is preceded by a determiner with a strong inflection. 
\end{styleFootnote}

\begin{styleFootnote}
Note again that this formulation does not explicitly mention adjectives that are not preceded by (inflected) determiners. However, since prenominal adjectives alternate only between weak and strong, the distribution of strong adjectives can be inferred from (61).
\end{styleFootnote}

\begin{styleFootnote}
\ \ It is clear that reference to lexical categories like adjective vs. determiner or determiner-like element is essential to the proper workings of this generalization. As documented above, inflections on elements of the same lexical category are the same. Specifically, independently of the presence or absence of \textit{ein}, an adjective that follows a strong adjective must also be strong (62a). As for determiners and determiner-like elements, they typically exhibit strong inflections. In fact, as long as the determiner(-like) elements appear in the order in (62b), any combination of them yields a grammatical string:
\end{styleFootnote}

\begin{styleFootnote}
(62)\ \ a.\ \ \textit{(ein) gut-er \ \ \ frisch-er Kaffee}
\end{styleFootnote}

\begin{styleFootnote}
\ \ \ \  a \ \ \ \ good-\textsc{st} fresh-\textsc{st} \ coffee.\textsc{masc}
\end{styleFootnote}

\begin{styleFootnote}
\ \ \ \ ‘(a) good fresh coffee’
\end{styleFootnote}

\begin{styleFootnote}
\ \ b.\ \ \textit{(all-e) (dies-e) \ (mein-e) Freunde}
\end{styleFootnote}

\begin{styleFootnote}
\ \ \ \  all-\textsc{st }\ these-\textsc{st} my-\textsc{st} \ \ friends.\textsc{masc}
\end{styleFootnote}

\begin{styleFootnote}
\ \ \ \ ‘all these friends of mine’
\end{styleFootnote}

\begin{styleFootnote}
To be clear, the generalization Weak After Strong must take the different lexical categories of the elements involved into consideration. Recall that determiners and determiner-like elements have the categorial feature [+D], but adjectives do not. This lexical distinction prevents us from expecting the second adjective in (62a) and the non-initial determiner(-like) elements in (62b) to exhibit weak inflections.
\end{styleFootnote}

\begin{styleFootnote}
\ \ Note that this generalization is surface-oriented making reference to the notion of precedence. While I already pointed out in Chapter 2, Section 2.1.1 that this generalization is, in and of itself, not an explanation, there are also two basic types of exceptions: (i) strong adjectives can be preceded by determiners with strong inflections; (ii) weak adjectives can occur without determiners with strong inflections preceding them.
\end{styleFootnote}

\begin{styleFootnote}
\ \ The first type of exception manifests itself when complex proper names are embedded in larger nominals. In these cases, strong adjectives that are part of proper names are preceded by the determiners of the larger nominals, where the determiners also have strong inflections:
\end{styleFootnote}

\begin{styleFootnote}
(63)\ \ a.\ \ \textit{d-em} [ \textit{Deutsch-e \ Bahn} ] \ \ \ \ \ \ \ \ \ \ \textit{Personal}
\end{styleFootnote}

\begin{styleFootnote}
\ \ \ \ the-\textsc{st} German-\textsc{st} Railroad.\textsc{fem} personnel.\textsc{neut}
\end{styleFootnote}

\begin{styleFootnote}
\ \ \ \ ‘(to) the German-Railroad personnel’
\end{styleFootnote}

\begin{styleFootnote}
\ \ b.\ \ \textit{d-es} \ [ \textit{Deutsch-e \ \ Bank} ] \ \ \ \ \ \textit{Chef-s}
\end{styleFootnote}

\begin{styleFootnote}
\ \ \ \ the-\textsc{st} German-\textsc{st} Bank.\textsc{fem} boss.\textsc{masc}{}-\textsc{gen}
\end{styleFootnote}

\begin{styleFootnote}
\ \ \ \ ‘(of) the German-Bank boss’
\end{styleFootnote}

\begin{styleFootnote}
It is clear that any generalization (or proposal) of the strong/weak alternation must take structure into account. I argued above that the determiners are in nominals different from those of the complex proper names in (63). As the determiners do not adjoin to any parts of the AgrPs containing the adjectives, Impoverishment does not occur. This explains the strong adjectives in (63).
\end{styleFootnote}

\begin{styleFootnote}
\ \ As to the second type of exception, weak adjectives routinely occur in the genitive masculine/neuter without any determiners being present (64a). Furthermore, they also occur in the presence of Saxon Genitives, items often taken to be determiner(-like) elements (64b). Observe that – what looks like – strong inflections (i.e., -\textit{s}) are present in both (64a) and (64b). Crucially though, these inflections are on elements that follow (but not precede) the relevant adjectives – they are on the head nouns. Third, weak adjectives can be preceded by pronominal determiners, which do not have inflections in the first and second person (64c). Finally, weak adjectives can also be preceded by uninflected \textit{dies} ‘this’ (64d):\footnote{\ I argue in detail in Chapter 4, Section 5 that uninflected \textit{dies} is not based on inflected \textit{dieses} such that the ending -\textit{es} has been deleted. This is different for \textit{die} ‘the/that/those’. Recall from Section 3 that schwa has been deleted here. Note that, if the generalization above only applies to surface strings, then nominals involving \textit{die} present another exception to it.}
\end{styleFootnote}

\begin{styleFootnote}
(64)\ \ a.\ \ \textit{trotz \ \ \ \ heiß-en Kaffee-s}
\end{styleFootnote}

\begin{styleFootnote}
\ \ \ \ despite hot-\textsc{wk} \ coffee.\textsc{masc}{}-\textsc{gen}
\end{styleFootnote}

\begin{styleFootnote}
\ \ \ \ ‘despite hot coffee’
\end{styleFootnote}

\begin{styleFootnote}
\ \ b.\ \ \textit{trotz \ \ \ \ Peters \ heiß-en Kaffee-s}
\end{styleFootnote}

\begin{styleFootnote}
\ \ \ \ despite Peter’s hot-\textsc{wk} \ coffee.\textsc{masc}{}-\textsc{gen}
\end{styleFootnote}

\begin{styleFootnote}
\ \ \ \ ‘despite Peter’s hot coffee’
\end{styleFootnote}

\begin{styleFootnote}
\ \ c.\ \ \textit{wir nett-en \ Studenten}
\end{styleFootnote}

\begin{styleFootnote}
\ \ \ \ we nice-\textsc{wk} students
\end{styleFootnote}

\begin{styleFootnote}
\ \ \ \ ‘we nice students’
\end{styleFootnote}

\begin{styleFootnote}
\ \ d.\ \ \textit{dies schön-e \ \ \ Kleid}
\end{styleFootnote}

\begin{styleFootnote}
\ \ \ \ this pretty-\textsc{wk} dress.\textsc{neut}
\end{styleFootnote}

\begin{styleFootnote}
\ \ \ \ ‘this pretty dress’
\end{styleFootnote}

\begin{styleFootnote}
The cases in (64) involve canonical structures. Above, I argued that Impoverishment Rule 2 is at work in (64a-b). Turning to (64c), the weak adjective is due to analogy with related definite DPs. As for (64d), I argued that it is the feature [+D] on the determiner stem that triggers Impoverishment on the adjective, not the inflection on the determiner – as is clear from (64d), such an inflection can be absent and yet, the adjective is weak. There are other issues with this generalization.
\end{styleFootnote}

\begin{styleFootnote}
\ \ Strictly speaking, Weak After Strong is a generalization about inflections on adjectives. More precisely, it describes the dependency of the inflections on the adjectives with regard to the inflections on determiners. The inflectional behavior of determiners or determiner-like elements themselves is not covered. Since inflections on determiners, determiner-like elements, and adjectives are identical, it is desirable to put forth a generalization that captures the distribution of the inflections on all these elements.
\end{styleFootnote}

\begin{styleFootnote}
\ \ A more general statement than the first generalization has come to be known as the Principle of Monoinflection, also repeated here from Chapter 2, Section 2.1.1 (see also, e.g., Darski 1979, Esau 1973, Helbig \& Buscha 2001: 274-76, Murphy 2018: 344, Roehrs 2009a: 135, Wegener 1995: 105, 153, cf. Nübling 2011: 178):\footnote{\ Note that a similar principle is also mentioned in the Duden reference grammar (\textit{Monoflexion}), but with a slightly different formulation, where the strong ending does not necessarily precede the weak one (e.g., Duden 2016: 954).}
\end{styleFootnote}

\begin{styleStandard}
(65)\textit{ \ \ Principle of Monoinflection}
\end{styleStandard}

\begin{styleFootnote}
The first inflected element within a noun phrase carries the strong and the second one the \ \ weak ending.
\end{styleFootnote}

\begin{styleFootnote}
As with the first generalization, reference to lexical categories must be made. In other words, the first element of a nominal string is taken to be of a different lexical category than the second element. Given that determiner(-like) elements and adjectives can each be stacked, the notion of element (of a certain lexical category) must be understood as group of elements (of the same category). To briefly illustrate, given that (66) involves determiner(-like) elements, they are all considered to be one type of element as regards the relevant lexical category. As such, they are all taken to be first in the relevant sense, and the generalization correctly describes the fact that all of them have a strong ending:
\end{styleFootnote}

\begin{styleFootnote}
(66)\ \ \textit{(all-e) (dies-e) \ (mein-e) Freunde}
\end{styleFootnote}

\begin{styleFootnote}
\ \  all-\textsc{st }\ these-\textsc{st} my-\textsc{st} \ \ friends
\end{styleFootnote}

\begin{styleFootnote}
\ \ ‘all these friends of mine’
\end{styleFootnote}

\begin{styleFootnote}
By contrast, if determiner(-like) elements are absent, then adjectives are the first inflected element. As expected under (65), they appear with strong inflections (cf. (60c) above).
\end{styleFootnote}

\begin{styleFootnote}
While this generalization covers more empirical ground, it faces the same general challenges as Weak After Strong. Besides the issues already mentioned above, a determiner-like element with a strong ending may precede an adjective with a strong ending (67a).\footnote{\ Note that this case cannot simply be discounted by stating that the adjective is not immediately preceded by \textit{dieses }‘this’. This is so because weak adjectives can be preceded by determiners with a strong inflection even if an uninflected element intervenes:\par \ \ (i)\ \ \textit{dies-e \ \ \ \ zehn klein-en \ \ \ Autos}\par \ \ \ \ these-\textsc{st} ten \ \ \ small-\textsc{wk} cars\par \ \ \ \ ‘these ten small cars’\par Again, a surface-oriented generalization involving precedence cannot capture cases like (67a) and others to be discussed momentarily.} Similarly, taking indefinite pronouns to be determiner-like elements (Section 2.2), indefinite pronoun constructions present a similar problem (67b):
\end{styleFootnote}

\begin{styleStandard}
(67)\ \ a. \ \ \ \textit{dies-es mein groß-es \ \ Glück}\ \ 
\end{styleStandard}

\begin{styleStandard}
\ \ \ \ this-\textsc{st} my \ \ \ great-\textsc{st} \ happiness.\textsc{neut}
\end{styleStandard}

\begin{styleStandard}
\ \ \ \ ‘this great happiness of mine’
\end{styleStandard}

\begin{styleStandard}
b. \ \%\ \ \textit{jemand-em} \ \ \ \ \ \ \ \ \ \ \ \ \textit{ander}{}-\textit{em}
\end{styleStandard}

\begin{styleStandard}
\ \ \ \ someone.\textsc{masc}{}-\textsc{st} different-\textsc{st}
\end{styleStandard}

\begin{styleStandard}
\ \ \ \ ‘someone different’
\end{styleStandard}

\begin{styleFootnote}
In Chapter 2, I argued that the demonstrative in (67a) is base-generated in LPP. As such, it can neither trigger nor undergo Impoverishment. As to (67b), I formulated in Section 2.2 three different paradigms for \textit{jemand} ‘someone’ and an optional phonetic rule, which has not applied to the inflection on the adjective in (67b). 
\end{styleFootnote}

\begin{styleFootnote}
\ \ The second type of exception concerns a weak ending on the first inflected element. This may hold with determiners (68a), indefinite pronouns (68b), and adjectives (68c):
\end{styleFootnote}

\begin{styleStandard}
(68)\ \ a.\ \ \textit{im \ \ \ \ \ Sommer dies-en Jahr-es}
\end{styleStandard}

\begin{styleStandard}
\ \ \ \ in.the summer this-\textsc{wk} year.\textsc{neut}{}-\textsc{gen}
\end{styleStandard}

\begin{styleStandard}
\ \ \ \ ‘in the summer of this year’
\end{styleStandard}

\begin{styleStandard}
b. \ \%\ \ \textit{jemand-en} \ \ \ \ \ \ \ \ \ \ \ \ \ \ \textit{ander}{}-\textit{en}
\end{styleStandard}

\begin{styleStandard}
\ \ \ \ someone.\textsc{masc}{}-\textsc{wk} different-\textsc{wk}
\end{styleStandard}

\begin{styleStandard}
\ \ \ \ ‘someone different’
\end{styleStandard}

\begin{styleStandard}
c. \ \%\ \ \textit{jemand} \ \ \ \ \ \ \ \ \ \ \ \ \ \textit{ander}{}-\textit{en}
\end{styleStandard}

\begin{styleStandard}
\ \ \ \ someone.\textsc{masc} different-\textsc{wk}
\end{styleStandard}

\begin{styleStandard}
\ \ \ \ ‘someone different’
\end{styleStandard}

\begin{styleStandard}
While the occurrence of these strings is admittedly restricted, these distributions do exist. In Section 4, I argued that determiners can optionally undergo Impoverishment in genitive masculine/neuter contexts (68a). In Section 2.2, I provided three paradigms for \textit{jemand} ‘someone’ (68b-c) and an optional phonetic rule, which has applied to the inflections on the adjectives in (68b-c). Finally, notice that both generalizations are silent about prenominal adjectives that can appear without inflections (e.g., \textit{das lila/prima/Berliner Auto} ‘the purple/great/Berlin car’). 
\end{styleStandard}

\begin{styleStandard}
Given these issues, I propose the following generalization. Note that the generalization below is not based on precedence but on structure, where canonical DPs play a crucial role:
\end{styleStandard}

\begin{styleStandard}
(69)\ \ \textit{Strong/Weak Alternation in German}\ \ 
\end{styleStandard}

\begin{styleStandard}
Disregarding a few (mostly) lexical exceptions, both adjectives and determiners have \ \ strong or weak inflections. Weak inflections only occur in canonical DPs, either 
\end{styleStandard}

\begin{styleStandard}
(a) \ in the context of certain lexical items, or
\end{styleStandard}

\begin{styleStandard}
(b) \ in certain featural contexts, or 
\end{styleStandard}

\begin{styleStandard}
(c) \ in a combination of (a) and (b).
\end{styleStandard}

\begin{styleStandard}
Strong inflections occur in all other environments.
\end{styleStandard}

\begin{styleStandard}
To briefly illustrate, (69a) refers to adjectives following \textit{der}{}-words, (69b) involves adjectives and determiners in genitive masculine/neuter contexts, and (69c) is instantiated by adjectives following \textit{ein}{}-words that occur in the feminine, plural, and in the two oblique cases. Furthermore, (69c) is also manifested by adjectives in dative masculine/neuter contexts that are preceded by nominal elements, and by adjectives in the dative singular and in the nominative plural preceded by pronominal determiners. These are the main contexts where weak inflections occur.
\end{styleStandard}

\begin{styleFootnote}
\ \ To sum up, the generalization Weak After Strong is a good initial description of the canonical facts, but it does not capture the whole spectrum. The Principle of Monoinflection is more general but faces the same general issues. Many (but not all) of the issues with these generalizations reveal themselves when we consider non-canonical constructions. An analysis that seeks to be more comprehensive has to include the latter cases as well. If so, the different structures involved must be part of the account of the strong/weak alternation. As mentioned before, I argue that the non-canonical constructions reveal the true nature of adjectival inflections.
\end{styleFootnote}

\begin{styleFootnote}
Thus far, I have discussed variation (and generalization) in the same dialect – colloquial Standard German. Next, I turn to variation of a different kind by discussing a regional variety. I chose Mannheim German for two reasons. First, the account developed above can, without adding too many details, be extended to this dialect showcasing its more general applicability. Second and more importantly, the discussion of this regional dialect will reiterate the special status of \textit{ein}{}-words in the four exceptional cases, delineated in Chapter 2, Section 2.2.2 as masculine/neuter in the nominative and accusative.
\end{styleFootnote}

\begin{styleFootnote}\bfseries
8.\ \ Dialectal Variation: Mannheim German
\end{styleFootnote}

\begin{styleFootnote}
In his critique of Bierwisch (1967), Blevins (1995), Wunderlich (1997), and B. Wiese (1999), G. Müller (2002b) discusses a regional variety of German spoken in and around Mannheim, \textit{Mannheimer Regionalsprache} ‘Mannheim regional dialect’, a Standard German dialect that is influenced by the local palatinate variety. For convenience, I refer to this dialect as Mannheim German. G. Müller observes that regular nominative masculine forms are used in accusative masculine contexts (data from G. Müller 2002b: 354):\footnote{\ G. Müller (2002b: 353-55) provides more details about this locally colored Standard German variety. There are three points of interest here: the accusative masculine personal pronoun is \textit{en} ‘him’ (not \textit{er}), endings including weak adjectival endings seem to be systematically reduced from -\textit{en} to -\textit{e} – a general feature of the dialect, and the weak ending -\textit{e} in the nominative masculine is often left out. This means that personal pronouns have their own insertion rules (at least in this dialect), that phonological rules may obscure the underlying inflectional paradigms, and that some endings may be optional (possibly analyzed as instances of Impoverishment). I abstract away from these finer points here. Also, as Russ (1989: 210-14) points out, this Mannheim German dialect should not be equated with regional varieties that are spoken in Mannheim or near this geographical area (see, for instance, Russ 1989: 231, 251 on the German spoken in Michelstadt in southern Hesse and in Kaulbach in western Palaninate). I thank a reviewer for making me aware of this issue.}
\end{styleFootnote}

\begin{styleFootnote}
(70)\ \ a.\ \ \textit{Ich kenn} [ \textit{d-er \ \ ander-e \ \ \ Mann} ].\ \ \ \ (Mannheim German)
\end{styleFootnote}

\begin{styleFootnote}
\ \ \ \ I \ \ \ know \ the-\textsc{st} other-\textsc{wk} man.\textsc{masc}
\end{styleFootnote}

\begin{styleFootnote}
\ \ \ \ ‘I know the other man.’
\end{styleFootnote}

\begin{styleFootnote}
\ \ b.\ \ \textit{Der hat} [ \textit{ein groß-er Hubser} ] \ \ \ \textit{gemacht}.
\end{styleFootnote}

\begin{styleFootnote}
\ \ \ \ he \ \ has \ \ a \ \ \ big-\textsc{st} \ \ jump.\textsc{masc} done
\end{styleFootnote}

\begin{styleFootnote}
\ \ \ \ ‘He did a big jump.’
\end{styleFootnote}

\begin{styleFootnote}
\ \ c.\ \ \textit{Ich mag} [ \textit{gut-er \ \ \ Wein} ].
\end{styleFootnote}

\begin{styleFootnote}
\ \ \ \ I \ \ \ like \ \ \ \ good-\textsc{st} wine.\textsc{masc}
\end{styleFootnote}

\begin{styleFootnote}
\ \ \ \ ‘I like good wine.’
\end{styleFootnote}

\begin{styleFootnote}
\ \ d.\ \ \textit{Ich seh} [ \textit{ein-er} ].
\end{styleFootnote}

\begin{styleFootnote}
\ \ \ \ I \ \ \ \ see \ \ one-\textsc{st}
\end{styleFootnote}

\begin{styleFootnote}
\ \ \ \ ‘I see one.’
\end{styleFootnote}

\begin{styleFootnote}
Müller concludes that all analyses including his own can straightforwardly account for this dialect by deleting a rule or changing the featural specification of a rule. We see below that the current proposal can also explain this variety. Unlike G. Müller’s discussion, here I also address the weak inflections and \textit{ein}{}-words.
\end{styleFootnote}

\begin{styleFootnote}
\ \ I start by updating the two inflectional paradigms. Considering (70a), note that the strong and weak ending -\textit{en} in the accusative masculine is replaced by the strong ending -\textit{er} and the weak ending -\textit{e}, respectively\textit{ }((70b-d) provide additional evidence that the nominative masculine form occurs in accusative contexts). The inflectional paradigm of Mannheim German is summarized in Table 11:
\end{styleFootnote}

\begin{styleStandard}
Table 11:\textit{ }Strong and Weak Endings in Mannheim German
\end{styleStandard}

\begin{flushleft}
\begin{tabular}{|m{0.7337598in}|m{0.5462598in}|m{0.6087598in}|m{0.6087598in}|m{0.6087598in}|m{0.6087598in}|m{0.6087598in}|m{0.6087598in}|}

\hline
\centering{\bfseries STRONG} &
\centering [-F, -N] &
\centering [-F, +N] &
\centering [+F, -N] &
\centering [+F, +N] &
\centering{\bfseries WEAK} &
\centering [-$\gamma $] &
\centering\arraybslash [+F, +N]\\\hline
[ -O, \ {}-S] &
\centering {}-\textit{er} &
\centering {}-\textit{es} &
\centering (-\textit{e}) &
\centering{\itshape {}-e} &
\centering [-O] &
\centering (\textit{{}-e}) &
\\\hline
[ -O, +S] &
\centering {}-\textit{er} &
\centering {}-\textit{es} &
\centering (-\textit{e}) &
\centering{\itshape {}-e} &
 &
 &
\\\hhline{-------~}
[+O, \ {}-S] &
\centering {}-\textit{em} &
\centering {}-\textit{em} &
\centering {}-\textit{er} &
\centering \{-\textit{en}\} &
\centering [+O] &
\multicolumn{2}{m{1.2962599in}|}{\centering \{-\textit{en}\}}\\\hline
[+O, +S] &
\centering {}-\textit{es} &
\centering {}-\textit{es} &
\centering {}-\textit{er} &
\centering{\itshape {}-er} &
 &
 &
\\\hhline{-------~}
\end{tabular}
\end{flushleft}
\begin{styleStandard}
The rules for vocabulary insertion are provided below. There are three differences to Standard German as regards adjectival inflections: (i) the rule inserting -\textit{en} in the accusative masculine has been deleted, (ii) both rules inserting -\textit{er} have been changed in their feature specifications, and (iii) due to the latter change, these two rules have been reordered using the same specificity metrics as above (in Table 12 below, I juxtapose the rule systems of Standard and Mannheim German):
\end{styleStandard}

\begin{styleStandard}
(71)\ \ Strong (except for feminine \textit{{}-e} and plural \textit{{}-en}):
\end{styleStandard}

\begin{styleStandard}
\ \ a.\ \ [+F, +N, +O, +S]\ \ →\ \ \textit{{}-er}
\end{styleStandard}

\begin{styleStandard}
\ \ b.\ \ [+F, +N, -O, $\alpha $S]\ \ →\ \ \textit{{}-e}
\end{styleStandard}

\begin{styleStandard}
\ \ c.\ \ [$\alpha $F, -N, $\alpha $O, $\beta $S]\ \ →\ \ \textit{{}-er}
\end{styleStandard}

\begin{styleStandard}
d.\ \ [-F, +O, -S]\ \ \ \ →\ \ \textit{{}-em}
\end{styleStandard}

\begin{styleStandard}
\ \ e.\ \ [-F, $\alpha $O, $\beta $S]\ \ \ \ →\ \ \textit{{}-es}
\end{styleStandard}

\begin{styleStandard}
(72) \ \ \ \ Weak (including strong feminine \textit{{}-e} and plural \textit{{}-en}):
\end{styleStandard}

\begin{styleStandard}
\ \ a.\ \ [-$\gamma $, -O]\ \ \ \ \ \ →\ \ \textit{{}-e}
\end{styleStandard}

\begin{styleStandard}
\ \ b.\ \ []\ \ \ \ \ \ →\ \ \textit{{}-en}
\end{styleStandard}

\begin{styleFootnote}
As regards \textit{ein}{}-words, the rule that inserts accusative masculine \textit{einen }‘a’ has been deleted yielding the following insertion rules. Recall that if (73a) does not apply, (73b) will in conjunction with the inflectional rules in (73c) or (73d):
\end{styleFootnote}

\begin{styleFootnote}
(73)\ \ a.\ \ [+D][-F, -O]\ \ \ \ → \ \ \textit{ein} / \_\_ word ]\textsubscript{$\Phi $}
\end{styleFootnote}

\begin{styleFootnote}
\ \ b.\ \ [+D]\ \ \ \ \ \ → \ \ \textit{ein}{}-
\end{styleFootnote}

\begin{styleFootnote}
c.\ \ [+F, +N, +O, +S]\ \ →\ \ \textit{{}-er}
\end{styleFootnote}

\begin{styleFootnote}
\ \ d.\ \ etc.
\end{styleFootnote}

\begin{styleFootnote}
Besides the three differences in adjectival inflections, this adds a fourth difference to Standard German.
\end{styleFootnote}

\begin{styleFootnote}
\ \ More generally, deleting the vocabulary insertion rules for accusative masculine -\textit{en} and for accusative masculine \textit{einen} ‘a’ highlights the fact that the feature combination of “nominative/accusative + masculine/neuter” has a special status. This is directly stated by the features [-F, -O] in (73a).
\end{styleFootnote}

\begin{styleFootnote}
The vocabulary insertion rules for the two varieties are provided in Table 12 juxtaposing the strong endings in (1), the weak endings in (2), and \textit{ein} in (3) in Standard German and their counterparts in (1’), (2’), and (3’) in Mannheim German: 
\end{styleFootnote}

\begin{styleFootnote}
Table 12: Comparison between Standard and Mannheim German
\end{styleFootnote}

\begin{flushleft}
\begin{tabular}{|m{3.24626in}|m{3.24626in}|}

\hline
\centering Standard German &
\centering\arraybslash Mannheim German\\\hline
1a. \ [+F, -N, +O, $\alpha $S]\ \  \ \ \ \ →\ \ \textit{{}-er}

\ \ \ \ \ \ \ [+F, +N, -O, $\alpha $S]\ \  \ \ \ \ →\ \ \textit{{}-e}

\ \ b. \ [$\alpha $F, $\alpha $N, $\alpha $O, $\alpha $S]\ \  \ \ \ \ →\ \ \textit{{}-er}

\ \ c. \ [-F, -N, -O]\ \  \ \ \ \ →\ \ \textit{{}-en} 

\ \ \ \ \ \ \ [-F, +O, -S]\ \  \ \ \ \ →\ \ \textit{{}-em}

\ \ d. \ [-F, $\alpha $O, $\beta $S]\ \  \ \ \ \ →\ \ \textit{{}-es}

2a. \ [-$\gamma $, -O]\ \ \ \  \ \ \ \ →\ \ \textit{{}-e}

\ \ b. \ []\ \ \ \ \ \  \ \ \ \ →\ \ \textit{{}-en} &
1a’. \ [+F, +N, +O, +S] →\ \ \textit{{}-er}

\ \ b’. \ [+F, +N, -O, $\alpha $S]\ \  →\ \ \textit{{}-e}

\ \ c’. \ [$\alpha $F, -N, $\alpha $O, $\beta $S]\ \  →\ \ \textit{{}-er}

\ \ d’. \ [-F, +O, -S]\ \  →\ \ \textit{{}-em}

\ \ e’. \ [-F, $\alpha $O, $\beta $S]\ \  →\ \ \textit{{}-es}

2a’. \ [-$\gamma $, -O]\ \ \ \  →\ \ \textit{{}-e}

\ \ b’. \ []\ \ \ \ \ \  →\ \ \textit{{}-en}\\\hline
{\fontsize{10pt}{12.0pt}\selectfont 3a. \ [+D]\ \  \ \ \ \ \ \ \ \ \ \ \ \ \ \ \ \ → \ \ \textit{ein- }/ [-F, -N, -O, +S]}

{\fontsize{10pt}{12.0pt}\selectfont \ \ b. \ [+D][-F, -O]\ \  \ \ \ \ → \ \ \textit{ein} / \_\_ word ]\textsubscript{$\Phi $}}

{\fontsize{10pt}{12.0pt}\selectfont \ \ c. \ [+D]\ \ \ \  \ \ \ \ → \ \ \textit{ein}{}-} &
{\fontsize{10pt}{12.0pt}\selectfont 3a’. \ [+D][-F, -O]\ \  → \ \ \textit{ein} / \_\_ word ]\textsubscript{$\Phi $}}

{\fontsize{10pt}{12.0pt}\selectfont \ \ b’. \ [+D]\ \ \ \  → \ \ \textit{ein}{}-}\\\hline
\end{tabular}
\end{flushleft}
\begin{styleFootnote}
Note again that the rule inserting -\textit{en} in (1c) in Table 12 has no counterpart in (1’). Given that, all vocabulary insertion rules for adjectival inflections in Mannheim German in (1a’-e’) have the feature [S]. Impoverishment applies as in Chapter 2 but now to all (unambiguously) strong inflections.
\end{styleFootnote}

\begin{styleFootnote}
To sum up, like the five other analyses, the current system can also account for the regional variety of Mannheim German, including the variation seen with weak inflections and \textit{ein}{}-words. Notice that I focused here on one regional dialect to show that the current system can capture dialectal variation as well. For the description of other dialects, see Schirmunski (2010); for some tentative discussion of Alemannic German, see Chapter 8, Section 2.1.
\end{styleFootnote}

\begin{styleFootnote}\bfseries
9. \ \ Conclusion
\end{styleFootnote}

\begin{styleFootnote}
This chapter continued the discussion of adjectival inflections begun in Chapter 2. Here I focused on variation in (mostly) canonical DPs where unexpected strong or unexpected weak adjectives occur. The unexpected strong adjectives are due to pronominal determiners not triggering Impoverishment. The unexpected weak adjectives follow from a phonetic rule, presumably a reflex of markedness reduction. The unexpected weak inflections on determiners are accounted for by a more general application of Impoverishment Rule 2. We arrive then at three different mechanisms that explain the weak inflections: one primary mechanism (Impoverishment Rule 1) and two secondary mechanisms (phonetic rule, Impoverishment Rule 2). Unlike the primary mechanism, the secondary ones apply in restricted contexts, that is, in certain constructions and/or specific featural contexts. Note that the secondary mechanisms are brought to light when the primary one does not occur. The main patterns of Chapters 2 and 3 along with their analyses are summarized in Table 13 (the inflection under consideration is set apart by a hyphen):
\end{styleFootnote}

\begin{styleStandard}
Table 13: Summary Chart
\end{styleStandard}

\begin{flushleft}
\begin{tabular}{|m{3.68376in}|m{2.80876in}|}

\hline
\centering Construction: Example &
\centering\arraybslash Inflection: Analysis\\\hline
Regular DP:\textit{ der heiß-e schwarz-e Kaffee} &
Weak: \textit{Impoverishment Rule 1 }\\\hline
Regular DP: (i) \textit{heiß-en Kaffees}; (ii)\textit{ dies-en Jahres} &
Weak: \textit{Impoverishment Rule 2}\\\hline
Regular DP:\textit{ }\textsuperscript{\%}\textit{mit frischem schwarz-en Kaffee} &
Weak: \textit{Phonetic Rule }\\\hline
Regular DP: \textit{ein heiß-er schwarz-er Kaffee} &
Strong: ein\textit{{}-words}\\\hline
Close Appositions:\textit{ der Indianer Groß-er Bär} &
Strong: \textit{Low Right-Adjunction }\\\hline
Indefinite Pronoun Constructions:\textit{ wer ander-er} &
Strong:\textit{ Mid Right-Adjunction }\\\hline
Noun-Adjective Exclamatives: \textit{Schwein, schwarz-es!} &
Strong: \textit{Mid Right-Adjunction}\\\hline
Loose Appositions:\textit{ wir, begeistert-e Linguisten} &
Strong: \textit{High Right-Adjunction }\\\hline
Agreement in Pronominal DPs: \textit{ich dumm-er Idiot} &
Strong: \textit{Pronominal Determiner}\\\hline
Dis-agreement in Pronominal DPs:\textit{ ihr jung-es Gemüse} &
Strong: \textit{Complex Specifier inside DP }\\\hline
Split Topicalizations:\textit{ bunt-e Hemden ... diese da} &
Strong: \textit{Separate Base-generation }\\\hline
Nominal Compounds: \textit{dem Deutsch-e Bank Chef} &
Strong: \textit{Complex Compound Modifier}\\\hline
Vocatives: \textit{Dumm-er Idiot!} &
Strong: \textit{Voc Head and Absence of Det}\\\hline
Predeterminers: \textit{dieses mein groß-es Glück} &
Strong: \textit{Determiner Outside DP Proper} \\\hline
\end{tabular}
\end{flushleft}
\begin{styleStandard}
Besides these cases, I also discussed inflectional variation on certain adjectives (e.g., \textit{lila} ‘purple’, \textit{prima} ‘great’, \textit{Berliner} ‘(from) Berlin’), and I addressed the uninflected forms of the definite article and its related distal demonstrative. Furthermore, I discussed the optional inflection on the predeterminer \textit{all-} ‘all’. On the basis of these cases, I demonstrated that the generalization Weak After Strong and the more general Principle of Monoinflection face a number of empirical challenges. Finally, I showed that the current system can be extended to Mannheim German, one of the dialects in Germany. The overall conclusion is that a purely surface-oriented account does not work – the consideration of structure appears to be an essential part of any proposal. The next chapter discusses consequences of the current proposal for other accounts. 
\end{styleStandard}

\clearpage\setcounter{page}{179}\begin{styleFootnote}
Chapter 4: Consequences for Other Analyses
\end{styleFootnote}

\begin{styleFootnote}\bfseries
1.\ \ Introduction
\end{styleFootnote}

\begin{styleFootnote}
In this chapter, I turn to some other consequences of the analysis developed in Chapter 2. Unlike the previous chapter, I now consider implications for other proposals. In so doing, I provide more evidence for the hypotheses proposed in Chapter 1. I consider three types of accounts where adjectival inflections are revealing as regards the proposed structure (Hypothesis 2a). Starting with analyses that seek to account for the presence of spurious indefinite articles, I show that weak adjectives present problems for structures involving Predicate Inversion and for analyses postulating null nouns. Second, discussing discontinuous noun phrases, I argue that strong endings on topicalized adjectives are only compatible with split topicalizations being analyzed as the base-generation of two separate nominals but not as movement involving one nominal. Third, returning to weak inflections, I show that adjectives involving a restrictive or a non-restrictive interpretation have the same basic structure. The discussion of these consequences documents some other properties of adjectival inflections. For instance, I demonstrate that strong endings are not “referential” but simply serve to make nominal features such as case, number, and gender visible (Hypothesis 2b). I begin by discussing the analysis involving Predicate Inversion and take up the discussion of the other points in the order mentioned above.
\end{styleFootnote}

\begin{styleStandard}\bfseries
2.\ \ Weak Adjectives in the Context of Spurious Indefinite Articles
\end{styleStandard}

\begin{styleStandard}
In this section, I discuss two types of analysis that seek to account for the occurrence of spurious indefinite articles in Dutch. These proposals involve Predicate Inversion, on the one hand, and null nouns, on the other hand. Extending the discussion to German, it is shown that weak adjectives follow the spurious articles. Given that both proposals involve structures different from canonical DPs, the weak adjectives indicate that these types of analysis are not compatible with the system laid out in Chapter 2.
\end{styleStandard}

\begin{styleStandard}\itshape
2.1.\ \ Weak Adjectives in Structures Involving Predicate Inversion
\end{styleStandard}

\begin{styleStandard}
Taking Dutch as their empirical basis, Bennis \textit{et al}. (1998) discuss – what they call – \textit{wat}{}-exclamative constructions (1a). Like some other constructions they investigate (1b), a singular indefinite article occurs with a plural noun. They call this element spurious article. This article is syntactically obligatory in the first construction but not in the second, where its presence or absence has semantic consequences (data are taken from Bennis \textit{et al}. 1998: 98, 101, see also den Dikken 2006):
\end{styleStandard}

\begin{styleStandard}
(1)\textit{ \ \ }a.\ \ \textit{Wat *(een) jongens!\ \ \ \ \ \ }(Dutch)
\end{styleStandard}

\begin{styleStandard}
\ \ \ \ what \ \ a \ \ \ \ \ boys
\end{styleStandard}

\begin{styleStandard}
\ \ \ \ ‘What boys!’
\end{styleStandard}

\begin{styleStandard}
\ \ b.\ \ \textit{idioten van (een) mannen}
\end{styleStandard}

\begin{styleStandard}
\ \ \ \ idiots \ \ of \ \ \ \ a \ \ \ \ \ men
\end{styleStandard}

\begin{styleStandard}
\ \ \ \ ‘idiots of men’
\end{styleStandard}

\begin{styleStandard}
I focus on the exclamative construction in (1a) as that has a counterpart in German.
\end{styleStandard}

\begin{styleStandard}
Adopting the general framework of den Dikken (1995), Bennis \textit{et al}. (1998) propose a small clause structure with some further functional positions on top. Considering (2) below, the small clause is represented by XP and the functional structure by DP. The nominal \textit{jongens} ‘boys’ is assumed to be the subject, \textit{wat} ‘what’ is the predicate, and \textit{een} ‘a’ is the head of the small clause. Bennis \textit{et al}. propose that D is an [+EXCL] operator, and this element needs to be lexicalized. As a consequence, \textit{een} raises from X to D. The predicate \textit{wat} undergoes Predicate Fronting to Spec,DP, a type of A’-movement. The authors interpret this as something similar to the Verb Second Constraint in the clause (their tree diagram on page 106 is slightly adapted here):
\end{styleStandard}

\begin{styleFootnote}
(2)\textit{ \ \ }\ \  \ DP
\end{styleFootnote}

\begin{styleFootnote}
[Warning: Draw object ignored][Warning: Draw object ignored]
\end{styleFootnote}

\begin{styleFootnote}
\ \  \textit{Wat}\textsubscript{k}\ \ \ \  \ \ D’
\end{styleFootnote}

\begin{styleStandard}
[Warning: Draw object ignored][Warning: Draw object ignored]
\end{styleStandard}

\begin{styleStandard}
\ \  \ \ \ \ \ \textit{een}\textsubscript{i}+[+\textsc{excl}]\ \  \textsubscript{\ }XP
\end{styleStandard}

\begin{styleStandard}
[Warning: Draw object ignored][Warning: Draw object ignored]
\end{styleStandard}

\begin{styleStandard}
\ \ \ \  \ \ \ \ \ \ \ \ \textit{jongens}\ \ \ \  \ X’
\end{styleStandard}

\begin{styleStandard}
[Warning: Draw object ignored][Warning: Draw object ignored]
\end{styleStandard}

\begin{styleStandard}
\ \ \ \ \ \ \ \  \ \ t\textsubscript{i}\ \ \ \  \ \ t\textsubscript{k} 
\end{styleStandard}

\begin{styleStandard}
The counterpart of the \textit{wat}{}-exclamative in German involves similar (but not identical) properties. First, unlike in Dutch, the prepositional element \textit{für} ‘for’ must be present (3a). Second, the spurious article is not obligatory in this construction. In fact, as pointed out in Chapter 1, Section 2.2, if it occurs, it typically appears in colloquial northern dialects (\% indicates dialectal variation). Note again that this \textit{ein} has a typical plural ending as seen in the attested example in (3b), taken from the Appendix:
\end{styleStandard}

\begin{styleStandard}
(3)\textit{ \ \ }a.\ \ \textit{Was *(für) Idioten!}
\end{styleStandard}

\begin{styleStandard}
\ \ \ \ what \ \ for \ \ idiots
\end{styleStandard}

\begin{styleStandard}
\ \ \ \ ‘What idiots!’
\end{styleStandard}

\begin{styleStandard}
\ \ b. \%\ \ \textit{{\textquotedbl}}\emph{Was \ \ für eine Idioten}\textit{{\textquotedbl} sagte Liam}
\end{styleStandard}

\begin{styleStandard}
\ \ what for \textsubscript{\ }a-\textsc{pl} idiots \ \ \ \ said \ \ Liam
\end{styleStandard}

\begin{styleStandard}
‘“What idiots!”, said Liam’
\end{styleStandard}

\begin{styleStandard}
After these preliminary observations, I return to the main line of investigation. To make these cases relevant, it is important to determine how adjectives, especially their inflections, fare when they are added to the subject of the small clause.
\end{styleStandard}

\begin{styleStandard}
Inserting an adjective yields an interesting asymmetry. With no determiner present, the adjective has a strong ending (4a). This is expected and follows from the system developed in Chapter 2. When the spurious article is present, the adjective is weak. In fact, all the examples I have found involve weak adjectives only. Consequently, I mark the strong inflection in the attested example in (4b) as ungrammatical:
\end{styleStandard}

\begin{styleStandard}
(4)\textit{ \ \ }a.\ \ \textit{Was \ für geil-e(*n) \ \ \ \ \ \ \ \ \ \ \ \ Bilder!}
\end{styleStandard}

\begin{styleStandard}
\ \ \ \ what for awesome-\textsc{st}/*\textsc{wk} pictures
\end{styleStandard}

\begin{styleStandard}
\ \ \ \ ‘What awesome pictures!’
\end{styleStandard}

\begin{styleStandard}
\ \ b. \ \%\ \ \textit{Was \ für ein-e geil-e*(n) \ \ \ \ \ \ \ \ \ \ \ \ Bilder \ \ \ wer \ hat die \ \ \ bloß gemacht}
\end{styleStandard}

\begin{styleStandard}
\ \ \ \ what for a-\textsc{pl} \ awesome-\textsc{wk}/*\textsc{st} pictures who has those \textsc{prt} \ made
\end{styleStandard}

\begin{styleStandard}
\ \ \ \ ‘What awesome pictures! Who in the world took those?’
\end{styleStandard}

\begin{styleStandard}
With Bennis \textit{et al}.’s (1998) discussion in mind, the structure of (4b) would presumably look as follows:\footnote{\ In their discussion of \textit{N of a N}{}-constructions, Bennis \textit{et al}. (1998) allow a preposition and a spurious article to occupy the same head position. In other words, the structure in the main text should, in principle, be fine.}
\end{styleStandard}

\begin{styleFootnote}
(5)\textit{ \ \ Predicate Inversion}
\end{styleFootnote}

\begin{styleFootnote}
\ \ \ \  \ DP
\end{styleFootnote}

\begin{styleFootnote}
[Warning: Draw object ignored][Warning: Draw object ignored]
\end{styleFootnote}

\begin{styleFootnote}
\ \  \textit{Was}\textsubscript{k}\ \ \ \  \ \ D’
\end{styleFootnote}

\begin{styleStandard}
[Warning: Draw object ignored][Warning: Draw object ignored]
\end{styleStandard}

\begin{styleStandard}
\ \  \ \ \ \ \ \ \textit{für}+\textit{eine}\textsubscript{i}\ \ \ \  \textsubscript{\ }XP
\end{styleStandard}

\begin{styleStandard}
[Warning: Draw object ignored][Warning: Draw object ignored]
\end{styleStandard}

\begin{styleStandard}
\ \ \ \  \ \ [\textit{geilen Bilder}]\ \  \ X’
\end{styleStandard}

\begin{styleStandard}
[Warning: Draw object ignored][Warning: Draw object ignored]
\end{styleStandard}

\begin{styleStandard}
\ \ \ \ \ \ \ \  \ \ t\textsubscript{i}\ \ \ \  \ \ t\textsubscript{k} 
\end{styleStandard}

\begin{styleStandard}
The question that arises now is how to account for the weak ending in (4b), given this structure.
\end{styleStandard}

\begin{styleStandard}
\ \ Note that the Predicate Inversion structure above is configurationally identical to pronominal DPs involving dis-agreement. This type of case was discussed in Chapter 2, Section 3.5 and is repeated below as (6). Notice first that both the spurious article in (5) and the pronoun in (6b) move from a lower position to the DP-level. Second and more importantly, in both cases, the adjective and noun form a complex specifier. In other words, unlike canonical DPs, the two cases in (5) and (6b) have an additional layer of embedding. However, unlike (4b), the nominal in (6a) involves a strong adjective:
\end{styleStandard}

\begin{styleStandard}
(6)\textit{ \ \ }a.\ \ \textit{ihr \ \ \ \ \ \ \ \ \ dumm-e*(s) \ \ \ Pack}
\end{styleStandard}

\begin{styleStandard}
\ \ \ \ you(\textsc{pl}) stupid-\textsc{st}/*\textsc{wk} gang.\textsc{neut}
\end{styleStandard}

\begin{styleStandard}
\ \ \ \ ‘you stupid gang’
\end{styleStandard}

\begin{styleFootnote}
\textit{\ \ }b.\ \  \ DP
\end{styleFootnote}

\begin{styleFootnote}
[Warning: Draw object ignored][Warning: Draw object ignored]
\end{styleFootnote}

\begin{styleFootnote}
\ \  \textit{ihr}\textsubscript{i}\ \ \ \ DisP
\end{styleFootnote}

\begin{styleFootnote}
[Warning: Draw object ignored][Warning: Draw object ignored]
\end{styleFootnote}

\begin{styleStandard}
\ \ \ \ \ \ \ \ \ \ \ [\textsubscript{AgrP} \textit{dummes Pack}] \ \ \ \ \ Dis’
\end{styleStandard}

\begin{styleStandard}
[Warning: Draw object ignored][Warning: Draw object ignored]
\end{styleStandard}

\begin{styleStandard}
\ \ \ \  \ \ \ \ \ \ Dis\ \ \ \ ArtP
\end{styleStandard}

\begin{styleStandard}
[Warning: Draw object ignored][Warning: Draw object ignored]
\end{styleStandard}

\begin{styleStandard}
\ \ \ \ \ \  \ \ \ \ \ \ \  \ t\textsubscript{t}\ \  \ \ \ \  XP
\end{styleStandard}

\begin{styleFootnote}
\ \ \ \ \ \ \ \ \ \  \ \ \ \ \ \ \ \ \ \ \ \ \ \textit{e}\textit{\textsubscript{N}}\textit{\ \ }
\end{styleFootnote}

\begin{styleStandard}
Before addressing the different inflections on the adjectives, it is instructive to consider a contrast between German and English.
\end{styleStandard}

\begin{styleStandard}
As mentioned above, German \textit{ein} exhibits a typical plural ending in these cases. It appears then as if the spurious article morphologically agrees with the plural noun. Now, Bennis \textit{et al}. (1998: 94, 97) explicitly claim that the indefinite article does not form a constituent with the following noun. In their base positions, though, the noun and the indefinite article are in a Spec-head relation. This, for instance, accounts for the absence of spurious articles in English, as witnessed in the ungrammatical string *\textit{What} \textit{a men!}. 
\end{styleStandard}

\begin{styleStandard}
In order to explain the contrast in the plural between English and German, we could assume that English \textit{a} is specified for singular morphological number but that German \textit{ein} is unspecified as regards morphological number: [$\alpha $PL morph] (see Chapter 5). Given the assumed Spec-head relation, English \textit{a} is only compatible with a singular noun but German \textit{ein} is with both a singular and plural noun. So far, so good. Note, however, that this Spec-head relation in and of itself does not explain the weak ending on the adjective.
\end{styleStandard}

\begin{styleStandard}
Recall from Chapter 2 that concord is not a sufficient condition for weak adjectives. What is required is an appropriate structure. As is clear from the above discussion, the Predicate Inversion structure is not a regular DP – the adjective is more deeply embedded. This leaves the occurrence of the weak adjective unexplained. However, it is possible that the structure in (5) is not correct. There are several options. For instance, one option might be to assume that a null article is present as part of the nominal in Spec,XP. As documented in Chapter 2, Section 2.3 though, null articles occur with strong adjectives (except in the genitive masculine/neuter). In other words, a null article cannot account for the weak adjective either. To keep Bennis \textit{et al}.’s structure, we would need to change the system of Chapter 2 to account for the weak adjectives under the assumption of a null article.
\end{styleStandard}

\begin{styleStandard}
As a second option, we could suggest that the adjective itself is in a position different from that assumed in (5). Interestingly, for ordinary DPs such as (7a), den Dikken (2006: 49) proposes the structure in (7b), where the adjective is in the specifier position of a Relator Phrase (RP), and the noun is in the complement position of the Relator head (non-pronunciation is marked by capital letters):
\end{styleStandard}

\begin{styleFootnote}
(7)\textit{ \ \ }a.\ \ a big butterfly\ \ 
\end{styleFootnote}

\begin{styleFootnote}
\ \ \ b. \ \ [\textsubscript{DP/NumP} \textit{a} [\textsubscript{RP} [\textsubscript{AP} \textit{big} ] [ RELATOR [\textsubscript{NP} \textit{butterfly} ]]]]
\end{styleFootnote}

\begin{styleStandard}
Importantly, the adjective is also in the specifier below the determiner, just as in the canonical cases in Chapter 2 (note that this holds independently of whether or not the adjective is base-generated or moved there). Updating the structure in (5), notice though that the entire RP, which includes the noun in (7b), would form the subject located in Spec,XP. This would essentially yield the same configuration as in (5), and the weak adjectives remain unexplained. 
\end{styleStandard}

\begin{styleStandard}
To sum up this subsection, at best, we can observe that the indefinite article in these constructions is not entirely spurious in German. Accepting Bennis \textit{et al}.’s (1998) structure (or adopting a similar structure), we would need to modify the analysis in Chapter 2 to account for the weak adjectives. It is currently not clear to me how to do that without losing the account of the canonical and other non-canonical constructions. If a solution can be found, this modification should preferably not be a (construction-specific) mechanism that simply serves to “save” the Predicate Inversion analysis. At worst, the weak endings on the adjectives hint at the fact that Bennis \textit{et al}.’s structure is not on the right track (for other issues, see Matushansky 2002). 
\end{styleStandard}

\begin{styleStandard}
Note that Chapters 2 and 3 documented a number of points that speak in favor of keeping the current analysis – it accounts not only for the canonical but also for many non-canonical cases. Indeed, if we accept the system laid out in Chapter 2, then this discussion reiterates the fact that the inflections of the adjectives are closely tied to the structure of the noun phrase as a whole (Hypothesis 2b). If this is on the right track, then we have a means to narrow down the choices of possible analyses of other constructions. In the next subsection, I consider another type of analysis that seeks to account for the occurrence of spurious articles. We will see that given current assumptions, this type of proposal is also ill-equipped to deal with weak adjectives.
\end{styleStandard}

\begin{styleStandard}\itshape
2.2.\ \ Weak Adjectives in Structures Involving Null Nouns
\end{styleStandard}

\begin{styleStandard}
Extending earlier work by Leu, later published as Leu (2008a,b), van Riemsdijk (2005) also discusses spurious articles in Dutch. Discussing various constructions, van Riemsdijk (2005: 165) provides the example in (8) involving a non-\textit{wh}{}-exclamative (see also Bennis \textit{et al}. 1998: 92 fn. 7):
\end{styleStandard}

\begin{styleFootnote}
(8)\ \ \textit{Die auto heeft een deuken!}\ \ \ \ \ \ (Dutch)
\end{styleFootnote}

\begin{styleFootnote}
\ \ that car \ \ has \ \ a \ \ \ \ dents
\end{styleFootnote}

\begin{styleFootnote}
\ \ ‘That car has dents!’
\end{styleFootnote}

\begin{styleFootnote}
Unlike the Predicate Inversion analysis, van Riemsdijk argues for the presence of a semi-lexical null noun. He labels this element TYPE (again, capitalization here indicates non-pronunciation). Based on his structure on page 173, the datum in (8) is analyzed as in (9), where !!! stands for an exclamative operator:
\end{styleFootnote}

\begin{styleFootnote}
(9)\ \ !!!.......[\textsubscript{DP} [ [\textsubscript{D} (\textit{een})] [ DEG [ [\textsubscript{n} TYPE] \textit{deuken} ]]]]
\end{styleFootnote}

\begin{styleFootnote}
With this in place, I return to the discussion of the spurious article and weak adjectives in German. Since I have not come across any examples like (8) in German that involve an adjective, I use a slightly different type of case as the basis for discussion – examples where \textit{so} ‘so’ precedes the indefinite article. As far as I can see, this additional element does not affect the point to be made.
\end{styleFootnote}

\begin{styleFootnote}
\ \ The presence of a null noun allows a different analysis of the spurious article. Consider (10a), an example taken from the Appendix. Unlike Predicate Inversion, here a null noun is present by assumption, and the ending on \textit{ein} could be interpreted as feminine agreeing with that null noun (recall that the strong inflection -\textit{e} is ambiguous between feminine and plural in the nominative/accusative). As for the postulated null noun with feminine gender, an overt counterpart in German would be \textit{Art} ‘kind’ (10b). In certain “affective” contexts, this noun would remain unpronounced (10c): 
\end{styleFootnote}

\begin{styleFootnote}
(10)\ \ a. \ \% \ \ \textit{ich hab \ mal~}[\emph{so'n{}-e \ \ \ \ \ \ \ \ \ \ \ ferien\textup{]}}\textit{~ \ \ }\textit{\textsubscript{\ }}\textit{in den alpen mit \ \ ner jugendgruppe gemacht~}
\end{styleFootnote}

\begin{styleStandard}
I \ \ \ have \textsc{prt} \ so.a-\textsc{pl/fem }holidays in the \ Alps \ with a \ \ \ \ youth.group \ \ made
\end{styleStandard}

\begin{styleStandard}
‘Once, I had such a vacation with a youth group in the Alps.’
\end{styleStandard}

\begin{styleFootnote}
\ \ b.\ \ \textit{so} \textit{eine Art \ \ Ferien}
\end{styleFootnote}

\begin{styleFootnote}
\ \ \ \ so a \ \ \ \ \textsubscript{\ }kind holidays
\end{styleFootnote}

\begin{styleFootnote}
\ \ \ \ ‘such a kind of holidays’
\end{styleFootnote}

\begin{styleFootnote}
\ \ \ c.\textit{ }\ \ \textit{so} \textit{eine ART \ \ \ Ferien} 
\end{styleFootnote}

\begin{styleFootnote}
\ \ \ \ \ so a \ \ \ \ \textsubscript{\ }TYPE holidays
\end{styleFootnote}

\begin{styleFootnote}
In other words, the bracketed string in (10a) is analyzed as in (10c). Note that the presence of a null noun leads to a split of these structures into two subparts, each involving a noun (i.e., \textit{ART}, \textit{Ferien}). To be clear, then, the postulation of a null noun involves an analysis different from the Predicate Inversion structure. 
\end{styleFootnote}

\begin{styleFootnote}
Notice, however, that this type of account also faces problems as regards weak adjectives. Considering an attested example with an adjective (11a), there are two ways to analyze this datum as regards the modifier. The adjective could precede the null noun (11b), or it could follow the null noun (11c):
\end{styleFootnote}

\begin{styleFootnote}
(11)\textit{ \ \ }a. \ \%\ \ \textit{ihr \ macht~}\emph{so ne geilen}\textit{~ \ \ \ \ \ lieder~}
\end{styleFootnote}

\begin{styleFootnote}
you make so a \ \ \ awesome songs
\end{styleFootnote}

\begin{styleFootnote}
‘You compose such awesome songs.’
\end{styleFootnote}

\begin{styleFootnote}
\ \ b. (*)\ \ \textit{so} \textit{eine }\emph{geilen}\textit{ ART Lieder}
\end{styleFootnote}

\begin{styleFootnote}
\ \ c. (*)\ \ \textit{so} \textit{eine ART }\emph{geilen}\textit{ Lieder}
\end{styleFootnote}

\begin{styleStandard}
There are strong indications that both analyses in (11b) and (11c) are not correct. Starting with (11b), it is clear that the feminine form \textit{eine} can only occur in the nominative/accusative. I point out that -\textit{en} is not a possible adjective ending, strong or weak, in those instances (it would be -\textit{e}). As for (11c), the adjective is located in the second subpart together with the plural noun. This type of structure is familiar from pseudo-partitive constructions. This seems an avenue worth taking. Löbel (1989) observes in this regard that the second nominal in pseudo-partitives can be in all four morphological cases in German. In light of this, I entertain the possibility that the adjective inflection -\textit{en} could be interpreted as weak or strong. 
\end{styleStandard}

\begin{styleStandard}
Starting with the strong endings, I observe that -\textit{en} can only occur in the dative plural (the strong endings in the other cases are -\textit{e} or -\textit{er}). In the dative plural, the head noun usually takes -\textit{n} (e.g., \textit{mit geilen Lieder-n} ‘with awesome songs’). Crucially though, \textit{Liedern} is not at all possible in (11a). This means that \textit{geilen} cannot be in the dative plural and thus cannot be strong. If so, then \textit{geilen} should be weak in these instances. 
\end{styleStandard}

\begin{styleStandard}
Note first that the weak ending -\textit{en} occurs in all morphological cases in the plural. As argued in Chapter 2, this implies the presence of a determiner triggering Impoverishment. However, \textit{geilen }in (11c) does not have its own relevant determiner (recall again that a null determiner, arguably present in the second nominal in (11c), does not trigger Impoverishment). This leaves the option of the weak inflection unexplained. I conclude then that the presence of a null feminine noun cannot explain (11a). In other words, weak adjectives also present a problem for this type of analysis (for other issues, see Corver \& van Koppen 2011a: 61 fn. 6, 69 fn. 24).
\end{styleStandard}

\begin{styleStandard}
\ \ \ In order to explain the weak adjective in the current system, it is clear that the indefinite article and the adjective must form a regular DP. While I cannot discuss all the above constructions in detail here, note that they are, in some sense, “affective” or “emotive”. As such, I would like to suggest that such contexts license the occurrence of the plural indefinite article in these constructions. In Chapter 8, Section 2.2.3, I propose that these contexts license the presence of certain null operators (cf. the operator !!! in (9)). These null operators are flagged by \textit{ein} yielding the instances of the plural article. The latter triggers Impoverishment on the adjectives as discussed in Chapter 2.
\end{styleStandard}

\begin{styleStandard}
Summarizing the last two subsections, I documented that overt plural indefinite articles are followed by weak adjectives in German. Given the current analysis, I pointed out that these weak adjectives pose a challenge for accounts involving Predicate Inversion or null nouns. There are two ways to proceed: either the current analysis of adjectival inflections needs to be modified, or the above accounts of the spurious articles need to be changed. In other words, proponents of analyses involving Predicate Inversion or null nouns need to find a solution to explain the weak adjectives in these non-canonical constructions (NB: this proposal should also explain the canonical and other non-canonical cases). Since I am not aware of any such attempts, the alterntive is to suggest that constructions with spurious articles involve regular structures licensed in certain contexts. 
\end{styleStandard}

\begin{styleStandard}
More generally, given that the current analysis has broad empirical coverage as regards adjectival inflections, it shows good promise for a (more) comprehensive account. Indeed, the current system has revealed a number of interesting consequences. Most importantly, adjectival inflections in German are related to abstract structure (Hypothesis 2b). To reiterate, if this turns out to be correct, then this account narrows down the options of possible structural analyses of nominals. In the next section, I discuss strong adjectives that raise issues for certain structural analyses of split topicalization.
\end{styleStandard}

\begin{styleStandard}\bfseries
3.\ \ Strong Adjectives in Structures Involving Split Topicalization
\end{styleStandard}

\begin{styleFooter}
Discontinuous phrases have received much attention in the literature. German is interesting in that it allows the lower part of a noun phrase to be left dislocated. Compare (12a) to (12b). The topicalized element in (12b) functions as a contrastive topic and the lower nominal forms a focus. These two parts are related by a bridge intonation contour indicated by slash signs below:
\end{styleFooter}

\begin{styleStandard}
(12)\ \ a.\ \ \ \ \textit{Ich habe keine Bücher gelesen}.
\end{styleStandard}

\begin{styleStandard}
\ \ \ \ \ \ I \ \ \ \ have no \ \ \ \ books \ \ read
\end{styleStandard}

\begin{styleStandard}
\ \ \ \ \ \ ‘I have read no books.’
\end{styleStandard}

\begin{styleTextbody}
\ \ \ \ b.\ \ /\textbf{\textit{BÜcher}}\textit{ habe ich }\textbf{\textit{KEI}}{\textbackslash}\textbf{\textit{ne}}\textit{ gelesen}.
\end{styleTextbody}

\begin{styleStandard}
\ \ \ \  books \ \ \ have \ I \ \ \ none \ \ \ \ read.
\end{styleStandard}

\begin{styleStandard}
\ \ \ \ ‘As for books, I have read none.’
\end{styleStandard}

\begin{styleStandard}
To establish some terminology, I refer to this construction as split topicalization, to the left nominal as split-off, and to the right one as source. 
\end{styleStandard}

\begin{styleStandard}
As is well known, discontinuous DPs exhibit a number of paradoxical properties indicating both movement and separate base-generation. With current purposes in mind, I focus on the intriguing behavior of adjectives (for paradoxical properties not related to adjectives, see, for instance, Ott 2011a and other references mentioned below). To account for these properties, different proposals have been made. In what follows, I discuss three types of analysis (see also van Hoof 2006 and references cited therein). I show that movement out of one noun phrase faces problems in the account of strong adjectives. Similarly, a copy-and-delete approach to split topicalization cannot account for strong adjectives either. However, the base-generation of two independent noun phrases is completely compatible with the discussion in Chapter 2.
\end{styleStandard}

\begin{styleFootnote}\itshape
3.1.\ \ Movement out of the Source
\end{styleFootnote}

\begin{styleTextbody}
Van Riemsdijk (1989: 122) observes that the linear order of the adjectives in split topicalizations corresponds to the one without a split. Compare the sequences of adjectives in the unsplit examples in (13) to those in the split ones in (14):\footnote{\ The judgments in (13b) and (14b) are not uncontroversial and probably too strong (see Fanselow \& Ćavar 2002: 79-80, Ott 2011a: 30). However, I continue providing the original judgments.}
\end{styleTextbody}

\begin{styleStandard}
(13)\ \ a.\ \ \textit{ein neues amerikanisches Auto}
\end{styleStandard}

\begin{styleFooter}
\ \ \ \ a \ \ \ new \ \ American \ \ \ \ \ \ \ \ \ car.\textsc{neut}
\end{styleFooter}

\begin{styleFooter}
\ \ \ \ ‘a new American car’
\end{styleFooter}

\begin{styleStandard}
\ \ b. *\ \ \textit{ein amerikanisches neues Auto}
\end{styleStandard}

\begin{styleFooter}
\ \ \ \ a \ \ \ American \ \ \ \ \ \ \ \ \ new \ \ car.\textsc{neut}
\end{styleFooter}

\begin{styleStandard}
(14)\ \ a.\ \ \textbf{\textit{Ein amerikanisches Auto}}\textit{ \ \ \ \ \ \ kann ich mir \ \ }\textbf{\textit{kein neues}}\textit{ leisten.}
\end{styleStandard}

\begin{styleFooter}
\ \ \ \ an \ \ American \ \ \ \ \ \ \ \ \ \ car\textsc{.neut} can \ \ I \ \ \ \ \textsc{refl} no \ \ \ new \ \ \ \ afford
\end{styleFooter}

\begin{styleFooter}
\ \ \ \ ‘As for an American car, I cannot afford a new one.’
\end{styleFooter}

\begin{styleStandard}
\ \ b. *\ \ \textbf{\textit{Ein neues Auto}}\textit{ \ \ \ \ \ \ kann ich mir \ \ }\textbf{\textit{kein amerikanisches}}\textit{ leisten.}
\end{styleStandard}

\begin{styleFooter}
\ \ \ \ an \ \ new \ \ \ car\textsc{.neut} can \ \ I \ \ \ \ \textsc{refl} no \ \ \ American \ \ \ \ \ \ \ \ \ \ afford
\end{styleFooter}

\begin{styleStandard}
This can straightforwardly be explained under a movement analysis. In more detail, van Riemsdijk (1989) and Bhatt (1990: 249-50) argue for movement of the split-off out of the source. Adopting the DP-hypothesis, the above data can be analyzed as follows (see also Pafel 1995, Murphy 2018):
\end{styleStandard}

\begin{styleStandard}
(15)\ \ a.\ \ \textbf{\textit{Ein amerikanisches Auto}}\textit{ \ \ \ \ \ \ kann ich mir \ \ }\textbf{\textit{kein neues}}\textit{ leisten.}
\end{styleStandard}

\begin{styleFooter}
\ \ \ \ an \ \ American \ \ \ \ \ \ \ \ \ \ car\textsc{.neut} can \ \ I \ \ \ \ \textsc{refl} no \ \ \ new \ \ \ \ afford
\end{styleFooter}

\begin{styleFooter}
\ \ \ \ ‘As for an American car, I cannot afford a new one.’
\end{styleFooter}

\begin{styleStandard}
\textit{\ \ }b.\textit{\ \ }[ \textbf{\textit{Ein amerikanisches Auto }}]\textsubscript{i}\textit{ kann ich mir }[\textsubscript{DP}\textit{ }\textbf{\textit{kein neues}}\textit{ }[ t\textsubscript{i} ]]\textit{ leisten..}
\end{styleStandard}

\begin{styleFooter}
Note that there are two determiners in (15), \textit{ein} ‘a’ and \textit{kein} ‘no’, where \textit{ein} in the split-off is apparently doubling the determiner element in the source. To explain the presence of \textit{ein} in the split-off, it is assumed that this element is inserted later in the derivation (in Section 3.4, I show that late insertion of \textit{ein} is unlikely to be correct).
\end{styleFooter}

\begin{styleStandard}
Indefinite articles are usually absent in the split-off. Crucially, the distribution of adjectival inflections in such cases indicates an analysis different from simple movement. Note in this regard that adjectival inflections differ in regular and discontinuous noun phrases: the adjective is weak in a regular, unsplit DP (16a); it is strong, if the adjective is topicalized (16b):
\end{styleStandard}

\begin{styleStandard}
(16)\ \ a.\ \ \textit{Ich habe kein-e bunt-en \ \ \ \ \ \ \ \ \ \ \ \ \ \ \ \ Blumen gekauft}.
\end{styleStandard}

\begin{styleStandard}
\ \ \ \ I \ \ \ \ have no-\textsc{st} \ multi.colored-\textsc{wk} flowers bought
\end{styleStandard}

\begin{styleStandard}
\ \ \ \ ‘I have bought no multi-colored flowers.’
\end{styleStandard}

\begin{styleStandard}
\ \ b.\ \ [ \textbf{\textit{Bunt-e \ \ \ \ \ \ \ \ \ \ \ \ \ \ \ Blumen}} ]\textsubscript{i} \textit{habe ich }[\textsubscript{DP} \textbf{\textit{kein-e}}\textit{ }[ t\textsubscript{i} ]] \textit{gekauft}..
\end{styleStandard}

\begin{styleStandard}
\ \ \ \  \ multi.colored-\textsc{st} flowers \ \ \ \ have I \ \ \ \ \ \ \ \ none-\textsc{st} \ \ \ \ \ \ bought
\end{styleStandard}

\begin{styleStandard}
\ \ \ \ ‘As for multi-colored flowers, I have bought none.’
\end{styleStandard}

\begin{styleStandard}
With Chapter 2 in mind, I point out that the weak ending in (16a) shows that Impoverishment has taken place. The strong ending in (16b) indicates that Impoverishment has not occurred. This implies that either (16b) involves a non-canonical structure or that no relevant determiner is present in the split-off (or both). 
\end{styleStandard}

\begin{styleStandard}
\ \ Given a movement type of analysis, note that the entire noun phrase is assembled first. This is followed by movement of the split-off. Observe though that late separation, that is, building a regular DP first and then moving the lower part out would bring about a (wrong) weak ending in (16b). In other words, assuming movement out of the DP, the change of the adjective ending from weak in (16a) to strong in (16b) becomes mysterious. As documented in Chapters 2 and 3, a simple surface-oriented account of the strong/weak alternation does not suffice to explain these data.\footnote{\ Note that proponents of this and other structures of split topicalization do usually not provide (m)any details of their account of the strong/weak alternation of adjectives.}
\end{styleStandard}

\begin{styleStandard}
\ \ However, a strong ending on an unpreceded adjective is exactly what we expect if the two noun phrases in (16b) are base-generated independently of each other; that is, these elements do not seem to be related by movement. In the latter case though, the structure is different – there are two separate nominals, and the determiner that could trigger Impoverishment on the adjective in the split-off is in the other noun phrase, the source. As a consequence, Impoverishment does not occur, and the adjective in the split-off surfaces with a strong inflection as in (16b).
\end{styleStandard}

\begin{styleStandard}
To drive this point home, note that two related adjectives may show different inflections when split up such that the adjective in the source is weak, but the adjective in the split-off is strong (cf. Haider 1993: 215 for similar data). Compare (17a) to (17b):
\end{styleStandard}

\begin{styleStandard}
(17)\ \ a.\ \ \textit{Ich habe kein-e groß-en bunt-en \ \ \ \ \ \ \ \ \ \ \ \ \ \ \ \ Blumen gekauft}.
\end{styleStandard}

\begin{styleStandard}
\ \ \ \ I \ \ \ \ have no-\textsc{st} \ big-\textsc{wk} \ multi.colored-\textsc{wk} flowers \ bought
\end{styleStandard}

\begin{styleStandard}
\ \ \ \ ‘I have bought no big multi-colored flowers.’
\end{styleStandard}

\begin{styleTextbody}
\ \ b.\ \ \textbf{\textit{Bunt-e \ \ \ \ \ \ \ \ \ \ \ \ \ \ \ \ Blumen}}\textit{ habe ich }\textbf{\textit{kein-e }}\textbf{\textit{\textsubscript{\ }}}\textbf{\textit{groß-en}}\textit{ gekauft}.
\end{styleTextbody}

\begin{styleStandard}
\ \ \ \ multi.colored-\textsc{st} flowers \ have I \ \ \ \ no-\textsc{st} \ \ big-\textsc{wk} \ bought
\end{styleStandard}

\begin{styleStandard}
\ \ \ \ ‘As for multi-colored flowers, I have not bought any big ones.’
\end{styleStandard}

\begin{styleStandard}
Again, a movement analysis cannot account for the different endings on the two adjectives. This is particularly clear here since the underlyingly higher adjective (\textit{groß} ‘big’) is weak, but the underlyingly lower adjective (\textit{bunt} ‘multi-colored’) is strong. This issue is brought out very clearly on current assumptions: with the determiner having moved to the DP-level in the source, it is the lower element(s) that should appear with a weak ending (usually adjectives) and the higher one(s) with a strong ending (usually determiner(-like) elements). Indeed, all adjectives should have the same inflection in a nominal, later split up by movement.\footnote{\ There is another type of analysis involving movement out of the source. Tappe (1989) argues for a combination of different base-generations and movement. In particular, the split-off is base-generated in Spec,CP, and the source is base-generated in situ. The lower part of the source is proposed to move into the complement position of the split-off. The example in (ia) is derived as in (ib):\par (i)\ \ a.\ \ \textit{So*(}\textbf{\textit{’nen}}\textit{) }\textbf{\textit{Wagen}}\textit{ \ \ \ kann ich mir \ \ }\textbf{\textit{keinen}}\textit{ leisten}\par \ \ \ \ \ \ such \ a \ \ \ \ \ car.\textsc{masc} can \ \ I \ \ \ \textsc{refl} none \ \ \ afford\par \ \ \ \ \ \ ‘As for such a car, I cannot afford one.’\par \ \ \ \ b.\ \ [\textsubscript{DP} \textit{so’nen} [\textsubscript{NP }\textit{Wagen} ]\textsubscript{i} ] \textit{kann ich mir} [\textsubscript{DP} \textit{keinen} t\textsubscript{i} ]\textit{ leisten}\par This proposal leads Tappe to revise standard assumptions about chains, which raises other issues. } 
\end{styleStandard}

\begin{styleFooter}\itshape
3.2.\ \ Movement but not out of the Source
\end{styleFooter}

\begin{styleFooter}
Fanselow and Ćavar (2002) hypothesize that split topicalizations involve movement but crucially \textit{not} out of the DP that will surface in two different parts. As a technical implementation, they argue for a different type of account adopting the copy-and-delete approach to movement (Chomsky 1995). Moving the entire DP, they propose that deletion may affect \textit{both} copies. Glossing over some of the details here, they suggest that the determiner is deleted in the higher copy and the head noun in the lower one. This derives (18a) as in (18b):
\end{styleFooter}

\begin{styleStandard}
(18)\ \ \ \ a. \ \ \ \ \textbf{\textit{Wagen}}\textit{ \ \ \ \ hat \ er sich \ noch k-}\textbf{\textit{einen}}\textit{ \ leisten können}.
\end{styleStandard}

\begin{styleStandard}
\ \ \ \ car.\textsc{masc} has he \textsc{refl} yet \ \ \ \textsc{neg}{}-one afford \ could
\end{styleStandard}

\begin{styleStandard}
\ \ \ \ ‘As for a car, he has not been able to afford one.’
\end{styleStandard}

\begin{styleStandard}
\ \ \ \ b.\ \ \{einen Wagen\} hat er sich noch k- \{einen Wagen\} leisten können 
\end{styleStandard}

\begin{styleStandard}
Note that these authors treat \textit{keinen} ‘no/none’ as a composite form consisting of negative \textit{k}{}- and the indefinite article \textit{einen} (Chapter 5).
\end{styleStandard}

\begin{styleStandard}
\ \ At first glance, this analysis of distributed deletion seems to receive strong confirmation from the fact that the deletion of the higher copy of the determiner may be suspended yielding two instances of \textit{ein}:
\end{styleStandard}

\begin{styleStandard}
(19)\ \ \ \ a. \ \ \ \ \textbf{\textit{Einen Wagen}}\textit{ \ \ \ hat \ er sich \ noch k-}\textbf{\textit{einen}}\textit{ \ \ leisten können}.
\end{styleStandard}

\begin{styleStandard}
\ \ \ \ a \ \ \ \ \ \ \ \ car.\textsc{masc} has he \textsc{refl} yet \ \ \ \textsc{neg}{}-one afford \ could
\end{styleStandard}

\begin{styleStandard}
\ \ \ \ ‘As for a car, he has not been able to afford one.’
\end{styleStandard}

\begin{styleStandard}
\ \ \ \ b.\ \ \{einen Wagen\} hat er sich noch k- \{einen Wagen\} leisten können 
\end{styleStandard}

\begin{styleStandard}
However, upon closer inspection, it turns out that both determiners do not have to be the same (20a). In fact, when the determiner in the source is definite, the one in the split-off cannot be definite (20b):
\end{styleStandard}

\begin{styleStandard}
(20)\ \ \ \ a.\ \ \textbf{\textit{Einen Wagen}}\textit{ \ \ \ \ hat er \ sich \ nur \ }\textit{\textsubscript{\ }}\textbf{\textit{diesen}}\textit{ leisten können}.
\end{styleStandard}

\begin{styleStandard}
\ \ \ \ a \ \ \ \ \ \ \ \ car.\textsc{masc} has he \textsc{refl} only this \ \ \ \ afford \ could 
\end{styleStandard}

\begin{styleStandard}
\ \ \ \ ‘As for a car, he has only been able to afford this one.’
\end{styleStandard}

\begin{styleStandard}
\ \ \ \ b. \ *\ \ \textit{\{}\textbf{\textit{Diesen}}\textit{ / }\textbf{\textit{Den}}\textit{\} }\textbf{\textit{Wagen}}\textit{ \ \ \ hat }\textit{\textsubscript{\ }}\textit{er sich \ nur \ }\textbf{\textit{diesen}}\textit{ leisten können}.
\end{styleStandard}

\begin{styleStandard}
\ \ \ \ \ \  \ this \ \ \ \ \ / the \ \ \ \ car.\textsc{masc} has he \textsc{refl} only this \ \ \ \ afford \ could 
\end{styleStandard}

\begin{styleFooter}
Now, if a copy-and-delete type of analysis were correct, we would expect the grammaticality judgments in (20a-b) to be the reverse. It could be suggested though that this type of contrast is handled by repair rules. However, we see in Section 3.4 that late insertion of \textit{ein} in (20a) is unlikely to be correct. 
\end{styleFooter}

\begin{styleFooter}
\ \ Returning to the main line of argument, we can observe that this type of proposal is not compatible with the analysis of adjectival inflections in Chapter 2 either. Specifically, constructing a regular DP first (\textit{keine bunt-en Blumen} ‘no multi-colored{}-\textsc{wk} flowers’), we would expect the weak ending -\textit{en} on the adjective in the split-off, contrary to fact. Note that this would hold independently of whether \textit{keine} is treated as a composite (21b) or not (21c):
\end{styleFooter}

\begin{styleStandard}
(21)\ \ a.\ \ \textbf{\textit{Bunt-e \ \ \ \ \ \ \ \ \ \ \ \ \ \ \ \ Blumen}}\textit{ habe ich }\textbf{\textit{kein-e}}\textit{ \ \ \ gekauft}.
\end{styleStandard}

\begin{styleStandard}
\ \ \ \ multi.colored-\textsc{st} flowers \ have I \ \ \ \ none-\textsc{st} bought
\end{styleStandard}

\begin{styleStandard}
\ \ \ \ ‘As for multi-colored flowers, I have bought none.’
\end{styleStandard}

\begin{styleStandard}
\ \ b. \ (*)\ \ \{eine bunt-en Blumen\} habe ich k- \{ein-e bunten Blumen\} gekauft.
\end{styleStandard}

\begin{styleStandard}
\ \ c. \ (*)\ \ \{keine bunt-en Blumen\} habe ich \{kein-e bunten Blumen\} gekauft.
\end{styleStandard}

\begin{styleFooter}
As mentioned already above, a surface-oriented explanation of the strong ending is not sufficient to explain the whole range of facts. In view of these and some other issues (e.g., the licensing of Negative Polarity Items, see Bosse 2009: 278), a different technical implementation of split topicalizations is called for.
\end{styleFooter}

\begin{styleStandard}\itshape
3.3.\ \ Separate Base-generation
\end{styleStandard}

\begin{styleStandard}
Some alternative analyses take as their point of departure Fanselow and Ćavar’s (2002) generalization that split topicalizations involve movement but crucially \textit{not} out of the DP that surfaces in two separate parts. Specifically, these proposals involve the base-generation of two separate nominals in the VP and moving one of them (or both) to the left. I agree with this assessment, and as far as I am aware, a consensus seems to be emerging that this is indeed the correct characterization of the facts. There have been several proposals to make this idea concrete.
\end{styleStandard}

\begin{styleStandard}
Specifically, Bosse (2009) proposes an analysis where she links split topicalizations to Restrictive Elliptical Appositives (van Riemsdijk 1998b). Ott (2011a) argues for an account that involves breaking up a symmetric bare-predication structure. Third, Roehrs (2011) also argues for the base-generation of two separate nominals. All three analyses are compatible with the discussion in Chapter 2. Given that the third analysis involves fewer new assumptions, I illustrate this proposal here in some detail.
\end{styleStandard}

\begin{styleStandard}
Basing the following account on Fanselow (1988), Roehrs (2011) proposes that there is a division of labor between the syntax and the semantics. In particular, it is suggested in that paper that split topicalizations involve the separate base-generation of an argumental DP and a related predicative nominal in the same local domain, the VP. The argumental part contains an empty noun (e\-\textsubscript{N}). The initial stage of the derivation of (22a) is illustrated in (22b):
\end{styleStandard}

\begin{styleStandard}
(22)\ \ a.\ \ \textbf{\textit{Bunt-e \ \ \ \ \ \ \ \ \ \ \ \ \ \ \ \ Blumen}} \textit{habe ich }\textbf{\textit{kein-e}}\textit{ \ \ \ gekauft}.
\end{styleStandard}

\begin{styleStandard}
\ \ \ \ multi.colored-\textsc{st} flowers \ have I \ \ \ \textsubscript{\ \ }none-\textsc{st} bought
\end{styleStandard}

\begin{styleStandard}
\ \ \ \ ‘As for multi-colored flowers, I have bought none.’
\end{styleStandard}

\begin{styleStandard}
\ \ b.\ \ \ \  \ \ \ \ \ VP
\end{styleStandard}

\begin{styleStandard}
[Warning: Draw object ignored][Warning: Draw object ignored]
\end{styleStandard}

\begin{styleStandard}
\ \ \ \ AgrP\ \ \ \ \ \  V’
\end{styleStandard}

\begin{styleStandard}
\textit{\ \ }[Warning: Draw object ignored][Warning: Draw object ignored]\textit{\ \ \ bunte Blumen}
\end{styleStandard}

\begin{styleStandard}
\ \ \ \ \ \ \ \ DP\ \ \ \  V
\end{styleStandard}

\begin{styleStandard}
\textit{\ \ \ \ \ \ \ \ keine e}\textit{\textsubscript{N}}\textit{ \ \  \ \ \ \ \ \ \ gekauft}
\end{styleStandard}

\begin{styleStandard}
Both the predicate split-off and the argumental source undergo movement to the left: AgrP undergoes topicalization to Spec,CP; DP moves for case. This yields the correct word order. Assuming that the overt nominal in the split-off and e\textsubscript{N} in the source are of the same semantic type, the “unsaturated” predicate in Spec,CP is closed off by interpreting it in e\textsubscript{N} of the argumental DP filling e\textsubscript{N} with semantic content at the same time. 
\end{styleStandard}

\begin{styleStandard}
Returning to the issue of inflection, both nominals involve regular structures where the adjective in the split-off is in Spec,AgrP. It is clear though that the adjective is in a nominal different from that of the determiner, and as such Impoverishment does not occur. As a consequence, the CNG features on the adjective are not reduced, and the adjective surfaces with a strong ending. This gives the desired result (see also Chapter 2, Section 3.6). More generally, proposals involving separate base-generation are fully compatible with the analysis in Chapter 2. I briefly return to the discussion of \textit{ein}.
\end{styleStandard}

\begin{styleStandard}
\textit{3.4.\ \ }Ein\textit{ in the Split-off}
\end{styleStandard}

\begin{styleStandard}
We saw in Section 2 that \textit{ein} is not entirely spurious in that it triggers a weak ending on a following adjective. This means that \textit{ein}, or more precisely its feature bundles, cannot be inserted late in a high position. There is independent evidence for this conclusion. It derives from indefinite pronoun constructions occurring as split topicalizations.
\end{styleStandard}

\begin{styleStandard}
\ \ As illustrated in Chapters 2 and 3, adjectives can follow an indefinite pronoun (23a). Interestingly, an adjective preceded by an indefinite article cannot (23b):
\end{styleStandard}

\begin{styleStandard}
(23) \ \ a.\ \ \textit{etwas \ \ \ \ \ \ \ \ \ \ \ \ \ \ \ \ \ Amerikanisches}
\end{styleStandard}

\begin{styleStandard}
\ \ \ \ something.\textsc{neut} American
\end{styleStandard}

\begin{styleStandard}
\ \ \ \ ‘something American’
\end{styleStandard}


\setcounter{listWWviiiNumxleveli}{1}
\begin{listWWviiiNumxleveli}
\item 
\begin{styleStandard}
*\ \ \textit{etwas \ \ \ \ \ \ \ \ \ \ \ \ \ \ \ \ \ ein Amerikanisches}
\end{styleStandard}
\end{listWWviiiNumxleveli}
\begin{styleStandard}
\ \ \ \ something.\textsc{neut} an \ American
\end{styleStandard}

\begin{styleStandard}
At first glance, we might claim that this is a phonetic restriction such that the pronoun and \textit{ein} cannot be adjacent. However, consistent with (23b), split topicalizations formed on the indefinite pronoun construction cannot involve \textit{ein} in the split-off either. Compare (24a) to (24b):
\end{styleStandard}

\begin{styleStandard}
(24)\ \ a.\ \ \textit{(}\textbf{\textit{Ein}}\textit{) }\textbf{\textit{amerikanisches}}\textit{ hat er sich \ \ nur \ \ }\textbf{\textit{eins}}\textit{ leisten können}. \ \ \ \ \ \ \ 
\end{styleStandard}

\begin{styleStandard}
\ \ an \ \ \ American \ \ \ \ \ \ \ \ \ \textsubscript{\ }has he \textsc{refl} only one \ afford \textsubscript{\ }could
\end{styleStandard}

\begin{styleStandard}
‘As for American ones, he has been able to afford only one.’
\end{styleStandard}

\begin{styleStandard}
\ \ b.\ \ \textit{(*}\textbf{\textit{Ein}}\textit{) }\textbf{\textit{Amerikanisches}}\textit{ hat er }\textit{\textsubscript{\ }}\textit{sich \ }\textit{\textsubscript{\ }}\textbf{\textit{etwas}}\textit{ \ \ \ \ \ \ \ \ \ \ \ \ \ \ \ \ \ leisten können}.
\end{styleStandard}

\begin{styleStandard}
\ \ \ \  \ \ an \ \ \ \ American \ \ \ \ \ \ \ \ \ \ \textsubscript{\ }has he \textsc{refl} something.\textsc{neut} afford could
\end{styleStandard}

\begin{styleStandard}
\ \ \ \ ‘As for American stuff, he has been able to afford something.’
\end{styleStandard}

\begin{styleStandard}
Considering \textit{nichts (*ein) Amerikanisches} ‘nothing American’, the same facts hold if the source involves the negative counterpart of \textit{etwas} ‘something’:
\end{styleStandard}

\begin{styleStandard}
(25)\ \ a.\ \ \textit{(}\textbf{\textit{Ein}}\textit{) }\textbf{\textit{amerikanisches}}\textit{ hat er sich \ \ noch }\textbf{\textit{keins}}\textit{ leisten können}. \ \ \ \ \ \ \ 
\end{styleStandard}

\begin{styleStandard}
\ \ an \ \ \ American \ \ \ \ \ \ \ \ \ \textsubscript{\ }has he \textsc{refl} yet \ \ \ none \ afford could
\end{styleStandard}

\begin{styleStandard}
‘As for American ones, he has not been able to afford one yet.’
\end{styleStandard}

\begin{styleStandard}
\ \ b.\ \ \textit{(*}\textbf{\textit{Ein}}\textit{) }\textbf{\textit{Amerikanisches}}\textit{ hat er }\textit{\textsubscript{\ }}\textit{sich \ }\textit{\textsubscript{\ }}\textit{noch }\textbf{\textit{nichts}}\textit{ \ \ \ \ \ \ \ \ \ \ \ leisten können}.
\end{styleStandard}

\begin{styleStandard}
\ \ \ \  \ \ an \ \ \ \ American \ \ \ \ \ \ \ \ \ \ \textsubscript{\ }has he \textsc{refl} yet \ \ \textsubscript{\ }nothing.\textsc{neut} afford \ could
\end{styleStandard}

\begin{styleStandard}
\ \ \ \ ‘As for American stuff, he has not been able to afford anyething.’
\end{styleStandard}

\begin{styleStandard}
In view of the ungrammaticality induced by \textit{ein} in (24b) and (25b), a phonetic restriction cannot explain all the ungrammatical cases: unlike in (23b), the pronoun and \textit{ein} are neither adjacent in (24b) nor in (25b). Note in this respect that late insertion of \textit{ein} does not help explain the ungrammaticality in a surface-oriented account: if \textit{ein} were inserted late (i.e., in PF), the local context of \textit{ein} inside the split-off in the grammatical (24a) and (25a) and in the ungrammatical (24b) and (25b) would be the same; that is, \textit{ein} would precede the adjectives (and be separated from the pronouns) in both the grammatical and ungrammatical cases of (24) and (25). Consequently, this cannot explain the difference between the (a)-examples and the (b)-examples above. Rather, the difference between these cases seems to be located in the source: while \textit{eins} ‘one’ and \textit{keins} ‘none’ are pronominal forms related to \textit{ein} ‘a’ and \textit{kein} ‘no’, \textit{etwas} ‘something’ and \textit{nichts} ‘nothing’ are inherent pronouns.
\end{styleStandard}

\begin{styleStandard}
Above, I suggested that the split-off is related to the source by interpreting the former inside the latter. Note that interpretation occurs at LF (and not PF). Now, the pronominal forms (\textit{eins}, \textit{keins}) presumably involve an internal makeup different from that of the inherent pronouns (\textit{etwas}, \textit{nichts}). While there are several options to implement this idea, it seems clear that interpreting the split-off containing \textit{ein} inside the source is compatible with the makeup of the pronominal forms but that this is not compatible with the makeup of the inherent pronouns.\footnote{\ One idea to make this more concrete is to assume that DPs and smaller elements can be interpreted in e\textsubscript{N} of the pronominal forms but that DPs are too large to be interpreted inside the inherent pronouns – only smaller elements, the ones lacking \textit{ein}, can do so. Alternatively, in the second part of the book, I argue that \textit{ein} flags the presence of a null operator. If so, it might be claimed that it is actually this null operator that is incompatible with the inherent pronouns.} If this is on the right track, then it seems clear that the presence of \textit{ein} cannot be a late phenomenon in the ungrammatical (24b) and (25b) – its presence at LF helps explain the ungrammaticality. Note in this regard that the presence of an element at LF implies the presence of that element earlier in the syntactic derivation. If so, then the occurrence of \textit{ein} in the grammatical (24a) and (25a) cannot involve a PF phenomenon on parity of assumption. Returning briefly to (23b), recall from Chapter 2, Section 2.3 that indefinite pronoun constructions involve adjunction mediated by ModP. With this in mind, the ungrammaticality of this example could be explained by assuming that the head Mod selects AgrP, which excludes preceding \textit{ein}.
\end{styleStandard}

\begin{styleStandard}
To sum up, I have demonstrated that split topicalizations are only compatible with the analysis of Chapter 2 if they involve two separate base-generations (but not if they involve movement out of the DP or a copy-and-delete approach). Next, I briefly discuss adjectival inflections as regards semantic concepts. Specifically, I discuss inflections on adjectives as regards the restrictive and non-restrictive interpretation of adjectives, and I examine inflections on determiners as regards “referentiality”. In keeping with the current discussion, we will see that independently of their (non-)restrictive interpretation, inflected adjectives are in their regular positions (i.e., Spec,AgrP) and that the inflections themselves make no semantic contribution in German.
\end{styleStandard}

\begin{styleFootnote}\bfseries
4. \ \ Weak Adjectives with Non-restrictive Interpretation
\end{styleFootnote}

\begin{styleFootnote}
As documented in detail in Chapter 2, adjectives are usually weak if they follow a determiner. Note that the adjective in these types of strings can actually have two interpretations, restrictive or non-restrictive (the latter is often called appositive). Considering the noun phrases in (26a-b), the restrictive interpretation of the adjective is given in the translation line in (26a), and the non-restrictive reading is provided in that of (26b). For clarity, I use a relative clause to translate the adjective. The non-restrictive interpretation is clearly brought out by the addition of \textit{übrigens} ‘by the way’ and is sometimes called a by-the-way remark. To repeat, the adjectives are weak in both contexts:\footnote{\ In other languages, the strong/weak alternation of the adjective does indicate a difference in (non-)restrictive interpretation. In Icelandic, adjectives with a weak ending are restrictive in interpretation (ia), but adjectives with a strong ending are not restrictive (ib) (see Delsing 1993: 132 fn. 25; Thráinsson 1994: 166, 2007: 3, 89; Sigurðsson 2006: 200 fn. 3): \par (i)\ \ a.\ \ \textit{gul-i \ \ \ \ \ \ \ \ \ \ \ bill-inn}\ \ \ \ \ \ \ \ (Icelandic)\par \ \ \ \ \ \ wellow\textsc{{}-wk} car-\textsc{def}\par \ \ \ \ \ \ ‘the car that is yellow’\par \ \ \ \ b.\ \ \textit{gul-ur \ \ \ \ \ \ bill-inn}\par \ \ \ \ \ \ yellow-\textsc{st} car-\textsc{def}\par \ \ \ \ \ \ ‘the car, which is yellow’\par For recent discussion, see also Pfaff (2017). As regards inflection and position of prenominal adjectives, note also Evans (2021), who discusses a correlation of inflected and uninflected adjectives with different interpretations in Dutch. He proposes that inflected and uninflected adjectives are in different positions yielding the different interpretations.}
\end{styleFootnote}

\begin{styleFootnote}
(26)\ \ a.\ \ \textit{der alt-e \ \ \ \ }\textit{\textsubscript{\ }}\textit{Mann}
\end{styleFootnote}

\begin{styleFootnote}
\ \ \ \ the \textsubscript{\ }old\textsc{{}-wk} man.\textsc{masc}
\end{styleFootnote}

\begin{styleFootnote}
‘the man that is old’
\end{styleFootnote}

\begin{styleFootnote}
b.\ \ \textit{der (übrigens) \ \ \ }\textit{\textsubscript{\ }}\textit{alt-e \ \ \ \ Mann}
\end{styleFootnote}

\begin{styleFootnote}
\ \ \ \ the \ \textsubscript{\ }incidentally old-\textsc{wk} man.\textsc{masc}
\end{styleFootnote}

\begin{styleFootnote}
‘the man, who is (by the way) old’
\end{styleFootnote}

\begin{styleFootnote}
With the above analysis of weak endings in mind, this implies that both restrictive and non-restrictive interpretations involve the same basic (regular) structure in which Impoverishment has occurred. In other words, inflected non-restrictive adjectives cannot be inserted late, that is, after Impoverishment has occurred. Rather, I suggest that the structure of non-restrictive adjectives involves an additional lexical element.
\end{styleFootnote}

\begin{styleFootnote}
\ \ In Roehrs (2009a: 104), I propose that the main difference between the restrictive and the non-restrictive interpretation of adjectives involves the absence or presence of a null \textit{pro }co-indexed with the hosting DP. This can be schematically represented as follows:
\end{styleFootnote}

\begin{styleFootnote}
(27)\ \ [ \textit{der} [ (pro\textsubscript{i}) \textit{alte} ] \textsubscript{\ }\textit{Mann }]\textsubscript{(i)}
\end{styleFootnote}

\begin{styleFootnote}
\ \  \ \textsubscript{\ }the \ \ \ \ \ \ \ \ \ \ \ \ old \ \ \ \ man.\textsc{masc}
\end{styleFootnote}

\begin{styleFootnote}
In order to account for the uniform inflectional behavior of the adjectives, it is important to be more precise about the relevant structures and derivations. With the focus on the morpho-syntax, I only briefly discuss the semantics (for the denotations, see Roehrs 2009a: Chapter 3). I start with the case involving a restrictive interpretation. 
\end{styleFootnote}

\begin{styleFootnote}
Recall from Chapter 2 that the determiner moves from below the adjective to the DP-level (28). Given this structure and the presence of a determiner, Impoverishment occurs bringing about the weak ending on the adjective. As to the semantics, I assume that kind nouns are of type {\textless}e{\textgreater}. Following de Swart \textit{et al}. (2007), I argue in Chapter 6 that a kind noun combines with the realization operator REL (type {\textless}e{\textless}e,t{\textgreater}{\textgreater}) to yield a predicate (type {\textless}e,t{\textgreater}). This predicate combines with the adjective by Predicate Modification. The resulting conjunction of the two predicates is combined with the determiner (type {\textless}{\textless}e,t{\textgreater}e{\textgreater}) by Functional Application yielding an individual (type {\textless}e{\textgreater}):
\end{styleFootnote}

\begin{styleFootnote}
(28)\ \ \textit{Restrictive Interpretation}
\end{styleFootnote}

\begin{styleFootnote}
\ \ \ \ DP\textsubscript{{\textless}e{\textgreater}}
\end{styleFootnote}

\begin{styleFootnote}
[Warning: Draw object ignored][Warning: Draw object ignored]\ \ \ \ \ \ \ \ \ \ (Functional Application)
\end{styleFootnote}

\begin{styleFootnote}
\ \ \textit{der}\textsubscript{{\textless}{\textless}e,t{\textgreater}e{\textgreater}i} \ \ \ \ \ \ \ AgrP\textsubscript{{\textless}e,t{\textgreater}}
\end{styleFootnote}

\begin{styleFootnote}
[Warning: Draw object ignored][Warning: Draw object ignored]\ \ \ \ \ \ \ \ \ \ \ \ (Predicate Modification)
\end{styleFootnote}

\begin{styleFootnote}
\ \ \ \ \textit{alte}\textsubscript{{\textless}e,t{\textgreater}}\ \  \ \ \ \ \ \ \ \ \ ArtP\textsubscript{{\textless}e,t{\textgreater}}
\end{styleFootnote}

\begin{styleFootnote}
[Warning: Draw object ignored][Warning: Draw object ignored]\ \ 
\end{styleFootnote}

\begin{styleFootnote}
\ \ \ \ \ \ \textit{der}\textsubscript{i}\ \  \ \ \ \ \ \ \ \ \ NumP\textsubscript{{\textless}e,t{\textgreater}}
\end{styleFootnote}

\begin{styleFootnote}
[Warning: Draw object ignored][Warning: Draw object ignored]\ \ \ \ \ \ \ \ \ \ \ \ \ \ (Application of REL)
\end{styleFootnote}

\begin{styleFootnote}
\ \ \ \ \ \  \ \ \ \ \ \ Num\textsubscript{REL{\textless}e{\textless}e,t{\textgreater}{\textgreater}}\ \ NP
\end{styleFootnote}

\begin{styleFootnote}
\ \ \ \ \ \ \ \  \ \  \ \ \ \ \ \ \ \ \ \textit{Mann}\textsubscript{{\textless}e{\textgreater}}
\end{styleFootnote}

\begin{styleFootnote}
The structure of the non-restrictive reading is basically the same. Unlike above, however, the determiner is interpreted in its base-position; that is, the adjective is outside the scope of the determiner (29) (see also Partee 1973: 54, Schoorlemmer 2009). Proceeding bottom-up and much like above, the kind noun combines with the realization operator REL to yield a predicate. Unlike above, the low copy of the determiner combines with the predicate noun yielding an element of type {\textless}e{\textgreater}. In order for this resultant nominal to combine with a clausal predicate (type {\textless}e,t{\textgreater}), I adopt a model of multiple semantic spell-out, whereby the higher by-the-way remark is sent off for interpretation separately (for details, see Roehrs 2009a: 102-06). Structurally, this remark consists of the adjective and \textit{pro}, indicated by square brackets below. These two elements combine by Functional Application yielding a truth value (I continue the discussion of this subcomponent further below). After this element is sent off to Spell-out, the remaining nominal can combine with a clausal predicate (not shown here): 
\end{styleFootnote}

\begin{styleFootnote}
(29)\ \ \textit{Non-restrictive Interpretation}
\end{styleFootnote}

\begin{styleFootnote}
\ \ \ \ DP\textsubscript{{\textless}e{\textgreater}}
\end{styleFootnote}

\begin{styleFootnote}
[Warning: Draw object ignored][Warning: Draw object ignored]\ \ \ \ \ \ \ \ \ \ 
\end{styleFootnote}

\begin{styleFootnote}
\ \ \textit{der}\textsubscript{i} \ \ \ \ \ \ \ \ \ \ \ \ \ \ \ \ AgrP\textsubscript{{\textless}e{\textgreater}}
\end{styleFootnote}

\begin{styleFootnote}
[Warning: Draw object ignored][Warning: Draw object ignored]\ \ \ \ \ \ \ \ \ \ 
\end{styleFootnote}

\begin{styleFootnote}
\ \ \ \ \ \ \ [pro\textsubscript{{\textless}e{\textgreater}} \textit{alte}\textsubscript{{\textless}e,t{\textgreater}}]\textsubscript{{\textless}t{\textgreater}} \ \ \ \ \ \ \ \ \ \ ArtP\textsubscript{{\textless}e{\textgreater}}
\end{styleFootnote}

\begin{styleFootnote}
[Warning: Draw object ignored][Warning: Draw object ignored][Warning: Draw object ignored]\ \ \ \ \ \ \ \ \ \ \ \ \ \ (Functional Application)
\end{styleFootnote}

\begin{styleFootnote}
\ \ \ \ \ \ \ \ sem. spell-out \ \ \ \textit{der}\textsubscript{{\textless}{\textless}e,t{\textgreater}e{\textgreater}i} \ \ \ \ \ \ NumP\textsubscript{{\textless}e,t{\textgreater}}
\end{styleFootnote}

\begin{styleFootnote}
[Warning: Draw object ignored][Warning: Draw object ignored]\ \ \ \ \ \ \ \ \ \ \ \ \ \ (Application of REL)
\end{styleFootnote}

\begin{styleFootnote}
\ \ \ \ \ \  \ \ \ \ \ \ Num\textsubscript{REL{\textless}e{\textless}e,t{\textgreater}{\textgreater}}\ \ NP
\end{styleFootnote}

\begin{styleFootnote}
\ \ \ \ \ \ \ \  \ \  \ \ \ \ \ \ \ \ \ \textit{Mann}\textsubscript{{\textless}e{\textgreater}}
\end{styleFootnote}

\begin{styleFootnote}
Recalling that \textit{pro} and \textit{der Mann} ‘the man’ are coindexed, the expression involving \textit{pro} and the adjective is interpreted as an additional, non-restrictive remark about \textit{der Mann}, something along the lines of ‘the man – he is old’. To be clear then, the determiner in both interpretations has moved to the DP-level in syntax triggering Impoverishment. However, the determiner is interpreted in different positions at LF. Consider the internal structure of the non-restrictive modifier in more detail.
\end{styleFootnote}

\begin{styleFootnote}
\ \ In Chapter 2, Section 2.4, I argued that adjectives involve extended projections. In particular, I suggested there that they consist of an adjective stem at the bottom and an inflectional head at the top. Note that adjectives can take dependents, for instance, an assumed \textit{pro}. In order for the stem to combine with its inflection, the stem along with its dependents moves to Spec,InflP. The minimal structure is as follows: 
\end{styleFootnote}

\begin{styleStandard}
(30)\textit{ \ \ }\ \ \ \  \ \ \ \ \ InflP
\end{styleStandard}

\begin{styleFootnote}\itshape
[Warning: Draw object ignored][Warning: Draw object ignored]
\end{styleFootnote}

\begin{styleFootnote}
\ \ \ \ AP\textsubscript{{\textless}t{\textgreater}k}\ \ \ \ \ \ Infl’
\end{styleFootnote}

\begin{styleFootnote}
[Warning: Draw object ignored][Warning: Draw object ignored][Warning: Draw object ignored][Warning: Draw object ignored]\ \ \ \ \ \ \ \  \ \ \ \ \ \ \ \ (Functional Application)
\end{styleFootnote}

\begin{styleFootnote}
pro\textsubscript{{\textless}e{\textgreater}}\ \  \ \ \ \ \ \ \ \ \textit{alt}\textsubscript{{\textless}e,t{\textgreater}} \ \ {}-\textit{e}\ \ \ \  \ t\textsubscript{k}
\end{styleFootnote}

\begin{styleFootnote}
Semantically, the pronominal element \textit{pro} (type {\textless}e{\textgreater}) combines with the adjective (type {\textless}e,t{\textgreater}) by Functional Application yielding a truth value (type {\textless}t{\textgreater}). This is the desired result for the by-the-way remark ‘he is old’.
\end{styleFootnote}

\begin{styleFootnote}
Observe that independently of the presence of \textit{pro}, Impoverishment can occur in a local fashion, namely on the highest head of the extended projection of the adjective (Infl). In other words, given this structure, Impoverishment is compatible with different interpretations of the adjective. If this is accepted, then I can continue claiming that weak adjectival inflections indicate one and the same structural constellation and are independent of the semantics (Hypothesis 1a). Finally, I turn to a case where an adjectival inflection seems to be related to “referentiality”.
\end{styleFootnote}

\begin{styleStandard}\bfseries
5.\ \ Adjectival Inflections Make Nominal Features Visible
\end{styleStandard}

\begin{styleFootnote}
Adjectival inflections occur on determiners and adjectives but not, for instance, on verbs. It is clear that they are nominal in nature. I argued above that adjectival inflections are semantically vacuous. I demonstrated that adjectival inflections do not indicate, or are a reflex of, (in-)definiteness or (non-)restrictiveness in interpretation. Considering certain other data, there is another possible semantic concept they could be associated with.
\end{styleFootnote}

\begin{styleFootnote}
\ \ Alexiadou \textit{et al.} (2011: 31) point out that inflected \textit{dieses} ‘this’ is an anaphor for noun phrases only, but uninflected \textit{dies} ‘this’ is an anaphor for both noun phrases and clauses. Compare (31a) to (31b) (data slightly adapted; see also Helbig \& Buscha 2001: 229-30):
\end{styleFootnote}

\begin{styleFootnote}
(31)\ \ a.\ \ \textit{Hans hat ein rotes Buch. \ \ \ \ \ \ \ \ \{Dies-es / Dies\} war sehr }\textit{\textsubscript{\ }}\textit{teuer}.
\end{styleFootnote}

\begin{styleFootnote}
\ \ \ \ Hans has a \ \ \ red \ \ \ book.\textsc{neut}. this-\textsc{st} \ \ \ \ \ \ \ \ \ \ \ \ \ was very expensive
\end{styleFootnote}

\begin{styleFootnote}
\ \ \ \ ‘Hans has a red book. It was very expensive.’
\end{styleFootnote}

\begin{styleFootnote}
\ \ b.\ \ \textit{Daß Maria bereits \ angekommen ist, \{*dieses / dies\} weiß \ }\textit{\textsubscript{\ }}\textit{ich genau}.
\end{styleFootnote}

\begin{styleFootnote}
\ \ \ \ that \textsubscript{\ }Mary \ \textsubscript{\ }already arrived \ \ \ \ \ \ \ \ \ \ is \ \ \ \ \ \textsubscript{\ }this \ \ \ \ \ \ \ \ \ \ \ \ \ \ \ know I \ \ \ \ well
\end{styleFootnote}

\begin{styleFootnote}
\ \ \ \ ‘I am positive that Mary has already arrived.’
\end{styleFootnote}

\begin{styleFootnote}
Given the data above, we might suggest that the two demonstrative forms are semantically different in that inflected \textit{dieses} must have a noun phrase as its antecedent, but uninflected \textit{dies} is less restricted and tolerates both a noun phrase and a clause as its antecedent. As these two elements differ in inflection, we might further suggest that adjectival endings have or are sensitive to the semantics after all. In particular, since inflected \textit{dieses} can only refer back to noun phrases, we could claim that inflectional endings are, in some sense, “referential”. I think this possible interpretation of the facts is not correct. 
\end{styleFootnote}

\begin{styleFootnote}
\ \ In Chapter 3, Section 3, I discussed adjectives like \textit{lila} ‘purple’ and \textit{rosa} ‘pink’. I pointed out that the inflections on these adjectives are optional when a noun follows (32a) but that they are obligatory when such a noun is absent (32b). Above, I argued that the source of split topicalization structures contains a null noun, which is licensed by the predicate split-off. With this in mind, I suggest that (32b) also involves a null noun, here licensed by the anaphoric relation with the noun in the previous sentence (Fanselow 1988: 101, Murphy 2018):
\end{styleFootnote}

\begin{styleFootnote}
(32)\ \ a.\ \ \textit{ein lila(-nes) \ \ \ \ Kleid}
\end{styleFootnote}

\begin{styleFootnote}
\ \ \ \ a \ \ \ purple-\textsc{infl} dress.\textsc{neut}
\end{styleFootnote}

\begin{styleFootnote}
\ \ \ \ ‘a purple dress’
\end{styleFootnote}

\begin{styleFootnote}
b.\ \ \textit{Da \ \ \ waren viele \ bunte \ \ \ \ \ \ \ \ \ \ \ \ \ Kleider. Ich} \textit{kaufte \ ein lila*(-nes).}
\end{styleFootnote}

\begin{styleFootnote}
\ \ \ \ there were \ \ many multi-colored dresses \ \ I \ \ \ bought a \ \ \ purple-\textsc{infl} 
\end{styleFootnote}

\begin{styleFootnote}
\ \ \ \ ‘There were many multi-colored dresses. I bought a purple one.’
\end{styleFootnote}

\begin{styleFootnote}
Notably, indeclinable elements never have endings including in anaphoric contexts. This can be illustrated with the numeral for ‘ten’:
\end{styleFootnote}

\begin{styleFootnote}
(33)\ \ a.\ \ \textit{zehn Kleider}
\end{styleFootnote}

\begin{styleFootnote}
\ \ \ \ \ ten \ \ dresses
\end{styleFootnote}

\begin{styleFootnote}
\ \ \ \ ‘ten dresses’
\end{styleFootnote}

\begin{styleFootnote}
\ \ b.\ \ \textit{Da \ \ \ waren viele} \ \textit{Kleider. Ich kaufte \ zehn}.
\end{styleFootnote}

\begin{styleFootnote}
\ \ \ \ \ there were \ \ many dresses \ I \ \ \ \ bought ten
\end{styleFootnote}

\begin{styleFootnote}
\ \ \ \ ‘There were many dresses. I bought ten.’
\end{styleFootnote}

\begin{styleFootnote}
The distinction of declinability makes \textit{lila}{}- ‘purple’ in (32) and \textit{zehn} ‘ten’ in (33) different. The generalization that seems to be emerging is that adjectival inflections are obligatory when the hosting elements are declinable and occur in anaphoric contexts; indeclinable elements remain uninflected in anaphoric contexts. If this holds more generally, then there is an interesting consequence for the two forms of the demonstrative \textit{dies} and \textit{dieses}. 
\end{styleFootnote}

\begin{styleFootnote}
At first glance, this demonstrative appears to be similar to the adjectives \textit{lila} and \textit{rosa} – when overt material follows, the inflection is optional (34). Note though that uninflected \textit{dies} can only occur in nominative/accusative contexts, both as a determiner (34a) or as a predeterminer (34b):\footnote{\ Recall that this restriction makes \textit{dies(es)} ‘this’ different from \textit{all(e) }‘all’. The inflection of the latter is not subject to a restriction in morphological case (see again Chapter 3, Section 6).}
\end{styleFootnote}

\begin{styleFootnote}
(34)\ \ a.\ \ \textit{dies(-es) \ schön-e \ \ \ Kleid}
\end{styleFootnote}

\begin{styleFootnote}
\ \ \ \ this-\textsc{infl} pretty-\textsc{wk} dress.\textsc{neut}
\end{styleFootnote}

\begin{styleFootnote}
\ \ \ \ ‘this pretty dress’
\end{styleFootnote}

\begin{styleFootnote}
\ \ b.\ \ \textit{dies(-es) \ mein Glück}.
\end{styleFootnote}

\begin{styleFootnote}
\ \ \ \ this-\textsc{infl} my \ \ \ happiness.\textsc{neut}
\end{styleFootnote}

\begin{styleFootnote}
\ \ \ \ ‘this happiness of mine’
\end{styleFootnote}

\begin{styleFootnote}
There are two differences to the adjectives above. First, both the inflected and the uninflected forms of the demonstrative are possible in anaphoric contexts involving noun phrases as in (31a). Note that the noun phrase in (31a) is in the neuter gender. Second, the demonstrative is also possible in anaphoric contexts involving clauses, but here only the uninflected form appears as in (31b). If the generalization above is correct, then we cannot say that the demonstrative involves one element with an optional inflection. If this were so, then we would expect only the inflected form \textit{dieses} to occur in anaphoric contexts, contrary to fact. In order to explain the occurrence of both the inflected \textit{and} the uninflected form of the demonstrative in (31a), we have to assume that there are actually two independent elements, a declinable \textit{dieses} and an indeclinable \textit{dies}. In other words, we cannot simply assume that \textit{dies} is based on \textit{dieses} such that the ending has been deleted in PF (for this idea, see Gallmann 2004: 154 fn. 3, Roehrs 2009a: 159 fn. 33, cf. also G. Müller 2002a: 117 fn. 8). Note that the proposal of two separate elements is consistent with the aforementioned fact that unlike inflected \textit{dieses}, uninflected \textit{dies} is restricted to nominative/accusative neuter contexts. 
\end{styleFootnote}

\begin{styleFootnote}
The structure of inflected \textit{dieses} is repeated in (35a); the relevant vocabulary insertion rules are given in (35b-d):
\end{styleFootnote}

\begin{styleStandard}
\ (35)\ \ \textit{Inflected }dieses
\end{styleStandard}

\begin{styleJBExample}
\ \ a. \ \ \ \ \ \ \  \ \ \ \ InflP\textsubscript{[+D; +DEF, +DEIX][F, N, O, S]}
\end{styleJBExample}

\begin{styleStandard}\bfseries
[Warning: Draw object ignored][Warning: Draw object ignored]
\end{styleStandard}

\begin{styleStandard}
\ \ \ \ \ \ \ \ \ \ [F, N, O, S]\ \ \ \ DemP
\end{styleStandard}

\begin{styleStandard}
\ \ \ \ \ \ \ \  \ \ \ {\textbar} \ \ \ \ \ \ 
\end{styleStandard}

\begin{styleStandard}
\ \ \ \ \ \ \ \ Dem
\end{styleStandard}

\begin{styleStandard}
\ \ \ \  \ \ \ \ \ \ \ \ \ [+D; +DEF, +DEIX]
\end{styleStandard}

\begin{styleStandard}
\ \ b.\ \ [+D; +DEF, +DEIX]\ \ →\ \ \textit{dies-}
\end{styleStandard}

\begin{styleStandard}
c.\ \ [+F, -N, +O, $\alpha $S] \ \ →\ \ \textit{{}-er}
\end{styleStandard}

\begin{styleStandard}
\ \ d.\ \ etc.
\end{styleStandard}

\begin{styleFootnote}
Recall that the stem \textit{dies}{}- spells out [+D; +DEF, +DEIX] by (35b) and that the varying inflections realize the CNG feature bundle in Infl (by 35c-d). As for uninflected \textit{dies}, I propose that this element is like the first and second-person pronominals discussed in Chapter 3, Section 5. As such, I assume that \textit{dies} involves a reduced demonstrative structure (36). The restriction of \textit{dies} to nominative/accusative neuter contexts can be captured by – what I call here – restriction features. These types of features are on [+DEIX], indicated below by the features following the colon sign after [+DEIX]. Specifically, on the one hand, the features [-F, +N, -O] are only compatible with the structure (36) being inserted in nominative/accusative neuter contexts; on the other hand, this type of deictic feature involving a restriction only projects a reduced structure:\footnote{\ These types of restriction features are discussed in more detail in Chapter 8, Sections 2.2.5 and 2.2.6. Among others, they account for the varying distribution of \textit{ein} in regular singular contexts (e.g., \textit{ein Auto} ‘a car’) vs. in plural contexts (e.g., \textit{m-eine Autos} ‘my cars’).}
\end{styleFootnote}

\begin{styleStandard}
(36)\ \ \textit{Uninflected }dies
\end{styleStandard}

\begin{styleStandard}
\ \ \ \ \ \ \ \ \ \ \ \ \ \ \ \ DemP\textsubscript{[+D; +DEF, +DEIX]}
\end{styleStandard}

\begin{styleStandard}
\ \ \ \ \ \  \ \ \ {\textbar} \ \ \ \ \ \ 
\end{styleStandard}

\begin{styleStandard}
\ \ \ \ \ \ Dem
\end{styleStandard}

\begin{styleStandard}
\ \  \ [+D; +DEF, +DEIX: -F, +N, -O]\ \ 
\end{styleStandard}

\begin{styleFootnote}
Note that the vocabulary insertion rule for \textit{dies}{}- in (35b) can apply to both (35a) and (36). If it spells out the features in (35a), then we obtain \textit{dies}{}-, where the CNG bundle is separately realized by the insertion rules in (35c-d); if it applies to (36), then we get an uninflected form of \textit{dies}{}-, with the proviso that the structure in (36) can only occur in nominative/accusative contexts. To be clear, the possibility of projecting two different demonstrative structures accounts for the two different forms – the form of \textit{dies}{}- simply depends on which structure, (35) or (36), is merged in Spec,ArtP of the syntactic representation. With this in mind, I return to the data in (31) from the beginning of this section.
\end{styleFootnote}

\begin{styleFootnote}
Above, I proposed that anaphoric contexts involve null nouns. I extend this claim now to the two forms of the demonstrative. Specifically, I propose that the structure of the demonstrative for the nominal case in (31a) involves a DP that contains either (35a) or (36) along with a null noun (37a-b). As for the clausal instance in (31b), I assume for concreteness that the two demonstrative structures combine with a propositional element, here illustrated with TP (37c-d):
\end{styleFootnote}

\begin{styleFootnote}
(37)\ \ a.\ \ [\textsubscript{DP} \textit{dieses }[\textsubscript{NP} e\textsubscript{N} ]]
\end{styleFootnote}

\begin{styleFootnote}
\ \ b.\ \ [\textsubscript{DP} \textit{dies} [\textsubscript{NP} e\textsubscript{N} ]]
\end{styleFootnote}

\begin{styleFootnote}
\ \ c.\ \ [\textsubscript{DP} \textit{dies} [\textsubscript{TP} e\textsubscript{T} ]]
\end{styleFootnote}

\begin{styleFootnote}
\ \ d. \ \ *\ \ [\textsubscript{DP} \textit{dieses }[\textsubscript{TP} e\textsubscript{T} ]]
\end{styleFootnote}

\begin{styleFootnote}
Starting with uninflected \textit{dies }in (37b-c), I propose that this element is not subject to any conditions as regards agreement. Given the absence of inflection, \textit{dies} does not have to undergo concord in agreement features. Consequently, it is fine in both nominal and clausal contexts. Note again that the restriction to nominative/accusative neuter environments follows from the restriction features on [+DEIX] in (36).
\end{styleFootnote}

\begin{styleFootnote}
\ \ Turning to \textit{dieses }in (37a,d), observe now that -\textit{es} is a neuter inflection. Given the presence of a null noun in (37a), the inflection on \textit{dieses} is licensed by concord in agreement features with this null noun (recall that this noun is part of the anaphor that is related to an antecedent in the neuter gender). Consequently, the demonstrative in (37a) can be part of a nominal anaphor. In order to explain the ungrammaticality of \textit{dieses} in the clausal context in (37d), I suggest that the nominal features of the inflection cannot be licensed as there is no noun present. As such, \textit{dieses} is ungrammatical in this context and cannot serve as a propositional anaphor.\footnote{\ There are dialects that do not make this difference; that is, \textit{dieses} is also possible as a propositional anaphor (see, e.g., Duden 1995: 336, Griesbach \& Schulz 1965: 148). I assume that in these dialects, the licensing conditions of the neuter inflection are different; for instance, neuter gender could be licensed as a default option.} 
\end{styleFootnote}

\begin{styleFootnote}
\ \ More generally, I propose that the varying inflectional morphology and the different semantic restrictions in (31) result from the different demonstrative structures and the element following the demonstrative, either a null nominal (37a-b) or a null clausal element (37c-d). If so, I can maintain the claim that adjectival endings are semantically vacuous (Hypothesis 1a). Indeed, as this section has made clear again, adjectival inflections are licensed in certain contexts only. Furthermore, I propose that adjectival inflections serve to make nominal features such as case, number, and gender visible (Hypothesis 2b). This is particularly clear in contexts where overt nouns are absent. Before I continue the exploration of Hypothesis 2b, I briely need to return to the discussion of weak adjectives.
\end{styleFootnote}

\begin{styleFootnote}
\ \ It is important to comment on the inflection of the adjective in (34a) above, restating that data point below by separating it according to the inflected vs. the uninflected form of the demonstrative. Note again that the two forms of the demonstrative, \textit{dieses }and\textit{ dies}, can only take a weak adjective:
\end{styleFootnote}

\begin{styleFootnote}
(38)\ \ a.\ \ \textit{dies-es schön-e \ \ \ Kleid}
\end{styleFootnote}

\begin{styleFootnote}
\ \ \ \ this-\textsc{st} pretty-\textsc{wk} dress.\textsc{neut}
\end{styleFootnote}

\begin{styleFootnote}
\ \ \ \ ‘this pretty dress’
\end{styleFootnote}

\begin{styleFootnote}
\ \ b.\ \ \textit{dies schön-e \ \ \ Kleid}
\end{styleFootnote}

\begin{styleFootnote}
\ \ \ \ this pretty-\textsc{wk} dress.\textsc{neut}
\end{styleFootnote}

\begin{styleFootnote}
\ \ \ \ ‘this pretty dress’
\end{styleFootnote}

\begin{styleFootnote}
If \textit{dieses }and\textit{ dies} are indeed two separate elements, then the weak inflection on the following adjective in (38b) is not a function of a preceding strong ending but rather of the preceding determiner stem. Note in this regard that both \textit{dieses} and \textit{dies} involve the feature [+DEF], a context where [+D] triggers Impoverishment. Another interesting point about Hypothesis 2b can be made more explicit here.
\end{styleFootnote}

\begin{styleFootnote}
\ \ Recall that Impoverishment deletes [S] from the syntactic representation – the other features remain intact. Furthermore, as mentioned before, the vocabulary insertion rules are underspecified for some of the CNG features. This means that overt inflections only make a subset of the underlying features visible. In other words, Hypothesis 2b does not claim that adjectival inflections make all CNG features visible. Rather, it claims that adjectival inflections make features visible that are left in the syntactic representation after Impoverishment has occurred. In fact, adjectival inflections only make features visible that are specified in their vocabulary insertion rules.
\end{styleFootnote}

\begin{styleFootnote}
Finally, I documented in Chapter 2 that \textit{ein}{}-words can involve pronominal forms too. Exemplifying with \textit{ein}, this element is different from the cases discussed above in certain ways. Similar to the demonstrative, \textit{ein} can also occur in anaphoric contexts (39). Unlike the demonstrative, \textit{ein} must have an ending in anaphoric contexts involving noun phrases:
\end{styleFootnote}

\begin{styleFootnote}
(39)\ \ \textit{Zwei Männer standen auf der Straße. Ein-*(er) von ihnen kam \ }\textit{\textsubscript{\ }}\textit{zur \ \ \ \ Tür}.
\end{styleFootnote}

\begin{styleFootnote}
\ \ two \ \textsubscript{\ }men \ \ \ \ \ \ stand \ \ \ \ in \ \ the \textsubscript{\ }street \ \ \textsubscript{\ }one-\textsc{st} \ \ \ \ of \ \ \ them came to.the door
\end{styleFootnote}

\begin{styleFootnote}
\ \ ‘Two men were standing in the street. One of them came to the door.’
\end{styleFootnote}

\begin{styleFootnote}
Assuming the presence of a null noun following \textit{einer} ‘one’, I argue that the obligatory inflection on \textit{ein} indicates that there can be only one type of \textit{ein}. As proposed in Chapter 2, Section 2.2, this element is a determiner. Second, like the inflection on \textit{dieses} and other elements, the ending on \textit{ein} in (39) makes nominal features visible. In the next chapter, I discuss \textit{ein} in more detail. We will see that there are actually two types of \textit{ein}. I hasten to add though that the other type of \textit{ein} is an adjective that follows definite determiners and is crucially not related to the article \textit{ein} in (39).
\end{styleFootnote}

\begin{styleFootnote}\bfseries
6. \ \ Conclusion
\end{styleFootnote}

\begin{styleStandard}
In this chapter, I turned to four consequences of the analysis laid out in Chapter 2. I discussed two influential proposals of spurious articles – Predicate Inversion and null nouns. I argued that the weak endings on adjectives in these structures indicate that the plural indefinite article in German is not entirely spurious. The two proposals just mentioned involve non-canonical structures. As such, the weak adjectives do not follow from the system developed in Chapter 2. Awaiting an account that can explain the weak adjectives in these non-canonical structures without losing the account of the canonical and other non-canonical instances, I assume that spurious articles actually involve regular structures. While the discussion of these structures is postponed to Chapter 8, I continue assuming that the current account of the strong/weak alternation is on the right track.
\end{styleStandard}

\begin{styleStandard}
\ \ Next, I turned to some consequences for the discussion of discontinuous noun phrases. Illustrating some paradoxical properties, I showed that analyses postulating movement that involve one underlying noun phrase cannot account for the strong adjective in the split-off under current assumptions. In contrast, accounts involving the base-generation of two separate underlying nominals and subsequent movement are completely compatible with the analysis of Chapter 2. I also provided evidence that \textit{ein} (or its feature bundles) is unlikely to be inserted late. 
\end{styleStandard}

\begin{styleStandard}
\ \ Finally, I suggested that non-restrictive adjectives have a similar structural analysis as restrictive ones, the main difference being the presence of a null co-indexed pronoun in the former. In other words, inflectional endings do not signal (non-)restrictiveness of the interpretation of adjectives. In a similar vein, I argued that adjectival inflections are not tied to concepts like “referentiality” but make nominal features like case, number, and gender visible (Hypothesis 2b). 
\end{styleStandard}

\begin{styleStandard}
More generally, I can maintain the claim that adjectival endings are a reflex of the structure (Hypothesis 1b) but not of the semantics (Hypothesis 1a). Furthermore, the discussion of Predicate Inversion has shown in more detail that the degree of the embedding of the adjective is important for the account of the strong/weak alternation (Hypothesis 2a). In the next chapter, I turn to the discussion of \textit{ein} in more detail. We will see that \textit{ein} shares some of the properties of adjectival inflections.
\end{styleStandard}

\clearpage\setcounter{page}{201}\section[Chapter 5: Ein{}-words and Adjectival eine]{\textmd{Chapter 5: }\textmd{\textit{Ein}}\textmd{{}-words and Adjectival }\textmd{\textit{eine}}}
\begin{styleStandard}\bfseries
1. \ \ Introduction
\end{styleStandard}

\begin{styleFooter}
The previous three chapters detailed the current account of the various properties of adjectival inflections. I now turn to \textit{ein}{}-words, which exhibit related properties. Note though that besides this general relatedness of the two types of elements, adjectival endings will continue to play an important part in the following discussion as they help narrow down the choices when proposing plausible structures. With that in mind, I turn to the main hypotheses about \textit{ein}, formulated in Chapter 1 and repeated here for convenience. Recall that (1b) can be fleshed out as in (2a):
\end{styleFooter}

\begin{styleStandard}
(1) \ \ \textit{Hypothesis 1}
\end{styleStandard}

\begin{styleStandard}
Adjectival inflections and \textit{ein}:
\end{styleStandard}

\begin{styleStandard}
\ \ \  \ \ a. \ \ \ \ are expletive elements and
\end{styleStandard}

\begin{styleStandard}
\ \ \  \ \ b. \ \ \ \ indicate abstract structure in the noun phrase.
\end{styleStandard}

\begin{styleStandard}
(2)\ \ \textit{Hypothesis 3}
\end{styleStandard}

\begin{styleStandard}
\ \ \textit{Ein}:
\end{styleStandard}

\begin{styleStandard}
\ \ \ a.\ \ indicates abstract structure in the lower layers of the noun phrase (NP vs. ArtP), \ \ \ \ \ \ and
\end{styleStandard}

\begin{styleStandard}
\ \ \ b.\ \ it supports overt semantic operators (e.g., NEG \textit{k}{}-) and flags the presence of \ \ \ \ \ \ covert semantic operators (e.g., REL).
\end{styleStandard}

\begin{styleFooter}
As already stated in Chapter 1, the difference between supporting and flagging the operator by \textit{ein} is that the presence of the operator itself is in some way detectable in the first case but not in the second. The hypotheses in (1) and (2) are addressed in detail in this and the next two chapters. I start with some general considerations and some basic data involving \textit{ein}.
\end{styleFooter}


\setcounter{listWWviiiNumxxleveli}{0}
\begin{listWWviiiNumxxleveli}
\item 

\setcounter{listWWviiiNumxxlevelii}{0}
\begin{listWWviiiNumxxlevelii}
\item 
\begin{styleFooter}\itshape
Preliminaries and Basic Data
\end{styleFooter}
\end{listWWviiiNumxxlevelii}
\end{listWWviiiNumxxleveli}
\begin{styleFooter}
As amply illustrated above, German noun phrases have determiners. For instance, depending on the context, a singular noun may occur with the appropriate form of the indefinite article. The latter is often reduced in speech (3a). This reduction of the stem from \textit{ein-} to \textit{n-} is marked by parentheses or apostrophe below.\footnote{\ I distinguish different kinds of \textit{ein }below. Parentheses indicate the optionality of the vocalic part of the stem provided that the element is not stressed (which is the case with the indefinite article \textit{ein}, but not with the singularity numeral \textit{EIN} or adjectival \textit{eine}). In some cases, I use an apostrophe to clearly indicate the (reduced) indefinite article. In the latter cases, the unreduced form of the article would lead to a difference in meaning and/or grammaticality.} Alternatively, such a noun can surface with the definite article or its almost homophonous demonstrative counterpart (3b) (the ambiguous status of \textit{die} as the article ‘the’ or the demonstrative ‘that’ is indicated as DET(erminer) in the gloss):
\end{styleFooter}

\begin{styleFooter}
(3)\ \ a.\ \ \textit{(ei)ne Freundin}
\end{styleFooter}

\begin{styleFooter}
\ \ \ \  a \ \ \ \ \ \ girlfriend.\textsc{fem}
\end{styleFooter}

\begin{styleFooter}
\ \ \ \ ‘a girlfriend’
\end{styleFooter}

\begin{styleFooter}
\ \ b.\ \ \textit{die \ Freundin}
\end{styleFooter}

\begin{styleFooter}
\ \ \ \ \textsc{det} girlfriend.\textsc{fem}
\end{styleFooter}

\begin{styleFooter}
\ \ \ \ ‘the / that girlfriend’
\end{styleFooter}

\begin{styleFooter}
As is well known, indefinite and definite determiners cannot co-occur. This applies to both reduced and unreduced indefinite articles as well as to definite articles and demonstratives (for (5b) below, see Pafel 1994: 251; I discuss the unreduced form of \textit{eine} in contexts like (5b) below):
\end{styleFooter}

\begin{styleFooter}
(4)\ \ a. \ *\ \ \textit{(ei)ne (ei)ne Freundin}
\end{styleFooter}

\begin{styleFooter}
\ \ \ \  an \ \ \ \ \ a \ \ \ \ \ \ \ girlfriend.\textsc{fem}
\end{styleFooter}

\begin{styleFooter}
\ \ b. \ *\ \ \textit{(ei)ne die \ Freundin}
\end{styleFooter}

\begin{styleFooter}
\ \ \ \  an \ \ \ \ \textsc{det} girlfriend.\textsc{fem}
\end{styleFooter}

\begin{styleFooter}
(5)\ \ a. \ *\ \ \textit{die \ die \ \ Freundin}
\end{styleFooter}

\begin{styleFooter}
\ \ \ \ \textsc{det} \textsc{det} girlfriend.\textsc{fem}
\end{styleFooter}

\begin{styleFooter}
b. \ *\ \ \textit{die ‘ne Freundin}
\end{styleFooter}

\begin{styleFooter}
\ \ \ \ \textsc{det} a \ \ girlfriend.\textsc{fem}
\end{styleFooter}

\begin{styleFooter}
The same distributional restrictions hold if \textit{die} is replaced by the proximal demonstrative \textit{diese} ‘this’.
\end{styleFooter}

\begin{styleFooter}
Casting the empirical net wider, it is not surprising that the negation particle \textit{nicht} ‘not’ can precede determiners (6). However, I point out that possessives like \textit{mein} ‘my’ and the negator \textit{kein} ‘no’ are in complementary distribution with the indefinite article, the definite article, and the related demonstrative; consider (7) and (8):
\end{styleFooter}

\begin{styleFooter}
(6)\ \ a.\ \ \textit{nicht ‘ne Freundin}\newline
\ \ not \ \ \ \ a \ \ girlfriend.\textsc{fem}
\end{styleFooter}

\begin{styleFooter}
\ \ \ \ ‘not a \ girlfriend’
\end{styleFooter}

\begin{styleFooter}
\ \ b.\ \ \textit{nicht die \ Freundin}\newline
\ \ not \ \ \ \textsc{det} girlfriend.\textsc{fem}
\end{styleFooter}

\begin{styleFooter}
\ \ \ \ ‘not the / that girlfriend’
\end{styleFooter}

\begin{styleFooter}
(7)\ \ a. \ *\ \ \textit{meine ‘ne Freundin}\newline
\ \ my \ \ \ \ \ \ a \ \ girlfriend.\textsc{fem}
\end{styleFooter}

\begin{styleFooter}
\ \ b. \ *\ \ \textit{meine die \ Freundin}\newline
\ \ my \ \ \ \ \textsc{det} girlfriend.\textsc{fem}
\end{styleFooter}

\begin{styleFooter}
(8)\ \ a. \ *\ \ \textit{keine ‘ne Freundin}\newline
\ \ no \ \ \ \ \ \ a \ \ girlfriend.\textsc{fem}
\end{styleFooter}

\begin{styleFooter}
b. \ *\ \ \textit{keine die \ Freundin}\newline
\ \ no \ \ \ \ \textsc{det} girlfriend.\textsc{fem}
\end{styleFooter}

\begin{styleFooter}
The same holds if the determiner precedes the possessive or negator in (7) and (8). In view of this mutually exclusive distribution, I point out again that the possessive pronominal and the negator are labeled possessive article and negative article, respectively. These names foreshadow the analysis to be developed below.
\end{styleFooter}

\begin{styleFooter}
Interestingly, the distribution is partially different when \textit{ein} is stressed and must occur in its unreduced form (stress is indicated here with capital letters). Similar to (6a), the negation particle \textit{nicht} is compatible with \textit{eine} (9a). Unlike (7a), \textit{ein} can, when stressed and unreduced, co-occur with the possessive article (9b). Importantly, this element is still impossible with the negative article or the indefinite article (9c-d) ((9c) is adopted from Fanselow 1988: 111 fn. 29):\footnote{\ Note already here that the two instances of \textit{ein} in (9a) and (9b) are not the same: (9a) involves the singularity numeral \textit{EIN}, but (9b) contains adjectival \textit{eine}. One of the differences is that (9b) has an additional interpretation that involves a duality presupposition roughly translated as ‘one of my two girlfriends’. The differences including the various interpretations are discussed below.}
\end{styleFooter}

\begin{styleFooter}
(9)\ \ a.\ \ \textit{nicht EINE Freundin}\newline
\ \ not \ \ \ one \ \ \ \ girlfriend.\textsc{fem}
\end{styleFooter}

\begin{styleFooter}
\ \ \ \ ‘not one girlfriend’
\end{styleFooter}

\begin{styleFooter}
\newline
b. \ \ \textit{meine EINE Freundin}\newline
\ \ my \ \ \ \ \ one \ \ \ girlfriend.\textsc{fem}
\end{styleFooter}

\begin{styleFooter}
\ \ \ \ ‘my one girlfriend’
\end{styleFooter}

\begin{styleFooter}
\newline
c. *\ \ \textit{keine EINE Freundin}\newline
\ \ no \ \ \ \ \ one \ \ \ girlfriend.\textsc{fem}
\end{styleFooter}

\begin{styleFooter}
d. *\ \ \textit{(ei)ne EINE Freundin}\newline
\ \  an \ \ \ \ one \ \ \ girlfriend.\textsc{fem}
\end{styleFooter}

\begin{styleFooter}
It seems clear that the difference between the determiner elements in (9b) and (9c-d) relates to definiteness: the possessive in (9b) is definite, but the negator and the indefinite article in (9c-d) are not. Thus, besides stress and phonetic non-reducibility, definiteness seems to be a relevant factor in the distribution of \textit{ein}. In other words, any account that seeks to be on the right track needs to take into account the different stress and reduction patterns of \textit{ein} and the syntactic-semantic context this article occurs in.
\end{styleFooter}

\begin{styleFooter}
\ \ It is apparent from the previous literature (see Section 7) that there is relatively little discussion about the morpho-syntax and semantics of all the different kinds of \textit{ein}.\footnote{\ To anticipate the discussion in Section 7, it will become clear there that when we consider some of the previous accounts, we cannot help but notice that opinions diverge considerably with regard to the nature of this kind of indefinite element. Furthermore, none of these proposals discusses all the different kinds of \textit{ein}. } As with adjectival inflections, only the canonical cases are usually discussed. In what follows, I seek to provide a more comprehensive overview of this type of element in German (for some brief cross-linguistic discussion, see again Chapter 1, Section 2.1.3). Similar to adjectival inflections, I propose that the discussion of the non-canonical cases reveals the true nature of \textit{ein}.
\end{styleFooter}

\begin{styleStandard}
\textit{1.2. \ \ Initial Taxonomy of }ein
\end{styleStandard}

\begin{styleFooter}
I assume the following initial classification. This taxonomy contains three main types of \textit{ein }(Karnowski \& Pafel 2004: 174-75) and some subtypes. Below, I propose in detail that certain elements are composite forms consisting of the article and another component:
\end{styleFooter}

\begin{styleStandard}
(10)\ \ a. \textit{ein} as an article:
\end{styleStandard}

\begin{styleStandard}
{}- indefinite article
\end{styleStandard}

\begin{styleStandard}
{}- vacuous article:
\end{styleStandard}

\begin{styleStandard}
{}- \textit{ein} as part of a composite:
\end{styleStandard}

\begin{styleStandard}
{}- possessive
\end{styleStandard}

\begin{styleStandard}
{}- negation
\end{styleStandard}

\begin{styleFooter}
{}- \textit{ein} in predicate noun phrases\newline
{}- \textit{ein} as part of a complex determiner 
\end{styleFooter}

\begin{styleStandard}
b. \textit{ein} as a numeral
\end{styleStandard}

\begin{styleFooter}
c. \textit{ein} as an adjective
\end{styleFooter}

\begin{styleFooter}
I refer to the types of \textit{ein} in (10a) and (10b) collectively as \textit{ein}{}-words; the type in (10c) is called adjectival \textit{eine}. In fact, reducing the numeral to the article, I propose below that there are just two types of \textit{ein}: the article and the adjective. Thus, the two corresponding designations, \textit{ein}{}-word and adjectival \textit{eine}, reflect the analysis to be developed below. Before I go into any specifics, consider first some illustrative examples of the above taxonomy. Note already here that while \textit{ein}{}-words have no ending in (11a-f), adjectival \textit{eine} has a weak ending in (11g):\footnote{\ The German noun \textit{Brot} ‘bread’ is countable. This count reading is translated into English using the classifier \textit{loaf} (e.g., \textit{zwei Brote} ‘(two breads =) two loaves of bread’). Also, I often use the [-figurative] noun \textit{Lehrer} ‘teacher’ in predicative contexts. As a term for a profession, it takes an optional indefinite article in such contexts (Chapter 6). This optionality is crucial for formulating hypotheses about the true nature of \textit{ein}.}
\end{styleFooter}

\begin{styleFooter}
(11)\ \ a.\ \ Indefinite Article
\end{styleFooter}

\begin{styleFooter}
\textit{Ich habe immer \ nur (ei)n frisch-es Brot \ \ \ \ \ \ \ \ \ \ \ }\textit{\textsubscript{\ }}\textit{mitgebracht}.
\end{styleFooter}

\begin{styleFooter}
I \ \ \ \ have always only a \ \ \ \ fresh-\textsc{st} \ bread.\textsc{neut} brought
\end{styleFooter}

\begin{styleFooter}
‘I have always brought only a fresh loaf of bread.’
\end{styleFooter}

\begin{styleFooter}
\ \ b.\ \ Possessive
\end{styleFooter}

\begin{styleFooter}
\textit{Ich habe immer \ nur \ mein frisch-es Brot \ }\textit{\textsubscript{\ \ \ \ \ \ \ \ \ \ \ \ \ \ \ \ }}\textit{mitgebracht}.
\end{styleFooter}

\begin{styleFooter}
I \ \ \ \ have always only my \ \ fresh-\textsc{st} \ bread.\textsc{neut} brought
\end{styleFooter}

\begin{styleFooter}
‘I have always brought only my fresh loaf of bread.’
\end{styleFooter}

\begin{styleFooter}
\ \ c.\ \ Negation
\end{styleFooter}

\begin{styleFooter}
\textit{Ich habe kein frisch-es Brot \ }\textit{\textsubscript{\ \ \ \ \ \ \ \ \ \ \ \ \ \ \ }}\textit{mitgebracht}.
\end{styleFooter}

\begin{styleFooter}
I \ \ \ \ have no \ \ fresh-\textsc{st} \ bread.\textsc{neut} brought
\end{styleFooter}

\begin{styleFooter}
‘I have brought no fresh loaf of bread.’
\end{styleFooter}

\begin{styleFooter}
\ \ d.\ \ Predicative
\end{styleFooter}

\begin{styleStandard}
\textit{Mein Vater \ ist (ein) Lehrer}.
\end{styleStandard}

\begin{styleStandard}
my \ \ \ \ father is \ \ a \ \ \ \ \ teacher.\textsc{masc}
\end{styleStandard}

\begin{styleStandard}
‘My father is a teacher.’
\end{styleStandard}

\begin{styleStandard}
e.\ \ Complex Determiner
\end{styleStandard}

\begin{styleStandard}
\ \ \ \ \textit{Ich habe ein jedes \ frisch-es Brot \ \ \ \ \ \ \ \ \ \ \ mitgebracht}.
\end{styleStandard}

\subsubsection[\ \ \ \ I \ \ \ have \ an \ every fresh{}-st \ bread.neut brought]{\textmd{\ \ \ \ I \ \ \ have \ an \ every fresh-}\textmd{\textsc{st}}\textmd{ \ bread.}\textmd{\textsc{neut}}\textmd{ brought}}
\begin{styleStandard}
\ \ \ \ ‘I have brought each fresh loaf of bread.’
\end{styleStandard}

\begin{styleFooter}
\ \ f.\ \ Numeral
\end{styleFooter}

\begin{styleFooter}
\textit{Ich habe immer \ nur \ EIN frisch-es Brot \ }\textit{\textsubscript{\ \ \ \ \ \ \ \ \ \ \ \ \ \ \ \ }}\textit{mitgebracht}.
\end{styleFooter}

\begin{styleFooter}
I \ \ \ \ have always only one fresh-\textsc{st} \ bread.\textsc{neut} brought
\end{styleFooter}

\begin{styleFooter}
‘I have always brought only one fresh loaf of bread.’
\end{styleFooter}

\begin{styleFooter}
\ \ g.\ \ Adjective
\end{styleFooter}

\begin{styleFooter}
\textit{Ich habe nur \ das ein-e \ \ \ \ frisch-e \ \ Brot \ \ \ \ \ \ \ \ \ \ \ \ mitgebracht}.
\end{styleFooter}

\begin{styleFooter}
I \ \ \ \textsubscript{\ }have only the one-\textsc{wk} fresh-\textsc{wk} bread.\textsc{neut} brought
\end{styleFooter}

\begin{styleFooter}
‘I have only brought the one fresh loaf of bread.’
\end{styleFooter}

\begin{styleFooter}
These different kinds of contexts \textit{ein} appears in will be discussed at length in what follows. The discussion of the current and the next two chapters can be anticipated as follows.
\end{styleFooter}

\begin{styleFooter}\itshape
1.3.\ \ Outlook
\end{styleFooter}

\begin{styleFooter}
One goal of this chapter is to provide a more comprehensive survey of the different types of \textit{ein} with the intention of ultimately reducing them in number. I propose that there are two lexical types of \textit{ein}: the article and the adjective (cf. Pafel 2005: 179). The second goal is to determine the properties of the different types of \textit{ein}. I argue that the article is semantically vacuous (Hypothesis 1a) but that adjectival \textit{eine} is not. In other words, the article is, in a number of ways, similar to adjectival inflections discussed in Chapter 2.
\end{styleFooter}

\begin{styleFooter}
In more detail, I propose that the possessive articles, the negative article, and the singularity numeral are composite forms consisting of the indefinite article and another component. Providing some brief remarks on the possessive and negative composites, I dedicate most of the discussion to the article, the numeral, and adjectival \textit{eine}. Some distinctions between the last three elements are argued to follow from their different featural specifications and others from their different positions in the syntactic tree. Specifically, deriving the numeral from the combination of the indefinite article and another component, I account for the identical morphology but the different semantics of the numeral and article. Thus, I propose that the numeral is related to the article in a way that adjectival \textit{eine} is not – the latter is a separate lexical element. 
\end{styleFooter}

\begin{styleFooter}
As to the remaining types in (11) above, predicative \textit{ein} is illustrated below but discussed in greater detail in Chapter 6. I do not address the complex determiner \textit{ein jeder }‘(an every =) each’ in much detail here (but see Roehrs 2012, Zimmermann 2011).\footnote{\ Note that \textit{ein} here occurs in the context of the semantically plural and definite element\textit{ jeder} ‘every’. It is proposed in Roehrs (2012) that \textit{ein} intensifies the distributivity of \textit{jeder}. I tentatively suggest in Chapter 8 that \textit{ein} may flag the presence of a null distributivity operator (cf. de Swart \textit{et al}. 2007: 218). This is consistent with the proposal that \textit{ein} is semantically vacuous, including in the string \textit{ein jeder}.} More generally, the current chapter discusses the morpho-syntax of \textit{ein }and its relation to indefiniteness, and the next two chapters focus on \textit{ein} in relation to semantic notions like emotiveness (Chapter 6) and number (Chapter 7).
\end{styleFooter}

\begin{styleFooter}
The current chapter is organized as follows. In order to motivate the approach that some types of \textit{ein} should be treated in the same way, I first illustrate certain morphological similarities between the different kinds of \textit{ein}. In Section 3, some phonetic and semantic differences are pointed out. These differences are summarized in Table 1 below. A bipartite proposal involving different feature specifications and different syntactic positions is discussed in Sections 4 and 5, respectively. Section 6 provides some diachronic and cross-linguistic evidence that \textit{ein}{}-words are indeed composite forms in (Modern) German. In Section 7, I critique a previous proposal in some detail, and Section 8 forms the conclusion.
\end{styleFooter}

\begin{styleFooter}\bfseries
2. \ \ Similarities
\end{styleFooter}

\begin{styleFooter}
In Section 1.2, I provided my taxonomy along with an illustration of the basic cases. With these reference points in mind, I discuss the occurrence of the different inflections on \textit{ein} in split topicalizations and with elided nouns. Note that these constructions require special contexts to be felicitous (see again Chapter 4, Section 3).
\end{styleFooter}

\begin{styleFooter}\itshape
2.1. \ \ Split Topicalization
\end{styleFooter}

\begin{styleFooter}
Comparing (11) to (12), split topicalizations with in-situ adjectives have the same morphology as non-split noun phrases; that is, \textit{ein}{}-words have no endings in (12a-e), but adjectival \textit{eine} has a weak inflection in (12f):\footnote{\ To investigate the relevant aspects of the morpho-syntax of predicative \textit{ein} in split topicalizations, I chose a non-canonical construction involving the modal particle \textit{vielleicht} ‘really’. This element makes the split of the predicative construction more felicitous. Note also that the translations of the following examples into English are not always straightforward – they often are approximations of the German originals.}
\end{styleFooter}

\begin{styleFooter}
(12)\ \ a.\ \ Indefinite Article
\end{styleFooter}

\begin{styleFooter}
\textit{Brot \ \ \ \ \ \ \ \ \ \ \ \ habe ich immer \ nur \ (ei)n \ frisch-es mitgebracht}.
\end{styleFooter}

\begin{styleFooter}
bread.\textsc{neut} have I \ \ \ \ always only \ a \ \ \ \ \textsubscript{\ }fresh-\textsc{st} \ brought
\end{styleFooter}

\begin{styleFooter}
‘As for bread, I have always brought only a fresh loaf.’
\end{styleFooter}

\begin{styleFooter}
\ \ b.\ \ Possessive
\end{styleFooter}

\begin{styleFooter}
\textit{Brot \ \ \ \ \ \ \ \ \ \ \ \ habe ich immer nur \ \ mein \ frisch-es mitgebracht}.
\end{styleFooter}

\begin{styleFooter}
bread.\textsc{neut} have I \ \ \ \ always only my \ \ \textsubscript{\ }fresh-\textsc{st} \ brought
\end{styleFooter}

\begin{styleFooter}
‘As for bread, I have always brought only my fresh loaf.’
\end{styleFooter}

\begin{styleFooter}
\ \ c.\ \ Negation
\end{styleFooter}

\begin{styleFooter}
\textit{Brot \ \ \ \ \ \ \ \ \ \ \ habe ich kein frisch-es mitgebracht}.
\end{styleFooter}

\begin{styleFooter}
bread.\textsc{neut} have I \ \ \ \ no \ \ fresh-\textsc{st} \ brought
\end{styleFooter}

\begin{styleFooter}
‘As for bread, I have brought no fresh loaf.’
\end{styleFooter}

\begin{styleFooter}
\ \ d.\ \ Numeral
\end{styleFooter}

\begin{styleFooter}
\textit{Brot \ \ \ \ \ \ \ \ \ \ \ \ habe ich immer \ nur \ EIN frisch-es mitgebracht}.
\end{styleFooter}

\begin{styleFooter}
bread.\textsc{neut} have I \ \ \ \ always only one fresh-\textsc{st} \ brought
\end{styleFooter}

\begin{styleFooter}
‘As for bread, I have always brought only one fresh loaf.’
\end{styleFooter}

\begin{styleStandard}
e.\ \ Predicative
\end{styleStandard}

\begin{styleStandard}\itshape
Brot \ \ \ \ \ \ \ \ \ \ \ \ ist das vielleicht (ei)n frisch-es!
\end{styleStandard}

\begin{styleStandard}
bread.\textsc{neut} is \ that \textsc{prt} \ \ \ \ \ \ \ \ \ \ a \ \ \ \ fresh-\textsc{st} \ 
\end{styleStandard}

\begin{styleStandard}
‘As for bread, this is really a fresh loaf.’
\end{styleStandard}

\begin{styleFooter}
\ \ f.\ \ Adjective
\end{styleFooter}

\begin{styleFooter}
\textit{Brot \ \ \ \ \ \ \ \ \ \ \ \ habe ich nur \ \ das ein-e \ \ \ \ frisch-e \ \ mitgebracht}.
\end{styleFooter}

\begin{styleFooter}
bread.\textsc{neut} have I \ \ \ \ only the \ one-\textsc{wk} fresh-\textsc{wk} brought
\end{styleFooter}

\begin{styleFooter}
‘As for bread, I have only brought the one fresh loaf.’
\end{styleFooter}

\begin{styleFooter}\itshape
2.2. \ \ Split Topicalization with Fronted Adjective
\end{styleFooter}

\begin{styleFooter}
If the adjective is part of the split-off, the \textit{ein}{}-word exhibits a strong ending (13a-e). Interestingly, the stranded \textit{ein}{}-words have an optional schwa (see also Roehrs 2009a: 156, Sternefeld 2008: 152), where the option with schwa seems to be, at least to my ears, of elevated style. With a determiner preceding, there is no change for \textit{eine} in (13f):\footnote{\ When \textit{ein} is stranded by itself (13a), it is actually stressed. This makes the indefinite article ambiguous with the numeral in (13d). Note that even if another stressed element such as \textit{so} ‘such’ is added, the unreduced form of \textit{ein} is still much better here:\par 
\setcounter{listWWviiiNumiileveli}{0}
\begin{listWWviiiNumiileveli}
\item 
\begin{styleFootnote}
\textit{Brot \ }\textit{\textsubscript{\ \ \ \ \ \ \ \ \ \ \ \ \ \ \ \ }}\textit{habe ich immer \ nur \ }\textit{\textsubscript{\ }}\textit{so \ \ \{ein(e)s / *?‘ns / *‘nes\} mitgebracht}.
\end{styleFootnote}
\end{listWWviiiNumiileveli}
bread.\textsc{neut} have I \ \ \ \textsubscript{\ }always only such a \ \ \ \ \ \ \ \ \ \ \ \ \ \ \ \ \ \ \ \ \ \ \ \ \ \ \ \ \ \ \ \ \ \ brought\par ‘As for bread, I have always brought only this kind of loaf.’\par Notice that \textit{ein} in (i) is used pronominally and as such, it has some stress on it. This presumably explains the degradedness of the reduced forms. It is not clear to me though if this instance of \textit{ein} is the unreduced article or the numeral (which consists of the article and an additional component).}
\end{styleFooter}

\begin{styleFooter}
(13)\ \ a.\ \ Indefinite Article
\end{styleFooter}

\begin{styleFooter}
\textit{(Frisch-es) Brot \ \ \ \ \ \ \ \ \ \ \ habe ich immer \ nur \ ein-(e)s mitgebracht}.
\end{styleFooter}

\begin{styleFooter}
\ fresh-\textsc{st} \ \ \ bread.\textsc{neut} have I \ \ \ \ always only a-\textsc{st} \ \ \ \ \ brought
\end{styleFooter}

\begin{styleFooter}
‘As for (fresh) bread, I have always brought only a loaf.’
\end{styleFooter}

\begin{styleFooter}
\ \ b.\ \ Possessive
\end{styleFooter}

\begin{styleFooter}
\textit{(Frisch-es) Brot \ \ \ \ \ \ \ \ \ \ \ \ habe ich immer \ nur \ mein-(e)s mitgebracht}.
\end{styleFooter}

\begin{styleFooter}
\ fresh-\textsc{st} \ \ \ \ bread.\textsc{neut} have I \ \ \ \ always only my\textsc{{}-st} \ \ \ \ brought
\end{styleFooter}

\begin{styleFooter}
‘As for (fresh) bread, I have always brought only my loaf.’
\end{styleFooter}

\begin{styleFooter}
\ \ c.\ \ Negation
\end{styleFooter}

\begin{styleFooter}
\textit{(Frisch-es) Brot \ \ \ \ \ \ \ \ \ \ \ \ habe ich kein-(e)s mitgebracht}.
\end{styleFooter}

\begin{styleFooter}
\ fresh-\textsc{st} \ \ \ \ bread.\textsc{neut} have I \ \ \ \ no-\textsc{st} \ \ \ \ \ brought
\end{styleFooter}

\begin{styleFooter}
‘As for (fresh) bread, I have brought no loaf.’
\end{styleFooter}

\begin{styleFooter}
\ \ d.\ \ Numeral
\end{styleFooter}

\begin{styleFooter}
\textit{(Frisch-es) Brot \ \ \ \ \ \ \ \ \ \ \ \ habe ich immer nur \ EIN-(E)S mitgebracht}.
\end{styleFooter}

\begin{styleFooter}
\ fresh-\textsc{st} \ \ \ \ bread.\textsc{neut} have I \ \ \ always only one-\textsc{st} \ \ \ \ brought
\end{styleFooter}

\begin{styleFooter}
‘As for (fresh) bread, I have always brought only one loaf.’
\end{styleFooter}

\begin{styleStandard}
e.\ \ Predicative
\end{styleStandard}

\begin{styleTextbodyindent}
\ \  \ \ \textit{(Frisch-es) Brot \ \ \ \ \ \ \ \ \ \ \ \ ist das vielleicht ein-(e)s!}
\end{styleTextbodyindent}

\begin{styleTextbodyindent}
\ \ \ \  fresh-\textsc{st} \ \ \ \ bread.\textsc{neut} is \ that \textsc{prt} \ \ \ \ \ \ \ \ \ one-\textsc{st} \ 
\end{styleTextbodyindent}

\begin{styleStandard}
‘As for (fresh) bread, this is really one [fresh] loaf.’
\end{styleStandard}

\begin{styleFooter}
\ \ f.\ \ Adjective
\end{styleFooter}

\begin{styleFooter}
\textit{(Frisch-es) Brot \ \ \ \ \ \ \ \ \ \ \ \ habe ich nur \ das ein-e \ \ \ \ mitgebracht}.
\end{styleFooter}

\begin{styleFooter}
\ fresh-\textsc{st} \ \ \ \ bread.\textsc{neut} have I \ \ \ only the \ one-\textsc{wk} brought
\end{styleFooter}

\begin{styleFooter}
‘As for (fresh) bread, I have only brought the one loaf.’
\end{styleFooter}

\begin{styleFooter}
Comparing the inflections on the adjectives in Section 2.1 to the ones on the \textit{ein}{}-words here, we can observe again that they are the same. In keeping with Chapter 4, Section 5, I assume that these inflections make nominal features like case, number, and gender visible (Hypothesis 2b). Recall also that the emergence of the strong ending on the \textit{ein}{}-words is due to a certain vocabulary insertion rule (Chapter 2, Section 2.2). Parallel facts hold when the noun is elided.
\end{styleFooter}

\begin{styleFooter}\itshape
2.3. \ \ Adjective Followed by Elided Noun
\end{styleFooter}

\begin{styleFooter}
When adjectives are present, noun phrases with elided nouns have the same morphology as those with non-elided nouns (Chapter 2); that is, the \textit{ein}{}-words have no endings in (14a-e), but adjectival \textit{eine} has a weak inflection in (14f):
\end{styleFooter}

\begin{styleFooter}
(14)\ \ a.\ \ Indefinite Article
\end{styleFooter}

\begin{styleFooter}
\textit{Ich habe immer \ nur (ei)n frisch-es mitgebracht}.
\end{styleFooter}

\begin{styleFooter}
I \ \ \ \ have always only a \ \ \ \ fresh-\textsc{st} \ brought
\end{styleFooter}

\begin{styleFooter}
‘I have always brought only a fresh loaf.’
\end{styleFooter}

\begin{styleFooter}
\ \ b.\ \ Possessive
\end{styleFooter}

\begin{styleFooter}
\textit{Ich habe immer \ nur \ mein frisch-es mitgebracht}.
\end{styleFooter}

\begin{styleFooter}
I \ \ \ \ have always only my \ \ fresh-\textsc{st} \ brought
\end{styleFooter}

\begin{styleFooter}
‘I have always brought only my fresh loaf.’
\end{styleFooter}

\begin{styleFooter}
\ \ c.\ \ Negation
\end{styleFooter}

\begin{styleFooter}
\textit{Ich habe kein frisch-es mitgebracht}.
\end{styleFooter}

\begin{styleFooter}
I \ \ \ \ have no \ \ \textsubscript{\ }fresh-\textsc{st} \ brought
\end{styleFooter}

\begin{styleFooter}
‘I have brought no fresh loaf.’
\end{styleFooter}

\begin{styleFooter}
\ \ d.\ \ Numeral
\end{styleFooter}

\begin{styleFooter}
\textit{Ich habe immer \ nur \ EIN frisch-es mitgebracht}.
\end{styleFooter}

\begin{styleFooter}
I \ \ \ \ have always only one fresh-\textsc{st} \ brought
\end{styleFooter}

\begin{styleFooter}
‘I have always brought only one fresh loaf.’
\end{styleFooter}

\begin{styleStandard}
e.\ \ Predicative
\end{styleStandard}

\begin{styleStandard}\itshape
Das ist vielleicht (ei)n frisch-es!
\end{styleStandard}

\begin{styleStandard}
that is \ \textsc{prt} \ \ \ \ \ \ \ \ \ \ a \ \ \ \ fresh-\textsc{st} \ 
\end{styleStandard}

\begin{styleStandard}
‘This is really a fresh loaf.’
\end{styleStandard}

\begin{styleFooter}
\ \ f.\ \ Adjective
\end{styleFooter}

\begin{styleFooter}
\textit{Ich habe nur \ \ das ein-e \ \ \ \ \ frisch-e \ \ mitgebracht}.
\end{styleFooter}

\begin{styleFooter}
I \ \ \ \ have only the \ one-\textsc{wk} fresh-\textsc{wk} brought
\end{styleFooter}

\begin{styleFooter}
‘I have only brought the one fresh loaf.’
\end{styleFooter}

\begin{styleStandard}
Finally, consider topicalization where the split-off contains an adjective and an elided noun.
\end{styleStandard}

\begin{styleFooter}\itshape
2.4. \ \ Split Topicalization with Fronted Adjective and Elided Noun
\end{styleFooter}

\begin{styleFooter}
If the noun phrases are split and the adjectives and elided nouns are part of the split-offs, then similar to Section 2.2, the \textit{ein}{}-words have strong endings (15a-e). Again, there is no change for \textit{eine} in (15f):
\end{styleFooter}

\begin{styleFooter}
(15)\ \ a.\ \ Indefinite Article
\end{styleFooter}

\begin{styleFooter}
\textit{Frisch-es habe ich immer \ nur \ ein-(e)s mitgebracht}.
\end{styleFooter}

\begin{styleFooter}
fresh-\textsc{st} \ \ have I \ \ \ \ always only a-\textsc{st} \ \ \ \ brought
\end{styleFooter}

\begin{styleFooter}
‘As for a fresh one, I have always brought only a loaf.’
\end{styleFooter}

\begin{styleFooter}
\ \ b.\ \ Possessive
\end{styleFooter}

\begin{styleFooter}
\textit{Frisch-es habe ich immer \ nur \ mein-(e)s mitgebracht}.
\end{styleFooter}

\begin{styleFooter}
fresh-\textsc{st} \ \ have I \ \ \ \ always only my-\textsc{st} \ \ \ \ brought
\end{styleFooter}

\begin{styleFooter}
‘As for a fresh one, I have always brought only my loaf.’
\end{styleFooter}

\begin{styleFooter}
\ \ c.\ \ Negation
\end{styleFooter}

\begin{styleFooter}
\textit{Frisch-es habe ich kein-(e)s mitgebracht}.
\end{styleFooter}

\begin{styleFooter}
fresh-\textsc{st} \ \ have I \ \ \ \ no-\textsc{st} \ \ \ \ \ brought
\end{styleFooter}

\begin{styleFooter}
‘As for a fresh one, I have brought no loaf.’
\end{styleFooter}

\begin{styleFooter}
\ \ d.\ \ Numeral
\end{styleFooter}

\begin{styleFooter}
\textit{Frisch-es habe ich immer \ nur \ EIN-(E)S mitgebracht}.
\end{styleFooter}

\begin{styleFooter}
fresh-\textsc{st} \ \ have I \ \ \ \ always only one-\textsc{st} \ \ \ \ brought
\end{styleFooter}

\begin{styleFooter}
‘As for a fresh one, I have always brought only one loaf.’
\end{styleFooter}

\begin{styleStandard}
e.\ \ Predicative
\end{styleStandard}

\begin{styleStandard}\itshape
Frisch-es ist das vielleicht ein-(e)s!
\end{styleStandard}

\begin{styleStandard}
fresh-\textsc{st} \ \ is \ that \textsc{prt} \ \ \ \ \ \ \ \ \ one-\textsc{st} \ 
\end{styleStandard}

\begin{styleStandard}
‘As for a fresh one, this is really one [fresh] loaf.’
\end{styleStandard}

\begin{styleFooter}
\ \ f.\ \ Adjective
\end{styleFooter}

\begin{styleFooter}
\textit{Frisch-es habe ich nur \ das ein-e \ \ \ \ mitgebracht}.
\end{styleFooter}

\begin{styleFooter}
fresh-\textsc{st} \ \ have I \ \ \ \ only the one-\textsc{wk} brought
\end{styleFooter}

\begin{styleFooter}
‘As for a fresh one, I have only brought the one loaf.’
\end{styleFooter}

\begin{styleFooter}
To summarize these sections, all \textit{ein}{}-words are marked by the emergence of strong endings when they are “stranded” by themselves in split topicalizations. Adjectival \textit{eine} is different – it remains unchanged. Similar facts of the \textit{ein}{}-words were observed when the latter occur in elliptical contexts; that is, when the adjective and the noun are not present at all (Chapter 2, Section 2.2.2). The emergence of the strong inflections on the \textit{ein}{}-words follow from the same vocabulary insertion rule stated in Section 2.2.2: it applies to \textit{ein}{}-words, independently of whether \textit{ein} is “stranded” by split topicalization or occurs in elliptical contexts – in each case, there is no overt material following \textit{ein}.\footnote{\ Recall from Chapter 4, Section 3 that \textit{ein} in split topicalizations is not actutally stranded by the adjective and/or noun moving out of the source DP. Rather, the adjective and/or noun are part of a separate nominal, the split-off. As such, on my assumptions, both split topicalization and ellipsis involve null nouns in the nominal containing \textit{ein}.}
\end{styleFooter}

\begin{styleFooter}\bfseries
3. \ \ Differences
\end{styleFooter}

\begin{styleFooter}
In this section, I focus on three phonetic and semantic differences: encliticization, stressability, and semantic singularity (see also Karnowski \& Pafel 2004: 174-75, Pafel 2005: 180). These differences are summarized in tabular form at the end of this section.
\end{styleFooter}

\begin{styleFooter}\itshape
3.1. \ \ Encliticization
\end{styleFooter}

\begin{styleFooter}
As seen above, reduced forms of the indefinite article are independently possible. Thus, I take encliticization of \textit{ein} to be instantiated when the indefinite article undergoes further phonetic changes due to its preceding element, its host. For instance, coronal \textit{‘n} as the reduced form of \textit{ein} becomes \textit{’m} when it is encliticized onto a word ending in a labial sound (R. Wiese 1996a: 166). For this purpose, I use, for the most part, verbal hosts with their endings apocopated (e.g., \textit{habe} {\textgreater} \textit{hab’ }‘have’). With this in mind, note that cliticization to a preceding word is only possible with an indefinite article (16a) and predicative \textit{ein }(16e) (encliticization is indicated in the gloss by a period; I do not use phonetic brackets to indicate the phonetic change):\footnote{\ In the predicative context, the copular \textit{sein} ‘to be’ was changed to \textit{bleiben} ‘to remain’. This was done to have a verbal stem ending in a non-coronal sound.}
\end{styleFooter}

\begin{styleFooter}
(16)\ \ a.\ \ Indefinite Article
\end{styleFooter}

\begin{styleStandard}
\textit{Ich hab’m (frisch-es) Brot \ \ \ \ \ \ \ \ \ \ \ mitgebracht}.
\end{styleStandard}

\subsubsection[\ \ \ \ I \ \ \ have.a \ \ fresh{}-st \ \ bread.neut brought]{\textmd{\ \ \ \ I \ \ \ have.a \ \ fresh-}\textmd{\textsc{st}}\textmd{ \ \ bread.}\textmd{\textsc{neut}}\textmd{ brought}}
\begin{styleStandard}
\ \ \ \ ‘I have brought a (fresh) loaf of bread.’
\end{styleStandard}

\begin{styleFooter}
\ \ b. \ \ \ Possessive
\end{styleFooter}

\begin{styleStandard}
\textit{\ \ \ \ *\ \ Ich hab’ m’m \ (frisch-es) Brot \ \ \ \ \ \ \ \ \ \ \ \ mitgebracht}.
\end{styleStandard}

\subsubsection[\ \ \ \ I \ \ \ have poss.a fresh{}-st \ \ bread.neut brought]{\textmd{\ \ \ \ I \ \ \ have }\textmd{\textsc{poss}}\textmd{.a fresh-}\textmd{\textsc{st}}\textmd{ \ \ bread.}\textmd{\textsc{neut}}\textmd{ brought}}
\begin{styleStandard}
\ \ \ \ ‘I have brought my (fresh) loaf of bread.’
\end{styleStandard}

\begin{styleFooter}
\ \ c. \ \ \ Negation
\end{styleFooter}

\begin{styleStandard}
\textit{\ \ \ \ *\ \ Ich hab’ k’ŋ \ \ \ (frisch-es) Brot \ \ \ \ \ \ \ \ \ \ \ mitgebracht}.
\end{styleStandard}

\subsubsection[\ \ \ \ I \ \ \ have \ neg.a fresh{}-st \ \ bread.neut brought]{\textmd{\ \ \ \ I \ \ \ have }\textmd{\textsubscript{\ }}\textmd{\textsc{neg}}\textmd{.a fresh-}\textmd{\textsc{st}}\textmd{ \ \ bread.}\textmd{\textsc{neut}}\textmd{ brought}}
\begin{styleStandard}
\ \ \ \ ‘I have brought no (fresh) loaf of bread.’
\end{styleStandard}

\begin{styleFooter}
\ \ d. \ \ \ Numeral
\end{styleFooter}

\begin{styleStandard}
\textit{\ \ \ \ *\ \ Ich hab’M \ \ \ (frisch-es) Brot \ \ \ \ \ \ \ \ \ \ \ mitgebracht}.
\end{styleStandard}

\subsubsection[\ \ \ \ I \ \ \ have.one \ fresh{}-st \ \ bread.neut brought]{\textmd{\ \ \ \ I \ \ \ have.one \ fresh-}\textmd{\textsc{st}}\textmd{ \ \ bread.}\textmd{\textsc{neut}}\textmd{ brought}}
\begin{styleStandard}
\ \ \ \ ‘I have brought one (fresh) loaf of bread.’
\end{styleStandard}

\begin{styleFooter}
\ \ e.\ \ Predicative
\end{styleFooter}

\begin{styleStandard}
\textit{Ich bleib’m \ (jung-er) \ \ Lehrer}.
\end{styleStandard}

\begin{styleStandard}
I \ \ \ \ remain.a \textsubscript{\ }young-\textsc{st} teacher.\textsc{masc}
\end{styleStandard}

\begin{styleStandard}
‘I remain a (young) teacher.’
\end{styleStandard}

\begin{styleFooter}
f.\ \ Adjective
\end{styleFooter}

\begin{styleStandard}
\ \  \ \ \ \ *\ \ \textit{Ich habe gestern \ \ \ \ nur \ dem’m-en \ (frisch-en) Brot \ }\textit{\textsubscript{\ \ \ \ \ \ \ \ \ \ \ \ \ \ \ \ }}\textit{zugesprochen}.
\end{styleStandard}

\begin{styleStandard}
\ \ \ \ I \ \ \ have \ yesterday only the.one-\textsc{wk} fresh-\textsc{wk} \ bread.\textsc{neut} eaten
\end{styleStandard}

\begin{styleStandard}
\ \ \ \ ‘Yesterday, I only ate the one (fresh) loaf of bread.’
\end{styleStandard}

\begin{styleFooter}
To be clear, encliticization of \textit{ein }is not possible when \textit{ein }is part of a composite (16b-c) or stressed (16d,f). I will not have much more to say here about the regularities that govern encliticization.
\end{styleFooter}

\begin{styleFooter}\itshape
3.2. \ \ Stressability
\end{styleFooter}

\begin{styleFooter}
With regard to the possibility of bearing stress, some of the judgments from the previous section reverse. The types of \textit{ein} fall into three groups: first, the indefinite article may not be stressed (17a); second, the possessive article, the negative article, and \textit{ein} in predicate noun phrases may be stressed (17b,c,e); and third, the numeral and adjectival \textit{eine} must be stressed (17d,f,f’):
\end{styleFooter}

\begin{styleFooter}
(17)\ \ a.\ \ Indefinite Article
\end{styleFooter}

\begin{styleStandard}
\textit{\ \ \ \ *\ \ Ich habe ‘N (frisch-es) Brot \ \ \ \ \ \ \ \ \ \ \ mitgebracht}.
\end{styleStandard}

\begin{styleStandard}
\ \ \ \ I \ \ \ \ have \ a \ \ fresh-\textsc{st} \ \ bread.\textsc{neut} brought
\end{styleStandard}

\begin{styleStandard}
\ \ \ \ ‘I have brought a (fresh) loaf of bread.’
\end{styleStandard}

\begin{styleFooter}
\ \ b.\ \ Possessive
\end{styleFooter}

\begin{styleStandard}
\textit{Ich habe MEIN (frisch-es) Brot \ \ \ \ \ \ \ \ \ \ \ mitgebracht}.
\end{styleStandard}

\begin{styleStandard}
\ \ \ \ I \ \ \ \ have my \ \ \ \ \textsubscript{\ }\ fresh-\textsc{st} \ \ bread.\textsc{neut} brought
\end{styleStandard}

\begin{styleStandard}
\ \ \ \ ‘I have brought my (fresh) loaf of bread.’
\end{styleStandard}

\begin{styleFooter}
\ \ c.\ \ Negation
\end{styleFooter}

\begin{styleStandard}
\textit{Ich habe KEIN }\textit{\textsubscript{\ }}\textit{(frisch-es) Brot \ \ \ \ \ \ \ \ \ \ \ mitgebracht}.
\end{styleStandard}

\begin{styleStandard}
\ \ \ \ I \ \ \ \ have no \ \ \ \ \ \textsubscript{\ }\ fresh-\textsc{st} \ \ bread.\textsc{neut} brought
\end{styleStandard}

\begin{styleStandard}
\ \ \ \ ‘I have brought no (fresh) loaf of bread.’
\end{styleStandard}

\begin{styleFooter}
\ \ d.\ \ Numeral
\end{styleFooter}

\begin{styleStandard}
\textit{\ \ \ \ *\ \ Ich habe ‘n}\textit{\textsubscript{ \ \ \ \ }}\textit{(frisch-es) Brot \ \ \ \ \ \ \ \ \ \ \ mitgebracht}.
\end{styleStandard}

\begin{styleStandard}
\ \ \ \ I \ \ \ \ have \ one fresh-\textsc{st} \ \ bread.\textsc{neut} brought
\end{styleStandard}

\begin{styleStandard}
\ \ \ \ ‘I have brought one (fresh) loaf of bread.’
\end{styleStandard}

\begin{styleFooter}
\ \ e.\ \ Predicative
\end{styleFooter}

\begin{styleStandard}
\textit{Mein Vater ist EIN Lehrer}.
\end{styleStandard}

\subparagraph[\ \ \ \ \ my \ \ \ father is \ one \ teacher.masc]{\textrm{\textmd{\textup{\ \ \ \ \ my \ \ \ father is }}}\textrm{\textmd{\textup{\textsubscript{\ }}}}\textrm{\textmd{\textup{one \ teacher.}}}\textrm{\textmd{\textsc{masc}}}}
\begin{styleStandard}
\ \ \ \ ‘My father is one teacher.’
\end{styleStandard}

\begin{styleFooter}
f.\ \ Adjective
\end{styleFooter}

\begin{styleStandard}
\ \  \ \ \ \ *\ \ \textit{Ich habe nur \ \ das ‘n-e \ \ \ \ \ \ (frisch-e) \ Brot \ \ \ \ \ \ \ \ \ \ \ mitgebracht}.
\end{styleStandard}

\begin{styleStandard}
\ \ \ \ I \ \ \ have \ only the \ \ one-\textsc{wk} fresh-\textsc{wk} bread.\textsc{neut} brought
\end{styleStandard}

\begin{styleStandard}
\ \ \ \ ‘I have brought only the one (fresh) loaf of bread.’
\end{styleStandard}

\begin{styleFooter}
f’.\ \ Adjective
\end{styleFooter}

\begin{styleStandard}
\ \  \ \ \ \ \ \ \textit{Ich habe nur \ \ das EIN-E \ (frisch-e) \ Brot \ \ \ \ \ \ \ \ \ \ \ mitgebracht}.
\end{styleStandard}

\begin{styleStandard}
\ \ \ \ I \ \ \ have \ only the \ one-\textsc{wk }fresh-\textsc{wk} bread.\textsc{neut} brought
\end{styleStandard}

\begin{styleStandard}
\ \ \ \ ‘I have brought only the one (fresh) loaf of bread.’
\end{styleStandard}

\begin{styleFooter}
Although stressed, I do not provide adjectival \textit{eine} in capital letters below in order to distinguish it better from the singularity numeral. More importantly, note that with the exception of the singularity numeral and adjectival \textit{eine }(which are both independently stressed), it seems clear that stress has a semantic effect. For instance, (17e) implies that my father is not just a teacher but one teacher among others (cf. Higginbotham’s 1987: 68 fn. 4 discussion of English).\footnote{\ In fact, stressed \textit{EIN} in (17e) might be a numeral in a predicative context. Note that with the exception of the singularity numeral and adjectival \textit{eine}, I do not discuss the effects of stress on the semantics.} 
\end{styleFooter}

\begin{styleFooter}\itshape
3.3. \ \ Semantic Singularity
\end{styleFooter}

\begin{styleFooter}
I turn to some semantic differences. While the indefinite article usually implies singularity of the entity (18a), the numeral emphasizes singularity as opposed to plurality (18b):
\end{styleFooter}

\begin{styleFooter}
(18)\ \ a.\ \ \textit{Ich habe (ei)n Mädchen geküßt}.
\end{styleFooter}

\begin{styleFooter}
\ \ \ \ I \ \ \ \ have \ a \ \ \ \ \ girl.\textsc{neut }kissed
\end{styleFooter}

\begin{styleStandard}
‘I have kissed a girl.’
\end{styleStandard}

\begin{styleFooter}
\ \ b.\ \ \textit{Ich habe EIN Mädchen geküßt (nicht ZWEI)}.
\end{styleFooter}

\begin{styleFooter}
\ \ \ \ I \ \ \ \ have one girl.\textsc{neut} kissed \ \ not \ \ \ two
\end{styleFooter}

\begin{styleFooter}
\ \ \ \ ‘I have kissed one girl (not two).’
\end{styleFooter}

\begin{styleFooter}
Below, I argue that singularity in (18a) does not come from \textit{ein} itself – the latter is proposed to be a semantically vacuous element. Note again that this is consistent with the indefinite article occuring in plural contexts as illustrated in the previous chapters and discussed in more detail in Section 4 below. In contrast, the singularity numeral has semantics, and it is proposed below that it consists of (vacuous) \textit{ein} and a contentful element.
\end{styleFooter}

\begin{styleFooter}
\ \ There are other cases where \textit{ein} is not associated with singularity (also Schoorlemmer 2009: 197 fn. 17). It is clear that \textit{ein} as part of the possessive articles or the negative article has no relevance with regard to semantic singularity either. In fact, these two \textit{ein}{}-words can take head nouns with plural morphology (19a-b). Note also that a predicate noun phrase does not denote an entity but a property (19c). As such, semantic singularity is not a relevant characteristic here either (see Chapter 7): 
\end{styleFooter}

\begin{styleFooter}
(19)\ \ a.\ \ \textit{Ich fahre m-eine Autos}.
\end{styleFooter}

\begin{styleFooter}
\ \ \ \ I \ \ \ drive \ \textsc{poss}{}-a cars
\end{styleFooter}

\begin{styleFooter}
\ \ \ \ ‘I drive my cars.’
\end{styleFooter}

\begin{styleFooter}
\ \ b.\ \ \textit{Ich fahre k-eine Autos}.
\end{styleFooter}

\begin{styleFooter}
\ \ \ \ I \ \ \ drive \ \textsc{neg}{}-a cars
\end{styleFooter}

\begin{styleFooter}
\ \ \ \ ‘I drive no cars.’
\end{styleFooter}

\begin{styleFooter}
\ \ c.\ \ \textit{BMW \ ist (ei)n Auto}.
\end{styleFooter}

\begin{styleFooter}
\ \ \ \ BMW is \ \ \textsubscript{\ }a \ \ \ \ car.\textsc{neut}
\end{styleFooter}

\begin{styleFooter}
\ \ \ \ ‘BMW is a car.’
\end{styleFooter}

\begin{styleFooter}
Note in passing that the cases above also show that \textit{ein} cannot be related to indefiniteness.
\end{styleFooter}

\begin{styleFooter}
Finally, nominalized infinitives and generic noun phrases make no claim about singularity of the relevant event or entity either (the data are adopted from Bisle-Müller 1991: 115, 151):\footnote{\ While a nominalized infinitive is not compatible with the singularity numeral (20b), it is fine with adjectival \textit{einmalig }‘one-time’ preceded by the indefinite article:\par \ \ \ (i)\ \ \textit{(Ei)n} \textit{einmaliges} \textit{Abweichen \ \ \ \ \ \ vom \ \ \ \ \ \ \ Kurs \ \ }\textit{\textsubscript{\ }}\textit{ist verzeihbar}.\par \ \ \ \  a \ \ \ \ \textsubscript{\ }one-time \ \ \ departing.\textsc{neut} from.the course is \ forgivable\par \ \ \ \ ‘Departing from one’s course one time is forgivable.’\par Presumably, this has to do with the event structure of the nominal. Also, note that (20a), (20c), and (i) allow \textit{ein} to be replaced by a definite article (triggering a weak inflection on the following adjective in (i)).}
\end{styleFooter}

\begin{styleFooter}
(20)\ \ a.\ \ \textit{(Ei)n Abweichen \ \ \ \ \ \ \ vom \ \ \ \ \ \ \ Kurs \ \ }\textit{\textsubscript{\ }}\textit{ist nicht gut}.
\end{styleFooter}

\begin{styleFooter}
\ \ \ \  \ a \ \ \ \ departing.\textsc{neut} from.the course is \ not \ \ \ good
\end{styleFooter}

\begin{styleFooter}
\ \ \ \ ‘Departing from one’s course is not good.’
\end{styleFooter}

\begin{styleFooter}
\ \ b. \textit{*\ \ EIN Abweichen \ \ \ \ \ \ \ \ vom \ \ \ \ \ \ \ Kurs \ \ }\textit{\textsubscript{\ }}\textit{ist nicht gut}.
\end{styleFooter}

\begin{styleFooter}
\ \ \ \ one \ departing.\textsc{neut} from.the course is \ not \ \ \ good
\end{styleFooter}

\begin{styleFooter}
\ \ \ \ ‘Departing from one’s course is not good.’
\end{styleFooter}

\begin{styleFooter}
\ \ c.\ \ \textit{(Ei)n Wal \ \ \ \ \ \ \ \ \ \ \ \ \ ist ein Säugetier.}
\end{styleFooter}

\begin{styleFooter}
\ \ \ \  \ a \ \ \ \ whale.\textsc{masc} is \ \textsubscript{\ }a \ \ \ mammal
\end{styleFooter}

\begin{styleFooter}
\ \ \ \ ‘A whale is a mammal.’
\end{styleFooter}

\begin{styleFooter}
\ \ d. \ \textit{*\ \ EIN Wal \ \ \ \ \ \ \ \ \ \ \ \ \ \ ist ein Säugetier.}
\end{styleFooter}

\begin{styleFooter}
\ \ \ \ one \ whale.\textsc{masc} is \ \textsubscript{\ }a \ \ \ mammal
\end{styleFooter}

\begin{styleFooter}
\ \ \ \ ‘A whale is a mammal.’
\end{styleFooter}

\begin{styleFooter}
As to adjectival \textit{eine}, note again that like the numeral \textit{EIN} in (21a), this type of \textit{ein} is stressed (21b) (recall though that I do not mark it as such). Crucially, unlike the numeral, adjectival \textit{eine} usually presupposes the existence of a second entity and thus implies a certain plurality of the members of the relevant kind. In fact, as observed by M. Müller (1986: 43), \textit{eine} in (21b) has a partitive sense, presupposing a set of typically two entities in the relevant world of discourse (cf. also Vater 1982: 71). I indicated this in the English translation:
\end{styleFooter}

\begin{styleFooter}
(21)\ \ a.\ \ \textit{EIN Mann}
\end{styleFooter}

\begin{styleFooter}
\ \ \ \ one \ man.\textsc{masc}
\end{styleFooter}

\begin{styleStandard}
‘one man’
\end{styleStandard}

\begin{styleFooter}
\ \ b.\ \ \textit{der eine Mann}
\end{styleFooter}

\begin{styleFooter}
\ \ \ \ the \textsubscript{\ }one \ man.\textsc{masc}
\end{styleFooter}

\begin{styleStandard}
‘one of the two men’
\end{styleStandard}

\begin{styleFooter}
Note that this duality presupposition cannot come from the definite article – the latter typically presupposes uniqueness in singular contexts. Importantly, \textit{eine} must be preceded by a definite element, and it is often contrasted with a second DP containing \textit{andere} ‘other’:\footnote{\ This construction already existed in OHG (ia). Furthermore, this distribution is also possible in related Yiddish (ib) (from Reershemius 1997: 362):\par 
\setcounter{listWWviiiNumxvileveli}{0}
\begin{listWWviiiNumxvileveli}
\item 
\begin{styleFootnote}
a.\ \ \textit{ther eino – ther ander}\ \ \ \ (OHG)
\end{styleFootnote}
\end{listWWviiiNumxvileveli}
the \ one \ \ \ \ \ the \ other\par ‘one (of the two) – the other’\par \ \ (Otfrid, Braune \& Reiffenstein 2004: 234)\par b.\ \ \textit{…tsvey brider …Der eyner hot }\textit{\textsubscript{\ }}\textit{zikh \ ungerufn Elon un \ }\textit{\textsubscript{\ }}\textit{der tsveyter Aladan}. \ \ \ (Yiddish)\par \ \ \ \textsubscript{\ }two \ \ brothers the \ \textsubscript{\ }one \ \ \ has \textsc{refl} called \ \ \ \ \ Elon and the second \ \textsubscript{\ }Aladan\par ‘…two brothers… One (of the two) was called Elon and the other Aladan.’}
\end{styleFooter}

\begin{styleFooter}
(22)\ \ \textit{Der eine Mann \ \ \ \ \ \ \ \ kam, der andere nicht}.
\end{styleFooter}

\begin{styleFooter}
\ \ the \ one \ man.\textsc{masc} came the other \ \ \ not
\end{styleFooter}

\begin{styleStandard}
‘One of the (two) men came, the other did not.’
\end{styleStandard}

\begin{styleFooter}
Finally, like \textit{ein} in the possessive and negative composites, adjectival \textit{eine} can also be morphologically plural. In this case, adjectival \textit{eine} presupposes two sets of elements. Compare (22) to (23a).\textstyleFootnoteSymbol{ }In fact, as pointed out by a reviewer, it is possible to combine a set involving one member with a set involving multiple members (23b) (for Swedish in this regard, see Börjars 1998: 18 fn. 7):
\end{styleFooter}

\begin{styleFooter}
(23)\ \ a.\ \ \textit{die \ \ \ \ \ einen, die \ \ \ \ \ anderen}
\end{styleFooter}

\begin{styleFooter}
\ \ \ \ the.\textsc{pl} one \ \ \ \ the.\textsc{pl} other
\end{styleFooter}

\begin{styleFooter}
\ \ \ \ ‘these, those’
\end{styleFooter}

\begin{styleFooter}
\ \ b.\ \ \textit{Der eine Sohn arbeitet und die drei \ anderen machen Pause}.
\end{styleFooter}

\begin{styleFootnote}
\ \ \ \ the \ one \ son \ \ works \ \ \ and the three others \ \ \ take \ \ \ \ \ \ break
\end{styleFootnote}

\begin{styleFootnote}
\ \ \ \ ‘One of the sons is working, and the three others are taking a break.’
\end{styleFootnote}

\begin{styleFooter}
This means that the duality has to do with two sets that may differ from one another in size. Finally, we see in Section 4.3.1 below that the duality presupposition associated with adjectival \textit{eine} can be cancelled.
\end{styleFooter}

\begin{styleFooter}
The differences discussed above are summarized in Table 1 (the properties are coded as follows: OK indicates an optional property; +/- signifies an inherent characteristic; N.A. means that this criterion is not applicable; I comment on the use of the parentheses in Table 1 in the next paragraph):
\end{styleFooter}

\begin{styleFooter}
Table 1: Summary of the Differences between the Types of \textit{ein}
\end{styleFooter}

\begin{flushleft}
\begin{tabular}{|m{0.5587598in}|m{0.6712598in}m{0.8587598in}|m{0.6087598in}|m{0.48375985in}|m{1.3587599in}|m{1.5587599in}|}

\hline
\multicolumn{3}{|m{2.24626in}|}{Kinds of \textit{ein}} &
\centering Enclitic &
\centering Stress &
\centering Sem. singularity &
\centering\arraybslash Morphological plural\\\hline
Article &
\multicolumn{2}{m{1.6087599in}|}{Indefinite} &
\centering OK &
\centering {}- &
\centering (+) &
\centering\arraybslash (-)\\\hline
 &
\multicolumn{1}{m{0.6712598in}|}{Vacuous} &
Possessive &
\centering {}- &
\centering OK &
\centering N.A. &
\centering\arraybslash OK\\\hline
 &
 &
Negative &
\centering {}- &
\centering OK &
\centering N.A. &
\centering\arraybslash OK\\\hline
 &
 &
Predicative &
\centering OK &
\centering OK &
\centering N.A. &
\centering\arraybslash (-)\\\hline
\multicolumn{3}{|m{2.24626in}|}{Numeral} &
\centering {}- &
\centering + &
\centering + &
\centering\arraybslash {}-\\\hline
\multicolumn{3}{|m{2.24626in}|}{Adjective} &
\centering {}- &
\centering + &
\centering OK (with canceled presupposition) &
\centering\arraybslash OK\\\hline
\end{tabular}
\end{flushleft}
\begin{styleFooter}
These are the most typical properties. Having set out the basic similarities and differences, I turn to accounting for them. In the course of the following discussion, I refine the statements about the indefinite article, especially with regard to semantic and morphological number (for predicative \textit{ein}, see Chapters 6 and 7). Specifically, I propose that the article \textit{ein} is semantically vacuous (Hypothesis 1a), and I illustrate again that it can occur in morphologically plural contexts. The upcoming refinement of the statements about number is indicated in Table 1 by parentheses (that is, it is argued below that the indefinite article is not specified for number). 
\end{styleFooter}

\begin{styleFooter}\bfseries
4. \ \ Step 1 of the Proposal: Morphology and Semantics
\end{styleFooter}

\begin{styleFooter}
Recall the generalization from Section 1 that determiners may, independently of word order, not co-occur. Focusing on \textit{ein}{}-words, the definite article, and the demonstratives, this is illustrated again below (recall that \textit{die}, glossed below as DET, comprises the definite article \textit{die} and its related distal demonstrative \textit{DIE}): 
\end{styleFooter}

\begin{styleFooter}
(24) \ *\ \ \textit{keine / meine / ‘ne / die \ / diese Freundin}
\end{styleFooter}

\begin{styleFooter}
\ \ no \ \ \ \ \ / my \ \ \ \ / \ \textsubscript{\ }a \ / \textsc{det} / this \ \ girlfriend.\textsc{fem}
\end{styleFooter}

\begin{styleFooter}
There were basically two potential exceptions to this generalization: (i) definite determiners may occur with stressed \textit{eine} (25a-c), and (ii) \textit{diese} ‘this’ can occur with a possessive article (25d). In Section 4.3.1, I discuss the different interpretations of the strings in (25a-b) in detail:\footnote{\ In poetic or elevated German, a possessive element can also be combined with a definite article (ia). Given the presence of the article and the weak inflection on the possessive element, this lower possessive is presumably an adjective. Note that this syntactic distribution is familiar from older varieties of German (ib) ((ib) is taken from Harbert 2007: 155):\par \ \ (i)\ \ a.\ \ \textit{Du \ bist} \textit{d-er \ \ \ mein-e.}\par \ \ \ \ \ \ you are \textsubscript{\ }the-\textsc{st} my-\textsc{wk} \par \ \ \ \ \ \ ‘You are mine.’\par b.\ \ \textit{in dheru sineru heilegun chiburdi\ \ }\ \ (OHG)\par in the \ \ \ \ \textsubscript{\ }his \ \ \ \ \ \textsubscript{\ }holy \ \ \ \ \ \ \textsubscript{\ }birth\par ‘in his holy birth’\par Given these points, I assume that (ia) is part of a different grammar.}
\end{styleFooter}

\begin{styleFooter}
(25)\ \ a.\ \ \textit{meine eine Freundin}
\end{styleFooter}

\begin{styleFooter}
\ \ \ \ my \ \ \ \ one \ girlfriend.\textsc{fem}
\end{styleFooter}

\begin{styleFooter}
‘one of the two of my girlfriends’
\end{styleFooter}

\begin{styleFooter}
\ \ \ \ ‘my one girlfriend’
\end{styleFooter}

\begin{styleFooter}
\ \ b.\ \ \textit{die \ eine Freundin}
\end{styleFooter}

\begin{styleFooter}
\ \ \ \ \textsc{det} one \ girlfriend.\textsc{fem}
\end{styleFooter}

\begin{styleFooter}
\ \ \ \ ‘one of the two girlfriends’ 
\end{styleFooter}

\begin{styleFooter}
‘that one girlfriend’
\end{styleFooter}

\begin{styleFooter}
\ \ c.\ \ \textit{diese eine Freundin}
\end{styleFooter}

\begin{styleFooter}
\ \ \ \ this \ \ one \ girlfriend.\textsc{fem}
\end{styleFooter}

\begin{styleFooter}
\ \ \ \ ‘this one girlfriend’
\end{styleFooter}

\begin{styleFooter}
\ \ d.\ \ \textit{diese meine Freundin}
\end{styleFooter}

\begin{styleFooter}
\ \ \ \ this \ \ my \ \ \ \ \ girlfriend.\textsc{fem}
\end{styleFooter}

\begin{styleFooter}
\ \ \ \ ‘this girlfriend of mine’
\end{styleFooter}

\begin{styleFooter}
Furthermore, I showed in Section 2 that the indefinite article (including predicative \textit{ein}), the possessive articles, the negative article, and the singularity numeral exhibit the same inflectional behavior. In other words, semantically quite diverse elements behave morphologically the same. In this and the next section, I account for these and some other facts. I provide a brief preview of the account of \textit{ein}. 
\end{styleFooter}

\begin{styleFooter}
Starting with (24), I follow much discussion in the literature and assume that indefinite and definite articles are in D and that demonstratives are in Spec,DP (e.g., Alexiadou \textit{et al}. 2007: 105-20; Bernstein 1997, 2001b; Giusti 1997, 2002; Leu 2007, 2015; Roehrs 2010; van Gelderen 2007; see also Chapter 1, Section 4.1.2). Furthermore, it is well documented for German that only one such element can be in the DP-level. If so, this restriction accounts for the non-co-occurrence of these three elements in the same DP.\footnote{\ Usually, the Doubly-filled DP Filter is brought into play here (for some discussion, see Abney 1987: 271; Giusti 1997: 109, 2002: 70). In Roehrs (2019), I propose that unlike the North Germanic languages, the West Germanic languages have only one (complex) feature bundle for definiteness explaining why there is only one element showing definiteness in the DP in the latter type of language. Note that this account makes no claim about the co-occurrence of a definite and a semantically vacuous element (see the discussion of \textit{m+ein} ‘my’ below).} Turning to the two remaining elements in (24), that is, to the negative and the possessive articles, I develop a composite analysis of \textit{keine }and \textit{meine} (and other elements) below arguing that these elements consist of vacuous \textit{ein} and an abstract component denoting negation or possession. Among others, this proposal explains, on the one hand, the non-co-occurrence of the latter two elements with articles and demonstratives and, on the other hand, the identical morphology of the various composites and \textit{ein}.
\end{styleFooter}

\begin{styleFooter}
\ \ If this is on the right track, then \textit{eine} in (25a-c) cannot be derived from vacuous \textit{ein} and must be a different element. I propose that this type of \textit{ein} is an adjective in a high specifier position.\footnote{\ That this type of \textit{ein} is indeed special is quite clear in Norwegian. As is well known, this language has a singularity numeral that has the neuter form \textit{ett} ‘one’ (ia). However, in the construction under discussion, the expected form \textit{ette} is impossible, and only a – what looks like – non-neuter form can be used (ib) (Marit Julien, p.c.):\par 
\setcounter{listWWviiiNumivleveli}{0}
\begin{listWWviiiNumivleveli}
\item 
\begin{styleStandard}
a.\ \ \textit{ett \ \ stor-t \ hus}\ \ \ \ \ \ \ \ (Norwegian)
\end{styleStandard}
\end{listWWviiiNumivleveli}
\ \ \ \ \ \ one big-\textsc{st} house.\textsc{neut}\par \ \ \ \ \ \ ‘one big house’\par \ \ \ \ b.\ \ \textit{det en-e \ \ \ \ \ \ stor-e \ \ hus-et}\par \ \ \ \ \ \ the one-\textsc{wk} big-\textsc{wk} house.\textsc{neut}{}-\textsc{def}\par \ \ \ \ \ \ ‘one of the two big houses’\par For Icelandic in this regard, see Pfaff (2015: 41 fn. 20).} If so, the distributions in (25a-c) do not present exceptions to the generalization about the non-co-occurrence of determiners. Turning to (25d), this datum has greater potential of being an exception to this generalization. However, as already discussed in Chapter 2, Section 4, this type of \textit{diese} is not in Spec,DP. Rather, it is a predeterminer, a semantic intensifier, in LPP (also Section 5.2.2). Thus, there is only one element in the DP-level here as well. 
\end{styleFooter}

\begin{styleFooter}
These initial remarks can be summarized in the following simplified structure where all elements just discussed are put in their surface positions. Disregarding restrictions on the co-occurrence of these elements, this one structure illustrates the left part of the noun phrase in German. For completeness, I have added the semantic component of the singularity numeral \textit{EIN} ‘one’, illustrated below as \textit{Ø}\textsubscript{[-PL]} (see Section 5.1). Note that NEG is the abstract element later to be spelled out as \textit{nicht} ‘not’ or \textit{k(-ein)} ‘no’ (see Section 4.1.2):
\end{styleFooter}

\begin{styleFootnote}
(26)\ \ \ \ LPP
\end{styleFootnote}

\begin{styleFootnote}
[Warning: Draw object ignored][Warning: Draw object ignored]
\end{styleFootnote}

\begin{styleFootnote}
\ \  NEG\ \ \ \ LPP
\end{styleFootnote}

\begin{styleFootnote}
[Warning: Draw object ignored][Warning: Draw object ignored]
\end{styleFootnote}

\begin{styleFootnote}
\ \  \ \ \ \ \ \ \ \ \ \ \ \textit{diese}\ \ \ \  DP
\end{styleFootnote}

\begin{styleFootnote}
[Warning: Draw object ignored][Warning: Draw object ignored]
\end{styleFootnote}

\begin{styleFootnote}
\ \ \ \ \ \ \textit{diese}\ \ \ \  \ D’
\end{styleFootnote}

\begin{styleFootnote}
[Warning: Draw object ignored][Warning: Draw object ignored]\ \ \ \ \ \ \textit{DIE}
\end{styleFootnote}

\begin{styleFootnote}
\ \ \ \ \ \ \textit{m}{}-\ \ D\ \ \ \ CardP
\end{styleFootnote}

\begin{styleFootnote}
[Warning: Draw object ignored][Warning: Draw object ignored]\ \ \ \ \ \  \ \ \ \ \ \ \ \ \ \ \textit{die}
\end{styleFootnote}

\begin{styleFootnote}
\ \ \ \ \ \  \ \ \ \ \ \ \ \textit{(ei)ne}\ \  \ \textit{Ø}\textsubscript{[-PL]}\ \ \ \ Card’
\end{styleFootnote}

\begin{styleFootnote}
[Warning: Draw object ignored][Warning: Draw object ignored]
\end{styleFootnote}

\begin{styleFootnote}
\ \ \ \ \ \ \ \ \ \ \ \ Card\ \ \ \ AgrP
\end{styleFootnote}

\begin{styleFootnote}
[Warning: Draw object ignored][Warning: Draw object ignored]
\end{styleFootnote}

\begin{styleFootnote}
\textit{\ \ \ \ \ \ \ \ \ \ eine}\textsubscript{ADJ}\ \ \ \ Agr’
\end{styleFootnote}

\begin{styleFootnote}
[Warning: Draw object ignored][Warning: Draw object ignored]\ \ 
\end{styleFootnote}

\begin{styleFootnote}
\ \ \ \ \ \ \ \ \ \ \ \ \ \ \ \ \ \ \ \ \ \ \ \ \ \ \ \ \ \ \ \ \ \ \ \ \ \ \ \ \ \ \ \ \ \ \ \ \ \ \ \ \ \ \ \ \ \ \ \ \ \ \ \ \ \ \ \ \ \ \ \ \ \ \ \ \ \ \ \ \ \ \ \ \ \ \ \ \ \ \ \ \ \ \ Agr\ \ \ \  …
\end{styleFootnote}

\begin{styleFootnote}
To anticipate the structural discussion below, I assume that only one element with the categorial feature [+D] can be merged in ArtP. Given that determiners move to the DP-level, this explains why only one determiner can surface in this layer: \textit{diese}, \textit{DIE}, \textit{die}, \textit{(ei)ne}. Note now that NEG, the predeterminer \textit{diese}, possessives, the singularity component \textit{Ø}\textsubscript{[-PL]}, and adjectival \textit{eine} are not merged in ArtP and can, at least in principle, co-occur with the determiners. Specifically, NEG is adjoined to LPP, the predeterminer \textit{diese} is base-generated in Spec,LPP, possessives move from Spec,NP to Spec,DP, and the singularity component is base-generated in Spec,CardP. Focusing on the different types of \textit{ein}, it will become clear in the next section that the negative article, the possessive articles, and the singularity numeral are restricted to the occurrence of \textit{ein }– these elements form composites. Finally, adjectival \textit{eine} is base-generated in the specifier position of a high AgrP and is restricted to definite contexts. These general ideas are fleshed out in what follows.
\end{styleFootnote}

\begin{styleFooter}
\textit{4.1.\ \ Composite Elements: Article }ein\textit{ as a Supporting and Flagging Element}
\end{styleFooter}

\begin{styleFooter}
I start by providing my basic proposal. This is followed by a detailed discussion of the derivations of the different \textit{ein}{}-words. Finally, I point out some advantages and initial consequences of this proposal.
\end{styleFooter}

\begin{styleFooter}
4.1.1. Basic Proposal
\end{styleFooter}

\begin{styleFooter}
Starting with \textit{ein} itself, I provided a fair amount of empirical evidence above that this element is special. Among others, it can occur not only in singular but also plural contexts. Furthermore, \textit{ein} can surface not only in indefinite but also definite environments. This means that \textit{ein} is not a singular indefinite article. Rather, I propose that it is a semantically vacuous element (Hypothesis 1a). This is in agreement with some of the semantic literature. For instance, Heim \& Kratzer (1998: 62) state that the indefinite article can be defined as follows: [[\textbf{a}]] = $\lambda $f ${\in}$ D\textsubscript{{\textless}e,t{\textgreater}} . f, which maps every function in D\textsubscript{{\textless}e,t{\textgreater}} to itself. In other words, \textit{ein} makes no new semantic contribution. Similarly, Coppock \& Beaver (2015: 400-01) assume that the indefinite article is a vacuous identity function on predicates: \textit{a(n)} = $\lambda $\textit{P.P} (where like in Heim \& Kratzer the input and the output are the same predicate).\footnote{\ The indefinite article is often related to its definite counterpart. For instance, Giusti (2015: 177-81) argues that definite articles have nothing to do with definiteness or the iota operator. She proposes that definite (but also indefinite) articles are not determiners but high segments of an N-projection; in other words, they spell out inflectional features of a scattered, reprojected N-head (and in some cases those of an adjective). Giusti (2015: 208) analyzes the definite article in German as \textit{d-er} ‘the’, where \textit{d}{}- is a dummy (see also Roehrs 2013a) and -\textit{er} involves features of N. Extending this to the indefinite article, as Guisti briefly mentions, this is compatible with the current claim that \textit{ein} is semantically vacuous (note though that in Giusti 2015: 208, the definite article and the indefinite article in German are in different positions – D vs. Spec,DP, respectively). Finally, note that the indefinite article is sometimes also assumed to be semantically vacuous in other languages. For instance, Matushansky \& Spector (2005: 245) state that the indefinite article in post-copular position in French is semantically vacuous and a reflex of a syntactic operation.} 
\end{styleFooter}

\begin{styleFooter}
Turning to the other \textit{ein}{}-words, it was suggested in Roehrs (2009a: chapter 4) that \textit{ein} is part of the negative article \textit{kein }‘no’, possessive articles like \textit{mein }‘my’,\textit{ }and the singularity numeral \textit{EIN} ‘one’ (27a-c). Below, this is instantiated by proposing that \textit{ein} supports certain operators that are detectable, segmentally as in the case of NEG \textit{k}{}- and POSS \textit{m}{}- or suprasegmentally as in the case of \textit{Ø}\textsubscript{[-PL]}, which induces stress.\footnote{\ I define operator broadly here. They involve scopal elements (negators, quantifiers), indexical elements (possessives), and other semantically active, functional elements (see REL below and certain other elements in subsequent chapters).} We will see in Chapter 6 that \textit{ein} also indicates the presence of the realization operator REL (27d). Unlike (27a-c), here \textit{ein} flags the presence of an operator that has no independent manifestation. I indicate the co-occurrence relation of \textit{ein} and REL by an underscore. Completing the picture, I propose that unlike these four elements, adjectival \textit{eine} is an independent element (27e):
\end{styleFooter}

\begin{styleFooter}
(27)\ \ a.\ \ (vacuous) \textit{ein} + NEG\textit{ \ \ \ \ \ \ }→ \ \ \textit{kein}
\end{styleFooter}

\begin{styleFooter}
b. \ \ (vacuous) \textit{ein }+ POSS\textsubscript{[1ST}\textsc{\textsubscript{ }}\textsubscript{PERS -PL]} \ \ → \ \ \textit{mein} 
\end{styleFooter}

\begin{styleFooter}
c. \ \ (vacuous) \textit{ein} + \textit{Ø}\textsubscript{[{}-PL]} \ \ \ \ \ \ → \ \ \textit{EIN} 
\end{styleFooter}

\begin{styleFooter}
d.\ \ (vacuous) \textit{ein }\_ REL\ \ \ \ \ \ → \ \ \textit{ein}
\end{styleFooter}

\begin{styleFooter}
\ \ e.\ \ (non-composite) \textit{eine}\textsubscript{ADJ}
\end{styleFooter}

\begin{styleFooter}
The claim that (27a-b) are composite forms is by no means new; for instance, Fanselow \& Cavar (2002) and Kobele \& Zimmermann (2012: 247) employ (27a), and Corver (2003: 4) and Corver \& van Koppen (2010: 114) assume (27b) in their discussion of Dutch (cf. also Julien 2016: 94-95 for certain possessives in Scandinavian). Murphy (2018: 331) and Schoorlemmer (2009: 197 fn. 17) accept both (27a) and (27b) for German. The analysis in (27c) might be relatable to Ackles (1996), who proposes that English \textit{a(n)} licenses NumP, an element without overt realization, in the context of singular count nouns. (27d) is also in good company; for instance, Bennis \textit{et al}. (1998) claim that the operator [+EXCL] needs to be lexicalized by \textit{een} ‘a’ in Dutch (Chapter 4, Section 2.1). Finally, adjectival \textit{eine} is proposed by Gallmann (2004: 155), Lindauer (1995: 160-61), M. Müller (1986: 45), and Pafel (2005: 179). Note that these related works lend some initial credence to the current proposal (more references are provided in the course of the following discussion). Consider the derivation of the composite elements in (27a-c) in the framework of DM.
\end{styleFooter}

\begin{styleFooter}
4.1.2. Derivation of Composite Forms
\end{styleFooter}

\begin{styleFooter}
Beginning with \textit{ein }itself, I proposed in Chapter 2, Section 2.1.6 that this element consists of the categorial feature [+D] and features for case, number, and gender (I leave the values unspecified below). Unlike the definite article and the demonstratives, \textit{ein} has no features for definiteness or deixis. As such, this is the least specified determiner. This fits well with the proposal above that \textit{ein} is a semantically vacuous element:
\end{styleFooter}

\begin{styleStandard}
(28)\ \ \textit{Indefinite Article} ein
\end{styleStandard}

\begin{styleStandard}
\ \ \ \ Art\textsubscript{[+D][F, N, O, S]}
\end{styleStandard}

\begin{styleStandard}
[Warning: Draw object ignored][Warning: Draw object ignored]
\end{styleStandard}

\begin{styleStandard}
\ \ [+D]\ \ \ \ [F, N, O, S]\ \ \ \ 
\end{styleStandard}

\begin{styleFooter}
I turn to the operators in (27a-c). Starting with the possessive articles, recall from Chapter 2, Section 2.2.3 that I analyze these elements as bound or free morphemes (29a-b).\footnote{\ A reviewer remarks that it is unusual that determiner stems of the same kind seem to fall into bound and free morphemes (but see also personal pronouns mentioned in Footnote Error: Reference source not found below). Note that bound in this case means a bipartite stem (e.g., \textit{m-ein}) and free signifies a monopartite stem (e.g., \textit{ihr}). While I flesh out this oft-suggested analysis below, there are alternatives. Note in this regard that unlike accusative/dative pronouns, all monopartite possessive articles consist of \textit{(e)r} at the end: \textit{ih-r} ‘her; their’ vs. \textit{ih-n/ih-m} ‘him’; \textit{uns-er} ‘our’ vs. \textit{uns} ‘us’; \textit{eu-er} ‘your’ vs. \textit{eu-ch} ‘you’. A different account than the one proposed below (where I suggest that the pronunciation of \textit{ein} is suppressed in the context of free/monopartite possessives) would be to claim that \textit{ein} is actually spelled out as -\textit{r} in the relevant possessive contexts – another case of contextually conditioned allomorphy. This would mean that all possessive articles involve bound/bipartite forms. I proceed on more traditional assumptions.} I point out now that these forms are more general (also Fischer 2006). For instance, the bound morphemes in (29a) also occur as part of reflexive pronominal forms (29c). In fact, this decomposition is quite pervasive with personal pronouns (29d) (I leave out the genitive forms, which are diachronically related to the possessive forms in (29a)):\footnote{\ Note that third-person \textit{s}{}- is somewhat special: it only involves masculine with possessives (29a) but masculine and/or other specifications with reflexives (29c) and personal pronouns (29d). Notice also that \textit{S-ie} ‘you(\textsc{formal})’, not provided in (29), is morphologically third but semantically second person (see Chapter 7). Furthermore, recall that the third-person pronoun \textit{sie} ‘she, her; they, them’ was claimed to take an adjectival inflection that is later deleted due to the avoidance of a hiatus (Chapter 3, Section 5.3.2). In other words, given current assumptions, this pronoun has the underlying form of \textit{s-ie-e}. Compared to third-person singular \textit{e-r} ‘he’, we find another case of bound/bipartite vs. free/monopartite determiner stems.} 
\end{styleFooter}

\begin{styleFootnote}
(29)\ \ a.\ \ \textit{m}{}-, \ \ \textit{d}{}-, \ \ \ \ \ \ \ \ \ \ \ \ \ \ \textit{s}{}-
\end{styleFootnote}

\begin{styleFootnote}
\ \ \ \ my-, your(\textsc{sgl})-, his-
\end{styleFootnote}

\begin{styleFootnote}
\ \ b.\ \ \textit{ihr}, \ \ \ \ \ \ \ \ \ \ \ \ \ \ \ \ \ \ \ \ \ \ \ \ \ \ \ \ \ \ \ \ \ \textit{unser}, \textit{euer}
\end{styleFootnote}

\begin{styleFootnote}
\ \ \ \ her/their/your(\textsc{formal}), our, \ \ \ \ your(\textsc{pl})
\end{styleFootnote}

\begin{styleFooter}
\ \ c.\ \ \textit{m-ich}, \ \ \textit{d-ich}, \ \ \ \ \ \textit{s-ich} 
\end{styleFooter}

\begin{styleFooter}
myself, yourself, himself/herself/themselves
\end{styleFooter}

\begin{styleFooter}
‘myself, yourself, himself/herself/themselves’
\end{styleFooter}

\begin{styleFooter}
\ \ d.\ \ \textit{m-ich}, \ \ \ \textit{m-ir}; \ \ \ \ \ \ \textit{d-u}, \ \ \ \ \ \ \ \ \ \ \ \ \ \ \ \ \ \ \textit{d-ich}, \ \ \ \ \ \ \ \ \ \ \ \ \ \ \textit{d-ir}; \ \ \ \ \ \ \ \ \ \ \ \ \ \ \ \ \textit{s-ie}
\end{styleFooter}

\begin{styleFooter}
me.\textsc{acc, }me.\textsc{dat}; you.\textsc{nom}(\textsc{sgl}), you.\textsc{acc}(\textsc{sgl}), you.\textsc{dat}(\textsc{sgl}); she.\textsc{nom}/her.\textsc{acc}/ they.\textsc{nom}/them.\textsc{acc}
\end{styleFooter}

\begin{styleFooter}
‘me; you; she/her/they/them’
\end{styleFooter}

\begin{styleFooter}
There is another subregularity here in that all elements in (29b) involve feminine and plural forms, which tend to pattern together in the nominal paradigms (Chapter 2, Section 2.1.5). Taken together, this provides some empirical motivation for separating possessive articles into bound and free morphemes in (29a) and (29b), respectively.
\end{styleFooter}

\begin{styleFooter}
\ \ In Roehrs (2020a), I propose that possessives involve a Possessive Phrase (PossP) and that they are base-generated low in the nominal structure. Following standard assumptions, I assume that prenominal possessives move from Spec,NP to Spec,DP (Chapter 1, Section 4.1.2). After movement of the article and PossP, the relevant part of the structure, the DP-level, can be illustrated as follows:
\end{styleFooter}

\begin{styleStandard}
(30)\ \ \ \ \ \ DP
\end{styleStandard}

\begin{styleStandard}
[Warning: Draw object ignored][Warning: Draw object ignored]
\end{styleStandard}

\begin{styleStandard}
\ \  \ \ \ \ \ \ \ \ PossP\textsubscript{k}\ \ \ \ D’\ \ 
\end{styleStandard}

\begin{styleFooter}
[Warning: Draw object ignored][Warning: Draw object ignored][Warning: Draw object ignored]\ \ \ \ 
\end{styleFooter}

\begin{styleFooter}
\ \  \ \ \ \ \ \ \ \ Poss\ \ D\ \  \ \ \ \ \ \ \ \ [… t\textsubscript{i} … t\textsubscript{k} …]
\end{styleFooter}

\begin{styleFooter}
[Warning: Draw object ignored][Warning: Draw object ignored]\ \ \ \ 
\end{styleFooter}

\begin{styleFooter}
\ \  \ \ \ \ \ \ \ \ \ Art\textsubscript{i}\ \ \ \ D
\end{styleFooter}

\begin{styleFooter}
[Warning: Draw object ignored][Warning: Draw object ignored]
\end{styleFooter}

\begin{styleStandard}
\ \ \ \ \ \ \ \ \ [+D]\ \  \ \ \ \ \ \ [F, N, O, S]\ \ \ \ 
\end{styleStandard}

\begin{styleFooter}
After Linearization of (30), Vocabulary Insertion occurs. There are two types of cases involving possessive articles.
\end{styleFooter}

\begin{styleFooter}
\ \ First, after bound possessive morphemes like \textit{m}{}- in (29a) are inserted in the left periphery of the nominal, the string in (31) is obtained. In Chapter 2, Section 2.2.2, I formulated specific vocabulary insertion rules for \textit{ein}. If an overt element like an adjective and/or noun follows, then \textit{ein} in nominative masculine and nominative/accusative neuter contexts spells out both the categorial feature [+D] and the CNG feature bundle (cf. (30) above) yielding uninflected \textit{ein}; if no such element follows, then \textit{ein} spells out [+D], and an inflectional suffix spells out [F, N, O, S]. The latter scenario yields inflected forms of \textit{ein}. Both options are illustrated below by putting the inflection in parentheses:
\end{styleFooter}

\begin{styleFooter}
[Warning: Draw object ignored][Warning: Draw object ignored][Warning: Draw object ignored](31)\ \ 
\end{styleFooter}

\begin{styleFooter}
\ \ \textit{m-\ \ ein} \ \ \ (-\textsc{infl})
\end{styleFooter}

\begin{styleFooter}
I propose that \textit{ein} supports both the bound morpheme \textit{m}{}- and the inflection (unless \textit{ein} spells out both [+D] and the CNG feature bundle). Note now that all relevant elements are adjacent to each other. Given these local, linear relations, I suggest that the operation of support is instantiated by Local Dislocation such that \textit{ein} undergoes this operation onto \textit{m}{}- and that the inflection, if present, does so onto \textit{m-ein }(cf. Murphy 2018). 
\end{styleFooter}

\begin{styleFooter}
Second, the free possessive morphemes in (29b) involve the same derivation as in (30) above. After Linearization and Vocabulary Insertion, the following string obtains exemplifying with \textit{ihr} ‘her’:
\end{styleFooter}

\begin{styleFooter}
[Warning: Draw object ignored][Warning: Draw object ignored][Warning: Draw object ignored](32)\ \ 
\end{styleFooter}

\begin{styleFooter}
\ \ \textit{ihr\ \ ein} \ \ \ (-\textsc{infl})
\end{styleFooter}

\begin{styleFooter}
This is not the correct surface string yet. Note that there are two free morphemes in (32): \textit{ihr} and \textit{ein}. I suggest that only one free form can surface in this local domain. Given that the possessive has semantics but \textit{ein} does not, I assume that the latter does not appear:\footnote{\ In a traditional tree structure, the Doubly-filled DP Filter could be brought into play here. However, the framework adopted here (DM) inserts vocabulary items after Linearization. If the tree structure is no longer accessible after Linearization, then a different way needs to be found to rule out the co-occurrence of the free possessive element and \textit{ein}. \par Note in this regard that we cannot claim that \textit{ein} is only merged as a Last Resort option (NB: this is consistent with Chapter 4, Section 3.4, where I argued that \textit{ein} is unlikely to be inserted late); for instance, we cannot claim that \textit{ein} is inserted unless there is already another element present that can perform \textit{ein}’s two main functions (supporting and flagging). This is so because, if \textit{ein} were not inserted in the context of (32), then the CNG bundle would remain untouched in the three (/four) exceptional cases. As a consequence, the inflection would be spelled out and would appear on (i.e., be supported by) the possessive in every instance yielding ungrammatical forms such as *\textit{ihr-er} in, say, a nominative masculine context. Given that, a more promising way to deal with this issue would be to delete \textit{ein} in the context of [Poss, pron, +F], which covers feminine and plural possessive pronouns. Alternatively, we could follow Matushanksy (2006: 86-87) in suggesting that \textit{ein} and the possessive element undergo Fusion (into one element) in feminine and plural contexts.} 
\end{styleFooter}

\begin{styleFooter}
[Warning: Draw object ignored][Warning: Draw object ignored][Warning: Draw object ignored](33)\ \ 
\end{styleFooter}

\begin{styleFooter}
\ \ \textit{ihr\ \ ein} \ \ \ (-\textsc{infl})
\end{styleFooter}

\begin{styleFooter}
If an inflection is present (i.e., the CNG bundle of the article structure of \textit{ein} remains untouched by the vocabulary insertion rules of \textit{ein}), it will undergo Local Dislocation onto the possessive element. 
\end{styleFooter}

\begin{styleFooter}
As to the two remaining composite elements in (27b-c), I assume that \textit{kein} ‘no’ is similar to \textit{mein }‘my’, the difference being that NEG is adjoined to the noun phrase in (30) and that it can be realized as either \textit{k}{}- or \textit{nicht}. The basic rule is provided below:
\end{styleFooter}

\begin{styleFooter}
(34)\ \ NEG \ \ → \ \ \textit{k}{}- \ \ / \_\_ (unstressed) \textit{ein}
\end{styleFooter}

\begin{styleFooter}
\ \ \ \ → \ \ \textit{nicht} \ \ (elsewhere)
\end{styleFooter}

\begin{styleFooter}
In words, NEG is spelled out as \textit{k}{}- in the context of unstressed \textit{ein} but as \textit{nicht} in the remaining instances (but see also Section 4.2.1 for some complications arising from the apparent optionality of \textit{kein} vs. \textit{nicht} in certain instances). As with the bound possessive morphemes, \textit{ein} undergoes Local Dislocation onto \textit{k}{}-. Finally, the singularity numeral \textit{Ø}\textsubscript{[-PL]} is also supported by \textit{ein }instantiated by Local Dislocation (for detailed discussion including a tree structure, see Section 5.1). To be clear, the elements on the right of the arrows in (27a-c) are taken to be the spell-out forms of the combination of the elements on the left of them. With regard to the workings of \textit{ein}, I continue speaking of support; that is, \textit{ein} supports other elements (instantiated by Local Dislocation).
\end{styleFooter}

\begin{styleFooter}
4.1.3. Some Advantages and Initial Consequences
\end{styleFooter}

\begin{styleFooter}
There are two immediate advantages of this proposal. With these composite elements made up of \textit{ein }and another component, it is easy to see how semantically quite diverse elements can share identical morphology. Furthermore, with the article \textit{ein }present, it follows that a second determiner cannot occur, and the complementary distribution with other determiners follows. Interestingly, noun phrases involving indefinite articles may have different interpretations with regard to specificity. As shown below, these interpretations are often claimed to be tied to the indefinite article being in different positions, for instance, D and Card. It could be suggested that vacuous \textit{ein} also supports D\textsubscript{[+}\textsc{\textsubscript{spec}}\textsubscript{]} and Card\textsubscript{[-}\textsc{\textsubscript{spec}}\textsubscript{]}. However, these are structural elements, and it is not likely that they are supported in the sense above. 
\end{styleFooter}

\begin{styleFooter}
\ \ In more detail, it is well known that indefinite articles are weak determiners (Milsark 1974). Furthermore, indefinite noun phrases may have different readings (see, among many others, Fodor \& Sag 1982, Harbert 2007: 140, Heusinger 2011, Zamparelli 2005: 760; also Diesing 1992 and Hallman 2004). For instance, they can have a non-specific (35a) or a presuppositional, specific (35b) interpretation:
\end{styleFooter}

\begin{styleFooter}
(35)\ \ a. \ \ \textit{Ich würde nur (ei)n Auto \ \ \ \ \ \ \ mit \ \ lila \ \ \ \ \ Punkten kaufen}.
\end{styleFooter}

\begin{styleFooter}
I \ \ \ \textsubscript{\ }would only a \ \ \ \ car.\textsc{neut} with purple dots \ \ \ \ \ \ buy
\end{styleFooter}

\begin{styleFooter}
‘I would only buy a car with purple dots.’
\end{styleFooter}

\begin{styleFooter}
\ \ b. \ \ \textit{Ich habe gestern \ \ (ei)n Auto \ \ \ \ \ \ \ mit \ \ lila \ \ \ \ \ Punkten gekauft}.
\end{styleFooter}

\begin{styleFooter}
\ \ \ \ I \ \ \ \textsubscript{\ }have yesterday a \ \ \ \ \textsubscript{\ }car.\textsc{neut}\textsubscript{ }with purple dots \ \ \ \ \ \ \textsubscript{\ }bought
\end{styleFooter}

\begin{styleFooter}
\ \ \ \ ‘I bought a car with purple dots yesterday.’
\end{styleFooter}

\begin{styleFooter}
Without going into much detail here, these readings are licensed by noun phrase-external factors, for instance, the different moods in the hosting clause (subjunctive vs. indicative). I basically follow Jackendoff (1977: 105) and Bowers (1988) in that weak determiners may be in different positions inside the noun phrase (licensed by these noun phrase-external factors). 
\end{styleFooter}

\begin{styleFooter}
To make the discussion concrete, I assume that there are two positions inside the DP relevant for the different readings: D, the head of DP, and Card, the head of CardP, which is located just below the DP (Chapter 1). In particular, on the weak, non-specific reading in (35a), I assume that the indefinite article surfaces in Card, and on the strong, specific reading in (35b), it appears in D (also Alexiadou \textit{et al}. 2007: 225, Borer 2005: 144-45, Chomsky 1995: 342, Zamparelli 2000: 264-65).\textbf{ }As proposed in Chapter 1, \textit{ein} originates in ArtP. With \textit{ein} lacking a feature for definiteness, it does not have to undergo movement to D. Thus, I propose that \textit{ein} moves to Card if the nominal has a weak reading and to D if the nominal has a strong reading. As such, I assume that vacuous \textit{ein} in Card\textsubscript{[-}\textsc{\textsubscript{spec}}\textsubscript{]} or in D\textsubscript{[+}\textsc{\textsubscript{spec}}\textsubscript{]} does not involve composite forms – the feature bundles later to be spelled out as \textit{ein} simply move to these positions. 
\end{styleFooter}

\begin{styleFooter}
Finally, as regards (27d), I claim in Chapter 6 that vacuous \textit{ein} (or a definite determiner) flags the presence of the realization operator REL. Recall that I suggested above that the negative, possessive, and singularity components of the \textit{ein}{}-words also involve operators. Given that, I reiterate the proposal that there are two ways to make operators visible by \textit{ein}: supporting where the operator itself is in some way detectable, as claimed for (27a-c), and flagging where the operator itself remains invisible as in (27d). If this is tenable, then the term indefinite article seems inappropriate, but I continue with the traditional terminology. As to (27e), adjectival \textit{eine} is not a composite form but a separate lexical item (for more details, see Section 5.2).
\end{styleFooter}

\begin{styleStandard}\itshape
4.2.\ \ Some Evidence for the Composite Analyses
\end{styleStandard}

\begin{styleStandard}
In the previous section, I provided the derivations that bring about the three composite forms. What they all have in common is that the two relevant elements are adjacent to one another. I consider this in more detail now providing some evidence for the composite analyses. I start with the most straightforward case.
\end{styleStandard}

\begin{styleFooter}
4.2.1. The Negative Article \textit{kein}
\end{styleFooter}

\begin{styleFooter}
It is a fairly standard proposal that \textit{kein} ‘no’ consists of NEG + \textit{ein }(e.g., Bech 1955/57, Jacobs 1980, Kratzer 1995: 144-47). I provide two morpho-syntactic arguments for the composite analysis. First, considering the data below, it seems clear that the contraction of NEG and \textit{ein} is obligatory in certain cases (36), optional in others (37), but cannot occur in yet other contexts (38):
\end{styleFooter}

\begin{styleFootnote}
(36)\ \ a. \ *\ \ \textit{Ich habe nicht (ei)n \ Buch \ \ \ \ \ \ \ \ \ gekauft}.
\end{styleFootnote}

\begin{styleFootnote}
\ \ \ \ I \ \ \ \ have not \ \ \ \ \textsubscript{\ }a \ \ \ \ \ book.\textsc{neut} bought
\end{styleFootnote}

\begin{styleFootnote}
\ \ \ \ ‘I did not buy a book.’
\end{styleFootnote}

\begin{styleFootnote}
\ \ b.\ \ \textit{Ich habe kein Buch \ \ \ \ \ \ \ \ \ \ gekauft}.
\end{styleFootnote}

\begin{styleFootnote}
\ \ \ \ I \ \ \ \ have no \ \ \ book.\textsc{neut} bought
\end{styleFootnote}

\begin{styleFootnote}
\ \ \ \ ‘I bought no book.’
\end{styleFootnote}

\begin{styleFootnote}
(37)\ \ a.\ \ \textit{Ich habe nicht (ei)n BUCH \ \ \ \ \ \ \ gekauft, sondern (ei)n HEFT}.
\end{styleFootnote}

\begin{styleFootnote}
\ \ \ \ I \ \ \ \ have not \ \ \ \ \textsubscript{\ }a \ \ \ \ book.\textsc{neut} bought \ \textsubscript{\ }but \ \ \ \ \ \ \ \ \ a \ \ \ \ \textsubscript{\ }booklet.\textsc{neut}
\end{styleFootnote}

\begin{styleFootnote}
\ \ \ \ ‘I did not buy a book but a booklet.’
\end{styleFootnote}

\begin{styleFootnote}
\ \ b.\ \ \textit{Ich habe kein BUCH \ \ \ \ \ \ \ gekauft, sondern (ei)n HEFT}.
\end{styleFootnote}

\begin{styleFootnote}
\ \ \ \ I \ \ \ \ have no \ \ \ book.\textsc{neut} bought \ \textsubscript{\ }but \ \ \ \ \ \ \ \ \ a \ \ \ \ \textsubscript{\ }booklet.\textsc{neut}
\end{styleFootnote}

\begin{styleFootnote}
\ \ \ \ ‘I did not buy a book but a booklet.’
\end{styleFootnote}

\begin{styleFootnote}
(38)\ \ a.\ \ \textit{(Ei)n Buch \ \ \ \ \ \ \ \ \ habe ich nicht gekauft}.
\end{styleFootnote}

\begin{styleFootnote}
\ \ \ \  \textsubscript{\ }a \ \ \ \ \textsubscript{\ }book.\textsc{neut} have I \ \ \ \ not \ \ \ bought
\end{styleFootnote}

\begin{styleFootnote}
\ \ \ \ ‘I bought no book/I did not buy a book.’
\end{styleFootnote}

\begin{styleFootnote}
b. \ *\ \ \textit{(Ei)n Buch \ \ \ \ \ \ \ \ \ habe ich kein gekauft}.
\end{styleFootnote}

\begin{styleFootnote}
\ \ \ \  \textsubscript{\ }a \ \ \ \ \textsubscript{\ }book.\textsc{neut} have I \ \ \ \ no \ \ \ bought
\end{styleFootnote}

\begin{styleFootnote}
\ \ \ \ ‘I bought no book/I did not buy a book.’
\end{styleFootnote}

\begin{styleFootnote}
Specifically, with ordinary stress, NEG and \textit{(ei)n} must form the negative article (36). This is different with contrastive stress on the noun where NEG and \textit{ein} can optionally be spelled out separately as \textit{nicht ein} or as the composite form \textit{kein} (37). Finally, when NEG and \textit{(ei)n} are not adjacent (38), both elements are spelled out separately, with NEG being realized as \textit{nicht}.\footnote{\ Note again that \textit{ein} and \textit{kein} can co-occur in split topicalizations:\par 
\setcounter{listWWviiiNumvleveli}{0}
\begin{listWWviiiNumvleveli}
\item 
\begin{styleFootnote}
\textit{(Ei)n Buch \ \ \ \ \ \ \ \ \ habe ich kein-s \ \ \ gekauft}.
\end{styleFootnote}
\end{listWWviiiNumvleveli}
\ \ \ \  \textsubscript{\ }a \ \ \ \ \textsubscript{\ }book.\textsc{neut} have I \ \ \ \ none-\textsc{st} bought\par ‘As for books, I bought none.’\par Considering the strong ending on \textit{kein} in (i), I assume that the negative article is followed by a null noun (see Chapter 4, Section 3.3). In other words, there are two separately base-generated nominals in (i), each with its own \textit{ein}. As such, the grammaticality of this example is not an argument against \textit{kein} requiring adjacency of NEG and \textit{(ei)n}, which is illustrated in (38).} Now, the fact that NEG can be realized in two different ways depending on the stress pattern and the adjacency with \textit{ein}, supports the claim that \textit{kein} is a composite form.
\end{styleFootnote}

\begin{styleFooter}
\ \ Consider a second piece of evidence for this composite analysis. The examples in (39a) and (39b) establish that negation is higher than the type particle \textit{so} ‘such’ and that the latter can intervene between negation and the indefinite article.\footnote{\ Note that \textit{nicht} ‘not’ can form a constituent with the nominal (with stress on \textit{so} ‘such’):\par \ \ (i)\ \ \textit{Nicht so \ \ \ ein Mann kam, sondern ein anderer}.\par \ \ \ \ not \ \ \ such a \ \ \ man \ \ came but \ \ \ \ \ \ \ \ a \ \ different\par \ \ \ \ ‘Not such a man came, but a different one.’} However, with an intervening \textit{so}, negation is not adjacent to \textit{ein}, and consequently, these two elements cannot be spelled out as \textit{kein} by the morphology as evidenced by (39c) and (39d):
\end{styleFooter}

\begin{styleFooter}
(39)\ \ a.\ \ \textit{nicht so \ \ \ (ei)n Idiot}
\end{styleFooter}

\begin{styleFooter}
\ \ \ \ not \ \ \ such an \ \ \ idiot.\textsc{masc}
\end{styleFooter}

\begin{styleStandard}
‘not such an idiot’
\end{styleStandard}

\begin{styleFooter}
\ \ b. *\ \ \textit{so nicht (ei)n Idiot}
\end{styleFooter}

\begin{styleFooter}
\ \ \ \ so not \ \ \ \ an \ \ idiot.\textsc{masc}
\end{styleFooter}

\begin{styleFooter}
\ \ c. ?*\ \ \textit{kein so \ \ \ \ Idiot}
\end{styleFooter}

\begin{styleFooter}
\ \ \ \ no \ \ \ such idiot.\textsc{masc}
\end{styleFooter}

\begin{styleFooter}
\ \ d. * \ \ \textit{so kein Idiot}
\end{styleFooter}

\begin{styleFooter}
\ \ \ \ so no \ \ \ idiot.\textsc{masc}
\end{styleFooter}

\begin{styleFooter}
This provides a second argument that \textit{kein} ‘no’ is a composite form subject to the adjacency of its two components. For other syntactic and semantic arguments for a composite analysis, see Pafel (2005: 186-87), von Fintel \& Iatridou (2007: 467-68), and Zeijlstra (2011) (for English, see Klima 1964: 273-76); for the discussion of \textit{kein so’n Idiot} ‘(NEG+a so an =) no such idiot’, see Chapter 8, Section 2.2.2.\footnote{\ There are different proposals to implement the claim that \textit{kein} ‘no’ is a composite form. For instance, like the current analysis, Zeijlstra (2011: 119, 131) assumes that \textit{kein} is a complex element. Unlike the current analysis, he argues that \textit{kein} is a complex head consisting of a top node that dominates a negation operator and an existential operator. These different implementations have other consequences, something I will not discuss here.} If \textit{kein} involves a composite analysis, then it is not implausible to propose this for the other complex \textit{ein}{}-words as well.
\end{styleFooter}

\begin{styleFooter}
4.2.2. Possessive Articles such as \textit{mein}
\end{styleFooter}

\begin{styleFooter}
There is also evidence that possessive articles are composite forms. While Saxon Genitives can both precede and follow their head nouns (40a-b) (e.g., Duden 2007: 366, Eisenberg \& Smith 2002: 125, Fuß 2011), possessive articles can only precede their head nouns (40c-d):
\end{styleFooter}

\begin{styleFooter}
(40)\ \ a.\ \ \textit{Magdalenas \ Buch \ \ \ \ \ \ \ \ \ ist schön}.
\end{styleFooter}

\begin{styleFooter}
\ \ \ \ Magdalena’s book.\textsc{neut} is \ beautiful
\end{styleFooter}

\begin{styleFooter}
\ \ \ \ ‘Mary’s book is beautiful.’
\end{styleFooter}

\begin{styleFooter}
\ \ \ b. \ \ \textit{Das Buch \ \ \ \ \ \ \ \ \ Magdalenas \ ist schön}.\ \ 
\end{styleFooter}

\begin{styleFooter}
\ \ \ \ the \ book.\textsc{neut} Magdalena’s is \ beautiful
\end{styleFooter}

\begin{styleFooter}
\ \ \ \ ‘The book of Mary’s is beautiful.’
\end{styleFooter}

\begin{styleFooter}
\ \ c.\ \ \textit{Mein Buch \ \ \ \ \ \ \ \ \ ist schön}.
\end{styleFooter}

\begin{styleFooter}
\ \ \ \ my \ \ \ book.\textsc{neut} is \ beautiful
\end{styleFooter}

\begin{styleFooter}
\ \ \ \ ‘My book is beautiful.’
\end{styleFooter}

\begin{styleFooter}
\ \ d. \ *\ \ \textit{Das Buch \ \ \ \ \ \ \ \ \ mein(s) ist schön}.\ \ 
\end{styleFooter}

\begin{styleFooter}
\ \ \ \ the \ book.\textsc{neut} mine \ \ \ \ \ is \ beautiful
\end{styleFooter}

\begin{styleFooter}
\ \ \ \ ‘The book of mine is beautiful.’
\end{styleFooter}

\begin{styleFooter}
These facts follow from the assumption that possessive articles consist of a possessive component and \textit{ein}. Specifically, \textit{ein} can, as an article, only precede its head noun explaining the restriction in (40d). There is other evidence for a composite analysis. 
\end{styleFooter}

\begin{styleFooter}
It is well known that the possessive component of the possessive article agrees in gender with its antecedent but that the remaining part of the possessive article agrees in case, number, and gender with the head noun (41a). Furthermore, besides gender, the possessive component also agrees in person and number with its antecedent, as can easily be seen in (41b). These agreement relations are illustrated with different indices below:\footnote{\ Georgi \& Salzmann (2011: 2077) observe the data points in (i), where a neuter possessor can occur with both a neuter and a feminine possessive article:\par \ \ (i)\ \ a.\ \ \textit{Das Mädchen hat seine / ihre Schuhe verloren}.\par \ \ \ \ \ \ the \ girl.\textsc{neut} has its \ \ \ \ / her \ \ shoes \ \ lost\par \ \ \ \ \ \ ‘The girl has lost her shoes.’\par b.\ \ \textit{dem \ \ \ \ \ \ Mädchen seine / ihre Schuhe}\par \ \ \ \ \ \ the.\textsc{dat} girl.\textsc{neut} its \ \ \ \ / her \ shoes\par \ \ \ \ \ \ ‘the girl’s shoes’\par They argue that the agreement relation between the possessor and the possessive article involves an anaphoric binding relation.}
\end{styleFooter}

\begin{styleFooter}
(41)\ \ a.\ \ \textit{Peter}\textsubscript{i}\textit{ hat s}\textsubscript{i}\textit{{}-einer}\textsubscript{k}\textit{ Freundin}\textsubscript{k}\textit{ \ \ \ \ \ \ geholfen}.
\end{styleFooter}

\begin{styleFooter}
\ \ \ \ Peter \ has \textsc{poss}{}-a \ \ girlfriend.\textsc{fem} helped
\end{styleFooter}

\begin{styleFooter}
\ \ \ \ ‘Peter helped his girlfriend.’
\end{styleFooter}

\begin{styleFooter}
\ \ b.\ \ \textit{Ich}\textsubscript{i}\textit{ habe m}\textsubscript{i}\textit{{}-einen}\textsubscript{k}\textit{ Freunden}\textsubscript{k}\textit{ geholfen}.
\end{styleFooter}

\begin{styleFooter}
\ \ \ \ I \ \ \ \ \textsubscript{\ }have \textsc{poss}{}-a \ \ \ \ friends \ \ \ \ \ \textsubscript{\ }helped
\end{styleFooter}

\begin{styleFooter}
\ \ \ \ ‘I helped my friends.’
\end{styleFooter}

\begin{styleFooter}
These different agreement relations accessing the same word are straightforwardly explained by the composite analysis.\footnote{\ Olsen (1989b: 139, 1991b: 53) also assumes decomposition of possessive articles, but it is different from the current analysis. Exemplifying with \textit{meinen} ‘my’, \textit{mein} is in Spec,DP, and -\textit{en} is in D. There are three issues here. First, Olsen also locates the possessive marker -\textit{s} in D. Note that D has the same features when the inflection -\textit{en} or the marker -\textit{s} is located in D (Olsen 1991b: 48, 53). This raises the question of what rules out the possessive marker surfacing on the possessive pronoun (cf. *\textit{meins neues Auto} ‘my-\textsc{poss} new-\textsc{st} car’ vs. \textit{Peters neues Auto} ‘Peter’s new-\textsc{st} car’). Second and more importantly, the lack of decomposition of the stem into \textit{m}{}- and \textit{ein} loses the generalization that all \textit{ein}{}-words behave the same morpho-syntactically. Third, we pointed out above that \textit{ein} and adjectival inflections have a status different from the possessive component: the first two elements are semantically vacuous, and they make linguistic items visible: \textit{ein} supports or flags operators and adjectival inflections provide overt exponents for abstract features. As such, \textit{ein} should be grouped with the inflection in D as in the current account (and not with the possessive component in Spec,DP as in Olsen’s work). Recall that here \textit{ein} involves a complex head consisting of [+D] and a CNG feature bundle.} 
\end{styleFooter}

\begin{styleFooter}
4.2.3. The Singularity Numeral \textit{EIN} 
\end{styleFooter}

\begin{styleFooter}
I discuss the advantages of a composite analysis of \textit{EIN} ‘one’ in detail in Section 5. Note already here that \textit{Ø}\textsubscript{[-PL]} is located in Spec,CardP in the weak reading of this indefinite element and that the supporting element \textit{ein} is in Card (recall that \textit{ein} lacks a definiteness feature and as such, it does not have to move to the DP-level). Something similar holds for the strong reading where \textit{Ø}\textsubscript{[-PL]} has moved to Spec,DP, and \textit{ein} is in D. Crucially, these are all Spec-head constellations, where adjacency holds inherently. Consequently, \textit{ein} can support \textit{Ø}\textsubscript{[-PL]}, and both elements are spelled out by the morphology as the numeral (note also that \textit{ein} supporting \textit{Ø}\textsubscript{[-PL]} licenses this null element at the same time)\textit{. }When \textit{Ø}\textsubscript{[-PL]} is\textit{ }not present, the vacuous article is spelled out as \textit{ein }under Card or D as discussed in Section 4.1.3. \ 
\end{styleFooter}

\begin{styleFooter}
\ \ To sum up, I take two points as established: first, the negative article \textit{kein }‘no’, possessive articles such as \textit{mein} ‘my’, and the singularity numeral \textit{EIN} ‘one’ are all composites; second, these elements involve post-syntactic spell-out forms. As we have seen and will see throughout this chapter, this analysis has a number of advantages allowing for a fairly simple account of the different kinds of \textit{ein}. Before I turn to some syntactic differences, I consider some of the semantic and morphological distinctions of these elements in more detail.
\end{styleFooter}

\begin{styleFooter}\itshape
4.3.\ \ Feature Specifications 
\end{styleFooter}

\begin{styleFooter}
First, I document that the presupposition induced by adjectival \textit{eine} can be cancelled under certain conditions. This lays the foundation for the second subsection, where I provide the feature specifications of the different types of \textit{ein}.
\end{styleFooter}

\begin{styleFooter}
4.3.1. Cancelling the Presupposition of Adjectival \textit{eine}
\end{styleFooter}

\begin{styleFooter}
Recall that adjectival \textit{eine} has to occur in definite contexts and that it involves a duality presupposition. This presupposition seems to be strongest with the definite article (42a), it is possible with the possessive article (42b), but it is absent with the demonstrative \textit{diese} ‘this’ (42c):
\end{styleFooter}

\begin{styleFooter}
(42)\ \ a.\ \ \textit{die \ ein-e \ \ \ Freundin}
\end{styleFooter}

\begin{styleFooter}
\ \ \ \ the one-\textsc{wk} girlfriend.\textsc{fem}
\end{styleFooter}

\begin{styleFooter}
\ \ \ \ ‘one of the two girlfriends’ 
\end{styleFooter}

\begin{styleFooter}
b.\ \ \textit{meine ein-e \ \ \ \ Freundin}
\end{styleFooter}

\begin{styleFooter}
\ \ \ \ my \ \ \ \ one-\textsc{wk} girlfriend.\textsc{fem}
\end{styleFooter}

\begin{styleFooter}
‘one of the two of my girlfriends’
\end{styleFooter}

\begin{styleFooter}
\ \ \ \ ‘my one girlfriend’
\end{styleFooter}

\begin{styleFooter}
\ \ c.\ \ \textit{diese ein-e \ \ \ \ Freundin}
\end{styleFooter}

\begin{styleFooter}
\ \ \ \ this \ \ one-\textsc{wk} girlfriend.\textsc{fem}
\end{styleFooter}

\begin{styleFooter}
\ \ \ \ ‘this one girlfriend’
\end{styleFooter}

\begin{styleFooter}
Recall also that the duality presupposition cannot stem from the singular definite article. On the one hand, this determiner carries a uniqueness presupposition; on the other hand, other definite elements like possessive articles can also license adjectival \textit{eine}. This means that the duality cannot come from the definite article itself. Rather, I pointed out above that in conjunction with a definite element, the presence of \textit{eine} brings about partitive semantics in that it presupposes the existence of two sets of elements where the sets may involve one or several members. 
\end{styleFooter}

\begin{styleFooter}
\ \ Besides the definite and the possessive articles, there are other definite elements that occur with adjectival \textit{eine}. Specifically, the periphrastic possessive in (43a) and the Saxon Genitive in (43b) can each involve a duality presupposition (for (43b), see Gunkel \textit{et al}. 2017: 1300):
\end{styleFooter}

\begin{styleFooter}
(43)\ \ a.\ \ \textit{Peter sein ein-er \ Sohn}
\end{styleFooter}

\begin{styleFooter}
\ \ \ \ Peter his \ \ one-\textsc{st} son.\textsc{masc}
\end{styleFooter}

\begin{styleFooter}
‘one of the two of Peter’s sons’
\end{styleFooter}

\begin{styleFooter}
\ \ \ \ ‘Peter’s one son’
\end{styleFooter}

\begin{styleFooter}
\ \ b.\ \ \textit{Peters \ ein-er \ Sohn}
\end{styleFooter}

\begin{styleFooter}
\ \ \ \ Peter’s one-\textsc{st} son.\textsc{masc}
\end{styleFooter}

\begin{styleFooter}
‘one of the two of Peter’s sons’
\end{styleFooter}

\begin{styleFooter}
\ \ \ \ ‘Peter’s one son’
\end{styleFooter}

\begin{styleFooter}
Before moving on, note already here that the weak inflection on \textit{eine} in (42) and the strong inflection on \textit{eine} in (43) indicate that \textit{eine} is indeed adjectival (see Section 5.2 for detailed morpho-syntactic discussion). Returning to the semantics, the question arises why the duality presupposition is absent with demonstratives.
\end{styleFooter}

\begin{styleFooter}
\ \ That the duality presupposition is missing with demonstratives is quite clear in (44a). In this case, there is no presupposition that there is a second boy with the relevant property. As might be expected, if the definite article is stressed and functions as the distal demonstrative, there is no presupposition either (44b).\footnote{\ Note that \textit{eine} in (44b) has fairly strong stress – \textit{eine} is inherently stressed and, additionally, it appears in phrase-final position. As a possible consequence of that, adjacent \textit{die} is not (strongly) stressed. } In a similar vein, (44c) consists of a singular subject and a predicate nominal modified by a relative clause. Indeed, there is no presupposition that there is a second man with the property denoted by the relative clause. Furthermore, Orrin Robinson (p.c.) raises the question as to whether the duality presupposition can disappear when one nominal containing adjectival \textit{eine} is coordinated with another involving a numeral (44d). This is indeed the case ((44b) is taken from M. Müller 1986: 45):\footnote{\ In Footnote Error: Reference source not found, we saw that \textit{ein} with duality presupposition could also follow a definite determiner in OHG. Apparently, this presupposition could also be cancelled with demonstratives (ia) and in contexts involving a unique entity (ib) in this older variety of German:\par \ \ (i)\ \ a.\ \ \textit{thiz eina j\=ar}\ \ \ \ (OHG)\par \ \ \ \ \ \ this one \ year.\textsc{neut}\par \ \ \ \ \ \ ‘this year’\par \ \ \ \ \ \ \ \ (Tatian, Braune \& Reiffenstein 2004: 234)\par \ \ \ \ b.\ \ \textit{der eino almahtico cot}\par \ \ \ \ \ \ the one \ almighty \ \ god.\textsc{masc}\par \ \ \ \ \ \ ‘the almighty God’\par \ \ \ \ \ \ \ \ (Wessobrunner Gebet 7, Schrodt 2004: 21)}
\end{styleFooter}

\begin{styleFooter}
(44)\ \ a.\ \ \textit{Dieser eine Junge \ \ \ \ \ \ \ hat viele \ Wunder \ vollbracht}.
\end{styleFooter}

\begin{styleFooter}
\ \ \ \ this \ \ \ \ \ one \ boy.\textsc{masc} has many miracles accomplished
\end{styleFooter}

\begin{styleStandard}
‘This one boy has performed many miracles.’
\end{styleStandard}

\begin{styleFooter}
\ \ b.\ \ \textit{alle Frauen wollen plötzlich \ \ nur \ \ das (/dies(es)) eine: Zärtlichkeit}
\end{styleFooter}

\begin{styleFooter}
\ \ \ \ all \ \ women want \ \ \ suddently only that / this \ \ \ \ \ \ \ \ one \ \ \ tenderness.\textsc{neut}
\end{styleFooter}

\begin{styleFooter}
\ \ \ \ ‘Suddenly, all women only want one thing: tenderness.’
\end{styleFooter}

\begin{styleFootnote}
\ \ c.\ \ \textit{Du \ bist der eine Mann, \ \ \ \ \ \ \ der mich geliebt hat}.
\end{styleFootnote}

\begin{styleFootnote}
\ \ \ \ you are the \ one \ man.\textsc{masc} that me \ \ \ loved \ \ has
\end{styleFootnote}

\begin{styleFooter}
\ \ \ \ ‘You are the one man that has loved me.’
\end{styleFooter}

\begin{styleFooter}
\ \ d.\ \ \textit{Meine eine Freundin \ \ \ \ \ \ \ \ \ und seine zwei sind ausgegangen.}
\end{styleFooter}

\begin{styleFooter}
\ \ \ \ my \ \ \ \ \ one girl-friend.\textsc{fem} and his \ \ \ \ two \ are \ \ gone out
\end{styleFooter}

\begin{styleFooter}
\ \ \ \ ‘My one girl-friend and his two went out.’
\end{styleFooter}

\begin{styleFooter}
To be clear, then, although \textit{eine} has adjectival morphology in each case and is presumably the same element, it may lack the presupposition property under certain conditions (also M. Müller 1986: 45). 
\end{styleFooter}

\begin{styleFooter}
I tentatively suggest for (44a-b) that demonstrative determiners, with their strong deictic force, cancel the presupposition of adjectival \textit{eine}. Furthermore, note that \textit{eine} does not seem to be (strongly) stressed in (44a), where in contrast to (44b), the head noun is present. This presumably is also a reflex of the presence of word stress on the preceding demonstrative. Possessives as in (42b) and (43) have been related to demonstratives in that possessives are also deictic elements. For instance, \textit{meine} ‘my’ can be compared to \textit{diese} ‘this’ such that both are associated with the speaker (see Lyons 1999: 18-19, Roehrs 2019: 378). This presumably explains why the duality presupposition can be absent with possessives (it is less clear to me though why this presupposition can sometimes be present with possessives).\footnote{\ The possible presence of a duality presupposition with possessives might be related to the fact that possessives do not presuppose singleton sets (Coppock \& Beaver 2015). } As to (44c), I assume that the duality presupposition of adjectival \textit{eine} is also cancelled when the nominal containing \textit{eine} forms the predicate of a singular subject. Finally turning to (44d), I assume that the coordination of the two nominals leads to a list-type effect where the different numbers of the people involved are contrasted. This resultant contrast also allows a singularity reading of adjective \textit{eine}.
\end{styleFooter}

\begin{styleFooter}
To sum up, I proposed that \textit{eine} is an adjective that can only occur in definite contexts. I assume that this distributional restriction has to do with \textit{eine}’s own (contribution to the) duality presupposition such that presuppositions in general often arise only in definite contexts. With these remarks in mind, I turn to the feature specifications.
\end{styleFooter}

\begin{styleFooter}
4.3.2. Feature Specifications of the Different Types of \textit{ein}
\end{styleFooter}

\begin{styleFooter}
As documented in Chapter 1, Section 2.2, \textit{ein} as an article can occur not only in singular but also in plural contexts (45). Note that in each of the two cases below, there is no duality presupposition. This is expected given that these are indefinite contexts:
\end{styleFooter}

\begin{styleCommentText}
(45)\ \ a. \ \%\ \ {\textquotedbl}\emph{Was \ \ für eine Idioten}{\textquotedbl}\textit{ sagte Liam}
\end{styleCommentText}

\begin{styleCommentText}
\ \ \ \  \ what for \ a \ \ \ \ idiots \ \ \ \ \ said \ Liam
\end{styleCommentText}

\begin{styleCommentText}
\ \ \ \ ‘“Such idiots!” said Liam.’
\end{styleCommentText}

\begin{styleStandard}
\ \ b. \ \%\ \ \emph{Das Personal war sehr unfreundlich. Ich hatte noch nie \ \ \ \ so eine Ferien\textup{.}}~
\end{styleStandard}

\begin{styleStandard}
\ \ \ \ the \ staff \ \ \ \ \ \ \ \ was very unfriendly \ \ \ \ I \ \ \ \ had \ \ \ still \ never so a \ \ \ \ \ holidays\ \ 
\end{styleStandard}

\begin{styleStandard}
\ \ \ \ ‘The staff was very unfriendly. I have never had such a (bad) vacation.’
\end{styleStandard}

\begin{styleFooter}
Again, it is clear that \textit{ein} can appear in plural contexts. Given the proposal that this element is a semantically vacuous element, I assume that \textit{ein} can be plural as regards its morphology.
\end{styleFooter}

\begin{styleFooter}
In contrast, adjectival \textit{eine} has semantics regarding number as just discussed. Thus, I make a distinction between morphological and semantic number (also Chapter 7), morph vs. sem for short. Consequently, I assume two types of plural features [PL] in the feature makeup where \textit{$\alpha $} may range over a negative or positive value:\footnote{\ With Chapter 2 in mind, plural actually consists of the features [+F, +N]. The difference between plural and singular is that in the plural, both $\alpha $ and $\beta $ in [$\alpha $F, $\beta $N] are valued positively but in the singular, $\alpha $ and/or $\beta $ is valued negatively (the latter was captured by a category variable: [-$\gamma $]). For simplicity, I use the feature [±PL] moving forward. } 
\end{styleFooter}

\begin{styleFooter}
(46)\ \ Number:\ \ [$\alpha $PL morph; $\alpha $PL sem]
\end{styleFooter}

\begin{styleFooter}
\ \ Starting with the feature specification of the indefinite article, I proposed that \textit{ein} is a semantically vacuous element. To review briefly, this element can occur in singular and plural contexts as well as in indefinite and definite environments. As such, this element cannot have a specification for semantic number or definiteness.\footnote{\ A reviewer points out that \textit{ein} occurring in plural contexts is not a strong argument that \textit{ein} is not associated with singularity, given that in some languages the plural seems to operate on the singular. As discussed in S. Grimm (2012: chapter 2), there are two cases as regards nouns: count nouns can be pluralized as in English, and collective nouns can be singularized and then pluralized as in Welsh (note also that plural can be subtractive in Murle, see Haspelmath \& Sims 2010: 37, or in Hessian German). Schematically, these two cases could be represented as follows: Stem+PL and Stem+SGL+PL. It could be claimed then that -\textit{e} in plural \textit{ein} spells out the plural part and that \textit{ein} realizes the Stem (and singular).\par \ \ There are certain points that seem to militate against this. First, the observation above was made on the basis of nouns. While singular \textit{ein} has traditionally been related to the feature [COUNT], it has not been associated with the feature [COLLECTIVE] (definite articles in German seem to be neutral in that they can occur in both of these singular contexts). Second, note that the plural form of the indefinite article is not morphologically more complex than its singular counterpart. In fact, both exhibit the same complexity and can even be identical (e.g., in the nominative feminine and plural: \textit{eine}). Third, if one or both of the two schematic structures above, Stem+PL and Stem+SGL+PL, is or are generally available, then we need to find a way to rule out spell-out forms such as *\textit{de-e} or *\textit{de-r-e} in the plural (nominative masculine) and *\textit{da-e} or *\textit{da-s-e} in the plural (nominative neuter). As far as I can see, these schematic structures do not capture the fact that in German, genders in the singular are not relevant in the plural. Finally, it remains unclear why plural \textit{ein} cannot occur in all plural contexts in Standard German.} Furthermore, given the composite analysis of the possessive and negative articles, which can each occur with a mass noun (e.g., \textit{meine / keine Milch} ‘my / no milk), \textit{ein} does not bring about countability either.\footnote{\ In Chapter 7, I discuss in more detail that countability in the current analysis is due to a side-effect, which is related to number agreement mitigated by the head of NumP.} Thus, this element has no feature for countability. However, we have seen that \textit{ein} morphologically agrees with its singular or plural head noun. I conclude again that this element must have morpho-syntactic features. Above, I proposed that \textit{ein} involves the feature [+D] and a feature bundle for case, number, and gender: 
\end{styleFooter}

\begin{styleStandard}
(47)\ \ \textit{Indefinite Overt Article}\ \ 
\end{styleStandard}

\begin{styleStandard}
\ \ \ \ Art\textsubscript{[+D][F, N, O, S]}
\end{styleStandard}

\begin{styleStandard}
[Warning: Draw object ignored][Warning: Draw object ignored]
\end{styleStandard}

\begin{styleStandard}
\ \ [+D]\ \ \ \ [F, N, O, S]\ \ →\ \ \textit{ein-}
\end{styleStandard}

\begin{styleFooter}
The CNG features in (47) are valued/checked during the derivation. Abstracting away from case and gender, morphological number has to do with [F, N], here more simply restated as [$\alpha $PL morph]. Thus, besides [+D], [$\alpha $PL morph] is part of the feature specification of \textit{ein}, specifically, the feature bundle CNG.\footnote{\ Note that the usual restriction of \textit{ein} to singular count contexts follows from a – what I call – restriction feature on the categorial feature yielding [+D: -PL] (cf. also uninflected \textit{dies} ‘this’ in Chapter 4, Section 5). This morpho-syntactic restriction feature can be deleted by the operators POSS, NEG, and certain others that are licensed in emotive/affective contexts (cf. the exclamative operators [+EXCL] and !!! dicussed in Chapter 4, Section 2). This deletion will widen the distribution of \textit{ein} allowing it to appear in certain non-singular contexts (for detailed discussion, see Chapter 8, Sections 2.2.5 and 2.2.6).}
\end{styleFooter}

\begin{styleFooter}
Turning to the singularity numeral \textit{EIN}, I proposed above that this element is a composite form consisting of vacuous \textit{ein} and the numeral component \textit{Ø}\textsubscript{[-PL]}. I suggested that \textit{Ø}\textsubscript{[-PL]} is the contentful part, a type of operator, and I assume that it is this component that is responsible for singularity and stress (cf. Barbiers 2005: 168). I propose that \textit{Ø}\textsubscript{[-PL]} is inherently specified for [-PL] semantically. Given that \textit{Ø}\textsubscript{[-PL]} and vacuous \textit{ein} form a composite, I take it that morphological number comes from \textit{ein} as specified above. Now, since the semantic number and the morphological number of \textit{EIN} have to match, I assume that there is a condition such that semantically singular \textit{Ø}\textsubscript{[-PL]} is only compatible with morphological singular contexts. The number specification for \textit{Ø}\textsubscript{[-PL]} can be stated as [-PL sem, where $\alpha $PL morph = $\alpha $PL sem]. I discuss the syntactic structure of \textit{EIN} in Section 5.1.2.
\end{styleFooter}

\begin{styleFooter}
\ \ Adjectival \textit{eine} is different. Like \textit{ein}, I suggest that \textit{eine}\textsubscript{ADJ} has an unvalued/unchecked feature for morphological number. Unlike \textit{ein}, it is an adjective, and it has a specification for semantic number as regards the cardinality of sets (n) presupposed. This can be stated as follows: [$\alpha $PL morph; n = 2 sem, where n is the number of presupposed sets of entities]. Morphological number is encoded in the CNG feature bundle, and the duality presupposition is located with the adjectival stem \textit{ein}:
\end{styleFooter}

\begin{styleStandard}
(48)\ \ \textit{Adjectival }eine
\end{styleStandard}

\begin{styleJBExample}
\ \ \ \ \ \ \ \ \ \ \ \  \ \ \ \ InflP
\end{styleJBExample}

\begin{styleStandard}\bfseries
[Warning: Draw object ignored][Warning: Draw object ignored]
\end{styleStandard}

\begin{styleStandard}
\ \ \ \ \ \ \ \ \ \ [F, N, O, S]\ \ \ \  \textsubscript{\ }AP
\end{styleStandard}

\begin{styleStandard}
\ \ \ \ \ \ \ \  \ \ \ {\textbar} \ \ \ \ \ \ 
\end{styleStandard}

\begin{styleStandard}
\ \ \ \ \ \ \ \  \ \textsubscript{\ }A
\end{styleStandard}

\begin{styleStandard}
\ \ \ \ \ \  \ \ \ \ \ \ \ \ \ \ \ \ \textit{ein}
\end{styleStandard}

\begin{styleFooter}
Abstracting away from the categorial features (and case and gender), I summarize the feature specifications of the three types of \textit{ein}:
\end{styleFooter}

\begin{styleFooter}
(49)\ \ a.\ \ \textit{ein}: \ \ [$\alpha $PL morph]
\end{styleFooter}

\begin{styleFooter}
\ \ b.\ \ \textit{Ø}\textsubscript{[-PL]}:\ \ [-PL sem, where $\alpha $PL morph = $\alpha $PL sem]
\end{styleFooter}

\begin{styleFooter}
\ \ c.\ \ \textit{eine}\textsubscript{ADJ}: [$\alpha $PL morph; n = 2 sem unless in the context of a demonstrative, etc.]
\end{styleFooter}

\begin{styleFooter}
To be clear, given the complex internal structure of determiners and adjectives (i.e., stem + inflection), the semantic specifications in (49) are assumed to be on the stems of the relevant elements, but the morphological ones are encoded in the inflections as unvalued/unchecked feature bundles. With this in mind, I briefly address the statement in (49b) again.
\end{styleFooter}

\begin{styleFooter}
There is an interesting interaction between (49a) and (49b). As proposed in more detail below, the numeral \textit{EIN} derives from the combination of (49a) and (49b), where the stem of \textit{ein} adds the categorial feature [+D], the inflection on \textit{ein} supplies unvalued/unchecked morphological features for case, number, and gender, and \textit{Ø}\textsubscript{[-PL]} imposes the restriction that only singular number will be compatible with other elements in the larger noun phrase. This means that in order to bring about a good derivation, the number features of all those elements must be valued/checked for singular by the more general operation that brings about concord in agreement features. If those number features are valued/checked for plural, incompability with the number specification of \textit{Ø}\textsubscript{[-PL]} arises.
\end{styleFooter}

\begin{styleFooter}
\ \ Finally, I suggested above that the wider linguistic context (e.g., different moods in the hosting clause) licenses the different readings of indefinites with regard to specificity. I proposed that this involves different positions of \textit{ein }inside the DP. If these claims are on the right track, then we can maintain the claim that \textit{ein} is semantically vacuous (Hypothesis 1a).
\end{styleFooter}

\begin{styleFooter}\bfseries
5. \ \ Step 2 of the Proposal: Syntax
\end{styleFooter}

\begin{styleFooter}
In the first subsection, I propose in more syntactic detail that synchronically, the numeral \textit{EIN} is, in certain respects, based on the indefinite article even though diachronically, the article derived from the numeral. With the different specifications from above in mind, the remaining differences between \textit{EIN} and \textit{ein} are argued to follow from the two different positions that \textit{Ø}\textsubscript{[-PL]} and \textit{ein} occupy in the syntactic tree. In the second subsection, adjectival \textit{eine} is shown to be independent of the indefinite article and numeral, and occupies a third position.
\end{styleFooter}

\begin{styleFooter}\itshape
5.1. \ \ Article vs. Numeral
\end{styleFooter}

\begin{styleFooter}
In this subsection, I first provide two pieces of evidence that the numeral \textit{EIN}, specifically the contentful component of it, is in a different position than the indefinite article. Then I proceed to derive certain aspects of the numeral from the indefinite article.
\end{styleFooter}

\begin{styleFooter}
5.1.1. Uniform Position(s) of all Numerals
\end{styleFooter}

\begin{styleFooter}
Recall from Chapter 1 that numerals are in a position different from determiners or determiner-like elements. Straightforward evidence for this comes from the following cases: 
\end{styleFooter}

\begin{styleFooter}
(50)\ \ a.\ \ \textit{die \ zwei Freunde}
\end{styleFooter}

\begin{styleFooter}
\ \ \ \ \textsc{det} two \ friends 
\end{styleFooter}

\begin{styleStandard}
‘the / those two friends’
\end{styleStandard}

\begin{styleFooter}
\ \ b.\ \ \textit{Peters \ zwei Freunde}
\end{styleFooter}

\begin{styleFooter}
\ \ \ \ Peter’s two \ friends 
\end{styleFooter}

\begin{styleStandard}
‘Peter’s two friends’
\end{styleStandard}

\begin{styleFooter}
I assumed that numerals are in Spec,CardP. Following previous work though, I also suggested that if there is no determiner, then numerals may be in Spec,DP yielding a specific interpretation. The picture is slightly more complicated with \textit{ein}{}-words, specifically the numeral \textit{EIN} and the article \textit{ein}. 
\end{styleFooter}

\begin{styleFooter}
In the last section, I discussed the different interpretations of nominals involving \textit{ein} with regard to specificity. In particular, I proposed that when the interpretation is non-specific, \textit{ein} is in Card, but when it is specific, \textit{ein} is in D. In other words, \textit{ein} in (51a) may be in two different positions. Furthermore, I proposed that \textit{EIN} consists of the singularity part \textit{Ø}\textsubscript{[-PL]} and vacuous \textit{ein}. As regards the different specificity interpretations, I pointed out that when the interpretation is non-specific, \textit{Ø}\textsubscript{[-PL]} is in Spec,CardP and supporting \textit{ein} is in Card, but when it is specific, \textit{Ø}\textsubscript{[-PL]} is in Spec,DP and \textit{ein} is in D. In other words, just like other numerals, \textit{Ø}\textsubscript{[-PL]} may be located in two phrasal positions in (51b):\footnote{\ Despite certain differences between the singularity numeral and the other numerals, Barbiers (2005: 171) also assumes the same position for all numerals (see also his footnote Error: Reference source not found). For typological discussion of numerals, see Corbett (2000) and Hurford (2003).}
\end{styleFooter}

\begin{styleFooter}
(51)\ \ a.\ \ \textit{(ei)n Freund}
\end{styleFooter}

\begin{styleFooter}
\ \ \ \  \textsubscript{\ }a \ \ \ \ friend.\textsc{masc}
\end{styleFooter}

\begin{styleStandard}
‘a friend’ 
\end{styleStandard}

\begin{styleFooter}
\ \ b.\ \ \textit{EIN Freund}
\end{styleFooter}

\begin{styleFooter}
\ \ \ \ one \ friend.\textsc{masc}
\end{styleFooter}

\begin{styleStandard}
‘one friend’
\end{styleStandard}

\begin{styleFooter}
Regarding \textit{ein}{}-words then, the relevant elements are in two positions involving phrases (Spec,CardP and Spec,DP for \textit{Ø}\textsubscript{[-PL]}) and in two positions involving heads (Card and D for \textit{ein}). There are two issues that need to be addressed now. 
\end{styleFooter}

\begin{styleFooter}
First, if the article and the numeral (or a part of it) are in different positions, then we need to explain why the two cannot co-occur (52a-b). This is presumably not due to semantic reasons as other (adjectival) elements emphasizing singularity are possible (52c-d). In fact, under the right (emotive) conditions, \textit{ein} can occur with a non-singularity numeral (52e), repeating an example from Chapter 1, Section 2.2 here (for the latter distribution in Dutch, see Bennis \textit{et al.} 1998: 112):\footnote{\ As is expected from the above discussion, adjectival \textit{eine} is also ungrammatical following \textit{ein} (ia) as the former is only licensed in a definite context (ib):\par \ \ (i) \ \ \ a. \ *\ \ \textit{ein einer Bruder}\par \ \ \ \ \ \ an \ one \ \ \textsubscript{\ }brother.\textsc{masc}\par \ \ \ \ b.\ \ \textit{Peters \ einer Bruder}\par \ \ \ \ \ \ Peter’s one \ \ \textsubscript{\ }brother.\textsc{masc}\par \ \ \ \ \ \ ‘Peter’s one brother’\par \ \ \ \ \ \ ‘one of the two of Peter’s brothers’\par Furthermore, an indication that adjectival \textit{eine} is different from \textit{einziger} ‘only’ can be seen in Negative Polarity contexts. Note first that neither a definite nor an indefinite article can license \textit{je} ‘ever’ (iia). However, adding \textit{einziger}, but not \textit{eine}, after the definite article results in a completely fine example (iib): \par \begin{listWWviiiNumvleveli}
\item 
\begin{styleFootnote}
a. \ *\ \ \textit{Du \ bist \{der / ein\} Mann, \ \ \ \ \ \ der mich je \ \ \ \ geliebt hat}.
\end{styleFootnote}
\end{listWWviiiNumvleveli}
\ \ \ \ \ \ you are \ \ the \ / a \ \ \ \ man.\textsc{masc} that me \ \ ever loved \ \ has\par \ \ \ \ b. \ \ \ \textit{Du \ bist der \{einzige / ??eine\} Mann, \ \ \ \ \ \ \ \ der mich je \ \ \ \ geliebt hat}.\par \ \ \ \ \ \ you are \ the \ \ only \ \ \ \ / \ \ \ \ one \ \ \ man.\textsc{masc} that me \ \ ever loved \ \ has\par \ \ \ \ \ \ ‘You are the only man that has ever loved me.’\par Finally, another indication that \textit{eine} is different from \textit{einziger} ‘only’ is that both elements can cooccur (iic) ((iic) is from Zifonun \textit{et al.} 1997: 1931; to my ears, this example sounds slightly marked without a pause between \textit{eine} and \textit{einzige}):\par \ \ (iii)\ \ \textit{der eine einzige Gegensatz}\par \ \ \ \ the one \ unique \ contrast.\textsc{masc}\par \ \ \ \ ‘the only contrast’}
\end{styleFooter}

\begin{styleFooter}
(52)\ \ a. \ *\ \ \textit{(ei)n EIN Freund}
\end{styleFooter}

\begin{styleFooter}
\ \ \ \  \textsubscript{\ }a \ \ \ \ one \ friend.\textsc{masc}
\end{styleFooter}

\begin{styleFooter}
\ \ b. \ *\ \ \textit{EIN (ei)n Freund}
\end{styleFooter}

\begin{styleFooter}
\ \ \ \ one \ \ a \ \ \ \ friend.\textsc{masc}
\end{styleFooter}

\begin{styleFooter}
\ \ c. \ \ \textit{(ei)n einziger Freund}
\end{styleFooter}

\begin{styleFooter}
\ \ \ \  \textsubscript{\ }a \ \ \ \ sole \ \ \ \ \ \ \ friend.\textsc{masc}
\end{styleFooter}

\begin{styleStandard}
‘a sole friend’
\end{styleStandard}

\begin{styleFooter}
\ \ d. \ \ \textit{(ei)n einzelner \ Mann}
\end{styleFooter}

\begin{styleFooter}
\ \ \ \  \textsubscript{\ }an \ \ individual man.\textsc{masc}
\end{styleFooter}

\begin{styleStandard}
‘a individual man’
\end{styleStandard}

\begin{styleStandard}
\ \ e. \ \%\ \ \textit{ach man wir sind schon so \ \ \ \ ne zwei wärmflaschen}
\end{styleStandard}

\begin{styleStandard}
\ \ \ \ oh \ boy \ \ we are \ \ \textsc{prt} \ \ \ \ such a \ \ two \ warm.bottles
\end{styleStandard}

\begin{styleStandard}
\ \ \ \ ‘Oh boy, we are such two hot-water bottles.’
\end{styleStandard}

\begin{styleFooter}
The ungrammaticality of (52a-b) follows from the assumptions above. Recall that \textit{ein} originates in Art and undergoes movement to Card or D. Proposing that \textit{Ø}\textsubscript{[-PL]} and \textit{ein} are spelled out together as \textit{EIN} under adjacency explains why an indefinite article cannot co-occur with the singularity numeral but only with other (non-composite) numerals (52e). I return to the discussion of (52e) at the end of this section. Adjectives like \textit{einziger} ‘sole’ (52c) and \textit{einzelner} ‘individual’ (52d) can occur with \textit{ein} as they are elements morpho-syntactically independent of \textit{ein}.
\end{styleFooter}

\begin{styleFooter}
\ \ Turning to the second issue, I documented above that the indefinite article and the singularity numeral have the same morphology. What is interesting to note is that under certain conditions, the German numerals for ‘two’ and ‘three’ can take endings in the genitive. In this instance, these two numerals and the following adjective have identical inflections (53a). This is in stark contrast to \textit{EIN}, which does not have the same ending as the following adjective (53b): 
\end{styleFooter}

\begin{styleFooter}
(53)\ \ a.\ \ \textit{das Auto zwei-er \ \ \ \ \ \ nett-er \ Freunde}
\end{styleFooter}

\begin{styleFooter}
\ \ \ \ the car \ \ \ two-\textsc{st.gen} nice-\textsc{st} friends 
\end{styleFooter}

\begin{styleStandard}
‘the car of two nice friends’
\end{styleStandard}

\begin{styleFooter}
\ \ b.\ \ \textit{das Auto EIN-ES \ \ \ \ \ \ nett-en \ \ Freundes}
\end{styleFooter}

\begin{styleFooter}
\ \ \ \ the car \ \ \ one-\textsc{st.gen} nice-\textsc{wk} friend.\textsc{neut}
\end{styleFooter}

\begin{styleFooter}
\ \ \ \ ‘the car of one nice friend’
\end{styleFooter}

\begin{styleFooter}
Assuming that all numerals are in the same positions (Spec,CardP or Spec,DP), we need to explain why the numerals for ‘two’ and ‘three’ have a different morphological impact on the following adjective than \textit{EIN}.
\end{styleFooter}

\begin{styleFooter}
\ \ Historically, the indefinite article derives from the numeral for ‘one’. Importantly, the inflectional properties of the singularity numeral have changed over time. In particular, OHG \textit{éin }behaves like Modern German \textit{zwei} ‘two’ and \textit{drei} ‘three’ but crucially not like Modern German \textit{EIN}: \textit{éin }takes a strong adjective like \textit{zwei} and \textit{drei}, but \textit{EIN} takes a weak one. Compare the dative strings in (54a) and (54b) (the (a)-example is taken from Demske 2001: 76):
\end{styleFooter}

\begin{styleFootnote}
(54)\ \ a.\ \ \textit{mít \ \ éin-emo rôt-emo tûoche}\ \ \ \ \ \ (OHG)
\end{styleFootnote}

\begin{styleFootnote}
\ \ with one-\textsc{st} \ \ red-\textsc{st} \ \ scarf.\textsc{neut}
\end{styleFootnote}

\begin{styleFooter}
\ \ \ \ ‘with one red scarf’
\end{styleFooter}

\begin{styleFootnote}
b.\ \ \textit{mit \ \ EIN-EM rot-en \ Tuch}
\end{styleFootnote}

\begin{styleFootnote}
\ \ with one-\textsc{st} \ \ red-\textsc{wk} scarf.\textsc{neut}
\end{styleFootnote}

\begin{styleFooter}
\ \ \ \ ‘with one red scarf’
\end{styleFooter}

\begin{styleFooter}
It is clear, then, that the singularity numeral in these two varieties of German is, in certain ways, different and should not receive the same account. 
\end{styleFooter}

\begin{styleFooter}
Recall that diachronically, the indefinite article derives from the numeral for ‘one’. On a somewhat speculative note, we could suggest that in the development from OHG to Modern German, \textit{éin} split into two parts: (vacuous) \textit{ein }+ \textit{Ø}\textsubscript{[-PL]}. Now, once these two separate elements are available, the indefinite article can occur with \textit{Ø}\textsubscript{[-PL]}\textit{\textsubscript{ }}(yielding the singularity numeral), but it can also start occurring by itself. In other words, the null part \textit{Ø}\textsubscript{[-PL]} can also be left out resulting in the emergence of the indefinite article. If we make this assumption, then it is not implausible to suggest that synchronically, Modern German \textit{EIN} is based on the indefinite article, at least in certain respects. This, in turn, provides an explanation of why Modern German \textit{EIN} has the morphological properties of a determiner – it consists in part of vacuous \textit{ein}. The latter element may trigger Impoverishment, and this accounts for the weak adjective in (53b) and (54b). \textit{EIN} has the semantic properties of a numeral because it involves \textit{Ø}\textsubscript{[-PL]}. In Section 6, I return to the diachronic split of \textit{éin} into two components suggesting that the same occurred with possessive and negative articles – the other two complex \textit{ein}{}-words.
\end{styleFooter}

\begin{styleFooter}
More generally, these morpho-syntactic and semantic differences indicate different positions of the two instances of \textit{ein}. Morpho-syntactically, adjectives following \textit{EIN} behave differently from those following other numerals. Assuming a composite analysis of \textit{EIN}, it is \textit{ein}, but not \textit{Ø}\textsubscript{[-PL]} or other numeral, that trigger Impoverishment. Note in this regard that \textit{ein} undergoes head movement from Art up the tree. Crucially though, it cannot move into Spec,CardP, which is occupied by \textit{Ø}\textsubscript{[-PL]} (see also next section). Semantically, \textit{EIN} involves singularity unlike \textit{ein}. If we assume that numerals are in Spec,CardP and specify the cardinality of a nominal, then that implies that \textit{ein} is in a different position. To be clear, like other articles, \textit{ein} is in a head position; like other numerals, \textit{Ø}\textsubscript{[-PL]} is in a phrasal position.
\end{styleFooter}

\begin{styleFooter}
5.1.2. Different Scope of \textit{mehr als}
\end{styleFooter}

\begin{styleFooter}
Above, I proposed that articles are in head positions but that numerals are in specifier positions. The second piece of evidence that the article and the numeral are in different positions derives from scopal facts. To begin, \textit{mehr als} ‘more than’ can take scope over the entire noun phrase where the nuclear stress is on the noun (55). In more detail, \textit{mehr als} in (55a-b) implies that the relevant person is more than just a student (perhaps he is also the speaker’s friend), and \textit{mehr als} in (55c) implies that not only exactly the one hundred students came but perhaps other students showed up or even other people that are not students:
\end{styleFooter}

\begin{styleFooter}
(55)\ \ a.\ \ \textit{Er ist }[\textit{ mehr als }\textit{\textsubscript{\ }}\textit{(ei)n Student }].
\end{styleFooter}

\begin{styleFooter}
\ \ \ \ he is \ \ \textsubscript{\ \ }more than a \ \ \ \ \ student.\textsc{masc}
\end{styleFooter}

\begin{styleStandard}
‘He is more than a student.’
\end{styleStandard}

\begin{styleFooter}
\ \ b.\ \ \textit{Er ist }[\textit{ mehr als \ }\textit{\textsubscript{\ }}\textit{mein Student }].
\end{styleFooter}

\begin{styleFooter}
\ \ \ \ he is \ \ \textsubscript{\ \ }more than my \ \ \ student.\textsc{masc}
\end{styleFooter}

\begin{styleStandard}
‘He is more than my student.’
\end{styleStandard}

\begin{styleFooter}
\ \ c.\ \ \textit{Es \ \ \ \ kamen }[\textit{ mehr als \ \ die hundert \ \ \ \ \ \ \ Studenten }].
\end{styleFooter}

\begin{styleFooter}
\ \ \ \ there came \ \ \ \textsubscript{\ \ }more than the one.hundred students
\end{styleFooter}

\begin{styleStandard}
‘More than the one hundred students came.’
\end{styleStandard}

\begin{styleFooter}
These scopal effects can be derived as follows.
\end{styleFooter}

\begin{styleFooter}
Illustrating with (55b), the possessive element has undergone movement to Spec,DP, and the article has moved to D. Consider the relevant portion of the syntactic tree in (56). Since Spec,DP is occupied, the scopal element must be outside the DP proper. For concreteness, I assume that it is adjoined to the DP:
\end{styleFooter}

\begin{styleFooter}
(56)\ \  \ \ \ \ \ \ \ \ \ \ DP
\end{styleFooter}

\begin{styleFooter}
[Warning: Draw object ignored][Warning: Draw object ignored]
\end{styleFooter}

\begin{styleFooter}
\ \ \ \ \ \ \textit{mehr als} \ \ \ \ \ \ \ \ \ \ \ \ \ \ DP
\end{styleFooter}

\begin{styleFooter}
[Warning: Draw object ignored][Warning: Draw object ignored]\ \ 
\end{styleFooter}

\begin{styleFooter}
\ \  \ \ \ \ \ \ \ \ PossP\ \ \ \ D’
\end{styleFooter}

\begin{styleFooter}
[Warning: Draw object ignored][Warning: Draw object ignored]\ \  \ \ \ \ \ \ \ \ \ \ \textit{m}{}-\textsubscript{k}
\end{styleFooter}

\begin{styleFooter}
\ \ \ \ \ \ D\ \  \ \ \ \ \ \ \ \ \ ArtP
\end{styleFooter}

\begin{styleFooter}
\ \ \ \  \ \ \ \ \ \ \ \ \textit{\ \ ein}\textsubscript{i}\ \  \ [ t\textsubscript{i} t\textsubscript{k} \textit{Student }]
\end{styleFooter}

\begin{styleFooter}
To be clear, \textit{mehr als} c-commands the entire DP. It takes scope over the possessive and the head noun allowing the various interpretations.
\end{styleFooter}

\begin{styleFooter}
\ \ In contrast, \textit{mehr als }may also take scope over numerals only (57). For the following sentences to be true, it must hold for (57a) that at least two students came and for (57b) that at least one hundred and one did:
\end{styleFooter}

\begin{styleFooter}
(57)\ \ a.\ \ \textit{Es \ \ \ \ kam }[\textit{ mehr als \ EIN }]\textit{ Student}.
\end{styleFooter}

\begin{styleFooter}
\ \ \ \ there came more than one \ \ student.\textsc{masc}
\end{styleFooter}

\begin{styleStandard}
‘More than one student came.’
\end{styleStandard}

\begin{styleFooter}
\ \ b.\ \ \textit{Es \ \ \ \ kamen die }[\textit{ mehr als \ \ HUNDERT }]\textit{ Studenten}.
\end{styleFooter}

\begin{styleFooter}
\ \ \ \ there came \ \ the \ \ more than one.hundred \ students.
\end{styleFooter}

\begin{styleFooter}
\ \ \ \ ‘The more than one hundred students came.’
\end{styleFooter}

\begin{styleFooter}
Returning briefly to the first set of data from this section, note that \textit{mehr als} in (55) occurs in the left periphery of the DP: it immediately precedes \textit{ein}{}-words as well as the definite article. In contrast, \textit{mehr als} in (57) immediately precedes the numerals – this is particularly clear in (57b), where \textit{mehr als} follows the definite article. In order to derive the different scopal readings in (55) vs. (57), I propose that \textit{mehr als} is in a different position when it precedes the numerals in (57). This, in turn, will provide an argument that the two instances of \textit{ein} (article vs. numeral) are in different positions. Consider this in more detail.
\end{styleFooter}

\begin{styleFooter}
\ \ To repeat, the scopal element \textit{mehr als} immediately precedes the numerals in (57). Given the readings above, it only affects the interpretation of the numerals. In order to find a plausible structure, I follow Svenonius’ (1993: 445-46) argumentation for adjectives in that the modifier and its modifiee must be embedded inside a specifier position. This prevents the modifier from taking scope over the entire noun phrase. If we assume that the same holds for modifiers and their numeral modifiees, then numerals are also in specifier positions in this case (rather than in a head positions in the extended projection of the noun), and their modifiers (including scopal elements) are part of the same specifiers. Again, I assume that this position is Spec,CardP (or after movement Spec,DP). I continue focusing on the singularity numeral.
\end{styleFooter}

\begin{styleFooter}
With this in mind, recall that the numeral \textit{EIN} consists of vacuous \textit{ein} and contentful \textit{Ø}\textsubscript{[-PL]}. The article \textit{ein} undergoes head movement from Art to Card (58). As to \textit{Ø}\textsubscript{[-PL]}, this component is responsible for singularity and stress (but has no independent segmental phonetic realization). I propose that \textit{Ø}\textsubscript{[-PL]}\textit{ }and the scopal element \textit{mehr als} are both part of a specifier, below illustrated with the specifier of CardP. Embedded in a specifier, \textit{mehr als} cannot take scope over the entire noun phrase:
\end{styleFooter}

\begin{styleFooter}
(58)\ \ \ \ DP
\end{styleFooter}

\begin{styleFooter}
[Warning: Draw object ignored][Warning: Draw object ignored]\ \ 
\end{styleFooter}

\begin{styleFooter}
\ \ \ \  \ \ \ \ \ \ \ \ \ \ D’
\end{styleFooter}

\begin{styleFooter}
[Warning: Draw object ignored][Warning: Draw object ignored]
\end{styleFooter}

\begin{styleFooter}
D\ \  \ \ \ \ \ CardP
\end{styleFooter}

\begin{styleFooter}
[Warning: Draw object ignored][Warning: Draw object ignored]\ \  \ \ \ \ \ \ \ 
\end{styleFooter}

\begin{styleFooter}
\ \ \ \  \ \ QP\ \ \ \  \ \ \ \ \ \ \ \ Card’
\end{styleFooter}

\begin{styleFooter}
[Warning: Draw object ignored][Warning: Draw object ignored][Warning: Draw object ignored][Warning: Draw object ignored]
\end{styleFooter}

\begin{styleFooter}
\ \ \ \ \ \ \ \ \ \ \ \ \textit{mehr als}\ \  \ Q \ \ \ \ \ Card\ \  \ \ \ \ \ \ \ ArtP
\end{styleFooter}

\begin{styleFooter}
[Warning: Draw object ignored][Warning: Draw object ignored]\ \ \ \  \ \ \ \ \ \ \ \ \ \ \ \textit{Ø}\textsubscript{[-PL]}\textit{\textsubscript{ }}\textit{\ \ ein}\textsubscript{i}
\end{styleFooter}

\begin{styleFooter}
\ \ \ \ \ \ \ \  \ \ \ \ \ \ \ \ \ \ \ \ \ \  \ \ \ \ \ \ Art’
\end{styleFooter}

\begin{styleFooter}
[Warning: Draw object ignored][Warning: Draw object ignored]\ \ 
\end{styleFooter}

\begin{styleFooter}
\ \ \ \ \ \ \ \ \ \  \ \ \ \ \ \ \ Art\ \  \ \ NumP
\end{styleFooter}

\begin{styleFooter}
\ \ \ \ \ \ \ \ \ \  \ \ \ \ \ \ \ \textit{ein}\textsubscript{i}\ \  \ \textit{Student}
\end{styleFooter}

\begin{styleFooter}
Some further remarks are in order. 
\end{styleFooter}

\begin{styleFooter}
Note again that the indefinite article has a feature for [+D], but not for definiteness. Consequently, \textit{ein} does not have to move to the DP-level but can surface in Card. Indeed, on the current analysis, \textit{Ø}\textsubscript{[-PL]}\textit{\textsubscript{ }}and \textit{ein} are separate elements in different positions: given standard assumptions about movement, \textit{ein} as a head moves to Card, but it cannot move into a specifier position, presumably to combine with \textit{Ø}\textsubscript{[-PL]}. However, there is another way to form a composite of these two elements. After Linearization, Vocabulary Insertion, and Copy Reduction, the relevant portion of (58) above can be illustrated as in (59):
\end{styleFooter}

\begin{styleFooter}
[Warning: Draw object ignored][Warning: Draw object ignored][Warning: Draw object ignored](59)\ \ 
\end{styleFooter}

\begin{styleFooter}
\ \ \textit{Ø}\textsubscript{[-PL]}\textit{\ \ ein} \ \ \ (-\textsc{infl})
\end{styleFooter}

\begin{styleFooter}
Null elements have to be licensed in some way (Kester 1996a, 1996b; Lobeck 1995; Rizzi 1986; but also Murphy 2018). This also applies to \textit{Ø}\textsubscript{[-PL]}. I proposed above that \textit{ein} licenses \textit{Ø}\textsubscript{[-PL]}\textit{\textsubscript{ }}by supporting it – both elements are adjacent to each other in (59). As with the possessive and negative articles, the operation of support can be instantiated by Local Dislocation (Section 4.1.2). This derives, in certain respects, the numeral from the indefinite article or, put differently, the numeral is based in part on the indefinite article.\footnote{\ Several authors (e.g., Bernstein 1993: 128, Julien 2002: 274) share the intuition that the indefinite article is merged lower and then raises to the DP-level. In contrast to the text proposal, these authors basically derive the indefinite article from the numeral. As far as I can see, there are a number of issues with this type of account, at least for Modern German (for some issues in English, see Perlmutter 1970: 239 fn. 10, 244 fn. 13): for instance, it is unclear to me how to account for the different morphological properties of \textit{ein} and \textit{EIN} vs. non-singularity, non-composite numerals as regards the inflections on the following adjectives (cf. (53) above) – if \textit{EIN} were a non-composite numeral, then the adjectives following \textit{EIN} and the (other) non-composite numerals should pattern the same in German, contrary to fact. Furthermore, deriving \textit{ein} from the singularity numeal does not explain why \textit{ein} can co-occur with other numerals under certain conditions (assuming that only one numeral may occupy Spec,CardP).}
\end{styleFooter}

\begin{styleFooter}
\ \ I summarize the discussion of scope. Unlike in (56), in (58) \textit{mehr als} is part of a specifier inside the DP c-commanding the numeral \textit{EIN}, or more precisely, the semantically active part \textit{Ø}\textsubscript{[-PL]}. In contrast, \textit{ein} was argued to be in Card. As such, similar to Section 5.1.1, these scopal data along with standard assumptions about movement present clear evidence that \textit{Ø}\textsubscript{[-PL]}\textit{\textsubscript{ }}is in a different location from \textit{ein}. Finally, recall that, in order to derive the strong reading of the numeral in (58), I assume that the article and \textit{Ø}\textsubscript{[-PL]}\textit{\textsubscript{ }}(or rather QP) move to D and Spec,DP, respectively; morphological spell-out occurs as in (59).
\end{styleFooter}

\begin{styleFooter}
\ \ To take stock thus far, deriving the Modern German singularity numeral from the combination of \textit{Ø}\textsubscript{[-PL]}\textit{\textsubscript{ }}and the vacuous article has a number of advantages. On the one hand, we can put all numerals in the same phrasal position(s), and we can explain the different scopal effects of \textit{mehr als} ‘more than’ on the various \textit{ein}{}-words and their related larger noun phrases. On the other hand, we can account for the facts that \textit{EIN} inflects like the indefinite article, that both of these elements cannot co-occur, and that both elements can take a weak adjective. In other words, the splitting of the numeral into two underlying parts and their subsequent composite spell-out accounts for the hybrid properties of \textit{EIN}\textstyleFootnoteSymbol{ }.\footnote{\ A composite analysis could also explain the alternation between the indefinite article and the numeral for ‘one’ in some other languages closely related to Standard German:\par (i)\ \ a.\ \ \textit{a(n)}\ \ vs.\ \ \textit{one}\ \ \ \ (English)\par \ \ \ \ b.\ \ \textit{a(n)}\ \ vs.\ \ \textit{eyn}\ \ \ \ (Yiddish)\par c.\textit{\ \ een}\ \ vs.\ \ \textit{één}\ \ \ \ (Dutch)\par d.\ \ \textit{a}\ \ vs.\ \ \textit{õa}\ \ \ \ (Bavarian German) \par In some of these cases, the spell-out of the relevant component(s) results in quite different surface forms.} To highlight the special status of \textit{EIN} further, consider the interaction between \textit{ein} and other numerals in more detail. 
\end{styleFooter}

\begin{styleFooter}
\ \ Returning briefly to the co-occurrence of \textit{ein} and non-singularity numerals, consider the examples in (60), where (60a) contains plural \textit{ein} and (60b) involves the negative article in a plural context:
\end{styleFooter}

\begin{styleStandard}
(60)\ \ a. \ \%\ \ \textit{und so ne zwei Kämpfer die \ \ alles umpflügen,}
\end{styleStandard}

\begin{styleStandard}
\ \ \ \ and so a \ two \ fighters \ \ who all \ \ \ \ under.plough
\end{styleStandard}

\begin{styleStandard}
\ \ \ \ ‘and such two guys, who create such chaos’
\end{styleStandard}

\begin{styleFooter}
\ \ b.\ \ \textit{Es waren k-eine zehn Leute \ da}.
\end{styleFooter}

\begin{styleFooter}
\ \ \ \ it \ \textsubscript{\ }were \ \ \textsc{neg}{}-a \textsubscript{\ }ten \ \ people there
\end{styleFooter}

\begin{styleFooter}
\ \ \ \ ‘There were not even ten people.’
\end{styleFooter}

\begin{styleFooter}
These distributions follow from the fact that unlike \textit{EIN}, other numerals are not composite forms – they are not morphologically related to \textit{ein}. Consequently, the non-composite numerals and the article \textit{ein} can co-occur. As to the structure, I suggest that the numerals in (60) are in Spec,CardP and that \textit{ein} is in a higher head position (D).
\end{styleFooter}

\begin{styleFooter}\itshape
5.2. \ \ Article / Numeral vs. Adjective 
\end{styleFooter}

\begin{styleFooter}
Above, I derived some of the morpho-syntactic aspects of the numeral from the indefinite article. Among others, this accounted for the fact that the numeral and the indefinite article may not (separately) co-occur although they are in different positions (specifier vs. head). This now makes the prediction that if a form of \textit{ein} were to occur with a determiner, then this \textit{ein} could not be the indefinite article (or the numeral). We will see again that such distributions are indeed possible, but I argue that this instance of \textit{ein} is adjectival. In what follows I provide more evidence for this categorically different \textit{ein} suggesting that it is in a different position. 
\end{styleFooter}

\begin{styleFooter}
5.2.1. Different Morphology
\end{styleFooter}

\begin{styleFooter}
Determiners do not have weak endings. There is just one exception. As discussed in Chapter 3, Section 4, some determiners may optionally have a weak ending in genitive masculine/neuter contexts. However, as seen there, this does not apply to \textit{ein}{}-words. Now, we saw in Section 3 above that adjectival \textit{eine} can have a weak ending in singular contexts (e.g., \textit{der ein-e}) or in plural ones (e.g., \textit{die ein-en}). Thus, this element behaves morphologically like a regular adjective (and not like an \textit{ein}{}-word). This provides some first motivation for the current classification of \textit{ein} as regards adjective vs. article/numeral. In addition to the difference in weak inflection, there are other inflectional distinctions. Consider this in more detail.
\end{styleFooter}

\begin{styleFooter}
\ \ As amply documented in Section 2, the indefinite article and the singularity numeral cannot have an ending in nominative masculine and nominative/accusative neuter contexts when an overt noun follows (61a-b). An account of this was provided in the context of adjectival inflections in Chapter 2, Section 2.2.2. We have also seen that adjectival \textit{eine} has an inflection when a noun follows. Recall though that is has to occur in a definite context – an indefinite environment yields ungrammaticality (61c):
\end{styleFooter}

\begin{styleFooter}
(61)\ \ a. \ \ \textit{(ei)n-(*er) Sohn}\ \ 
\end{styleFooter}

\begin{styleFooter}
\ \ \ \  a-\textsc{st} \ \ \ \ \ \ \ \ \ son.\textsc{masc}
\end{styleFooter}

\begin{styleFooter}
‘a son’
\end{styleFooter}

\begin{styleFooter}
\ \ b. \ \ \textit{EIN-(*ER) Sohn}\ \ 
\end{styleFooter}

\begin{styleFooter}
\ \ \ \ one-\textsc{st} \ \ \ \ \ \ \ son.\textsc{masc}
\end{styleFooter}

\begin{styleFooter}
‘one son’
\end{styleFooter}

\begin{styleFooter}
\ \ c. \ \textit{*\ \ ein-er \ Sohn}\ \ 
\end{styleFooter}

\begin{styleFooter}
\ \ \ \ one-\textsc{st} son.\textsc{masc}
\end{styleFooter}

\begin{styleFooter}
It is less well known that PP can be added to the left periphery of the noun phrase (Bhatt 1990, Fortmann 1996, Haider 1992, Roehrs 2020a). This includes \textit{von}{}-possessives. Interestingly, this context yields the same facts as regards the different kinds of \textit{ein}. Compare the examples in (61) to their respective counterparts below: 
\end{styleFooter}

\begin{styleFooter}
(62)\ \ a.\ \ \textit{von Peter (ei)n Sohn}
\end{styleFooter}

\begin{styleFooter}
\ \ \ \ of \ \ \ Peter \textsubscript{\ }\ a \ \ \ \ son.\textsc{masc}
\end{styleFooter}

\begin{styleFooter}
‘a son of Peter’
\end{styleFooter}

\begin{styleFooter}
b.\ \ \textit{von Peter EIN Sohn}
\end{styleFooter}

\begin{styleFooter}
\ \ \ \ of \ \ \ Peter one \ son.\textsc{masc}
\end{styleFooter}

\begin{styleFooter}
‘one son of Peter’
\end{styleFooter}

\begin{styleFooter}
c. \ *\ \ \textit{von Peter ein-er \ Sohn}
\end{styleFooter}

\begin{styleFooter}
\ \ \ \ of \ \ \ Peter one-\textsc{st} son.\textsc{masc}
\end{styleFooter}

\begin{styleFooter}
The judgments reverse with a Saxon Genitive, a context involving definiteness. Compare (62a-c) to (63a-c). Note also that adjectival \textit{eine} has a strong ending in this context (63c), again exhibiting similarities with a regular adjective (63d):
\end{styleFooter}

\begin{styleFooter}
(63)\ \ a. \ *\ \ \textit{Peters (ei)n Sohn}
\end{styleFooter}

\begin{styleFooter}
\ \ \ \ Peter’s \textsubscript{\ }a \ \ \ \ son.\textsc{masc}
\end{styleFooter}

\begin{styleFooter}
\ \ b. \ *\ \ \textit{Peters EIN Sohn}
\end{styleFooter}

\begin{styleFooter}
\ \ \ \ Peter’s one son.\textsc{masc}
\end{styleFooter}

\begin{styleFooter}
\ \ c.\ \ \textit{Peters \ ein-er \ Sohn}
\end{styleFooter}

\begin{styleFooter}
\ \ \ \ Peter’s one-\textsc{st} son.\textsc{masc}
\end{styleFooter}

\begin{styleStandard}
‘one of the two of Peter’s sons’ 
\end{styleStandard}

\begin{styleStandard}
‘Peter’s one son’
\end{styleStandard}

\begin{styleFooter}
\ \ d.\ \ \textit{Peters \ groß-er Sohn}
\end{styleFooter}

\begin{styleFooter}
\ \ \ \ Peter’s big-\textsc{st} \ \ son.\textsc{masc}
\end{styleFooter}

\begin{styleStandard}
‘Peter’s big son’
\end{styleStandard}

\begin{styleFooter}
As is expected, concomitant with the different distributions in (61-62) vs. (63), there is a difference in the semantics: while there is no duality presupposition in the first two sets of data, it is present in (63c) of the third paradigm. This is confirmed by the fact that this nominal can straightforwardly be followed by \textit{Peters anderer Sohn} ‘Peter’s other son’. 
\end{styleFooter}

\begin{styleFooter}
The different inflections on \textit{ein }and the different semantics just discussed correlate with another morpho-syntactic distinction: the definite article can replace \textit{ein} in (61a-b) and (62a-b) but not \textit{einer} in (63c). Consider (64a-b) vs. (64c). With a definite article present in (64a-b), adjectival \textit{eine} can be added to these strings yielding (64d):
\end{styleFooter}

\begin{styleFooter}
(64)\ \ a.\ \ \textit{der Sohn}
\end{styleFooter}

\begin{styleFooter}
\ \ \ \ the son.\textsc{masc}
\end{styleFooter}

\begin{styleFooter}
‘the son’
\end{styleFooter}

\begin{styleFooter}
\ \ b.\ \ \textit{von Peter der Sohn}
\end{styleFooter}

\begin{styleFooter}
\ \ \ \ of \ \ Peter \textsubscript{\ }the son.\textsc{masc}
\end{styleFooter}

\begin{styleFooter}
‘Peter’s son’
\end{styleFooter}

\begin{styleFooter}
\ \ c. *\ \ \textit{Peters \ der Sohn}
\end{styleFooter}

\begin{styleFooter}
\ \ \ \ Peter’s the son.\textsc{masc}
\end{styleFooter}

\begin{styleFooter}
\ \ d.\ \ \textit{(von Peter) der eine Sohn}
\end{styleFooter}

\begin{styleFooter}
\ \ \ \  \ of \ \ Peter \ \textsubscript{\ }the one \ son.\textsc{masc}
\end{styleFooter}

\begin{styleFooter}
‘one of the two sons (of Peter’s)’
\end{styleFooter}

\begin{styleFooter}
Thus far, we have seen that the indefinite article and the singularity numeral exhibit different inflections than adjectival \textit{eine} – the latter inflects like an adjective (and must appear in definite contexts). Before I continue the discussion of the inflections on the different kinds of \textit{ein}, I provide a interim summary of the data and account for the different distributions of the possessives and their following elements.
\end{styleFooter}

\begin{styleFooter}
\ \ To take stock thus far, \textit{von}{}-possessives can occur with the definite article, the indefinite article, the singularity numeral, but not with adjectival \textit{eine} (unless a definite article is also present). In contrast, Saxon Genitives can occur with adjectival \textit{eine}, but not with the definite article, the indefinite article, and the singularity numeral. In Roehrs (2020a), I propose that \textit{von}{}-possessives are outside the DP proper but that Saxon Genitives are in Spec,DP. Given this analysis, it follows that the former allows different determiners including the singularity numeral but that the latter does not (recall also from Chapter 2, Section 2.3 that Saxon Genitives involve null articles). The reason why \textit{von}{}-possessives do not tolerate adjectival \textit{eine}, but Saxon Genitives do, is that the former do not involve a definite context but that the latter do.
\end{styleFooter}

\begin{styleFooter}
\ \ Returning to the inflections, the morphological parallelism of adjectival \textit{eine} and regular adjectives can be strengthened further. As already seen above, the indefinite article and the singularity numeral have endings different from the following adjective. This is illustrated here again in the nominative and dative neuter: 
\end{styleFooter}

\begin{styleFooter}
(65)\ \ a.\ \ \textit{(ei)n frisch-es \ \ \ \ \ \ \ Brot}
\end{styleFooter}

\begin{styleFooter}
\ \ \ \  a \ \ \ \ fresh-\textsc{nom.st} bread.\textsc{neut}
\end{styleFooter}

\begin{styleStandard}
‘a loaf of fresh bread’
\end{styleStandard}

\begin{styleFooter}
\ \ b.\ \ \textit{EIN frisch-es \ \ \ \ \ \ \ Brot}
\end{styleFooter}

\begin{styleFooter}
\ \ \ \ one fresh-\textsc{nom.st} bread.\textsc{neut}
\end{styleFooter}

\begin{styleStandard}
‘one loaf of fresh bread’
\end{styleStandard}

\begin{styleFooter}
\ \ c.\ \ \textit{(ei)n-em \ \ frisch-en Brot}
\end{styleFooter}

\begin{styleFooter}
\ \ \ \  a-\textsc{dat.st} fresh-\textsc{wk} bread.\textsc{neut}
\end{styleFooter}

\begin{styleStandard}
‘a loaf of fresh bread’
\end{styleStandard}

\begin{styleFooter}
\ \ d.\ \ \textit{EIN-EM \ \ \ \ \ frisch-en Brot}
\end{styleFooter}

\begin{styleFooter}
\ \ \ \ one-\textsc{dat.st} fresh-\textsc{wk} bread.\textsc{neut}
\end{styleFooter}

\begin{styleStandard}
‘one loaf of fresh bread’
\end{styleStandard}

\begin{styleFooter}
Above, we saw that adjectival \textit{eine} and a regular adjective can each have a strong inflection in isolation. We expect then that when both of these elements co-occur, they both have the same endings. This is indeed borne out: both elements are weak in (66a-b) but strong in (66c):
\end{styleFooter}

\begin{styleFooter}
(66)\ \ a.\ \ \textit{da-s \ \ \ \ \ \ \ \ \ \ \ \ ein-e \ \ \ \ frisch-e \ \ Brot}
\end{styleFooter}

\begin{styleFooter}
\ \ \ \ the-\textsc{nom.st} one-\textsc{wk} fresh-\textsc{wk} bread.\textsc{neut}
\end{styleFooter}

\begin{styleStandard}
‘one of the two loaves of fresh bread’
\end{styleStandard}

\begin{styleFooter}
\ \ b.\ \ \textit{de-m \ \ \ \ \ \ \ \ \ ein-en \ \ \ frisch-en Brot}
\end{styleFooter}

\begin{styleFooter}
\ \ \ \ the-\textsc{dat.st} one-\textsc{wk} fresh-\textsc{wk} bread.\textsc{neut}
\end{styleFooter}

\begin{styleStandard}
‘one of the two loaves of fresh bread’
\end{styleStandard}

\begin{styleFooter}
\ \ c.\ \ \textit{Peters \ ein-er \ lieb-er \ Sohn \ \ \ \ \ \ \ und Peters \ ander-er lieb-er \ Sohn \ \ \ \ \ \ \ \ \ \ }
\end{styleFooter}

\begin{styleFooter}
Peter’s one-\textsc{st} nice-\textsc{st}\textsubscript{ }son.\textsc{masc} and Peter’s other-\textsc{st} \ nice-\textsc{st} son.\textsc{masc} 
\end{styleFooter}

\begin{styleFooter}
\textit{verstehen \ \ sich \ \ gut}.
\end{styleFooter}

\begin{styleFooter}
understand \textsc{refl} well 
\end{styleFooter}

\begin{styleFooter}
\ \ \ \ ‘Peter’s first nice son and Peter’s second nice son get along well.’
\end{styleFooter}

\begin{styleFooter}
This juxtaposition of \textit{ein} and regular adjectives presents more evidence that \textit{ein} as an article or numeral in (65) is different from \textit{ein} as an adjective in (66). I return to the parallelism of adjectival \textit{eine} and \textit{andere} ‘other’ seen in (66c) in Section 5.2.4 below. 
\end{styleFooter}

\begin{styleFooter}
Recall from the discussion of feature specifications in Section 4.3.2 that adjectival \textit{eine} involves an adjectival structure (i.e., it has InflP on the top of AP). Now, in order to derive the adjectival endings in a uniform way (compatible with Chapter 2), I propose that the structure of adjectival \textit{eine} is merged in a position similar to that of other adjectives. Presumably, this is the highest Spec,AgrP in the nominal structure (see also Gallmann 2004: 155, Pafel 2005: 179). The tree representation for adjectival \textit{eine} is provided in the next subsection.
\end{styleFooter}

\begin{styleFooter}
5.2.2. Co-occurrence with Possessive\textit{ ein}{}-words 
\end{styleFooter}

\begin{styleFooter}
Above, \textit{ein} was discussed in the context of prenominal possessives like \textit{von}{}-phrases and Saxon Genitives. Turning to possessive articles, it is worth pointing out again that unlike the indefinite and negative articles in (67a-b), the possessive article in (67c) may occur with adjectival \textit{eine}. This includes cases where the possessive article is part of a periphrastic possessive construction as in (67d):
\end{styleFooter}

\begin{styleFooter}
(67)\ \ a. *\ \ \textit{(ei)ne eine Freundin}\newline
\ \ \ \  an \ \ \ \ \textsubscript{\ }one \ girlfriend.\textsc{fem}
\end{styleFooter}

\begin{styleFooter}
\ \ b. *\ \ \textit{keine eine Freundin}\newline
\ \ \ \ no \ \ \ \ \textsubscript{\ }one \ girlfriend.\textsc{fem}
\end{styleFooter}

\begin{styleFooter}
c. \ \ \textit{meine eine Freundin}\newline
\ \ my \ \ \ \ \textsubscript{\ }one \ girlfriend.\textsc{fem}
\end{styleFooter}

\begin{styleFooter}
\ \ \ \ ‘one of the two of my girlfriends’ 
\end{styleFooter}

\begin{styleFooter}
‘my one girlfriend’
\end{styleFooter}

\begin{styleFooter}
\ \ d.\ \ \textit{Peter seine eine Freundin}
\end{styleFooter}

\begin{styleFooter}
\ \ \ \ Peter his \ \ \ \ one \ girlfriend.\textsc{fem}
\end{styleFooter}

\begin{styleFooter}
\ \ \ \ ‘one of the two of Peter’s girlfriends’ 
\end{styleFooter}

\begin{styleFooter}
‘Peter’s girlfriend’
\end{styleFooter}

\begin{styleFooter}
The sequence of two instances of \textit{ein} in (67a) and (67b) cannot simply be ruled out by haplology. Roughly, this describes the reduction of identical sequences of sounds (for the discussion of (67a) in this regard, see Bhatt 1990: 201; more generally, see Neeleman \& van de Koot 2006). If this were the case, we would expect (67c-d) to be ungrammatical as well, similar to (67a-b), but contrary to fact. The difference between (67a-b) and (67c-d) has to do with definiteness. Again, I conclude that adjectival \textit{eine} must occur in definite contexts.
\end{styleFooter}

\begin{styleFooter}
\ \ Note though that \textit{ein} preceding the noun in (67) is actually morphologically ambiguous in the feminine between the singularity numeral and the adjective (indeed, both of these elements are also stressed). Thus, it could be objected that \textit{eine} here is not the adjective. To confirm the analysis of \textit{ein} as an adjective in (67), I consider each of the two analytical options of the \textit{ein} before the noun (i.e., numeral vs. adjective) more carefully.
\end{styleFooter}

\begin{styleFooter}
\ \ I start with the first option, where lower \textit{ein} could be taken to be the numeral \textit{EIN}. Consider cases in the nominative masculine, where the morphology clearly distinguishes between \textit{ein}{}-words and adjectives. As can be seen in (68), these instances with the numeral \textit{EIN} are ungrammatical:
\end{styleFooter}

\begin{styleFooter}
(68)\ \ a. \ \textit{*\ \ (ei)n EIN Freund} 
\end{styleFooter}

\begin{styleFooter}
\ \ \ \ \  a \ \ \ \ one \ friend.\textsc{masc}
\end{styleFooter}

\begin{styleFooter}
\ \ b. \ \textit{*\ \ kein EIN Freund} 
\end{styleFooter}

\begin{styleFooter}
\ \ \ \ \ no \ \ \ one \ friend.\textsc{masc}
\end{styleFooter}

\begin{styleFooter}
\ \ c. \ \textit{*\ \ mein EIN Freund} 
\end{styleFooter}

\begin{styleFooter}
\ \ \ \ \ my \ \ one \ friend.\textsc{masc}
\end{styleFooter}

\begin{styleFooter}
\ \ d. \ \textit{*\ \ Peter sein EIN Freund} 
\end{styleFooter}

\begin{styleFooter}
\ \ \ \ \ Peter \textsubscript{\ }his \ one \ friend.\textsc{masc}
\end{styleFooter}

\begin{styleFooter}
Given the ungrammaticality of (67a-b), the status of (68a-b) may not be unexpected. However, note that unlike the grammatical cases in (67c-d), their counterparts in (68c-d) are ungrammatical here. This means that the lower \textit{ein} in (67c-d) cannot involve the numeral \textit{EIN}. 
\end{styleFooter}

\begin{styleFooter}
In order to account for (68), observe that there are two copies of \textit{ein} in these instances, one before \textit{EIN} and one that has – apparently – formed a composite with \textit{Ø}\textsubscript{[-PL]} yielding the singularity numeral \textit{EIN} itself. In order to account for the ungrammaticality, I suggest that after Copy Reduction, only the higher copy of \textit{ein} is present. This means that (lower) \textit{Ø}\textsubscript{[-PL]} cannot form a composite with \textit{ein}; that is, the distributions in (68) cannot be generated. This explains the ungrammaticality of (68).\footnote{\ In Chapter 8, Section 2.2.2, we will see that there are instances where a second copy of \textit{ein} can occur (ia). Crucially, this second copy must be a reduced form of \textit{ein} as can be verified in (ib):\par \ \ (i)\ \ a.\ \ \textit{EINE so’ne Tochter}\par \ \ \ \ \ \ one \ \ \ so.a \ \ daughter.\textsc{fem}\par \ \ \ \ \ \ ‘one such daughter’\par \ \ \ \ b. \ *\ \ \textit{EINE so \ \ \ \ eine Tochter}\par \ \ \ \ \ \ one \ \ \ such a \ \ \ \ \ daughter.\textsc{fem}\par Following Nunes (2001), I argue in Chapter 8 that encliticized elements can escape Copy Reduction. Note that a reduced copy of \textit{ein} is not possible with \textit{Ø}\textsubscript{[-PL]} as \textit{‘n} cannot encliticize to a null element. This can clearly be seen in the feminine (which involves adjectival inflections, thus distinguishing the two instances of \textit{ein}): *\textit{meine }[Ø\textsubscript{[-PL]}]\textit{’ne Tochter} ‘my one daughter’. \ } \ \ 
\end{styleFooter}

\begin{styleFooter}
This is different for adjectival \textit{eine}, the second option for lower \textit{ein} in (67). While this element cannot occur with an indefinite or a negative article in the nominative masculine either (69a-b), it can appear with the possessives under discussion here (69c-d), much like the Saxon Genitives discussed in Section 5.2.1:
\end{styleFooter}

\begin{styleFooter}
(69)\ \ a. \ \textit{*\ \ (ei)n ein-er \ Freund} 
\end{styleFooter}

\begin{styleFooter}
\ \ \ \  a \ \ \ \ one-\textsc{st} friend.\textsc{masc}
\end{styleFooter}

\begin{styleFooter}
\ \ b. \ \textit{*\ \ kein ein-er \ \ Freund} 
\end{styleFooter}

\begin{styleFooter}
\ \ \ \ no \ \ \ one-\textsc{st} friend.\textsc{masc}
\end{styleFooter}

\begin{styleFooter}
\ \ c.\ \ \textit{mein ein-er Freund} 
\end{styleFooter}

\begin{styleFooter}
\ \ \ \ my \ \ one-\textsc{st} friend.\textsc{masc}
\end{styleFooter}

\begin{styleFooter}
\ \ \ \ ‘one of the two of my friends’ 
\end{styleFooter}

\begin{styleFooter}
‘my one friend’
\end{styleFooter}

\begin{styleFooter}
\ \ d.\ \ \textit{Peter sein ein-er Freund} 
\end{styleFooter}

\begin{styleFooter}
\ \ \ \ Peter \textsubscript{\ }his \ one-\textsc{st} friend.\textsc{masc}
\end{styleFooter}

\begin{styleFooter}
\ \ \ \ ‘one of the two of Peter’s friends’ 
\end{styleFooter}

\begin{styleFooter}
‘Peter’s one friend’
\end{styleFooter}

\begin{styleFooter}
This essentially yields the distribution in (67) above confirming the claim that (67) involves adjectival \textit{eine}. Note again that (69a-b) are ungrammatical as adjectival \textit{eine} is not in definite contexts here. No such semantic problems arise with the possessives in (69c-d), which are definite in interpretation. I turn to the syntactic derivation of adjectival \textit{eine}.
\end{styleFooter}

\begin{styleFooter}
\ \ I have argued that \textit{ein} as part of the possessive article and \textit{eine} as an adjective are of different lexical categories. I can point out then that they do not stand in a relevant morpho-syntactic relation with one another. If so, it is expected that these types of \textit{ein} can co-occur. Recalling that adjectival \textit{eine} involves InfP at the top of its AP, the syntactic distribution can be illustrated as follows:
\end{styleFooter}

\begin{styleFooter}
(70)\ \  \ \ \ \ \ \ \ \ \ \ DP
\end{styleFooter}

\begin{styleFooter}
[Warning: Draw object ignored][Warning: Draw object ignored]\ \ 
\end{styleFooter}

\begin{styleFooter}
\ \ \ \ \ \ \ \ \ PossP\ \ \ \ D’
\end{styleFooter}

\begin{styleFooter}
[Warning: Draw object ignored][Warning: Draw object ignored]\textit{m{}-}
\end{styleFooter}

\begin{styleFooter}
D\ \  \ \ \ \ \ \ \ \ AgrP
\end{styleFooter}

\begin{styleFooter}
[Warning: Draw object ignored][Warning: Draw object ignored]\ \  \ \ \ \ \ \ \ \ \ \textit{eine}\textsubscript{i}
\end{styleFooter}

\begin{styleFooter}
\ \ \ \  \ \ \ \ \ \ \ \ InflP\ \  \ \ \ \ \ \ \ \ Agr’
\end{styleFooter}

\begin{styleFooter}
[Warning: Draw object ignored][Warning: Draw object ignored]\ \ \ \  \ \ \ \ \ \ \textit{eine}\textsubscript{ADJ}
\end{styleFooter}

\begin{styleFooter}
\ \ \ \  \ \ \ \ \ \ \ \ \ Agr\ \  \ \ \ \ \ \ \ \ ArtP
\end{styleFooter}

\begin{styleFooter}
[Warning: Draw object ignored][Warning: Draw object ignored]\ \ \ \ \ \  \ \ \ \ \ \ \ \ \ \textit{eine}\textsubscript{i}
\end{styleFooter}

\begin{styleFooter}
\ \ \ \ \ \ \ \ \ \ \ \  \ \ \ \ \ \ \ \ Art’
\end{styleFooter}

\begin{styleFooter}
[Warning: Draw object ignored][Warning: Draw object ignored]
\end{styleFooter}

\begin{styleFooter}
\ \ \ \ \ \ \ \ \ \  \ \ \ \ \ \ \ Art\ \  \ \ \ \ \ \ \ \ NP
\end{styleFooter}

\begin{styleFooter}
\ \ \ \ \ \ \ \ \ \  \ \ \ \ \ \ \ \textit{eine}\textsubscript{i}\ \  \ \ \ \textit{Freundin}
\end{styleFooter}

\begin{styleFooter}
This derivation also applies to more complex cases.
\end{styleFooter}

\begin{styleFooter}
Unsurprisingly, like possessive articles, \textit{diese} ‘this’ can also co-occur with adjectival \textit{eine} (71a). As discussed in Chapter 2, \textit{diese} and \textit{meine} can co-occur as well (71b). In fact, adjectival \textit{eine} can appear with both of these elements at the same time (71c), and an adjective can also be added (71d) (recall that there is no duality presupposition with demonstratives):
\end{styleFooter}

\begin{styleFooter}
(71)\ \ a.\ \ \textit{diese eine Freundin}
\end{styleFooter}

\begin{styleFooter}
\ \ \ \ this \ \ \textsubscript{\ }one \ girlfriend.\textsc{fem}
\end{styleFooter}

\begin{styleFooter}
\ \ \ \ ‘this one girlfriend’
\end{styleFooter}

\begin{styleFooter}
\ \ b.\ \ \textit{diese meine Freundin}
\end{styleFooter}

\begin{styleFooter}
\ \ \ \ this \ \ my \ \ \ \ \ girlfriend.\textsc{fem}
\end{styleFooter}

\begin{styleFooter}
\ \ \ \ ‘this girlfriend of mine’
\end{styleFooter}

\begin{styleFooter}
\ \ c. \ \ \textit{diese meine eine Freundin}
\end{styleFooter}

\begin{styleFooter}
\ \ \ \ this \ \ my \ \ \ \ \ one \ girlfriend.\textsc{fem}
\end{styleFooter}

\begin{styleFooter}
\ \ \ \ ‘this one girlfriend of mine’
\end{styleFooter}

\begin{styleFooter}
\ \ d.\ \ \textit{und diese meine eine große Liebe \ \ \ \ \ \ zu sehen}
\end{styleFooter}

\begin{styleFooter}
\ \ \ \ and this \ \ my \ \ \ \ \ one \ great \ love.\textsc{fem} to see
\end{styleFooter}

\begin{styleFooter}
\ \ \ \ ‘and to see this one great love of mine’
\end{styleFooter}

\begin{styleFooter}
(\href{https://www.kirmesforum.de/threads/faszination-break-%09dancer.1520/page-2}{\textstyleInternetlink{https://www.kirmesforum.de/threads/faszination-break-\ \ dancer.1520/page-2}})
\end{styleFooter}

\begin{styleFooter}
On current assumptions, \textit{diese} is in Spec,DP in (71a) but in Spec,LPP in (71b-d); \textit{meine} is in Spec,DP in (71b-d). Thus, these distributions are expected to be possible. Conversely, if we were to assume just one type of \textit{ein}; that is, if we were to try and derive both the numeral \textit{EIN} and adjectival \textit{eine} from (vacuous) \textit{ein}, then a number of distributions would be hard to account for. This is what I focus on next.
\end{styleFooter}

\begin{styleFooter}
5.2.3. Co-occurrence of \textit{ein} and Determiners Revisited
\end{styleFooter}

\begin{styleFooter}
In this subsection, I make the point that adjectival \textit{eine} is a different element from another perspective. Recall that the indefinite article and the numeral have no ending when they appear in front of a noun in certain morpho-syntactic contexts (72a-b). As stated before, a definite determiner cannot precede either of them (72c-d):
\end{styleFooter}

\begin{styleFooter}
(72)\ \ a.\ \ \textit{(ei)n Mann}
\end{styleFooter}

\begin{styleFooter}
\ \ \ \  a \ \ \ \ \textsubscript{\ }man.\textsc{masc}
\end{styleFooter}

\begin{styleFooter}
\ \ \ \ ‘a man’
\end{styleFooter}

\begin{styleFooter}
\ \ b.\ \ \textit{EIN Mann}
\end{styleFooter}

\begin{styleFooter}
\ \ \ \ one \textsubscript{\ }man.\textsc{masc}
\end{styleFooter}

\begin{styleFooter}
\ \ \ \ ‘one man’
\end{styleFooter}

\begin{styleFooter}
\ \ \ c. \ *\ \ \textit{der (ei)n Mann}
\end{styleFooter}

\begin{styleFooter}
\ \ \ \ the \ one \ man.\textsc{masc}
\end{styleFooter}

\begin{styleFooter}
\ \ d. \ *\ \ \textit{der EIN Mann}
\end{styleFooter}

\begin{styleFooter}
\ \ \ \ the one \ man.\textsc{masc}
\end{styleFooter}

\begin{styleFooter}
Note that \textit{ein }with a weak inflection is possible here (73a). Finally, recall again that Saxon Genitives can occur with \textit{ein }provided the latter has a strong inflection. Compare (73b) to (73c-d):
\end{styleFooter}

\begin{styleFooter}
(73)\ \ a.\ \ *\textit{(der) ein-e \ \ \ \ Mann}
\end{styleFooter}

\begin{styleFooter}
\ \ \ \  \ \ the \ \ one-\textsc{wk}\textsubscript{ }man.\textsc{masc}
\end{styleFooter}

\begin{styleFooter}
\ \ \ \ ‘one of the two men’
\end{styleFooter}

\begin{styleFooter}
\ \ b.\ \ \textit{Peters \ ein-er Sohn} 
\end{styleFooter}

\begin{styleFooter}
\ \ \ \ Peter’s one-\textsc{st} son.\textsc{masc}
\end{styleFooter}

\begin{styleFooter}
‘one of the two of Peter’s sons’ 
\end{styleFooter}

\begin{styleFooter}
‘Peter’s one son’
\end{styleFooter}

\begin{styleFooter}
\ \ c. \ *\ \ \textit{Peters \ (ei)n Sohn} 
\end{styleFooter}

\begin{styleFooter}
\ \ \ \ Peter’s \ a \ \ \ \ \ son.\textsc{masc}
\end{styleFooter}

\begin{styleFooter}
\ \ d. \ *\ \ \textit{Peters \ EIN Sohn} 
\end{styleFooter}

\begin{styleFooter}
\ \ \ \ Peter’s one \ son.\textsc{masc}
\end{styleFooter}

\begin{styleFooter}
Consider (72) and (73) in the context of the current system.
\end{styleFooter}

\begin{styleFooter}
Under current assumptions, the indefinite article in (72c) is ruled out because only one article can appear in a noun phrase. If we assume again that \textit{EIN} consists, in part, of (vacuous) \textit{ein}, then we can rule out the option of (72d) under the same assumption. In other words, these cases are not ruled out because \textit{ein} does not have a weak adjectival ending. Rather, there are two articles in the DP, \textit{der} and (vacuous) \textit{ein}, but only one of them can originate in ArtP and move to the left periphery. Turning to (73), if we accept the assumption that adjectival \textit{eine} is a non-composite form, then this element is predicted to occur with another article in (73a). Similarly, if we assume that Saxon Genitives as in (73b) involve null articles (Chapters 2), then this distribution follows from the same assumptions. This means that (73c-d) are out as there are two articles, vacuous \textit{ein} and a null article. 
\end{styleFooter}

\begin{styleFooter}
To be clear, then, assuming two basic types of \textit{ein}, we can explain why sometimes a form of \textit{ein} cannot occur with another article but sometimes it can, provided that form of \textit{ein} appears in a definite context. These distributions are hard to account for if we assume just one type of \textit{ein}. Indeed, if \textit{EIN} and adjectival \textit{eine} were both derived from vacuous \textit{ein}, then (73a-b) should be, under these assumptions, ungrammatical – there would be two articles just like (73c-d). Given the different grammaticalities, this, however, does not seem to be the case. 
\end{styleFooter}

\begin{styleFooter}
Finally, if a different analysis than the current one is to be pursued, then assumptions about the base-generated position of \textit{ein} and its different possible landing sites need to be layed out clearly. Additionally, the assumptions about the inflections on \textit{ein} need to be spelled out in detail. As demonstrated in previous chapters, a simple surface-oriented explanation does not suffice. This means that if the inflection on \textit{ein} is invoked as an explanation for some of the patterns above, it has to be part of a (more) comprehensive account of \textit{ein} itself and adjectival inflections in general. In addition to making the different syntactic and morphological assumptions explicit, such an account also needs to capture the different semantics of the various types of \textit{ein}. As of now, I am not aware of the existence of such an alternative account. I return to the parallelism between adjectival \textit{eine} and \textit{andere} ‘other’.
\end{styleFooter}

\begin{styleFooter}
5.2.4. More Evidence for Adjectival \textit{eine}
\end{styleFooter}

\begin{styleFooter}
The claim that \textit{ein} can be an adjective is further strengthened if \textit{eine} is treated as categorially parallel to \textit{andere} ‘other’. Consider (74a), where \textit{eine} and \textit{andere} are in different, coordinated DPs showing similar morpho-syntactic behavior. Now, as briefly discussed in Chapter 2, Section 2.3, there are certain types of adjectives – often called definite adjectives – that can license singular count nouns (74b-b’). The adjective \textit{andere} is different and requires the presence of a determiner (74c-c’):\footnote{\ Another indication of categorial parallelism comes from the fact that adjectival \textit{eine} can also be co-ordinated with \textit{andere}, as in the following idiom:\par (i)\ \ \textit{der }[\textit{ eine oder andere }]\textit{ Mann}\par \ \ the \ \ \ one \ or \ \ \ \ other \ \ \ \ \ man.\textsc{masc}\par \ \ ‘some men’\par This argument must be used with caution though as the relevant interpretation of \textit{eine} is different here; that is, the idiom does not denote just two people.} 
\end{styleFooter}

\begin{styleFooter}
(74)\ \ a.\ \ \textit{Meine ein-e \ \ \ \ Tochter \ }\textit{\textsubscript{\ \ \ \ \ \ \ \ \ \ \ \ }}\textit{kam, \ \ mein-e andere \ \ \ nicht}.
\end{styleFooter}

\begin{styleFooter}
\ \ \ \ my \ \ \ \ \ one-\textsc{wk} daughter.\textsc{fem} came, my \ \ \ \ \ other-\textsc{wk} not
\end{styleFooter}

\begin{styleStandard}
‘One of my daughters came, the other did not.’
\end{styleStandard}

\begin{styleFooter}
b. \ \ \ \textit{Folgendes Beispiel \ \ \ \ \ \ \ \ \ \ illustriert das.}
\end{styleFooter}

\begin{styleFooter}
following \ example.\textsc{neut} illustrate \ this 
\end{styleFooter}

\begin{styleFooter}
‘The following example illustrates this.’
\end{styleFooter}

\begin{styleFooter}
\ \ b’.\ \ \textit{Das folgende \ Beispiel \ \ \ \ \ \ \ \ \ \ \ illustriert das.}
\end{styleFooter}

\begin{styleFooter}
the \ following example.\textsc{neut} illustrate \ this
\end{styleFooter}

\begin{styleFooter}
‘The following example illustrates this.’
\end{styleFooter}

\begin{styleFooter}
c. \ \textit{*\ \ Anderes Beispiel \ \ \ \ \ \ \ \ \ \ illustriert das.}
\end{styleFooter}

\begin{styleFooter}
other \ \ \ \ \ example.\textsc{neut} illustrate \ this 
\end{styleFooter}

\begin{styleFooter}
\ \ c’.\ \ \textit{Das andere Beispiel \ \ \ \ \ \ \ \ \ \ \ illustriert das.}
\end{styleFooter}

\begin{styleFooter}
the \ other \ \ \ \textsubscript{\ }example.\textsc{neut} illustrate \ this
\end{styleFooter}

\begin{styleFooter}
‘The other example illustrates this.’
\end{styleFooter}

\begin{styleFooter}
The need for the presence of a determiner in (74c-c’) is exactly what we saw with adjectival \textit{eine} above. In other words, both adjectival \textit{eine} and \textit{andere} ‘other’ require a determiner. Thus, both of these elements are not definite adjectives. Interestingly, the adjective \textit{andere} can also have the meaning ‘different’. This meaning is subject to a certain context, and this reveals another parallelism of \textit{eine} and \textit{andere} in the meaning of ‘other’.
\end{styleFooter}

\begin{styleFooter}
5.2.5. Different Semantics: \textit{EIN} vs. \textit{eine} and \textit{andere }meaning ‘different’ vs. ‘other’
\end{styleFooter}

\begin{styleFooter}
In this final subsection, I make more remarks on the different semantics of \textit{EIN} and adjectival \textit{eine} and on the different readings of \textit{andere} meaning ‘different’ or ‘other’. As seen above, adjectival \textit{eine} and \textit{andere} ‘other’ are related in that they often co-occur. To set the stage for the discussion of \textit{andere }in its different meanings, I start with \textit{EIN} and adjectival \textit{eine}.
\end{styleFooter}

\begin{styleFooter}
\ \ As discussed above, the numeral \textit{EIN} involves singularity, but adjectival \textit{eine} presupposes duality. Besides this semantic difference, numerals can be modified by \textit{mehr als} ‘more than’ (75a,b,d) whereas adjectival \textit{eine} cannot (75c):
\end{styleFooter}

\begin{styleFooter}
(75)\ \ a.\ \ [\textit{ Mehr als \ EIN Student }]\textit{ \ \ \ \ \ \ \ kam \ \ zur \ \ \ \ Party}.
\end{styleFooter}

\begin{styleFooter}
\ \ \ \  \ more than one student.\textsc{masc} came to.the party
\end{styleFooter}

\begin{styleStandard}
‘More than one student came to the party.’
\end{styleStandard}

\begin{styleFooter}
\ \ b.\ \ [\textit{ Mehr als \ \ HUNDERT \ Studenten }]\textit{ kamen zur \ \ \ \ Party}.
\end{styleFooter}

\begin{styleFooter}
\ \ \ \  \ more than one.hundred students \ \ \ \ \ came \ \ to.the party
\end{styleFooter}

\begin{styleStandard}
‘More than one hundred students came to the party.’
\end{styleStandard}

\begin{styleFooter}
\ \ c. *\ \ \textit{Der }[\textit{ mehr als \ \ eine }]\textit{ Student \ \ \ \ \ \ \ \ \ kam \ \ zur \ \ \ \ Party}.
\end{styleFooter}

\begin{styleFooter}
\ \ \ \ the \ \ \ more than one \ \ \ student.\textsc{masc} came to.the party
\end{styleFooter}

\begin{styleFooter}
\ \ d.\ \ \textit{Die }[\textit{ mehr als \ HUNDERT }]\textit{ Studenten kamen zur \ \ \ \ \ Party}.
\end{styleFooter}

\begin{styleFooter}
\ \ \ \ the \ \ \ more than one.hundred students \ \ came \ \ \ to.the party 
\end{styleFooter}

\begin{styleStandard}
‘The more than one hundred students came to the party.’
\end{styleStandard}

\begin{styleFooter}
Again, adjectival \textit{eine} is an element semantically different from numerals including \textit{EIN}. With this in place, I turn to the different meanings of \textit{andere}.
\end{styleFooter}

\begin{styleFooter}
\ \ As just observed, adjectival \textit{eine} cannot be modified by \textit{mehr als} ‘more than’. Something similar holds for adjectival \textit{eine} when modified by the type/degree particle \textit{so }‘such’. Compare (75a) to (76a) and (75c) to (76b).\footnote{\ Note that (75a) involves the singularity numeral \textit{EIN} but that (76a) contains the article \textit{ein}. This is due to the fact that the scopal element \textit{mehr als} ‘more than’ modifies a numeral in (75a) but that the type/degree particle \textit{so }‘such’ is supported by \textit{ein} in (76a) (see also Chapter 8, Sections 2.2.2).} Interestingly, when \textit{eine} in (76b) is replaced by \textit{andere}, the example becomes grammatical, but \textit{andere} can only be interpreted as ‘different’ (but not as ‘other’) in this context (76c):
\end{styleFooter}

\begin{styleFooter}
(76)\ \ a. \ \ \textit{so (ei)n Mann}
\end{styleFooter}

\begin{styleFooter}
\ \ so \ \textsubscript{\ }a \ \ \ \ man.\textsc{masc}
\end{styleFooter}

\begin{styleFooter}
\ \ ‘such a man/a man like that’
\end{styleFooter}

\begin{styleFooter}
b. \ *\ \ \textit{der so eine Mann}
\end{styleFooter}

\begin{styleFooter}
\ \ the so one \ man.\textsc{masc}
\end{styleFooter}

\begin{styleFooter}
c.\ \ \textit{der so andere \ \ Mann}
\end{styleFooter}

\begin{styleFooter}
\ \ the so different man.\textsc{masc}
\end{styleFooter}

\begin{styleFooter}
\ \ ‘the so different man’
\end{styleFooter}

\begin{styleFooter}
\ \ \#‘the so other man’
\end{styleFooter}

\begin{styleFooter}
Thus, the ungrammaticality of (76b) and the interpretative restriction in (76c) fit well with the discussion above, where I showed that adjectival \textit{eine} often co-occurs with \textit{andere} in the meaning of ‘other’. More generally, this provides additional evidence that this type of \textit{ein}, just like \textit{andere} ‘other’, is a special kind of adjective. 
\end{styleFooter}

\begin{styleFooter}
\ \ To sum up this section, I discussed more evidence that there are two main types of \textit{ein}: the indefinite article and adjectival \textit{eine}. The singularity numeral \textit{EIN} is based, in part, on the indefinite article. Each of these three elements is in a different position: the indefinite article is in Card or D, the singularity numeral, that is, its semantic component, is in Spec,CardP or Spec,DP, and adjectival \textit{eine} is in a high Spec,AgrP position. The adjectival inflections on these different elements support the classification into the two main types. The different inflections follow from the system developed in Chapter 2, again highlighting the fact that adjectival inflections and \textit{ein} can (and should) be discussed in tandem. Before I review some previous work on \textit{ein}, I briefly return to the discussion of \textit{ein}{}-words as composite forms. 
\end{styleFooter}

\begin{styleNewTimesRoman}
\textbf{6.\ \ Diachronic and Cross-linguistic Evidence for }\textbf{\textit{ein}}\textbf{{}-words as Composites}
\end{styleNewTimesRoman}

\begin{styleNewTimesRoman}
As mentioned in Section 4.2.2, there is an interesting asymmetry in the syntactic distribution of Saxon Genitives and possessive articles. Whereas the former can follow their head nouns, the latter cannot:
\end{styleNewTimesRoman}

\begin{styleStandard}
(77) \ \ a.\ \ \textit{das Buch \ \ \ \ \ \ \ \ \ Maximilians}
\end{styleStandard}

\begin{styleStandard}
\ \ \ \ the book.\textsc{neut} Maximilian’s 
\end{styleStandard}

\begin{styleStandard}
\ \ \ \ ‘Maximilian’s book’\ \ 
\end{styleStandard}

\begin{styleStandard}
\ \ b. \ *\ \ \textit{das Buch \ \ \ \ \ \ \ \ \ sein} 
\end{styleStandard}

\begin{styleStandard}
\ \ \ \ the book.\textsc{neut} his
\end{styleStandard}

\begin{styleStandard}
This distribution was used to motivate the composite analysis of possessive articles – since \textit{ein} must precede the head noun, the possessive article must precede it as well. This accounts for the ungrammaticality in (77b). Interestingly, other languages seem to be revealing in this respect documenting that the lower position of the possessive pronominal in (77b) is not ungrammatical in principle. This means that (77b) requires an explanation in German. I consider this in more detail.
\end{styleStandard}

\begin{styleStandard}
Unlike Modern Standard German, OHG allows possessive pronominals to follow their head nouns (78a). Furthermore, note that Norwegian can also have possessive pronominals in postnominal position (78b) ((78a) is from Demske 2001: 173; (78b) is from Julien 2005a: 140):
\end{styleStandard}

\begin{styleNewTimesRoman}
(78)\ \ a.\ \ \textit{(ther) fater \ \ \ \ \ \ \ \ \ \ \ min}\textbf{\textit{ }}\textbf{\ \ \ \ }(OHG)
\end{styleNewTimesRoman}

\begin{styleNewTimesRoman}
\ \ \ \ \textsubscript{ \ }the \ \ father.\textsc{masc} my
\end{styleNewTimesRoman}

\begin{styleNewTimesRoman}
\ \ \ \ ‘my father’
\end{styleNewTimesRoman}

\begin{styleNewTimesRoman}
\ \ b.\ \ \textit{katt-a \ \ \ \ \ \ \ \ \ \ mi}\ \ \ \ \ \ (Norwegian)
\end{styleNewTimesRoman}

\begin{styleNewTimesRoman}
\ \ \ \ cat.\textsc{fem}{}-\textsc{def} my
\end{styleNewTimesRoman}

\begin{styleNewTimesRoman}
\ \ \ \ ‘my cat’
\end{styleNewTimesRoman}

\begin{styleStandard}
I reiterate the proposal that the properties of the possessive articles in Modern Standard German follow from the decompositional analysis discussed above. Specifically, since \textit{ein} itself cannot follow the head noun (79a), other \textit{ein}{}-words cannot either. This was illustrated for the possessive article in (77b) above and is illustrated for the singularity numeral and the negative article in (79b-c) below:
\end{styleStandard}

\begin{styleNewTimesRoman}
(79)\ \ a. \ *\ \ \textit{\{das} / \textit{ein / Ø}\textsubscript{D}\textit{\} Buch \ \ \ \ \ \ \ \ \ ein} 
\end{styleNewTimesRoman}

\begin{styleNewTimesRoman}
\ \textsubscript{\ }the \textsubscript{\ }/ a \ \ \ \ \textsubscript{\ \ \ \ \ \ \ \ \ \ \ \ \ }book.\textsc{neut} a
\end{styleNewTimesRoman}

\begin{styleNewTimesRoman}
\ \ b. \ *\ \ \textit{\{das} / \textit{ein / Ø}\textsubscript{D}\textit{\} Buch \ \ \ \ \ \ \ \ \ EIN} 
\end{styleNewTimesRoman}

\begin{styleNewTimesRoman}
\ \textsubscript{\ }the \textsubscript{\ }/ a \ \ \ \ \textsubscript{\ \ \ \ \ \ \ \ \ \ \ \ \ }book.\textsc{neut} one
\end{styleNewTimesRoman}

\begin{styleNewTimesRoman}
\ \ c. \ *\ \ \textit{\{das} / \textit{ein / Ø}\textsubscript{D}\textit{\} Buch \ \ \ \ \ \ \ \ \ \ kein} 
\end{styleNewTimesRoman}

\begin{styleNewTimesRoman}
\ \textsubscript{\ }the \textsubscript{\ }/ a \ \ \ \ \textsubscript{\ \ \ \ \ \ \ \ \ \ \ \ \ }book.\textsc{neut} no
\end{styleNewTimesRoman}

\begin{styleNewTimesRoman}
Now, while (79b-c) are also out for independent reasons (i.e., the high base positions of \textit{Ø}\textsubscript{[-PL]} and NEG), the ungrammaticality of (79a) suggests a straightforward account of the absence of postnominal possessive pronominals in Modern Standard German as in (77b). 
\end{styleNewTimesRoman}

\begin{styleNewTimesRoman}
In more detail, as seen in (77a) and (78) above, possessives can be base-generated in a low position. As such, possessive pronominals in Modern Standard German could also be expected to be grammatical in such a low position. This, however, is not the case. Given the composite analysis, note though that the possessive component of the article (e.g., \textit{s}{}- in (77b)) cannot be supported by \textit{ein }in Modern German if the possessive stays in situ – as indicated by (79a), \textit{ein} itself cannot occur in such a low position since \textit{ein}, like all determiners, is base-generated in ArtP, a position higher than the (surface) position of the noun. Assuming that the possessive pronominals in OHG and Norwegian are not composite forms involving an indefinite article, they can surface in postnominal position.\footnote{\ Interestingly, prenominal possessives can also occur lower in the structure in OHG, namely between the determiner and the head noun (see also Footnote Error: Reference source not found; (i) is taken from Demske 2001: 227):\par (i)\ \ \textit{in dhemu heilegin daniheles chiscribe}\ \ (OHG)\par in the \ \ \ \ \ \textsubscript{\ }holy \ \ \ \ \ \ Daniel’s \ \ scripture.\textsc{neut}\par ‘in Daniel’s holy scriptures’\par Again, this indicates that possessives have a different analysis in the older varieties of German.} This diachronic and cross-linguistic difference lends more credence to the proposal that possessive articles in (Modern) Standard German are composite forms.\footnote{\ Note that dialectal forms of the possessive article (e.g., Northern German \textit{min} ‘my’) may also involve a decompositional analysis. Their spell-out forms, however, may involve more than simple Local Dislocation to support the possessive element.} I briefly sketch the diachronic path of the three \textit{ein}{}-words involving composites.
\end{styleNewTimesRoman}

\begin{styleFootnote}
In Section 5.1.1, I tentatively suggested that OHG \textit{éin} ‘one’ was split into two components in the development of the language (80a) leading to the emergence of the indefinite article. With this in mind, I suggest that the other \textit{ein}{}-words were also split into two parts over time. Specifically, possessive pronominals were split into a possessive component and vacuous \textit{ein} (80b).\footnote{\ This process is possibly tied to ENHG Diphthongization, which changed [i:] to [aI]. Furthermore, the split of the possessive pronoun could have come about by analogy with the splits of the singularity numeral and/or the negative article.} As for the negative article, judging from the diachronic development of \textit{kein }‘no’\textit{ }described in Paul \textit{et al}. (1989: 235), it appears as if the negative article consisted of two components early on (80c). Importantly though, the inner makeup seems to have changed from the combination of a negative element with the singularity numeral to a negative element with the indefinite article:
\end{styleFootnote}

\begin{styleFootnote}
(80)\ \ a.\ \ \textit{éin}\ \ →\ \ \textit{Ø}\textsubscript{[-PL]}+\textit{ein}
\end{styleFootnote}

\begin{styleFootnote}
\ \ \ \ one\ \ \ \ \textit{Ø}\textsubscript{[-PL]}+a
\end{styleFootnote}

\begin{styleFootnote}
b.\ \ \textit{m\=\in}\ \ →\ \ \textit{m \ \ \ }\textit{\textsubscript{\ }}+\textit{ein}
\end{styleFootnote}

\begin{styleFootnote}
\ \ \ \ my\ \ \ \ \textsc{poss}+a
\end{styleFootnote}

\begin{styleFootnote}
\ \ c.\ \ \textit{ne(c)h+ein} \ \ \ \ → \ \ \ \ \textit{nekein \ \ \ \ }→ \ \ \ \ \textit{k \ \ \ }+\textit{ein}
\end{styleFootnote}

\begin{styleFootnote}
\ \ \ \ not \ \ \ \ \textsubscript{\ }+one\ \ \ \  \ \ \  \ \ \ \ \ \ \ \textsc{neg}+a
\end{styleFootnote}

\begin{styleFootnote}
To sum up, unlike vacuous \textit{ein}, the other Modern German \textit{ein}{}-words are composite forms. In contrast, the counterparts of the latter elements in OHG (and Norwegian) are not. Note that this difference in decompositionality is also compatible with the change in inflection that adjectives went through in the history of German; for instance, recall from Chapter 2, Section 1.2 that possessive pronominals in OHG (and Norwegian) take adjectives with weak inflections but that possessive articles in Modern German take adjectives with mixed inflections. Thus, while the old possessive pronominals pattern with the definite article, the Modern German counterparts are related to \textit{ein}. It is clear that German has undergone some changes in this empirical domain. In other words, OHG and Modern German must have different accounts of these phenomena.
\end{styleFootnote}

\begin{styleFootnote}
In the next section, I turn to some previous analyses examining one proposal in more detail. We will see that unlike adjectival endings discussed in Chapter 2, there are much fewer contributions investigating the different kinds of \textit{ein}.
\end{styleFootnote}

\begin{styleStandard}\bfseries
7.\ \ A Brief Critique of a Previous Proposal\newline

\end{styleStandard}

\begin{styleStandard}
In the descriptive literature, the indefinite and the definite article are usually discussed together, with the former taken to be the indefinite counterpart of the latter (for German, see Duden 1995: 303-21). Recognizing certain shortcomings of this juxtaposition, a number of other views have emerged. For example, Perlmutter (1970) derives English \textit{a(n)} as an unstressed version of the numeral \textit{one}, Higginbotham (1987: 47) argues that the indefinite article in predicate nominals is an adjective meaning ‘one’, and Ackles (1996) argues that \textit{a(n)} marks the presence of NumP with singular count nouns.\footnote{\ For the discussion of English \textit{a(n)} in quantifying expressions (e.g., \textit{a lot}, \textit{a few}), see Klockmann (2020). The author tentatively extends the discussion to ordinary noun phrases like \textit{a book} claiming that \textit{a(n)} is not an indefinite element – it is featureless but involves countability. This is partially different from the current discussion of German. Here, \textit{ein} is also claimed not to be related to indefiniteness, but \textit{ein} is proposed to have a feature (i.e., [+D]), and it is held not to be responsible for countabilily.} Elaborating on work by Oomen (1977), Vater (1982, 1984, 2002) proposes that there is no indefinite article at all but only a numeral/quantifier (“Quantor”). He mainly discusses German but his empirical coverage is meant to be wider.
\end{styleStandard}

\begin{styleStandard}
\ \ It is interesting to point out that all proposals just mentioned make different claims. No consensus has emerged. Recall now that I showed in Chapter 1, Section 2.1.3 that German, Nowegian, and Yiddish seem to differ in some of the empirical details. Similar to adjectival inflections in Chapter 2, I have proposed that it is these – on the surface minor – details that show the true nature of the indefinite article in German. In what follows, I review only Vater’s proposal, which explicitly discusses German. As far as I am aware, this proposal seems to have received fairly wide acceptance, at least for German. To give just one example, although providing a critique of some points of Vater (1982, 1984), Bisle-Müller (1991: 100-16) reaches a similar conclusion.
\end{styleStandard}

\begin{styleStandard}
\ \ In a series of papers (1982, 1984, 2002), Vater argues against the traditional opposition of the indefinite vis-à-vis the definite article (I refer only to Vater 1982 as that contains the main relevant insights). He proposes that \textit{ein} is not the indefinite counterpart of definite \textit{der} ‘the’. According to Vater, this element is not indefinite as \textit{ein} can lead not only to a non-specific but also to a specific interpretation of the containing noun phrase (for examples, see Section 4.1.3 above). In addition, it denotes a specific amount, namely singularity. Furthermore, this element is not an article as \textit{ein} is not necessarily “localizing” in function in the sense of Hawkins (1978). 
\end{styleStandard}

\begin{styleStandard}
Rather, Vater (1982) proposes that \textit{ein} is a cardinal numeral, that is, a type of quantifier that denotes a specific number of entities (also Oomen 1977). Generalizing his discussion, he claims that determiners or articles only involve definite elements. In contrast, the other determiner elements belong to a different part of speech – numerals/quantifiers. He provides some syntactic and semantic arguments for this claim. After illustrating some of these arguments, I return to them showcasing some shortcomings. 
\end{styleStandard}

\begin{styleStandard}
\ \ Starting with some syntactic arguments, Vater (1982: 71) points out that like other quantifiers, \textit{ein} can undergo quantifier float (note that the translations of the examples here and below are my own):
\end{styleStandard}

\begin{styleStandard}
(81)\ \ a.\ \ \textit{Antrag \ \ \ \ \ \ }\textit{\textsubscript{\ \ \ \ \ \ \ \ \ \ \ \ \ \ \ \ }}\textit{habe ich keinen gestellt}.
\end{styleStandard}

\begin{styleStandard}
\ \ \ \ application.\textsc{masc} have I \ \ \ \textsubscript{\ }no \ \ \ \ \ \ \ submitted
\end{styleStandard}

\begin{styleStandard}
\ \ \ \ ‘As for applications, I have submitted none.’
\end{styleStandard}

\begin{styleStandard}
\ \ b.\ \ \textit{Antrag \ \ \ \ \ \ \ \ \ \ \ \ \ \ \ \ }\textit{\textsubscript{\ }}\textit{habe ich einen gestellt}.
\end{styleStandard}

\begin{styleStandard}
\ \ \ \ application.\textsc{masc} have I \ \ \ \textsubscript{\ }one \ \ \ submitted
\end{styleStandard}

\begin{styleStandard}
\ \ \ \ ‘As for applications, I have submitted one.’
\end{styleStandard}

\begin{styleStandard}
Second, like other numerals (82a), \textit{ein} can co-occur with a determiner (82b):
\end{styleStandard}

\begin{styleStandard}
(82)\ \ a.\ \ \textit{die zwei Bücher}
\end{styleStandard}

\begin{styleStandard}
\ \ \ \ the two \ books
\end{styleStandard}

\begin{styleStandard}
\ \ \ \ ‘the two books’
\end{styleStandard}

\begin{styleStandard}
\ \ b.\ \ \textit{das eine Buch}
\end{styleStandard}

\begin{styleStandard}
\ \ \ \ the \textsubscript{\ }one \ book.\textsc{neut}
\end{styleStandard}

\begin{styleStandard}
\ \ \ \ ‘one of the two books’
\end{styleStandard}

\begin{styleStandard}
Turning to some semantic arguments, Vater (1982: 71-72) argues that like other numerals, \textit{ein} can individuate mass nouns. Glossing over some of the details, the interpretation in (83a) can involve certain types of bread or certain amounts of bread. Modulo the singular, the example in (83b) has a similar range of readings. Importantly, the definite article by itself does not have this individuating function. Note though that the example in (83c) is still ambiguous, here between a mass and a count reading:
\end{styleStandard}

\begin{styleStandard}
(83)\ \ a.\ \ \textit{zwei Brote}
\end{styleStandard}

\begin{styleStandard}
\ \ \ \ two \ breads
\end{styleStandard}

\begin{styleStandard}
\ \ \ \ ‘two types of bread’
\end{styleStandard}

\begin{styleStandard}
‘two loaves of bread’
\end{styleStandard}

\begin{styleStandard}
\ \ b.\ \ \textit{ein \ Brot}
\end{styleStandard}

\begin{styleStandard}
\ \ \ \ one bread.\textsc{neut}
\end{styleStandard}

\begin{styleStandard}
\ \ \ \ ‘one type of bread’
\end{styleStandard}

\begin{styleStandard}
‘one loaf of bread’
\end{styleStandard}

\begin{styleStandard}
\ \ c.\ \ \textit{das Brot}
\end{styleStandard}

\begin{styleStandard}
\ \ \ \ \ the bread.\textsc{neut}
\end{styleStandard}

\begin{styleStandard}
\ \ \ \ ‘the (substance of) bread’
\end{styleStandard}

\begin{styleStandard}
‘the loaf of bread’
\end{styleStandard}

\begin{styleStandard}
In order to derive the count reading in (83c), Vater follows certain aspects of Perlmutter (1970). Specifically, Vater (1982: 71-72, 1984: 39) proposes that the mass interpretation in (83c) only involves a definite article (84a). In contrast, the definite singular count reading in (83c) involves \textit{ein}, which is deleted (84b). Interestingly, this \textit{ein} can also surface. In this scenario, the interpretation is, according to Vater, partitive. This is indicated by the translation in (84c). Below, I return to the word put in square brackets:
\end{styleStandard}

\begin{styleStandard}
(84)\ \ a.\ \ \textit{das Brot}
\end{styleStandard}

\begin{styleStandard}
\ \ \ \ the bread.\textsc{neut}
\end{styleStandard}

\begin{styleStandard}
\ \ \ \ ‘the (substance of) bread’
\end{styleStandard}

\begin{styleStandard}
\ \ b.\ \ \textit{das eine Brot}
\end{styleStandard}

\begin{styleStandard}
\ \ \ \ the one \ bread.\textsc{neut}
\end{styleStandard}

\begin{styleStandard}
\ \ \ \ ‘the loaf of bread’
\end{styleStandard}

\begin{styleStandard}
\ \ c.\ \ \textit{das eine Brot}
\end{styleStandard}

\begin{styleStandard}
\ \ \ \ the \textsubscript{\ }one \ bread.\textsc{neut}
\end{styleStandard}

\begin{styleStandard}
\ \ \ \ ‘one of the [two] breads’
\end{styleStandard}

\begin{styleStandard}
After this brief illustration, I return to the above arguments pointing out some shortcomings.
\end{styleStandard}

\begin{styleStandard}
\ \ Note that the first syntactic argument does not involve quantifier float but rather a different type of discontinuous noun phrase. In Chapter 4, Section 3, I labeled this construction split topicalization. One argument against Vater’s view is that under certain conditions, this type of split does allow (other) determiners to be stranded: 
\end{styleStandard}

\begin{styleStandard}
(85)\ \ \textit{Hemden habe ich immer \ nur \ }\textit{\textsubscript{\ }}\textit{diese da \ \ \ \ getragen}
\end{styleStandard}

\begin{styleStandard}
\ \ shirts \ \ \ \ \ have I \ \ \ \textsubscript{\ }always only these there worn
\end{styleStandard}

\begin{styleStandard}
\ \ ‘As for shirts, I have always worn only these there.’
\end{styleStandard}

\begin{styleStandard}
Given that either \textit{ein} or \textit{diese} ‘these’ can occur in a source DP, we can maintain the claim that \textit{ein} is a determiner.
\end{styleStandard}

\begin{styleStandard}
Turning to the second syntactic argument, it is true that determiners and quantifiers can co-occur. However, there are also cases where two determiner elements can be combined (86a). The same holds for two quantifiers on Vater’s assumptions or, alternatively, two determiner elements on my assumptions (86b):
\end{styleStandard}

\begin{styleStandard}
(86)\ \ a.\ \ \textit{diese meine Freunde}
\end{styleStandard}

\begin{styleStandard}
\ \ \ \ these my \ \ \ \ friends
\end{styleStandard}

\begin{styleStandard}
\ \ \ \ ‘these friends of mine’
\end{styleStandard}

\begin{styleStandard}
b.\ \ \textit{ein jeder von uns}
\end{styleStandard}

\begin{styleStandard}
\ \ an \ every of \ \ us
\end{styleStandard}

\begin{styleStandard}
\ \ \ \ ‘each of us’
\end{styleStandard}

\begin{styleStandard}
Again, given the possibility of co-occurrence, I continue claiming that \textit{ein} is a determiner(-like) element when it occurs with another determiner as in (86b) (specifically, this type of \textit{ein} is a predeterminer, see also Footnote Error: Reference source not found). As for the semantic arguments, I agree that \textit{ein} seems to be individuating in nature. However, I suggest in Chapter 8, Section 2.2.3 that \textit{ein} is not responsible for this effect, but rather it flags the presence of an operator. Finally, I discuss the partitive interpretation in more detail.
\end{styleStandard}

\begin{styleStandard}
\ \ Vater (1982: 71) claims that \textit{ein} preceded by a definite article can be paraphrased as \textit{ein} followed by a genitive noun phrase in the plural; that is, \textit{ein} is taken to be the singularity numeral. Consider (87a-b). Vater (1982: 72) explicitly states that \textit{ein} here presupposes a larger set of elements. However, this claim is not quite accurate. Rather, while the paraphrase does indeed imply a larger set, \textit{ein} preceded by the definite article as in (87a) typically presupposes just two elements. Furthermore, as mentioned above, this duality presupposition can be cancelled when a demonstrative replaces the definite article (87b). Importantly, the partitive reading in the paraphrase is not cancelled when the demonstrative replaces the definite article there (note again that the translations are mine):
\end{styleStandard}

\begin{styleStandard}
(87)\ \ a.\ \ \textit{das eine Buch} \ \ \ \ \ \ \ \ \ – \textit{eins der \ \ \ \ Bücher }
\end{styleStandard}

\begin{styleStandard}
\ \ \ \ the \textsubscript{\ }one \textsubscript{\ }book.\textsc{neut} – one \ of.the books
\end{styleStandard}

\begin{styleStandard}
\ \ \ \ ‘one of the two books – one of the books’
\end{styleStandard}

\begin{styleStandard}
\ \ b.\ \ \textit{dieses eine Buch \ \ \ \ \ \ \ \ \ – eins dieser \ \ \ Bücher} 
\end{styleStandard}

\begin{styleStandard}
\ \ \ \ this \ \ \ \ one \textsubscript{\ }book.\textsc{neut} – one \ of.these books
\end{styleStandard}

\begin{styleStandard}
\ \ \ \ ‘this one book – one of these books’
\end{styleStandard}

\begin{styleStandard}
I take this to mean that the paraphrases do not capture the semantics of this \textit{ein }accurately. In other words, \textit{ein} is not the singularity numeral here. Furthermore, it appears that other numerals do not invoke this partitivity either; that is, \textit{die zwei Bücher }‘the two books’ does not mean ‘two of the books’. I conclude that the \textit{ein} in (87) is of a different type. Above, I proposed that this \textit{ein} is adjectival, an element only licensed in definite contexts. Besides these issues, it is also worth pointing out that a number of other, important types of \textit{ein} are not discussed in Vater. This shows the limited empirical coverage of that proposal.
\end{styleStandard}

\begin{styleStandard}
\ \ Predicate nominals and cases in the plural are not mentioned in Vater’s discussion at all:
\end{styleStandard}

\begin{styleStandard}
(88)\ \ a.\ \ \textit{Er ist (ein) Lehrer} 
\end{styleStandard}

\begin{styleStandard}
he is \ \ \textsubscript{\ }a \ \ \ \ teacher.\textsc{masc}
\end{styleStandard}

\begin{styleStandard}
‘He is a teacher.’
\end{styleStandard}

\begin{styleStandard}
\ \ b. \ \%\ \ \textit{So \ \ \ ein-e Idioten!} 
\end{styleStandard}

\begin{styleStandard}
\ \ \ \ \ such a-\textsc{pl} \ idiots
\end{styleStandard}

\begin{styleStandard}
\ \ \ \ \ ‘Such Idiots!’
\end{styleStandard}

\begin{styleStandard}
It is quite clear that the noun phrases in (88a) and (88b) have nothing to do with singularity: (88a) denotes a property, and (88b) involves a plurality. 
\end{styleStandard}

\begin{styleStandard}
Furthermore, note also that Vater does not discuss reduced forms. However, as can be seen in (89), the form of \textit{ein}, reduced or unreduced, \textit{can} make an important difference with regard to the grammaticality judgments and with regard to the interpretations (for (89c), see Pafel 1994: 251):
\end{styleStandard}

\begin{styleStandard}
(89)\ \ a.\ \ \textit{Geben Sie }\textit{\textsubscript{\ }}\textit{mir (ei)nen Apfel!} 
\end{styleStandard}

\begin{styleStandard}
give \ \ \ you me \ \ an \ \ \ \ \ \ apple.\textsc{masc}
\end{styleStandard}

\begin{styleStandard}
‘Give me an apple!’
\end{styleStandard}

\begin{styleStandard}
b.\ \ \textit{Geben Sie }\textit{\textsubscript{\ }}\textit{mir *(EI)NEN Apfel!} 
\end{styleStandard}

\begin{styleStandard}
give \ \ \ you me \ \ \ \ one \ \ \ \ \ \ \ apple.\textsc{masc}
\end{styleStandard}

\begin{styleStandard}
‘Give me one apple!’
\end{styleStandard}

\begin{styleStandard}
\ \ c.\ \ \textit{Geben Sie }\textit{\textsubscript{\ }}\textit{mir *(ei)nen!} 
\end{styleStandard}

\begin{styleStandard}
give \ \ \ you me \ \ \ \ one
\end{styleStandard}

\begin{styleStandard}
‘Give me one!’
\end{styleStandard}

\begin{styleStandard}
The distinction of reduced and unreduced forms has played an important role in categorizing the different kinds of \textit{ein} in the current account. (89a) involves the indefinite article \textit{ein}, and (89b) shows the singularity numeral \textit{EIN}. As for (89c), note that \textit{ein} is used pronominally here. I reiterate the idea that the reduced form of \textit{ein} is ungrammatical as pronominal forms of \textit{ein} involve some stress, which prevents the reduced forms from occurring.
\end{styleStandard}

\begin{styleStandard}
Finally, Vater treats the negator \textit{kein} ‘no’ as an unanalyzed form, and possessives such as \textit{mein} ‘my’ are not discussed at all. As documented above, \textit{ein} and the other \textit{ein}{}-words exhibit a number of morpho-syntactic similarities; for instances, as regards their inflections. Since inflections are not addressed by Vater, an important morpho-syntactic generalization over the different \textit{ein}{}-words is being missed here. In my view, adjectival inflections and \textit{ein}{}-words should be discussed in tandem. In this respect, I point out again that adjectival inflections provide important clues about the lexical categories of items in the noun phrase – they only surface on determiners, certain quantifiers, and adjectives – and they provide indications of the various structures of the noun phrase as a whole.
\end{styleStandard}

\begin{styleStandard}
\ \ To sum up, I agree with Vater (and others) that \textit{ein} is not an indefinite article (although I keep the name for convenience). I disagree with him (and others) that \textit{ein} is the singularity numeral. Rather, I claim that there are two types of \textit{ein}: one is a semantically vacuous element, and the other is adjectival \textit{eine}, which has semantics.
\end{styleStandard}

\begin{styleStandard}\bfseries
8.\ \ Conclusion 
\end{styleStandard}

\begin{styleStandard}
One goal of this chapter was to provide a more comprehensive survey of the different types of \textit{ein}. Arguing that \textit{ein} cannot involve the singularity numeral only, I examined three types: the article, the numeral, and the adjective. In order to capture the morphological similarities and the non-co-occurrence of the article and the numeral, I proposed that the numeral is a composite form – it involves the null element \textit{Ø}\textsubscript{[-PL]}, and the latter is supported by the article. Something similar holds for the possessive and negative articles. Adjectival \textit{eine} was proposed to be an independent lexical element. In order to account for the differences, I suggested that the three elements have different feature specifications and are in different positions in the syntactic tree. Table 2 below summarizes these differences:
\end{styleStandard}

\begin{styleFooter}
Table 2: Summary of the Properties of the Types of \textit{ein}
\end{styleFooter}

\begin{flushleft}
\begin{tabular}{|m{0.8087598in}m{1.5462599in}|m{0.9212598in}|m{1.2337599in}|m{1.6712599in}|}

\hline
\multicolumn{2}{|m{2.43376in}|}{Kinds of \textit{ein}} &
\centering Morphology &
\centering Semantics &
\centering\arraybslash Position\\\hline
\multicolumn{1}{|m{0.8087598in}|}{Article \textit{ein}} &
Vacuous (indefinite, possessive, negative, predicative) &
\centering $\alpha $PL morph &
 &
\centering\arraybslash Card or D (depending on the reading, weak vs. strong)\textsuperscript{a}\\\hline
 &
Numeral \textit{Ø}\textsubscript{[-PL]} &
\centering ($\alpha $PL morph = $\alpha $PL sem) &
\centering {}-PL sem &
\centering\arraybslash Spec,CardP or Spec,DP\\\hline
\multicolumn{2}{|m{2.43376in}|}{Adjectival \textit{eine}} &
\centering $\alpha $PL morph &
\centering n = 2 (unless in certain contexts) &
\centering\arraybslash (high) Spec,AgrP\\\hline
\end{tabular}
\end{flushleft}
\begin{styleStandard}
\textsuperscript{a} In Chapter 6, I propose that predicative \textit{ein} can also occur in Art (the head of ArtP).
\end{styleStandard}

\begin{styleStandard}
It is clear that the conditions under which adjectival \textit{eine} induces and loses its duality presupposition require more investigation, something I will no pursue here.
\end{styleStandard}

\begin{styleStandard}
\ \ The second goal was to identify more general properties of \textit{ein}. Similar to adjectival inflections in Chapter 2, I concluded here that \textit{ein} is also semantically vacuous (Hypothesis 1a). In the next two chapters, I turn to more properties of \textit{ein} by discussing some consequences of the current analysis. In particular, I discuss \textit{ein} in relation to semantic concepts such as emotiveness and number. This includes the discussion of \textit{ein} in predicative contexts. We will find confirmation of the conclusions reached thus far. Furthermore, we will see more evidence that \textit{ein} can also indicate a certain amount of structure in the noun phrase and that it can flag the presence of covert operators (Hypothesis 3).
\end{styleStandard}

\clearpage\setcounter{page}{258}\begin{styleJBHeadingii}
\textup{Chapter 6: }Ein\textup{ and Emotiveness}
\end{styleJBHeadingii}

\begin{styleStandard}\bfseries
1. \ \ Introduction
\end{styleStandard}

\begin{styleStandard}
In Chapter 5, I argued that \textit{ein} is semantically vacuous (Hypothesis 1a). I now turn to two consequences. In this and the next chapter, I discuss \textit{ein} with regard to emotiveness and number. These next two chapters are also meant to be broader in their empirical and theoretical range; that is, they discuss \textit{ein} in the context of both the noun phrase (DP) and the clause (TP). Specifically, I provide a more detailed survey of two (partially known) restrictions. Focusing on the contribution of \textit{ein}, I identify three types of readings in this chapter where the notion of emotiveness is of importance in singular, out-of-the-blue contexts. In the next chapter, I isolate restrictions with regard to both morphological and semantic number such that singular contexts are also more restricted than plural environments. Consistent with the previous discussion, I reach the conclusion that \textit{ein} does not make a semantic contribution. To contextualize the following discussion, I begin with some general remarks. 
\end{styleStandard}

\begin{styleStandard}\itshape
1.1.\ \ General Structural Similarities Between the DP and TP
\end{styleStandard}

\begin{styleStandard}
It was pointed out in Chapter 1 that it is generally assumed that clauses and noun phrases are parallel in meaning and structure (e.g., Abney 1987, Chomsky 1970, Iordăchioaia 2020). To repeat the relevant data, consider the following juxtaposition: 
\end{styleStandard}

\begin{styleStandard}
(1) \ \ a.\ \ The Romans destroyed the city.
\end{styleStandard}

\begin{styleStandard}
\ \ b.\ \ the Romans’ destruction of the city 
\end{styleStandard}

\begin{styleStandard}
In view of the similar interpretations of the arguments in (1), the structural parallelism in (2) has come to be established:\footnote{\ As mentioned in Chapter 1, this alignment is not uncontroversial. Recall that I take DP to be parallel to TP. In this and the next chapter, I use Type Theory (Chapter 1, Section 4.2.2) to highlight certain similarities and differences in the DP and TP. Note in this regard that Heim \& Kratzer (1998) take (root) clauses to be of category S. As is clear from their discussion of relative clauses (their chapter 5), S is under CP. Given the alignment in (2), I take S to be TP here.}
\end{styleStandard}

\begin{styleStandard}
(2) \ \ a.\ \ CP\ \ TP\ \ AgrP\ \  \ \ \ \ VP
\end{styleStandard}

\begin{styleStandard}
\ \ b.\ \ PP\ \ DP\ \ NumP\ \  \ \ \ \ NP
\end{styleStandard}

\begin{styleStandard}
The investigation of this parallel has been very fruitful in its empirical discoveries and theoretical innovations (for a survey, see Alexiadou \textit{et al}. 2007). This and the next chapter intend to add to this body of work by focusing on certain combinations of “pronoun + (copular verb) + noun” in simple DPs and copular TPs. Unlike the cases in (1) above, I investigate noun phrases and copular sentences that involve non-theta nouns; for instance, the noun Idiot ‘idiot’ in (3) and (4) below. 
\end{styleStandard}

\begin{styleStandard}
\ \ Observe first that the DPs and TPs under discussion here involve a pronoun and a noun (3a-b). The difference is that the TPs also contain an auxiliary:
\end{styleStandard}

\begin{styleStandard}
(3)\ \ a.\ \ \textit{wir Idioten}
\end{styleStandard}

\begin{styleStandard}
\ \ \ \ we idiots
\end{styleStandard}

\begin{styleStandard}
\ \ \ \ ‘us idiots’
\end{styleStandard}

\begin{styleStandard}
\ \ b.\ \ \textit{Wir sind Idioten}.
\end{styleStandard}

\begin{styleStandard}
\ \ \ \ we are \ \ idiots
\end{styleStandard}

\begin{styleStandard}
\ \ \ \ ‘We are idiots.’
\end{styleStandard}

\begin{styleStandard}
Unlike in the plural above, \textit{ein} is added in TPs involving the singular:\footnote{\ Recall that I put the nominal complement of singular pronominal determiners in parentheses in the English translations.}
\end{styleStandard}

\begin{styleStandard}
(4)\ \ a.\ \ \textit{du \ \ Idiot}
\end{styleStandard}

\begin{styleStandard}
\ \ \ \ you idiot.\textsc{masc}
\end{styleStandard}

\begin{styleStandard}
\ \ \ \ ‘you (idiot)’
\end{styleStandard}

\begin{styleStandard}
\ \ b.\ \ \textit{Du \ bist (ei)n Idiot}.
\end{styleStandard}

\begin{styleStandard}
\ \ \ \ you are \ \ an \ \ idiot.\textsc{masc}
\end{styleStandard}

\begin{styleStandard}
\ \ \ \ ‘You are an idiot.’
\end{styleStandard}

\begin{styleStandard}
Considering (3) and (4), we can observe that these constructions are fairly similar to each other and involve comparatively little complexity. Specifically, taking the auxiliary to be semantically vacuous (Coppock \& Beaver 2015: 429; Heim \& Kratzer 1998: 61-62), I assume that this verbal element simply indicates the presence of more structure, for instance, the TP-layer. As we will see below, \textit{ein} is similar to the auxiliary in this regard. Focusing on these types of constructions affords us a fairly simple and direct comparison of the workings of \textit{ein} in the nominal and clausal domains. Note also that some of the cases discussed below involve non-canonical constructions.
\end{styleStandard}

\begin{styleStandard}
\ \ Overall, I reach the conclusion that \textit{ein} makes no semantic contribution in these constructions either and that the nominal and clausal combinations of “pronoun + (copular verb) + noun” are quite similar albeit not entirely identical. The few differences between the two domains follow from certain assumptions about the different pragmatics involved (Rauh 2004), lexical differences between certain overt and covert elements (\textit{als} ‘as’ vs. ALS), and the obligatoriness of NumP in the DP but its syntactic optionality in the predicate nominal of the TP. If this discussion is on the right track, then I provide additional evidence that the nominal and clausal domains are essentially parallel. With these general remarks in mind, I begin the investigation. Postponing the discussion of number to the next chapter, I start by examining \textit{ein} in emotive contexts.
\end{styleStandard}

\begin{styleStandard}\itshape
1.2. \ \ \ General Interpretatory Differences Between the DP and TP
\end{styleStandard}

\begin{styleStandard}
In this chapter, I compare the structural and interpretatory differences of constructions where pronouns combine with bare nouns (5) or with nouns preceded by overt elements like \textit{als} ‘as’ or \textit{ein} ‘a’ (6). To avoid certain lexical and structural ambiguities of \textit{ein} (Chapter 5), the indefinite article is provided in its reduced and unstressable form \textit{‘n}. Interestingly, nouns such as \textit{Bauer} are ambiguous in that they have a neutral/literal meaning (‘farmer’) and an emotive/figurative one (‘peasant’) (e.g., Harbert 2007: 147, de Swart \textit{et al}. 2007, and references below). However, this ambiguity disappears in these different types of DP and TP in interesting ways. Specifically, bare nouns have an emotive/figurative meaning in the DP (5a) but a neutral/literal meaning in the TP (5b). In contrast, nouns preceded by \textit{als} have a neutral/literal meaning in the DP (6a), but nouns preceded by \textit{ein} have a (predominantly) emotive/figurative meaning in the TP (6b) (\# indicates that the interpretation is not available; \% means that the interpretation is possible but somewhat less prevalent or for some speakers not available at all):\footnote{\ A brief note on the data in (6) is in order here. The string in (6a) can felicitously be used in a sentence like \textit{Was würdest du als Bauer dazu sagen?} ‘What would you as a farmer say about this?’. As to (6b), Harbert (2007: 147-48) states that the indefinite article is not possible with role nouns in predicative contexts in German, a statement also often found in textbooks of German. As we see here and below, this statement is too strong. Indeed, the use of \textit{ein} with terms of professions is already attested in ENHG (Ebert \textit{et al}. 1993: 316). \par Cross-linguistically, predicate nouns in English typically have an indefinite article in cases like (5b) yielding (6b), but their Romance counterparts do not. Zamparelli’s (2008) explanation that this is due to lack of morphological gender in English is probably not the whole story as Yiddish, which does have morphological gender, patterns with English (see Lockwood 1995: 112, also\textbf{ }Harbert 2007: 148). For the discussion of morphological case on predicate nominals, see Maling \& Sprouse (1995).}
\end{styleStandard}

\begin{styleStandard}
(5) \ \ a.\ \ \textit{du \ \ Bauer}
\end{styleStandard}

\begin{styleStandard}
\ \ \ \ you peasant.\textsc{masc}
\end{styleStandard}

\begin{styleStandard}
\ \ \ \ ‘you (peasant)’
\end{styleStandard}

\begin{styleStandard}
\ \ \ \ \#‘you (farmer)’
\end{styleStandard}

\begin{styleStandard}
\ \ b.\ \ \textit{Du \ bist Bauer.}
\end{styleStandard}

\begin{styleStandard}
\ \ \ \ you are \ farmer.\textsc{masc}
\end{styleStandard}

\begin{styleStandard}
\ \ \ \ ‘You are a farmer.’
\end{styleStandard}

\begin{styleStandard}
\ \ \ \ \#‘You are a peasant.’
\end{styleStandard}

\begin{styleStandard}
(6)\ \ a.\ \ \textit{du \ \ als Bauer}
\end{styleStandard}

\begin{styleStandard}
\ \ \ \ you as \ farmer.\textsc{masc}
\end{styleStandard}

\begin{styleStandard}
\ \ \ \ ‘you as a farmer’
\end{styleStandard}

\begin{styleStandard}
\ \ \ \ \#‘you as a peasant’
\end{styleStandard}

\begin{styleStandard}
\ \ b.\ \ \textit{Du \ bist ‘n Bauer.}
\end{styleStandard}

\begin{styleStandard}
\ \ \ \ you are \ \ a peasant/farmer.\textsc{masc}
\end{styleStandard}

\begin{styleStandard}
\ \ \ \ ‘You are a peasant.’
\end{styleStandard}

\begin{styleStandard}
\ \ \ \ \%‘You are a farmer.’
\end{styleStandard}

\begin{styleStandard}
Comparing the (a)-examples to the (b)-examples, we find a near complementary distribution of the two readings of the noun. Indeed, comparing the (a)-examples to each other and the (b)-examples to each other, this difference in interpretation correlates with the presence or absence of \textit{als} ‘as’ and \textit{ein} ‘a’. Focusing on the latter, this means that \textit{ein} seems to have, at least in some contexts, semantic import. Given the proposal that \textit{ein} is semantically vacuous, this is unexpected and in need of an explanation. 
\end{styleStandard}

\begin{styleStandard}
\ \ To anticipate the discussion, I follow Rauh (2004) for (5a) and (6a) in arguing that the pragmatics involved here need to be taken into consideration to explain this difference. As to (5b) and (6b), it could be claimed that \textit{ein} brings about the semantic difference. However, I suggest that the added emotiveness in (6b) stems from the interaction between an operator, specifically the realization operator REL (de Swart \textit{et al}. 2007), and nouns lexically specified as [+figurative]. I propose that on the one hand, \textit{ein} indicates the presence of syntactic structure above NP (Hypothesis 3a) and that on the other hand, it flags the presence of the realization operator REL (Hypothesis 3b). As such, similar to adjectival inflections, I maintain that \textit{ein} is a reflex of the structure but not of the semantics. 
\end{styleStandard}

\begin{styleStandard}
\ \ The chapter is organized as follows. Section 2 provides the data discussing three different readings in the DP and TP, and the role emotiveness plays in them. In Section 3, I first lay out my main assumptions, and then I provide the proposal and detailed derivations of these different readings. Section 4 summarizes the main findings of this chapter.
\end{styleStandard}

\begin{styleStandard}\bfseries
2.\ \ Data
\end{styleStandard}

\begin{styleStandard}
This section investigates the parallels between the nominal and clausal domains with regard to certain readings. Of particular interest here are interpretations that seem to relate the presence of \textit{ein} to emotiveness. In order to understand and classify the data better, I start with some preliminary remarks, where I dicuss the different types of nouns involved in pronominal DPs and copular TPs and how their various meanings relate to different readings. In the second section, I turn to the data in more detail, and I provide a summary of the different readings in the nominal and clausal domains. 
\end{styleStandard}

\begin{styleStandard}\itshape
2.1.\ \ Preliminary Remarks
\end{styleStandard}

\begin{styleStandard}
In this section, I illustrate some of the properties of the different types of nouns. Specifically, I discuss how role and kind nouns fare in pronominal DPs and copular TPs. In the second subsection, I expand on concepts like emotiveness and figurative extension. In the final subsection, I establish three readings that help classify and analyze the data.
\end{styleStandard}

\begin{styleStandard}
2.1.1. Basic Differences Between Role and Kind Nouns in the DP and TP
\end{styleStandard}

\begin{styleStandard}
Nouns can be classified along different dimensions: role vs. kind, neutral vs. emotive, and related to neutral/emotive, literal vs. figurative (for the latter, see next subsection). It is important to point out that the restrictions to be discussed reveal themselves with certain types of nouns only. 
\end{styleStandard}

\begin{styleStandard}
\ \ Starting with role nouns, de Swart \textit{et al}. (2005: 453) provide a convenient summary of the properties of bare predicate nominals such as \textit{Landwirt} ‘farmer’:\footnote{\ For relevant discussion, see also Kupferman (1991), Stowell (1989, 1991), and more recently Alexiadou (2005: 814-17), Déprez (2005), Matushansky \& Spector (2005), Munn \& Schmitt (2005: 843-49), Winter (2005: 779-83), and Zamparelli (2008). The latter paper also contains detailed critical discussion of earlier work on this topic.} (i) they form a restricted class of nouns (typically, names for professions, nationalities, and religions), (ii) they only exhibit a restricted range of interpretations (what the authors call capacities), (iii) they allow capacity qualifiers such as \textit{by profession}, (iv) they exhibit number neutrality, and (v) they combine with certain adjectives, which remain uninflected (at least, in Dutch). In the discussion to follow, I revisit (i), (ii), and (iv). Furthermore, related to (ii), I add a sixth characteristic to this: in their literal meaning, bare role nouns are never emotive in pragmatically neutral contexts.
\end{styleStandard}

\begin{styleStandard}
\ \ Kind nouns comprise a more diverse group; for instance, they involve names for humans, animals, and inanimates. For current purposes, animate nouns can be classified as regards neutral (\textit{Mann} ‘man’) and emotive (\textit{Idiot} ‘idiot’). With this in place, I turn to the combinations of “pronoun + (copular verb) + noun” to identify some initial properties and restrictions. I focus for now on pragmatically neutral, out-of-the-blue contexts (but see Section 3.1.3).
\end{styleStandard}

\begin{styleStandard}
\ \ It is well known that pronominal DPs in the singular only allow emotive kind nouns like \textit{Idiot }‘idiot’; (neutral) role nouns like \textit{Landwirt} ‘farmer’ and neutral kind nouns like \textit{Mann} ‘man’ are marked (7): 
\end{styleStandard}

\begin{styleStandard}
(7) \ \ a.\ \ \textit{du }\textit{\textsubscript{\ \ \ }}\textit{Idiot }
\end{styleStandard}

\begin{styleStandard}
\ \ \ \ you idiot.\textsc{masc} 
\end{styleStandard}

\begin{styleStandard}
\ \ \ \ ‘you (idiot)’
\end{styleStandard}

\begin{styleStandard}
\ \ \ b. \ ??\textit{\ \ du }\textit{\textsubscript{\ \ \ }}\textit{Landwirt }
\end{styleStandard}

\begin{styleStandard}
\ \ \ \ you farmer.\textsc{masc} \ \ \ 
\end{styleStandard}

\begin{styleStandard}
\ \ \ \ ‘you (farmer)’
\end{styleStandard}

\begin{styleStandard}
\ \ \ \ \ c. \ ??\textit{\ \ du }\textit{\textsubscript{\ \ }}\textit{Mann}
\end{styleStandard}

\begin{styleStandard}
\ \ \ \ you man.\textsc{masc}
\end{styleStandard}

\begin{styleStandard}
\ \ \ \ ‘you (man)’
\end{styleStandard}

\begin{styleStandard}
In contrast, copular constructions can involve bare predicate nouns only if the relevant nominals consist of role nouns like \textit{Landwirt}. Note again though that the indefinite article is optional here for many speakers (8a). Unlike role nouns, kind nouns like emotive \textit{Idiot} and non-emotive \textit{Mann} require an indefinite article (8b-c):
\end{styleStandard}

\begin{styleStandard}
(8)\ \ a.\ \ \textit{Du \ bist (‘n) Landwirt}.
\end{styleStandard}

\begin{styleStandard}
\ \ \ \ you are \ \ \ a \ \ farmer.\textsc{masc} \ \ \ \ \ 
\end{styleStandard}

\begin{styleStandard}
\ \ \ \ ‘You are a farmer.’
\end{styleStandard}

\begin{styleStandard}
\ \ b.\ \ \textit{Du \ bist *(‘n) Idiot}.
\end{styleStandard}

\begin{styleStandard}
\ \ \ \ you are \ \ \ \textsubscript{\ \ \ }an idiot.\textsc{masc} \ 
\end{styleStandard}

\begin{styleStandard}
\ \ \ \ ‘You are an idiot.’
\end{styleStandard}

\begin{styleStandard}
\ \ c.\ \ \textit{Du \ bist *(‘n) Mann}.
\end{styleStandard}

\begin{styleStandard}
\ \ \ \ you are \ \ \ \textsubscript{\ \ \ }a \ \ man.\textsc{masc}
\end{styleStandard}

\begin{styleStandard}
\ \ \ \ ‘You are a man.’
\end{styleStandard}

\begin{styleStandard}
To be clear, abstracting away from the determiners in (7a) and (8a), bare nouns involve emotive kind nouns in the DP but (neutral) role nouns in the TP. Furthermore, role nouns in the DP (7b) and emotive kind nouns in the TP (8b) pattern like neutral kind nouns such as \textit{Mann} in the (c)-examples. 
\end{styleStandard}

\begin{styleStandard}
\ \ Examining these cases, it becomes clear that true minimal pairs between singular DPs and related TPs are not possible. In other words, in pragmatically neutral, out-of-the-blue contexts, there do not seem to be lexical items that share properties of being both emotive nouns and role nouns; that is, there are no nouns that allow licit patterns such as \textit{du Noun}\textit{\textsubscript{1}} ‘you Noun’ as well as \textit{Du bist Noun}\textit{\textsubscript{1}} ‘You are a Noun’. The generalization to be explained, then, is that emotive nouns cannot involve role nouns and that role nouns (in their literal meaning) are not emotive. Expressed in more structural terms, emotive nouns can occur in DPs but require \textit{ein} in TPs, and role nouns cannot occur in DPs but can be bare in TPs. As will become clear, these restrictions only hold in singular, pragmatically neutral contexts.
\end{styleStandard}

\begin{styleStandard}
\ \ Before moving on, note that adding an evaluative adjective creates minimal pairs. However, this covers up the differences in (7) and (8) above, lexico-semantically (all nouns are equally fine now) and syntactically (all copulative cases must have \textit{ein} now):\textstyleFootnoteSymbol{ }
\end{styleStandard}

\begin{styleStandard}
(9) \ \ a.\ \ \textit{du \ \ blöder Idiot}
\end{styleStandard}

\begin{styleStandard}
\ \ \ \ you stupid \ \ idiot.\textsc{masc} 
\end{styleStandard}

\begin{styleStandard}
\ \ \ \ ‘you (stupid idiot)’
\end{styleStandard}

\begin{styleStandard}
\ \ \ b.\ \ \textit{du \ \ blöder Landwirt }
\end{styleStandard}

\begin{styleStandard}
\ \ \ \ you stupid \ farmer.\textsc{masc} \ \ \ \ 
\end{styleStandard}

\begin{styleStandard}
\ \ \ \ ‘you (stupid farmer)’
\end{styleStandard}

\begin{styleStandard}
\ \ c.\ \ \textit{du \ \ blöder Mann }
\end{styleStandard}

\begin{styleStandard}
\ \ \ \ you stupid \ man.\textsc{masc} \ \ \ \ 
\end{styleStandard}

\begin{styleStandard}
\ \ \ \ ‘you (stupid man)’
\end{styleStandard}

\begin{styleStandard}
(10)\ \ a.\ \ \textit{Du \ bist *(‘n) blöder Landwirt}.
\end{styleStandard}

\begin{styleStandard}
\ \ \ \ you are \ \ \ \ \ \textsubscript{\ }a \ \textsubscript{\ }stupid farmer.\textsc{masc} \ \ \ 
\end{styleStandard}

\begin{styleStandard}
\ \ \ \ ‘You are a stupid farmer.’
\end{styleStandard}

\begin{styleStandard}
\ \ b.\ \ \textit{Du \ bist *(‘n) blöder Idiot}.
\end{styleStandard}

\begin{styleStandard}
\ \ \ \ you are \ \ \ \ \ \textsubscript{\ }a \ \textsubscript{\ }stupid idiot.\textsc{masc} 
\end{styleStandard}

\begin{styleStandard}
\ \ \ \ ‘You are a stupid idiot.’
\end{styleStandard}

\begin{styleStandard}
\ \ c.\ \ \textit{Du \ bist *(‘n) blöder Mann}.
\end{styleStandard}

\begin{styleStandard}
\ \ \ \ you are \ \ \ \ \ \textsubscript{\ }a \ \textsubscript{\ }stupid man.\textsc{masc}
\end{styleStandard}

\begin{styleStandard}
\ \ \ \ ‘You are a stupid man.’
\end{styleStandard}

\begin{styleStandard}
As can be seen in (10a), modified role nouns must also take \textit{ein }in predicative contexts. This is explained in Section 3.4.1. Thus, in order to probe into the restrictions, I use unmodified nouns.\footnote{\ I concentrate on countable nouns (for the discussion of the concept of countability, see Allan 1980, also Chapters 7 and 8), and I basically leave aside group nouns and \textit{pluralia tantum }nouns. Also, I abstract away from special cases such as “royal” \textit{ihr} ‘you’ and “nursely” \textit{wir }‘we’.} In the introduction, I made a distinction of nouns as regards their meanings: neutral/literal vs. emotive/figurative. It is important to be more precise about this. We will see that emotiveness and figurativeness are not the same.
\end{styleStandard}

\begin{styleStandard}
2.1.2. Emotiveness and Figurative Extension
\end{styleStandard}

\begin{styleStandard}
Role and kind nouns can have two subtypes as regards their meaning. Focusing on the main cases discussed below, role nouns like \textit{Landwirt} have a neutral-only meaning (‘farmer’); others like \textit{Bauer} can take on an additional figurative meaning: besides ‘farmer’, \textit{Bauer} can also mean ‘peasant’ in certain contexts (Section 2.2).\footnote{\ Besides figurative, there are other terms found in the literature: metaphorical, extended, approximative, subjective, descriptive, gradable, and [+scalar]. Also, note that this difference in interpretative possibilities also holds true of regular kind nouns and proper names; for instance, while \textit{Obelisk} ‘obelisk’ and \textit{Joachim }‘Joachim’ pattern with \textit{Landwirt} ‘farmer’, \textit{Säule} ‘column/pillar’ and \textit{Willi} ‘Willy/idiot’ align with \textit{Bauer} ‘farmer/peasant’.} Briefly, the figurative meaning of a role noun like \textit{Bauer} denotes certain stereotypical properties of a farmer (but there is no implication that this person is a farmer; more on this below). This figurative meaning is emotive. This means that while role nouns are not emotive in their literal meaning, they are in their figurative meaning. Kind nouns like \textit{Schwein} have a neutral-only meaning (‘pig’), but if applied to contexts involving humans, they can also take on a figurative meaning (‘swine/pig-like person’); other kind nouns like \textit{Idiot} have an emotive-only meaning (‘idiot’). 
\end{styleStandard}

\begin{styleStandard}
\ \ As regards emotiveness then, some nouns become emotive when “figuratively extended” in certain pragmatic contexts (\textit{Schwein}, \textit{Bauer}), but others are inherently emotive (\textit{Idiot}). This means that not all emotive readings are due to figurative extension. That said, it is important to elaborate on the figurative extension of \textit{Schwein} and \textit{Bauer} some more. In their neutral/literal meaning, these nouns denote properties like ‘pig’ or ‘farmer’; in their emotive/figurative meaning, these nouns denote certain features of a stereotypical representative of the relevant kind. In other words, while the individuals under discussion are ascribed certain features of a pig or farmer, there is no implication here that the individuals are actually pigs or farmers. Below, this type of reading is labeled ‘comparative’.
\end{styleStandard}

\begin{styleStandard}
\ \ \ In sum, while certain role nouns only have a neutral/literal meaning (\textit{Landwirt}), other such nouns may have an additional emotive/figurative one (\textit{Bauer}). Emotive kind nouns have an emotive meaning only (\textit{Idiot}), but neutral kind nouns may have an additional emotive/figurative meaning (\textit{Schwein}). Importantly, while a figurative meaning is emotive (\textit{Bauer}/\textit{Schwein}), an unambiguous emotive noun does not have a figurative meaning (\textit{Idiot}). In the summary Table 1, I distinguish the different types of nouns by the terms unambiguous and ambiguous (NB: since the neutral meaning of \textit{Schwein} ‘pig’ cannot apply to a human, I put it in parentheses; the negative value in parentheses indicates that this is not a possible reading in pragmatically neutral contexts):\footnote{\ As we will see below, role nouns in their literal meaning (e.g., \textit{Landwirt/Bauer} ‘farmer’) can take on an emotive component (‘very good/bad farmer’). The same goes for kind nouns (but this is not relevant for current purposes as humans cannot be pigs even in emotive contexts). Furthermore, emotive nouns like \textit{Idiot} ‘idiot’ cannot have a neutral reading and, as far as I can tell, they cannot have a figurative reading. As to the latter, note that \textit{Du bist ‘n Idiot!} does not seem to mean ‘You are an idiot-like person!’, which states that the person addressed only has certain features of an idiot (whatever those may be).}
\end{styleStandard}

\begin{styleStandard}
Table 1: Different Types of Count Nouns
\end{styleStandard}

\begin{flushleft}
\begin{tabular}{|m{0.6837598in}|m{1.0462599in}|m{0.7962598in}|m{1.6087599in}|m{0.6712598in}|m{1.3712599in}|}

\hline
\multicolumn{2}{|m{1.8087599in}|}{Type of noun} &
\multicolumn{2}{m{2.4837599in}|}{\centering Role} &
\multicolumn{2}{m{2.12126in}|}{\centering Kind}\\\hline
\multicolumn{2}{|m{1.8087599in}|}{(Non-)ambiguity of the noun} &
{\centering\itshape Landwirt\par}

\centering Unamb. &
{\centering\itshape Bauer\par}

\centering Ambiguous &
{\centering\itshape Idiot\par}

\centering Unamb. &
{\centering\itshape Schwein\par}

\centering\arraybslash Ambiguous\\\hline
Meaning &
Neutral/literal &
\centering ‘farmer’ &
\centering ‘farmer’ &
\centering {}- &
\centering\arraybslash (‘pig’)\\\hline
 &
Emotive &
\centering (-) &
\centering (-) &
\centering ‘idiot’ &
\centering\arraybslash (-)\\\hline
 &
Figur. (emot.) 

[Comparative] &
\centering {}- &
{\centering ‘peasant’\par}

\centering [“farmer-like person”] &
\centering {}- &
{\centering ‘swine’\par}

\centering\arraybslash [“pig-like person”]\\\hline
\end{tabular}
\end{flushleft}
\begin{styleStandard}
Note that the word pairs \textit{Landwirt/Bauer} and \textit{Idiot/Schwein} in Table 1 are intriguing in that one element is unambiguous and the other is ambiguous (again, with the qualification that the literal meaning of \textit{Schwein} ‘pig’ will not be relevant below). To account for this difference, some amount of lexical specification must be assumed (if so, we might also expect some dialectal and cross-linguistic variation in this regard, which I think is true; I return to the issue of lexical specification in Section 3.1.2). In combination with pronominal determiners, these different types of nouns yield three different readings, as explicated in the next subsection. These readings help organize the data in Section 2.2 and the subsequent analysis in Section 3.
\end{styleStandard}

\begin{styleStandard}
2.1.3. Three Different Readings
\end{styleStandard}

\begin{styleStandard}
Basically following den Dikken’s (2006: chapter 5) discussion of binominal structures of the type \textit{an idiot of a doctor}, I point out that there are three relevant interpretations in the cases under discussion here. Focusing for now on pronominal DPs (11), I label the first case as ordinary reading. The second and the third case are called comparative and capacity readings:\footnote{\ Den Dikken labels the capacity reading as attributive, a term I would like to avoid.}
\end{styleStandard}

\begin{styleStandard}
(11) \ \ a. Ordinary reading: 
\end{styleStandard}

\begin{styleStandard}
\ \ \ \ \textit{du \ \ Idiot}
\end{styleStandard}

\begin{styleStandard}
\ \ \ \ you idiot.\textsc{masc}
\end{styleStandard}

\begin{styleStandard}
\ \ \ \ ‘you (idiot)’
\end{styleStandard}

\begin{styleStandard}
\ \ b. Comparative reading: 
\end{styleStandard}

\begin{styleStandard}
\ \ \ \ \textit{du \ \ Schwein} \ \ \ \ \ \ 
\end{styleStandard}

\begin{styleStandard}
\ \ \ \ you pig.\textsc{neut}
\end{styleStandard}

\begin{styleStandard}
\ \ \ \ ‘you (idiot)’
\end{styleStandard}

\begin{styleStandard}
\ \ c. Capacity reading: 
\end{styleStandard}

\begin{styleStandard}
\ \ \ \ \textit{du \ \ als Landwirt} \ \ \ \ 
\end{styleStandard}

\begin{styleStandard}
\ \ \ \ you as \ farmer.\textsc{masc}
\end{styleStandard}

\begin{styleStandard}
\ \ \ \ ‘you as a farmer’
\end{styleStandard}

\begin{styleStandard}
I paraphrase each of these three readings as follows:
\end{styleStandard}

\begin{styleStandard}
(12) \ \ the unique x (informally addressed) such that
\end{styleStandard}

\begin{styleStandard}
\ \  \ \ a.\ \ x is an idiot
\end{styleStandard}

\begin{styleStandard}
\ \  \ \ b.\ \ x is, in a certain way, similar to a pig
\end{styleStandard}

\begin{styleStandard}
\ \  \ \ c.\ \ x is being singled out in the capacity of a farmer
\end{styleStandard}

\begin{styleStandard}
As we will see in the data section below, similar readings can also be identified in the TP. Before I turn to the data, some further remarks about the readings are in order here.
\end{styleStandard}

\begin{styleStandard}
\ \ I point out that the ordinary reading in the DP is most often negatively emotive (Vater 2000), but a positive connotation is also possible. In this regard, Rauh (2004) provides the following example: \textit{du Glückspilz} ‘(you luck.mushroom =) lucky you’. Assuming that the pronoun and noun stand in a predication relation with one another (see Section 3.2), I suggest that this type involves direct predication. 
\end{styleStandard}

\begin{styleStandard}
\ \ The comparative reading is also emotive but involves nouns with emotive/figurative meanings. Unlike the first case, it involves indirect predication; that is, the predication seems to be mediated by a predicate such as \textit{like }or\textit{ similar to}.\footnote{\ This is what I assume for now. Below, I suggest that the comparative reading follows from the realization operator REL.} Note again that epithetically used nouns like \textit{Schwein} ‘(pig =) idiot’ in (11b) characterize the individual in their entirety, but there is no implication that the individual being addressed is actually a pig, as reflected in the translation.\footnote{\ Recall that I translate the figurative meaning of animal names simply as ‘idiot’ (although such names may imply different attributes). } 
\end{styleStandard}

\begin{styleStandard}
\ \ Finally, in contrast to the first two cases, the capacity reading is neutral and features nouns with neutral/literal meanings. Like the second reading, it also involves indirect predication, which in this case is mediated by the element \textit{als} ‘as; in the capacity of’. As noted by Rauh (2004: 85-86, 94), these \textit{als-}nominals clearly presuppose the presence of other properties, but only the overt one following \textit{als} is taken to be relevant. In contrast to the comparative reading, which characterizes individuals in their entirety, the capacity reading zooms in on a certain aspect of the individual. 
\end{styleStandard}

\begin{styleStandard}
\ \ With this in place, I turn to the data in more detail to see how role nouns and emotive or figurative kind nouns pattern as regards the three readings in various morpho-syntactic contexts. I compare the relevant readings in the DP to those of the TP. 
\end{styleStandard}

\begin{styleStandard}\itshape
2.2.\ \ Data
\end{styleStandard}

\begin{styleStandard}
I start with the data involving nouns in the plural. We will see that with the exception of the capacity reading, there are no restrictions in these cases. This is followed by unambiguous nouns in the singular, which reveal more restrictions. After providing an interim summary, I discuss ambiguous nouns in the singular, which exhibit a complementary loss of their potential meanings.
\end{styleStandard}

\begin{styleStandard}
2.2.1. Plural
\end{styleStandard}

\begin{styleStandard}
I begin with the two unambiguous nouns from Table 1. As can be seen in (13) and (14), role as well as emotive kind nouns are equally possible in ordinary readings:
\end{styleStandard}

\begin{styleStandard}
(13) \ \ a.\ \ \textit{ihr \ \ Landwirte}
\end{styleStandard}

\begin{styleStandard}
\ \ \ \ you farmers
\end{styleStandard}

\begin{styleStandard}
\ \ \ \ ‘you farmers’
\end{styleStandard}

\begin{styleStandard}
\ \ b.\ \ \textit{Ihr \ seid Landwirte}.
\end{styleStandard}

\begin{styleStandard}
\ \ \ \ you are \ farmers
\end{styleStandard}

\begin{styleStandard}
\ \ \ \ ‘You are farmers.’
\end{styleStandard}

\begin{styleStandard}
(14) \ \ a.\ \ \textit{ihr \ }\textit{\textsubscript{\ }}\textit{Idioten}
\end{styleStandard}

\begin{styleStandard}
\ \ \ \ you idiots
\end{styleStandard}

\begin{styleStandard}
\ \ \ \ ‘you idiots’
\end{styleStandard}

\begin{styleStandard}
\ \ b.\ \ \textit{Ihr \ seid Idioten}.
\end{styleStandard}

\begin{styleStandard}
\ \ \ \ you are \ idiots
\end{styleStandard}

\begin{styleStandard}
\ \ \ \ ‘You are idiots.’
\end{styleStandard}

\begin{styleStandard}
\ \ Continuing with ambiguous kind nouns, something similar holds for the comparative reading with figuratively extended nouns:
\end{styleStandard}

\begin{styleStandard}
(15) \ \ a.\ \ \textit{ihr \ Schweine }
\end{styleStandard}

\begin{styleStandard}
\ \ \ \ you pigs
\end{styleStandard}

\begin{styleStandard}
\ \ \ \ ‘you idiots’
\end{styleStandard}

\begin{styleStandard}
\ \ b.\ \ \textit{Ihr \ seid Schweine.}
\end{styleStandard}

\begin{styleStandard}
\ \ \ \ you are \ pigs 
\end{styleStandard}

\begin{styleStandard}
\ \ \ \ ‘You are idiots.’
\end{styleStandard}

\begin{styleStandard}
\ \ As for the capacity reading, I note that only certain predicate nouns are possible. In particular, role nouns (and certain kind nouns, see below) are fine (16a). However, as pointed out in Lawrence (1993: 93) and Rauh (2004: 86), emotive and figuratively extended kind nouns are marked (16b-c). It seems to me that the latter also holds true of kind nouns that have a more general, hypernym-like denotation (16d):
\end{styleStandard}

\begin{styleStandard}
(16) \ \ a.\ \ \textit{ihr \ als Landwirte }
\end{styleStandard}

\begin{styleStandard}
\ \ \ \ you as \ farmers \ \ \ 
\end{styleStandard}

\begin{styleStandard}
\ \ \ \ ‘you as farmers’
\end{styleStandard}

\begin{styleStandard}
\ \ b. ??\ \ \textit{ihr \ als Idioten}
\end{styleStandard}

\begin{styleStandard}
\ \ \ \ you as \ \ \textsubscript{\ }idiots \ \ 
\end{styleStandard}

\begin{styleStandard}
\ \ c. ??\ \ \textit{ihr \ als Schweine }
\end{styleStandard}

\begin{styleStandard}
\ \ \ \ you as \ pigs \ \ \ \ \ \ \ \ 
\end{styleStandard}

\begin{styleStandard}
\ \ d. ??\ \ \textit{ihr \ als Personen}
\end{styleStandard}

\begin{styleStandard}
\ \ \ \ you as \ persons \ 
\end{styleStandard}

\begin{styleStandard}
The restriction shown in (16b-d) is probably due to the fact that \textit{als} ‘as’ singles out a person with regard to a specific capacity or skill. Importantly, this quality must be restrictive enough such that only some people typically have it in a given context. In other words, while it is clear that not each and every person is a farmer (16a), every person can presumably be in someone else’s bad books (16b-c). As to (16d), every person is a human making \textit{Personen} ‘persons’ too general in meaning to be restrictive in the relevant way. To be clear, only \textit{Landwirte} ‘farmers’ in (16a) singles out the addressed individuals in the relevant way but the other nouns in (16) do not. I turn to the capacity reading in the clause after I discuss ambiguous nouns.
\end{styleStandard}

\begin{styleStandard}
\ \ Turning to ambiguous role nouns such as \textit{Bauer} ‘farmer/peasant’, they are possible in the neutral/literal and emotive/figurative meanings in simple DPs and copular cases (17a-b), where both an ordinary and a comparative reading are possible. In contrast, the restriction discussed for (16) emerges again in \textit{als-}nominals (17c). In other words, the capacity reading in (17c) is the sum of \textit{als} ‘as’ and the neutral/literal reading of the noun:
\end{styleStandard}

\begin{styleStandard}
(17) \ \ a.\ \ \textit{ihr \ }\textit{\textsubscript{\ }}\textit{Bauern \ \ \ \ \ \ \ \ \ \ }
\end{styleStandard}

\begin{styleStandard}
\ \ \ \ you farmers/peasants
\end{styleStandard}

\begin{styleStandard}
\ \ \ \ ‘you farmers/peasants’
\end{styleStandard}

\begin{styleStandard}
\ \ b.\ \ \textit{Ihr \ seid Bauern.}
\end{styleStandard}

\begin{styleStandard}
\ \ \ \ you are \ farmers/peasants
\end{styleStandard}

\begin{styleStandard}
\ \ \ \ ‘You are farmers/peasants.’
\end{styleStandard}

\begin{styleStandard}
\ \ c.\ \ \textit{ihr \ als Bauern \ \ \ \ \ \ \ \ \ \ }
\end{styleStandard}

\begin{styleStandard}
\ \ \ \ you as \ farmers
\end{styleStandard}

\begin{styleStandard}
\ \ \ \ ‘you as farmers’
\end{styleStandard}

\begin{styleStandard}
\ \ \ \ \#‘you as peasants’
\end{styleStandard}

\begin{styleStandard}
\ \ Unlike the DP cases as in (17c), clauses do not involve \textit{als} ‘as’ in the capacity reading (18a). However, this reading can, as an approximation, be brought out when the modal particle \textit{vielleicht} ‘really’ is added (18b). Note that this example also gets an intensified comparative reading with \textit{Bauern} meaning ‘peasants’. What is interesting to note is that even an unambiguous role noun like \textit{Landwirt} ‘farmer’ can take on an emotive component in such a context, with a negative reading being somewhat easier to obtain (18c):\footnote{\ Again, translations are not straightforward. Note that the element \textit{some} in the English translations is intended to convey the emotive flavor. Also, these constructions can take – what Delsing (1993: 36) calls – an implicit argument. To illustrate with German, the dative pronoun \textit{mir} ‘me’ can be added to (18b):\par \ \ (i)\ \ \textit{Ihr \ seid mir vielleicht ‘n paar \ \ \ \ Bauern!}\par \ \ \ \ you are \ me \ \textsc{prt} \ \ \ \ \ \ \ \ \ \ a couple farmers/peasants\par \ \ \ \ ‘(To me,) you are some farmers/peasants!’\par Note that this often called ethical dative seems to bring out more clearly the capacity reading of \textit{Bauern} meaning ‘farmers’ such that the presence of other properties is presupposed, but only the overt predicate is taken to be relevant. I stick to the simpler cases in the main text.}
\end{styleStandard}

\begin{styleStandard}
(18)\ \ a. \ *\ \ \textit{Ihr \ seid als Bauern}.
\end{styleStandard}

\begin{styleStandard}
\ \ \ \ you are \ as \ farmers/peasants
\end{styleStandard}

\begin{styleStandard}
\ \ b.\ \ \textit{Ihr \ seid vielleicht ‘n paar \ \ \ \ Bauern!}
\end{styleStandard}

\begin{styleStandard}
\ \ \ \ you are \ \textsc{prt} \ \ \ \ \ \ \ \ \ \ \ a couple farmers/peasants
\end{styleStandard}

\begin{styleStandard}
\ \ \ \ ‘You are some farmers/peasants!’
\end{styleStandard}

\begin{styleStandard}
\ \ c.\ \ \textit{Ihr \ seid vielleicht ‘n paar \ \ \ Landwirte!}
\end{styleStandard}

\begin{styleStandard}
\ \ \ \ you are \ \textsc{prt} \ \ \ \ \ \ \ \ \ \ \ a couple farmers
\end{styleStandard}

\begin{styleStandard}
\ \ \ \ ‘You are some farmers!’
\end{styleStandard}

\begin{styleStandard}
Observe that the emotive reading in (18c) is not due to the properties of the noun itself or figurative extension. Rather, the modal particle \textit{vielleicht} plays a role here. Specifically, this particle invokes a [+scalar] capacity reading; that is, the persons being addressed are described as being good or bad in the capacity of farmers.
\end{styleStandard}

\begin{styleStandard}
2.2.2. Singular
\end{styleStandard}

\begin{styleStandard}
The cases in the singular are more restricted. While the same readings as above are in principle possible, some well-known restrictions with regard to the predicate noun emerge. As with the plural cases above, I start with unambiguous nouns in the ordinary reading. 
\end{styleStandard}

\begin{styleStandard}
\ \ Recalling that I focus for now on pragmatically neutral contexts, I already noted in the introduction that role nouns are not possible in the DP (19a) but that they pattern more freely in the TP. As also mentioned above, many speakers allow these nouns not only to occur without \textit{ein} but also with the indefinite article (19b). Unlike role nouns, emotive kind nouns are possible in the DP (20a) but must co-occur with an article in the TP (20b):
\end{styleStandard}

\begin{styleStandard}
(19) \ \ a. ??\ \ \textit{du \ \ Landwirt}
\end{styleStandard}

\begin{styleStandard}
\ \ \ \ you farmer.\textsc{masc}
\end{styleStandard}

\begin{styleStandard}
\ \ b.\ \ \textit{Du \ bist (‘n) Landwirt}.
\end{styleStandard}

\begin{styleStandard}
\ \ \ \ you are \ \ \ \textsubscript{\ }a \ farmer.\textsc{masc}
\end{styleStandard}

\begin{styleStandard}
\ \ \ \ ‘You are a farmer.’
\end{styleStandard}

\begin{styleStandard}
(20) \ \ a.\ \ \textit{du \ \ Idiot}
\end{styleStandard}

\begin{styleStandard}
\ \ \ \ you idiot.\textsc{masc}
\end{styleStandard}

\begin{styleStandard}
\ \ \ \ ‘you (idiot)’
\end{styleStandard}

\begin{styleStandard}
\ \ b.\ \ \textit{Du \ bist *(‘n) \ Idiot}.
\end{styleStandard}

\begin{styleStandard}
\ \ \ \ you are \ \ \ \ \ \textsubscript{\ }an idiot.\textsc{masc}
\end{styleStandard}

\begin{styleStandard}
\ \ \ \ ‘You are an idiot.’
\end{styleStandard}

\begin{styleStandard}
\ \ In the comparative reading in (21), the distribution is identical to the cases involving emotive kind nouns with ordinary readings (as just seen):
\end{styleStandard}

\begin{styleStandard}
(21) \ \ a.\ \ \textit{du \ \ Schwein}
\end{styleStandard}

\begin{styleStandard}
\ \ \ \ you pig.\textsc{neut}
\end{styleStandard}

\begin{styleStandard}
\ \ \ \ ‘you (idiot)’
\end{styleStandard}

\begin{styleStandard}
\ \ b.\ \ \textit{Du \ bist *(‘n) Schwein}
\end{styleStandard}

\begin{styleStandard}
\ \ \ \ you are \ \ \ \ \ \textsubscript{\ }a \ pig.\textsc{neut}
\end{styleStandard}

\begin{styleStandard}
\ \ \ \ ‘You are an idiot.’
\end{styleStandard}

\begin{styleStandard}
In other words, kind nouns, be they inherently emotive or figuratively extended, pattern alike. Finally, I turn to the capacity reading.
\end{styleStandard}

\begin{styleStandard}
\ \ Similar to the cases in the plural, only certain nouns are possible in \textit{als-}nominals:
\end{styleStandard}

\begin{styleStandard}
(22) \ \ a.\ \ \textit{du \ \ als Landwirt }
\end{styleStandard}

\begin{styleStandard}
\ \ \ \ you as \ farmer.\textsc{masc} \ \ \ 
\end{styleStandard}

\begin{styleStandard}
\ \ \ \ ‘you as a farmer’
\end{styleStandard}

\begin{styleStandard}
\ \ b. ??\ \ \textit{du \ \ als Idiot }
\end{styleStandard}

\begin{styleStandard}
\ \ \ \ you as \ idiot.\textsc{masc}\ \ 
\end{styleStandard}

\begin{styleStandard}
\ \ c. ??\ \ \textit{du \ \ als Schwein }
\end{styleStandard}

\begin{styleStandard}
\ \ \ \ you as \ \ pig.\textsc{neut}\ \ 
\end{styleStandard}

\begin{styleStandard}
\ \ d. ??\ \ \textit{du \ \ als Person}
\end{styleStandard}

\begin{styleStandard}
\ \ \ \ you as \ person.\textsc{fem}\ \ 
\end{styleStandard}

\begin{styleStandard}
I assume that the restriction involved in (22b-d) is the same as that seen in the plural above.
\end{styleStandard}

\begin{styleStandard}
\ \ In the clausal counterpart, the indefinite article is obligatory with role nouns, and these cases are emotive (23a). Notice also that when emotive or figuratively extended kind nouns combine with the modal particle, the ordinary and comparative readings get intensified (23b-c):
\end{styleStandard}

\begin{styleStandard}
(23)\ \ a.\ \ \textit{Du \ bist vielleicht *(‘n) Landwirt!}
\end{styleStandard}

\begin{styleStandard}
\ \ \ \ you are \ \textsc{prt} \ \ \ \ \ \ \ \ \ \ \ \ \ a \ \textsubscript{\ }farmer.\textsc{masc}
\end{styleStandard}

\begin{styleStandard}
\ \ \ \ ‘You are some farmer!’
\end{styleStandard}

\begin{styleStandard}
\ \ \ b.\ \ \textit{Du \ bist vielleicht ‘n \ \ Idiot!}\ \ \ \ \ \ 
\end{styleStandard}

\begin{styleStandard}
\ \ \ \ you are \ \textsc{prt} \ \ \ \ \ \ \ \ \ \ an idiot.\textsc{masc}
\end{styleStandard}

\begin{styleStandard}
\ \ \ \ ‘You are really an idiot!’
\end{styleStandard}

\begin{styleStandard}
\ \ \ c.\ \ \textit{Du \ bist vielleicht ‘n Schwein!}\ \ \ \ \ \ 
\end{styleStandard}

\begin{styleStandard}
\ \ \ \ you are \ \textsc{prt} \ \ \ \ \ \ \ \ \ \ a pig.\textsc{neut}
\end{styleStandard}

\begin{styleStandard}
\ \ \ \ ‘You are really an idiot!’
\end{styleStandard}

\begin{styleStandard}
Finally, for completeness’s sake, I discuss neutral kind nouns that involve words for humans.
\end{styleStandard}

\begin{styleStandard}
\ \ As already seen in Section 2.1.1, kind nouns such as \textit{Mann} ‘man’ are awkward in the DP (24a). Importantly, although non-emotive, they nonetheless require \textit{ein} in the TP (24b):
\end{styleStandard}

\begin{styleStandard}
(24) \ \ a. ??\ \ \textit{du \ \ Mann}
\end{styleStandard}

\begin{styleStandard}
\ \ you man.\textsc{masc}
\end{styleStandard}

\begin{styleStandard}
\ \ b. \ \ \textit{Du \ bist *(‘n) Mann}.
\end{styleStandard}

\begin{styleStandard}
\ \ you are \ \ \ \ \ \textsubscript{\ }a \ \textsubscript{\ }man.\textsc{masc}
\end{styleStandard}

\begin{styleStandard}
\ \ \ \ ‘You are a man.’
\end{styleStandard}

\begin{styleStandard}
These types of nouns do not occur with a comparative reading, but they do have a capacity reading:
\end{styleStandard}

\begin{styleStandard}
(25)\ \ a.\ \ \textit{du \ \ als Mann}
\end{styleStandard}

\begin{styleStandard}
\ \ \ \ you as \ man.\textsc{masc} \ \ \ 
\end{styleStandard}

\begin{styleStandard}
\ \ \ \ ‘you as a man’
\end{styleStandard}

\begin{styleStandard}
\ \ b.\ \ \textit{Du \ bist vielleicht ‘n Mann!}
\end{styleStandard}

\begin{styleStandard}
\ \ \ \ you are \ \textsc{prt} \ \ \ \ \ \ \ \ \ \ a \ man.\textsc{masc}
\end{styleStandard}

\begin{styleStandard}
\ \ \ \ ‘You are some man!’
\end{styleStandard}

\begin{styleStandard}
Before I discuss ambiguous role nouns like \textit{Bauer} ‘farmer/peasant’ in the singular, I briefly summarize the discussion thus far.
\end{styleStandard}

\begin{styleStandard}
2.2.3. Interim Summary
\end{styleStandard}

\begin{styleStandard}
Starting with some remarks on emotiveness, the last two subsections have shown that unambiguous role nouns can take on an emotive component in the clausal capacity reading. The latter involves the modal particle \textit{vielleicht} ‘really’. In other words, emotiveness can be distinguished into an inherent property (\textit{Idiot}), it can be due to figurative extension of a lexical meaning (\textit{Schwein} ‘pig’ {\textgreater} ‘swine’, \textit{Bauer }‘farmer’ {\textgreater} ‘peasant’), and it can be brought about in certain contexts where in conjunction with a modal particle, a neutral-only role noun may describe an individual performing their profession in an amazing or appalling way (\textit{Landwirt }‘farmer’ {\textgreater} ‘some farmer’). The latter becomes particularly clear when evaluative adjectives like \textit{toller} ‘great’ or \textit{miserabler} ‘poor’ are added. In other words, the context where a neutral noun appears may change the reading of the noun. With this in place, I summarize the nominal and clausal cases according to morphological number and the three readings.
\end{styleStandard}

\begin{styleStandard}
\ \ As can be seen in Table 2, DPs and TPs in the plural have basically the same readings – only the capacity reading differs in emotiveness:
\end{styleStandard}

\begin{styleStandard}
Table 2: Summary of the Readings in the Plural
\end{styleStandard}

\begin{flushleft}
\begin{tabular}{|m{0.37125984in}|m{1.9837599in}|m{1.9837599in}|m{1.9962599in}|}

\hline
 &
\centering Ordinary reading &
\centering Comparative reading &
\centering\arraybslash Capacity reading\\\hline
DP &
Emotive/Neutral &
Emotive &
Neutral [\textit{als}]\\\hline
TP &
Emotive/Neutral &
Emotive &
Emotive [\textit{vielleicht}]\\\hline
\end{tabular}
\end{flushleft}
\begin{styleStandard}
The singular cases are more restricted, especially in the ordinary reading. In more detail, the ordinary reading in the DP only allows (inherent) emotive kind nouns, but the two other readings in the DP are as in the plural. As for the TP, role nouns take an optional article in the ordinary reading, but kind nouns must have \textit{ein}. In the two other readings, nouns are emotive and require a determiner. Considering all three readings, \textit{ein} appears in both neutral and emotive contexts. This is summarized in Table 3 (the subscript K stands for kind noun and the subscript R for role noun):
\end{styleStandard}

\begin{styleStandard}
Table 3: Summary of the Readings in the Singular
\end{styleStandard}

\begin{flushleft}
\begin{tabular}{|m{0.37125984in}|m{1.9837599in}|m{1.9837599in}|m{1.9962599in}|}

\hline
 &
\centering Ordinary reading &
\centering Comparative reading &
\centering\arraybslash Capacity reading\\\hline
DP &
Emotive [\textit{du Idiot}\textsubscript{K}] &
Emotive [\textit{du Schwein}\textsubscript{K}] &
Neutral [\textit{du als Landwirt}\textsubscript{R}\textit{/Mann}\textsubscript{K}]\\\hline
TP &
Neutral [(\textit{ein}) \textit{Landwirt}\textsubscript{R}]

Emotive [\textit{ein Idiot}\textsubscript{K}]

Neutral [\textit{ein Mann}\textsubscript{K}] &
Emotive [\textit{ein Schwein}\textsubscript{K}] &
Emotive [\textit{vielleicht ein Landwirt}\textsubscript{R}\textit{/Mann}\textsubscript{K}]\\\hline
\end{tabular}
\end{flushleft}
\begin{styleStandard}
Note that all comparative readings involve figurative extension and consequently are emotive. I make some more remarks. 
\end{styleStandard}

\begin{styleStandard}
Starting with the singular (Table 3), I can point out again that the ordinary reading and the capacity reading are complementary as regards emotiveness in the DP. The absence vs. presence of \textit{als} ‘as’ seems to play a role here. Furthermore, emotiveness has less structure in the nominal domain (\textit{als} is not present) but more structure in the clausal domain (\textit{ein} must be present). In fact, the bidirectional entailment “non-emotive $\leftrightarrow $ \textit{als}” holds in the DP, and assuming that pronouns are determiners (Chapter 3, Section 5), the unidirectional entailment “emotive → determiner” holds in both domains. Finally, as mentioned above, the unidirectional entailment “no determiner → role noun” holds in the predicate nominal of the TP. 
\end{styleStandard}

\begin{styleStandard}
\ \ Interestingly, these entailments allow the combination of a role noun and \textit{ein }in the TP. This state-of-affairs is neatly summarized for the clause by de Swart \textit{et al}. (2005: 451), who point out that the reading of a bare (role) noun is basically a subset of the readings of the corresponding “\textit{ein }+ noun” combination (also Zamparelli 2008: 114). In other words, while a bare (role) noun can only be neutral/literal in meaning, the same noun preceded by \textit{ein} can be both neutral/literal (for many speakers) and emotive/figurative (for all speakers).\footnote{\ Note that de Swart \textit{et al}. (2007) seem to have retreated from this position. However, I believe their former generalization is correct. In this regard, see, for instance, the discussion of John F. Kennedy’s famous sentence \textit{Ich bin ein Berliner} in Eichhoff (1993), which for many speakers does not only have the emotive/figurative reading but also the neutral/literal one. I thank Veronika Ehrich for this reference and some discussion of this topic. } Below, we will see that the occurrence of \textit{ein} is simply a side-effect and does not cause this figurative extension in interpretation.
\end{styleStandard}

\begin{styleStandard}
\ \ Returning briefly to the plural (Table 2), the facts are somewhat different. First, \textit{als} ‘as’ does not have to be present for a neutral ordinary reading in the DP (e.g., \textit{ihr Linguisten} ‘you linguists’), and second, \textit{ein} is not present in the plural.\footnote{\ Note that non-canonical predicative contexts may involve plural \textit{ein}. This can be seen in the constructed example below:\par \textit{\ \ \ }(i) \ (?)\ \ \textit{Ihr \ seid vielleicht ein-e Bauern!}\par \ \ \ \ you are \ \ \textsc{prt \ \ \ \ \ \ \ \ \ \ \ }a-\textsc{pl }\ peasants/farmers\par \ \ \ \ ‘You are some peasants!’\par \ \ \ \ \%‘You are some farmers!’} With this in place, I finally turn to ambiguous role nouns in singular contexts. 
\end{styleStandard}

\begin{styleStandard}
2.2.4. \ Ambiguous Role Nouns in the Singular
\end{styleStandard}

\begin{styleStandard}
With the above discussion in mind, I consider examples involving \textit{Bauer} ‘farmer/peasant’. The expectation is that only some readings are possible in certain constructions. In fact, in the relevant cases, there should be a complementary loss of ambiguity. This is indeed borne out. To compare the distribution of the readings, I organize the data here according to syntactic domain.
\end{styleStandard}

\begin{styleStandard}
\ \ Starting with the nominal domain, I showed in section 2.2.2 that unambiguous role nouns (e.g., \textit{Landwirt} ‘farmer’) exhibit degraded grammaticality in the ordinary reading but are fine in the capacity reading. If so, we expect that ambiguous nouns have a complementary loss of meaning. Specifically, we predict that ambiguous nouns only involve an (emotive) comparative reading in the DP and that they have a neutral capacity reading in \textit{als-}nominals. This is exactly what we find in (26a) and (26b), respectively:
\end{styleStandard}

\begin{styleStandard}
(26) \ \ a.\ \ \textit{du \ \ Bauer}
\end{styleStandard}

\begin{styleStandard}
\ \ \ \ you peasant.\textsc{masc}
\end{styleStandard}

\begin{styleStandard}
\ \ \ \ ‘you (peasant)’
\end{styleStandard}

\begin{styleStandard}
\ \ \ \ \#‘you (farmer)’
\end{styleStandard}

\begin{styleStandard}
\ \ b.\ \ \textit{du \ \ als Bauer}
\end{styleStandard}

\begin{styleStandard}
\ \ \ \ you as \ farmer.\textsc{masc}
\end{styleStandard}

\begin{styleStandard}
\ \ \ \ ‘you as a farmer’
\end{styleStandard}

\begin{styleStandard}
\ \ \ \ \#‘you as a peasant’
\end{styleStandard}

\begin{styleStandard}
\ \ Turning to the clausal cases, a bare role noun should only have a neutral ordinary reading (27a), a role noun with an indefinite article should be ambiguous between a comparative and a neutral ordinary reading (27b), and a role noun in the context of the modal particle \textit{vielleicht} ‘really’ should be ambiguous between an intensified comparative and an emotive capacity reading (27c). Again, this is exactly what the data bear out:
\end{styleStandard}

\begin{styleStandard}
(27) \ \ a.\ \ \textit{Du \ bist Bauer.}
\end{styleStandard}

\begin{styleStandard}
\ \ \ \ you are \ farmer.\textsc{masc}
\end{styleStandard}

\begin{styleStandard}
\ \ \ \ ‘You are a farmer.’
\end{styleStandard}

\begin{styleStandard}
\ \ \ \ \#‘You are a peasant.’
\end{styleStandard}

\begin{styleStandard}
\ \ b.\ \ \textit{Du \ bist ‘n Bauer.}
\end{styleStandard}

\begin{styleStandard}
\ \ \ \ you are \ \ a peasant/farmer.\textsc{masc}
\end{styleStandard}

\begin{styleStandard}
\ \ \ \ ‘You are a peasant.’
\end{styleStandard}

\begin{styleStandard}
\ \ \ \ \%‘You are a farmer.’
\end{styleStandard}

\begin{styleStandard}
\ \ c.\ \ \textit{Du \ bist vielleicht ‘n Bauer!}\ \ \ \ 
\end{styleStandard}

\begin{styleStandard}
\ \ \ \ you are \ \textsc{prt} \ \ \ \ \ \ \ \ \ \ a peasant/farmer.\textsc{masc}
\end{styleStandard}

\begin{styleStandard}
\ \ \ \ ‘You are some peasant/farmer!’
\end{styleStandard}

\begin{styleStandard}
The discussion can be summarized in Table 4:
\end{styleStandard}

\begin{styleStandard}
Table 4: Summary of the Readings of Ambiguous Role Nouns in the Singular
\end{styleStandard}

\begin{flushleft}
\begin{tabular}{|m{0.37125984in}|m{1.9212599in}|m{1.9212599in}|m{2.12126in}|}

\hline
 &
\centering Ordinary reading &
\centering Comparative reading &
\centering\arraybslash Capacity reading\\\hline
DP &
 &
Emotive [\textit{du Bauer}] &
Neutral [\textit{du als Bauer}]\\\hline
TP &
Neutral [(\textit{ein}) \textit{Bauer}] &
Emotive [\textit{ein Bauer}] &
Emotive [\textit{vielleicht ein Bauer}]\\\hline
\end{tabular}
\end{flushleft}
\begin{styleStandard}
Table 4 is similar to Table 3, except for the ordinary reading. Unlike the kind noun \textit{Idiot} ‘idiot’, the role noun \textit{Bauer} ‘farmer/peasant’ cannot have an ordinary reading – the latter is only available in the TP, and it must be neutral there. Furthermore, note that \textit{Bauer} is only ambiguous in the presence of \textit{ein}. However, the article by itself does not have an “emotivizing” function as it does allow a neutral ordinary reading of \textit{Bauer }in the TP. 
\end{styleStandard}

\begin{styleStandard}
\ \ Above, we saw that role nouns in clauses have optional indefinite articles but that kind nouns, independently of emotiveness, have obligatory indefinite articles. This difference between role and kind nouns is taken up in the next section. We will see that the appearance of \textit{ein} is a mere side-effect of the presence of a certain semantic operator such that \textit{ein} flags the presence of that operator (Hypothesis 3b). Thus, I continue to claim that \textit{ein} is semantically vacuous (Hypothesis 1a). I also provide an explanation of the other highlighted properties of the aforementioned constructions. The proposal is based on two important works, de Swart \textit{et al}. (2007) and Rauh (2004), that manifest an interesting division of labor, at least in the DP.
\end{styleStandard}

\begin{styleStandard}\bfseries
3.\ \ Proposal
\end{styleStandard}

\begin{styleStandard}
In this section, I account for the similarities and differences between pronominal DPs and their clausal counterparts. First, I lay out my assumptions. This includes the discussion of two previous proposals. Then I turn to the three readings in detail.
\end{styleStandard}

\begin{styleStandard}\itshape
3.1. \ \ Combining Two Previous Proposals
\end{styleStandard}

\begin{styleStandard}
In order to account for the readings in the nominal and clausal domains, I basically follow de Swart \textit{et al}.’s (2007) discussion of the copular cases and extend it to pronominal DPs. Before I get into the specifics, I lay the groundwork.
\end{styleStandard}

\begin{styleStandard}
3.1.1. De Swart, Winter \& Zwarts (2007)
\end{styleStandard}

\begin{styleStandard}
As mentioned before, this book is not about the semantics of the DP \textit{per se} (or the semantics of the TP, for that matter). Rather than providing detailed denotations, I use Type Theory to show that the relevant elements are semantically compatible with one another (for general background, see Chapter 1, Section 4.2.2; Heim \& Kratzer 1998). As discussed in Chapter 3, Section 5, I take pronominal elements like \textit{du} ‘you’ to be similar to definite articles like \textit{der} ‘the’; that is, pronouns are determiners.
\end{styleStandard}

\begin{styleStandard}
\ \ As discussed in Section 2.1.1, common nouns come in two types: kind nouns and (a specific set of) role nouns.\footnote{\ De Swart \textit{et al.} (2007) call the latter capacity nouns. There are two notions of capacity now: an interpretative one (a certain type of reading) and a lexical one (certain groups of nouns). These notions are not identical. For instance, certain kind nouns may also occur in capacity readings (Section 2). In order to avoid confusion, I continue calling the relevant lexical items role nouns.} As is well known, kind nouns can be used to generically refer to a species (but not to individuals of that species) in cases like (28a). Carlson (1980) proposed that kind nouns are like names; that is, they are of type {\textless}e{\textgreater}. Assuming that generic determiners are expletives (cf. Longobardi 1994: 655-59), kind nouns and verbal predicates (type {\textless}e,t{\textgreater}) can combine to yield a truth value (type {\textless}t{\textgreater}). Turning to (28b), these types of cases are different. Here, the nominal refers to an individual of that species. De Swart \textit{et al}. (2007) follow Carlson (1980) in arguing that the realization operator REL takes a kind noun (type {\textless}e{\textgreater}) and maps it to a set of individuals (type {\textless}e,t{\textgreater}). The (substantive) determiner (type {\textless}{\textless}e,t{\textgreater}, e{\textgreater}) combines with this nominal predicate returning an entity (type {\textless}e{\textgreater}). The latter can combine with a verbal predicate to yield a truth value:
\end{styleStandard}

\begin{styleStandard}
(28)\ \ a.\ \ The dinosaur is a big animal / is extinct.
\end{styleStandard}

\begin{styleStandard}
\ \ b.\ \ The dinosaur ate some fruit.
\end{styleStandard}

\begin{styleStandard}
The novel claim of de Swart \textit{et al}. (2007) is that role nouns are also of type {\textless}e{\textgreater}. To map them into sets of individuals, they have to combine with the capacity operator CAP, a realization operator restricted to role nouns. To be clear then, both sorts of nouns are of type {\textless}e{\textgreater} and have to be mapped to type {\textless}e,t{\textgreater} if they are to function as predicate nouns. 
\end{styleStandard}

\begin{styleStandard}
\ \ De Swart \textit{et al}. (2007) relate the different realization operators to different syntactic structures. Kind nouns combine with REL, where the latter is assumed to be in NumP (29a). In contrast, de Swart \textit{et al}. (2007) propose that there are actually two options for role nouns. The first option involves a direct path, whereby role nouns combine with CAP. Note that CAP is in NP (29b). The second option constitutes an indirect path, whereby role nouns undergo kind coercion resulting in kind nouns. De Swart \textit{et al.} (2007: 213) state that \textit{kind} coercion involves an operator as well (but I will not go into the specifics here to keep the discussion simple). The latter elements then combine with REL in NumP (29b’): 
\end{styleStandard}

\begin{styleStandard}
(29)\ \ a.\ \ kind nouns:\ \ N\textsubscript{{\textless}e{\textgreater}} + REL \ \ \ \ \ \ → NumP\textsubscript{{\textless}e,t{\textgreater}}
\end{styleStandard}

\begin{styleStandard}
\ \ b.\ \ role nouns: \ \ N\textsubscript{{\textless}e{\textgreater}} + CAP \ \ \ \ \ \ → NP\textsubscript{{\textless}e,t{\textgreater}}
\end{styleStandard}

\begin{styleStandard}
\ \ b’.\ \ role nouns:\ \ N\textsubscript{{\textless}e{\textgreater}} → kind coercion + REL\ \ → NumP\textsubscript{{\textless}e,t{\textgreater}}
\end{styleStandard}

\begin{styleStandard}
Note that REL and CAP are both of type {\textless}e{\textless}e,t{\textgreater}{\textgreater} – these operators take an entity as their argument and return a predicate.\footnote{\ De Swart \textit{et al}. (2007: 217) argue that CAP is actually more complicated than {\textless}e{\textless}e,t{\textgreater}{\textgreater}. However, the latter simplication is sufficient for the current account (keeping in mind that CAP only combines with role nouns).} Furthermore, these operators are mutually exclusive; that is, when one applies, the other cannot. Crucially for the current discussion, REL is part of NumP, which triggers the presence of the indefinite article \textit{ein} ‘a’ (note that this is not further elaborated on in de Swart \textit{et al.}). For the most part, I follow Swart \textit{et al.}’s proposal. However, I refine it in certain ways below. 
\end{styleStandard}

\begin{styleStandard}
\ \ The two different paths to derive a predicate nominal in (29b-b’) are argued to explain the fact that bare role nouns are only neutral/literal in meaning, but role nouns preceded by \textit{ein} ‘a’ are both neutral/literal and emotive/figurative (Swart \textit{et al.}’s 2005 generalization, see Section 2.2.3 again).\footnote{\ A different proposal is made by Zamparelli (2008: 125), who claims that role nouns are ambiguous between nominals that denote certain classes of human beings and nominals that denote abstract, well-established activities that identify those classes of human beings. Unlike the former, the latter nominals lack an inherent value for abstract gender, which, in turn, is taken to explain the absence of the indefinite article in predicative contexts. To the extent that I understand this correctly, this basically implies the existence of two lexical items for every role noun. In contrast, de Swart \textit{et al.}’s analysis takes one relevant lexical item as its point of departure and proposes two operations for the role noun to be able to function as a predicate. The latter approach appears to be more desirable.} As such, de Swart \textit{et al.} relate overt morphological clues directly to the relevant interpretative differences. Specifically, as predicate nominals, bare (role) nouns imply the presence of an NP, which in turn implies CAP (30a). In contrast, plural suffixes on predicate nouns, exemplified in (30b) by -\textit{s}, are taken to be in NumP, and \textit{ein} is assumed to be in D in de Swart \textit{et al}. Recall from above though that I proposed that singular \textit{ein} is in ArtP, and this is what I assume for unmodified predicative nominals (30c). Note that both (30b) and (30c) involve REL. Thus, the minimal structures of predicate nominals are as follows (also Borer 2005: 67 fn. 4, Munn \& Schmitt 2005: 827,\textbf{ }cf. Déprez 2005: 866):\footnote{\ De Swart \textit{et al}. (2007) make another important claim about these data. They propose that morphological number is actually not specified on nouns. In particular, bare nouns (30a) are number neutral (this is one of the properties mentioned in Section 2.1.1, discussed in more detail in Chapter 7), but nouns with plural morphology (30b) and nouns preceded by \textit{ein} (30c) are specified for plural and singular, respectively (the latter two are called ‘Marked Nominals’ by de Swart \textit{et al}.). Given the presence of NumP in the latter two cases, de Swart \textit{et al}. (2007: 214, 217) take morphological and semantic number to originate in NumP. This is compatible with my assumptions in Chapter 7.}\textstyleFootnoteSymbol{ }
\end{styleStandard}

\begin{styleStandard}
(30) \ \ a.\ \ N\ \ → [ NP ]\ \ \ \ \ \ → CAP
\end{styleStandard}

\begin{styleStandard}
\ \ b.\ \ N-\textit{s} \ \ → [ NumP [ NP ]]\ \ \ \ → REL
\end{styleStandard}

\begin{styleStandard}
\ \ c.\ \ \textit{ein} N\ \ → [ ArtP [ NumP [ NP ]]]\ \ → REL
\end{styleStandard}

\begin{styleStandard}
Given (30a-c), it is clear that unmodified predicate nominals do not involve DPs. Rather, such nominals are of a smaller size; that is, in the absence of adjectival modifiers, predicate nominals preceded by \textit{ein} do not involve DP but ArtP (recall in this regard that \textit{ein} has no feature for definiteness and does not have to move to the DP-level).\footnote{\ In fact, even in the presence of adjectival modifiers, these nominals can be smaller than DP – \textit{ein} could be in Card, which is above AgrP. If this turns out to be tenable, then we could claim that DPs involve arguments, but not predicates (e.g., Borer 2005: 65-66, Déchaine \& Wiltschko 2002: 419, Stowell 1989; cf. also Longobardi 1994: 628).} If so, then this allows me to state that \textit{ein} and REL are in a local relation (i.e., in adjacent phrases in underlying structure).\footnote{\ If REL were to turn out to be in ArtP (at least in the singular cases), both \textit{ein} and REL would be in the same phrase.} To be clear, considering (30), predicate nominals can be of different sizes: NP, NumP, ArtP, and larger structures (if adjectival modifiers are present).
\end{styleStandard}

\begin{styleStandard}
\ \ Comparing (30a) to (30c), I note already here that \textit{ein} indicates the presence of structure beyond NP (Hypothesis 3a) and that \textit{ein} flags the presence of REL (Hypothesis 3b). The latter claim provides a reason for the appearance of \textit{ein}.\footnote{\ Notice that this comes close to one of Ackles’ (1996: 15) tentative suggestions that English \textit{a(n)} might be a dummy that makes the presence of NumP in an indefinite singular count noun phrase identifiable. Assuming the structural sequence of Q D Num N, she proposes that \textit{a(n)} is the minimal marker that is inserted in NumP when the latter is the leftmost empty phrase of a noun phrase.} Thus far, the proposal seems straightforward. 
\end{styleStandard}

\begin{styleStandard}
However, it is not entirely clear which part of de Swart \textit{et al.}’s (2007) account derives the emotive/figurative reading. Although intuitively attractive at first glance, I show that the emotive/figurative reading does, most likely, not derive from kind coercion but from the presence of REL. Furthermore, as already briefly pointed out in Section 2.1.2, not all nouns undergo figurative extension (e.g., \textit{Landwirt} ‘farmer’). Thus, the figurative extension of the literal meaning (e.g., \textit{Bauer} ‘farmer’ {\textgreater} ‘peasant’) must take a certain lexical specification of the head noun into account. Finally, I extend de Swart \textit{et al.}’s (2007) proposal to pronominal DPs (note in this regard that de Swart \textit{et al}. 2007: 215 assume that ordinary DPs like \textit{the child} also involve REL). Consider these points in more detail.
\end{styleStandard}

\begin{styleStandard}
3.1.2. Lexical Features on Nouns
\end{styleStandard}

\begin{styleStandard}
As seen above, certain nouns can only be neutral/literal in meaning (e.g., \textit{Landwirt }‘farmer’), but others can, under certain conditions, be ambiguous (e.g., \textit{Bauer }‘farmer/peasant’). I start with the latter. It is clear from de Swart \textit{et al.}’s (2007) proposal that the emotive/figurative reading is not derived from the direct path involving CAP. If this were the case, a bare role noun in the TP could have an emotive/figurative reading, contrary to fact. This leaves the indirect path where kind coercion and REL are at work. With REL present, NumP will be projected, and the indefinite article will be present in ArtP too. Unlike the bare noun counterpart, \textit{ein Bauer} can have both a neutral/literal and a emotive/figurative reading, which is correct. The question then arises whether kind coercion or REL is responsible for the figurative reading. As we see momentarily, the answer is not entirely straightforward and involves several considerations. I wind up suggesting that in combination with a lexical feature, REL brings about the interpretative extension (\textit{pace} de Swart \textit{et al}. 2007).
\end{styleStandard}

\begin{styleStandard}
\ \ First, considering \textit{ein Landwirt} ‘a farmer’, we observe that an article is present with a role noun. This means that ArtP, NumP, and REL are present. With a structure larger than NP, kind coercion must have occurred. However, this nominal has a neutral/literal meaning only. This means that certain nouns do not undergo extension of their neutral/literal meaning. As such, nouns must be lexically specified as to whether or not they can undergo this interpretative extension. In other words, neither kind coercion nor REL automatically invokes the figurative extension.
\end{styleStandard}

\begin{styleStandard}
\ \ Second, when applied to a person, names for animals (e.g., \textit{Schwein} ‘pig’) can only have an emotive/figurative meaning. Since these nouns are kind nouns to begin with, they do not undergo kind coercion. As such and making minimal assumptions, REL must be responsible for this extension in meaning (‘pig-like person/swine’).\footnote{\ Alternatively, it is possible to suggest here that a null predicate, call it LIKE, is responsible for the extension in meaning (cf. den Dikken’s 2006: 178 predicate SIMILAR in the comparative binominals). However, I proceed on minimal assumptions; that is, I assume that REL brings about the figurative extension.} However, REL performing this task cannot be the whole story. Certain other kind nouns (e.g., \textit{Katze} ‘cat’) do not seem to be able to undergo this extension easily or at all. This makes these kind nouns similar to the role noun \textit{Landwirt} (see also the word pairs in Footnote Error: Reference source not found). 
\end{styleStandard}

\begin{styleStandard}
\ \ I propose that certain vocabulary items (\textit{Bauer}, \textit{Schwein}) involve a lexical feature that allows figurative extension; other nouns (\textit{Landwirt}, \textit{Katze}) do not have this feature and cannot undergo this extension. With this in mind, I make the plausible assumption that \textit{Bauer} and \textit{Schwein} are similar in the relevant way (recall that the literal meaning of \textit{Schwein} ‘pig’ is excluded for pragmatic reasons when the relevant entity is anatomically not a pig). For concreteness, I assume that the relevant lexical items are specified as [+figurative] if they are able to undergo the extension. Thus, while both \textit{Bauer} and \textit{Schwein} are specified with this feature, \textit{Landwirt} and \textit{Katze} are not. Given that \textit{ein Bauer} can, for many speakers, also have a neutral/literal meaning, I assume that figurative extension is not obligatory (unless forced by the pragmatics).
\end{styleStandard}

\begin{styleStandard}
\ \ To be clear then, I basically follow de Swart \textit{et al}. (2007) with the refinement that figurative extension is not obligatory, that it is due to REL (and not kind coercion), and that it is constrained by the feature [+figurative] present on certain nouns.\footnote{\ Notice that kind coercion is still needed to map a role noun to a kind noun, so that the noun can combine with REL.} Note again that kind coercion only applies to role nouns but figurative extension to both role and kind nouns. In other words, figurative extension does not entail kind coercion. However, figurative extension does yield a special case of emotiveness. More generally, if REL is indeed responsible for the (optional) figurative extension, then I can continue to claim that \textit{ein} makes no semantic, but only a morpho-syntactic, contribution. I reiterate the proposal that the main functions of \textit{ein} are indicating the size of the structure (Hypothesis 3a) and flagging the presence of an operator (Hypothesis 3b). In other words, flagging REL is the reason why \textit{ein} occurs in these contexts. Before I extend de Swart \textit{et al.}’s (2007) discussion to pronominal DPs, it is important to point out that the full explanation of the nominal cases involves a(nother) pragmatic aspect as part of the account.
\end{styleStandard}

\begin{styleStandard}
3.1.3. Rauh (2004) 
\end{styleStandard}

\begin{styleStandard}
Thus far, I have only considered pragmatically neutral, out-of-the-blue examples for the ordinary reading in the DP (31a). However, as Rauh (2004) discusses in detail, these cases are perfectly fine given an appropriate context. Two of her examples may suffice here (31b-c) (stress is indicated here by capital letters):
\end{styleStandard}

\begin{styleStandard}
(31) \ \ a. ??\ \ \textit{\{ich} / \textit{du\} Linguist}\ \ 
\end{styleStandard}

\begin{styleStandard}
\ \ \ \  \textsubscript{\ }I \ \ \ / you linguist.\textsc{masc} \ 
\end{styleStandard}

\begin{styleStandard}
\ \ b.\ \ \textit{Ihr Literaturwissenschaftler mögt den jetzigen Zustand für angemessen halten, \ \ \ \ \ \ aber ich LINGUIST halte die Linguistik für weit unterrepräsentiert}.
\end{styleStandard}

\begin{styleStandard}
\ \ \ \ ‘You literature scholars may consider the current status quo as adequate but I\ \ \ \ \ \  (linguist) consider linguistics as quite underrepresented.’
\end{styleStandard}

\begin{styleStandard}
\ \ c.\ \ \textit{Wenn noch nicht einmal du LINGUIST die neue Rechtschreibung beherrschst, \ \ \ \ \ \ wer sollte es dann}?
\end{styleStandard}

\begin{styleStandard}
\ \ \ \ ‘If not even you (linguist) have mastered the new spelling rules, who else could \ \ \ \ \ \ do that?’
\end{styleStandard}

\begin{styleStandard}
While I cannot do full justice to all the facets of her account here (e.g., the different stress patterns involved), Rauh’s basic proposal is based on the different deictic qualities of the personal pronouns involved and two Gricean maxims (Grice 1975, 1978). 
\end{styleStandard}

\begin{styleStandard}
\ \ As a rule, \textit{ich} ‘I’ and \textit{du} ‘you(\textsc{sgl})’ are disambiguously specified in their deixis such that the identity of the person concerned is clear in a given context. Additional restrictive information provided by an NP complement (e.g., \textit{Linguist}) is not needed. In fact, by Grice’s maxim of quantity (informally: “be as informative as required but not more”), it is redundant and thus yields a marked string. Given a different, more involved context as in (31b-c) above, such an NP complement may make a relevant contribution, and as such it is allowed by Grice’s maxim of relation (informally: “be relevant”). Specifically, this contribution involves a contrastive or emotive component; compare (31b) to (31c). With this in mind, emotive NP complements such as \textit{Idiot} ‘idiot’ are always relevant and thus always possible. Finally, plural pronouns are less deictically specified and thus allow restrictive material in the complement position. I relate Rauh’s (2004) proposal for the DP to de Swart \textit{et al}.’s (2007) account of the TP.
\end{styleStandard}

\begin{styleStandard}
\ \ As I discuss in detail in Chapter 7, all DPs including pronominal DPs must involve NumP, and as such REL is always present there. With de Swart \textit{et al}.’s discussion of the clause in mind, this makes both a neutral and an emotive noun, at least in principle, possible inside the complement position of a pronominal determiner. Now, recalling Rauh’s (2004) pragmatic account of singular DPs, only an emotive complement remains possible in pragmatically neutral, out-of-the-blue contexts. In contrast, singular DPs in a different, appropriate context or plural DPs in general can have neutral or emotive NP complements. To be clear, then, the explanation of pronominal DPs involves a straightforward extension of Swart \textit{et al.} (2007), once we assume that NumP (and thus REL) is always present there and that the explanation must also involve pragmatic considerations. Conversely, predicate nominals in TPs make an assertion, and as such, they provide informative and relevant information. Consequently, they are in agreement with Rauh’s (2004) assumptions, here extended to TPs. Finally, combining these two proposals also explains the fact that in pragmatically neutral, out-of-the-blue contexts, there are no true minimal pairs between pronominal DPs and copular TPs when both involve bare predicate nouns (for more details, see Section 3.3). 
\end{styleStandard}

\begin{styleStandard}
\ \ In the next section, I turn to the specifics of my proposal. I focus on the cases in the singular where I have identified most restrictions and where \textit{ein} is present in emotive contexts. Before doing so, I summarize in Table 5 the cases in the singular (without \textit{als} ‘as’), discussed in this and the previous section. Note that the first column indicates the inherent (non-)emotiveness of the head noun but that the second column shows the readings that noun may take on in DPs and TPs. These readings are exemplified in the third and fourth columns:
\end{styleStandard}

\begin{styleStandard}
Table 5: Nouns and their Readings in the DP and TP
\end{styleStandard}

\begin{flushleft}
\begin{tabular}{|m{0.63655984in}|m{1.3434598in}|m{1.7337599in}|m{2.62126in}|}

\hline
Noun &
Reading &
\centering DP &
\centering\arraybslash TP\\\hline
Neutral &
Literal &
{\centering\itshape du LANDWIRT\par}

{\centering ‘you farmer’\par}

\centering (stressed - contrastive) &
{\centering \textit{Du bist (‘n) Landwirt}.\par}

\centering\arraybslash ‘You are a farmer.’\\\hline
Emotive &
Emotive &
{\centering\itshape du Idiot\par}

\centering ‘you idiot’ &
{\centering\itshape Du bist ‘n Idiot!\par}

\centering\arraybslash ‘You are an Idiot!’\\\hline
Neutral  &
Emotive/figurative [comparative] &
{\centering\itshape du Schwein/Bauer\par}

\centering ‘you idiot/peasant’ &
{\centering\itshape Du bist ‘n Schwein/Bauer!\par}

\centering\arraybslash ‘You are an idiot/peasant!’\\\hline
Neutral  &
Emotive  &
{\centering\itshape du LANDWIRT\par}

{\centering ‘you farmer’\par}

\centering (stressed - emotive) &
{\centering\itshape Du bist vielleicht ‘n Landwirt!\par}

\centering\arraybslash ‘You are some farmer!’\\\hline
\end{tabular}
\end{flushleft}
\begin{styleStandard}
Note again that all the emotive readings have a determiner including \textit{ein} in singular predicate nominals. However, it is clear from the optionality of \textit{ein} in clausal cases like \textit{Du bist (‘n) Landwirt} that the presence of \textit{ein} does not necessarily lead to emotiveness. This fact is consistent with the proposal that \textit{ein} is semantically vacuous. This means that in instances where \textit{ein} is present in emotive contexts, \textit{ein} must be due to a different reason. I proposed above that \textit{ein} flags the presence of a null operator. With this in place, I turn to the explanation of the individual readings in the DP and TP starting with the – what I have called – ordinary reading. 
\end{styleStandard}

\begin{styleStandard}\itshape
3.2.\ \ Ordinary Reading
\end{styleStandard}

\begin{styleStandard}
I propose that the ordinary reading involves direct predication. What I mean by that is that the pronoun and noun combine with one another without invoking a figurative extension of the literal meaning of the head noun. This is different from the comparative reading discussed in the next section. First, I address the nominal domain and then the clausal one.
\end{styleStandard}

\begin{styleStandard}
3.2.1. Ordinary Reading in the DP
\end{styleStandard}

\begin{styleStandard}
As discussed in Chapter 3, I assume that pronominal DPs consist of a head noun and a pronominal determiner. To keep the exposition simple, I abstract away from the lower base position of the determiner and its movement to the DP-level. I assume here and argue below that NumP is sandwiched between the head noun and the pronominal determiner. As such, pronominal DPs exhibit a regular DP structure (see Chapter 1, Section 4). Consider the example in (32a) and its structural analysis in (32b):
\end{styleStandard}

\begin{styleStandard}
(32) \ \ a.\ \ \textit{du }\textit{\textsubscript{\ \ \ }}\textit{Idiot }
\end{styleStandard}

\begin{styleStandard}
\ \ \ \ you idiot.\textsc{masc}
\end{styleStandard}

\begin{styleStandard}
\ \ \ \ ‘you (idiot)’
\end{styleStandard}

\begin{styleStandard}
\ \ b. \textit{Ordinary Reading in DP}
\end{styleStandard}

\begin{styleStandard}
\ \ \ \ DP\textsubscript{{\textless}e{\textgreater}}
\end{styleStandard}

\begin{styleStandard}
[Warning: Draw object ignored][Warning: Draw object ignored]\ \ \ \ \ \ \ \ \ \ (Funtional Application)
\end{styleStandard}

\begin{styleStandard}
\ \ \textit{du}\textsubscript{{\textless}{\textless}e,t{\textgreater}e{\textgreater}} \ \ \ \ \ \ NumP\textsubscript{{\textless}e,t{\textgreater}}
\end{styleStandard}

\begin{styleStandard}
[Warning: Draw object ignored][Warning: Draw object ignored]\ \ \ \ \ \ \ \ \ \ \ \ (Functional Application)
\end{styleStandard}

\begin{styleStandard}
\ \  \ \ \ \ Num\textsubscript{REL{\textless}e{\textless}e,t{\textgreater}{\textgreater}} \ \ \ \ \ \ NP
\end{styleStandard}

\begin{styleStandard}
\ \ \ \  \ \ \  \ \ \ \ \ \ \ \ \textit{Idiot}\textsubscript{{\textless}e{\textgreater}} \ \ \ \ \ \ 
\end{styleStandard}

\begin{styleStandard}
I turn to the semantics. With Section 3.1.1 in mind, I restrict myself to giving the semantic types of the relevant elements and how they are combined. This is shown in (32b). Putting it into words and starting at the bottom of the tree, the kind noun \textit{Idiot} ‘idiot’ combines with the REL operator in NumP to return a predicate nominal. I take Postal’s (1966) claim that pronouns are determiners at face value. In other words, I assume that \textit{du} ‘you(\textsc{sgl})’ is a pronominal determiner that is semantically similar to an ordinary definite determiner.\footnote{\ In Chapter 7, I discuss the ungrammaticality of cases like \textit{*du ‘n Idiot} ‘(lit.) you an idiot’. I assume that pronominal determiners can flag the presence of REL preventing the occurrence of \textit{ein }(also Section 3.4.1). This makes these types of determiners similar to ordinary determiners like definite articles (see de Swart \textit{et al}. 2007: 215).} Given that, Functional Application can apply to the predicate nominal and the pronominal determiner resulting in an entity with the following reading: the unique x (informally addressed) such that x is an idiot. The derivation is the same for \textit{Mann} ‘man’ but requires a special pragmatic context (Rauh 2004). Similar assumptions hold for \textit{Landwirt} ‘farmer’, but the latter undergoes kind coercion. Since \textit{Landwirt} is not marked by [+figurative], figurative extension does not occur.
\end{styleStandard}

\begin{styleStandard}
\ \ Before I discuss the clausal cases, I need to be more explicit about pronouns that have no noun as their complement. Similar to what we have just seen, I assume that they are not simply “intransitive” but more complex. In particular, consistent with Chapter 3, Section 5, I assume that there is actually a head noun. I suggest that this head is an unpronounced kind noun, indicated below as \textit{e}\textit{\textsubscript{N}} (cf. PERSON in Rauh 2004: 89). The latter combines with REL to give a predicate nominal. This complex null element, semantically REL(\textit{e}\textit{\textsubscript{N}}), can now function as an argument to the pronominal functor resulting in an entity:
\end{styleStandard}

\begin{styleStandard}
(33)\ \ \textit{Pronouns Without Overt Complement}
\end{styleStandard}

\begin{styleStandard}
\ \ \ \ DP\textsubscript{{\textless}e{\textgreater}}
\end{styleStandard}

\begin{styleStandard}
[Warning: Draw object ignored][Warning: Draw object ignored]
\end{styleStandard}

\begin{styleStandard}
\ \ \textit{du}\textsubscript{{\textless}{\textless}e,t{\textgreater}e{\textgreater}} \ \ \ \ \ \ NumP\textsubscript{{\textless}e,t{\textgreater}}
\end{styleStandard}

\begin{styleStandard}
[Warning: Draw object ignored][Warning: Draw object ignored]
\end{styleStandard}

\begin{styleStandard}
\ \  \ \ \ \ Num\textsubscript{REL{\textless}e{\textless}e,t{\textgreater}{\textgreater}} \ \ \ \ \ \ NP
\end{styleStandard}

\begin{styleStandard}
\ \ \ \  \ \ \  \ \ \ \ \ \ \ \ \ \ \textit{e}\textit{\textsubscript{N}}\textsubscript{{\textless}e{\textgreater}} \ \ \ \ \ \ 
\end{styleStandard}

\begin{styleStandard}
Note that this is the same abstract structure and derivation as in (32b) above.
\end{styleStandard}

\begin{styleStandard}
3.2.2. Ordinary Reading in the TP
\end{styleStandard}

\begin{styleStandard}
We saw in Section 3.1.1 that predicate nominals can be of different sizes: NP, NumP, ArtP, or larger (if an adjectival modifier is present). As proposed in the previous subsection, pronominal DPs involve regular structures that consist of NPs, NumPs, ArtPs, and other phrases. I propose that predicate nominals in copular TPs are different. Unlike DPs, I claim that predicate nominals in copular clauses do not necessarily involve NumPs, ArtPs, or other phrases – they can just be NPs. Now, similar to the DP, the ordinary reading in the clause does not involve figurative extension either. Adopting fairly traditional tree representations, I assume that the clause involves a Predication Phrase (PrP, Bowers 1993) where the head Pr selects NP.\footnote{\ There are other possibilities; for instance, the auxiliary could select a Small Clause (e.g., Hoekstra 1984: 231). Note also that Bowers (1993) employs PrP in the DP. However, I follow here the more familiar structural view of DPs already discussed above. The resulting differences in names for functional phrases should not be taken as an indication that the DP and TP are not parallel.} The underlying structure of (34a) is given in (34b) (for simplicity, I abstract away here from further movements to the left, see Chapter 7):
\end{styleStandard}

\begin{styleStandard}
(34) \ \ a.\ \ \textit{Du \ bist Bauer}.
\end{styleStandard}

\begin{styleStandard}
\ \ \ \ you are \ farmer.\textsc{masc} 
\end{styleStandard}

\begin{styleStandard}
\ \ \ \ ‘You are a farmer.’
\end{styleStandard}

\begin{styleStandard}
\ \ b. \textit{Ordinary Reading in TP with Role Nouns}
\end{styleStandard}

\begin{styleStandard}
\ \ \ \ PrP\textsubscript{{\textless}t{\textgreater}}
\end{styleStandard}

\begin{styleStandard}
[Warning: Draw object ignored][Warning: Draw object ignored]
\end{styleStandard}

\begin{styleStandard}
\ \ \ \ [\textit{du }REL e\textsubscript{N}]\textsubscript{{\textless}e{\textgreater}}\ \ Pr’\textsubscript{{\textless}e,t{\textgreater}}
\end{styleStandard}

\begin{styleStandard}
[Warning: Draw object ignored][Warning: Draw object ignored]
\end{styleStandard}

\begin{styleStandard}
\ \ \ \ \textit{bist}\ \ \ \ NP\textsubscript{CAP{\textless}e{\textless}e,t{\textgreater}{\textgreater}}
\end{styleStandard}

\begin{styleStandard}
\ \ \ \ \ \ \ \  \ {\textbar}
\end{styleStandard}

\begin{styleStandard}
\textit{\ \ \ \ \ \  \ \ \ \ \ \ \ \ }\textit{\textsubscript{\ }}\textit{Bauer}\textsubscript{{\textless}e{\textgreater}}
\end{styleStandard}

\begin{styleStandard}
As to the semantics, \textit{Bauer} is a role noun, and unlike kind nouns it can combine with CAP. The latter is located in NP and brings about a predicate nominal. Above, we saw that “intransitive” pronouns like \textit{du} ‘you(\textsc{sgl})’ actually involve pronominal DPs; that is, they involve entities enclosed by square brackets in (34b). Recalling that the copula is semantically vacuous, the pronominal DP and the predicate nominal combine by Functional Application to yield a truth value. Note now that (34a) cannot have the emotive/figurative reading ‘peasant’. This is so because CAP (but not REL) brought about the predicate nominal involving \textit{Bauer}. Finally, in order to account for the full grammaticality of these types of examples, observe that the predicate noun makes an assertion about the subject, and as such it is informative and relevant from a pragmatic point of view. The same holds if we exchange \textit{Bauer} with \textit{Landwirt} ‘farmer’.
\end{styleStandard}

\begin{styleStandard}
Finally, it is worth mentioning again that the combination of “\textit{ein} + N” can also result in ordinary readings (35a). With \textit{Idiot }‘idiot’ an emotive kind noun, it combines with REL in NumP bringing about \textit{ein }in ArtP. The remainder of the derivation proceeds as above and is illustrated in (35b): 
\end{styleStandard}

\begin{styleStandard}
(35) \ \ a.\ \ \textit{Du \ bist ‘n \ Idiot.}
\end{styleStandard}

\begin{styleStandard}
\ \ \ \ you are \ \ an idiot.\textsc{masc} 
\end{styleStandard}

\begin{styleStandard}
\ \  \ \ \ \ \ \ \ \ \ ‘You are an idiot.’
\end{styleStandard}

\begin{styleStandard}
\ \ b. \textit{Ordinary Reading in TP with Kind Nouns}
\end{styleStandard}

\begin{styleStandard}
\ \ \ \ PrP\textsubscript{{\textless}t{\textgreater}}
\end{styleStandard}

\begin{styleStandard}
[Warning: Draw object ignored][Warning: Draw object ignored]
\end{styleStandard}

\begin{styleStandard}
\ \ \ \ [\textit{du }REL e\textsubscript{N}]\textsubscript{{\textless}e{\textgreater}}\ \ Pr’
\end{styleStandard}

\begin{styleStandard}
[Warning: Draw object ignored][Warning: Draw object ignored]
\end{styleStandard}

\begin{styleStandard}
\ \ \ \ \textit{bist}\ \  \ \ \ \ \ \ \ \ \ ArtP
\end{styleStandard}

\begin{styleStandard}
[Warning: Draw object ignored][Warning: Draw object ignored]
\end{styleStandard}

\begin{styleStandard}
\ \ \ \ \ \ \textit{‘n}\ \  \ \ \ \ \ \ \ NumP\textsubscript{{\textless}e,t{\textgreater}}
\end{styleStandard}

\begin{styleStandard}
[Warning: Draw object ignored][Warning: Draw object ignored]
\end{styleStandard}

\begin{styleStandard}
\ \ \ \ \ \  \ \ \ \ Num\textsubscript{REL{\textless}e{\textless}e,t{\textgreater}{\textgreater}}\ \ NP
\end{styleStandard}

\begin{styleStandard}
\ \ \ \ \ \ \ \ \ \  \ \ \ \ \ \ \ \ \ \textit{Idiot}\textsubscript{{\textless}e{\textgreater}}
\end{styleStandard}

\begin{styleStandard}
The same holds if the noun involves \textit{Mann} ‘man’. As to \textit{ein}, note again that this element does not have an emotivizing function (e.g., \textit{ein Mann} ‘a man’); that is, \textit{ein} is an expletive element and does not participate in the semantic derivation. The ambiguous noun \textit{Bauer} ‘farmer/peasant’ preceded by \textit{ein} will be discussed in the next section (comparative readings). This is also where I comment on the optional presence of \textit{ein} in the ordinary reading in the TP (e.g., \textit{Du bist (‘n) Bauer}. ‘You are a farmer.’).
\end{styleStandard}

\begin{styleStandard}
\ \ To summarize, the ordinary reading is emotive in the nominal domain but neutral or emotive in the clausal domain. Both cases involve direct predication; that is, there is no figurative extension in the DP (i.e., the head noun is inherently emotive), and the TP involves a role noun in combination with CAP or an emotive or neutral kind noun combining with REL. Next, I discuss cases with figurative extension.
\end{styleStandard}

\begin{styleStandard}\itshape
3.3.\ \ Comparative Reading
\end{styleStandard}

\begin{styleStandard}
Before I provide the structures of the relevant DPs and TPs, I start with some general considerations.
\end{styleStandard}

\begin{styleStandard}
3.3.1. Preliminaries
\end{styleStandard}

\begin{styleStandard}
As a point of departure, I begin with more complex comparative structures. It is well documented that an attributive adjective and its head noun must agree in phi-features yielding concord. This means that the adjectives in (36) below are in the extended projection of their following noun (Chapter 1, Section 4). In Chapter 3, Section 5, I proposed that pronominal DPs involve canonical structures where the adjective is in the specifier just below the determiner:
\end{styleStandard}

\begin{styleFootnote}
(36) \ \ a.\ \ \textit{ihr \ \ \ \ \ \ \ \ \ \ \ \ \ \ dumm-en \ \ Schweine}
\end{styleFootnote}

\begin{styleFootnote}
\ \ \ \ you\textsc{.pl.nom} stupid-\textsc{wk} pigs
\end{styleFootnote}

\begin{styleFootnote}
\ \ \ \ ‘you stupid idiots’\ \ \ \ 
\end{styleFootnote}

\begin{styleFootnote}
b.\ \ \textit{mir \ \ \ \ \ \ \ \ \ \ \ \ \ \ groß-en \ \ Gans}
\end{styleFootnote}

\begin{styleFootnote}
\ \ \ \ me\textsc{.sgl.dat} great-\textsc{wk} goose.\textsc{fem} \ \ \ \ \ \ \ \ \ \ 
\end{styleFootnote}

\begin{styleFootnote}
\ \ \ \ ‘me (stupid idiot)’
\end{styleFootnote}

\begin{styleStandard}
In other words, these comparative constructions have the pronominal determiner, the adjective, and the noun in a regular DP structure, just like with the ordinary reading. I proceed with examples involving unmodified nouns.
\end{styleStandard}

\begin{styleStandard}
3.3.2. Comparative Reading in the DP
\end{styleStandard}

\begin{styleStandard}
Recall that I proposed that the comparative reading involves a figurative extension of the neutral/literal meaning of the head noun; for example, the neutral/literal meaning of \textit{Bauer} ‘farmer’ is extended to ‘farmer-like person’ or ‘peasant’. I proposed that this is due to REL. This fits well with the assumption that NumP is present in the DP. The example in (37a) is derived as in (37b):
\end{styleStandard}

\begin{styleStandard}
(37) \ \ a.\ \ \textit{du \ \ Bauer}
\end{styleStandard}

\begin{styleStandard}
\ \ \ \ you peasant.\textsc{masc}
\end{styleStandard}

\begin{styleStandard}
\ \  \ \ \ \ \ \ \ \ \ \ \ ‘you (peasant)’
\end{styleStandard}

\begin{styleStandard}
\ \ \ \ \#‘you (farmer)’
\end{styleStandard}

\begin{styleStandard}
\ \ b. \textit{Comparative Reading in DP}
\end{styleStandard}

\begin{styleStandard}
\ \ \ \ DP\textsubscript{{\textless}e{\textgreater}}
\end{styleStandard}

\begin{styleStandard}
[Warning: Draw object ignored][Warning: Draw object ignored]
\end{styleStandard}

\begin{styleStandard}
\ \ \textit{du}\textsubscript{{\textless}{\textless}e,t{\textgreater}e{\textgreater}} \ \ \ \ \ \ NumP\textsubscript{{\textless}e,t{\textgreater}}
\end{styleStandard}

\begin{styleStandard}
[Warning: Draw object ignored][Warning: Draw object ignored]
\end{styleStandard}

\begin{styleStandard}
\ \  \ \ \ \ \ Num\textsubscript{REL{\textless}e{\textless}e,t{\textgreater}{\textgreater}}\ \ NP
\end{styleStandard}

\begin{styleStandard}
\ \ \ \ \ \  \ \ \ \ \ \ \ \ \textsubscript{\ }\textit{Bauer}\textsubscript{{\textless}e{\textgreater}}
\end{styleStandard}

\begin{styleStandard}
As for the interpretation, the role noun \textit{Bauer} undergoes kind coercion and combines with REL in NumP. As the noun is lexically marked as [+figurative], it can undergo figurative extension. Unlike the neutral/literal reading, this extension yields emotiveness and is felicitous in pragmantically neutral, out-of-the-blue contexts (Rauh 2004). Next, NumP combines with the pronoun resulting in an entity with the following reading: the unique x (informally addressed) such that x is, in some way, similar to a farmer. Similar considerations apply to \textit{Schwein} ‘pig’, with the proviso that the interpretation can only be ‘pig-like person’ or ‘swine’ for pragmatic reasons.
\end{styleStandard}

\begin{styleStandard}
3.3.3. Comparative Reading in the TP
\end{styleStandard}

\begin{styleStandard}
If the above discussion of the DP is tenable, then the comparative reading in the TP should involve a similar account. The example in (38a) is derived as in (38b):
\end{styleStandard}

\begin{styleStandard}
(38) \ \ a.\ \ \textit{Du \ bist ‘n Bauer.}
\end{styleStandard}

\begin{styleStandard}
\ \ \ \ you are \ \ a peasant/farmer.\textsc{masc}
\end{styleStandard}

\begin{styleStandard}
\ \ \ \ ‘You are a peasant.’
\end{styleStandard}

\begin{styleStandard}
\ \ \ \ \%‘You are a farmer.’
\end{styleStandard}

\begin{styleStandard}
\ \ b. \textit{Comparative Reading in TP}
\end{styleStandard}

\begin{styleStandard}
\ \ \ \ PrP\textsubscript{{\textless}t{\textgreater}}
\end{styleStandard}

\begin{styleStandard}
[Warning: Draw object ignored][Warning: Draw object ignored]
\end{styleStandard}

\begin{styleStandard}
\ \ \ \ [\textit{du }REL e\textsubscript{N}]\textsubscript{{\textless}e{\textgreater}}\ \ Pr’
\end{styleStandard}

\begin{styleStandard}
[Warning: Draw object ignored][Warning: Draw object ignored]
\end{styleStandard}

\begin{styleStandard}
\ \ \ \ \textit{bist}\ \  \ \ \ \ \ \ \ \ \ ArtP
\end{styleStandard}

\begin{styleStandard}
[Warning: Draw object ignored][Warning: Draw object ignored]
\end{styleStandard}

\begin{styleStandard}
\ \ \ \ \ \ \textit{‘n}\ \  \ \ \ \ \ \ \ NumP\textsubscript{{\textless}e,t{\textgreater}}
\end{styleStandard}

\begin{styleStandard}
[Warning: Draw object ignored][Warning: Draw object ignored]
\end{styleStandard}

\begin{styleStandard}
\ \ \ \ \ \  \ \ \ \ Num\textsubscript{REL{\textless}e{\textless}e,t{\textgreater}{\textgreater}}\ \ NP
\end{styleStandard}

\begin{styleStandard}
\ \ \ \ \ \ \ \ \ \  \ \ \ \ \ \ \ \ \textit{Bauer}\textsubscript{{\textless}e{\textgreater}}
\end{styleStandard}

\begin{styleStandard}
Like above, \textit{Bauer} undergoes kind coercion and then combines with REL in NumP. Structurally, this brings about \textit{ein} ‘a’ in ArtP. Semantically, this complex element, namely REL(\textit{Bauer}), combines with the pronominal DP to return a truth value. Note that \textit{Bauer} has the option of keeping its neutral/literal meaning or undergoing figurative extension. This is so because this noun is specified [+figurative], but the figurative extension by REL is not obligatory. This derives the two readings in (38a) above. The same basic derivation as in (38b) applies to \textit{Schwein} ‘pig’, with the same proviso mentioned for the DP above.
\end{styleStandard}

\begin{styleStandard}
Returning to an issue left open in the last section, notice that the ordinary reading of the role noun involves the indirect path in (38), and as such it entails more structure and more operations than the direct path (Section 3.1.1). This might explain why the combination of \textit{ein} and \textit{Bauer} in its neutral/literal meaning is somewhat less easily available or even absent for some speakers. 
\end{styleStandard}

\begin{styleStandard}
\ \ Finally, recall also that the modal particle \textit{vielleicht} ‘really’ can intensify the comparative reading resulting in an interpretation close to ‘really like’. Consonant with the above discussion and adding an adjective to the predicate nominal (39a), the noun and its modifier are in their regular position (39b): 
\end{styleStandard}

\begin{styleStandard}
(39)\ \ a.\ \ \textit{Du \ bist vielleicht ‘n echter Bauer!}\ \ 
\end{styleStandard}

\begin{styleStandard}
\ \ you are \ \textsc{prt} \ \ \ \ \ \ \ \ \ \ a \ real \ \ \ \ peasant.\textsc{masc}
\end{styleStandard}

\begin{styleStandard}
\ \ ‘You are really like a peasant!’
\end{styleStandard}

\begin{styleStandard}
\ \ b. \ \ \ \ \ \ CardP
\end{styleStandard}

\begin{styleStandard}
[Warning: Draw object ignored][Warning: Draw object ignored]
\end{styleStandard}

\begin{styleStandard}
\ \ \ \ \ \ \textit{‘n} \ \  \ \ \ \ \ \ \ \ AgrP\textsubscript{{\textless}e,t{\textgreater}}
\end{styleStandard}

\begin{styleStandard}
[Warning: Draw object ignored][Warning: Draw object ignored]\ \ \ \ \ \ \ \ \ \ \ \ (Predicate Modification)
\end{styleStandard}

\begin{styleStandard}
\ \  \ \ \ \ \ \ \ \textit{echter}\textsubscript{{\textless}e,t{\textgreater}}\ \  \ \ \ \ \ \ \ NumP\textsubscript{{\textless}e,t{\textgreater}}
\end{styleStandard}

\begin{styleStandard}
[Warning: Draw object ignored][Warning: Draw object ignored]
\end{styleStandard}

\begin{styleStandard}
\ \ \ \  \ \ \ \ Num\textsubscript{REL{\textless}e{\textless}e,t{\textgreater}{\textgreater}}\ \ NP
\end{styleStandard}

\begin{styleStandard}
\ \ \ \ \ \ \ \  \ \ \ \ \ \ \ \ \textit{Bauer}\textsubscript{{\textless}e{\textgreater}}
\end{styleStandard}

\begin{styleStandard}
Note that the adjectival predicate and the nominal predicate combine by Predicate Modification.
\end{styleStandard}

\begin{styleStandard}
\ \ Taking stock, the comparative reading in the nominal and clausal domain involves the operator REL in NumP. In combination with a [+figurative] role noun, this results in an emotive/figurative reading in the DP and in the TP (although a neutral/literal reading is not excluded in the latter domain). Note now that the interaction between de Swart \textit{et al}. (2007) and Rauh (2004) also affords us an explanation of why there are no true minimal pairs of the type \textit{du Noun}\textit{\textsubscript{1}} ‘you Noun’ and \textit{Du bist Noun}\textit{\textsubscript{1}} ‘You are a Noun’ in pragmatically neutral contexts. 
\end{styleStandard}

\begin{styleStandard}
\ \ Specifically, all pronominal DPs involve NumP and thus REL. The latter may bring about the emotive/figurative extension of the meaning of the head noun. Now, Rauh’s pragmatic proposal explains why only inherently emotive kind nouns (\textit{Idiot}), [+figurative] kind nouns (\textit{Schwein}), and [+figurative] role nouns (\textit{Bauer}) are possible in the DP. In contrast, de Swart \textit{et al.} argue that bare predicate nouns in the TP only involve NP and thus CAP. With REL absent here, this only allows the occurrence of role nouns, with the feature [+figurative] (\textit{Bauer}) or without (\textit{Landwirt}). Due to the absence of REL, \textit{Bauer} can only have the neutral/literal meaning like \textit{Landwirt}. To be clear, then, Rauh’s proposal forces the bare noun to be emotive in the DP, and de Swart \textit{et al}.’s analysis explains why the bare noun can only be neutral in the TP. If so, the absence of minimal pairs should not be taken as an argument that DPs and TPs are fundamentally different after all – the difference is explained by the interplay of the two proposals above, which involve pragmatic considerations. I turn to the last reading.
\end{styleStandard}

\begin{styleStandard}\itshape
3.4.\ \ Capacity Reading
\end{styleStandard}

\begin{styleStandard}
Before I discuss the detailed analyses of the DP and TP, I need to provide some background information motivating the relevant structures.
\end{styleStandard}

\begin{styleStandard}
3.4.1. Preliminaries
\end{styleStandard}

\begin{styleStandard}
We have seen that DPs involving \textit{als} ‘as’ have a (neutral) capacity reading (40a). The latter can be paraphrased as ‘in the capacity of / with regard to being’. In the TP (40b), the modal particle \textit{vielleicht} ‘really’ invokes an emotive capacity reading similar to ‘bad/good in the capacity of / with regard to being’, and it intensifies the comparative reading (for the structure of the latter, see previous section):\footnote{\ De Swart \textit{et al}. (2007: 218) assume that the combination of “\textit{als }+ NP” is a verb modifier. Note though that this does not work for \textit{als}{}-nominals that are part of pronominal DPs: given the Verb Second Constraint in German, the latter clearly involve constituents: \textit{Du als Arzt solltest das nicht machen} ‘You as a doctor should not do that’.}
\end{styleStandard}

\begin{styleStandard}
(40) \ \ a.\ \ \textit{du \ \ als Bauer}
\end{styleStandard}

\begin{styleStandard}
\ \ \ \ you as \ farmer.\textsc{masc}
\end{styleStandard}

\begin{styleStandard}
\ \ \ \ ‘you as a farmer’
\end{styleStandard}

\begin{styleStandard}
\ \ \ \ \#‘you as a peasant’
\end{styleStandard}

\begin{styleStandard}
\ \ b.\ \ \textit{Du \ bist vielleicht ‘n Bauer!}\ \ \ \ 
\end{styleStandard}

\begin{styleStandard}
\ \ \ \ you are \ \textsc{prt} \ \ \ \ \ \ \ \ \ \ a \ farmer/peasant.\textsc{masc}
\end{styleStandard}

\begin{styleStandard}
\ \ \ \ ‘You are some farmer/peasant!’
\end{styleStandard}

\begin{styleStandard}
Starting with the DP, I make the fairly straightforward proposal that \textit{als} ‘as’ brings about the capacity reading. As to the TP, I would like to suggest that in combination with the modal particle, the null equivalent of \textit{als} (i.e., ALS) derives the corresponding clausal reading. Like all null elements, ALS has to be licensed. I assume that the modal particle is responsible for that. Schematically, we arrive at the following, where the relevant noun phrases of the data in (40a-b) are put in brackets in (41a-b):
\end{styleStandard}

\begin{styleStandard}
(41) \ \ a.\ \ [\textit{du als Bauer}]
\end{styleStandard}

\begin{styleStandard}
\ \ b.\ \ \textit{vielleicht} [\textit{ein }ALS\textit{ Bauer}]
\end{styleStandard}

\begin{styleStandard}
The bracketed strings are of particular relevance below. To determine the syntactic properties of these two capacity constructions, I consider the latter in combination with third-person pronouns. As is well known, these pronouns are syntactically very restrictive and therefore help us find a plausible structural analysis.
\end{styleStandard}

\begin{styleStandard}
\ \ Unlike pronouns of the first and second person, pronouns of the third person cannot directly combine with adjectives and/or overt nouns (see Höhn 2020 and Chapter 3, Section 5). Compare (42a) to (42b-d): 
\end{styleStandard}

\begin{styleStandard}
(42) \ \ a.\ \ \textit{er als Bauer}
\end{styleStandard}

\begin{styleStandard}
\ \ \ \ he as \ farmer.\textsc{masc}
\end{styleStandard}

\begin{styleStandard}
\ \ \ \ ‘he as a farmer’ 
\end{styleStandard}

\begin{styleStandard}
\ \ b. *\ \ \textit{er Gute(r) }
\end{styleStandard}

\begin{styleStandard}
\ \ \ \ he \ good \ \ \ \ 
\end{styleStandard}

\begin{styleStandard}
\ \ c. *\ \ \textit{er Bauer }\textit{\textsubscript{\ }}
\end{styleStandard}

\begin{styleStandard}
\ \ \ \ he farmer.\textsc{masc} \ 
\end{styleStandard}

\begin{styleStandard}
\ \ d. *\ \ \textit{er gute(r) Bauer}
\end{styleStandard}

\begin{styleStandard}
\ \ \ \ he good \ \ \ farmer.\textsc{masc}
\end{styleStandard}

\begin{styleStandard}
However, they can be modified by relative clauses (43a) (see Rauh 2003, Vater 1985). In other words, I take (42a) to be on a par with (43a). Assuming the traditional adjunction analysis for relative clauses, I propose that \textit{als Bauer} is also adjoined to the pronoun (cf. Rauh 2004: 94), schematically illustrated in (43b):
\end{styleStandard}

\begin{styleStandard}
(43)\ \ a.\ \ \textit{er, der \ gerade durch \ \ \ die Tür \ gekommen ist}\ \ 
\end{styleStandard}

\begin{styleStandard}
\ \ \ \ he who just \ \ \ \ \ through the door come \ \ \ \ \ \ \ \ is
\end{styleStandard}

\begin{styleStandard}
\ \ \ \ ‘he, who just came through the door’
\end{styleStandard}

\begin{styleStandard}
\ \ b.\ \ Adjunction-type Analysis
\end{styleStandard}

\begin{styleStandard}
\ \ \ \ \textit{du} [ \textit{als Bauer} ]
\end{styleStandard}

\begin{styleStandard}
Turning to the clausal counterpart (41b), that is, \textit{ein }ALS\textit{ Bauer}, recall from Chapter 2 that \textit{ein} ‘a’ cannot have an ending in certain morpho-syntactic contexts when an adjective and/or a noun follows. This holds independently of the presence or absence of the modal particle. Compare (44a) to (44b-d):
\end{styleStandard}

\begin{styleStandard}
(44) \ \ a.\ \ \textit{Er ist vielleicht ein \ \ \ (guter) Bauer!}
\end{styleStandard}

\begin{styleStandard}
\ \ \ \ he is \ \textsc{prt} \ \ \ \ \ \ \ \ \ \ a/one good \ \ farmer.\textsc{masc}
\end{styleStandard}

\begin{styleStandard}
\ \ \ \ ‘He is some great farmer!’
\end{styleStandard}

\begin{styleStandard}
\ \ b. *\ \ \textit{Er ist (vielleicht) ein-er \ \ \ \ Gute(r) }
\end{styleStandard}

\begin{styleStandard}
\ \ \ \ he is \ \ \ \textsc{prt} \ \ \ \ \ \ \ \ \ \ a/one\textsc{{}-st} good \ \ \ \ 
\end{styleStandard}

\begin{styleStandard}
\ \ c. *\ \ \textit{Er ist (vielleicht) ein-er \ \ \ \ Bauer }\textit{\textsubscript{\ }}
\end{styleStandard}

\begin{styleStandard}
\ \ \ \ he is \ \ \ \textsc{prt} \ \ \ \ \ \ \ \ \ \ a/one-\textsc{st} farmer.\textsc{masc} 
\end{styleStandard}

\begin{styleStandard}
\ \ d. *\ \ \textit{Er ist (vielleicht) ein-er \ \ \ \ gute(r) Bauer}
\end{styleStandard}

\begin{styleStandard}
\ \ \ \ he is \ \ \ \textsc{prt} \ \ \ \ \ \ \ \ \ \ a/one-\textsc{st} good \ \ \ farmer.\textsc{masc}
\end{styleStandard}

\begin{styleStandard}
Rather, inflected \textit{ein} is only possible here when the modifier is a relative clause (a genitive DP, or a PP) or when no relevant element follows at all (45a). Thus, in order to explain the absence of the inflection on \textit{ein }in (41b), I propose that similar to regular adjectives, ALS\textit{ Bauer} is in a specifier. This is shown in simplified form in (45b):
\end{styleStandard}

\begin{styleStandard}
(45)\ \ a.\ \ \textit{Er ist ein-er (, der \ viel \ \ \ \ arbeitet)}.
\end{styleStandard}

\begin{styleStandard}
\ \ \ \ he is \ one-\textsc{st} \ \ who much works
\end{styleStandard}

\begin{styleStandard}
\ \ \ \ ‘He is one that works a lot.’
\end{styleStandard}

\begin{styleStandard}
\ \ b.\ \ Specifier-type Analysis
\end{styleStandard}

\begin{styleStandard}
\ \ \ \ \textit{ein} [ ALS \textit{Bauer} ] \textit{e}\textit{\textsubscript{N}}
\end{styleStandard}

\begin{styleStandard}
\ \ To sum up thus far, we have seen some evidence that \textit{als Bauer} is most likely in an adjoined position and that ALS \textit{Bauer} is in a specifier position. Importantly, de Swart \textit{et al.} (2007: 200) and others have noted that modifiers such as relative clauses and adjectives make optional determiners obligatory. This is particularly clear with role nouns (46a-b), even if such a noun is not pronounced (46c):
\end{styleStandard}

\begin{styleStandard}
(46) \ \ a.\ \ \textit{Er ist *(‘n) Bauer, \ \ \ \ \ \ \ \ \ \ der }\textit{\textsubscript{\ }}\textit{viel \ \ \ arbeitet}.
\end{styleStandard}

\begin{styleStandard}
\ \ \ \ he is \ \ \ \ \ \ a \ \textsubscript{\ }farmer.\textsc{masc} that much works
\end{styleStandard}

\begin{styleStandard}
\ \ \ \ ‘He is a farmer that works a lot.’
\end{styleStandard}

\begin{styleStandard}
\ \ b.\ \ \textit{Er ist *(‘n) guter Bauer}.
\end{styleStandard}

\begin{styleStandard}
\ \ \ \ he is \ \ \ \ \ \ a \ \textsubscript{\ }good farmer.\textsc{masc}
\end{styleStandard}

\begin{styleStandard}
\ \ \ \ ‘He is a good farmer.’
\end{styleStandard}

\begin{styleStandard}
\ \ c.\ \ \textit{Peter} \textit{ist ‘n schlechter Bauer, \ \ \ \ \ \ \ \ \ aber Hans ist *(‘n) guter}.
\end{styleStandard}

\begin{styleStandard}
\ \ \ \ Peter is \ \ \ a bad \ \ \ \ \ \ \ \ \ \ farmer.\textsc{masc} but \ \ Hans is \ \ \ \ \ \ a \ \textsubscript{\ }good 
\end{styleStandard}

\begin{styleStandard}
\ \ \ \ ‘Peter is a bad farmer, but Hans is a good one.’
\end{styleStandard}

\begin{styleStandard}
It is clear then that modifiers in both adjoined and specifier positions require the presene of \textit{ein}. I propose that \textit{ein} is present because the modifiers in (46) combine with their nouns by Predicate Modification. Notice that this operation conjoins elements of type {\textless}e,t{\textgreater}. \ Now, for the noun to be of this type, it must have combined with REL first. As we know from above, this operator implies the presence of NumP and thus \textit{ein}.\footnote{\ For some discussion of why CAP cannot be involved here, see Chapter 7.} Now, assuming that nominals involving a noun and ALS (or \textit{als} ‘as’) are also predicates, I point out that REL is also present, and this explains the presence of \textit{ein} in (45b). 
\end{styleStandard}

\begin{styleStandard}
I briefly return to the analysis of the first case, \textit{du als Bauer} in (43b). Observe that this string involves adjunction just like (46a) above. However, unlike (46a), there is no \textit{ein }in (43b). In order to address this difference, recall the discussion from Section 3.2.1, where \textit{du} itself was proposed to involve a complex structure. Specifically, \textit{du} is a determiner with NumP and a null noun in its complement structure. I update (43b) as in (47a). Notice now that this makes (47a) and (46a) completely parallel in that the adjoined material (in brackets) is preceded by a structurally complex nominal:\footnote{\ To the extent that \textit{ein} can be followed by \textit{als Bauer}, it must have an inflection: \textit{ein*(-er) als Bauer}. This fits well with the adjunction-type analysis of \textit{als Bauer }in (47a).}
\end{styleStandard}

\begin{styleStandard}
(47)\ \ a.\ \ \textit{du} REL e\textsubscript{N} [\textit{als Bauer}]
\end{styleStandard}

\begin{styleStandard}
\ \ b.\ \ \textit{‘n} REL \textit{Bauer} [\textit{der viel arbeitet}]
\end{styleStandard}

\begin{styleStandard}
To explain the absence of \textit{ein} in the first nominal in (47a), I suggest that elements like pronominal determiners can also flag REL. This is compatible with de Swart \textit{et al}.’s assumptions about regular DPs like \textit{the child}. In other words, besides \textit{ein} and the definite article, other determiners can also indicate the presence of REL. 
\end{styleStandard}

\begin{styleStandard}
To sum up, \textit{du als Bauer} involves adjunction of \textit{als Bauer}, and \textit{ein }ALS\textit{ Bauer} contains a complex specifier occupied by ALS \textit{Bauer}. The determiners \textit{du} and \textit{ein} are present due to REL. The absence of the inflection on \textit{ein} is explained by the presence of an overt element in the specifier just below \textit{ein}. With this much in place, I consider the relevant analyses in more detail. I start with the (neutral) capacity reading in the DP.
\end{styleStandard}

\begin{styleStandard}
3.4.2. Capacity Reading in the DP
\end{styleStandard}

\begin{styleStandard}
I have proposed that \textit{du} itself involves a complex structure and that \textit{als Bauer} is adjoined to that structure as in (47a) above. I assume that this type of adjunction is instantiated by a Modifier Phrase (ModP; see Chapter 2, Sections 3.2 and 3.3). I interpret \textit{als} as the head of ModP, and this head takes \textit{Bauer} as a complement. The head \textit{als} brings about the capacity reading of the DP. I derive (48a) as (48b):
\end{styleStandard}

\begin{styleStandard}
(48)\ \ a.\textit{ \ \ du \ \ als Bauer}
\end{styleStandard}

\begin{styleStandard}
\ \ \ \ you as \ farmer.\textsc{masc}
\end{styleStandard}

\begin{styleStandard}
\ \ \ \ ‘you as a farmer’
\end{styleStandard}

\begin{styleStandard}
\ \ \ \ \#‘you as a peasant’
\end{styleStandard}

\begin{styleStandard}
\ \ b. \textit{(Neutral)} \textit{Capacity Reading Licensed DP-internally}
\end{styleStandard}

\begin{styleStandard}
\ \ \ \ \  \ \ \ DP\textsubscript{{\textless}e{\textgreater}}
\end{styleStandard}

\begin{styleStandard}
[Warning: Draw object ignored][Warning: Draw object ignored]
\end{styleStandard}

\begin{styleStandard}
\ \ \textit{du}\textsubscript{{\textless}{\textless}e,t{\textgreater}e{\textgreater}} \ \ \ \ \ \ \ \ \ \ \ NumP\textsubscript{{\textless}e,t{\textgreater}}
\end{styleStandard}

\begin{styleStandard}
[Warning: Draw object ignored][Warning: Draw object ignored]\ \ \ \ \ \ \ \ \ \ \ \ \ \ (Predicate Modification)
\end{styleStandard}

\begin{styleStandard}
\ \  \ \ \ \ \ \ \ NumP\textsubscript{{\textless}e,t{\textgreater}}\ \  \ \ \ \ \ \ \ \ \ \ \ \ \ \ \ \ \ \ ModP\textsubscript{{\textless}e,t{\textgreater}}
\end{styleStandard}

\begin{styleStandard}
[Warning: Draw object ignored][Warning: Draw object ignored][Warning: Draw object ignored][Warning: Draw object ignored]\ \ \ \  \ \ \ \ \ 
\end{styleStandard}

\begin{styleStandard}
\ \ \ \ \ \ Num\textsubscript{REL{\textless}e{\textless}e,t{\textgreater}{\textgreater}} \ \ \ \ NP \ \ \ \ \ \ \textit{als}\ \  \ \ \ \ \ \ \ \ NP
\end{styleStandard}

\begin{styleStandard}
\ \  \ \ \ \ \ \ \ \ \ \ \ \ \ \ \ \ \ \ \ \ \textit{e}\textit{\textsubscript{N}}\textsubscript{{\textless}e{\textgreater}}\ \  \ \ \ \ \ \ \ \ \ \ \ \ \ \ \ \ \ \textit{Bauer}\textsubscript{{\textless}e{\textgreater}}
\end{styleStandard}

\begin{styleStandard}
Observe that the lower NumP and ModP are both of type {\textless}e,t{\textgreater} (I comment on the role of \textit{als} momentarily). I assume that they combine by Predicate Modification. 
\end{styleStandard}

\begin{styleStandard}
I point out that \textit{als-}nominals are special in that \textit{als} also combines with bare kind nouns such as \textit{Mann} ‘man’ in appropriate contexts (49a). In fact, the presence of \textit{ein} leads to awkwardness (49b): 
\end{styleStandard}

\begin{styleStandard}
(49)\ \ a.\ \ \textit{Was \ würdest du \ \ als Mann \ \ \ \ \ \ \ dazu sagen?}
\end{styleStandard}

\begin{styleStandard}
\ \ \ \ what would \ \ \ you as \ man.\textsc{masc}\textsubscript{ }it.to \ say
\end{styleStandard}

\begin{styleStandard}
\ \ \ \ ‘What would you as a male say about this?’
\end{styleStandard}

\begin{styleStandard}
\ \ b. \ ?\ \ \textit{Was \ würdest du \ \ als ‘n Mann \ \ \ \ \ \ \ \ dazu sagen?}
\end{styleStandard}

\begin{styleStandard}
\ \ \ \ what would \ \ \ you as \ \ a \ man.\textsc{masc}\textsubscript{ }it.to \ say
\end{styleStandard}

\begin{styleStandard}
\ \ \ \ ‘What would you as a male say about this?’
\end{styleStandard}

\begin{styleStandard}
Since a kind noun is involved here, CAP cannot be at work in the adjoined structure. Furthermore, if REL were at work here, we would expect that the presence of \textit{ein} is fully grammatical, contrary to fact. I assume that \textit{als} is a general(ized) capacity operator that maps role and kind nouns alike into predicate nominals; that is, \textit{als} is of type {\textless}e{\textless}e,t{\textgreater}{\textgreater}. I discuss this in more detail in Chapter 7. I turn to the data from the clausal domain.
\end{styleStandard}

\begin{styleStandard}
3.4.3. Capacity Reading in the TP
\end{styleStandard}

\begin{styleStandard}
As illustrated many times, adjectives exhibit strong endings after uninflected \textit{ein}. In Chapter 2, I discussed two options to analyze this: either the adjective is in the regular specifier position, and the strong ending follows from the special morpho-syntactic properties of \textit{ein}, or alternatively, the adjective is in a non-canonical position where its ending cannot undergo Impoverishment in the first place. As discussed in that chapter, one such non-canonical position involves the adjective to be deeply embedded in a complex specifier. 
\end{styleStandard}

\begin{styleStandard}
Recall now that the combination of “\textit{vielleicht} + \textit{ein} N” has two readings. I made use of the first option mentioned above (the adjective is in its regular position) with the intensified comparative reading in Section 3.3.3. The second option (the adjective is more deeply embedded) was proposed for ALS-nominals in the preliminary section (see (45b) above). I now turn to the second option in more detail.
\end{styleStandard}

\begin{styleStandard}
\ \ Specifically, I propose that cases with the emotive capacity reading (50a) have the adjective and noun embedded in a specifier position as in (50b). As discussed in Section 3.4.1 above, this complex specifier contains the null element ALS. I propose that similar to the DP, the embedded nominal involves ModP, and I assume that ALS is its head. Again, like the DP, Predicate Modification combines the two nominals, here ModP and Agr’:
\end{styleStandard}

\begin{styleStandard}
(50)\ \ a.\ \ \textit{Du \ bist vielleicht ‘n \ erstaunlicher Bauer!}
\end{styleStandard}

\begin{styleStandard}
\ \ you are \ \textsc{prt} \ \ \ \ \ \ \ \ \ \ an amazing \ \ \ \ \ \ \ \ farmer.\textsc{masc}
\end{styleStandard}

\begin{styleStandard}
\ \ ‘You are really an amazing farmer!’
\end{styleStandard}

\begin{styleStandard}
\ \ b. \textit{Emotive} \textit{Capacity Reading Licensed DP-externally}
\end{styleStandard}

\begin{styleStandard}
\ \ \ \ \  \ \ \ \ \ \ \ CardP
\end{styleStandard}

\begin{styleStandard}
[Warning: Draw object ignored][Warning: Draw object ignored]
\end{styleStandard}

\begin{styleStandard}
\ \  \textsubscript{\ }\textit{‘n}\ \  \ \ \ \ \ \ \ \ \ \ \ \ \ \ \ \ \ \ \ \ \ \ \ \ \ AgrP\textsubscript{{\textless}e,t{\textgreater}}
\end{styleStandard}

\begin{styleStandard}
[Warning: Draw object ignored][Warning: Draw object ignored]\ \ \ \ \ \ \ \ \ \ \ \ \ \ \ \ (Predicate Modification)
\end{styleStandard}

\begin{styleStandard}
\ \  \ \ \ \ \ \ \ \ \ ModP\textsubscript{{\textless}e,t{\textgreater}}\ \  \ \ \ \ \ \ \ \ \ \  \ \ \ \ \ \ \ \ \ \ \ \ \ \ \ Agr’\textsubscript{{\textless}e,t{\textgreater}}
\end{styleStandard}

\begin{styleStandard}
[Warning: Draw object ignored][Warning: Draw object ignored]\ \ \ \ \ \ \ \ \ \  \ \ \ \ \ \ \ \ \ \ \ \ \ \ \ \ \ \ \ {\textbar}\ \ 
\end{styleStandard}

\begin{styleStandard}
\ \ \ \ \ \ \ \ ALS\ \  \ \ \ \ \ \ \ \ \ \ \ AgrP\textsubscript{{\textless}e,t{\textgreater}} \ \ \ \ \ \ \ \ \ \ \ \ \ \ \  NumP\textsubscript{REL}
\end{styleStandard}

\begin{styleStandard}
\ \  \ \ \ \ \ \ \ \ [\textit{erstaunlicher }REL \textit{Bauer}] \ \ \ \ \ \textit{ \ \ \ \ \ \ \ }{\textbar}
\end{styleStandard}

\begin{styleStandard}
\ \ \ \ \ \ \ \  \ \ \ \ \ \ \ \ \ \ \ \ \  \ \ \ \ \ NP
\end{styleStandard}

\begin{styleStandard}
\ \ \ \ \ \ \ \  \ \ \ \ \ \ \ \ \ \ \ \ \ \ \ \ \ \ \ \ \ \ \ \ \ \ \ \textit{e}\textit{\textsubscript{N}}\textsubscript{{\textless}e{\textgreater}}
\end{styleStandard}

\begin{styleStandard}
Recall that ALS is licensed by the modal particle \textit{vielleicht}. Furthermore, \textit{ein} is present due to REL in the matrix nominal, and \textit{ein} is uninflected due to the presence of an overt element in the specifier of the higher AgrP.\footnote{\ Note that the instances with an adjective involve REL in the embedded nominal as in (50b), but I assume that cases without an adjective also involve REL. In other words, ALS always involves REL and thus NumP as part of its complement. Observe in this regard that these cases involve concord in agreement features between \textit{ein} in the matrix nominal and the (adjective and) noun embedded in Spec,AgrP. I take it that the head Mod (ALS) mediates the agreement between these two nominals (also Chapter 8, Section 3.1).} Finally, I assume that the emotive flavor originates with the modal particle.
\end{styleStandard}

\begin{styleStandard}
\ \ For clarity, I review the capacity reading in the nominal and clausal domains adding a few more details in the process. There are two differences between the DP and the TP. First, the DP has a neutral interpretation, but the TP has an emotive one. This difference in emotiveness follows from the need to be restrictive enough in the former case (namely when the speaker singles someone out in a certain capacity, see Section 2.2.1) and the presence of the modal particle in the latter case. 
\end{styleStandard}

\begin{styleStandard}
Second, the indefinite determiner is absent in the DP (\textit{du als Bauer}) but present in the TP (\textit{ein }ALS\textit{ erstaunlicher Bauer }e\textsubscript{N}). I consider two structural domains here, delineated by square brackets in the following examples where the embedded (but not the matrix) nominal is enclosed by the brackets. In the matrix nominal, \textit{ein} is absent in \textit{du }[\textit{als Bauer}] as \textit{du} itself flags REL of the matrix nominal; \textit{ein} is present in \textit{ein }[\textit{erstaunlicher Bauer}] e\textsubscript{N} as it flags REL. In the embedded nominal, \textit{ein} is absent in [\textit{als Bauer}]\textit{ }since \textit{als} is a general(ized) capacity operator that takes role and kind nouns as arguments. As to the absence of \textit{ein} in [\textit{erstaunlicher Bauer}], I assume that this is\textit{ }due to haplology with \textit{ein} in the matrix nominal. 
\end{styleStandard}

\begin{styleStandard}
\ \ To sum up this section, I derived the three readings established at the beginning of this chapter. In the DP, the ordinary and comparative readings involve canonical, simple DPs, but the capacity reading manifests an \textit{als-}nominal in an adjoined position. As for the TP, the first two readings exhibit regular copular structures, but the capacity reading involves a modal particle licensing an unpronounced ALS. The latter is part of a complex specifier, and this specifier forms part of a non-canonical nominal. 
\end{styleStandard}

\begin{styleStandard}\bfseries
4.\ \ Conclusion
\end{styleStandard}

\begin{styleStandard}
This chapter discussed the first consequence of the proposal laid out Chapter 5 – \textit{ein} in the context of emotive constructions. I started with the observation that the DP and the TP are parallel in structure and interpretation. Making this my general heuristic methodology, I extended the discussion to pronominal DPs and copular TPs containing non-theta nouns. Establishing three basic readings (cf. den Dikken 2006), I provided a more detailed investigation of how \textit{ein} combines with certain role and kind nouns and how these combinations fare with regard to emotiveness in the nominal and clausal domains. 
\end{styleStandard}

\begin{styleStandard}
In order to explain the commonalities and slight differences, I employed, with a few refinements, de Swart \textit{et al}.’s (2007) account of the clause and extended it to pronominal DPs. In order to explain the whole range of data, the explanation of the DP had to include some pragmatic considerations (Rauh 2004). In fact, the interaction between these two proposals also explained the absence of true minimal pairs of pronominal DPs and copular DPs involving bare predicate nouns. More importantly, I showed that \textit{ein} is not responsible for the emotiveness property. Rather, this property is due to REL in combination with a [+figurative] noun. It was argued that \textit{ein} indicates the presence of structure on top of NP (Hypothesis 3a) and the presence of REL (Hypothesis 3b). In other words, I can maintain the claim that \textit{ein} is semantically vacuous (Hypothesis 1a).
\end{styleStandard}

\begin{styleStandard}
So far, I only discussed cases where we find morphological agreement between the pronoun and the predicate noun. Next I turn to cases of – what looks like – dis-agreement in number.
\end{styleStandard}

\clearpage\setcounter{page}{292}\begin{styleStandard}
Chapter 7: \textit{Ein} and Number 
\end{styleStandard}

\begin{styleStandard}\bfseries
1.\ \ Introduction
\end{styleStandard}

\begin{styleStandard}
In this chapter, I continue the investigation begun in Chapter 5 and turn to the second and final consequence of the proposals about \textit{ein}. I argue that \textit{ein} is not a reflex of number, neither morphologically nor semantically. This is consistent with the claim that \textit{ein} is semantically vacuous (Hypothesis 1a).
\end{styleStandard}

\begin{styleStandard}\itshape
1.1.\ \ A Brief Review
\end{styleStandard}

\begin{styleStandard}
Recall from Chapter 6 that true minimal pairs between the DP and TP are not possible in singular, out-of-the-blue contexts. Specifically, although a bare role noun such as \textit{Bauer} is possible in both the DP and TP, it crucially has different interpretations: it is emotive/figurative in the DP (meaning ‘peasant’) and neutral/literal in the TP (meaning ‘farmer’). In other words, bare nouns are fairly restricted with regard to (non-)emotiveness in the two domains. I proposed that factors involving both pragmatics and semantics explain these restrictions, the former (Grician maxims) being particularly relevant for the DP and the latter (the operator CAP) for the TP. When \textit{ein} appears in front of a role noun, the facts fall differently.
\end{styleStandard}

\begin{styleStandard}
\ \ In particular, I noted that \textit{ein} cannot surface in pronominal DPs. Compare (1a) to (1b). In order to indicate that \textit{ein} is the indefinite article here, I provide the latter in its reduced, unstressable form \textit{‘n}:\footnote{\ As might be expected, the singularity numeral \textit{EIN} ‘one’ is also ungrammatical here (ia), but adjectival \textit{eine} is not (ib). Note that the second example involves a contrast cancelling the duality presupposition:\par \ \ (i)\ \ a. \ \ *\ \ \textit{du \ \ EIN Bauer}\par \ \ \ \ \ \ you one peasant.\textsc{masc }\par \ \ \ \ b.\ \ \textit{wir zwei Bauern und du \ \ einer Bauer}\par \ \ \ \ \ \ we two \ farmers and you one \ \ farmer.\textsc{masc}\par \ \ \ \ \ \ ‘us two farmers and you one farmer’\par In the main text below, (1b) is ruled out due to the presence of two determiners in one DP. Given the composite analysis of \textit{EIN} ‘one’ in German (Chapter 5), the same explanation extends to (ia). As adjectival \textit{eine} is a lexical item, morphologically unrelated to \textit{ein}, it can occur after a definite determiner as in (ib).}
\end{styleStandard}

\begin{styleStandard}
(1) \ \ a.\ \ \textit{du \ \ \ \ \ \ \ \ \ \ \ Bauer }
\end{styleStandard}

\begin{styleStandard}
\ \ \ \ you(\textsc{sgl}) peasant.\textsc{masc} \ \ \ \ \ \ \ \ \ 
\end{styleStandard}

\begin{styleStandard}
\ \ \ \ ‘you (peasant)’
\end{styleStandard}

\begin{styleStandard}
\ \ \ b. *\ \ \textit{du \ \ \ \ \ \ \ \ \ \ ‘n Bauer}
\end{styleStandard}

\begin{styleStandard}
\ \ \ \ you(\textsc{sgl}) a peasant.\textsc{masc} \ \ \ \ \ \ \ \ 
\end{styleStandard}

\begin{styleStandard}
However, \textit{ein} is possible in predicate nominals in the TP. Specifically, if \textit{ein} appears in these copular cases, the role noun may undergo an interpretative extension from its neutral/literal meaning in (2a) to an emotive/figurative reading in (2b). 
\end{styleStandard}

\begin{styleStandard}
(2) \ \ a.\ \ \textit{Du \ \ \ \ \ \ \ \ \ \ bist Bauer.}
\end{styleStandard}

\begin{styleStandard}
\ \ \ \ you(\textsc{sgl}) are farmer.\textsc{masc} 
\end{styleStandard}

\begin{styleStandard}
\ \ \ \ ‘You are a farmer.’
\end{styleStandard}

\begin{styleStandard}
\ \ b.\ \ \textit{Du \ \ \ \ \ \ \ \ \ \ bist ‘n Bauer.}
\end{styleStandard}

\begin{styleStandard}
\ \ \ \ you(\textsc{sgl}) are \ \ a peasant/farmer.\textsc{masc}
\end{styleStandard}

\begin{styleStandard}
\ \ \ \ ‘You are a peasant.’
\end{styleStandard}

\begin{styleStandard}
\ \ \ \ \%‘You are a farmer.’
\end{styleStandard}

\begin{styleStandard}
I proposed that this effect is not due to the presence of \textit{ein} itself. Rather, this extension in interpretatory possibilities is a function of the null operator REL, the presence of which is flagged by \textit{ein}. This is consistent with the claim that \textit{ein} is a semantically vacuous element.
\end{styleStandard}

\begin{styleStandard}\itshape
1.2.\ \ Number in the DP and TP
\end{styleStandard}

\begin{styleStandard}
This chapter abstracts away from the different meanings of the noun. Below, I focus on issues related to number (for general typological discussion, see Corbett 2000 and Hurford 2003; for a survey of the formal semantics, see Link 1998). To start with some simple examples, consider plural contexts where nouns and pronouns of the same number can combine in both the DP and TP:
\end{styleStandard}

\begin{styleStandard}
(3)\ \ a.\ \ \textit{ihr \ \ \ \ \ \ \ \ Schweine}
\end{styleStandard}

\begin{styleStandard}
\ \ \ \ you(\textsc{pl}) pigs
\end{styleStandard}

\begin{styleStandard}
\ \ \ \ ‘you idiots’\ \ 
\end{styleStandard}

\begin{styleStandard}
\ \ b.\ \ \textit{Ihr \ \ \ \ \ \ \ \ seid \ Ärzte.}
\end{styleStandard}

\begin{styleStandard}
\ \ \ \ you(\textsc{pl}) are \ \ doctors
\end{styleStandard}

\begin{styleStandard}
\ \ \ \ ‘You are doctors.’\ \ 
\end{styleStandard}

\begin{styleStandard}
These distributions are as we would expect. 
\end{styleStandard}

\begin{styleStandard}
Turning to singular nouns, it is also expected that such nouns cannot combine with plural pronouns to form pronominal DPs. This holds independently of the presence of \textit{ein}: 
\end{styleStandard}

\begin{styleStandard}
(4)\ \ a. *\ \ \textit{ihr \ \ \ \ \ \ \ \ Schwein}
\end{styleStandard}

\begin{styleStandard}
\ \ \ \ you(\textsc{pl}) pig.\textsc{neut} \ \ \ \ \ \ \ \ 
\end{styleStandard}

\begin{styleStandard}
\ \ b. *\ \ \textit{ihr \ \ \ \ \ \ \ ‘n Schwein}
\end{styleStandard}

\begin{styleStandard}
\ \ \ \ you(\textsc{pl}) a pig.\textsc{neut} \ \ \ \ \ \ \ \ 
\end{styleStandard}

\begin{styleStandard}
However, singular nouns behave differently in the TP. Surprisingly, a singular noun is possible in copular contexts (5a). At face value, this presents a case of dis-agreement between a singular noun and a plural pronoun. Crucially though, when \textit{ein} appears, restrictions in number reveal themselves again (5b):
\end{styleStandard}

\begin{styleStandard}
(5)\ \ a.\ \ \textit{Ihr \ \ \ \ \ \ \ \ seid alle Arzt.}
\end{styleStandard}

\begin{styleStandard}
\ \ \ \ you(\textsc{pl}) are \ all \ \ doctor.\textsc{masc} 
\end{styleStandard}

\begin{styleStandard}
\ \ \ \ ‘You are all doctors.’\ \ 
\end{styleStandard}

\begin{styleStandard}
\ \ b. ?*\ \ \textit{Ihr \ \ \ \ \ \ \ \ seid alle ‘n Arzt.}
\end{styleStandard}

\begin{styleStandard}
\ \ \ \ you(\textsc{pl}) are \ all \ \ \ \ a doctor.\textsc{masc}
\end{styleStandard}

\begin{styleStandard}
To sum up thus far, as seen in Chapter 6, bare nouns are restricted to non-emotiveness in singular TPs. As just seen though, they are unrestricted in number in TPs: bare nouns can occur not only in singular but also plural contexts – they are number-neutral. To repeat, bare nouns in TPs are restricted as regards emotiveness but unrestricted as regards number.
\end{styleStandard}

\begin{styleStandard}
It is clear that the grammaticality status of (5a) vs. (5b) correlates with the absence vs. presence of \textit{ein}. It could be claimed then that this difference is due to the morphology and/or semantics of \textit{ein}. However, I proposed in Chapter 5 that \textit{ein} itself is not specified for morphological number; that is, it has the specification [$\alpha $PL morph]. Furthermore, I argued that \textit{ein} has no semantics. If so, we cannot rule out (5b) by resorting to problems in morphological or semantic number induced by the presence of \textit{ein}. 
\end{styleStandard}

\begin{styleStandard}
\ \ To anticipate the following discussion, I reiterate the claim that morphological number is actually not specified on nouns in German. In Chapter 6, I followed de Swart \textit{et al}. (2007) in assuming that number morphology has to do with NumP. Specifically, bare nouns such as the role noun \textit{Arzt} ‘doctor’ are proposed to be unmarked for number and only project NP (6a). In contrast, plural nouns like \textit{Ärzte} ‘doctors’ are argued to involve NumP (6b), and singular nominals such as \textit{(ei)n Arzt} ‘a doctor’ are claimed to project more structure. I proposed in Chapter 6 that \textit{pace} de Swart \textit{et al.}, they may involve ArtP as in (6c). In order to differentiate between bare role nouns (6a) and singular nouns (6c), I label the first as non-plural reserving the term singular for the second. In other words, what has traditionally been classified as a “singular” noun may have two structural interpretations depending on the presence of \textit{ein}: 
\end{styleStandard}

\begin{styleStandard}
(6)\ \ a.\ \ bare/non-plural \textit{Arzt}:\ \ N\ \ → [ NP ]
\end{styleStandard}

\begin{styleStandard}
\ \ b.\ \ plural \textit{Ärzte}:\ \ \ \ N-\textsc{pl}\ \ → [ NumP [ NP ]]
\end{styleStandard}

\begin{styleStandard}
\ \ c.\ \ singular \textit{‘n Arzt}:\ \ \textit{ein} N\ \ → [[[ ArtP [ NumP [ NP ]]]
\end{styleStandard}

\begin{styleStandard}
To explain the difference in grammaticality in (5), I make use of the structural proposal in (6), where NumP is absent with non-plural nouns (6a) but present with plural and singular nouns (6b-c). Specifically, (5a) involves a non-plural noun as in (6a), but (5b) contains a singular noun as in (6c). I propose that (5a) is grammatical as NumP is absent in the predicate nominal. In other words, the predicate noun is number-neutral as no agreement relation between the predicate nominal and the subject pronoun is established.\footnote{\ That bare nouns as in (5a) are neither singular nor plural in interpretation (i.e., they are number-neutral) can also be seen in contexts other than the ones discussed in the main text:\par 
\setcounter{listWWviiiNumxvleveli}{0}
\begin{listWWviiiNumxvleveli}
\item 
\begin{styleFootnote}
a.\ \ flea(*s) infestation
\end{styleFootnote}
\end{listWWviiiNumxvleveli}
b.\ \ stone carving\par For the discussion of (i) and some other cases, see Borer (2005: 132-35).} In contrast, (5b) is ungrammatical as NumP is present in the predicate nominal. Here, an agreement relation between the predicate nominal and the subject pronoun is established resulting in a mismatch in number. In other words, it is the absence vs. presence of NumP (but not \textit{ein }itself) that is proposed to explain the difference in grammaticality in (5). 
\end{styleStandard}

\begin{styleStandard}
As for the DP cases in (4), I propose that NumP is always present; that is, predicate nominals within DPs always involve NumP. Here, NumP is present for cartographic reasons. Unlike in the TP, where the presence of NumP is indicated by overt elements, in the DP higher, obligatory phrases like ArtP entail the presence of lower NumP. If so, the presence of NumP rules out (4a) due to a mismatch in number between a plural pronoun and a singular noun. Note now that what may look like a non-plural noun in (4a) actually involves a singular noun – NumP entails REL and thus \textit{ein} should be present as in (4b) (NB: I usually provide the label for the surface forms given the terminology provided in (6)). 
\end{styleStandard}

\begin{styleStandard}
I suggest that the reason \textit{ein} can never be present inside pronominal DPs, be they singular as in (1b) or plural as in (4b), is that the null operator REL can be flagged not only by expletive \textit{ein} but also by other, substantive elements (e.g., pronominal determiners). Assuming that both types of determiner elements are merged in ArtP, only one of them can be inserted and consequently appear. If these considerations are on the right track, then I can maintain the claim that \textit{ein} does not determine number, neither morphologically nor semantically. As in the last chapter, I suggest that \textit{ein} indicates more structure on top of NP (Hypothesis 3a) and that it flags the presence of a null operator (Hypothesis 3b).
\end{styleStandard}

\begin{styleStandard}
\ \ The chapter is organized as follows. First, I present the data illustrating the restrictions on morphological and semantic number in more detail. In Section 3, I lay out my proposal to account for these facts. Section 4 summarizes the main findings.
\end{styleStandard}

\begin{styleStandard}\bfseries
2.\ \ Data
\end{styleStandard}

\begin{styleStandard}
I start with nominal and clausal contexts involving agreement in number. After that, I consider different instances of dis-agreement. The data are summarized in Tables 1 and 2 below. Given the discussion of the last chapter, I present the data of the DP with figuratively extended nouns and the examples of the TP with role nouns. Recall that I refer to “singular” forms of the noun without \textit{ein} as non-plural and to those with \textit{ein} as singular.
\end{styleStandard}

\begin{styleStandard}\itshape
2.1.\ \ Cases of Agreement 
\end{styleStandard}

\begin{styleStandard}
To begin, consider cases involving agreement in number. As already seen in the introduction, a non-plural noun can combine with a singular pronoun (7a) but a singular noun cannot (7b). Furthermore, a plural noun can combine with a plural pronoun (7c):
\end{styleStandard}

\begin{styleStandard}
(7) \ \ a.\ \ \textit{du \ \ \ \ \ \ \ \ \ \ \ Schwein }
\end{styleStandard}

\begin{styleStandard}
\ \ \ \ you(\textsc{sgl}) pig.\textsc{neut} \ \ \ \ \ \ \ \ \ 
\end{styleStandard}

\begin{styleStandard}
\ \ \ \ ‘you (idiot)’
\end{styleStandard}

\begin{styleStandard}
\ \ b. \ *\ \ \textit{du \ \ \ \ \ \ \ \ \ \ ‘n Schwein}
\end{styleStandard}

\begin{styleStandard}
\ \ \ \ you(\textsc{sgl}) a pig.\textsc{neut} \ \ \ \ \ 
\end{styleStandard}

\begin{styleStandard}
\ \ c.\ \ \textit{ihr \ \ \ \ \ \ \ \ Schweine}
\end{styleStandard}

\begin{styleStandard}
\ \ \ \ you(\textsc{pl}) pigs
\end{styleStandard}

\begin{styleStandard}
\ \ \ \ ‘you idiots’\ \ 
\end{styleStandard}

\begin{styleStandard}
As expected, (7a) can only be singular in interpretation and (7c) only plural. In other words, the interpretations in (7a) and (7c) are parallel to the morphology on the noun. I discuss the ungrammatical case of (7b) in Section 3.2.5 in detail proposing that (7b) involves two determiners in the same DP. 
\end{styleStandard}

\begin{styleStandard}
\ \ Turning to the TP, we find similar facts with the exception of singular nouns. Specifically, a non-plural as well as a singular noun can combine with a singular pronoun (8a-b). Again, a plural noun can occur with a plural pronoun (8c):\footnote{\ Recall from Chapter 6 that a predicative [-figurative] role noun in German can, for many speakers, take an indefinite article. In other words, these speakers allow both (8a) and (8b). Below, I focus on those speakers.} 
\end{styleStandard}

\begin{styleStandard}
(8) \ \ a.\ \ \textit{Du \ \ \ \ \ \ \ \ \ bist Arzt.}
\end{styleStandard}

\begin{styleStandard}
\ \ \ \ you(\textsc{sgl}) are doctor.\textsc{masc} \ 
\end{styleStandard}

\begin{styleStandard}
\ \ \ \ ‘You are a doctor.’
\end{styleStandard}

\begin{styleStandard}
\ \ \ b.\ \ \textit{Du \ \ \ \ \ \ \ \ \ \ \ bist ‘n Arzt.}
\end{styleStandard}

\begin{styleStandard}
\ \ \ \ you(\textsc{sgl}) are \ \ \ a doctor.\textsc{masc} 
\end{styleStandard}

\begin{styleStandard}
\ \ \ \ ‘You are a doctor.’
\end{styleStandard}

\begin{styleStandard}
\ \ c.\ \ \textit{Ihr \ \ \ \ \ \ \ \ seid Ärzte.}
\end{styleStandard}

\begin{styleStandard}
\ \ \ \ you(\textsc{pl}) are \ doctors
\end{styleStandard}

\begin{styleStandard}
\ \ \ \ ‘You are doctors.’\ \ 
\end{styleStandard}

\begin{styleStandard}
Again, the semantics runs parallel to the morphology. Next, I illustrate the various cases involving dis-agreement in number in more detail.
\end{styleStandard}

\begin{styleStandard}\itshape
2.2.\ \ Cases of Dis-agreenent
\end{styleStandard}

\begin{styleStandard}
Starting with the DP, a plural noun cannot combine with a singular pronoun (9a). Conversely, a singular or a non-plural noun cannot combine with a plural pronoun either (9b-c):\footnote{\ In Section 3.3.2, I discuss special cases such as \textit{Sie Schwein} ‘you (pig =) idiot’, which are characterized by semantic agreement but morphological dis-agreement (the latter though crucially holds between two different nominals as explicated in detail in that section).}
\end{styleStandard}

\begin{styleStandard}
(9)\ \ a. \ *\ \ \textit{du \ \ \ \ \ \ \ \ \ \ \ Schweine}
\end{styleStandard}

\begin{styleStandard}
you(\textsc{sgl}) pigs
\end{styleStandard}

\begin{styleStandard}
\ \ \ b. \ *\ \ \textit{ihr \ \ \ \ \ }\textit{\textsubscript{\ \ }}\textit{‘n Schwein}
\end{styleStandard}

\begin{styleStandard}
\ \ \ \ you(\textsc{pl}) a \ pig.\textsc{neut}\ \ \ \ \ \ 
\end{styleStandard}

\begin{styleStandard}
\ \ c. \ *\ \ \textit{ihr \ \ \ \ \ }\textit{\textsubscript{\ \ }}\textit{\ Schwein}
\end{styleStandard}

\begin{styleStandard}
\ \ \ \ you(\textsc{pl}) pig.\textsc{neut}\ \ \ \ \ \ 
\end{styleStandard}

\begin{styleStandard}
So far, the facts are as we might expect. 
\end{styleStandard}

\begin{styleStandard}
\ \ Turning to the TP, it may not be surprising either that a plural noun cannot combine with a singular pronoun (10a) or a singular noun with a plural pronoun (10b). However, it is unexpected that a non-plural noun can combine with a plural pronoun (10c). Importantly, the interpretation here is plural (for similar facts in Dutch, see de Swart \textit{et al}. 2005: 451):\footnote{\ There seem to be some interesting cross-linguistic differences with this type of data. For example, the Romance languages are different in that they require agreement between the pronoun and the nominal (see Zamparelli 2008: 107 fn. 4; also Munn \& Schmitt 2005: 839). }
\end{styleStandard}

\begin{styleStandard}
(10)\ \ a. \ *\ \ \textit{Du \ \ \ \ \ \ \ \ \ \ bist Ärzte.}
\end{styleStandard}

\begin{styleStandard}
\ \ \ \ you(\textsc{sgl}) are doctors
\end{styleStandard}

\begin{styleStandard}
\ \ b. ?*\ \ \textit{Ihr \ \ \ \ \ \ \ \ seid alle ‘n Arzt.}
\end{styleStandard}

\begin{styleStandard}
\ \ \ \ you(\textsc{pl}) are \ \ all \ \ \ a doctor.\textsc{masc}
\end{styleStandard}

\begin{styleStandard}
\ \ c.\ \ \textit{Ihr \ \ \ \ \ \ \ \ }\textit{\textsubscript{\ }}\textit{seid alle Arzt.}
\end{styleStandard}

\begin{styleStandard}
\ \ \ \ you(\textsc{pl}) are \ \ all doctor.\textsc{masc} 
\end{styleStandard}

\begin{styleStandard}
\ \ \ \ ‘You are all doctors.’\ \ 
\end{styleStandard}

\begin{styleStandard}
Note that there is a floating quantifier in the grammatical instance (10c) above. This element can also be in a different position (11a) or be exchanged by another quanfitier (11b). Interestingly, when such an element is missing, the example becomes a bit marked, and the judgments are somewhat unstable (11c) (I indicate this by \textsuperscript{m} here):\footnote{\ Below I argue that the (overt) quantifiers in (11a-b) are distributivity operators: they “single out” the individuals inside the set presupposed by the plural pronoun, and they distribute the relevant predicate over those individuals. As to (11c), I tentatively suggest below that there is a null distributivity operator present (also de Swart \textit{et al}. 2017: 218). The difference in overtness of the operator may explain the more stable judgments in (11a-b) vs. (11c).}
\end{styleStandard}

\begin{styleStandard}
(11)\ \ a.\ \ \textit{Ihr \ \ \ \ \ \ \ alle seid Arzt.}
\end{styleStandard}

\begin{styleStandard}
\ \ \ \ you(\textsc{pl}) all \ \ are \ doctor.\textsc{masc} 
\end{styleStandard}

\begin{styleStandard}
\ \ \ \ ‘You are all doctors.’\ \ 
\end{styleStandard}

\begin{styleStandard}
\ \ b.\ \ \textit{Ihr \ \ \ \ \ \ \ \ seid jeder Arzt}.
\end{styleStandard}

\begin{styleStandard}
\ \ \ \ you(\textsc{pl}) are \ \ each \ doctor.\textsc{masc} 
\end{styleStandard}

\begin{styleStandard}
\ \ \ \ ‘You are each a doctor.’\ \ 
\end{styleStandard}

\begin{styleStandard}
\ \ c. \ \textsuperscript{m}\ \ \textit{Ihr \ \ \ \ \ \ \ \ seid Arzt.}
\end{styleStandard}

\begin{styleStandard}
\ \ \ \ you(\textsc{pl}) are \ \ doctor.\textsc{masc} 
\end{styleStandard}

\begin{styleStandard}
\ \ \ \ ‘You are doctors.’\ \ 
\end{styleStandard}

\begin{styleStandard}
Similar facts hold in capacity constructions involving \textit{als} ‘as’. This is what I turn to next.
\end{styleStandard}

\begin{styleStandard}
\textit{2.3.\ \ More Cases Of Dis-agreement: Instances Involving }als\textit{{}-nominals}
\end{styleStandard}

\begin{styleStandard}
Recall that I refer to nouns preceded by \textit{als} ‘as’ as \textit{als}{}-nominals. As we will see, \textit{als}{}-nominals can occr in the context of pronominal DPs and non-copular TPs. First, I discuss cases of agreement and then instances of dis-agreement. Starting with the DPs, \textit{als}{}-nominals agree with their preceding pronoun, independently of whether the pronoun is in the singular (12a) or in the plural (12c), with the qualification that the presence of \textit{ein} leads to slight markedness (12b):
\end{styleStandard}

\begin{styleStandard}
(12)\ \ a.\ \ \textit{du \ \ \ \ \ \ \ \ \ \ \ als Arzt}
\end{styleStandard}

\begin{styleStandard}
\ \ \ \ you(\textsc{sgl}) as \ doctor.\textsc{masc}
\end{styleStandard}

\begin{styleStandard}
\ \ \ \ ‘you as a doctor’
\end{styleStandard}

\begin{styleStandard}
\ \ b. \ ?\ \ \textit{du \ \ \ \ \ \ \ \ \ \ \ als ‘n Arzt}
\end{styleStandard}

\begin{styleStandard}
\ \ \ \ you(\textsc{sgl}) as \ \ \ a doctor.\textsc{masc}
\end{styleStandard}

\begin{styleStandard}
\ \ \ \ ‘you as a doctor’
\end{styleStandard}

\begin{styleStandard}
\ \ c.\ \ \textit{ihr \ \ \ \ \ \ \ \ als Ärzte}
\end{styleStandard}

\begin{styleStandard}
\ \ \ \ you(\textsc{pl}) as \ doctors
\end{styleStandard}

\begin{styleStandard}
\ \ \ \ ‘you as doctors’
\end{styleStandard}

\begin{styleStandard}
\ \ The same grammaticality judgments hold for the corresponding sentences involving a non-copular verb:\footnote{\ The sentences with non-copular verbs involve \textit{als} ‘as’ and have capacity readings. In contrast, the sentences with copular verbs discussed in Sections 2.1 and 2.2 above do not involve ALS ‘as’ and do not have capacity readings (recall from Chapter 6, Section 3.4.1 that copular TPs can have a capacity reading – they involve null ALS licensed by the modal particle \textit{vielleicht} ‘really’). The different readings do not play a role here – the focus is on number. }
\end{styleStandard}

\begin{styleStandard}
(13)\ \ a.\ \ \textit{Du \ \ \ \ \ \ \ \ \ \ sprichst als Arzt}.
\end{styleStandard}

\begin{styleStandard}
\ \ \ \ you(\textsc{sgl}) speak \ \ \ \ as \ doctor.\textsc{masc}
\end{styleStandard}

\begin{styleStandard}
\ \ \ \ ‘You speak as a doctor.’
\end{styleStandard}

\begin{styleStandard}
\ \ b. \ ?\ \ \textit{Du \ \ \ \ \ \ \ \ \ \ sprichst als ‘n Arzt}.\ \ \ \ 
\end{styleStandard}

\begin{styleStandard}
\ \ \ \ you(\textsc{sgl}) speak \ \ \ \ as \ \ a doctor.\textsc{masc}
\end{styleStandard}

\begin{styleStandard}
\ \ \ \ ‘You speak as a doctor.’
\end{styleStandard}

\begin{styleStandard}
\ \ c.\ \ \textit{Ihr \ \ \ \ \ \ \ \ sprecht als Ärzte.}
\end{styleStandard}

\begin{styleStandard}
\ \ \ \ you(\textsc{pl}) speak \ \ \ as \ doctors
\end{styleStandard}

\begin{styleStandard}
\ \ \ \ ‘You speak as doctors.’
\end{styleStandard}

\begin{styleStandard}
\ \ Turning to the cases involving dis-agreement, consider nominals where a singular pronoun is combined with a plural noun (14a), and a plural pronoun is followed by a singular noun (14b). While these two combinations lead to ungrammaticality, there is one exception: a plural pronoun can be followed by an \textit{als}{}-nominal with a non-plural noun (14c):
\end{styleStandard}

\begin{styleStandard}
(14)\ \ a. *\ \ \textit{du \ \ \ \ \ \ \ \ \ \ }\textit{\textsubscript{\ }}\textit{als Ärzte}
\end{styleStandard}

\begin{styleStandard}
\ \ \ \ you(\textsc{sgl}) as \ doctors
\end{styleStandard}

\begin{styleStandard}
\ \ b. ?*\ \ \textit{ihr \ }\textit{\textsubscript{\ \ \ \ \ \ \ \ \ \ }}\textit{alle als ‘n Arzt}
\end{styleStandard}

\begin{styleStandard}
\ \ \ \ you(\textsc{pl}) all \ \ as \ \ \textsubscript{\ }a doctor.\textsc{masc}\ \ \ \ \ \ \ \ 
\end{styleStandard}

\begin{styleStandard}
\ \ c.\ \ \textit{ihr \ \ \ \ \ \ \ \ alle }\textit{\textsubscript{\ }}\textit{als Arzt \ \ \ }\ \ \ \ \ \ 
\end{styleStandard}

\begin{styleStandard}
\ \ \ \ you(\textsc{pl}) all \ \ \ as \ doctor.\textsc{masc}\ \ \ \ \ \ \ \ 
\end{styleStandard}

\begin{styleStandard}
\ \ \ \ ‘you all as doctors’
\end{styleStandard}

\begin{styleStandard}
Note that the interpretation of the non-plural nominal in (14c) is indeed plural. The same holds for the clausal counterparts (the datum in (15c) is adapted from de Swart \textit{et al}. 2007: 206):
\end{styleStandard}

\begin{styleStandard}
(15)\ \ a. *\ \ \textit{Du \ \ \ \ \ \ \ \ \ \ sprichst als Ärzte}.
\end{styleStandard}

\begin{styleStandard}
\ \ \ \ you(\textsc{sgl}) speak \ \ \ as \ \ doctors
\end{styleStandard}

\begin{styleStandard}
\ \ b. ??\ \ \textit{Ihr \ \ \ \ \ \ \ }\textit{\textsubscript{\ }}\textit{sprecht alle als ‘n Arzt}.\ \ \ \ 
\end{styleStandard}

\begin{styleStandard}
\ \ \ \ you(\textsc{pl}) speak \ \ all \ \ \ as \ \ \ a doctor.\textsc{masc}\ \ \ \ \ \ 
\end{styleStandard}

\begin{styleStandard}
\ \ c.\ \ \textit{Ihr \ \ \ \ \ \ \ sprecht alle als Arzt}.\ \ \ \ 
\end{styleStandard}

\begin{styleStandard}
\ \ \ \ you(\textsc{pl}) speak \ \ all \ \ as \ \ doctor.\textsc{masc}\ \ \ \ \ \ 
\end{styleStandard}

\begin{styleStandard}
\ \ \ \ ‘You speak all as doctors.’
\end{styleStandard}

\begin{styleStandard}
To review, this last section discussed \textit{als}{}-nominals in the context of pronominal DPs and non-copular TPs. These two cases show the same syntactic distributions and corresponding semantic interpretations as the copular TPs illustrated in the two preceding sections – these three cases may involve morphological dis-agreement. This is in stark contrast to simple DPs, that is, DPs without \textit{als}{}-nominals, which have to exhibit agreement. Before moving on, I provide a more detailed summary of the individual cases in tabular form.
\end{styleStandard}

\begin{styleStandard}\itshape
2.4.\ \ Summary of the Data
\end{styleStandard}

\begin{styleStandard}
I summarize the syntactic distributions in the following charts. I mark the unexpected instances by (!), the latter all restricted to the occurrence of non-plural nouns. Note that if there is a difference in judgments between nominals with or without \textit{als}, an explicit reference to \textit{als} is provided. Table 1 shows the facts involving the DP (the shadings in Table 1 are commented on below):
\end{styleStandard}

\begin{styleStandard}
Table 1: Summary of the Judgments in the DP
\end{styleStandard}

\begin{flushleft}
\begin{tabular}{|m{2.12126in}|m{1.4212599in}|m{1.4212599in}|m{1.3712599in}|}

\hline
 &
\centering Non-plural N &
\centering \textit{(ei)n} N &
\centering\arraybslash N-\textsc{pl}\\\hline
\textit{du} ‘you(\textsc{sgl})’ &
\centering ${\surd}$ &
{\centering * [without \textit{als}]\par}

\centering ? [with \textit{als}] &
\centering\arraybslash *\\\hline
\textit{ihr} ‘you(\textsc{pl})’ &
{\centering * [without \textit{als}]\par}

\centering ${\surd}$ (!) [with \textit{als}]  &
{\centering * [without \textit{als}]\par}

\centering ?* [with \textit{als}] &
\centering\arraybslash ${\surd}$\\\hline
\end{tabular}
\end{flushleft}
\begin{styleStandard}
Table 2 shows the facts of the TP. Observe again that non-plural nouns are surprising as they are grammatical in basically all clausal contexts:
\end{styleStandard}

\begin{styleStandard}
Table 2: Summary of the Judgments in the TP
\end{styleStandard}

\begin{flushleft}
\begin{tabular}{|m{2.12126in}|m{1.4212599in}|m{1.4212599in}|m{1.3712599in}|}

\hline
 &
\centering Non-plural N &
\centering \textit{(ei)n} N &
\centering\arraybslash N-\textsc{pl}\\\hline
\textit{du} ‘you(\textsc{sgl})’ &
\centering ${\surd}$ &
{\centering ${\surd}$ [without \textit{als}]\par}

\centering ? [with \textit{als}] &
\centering\arraybslash *\\\hline
\textit{ihr} ‘you(\textsc{pl})’ &
\centering ${\surd}$ (!) &
{\centering ?* [without \textit{als}]\par}

\centering ?? [with \textit{als}] &
\centering\arraybslash ${\surd}$\\\hline
\end{tabular}
\end{flushleft}
\begin{styleStandard}
To repeat, the unexpected distributions occur when a plural pronoun is combined with a non-plural noun, both in the DP mediated by \textit{als} ‘as’ and in the TP more generally. Furthermore, the judgments are basically the same with or without \textit{als }(and these cases are treated below as the same). There are two exceptions to this that both hold in the DP (indicated by shading of the relevant cells in Table 1): depending on the presence of \textit{als}, the judgments differ markedly if a singular pronoun combines with a singular noun or a plural pronoun occurs with a non-plural noun. 
\end{styleStandard}

\begin{styleStandard}
\ \ In view of these data, I conclude that number is very restricted in simple DPs such that pronouns and nouns have to match in number. Syntactically, singular pronouns cannot combine with plural nouns, and plural pronouns cannot combine with singular or non-plural nouns.\footnote{\ The only potential exception that I am aware of comes from discontinuous noun phrases, where a plural noun is compatible with both a plural and a singular determiner element:\par (i)\ \ \textit{Hemden habe ich \{kein-e \ \ / ?kein-es\} \ getragen}.\par \ \ \ \ shirts \ \ \ \ have \ I \ \ \ \ \ none-\textsc{pl}/ \ none-\textsc{sgl} worn\par \ \ \ \ ‘As for shirts, I have worn none.’\par I argued in Chapter 4, Section 3 that the source and the split-off involve two separately base-generated nominals. Furthermore, Ott \& Nicolae (2010) suggest that the plural in the split-off presumably has to do with frame-setting in topicalization, and as such the “dis-agreement” here is a different phenomenon.} Semantically, singular and plural pronouns are interpreted as singular and plural, respectively; singular/non-plural and plural nouns are understood as singular and plural, respectively.
\end{styleStandard}

\begin{styleStandard}
\ \ In contrast, copular clauses, non-copular clauses, and pronominal DPs, where the latter two involve \textit{als}, are less restricted, both syntactically and semantically. Syntactically, plural pronouns can, under certain conditions, occur with non-plural nouns. Semantically, non-plural nouns are compatible with a plural interpretation. In the next section, I turn to an explanation of these syntactic and semantic facts.
\end{styleStandard}

\begin{styleStandard}\bfseries
3.\ \ Proposal
\end{styleStandard}

\begin{styleStandard}
In this section, I account for the different morphological and semantic possibilities licensed in the different syntactic domains. Just as in Chapter 6, I follow de Swart \textit{et al.} (2007) and extend their proposal to pronominal DPs explaining the differences between predicate nominals in TPs and those in DPs.
\end{styleStandard}

\begin{styleStandard}\itshape
3.1.\ \ Basic Assumptions
\end{styleStandard}

\begin{styleStandard}
To begin, I lay out my assumptions of the simple structures; the instances involving \textit{als}{}-nominals are discussed at the end of this chapter. Starting with the syntax and simplifying somewhat for now, I assume again that all DPs, including pronominal DPs, consist of a head noun projecting an NP, a Number head projecting a NumP, and a (pronominal) determiner surfacing in the DP-level (16a). In keeping with the discussion in Chapter 6, the predicate nominal inside the DP always involves NumP (where REL maps an element of type {\textless}e{\textgreater} to type {\textless}e,t{\textgreater}). TPs are different: while they all involve a subject DP, a copular verb, and a predicate nominal, the latter may vary in size. As in Chapter 6, I assume that NumP is syntactically optional in predicate nominals following a copular auxiliary. Specifically, a non-plural nominal involves NP (16b), a plural nominal consists of NumP (16c), and a singular nominal projects ArtP (16d) (Aux = stands for the auxiliary \textit{sein} ‘to be’):
\end{styleStandard}

\begin{styleStandard}
(16) \ \ a.\ \ DP\textsubscript{{\textless}e{\textgreater}}:\ \ [\textsubscript{DP} [\textsubscript{NumP} [\textsubscript{NP} ]]]\textsubscript{{\textless}e{\textgreater}}
\end{styleStandard}

\begin{styleStandard}
\ \ b.\ \ TP\textsubscript{{\textless}t{\textgreater}}:\ \ [\textsubscript{DP} [\textsubscript{NumP} [\textsubscript{NP} ]]]\textsubscript{{\textless}e{\textgreater}} Aux [\textsubscript{NP} ]\textsubscript{{\textless}e,t{\textgreater}}
\end{styleStandard}

\begin{styleStandard}
\ \ c.\ \ TP\textsubscript{{\textless}t{\textgreater}}:\ \ [\textsubscript{DP} [\textsubscript{NumP} [\textsubscript{NP} ]]]\textsubscript{{\textless}e{\textgreater}} Aux [\textsubscript{NumP} Num\textsubscript{[+PL]} [\textsubscript{NP} ]]\textsubscript{{\textless}e,t{\textgreater}}
\end{styleStandard}

\begin{styleStandard}
\ \ d.\ \ TP\textsubscript{{\textless}t{\textgreater}}:\ \ [\textsubscript{DP} [\textsubscript{NumP} [\textsubscript{NP} ]]]\textsubscript{{\textless}e{\textgreater}} Aux [\textsubscript{ArtP} \textit{ein} [\textsubscript{NumP} Num\textsubscript{[-PL]} [\textsubscript{NP} ]]]\textsubscript{{\textless}e,t{\textgreater}}
\end{styleStandard}

\begin{styleStandard}
These are the basic assumptions about the syntax. 
\end{styleStandard}

\begin{styleStandard}
\ \ As for the semantics, I again follow de Swart \textit{et al.}’s (2007) discussion of the clause in (16b-d) and assume that the predicate nominal must be of type {\textless}e,t{\textgreater} to combine with the subject DP (type {\textless}e{\textgreater}). Recalling that the copular auxiliary is semantically vacuous, this means that NP, NumP, and ArtP must all be of type {\textless}e,t{\textgreater}. As already discussed above, de Swart \textit{et al. }propose that role nouns combine with CAP, which is in NP, and that kind nouns, inherent or coerced, combine with REL, which is in NumP. The first option holds for (16b); the second option applies to plural (16c) and to singular (16d), with singular involving an additional structural layer (ArtP). To be clear, these types of predicate nominals are all syntactically different but semantically the same (i.e., type {\textless}e,t{\textgreater}). As such, they can combine with the subject DP to yield a truth value (type {\textless}t{\textgreater}). Note again that \textit{ein} indicates the presence of a certain amount of structure on top of NP (Hypothesis 3a) and that it flags the presence of the operator REL (Hypothesis 3b).
\end{styleStandard}

\begin{styleStandard}
\ \ As for the DP in (16a), I suggested that NumP (and thus REL) is always present. I argue in more detail below that this is so for cartographic reasons – higher, obligatory phrases entail the presence of lower ones. Semantically, combining the head noun and REL yields an element of type {\textless}e,t{\textgreater}. As determiners, including pronominal determiners, are of type {\textless}{\textless}e,t{\textgreater}e{\textgreater}, predicate nominals and determiners can combine to return an entity (type {\textless}e{\textgreater}).
\end{styleStandard}

\begin{styleStandard}
\ \ \ With these general points in place, I turn to the cases from the data section. First, I discuss DPs and TPs involving agreement and dis-agreement. Then, I turn to some special cases involving the pronoun \textit{Sie} ‘you(\textsc{formal})’, identifying the parts of the structure where morphological and semantic number originate. In the final section, I address the more complex DPs and TPs involving \textit{als}{}-nominals. 
\end{styleStandard}

\begin{styleStandard}
\textit{3.2.\ \ Agreement in Constructions Without }als
\end{styleStandard}

\begin{styleStandard}
First, I review some general facts about agreement in the TP. This is followed by proposing that NumP plays a crucial role in agreement. Finally, I detail the account of agreement as regards plural nouns, non-plural nouns, and singular nouns.
\end{styleStandard}

\begin{styleStandard}
3.2.1. Agreement and Dis-agreement
\end{styleStandard}

\begin{styleStandard}
Consider the data involving copular TPs below. Recall from above that non-plural nouns are fine in all contexts (17). In contrast, singular nouns can only combine with singular pronouns (18) and plural nouns only with plural pronouns (19):
\end{styleStandard}

\begin{styleStandard}
(17) \ \ a.\ \ \textit{Du \ \ \ \ \ \ \ \ \ \ bist} \textit{Arzt.}
\end{styleStandard}

\begin{styleStandard}
\ \ \ \ you(\textsc{sgl}) are \ \textsubscript{\ }doctor.\textsc{masc}
\end{styleStandard}

\begin{styleStandard}
\ \ \ \ ‘You are a doctor.’\ \ 
\end{styleStandard}

\begin{styleStandard}
\ \ b.\ \ \textit{Ihr \ \ \ \ \ \ \ seid alle Arzt.}
\end{styleStandard}

\begin{styleStandard}
\ \ \ \ you(\textsc{pl}) are \ all \ doctor.\textsc{masc} 
\end{styleStandard}

\begin{styleStandard}
\ \ \ \ ‘You are all doctors.’\ \ 
\end{styleStandard}

\begin{styleStandard}
(18)\ \ a.\ \ \textit{Du \ \ \ \ \ \ \ \ \ \ bist} \textit{‘n Arzt.}
\end{styleStandard}

\begin{styleStandard}
\ \ \ \ you(\textsc{sgl}) are \ \textsubscript{\ }a doctor.\textsc{masc}
\end{styleStandard}

\begin{styleStandard}
\ \ \ \ ‘You are a doctor.’\ \ 
\end{styleStandard}

\begin{styleStandard}
b. \ ?*\ \ \textit{Ihr \ \ \ \ \ \ \ seid alle ‘n Arzt.}
\end{styleStandard}

\begin{styleStandard}
\ \ \ \ you(\textsc{pl}) are \ all \ \ \ a doctor.\textsc{masc} 
\end{styleStandard}

\begin{styleStandard}
(19)\ \ a. \ *\ \ \textit{Du \ \ \ \ \ \ \ \ \ \ bist} \textit{Ärzte.}
\end{styleStandard}

\begin{styleStandard}
\ \ \ \ you(\textsc{sgl}) are \ doctors
\end{styleStandard}

\begin{styleStandard}
b.\ \ \textit{Ihr \ \ \ \ \ \ \ seid Ärzte.}
\end{styleStandard}

\begin{styleStandard}
\ \ \ \ you(\textsc{pl}) are \ doctors
\end{styleStandard}

\begin{styleStandard}
\ \ \ \ ‘You are doctors.’\ \ 
\end{styleStandard}

\begin{styleStandard}
To be clear, non-plural nouns can occur in both singular and plural contexts, but singular and plural nouns are in complementary distribution as regards number.
\end{styleStandard}

\begin{styleStandard}
\ \ To see whether this is a morphological or semantic restriction, I discuss \textit{jeder} ‘each’ in this regard. This quantificational element is morphologically singular but semantically plural; that is, it takes a non-plural restriction but presupposes a plurality of entities. As might be expected, this element can combine with a non-plural noun via an auxiliary (20a). However, while \textit{jeder} can also occur with a singular noun (20b), it cannot occur with a plural one (20c):\footnote{\ Other quantifiers also presuppose a plurality of entities, but they are morphologically plural and can combine with plural nouns:\par \ \ (i)\ \ \textit{Einige (Männer hier) sind} \textit{Ärzte.}\par \textit{\ \ \ \ }some \ \ \ men \ \ \ \ \ \ here \ are \ doctors\par \ \ \ \ ‘Some (men here) are doctors.’}
\end{styleStandard}

\begin{styleStandard}
(20)\ \ a.\ \ \textit{Jeder (Mann hier) ist Arzt.}
\end{styleStandard}

\begin{styleStandard}
\ \ \ \ each \ \ \ man \ \ here \ is \ doctor.\textsc{masc} 
\end{styleStandard}

\begin{styleStandard}
\ \ \ \ ‘Each (man here) is a doctor.’
\end{styleStandard}

\begin{styleStandard}
\ \ b.\ \ \textit{Jeder (Mann hier) ist ‘n Arzt.}
\end{styleStandard}

\begin{styleStandard}
\ \ \ \ each \ \ \ man \ \ here \ is \ \ a doctor.\textsc{masc} 
\end{styleStandard}

\begin{styleStandard}
\ \ \ \ ‘Each (man here) is a doctor.’
\end{styleStandard}

\begin{styleStandard}
\ \ c. \ *\ \ \textit{Jeder (Mann hier) ist Ärzte.}
\end{styleStandard}

\begin{styleStandard}
\ \ \ \ each \ \ \ man \ \ here \ is \ doctors
\end{styleStandard}

\begin{styleStandard}
Considering that \textit{jeder} is semantically plural, the incompatibility with a plural noun in (20c) is somewhat surprising. I make the strongest claim and propose that the number restrictions shown in (17) through (20) are morphological \textit{and} semantic in nature. The lack of morphological agreement between singular \textit{jeder} and the plural predicate nominal explains the ungrammaticality of (20c). As for the grammatical (20a-b), I propose that \textit{jeder} is semantically compatible with non-plural and singular nouns. Specifically, I suggest that \textit{jeder}, itself a distributivity operator, distributes the predicate over the set presupposed by itself.\footnote{\ Alternatively, there could be an additional null distributivity operator involved here. Note in this regard that de Swart \textit{et al}. (2007: 218) propose a null distributivity operator for cases like \textit{Jan und Sofie sind Arzt} ‘Jan and Sofie are (doctor =) doctors’ in Dutch (more on null distributivity operators in Section 3.3.2). } In what follows, I focus on morphological number, but I return to semantic number in Section 3.3.
\end{styleStandard}

\begin{styleStandard}
3.2.2. Agreement and NumP
\end{styleStandard}

\begin{styleStandard}
Starting with the TP, recall that predicate nominals in the clause involve NP, NumP, or ArtP depending on the morphology inside the predicate nominal. Specifically, while all nouns involve at least NP, plural morphology on the noun indicates the presence of NumP, and the presence of \textit{ein} shows ArtP (containing NumP). Furthermore, I propose that when NumP is present, an agreement relation must be established between the predicate nominal and the subject DP. In contrast, a non-plural noun does not project NumP – it is a predicate nominal involving NP. I propose that with NumP absent, such a predicate nominal does not have to undergo an agreement relation, and consequently its distribution is much freer (see also den Dikken 2006: 210). 
\end{styleStandard}

\begin{styleStandard}
DPs are different. I propose that NumP must be present for cartographic reasons: higher phrases such as ArtP entail the presence of lower phrases such as NumP. Note in this regard that DPs involve determiners and that these determiners originate in ArtP and move to the DP-level. Now, as is well known, elements in the DPs must exhibit concord in agreement features like case, number, and gender. I follow the literature in making the standard assumption that number features originate in NumP (e.g., de Swart \textit{et al}. 2007, Julien 2005a, Ritter 1991, Roehrs 2006b). I assume that the values on Num are morphological (i.e., [$\alpha $PL morph]) and that NumP mediates concord within the DP. For concreteness, I assume here that the value of the Num head “percolates” up the nominal tree by some concord mechanism.\footnote{\ As mentioned in Chapter 1, Section 4.1.1, there are different mechanisms that have been claimed to bring about concord. Note also that the obligatory presence of NumP is presumably not due to concord itself. This is so because I suggest below that mass nouns receive singular morphology by default (as [$\alpha $PL morph] is proposed to be absent on Num in those cases). } As to the head noun, I proposed in Chapter 1, Section 4 that it moves to adjoin to Num (also Julien 2005a), where the head noun establishes an agreement relation with Num. To repeat, NumP is always present in the DP.
\end{styleStandard}

\begin{styleStandard}
Taking stock thus far, there are two types of agreement: concord in agreement features within the DP, indicated below by subscript alphas (21a), and agreement between the subject and the predicate nominal within the copular TP, marked below by subscript betas (21b). The latter type of agreement only holds if NumP is present and if the morphological number of the predicate nominal agrees with that of the subject. Note that (21b) also involves concord in agreement features, just like (21a), if NumP and higher phrases are present within the predicate nominal:
\end{styleStandard}

\begin{styleStandard}
(21)\ \ a.\ \ DP:\ \ [ Det\textsubscript{$\alpha $} Num\textsubscript{$\alpha $} N\textsubscript{$\alpha $} ]
\end{styleStandard}

\begin{styleStandard}
\ \ b.\ \ TP:\ \ Subj\textsubscript{$\beta $} Aux [ … (Num\textsubscript{$\alpha $}) N\textsubscript{$\alpha $} ]\textsubscript{$\beta $}\ \  \ \ (where $\beta $ = $\alpha $ on Num)
\end{styleStandard}

\begin{styleStandard}
It is clear that NumP is crucial in establishing agreement relations both within the nominal domain and the clausal domain (involving copulas). For the TP, I focus on the agreement relation between the subject and the predicate nominal.
\end{styleStandard}

\begin{styleStandard}
With these points about NumP in mind, we can observe that there are two notions of dis-agreement. Starting with the DP, all ungrammatical cases in the nominal domain involve (true) morphological dis-agreement where the combination of singular and plural elements cannot establish an agreement relation (22). This failure to establish concord in agreement features is due to the obligatory presence of NumP in the DP, which is either specified as singular or plural (for details, see Section 3.3): 
\end{styleStandard}

\begin{styleStandard}
(22)\ \ a. \ *\ \ \textit{du \ \ \ \ \ \ \ \ \ \ \ Ärzte}
\end{styleStandard}

\begin{styleStandard}
\ \ \ \ you(\textsc{sgl}) doctors
\end{styleStandard}

\begin{styleStandard}
b. \ *\ \ \textit{ihr \ \ \ \ \ \ \ \ Arzt}
\end{styleStandard}

\begin{styleStandard}
\ \ \ \ you(\textsc{pl}) doctor.\textsc{masc}
\end{styleStandard}

\begin{styleStandard}
This is different for the TP. Given certain conditions, we may find instances of true dis-agreement (23a) but also cases of \textit{apparent} dis-agreement (23b):
\end{styleStandard}

\begin{styleStandard}
(23)\ \ a. \ *\ \ \textit{Du \ \ \ \ \ \ \ \ \ bist Ärzte.}
\end{styleStandard}

\begin{styleStandard}
\ \ \ \ you(\textsc{sgl}) are doctors
\end{styleStandard}

\begin{styleStandard}
b.\ \ \textit{Ihr \ \ \ \ \ \ \ \ seid alle Arzt}. 
\end{styleStandard}

\begin{styleStandard}
\ \ \ \ you(\textsc{pl}) are \ \ all \ doctor.\textsc{masc}
\end{styleStandard}

\begin{styleStandard}
\ \ \ \ ‘You are all doctors.’
\end{styleStandard}

\begin{styleStandard}
Importantly, true dis-agreement in copular contexts also involves the fact that NumP is present, here in the plural predicate nominal in (23a). This leads to ungrammaticality as an agreement relation between the singular subject pronoun and the plural predicate nominal cannot be established. Cases of apparent dis-agreement are different. In (23b), NumP is absent with non-plural nominals, and no agreement relation has to be established. This allows the combination of a plural subject pronoun and a non-plural noun to surface. Consider the individual derivations involving the different types of nouns.
\end{styleStandard}

\begin{styleStandard}
3.2.3. Plural Nouns
\end{styleStandard}

\begin{styleStandard}
DPs involve structures where NumP is always present. Assuming again that determiners orginate in ArtP, the tree diagram of (24b) is given in (24c) (in the discussion, I abstract away from the traces/copies left by movement):
\end{styleStandard}

\begin{styleStandard}
(24) \ \ a. * \ \ \textit{du \ \ \ \ \ \ \ \ \ \ \ Schweine}
\end{styleStandard}

\begin{styleStandard}
\ \ \ \ you(\textsc{sgl}) pigs
\end{styleStandard}

\begin{styleStandard}
\ \ \ b. \ \ \textit{ihr \ \ \ \ \ \ \ }\textit{\textsubscript{\ }}\textit{Schweine}
\end{styleStandard}

\begin{styleStandard}
\ \ \ \ you(\textsc{pl}) pigs
\end{styleStandard}

\begin{styleStandard}
\ \ \ \ ‘you idiots’
\end{styleStandard}

\begin{styleStandard}
\ \ c. \textit{Agreement in the DP}
\end{styleStandard}

\begin{styleStandard}
\ \ \ \ DP\textsubscript{{\textless}e{\textgreater}}
\end{styleStandard}

\begin{styleStandard}
[Warning: Draw object ignored][Warning: Draw object ignored]
\end{styleStandard}

\begin{styleStandard}
\ \ \ \ \ \ \ \ \ \ \ \textit{ihr}\textsubscript{{\textless}{\textless}e,t{\textgreater}e{\textgreater}k} \ \ \ \ \ \ ArtP
\end{styleStandard}

\begin{styleStandard}
[Warning: Draw object ignored][Warning: Draw object ignored]
\end{styleStandard}

\begin{styleStandard}
\ \ \ \ t\textsubscript{k} \ \  \ \ \ \ \ NumP\textsubscript{{\textless}e,t{\textgreater}}
\end{styleStandard}

\begin{styleStandard}\itshape
[Warning: Draw object ignored][Warning: Draw object ignored]
\end{styleStandard}

\begin{styleStandard}
\ \ \ \  \ \ \ \ Num\textsubscript{REL{\textless}e{\textless}e,t{\textgreater}{\textgreater}} \ \ \ NP
\end{styleStandard}

\begin{styleStandard}
\ \ \ \ \ \  \ \ \  \ \textit{Schweine}\textsubscript{{\textless}e{\textgreater}} \ \ \ \ \ \ 
\end{styleStandard}

\begin{styleStandard}
As mentioned in the previous subsection, all elements inside DP have to establish an agreement relation mediated by NumP. If so, it is easy to rule out (24a), where such a relation between the pronominal determiner, Num, and the noun cannot be established. In contrast, (24b) is fine as the relevant elements can establish an agreement relation in number. I turn to the clausal domain.
\end{styleStandard}

\begin{styleStandard}
As in Chapter 6, I follow Bowers (1993) in assuming that copular structures involve a Predication Phrase (PrP). The latter is embedded under a TP. I propose that the head Pr can take different elements as its complement. With plural morphology present on the nouns in (25a-b), I suggest that Pr takes NumP as its predicative complement. I provide the tree diagram of (25b) in (25c):
\end{styleStandard}

\begin{styleStandard}
(25)\ \ a. *\ \ \textit{Du \ \ \ \ \ \ \ \ \ \ bist Ärzte.}
\end{styleStandard}

\begin{styleStandard}
\ \ \ \ you(\textsc{sgl}) are doctors
\end{styleStandard}

\begin{styleStandard}
\ \ \ b.\ \ \textit{Ihr \ \ \ \ \ \ \ \ seid Ärzte.}
\end{styleStandard}

\begin{styleStandard}
\ \ \ \ you(\textsc{pl}) are \ doctors
\end{styleStandard}

\begin{styleStandard}
\ \ \ \ ‘You are doctors.’
\end{styleStandard}

\begin{styleStandard}
\ \ c. \textit{Agreement in the TP}
\end{styleStandard}

\begin{styleStandard}
\ \ \ \ TP\textsubscript{{\textless}t{\textgreater}}
\end{styleStandard}

\begin{styleStandard}
[Warning: Draw object ignored][Warning: Draw object ignored]
\end{styleStandard}

\begin{styleStandard}
\ \ \ \ \ [\textit{ihr }REL e\textsubscript{N}]\textsubscript{{\textless}e{\textgreater}k} \ \ T’
\end{styleStandard}

\begin{styleStandard}
[Warning: Draw object ignored][Warning: Draw object ignored]
\end{styleStandard}

\begin{styleStandard}
\ \  \ \ \ \ \ \ \ \ \ \textit{seid}\textsubscript{i}\ \  \ \ \ \ \ PrP
\end{styleStandard}

\begin{styleStandard}
[Warning: Draw object ignored][Warning: Draw object ignored]\ \  \ \ \ \ \ \ \ \ 
\end{styleStandard}

\begin{styleStandard}
\ \ \ \  \ \ \ \ \ \ \ \ \ t\textsubscript{k}\ \  \ \ Pr’
\end{styleStandard}

\begin{styleStandard}
[Warning: Draw object ignored][Warning: Draw object ignored]\ \  \ \ \ \ \ \ \ 
\end{styleStandard}

\begin{styleStandard}
\ \ \ \ \ \  \ \ \ \ \ \ t\textsubscript{i}\ \  \ \ \ \ \ \ \ \ NumP\textsubscript{{\textless}e,t{\textgreater}}\ \ 
\end{styleStandard}

\begin{styleStandard}
\ \ \ \ \ \ \ \  \ \ \ \ \ \ \ \ \ \ \textit{Ärzte}
\end{styleStandard}

\begin{styleStandard}
Recall from the previous chapter that pronominal determiners (\textit{ihr}) are taken to be of type {\textless}{\textless}e,t{\textgreater}e{\textgreater} but that pronominal DPs as a whole (e.g., \textit{ihr }REL e\textsubscript{N}) are of type {\textless}e{\textgreater}. As regards morphological number, I proposed above that the NumP of the predicate nominal has to establish an agreement relation with the subject DP. As plural \textit{Ärzte} ‘doctors’ can entered into such a relation with plural \textit{ihr} but not singular \textit{du}, the difference in grammaticality in (25a-b) is accounted for. As just seen, we can observe that the cases involving plural nouns are straightforward.
\end{styleStandard}

\begin{styleStandard}
3.2.4. Non-plural Nouns
\end{styleStandard}

\begin{styleStandard}
Again, all DPs contain NumP. As such, a singular pronoun is grammatical with a non-plural noun (26a), but a plural pronoun is not (26b). I derive (26a) as in (26c):
\end{styleStandard}

\begin{styleStandard}
(26) \ \ a.\ \ \textit{du \ \ \ \ \ \ \ \ \ \ \ Schwein }
\end{styleStandard}

\begin{styleStandard}
\ \ \ \ you(\textsc{sgl}) pig.\textsc{neut} \ \ \ \ \ \ \ \ \ 
\end{styleStandard}

\begin{styleStandard}
\ \ \ \ ‘you (idiot)’
\end{styleStandard}

\begin{styleStandard}
\ \ b. *\ \ \textit{ihr \ \ \ \ \ \ \ }\textit{\textsubscript{\ \ }}\textit{Schwein}
\end{styleStandard}

\begin{styleStandard}
\ \ \ \ you(\textsc{pl}) pig.\textsc{neut}\ \ \ \ \ \ 
\end{styleStandard}

\begin{styleStandard}
\ \ c. \textit{Agreement in the DP}
\end{styleStandard}

\begin{styleStandard}
\ \ \ \ DP\textsubscript{{\textless}e{\textgreater}}
\end{styleStandard}

\begin{styleStandard}
[Warning: Draw object ignored][Warning: Draw object ignored]
\end{styleStandard}

\begin{styleStandard}
\ \ \ \ \ \ \ \ \ \ \ \textit{du}\textsubscript{{\textless}{\textless}e,t{\textgreater}e{\textgreater}k} \ \ \ \ \ \ ArtP
\end{styleStandard}

\begin{styleStandard}
[Warning: Draw object ignored][Warning: Draw object ignored]
\end{styleStandard}

\begin{styleStandard}
\ \ \ \ t\textsubscript{k} \ \  \ \ \ \ \ NumP\textsubscript{{\textless}e,t{\textgreater}}
\end{styleStandard}

\begin{styleStandard}\itshape
[Warning: Draw object ignored][Warning: Draw object ignored]
\end{styleStandard}

\begin{styleStandard}
\ \ \ \  \ \ \ \ Num\textsubscript{REL{\textless}e{\textless}e,t{\textgreater}{\textgreater}} \ \ \ NP
\end{styleStandard}

\begin{styleStandard}
\ \ \ \ \ \  \ \ \  \ \textit{Schwein}\textsubscript{{\textless}e{\textgreater}} \ \ \ \ \ \ 
\end{styleStandard}

\begin{styleStandard}
The ungrammaticality in (26b) is due to the failure of establishing the relevant agreement relation between the pronominal determiner, Num, and the noun. Note now that with NumP present, REL is present too, and we would expect \textit{ein} to occur. In the next subsection, I discuss why singular nouns are not possible here.
\end{styleStandard}

\begin{styleStandard}
\ \ As regards the TP, I followed de Swart \textit{et al}. (2007) in that NumP is absent if no number-related elements are present in the predicate nominal. This is the case for (27a-b). I propose that the head Pr takes NP as its predicative complement here. Recall that bare role nouns combine with CAP. I derive (27a-b) as (27c):
\end{styleStandard}

\begin{styleStandard}
\ (27)\ \ a.\ \ \textit{Du \ \ \ \ \ \ \ \ \ \ bist Arzt.}
\end{styleStandard}

\begin{styleStandard}
\ \ \ \ you(\textsc{sgl}) are doctor.\textsc{masc} 
\end{styleStandard}

\begin{styleStandard}
\ \ \ \ ‘You are a doctor.’
\end{styleStandard}

\begin{styleStandard}
\ \ \ b.\ \ \textit{Ihr \ \ \ \ \ \ \ \ seid alle Arzt.}
\end{styleStandard}

\begin{styleStandard}
\ \ \ \ you(\textsc{pl}) are \ all \ \ doctor.\textsc{masc}
\end{styleStandard}

\begin{styleStandard}
\ \ \ \ ‘You are doctors.’
\end{styleStandard}

\begin{styleStandard}
\ \ c. \textit{Apparent (Dis-)Agreement in the TP}
\end{styleStandard}

\begin{styleStandard}
\ \ \ \ TP\textsubscript{{\textless}t{\textgreater}}
\end{styleStandard}

\begin{styleStandard}
[Warning: Draw object ignored][Warning: Draw object ignored]
\end{styleStandard}

\begin{styleStandard}
\ \ \ \ \ \ \ \ \ \textit{du}/\textit{ihr}\textsubscript{{\textless}e{\textgreater}k} \ \ \ \ \ \ \ \ \ \ T’
\end{styleStandard}

\begin{styleStandard}
[Warning: Draw object ignored][Warning: Draw object ignored]
\end{styleStandard}

\begin{styleStandard}
\ \ \ \ \textit{seid}\textsubscript{i}\ \  \ \ \ \ \ \ PrP
\end{styleStandard}

\begin{styleStandard}
[Warning: Draw object ignored][Warning: Draw object ignored]\ \  \ \ \ \ \ \ \ \ 
\end{styleStandard}

\begin{styleStandard}
\ \ \ \ \ \ t\textsubscript{k}\ \  \ \ \ \ Pr’\textsubscript{{\textless}e,t{\textgreater}}
\end{styleStandard}

\begin{styleStandard}
[Warning: Draw object ignored][Warning: Draw object ignored]
\end{styleStandard}

\begin{styleStandard}
\ \ \ \ \ \  \ \ \ \ \ \ \ \ t\textsubscript{i}\ \ \ \  NP\textsubscript{CAP{\textless}e{\textless}e,t{\textgreater}{\textgreater}}\ \ 
\end{styleStandard}

\begin{styleStandard}
\ \ \ \ \ \ \ \  \ \ \textit{Arzt}\textsubscript{{\textless}e{\textgreater}}
\end{styleStandard}

\begin{styleStandard}
With NumP absent, the predicate nominal does not have to enter into an agreement relation. In other words, no agreement relation between the predicate nominal and the subject DP has to be established, and both singular and plural subject pronouns are grammatical here. (27b), then, is a case of apparent dis-agreement as no relevant agreement relation is established in the first place. In order to obtain a plural interpretation of the non-plural noun in (27b), I follow de Swart \textit{et al.} (2007) in that NP may involve a distributivity operator. I take the floating quantifier in (27b) to be this element. 
\end{styleStandard}

\begin{styleStandard}
\ \ This discussion makes an interesting prediction: if a predicate nominal in a copular TP does not involve NumP but just NP, then the head noun can only combine with CAP to yield an element of type {\textless}e,t{\textgreater} (see (27c) again). This in turn should allow only a neutral/literal meaning of a role noun that can, at least potentially, undergo figurative extension. In keeping with what we saw in Chapter 6, we find the same interpretatory restriction in the (apparent) dis-agreement cases discussed here:
\end{styleStandard}

\begin{styleStandard}
(28) \ \ \textit{Ihr \ \ \ \ \ \ \ \ seid jeder Bauer.}
\end{styleStandard}

\begin{styleStandard}
\ \ you(\textsc{pl}) are \ \ each \ farmer.\textsc{masc}
\end{styleStandard}

\begin{styleStandard}
\ \ ‘You (each) are farmers.’
\end{styleStandard}

\begin{styleStandard}
\ \ \#‘You (each) are peasants.’
\end{styleStandard}

\begin{styleStandard}
In the next section, I disucss singular nouns, and I provide an explanation as to why \textit{ein} is not possible in pronominal DPs.
\end{styleStandard}

\begin{styleStandard}
3.2.5. Singular Nouns
\end{styleStandard}

\begin{styleStandard}
Starting with the DP, I return to the question left unanswered in the last subsection, namely why \textit{ein} cannot occur in pronominal DPs despite the fact that REL is present. Note that this not only applies to pronominal DPs (29a) but also to ordinary DPs (29b):
\end{styleStandard}

\begin{styleStandard}
(29) \ \ a. *\ \ \textit{du \ \ \ \ \ \ \ \ \ ‘n Schwein}
\end{styleStandard}

\begin{styleStandard}
\ \ \ \ you(\textsc{sgl}) a pig.\textsc{neut} \ \ \ \ \ 
\end{styleStandard}

\begin{styleStandard}
\ \ b. *\ \ \textit{das} \textit{‘n Schwein}
\end{styleStandard}

\begin{styleStandard}
\ \ \ \ the \ \ a pig.\textsc{neut}
\end{styleStandard}

\begin{styleStandard}
I proposed in the last chapter that REL is not only flagged by \textit{ein} but also by other elements. Specifically, definite determiners, pronominal or ordinary, can also flag the presence of REL. Recall that this is consistent with de Swart \textit{et al}.’s (2007) assumption about regular DPs like \textit{the child}. Now, assuming that determiners and \textit{ein} originate in the same position (i.e., ArtP), only one such element can occur. This explains the ungrammaticality of the cases in (29), where two determiners are present. Finally, given the current analysis, the distributional interaction between \textit{ein} and personal pronouns in pronominal DPs provides evidence that the latter are indeed determiners. I turn to the TP.
\end{styleStandard}

\begin{styleStandard}
\ \ In Chapter 6, I discussed in detail the proposal that for nouns to be predicates, kind nouns combine with REL directly, but role nouns do so indirectly (via kind coercion and REL). REL triggers the presence of \textit{ein} in singular contexts, which I explained by proposing that \textit{ein} flags the presence of REL. Consider (30a-b). Given the absence of a feature for definiteness on \textit{ein}, I also proposed that this element may surface in ArtP (and not necessarily in DP). Given that, I propose that besides NP and NumP, the head Pr can also take ArtP as its predicative complement. The tree diagram of (30a) is provided in (30c):
\end{styleStandard}

\begin{styleStandard}
(30)\ \ a.\ \ \textit{Du \ \ \ \ \ \ \ \ \ \ bist ‘n Arzt.}
\end{styleStandard}

\begin{styleStandard}
\ \ \ \ you(\textsc{sgl}) are \ \ a doctor.\textsc{masc} 
\end{styleStandard}

\begin{styleStandard}
\ \ \ \ ‘You are a doctor.’
\end{styleStandard}

\begin{styleStandard}
\ \ b. ?*\ \ \textit{Ihr \ \ \ \ \ \ \ seid alle ‘n Arzt.}
\end{styleStandard}

\begin{styleStandard}
\ \ \ \ you(\textsc{pl}) are \ alle \ a doctor.\textsc{masc}
\end{styleStandard}

\begin{styleStandard}
\ \ c. \textit{Agreement in the TP}
\end{styleStandard}

\begin{styleStandard}
\ \ \ \ TP\textsubscript{{\textless}t{\textgreater}}
\end{styleStandard}

\begin{styleStandard}
[Warning: Draw object ignored][Warning: Draw object ignored]
\end{styleStandard}

\begin{styleStandard}
\textit{\ \ \ \ \ \ \ \ \ \ \ \ du}\textsubscript{{\textless}e{\textgreater}k} \ \ \ \ \ \ \ \ \ \ \ \ T’
\end{styleStandard}

\begin{styleStandard}
[Warning: Draw object ignored][Warning: Draw object ignored]
\end{styleStandard}

\begin{styleStandard}
\ \ \ \ \textit{bist}\textsubscript{i}\ \  \ \ \ \ \ \ PrP
\end{styleStandard}

\begin{styleStandard}
[Warning: Draw object ignored][Warning: Draw object ignored]\ \  \ \ \ \ \ \ \ \ 
\end{styleStandard}

\begin{styleStandard}
\ \ \ \ \ \ t\textsubscript{k}\ \  \ \ \ \ Pr’
\end{styleStandard}

\begin{styleStandard}
[Warning: Draw object ignored][Warning: Draw object ignored]\ \  \ \ \ \ \ \ \ 
\end{styleStandard}

\begin{styleStandard}
\ \ \ \ \ \  \ \ \ \ \ \ \ \ t\textsubscript{i}\ \ \ \ ArtP\textsubscript{{\textless}e,t{\textgreater}}
\end{styleStandard}

\begin{styleStandard}
[Warning: Draw object ignored][Warning: Draw object ignored]
\end{styleStandard}

\begin{styleStandard}
\ \ \ \ \ \ \ \  \ \ \ \textit{‘n}\ \  \ \ \ \ \ \ NumP\textsubscript{REL{\textless}e{\textless}e,t{\textgreater}{\textgreater}}\ \ 
\end{styleStandard}

\begin{styleStandard}
\ \ \ \ \ \ \ \  \ \  \ \ \ \ \ \ \ \textsubscript{\ }\textit{Arzt}\textsubscript{{\textless}e{\textgreater}}
\end{styleStandard}

\begin{styleStandard}
Observe that the subject pronoun \textit{du} ‘you’ and \textit{ein} in (30a) are part of different nominals – the pronominal one and the predicate one, where the former is separated from the latter by an auxiliary. As such, \textit{du} and \textit{ein} do not compete for insertion in the same ArtP, and they can co-occur. Now, notice that with NumP present, a singular predicate nominal can establish an agreement relation with a singular, but not a plural, subject pronoun. As such, the latter case presents an instance of true dis-agreement, and the failure to establish an agreement relation explains the ungrammaticality of (30b). I return to the main focus of this book summarizing the discussion so far and drawing some conclusions. 
\end{styleStandard}

\begin{styleStandard}
Starting with the DP, I proposed that NumP is always present, and a relevant agreement relation between a determiner, Num, and a noun has to be established. Furthermore, although REL is always present in the DP, \textit{ein} does not surface when another determiner occurs. As suggested above, elements other than \textit{ein} can flag the presence of REL in singular contexts. As for the TP, I proposed that the head Pr can take NP, NumP, or ArtP as its predicative complement. Importantly, when NumP is present in the predicative complement, an agreement relation with the subject DP has to be established. In contrast, when NumP is absent, no agreement relation has to be established, and a number-neutral predicate comes about.
\end{styleStandard}

\begin{styleStandard}
More generally, these assumptions allow us to account for the diverse agreement phenomena illustrated above while maintaining the claim that \textit{ein} has no specifications for number. In keeping with Chapter 5, I provide more details in the next section that morphological number is not due to \textit{ein} but due to the number specification on NumP. In addition, I relate morphological number to semantic number and the mass/count distinction of nouns.
\end{styleStandard}

\begin{styleStandard}\itshape
3.3.\ \ Morphological and Semantic Number
\end{styleStandard}

\begin{styleStandard}
Considerations regarding number and the related mass/count distinction are notoriously complex and difficult. For convenience, I employ my own system developed in Roehrs (2006b).\footnote{\ There are many other proposals. To name just a few: Borer (2005), Eynde (2020), Ott (2011b), Watanabe (2006), and H. Wiese \& Maling (2005) (see also references cited in these works). Also, Hachem (2015) and Rehn (2019: 181-86) argue that a simple dichotomy of count vs. mass nouns is not correct (also S. Grimm 2012, Zhang 2012). On the basis of diachronic facts, Hachem (2015) proposes that grammatical gender functions as a mass quantifier; that is, gender is a semantically meaningful category that creates different types of mass distinctions (her page 100): neuter involves unbounded mass (e.g., \textit{snow}), masculine forms individuative mass (e.g., \textit{frost}), and feminine is collective mass (e.g., \textit{winter}); plural is a separate category. \par Rehn (2019) points out that these fine-grained distinctions present a problem for Borer (2005) (for other issues with Borer’s work, see also Ott 2011b, Zhang 2012). To the extent that these finer distinctions are correct, the current analysis could break NumP into several phrases (as is done in Hachem 2015 and Rehn 2019). Having said that, Hachem states that this proposed correlation between gender of the noun and its interpretation is no longer obvious in Modern German – the correlation has been regrammaticalized (see Leiss 1997). Furthermore, these finer distinctions do not seem to have an influence on the distribution of adjectival inflections or \textit{ein }in Standard German. As such, I do not take them into consideration here.} My primary goal here is not to argue that this is the correct way to account for morphological and semantic number. Rather, I intend to show that elements other than \textit{ein} can be held responsible for morphological and semantic number thus defending the claim that \textit{ein} is not a reflex of number. 
\end{styleStandard}

\begin{styleStandard}
\ \ In the first subsection, I relate morphological and semantic number. With this in place, I turn to cases of dis-agreement in pronominal DPs containing \textit{Sie} ‘you(\textsc{formal})’. In the third subsection, I provide more details of the derivation of these cases as regards number. The discussion of these non-canonical structures reveals a number of interesting points.
\end{styleStandard}

\begin{styleStandard}
3.3.1. Relating Morphological and Semantic Number
\end{styleStandard}

\begin{styleStandard}
I begin by illustrating the relevant issues with an ordinary noun like \textit{Schwein} ‘pig’. As is well known, this type of noun has three readings. It can have a mass (31a), a singular count (31b), or a plural count (31c) interpretation:
\end{styleStandard}

\begin{styleStandard}
(31) \ \ a.\ \ \textit{Schwein \ \ schmeckt gut}.
\end{styleStandard}

\begin{styleStandard}
\ \ \ \ pig.\textsc{neut} tastes \ \ \ \ \ \ good
\end{styleStandard}

\begin{styleStandard}
\ \ \ \ ‘Pork tastes good.’
\end{styleStandard}

\begin{styleStandard}
\ \ b.\ \ \textit{du \ \ \ \ \ \ \ \ \ \ \ Schwein }
\end{styleStandard}

\begin{styleStandard}
\ \ \ \ you(\textsc{sgl}) pig.\textsc{neut} \ \ \ \ \ \ \ \ \ 
\end{styleStandard}

\begin{styleStandard}
\ \ \ \ ‘you (idiot)’
\end{styleStandard}

\begin{styleStandard}
\ \ c. \ \ \textit{ihr \ \ \ \ \ \ \ }\textit{\textsubscript{\ \ }}\textit{Schweine}
\end{styleStandard}

\begin{styleStandard}
\ \ \ \ you(\textsc{pl}) pigs\ \ \ \ \ \ 
\end{styleStandard}

\begin{styleStandard}
\ \ \ \ ‘you idiots’
\end{styleStandard}

\begin{styleStandard}
In order to account for these readings, I propose that there is an intricate interplay between the number specifications on the noun and the Number head (Num). Specifically, head nouns have a statement for semantic number, and the latter interacts with the morphological number on Num.
\end{styleStandard}

\begin{styleStandard}
\ \ In more detail, I assume with Borer (2005: 94) that nouns are unmarked for the mass-count distinction. Unlike Borer, I assume that all regular nouns involve two statements as regards number (to be collapsed into one statement below). The first consists of morphological and semantic number with unspecified values (32a); the second states that both of these values have to coincide (32b); that is, either both values for number are positive or both are negative.\footnote{\ It is not likely that these two statements are acquired for each and every noun. Rather, they are probably provided to all regular nouns by a default mechanism. Note in this regard that we will see below that there are certain elements where the specifications for morphological and semantic number can diverge. Presumably, these specifications are lexically marked on those elements and have to be acquired individually.} As for NumP, its head has the morphological number feature [$\alpha $PL morph], which enters the derivation specified as positive or negative. Syntactically, recall from Chapter 1, Section 4 that the head noun moves to adjoin to Num. With these points in mind, the lower part of the DP structure can be fleshed out as in (32c), where the two statements of the noun in (32a-b) are collapsed into one:
\end{styleStandard}

\begin{styleStandard}
(32) \ \ \textit{Noun }
\end{styleStandard}

\begin{styleStandard}
\ \ a. \ \ [$\alpha $PL morph; $\beta $PL sem] 
\end{styleStandard}

\begin{styleStandard}
\ \ b. \ \ (where $\alpha $ = $\beta $)
\end{styleStandard}

\begin{styleStandard}
c.\ \ \ \  \ NumP
\end{styleStandard}

\begin{styleStandard}
[Warning: Draw object ignored][Warning: Draw object ignored]
\end{styleStandard}

\begin{styleStandard}
\ \  \ \ \ \ \ \ \ \ \ \  \ Num’
\end{styleStandard}

\begin{styleStandard}
[Warning: Draw object ignored][Warning: Draw object ignored]
\end{styleStandard}

\begin{styleStandard}
\ \ \ \  \ \ \  \ \ \ \ \ \ \ \ \ \ Num \ \ \ \  \ \ \ \ \ \ NP \ 
\end{styleStandard}

\begin{styleStandard}
[Warning: Draw object ignored][Warning: Draw object ignored]\ \ \ \ \ \ \ \ \ \  \ \ \ \ \ \ \ \ t\textsubscript{k}
\end{styleStandard}

\begin{styleStandard}
\ \ \ \ \ \ N\textsubscript{[$\alpha $PL morph; $\beta $PL sem; $\alpha $ = $\beta $]k} \ \ \ Num\textsubscript{[$\alpha $PL morph]}
\end{styleStandard}

\begin{styleStandard}
With the head noun and Num in a local relation, I assume that the morphological value for number on Num and that on the noun establish an agreement relation. After the morphological number of the noun has agreed with Num, the equality statement in (32b) brings about the semantic number of the noun. This means that there is an interplay between Num and the head noun as regards number, morphologically and consequently semantically.
\end{styleStandard}

\begin{styleStandard}
\ \ Before providing the detailed derivations of (31a-c) above, I state the following vocabulary entries for second-person informal pronouns, the pronominal determiners in (31b-c) above (see also Chapter 3, Section 5):
\end{styleStandard}

\begin{styleFootnote}
(33)\ \ a.\ \ [+D; -AUTH, +PART; INFORMAL]\ \  \ \ \ →\ \ \textit{ihr} \textsubscript{\ \ }/ \ \ \textsubscript{\ }[+F, +N, -O,\textsubscript{ }\ {}-S]
\end{styleFootnote}

\begin{styleStandard}
\ \ b.\ \ [+D; -AUTH, +PART; INFORMAL]\ \  \ \ \ →\ \ \textit{du} \ \ / \ \ \textsubscript{\ }[ -O, \ {}-S]
\end{styleStandard}

\begin{styleStandard}
The formal pronoun of address (i.e., \textit{Sie} ‘you(\textsc{formal})’) is discussed in the next subsection.
\end{styleStandard}

\begin{styleStandard}
Returning to the data in (31), recall that \textit{Schwein} can have three manifestations: it can be a mass, a singular count, or a plural count noun. I assume that the morphological feature [$\alpha $PL morph]\textsubscript{ }on NumP is syntactically optional. If it is absent, \textit{Schwein} is interpreted as a mass noun, and it occurs with a null article. The subject DP in (31a) can be illustrated in the following simplified structure:
\end{styleStandard}

\begin{styleStandard}
(34)\ \ \ \  \ DP
\end{styleStandard}

\begin{styleStandard}
[Warning: Draw object ignored][Warning: Draw object ignored]
\end{styleStandard}

\begin{styleStandard}
\ \  \ \ \textit{Ø}\textit{\textsubscript{D}}\ \ \ \ NumP
\end{styleStandard}

\begin{styleStandard}
\ \ \ \ \ \ \ \ [Warning: Draw object ignored][Warning: Draw object ignored]
\end{styleStandard}

\begin{styleStandard}
\ \ \ \ Num\ \ \ \  NP
\end{styleStandard}

\begin{styleStandard}
[Warning: Draw object ignored][Warning: Draw object ignored]\ \ \ \ \ \ \ \  \ t\textsubscript{k}
\end{styleStandard}

\begin{styleStandard}
\textit{Schwein}\textsubscript{k}\ \ Num\ \ \ \  \ \ 
\end{styleStandard}

\begin{styleStandard}
\textsubscript{\ \ [$\alpha $PL morph; $\beta $PL sem; $\alpha $ = $\beta $]}\ \ \textsubscript{ \ \ \ \ \ \ \ \ \ \ \ \ \ \ \ \ }
\end{styleStandard}

\begin{styleStandard}
Note that the number values of the noun are not specified. I suggest that this yields a (number-neutral) mass interpretation. The apparent singularity of the mass nominal in (31a) is due to singular being a default value here. 
\end{styleStandard}

\begin{styleStandard}
If [$\alpha $PL morph]\textsubscript{ }on NumP is present, it can have two values. [-PL] results in a singular count noun, and [+PL] brings about a plural count noun. The singular pronominal DP in (31b) is illustrated as follows:
\end{styleStandard}

\begin{styleStandard}
(35)\ \ \ \  \ DP
\end{styleStandard}

\begin{styleStandard}
[Warning: Draw object ignored][Warning: Draw object ignored]
\end{styleStandard}

\begin{styleStandard}
\ \  \ \textit{du}\ \ \ \ NumP
\end{styleStandard}

\begin{styleStandard}
\ \ \ \ \ \ \ \ [Warning: Draw object ignored][Warning: Draw object ignored]
\end{styleStandard}

\begin{styleStandard}
\ \  \ \ Num\ \ \ \  NP
\end{styleStandard}

\begin{styleStandard}
[Warning: Draw object ignored][Warning: Draw object ignored]\ \ \ \ \ \ \ \  \ t\textsubscript{k}
\end{styleStandard}

\begin{styleStandard}
\textit{Schwein}\textsubscript{k}\ \ Num\ \ \ \  \ \ 
\end{styleStandard}

\begin{styleStandard}
\textsubscript{\ \ [-PL morph; {}-PL sem; $\alpha $ = $\beta $]}\ \ \textsubscript{[-PL morph] \ \ \ \ \ \ \ \ \ }
\end{styleStandard}

\begin{styleStandard}
The plural counterpart in (31c) is similar, but has positive values for number. This feature specification on Num percolates up to the other terminal nodes of the structure bringing about concord in agreement features. Depending on this specification (singular or plural), the corresponding pronominal determiner in (33a) or (33b) is inserted. Observe now that in the singular and plural cases, the morphological and semantic number coincide by (32b). Note that this yields countability as a side-effect and makes individuative or mass quantifiers (e.g., \textit{mehrere} ‘several’ vs. \textit{viel} ‘much’) compatible with individuative or mass nominals, respectively.\footnote{\ Again, there are other proposals. H. Wiese \& Maling (2005: 8) claim that countability stems from the head noun. Second, Crisma (1999: 122) argues that the indefinite article does not mark indefiniteness but brings about countability. Third, developing ideas of Delfitto \& Schroten (1991), Panagiotidis (2002: 58, 2003a: 421) suggests that countability originates with Num. The latter is similar to what is claimed in the main text. Also, as far as I can see, the current discussion does not need to make use of the countability feature [±COUNT]. However, the adoption of such a feature makes the formulation of the vocabulary insertion rules of determiners more straightforward (see Chapter 8, Section 2.2.4).} 
\end{styleStandard}

\begin{styleStandard}
\ \ I argued in Chapter 6 that bare role nouns in copular clauses do not involve NumP. Now, if there is no NumP in these instances at all, then the number values on these types of head nouns do not get specified resulting in a number neutral element (cf. also Munn \& Schmitt 2005: 827). This makes bare role nouns similar to the mass nominals discussed in (34) above.\footnote{\ Presumably, the pragmatics will force a bare role noun to have a individuative (rather than mass) interpretation in corpular TPs. Furthermore, one might suggest that nouns on their mass interpretation also lack NumP (cf. Borer 2005). Note, however, that these nouns can combine with adjectives. With AgrP – the location of adjectives – high in the structure, NumP is also present. Notice though that the mass interpretation of the modified noun is not necessarily turned into a count interpretation (perhaps due to the mere presence of NumP). Thus, I continue to assume that with the exception of bare role nouns, NumP is present and that the number feature on Num is syntactically optional. Having said that, note that in pronominal DPs like \textit{du Schwein}, the number feature on Num must be present – this nominal does not mean ‘you pork’, but rather ‘you (pig =) idiot’. In addition, we utilized NumP (and its number feature) to account for cases of true dis-agreement in the previous section. It seems clear then that nouns do not get an individuative interpretation in the latter instances by the pragmatics.} 
\end{styleStandard}

\begin{styleStandard}
With these general remarks in mind, I continue illustrating the relevant issues with some special cases involving the pronoun \textit{Sie} ‘you(\textsc{formal})’, an element rarely discussed in the literature in this respect. As before, I follow the idea that non-canonical cases may reveal the true nature of the phenomenon.
\end{styleStandard}

\begin{styleStandard}
3.3.2. Dis-agreement Revisited
\end{styleStandard}

\begin{styleStandard}
As detailed in Section 2, DPs involving \textit{als}{}-nominals and TPs more generally may exhibit morphological dis-agreement in number. Extending this empirical discussion, it is clear from the verbal agreement in (36a) that \textit{Sie} ‘you’ is morphologically plural. As is well known, \textit{Sie} is semantically ambiguous between singular and plural; that is, \textit{Sie} can be used to address one or several individuals. It is interesting to note that when this pronoun occurs with a non-plural noun in a copular sentence, it can have both interpretations (36a). The presence of the floating quantifier disambiguates \textit{Sie} as having plural semantics and makes the latter reading more easily available. The DP is different. Although both \textit{Sie} and a non-plural noun can co-occur, this string only has a singular interpretation (36b). This interpretative difference is indicated in the respective translations:\footnote{\ Note that the combination of formal \textit{Sie} ‘you’ and (negatively) emotive \textit{Schwein} ‘(swine =) idiot’ leads to a stylistic clash. This makes (36b) marked for some speakers. Cases similar to (36b) involve the fairly common strings \textit{Sie Armleuchter} ‘you (chandelier =) idiot’ or \textit{Sie Arsch} ‘you ass’ (see also Darski 1979, Roehrs 2006b).}
\end{styleStandard}

\begin{styleStandard}
(36) \ \ a.\ \ \textit{Sie \ sind \ \ (alle) Arzt}.
\end{styleStandard}

\begin{styleStandard}
\ \ \ \ you are.\textsc{pl} all \ \ \ doctor.\textsc{masc}
\end{styleStandard}

\begin{styleStandard}
\ \ \ \ ‘You are a doctor.’
\end{styleStandard}

\begin{styleStandard}
\ \ \ \ ‘You are (all) doctors.’
\end{styleStandard}

\begin{styleStandard}
\ \ b.\ \ \textit{Sie \ Schwein}
\end{styleStandard}

\begin{styleStandard}
\ \ \ \ you pig.\textsc{neut}
\end{styleStandard}

\begin{styleStandard}
\ \ \ \ ‘you (idiot)’
\end{styleStandard}

\begin{styleStandard}
\ \ \ \ \#’you idiots’
\end{styleStandard}

\begin{styleStandard}
This difference in the semantics between (36a) and (36b) resembles the cases from Section 2. However, here \textit{both} the TP and the DP involve morphological dis-agreement. This is of special interest for the nominal domain as (36b) does not involve an \textit{als}{}-nominal. 
\end{styleStandard}

\begin{styleStandard}
In keeping with the discussion above, I suggest below that the semantic difference between (36a) and (36b) follows from the optional presence of NumP in the clausal domain but its obligatory presence in the nominal domain. Specifically, I proposed above that semantic number is due to an interaction between Num and the head noun. I propose that NumP is absent in (36a). Consequently, the predicate noun is number-neutral. In contrast, NumP is present in (36b), and the unexpected grammaticality is accounted for by a structure involving two separate nominals on a par with cases discussed in Chapter 2, Section 3.5. Consider this in more detail.
\end{styleStandard}

\begin{styleStandard}
\ \ Starting with the copular case in (36a), recall from above that if there is no number morphology on the noun, then there is no NumP in the predicate nominal. Consequently, no agreement relation is established, and there is no true dis-agreement. To derive the difference between the singular and plural interpretations of the predicate nominal, I assume that there is a null counterpart of the distributivity operator \textit{alle} ‘all’ (de Swart et al. 2007: 218). I label this null element as DIST. I suggest that it is optionally present (when \textit{alle} is absent):
\end{styleStandard}

\begin{styleStandard}
(37)\ \ \textit{Sie \ sind \ \ }(DIST)\textit{ Arzt}.
\end{styleStandard}

\begin{styleStandard}
\ \ you are.\textsc{pl} \ \ \ \ \ \ \ \ \ \ \ doctor.\textsc{masc}
\end{styleStandard}

\begin{styleStandard}
If this operator is absent, we derive the singular interpretation; if the operator is present, we obtain the plural reading. Note in passing that this distributivity operator is incompatible with subjects that have singular semantics (i.e., the operator cannot distribute the relevant predicate over just one entity). As such, unlike \textit{Sie}, \textit{du} ‘you(\textsc{sgl})’ combined with a non-plural noun\textit{ }(e.g., \textit{Du bist Arzt} ‘You are a doctor’) can only be singular in interpretation.
\end{styleStandard}

\begin{styleStandard}
\ \ As for (36b), I argued above that DPs always contain NumP. At first glance, this then appears to be a case of true morphological dis-agreement between the plural pronominal determiner and the singular nominal. In Section 3.2, similar cases were ruled out by suggesting that the singular nominal cannot establish an agreement relation with a plural pronominal determiner. Now, in order to explain the unexpected grammaticality of this example, I suggested in Roehrs (2006b) that these types of DPs have a structural analysis different from the canonical DPs discussed earlier. 
\end{styleStandard}

\begin{styleStandard}
Specifically, it is proposed in that paper that this construction involves two nominals, a matrix nominal containing the pronominal determiner and an embedded nominal containing the dis-agreeing predicate nominal. The latter is proposed to be located in a specifier inside the matrix pronominal DP. For lack of a better term, this position was labeled as specifier of a Disagreement Phrase (DisP) in Chapter 2, Section 3.5, where cases like \textit{ihr junges} \textit{Gemüse} ‘you young (vegetable =) folks’ were discussed. Abstracting away from movement, the pronominal DP in (36b), repeated below as (38a), can be illustrated as a first approximation as in (38b):
\end{styleStandard}

\begin{styleStandard}
(38) \ \ a.\ \ \textit{Sie \ Schwein}
\end{styleStandard}

\begin{styleStandard}
\ \ \ \ you pig.\textsc{neut}
\end{styleStandard}

\begin{styleStandard}
\ \ \ \ ‘you (idiot)’
\end{styleStandard}

\begin{styleStandard}
\ \ b.\ \ \textit{Apparent} \textit{Dis-agreement in the DP}
\end{styleStandard}

\begin{styleStandard}
\ \ \ \  \ \ \ \ \ \ DP\textsubscript{{\textless}e{\textgreater}}
\end{styleStandard}

\begin{styleStandard}
[Warning: Draw object ignored][Warning: Draw object ignored]
\end{styleStandard}

\begin{styleStandard}
\ \  \ \ \ \ \ \textit{Sie}\ \  \ \ \  \ \ \ \ \ DisP
\end{styleStandard}

\begin{styleStandard}
[Warning: Draw object ignored][Warning: Draw object ignored]
\end{styleStandard}

\begin{styleStandard}
\ \ \ \ [\textsubscript{NumP} Num\textsubscript{[-PL] }[\textsubscript{NP} \textit{Schwein}]]\textsubscript{{\textless}e,t{\textgreater}}\ \ Dis’
\end{styleStandard}

\begin{styleStandard}
[Warning: Draw object ignored][Warning: Draw object ignored]
\end{styleStandard}

\begin{styleStandard}
\ \ \ \ \ \ \ \ Dis\ \  \ \ \ \ \ \ \ \ NumP\textsubscript{{\textless}e,t{\textgreater}}
\end{styleStandard}

\begin{styleStandard}
[Warning: Draw object ignored][Warning: Draw object ignored]
\end{styleStandard}

\begin{styleStandard}
\ \ \ \ \ \ \ \  \ \ \ \ \ \ \ \ \ Num\textsubscript{[+PL]} \ \ NP
\end{styleStandard}

\begin{styleStandard}
\ \ \ \ \ \ \ \ \ \ \ \ \ \ \textit{e}\textit{\textsubscript{N}}
\end{styleStandard}

\begin{styleStandard}
Having claimed above that NumP has a specification for morphological number, I propose that both the matrix and the embedded nominal have a NumP where the value of the former is plural, but the value of the latter is singular. In order to account for the absence of a featural clash between these two nominals in (38b), I blame the lack of obligatory concord between these two nominals on the absence of a head (Mod) that mediates concord in agreement features between these nominals (for more details and other cases, see Chapter 8, Section 3.1). To be clear, then, both nominals in (38b) form independent agreement domains, and consequently, there is no true morphological dis-agreement. These are the basic assumptions of pronominal DPs involving \textit{Sie} ‘you’. Next, I flesh out some of the details of the derivation in (38b), specifically those about number.
\end{styleStandard}

\begin{styleStandard}
3.3.3. Pronominal DPs Involving \textit{Sie} in More Detail
\end{styleStandard}

\begin{styleStandard}
As briefly alluded to above, the two number values, morphological and semantic, can also diverge; for instance, pluralia tantum nouns such as \textit{Ferien} ‘vacation/holidays’ are morphologically plural but semantically singular (i.e., they involve one entity or event).\footnote{\ For Pesetsky \& Torrego (2007: 263), English pluralia tantum nouns like \textit{scissors} are also taken to have a lexically valued feature for number (see also Borer 2005: 105).} In other words, they are lexically specified such that they can only occur in the corresponding context in (39a). Furthermore, I assume that \textit{Sie} ‘you’ has a specification for morphological number but that it does not have a fixed value for semantic number – it can be used to address one or several individuals (39b):\footnote{\ Note that it might be possible to restate the restriction to a certain morphological number in (39a-b) by employing restriction features (see the discussion of uninflected \textit{dies} ‘this’ in Chapter 4, Section 5 and Chapter 8, Sections 2.2.5 and 2.2.6). However, I continue with the system laid out in Roehrs (2006b). Notice also that since \textit{Sie} can be merged in semantically singular or plural contexts, the specification for semantic number involving a variable in (39b) could presumably be left out. For reasons of parsimony, I continue with the statement in (39b). Interestingly, the Dutch counterpart \textit{u} ‘you(\textsc{formal})’ is morphologically singular and semantically singular or plural. On current assumptions, its feature specification would be [-PL morph; $\beta $PL sem].} 
\end{styleStandard}

\begin{styleStandard}
(39)\ \ a.\ \ pluralia tantum noun\ \ 
\end{styleStandard}

\begin{styleStandard}
\ \ \ \ [+PL morph; -PL sem]
\end{styleStandard}

\begin{styleStandard}
\ \ b.\ \ \textit{Sie}
\end{styleStandard}

\begin{styleStandard}
\ \ \ \ [+PL morph; $\beta $PL sem]
\end{styleStandard}

\begin{styleStandard}
To be clear, compared to regular nouns (Section 3.3.1), these elements have different values for morphological and semantic number, and (consequently) they lack the equality statement “$\alpha $ = $\beta $”. Furthermore, while both values are fixed with pluralia tantum nouns, the value for semantic number of \textit{Sie} is unspecified. I assume that this value is determined by the head noun in the complement position of \textit{Sie} mediated by NumP as discussed in Section 3.3.1. To be clear, these lexically specified elements can only be inserted in contexts that are compatible with the featural specification of their vocabulary entries. Before I return to the intriguing case of \textit{Sie} occurring with a non-plural noun, I discuss the derivation of \textit{Sie} in the environment of a plural noun (where both elements agree in morphological number). This lays the groundwork for the remaining discussion.
\end{styleStandard}

\begin{styleStandard}
\ \ Considering (40a), note that all number values of \textit{Sie} in (40b) are positive. The morphological number [+PL morph] on \textit{Sie} is compatible with the morphological number on Num. The semantic number [$\beta $PL sem] on \textit{Sie} is positive. Specifically, due to the interaction of the head noun and Num, the semantic number on the noun is positive and, consequently, only a semantically plural number on \textit{Sie} is compatible with a semantically plural number on the noun:
\end{styleStandard}

\begin{styleStandard}
(40)\ \ a.\ \ \textit{Sie \ \ \ \ \ \ \ \ Schweine}
\end{styleStandard}

\begin{styleStandard}
\ \ \ \ you(\textsc{pl}) pigs\ \ \ \ \ \ 
\end{styleStandard}

\begin{styleStandard}
\ \ \ \ ‘you idiots’
\end{styleStandard}

\begin{styleStandard}
\ \ b.\ \  \ DP
\end{styleStandard}

\begin{styleStandard}
[Warning: Draw object ignored][Warning: Draw object ignored]
\end{styleStandard}

\begin{styleStandard}
\ \  \ \textit{Sie}\ \ \ \ NumP
\end{styleStandard}

\begin{styleStandard}
\textsubscript{\ \ [+PL morph; +PL sem}[Warning: Draw object ignored][Warning: Draw object ignored]\textsubscript{]}
\end{styleStandard}

\begin{styleStandard}
\ \ \ \ Num\ \ \ \  NP
\end{styleStandard}

\begin{styleStandard}
[Warning: Draw object ignored][Warning: Draw object ignored]\ \ \ \ \ \ \ \  \ t\textsubscript{k}
\end{styleStandard}

\begin{styleStandard}
\textit{\ \ \ \ \ \ \ \ Schweine}\textsubscript{k}\ \ Num\ \ \ \  \ \ 
\end{styleStandard}

\begin{styleStandard}
\textsubscript{\ \ [+PL morph; +PL sem; $\alpha $ = $\beta $]}\ \ \textsubscript{[+PL morph] \ \ \ \ \ \ \ \ \ }
\end{styleStandard}

\begin{styleStandard}
This derivation involving formal \textit{Sie} ‘you’ is similar to informal \textit{ihr Schweine} ‘you idiots’, briefly discussed in Section 3.3.1. With this in place, I provide more details of the derivation in (38b), where \textit{Sie} occurs in the context of a non-plural noun. 
\end{styleStandard}

\begin{styleStandard}
For easier readability, I illustrate the pronominal DP in (41a) in two parts: (41b) shows the matrix nominal, and (42) further below illustrates the embedded part XP located in Spec,DisP in (41b). Starting with the matrix nominal, I assume that the complement of \textit{Sie} in (41a) is not \textit{Schwein} itself but involves a null pluralia tantum noun. Since both \textit{Sie} and this type of noun are lexically specified for [+PL morph], both elements are morphologically compatible. However, a pluralia tantum noun is only compatible with a negative value on \textit{Sie}’s [$\beta $PL sem] statement. This leads to singular semantics of \textit{Sie}:\footnote{\ This is different from the copular case \textit{Sie sind Arzt} ‘(you are doctor =) you (all) are doctors’ discussed earlier. Here, \textit{Sie}, the subject DP, has an ordinary null noun as part of its complement structure. This null noun has the familiar specification [$\alpha $PL morph; $\beta $PL sem; $\alpha $ = $\beta $]. Mediated by NumP, the subject DP (i.e., \textit{Sie} Num e\textsubscript{N}) gets plural semantics (note also that given the null elements, this makes \textit{Sie} structurally similar to \textit{Sie Schweine} ‘you idiots’). The subject DP is then combined with the non-plural predicate noun \textit{Arzt} ‘doctor’ via the head Pr and the auxiliary (Chapter 6). An assumed null distributivity operator accounts for the compatibility of plural \textit{Sie} and the non-plural noun (cf. (37)).}
\end{styleStandard}

\begin{styleStandard}
(41)\ \ a.\ \ \textit{Sie \ Schwein}
\end{styleStandard}

\begin{styleStandard}
\ \ \ \ you pig.\textsc{neut}
\end{styleStandard}

\begin{styleStandard}
\ \ \ \ ‘you (idiot)’
\end{styleStandard}

\begin{styleStandard}
\ \ b.\ \  \ DP\textsubscript{{\textless}e{\textgreater}}
\end{styleStandard}

\begin{styleStandard}
[Warning: Draw object ignored][Warning: Draw object ignored]
\end{styleStandard}

\begin{styleStandard}
\ \ \textit{Sie}\textsubscript{{\textless}{\textless}e,t{\textgreater}e{\textgreater}} \ \ \ DisP\textsubscript{{\textless}e,t{\textgreater}}
\end{styleStandard}

\begin{styleStandard}
\textsubscript{\ \ [+PL morph; -PL sem] \ \ \ \ \ \ \ }[Warning: Draw object ignored][Warning: Draw object ignored]
\end{styleStandard}

\begin{styleStandard}
\ \ \ \  \ XP\textsubscript{{\textless}e,t{\textgreater}}\ \ Dis’
\end{styleStandard}

\begin{styleStandard}
[Warning: Draw object ignored][Warning: Draw object ignored]
\end{styleStandard}

\begin{styleStandard}
\ \ \ \ \ \ Dis\ \  \ \ \ \ \ \ \ \ NumP\textsubscript{{\textless}e,t{\textgreater}}
\end{styleStandard}

\begin{styleStandard}
[Warning: Draw object ignored][Warning: Draw object ignored]
\end{styleStandard}

\begin{styleStandard}
\ \ \ \ \ \  \ \ \ \ \ \ \ \ \ Num\textsubscript{\ \ } \ \ NP
\end{styleStandard}

\begin{styleStandard}
[Warning: Draw object ignored][Warning: Draw object ignored]\ \ \ \ \ \ \ \ \ \ \ \  \ t\textsubscript{i}
\end{styleStandard}

\begin{styleStandard}
\ \ \ \ \ \ \textit{e}\textit{\textsubscript{N}}\textsubscript{i\ \ } \ \ \ \ \ \ \ \ Num
\end{styleStandard}

\begin{styleStandard}
\textsubscript{\ \ \ \ \ \ \ \ \ \ \ \ \ \ \ \ \ \ \ [+PL morph; -PL sem] \ \ \ \ \ \ \ [+PL morph]}
\end{styleStandard}

\begin{styleStandard}
As for the embedded nominal, this NumP – distinguished by the subscript 2 below – is specified as [-PL morph], and the ordinary noun \textit{Schwein} comes out as a singular count element:
\end{styleStandard}

\begin{styleStandard}
(42)\ \ \ \ \ \  \ NumP\textsubscript{2} (= XP)
\end{styleStandard}

\begin{styleStandard}
[Warning: Draw object ignored][Warning: Draw object ignored]
\end{styleStandard}

\begin{styleStandard}
\ \  \ \ \ \ \ \ \ \ \ \  \ Num’\textsubscript{2}
\end{styleStandard}

\begin{styleStandard}
[Warning: Draw object ignored][Warning: Draw object ignored]
\end{styleStandard}

\begin{styleStandard}
\ \  \ \ \  \ \ \ \ \ \ \ Num\textsubscript{2\ \ }\ \  \ \ \ \ \ \ NP \ 
\end{styleStandard}

\begin{styleStandard}
[Warning: Draw object ignored][Warning: Draw object ignored]\ \ \ \ \ \ \ \ \ \  \ \ \ \ \ \ \ \ t\textsubscript{k}
\end{styleStandard}

\begin{styleStandard}
\ \ \ \textit{Schwein}\textsubscript{k} \ \ \ \ \  \ \ \ \ Num\textsubscript{2}
\end{styleStandard}

\begin{styleStandard}
\textsubscript{\ \ \ \ \ \ \ \ \ \ \ [-PL morph; {}-PL sem; $\alpha $ = $\beta $] \ \ \  \ \ \ [-PL morph]}
\end{styleStandard}

\begin{styleStandard}
To recapitulate, the matrix nominal involves a plural NumP and a (null) pluralia tantum head noun, and the embedded nominal contains a singular NumP and an (overt) ordinary head noun. In other words, although there are two NumPs in (41b) and (42) with different (morphological) specifications for number, both nominals are semantically singular. I now turn to the question as to how the two nominals in (41b) and (42) are semantically combined.
\end{styleStandard}

\begin{styleStandard}
\ \ I assume that the null noun in the matrix nominal is of the kind type. For this element to function as a predicate, it must combine with REL in NumP. Suppose now that the matrix nominal and the embedded nominal in (41b) and (42) are combined by Predicate Modification, which conjoins two elements of the same semantic type (i.e., {\textless}e,t{\textgreater}). In order to avoid type mismatch, the embedded predicate nominal must be of the same type. This is true. With the kind noun \textit{Schwein} ‘(pig =) idiot’ present there, the latter must also involve REL in NumP (to bring about a predicate), which is in line with the assumptions of Chapter 6. Thus, we have justification for the assumption of two NumPs, one in the matrix nominal and one in the embedded one. Furthermore, assuming that Predicate Modification only combines two nominals of the same general semantics, I can also account for the fact that the semantically singular nominal in Spec,DisP is only compatible with a semantically singular matrix nominal (for more details, see Roehrs 2006b: 168). This explains the lack of a plural reading in \textit{Sie Schwein} ‘you (pig =) idiot’.
\end{styleStandard}

\begin{styleStandard}
To sum up, following the literature, I suggested that morphological number originates with NumP. Furthermore, I proposed that semantic number is the result of an intricate interplay between Num and the head noun. In addition, I considered some non-canonical cases in copular TPs and pronominal DPs, where \textit{Sie} occurs in the context of a non-plural noun. I explained the grammaticality of \textit{Sie} in the copular cases in the same fashion as in Section 3.2, namely by the syntactic optionality of NumP in the clause: in this particular instance, NumP is absent. As for the pronominal DPs, I accounted for the grammaticality of \textit{Sie} here by proposing a different structure where the pronominal determiner is in the matrix DP, but the non-plural noun forms a complex specifier embedded in the matrix DP. 
\end{styleStandard}

\begin{styleStandard}
More generally, I proposed that NumP is present for cartographic reasons in noun phrases bigger than NP. This means that all noun phrases (except those with mass and bare role nouns) involve morphological number. Semantic number may come about in one of three ways: (i) it is lexically specified on the head noun (pluralia tantum), (ii) it originates with the (regular) kind noun if [$\alpha $PL morph; $\beta $PL sem] is specified via the morphological value of Num and the equality statement “$\alpha $ = $\beta $”, or (iii) semantic number of bare role nouns in copular sentences stems from the pragmatics (in the singular cases; see Footnote Error: Reference source not found) or a distributivity operator (in the plural cases; see (37) above). 
\end{styleStandard}

\begin{styleStandard}
If correct, this discussion leads to some interesting issues.\footnote{\ I mention just one issue in this context. As pointed out in the main text, the kind noun in the embedded nominal in (42) requires the presence of REL to supply a predicate. REL, in turn, should trigger the presence of \textit{ein}. However, this yields an ungrammatical example:\par \ \ (i) \ * \ \ \textit{Sie }[\textsubscript{ArtP}\textit{ ‘n} \textit{Schwein }] \textit{e}\textit{\textsubscript{N}}\par \ \ \ \ you \ \ \ \ \ \ \ \textsubscript{\ }a pig.\textsc{neut}\par To explain the ungrammaticality, there is more involved here than in cases like *\textit{du ‘n Schwein }‘you a (pig =) idiot’ (see Section 3.2.5) – notice that \textit{Sie} and \textit{ein} in (i) are in different nominals. Without going into too much detail, I suggest that \textit{ein} is deleted in the presence of the adjacent pronominal determiner. Note in this regard that presumably, the latter is retained as it has semantics unlike vacuous \textit{ein}.} I will not pursue these questions here. My main goal in this section was to show that elements other than \textit{ein} can be held responsible for morphological and semantic number. If this is tenable, then I can continue claiming that \textit{ein} is not a reflex of number, be it morphological or semantic – it is a semantically vacuous element (Hypothesis 1a). Next, I consider constructions involving \textit{als}{}-nominals in more detail. While these instances have certain similarities to the dis-agreement cases discussed above, they are different in a number of ways.
\end{styleStandard}

\begin{styleStandard}
\textit{3.4.\ \ Agreement in Constructions Involving als}{}-nominals
\end{styleStandard}

\begin{styleStandard}
Finally, I discuss \textit{als}{}-constructions, nominal and clausal constructions involving \textit{als} ‘as’ and a following noun, the latter two elements labeled as \textit{als}{}-nominals. I begin by showing how \textit{als}{}-nominals combine with the preceding part of their structures. This includes a brief discussion of the semantic types and \textit{als} as a realization operator. Then I address the agreement facts with regard to number.
\end{styleStandard}

\begin{styleStandard}
3.4.1. Structure and Semantics
\end{styleStandard}

\begin{styleStandard}
In Chapter 6, Section 3.4.2, I offered the tentative claim that \textit{als} ‘as’ in the DP is a generalized capacity operator that combines with a noun, be it a role or a kind noun, to return an element of type {\textless}e,t{\textgreater}. I suggested there that syntactically, the \textit{als}{}-nominal is right-adjoined to its preceding structure and that semantically, this \textit{als}{}-nominal combines with another element by Predicate Modification (which combines two elements of type {\textless}e,t{\textgreater}). As such, I proposed that the \textit{als}{}-nominal is adjoined to an element of type {\textless}e,t{\textgreater}. Recalling that \textit{du} ‘you’ itself is argued to have internal structure, NumP is such an element. I also suggested in that section that \textit{als} is the head of ModP and I proposed that this ModP is adjoined to NumP of the pronominal DP. The resultant structure is slightly adapted here, deriving (43a) as in (43b): 
\end{styleStandard}

\begin{styleStandard}
(43)\ \ a.\ \ \textit{du \ \ \ \ \ \ \ \ \ \ als Arzt}
\end{styleStandard}

\begin{styleStandard}
\ \ \ \ you(\textsc{sgl}) as \ doctor.\textsc{masc}
\end{styleStandard}

\begin{styleStandard}
\ \ \ \ ‘you as a doctor’
\end{styleStandard}

\begin{styleStandard}
\ \ b.\ \  \ \ \ DP\textsubscript{{\textless}e{\textgreater}}
\end{styleStandard}

\begin{styleStandard}
[Warning: Draw object ignored][Warning: Draw object ignored]
\end{styleStandard}

\begin{styleStandard}
\ \ \textit{du}\textsubscript{{\textless}{\textless}e,t{\textgreater}e{\textgreater}} \ \ \ \ \ \ \ \ \ \ \ NumP\textsubscript{{\textless}e,t{\textgreater}}
\end{styleStandard}

\begin{styleStandard}
[Warning: Draw object ignored][Warning: Draw object ignored]\ \ \ \ \ \ \ \ \ \ \ \ \ \ (Predicate Modification)
\end{styleStandard}

\begin{styleStandard}
\ \  \ \ \ \ \ \ \ NumP\textsubscript{{\textless}e,t{\textgreater}}\ \  \ \ \ \ \ \ \ \ \ \ \ \ \ \ \ \ \ \ ModP\textsubscript{{\textless}e,t{\textgreater}}
\end{styleStandard}

\begin{styleStandard}
[Warning: Draw object ignored][Warning: Draw object ignored][Warning: Draw object ignored][Warning: Draw object ignored]\ \ \ \  \ \ \ \ \ 
\end{styleStandard}

\begin{styleStandard}
\ \ \ \ \ \ Num\textsubscript{REL{\textless}e{\textless}e,t{\textgreater}{\textgreater}} \ \ \ \ NP \ \ \ \ \ \ \textit{als}\ \  \ \ \ \ \ \ \ \ NP
\end{styleStandard}

\begin{styleStandard}
\ \  \ \ \ \ \ \ \ \ \ \ \ \ \ \ \ \ \ \ \ \ \textit{e}\textit{\textsubscript{N}}\textsubscript{{\textless}e{\textgreater}}\ \  \ \ \ \ \ \ \ \ \ \ \ \ \ \ \ \ \ \ \textit{Arzt}\textsubscript{{\textless}e{\textgreater}}
\end{styleStandard}

\begin{styleStandard}
As for the non-copular TP, de Swart \textit{et al}. (2007: 218) assume that \textit{als}{}-nominals are verb modifiers. I propose that the \textit{als}{}-nominal in the clausal cases is adjoined to the VP. Note that intransitive verbal predicates are of type {\textless}e,t{\textgreater}. Semantically, this is the same type as that of \textit{als}{}-nominals. The simplified structure of (44a) is provided in (44b):
\end{styleStandard}

\begin{styleStandard}
(44)\ \ a.\ \ \textit{Du \ \ \ \ \ \ \ \ \ }\textit{\textsubscript{\ }}\textit{sprichst als Arzt.}
\end{styleStandard}

\begin{styleStandard}
\ \ \ \ you(\textsc{sgl}) speak \ \ \ \ as \ doctor.\textsc{masc}
\end{styleStandard}

\begin{styleStandard}
\ \ \ \ ‘You speak as a doctors.’
\end{styleStandard}

\begin{styleStandard}
\ \ b.\ \ TP
\end{styleStandard}

\begin{styleStandard}
[Warning: Draw object ignored][Warning: Draw object ignored]
\end{styleStandard}

\begin{styleStandard}
\textit{\ \ \ \ \ \ \ \ \ \ \ \ du}\textsubscript{k} \ \ \ \ \ \ \ \ \ \ \ \ \ \ \ \ T’
\end{styleStandard}

\begin{styleStandard}
[Warning: Draw object ignored][Warning: Draw object ignored]
\end{styleStandard}

\begin{styleStandard}
\ \  \ \ \ \ \ \ \ \ \textit{sprichst}\textsubscript{i}\ \  \ \ \ \ \ \ \ \textit{v}P
\end{styleStandard}

\begin{styleStandard}
[Warning: Draw object ignored][Warning: Draw object ignored]\ \  \ \ \ \ \ \ \ \ 
\end{styleStandard}

\begin{styleStandard}
\ \ \ \  \ \ \ \ \ \ \ \ \ \ t\textsubscript{k}\ \  \ \ \ VP\textsubscript{{\textless}e,t{\textgreater}}
\end{styleStandard}

\begin{styleStandard}
[Warning: Draw object ignored][Warning: Draw object ignored]\ \  \ \ \ \ \ \ \ 
\end{styleStandard}

\begin{styleStandard}
\ \  \ \ \ \ \ \ \ \ \ \ \ \ \ \ \  \ \ \ \ \ VP\textsubscript{{\textless}e,t{\textgreater}}\ \ [\textsubscript{ModP }\textit{als Arzt}]\textsubscript{{\textless}e,t{\textgreater}}
\end{styleStandard}

\begin{styleStandard}
\ \ \ \ \ \  \ \ \ \ \ \ \ t\textsubscript{i}
\end{styleStandard}

\begin{styleStandard}
These are the basic assumptions about the syntax and semantics. Before moving on to the agreement facts, I specify the role of \textit{als }in more detail.
\end{styleStandard}

\begin{styleStandard}
\ \ In Chapter 6, we saw that CAP and REL are both possible with role nouns. Recall that CAP involves direct predication but that REL involves an indirect path to predication. Note now that these realization operators are mutually exclusive – only one is needed to supply a predicate, and the presence of a second would lead to type mismatch. In other words, CAP must be, at least in principle, optional. I now make the assumption that all realization operators are optional in type but not in number; that is, while there is a potential choice between different operators, the presence of one such operator is required to supply a predicate. This might lead us to expect that there are other realization operators. 
\end{styleStandard}

\begin{styleStandard}
\ \ As tentatively suggested in Chapter 6, Section 3.4.2, I take \textit{als} to be such an element. Similar to the two other elements, if \textit{als} is present, both CAP and REL cannot be. I reiterate the proposal that \textit{als} takes all elements of type {\textless}e{\textgreater}, that is, role nouns or kind nouns alike, as arguments and returns predicate nominals (type {\textless}e,t{\textgreater}). In other words, \textit{als} is a general(ized) capacity operator of type {\textless}e,{\textless}e,t{\textgreater}{\textgreater}. While of the same semantic type, this makes this element different from CAP and REL. I turn to the agreement facts.
\end{styleStandard}

\begin{styleStandard}
3.4.2. Agreement 
\end{styleStandard}

\begin{styleStandard}
Starting with the (non-copular) TPs, I assume that agreement in morphological number between the subject and the \textit{als}{}-nominal is established independently of the presence of the verb. In a way, this makes the adjunction of the \textit{als}{}-nominal in the clause similar to a displaced relative clause – in both instances, the two related elements are separated by a verb.\footnote{\ Below is an example of a displaced relative clause:\par \ \ (i)\ \ \textit{Er hat das Haus gekauft, das nicht so teuer \ \ \ \ \ \ \ \ war}.\par \ \ \ \ he has the house bought \ that not \ \ \ so expensive was\par \ \ \ \ ‘He bought the house that was not so expensive.’} Note that this parallel pattern fits well with the discussion in Chapter 6, Section 3.4.1, where I argued that \textit{als}{}-nominals in pronominal DPs are similar to relative clauses. Consequently, I treat the \textit{als}{}-nominals in the non-copular TPs and the ones in the pronominal DPs in a related way. Below, I put the verb in parentheses discussing both the TPs and the DPs at the same time.
\end{styleStandard}

\begin{styleStandard}
We saw in Section 3.1 that plural morphology on the noun indicates the presence of NumP. Assuming that the number specification of the predicate nominal inside the \textit{als}{}-nominal has to establish an agreement relation with the pronominal element, this rules out (45a) but allows (45b):
\end{styleStandard}

\begin{styleStandard}
(45)\ \ a. *\ \ \textit{du \ \ \ \ \ \ \ \ \ (sprichst) als Ärzte}
\end{styleStandard}

\begin{styleStandard}
\ \ \ \ you(\textsc{sgl}) speak \ \ \ \ \ as \ \ doctors
\end{styleStandard}

\begin{styleStandard}
\ \ b. \ \ \textit{ihr \ \ \ \ \ }\textit{\textsubscript{\ \ }}\textit{(sprecht) als Ärzte}
\end{styleStandard}

\begin{styleStandard}
\ \ \ \ you(\textsc{pl}) speak \ \ \ \ as \ doctors
\end{styleStandard}

\begin{styleStandard}
\ \ \ \ ‘you (speak) as doctors’
\end{styleStandard}

\begin{styleStandard}
In contrast, a non-plural noun in the \textit{als}{}-nominal does not project NumP. It involves NP and does not enter into an agreement relation with the pronominal element. As such, both (46a) and (46b) are fine:
\end{styleStandard}

\begin{styleStandard}
(46)\ \ a.\ \ \textit{du \ \ \ \ \ \ \ \ \ (sprichst) als Arzt}
\end{styleStandard}

\begin{styleStandard}
\ \ \ \ you(\textsc{sgl}) speak \ \ \ \ \ as \ doctor.\textsc{masc}
\end{styleStandard}

\begin{styleStandard}
\ \ \ \ ‘you (speak) as a doctor’
\end{styleStandard}

\begin{styleStandard}
\ \ b.\ \ \textit{ihr \ \ \ \ \ \ }\textit{\textsubscript{\ }}\textit{(sprecht) als Arzt}
\end{styleStandard}

\begin{styleStandard}
\ \ \ \ you(\textsc{pl}) speak \ \ \ \ as \ \ doctor.\textsc{masc}\ \ \ \ \ \ 
\end{styleStandard}

\begin{styleStandard}
\ \ \ \ ‘you (speak) as doctors’
\end{styleStandard}

\begin{styleStandard}
I assume that the plural interpretation in (46b) follows from the presence of a null distributivity operator just like in the copular cases. So far, the agreement facts are similar to what we saw above. 
\end{styleStandard}

\begin{styleStandard}
\ \ Turning to singular nouns, I observed in Section 2.3 that the presence of \textit{ein} leads to awkwardness in a singular context (47a). As also pointed out in that section, a singular \textit{als}{}-nominal is ungrammatical with a plural pronominal (47b), where the presence of a non-copular verb and quantifier yields a somewhat better result than the counterpart without those two elements (this correlation in (47b) is indicated by two sets of parentheses):
\end{styleStandard}

\begin{styleStandard}
(47)\ \ a. \ ?\ \ \textit{du \ \ \ \ \ \ \ \ \ \ \ (sprichst) als ‘n Arzt}
\end{styleStandard}

\begin{styleStandard}
\ \ \ \ you(SGL) speak \ \ \ \ \ as \ \textsubscript{\ }\ a doctor.\textsc{masc}
\end{styleStandard}

\begin{styleStandard}
\ \ \ \ ‘you (speak) as a doctor’
\end{styleStandard}

\begin{styleStandard}
\ \ b.(??)?*\textit{\ \ ihr \ \ \ \ \ \ }\textit{\textsubscript{\ }}\textit{(sprecht alle) als ‘n Arzt}
\end{styleStandard}

\begin{styleStandard}
\ \ \ \ you(PL) speak \ \ \ all \ \ \ as \ \ \ a doctor.\textsc{masc}\ \ \ \ \ \ 
\end{styleStandard}

\begin{styleFootnote}
The agreement facts in (47) are similar to the copular constructions discussed in Section 3.2 (cf. also Tables 1 and 2). There is only one exception: the presence of \textit{ein} in a copular construction such as \textit{Du bist ‘n Arzt} ‘You are a doctor’ is completely fine for many speakers, but the presence of \textit{ein} in an \textit{als}{}-nominal like (47a) leads to a certain degree of awkwardness. Note though that the latter case involves \textit{als} ‘as’.
\end{styleFootnote}

\begin{styleFootnote}
In order to account for (47), I propose that syntactically, \textit{als} has intermediate characteristics as regards CAP and REL. To see this, I briefly return to (45) and (46). I suggest for (45) that \textit{als} is in NumP with a plural noun (similar to REL) but I assume for (46) that it is in NP with a non-plural noun (similar to CAP) – the first case will force morphological agreement, the second will not.\footnote{\ This means that Mod takes different complements (NP, NumP) and that \textit{als} might move to Mod.} This rules out the plural noun in (45a) but allows the non-plural noun in (46b). Turning to the singular nouns in (47), I propose that \textit{als} does not trigger \textit{ein}. This makes \textit{als} similar to CAP and explains the markedness of (47a). As to the more marked (47b), \textit{als} does not bring about \textit{ein} either, and additionally, no agreement relation between the plural pronominal element and the singular noun in the \textit{als}{}-nominal is established.
\end{styleFootnote}

\begin{styleStandard}
\ \ To sum up the entire section, the preceding discussion revealed an interesting syntax-semantics correlation: cases of morphological agreement are very restricted in their interpretations, but instances of apparent morphological dis-agreement (copular cases and\textit{ als-}nominals) are fairly free. Given the proposed interaction between Num and the head noun, this difference follows from the presence of NumP in the first cases but its absence in the second. Importantly, the role of \textit{ein} is only indirect. Its presence allows us to draw conclusions about the syntactic structure and the type of realization operator present in that structure (i.e., REL). As seen throughout this section, \textit{ein} does not specifify number, neither morphologically nor semantically.
\end{styleStandard}

\begin{styleStandard}\bfseries
4.\ \ Conclusion
\end{styleStandard}

\begin{styleStandard}
This chapter provided the second and final consequence of the proposals involving \textit{ein}, here focusing on number. Continuing the investigation from the previous chapter, I provided a survey of the data and discussed issues pertaining to morphological and semantic number in the nominal and clausal domains. I showed that the DP is more restricted in this regard than the TP. I proposed that this is accounted for by the obligatory presence of NumP in the DP (a reflex of the cartography of the DP) and its optional syntactic presence in the TP. 
\end{styleStandard}

\begin{styleStandard}
Specifically, I proposed that morphological number resides in Num and semantic number is a result of an intricate interaction between Num and the head noun. With current purposes in mind, I can conclude that \textit{ein} itself does not determine number, neither morphologically nor semantically. Rather, we have seen that \textit{ein} indicates the presence of a certain amount of structure on top of NP (Hypothesis 3a) and that it flags the presence of the operator REL (Hypothesis 3b). I also showed that elements other than \textit{ein} can flag the presence of REL precluding \textit{ein} from occurring. Again, these facts find a natural explanation if \textit{ein} is semantically vacuous (Hypothesis 1a); that is, \textit{ein} is an element that can be absent without a loss in meaning. 
\end{styleStandard}

\begin{styleStandard}
\ \ More generally, to the extent that the discussion here proves tenable, it provides further support for de Swart \textit{et al}.’s (2007) proposal, and more generally, it strengthens the parallels between the DP and TP. 
\end{styleStandard}

\clearpage\setcounter{page}{323}\begin{styleStandard}
Chapter 8: Discussion and Concluding Remarks
\end{styleStandard}

\begin{styleStandard}\bfseries
1.\ \ The Bigger Picture
\end{styleStandard}

\begin{styleStandard}
Focusing on German, one goal of this book was to provide a more comprehensive discussion of adjectival inflections and \textit{ein} compared to that which exists in the literature. Beside an overview of these important empirical subdomains of the noun phrase, the second goal was to find commonalities and differences between these two types of elements and to discuss certain theoretical issues that arise in these contexts. It is important to point out again that both goals go in partially different directions. 
\end{styleStandard}

\begin{styleStandard}
Providing an overview aims to be fairly exhaustive, but painting the big picture is an attempt to see what different domains, empirical or theoretic, have in common. As such, discussing issues to some comprehensive degree in one domain tends to move the focus away from traits shared by the different domains. In this final chapter, I try to make more headway toward the second goal; that is, toward comparing both adjectival inflections and \textit{ein} more directly. In the second part of this chapter, I discuss some further consequences of the current analysis, show avenues for future research, and draw some more general conclusions. Before I compare adjectival inflections and \textit{ein} in more detail, recall the general contents of the previous chapters. 
\end{styleStandard}

\begin{styleStandard}
In Chapter 1, I laid out my general assumptions and formulated the main claims of this book. Chapter 2 focused on adjectival inflections and included a detailed investigation of the special role of \textit{ein}{}-words in this regard. In Chapter 3, variation was discussed leading to the postulation of two secondary mechanisms that regulate adjectival endings, and in Chapter 4, some consequences of the analysis for other accounts were addressed. Chapter 5 provided an overview of the different types of \textit{ein}. This was followed by \textit{ein} being discussed with regard to emotiveness and number in Chapters 6 and 7, respectively. The analysis of adjectival inflections developed earlier helped narrow down the choices of plausible analyses in later chapters.
\end{styleStandard}

\begin{styleStandard}
Contrasting these phenomena, we observed that the analysis of adjectival inflections and that of \textit{ein} are closely connected. In fact, I argued that adjectival inflections and \textit{ein} are similar in certain aspects but different in others. I review the main hypotheses discussed throughout this book starting with the similarities followed by the differences. Besides providing a summary and drawing some conclusions, I add a few other details.
\end{styleStandard}

\begin{styleStandard}
\textit{1.1. \ \ Commonalities of Adjectival Inflections and }ein
\end{styleStandard}

\begin{styleStandard}
I formulated the following two main claims:
\end{styleStandard}

\begin{styleStandard}
(1) \ \ \textit{Hypothesis 1}
\end{styleStandard}

\begin{styleStandard}
Adjectival inflections and \textit{ein}:
\end{styleStandard}

\begin{styleStandard}
\ \ \  \ \ a. \ \ \ \ are expletive elements and
\end{styleStandard}

\begin{styleStandard}
\ \ \  \ \ b. \ \ \ \ indicate abstract structure in the noun phrase.
\end{styleStandard}

\begin{styleStandard}
Beginning with Hypothesis 1a, we have seen that both adjectival endings and \textit{ein} are not a reflex of (in-)definiteness. While both elements share this characteristic, evidence for Hypothesis 1a has also come from other, different empirical domains. Specifically, adjectival inflections have nothing to do with the restrictiveness of the interpretation of modifiers or “referentiality”, and \textit{ein} is not an exponent of emotiveness and number. These points were summarized in the statement that adjectival inflections and \textit{ein} are semantically vacuous.
\end{styleStandard}

\begin{styleStandard}
\ \ As for Hypothesis 1b, both adjectival inflections and \textit{ein} indicate abstract structure. However, they do so in different ways. Compare Hypothesis 2a to Hypothesis 3a:
\end{styleStandard}

\begin{styleStandard}
(2)\ \ \textit{Hypothesis 2}
\end{styleStandard}

\begin{styleStandard}
\ \ Adjectival inflections:
\end{styleStandard}

\begin{styleStandard}
\ \ \ a.\ \ indicate abstract structure in the higher layers of the noun phrase (DP vs. LPP), \ \ \ \ \ \ [and] they provide clues about structures involving various degrees of embedding \ \ \ \ \ \ of adjectives (simple vs. complex DPs).
\end{styleStandard}

\begin{styleStandard}
(3)\ \ \textit{Hypothesis 3}
\end{styleStandard}

\begin{styleStandard}
\ \ \textit{Ein}:
\end{styleStandard}

\begin{styleStandard}
\ \ \ a.\ \ indicates abstract structure in the lower layers of the noun phrase (NP vs. ArtP).
\end{styleStandard}

\begin{styleStandard}
While both adjectival inflections and \textit{ein} indicate certain sizes of abstract structure, the former also provides clues about different types of embeddings. 
\end{styleStandard}

\begin{styleStandard}
In what follows, I briefly review some of the evidence for Hypotheses 1, 2a, and 3a below. Hypotheses 2b (adjectival endings make features visible) and 3b (\textit{ein} supports or flags operators), which deal with differences between adjectival inflections and \textit{ein}, are discussed in Section 1.2. I begin with the intersecting semantic concept of (in-)definiteness, where both types of elements not only show a similar lack of semantic import but also directly interact with one another. Following that, I repeat one more argument from the other, different empirical domains showing that both types of elements are semantically vacuous: adjectival inflections do not indicate the restrictiveness of the interpretation of modifiers, and \textit{ein }is not related to number. Finally, I review the main facts that show that both types of elements provide clues about abstract structure (the indication of different types of embeddings by adjectival inflections is left for the section dedicated to the differences – Section 1.2).
\end{styleStandard}

\begin{styleStandard}
1.1.1. Some Evidence for Hypothesis 1a
\end{styleStandard}

\begin{styleStandard}
Starting with (in-)definiteness, consider again cases in the nominative, the (a)-examples, and in the dative, the (b)-examples: 
\end{styleStandard}

\begin{styleStandard}
(4)\ \ a.\ \ \textit{ein gut-er \ \ \ Wein}
\end{styleStandard}

\begin{styleStandard}
\ \ \ \ a \ \ \ good-\textsc{st} wine.\textsc{masc}
\end{styleStandard}

\begin{styleStandard}
\ \ \ \ ‘a good wine’
\end{styleStandard}

\begin{styleStandard}
\ \ b.\ \ \textit{mit \ \ ein-em gut-en \ \ \ \ Wein}
\end{styleStandard}

\begin{styleStandard}
\ \ \ \ with a-\textsc{st} \ \ \ \ good-\textsc{wk} wine.\textsc{masc}
\end{styleStandard}

\begin{styleStandard}
\ \ \ \ ‘with a good wine’
\end{styleStandard}

\begin{styleStandard}
(5)\ \ a.\ \ \textit{sein gut-er \ \ \ Wein}
\end{styleStandard}

\begin{styleStandard}
\ \ \ \ his \ \ good-\textsc{st} wine.\textsc{masc}
\end{styleStandard}

\begin{styleStandard}
\ \ \ \ ‘his good wine’
\end{styleStandard}

\begin{styleStandard}
\ \ b.\ \ \textit{mit \ \ sein-em gut-en \ \ \ \ Wein}
\end{styleStandard}

\begin{styleStandard}
\ \ \ \ with his-\textsc{st} \ \ \ good-\textsc{wk} wine.\textsc{masc}
\end{styleStandard}

\begin{styleStandard}
\ \ \ \ ‘with his good wine’
\end{styleStandard}

\begin{styleStandard}
Traditionally, the article \textit{ein} ‘a’ is taken to be an indefinite element, but the possessive article \textit{sein} ‘his’ is a definite one. With that in mind, compare the (a)-examples to their (b)-counterparts. We observe that indefinite contexts exhibit strong and weak adjectives (4) but that definite environments also show strong and weak adjectives (5). Put differently and contrasting the (a)-examples to each other and the (b)-examples to each other, strong adjectives occur in both indefinite and definite contexts, and weak adjectives also surface in both indefinite and definite environments. 
\end{styleStandard}

\begin{styleStandard}
\ \ Furthermore, what is sometimes left unmentioned in this context is that determiners can have the exact same inflections as adjectives (and that is why I referred to these inflections as adjectival, rather than adjective, endings). Considering the (b)-examples above, we note that both the (indefinite) article and the (definite) possessive article have strong inflections as their own endings. Indeed, I discussed one case in Chapter 3, Section 4, where certain (definite) \textit{der}{}-words may optionally have a strong or a weak ending. Considering the distributions of these inflections, we can observe again that these are clearly contradictory patterns – there is no association of adjectival inflections with (in-)definiteness.
\end{styleStandard}

\begin{styleStandard}
\ \ Recall though that I proposed in Chapter 5 that possessive articles consist of a possessive component and \textit{ein} (e.g., \textit{s-ein} ‘his’). If we assume that \textit{ein} itself determines the adjectival inflections, then we can observe that \textit{ein} occurs with a strong adjective in the nominative but with a weak adjective in the dative. This would be consistent with the claim that \textit{ein} itself is indefinite, and the inflections on adjectives would vary according to morphological case. However, weak adjectives are also possible in the nominative (6a), and strong adjectives can also appear in the dative (6b):
\end{styleStandard}

\begin{styleStandard}
(6)\ \ a.\ \ \textit{seine gut-en \ \ \ \ Weine}
\end{styleStandard}

\begin{styleStandard}
\ \ \ \ his \ \ \ good-\textsc{wk} wines
\end{styleStandard}

\begin{styleStandard}
\ \ \ \ ‘his good wines’
\end{styleStandard}

\begin{styleStandard}
\ \ b.\ \ \textit{mit \ \ gut-em \ \ Wein}
\end{styleStandard}

\begin{styleStandard}
\ \ \ \ with good-\textsc{st} wine.\textsc{masc}
\end{styleStandard}

\begin{styleStandard}
\ \ \ \ ‘with good wine’
\end{styleStandard}

\begin{styleStandard}
Thus, there is no correlation between case and the strong/weak alternation on the adjective either. \ \ These contradictory points disappear once we claim that neither adjectival inflections nor \textit{ein} are related to (in-)definiteness (or case). Rather, both of these elements are semantically vacuous and interact with each other morpho-syntactically. I proposed in Chapter 2 that determiners including \textit{ein} trigger Impoverishment on the adjective yielding the weak endings. There is one exception: instances of \textit{ein} specified as [-N, -O] are special – in this featural context (along with the absence of positive features for definiteness and deixis), [+D] does not trigger Impoverishment on the adjective. It is clear that adjectival inflections and \textit{ein} are directly related albeit not by the semantics.
\end{styleStandard}

\begin{styleStandard}
\ \ Turning to the other, differing empirical domains, I point out again that adjectival endings do not indicate the restrictiveness of the interpretation of modifiers. Considering (7), notice that both a restrictive and a non-restrictive adjective can have a weak ending:
\end{styleStandard}

\begin{styleFootnote}
(7)\ \ a.\ \ \textit{der alt-e \ \ \ \ }\textit{\textsubscript{\ }}\textit{Mann}
\end{styleFootnote}

\begin{styleFootnote}
\ \ \ \ the \textsubscript{\ }old\textsc{{}-wk} man.\textsc{masc}
\end{styleFootnote}

\begin{styleFootnote}
‘the man that is old’
\end{styleFootnote}

\begin{styleFootnote}
b.\ \ \textit{der (übrigens) \ \ \ }\textit{\textsubscript{\ }}\textit{alt-e \ \ \ \ Mann}
\end{styleFootnote}

\begin{styleFootnote}
\ \ \ \ the \ \textsubscript{\ }incidentally old-\textsc{wk} man.\textsc{masc}
\end{styleFootnote}

\begin{styleFootnote}
‘the man, who is (by the way) old’
\end{styleFootnote}

\begin{styleStandard}
Recalling that determiners move from ArtP to DP, I proposed in Chapter 4 that the determiner is interpreted above the adjective in (7a) but below the adjective in (7b). The latter also involves a coindexed \textit{pro}. The weak endings on the adjectives are the result of Impoverishment applying in a canonical DP structure. This analysis is consistent with the fact that adjectival endings involve no semantics.
\end{styleStandard}

\begin{styleStandard}
\ \ As for \textit{ein}, I claimed that this element is not associated with number (and related countability). To illustrate this claim again, notice that the example in (8a) denotes a singularity, and the ones in (8b-c), taken from the Appendix, involve pluralities:
\end{styleStandard}

\begin{styleStandard}
(8)\ \ a.\ \ \textit{Nur} \ \textit{EINE Katze \ \ \ war auf dem Hof!}
\end{styleStandard}

\begin{styleStandard}
\ \ \ \ only one \textsubscript{\ \ \ \ }cat.\textsc{fem} was in \ \ the \ yard
\end{styleStandard}

\begin{styleStandard}
\ \ \ \ ‘Only one cat was in the yard!’
\end{styleStandard}

\begin{styleStandard}
\ \ b. \ \%\ \ \textit{Och~}\emph{so n-e \ süssen}\textit{~Katzis... :-)}
\end{styleStandard}

\begin{styleStandard}
oh, \ so a-\textsc{pl} cute \ \ \ kittens
\end{styleStandard}

\begin{styleStandard}
‘Oh, such cute kittens!’
\end{styleStandard}

\begin{styleStandard}
\ \ \ \ c. \ \%\ \ \textit{Gerade gesehen... was \ \ für eine Idioten, die \ \ so Aufmerksamkeit wollen}
\end{styleStandard}

\begin{styleStandard}
\ \ \ \ just \ \ \ \ \ \ seen \ \ \ \ \ \ \ \ \ what for \ a-\textsc{pl} idiots \ \ \ who so attention \ \ \ \ \ \ \ \ \ \ \ \ want
\end{styleStandard}

\begin{styleStandard}
\ \ \ \ ‘Just seen… what kind of idiots that want attention like that’
\end{styleStandard}

\begin{styleStandard}
Contrasting these two types of examples, we observe that \textit{ein} can appear in very diverse semantic contexts. Again, given that \textit{ein} appears to be associated with different, contradictory semantics, I proposed that this element is also semantically vacuous. I suggested in Chapter 5 that \textit{EIN} in (8a) consists of \textit{ein }and the null operator \textit{Ø}\textsubscript{[-PL]} inducing singularity, that \textit{ein} makes null operators visible (8b-c), and more generally that number is due to an interaction between the head noun and Num (Chapter 7). I return to the discussion of (8b-c) in Section 2.2.3 below.
\end{styleStandard}

\begin{styleStandard}
Given these facts, we expect that adjectival inflections and \textit{ein} can be left out without a change in meaning. I showed that this is indeed the case in certain lexical contexts, with specific sets of adjectives in non-elliptical contexts (9a) and with role nouns in predicate nominals (9b):
\end{styleStandard}

\begin{styleStandard}
(9)\ \ a.\ \ \textit{Das ist ein lila(nes) \ \ \ \ \ Buch}.
\end{styleStandard}

\begin{styleStandard}
\ \ \ \ this is \ \ a \ \ \ purple-\textsc{infl} book.\textsc{neut}
\end{styleStandard}

\begin{styleStandard}
\ \ \ \ ‘This is a purple book.’
\end{styleStandard}

\begin{styleStandard}
\ \ b.\ \ \textit{Er ist (ein) Lehrer}.
\end{styleStandard}

\begin{styleStandard}
\ \ \ \ he is \ \ \ a \ \ \ \ teacher.\textsc{masc}
\end{styleStandard}

\begin{styleStandard}
\ \ \ \ ‘He is a teacher.’
\end{styleStandard}

\begin{styleStandard}
Furthermore, the suggestion that both types of elements are semantically vacuous also allows us to avoid the conclusion that certain semantic features are redundantly present when these elements do co-occur. Again, this holds for both adjectival inflections and \textit{ein} (I return to (10b) in Section 2.2.2):
\end{styleStandard}

\begin{styleStandard}
(10)\ \ a.\ \ \textit{gut-e \ \ \ \ \ \ \ \ teur-e \ \ \ \ \ \ \ \ \ \ \ \ \ \ \ Autos} 
\end{styleStandard}

\begin{styleStandard}
\ \ \ \ good-\textsc{infl} expensive-\textsc{infl} cars
\end{styleStandard}

\begin{styleStandard}
\ \ \ \ ‘good expensive cars’
\end{styleStandard}

\begin{styleStandard}
\ \ b. \ \%\ \ \textit{k-eine so’ne Autos}
\end{styleStandard}

\begin{styleStandard}
\ \ \ \ \textsc{neg}{}-a so a \ \ cars
\end{styleStandard}

\begin{styleStandard}
\ \ \ \ ‘no such cars’
\end{styleStandard}

\begin{styleStandard}
If these elements are indeed semantically vacuous, then the actual semantics involved is brought about in a different way. Discussing the various cases, I made remarks to such effect in the previous chapters.
\end{styleStandard}

\begin{styleStandard}
1.1.2. Some Evidence for Hypothesis 1b: Hypothesis 2a vs. 3a
\end{styleStandard}

\begin{styleStandard}
Hypothesis 1b, the indication of abstract structure, consists of two subclaims: Hypothesis 2a and Hypothesis 3a. I proposed that adjectival inflections and \textit{ein} indicate differences in nominal structures with regard to size. For instance, the strong/weak alternation supports the claim that synactic arguments are not only DPs (11a) but also LPPs (11b). Note in this regard that \textit{alle} ‘all’ is followed by an element with a weak inflection in (11a) but by an element with a strong ending in (11b):
\end{styleStandard}

\begin{styleStandard}
(11)\ \ a.\ \ [\textsubscript{DP }\textit{all-e \ nett-en \ \ Studenten }]
\end{styleStandard}

\begin{styleStandard}
\ \ \ \  \ \ \ \ all-\textsc{st} nice-\textsc{wk} students
\end{styleStandard}

\begin{styleFootnote}
\ \ \ \ ‘all nice students’
\end{styleFootnote}

\begin{styleStandard}
\ \ b.\ \ [\textsubscript{LPP }\textit{all-e \ \ }[\textsubscript{DP }\textit{dies-e \ \ \ nett-en \ \ Studenten }]]
\end{styleStandard}

\begin{styleStandard}
\ \ \ \  \ \ \ \ \ all-\textsc{st} \ \ \ \ \ \ these-\textsc{st} nice-\textsc{wk} students
\end{styleStandard}

\begin{styleFootnote}
\ \ \ \ ‘all these nice students’
\end{styleFootnote}

\begin{styleStandard}
I argued in Chapter 2 that Impoverishment only occurs in simple, canonical DPs (11a), the domain in which the determiner undergoes movement. The strong ending on \textit{diese} ‘these’ in (11b) followed from the assumption that this nominal involves a non-canonical structure. Unlike in (11a), \textit{alle} ‘all’ in (11b) is base-generated in LPP. Consequently, Impoverishment does not occur.
\end{styleStandard}

\begin{styleStandard}
Turning to structural differences indicated by \textit{ein}, I sided with de Swart \textit{et al}. (2007) in arguing that predicate nominals without \textit{ein} involve NPs (12a) and that those with \textit{ein }are bigger – under my assumptions (at least) ArtPs (12b):
\end{styleStandard}

\begin{styleStandard}
(12)\ \ a.\ \ \textit{Er ist }[\textsubscript{NP} \textit{Lehrer }].
\end{styleStandard}

\begin{styleStandard}
\ \ \ \ he is \ \ \ \ \ \ \ teacher.\textsc{masc}
\end{styleStandard}

\begin{styleStandard}
\ \ \ \ ‘He is a teacher.’
\end{styleStandard}

\begin{styleStandard}
\ \ b.\ \ \textit{Er ist }[\textsubscript{ArtP} \textit{ein Mann }].
\end{styleStandard}

\begin{styleStandard}
\ \ \ \ he is \ \textsubscript{\ \ \ \ \ \ \ \ \ \ \ \ }a \ \ \ man.\textsc{masc}
\end{styleStandard}

\begin{styleStandard}
\ \ \ \ ‘He is a man.’
\end{styleStandard}

\begin{styleStandard}
I followed the above-mentioned authors in that (12a) involves a role noun with CAP in NP but that (12b) contains a kind noun with REL in NumP, the latter being flagged by \textit{ein }in ArtP. More generally, observe again that adjectival inflections indicate differences in structure in the higher layers of the noun phrase but that \textit{ein} shows distinctions in the lower projections. This is consistent with another fact.
\end{styleStandard}

\begin{styleStandard}
\ \ As just seen, role nouns can appear in predicative contexts without an article. Importantly, role nouns require a determiner when they appear in argument position: 
\end{styleStandard}

\begin{styleFootnote}
(13)\ \ \textit{*(Ein) Lehrer \ \ \ \ \ \ \ \ \ braucht Unterstützung}.
\end{styleFootnote}

\begin{styleFootnote}
\ \  \ \ a \ \ \ \ \ teacher.\textsc{masc} needs \ \ \ support
\end{styleFootnote}

\begin{styleFootnote}
\ \ ‘A teacher needs support.’
\end{styleFootnote}

\begin{styleFootnote}
Considering (13), it could be claimed that \textit{ein} brings about the semantic change from a predicate nominal (12a) to an argumental expression (13) and that \textit{ein} would then be tied to the semantics after all. However, there is an independent explanation for the occurrence of \textit{ein} in (13). 
\end{styleFootnote}

\begin{styleFootnote}
Longobardi (1994) proposed that argumental nominals project DPs (also Borer 2005: 65-66, Stowell 1989). For role nouns to occur in argumental expressions, more structure beyond NP must be projected. As discussed in Chapter 6, part of this structure is NumP which involves REL. The latter is flagged by \textit{ein}. We can point out that it is not \textit{ein} itself that turns a predicate nominal into an argumental expression – it is more structure. Having said that, \textit{ein} does not indicate the presence of the DP-level either: many speakers allow \textit{ein} to appear with role nouns in predicative contexts, and kind nouns appear with \textit{ein} in such contexts more generally. As discussed above, both of the latter nominals only involve ArtP (not DP) and yet, they contain \textit{ein }(due to REL). This means that \textit{ein} does not necessarily indicate a DP. The reason why argumental noun phrases require a DP-level is that this phrasal layer involves a person feature (Longobardi 1994). In other words, it is the structure involving the person feature (and not \textit{ein}) that brings about an argumental noun phrase.
\end{styleFootnote}

\begin{styleStandard}
\ \ Finally, I restate the observation that adjectival inflections and \textit{ein} interact morpho-syntactically calling into question certain structural claims of other proposals. In Chapter 4, Section 2, I pointed out that \textit{ein} is possible in certain \textit{wh}{}-exclamatives (14a) and in constructions involving \textit{so} ‘such’ (14b). Now, while not all speakers of German allow \textit{ein} to occur in these contexts, those who do have a weak ending on the adjective as shown in the following attested examples:
\end{styleStandard}

\begin{styleFootnote}
(14)\ \ a. \ \%\ \ \textit{NFU und NFSU2 (was \ für n-e \ \ dumm-en \ Abkürzungen)}~
\end{styleFootnote}

\begin{styleStandard}
NFU and NFSU2 \ what for a-\textsc{pl }stupid-\textsc{wk} abbreviations
\end{styleStandard}

\begin{styleStandard}
‘NFU and NFSU2 (what stupid abbreviations!)’
\end{styleStandard}

\begin{styleFootnote}
\ \ b. \ \%\ \ \emph{So n-e \ süß-en}\textit{~ \ \ \ klein-en \ \ Pfoten}
\end{styleFootnote}

\begin{styleFootnote}
so a-\textsc{pl} cute-\textsc{wk} little-\textsc{wk} paws
\end{styleFootnote}

\begin{styleFootnote}
‘Such cute little paws!’
\end{styleFootnote}

\begin{styleStandard}
I demonstrated in the preceding chapters that a simple, surface-oriented account of the strong/weak alternation cannot explain all the cases. Rather, determiners including \textit{ein} bring about a weak ending on the adjective in a specific structural constellation – simple DPs. In other words, \textit{ein} interacts with the inflection on the adjective morpho-syntactically. 
\end{styleStandard}

\begin{styleStandard}
If this is so, then the current account also raises some questions for other proposals. Specifically, the weak inflections on the adjectives above raise issues for a Predicate Inversion analysis of (14a) as discussed in Bennis \textit{et al}. (1998), and they argue against the assumption of a null noun in (14b), either after \textit{ein} or after the adjectives (cf. van Riemsdijk 2005). Furthermore, the strong inflections on adjectives in the split-off of discontinuous noun phrases are only consistent with split topicalization being analyzed as involving two separately base-generated nominals (e.g., Fanselow 1988). In the latter context, I also argued that \textit{ein} (i.e., its feature bundles) is unlikely to be inserted late in split topicalizations involving indefinite pronouns.
\end{styleStandard}

\begin{styleStandard}
Returning briefly to (14), it is interesting to observe that \textit{ein} does not only determine the inflection on the adjectives but that \textit{ein} also seems to have some semantic effect there. For instance, van Riemsdijk (2005) points out for similar cases in Dutch that these types of constructions can express some relatively excessive property of the entities under discussion. These cases were not addressed in detail in Chapter 4. In Section 2.2.3 below, I suggest again that \textit{ein} itself does not have an influence on the interpretation in these cases. Rather, an operator is held responsible for this effect, and this operator is flagged by \textit{ein}. 
\end{styleStandard}

\begin{styleStandard}
More generally, given these shared properties and the interwoven argumentation, I reiterate my proposal that adjectival inflections and \textit{ein} deserve to be discussed in tandem. Importantly, these elements also differ in certain ways. 
\end{styleStandard}

\begin{styleStandard}
\textit{1.2. \ \ Differences between Adjectival Inflections and }ein
\end{styleStandard}

\begin{styleStandard}
In the previous section, I reviewed the claim that adjectival inflections and \textit{ein} indicate abstract structure. I left open there that the former also provide clues about the different types of embeddings, namely clues that indicate simple vs. complex DPs. To illustrate this again, consider (15). The example in (15a) involves a simple, canonical DP where the adjective is in Spec,AgrP. In contrast, the example in (15b) is a case of morphological dis-agreement. I argued in Chapter 2, Section 3.5 that it involves a different structure where both the adjective and the noun are in Spec,DisP, and the head noun of the larger structure is a null element. This yields a non-canonical structure (see also the discussion of pronominal DPs involving \textit{Sie} ‘you’ and non-plural nouns like \textit{Schwein} ‘(pig =) idiot’ in Chapter 7, Section 3.3.2):
\end{styleStandard}

\begin{styleStandard}
(15)\ \ a.\ \ \textit{ihr }[\textsubscript{AgrP}\textit{ blöd-en }[\textsubscript{NP}\textit{ Schweine }]]
\end{styleStandard}

\begin{styleStandard}
\ \ \ \ you \ \ \ \ \ \ \ stupid-\textsc{wk} \ pigs
\end{styleStandard}

\begin{styleStandard}
\ \ \ \ ‘you stupid idiots’
\end{styleStandard}

\begin{styleStandard}
\ \ b.\ \ \textit{ihr }[\textsubscript{DisP} [\textsubscript{AgrP}\textit{ blöd-e}\textit{\textsubscript{ }}[\textsubscript{NP}\textit{ Bande }]] [\textsubscript{NP}\textit{ e}\textit{\textsubscript{N}}\textit{ }]]
\end{styleStandard}

\begin{styleStandard}
\ \ \ \ you \ \ \ \ \ \ \ \ \ \ \ \ \ \ stupid-\textsc{st} \ gang.\textsc{fem}
\end{styleStandard}

\begin{styleStandard}
\ \ \ \ ‘you stupid gang’
\end{styleStandard}

\begin{styleStandard}
Importantly, although the adjectives in both (15a) and (15b) are in Spec,AgrP, the one in (15b) is more deeply embedded – this AgrP is in Spec,DisP. Consequently, Impoverishment cannot occur in the latter case resulting in a strong ending on the adjective. In addition to complex specifiers, it was also shown in Chapter 2 that strong adjectival endings surface in several other non-canonical structures, for instance, in different kinds of adjunction. In each case, Impoverishment does not occur, and I concluded that weak inflections only surface in simple, regular DPs. More generally, the strong/weak alternation on the adjective was employed throughout the book as a means to find plausible structures for related constructions (many containing \textit{ein}) and as briefly mentioned just above, to evaluate other proposals as regards their structural claims.
\end{styleStandard}

\begin{styleStandard}
\ \ Both adjectival inflections and \textit{ein} also make contributions in their own right. While both types of elements indicate the presence of certain elements, they differ in that adjectival endings make morphological features visible but that \textit{ein} supports or flags semantic operators. Consider the remaining claims of Hypotheses 2 and 3.
\end{styleStandard}

\begin{styleStandard}
1.2.1. Some Evidence for Hypothesis 2b
\end{styleStandard}

\begin{styleStandard}
I put forth the following hypothesis for adjectival inflections:
\end{styleStandard}

\begin{styleStandard}
(16)\ \ \textit{Hypothesis 2}
\end{styleStandard}

\begin{styleStandard}
\ \ Adjectival inflections:
\end{styleStandard}

\begin{styleStandard}
\ \ \ b.\ \ […] make nominal features like case, number, and gender visible.
\end{styleStandard}

\begin{styleStandard}
To briefly illustrate, this claim receives credence from the comparison of a regular noun phrase as in (17a) and its discontinuous counterpart as in (17b):
\end{styleStandard}

\begin{styleFootnote}
(17)\ \ a.\ \ \textit{ein lila-(n-es) \ \ Kleid}
\end{styleFootnote}

\begin{styleFootnote}
\ \ \ \ a \ \ \ purple-n-\textsc{st} dress.\textsc{neut}
\end{styleFootnote}

\begin{styleFootnote}
\ \ \ \ ‘a purple dress’
\end{styleFootnote}

\begin{styleFootnote}
\ \ b.\ \ \textit{Kleid \ \ \ \ \ \ \ \ \ habe ich ein lila-*(n-es)}.
\end{styleFootnote}

\begin{styleFootnote}
\ \ \ \ dress.\textsc{neut} have \textsubscript{\ }I \ \ \ \textsubscript{\ }a \ \ \ purple-n-\textsc{st}
\end{styleFootnote}

\begin{styleFootnote}
\ \ \ \ ‘As for dresses, I have a purple one’
\end{styleFootnote}

\begin{styleStandard}
I proposed that the adjectival ending in (17b) makes the features of the displaced noun visible; that is, the adjectival inflection is an exponent of case, number, and gender (specifically for (17b): nominative, singular, neuter). In fact, due to concord inside the noun phrase, adjectival inflections make features of the entire DP visible. The same holds for the inflection on \textit{ein}:
\end{styleStandard}

\begin{styleStandard}
(18)\ \ a.\ \ \textit{ein Kleid}
\end{styleStandard}

\begin{styleStandard}
\ \ \ \ a \ \ \ dress.\textsc{neut}
\end{styleStandard}

\begin{styleFootnote}
\ \ \ \ ‘a dress’
\end{styleFootnote}

\begin{styleFootnote}
\ \ b.\ \ \textit{Kleid \ \ \ \ \ \ \ \ \ habe ich nur \ }\textit{\textsubscript{\ }}\textit{ein-*(es)}.
\end{styleFootnote}

\begin{styleFootnote}
\ \ \ \ dress.\textsc{neut} have \textsubscript{\ }I \ \ \ \textsubscript{\ }only one-\textsc{st}
\end{styleFootnote}

\begin{styleFootnote}
\ \ \ \ ‘As for dresses, I have only one.’
\end{styleFootnote}

\begin{styleStandard}
Comparing (17b) and (18b), note again that adjectival inflections and \textit{ein} interact in that \textit{ein} determines the ending on the following adjective (17b) but that this very ending may also appear on \textit{ein} itself if the adjective (and noun) is absent (18b). Recall from Chapter 1, Section 3.2.2 that the additional adjectival ending in (18b) is only possible on \textit{ein} under very specific conditions (which makes \textit{ein} different from other determiners). This was made formal in Chapter 2, Section 2.2 postulating certain vocabulary insertion rules for \textit{ein}.
\end{styleStandard}

\begin{styleStandard}
1.2.2. Some Evidence for Hypothesis 3b
\end{styleStandard}

\begin{styleStandard}
The element \textit{ein} also makes certain elements visible. Unlike adjectival inflections, \textit{ein} indicates the presence of semantic operators.\footnote{\ It may turn out that adjectival inflections are similar to \textit{ein} here as well. Specifically, adjectival inflections might also be able to flag the presence of an operator in certain contexts. In this regard, Merchant (1996: 183) points out that uninflected \textit{all} ‘all’ tends to induce a collective reading (ia) but that its inflected counterpart seems to yield a distributive reading (ib):\par (i)\ \ a.\ \ \textit{all diese guten Freunde}\ \ \ \ \ \ \ \ \par \ \ \ \ \ \ all these good \ friends\par \ \ \ \ \ \ ‘all these good friends’\par b.\ \ \textit{all-e \ \ \ \ \ }\textit{\textsubscript{\ }}\textit{diese guten Freunde}\ \ \ \ \ \ \ \ \par \ \ \ \ \ \ all-\textsc{infl} these \ good \ friends\par \ \ \ \ \ \ ‘all these good friends’\par Taking this at face value, the presence or absence of the inflection seems to correlate with different readings. Rather than suggesting that adjectival inflections are associated with the semantics after all (e.g., Roehrs 2015: 264), we might tentatively suggest that they make a distributivity operator in (ib) visible. } I formulated the following general claim:
\end{styleStandard}

\begin{styleStandard}
(19)\ \ \textit{Hypothesis 3}
\end{styleStandard}

\begin{styleStandard}
\ \ \textit{Ein}:
\end{styleStandard}

\begin{styleStandard}
\ \ \ b.\ \ […] supports overt semantic operators (e.g., NEG \textit{k}{}-) and flags the presence of \ \ \ \ \ \ covert semantic operators (e.g., REL).
\end{styleStandard}

\begin{styleStandard}
I argued above that the possessive articles, the negative article, and the singularity numeral are composite forms. These complex elements involve possessor, negator, or singularity components that, if supported by \textit{ein}, receive the following spell-out forms in a noun phrase: 
\end{styleStandard}

\begin{styleStandard}
(20)\ \ a.\ \ \textit{m-eine Frau}
\end{styleStandard}

\begin{styleStandard}
\ \ \ \ \textsc{poss}{}-a woman.\textsc{fem}
\end{styleStandard}

\begin{styleStandard}
\ \ \ \ ‘my woman’
\end{styleStandard}

\begin{styleStandard}
\ \ b.\ \ \textit{k-eine Frau}
\end{styleStandard}

\begin{styleStandard}
\ \ \ \ \textsc{neg}{}-a woman.\textsc{fem}
\end{styleStandard}

\begin{styleStandard}
\ \ \ \ ‘no woman’
\end{styleStandard}

\begin{styleStandard}
\ \ c.\ \ \textit{EINE \ \ \ Frau}
\end{styleStandard}

\begin{styleStandard}
\ \ \ \ \textit{Ø}\textsubscript{[-}\textsc{\textsubscript{pl}}\textsubscript{]}+a woman.\textsc{fem}
\end{styleStandard}

\begin{styleStandard}
\ \ \ \ ‘one woman’
\end{styleStandard}

\begin{styleStandard}
Note again that supporting involves overt operators. Overtness of the operator is a convenient shorthand for stating that the relevant semantic components have a detectible manifestation: possessors and negation have an overt segmentable element (e.g., \textit{m}{}-, \textit{k}{}-), but the singularity numeral involves a non-segmentable element, namely stress. I suggested above that these three components are operators. This not only allows us to relate the three instances to each other but also to other cases.
\end{styleStandard}

\begin{styleStandard}
In particular, basically following de Swart \textit{et al}. (2007), I proposed that \textit{ein} indicates the presence of the realization operator REL, an element that does not have an independent overt manifestation (21). With \textit{ein} appearing in this context, this is what I referred to as flagging. Notice also that the flagging of an operator and the indication of a certain amount of structure are related here. Assuming with de Swart \textit{et al}. (2007) that REL is in NumP, \textit{ein} indicates the presence of REL and a certain abstract structure in singular contexts (i.e., ArtP):
\end{styleStandard}

\begin{styleStandard}
(21)\ \ [\textsubscript{ArtP} \textit{eine }[\textsubscript{NumP} REL\textsubscript{{\textless}e{\textless}e,t{\textgreater}{\textgreater}}\textit{ }[\textsubscript{NP} \textit{Frau}\textsubscript{{\textless}e{\textgreater}}\textit{ }]]]\textsubscript{{\textless}e,t{\textgreater}}
\end{styleStandard}

\begin{styleStandard}
\ \  \ \ \ \ \ \ \textsubscript{\ }a \ \ \ \ \ \ \ \ \ \ \ \ \ \ \ \ \ \ \ \ \ \ \ \ \ \ \ \ \ \ \ \ \ \ \ \ \ \textsubscript{\ }woman
\end{styleStandard}

\begin{styleStandard}
\ \ ‘a woman’
\end{styleStandard}

\begin{styleStandard}
Finally, it is worth pointing out that the cases in (20) above also involve REL – the head nouns (type {\textless}e{\textgreater}) must be mapped to sets of individuals (type {\textless}e,t{\textgreater}). In other words, \textit{ein} does not only support an overt operator there, but it also flags a covert one at the same time. In Sections 2.2.2 and 2.2.3 below, I list more cases involving supporting and flagging and discuss some of the related issues in more detail. 
\end{styleStandard}

\begin{styleStandard}\itshape
1.3.\ \ Summary of the Main Claims
\end{styleStandard}

\begin{styleStandard}
In this last subsection, I provide a convenient summary of the main insights in bullet-point format:
\end{styleStandard}

\begin{styleStandard}
Adjectival inflections are not a reflex of:
\end{styleStandard}

\begin{listWWviiiNumxileveli}
\item 
\begin{styleStandard}
(in-)definiteness 
\end{styleStandard}
\item 
\begin{styleStandard}
(non-)restrictiveness of interpretation of modifiers 
\end{styleStandard}
\item 
\begin{styleStandard}
referentiality
\end{styleStandard}
\end{listWWviiiNumxileveli}
\begin{styleStandard}
The article \textit{ein} is not a reflex of:
\end{styleStandard}

\begin{listWWviiiNumxileveli}
\item 
\begin{styleStandard}
indefiniteness 
\end{styleStandard}
\item 
\begin{styleStandard}
emotiveness
\end{styleStandard}
\item 
\begin{styleStandard}
singular number/countability 
\end{styleStandard}
\end{listWWviiiNumxileveli}
\begin{styleStandard}
Both adjectival inflections and \textit{ein} are semantically vacuous and can:
\end{styleStandard}

\begin{listWWviiiNumxiileveli}
\item 
\begin{styleStandard}
appear in contradictory semantic contexts
\end{styleStandard}
\item 
\begin{styleStandard}
be left out in certain, well-defined contexts without loss of meaning
\end{styleStandard}
\item 
\begin{styleStandard}
have multiple occurrences without leading to semantic redundancy
\end{styleStandard}
\end{listWWviiiNumxiileveli}
\begin{styleStandard}
Furthermore, both adjectival inflections and \textit{ein} can each be in different positions:
\end{styleStandard}

\begin{listWWviiiNumixleveli}
\item 
\begin{styleStandard}
strong endings are in AgrP, CardP, DP, and LPP; weak endings are in AgrP, CardP, and DP
\end{styleStandard}
\item 
\begin{styleStandard}
\textit{ein} is in Art, Card, and D
\end{styleStandard}
\end{listWWviiiNumixleveli}
\begin{styleStandard}
Abstracting away from adjectival inflections indicating different degrees of embedding, these two types of elements indicate abstract structure:
\end{styleStandard}

\begin{listWWviiiNumviileveli}
\item 
\begin{styleStandard}
adjectival inflections:\ \ DP vs. LPP 
\end{styleStandard}
\item 
\begin{styleStandard}
\textit{ein}: \ \ \ \ \ \ NP vs. ArtP 
\end{styleStandard}
\end{listWWviiiNumviileveli}
\begin{styleStandard}
These two types of items make other elements visible:
\end{styleStandard}

\begin{listWWviiiNumviileveli}
\item 
\begin{styleStandard}
adjectival inflections:\ \ morphological features
\end{styleStandard}
\item 
\begin{styleStandard}
\textit{ein}: \ \ \ \ \ \ semantic operators
\end{styleStandard}
\end{listWWviiiNumviileveli}
\begin{styleStandard}
Miscellaneous for adjectival inflections:
\end{styleStandard}

\begin{listWWviiiNumviileveli}
\item 
\begin{styleStandard}
concord is a necessary (but not sufficient) condition for weak endings\textit{ }
\end{styleStandard}
\item 
\begin{styleStandard}
Impoverishment occurs locally and in a bottom-up fashion
\end{styleStandard}
\end{listWWviiiNumviileveli}
\begin{styleStandard}
Miscellaneous for \textit{ein}:
\end{styleStandard}

\begin{listWWviiiNumviileveli}
\item 
\begin{styleStandard}
it is the least specified determiner\footnote{\ Besides having no semantics, there are some other advantages of assuming that \textit{ein} has no feature for definiteness. On the one hand, \textit{ein} does not have to move to the DP-level allowing us to claim that predicates can be of a smaller structural size; on the other hand, if \textit{ein} can stay low in the structure, we can propose that \textit{ein} always surfaces on the right side of the operator it supports (e.g., with \textit{EIN} ‘one’, see also Section 2.2.5). This, in turn, allows us to use the same mechanism (Local Dislocation) to instantiate this support.}
\end{styleStandard}
\item 
\begin{styleStandard}
it is not inserted late\footnote{\ Recall that with Vocabulary Items inserted late in DM, this statement actually has to do with the feature bundles that make up \textit{ein(-er) }where the stem involves [+D] and its adjectival inflection contains features for case, number, and gender. In other words, both the stem and the inflection are treated in the same way – no features are inserted late, only the related Vocabulary Items.}
\end{styleStandard}
\item 
\begin{styleStandard}
it is triggered by REL (but not by CAP or \textit{als }‘as’)
\end{styleStandard}
\end{listWWviiiNumviileveli}
\begin{styleStandard}
Overall, I conclude that although semantically vacuous, these two types of elements provide clues about abstract structure and the presence of covert elements (features and operators) in the German noun phrase. 
\end{styleStandard}

\begin{styleStandard}\bfseries
2. \ \ Some Extensions and Further Consequences 
\end{styleStandard}

\begin{styleStandard}
In this section, I consider some possible extensions and more consequences. Given the current state of the investigation, I only briefly discuss the relevant issues and tentatively hint at some solutions. A full-blown account of all the relevant aspects must await another occasion. I start with adjectival inflections and then move on to \textit{ein}.
\end{styleStandard}

\begin{styleFootnote}\itshape
2.1.\ \ Extensions and Consequences of Adjectival Inflections: German Dialects
\end{styleFootnote}

\begin{styleFootnote}
In Chapter 3, Section 7, I provided my account of adjectival inflections in one dialectal variety – Mannheim German. I have not investigated and analyzed other regional forms of German in this regard. In what follows, I make some brief remarks about adjectival inflections in one well-studied dialect, Alemannic German, showing how this type of dialect could be accommodated in the current discussion.
\end{styleFootnote}

\begin{styleFootnote}
Rehn (2019) provides an interesting overview and in-depth analysis of prenominal adjectives in Alemannic (for different subdialects of Alemannic, see Baechler 2017).\footnote{\ Baechler (2017) provides convenient paradigms of adjectival inflections (her pages 305-13) and definite articles and demonstratives (her pages 339-44) in a number of different Alemannic subdialects (note though that unlike the current analysis, she does not consider definite articles and demonstratives as bipartite forms).} She observes that they can appear without an inflection in both indefinite and definite contexts. Rehn (2019: 114) points out that this cannot be a continuation of ENHG as uninflected adjectives in Alemannic occur in more contexts than in ENHG. Rather than involving a null ending, she argues that these adjectives are truly uninflected. Rehn (2019: 103-04) argues that uninflected adjectives and nouns could be analyzed as compound-like elements in some instances but crucially not in all cases and dialects. As such, the distribution of the inflections on adjectives (and other elements in the noun phrase) is proposed to follow from certain features that Alemannic (and German more generally) must mark overtly. In what follows, I briefly summarize some of the main aspects of Rehn’s work (but I am not able here to do full justice to all the intricacies of her multi-faceted proposal).
\end{styleFootnote}

\begin{styleFootnote}
Rehn (2019) observes that inflections on adjectives are basically optional. This is shown with examples in the nominative/accusative in (22a-b). There is no difference in the interpretation that corelates with the presence or absence of the inflections, and there are no morpho-syntactic restrictions. There is one exception. In the absence of a determiner, the inflection on the adjective is obligatory illustrating this with an example in the nominative in (22c) (examples are from Rehn 2019: 133, 121):
\end{styleFootnote}

\begin{styleFootnote}
(22)\ \ a.\ \ \textit{des lang(-e) \ Soil}\ \ \ \ (Alemannic German)
\end{styleFootnote}

\begin{styleFootnote}
\ \ \ \ the long-\textsc{wk} rope.\textsc{neut}
\end{styleFootnote}

\begin{styleFootnote}
\ \ \ \ ‘the long rope’
\end{styleFootnote}

\begin{styleFootnote}
b.\ \ \textit{a guat(-s) Buach}\ \ \ \ 
\end{styleFootnote}

\begin{styleFootnote}
\ \ \ \ a good-\textsc{st} book.\textsc{neut}
\end{styleFootnote}

\begin{styleFootnote}
\ \ \ \ ‘a good book’
\end{styleFootnote}

\begin{styleFootnote}
\ \ c.\ \ \textit{guad*(-er) Wei}
\end{styleFootnote}

\begin{styleFootnote}
\ \ \ \ good-\textsc{st} \ \ \ \ wine.\textsc{masc}
\end{styleFootnote}

\begin{styleFootnote}
\ \ \ \ ‘good wine’
\end{styleFootnote}

\begin{styleFootnote}
In order to explain the distribution of the adjectival inflections in (22), Rehn (2019: 122-26) proposes that German DPs must mark features for number and oblique case overtly. 
\end{styleFootnote}

\begin{styleFootnote}
\ \ Considering the distribution of the indefinite and definite articles, Rehn observes that certain features are consistently marked overtly (as different). To exemplify with Standard German, number is marked differently with the indefinite article in the singular (\textit{ein}) and plural (\textit{Ø}\textit{\textsubscript{D}}). Similarly, the definite article involves \textit{der}, \textit{das}, \textit{die }in the singular and \textit{die} in the plural (NB: feminine singular and plural forms are not entirely straightforward as the nominative/accusative forms are all \textit{die}, and the dative forms are both \textit{der}). Turning to case, Rehn points out that there is syncretism in the nominative and accusative in both the singular and plural. In fact, only the oblique cases are consistently different, for instance, nominative/accusative neuter \textit{das} vs. dative neuter \textit{dem} and genitive neuter \textit{des} (NB: feminine singular has the form \textit{der} in both the dative and genitive). As for gender, this category is not consistently different. Masculine and neuter are the same with \textit{ein} in the nominative, and masculine and neuter are the same more generally with \textit{einem/dem} in the dative and\textit{ eines/des} in the genitive. Rehn draws the conclusion that only number and oblique case must be marked overtly in German DPs.\footnote{\ Given the \textit{nota bene} remarks in the main text, the difference between number and oblique case vs. non-oblique case and gender is not entirely clearcut. I put this potential issue aside here.}
\end{styleFootnote}

\begin{styleFootnote}
\ \ Rehn (2019: 125) formulates the following principle:
\end{styleFootnote}

\begin{styleFootnote}
(23)\ \ \textit{Feature Specification in German DPs:}
\end{styleFootnote}

\begin{styleFootnote}
\ \ Within a DP in German, number and oblique case must be marked overtly either
\end{styleFootnote}


\setcounter{listWWviiiNumxiiileveli}{0}
\begin{listWWviiiNumxiiileveli}
\item 
\begin{styleFootnote}
on an article,
\end{styleFootnote}
\item 
\begin{styleFootnote}
through inflection realized in Mod\textsuperscript{0} [here Agr, DR] when no article is realized,
\end{styleFootnote}
\item 
\begin{styleFootnote}
or on the noun itself when neither of the above is realized.
\end{styleFootnote}
\end{listWWviiiNumxiiileveli}
\begin{styleFootnote}
This principle is meant to hold for both Standard German and the regional dialects alike. Note that the obligatory inflections in Standard German are, in Rehn’s (2019: 135-39) view, due to a standardization process in the 18\textsuperscript{th} and early 19\textsuperscript{th} centuries. For Rehn, inflections on adjectives are generally optional in German, and the obligatory inflections in Standard German are “a mere PF-phenomenon”.\footnote{\ Rehn (2019: 133) observes that inflections on adjectives are also optional in other dialectal varieties of German. Note though that this does not hold for all dialects: Berlin German has obligatory inflections.}
\end{styleFootnote}

\begin{styleFootnote}
\ \ Returning to Alemannic, the principle in (23) accounts for the data in (22). In the non-oblique cases provided in (22), only number must be marked overtly. This marking is supplied by the articles in (22a-b), and as a consequence, the inflections on the adjectives are optional. As for the obligatory inflection on the adjective in (22c), there is no article, and the inflection on the adjective itself is the only way to mark number (Rehn 2019: 124). 
\end{styleFootnote}

\begin{styleFootnote}
While the principle in (23) may account for the data in (22), it raises questions about DPs involving bare mass nouns. In the non-oblique cases as in (24a), number must be overtly marked; in the oblique cases as in (24b), both number and the oblique case must be overtly marked. As far as I am aware, there is no overt marking in (24a-b), and yet, the DPs are grammatical: 
\end{styleFootnote}

\begin{styleFootnote}
(24)\ \ a.\ \ \textit{Wei}\ \ \ \ \ \ \ \ (Alemannic German)
\end{styleFootnote}

\begin{styleFootnote}
\ \ \ \ wine.\textsc{masc}
\end{styleFootnote}

\begin{styleFootnote}
\ \ \ \ ‘wine’
\end{styleFootnote}

\begin{styleFootnote}
b.\ \ \textit{mit} \ \ \textit{Wei}\ \ \ \ \ \ 
\end{styleFootnote}

\begin{styleFootnote}
\ \ \ \ with wine.\textsc{masc}
\end{styleFootnote}

\begin{styleFootnote}
\ \ \ \ ‘with wine’
\end{styleFootnote}

\begin{styleFootnote}
In my view, the grammaticality of these cases is telling. To explain their status, we have to assume that features for case, number, and gender are only spelled out by adjectival inflections (but not endings on nouns). An alternative to Rehn’s proposal could be to extend the current analysis to Alemannic German.
\end{styleFootnote}

\begin{styleFootnote}
In the current system, null articles and nouns have no feature bundles for case, number, and gender – only overt determiners and adjectives do (I follow Rehn 2019: 155 in that case inflections on nouns are under Cl, in the current system Num, signaling case and number at the same time). I assume the following feature realization rule for German DPs, for both the standard and regional varieties:
\end{styleFootnote}

\begin{styleFootnote}
(25)\ \ \textit{Feature Specification of German DPs:}
\end{styleFootnote}

\begin{styleFootnote}
If CNG bundles are present, at least one of them must be made overt.
\end{styleFootnote}

\begin{styleFootnote}
Note that this is a stronger claim than Hypothesis 2b (i.e., adjectival inflections make nominal features like case, number, and gender visible) in that this conditional is meant to help explain the distribution of adjectival inflections. Consider how Standard and Alemannic German can be accounted for by (25).
\end{styleFootnote}

\begin{styleFootnote}
\ \ As discussed above, overt determiners have CNG features as part of their structures. Furthermore, (regular) adjectives project InflP as part of their extended projection. As such, the rule in (25) is adhered to in all cases involving overt determiners and/or adjectives in the counterparts of (22) in Standard German – CNG features are present and made overt (I comment on the obligatory presence of the inflection below). Bare mass nominals as in (24) are predicted to be fine too as they do not involve CNG bundles in the first place. As for Alemannic, recall that there is no difference in the interpretation that corelates with the presence or absence of the inflections on adjectives (for such a case in Dutch, see Evans 2021). As such, I agree with Rehn (2019: 194) that there is no difference in the syntactic position of inflected vs. uninflected adjectives. One way to account for the obligatory presence of inflections on adjectives in Standard German and the optional absence of inflections on adjectives in Alemannic German is to assume that like in Standard German, InflP is projected with (regular) adjectives in Alemannic German. Unlike in Standard German, the inflections on the adjectives are optionally deleted in Alemannic German unless the deletion violates the rule in (25) – the latter scenario involves the presence of only one CNG feature bundle as in (22c).
\end{styleFootnote}

\begin{styleFootnote}
\ \ Finally, I make some remarks on the indefinite article. Note that DPs like \textit{ein Buch} or \textit{a Buach} ‘a book’ are grammatical in both Standard and Alemannic German. Recall that I claimed in Chapter 2, Section 2.2.2 that the indefinite article spells out CNG features. This includes the non-oblique instances where this element has no overt inflections in three feature combinations. Given the vocabulary insertion rules provided in that section, the uninflected indefinite article \textit{ein} spells out both [+D] and [CNG] at the same time. If this is so, then this means that the non-oblique cases are not special. Rather, we can maintain that all features of the CNG bundles must be spelled out. In other words, non-oblique case features are on a par with oblique case features making the account of the inflectional distribution more general. If the sketch above turns out to be a viable alternative account of the Alemannic data, it supports the current analysis of the indefinite article. 
\end{styleFootnote}

\begin{styleFootnote}
\ \ To sum up, while I have not examined adjectival inflections in dialectal varieties of German in much detail here, I believe the current system can be extended to those cases. Above, this was exemplified by a brief discussion of Alemannic German. Admittedly, the details of this extension still need to be worked out more carefully. This goes beyond the scope of this book. Besides certain theoretical questions that arise, there are also empirical points that need to be clarified. For instance, it is important to determine the distribution of inflections on multiple adjectives in cases where determiners are absent. Such data will help narrow down the analytical options. Next, I consider \textit{ein} in more detail. 
\end{styleFootnote}

\begin{styleFootnote}
\textit{2.2.\ \ Extensions and Consequences Involving} ein
\end{styleFootnote}

\begin{styleFootnote}
In this section, I turn to some extensions and consequences of the discussion of \textit{ein}. At the end, I provide the updated vocabulary insertion rules of the main types of articles. I begin by reviewing certain parts of the previous discussion.
\end{styleFootnote}

\begin{styleStandard}
2.2.1. Reviewing Cases of Supporting and Flagging
\end{styleStandard}

\begin{styleStandard}
As argued in Chapter 5, \textit{ein} can support POSS of the possessive articles, NEG of the negative article, and \textit{Ø}\textsubscript{[-PL]}\textit{ }of the singularity numeral: 
\end{styleStandard}

\begin{styleFootnote}
(26)\ \ a.\ \ \textit{m-eine Frau}
\end{styleFootnote}

\begin{styleFootnote}
\ \ \ \ \textsc{poss}{}-a woman.\textsc{fem}
\end{styleFootnote}

\begin{styleFootnote}
\ \ \ \ ‘my woman’
\end{styleFootnote}

\begin{styleFootnote}
\ \ b.\ \ \textit{k-eine Frau}
\end{styleFootnote}

\begin{styleFootnote}
\ \ \ \ \textsc{neg}{}-a woman.\textsc{fem}
\end{styleFootnote}

\begin{styleFootnote}
\ \ \ \ ‘no woman’
\end{styleFootnote}

\begin{styleFootnote}
\ \ c.\ \ \textit{EINE \ \ \ Frau}
\end{styleFootnote}

\begin{styleFootnote}
\ \ \ \ \textit{Ø}\textsubscript{[-}\textsc{\textsubscript{pl}}\textsubscript{]}+a woman.\textsc{fem}
\end{styleFootnote}

\begin{styleFootnote}
\ \ \ \ ‘one woman’
\end{styleFootnote}

\begin{styleFootnote}
In Chapter 6, I followed work by de Swart \textit{et al}. (2007) such that kind nouns like \textit{Frau} ‘woman’ combine with the realization operator REL returning predicate nominals. Note though that combinations of REL(\textit{Frau}) can occur not only as part of predicate nominals (27a) but also as part of argumental expressions (27b). In either case, REL is flagged by \textit{ein}:
\end{styleFootnote}

\begin{styleStandard}
(27)\ \ a.\ \ \textit{Die Person da \ \ \ \ hinten ist eine Frau}.
\end{styleStandard}

\begin{styleStandard}
\ \ \ \ the \ person there back \ \ \ is \ a \ \ \ \ \ woman.\textsc{fem}
\end{styleStandard}

\begin{styleStandard}
\ \ \ \ ‘The person back there is a woman.’
\end{styleStandard}

\begin{styleStandard}
\ \ b.\ \ \textit{Eine Frau \ \ \ \ \ \ \ \ \ \ \ kam \ \ durch \ \ \ die Tür}.
\end{styleStandard}

\begin{styleStandard}
\ \ \ \ a \ \ \ \ \ woman.\textsc{fem} came through the door
\end{styleStandard}

\begin{styleStandard}
\ \ \ \ ‘A woman came through the door.’
\end{styleStandard}

\begin{styleStandard}
Indeed, as briefly pointed out in Section 1.2, one instance of \textit{ein} can make two operators visible at the same time. To see this again, consider the example in (28a). Having added an adjective, it is clear that the kind noun \textit{Frau} ‘woman’ must combine with REL in NumP to yield a predicate. With both elements of type {\textless}e,t{\textgreater}, Predicate Modification can combine the predicate nominal and its adjectival modifier. REL triggers the presence of \textit{ein}. Furthermore, the singularity operator must be supported by \textit{ein}. Thus, (28a) has the more abstract representation in (28b):
\end{styleStandard}

\begin{styleStandard}
(28)\ \ a.\ \ \textit{EINE blonde Frau}
\end{styleStandard}

\begin{styleStandard}
\ \ \ \ one \textsubscript{\ \ \ \ }blonde woman.\textsc{fem}
\end{styleStandard}

\begin{styleStandard}
\ \ \ \ ‘one blonde woman’
\end{styleStandard}

\begin{styleStandard}
\ \ b.\ \ \textit{Ø}\textsubscript{[-}\textsc{\textsubscript{pl}}\textsubscript{]} Adjective\textit{ }REL Noun
\end{styleStandard}

\begin{styleStandard}
To be clear, \textit{ein} flags the presence of REL and supports the singularity operator at the same time. The same applies to \textit{mein} ‘my’ and \textit{kein} ‘no’ discussed above. This means that cases of supporting also involve flagging. In fact, cases of supporting seem to entail instances of flagging in that all overt operators occur with REL:\footnote{\ Currently, I am not aware of cases where \textit{ein} only supports an operator but does not flag a covert operator at the same time. In other words, all nouns in strings involving “… \textit{ein} … Noun” seem to involve REL (or another covert operator, see below). } 
\end{styleStandard}

\begin{styleStandard}
(29)\ \ Supporting and Flagging:
\end{styleStandard}

\begin{styleStandard}
\ \  \ \ a.\ \ (\textit{m-eine}) \ \ \ \textsubscript{\ }\ \ \ \ REL
\end{styleStandard}

\begin{styleStandard}
\ \  \ \ b.\ \ (\textit{k-eine}) \ \ \ \ \ \ \ \ \textsubscript{\ }REL
\end{styleStandard}

\begin{styleStandard}
\ \  \ \ c.\ \ (\textit{Ø}\textsubscript{[-}\textsc{\textsubscript{pl}}\textsubscript{]}\textit{+eine})\textit{ \ }REL
\end{styleStandard}

\begin{styleStandard}
Before moving on, consider some other points about REL.
\end{styleStandard}

\begin{styleStandard}
\ \ I proposed in Chapter 7 that other determiners can also flag REL. Besides regular definite determiners like \textit{der} ‘the’ as in de Swart \textit{et al}. (2007), I proposed that pronominal determiners like \textit{du} ‘you’ can do so as well (e.g., \textit{der Wagen} ‘the car’; \textit{du Schwein} ‘you (pig =) idiot’). This means that \textit{ein} surfaces in indefinite contexts but \textit{der} or \textit{du} in definite environments. Given that only one determiner can be merged in ArtP, the distinction followed from the different vocabulary entries where \textit{ein} is not marked for (in-)definiteness but \textit{der} has a definiteness feature and \textit{du} has features for author and participant (entailing definiteness). More generally, it seems clear that different elements can flag the same operator and that the presence of REL is not a sufficient condition for the occurrence of \textit{ein }(we see in Section 2.2.3 below that the presence of REL is not a necessary condition for the occurrence of \textit{ein} either). With this brief review in place, consider more complex cases.
\end{styleStandard}

\begin{styleStandard}
2.2.2. Nominals with Two Overt Operators
\end{styleStandard}

\begin{styleStandard}
As just seen again, there are two ways for \textit{ein} to make operators visible: supporting and flagging. Interestingly, the cases discussed above can be more complex – they can involve two overt operators in one nominal, both supported by \textit{ein}. To set the stage, recall that we saw above that \textit{so} ‘such’ can precede \textit{ein} where both unreduced and reduced forms of \textit{ein} are possible (for (30b), cf. Elmentaler \& Rosenberg 2015: 389):\footnote{\ Hole \& Klumpp (2000) propose that \textit{son}, that is, \textit{so’n} in (30), is an article that simultaneously expresses definite type reference and indefinite token reference. One of their main arguments is that since \textit{son} can appear in plural contexts but the indefinite article \textit{ein} cannot, \textit{son} must be an atomic element. They wind up suggesting that German has three articles: the definite article, the indefinite article, and \textit{son}. Note, however, that German allows the following strings:\par 
\setcounter{listWWviiiNumxivleveli}{0}
\begin{listWWviiiNumxivleveli}
\item 
\begin{styleFootnote}
a.\ \ \textit{so jemand}
\end{styleFootnote}
\end{listWWviiiNumxivleveli}
so someone.\textsc{masc}\par ‘someone like that’\par \ \ \ \ b.\ \ \textit{so (et)was}\par \ \ \ \ \ \ so something.\textsc{neut}\par \ \ \ \ \ \ ‘something like that’\par There are different kinds of \textit{so} (Heusinger 2011: 20-21). Taking into consideration the different semantic contributions of \textit{jemand} ‘someone’ and \textit{(et)was} ‘something’ (and abstracting away from the independent non-reducibility of \textit{jemand} and \textit{(et)was}), the element \textit{so} in (i) appears to be, as far as I can tell, the same as in (30) (admittedly, this intuition has to be investigated in more detail). If so, this parallel of (i) and (30) makes atomicity of \textit{so’n} less likely. More importantly, \textit{ein} can also surface as an unreduced form after \textit{so} (also in the plural, Chapter 1, Section 2.2), and \textit{so’n} can follow another article and/or a numeral in the same noun phrase (as shown in (31) below). Thus, the occurrenc of two instances of \textit{ein}, one after \textit{k}{}- and one after \textit{so}, and the low position of \textit{so’n} itself indicate that \textit{son} is not an atomic article but a bipartite form (as indicated in the alternative spelling). Finally, as shown in Chapter 1, Section 2.2, plural \textit{ein} can also appear in \textit{was-für} contexts, both in reduced and unreduced form. As far as I know, there are no claims that \textit{(was) für} and \textit{ein} form an atomic article. For cross-linguistic discussion of \textit{so} and the indefinite article, see Wood \& Vikner (2011, 2013).}
\end{styleStandard}

\begin{styleFootnote}
(30)\ \ a.\ \ \textit{so (ei)ne (nette) Frau}
\end{styleFootnote}

\begin{styleFootnote}
\ \ \ \ so \ a \ \ \ \ \ \ \ \ nice \ \ woman.\textsc{fem}
\end{styleFootnote}

\begin{styleFootnote}
\ \ \ \ ‘such a nice woman’
\end{styleFootnote}

\begin{styleFootnote}
\ \ b.\ \ \textit{irgend so (ei)ne Frau} 
\end{styleFootnote}

\begin{styleFootnote}
\ \ \ \ any \ \ \ \ so \ \ a \ \ \ \ \ \ woman.\textsc{fem}
\end{styleFootnote}

\begin{styleFootnote}
\ \ \ \ ‘some woman or other’
\end{styleFootnote}

\begin{styleFootnote}
The element \textit{so} can roughly be rendered as ‘such’ or ‘of that type’. If we interpret this element as an overt type operator, then a number of other interesting points can be made.
\end{styleFootnote}

\begin{styleStandard}
Putting the possessive articles aside for a moment (but see below), we have seen that \textit{ein} can support NEG and \textit{Ø}\textsubscript{[-PL]}. I assume that this also holds for \textit{so }‘such’. Considering (31a) and (31b), note that there are two operators in each case and that \textit{ein} follows both in each case yielding two instances of the indefinite article. Notice that the example in (31b) is parallel to an example from the Appendix, (31c), where a numeral other than \textit{EIN} ‘one’ precedes \textit{so} suggesting that \textit{EIN} is the regular singularity numeral in (31b). Indeed, \textit{so’n} can follow another article and a numeral as in the constructed example in (31d) (the example in (31b) is adapted from Hole \& Klumpp 2000: 238, also Elmentaler \& Rosenberg 2015: 383, 389; for Dutch, see Barbiers 2008: 6; note that the numeral \textit{zwei} ‘two’ should be stressed in (31d)):
\end{styleStandard}

\begin{styleFootnote}
(31)\ \ a.\ \ \textit{k-eine so’ne Frau}
\end{styleFootnote}

\begin{styleFootnote}
\ \ \ \ \textsc{neg}{}-a so.a \ \textsubscript{\ }woman.\textsc{fem}
\end{styleFootnote}

\begin{styleFootnote}
\ \ \ \ ‘no such woman’
\end{styleFootnote}

\begin{styleFootnote}
\ \ b.\ \ \textit{EINE \ \ \ so’ne Frau}
\end{styleFootnote}

\begin{styleFootnote}
\ \ \ \ \textit{Ø}\textsubscript{[-}\textsc{\textsubscript{pl}}\textsubscript{]}+a so.a \ \ woman.\textsc{fem}
\end{styleFootnote}

\begin{styleFootnote}
\ \ \ \ ‘one such woman’
\end{styleFootnote}

\begin{styleStandard}
\ \ c. \ \%\ \ \textit{drei}~ \ \emph{so ne geilen}~ \ \ \ \ \textit{Mädels}
\end{styleStandard}

\begin{styleStandard}
\ \ \ \ three so.a \ \ awesome girls
\end{styleStandard}

\begin{styleStandard}
\ \ \ \ ‘three such awesome girls’
\end{styleStandard}

\begin{styleStandard}
\ \ d. \ \%\ \ \textit{Es gibt \ \ keine \ }\textit{\textsubscript{\ }}\textit{zwei so’ne Frauen auf der Welt}.
\end{styleStandard}

\begin{styleStandard}
\ \ \ \ it \ \ gives \textsc{neg}{}-a two \ so.a \ \ \ women in \ \ the world
\end{styleStandard}

\begin{styleStandard}
\ \ \ \ ‘There are not two such women in the world.’
\end{styleStandard}

\begin{styleFootnote}
The current account is able to throw some light on these multiple occurrences of \textit{ein}. Before I provide the derivations, I identify some restrictions. 
\end{styleFootnote}

\begin{styleFootnote}
First, unlike (30), (31) cannot involve an unreduced form of \textit{ein} as a second instance. This is particularly clear with the singularity numeral (32a). Furthermore, the first instance of \textit{ein} cannot be a reduced form either (32b) ((32a) is based on Elmentaler \& Rosenberg 2015: 383):
\end{styleFootnote}

\begin{styleFootnote}
(32)\ \ a. \ *\ \ \textit{EINE so eine Frau}
\end{styleFootnote}

\begin{styleFootnote}
\ \ \ \ one \ \ \ so a \ \ \ \ woman.\textsc{fem}
\end{styleFootnote}

\begin{styleFootnote}
\ \ b. \ *\ \ \textit{‘ne so eine Frau}
\end{styleFootnote}

\begin{styleFootnote}
\ \ \ \  a \ \ so a \ \ \ \ \ woman.\textsc{fem}
\end{styleFootnote}

\begin{styleFootnote}
Second, \textit{so} ‘such’ cannot be left out:
\end{styleFootnote}

\begin{styleFootnote}
(33)\ \ a. \ *\ \ \textit{keine’ne Frau}
\end{styleFootnote}

\begin{styleFootnote}
\ \ \ \ no.a \ \ \ \ \ \ \textsubscript{\ }woman.\textsc{fem}
\end{styleFootnote}

\begin{styleFootnote}
\ \ b. \ *\ \ \textit{EINE’ne Frau}
\end{styleFootnote}

\begin{styleFootnote}
\ \ \ \ one.a \ \ \ \ \ woman.\textsc{fem}
\end{styleFootnote}

\begin{styleFootnote}
Third, the second instance of \textit{ein} cannot be left out: 
\end{styleFootnote}

\begin{styleFootnote}
(34)\ \ a. \ *\ \ \textit{keine so Frau}
\end{styleFootnote}

\begin{styleFootnote}
\ \ \ \ no \ \ \ \ so woman.\textsc{fem}
\end{styleFootnote}

\begin{styleFootnote}
\ \ b. \ *\ \ \textit{EINE so Frau}
\end{styleFootnote}

\begin{styleFootnote}
\ \ \ \ one \ \ \ so woman.\textsc{fem}
\end{styleFootnote}

\begin{styleFootnote}
Fourth, \textit{so’n} is not possible in definite contexts such as possessives (35a-b) or definite determiners (35c-d):
\end{styleFootnote}

\begin{styleFootnote}
(35)\ \ a. \ *\ \ \textit{meine so’ne Frau}
\end{styleFootnote}

\begin{styleFootnote}
\ \ \ \ my \ \ \ \ so.a \ \ woman.\textsc{fem}
\end{styleFootnote}

\begin{styleFootnote}
\ \ b. \ *\ \ \textit{Peters \ so’ne Frau}
\end{styleFootnote}

\begin{styleFootnote}
\ \ \ \ Peter’s so.a \ \ woman.\textsc{fem}
\end{styleFootnote}

\begin{styleFootnote}
\ \ c. \ *\ \ \textit{die so’ne Frau}
\end{styleFootnote}

\begin{styleFootnote}
\ \ \ \ the so.a \ \ woman.\textsc{fem}
\end{styleFootnote}

\begin{styleFootnote}
\ \ d. \ *\ \ \textit{diese so’ne Frau}
\end{styleFootnote}

\begin{styleFootnote}
\ \ \ \ this \ \ so.a \ \ \ woman.\textsc{fem}
\end{styleFootnote}

\begin{styleFootnote}
Summarizing these sets of data, note that these multiple occurrences of \textit{ein} only surface in indefinite contexts where one instance of \textit{ein} occurs before \textit{so} and one after it. Furthermore, when both instances of \textit{ein} are present, \textit{so} cannot be left out; that is, both instances of \textit{ein} cannot be adjacent. When the first instance of \textit{ein} and \textit{so} are present, the second instance of \textit{ein} must be present as well. Finally, the first instance of \textit{ein} involves an unreduced form, but the second instance of \textit{ein} is a reduced form. To repeat, this is different when the first instance of \textit{ein} is missing as in (30). In the latter case, \textit{ein} can appear in both reduced or unreduced forms.
\end{styleFootnote}

\begin{styleFootnote}
\ \ I argued in Chapter 1, Section 4 that determiners are base-generated in ArtP and move successive cyclically up the tree. I propose that the multiple occurrences of \textit{ein} above are also related by movement such that two copies of \textit{ein} are pronounced. For concreteness, I assume that the operator \textit{so} is in the specifier position of a Type Phrase (TypeP). I repeat (31b) below as (36a) and derive it as in (36b). I suggest that \textit{Ø}\textsubscript{[-PL]} and \textit{so} each get supported by \textit{ein}. The relevant elements are linearly adjacent:\footnote{\ For the counterpart in Dutch, Barbiers (2005: 171) suggests that NumP (CardP in the current system) is base-generated above DP. Such a strong (non-standard) claim does not have to be made in the current account.}
\end{styleFootnote}

\begin{styleStandard}
(36)\ \ a.\ \ \textit{EINE \ \ \ so’ne Frau}
\end{styleStandard}

\begin{styleFootnote}
\ \ \ \ \textit{Ø}\textsubscript{[-}\textsc{\textsubscript{pl}}\textsubscript{]}+a so.a \ \ woman.\textsc{fem}
\end{styleFootnote}

\begin{styleFootnote}
\ \ \ \ ‘one such woman’
\end{styleFootnote}

\begin{styleStandard}
\ \ b. \ \ \ \ \ CardP
\end{styleStandard}

\begin{styleStandard}
[Warning: Draw object ignored][Warning: Draw object ignored]
\end{styleStandard}

\begin{styleStandard}
\textit{\ \ \ \ \ \ \ \ \ \ \ \ \ \ Ø}\textsubscript{[-}\textsc{\textsubscript{pl}}\textsubscript{]} \ \ \ \ \ \ \ \ \ Card’
\end{styleStandard}

\begin{styleStandard}
[Warning: Draw object ignored][Warning: Draw object ignored]
\end{styleStandard}

\begin{styleStandard}
\ \  \ \ \ \ \ \ \ \ \ \ \textit{eine}\textsubscript{i} \ \ \ \ \ \  \ \ \ TypeP
\end{styleStandard}

\begin{styleFootnote}
[Warning: Draw object ignored][Warning: Draw object ignored]
\end{styleFootnote}

\begin{styleFootnote}
\ \ \ \  \ \ \ \ \ \ \ \ \textit{so}\ \  Type’
\end{styleFootnote}

\begin{styleFootnote}
[Warning: Draw object ignored][Warning: Draw object ignored]
\end{styleFootnote}

\begin{styleFootnote}
\ \ \ \ \ \  \ \ \ \ \ \textit{’ne}\textsubscript{i}\ \ \ \ ArtP
\end{styleFootnote}

\begin{styleFootnote}
[Warning: Draw object ignored][Warning: Draw object ignored]
\end{styleFootnote}

\begin{styleFootnote}
\ \ \ \ \ \ \ \  \textit{eine}\textsubscript{i}\ \  \ \ \ \ \ [\textit{Frau}]
\end{styleFootnote}

\begin{styleFootnote}
Some remarks are in order here. 
\end{styleFootnote}

\begin{styleFootnote}
\ \ I argued in Chapter 2, Section 2.1.6 that \textit{ein} involves a complex head consisting of the categorial feature [+D] and a separate feature bundle for case, number, and gender. Notice that this complex head undergoing movement immediately explains that the second instance of \textit{ein} is inflected and that it is inflected in the same way as the first instance. More tentatively, I assume that the operator \textit{so} is a deficient element (i.e., a bound stem) in that it must be supported by \textit{ein}. This accounts for the fact that \textit{ein} must follow \textit{so }– just like with the other overt operators. There are two kinds of derivation. Starting with (30), where only one overt operator is present, either an unreduced or a reduced form of \textit{ein} follows \textit{so}. I assume that in the reduced instances, \textit{ein} has encliticized to the operator.
\end{styleFootnote}

\begin{styleFootnote}
Second, unlike (30), (31a-b) involve two instances of \textit{ein}, and the \textit{ein} following \textit{so} is a reduced form. Nunes (2001) argues that Move is not a primitive operation but rather the output of Copy, Merge, Form Chain, and Copy Reduction (Chapter 1). Assuming that Copy Reduction applies to free, unbound copies, unreduced forms of lower copies are deleted. Extending this discussion, if \textit{ein} is an unreduced copy, it gets deleted by Copy Reduction. This explains the ungrammaticality of (32a-b) as the second instance of \textit{ein} should have been deleted. Furthermore, it explains the ungrammaticality of (34) as \textit{ein} has been deleted but \textit{so} remains unsupported here. The examples in (33) are presumably out as \textit{ein} cannot encliticize to a copy of itself. Note though that \textit{ein} can encliticize to \textit{so}. This reduced form of \textit{ein} is a bound copy and becomes invisible to Copy Reduction (cf. Nunes 2001: 311-12). This derivation allows a second \textit{ein} to surface, and this lower copy can support \textit{so}.\footnote{\ Discussing similar cases, Wood \& Vikner (2011, 2013) also locate lower instances of \textit{ein} in a lower positon of the extended projection of the noun. } \ 
\end{styleFootnote}

\begin{styleFootnote}
Note that I have not discussed cases where both instances of \textit{ein} are reduced. While perhaps not fully grammatical, they seem to be possible:
\end{styleFootnote}

\begin{styleFootnote}
(37) \ ?\ \ \textit{‘ne so’ne Frau}
\end{styleFootnote}

\begin{styleFootnote}
\ \  a \ \ so.a \ \ woman.\textsc{fem}
\end{styleFootnote}

\begin{styleFootnote}
\ \ ‘such a woman.
\end{styleFootnote}

\begin{styleFootnote}
Similar to the discussion above, these cases can be accounted for by movement and enclitization of the second instance of \textit{ein} to \textit{so} ‘such’. Additionally, we can suggest that the first instance of \textit{ein} is reduced to \textit{‘n}, just like other instances of the (reduced) article.
\end{styleFootnote}

\begin{styleFootnote}
Finally, consider the data in (35) again. I assume that \textit{mein} ‘my’ is not possible in the construction in (35a) as this type of \textit{so} is incompatible with definiteness (more on this in Section 2.2.5). This also rules out the cases of the other definite determiner(-like) elements in (35b-d) in addition to the fact that they do not involve (a higher instance of) \textit{ein} to begin with; that is, the cases in (35b-d) are also ungrammatical as there are two independent determiners in the noun phrase at the same time (e.g., \textit{Ø}\textit{\textsubscript{D}}/\textit{der}/\textit{dieser} and \textit{ein}).\footnote{\ Above, we saw that the singularity numeral cannot be followed by \textit{so} ‘such’ and unreduced \textit{ein} (32a). The same holds for possessive articles (ia). The question arises as to why the negative article followed by \textit{so} and unreduced \textit{ein} as in (ib) sounds better than its two counterparts:\par (i) \ \ \ \ a. \ \ *\ \ \textit{meine so eine Frau}\par \ \ \ \ \ \ my \ \ \ \ \textsubscript{\ }so a \ \ \ \ woman.\textsc{fem}\par \ \ \ \ ‘my such woman’\par \ \ b. \ ?\ \ \textit{keine so eine Frau}\par \ \ \ \ \ \ no \ \ \ \ \textsubscript{\ }so a \ \ \ \ woman.\textsc{fem}\par \ \ \ \ \ \ ‘no such woman’\par Note that an informal \textit{google} search retrieved hits such as (ib). The fairly good status of this example is surprising as regards the previous discussion. While I do not have a definite answer here, there appears to be a tension between the need to support \textit{so} and the requirement to delete lower unreduced copies of \textit{ein }(as discussed in the main text). This presumably accounts for the markedness of this type of example. While this tension also holds for the singularity numeral and the possessive articles, there are additional factors that make the latter two cases worse: the case with the singularity numeral involves two (almost) homophonous instances of \textit{ein} and the possessive articles induce definite contexts.} 
\end{styleFootnote}

\begin{styleFootnote}
\ \ To sum up thus far, \textit{ein} appears to be able to support multiple instances of overt operators, and it can flag a single instance of a covert one:
\end{styleFootnote}

\begin{styleStandard}
(38)\ \ Supporting and Flagging:
\end{styleStandard}

\begin{styleStandard}
\ \  \ \ a.\ \ (\textit{m-eine}) \ \  \ \ \ \ \textsubscript{\ \ \ \ \ \ \ \ \ \ }REL
\end{styleStandard}

\begin{styleStandard}
\ \  \ \ b.\ \ (\textit{k-eine}) \ \ \ \ \ \ \ \ \ (\textit{so-ne}) \ REL
\end{styleStandard}

\begin{styleStandard}
\ \  \ \ c.\ \ (\textit{Ø}\textsubscript{[-PL]}\textit{+eine}) \ (\textit{so-ne}) \ REL
\end{styleStandard}

\begin{styleFootnote}
Considering (38), we observe that \textit{ein} can support (at least) four different kinds of overt operators. The question arises if \textit{ein} can also flag different kinds of covert operators. I tentatively suggest in the next section that besides REL, \textit{ein} can also indicate the presence of EXIST, QUAL, QUANT, SORT, and CONT. However, so far, I have not been able to identify a clear case indicating that \textit{ein} can flag multiple instances of covert operators at the same time.
\end{styleFootnote}

\begin{styleFootnote}
2.2.3. Flagging of Other Covert Operators
\end{styleFootnote}

\begin{styleStandard}
There are some other well-known cases where \textit{ein} seems to make a semantic contribution, at least at first glance. I discuss split-scope phenomena, I return to exclamatives, and I briefly discuss \textit{ein} in mass nominals. For each of these cases, I suggest that it is not \textit{ein} that brings about the different interpretations. Rather, different covert operators are taken to be responsible, and \textit{ein} flags the presence of these null elements. One result of this discussion is that I identify more cases involving flagging. Furthermore, I draw the tentative conclusion that there do not seem to be cases where \textit{ein} flags multiple covert operators at the same time.
\end{styleStandard}

\begin{styleStandard}
\ \ I have argued at length that \textit{ein} is not a reflex of indefiniteness. It could be suggested that the absence of definite elements such as \textit{der} ‘the’ or \textit{dieser} ‘this’ leads to indefiniteness, perhaps as a default option. However, this is, most likely, not correct. There is evidence that indefiniteness is not a default option but rather that it involves an actual element. This evidence comes from split-scope phenomena (e.g., Bech 1955/57, Jacobs 1980, data taken from Zeijlstra 2011: 113):
\end{styleStandard}

\begin{styleStandard}
(39)\ \ \textit{Du \ musst keine Krawatte tragen}.
\end{styleStandard}

\begin{styleStandard}
\ \ you must \ no \ \ \ \ tie.\textsc{fem} \ \ \ \ wear
\end{styleStandard}

\begin{styleStandard}
a. \ \ ‘It is not required that you wear a tie.’\ \ \ \ (¬ {\textgreater} must {\textgreater} \textstyleunicodei{\textrm{${\exists}$})}
\end{styleStandard}

\begin{styleStandard}
b. \ \ ‘There is no tie that you are required to wear.’\ \ (¬ {\textgreater} \textstyleunicodei{\textrm{${\exists}$} }{\textgreater} must\textstyleunicodei{)}
\end{styleStandard}

\begin{styleStandard}
c. \ \ ‘It is required that you don’t wear a tie.’\ \ \ \ (must {\textgreater} ¬ {\textgreater} \textstyleunicodei{\textrm{${\exists}$})}
\end{styleStandard}

\begin{styleStandard}
\textstyleunicodei{The most }salient reading is paraphrased in (39a), where the negation has the widest scope, and the existential operator has the lowest scope. In other words, the negative article \textit{kein} ‘no’ is split into two parts, and both elements have semantic import. Note first that split scope provides more evidence for the composite analysis of the negative article (for a different view, see Jäger \& Doris 2012). Now, given that the different scope options are derived from the various positions of the existential operator in (39a-c), I conclude that this operator involves an actual element. In keeping with the claims made thus far, I suggest that this existential operator is a null element and that this element is responsible for indefiniteness. For convenience, I relabel \textstyleunicodei{\textrm{${\exists}$}} as EXIST(ence). I assume that the presence of this covert operator is flagged by \textit{ein}. 
\end{styleStandard}

\begin{styleFootnote}
\ \ Second, as briefly discussed in Chapter 4, Section 2 special contexts such as exclamatives license the presence of \textit{ein} with plural nouns as in the (constructed) example in (40a). To repeat, van Riemsdijk (2005: 167) points out that these – often called – emotive or affective constructions can express some relatively excessive property of the entities under discussion. Mass nouns can also occur with \textit{ein} (40b). Note that both examples have a clear quality reading expressed in the translation as ‘great’. While the latter reading may be dominant, a quantity interpretation also seems possible rendered in the translation as ‘a lot of’:
\end{styleFootnote}

\begin{styleFootnote}
(40)\ \ a. \%\ \ \textit{Was \ für (ei)ne Autos!}
\end{styleFootnote}

\begin{styleFootnote}
\ \ \ \ what for \ a \ \ \ \ \ \ cars
\end{styleFootnote}

\begin{styleFootnote}
\ \ \ \ ‘Such great cars!’
\end{styleFootnote}

\begin{styleFootnote}
\ \ \ \ ‘Such a lot of cars!’
\end{styleFootnote}

\begin{styleFootnote}
\ \ b. \ \ \textit{Was \ für (ei)n Bier!}
\end{styleFootnote}

\begin{styleFootnote}
\ \ \ \ what for \ a \ \ \ \ \ beer.\textsc{neut}
\end{styleFootnote}

\begin{styleFootnote}
\ \ \ \ ‘Such great beer!’
\end{styleFootnote}

\begin{styleFootnote}
\ \ \ \ ‘Such a lot of beer!’
\end{styleFootnote}

\begin{styleFootnote}
The presence of \textit{ein} is unexpected with plural nouns in indefinite contexts and with mass nouns more generally.\footnote{\ As briefly discussed in Chapter 1, Section 2.2.1, southern dialects of German allow indefinite articles to occur in plural contexts as part of argumental expressions (without \textit{so} ‘such’). The same holds true of mass nouns (Haider 1988: 52 fn. 14):\par \ \ (i)\ \ \textit{wos \ \ moch i den mid \ an wossa}\ \ \ \ (Bavarian German)\par \ \ \ \ what make I \textsc{prt} with a \ \ water\par \ \ \ \ ‘What do I do with water?’\par Unlike Standard German, the indefinite article here is not restricted to singular (non-emotive) contexts.} I argued in Chapter 4, Section 2.2 that cases like (40a) cannot involve a null noun between the article and the overt noun; that is, \textit{ein} cannot be due to the presence of a null noun. I assume the same for (40b). While \textit{was für }‘what kind’ is an (interrogative) operator in questions, the examples in (40) involve exclamatives. Thus, it seems less likely that \textit{was} \textit{für }is the operator here. I suggest that there are two covert semantic operators at work, QUAL(ity) and QUANT(ity) (cf. Chapter 4, Section 2, where Bennis \textit{et al}. 1998 and van Riemsdijk 2005 also claimed null operators, their [+EXCL] and !!!, respectively, to be present in the related constructions in Dutch).\footnote{\ Note that besides \textit{was für }‘what kind’, the cases in (40) are also possible with \textit{so} ‘such’ instead. It might be claimed that \textit{so} is the relevant operator here (similar to Section 2.2.2 above). } Depending on which operator is present, a different interpretation obtains. The presence of these operators is flagged by \textit{ein}. 
\end{styleFootnote}

\begin{styleFootnote}
\ \ Third, mass nouns are interesting in other ways. When they occur with an indefinite article, they can also have an effect different from (40b) above. Specifically, (41a) has a sortal reading, and (41b) has a container reading as is clear from the translation involving \textit{loaf}:
\end{styleFootnote}

\begin{styleFootnote}
(41)\ \ a.\ \ \textit{Pumpernickel ist (ei)n Brot}.
\end{styleFootnote}

\begin{styleFootnote}
\ \ \ \ pumpernickel is \ \ \textsubscript{\ }a \ \ \ \ bread.\textsc{neut} 
\end{styleFootnote}

\begin{styleFootnote}
\ \ \ \ ‘Pumpernickel is a bread (= certain kind of that substance).’
\end{styleFootnote}

\begin{styleFootnote}
\ \ b.\ \ \textit{Ich möchte \ \ \ \ \ (ei)n Brot}.
\end{styleFootnote}

\begin{styleFootnote}
\ \ \ \ I \ \ \ \ would.like \textsubscript{\ }a \ \ \ \ bread.\textsc{neut}
\end{styleFootnote}

\begin{styleFootnote}
\ \ \ \ ‘I would like a loaf of bread (= a certain quantity of that substance).’
\end{styleFootnote}

\begin{styleStandard}
Notice that unlike (40b), these instances involve non-exclamative contexts. I suggest that semantic operators also account for these data. To be concrete, I simply label these elements as SORT(er) and CONT(ainer) (cf. ‘univeral sorter’ and ‘universal packer’, respectively; see H. Wiese \& Maling 2005 for discussion). Again, I take it that \textit{ein} flags the presence of the operators that bring about the different readings.
\end{styleStandard}

\begin{styleStandard}
\ \ To sum up, I made the claim more general that \textit{ein} indicates the presence of covert operators. In particular, I extended this claim to split-scope phenomena (EXIST), to plural indefinite nouns in exclamatives (QUAL, QUANT), and to mass nouns (QUAL, QUANT, SORT, CONT). This means that besides REL, other covert operators appear to be flagged by \textit{ein}:
\end{styleStandard}

\begin{styleStandard}
(42)\ \ Flagging:
\end{styleStandard}

\begin{styleStandard}
\ \  \ \ a.\ \ \textit{(ei)n}\ \ \ \ REL
\end{styleStandard}

\begin{styleStandard}
\ \  \ \ b.\ \ \textit{(ei)n}\ \ \ \ EXIST
\end{styleStandard}

\begin{styleStandard}
\ \  \ \ c.\ \ \textit{(ei)n}\ \ \ \ QUAL
\end{styleStandard}

\begin{styleStandard}
\ \  \ \ d.\ \ \textit{(ei)n}\ \ \ \ QUANT
\end{styleStandard}

\begin{styleStandard}
\ \ \ e.\ \ \textit{(ei)n}\ \ \ \ SORT
\end{styleStandard}

\begin{styleStandard}
\ \ \ f.\ \ \textit{(ei)n} \ \ \ \ CONT
\end{styleStandard}

\begin{styleStandard}
Thus far, I have not identified any cases where \textit{ein} flags two covert operators at the same time.\footnote{\ A case with two covert operators might involve the complex determiner \textit{ein jeder} ‘(an every =) each’, where Roehrs (2012) suggests that \textit{ein} intensifies the distributive reading of \textit{jeder}. Accepting the proposal that \textit{ein} itself is not responsible for the semantics, a null distributivity operator (DIST) could be postulated. The example in (ia) could then be schematically analyzed as in (ib):\par (i)\ \ a.\ \ \textit{ein jeder \ nette Student}\par \ \ \ \ \ \ an \ every nice \ student.\textsc{masc}\par \ \ \ \ \ \ ‘each nice student’\par \ \ \ \ b.\ \ \textit{ein} DIST \textit{jeder} Adjective REL Noun\par This presents a case of multiple covert operators in one nominal. However, it is unlikely that \textit{ein} flags both DIST and REL at the same time, as the definite determiner \textit{jeder} is present; that is, \textit{jeder} presumably flags REL on par with \textit{der} ‘the’ or \textit{du} ‘you’ as discussed above. Notice also that DIST is different from the other covert operators above in that it intensifies the reading of a specific determiner (rather than determines the reading of the nominal as a whole). As such, I do not add it to the list in (42).} More generally, if the interpretation of the above empirical observations is tenable, then I can state again that the traditional term indefinite article is a misnomer. The fact though that one covert operator (CAP) is not indicated by \textit{ein}, but the operators of the rather long list in (42) are is not very satisfying. Some more remarks are in order.
\end{styleStandard}

\begin{styleStandard}
That CAP is not flagged by \textit{ein} but that the types of operator in (42) are should, ideally, be related to an independent factor. If possible, we might like to claim that the respective operators have something in common. Pending further investigation, I think it is not impossible to make such a claim. In particular, we could suggest that the position the operator appears in is related to it being flagged by \textit{ein}. De Swart \textit{et al}. (2007) proposed that CAP is in NP and that REL is in NumP, the latter being indicated in singular contexts by \textit{ein}. Note that NumP is closer to the base-position of \textit{ein} (ArtP) than NP. Assuming that locality plays a role here, this could explain why REL (but not CAP) is flagged by \textit{ein}. If so, we could suggest further that all covert operators in (42) are in NumP. This, in turn, might open another avenue of research.
\end{styleStandard}

\begin{styleStandard}
\ \ It is linguistically neither very interesting nor very elegant to postulate a different operator for every different reading. This is basically what I did at the beginning of this subsection. In my view, it is desirable to pursue the idea that these operators can be related or even collapsed into fewer elements. While I have to leave the ultimate solution of this issue to semanticists, there are a couple of morpho-syntactic observations worth making here.
\end{styleStandard}

\begin{styleStandard}
\ \ REL and EXIST typically occur with count nouns. While REL was originally motivated for predicative contexts, EXIST (or \textstyleunicodei{\textrm{${\exists}$}}) is postulated for argumental expressions. At the very least, these cases seem structurally related in that predicative NumP (REL) is part of argumental DP (EXIST). At best, these two operators could be collapsed into one if the syntax of predicate vs. argumental nominals could explain the remaining properties (e.g., figurative extension, scope, etc.).
\end{styleStandard}

\begin{styleStandard}
\ \ By contrast, QUAL, QUANT, SORT, and CONT typically occur with mass nouns. If these four covert operators are indeed all in NumP as tentatively suggested above, then they could be collapsed into one more general operator that gets specified in a given context. In fact, it might turn out that the pragmatics specifies this general operator as regards the different readings. While I have to leave a more robust discussion of these issues for future research, it does not seem impossible to simplify the account above. Pending a solution of these issues, I continue with the operators as presented in (42).
\end{styleStandard}

\begin{styleFootnote}
The result of the discussion above is that we can observe that all nominals containing \textit{ein} involve at least one operator. I believe this is an interesting claim. Crucially though, \textit{ein} occurs with a number of different operators. This means that stating the vocabulary insertion rule for \textit{ein} and accounting for the presence or absence of \textit{ein} in the various contexts is not trivial. A number of complications arise. As such, the discussion in the next three subsections is, admittedly, somewhat involved. However, I think that it serves well to highlight the complexities that arise. Indeed, as should be clear, any account espousing the composite analysis of complex \textit{ein}{}-words has to deal with these complications. First, I address \textit{ein} in the context of covert operators, then I add overt operators to the discussion, and finally, I provide my account involving feature deletion, which may widen the distribution of \textit{ein}.
\end{styleFootnote}

\begin{styleStandard}
2.2.4. Contexts of Vocabulary Insertion for \textit{ein}: Covert Operators
\end{styleStandard}

\begin{styleFootnote}
In the previous chapters, I argued that the indefinite article does not have semantic import. Rather, it flags and supports operators. I proposed that \textit{ein} involves the categorial feature [+D] and a separate feature bundle for case, number, and gender. I took [+D] to be the semantically vacuous element that is spelled out as the stem \textit{ein}; the features for case, number, and gender are spelled out as its inflection. If features in addition to [+D], case, number, and gender are present, different elements are spelled out. To give some examples: if [+DEF] is present, the definite article \textit{der} ‘the’ is spelled out; if [+DEF; +DEIX] are present, the demonstrative \textit{dieser} ‘this’ is realized; if features for [AUTH] and [PART] are present, pronominal determiners surface. Note though that the distribution of determiners is more complicated when we consider the indefinite articles \textit{Ø}\textit{\textsubscript{D}} and \textit{ein} in the context of operators. 
\end{styleFootnote}

\begin{styleFootnote}
For instance, if \textit{ein} flags REL, it only occurs in singular contexts. However, if \textit{ein} supports an overt operator (in addition to flagging REL), it can occur in mass, singular, and plural contexts. Leaving the discussion of overt operators to the next subsection, note that all covert operators flagged by \textit{ein} involve indefinite contexts (see previous subsection). Before I turn to a more detailed discussion of the vocabulary insertion rules of the articles in indefinite contexts, I review the cases above and add a few minor points to the investigation. This lays the foundation for the account of Vocabulary Insertion of determiners. Given the complex sets of data, the discussion is summarized at various points in Tables 1 through 3 below.
\end{styleFootnote}

\begin{styleFootnote}
Nouns become instantiated as mass, singular count, or plural count nominals during the derivation. I proposed in Chapter 7 that the interplay between the head noun and Num as regards morphological and consequently semantic number leads to the three manifestations. Importantly, independent of number (and thus manifestation of the noun as mass, singular, or plural), all types of nominals involve an operator. Focusing on kind nouns, recall that they are of type {\textless}e{\textgreater}. In order to become predicates (type {\textless}e,t{\textgreater}), they combine with REL. This step is clearly needed so that these nominals can combine with adjectival modifiers (type {\textless}e,t{\textgreater}) by Predicate Modification. De Swart \textit{et al}. (2007) did not discuss mass nouns in this regard. However, since they can also be modified, I assume that they also involve REL. Furthermore, mass nouns can also occur in the context of existential operators. As such, they involve EXIST – just like count nouns. As is well known, in indefinite, non-exclamative contexts, mass and plural count nominals involve null articles, and singular count nominals contain \textit{ein}. This is summarized in Table 1 (recall that mass nouns involve Num without a specification for morphological number): 
\end{styleFootnote}

\begin{styleFootnote}
Table 1: Covert Operators in Indefinite, Non-exlamative Contexts
\end{styleFootnote}

\begin{flushleft}
\begin{tabular}{|m{1.5837599in}|m{1.5837599in}|}

\hline
\centering Type of nominal &
\centering\arraybslash Non-exclamative\\\hline
mass (Num) &
{\fontsize{10pt}{12.0pt}\selectfont \textit{Ø}\textit{\textsubscript{D}} (REL/EXIST)}\\\hline
{\fontsize{10pt}{12.0pt}\selectfont singular (Num\textsubscript{[-PL]})} &
{\fontsize{10pt}{12.0pt}\selectfont \textit{ein} (REL/EXIST)}\\\hline
{\fontsize{10pt}{12.0pt}\selectfont plural (Num\textsubscript{[+PL]})} &
{\fontsize{10pt}{12.0pt}\selectfont \textit{Ø}\textit{\textsubscript{D}} (REL/EXIST)}\\\hline
\end{tabular}
\end{flushleft}
\begin{styleFootnote}
More needs to be said about mass nouns.
\end{styleFootnote}

\begin{styleFootnote}
Mass nouns on their sortal (SORT) and container (CONT) reading as in (41a-b) occur in non-exclamative contexts and most likely involve nominals where Num is specified as [-PL morph]. This is so as these nominals can be pluralized (e.g., \textit{zwei verschiedene Zucker} ‘two different sugars’; \textit{zwei Brote} ‘two loaves of bread’). As such, I put the singular instances of these readings with the singular nominals in Table 2 below. I assume that the distinction between the ordinary count reading of \textit{ein Schwein} ‘a pig’ and the special sortal reading of \textit{ein Zucker} ‘a (type of) sugar’ follows from the different lexical items involved and their preferred respective interpretation (Borer 2005: 102, see also H. Wiese \& Maling 2005: 8). In Table 2, the singular sortal and container readings are put in a different row from the ordinary readings involving REL/EXIST. To be clear, compared to Table 1, Table 2 involves an additional row in the non-exclamative column.
\end{styleFootnote}

\begin{styleFootnote}
As to exclamative contexts (see column three of Table 2), it is not obvious if the quantity (QUANT) or quality (QUAL) reading of mass nouns involving \textit{ein} as in (40b) is singular. The pluralization (i.e., \textit{Was für eine Biere!} ‘Such beers!’) seems a bit marked and has a different, sortal reading. Furthermore, it does not seem plausible to relate the unspecific quantity reading, rendered in (40b) above as ‘a lot of’, to semantic singularity and countability. Rather, it seems more straightforward to interpret these cases as actual mass nominals; that is, as nominals that involve Num with no morphological specification at all. If so, these nominals are number neutral and can be interpreted in different ways. These nominals are put in the row of mass nominals in the exclamative column of Table 2. Plural count nominals involving \textit{ein} as in (40a) are put with the plural instances. Perhaps unsurprisingly, singular cases also occur in exclamative contexts. Given the singular context, they only have a quality reading (e.g., \textit{Was für ein Auto!} ‘Such a great car!’). The remarks above can be summarized as follows:
\end{styleFootnote}

\begin{styleFootnote}
Table 2: Covert Operators in Indefinite Contexts
\end{styleFootnote}

\begin{flushleft}
\begin{tabular}{|m{1.5837599in}|m{1.5837599in}|m{1.5837599in}|}

\hline
\centering Type of nominal &
\centering Non-exclamative &
\centering\arraybslash Exclamative\\\hline
mass (Num) &
{\fontsize{10pt}{12.0pt}\selectfont \textit{Ø}\textit{\textsubscript{D}} (REL/EXIST)} &
{\fontsize{10pt}{12.0pt}\selectfont \textit{ein} (QUANT, QUAL)}\\\hline
{\fontsize{10pt}{12.0pt}\selectfont singular (Num\textsubscript{[-PL]})} &
{\fontsize{10pt}{12.0pt}\selectfont \textit{ein} (REL/EXIST)} &
{\fontsize{10pt}{12.0pt}\selectfont \textit{ein} (QUAL)}\\\hline
 &
{\fontsize{10pt}{12.0pt}\selectfont \textit{ein} (SORT, CONT)} &
\\\hline
{\fontsize{10pt}{12.0pt}\selectfont plural (Num\textsubscript{[+PL]})} &
{\fontsize{10pt}{12.0pt}\selectfont \textit{Ø}\textit{\textsubscript{D}} (REL/EXIST)} &
{\fontsize{10pt}{12.0pt}\selectfont \textit{ein} (QUANT, QUAL)}\\\hline
\end{tabular}
\end{flushleft}
\begin{styleFootnote}
Considering Table 2, note again that the articles occur in the context of certain operators. To repeat, articles in non-exclamative contexts are inserted in their usual contexts: the null articles are inserted in mass and plural contexts, and \textit{ein} in singular environments. Exclamative contexts are different. Here, \textit{ein} occurs in all number contexts. Given this complexity, the vocabulary insertion rules for the indefinite articles \textit{Ø}\textit{\textsubscript{D}} and \textit{ein }are developed in two steps, the first step is discussed in the remaining part of this subsection and the second step in the next subsection.
\end{styleFootnote}

\begin{styleFootnote}
All determiners including articles involve the categorial feature [+D]. \textit{Ein} is the least specified determiner – it only has the categorial feature (abstracting away from the inflection). As mentioned above, other determiners including the null articles have more features. In order to restrict the null articles to non-exlamative contexts, I assume for now that these contexts involve an exclamative feature [+EXCL(amative)] (but see the next subsection where this feature is made unnecessary). Postulating that the null articles have the negative counterpart of [+EXCL] as part of their vocabulary insertion rules, this feature will block the occurrence of the null articles in exclamative contexts. Before I provide the preliminary set of vocabulary insertion rules, I need to make some other remarks. 
\end{styleFootnote}

\begin{styleFootnote}
I argued above that countability is a side-effect of Num being specified for singular or plural. While the feature [±COUNT] has not done any work in the current system so far, I make use of it here as it allows me to state the vocabulary insertion rules for determiners more straightforwardly.\footnote{\ Note that with features having only two values (e.g., [±PL]), a three way distinction such as mass/singular/plural cannot be expressed by just one feature. As will become clear, underspecification does not work in these cases – the feature [-COUNT] has to be present.} In order to restrict the occurrence of the articles to their usual contexts (i.e., the non-exclamative cases in Table 1), I assume a special type of feature for all indefinite articles. I label this specification as restriction feature. This restriction feature can be of two types: [±COUNT] and [±PL]. This special feature restricts the occurrence of the articles to contexts as regards morphological number, so that the mass article only occurs in mass contexts, the singular article only in singular contexts, and the plural article only in plural contexts. I assume that this specification is on the categorial feature [+D], and I put it after the colon following the categorial feature (i.e. [+D:\_\_\_]). Note that this feature is not a regular concord feature (the latter being spelled out by the inflection) – it is a compatibility feature (see also the discussion of uninflected \textit{dies} ‘this’ in Chapter 4, Section 5). 
\end{styleFootnote}

\begin{styleFootnote}
Again, restricting the discussion to indefinite contexts for now, I assume that the mass article \textit{Ø}\textit{\textsubscript{D}} involves [+D: -COUNT] and that the plural article \textit{Ø}\textit{\textsubscript{D}} has [+D: +PL]. The above-mentioned feature [-EXCL] confines the null articles to non-exclamative contexts. The first versions of the vocabulary insertion rules for mass and plural \textit{Ø}\textit{\textsubscript{D}} are provided in (43a-b). The singular article \textit{ein} involves the feature [+D: -PL], and its preliminary vocabulary insertion rule is given in (43c) (NB: this number restriction is neither on Num nor on N and will not lead to semantic singularity of \textit{ein}):\footnote{\ Recall from Footnote Error: Reference source not found that \textit{ein} in southern German dialects has a wider distribution, something I will not make more formal here. Also, recall from Chapter 2, Section 2.2.2 that the vocabulary insertion rules for \textit{ein} consisted of three entries. (43c) in the main text presents the elsewhere case. The remaining two rules are repeated here for convenience and updated with the restriction feature [-PL]:\par (i)\ \ a.\ \ [+D: -PL]\ \ \ \ → \ \ \textit{ein- }/ [-F, -N, -O, +S]\par \ \ \ \ b.\ \ [+D: -PL][-F, -O]\ \ \ \ → \ \ \textit{ein} / \_\_ word ]\textsubscript{$\Phi $}}
\end{styleFootnote}

\begin{styleFootnote}
(43)\ \ \textit{Preliminary Vocabulary Insertion Rules for:}
\end{styleFootnote}

\begin{styleFootnote}
a.\ \ [+D: -COUNT; -DEF; -EXCL] \ \ → \ \ \textit{Ø}\textit{\textsubscript{D}}\ \ 
\end{styleFootnote}

\begin{styleFootnote}
\ \ b.\ \ [+D: +PL; -DEF; -EXCL]\ \ \ \ → \ \ \textit{Ø}\textit{\textsubscript{D}}\textit{ }
\end{styleFootnote}

\begin{styleFootnote}
\ \ c.\ \ [+D: \ {}-PL] \ \ \ \ \ \ \ \ → \ \ \textit{ein-\ \ \ \ }
\end{styleFootnote}

\begin{styleStandard}
Returning to the co-occurrence of the operators and articles (Table 2), the null articles \textit{Ø}\textit{\textsubscript{D}} and \textit{ein} are inserted if the features of their vocabulary entries in (43) are present in the syntactic representation, provided their restriction feature on [+D] is compatible with the concord feature on Num of the noun phrase as a whole. Note in this regard that I assume that [-COUNT] in (43a) is only compatible with the absence of [$\alpha $PL morph] on Num. This accounts for the distribution of the different articles occurring with REL/EXIST in the three number contexts (mass, singular, plural) in non-exclamative environments. SORT, CONT, QUANT, and QUAL are different. They only occur with \textit{ein }(Table 2). SORT and CONT occur with \textit{ein} in singular contexts; that is, they occur with \textit{ein} when Num is specified as [-PL morph] – note that only \textit{ein} is compatible with this number specification on Num.\footnote{\ As mentioned above, SORT and CONT also occur in plural contexts (e.g., \textit{zwei verschiedene Zucker} ‘two different sugars’; \textit{zwei Brote} ‘two loaves of bread’). Here, Num is specified as [+PL morph] leading to the insertion of different determiners, provided they are compatible with this number specification (e.g., plural \textit{Ø}\textit{\textsubscript{D}}\textit{ }and \textit{diese} ‘these’).} 
\end{styleStandard}

\begin{styleStandard}
As to QUANT and QUAL, they occur in mass, singular, and plural contexts in exclamative environments. This means that null articles cannot be inserted here – they have the feature [-EXCL]. Note though that \textit{ein} cannot be inserted in mass and plural contexts either, as \textit{ein} is restricted to singular environments by [-PL] on [+D]. The question arises how \textit{ein} can appear in mass and plural contexts with QUANT and QUAL (but not with REL/EXIST, SORT, or CONT in non-exclamative contexts). In the next subsection, I propose that certain operators can delete the restriction feature on [+D]. This allows \textit{ein} (and \textit{Ø}\textit{\textsubscript{D}}) to occur in other number contexts.
\end{styleStandard}

\begin{styleStandard}
2.2.5. Contexts of Vocabulary Insertion for \textit{ein}: Covert and Overt Operators
\end{styleStandard}

\begin{styleFootnote}
As discussed above, in non-exclamative contexts, the null articles are restricted to mass and plural contexts, and \textit{ein} is restricted to singular contexts. Note again that these articles can be used in other environments when certain operators are present, operators that occur in exclamative contexts (e.g., QUANT), but also operators that occur more generally (e.g., POSS). This means that certain combinations of operators and articles have to be allowed while others have to be excluded. Table 3 below repeats the covert operators occurring in exclamative contexts (i.e., QUANT, QUAL), and it also contains the overt operators and their co-occurring articles. Note that POSS (\textit{m}{}-) and POSS (-\textit{s}) stand for possessive articles and Saxon Genitives, respectively:
\end{styleFootnote}

\begin{styleFootnote}
Table 3: Certain Operators and Their Co-occurring Articles
\end{styleFootnote}

\begin{flushleft}
\begin{tabular}{|m{0.6941598in}|m{1.1643599in}|m{0.35525984in}|m{0.46635982in}|m{0.8587598in}|m{0.7962598in}|m{0.7337598in}|m{0.9288598in}|}

\hline
 &
QUANT/QUAL &
{\itshape so} &
{\fontsize{10pt}{12.0pt}\selectfont \textit{Ø}\textsubscript{[-PL]}} &
{\fontsize{10pt}{12.0pt}\selectfont POSS (\textit{m}{}-)} &
{\fontsize{10pt}{12.0pt}\selectfont POSS (-\textit{s})} &
{\fontsize{10pt}{12.0pt}\selectfont NEG (\textit{k}{}-)} &
{\fontsize{10pt}{12.0pt}\selectfont NEG (\textit{nicht})}\\\hline
mass &
\centering{\itshape ein} &
\centering{\itshape ein} &
\centering{\itshape {}-} &
\centering{\itshape ein} &
\centering \textit{Ø}\textit{\textsubscript{D}} &
\centering{\itshape ein} &
\centering\arraybslash{\itshape der}\\\hline
singular &
\centering{\itshape ein} &
\centering{\itshape ein} &
\centering{\itshape ein} &
\centering{\itshape ein} &
\centering \textit{Ø}\textit{\textsubscript{D}} &
\centering{\itshape ein} &
\centering\arraybslash{\itshape der}\\\hline
plural &
\centering{\itshape ein} &
\centering{\itshape ein} &
\centering{\itshape {}-} &
\centering{\itshape ein} &
\centering \textit{Ø}\textit{\textsubscript{D}} &
\centering{\itshape ein} &
\centering\arraybslash{\itshape der}\\\hline
\end{tabular}
\end{flushleft}
\begin{styleFootnote}
These are the main combinations of operators and articles. This means that the following account has to rule out strings with combinations like *\textit{so-Ø}\textit{\textsubscript{D}}\textit{ Milch/Autos} ‘such milk/cars’, *\textit{m-der Freund} ‘my friend’, *\textit{Peters der Freund} ‘Peter’s friend’, *\textit{k-Ø}\textit{\textsubscript{D}}\textit{ Milch/Autos} ‘no milk/cars’, and others.
\end{styleFootnote}

\begin{styleFootnote}
\ \ I propose that the deletion of features under Art in the syntactic representation may widen and narrow the distribution of determiners.\footnote{\ This deletion could be instantiated by Impoverishment.} Specifically, deleting the restriction feature on [+D] will allow all relevant articles to occur in all three number contexts (i.e., mass, singular, plural). To narrow the distribution of articles again, another feature under Art in the syntactic representation can also be deleted. For instance, deleting the categorial feature [+D] under certain conditions will prevent \textit{Ø}\textit{\textsubscript{D}} and \textit{der} from surfacing but not \textit{ein}. 
\end{styleFootnote}

\begin{styleFootnote}
The account of REL/EXIST, SORT and CONT is as above – no features are deleted. However, the analysis of QUANT and QUAL is different. Given the suggestion just made, this alternative will account not only for the occurrence of \textit{ein} with these two operators but also for the distribution of determiners with the overt operators more generally. In addition, this deletion of features will make the use of the feature [EXCL] unnecessary. Before I turn to the details, I lay out some other assumptions. \ 
\end{styleFootnote}

\begin{styleFootnote}
\ \ All argument DPs have determiners. In what follows, I focus on the three types of articles: the null articles as in (44) below, \textit{ein }provided in\textit{ }(45), and the definite article given in (46). As proposed in Chapter 2, Section 2.1.6, these three types of articles have slightly different structures and features under Art (see the (a)-examples below). Each article has its own corresponding vocabulary insertion rule (see the (b)-examples below). This is consistent with the general assumptions of DM, where a difference is made between the syntactic representation and subsequent Vocabulary Insertion. To be clear, Art with its varying feature bundles is in the derivation at the beginning, but the corresponding vocabulary insertion rules of the articles supply the spell-out forms later. Recall that all these articles are merged in ArtP and move to DP (triggering Impoverishment as discussed in Chapter 2). Below are the final structures and vocabulary insertion rules of the three types of articles (for the vocabulary insertion rules of adjectival inflections, see Chapter 2, Section 2.1.5; the forward slash sign in (44a) is explained below):
\end{styleFootnote}

\begin{styleStandard}
(44)\ \ a.\ \ \textit{Indefinite Null Articles}
\end{styleStandard}

\begin{styleStandard}
\ \  \ \ \ \ \ \ \  \ \ \ \ Art\textsubscript{[+D: -COUNT/+PL; -DEF]}
\end{styleStandard}

\begin{styleStandard}
[Warning: Draw object ignored]
\end{styleStandard}

\begin{styleStandard}
\ \  \ \ \ [+D: -COUNT/+PL; -DEF]\ \ \ \ 
\end{styleStandard}

\begin{styleStandard}
\ \ b.\ \ [+D; -DEF]\ \ \ \ \ \ →\ \ \textit{Ø}\textit{\textsubscript{D}}
\end{styleStandard}

\begin{styleStandard}
(45)\ \ a.\ \ \textit{Indefinite Overt Article}
\end{styleStandard}

\begin{styleStandard}
\ \ \ \  \ \ \ \ \  \ \ Art\textsubscript{[+D: -PL][CNG]}
\end{styleStandard}

\begin{styleStandard}
[Warning: Draw object ignored][Warning: Draw object ignored]
\end{styleStandard}

\begin{styleStandard}
\ \ \ \ [+D: -PL] \ \ \ \ \ \ [CNG]\ \ \ \ 
\end{styleStandard}

\begin{styleStandard}
\ \ b.\ \ [+D]\ \ \ \ \ \ \ \ →\ \ \textit{ein-}
\end{styleStandard}

\begin{styleStandard}
(46)\ \ a. \ \ \textit{Definite Article}
\end{styleStandard}

\begin{styleStandard}
\ \ \ \ \ \ \ \ \ \ \  \ \ \ Art\textsubscript{[+D; +DEF][CNG]}
\end{styleStandard}

\begin{styleStandard}
[Warning: Draw object ignored][Warning: Draw object ignored]
\end{styleStandard}

\begin{styleStandard}
\ \ \ \ [+D; +DEF]\ \  [CNG]\ \ \ \ 
\end{styleStandard}

\begin{styleStandard}
\ \ b.\ \ [+D; +DEF]\ \ \ \ \ \ →\ \ \textit{d-}
\end{styleStandard}

\begin{styleFootnote}
Recall that the features bundles under Art undergo feature union. I comment on the feature bundles under Art and on the feature bundles in the vocabulary insertion rules.
\end{styleFootnote}

\begin{styleFootnote}
Recall that the definiteness feature [DEF] is not a concord feature of the noun phrase as a whole. Rather, it is part of specific determiners. The two null indefinite articles involve [-DEF] (44). For convenience, I provided both null articles in (44), but it should be kept in mind that they differ in their restriction feature on [+D] (the two articles are distinguished by the different restriction feature separated by a forward slash sign in (44a)). As to the two other articles, \textit{ein} lacks a definiteness feature (45), and the definite article has [+DEF] (46). Furthermore, unlike the two types of indefinite articles (\textit{Ø}\textit{\textsubscript{D}}, \textit{ein}{}-), there is only one definite article (\textit{d}{}-). The latter element can be inserted in all number contexts. As such, I assume that there is no restriction feature on [+D] with the definite article (46). Importantly, observe that none of the vocabulary insertion rules involve restriction features. Other than that, the features of the vocabulary insertion rules are identical to the features of their corresponding determiner stems under Art. Finally, recall that CNG stands for the inflectional feature bundle involving case, number, and gender. 
\end{styleFootnote}

\begin{styleFootnote}
Crucially, I assume that the determiner structures are all merged freely; that is, either the structure of a null article (44a), the structure of \textit{ein }(45a), or the structure of the definite article (46a) is merged (projecting ArtP in the process). As discussed in detail below, this will allow some articles – specifically the null articles and \textit{ein} – to appear in some other contexts.\footnote{\ For instance, this will allow \textit{ein}, usually restricted to singular contexts, to appear in plural environments (where Num involves a positive feature for [PL]). As made more precise below, this means that the compatibility of determiners and Num in number is checked after the deletion of the restriction feature on [+D]. This will also result in some cases of overgeneration. The latter are accounted for by the deletion of the feature [+D] on the determiner in certain contexts leading to bad derivations. } Furthermore, I assume that all features of a determiner structure must be spelled out or supported. This means that the categorial feature [+D] and, if present, the definiteness feature must be spelled out by a determiner stem. Furthermore, the CNG feature bundle, if present, must be supported by an (overt) determiner stem. Since every type of Art above involves features that must be spelled out, every nominal that involves ArtP has a determiner. 
\end{styleFootnote}

\begin{styleStandard}
In addition, I assume that all operators surface in the left periphery of the noun phrase: either they are base-generated there (e.g., NEG), or they move there (e.g., POSS).\footnote{\ This high position also licenses the occurrence of adjectival inflections and \textit{ein} as expletive elements (see Section 3.2 below).} As a consequence, all operators precede the determiner. This can schematically be illustrated as follows where OP stands for operator, and $\alpha $ may be XP in case the operator is adjoined (e.g., NEG) or X’ in case the operator is in a specifier position (e.g., POSS): 
\end{styleStandard}

\begin{styleStandard}
(47)\ \  \ \ \ \ \ \ \ \ \ \ \ XP
\end{styleStandard}

\begin{styleStandard}
[Warning: Draw object ignored][Warning: Draw object ignored]
\end{styleStandard}

\begin{styleStandard}
\textit{\ \ \ \ \ \ \ \ \ \ \ \ \ \ \ OP} \ \ \ \ \ \ \ \ \ \ \ \ \ \ $\alpha $ \ 
\end{styleStandard}

\begin{styleStandard}
[Warning: Draw object ignored][Warning: Draw object ignored]
\end{styleStandard}

\begin{styleStandard}
\ \  \ \ \ \ \ \ \ \ \ \ \ Art \ \ \ \ \  \ \ \ …
\end{styleStandard}

\begin{styleStandard}
Now, in order to widen the distribution of articles in certain contexts, I assume that all operators (except REL/EXIST, SORT and CONT) delete the restriction feature on [+D] under Art before the determiner structure is checked for compatibility with Num of the larger noun phrase and before Vocabulary Insertion occurs. While this does not affect the distribution of the definite article in (46), this allows the indefinite articles in (44) and (45) to occur in other number contexts with specific operators: the null article(s) with POSS (Saxon Genitives), and \textit{ein} in most of the other cases. With this in place, I continue the discussion of covert QUANT and QUAL and extend it to the overt operators. 
\end{styleStandard}

\begin{styleStandard}
2.2.6. Contexts of Vocabulary Insertion for \textit{ein}: Feature Deletions
\end{styleStandard}

\begin{styleStandard}
As just proposed, the deletion of the restriction feature on [+D] under Art widens the distribution of the indefinite articles. To constrain this widening in the relevant way, there are several options to narrow the distribution of articles. As far as I can see, the most straightforward option is to assume that the categorial feature [+D] is deleted in certain featural contexts by specific operators.\footnote{\ Note that if other features than [+D] were deleted, then this would, in certain contexts, allow the least specified element\textit{ ein} in (45b) to be inserted in the null article structure in (44a) or in the definite article structure in (46a).} For instance, QUANT and QUAL trigger the deletion of [+D] in the context of a definiteness feature, where [DEF] includes [-DEF] and [+DEF]. This is shown in (48a), along with the other operators that work the same way. (48b-c) show the remaining cases. In (48b), [+D] is deleted in the context of the [CNG] feature bundle and the operator POSS (-\textit{s}); in (48c), [+D] is deleted in the context of the feature [-DEF] and the operator NEG (deletion of a feature is marked below by strikethrough):
\end{styleStandard}

\begin{styleStandard}
(48)\ \ \textit{Deletion of [+D] under Art in the Contexts of […]:}
\end{styleStandard}

\begin{styleStandard}
\ \ a.\ \ [+D] \ \ / \ \ [DEF] triggered by QUANT, QUAL, \textit{so}, \textit{Ø}\textsubscript{[-PL]}, POSS (\textit{m}{}-)
\end{styleStandard}

\begin{styleStandard}
\ \ b.\ \ [+D] \ \ / \ \ [CNG] triggered by POSS (-\textit{s})
\end{styleStandard}

\begin{styleStandard}
\ \ c.\ \ [+D] \ \ / \ \ [-DEF] triggered by NEG
\end{styleStandard}

\begin{styleStandard}
These are the deletions that narrow the distributions of the articles. As explicated in more detail below, (48a) only allows \textit{ein }to be inserted in the relevant contexts, (48b) only allows \textit{Ø}\textit{\textsubscript{D}}, and (48c) only allows \textit{ein} or \textit{d}{}- to be inserted in the relevant context. The articles that cannot be inserted into their corresponding structure will lead to bad derivations as the undeleted features of their corresponding structure will not be spelled out or supported. Also, note that each of the “surviving” determiners in (48) still has the feature [+D] in tact under Art. This means that Impoverishment will not be affected (by this feature deletion on the other determiners). Consider the effects of the deletion of the restriction feature and other features in more detail.
\end{styleStandard}

\begin{styleStandard}
\ \ As mentioned above, QUANT and QUAL delete the restriction feature on [+D]. This brings about [D: {}-COUNT], [D: {}- PL], or [D: +PL] in the relevant syntactic representations (Art). Consequently, all three article structures in (44a), (45a), and (46a) can, at least in principle, occur in the three morphological contexts as regards number (mass, singular, plural). In other words, this deletion widens the distribution of the indefinite articles (recall that the definite aricle has no restriction feature to begin with). In addition, I suggested in (48a) above that these two operators delete the categorial feature [+D] under Art in the context of [DEF], which includes [-DEF] and [+DEF]. This allows the structure in (45a) to be spelled out, but not the structures in (44a) and (46a). This, in turn, narrows the distribution of articles. 
\end{styleStandard}

\begin{styleStandard}
\ \ Specifically, if Art in the syntactic representation involves [+D: -COUNT; -DEF] or [+D: +PL; -DEF] as in (44a) above (modulo the deletion), then \textit{Ø}\textit{\textsubscript{D}} cannot be inserted by (44b) as [+D] was deleted by the relevant operator. Similar assumptions hold for the definite article where Art involves [+D; +DEF] [CNG] as in (46a) above, again modulo the deletion. Note that these two cases lead to bad derivations as the remaining definiteness feature under Art cannot be spelled out by a determiner. This is different for \textit{ein}, where Art involves [+D: {}-PL] [CNG] as in (45a). Here, there is no definiteness feature under Art and in the corresponding vocabulary insertion rule. Consequently, \textit{ein} can be inserted. This type of account carries over to the overt operators and their co-occuring determiners.
\end{styleStandard}

\begin{styleStandard}
Unless independently ruled out (e.g., \textit{Ø}\textsubscript{[-PL]}), all operators occur in all three number contexts (Tables 2 and 3). Like QUANT and QUAL, I suggest that \textit{so}, POSS, and NEG also delete the restriction feature on [+D]. Like above, this allows all three article structures to occur in the three number contexts with these operators as well. Furthermore, similar to QUANT and QUAL just discussed, overt operators also delete [+D] under Art in certain contexts narrowing the distribution of the articles, as stated in (48) above.
\end{styleStandard}

\begin{styleStandard}
In more detail, \textit{so} ‘such’ also deletes the categorial feature [+D] under Art in the context of [DEF]. As with QUANT and QUAL, if (44a) or (46a) is merged in ArtP and moves to DP, their corresponding vocabulary items cannot be inserted. As the definiteness feature under Art cannot be spelled out, this leads to bad derivations. By contrast, if (45a) is merged in ArtP and moves to DP, \textit{ein} can be inserted by (45b), and all the features under Art are spelled out. This results in a good derivation. Similar considerations hold for \textit{Ø}\textsubscript{[-PL]}, with the proviso that this element itself is restricted to occur in singular contexts (Chapter 5). There are two types of derivations for POSS and NEG each.
\end{styleStandard}

\begin{styleStandard}
Like \textit{so} above, the possessive component (POSS) of the possessive articles deletes [+D] under Art in the context of [DEF]. This allows (45a) to be spelled out, but not (44a) or (46a). The latter two lead to bad derivations as the definiteness feature under Art cannot be spelled out. In contrast, I suggest that the possessive marker -\textit{s }(POSS) of Saxon Genitives deletes [+D] under Art in the context of the [CNG] feature bundle. This allows (44a) to be spelled out (NB: there is no deletion of [+D] here – the null articles have no CNG features; only the restriction feature on [+D] is deleted). In contrast, the articles in (45b) and (46b) cannot be inserted – [+D] has been deleted under Art. These derivations are bad as the [CNG] feature bundles of (45a) and (46a) cannot be supported by the Saxon Genitive (e.g., *\textit{Peters-er}). In addition, the definiteness feature in (46a) cannot be spelled out.\footnote{\ Recall from Chapter 1, Footnote Error: Reference source not found that definiteness spread, if instantiated by Spec-head agreement, leads to incompatibility between definite Saxon Genitives and the null articles (which are specified as [-DEF]). There are two possible solutions. First, similar to the head D (which involves [uDEF]; see Chapter 1, Section 4.1.1), we could assume that the null articles do not have an inherent feature for definiteness but rather that they get specified in this regard during the derivation. Note though that if the definiteness feature of the null articles gets specified as positive, this presumably turns \textit{Ø}\textit{\textsubscript{D}} into triggers for Impoverishment, resulting in weak adjectives also in nominative/accusative/dative contexts, contrary to fact (see again Chapter 2, Section 2.2.2). Clearly, such a specification process must be prevented. Alternatively, the feature [DEF] on \textit{Ø}\textit{\textsubscript{D}} could be of a different kind. Similar to some vocabulary insertion rules discussed above, this feature could involve an (unvaluable) variable (i.e., [$\alpha $DEF]), which is compatible with both a positive or a negative value of definiteness. In other words, \textit{Ø}\textit{\textsubscript{D}} could occur in either indefinite or definite contexts (note that similar to \textit{ein}, the (in-)definiteness of the larger noun phrase would depend on the presence of possessors and other elements). Since this type of feature does not get specified, Impovishment would not be triggered. I do not attempt here to decide between these two solutions. Note though that either solution involves a definiteness feature on \textit{Ø}\textit{\textsubscript{D}}, and the structures of the null articles, their vocabulary insertion rules, and the relevant deletion rules above would only have to be changed from [-DEF] to [uDEF] or [$\alpha $DEF].}
\end{styleStandard}

\begin{styleStandard}
Finally, I assume that NEG deletes [+D] under Art in the context of [-DEF]. This allows both (45a) and (46a) to be spelled out. Note that (46b) is more specific than (45b). This means that (46a) is spelled out by (46b) and (45a) by (45b). The null articles in (44a) lead to bad derivations. In the latter cases, the definiteness feature under Art cannot be spelled out by (44b). The vocabulary insertion rules from Chapter 5, Section 4.1.2, repeated below for convenience, account for the overt realization of NEG depending on the context: if (45a) is merged and spelled out by (45b), then NEG is realized as \textit{k}{}-; if (46a) is merged and spelled out by (46b), then NEG is spelled out as \textit{nicht}:
\end{styleStandard}

\begin{styleFooter}
(49)\ \ NEG \ \ → \ \ \textit{k}{}- \ \ / \_\_ (unstressed) \textit{ein}
\end{styleFooter}

\begin{styleFooter}
\ \ \ \ → \ \ \textit{nicht} \ \ (elsewhere)
\end{styleFooter}

\begin{styleStandard}
To sum up this section on \textit{ein}, I reviewed the combinations of operators and articles and accounted for them. It is probably fair to state that this discussion provided a somewhat involved solution, and certain issues remain. Having said that, I hope that it makes clear what complexities arise in the discussion. As already mentioned above, any account adopting a composite analysis of complex \textit{ein}{}-words such as \textit{kein} ‘(NEG+a =) no’ has to come to terms with these issues. Some other conclusions can be drawn.
\end{styleStandard}

\begin{styleStandard}
Given the discussion above, it appears that determiners are not only inserted to flag and support operators. Rather, Art in the syntactic representation involves features, and these features must be spelled out by a determiner to yield a good derivation. All determiners including \textit{ein} have [+D] as part of their vocabulary entries. Consequently, if Art is present in the noun phrase, [+D] is spelled out yielding the fact that all nominals bigger than NumP have a determiner (in fact, since REL and other operators flagged by determiners reside in NumP, it may turn out that all nominals bigger than NP involve determiners). 
\end{styleStandard}

\begin{styleStandard}
As to \textit{ein} itself, this element is usually inserted in singular contexts given the restriction feature [-PL] on [+D]. However, when certain operators are present, the restriction feature is deleted, and \textit{ein} can occur in other number contexts. The side-effect is that \textit{ein} can not only flag and support operators in singular but also in mass and plural environments. This means that \textit{ein} is probably not due to a simple Last-Resort operation – its occurrence (and that of other determiners) follows from the specific features present under Art, depending on the presence of certain operators. 
\end{styleStandard}

\begin{styleStandard}
To sum up the entire Section 2, I addressed some further consequences of the current proposal. Specifically, I briefly discussed how the current analysis could be extended to the distribution of adjectival inflections in Alemannic German. Furthermore, I briefly examined more cases of supporting and flagging. Thus far, I have identified cases that only involve flagging, but there do not seem to be instances that only show supporting. Furthermore, I have found instances where overt and covert operators are supported and flagged by \textit{ein} at the same time. This included cases where two overt operators are supported (and a covert operator is flagged) by \textit{ein}. However, two covert operators do not seem to be flagged by \textit{ein} at the same time. Finally, I provided an account of the insertion of \textit{ein} (and other articles) in the contexts of overt and covert operators. The postulation of a restriction feature and the deletion of it and that of [+D] accounted for the varying distribution of \textit{ein} (and other articles) as regards the presence of different operators.
\end{styleStandard}

\begin{styleStandard}\bfseries
3. \ \ More General Considerations
\end{styleStandard}

\begin{styleStandard}
In this final section, I return to complex nominals in the context of concord in agreement features, and I make suggestions as to what all expletives, clausal and nominal, may have in common.
\end{styleStandard}

\begin{styleStandard}\itshape
3.1.\ \ Nominal Structure and Concord
\end{styleStandard}

\begin{styleStandard}
In the previous chapters, a variety of syntactic phrases played a role. These phrases can be assembled into one hierarchical, abstract structure:
\end{styleStandard}

\begin{styleStandard}
(50)\ \ [ LPP [ D [ Card [ Agr [ Art [ Num [ N ]]]]]]]
\end{styleStandard}

\begin{styleStandard}
I note two things about this structure: I assumed that nouns move to Num and that articles merged in Art (or more precisely, merged as Art) also undergo movement to higher positions in certain cases. With minor differences, this is consistent with the proposals by Julien (2005a), Roehrs (2009a), and others. Importantly, the structure in (50) received further application in this book. I take this to indicate that this structure shows good promise for future investigations. I turn to some issues related to agreement in non-canonical noun phrases.
\end{styleStandard}

\begin{styleStandard}
\ \ As is well known, elements in the noun phrase are subject to concord in agreement features. For example, determiners, adjectives, and nouns in structures like (50) agree in case, number, and gender. In the course of this book, I also discussed a number of more complex cases where a second nominal is combined with the matrix DP by way of an embedded complex specifier or right adjunction. In the literature, these non-canonical nominals have not received much attention. I return to some of them briefly here focusing on agreement. We will see that all these constructions agree in semantic number but not necessarily in morphological number. The relevant structures are illustrated here again but in a simplified way.
\end{styleStandard}

\begin{styleStandard}
\ \ In the context of capacity readings, I discussed DPs that involve \textit{als} ‘as’ and its null counterpart ALS. To recapitulate, the \textit{als-}nominal in (51a) involves adjunction to the pronoun where \textit{als} is the head of ModP, and the role noun \textit{Arzt} ‘doctor’ is an NP complement of that head. I proposed that the adjoined nominal does not contain NumP, and as such the number mismatch with a plural pronoun is only apparent (the plural interpretation follows from the presence of a distributivity operator as postulated by de Swart \textit{et al}. 2007: 218). In contrast, while the ALS{}-nominal in (51b) also contains ModP, this contruction does exhibit concord. I suggested that the ALS{}-nominal must contain NumP and that it is located in the specifier of AgrP:
\end{styleStandard}

\begin{styleStandard}
(51) \ \ a.\ \ \textit{ihr} [\textsubscript{ModP}\textit{ als}\textsubscript{Mod}\textit{ Arzt }]
\end{styleStandard}

\begin{styleStandard}
\ \ \ \ you(\textsc{pl}) \ as \ \ \ \ \ \ doctor.\textsc{masc}
\end{styleStandard}

\begin{styleStandard}
\ \ \ \ ‘you as doctors’
\end{styleStandard}

\begin{styleStandard}
\ \ b.\ \ (\textit{vielleicht}) \textit{ein} [\textsubscript{ModP}\textit{ }ALS\textsubscript{Mod}\textit{ Landwirt }] Agr\textit{ e}\textit{\textsubscript{N}}\textsubscript{\ \ }
\end{styleStandard}

\begin{styleStandard}
\ \ \ \  really \ \ \ \ \ \ \ a \ \ \ \ \ \ \ \ \ \ \ \ \ \ \ \ \ \ \ \ \ \ \ \ \ farmer.\textsc{masc}
\end{styleStandard}

\begin{styleStandard}
\ \ \ \ ‘(really) some farmer’
\end{styleStandard}

\begin{styleStandard}
Two other types of nominals should be discussed in this context.
\end{styleStandard}

\begin{styleStandard}
\ \ Indefinite pronoun constructions also involve complex structures (52a). As in the first two cases, these structures also involve ModP. Like (51a), this ModP was also argued to involve adjunction. However, unlike (51a), this nominal exhibits concord. In view of the presence of an inflected adjective and a null noun, I argued that REL and thus NumP must be in the structure. Finally, I discussed a case of morphological dis-agreement but semantic agreement (52b). Given the figuratively extended meaning of the noun, I concluded that REL and thus NumP must be present. I proposed that the dis-agreeing nominal is embedded in the specifier of DisP:
\end{styleStandard}

\begin{styleStandard}
(52) \ \ a.\ \ \textit{jemand} [\textsubscript{ModP}\textit{ Ø}\textsubscript{Mod}\textit{ ander-er}\textit{\textsubscript{ }}\textit{e}\textit{\textsubscript{N}} ]
\end{styleStandard}

\begin{styleStandard}
\ \ \ \ someone.\textsc{masc} \ \ \ \ \ different-\textsc{st}
\end{styleStandard}

\begin{styleStandard}
\ \ \ \ ‘someone different’
\end{styleStandard}

\begin{styleStandard}
\ \ b. \ \ \ \ \textit{Sie }[\textsubscript{NumP}\textit{ Bauer }] Dis\textit{ e}\textit{\textsubscript{N}}
\end{styleStandard}

\begin{styleStandard}
\ \ \ \ you \ \ \ \ \ \ \ \ peasant.\textsc{masc}
\end{styleStandard}

\begin{styleStandard}
\ \ \ \ ‘you peasant’
\end{styleStandard}

\begin{styleStandard}
I summarize the properties of these different cases in Table 4:
\end{styleStandard}

\begin{styleStandard}
Table 4: Summary of the Constructions and Their Properties
\end{styleStandard}

\begin{flushleft}
\begin{tabular}{|m{1.3712599in}|m{0.8587598in}|m{2.0462599in}|m{1.3587599in}|m{0.6212598in}|}

\hline
Construction &
\centering Structure &
\centering Agreement b/w nominals &
\centering NumP &
\centering\arraybslash Head\\\hline
\textit{als}{}-capacity &
\centering adjunction &
\centering apparent morphological and semantic dis-agreement &
\centering no NumP (but distributivity OP)\textsuperscript{a} &
\centering\arraybslash \textit{als}\textsubscript{Mod}\\\hline
ALS-capacity &
\centering specifier &
\centering morphological and semantic agreement &
\centering NumP &
\centering\arraybslash ALS\textsubscript{Mod}\\\hline
Indefinite pronoun construction  &
\centering adjunction &
\centering morphological and semantic agreement &
\centering NumP &
\centering\arraybslash Ø\textsubscript{Mod}\\\hline
DisP &
\centering specifier &
\centering morphological dis-agreement but semantic agreement &
\centering NumP &
\centering\arraybslash {}-\\\hline
\end{tabular}
\end{flushleft}
\begin{styleStandard}
\textsuperscript{a} The \textit{als}{}-capacity construction involves NumP in the plural cases (e.g., \textit{ihr als Ärzte} ‘you as doctors’).
\end{styleStandard}

\begin{styleStandard}
It is interesting to note that all four constructions agree in semantic number, either mediated through NumP or a distributivity operator. As for concord, agreement in phi-features is not an issue if NumP is absent in the embedding (\textit{als}{}-capacity).\footnote{\ Recall from Chapter 7 that this is similar to clausal structures involving copular verbs and non-plural role nouns (e.g., \textit{Ihr seid alle Arzt.} ‘You are all (doctor =) doctors.’), where NumP is also absent.} Such agreement holds though if NumP is present (ALS-capacity, indefinite pronoun construction). The only true exception is DisP. The latter involves NumP but does not show concord in agreement features. I briefly explore why this might be so.
\end{styleStandard}

\begin{styleStandard}
\ \ It is clear that the position of the embedded nominal \textit{per se} cannot account for these agreement patterns as both adjuncts and specifiers may exhibit concord (indefinite pronoun construction, ALS-capacity) or not (\textit{als}{}-capacity, DisP). Rather, it seems to be a property of the head of the embedded nominal that determines the agreement possibilities. In particular, ALS\textsubscript{Mod} and Ø\textsubscript{Mod} could be taken to mediate concord. I also argued that \textit{als}\textsubscript{Mod} is an operator that may take an NP complement, as in the case in (51a), but it may also embed a NumP complement, as in the case of plural nouns (see superscript a in Table 4). With NumP present in the plural, the \textit{als}{}-capacity construction also shows agreement. All these three constructions involve the head Mod.
\end{styleStandard}

\begin{styleStandard}
Unlike the first three cases, the nominal embedded in Spec,DisP does not contain the head Mod. Given this distinction, I would like to suggest that all three Mod heads mediate concord between the two nominals provided NumP is present.\footnote{\ I briefly discussed in Chapter 2, Section 3.2 that Mod is instantiated by the prepositional element \textit{de} ‘of’ in certain Romance languages. As suggested in Roehrs (2008: 21-23), this element mediates concord between different nominals, for instance, in French indefinite pronoun constructions of the type \textit{quelque chose}\textsubscript{MASC}\textit{ de grand}\textsubscript{MASC} ‘something big’.} Due to the absence of such a head in the nominal inside Spec,DisP, the latter does not have to undergo concord with the matrix nominal. Currently, it is not clear to me how to implement this idea. As such, I have to leave the details of this suggestion for future research.
\end{styleStandard}

\begin{styleStandard}\itshape
3.2.\ \ Expletive Elements more Generally
\end{styleStandard}

\begin{styleStandard}
As mentioned at the beginning, this book is not meant to contribute to the theory of expletive elements \textit{per se}. Rather, it seeks to identify more elements that share some of the properties of generally accepted expletives. \textit{There} in existential constructions and, to a lesser degree, definite articles in the context of proper names seem to be such established elements. This book argued that adjectival inflections and \textit{ein} are also expletive elements. In this final section, I compare these four elements in a bit more detail with the goal of identifying more common properties.
\end{styleStandard}

\begin{styleStandard}
Recall that Chomsky (1995) pointed out that expletive elements seem to violate the Principle of Full Interpretation. He proposed that the associate noun phrase in existential constructions moves at LF to license \textit{there}. In other words, a substantive element moves to license a vacuous one. Longobardi (1994) suggested something similar for proper names and proprial articles. To motivate the movement of the proper name to the article, I follow Chomsky by claiming that all expletives are deficient elements (note that Chomsky takes \textit{there} to be a LF-affix). 
\end{styleStandard}

\begin{styleStandard}
Making this more general, this means that adjectival inflections and \textit{ein} also need to be licensed by a contentful element. For adjectives, I proposed in Chapter 2, Section 2.4 that the adjective stem moves to provide a host for the inflection. As to \textit{ein}, I tentatively suggested in Section 2.2.5 above that operators surface in the left periphery of the noun phrase: either they are base-generated there, or they move there to precede \textit{ein}. I assume that this local relation licenses \textit{ein}. This means that \textit{ein} makes an operator visible but gets licensed by it at the same time. To repeat from Chapter 1, there is a division of labor: syntactically, the expletive indicates the substantive element; semantically, the substantive element identifies the expletive. These relations are shown below before movement of the licenser:
\end{styleStandard}

\begin{styleStandard}
\ \ \ \  \ \ \ \ \ \ \ \ Indicate (syntax)
\end{styleStandard}

\begin{styleStandard}
[Warning: Draw object ignored][Warning: Draw object ignored](53)\ \ EXPL\ \ \ \ \ \ \ \ \ \ SUBST
\end{styleStandard}

\begin{styleStandard}
\ \ \ \  \ \ \ \ \ \ Identify (semantics)
\end{styleStandard}

\begin{styleStandard}
There is a second point worth making.
\end{styleStandard}

\begin{styleStandard}
Recall also that one of the conclusions of this book is that adjectival inflections and \textit{ein} make different abstract structures visible. Again, I suggest that this is a general property of all expletives. In other words, I would like to claim that both \textit{there} and the proprial article also indicate abstract structure. Consider this point in more detail.
\end{styleStandard}

\begin{styleStandard}
Above, I made a distinction between supporting and flagging. Starting with the latter, this mechanism seems to exhibit certain similarities to what is involved with \textit{there} and its associate and with proprial articles and proper names. Specifically, in all these cases, the expletive element is a free, unbound morpheme (at least in the overt component of the derivation). I observe now that the expletives and the contentful elements involve two different positions where the expletives are higher in the structure than the substantive elements:
\end{styleStandard}

\begin{styleStandard}
[Warning: Draw object ignored][Warning: Draw object ignored](54)\ \ \ \  \ \ \ \ \ \ \ \ \ \ $\alpha $
\end{styleStandard}

\begin{styleStandard}
\ \ EXPL\ \ \ \ \ \  \ \ \ \ \ $\beta $
\end{styleStandard}

\begin{styleStandard}
[Warning: Draw object ignored][Warning: Draw object ignored][Warning: Draw object ignored]
\end{styleStandard}

\begin{styleStandard}
[Warning: Draw object ignored]\ SUBST\ \ 
\end{styleStandard}

\begin{styleStandard}
\ \ \textit{there}\ \ \ \ associate
\end{styleStandard}

\begin{styleStandard}
\ \ prop. art.\ \ proper name
\end{styleStandard}

\begin{styleStandard}
\ \ \textit{ein}\ \ \ \ operator
\end{styleStandard}

\begin{styleStandard}
The contentful licensers are merged lower in the structure due to semantico-syntactic factors; that is, they are base-generated low for independent reasons. Specifically, associates are at the bottom of existential constructions given local theta-role assignment (55a), nouns including proper names form the bottom of their extended projections (55b), and semantic operators like REL turn nouns into predicates before these nominals can combine with other elements (55c). Notice that the expletives and their licensers are not linearly adjacent in their positions as other elements can intervene: 
\end{styleStandard}

\begin{styleStandard}
(55)\ \ a.\ \ [ There ] is [ a man ] in the garden.
\end{styleStandard}

\begin{styleStandard}
\ \ b.\ \ [ \textit{der} ] \textit{alte} [ \textit{Peter} ]
\end{styleStandard}

\begin{styleStandard}
\ \ \ \  \ the \ \ \ old \ \ \ Peter.\textsc{masc}
\end{styleStandard}

\begin{styleStandard}
\ \ \ \ ‘Peter, who is old’
\end{styleStandard}

\begin{styleStandard}
\ \ c.\ \ [ \textit{ein} ] \textit{netter} [ REL ] \textit{Freund}
\end{styleStandard}

\begin{styleStandard}
\ \ \ \  \ a \ \ \ \ \ nice \ \ \ \ \ \ \ \ \ \ \ \ \ \ \ \ friend.\textsc{masc}
\end{styleStandard}

\begin{styleStandard}
\ \ \ \ ‘a nice friend’
\end{styleStandard}

\begin{styleStandard}
Given these data, I reiterate the claim that these expletives and their licensers are related by movement. Recall that expletives are deficient elements and that the substantive elements are in a lower position for independent reasons. Now, as syntactic movement only proceeds upwards, we have an explanation as to why the expletives must be in the higher position (and not the lower one) – they are licensed by the movement of the substantive elements. Being in a high position, expletives indicate abstract structure. This includes \textit{there} and proprial articles.
\end{styleStandard}

\begin{styleStandard}
\ \ The mechanism of supporting is partially different. It, too, makes operators visible but forms overt composites of the two relevant elements under linear adjacency. This involves the complex \textit{ein}{}-words and inflected adjectives. Like flagging, the substantive element is base-generated low in the structure and undergoes movement to the left periphery. This is clearly the case with the possessive component. After movement, it combines with expletive \textit{ein} (56a). Similary, the adjective stem is base-generated below its inflection and moves to combine with it (56b):
\end{styleStandard}

\begin{styleStandard}
(56)\ \ a.\ \ [\textsubscript{DP} \textit{m}\textsubscript{i}{}-\textit{ein} \ \textit{Auto} t\textsubscript{i} ]
\end{styleStandard}

\begin{styleStandard}
\ \ \ \  \ \ \ \ \ \textsc{poss}{}-a car.\textsc{neut}
\end{styleStandard}

\begin{styleStandard}
\ \ \ \ ‘my car’
\end{styleStandard}

\begin{styleStandard}
\ \ b. \ \ \ \textit{das} [\textsubscript{InflP} \textit{groß}\textsubscript{i}{}-\textit{e} t\textsubscript{i} ] \textit{Auto}
\end{styleStandard}

\begin{styleStandard}
\ \ \ \ the \ \ \ \ \ \ \ \ big-\textsc{infl} \ \ \ car.\textsc{neut}
\end{styleStandard}

\begin{styleStandard}
\ \ \ \ ‘the big car’
\end{styleStandard}

\begin{styleStandard}
In each case, the substantive part originates low in the structure, and the expletive element is located higher in the structure.
\end{styleStandard}

\begin{styleStandard}
\ \ This leaves the three overt operators that are part of the negative article \textit{kein} ‘no’, the singularity numeral \textit{EIN} ‘one’, and \textit{so’n} ‘such a’ to be discussed. Unlike the cases above, these three operators are base-generated higher in the structure than \textit{ein}: NEG is outside the DP proper, \textit{Ø}\textsubscript{[-PL]} is in Spec,CardP, and \textit{so} is in Spec,TypeP. Recall from Section 2.2.2 though that I have, thus far, not been able to identity cases that only involve supporting. In other words, all cases involve flagging and may, additionally, also show supporting. If this turns out to be correct, then we can claim that flagging is the primary mechanism to indicate operators, and supporting is a secondary, additional mechanism.\footnote{\ This could mean that flagging is a universal process but supporting is not.} Consequently, I take flagging, as discussed above, to manifest the typical properties of the relationship between expletives and their licensers. To be clear, the cases involving supporting do not seem to be telling as regards the relation between expletives and licensers.
\end{styleStandard}

\begin{styleStandard}
\ \ With this in mind, I summarize the discussion of all expletives. I start by providing a list of properties of \textit{there}, an established expletive discussed in the introduction of this book. Besides the properties mentioned there, other features have been identified. Lasnik \& Uriagereka with Boeckx (2005: 134, 150-51) point out the following traits in their overview discussion of the relevant construction:\footnote{\ There are other properties of \textit{there} that are not relevant to the discussion of adjectival inflections and \textit{ein} (nor are they relevant to proprial articles); for instance, a third element – the verb – does not agree with \textit{there}, but with the associate.} (i) \textit{there} does not have a meaning and has to be licensed, (ii) \textit{there} and the associate are in two different positions, (iii) \textit{there} is in the highest site and the associate is in the lowest site, (iv) \textit{there} stands in a certain formal relation with the associate such that both sites are related by movement, and (v) the higher position is made visible, either by \textit{there} or the associate indicating the abstract subject position.
\end{styleStandard}

\begin{styleStandard}
\ \ Comparing these properties to those of adjectival inflections and \textit{ein}, we notice that they are basically identical. I would like to point out though that the last point, indicating abstract structure, is often mentioned but does not seem to have attracted much attention (unlike Hypothesis 1b in the current discussion). It is possible that this property can ultimately be related to other phenomena.\footnote{\ One possible avenue of research might be to relate the indication of abstract structure to language acquisition.} Be that as it may, there are some further characteristics that have been pointed out for established expletives: (vi) unlike the current cases, the associate moves at LF to license the expletive, (vii) unlike the current cases, the expletive can be null in some languages (e.g., Italian), and (viii) unlike the current cases, there is an alternative construction whereby the associate occupies the higher position and the expletive does not appear at all. 
\end{styleStandard}

\begin{styleStandard}
Of particular interest here is the last property in (viii). As far as I am aware, adjectival inflections or \textit{ein} are never in complementary distribution with their licensers as regards the higher position. In my view, this is a difference that might indicate that there are two types of expletives after all: \textit{there} and proprial articles vs. adjectival inflections and \textit{ein}. Having said that, it might turn out that independent factors such as the requirement to spell out CNG features or to overtly indicate operators could explain the obligatory presence of the latter two elements. I leave a more detailed exploration of these ideas for future research.
\end{styleStandard}

\begin{styleStandard}
Finally, the main empirical goal of this book was to provide a more comprehensive discussion of adjectival inflections and \textit{ein}. I believe this pursuit has lead to some interesting theoretical proposals. At this point, I cannot claim to have resolved all issues surrounding adjectival inflections and \textit{ein}. However, I hope to have made a good argument for the main hypothesis of this book – these two elements are semantically vacuous in German. Undoubtedly, further investigations and the detailed discussion of other languages will yield new insights into these empirical domains.
\end{styleStandard}

\clearpage\setcounter{page}{361}\begin{styleStandard}
Appendix: Examples of Plural \textit{ein}
\end{styleStandard}

\begin{styleStandard}
The examples in Sections 1-3 are presented first from google, then from Twitter, and finally from Facebook. Within each of these three search domains, the examples are provided in the following order: \textit{was für (ei)ne}, \textit{was für} \textit{(ei)ne} with adjective, \textit{so (ei)ne}, and \textit{so} \textit{(ei)ne} with adjective. Similarly, the examples in Section 4 are presented according to the alphabetical order of the different corpora. Within the corpora, the examples involving \textit{was für} are given before those involving \textit{so (ei)ne}. Again, instances without adjectives precede those with adjectives.
\end{styleStandard}

\begin{styleStandard}
\textit{1.1.\ \ Googled: }was für (ei)ne \textit{without adjective}
\end{styleStandard}

\begin{styleStandard}
{\textquotedbl}\emph{\textbf{\textup{Was für eine Idioten}}}{\textquotedbl} sagte Liam
\end{styleStandard}

\begin{styleStandard}
{\textquotedbl}\emph{\textbf{\textup{Was für eine Arschlöcher}}}.{\textquotedbl} waren meine letzten worte und somit verschwand ich auch.~
\end{styleStandard}

\begin{styleStandard}
\textbf{WAS FÜR NE VÖGEL} ALTER SCHWEDE !!!!
\end{styleStandard}

\begin{styleStandard}
\textbf{Was für ne Vögel}...aber Glück für dich eben.~  [Warning: Image ignored] % Unhandled or unsupported graphics:
%\includegraphics[width=0.1563in,height=0.1563in,width=\textwidth]{roehrs-img001.png}
 
\end{styleStandard}

\begin{styleStandard}
\textbf{Was für ne Vögel} habts denn da aufgetrieben?
\end{styleStandard}

\begin{styleStandard}
\textbf{Was für'ne Idioten}...!!!
\end{styleStandard}

\begin{styleStandard}
Oh gott, \textbf{was für ne idioten}, es gibt klar ne menge websiten,
\end{styleStandard}

\begin{styleStandard}
So nun rief er beim Freundlichen an,~\emph{\textbf{\textup{was für ne Idioten}}}~sie sind, den Ölkühler zu wechseln und kein Öl nachzufüllen,~
\end{styleStandard}

\begin{styleStandard}
wenn du da bist dann würde ich denen (besser nach dem kauf der siebchen) sagen \textbf{was für ne idioten} das sind
\end{styleStandard}

\begin{styleStandard}
Da fragte ich mich auch, \textbf{was für ne Idioten}, vorallem hatte ich immer Luft in der Anlage.
\end{styleStandard}

\begin{styleStandard}
Also beim MM in Lübeck gabs gestern noch genug laptops von Samsung... liegt wohl auch daran das die das ding falsch ausgeschildert haben, denn dort steht 1,8 Ghz auf dem Schild~  [Warning: Image ignored] % Unhandled or unsupported graphics:
%\includegraphics[width=0.0173in,height=0.0173in,width=\textwidth]{roehrs-img002.png}
 ~vom Despina. Und das Devin verkaufen sie zum gleichen Preis.... \textbf{was für ne idioten}
\end{styleStandard}

\begin{styleStandard}
Gute Idee aber du beschimpfst jeden und flames \textbf{was für ne Idioten} sie sind da machts kein Spaß dir zuzuckuken
\end{styleStandard}

\begin{styleStandard}
die wollten das wir die Linke oder die NPD wählen,aber sie sagten die Linke währe für sie besser…lol \textbf{was für ne Idioten}…
\end{styleStandard}

\begin{styleStandard}
\textbf{Was für ne Spinner} bei Lyon, 
\end{styleStandard}

\begin{styleStandard}
\textbf{Was für ne Spinner} gibts hier eigentlich?
\end{styleStandard}

\begin{styleStandard}
Die zwei hab' ich auch gesehen. \textbf{Was für ne Spinner}. Der Typ der das mamor im Auto hatte
\end{styleStandard}

\section[...ach gott wie lustig, was für ne Typen es hier so gibt .... :{}-) o.T.]{...ach gott wie lustig, \textbf{was für ne Typen} es hier so gibt .... :-) o.T.}
\section{}
\begin{styleStandard}
Dass muss man sich mal überlegen, \textbf{was für ne Schweine} dass sind.
\end{styleStandard}

\begin{styleStandard}
hm ...\textbf{was für ne zicken}... !!!~  [Warning: Image ignored] % Unhandled or unsupported graphics:
%\includegraphics[width=0.1563in,height=0.1563in,width=\textwidth]{roehrs-img003.png}
 
\end{styleStandard}

\begin{styleStandard}
\textit{1.2.\ \ Googled: }was für (ei)ne \textit{with adjective}
\end{styleStandard}

\begin{styleStandard}
Auf \textbf{was für ne‘ coolen Ideen} kommt ihr denn noch! LG
\end{styleStandard}

\begin{styleStandard}
NFU und NFSU2 (\textbf{was für ne dummen Abkürzungen})~
\end{styleStandard}

\begin{styleStandard}
Ich empfand dieses ignorante Verhalten als sehr unverschämt und beleidigend auf meine Person bezogen und bin gegangen!!!~\emph{\textbf{\textup{Was für eine dummen}}}\textbf{~Hühner} dort ;).~
\end{styleStandard}

\begin{styleStandard}
\emph{\textbf{\textup{Was für eine hübschen}}}\textbf{~Jungs} kennst du? Namen?~
\end{styleStandard}

\begin{styleStandard}
Woow \textbf{was für eine schönen Beine} hast du wirklich Frau \textsf{SMILEY}
\end{styleStandard}

\begin{styleStandard}
\emph{\textbf{\textup{Was für eine geilen}}}\textbf{~Bilder} wer hat die bloß gemacht
\end{styleStandard}

\begin{styleStandard}
\textbf{Was für ne geilen Threads} ey!
\end{styleStandard}

\begin{styleStandard}
\textbf{was} hast du immer~\emph{\textbf{\textup{für ne krassen}}}\textbf{~anfälle} im voice :d
\end{styleStandard}

\begin{styleStandard}
\textbf{Was} mach ich heut~\emph{\textbf{\textup{für ne krassen}}}\textbf{~Fehler} LOL
\end{styleStandard}

\begin{styleStandard}
\textbf{was} meine Unachtsamkeit~\emph{\textbf{\textup{für ne krassen}}}\textbf{~Folgen} hat
\end{styleStandard}

\begin{styleStandard}
Wow heftig dass ist voll komisch zu sehen \textbf{was~}\emph{\textbf{\textup{für ne krassen}}}\textbf{~Menschen} albanische Wurzeln haben
\end{styleStandard}

\begin{styleStandard}
\textbf{Was} denn~\emph{\textbf{\textup{für ne verdammten}}}\textbf{~Ringe}?
\end{styleStandard}

\begin{styleStandard}
\textit{1.3.\ \ Googled: }so (ei)ne \textit{without adjective}
\end{styleStandard}

\begin{styleStandard}
\emph{\textbf{\textup{So eine Ferien}}}~könnten darauf deuten, dass sie wünscht, sich selbst zu erholen.~
\end{styleStandard}

\begin{styleStandard}
Das Personal war sehr unfreundlich. Ich hatte noch nie~\emph{\textbf{\textup{so eine Ferien}}}.~
\end{styleStandard}

\begin{styleStandard}
Nich schön,~\emph{\textbf{\textup{so ne Ferien}}}.~
\end{styleStandard}

\begin{styleStandard}
ich hab mal~\emph{\textbf{\textup{so'ne ferien}}}~in den alpen mit ner jugendgruppe gemacht~
\end{styleStandard}

\begin{styleStandard}
\textit{1.4.\ \ Googled: }so (ei)ne \textit{with adjective}
\end{styleStandard}

\begin{styleStandard}
und denn kommen dir~\emph{\textbf{\textup{so ne geilen}}}\textbf{~Typen} entgegnen gestolpert
\end{styleStandard}

\begin{styleStandard}
ihr macht~\emph{\textbf{\textup{so ne geilen}}}\textbf{~lieder}~
\end{styleStandard}

\begin{styleStandard}
~Hahahahah~\emph{\textbf{\textup{so ne geilen}}}\textbf{~vids} hahahah.
\end{styleStandard}

\begin{styleStandard}
\emph{\textbf{\textup{So ne geilen}}}\textbf{~Leute} echt jeder auf seiner Art so sympathisch!!~
\end{styleStandard}

\begin{styleStandard}
\textbf{drei~}\emph{\textbf{\textup{so ne geilen}}}\textbf{~Mädels}
\end{styleStandard}

\begin{styleStandard}
Wie kommt man auf~\emph{\textbf{\textup{so ne geilen}}}\textbf{~Ideen}?~
\end{styleStandard}

\begin{styleStandard}
Hi,wie kriegen die~\emph{\textbf{\textup{so ne geilen}}}\textbf{~Lichteffekte} hin?~
\end{styleStandard}

\begin{styleStandard}
Dass es hier auf dieser Seite~\emph{\textbf{\textup{so ne geilen}}}\textbf{~Schlampen} wie mich gibt?
\end{styleStandard}

\begin{styleStandard}
Ich hoffe die TranX macht dann auch~\emph{\textbf{\textup{so ne geilen}}}\textbf{~Geräusche}.
\end{styleStandard}

\begin{styleStandard}
warum die Phiole Rechts im Bild nicht~\emph{\textbf{\textup{so ne geilen}}}\textbf{~Reflection Effekte} hat wie die in der Mitte
\end{styleStandard}

\begin{styleStandard}
Herzlichen Glückwunsch und macht weiter~\emph{\textbf{\textup{so ne geilen}}}\textbf{~Videos}!~
\end{styleStandard}

\begin{styleStandard}
Neue Sounddesigner und dann~\emph{\textbf{\textup{so'ne geilen}}}\textbf{~Kisten}?~
\end{styleStandard}

\begin{styleStandard}
ich finde ihr solltet~\emph{\textbf{\textup{so ne geilen}}}\textbf{~partys} mal am samstag machen~
\end{styleStandard}

\begin{styleStandard}
Und nur~\emph{\textbf{\textup{so ne geilen}}}\textbf{~Antworten}
\end{styleStandard}

\begin{styleStandard}
textlich hat der~\emph{\textbf{\textup{so ne geilen}}}\textbf{~dinger} dabei das ich manchmal mit grinsendem gesicht dasitze und denke {\textquotedbl}geil{\textquotedbl}.
\end{styleStandard}

\begin{styleStandard}
wieso gibts~\emph{\textbf{\textup{so'ne geilen}}}\textbf{~typen} nur in japan
\end{styleStandard}

\begin{styleStandard}
Mushido hat immer~\emph{\textbf{\textup{so ne geilen}}}\textbf{~Beats} :D
\end{styleStandard}

\begin{styleStandard}
\emph{\textbf{\textup{So ne geilen}}}\textbf{~Geländewagen} find ich richig geil
\end{styleStandard}

\begin{styleStandard}
es gibt ja im BC bei den ganzen objekten~\emph{\textbf{\textup{so ne geilen}}}\textbf{~schilder} wie {\textquotedbl}berlin Gasthof{\textquotedbl}
\end{styleStandard}

\begin{styleStandard}
die hat~\emph{\textbf{\textup{so ne geilen}}}\textbf{~Beine}
\end{styleStandard}

\begin{styleStandard}
wenn da~\emph{\textbf{\textup{so ne geilen}}}\textbf{~Sachen} Rauskommen~
\end{styleStandard}

\begin{styleStandard}
woher kriegsten du~\emph{\textbf{\textup{so ne geilen}}}\textbf{~witze}?
\end{styleStandard}

\begin{styleStandard}
jetzt nutze ich~\emph{\textbf{\textup{so ne geilen}}}\textbf{~dinger} für den PC~
\end{styleStandard}

\begin{styleStandard}
Man würd'~\emph{\textbf{\textup{so 'ne süßen}}}\textbf{~Sachen} natürlich niemals tun.~
\end{styleStandard}

\begin{styleStandard}
Och~\emph{\textbf{\textup{so ne süssen}}}\textbf{~Katzis}... :-)
\end{styleStandard}

\begin{styleStandard}
\emph{\textbf{\textup{So'ne süßen}}}\textbf{~Dreckspätze} und selbst bei so einem. Mistwetter macht Doris noch ne tolle Geschichte draus.~
\end{styleStandard}

\begin{styleStandard}
Unsere Sabine macht immer~\emph{\textbf{\textup{so ne süßen}}}\textbf{~Geschenke}.
\end{styleStandard}

\begin{styleStandard}
Ihr seid doch alles~\emph{\textbf{\textup{so ne süßen}}}\textbf{~Krümels}
\end{styleStandard}

\begin{styleStandard}
das sind~\emph{\textbf{\textup{so ne süßen}}}\textbf{~hundewelpen}
\end{styleStandard}

\begin{styleStandard}
\emph{\textbf{\textup{So ne süßen}}}\textbf{~kleinen Pfoten}
\end{styleStandard}

\begin{styleStandard}
Ja, ja,~\emph{\textbf{\textup{so'ne süßen}}}\textbf{~kleinen Löckenwölfe} sind schon entzückend!
\end{styleStandard}

\begin{styleStandard}
mit ihm~\emph{\textbf{\textup{so 'ne süßen}}}\textbf{~Momente} zu haben
\end{styleStandard}

\begin{styleStandard}
als ich schwanger war (was ja auch noch nicht sooo lang her ist) gabs~\emph{\textbf{\textup{so ne süßen}}}\textbf{~teile} leider nicht
\end{styleStandard}

\begin{styleStandard}
\textit{2.1.\ \ Twitter: }was für (ei)ne \textit{without adjective}
\end{styleStandard}

\begin{styleStandard}
\textbf{Wasfür eine Idioten} leben hier? Wahnsinn!
\end{styleStandard}

\begin{styleStandard}
Was glaubst Du denn, \textbf{was }da\textbf{ für eine Leute} wohnen?
\end{styleStandard}

\begin{styleStandard}
\textbf{was} das \textbf{für eine Leute} sind, mit denen sich die Polizei auseinandersetzen muss.
\end{styleStandard}

\begin{styleStandard}
\textbf{Was} sind das denn jetzt \textbf{für ne Leute}, die
\end{styleStandard}

\begin{styleStandard}
\textbf{Was} waren denn das \textbf{für ne Leute} eben auf Periscope. Schlimm.
\end{styleStandard}

\begin{styleStandard}
man man man \textbf{was für ne idioten} es dich gibt
\end{styleStandard}

\begin{styleStandard}
\textbf{Was für ne Idioten} und danke
\end{styleStandard}

\begin{styleStandard}
\textbf{Was für ne idioten} seit ihr eigentlich habt ihr noch alle tassen im schrank
\end{styleStandard}

\begin{styleStandard}
\textbf{Was für eine Idioten} wohnen nur in Sachsen-Anhalt? Ich explodiere.
\end{styleStandard}

\begin{styleStandard}
\textbf{was für eine Idioten}! Die Bude aber noch an den größeren Deppen Fratzenbuch für 16Milliarden zu verkaufen,ist Realsatire
\end{styleStandard}

\begin{styleStandard}
Gerade gesehen... \textbf{was für eine Idioten}, die so Aufmerksamkeit wollen
\end{styleStandard}

\begin{styleStandard}
\textbf{Was für eine Idioten} , wir können in Deutschland gar keine Schraube drehen.
\end{styleStandard}

\begin{styleStandard}
\textbf{Was} sind denn das \textbf{für ne vögel} am alex? Oo [Berlin]
\end{styleStandard}

\begin{styleStandard}\bfseries
wat für ne spinner
\end{styleStandard}

\begin{styleStandard}\bfseries
was für ne spinner
\end{styleStandard}

\begin{styleStandard}
Lateline man \textbf{was für eine spinner} von Anrufer
\end{styleStandard}

\begin{styleStandard}
\textbf{Was für eine spinner} er hätte sich lieber ne andere Sportart suchen sollen
\end{styleStandard}

\begin{styleStandard}
DPA, \textbf{was für eine Arschlöcher} seid ihr denn?
\end{styleStandard}

\begin{styleStandard}
\textbf{Was} sind das \textbf{für ne Arschlöcher}, die IMMER NOCH megalaute Böller zünden?
\end{styleStandard}

\begin{styleStandard}\bfseries
Wat für ne Kunden
\end{styleStandard}

\begin{styleStandard}
Die sind ja picky. \textbf{Was für ne Säcke}. :D
\end{styleStandard}

\begin{styleStandard}
\textbf{Was für ne Säcke} sind das denn bitte? WTF? D:
\end{styleStandard}

\begin{styleStandard}
viel spass mit deiner Familie, yaaa. \textbf{was fuer eine Ferien}! Hier habe ich auch schoene Ferien. :D
\end{styleStandard}

\begin{styleStandard}
Ferien???????????????????????? \textbf{Was für ne Ferien}????????????????????????
\end{styleStandard}

\begin{styleStandard}
\textbf{was für ne ferien} habt ihr!? :o
\end{styleStandard}

\begin{styleStandard}
\textbf{was für ne ferien} ?
\end{styleStandard}

\begin{styleStandard}
\textbf{wat fuer ne spasten}. das ergibt doch keinen sinn xd
\end{styleStandard}

\begin{styleStandard}
echt jetzt?? \textbf{Was für ne spasten} sind das bitte -.- arghhh
\end{styleStandard}

\begin{styleStandard}
\textit{2.2.\ \ Twitter: }was für (ei)ne \textit{with adjective}
\end{styleStandard}

\begin{styleStandard}
Ich wurde auf Twitch gebannt weil ich geschrieben hab: Schöne achseln diggah!? \textbf{was für eine dummen mods}?!
\end{styleStandard}

\begin{styleStandard}
Community schrumpft 60\%...lol...\textbf{was für ne verdammten idioten}...
\end{styleStandard}

\begin{styleStandard}\bfseries
Wat für ne furchtbaren Trikots
\end{styleStandard}

\begin{styleStandard}
\textbf{Was für eine bescheuerten Haufen} sitzen in unsere Ministerien. 
\end{styleStandard}

\begin{styleStandard}
\textit{2.3.\ \ Twitter: }so (ei)ne \textit{without adjective (but with numeral)}
\end{styleStandard}

\begin{styleStandard}
ach man wir sind schon \textbf{so ne zwei wärmflaschen}
\end{styleStandard}

\begin{styleStandard}
und \textbf{so ne zwei Kämpfer} die alles umpflügen,
\end{styleStandard}

\begin{styleStandard}
\textbf{So 'ne zwei Bubis}... Denken auch ich halt mein maul ...
\end{styleStandard}

\begin{styleStandard}
\textit{2.4.\ \ Twitter: }so (ei)ne \textit{with adjective}
\end{styleStandard}

\begin{styleStandard}
Selten \textbf{so ne geilen Livebands} gehört.
\end{styleStandard}

\begin{styleStandard}
{\textquotedbl}wir sind \textbf{so ne geilen Socken}{\textquotedbl}
\end{styleStandard}

\begin{styleStandard}
Wir haben hier \textbf{so 'ne geilen Lehrer-Praktikanten}.
\end{styleStandard}

\begin{styleStandard}
\textbf{So ne geilen Bilder}...
\end{styleStandard}

\begin{styleStandard}
Ich bins als Dorfkind ja nicht gewohnt \textbf{so ne geilen läden} zu habn
\end{styleStandard}

\begin{styleStandard}
aber als ich in meiner emo-phase war hab ich mir immer \textbf{so'ne geilen stories} ausgedacht und muss heute noch dran denken 8)
\end{styleStandard}

\begin{styleStandard}
Ich will \textbf{so ne geilen smileys} BEI DER TASTATUR DABEI
\end{styleStandard}

\begin{styleStandard}
und dann noch \textbf{so'ne geilen schuhe}!!
\end{styleStandard}

\begin{styleStandard}
wie kommt man sonst auf \textbf{so'ne geilen Ideen}?!
\end{styleStandard}

\begin{styleStandard}
Aber wie können die \textbf{so ne geilen Tänzer} rauswerfen???
\end{styleStandard}

\begin{styleStandard}
woa cheng \textbf{so ne geilen weißen schuhe} hab ich ja noch nie gesehen xd das strahlt richtig
\end{styleStandard}

\begin{styleStandard}
Kann ja nicht jeder \textbf{so ne geilen Videos} machen wie du :D
\end{styleStandard}

\begin{styleStandard}
Aber du hast doch auch ohne den Anzug ein Sixpack und \textbf{so'ne geilen Muskeln}!
\end{styleStandard}

\begin{styleStandard}
wo bekommt man den \textbf{so ne geilen Lollys}?
\end{styleStandard}

\begin{styleStandard}
Doch, der Tag war okay. Nur hatte ich bisher nicht \textbf{so ne geilen Nachrichten} gehört
\end{styleStandard}

\begin{styleStandard}
Das kommt jetzt mega weird..aber so'n Stripper als Freund......also weil die machen immer \textbf{so ne geilen Bewegungen},weißt du...
\end{styleStandard}

\begin{styleStandard}
Du machst immer \textbf{so ne geilen Verlosungen} !
\end{styleStandard}

\begin{styleStandard}
\textbf{so'ne geilen Tage} muss es auch mal geben
\end{styleStandard}

\begin{styleStandard}
uf \textbf{so ne geilen ideen} kommst auch nur du!
\end{styleStandard}

\begin{styleStandard}
ey ich schwör Dir Freitag und Montag werden \textbf{so 'ne geilen Tage} !! 
\end{styleStandard}

\begin{styleStandard}
und immer \textbf{so ne geilen sachen} ;)
\end{styleStandard}

\begin{styleStandard}
hör auf hier \textbf{so'ne geilen Bands} aufzuzählen,sonst muss ich leider kurz ausflippen,verdammte Axxxt.
\end{styleStandard}

\begin{styleStandard}
Und dann auch immer \textbf{so ne geilen Argumente} bringen, wie {\textquotedbl}FRÜHER war das auch nicht so{\textquotedbl}
\end{styleStandard}

\begin{styleStandard}
Alter \textbf{irgend so ne dummen Hunde} haben direkt neben mir n Böller gezündet
\end{styleStandard}

\begin{styleStandard}
\textbf{so ne dummen aufgaben} kann sich auch nur klaas ausdenken
\end{styleStandard}

\begin{styleStandard}
als ob brendon jetzt ernsthaft VON DER QUEERCOMMUNITY \textbf{so'ne dummen kommentare} dafür abbekommt dass er sich als pan geouted hat
\end{styleStandard}

\begin{styleStandard}
komm jetzt noch mehr \textbf{so ne dummen machosprüche} von dir?
\end{styleStandard}

\begin{styleStandard}
{\textquotedbl}Selbst schuld, wenn du immer \textbf{so'ne dummen Kommentare} gibst.{\textquotedbl}
\end{styleStandard}

\begin{styleStandard}
Ich war eben kurz davor \textbf{so 'ne dummen Kinder} zu fragen,
\end{styleStandard}

\begin{styleStandard}
Warum passieren mir immer \textbf{so ne dummen Sachen}?!
\end{styleStandard}

\begin{styleStandard}
Klar, streut noch \textbf{so 'ne dummen Gerüchte}. :O
\end{styleStandard}

\begin{styleStandard}
Da waren heute \textbf{2 so'ne dummen Weiber} beim Zumba-Kurs, die schrien ständig wie \textbf{so'ne Fangirls }[NB: this token was not counted at all, DR]. -\_\_\_\_- Das hat genervt!
\end{styleStandard}

\begin{styleStandard}
Gestern so um 23 Uhr waren \textbf{so'ne dummen Jugendlichen} vor meinem Febster und haben laut Musik gespielt + laut geredet.
\end{styleStandard}

\begin{styleStandard}
Sollte ich öfter machen, dann krieg ich auch nicht \textbf{so 'ne dummen Nachrichten} von Mitschülern.
\end{styleStandard}

\begin{styleStandard}
ich hab ja früher \textbf{so ne dummen Digitalzeichnungen} gemacht
\end{styleStandard}

\begin{styleStandard}
komm jetzt nicht noch auf \textbf{so'ne dummen Ideen}... ;o)
\end{styleStandard}

\begin{styleStandard}
Boa. \textbf{So ne dummen Eltern} hätte ich auch gern.
\end{styleStandard}

\begin{styleStandard}
ich schon \textbf{so ne dummen Vögel} von der CDU können mir gestohlen bleiben
\end{styleStandard}

\begin{styleStandard}
Ich wer dir \textbf{so ne dummen Sprüche} noch austreiben !
\end{styleStandard}

\begin{styleStandard}
\textbf{so'ne dummen Menschen}!
\end{styleStandard}

\begin{styleStandard}
Wieso gibt es \textbf{so ne dummen, bescheuerten Leute} die sowas machen?? D:
\end{styleStandard}

\begin{styleStandard}
hahaha und \textbf{so ne dummen leute} am tisch ehy
\end{styleStandard}

\begin{styleStandard}
es gibt halt \textbf{so ne dummen und abgehobenen leute}
\end{styleStandard}

\begin{styleStandard}
ach \textbf{so 'ne dummen} gibt es öfter. xD
\end{styleStandard}

\begin{styleStandard}
\textit{3.1.\ \ Facebook: }so (ei)ne \textit{with adjective}
\end{styleStandard}

\begin{styleStandard}
\textbf{so ne dummen Schweine} !!! wie kann man so bloß mit den Tieren umgehen !!!! HASS !!!!!!!!!
\end{styleStandard}

\begin{styleStandard}
Hey Schnitzel, hoffe du bist bald wieder wohl auf. \textbf{So ne dummen idioten} ey.....
\end{styleStandard}

\begin{styleStandard}
Wie kann man nur \textbf{so ne dummen Idioten} wählen.....
\end{styleStandard}

\begin{styleStandard}
Bitte blockiert endlich \textbf{so ne blöden Tussen}!!! Wir sind hier nicht beim Werbe-Flohmarkt! [Hamburg]
\end{styleStandard}

\begin{styleStandard}
Ich treffe täglich auf \textbf{so ne ekelhaften Leute} ohne Achtung...
\end{styleStandard}

\begin{styleStandard}\itshape
4. \ \ Datenbank für Gesprochenes Deutsch
\end{styleStandard}

\begin{styleStandard}\itshape
Australiendeutsch
\end{styleStandard}

\begin{styleStandard}
\textstylew{und}~\textstylew{deshalb}~\textstylew{kann}~\textstylew{ich}~\textstylew{Ihnen}~\textstylew{nich}~\textstylew{sagen},~\textstylew{was}~\textstylew{\textbf{was}}\textbf{~}\textstylew{\textbf{für}}\textbf{~}\textstylew{\textbf{eine}}\textbf{~}\textstylew{\textbf{Geschäfte}}~\textstylew{sie}~\textstylew{hier}~\textstylew{gehabt}~\textstylew{haben}.\textbf{ }
\end{styleStandard}

\begin{styleStandard}
(pseudonym: Bastian; born in Timkenberg, Mecklenburg-Vorpommern)
\end{styleStandard}

\begin{styleStandard}
\textbf{Was für 'ne}~\textbf{Fächer} ?~
\end{styleStandard}

\begin{styleStandard}
(born: no information, but probably an interviewer from Germany)
\end{styleStandard}

\begin{styleStandard}\bfseries
sone~dumme Bengels 
\end{styleStandard}

\begin{styleStandard}
(pseudonym: Patrick; born in Tanunda, Australia; ancestors from Silesia)
\end{styleStandard}

\begin{styleStandard}\itshape
Berliner Wendekorpus 
\end{styleStandard}

\begin{styleStandard}
\textstylew{\textbf{was}}~\textstylew{hast-e}~\textstylew{dir~}\textstylew{\textbf{für-ne}}~\textstylew{\textbf{gedanken}}~\textstylew{gemacht}~\textstylew{wie-s}~\textstylew{weitergeht}? 
\end{styleStandard}

\begin{styleStandard}
\ \ (no information about speaker available)
\end{styleStandard}

\begin{styleStandard}
\textstylew{und}~\textstylenv{*}\textstylew{~\textbf{so-ne}}~\textstylew{\textbf{sachen}}~\textstylew{ja\^{}}~\textstylew{ooch}~\textstylew{n}~\textstylew{bißel}~
\end{styleStandard}

\begin{styleStandard}
\textstylew{\ \ (}pseudonym: Dirk; lives in~Friedrichshain, Berlin\textstylew{)}
\end{styleStandard}

\begin{styleStandard}
\textstylew{und}~\textstylew{allet~\textbf{so-ne}~\textbf{so-ne}}~\textstylew{\textbf{späßchen} }
\end{styleStandard}

\begin{styleStandard}
\textstylew{\ \ (}pseudonym: Dirk; lives in~Friedrichshain, Berlin\textstylew{)}
\end{styleStandard}

\begin{styleStandard}
\textstylew{und}~\textstylew{allet~}\textstylew{\textbf{so-ne}}~\textstylew{\textbf{späße}}~\textstylew{wa}
\end{styleStandard}

\begin{styleStandard}
\textstylew{\ \ (}pseudonym: Dirk; lives in~Friedrichshain, Berlin\textstylew{)}
\end{styleStandard}

\begin{styleStandard}
\textstylew{dit}~\textstylew{sind}~\textstylew{allet~}\textstylew{\textbf{so-ne}}~\textstylew{\textbf{dinger}}~\textstylew{ja }
\end{styleStandard}

\begin{styleStandard}
\textstylew{\ \ (}pseudonym: Dirk; lives in~Friedrichshain, Berlin\textstylew{)}
\end{styleStandard}

\begin{styleStandard}
\textstylew{dit}~\textstylew{sind}~\textstylew{wie}~\textstylew{jesacht~}\textstylew{\textbf{so-ne}}~\textstylew{\textbf{dinge}}~
\end{styleStandard}

\begin{styleStandard}
\textstylew{\ \ (}pseudonym: Dirk; lives in~Friedrichshain, Berlin\textstylew{)}
\end{styleStandard}

\begin{styleStandard}
\textstylew{wenn}~\textstylew{ick}~\textstylew{sehe}~\textstylew{zehn~}\textstylew{\textbf{so-ne}}~\textstylew{\textbf{stullchen}}~\textstylew{daa}~
\end{styleStandard}

\begin{styleStandard}
\textstylew{\ \ (}pseudonym: Dirk; lives in~Friedrichshain, Berlin\textstylew{)}
\end{styleStandard}

\begin{styleStandard}
\textstylew{dit}~\textstylew{sind}~\textstylew{ja}~\textstylew{ooch}~\textstylew{nich~}\textstylew{\textbf{so-ne}}~\textstylew{\textbf{sachen} }
\end{styleStandard}

\begin{styleStandard}
\textstylew{\ \ (}pseudonym: Dirk; lives in~Friedrichshain, Berlin\textstylew{)}
\end{styleStandard}

\begin{styleStandard}
\textstylew{aber}~\textstylew{äh~}\textstylew{\textbf{so-ne}}\textstylew{~}\textstylew{\textbf{so-ne}}~\textstylew{\textbf{dinge}}~\textstylenv{*}~\textstylew{dit}~\textstylew{stimmt }
\end{styleStandard}

\begin{styleStandard}
\textstylew{\ \ (}pseudonym: Dirk; lives in~Friedrichshain, Berlin\textstylew{)}
\end{styleStandard}

\begin{styleStandard}
\textstylew{und}~\textstylew{dazu}~\textstylew{noch~\textbf{so-ne}~\textbf{so-ne}}~\textstylew{\textbf{tochterfirmen} }
\end{styleStandard}

\begin{styleStandard}
\textstylew{\ \ (pseudonym: Gina; lives in }Hennigsdorf, Berlin\textstylew{)}
\end{styleStandard}

\begin{styleStandard}
\textstylew{oder}~\textstylenv{*}~\textstylew{so~}\textstylew{\textbf{so-ne}}~\textstylew{\textbf{antworten}}~\textstylew{kriegt}~\textstylew{man}~\textstylew{da}~\textstylew{halt}~\textstylew{ne }
\end{styleStandard}

\begin{styleStandard}
\textstylew{\ \ (pseudonym: Gina; lives in }Hennigsdorf, Berlin\textstylew{)}
\end{styleStandard}

\begin{styleStandard}
\textstylew{und~}\textstylew{\textbf{so-ne}}~\textstylew{\textbf{dinger}}~\textstylew{ja }
\end{styleStandard}

\begin{styleStandard}
\textstylew{\ \ (pseudonym: Gina; lives in }Hennigsdorf, Berlin\textstylew{)}
\end{styleStandard}

\begin{styleStandard}
\textstylew{da}~\textstylew{jabt-s}~\textstylew{auch}~\textstylew{jetzt}~\textstylenv{((unverständlich))}\textstylew{~}\textstylew{\textbf{so-ne}}~\textstylew{\textbf{shops}}~\textstylew{da }
\end{styleStandard}

\begin{styleStandard}
\textstylew{\ \ (pseudonym: Gina; lives in }Hennigsdorf, Berlin\textstylew{)}
\end{styleStandard}

\begin{styleStandard}
\textstylew{die}~\textstylew{sind}~\textstylew{da}~\textstylew{allet~\textbf{so-ne}}~\textstylew{\textbf{iischen}}~\textstylew{da }
\end{styleStandard}

\begin{styleStandard}
\textstylew{\ \ (pseudonym: Gina; lives in }Hennigsdorf, Berlin\textstylew{)}
\end{styleStandard}

\begin{styleStandard}
\textstylew{ürgendwie}~\textstylenv{*}~\textstylew{manchmal}~\textstylew{hab}~\textstylew{ick}~\textstylew{ooch~\textbf{so-ne}~\textbf{so-ne}}~\textstylew{\textbf{dinger}}~\textstylew{jehört}~
\end{styleStandard}

\begin{styleStandard}
\textstylew{\ \ (pseudonym: Gina; lives in }Hennigsdorf, Berlin\textstylew{)}
\end{styleStandard}

\begin{styleStandard}
rundrum und~\textbf{sone~sachn} die 
\end{styleStandard}

\begin{styleStandard}
\ \ (pseudonym: Lore; lives in Treptow, Berlin)
\end{styleStandard}

\begin{styleStandard}
jetzt in~\textbf{sone~kategorien} einteilen 
\end{styleStandard}

\begin{styleStandard}
\ \ (no information about speaker available)
\end{styleStandard}

\begin{styleStandard}
immer nur~\textbf{sone~fetzen} aufgefangen...dit sind~\textbf{sone~wörter} die 
\end{styleStandard}

\begin{styleStandard}
\ \ (pseudonym: Jenny; lives in Mitte, Berlin)
\end{styleStandard}

\begin{styleStandard}
\textstylew{und}~\textstylew{nich}~\textstylew{\textbf{so-ne}}~\textstylew{\textbf{zehn}}\textbf{~}\textstylew{\textbf{abjepackten}}\textbf{~}\textstylew{\textbf{stullchen}}~\textstylew{fürn}~\textstylew{gleichen}~\textstylew{preis}~
\end{styleStandard}

\begin{styleStandard}
\textstylew{\ \ (}pseudonym: Dirk; lives in~Friedrichshain, Berlin\textstylew{)}
\end{styleStandard}

\begin{styleStandard}
\textstylew{weil}~\textstylew{ooch~}\textstylew{\textbf{so-ne}}~\textstylenv{*}~\textstylew{\textbf{blöden}}\textbf{~}\textstylew{\textbf{zuchverbindungen}}~\textstylew{warn }
\end{styleStandard}

\begin{styleStandard}
\textstylew{\ \ (pseudonym: Gina; lives in }Hennigsdorf, Berlin\textstylew{)}
\end{styleStandard}

\begin{styleStandard}
\textstylew{na}~\textstylew{wie}~\textstylew{sacht}~\textstylew{man}~\textstylenv{*}\textstylew{~}\textstylew{\textbf{so-ne}}~\textstylew{\textbf{ausrangierten}}\textbf{~}\textstylew{\textbf{sachen} }
\end{styleStandard}

\begin{styleStandard}
\textstylew{\ \ (pseudonym: Gina; lives in }Hennigsdorf, Berlin\textstylew{)}
\end{styleStandard}

\begin{styleStandard}
\textstylew{ne\^{}}~\textstylenv{*}~\textstylew{imma}~\textstylew{nur~}\textstylew{\textbf{so-ne}}~\textstylew{\textbf{ollen}}\textbf{~}\textstylew{\textbf{äppel} }
\end{styleStandard}

\begin{styleStandard}
\textstylew{\ \ (pseudonym: Gina; lives in }Hennigsdorf, Berlin\textstylew{)}
\end{styleStandard}

\begin{styleStandard}
\textstylew{weil}~\textstylenv{*}~\textstylew{ick}~\textstylew{hatte~\textbf{so}}\textbf{~}\textstylew{\textbf{ne}}~\textstylew{\textbf{panischen}}\textbf{~}\textstylew{\textbf{ängste}}~\textstylew{in}~\textstylew{mir }
\end{styleStandard}

\begin{styleStandard}
\textstylew{\ \ (pseudoym: Maria; lives in }Marzahn, Berlin\textstylew{)}
\end{styleStandard}

\begin{styleStandard}
\textstylew{dit}~\textstylew{sind}~\textstylew{ja~\textbf{so}}\textbf{~}\textstylew{\textbf{ne}}~\textstylew{\textbf{richtchen}}\textbf{~}\textstylew{\textbf{arroganten}}\textbf{~}\textstylew{\textbf{scheißer}}~\textstylenv{*}~\textstylew{die}~\textstylew{ham }
\end{styleStandard}

\begin{styleStandard}
\textstylew{\ \ (pseudonym: Kira; lives in }Hellersdorf, Berlin\textstylew{)}
\end{styleStandard}

\begin{styleStandard}
durch~\textbf{sone~engen gassen }
\end{styleStandard}

\begin{styleStandard}
\ \ (pseudonym: Micha; lives in Lichtenberg, Berlin)
\end{styleStandard}

\begin{styleStandard}
daß auch~\textbf{sone~auswüchse} [NB: this token was not counted at all, DR] von...ich \textbf{sone} \textbf{übergroßen erwartungen }
\end{styleStandard}

\begin{styleStandard}
\ \ (pseudonym: Willy; lives in Köpenick, Berlin)
\end{styleStandard}

\begin{styleStandard}
warn ja~\textbf{sone~weißen schnecken }
\end{styleStandard}

\begin{styleStandard}
\ \ (pseudonym: Aldi; lives in Lichtenrade, Berlin)
\end{styleStandard}

\begin{styleStandard}\itshape
Biographische und Reiseerzählungen
\end{styleStandard}

\begin{styleStandard}
sonst auch~\textbf{sone~bekannten Losungen}
\end{styleStandard}

\begin{styleStandard}
\ \ (pseudonym: Gisela; born in Potsdam)
\end{styleStandard}

\begin{styleStandard}\itshape
Deutsch heute
\end{styleStandard}

\begin{styleStandard}
also~\textbf{sone~Sachen }
\end{styleStandard}

\begin{styleStandard}
\ \ (pseudonym: none; born in Berlin)
\end{styleStandard}

\begin{styleStandard}
\textstylew{also}~\textstylew{da}~\textstylew{kommen}~\textstylew{\textbf{so}}\textbf{~}\textstylew{\textbf{ne}}\textbf{~}\textstylew{\textbf{Nägel}}~\textstylew{nachher}
\end{styleStandard}

\begin{styleStandard}
\ \ (pseudonym: none; born in Crivitz, Mecklenburg-Vorpommern)
\end{styleStandard}

\begin{styleStandard}\itshape
Deutsch in Namibia
\end{styleStandard}

\begin{styleStandard}
\textstylew{\textbf{was}}\textbf{~}\textstylew{\textbf{für}}\textbf{~}\textstylew{\textbf{ne}}\textbf{~}\textstylew{\textbf{kinderspiele}}\textbf{~}\textstylew{ham}~(\textstylew{wir})~\textstylew{gespielt }
\end{styleStandard}

\begin{styleStandard}
\textstylew{(pseudonym: none; pupil born in }Otjiwarongo, Namibia; parents speak German\textstylew{)}
\end{styleStandard}

\begin{styleStandard}
\textstylew{un}~\textstylew{was}~\textstylew{is}~\textstylew{dem}~\textstylew{seine}~\textstylew{story}\textbf{~}\textstylew{\textbf{was}}\textbf{~}\textstylew{\textbf{für}}\textbf{~}\textstylew{\textbf{ne}}\textbf{~}\textstylew{\textbf{manners}}\textbf{~}\textstylew{hat}~\textstylew{der }
\end{styleStandard}

\begin{styleStandard}
\textstylew{(pseudonym: none; pupil born in }Gobabis,\textstylew{ Namibia; parents speak German)}
\end{styleStandard}

\begin{styleStandard}
\textstylew{\textbf{was}}\textbf{ }\textstylew{\textbf{für}}\textbf{\_}\textstylew{\textbf{ne}}\textbf{~}\textstylew{\textbf{spiele }}
\end{styleStandard}

\begin{styleStandard}
\textstylew{(pseudonym: none; pupil born in }Windhoek,\textstylew{ Namibia; mother speaks German)}
\end{styleStandard}

\begin{styleStandard}
sprechn un~\textbf{sone~sachn }
\end{styleStandard}

\begin{styleStandard}
\ \ (\textstylew{pseudonym: none; lab assistant; parents speak German})
\end{styleStandard}

\begin{styleStandard}
nich~\textbf{sone~leute }
\end{styleStandard}

\begin{styleStandard}
\ \ (pseudonym: none (but same as just above); lab assistant; parents speak German)
\end{styleStandard}

\begin{styleStandard}
ich will~\textbf{sone~haare }
\end{styleStandard}

\begin{styleStandard}
\ \ (pseudonym: none; pupil born in Pietermaritzburg, South Africa; parents speak German)
\end{styleStandard}

\begin{styleStandard}
du un~\textbf{sone~sachn} 
\end{styleStandard}

\begin{styleStandard}
\ \ (pseudonym: none; adult born in Windhoek, Namibia; parents speak German)
\end{styleStandard}

\begin{styleStandard}
un un~\textbf{sone~sachn} dabei 
\end{styleStandard}

\begin{styleStandard}
\ \ (pseudonym: none; self-employed; parents speak German)
\end{styleStandard}

\begin{styleStandard}
will da~\textbf{sone~stückn} rausschneidn 
\end{styleStandard}

\begin{styleStandard}
\ \ (pseudonym: none; accountant; parents speak German)
\end{styleStandard}

\begin{styleStandard}\itshape
Flucht und Emigration nach Großbritannien
\end{styleStandard}

\begin{styleStandard}
Aber \textbf{all}~\textbf{sone~Probleme} gab es 
\end{styleStandard}

\begin{styleStandard}
\ \ (name: Margarete; born in Berlin)
\end{styleStandard}

\begin{styleStandard}\itshape
Forschungs- und Lehrkorpus Gesprochenes Deutsch
\end{styleStandard}

\begin{styleStandard}
\textstylew{un}\_\textstylew{da}~\textstylew{gibts~}\textstylew{\textbf{so}}\textbf{~}\textstylew{\textbf{ne}}~\textstylew{und~}\textstylew{\textbf{so}}\textbf{~}\textstylew{\textbf{ne}}~\textstylew{\textbf{urteile} }
\end{styleStandard}

\begin{styleStandard}
\textstylew{\ \ (pseudonym: Rainer; dialect: Westphalian)}
\end{styleStandard}

\begin{styleStandard}
\textstylew{gummimanschette}~\textstylew{mit~}\textstylew{\textbf{so}}\textbf{~}\textstylew{\textbf{ne}}~\textstylew{\textbf{gummikragen} }
\end{styleStandard}

\begin{styleStandard}
\textstylew{\ \ (pseudonym: Albrecht; dialect: Eastphalian)}
\end{styleStandard}

\begin{styleStandard}
\textstylew{außer}\textstyleolmarker{[}\textstylew{dem}~\textstylew{wir}~\textstylew{können}~\textstylew{ja}\textstyleolmarker{]~}\textstylew{\textbf{so}}\textbf{~}\textstylew{\textbf{ne}}~\textstylew{\textbf{romantiker}}~\textstylew{we}~\textstylew{ge}~\textstylew{machen }
\end{styleStandard}

\begin{styleHTMLPreformatted}
\textstylew{\textrm{\ \ }\textrm{(pseudonym: Anita; dialect: Rhine-Franconian)}}
\end{styleHTMLPreformatted}

\begin{styleStandard}
\textstylew{ja}~\textstylew{die}~\textstylew{ham}~\textstylew{da}~\textstylew{normalerweise}~\textstylew{\textbf{so}}\textbf{\_}\textstylew{\textbf{ne}}~\textstylew{\textbf{mützen}}~\textstylew{auf }
\end{styleStandard}

\begin{styleStandard}
\textstylew{\ \ (pseudonym: Christina; dialect: Rhine-Franconian})
\end{styleStandard}

\begin{styleStandard}
\textstylew{so}~\textstylew{wie~\textbf{so}}\textbf{~}\textstylew{\textbf{ne}}~\textstylew{\textbf{nägel} }
\end{styleStandard}

\begin{styleStandard}
\textstylew{\ \ (pseudonym: Henrike; dialect: }Swabian\textstylew{)}
\end{styleStandard}

\begin{styleStandard}
\textstylew{weißt}~\textstylew{du}~\textstylew{da}~\textstylew{sitz}~\textstylew{er}~\textstylew{sogar}~\textstylew{\textbf{so}}\textbf{~}\textstylew{\textbf{ne}}~\textstylew{\textbf{kleinen}}\textbf{~}\textstylew{\textbf{rahmen}}~\textstylenv{(.)}~
\end{styleStandard}

\begin{styleStandard}
\ \ (pseudonym: Annabelle; dialect: \textstylew{Rhine-Franconian})
\end{styleStandard}

\begin{styleStandard}
also~\textbf{sone~flachen wandvorlagen }
\end{styleStandard}

\begin{styleStandard}
\ \ (pseudonym: Tobias; dialect: Northern Low German)
\end{styleStandard}

\begin{styleStandard}
also so~\textbf{sone~freien schulen }
\end{styleStandard}

\begin{styleStandard}
\ \ (pseudonym: Birgit; dialect: Mecklenburg-Western Pomeranian)
\end{styleStandard}

\begin{styleStandard}\itshape
Gesprochene Wissenschaftssprache Kontrastiv
\end{styleStandard}

\begin{styleStandard}
gibt s~\textbf{sone~karten} im 
\end{styleStandard}

\begin{styleStandard}
\ \ (pseudonym: Karen; dialect: not available)
\end{styleStandard}

\begin{styleStandard}\itshape
Mennonitenplautdietsch in Nord- und Südamerika
\end{styleStandard}

\begin{styleStandard}
Ik gleich~\textbf{sone~Menschen} nich 
\end{styleStandard}

\begin{styleStandard}
\ \ (pseudonym: none; born in Ciudad Cuauhtémoc, Mexico)
\end{styleStandard}

\begin{styleStandard}
gleich nich~\textbf{sone~Menschen} wat 
\end{styleStandard}

\begin{styleStandard}
\ \ (pseudonym: none (but different from just above); born in Ciudad Cuauhtémoc, Mexico)
\end{styleStandard}

\begin{styleStandard}\itshape
Russlanddeutsche Dialekte
\end{styleStandard}

\begin{styleStandard}
\textbf{viele~sone~Tänze} und 
\end{styleStandard}

\begin{styleStandard}
\ \ (pseudonym: none: born in Ukraine)
\end{styleStandard}

\begin{styleStandard}\bfseries
sone~scheene Jeschichten 
\end{styleStandard}

\begin{styleStandard}
\ \ (pseudonym: none (but same as just above); born in Ukraine)
\end{styleStandard}

\begin{styleStandard}
jesungen~\textbf{scheene~}\textbf{sone}\textbf{~Lieder }
\end{styleStandard}

\begin{styleStandard}
\ \ (pseudonym: none (but same as just above); born in Ukraine)
\end{styleStandard}

\begin{styleStandard}\itshape
Zwirner-Korpus
\end{styleStandard}

\begin{styleStandard}
und alle~\textbf{sone~Scherze} gemacht 
\end{styleStandard}

\begin{styleStandard}
\ \ (pseudonym: none; born in Schessinghausen, Lower Saxony)
\end{styleStandard}

\begin{styleStandard}\bfseries
alle~sone~ähnlichen Sachen 
\end{styleStandard}

\begin{styleStandard}
\ \ (pseudonym: none; born in Rosgaard, Schleswig-Holstein)
\end{styleStandard}

\clearpage\begin{styleNewTimesRoman}
References
\end{styleNewTimesRoman}

\begin{styleSubtitle}
\textmd{\textup{Abney, Steven. 1987, The English Noun Phrase in its Sentential Aspect. Doctoral Dissertation, \ \ MIT.}}
\end{styleSubtitle}

\begin{styleStandard}
Abraham, Werner. 2013. Dialect as a spoken-only medium: what it means – and what it does not \ \ mean. \textit{Linguistische Berichte}, \textit{Sonderheft} 19: 247-271. 
\end{styleStandard}

\begin{styleStandard}
Ackles, Nancy. 1996. The Indefinite Article, Historical Syntax and the Noun Phrase Structure. \ \ \textit{Working Papers in Linguistics} 14: 1-35.
\end{styleStandard}

\begin{styleStandard}
Alexiadou, Artemis. 2004. On the development of possessive determiners: Consequences for DP \ \ structure. In Eric Fuß and Carola Trips (eds.) \textit{Diachronic clues to synchronic grammar}, \ \ pp. 31-58. Amsterdam: John Benjamins.
\end{styleStandard}

\begin{styleStandard}
Alexiadou, Artemis. 2005. Possessors and (in)definiteness. \textit{Lingua} 115: 787-819.
\end{styleStandard}

\begin{styleStandard}
Alexiadou, Artemis. 2014. \textit{Multiple Determiners and the Structure of DPs}. Amsterdam: John \ \ Benjamins.
\end{styleStandard}

\begin{styleStandard}
Alexiadou, Artemis, Liliane Haegeman, and Melita Stavrou. 2007. \textit{Noun Phrase in the \ \ Generative Perspective}. Berlin: Mouton de Gruyter.
\end{styleStandard}

\begin{styleStandard}
Alexiadou, Artemis, Gianina Iordăchioaia \& Florian Schäfer. 2011. Scaling the variation in \ \ Romance and Germanic nominalizations. In Petra Sleeman \& Harry Perridon (eds.) \textit{The \ \ Noun Phrase in Romance and Germanic: Structure, variation, and change}, pp. 25-40. \ \ Amsterdam: John Benjamins.
\end{styleStandard}

\begin{styleStandard}
Allan, Keith. 1980. Nouns and Countability. \textit{Language} 56 (3): 541-67.
\end{styleStandard}

\begin{styleStandard}
Anderson, Stephen. 1992. \textit{A-morphous Morphology}. Cambridge: Cambridge University Press.
\end{styleStandard}

\begin{styleStandard}
Arregi, Karlos \& Andrew Nevins. 2012. \textit{Morphotactics. Basque Auxiliaries and the Structure of \ \ Spellout} [Studies in Natural Language and Linguistic Theory 86]. Dordrecht: Springer.
\end{styleStandard}

\begin{styleStandard}
Baechler, Raffaela. 2017. \textit{Absolute Komplexität in der Nominalflexion: Althochdeutsch,}
\end{styleStandard}

\begin{styleStandard}
\textit{Mittelhochdeutsch, Alemannisch und deutsche Standardsprache}. Berlin: Language
\end{styleStandard}

\begin{styleStandard}
Science Press.
\end{styleStandard}

\begin{styleStandard}
Barbiers, Sjef. 2005. Variation in the morphosyntax of \textit{one}. \textit{Journal of Comparative Germanic \ \ Linguistics} 8: 159-183.
\end{styleStandard}

\begin{styleStandard}
Barbiers, Sjef. 2008. Microvariation in syntactic doubling – an introduction. In Sjef Barbiers, \ \ Olaf Koeneman, Marika Lekakou \& Margreet van der Ham (eds.) \textit{Microvariation in \ \ Syntactic Doubling}. [Syntax and Semantix 36], pp. 1-34. United Kingdom: Emerald.
\end{styleStandard}

\begin{styleStandard}
Bayer, Josef. 2015. A Note on Possessor Agreement. In Hiroki Egashira, Hisatsugu Kitahara, \ \ Kazuo Nakazawa, Tadao Nomura, Masayuki Oishi, Akira Saizen, \& Motoko Suzuki \ \ (eds.) \textit{In Untiring Pursuit of Better Alternatives}, pp. 2-11. Tokyo: Kaitakusha.
\end{styleStandard}

\begin{styleStandard}
Bayer, Josef, Markus Bader, \& Michael Meng. 2001. Morphological underspecification meets \ \ oblique case: Syntactic and processing effects in German. \textit{Lingua} 111: 465-514.
\end{styleStandard}

\begin{styleStandard}
Bech, Gunnar. 1955/57. \textit{Studien über das deutsche verbum infinitum}. København: Det Kongelige \ \ Danske Akademie av Videnskaberne.
\end{styleStandard}

\begin{styleStandard}
Bennis, Hans, Norbert Corver \& Marcel den Dikken. 1998. Predication in nominal phrases. \textit{The \ \ Journal of Comparative Germanic Linguistics} 1: 85-117.
\end{styleStandard}

\begin{styleSubtitle}
\textmd{\textup{Bernstein, Judy B. 1993. Topics in the Syntax of Nominal Structure across Romance. Doctoral \ \ Dissertation, The City University of New York}}\textmd{.}
\end{styleSubtitle}

\begin{styleSubtitle}
\textmd{\textup{Bernstein, Judy B. 1997. Demonstratives and reinforcers in Romance and Germanic languages. \ \ }}\textmd{Lingua}\textmd{\textup{ 102: 87-113.}}
\end{styleSubtitle}

\begin{styleSubtitle}
\textmd{\textup{Bernstein, Judy B. 2001a. The DP Hypothesis: Identifying Clausal Properties in the Nominal \ \ Domain. In Mark Baltin \& Chris Collins (eds.) }}\textmd{The Handbook of Contemporary Syntactic \ \ Theory}\textmd{\textup{, pp. 536-561. Malden, MA: Blackwell Publishing Ltd.}}
\end{styleSubtitle}

\section[Bernstein, Judy B. 2001b. Focusing the “right” way in Romance determiner phrases. Probus 13: \ \ 1{}-29.]{\textmd{Bernstein, Judy B. 2001b. Focusing the “right” way in Romance determiner phrases. }\textmd{\textit{Probus}}\textmd{ 13: \ \ 1-29.}}
\begin{styleStandard}
Bhatt, Christa. 1990. \textit{Die syntaktische Struktur der Nominalphrase im Deutschen}. [Studien zur \ \ deutschen Grammatik 38]. Tübingen: Gunter Narr Verlag.
\end{styleStandard}

\begin{styleStandard}
Bierwisch, Martin. 1967. Syntactic Features in Morphology: General Problems of so-called \ \ Pronominal Inflection in German. In \textit{To Honor Roman Jacobson}: \textit{Essays on the Occasion \ \ of His Seventieth Birthday}. Vol. I, pp. 239-270. Den Haag: Mouton.
\end{styleStandard}

\section[Bisle{}-Müller, Hansjörg. 1991. Artikelwörter im Deutschen. Semantische und prakmatische \ \ Aspekte ihrer Verwendung. Tübingen: Niemeyer.]{\textmd{Bisle-Müller, Hansjörg. 1991. }\textmd{\textit{Artikelwörter im Deutschen. Semantische und prakmatische \ \ Aspekte ihrer Verwendung}}\textmd{. Tübingen: Niemeyer.}}
\begin{styleStandard}
Bittner, Dagmar. 2006. Was motiviert die partielle Unflektiertheit des indefiniten Artikels?\ \ \ \ Markiertheitstheoretische und sprachhistorische Überlegungen. \textit{Zeitschrift für \ \ germanistische Linguistik }34(3), 354-373.
\end{styleStandard}

\begin{styleStandard}
Blevins, James B. 1995. Syncretism and Paradigmatic Opposition. \textit{Linguistics and Philosophy} \ \ 18: 113-152.
\end{styleStandard}

\begin{styleStandard}
Borer, Hagit. 2005. \textit{In Name Only. Structuring Sense}, Volume I. Oxford: Oxford University \ \ Press.
\end{styleStandard}

\begin{styleStandard}
Borer, Hagit. 2013. \textit{Taking Form. Structuring Sense}, Volume III.\textit{ }Oxford: Oxford University \ \ Press.
\end{styleStandard}

\begin{styleStandard}
Börjars, Kersti. 1998. \textit{Feature Distribution in Swedish Noun Phrases}. Publications of the \ \ Philological Society 32. Boston: Blackwell.
\end{styleStandard}

\begin{styleStandard}
Bošković, Željko. 2004. Be careful where you float your quantifiers. \textit{Natural Language \& \ \ Linguistic Theory} 22 (4): 681-742.
\end{styleStandard}

\begin{styleFooter}
Bosse, Solveig. 2009. Split DPs in (Northern) German as Derived Structures. \textit{Interdisciplinary \ \ Journal for Germanic Linguistics and Semiotic Analysis} 14 (2): 251-287.
\end{styleFooter}

\begin{styleStandard}
Bowers, John. 1988. Extended X-Bar Theory, the ECP, and the Left Branch Condition. In\ \ \ \ \textit{Proceedings of the West Coast Conference on Formal Linguistics} 7: 47-62. Stanford \ \ Linguistics Association, Stanford University.
\end{styleStandard}

\begin{styleStandard}
Bowers, John. 1993. The Syntax of Predication. \textit{Linguistic Inquiry} 24 (4): 591-656.
\end{styleStandard}

\begin{styleStandard}
Braune, Wilhelm \& Ingo Reiffenstein. 2004. \textit{Althochdeutsche Grammatik I}. (15\textsuperscript{th} ed) Tübingen: \ \ Max Niemeyer Verlag.
\end{styleStandard}

\begin{styleStandard}
Brugè, Laura. 1996. Demonstrative movement in Spanish: A comparative approach. \textit{University \ \ of Venice Working Papers in Linguistics} 6(1): 1-61.
\end{styleStandard}

\section[Brugè, Laura. 2002. The Positions of Demonstratives in the Extended Nominal Projection. In \ \ Guglielmo Cinque (ed.) Functional Structure in DP and IP: The Cartography of \ \ Syntactic Structures. vol.1, pp. 15{}-53. Oxford: Oxford University Press.]{\textmd{Brugè, Laura. 2002. The Positions of Demonstratives in the Extended Nominal Projection. In \ \ Guglielmo Cinque (ed.) }\textmd{\textit{Functional Structure in DP and IP: The Cartography of \ \ Syntactic Structures}}\textmd{. vol.1, pp. 15-53. Oxford: Oxford University Press.}}
\begin{styleFooter}
Carlson, Gregory N. 1980. \textit{Reference to kinds in English}. New York: Garland.
\end{styleFooter}

\begin{styleFooter}
Carnie, Andrew. 2021. \textit{Syntax. A Generative Introduction}. 4\textsuperscript{th} ed. Cambridge: Blackwell \ \ Publishers.
\end{styleFooter}

\begin{styleSubtitle}
\textmd{\textup{Chomsky, Noam. 1957. }}\textmd{Syntactic structures}\textmd{\textup{. The Hague: Mouton.}}
\end{styleSubtitle}

\begin{styleStandard}
Chomsky, Noam. 1970. Remarks on nominalization. In R. Jacobson \& P. Rosenbaum (eds.) \ \ \textit{Readings in English Transformational Grammar}, pp. 184-221. Waltham, MA, Ginn \& \ \ Co.
\end{styleStandard}

\begin{styleStandard}
Chomsky, Noam. 1981. \textit{Lectures on Government and Binding}. Dordrecht: Foris.
\end{styleStandard}

\begin{styleStandard}
Chomsky, Noam. 1991. Some Notes on Economy of Derivation and Representation. [reprinted in \ \ Noam Chomsky. 1995. \textit{The Minimalist Program}, pp. 129-166.]
\end{styleStandard}

\begin{styleSubtitle}
\textmd{\textup{Chomsky, Noam. 1995. }}\textmd{Minimalist Program.}\textmd{\textup{ Cambridge: MIT Press.}}
\end{styleSubtitle}

\begin{styleStandard}
Chomsky, Noam. 2000. Minimalist Inquiries: The Framework. In Roger Martin, David \ \ Micheals \& Juan Uriagereka (eds.) \textit{Step by Step: Essays on Minimalist Syntax in Honor \ \ of Howard Lasnik}, pp. 89-155. Cambridge, Mass.: MIT Press.
\end{styleStandard}

\begin{styleStandard}
Chomsky, Noam \& Howard Lasnik. 1993. The Theory of Principles and Parameters. [reprinted \ \ in Noam Chomsky. 1995. \textit{The Minimalist Program}, pp. 13-127.]
\end{styleStandard}

\begin{styleStandard}
Cinque, Guglielmo. 1994. On the Evidence for Partial N-Movement in the Romance DP. In \ \ Guglielmo Cinque, Jan Koster, Jean-Yves Pollock, Luigi Rizzi \& Raffaella Zanuttini \ \ (eds.) \textit{Paths Towards Universal Grammar. Studies in Honor of Richard S. Kayne}, pp. 85-\ \ 110. Georgetown University Press.
\end{styleStandard}

\begin{styleStandard}
Cinque, Guglielmo. 2010. \textit{The Syntax of Adjectives. A Comparative Study}. Cambridge, MA: MIT \ \ Press.
\end{styleStandard}

\begin{styleStandard}
Cirillo, Robert. 2016. Why all John’s friends are Dutch, not German: On the determiner-like \ \ characteristics of the inflection on the universal quantifier in West Germanic. \textit{Journal of \ \ Germanic Linguistics} 28(3): 179-218.
\end{styleStandard}

\begin{styleStandard}
Coppock, Elizabeth \& David Beaver. 2015. Definiteness and Determinacy. Linguistics and \ \ Philosophy 38(5): 377-435.
\end{styleStandard}

\begin{styleStandard}
Corbett, Greville G. 2000. \textit{Number}. Cambridge: Cambridge University Press.
\end{styleStandard}

\begin{styleStandard}
Corver, Norbert. 1991. Evidence for DegP. \textit{Proceedings of NELS} 21: 33-47.
\end{styleStandard}

\begin{styleBodyTextIndentii}
Corver, Norbert. 1997. The Internal Structure of the Dutch Extended Adjectival Projection. \ \ \textit{Natural Language and Linguistic Theory} 15: 289-368.\newline
Corver, Norbert. 2003. A Note on Micro-dimensions of Possession in Dutch and Related \ \ Languages. In Jan Koster and Henk van Riemsdijk (eds.) \textit{Germania et alia. A linguistic \ \ Webschrift for Hans den Besten}. [available at: \ \ http://odur.let.rug.nl/koster/DenBesten/contents.htm]
\end{styleBodyTextIndentii}

\begin{styleStandard}
Corver, Norbert. 2006. Proleptic Agreement as a good design property. In João Costa \textstyleptbrandiii{\& Maria \ \ Cristina Figueiredo Silva} (eds.) \textit{Studies on Agreement}, pp. 47-73. Amsterdam: John \ \ Benjamins.
\end{styleStandard}

\begin{styleStandard}
Corver, Norbert \& Marjo van Koppen. 2010. Ellipsis in Dutch possessive noun phrases: a \ \ micro-comparative approach. \textit{The Journal of Comparative Germanic Linguistics} 13(2): \ \ 99-140.
\end{styleStandard}

\begin{styleFooter}
Corver, Norbert \& Marjo van Koppen. 2011a. Micro-diversity in Dutch interrogative DPs. A \ \ case study on the (dis)continuous \textit{wat voor ’n N}{}-construction. In Petra Sleeman \& Harry \ \ Perridon (eds.) \textit{The Noun Phrase in Romance and Germanic: Structure, variation, and \ \ change}, pp. 57-87. Amsterdam: John Benjamins.
\end{styleFooter}

\begin{styleStandard}
Corver, Norbert \& Marjo van Koppen. 2011b. NP-ellipsis with adjectival remnants: a micro-\ \ comparative perspective. \textit{Natural Language \& Linguistic Theory} 29 (2): 371-421.
\end{styleStandard}

\begin{styleStandard}
Crisma, Paola. 1999. Nominals without the article in the Germanic languages. \textit{Rivista di \ \ Grammatica Generativa} 24: 105-125.
\end{styleStandard}

\begin{styleStandard}
Curme, George O. 1910. The Origin and Growth of the Adjective Declension in Germanic. \ \ \textit{Journal of English and Germanic Philology} 9: 439-482
\end{styleStandard}

\begin{styleSubtitle}
\textmd{\textup{Darski, Józef. 1979. Die Adjektivdeklination im Deutschen. }}\textmd{Sprachwissenschaft}\textmd{\textup{ 4: 190-205.}}
\end{styleSubtitle}

\begin{styleSubtitle}
\textmd{\textup{Déchaine, Rose-Marie \& Martina Wiltschko. 2002. Decomposing Pronouns. }}\textmd{Linguistic Inquiry}\textmd{\textup{ \ \ 33 (3): 409-442.}}
\end{styleSubtitle}

\begin{styleStandard}
De Lacy, Paul. 2006. \textit{Markedness. Reduction and preservation in phonology}. Cambridge: \ \ Cambridge University Press
\end{styleStandard}

\begin{styleStandard}
Delfitto, Denis \& Jan Schroten. 1991. Bare Plurals and the Number Affix in DP. \textit{Probus} 3(2): \ \ 155-85.
\end{styleStandard}

\begin{styleStandard}
Delorme, Evelyne \& Ray C. Dougherty. 1972. Appositive NP Constructions: \textit{we, the men; we \ \ men; I, a man}; ETC. \textit{Foundations of Language} 8: 2-29.
\end{styleStandard}

\begin{styleStandard}
Delsing, Lars-Olof. 1993. \textit{The Internal Structure of Noun Phrases in the Scandinavian \ \ Languages. A Comparative Study}. Doctoral Dissertation, University of Lund.
\end{styleStandard}

\begin{styleStandard}
Demske, Ulrike. 2001. \textit{Merkmale und Relationen: Diachrone Studien zur Nominalphrase des \ \ Deutschen}. Berlin and New York: Walter de Gruyter.
\end{styleStandard}

\begin{styleStandard}
Déprez, Viviane. 2005. Morphological number, semantic number and bare nouns. \textit{Lingua} 115: \ \ 857-883.
\end{styleStandard}

\begin{styleStandard}
Diesing, Molly. 1992. \textit{Indefinites}. Cambridge: MIT Press.
\end{styleStandard}

\begin{styleStandard}
Dikken, Marcel den. 1995. Copulas. Paper presented at GLOW 18, Tromsø. Unpublished \ \ manuscript, Vrije Universiteit Amsterdam/HIL.
\end{styleStandard}

\begin{styleNewTimesRoman}
Dikken, Marcel den. 2006. \textit{Relators and Linkers. The Syntax of Predication, Predicate Inversion, \ \ and Copulas}. Cambridge: MIT Press.
\end{styleNewTimesRoman}

\begin{styleStandard}
Duden. 1989. \textit{Deutsches Universalwörterbuch}. (2nd. ed) Mannheim: Bibliographisches Institut.
\end{styleStandard}

\begin{styleStandard}
Duden. 1995. \textit{Duden: Grammatik der deutschen Gegenwartsprache}. Vol 4. (5th ed.) Mannheim: Dudenverlag.
\end{styleStandard}

\begin{styleStandard}
Duden. 2007. \textit{Duden:} \textit{Richtiges und gutes Deutsch}. Vol. 9. (6th ed.) Mannheim: Dudenverlag.
\end{styleStandard}

\begin{styleSubtitle}
\textmd{\textup{Duden. 2016. }}\textmd{Duden: Grammatik der deutschen Gegenwartsprache}\textmd{\textup{. Vol 4. (9th ed.) Mannheim: \ \ Dudenverlag.}}
\end{styleSubtitle}

\begin{styleSubtitle}
\textmd{\textup{Durrell, Martin. 2002. }}\textmd{Hammer’s German grammar and usage}\textmd{\textup{. 4th ed. Chicago: McGraw-Hill.}}
\end{styleSubtitle}

\begin{styleStandard}
Dürscheid, Christa. 2002. “Polemik satt und Wahlkampf pur” – Das postnominale Adjektiv im \ \ Deutschen. \textit{Zeitschrift für Sprachwissenschaft} 21 (1): 57-81.
\end{styleStandard}

\begin{styleStandard}
Ebert, Robert Peter, Oskar Reichmann, Hans-Joachim Solms, \& Klaus-Peter Wegera. 1993.\textit{ Frühneuhochdeutsche Grammatik. }Tübingen: Max Niemeyer Verlag.
\end{styleStandard}

\begin{styleStandard}
Eichhoff, Jürgen. 1993. “Ich bin ein Berliner”: A History and a Linguistic Clarification. \ \ \textit{Monatshefte} 85 (1): 71-80.
\end{styleStandard}

\begin{styleStandard}
Eisenberg, Peter. 1998. \textit{Grundriß der deutschen Grammatik. Band 1: Das Wort}. Stuttgart: \ \ Metzler.
\end{styleStandard}

\begin{styleStandard}
Eisenberg, Peter. 1999. \textit{Grundriß der deutschen Grammatik}. \textit{Band 2: Der Satz}. Stuttgart: \ \ Metzler.
\end{styleStandard}

\begin{styleStandard}
Eisenberg, Peter \& George Smith. 2002. Der einfache Genitiv. Eigennamen als Attribute. In \ \ Corinna Peschel (ed.) \textit{Grammatik und Grammatikvermittlung}, pp. 113-126. Frankfurt: \ \ Lang.
\end{styleStandard}

\begin{styleStandard}
Elmentaler, Michael, and Peter Rosenberg. 2015. \textit{Norddeutscher Sprachatlas (NOSA). Band 1: \ \ Regionale Sprachlagen}. Hildesheim: Olms. 
\end{styleStandard}

\begin{styleStandard}
Embick, David \& Rolf Noyer. 2007. Distributed Morphology and the Syntax – Morphology \ \ Interface. In Gillian Ramchand and Charles Reiss (eds.) \textit{Oxford Handbook of Linguistic \ \ Interfaces}, pp. 289-324. Oxford: Oxford University Press.
\end{styleStandard}

\begin{styleStandard}
Emonds, Joseph E. 1987. The invisible Category Principle. \textit{Linguistic Inquiry} 18: 613-632.
\end{styleStandard}

\begin{styleStandard}
Esau, Helmut. 1973. Form and Function of German Adjectival endings. \textit{Folia Linguistica }6 \ \ (1/2): 136-45.
\end{styleStandard}

\begin{styleStandard}
Evans, Elliott. 2019. The Origin, Functions, and Histories of Germanic Adjective Endings. \ \ Doctoral dissertation, Indiana University.
\end{styleStandard}

\begin{styleStandard}
Evans, Elliott. 2021. Prenominal French and Uninflected Dutch Adjective. \textit{Studia Linguistica} 75 \ \ (3): 511-537.
\end{styleStandard}

\begin{styleStandard}
Eynde, Frank van. 2020. Agreement, disagreement and the NP vs. DP debate. \textit{Glossa: A Journal \ \ of General Linguistics} 5(1): 65. 1-23.
\end{styleStandard}

\begin{styleStandard}
Fanselow, Gisbert. 1988. Aufspaltung von NPn und das Problem der “freien” Wortstellung. \ \ \textit{Linguistische Berichte} 114: 91-113.
\end{styleStandard}

\begin{styleref}
Fanselow, Gisbert, and Damir Ćavar. 2002. Distributed deletion. In \textit{Theoretical Approaches to \ \ Universals}, ed. A. Alexiadou, 65–107. Amsterdam: John Benjamins.
\end{styleref}

\begin{styleSubtitle}
\textmd{\textup{Felix, Sascha W. 1990. The Structure of functional categories. }}\textmd{Linguistische Berichte}\textmd{\textup{ 125: 46-\ \ 71.}}
\end{styleSubtitle}

\begin{styleStandard}
Fintel, Kai von \& Sabine Iatridou. 2007. Anatomy of a Modal Construction. \textit{Linguistic Inquiry} \ \ 38 (3): 445-483.
\end{styleStandard}

\begin{styleStandard}
Fischer, Silke. 2006. Zur Morphologie der deutschen Personalpronomina - eine \ \ Spaltungsanalyse. In Gereon Müller \& Jochen Trommer (eds.) \textit{Subanalysis of Argument \ \ Encoding in Distributed Morphology}. [Linguistische Arbeitsberichte 84], 77-101. \ \ Universität Leipzig.
\end{styleStandard}

\begin{styleStandard}
Fiva, Toril. 1985. NP-Internal Chains in Norwegian. \textit{Nordic Journal of Linguistics} 8 (1): 25-47.
\end{styleStandard}

\begin{styleStandard}
Fodor, Janet Dean \& Ivan A. Sag. 1982. Referential and Quantificational Indefinites. \textit{Linguistics \ \ and Philosophy} 5: 335-398.
\end{styleStandard}

\begin{styleStandard}
Fortmann, Christian. 1996. \textit{Konstituentenbewegung in der DP-Struktur. Zur funktionalen \ \ Analyse der Nominalphrase im Deutschen}. Linguistische Arbeiten 347. Tübingen: \ \ Niemeyer.
\end{styleStandard}

\begin{styleStandard}
Fuhrhop, Nanna. 2003. Zur Grammatik der Stadtadjektive. \textit{Linguistische Berichte }193: 91-108.
\end{styleStandard}

\begin{styleStandard}
Fuß, Eric. 2011. Eigennamen und adnominaler Genitiv im Deutschen. \textit{Linguistische Berichte} \ \ \ 225: 19-42.
\end{styleStandard}

\begin{styleStandard}
Gallmann, Peter. 1990. \textit{Kategoriell komplexe Wortformen}. Tübingen: Niemeyer.
\end{styleStandard}

\begin{styleStandard}
Gallmann, Peter. 1996. Die Steuerung der Flexion in der DP. \textit{Linguistische Berichte} 164: 283-\ \ 314.
\end{styleStandard}

\begin{styleNewTimesRoman}
Gallmann, Peter. 1998. Case Underspecification in Morphology, Syntax and the Lexicon. In \ \ Artemis Alexiadou \& Chris Wilder (eds.) \textit{Possessors, Predicates and Movement in the \ \ Determiner Phrase}, pp. 141-175. Amsterdam: John Benjamins Publishing Company.
\end{styleNewTimesRoman}

\begin{styleStandard}
Gallmann, Peter. 2004. Feature Sharing in DPs. In Gereon Müller, Lutz Gunkel, and Gisela \ \ Zifonun (eds.) \textit{Explorations in Nominal Inflection}. Berlin: Mouton de Gruyter. pp. 121-\ \ 160.
\end{styleStandard}

\begin{styleStandard}
Gallmann, Peter. 2018. The genitive rule and its background. In Tanja Ackermann, Horst J. \ \ Simon \& Christian Zimmer (eds) \textit{Germanic Genitives}, pp. 149-188. Amsterdam: John \ \ Benjamins.
\end{styleStandard}

\begin{styleStandard}
Gallmann, Peter, and Thomas Lindauer. 1994. Funktionale Kategorien in Nominalphrasen. \ \ \textit{Beiträge zur Geschichte der Deutschen Sprache und Literatur} (PBB) 116(1): 1-27.
\end{styleStandard}

\begin{styleStandard}
Gelderen, Elly van. 2007. The Definiteness Cycle in Germanic. \textit{Journal of Germanic Linguistics} \ \ 19(4): 275-308.
\end{styleStandard}

\begin{styleStandard}
Georgi, Doreen, \& Salzmann, Martin. 2011. DP-internal double agreement is not double Agree: \ \ Consequences of Agree-based case assignment within DP. \textit{Lingua }121(14): 2069-2088.
\end{styleStandard}

\begin{styleStandard}
Giusti, Giuliana. 1997. The categorial status of determiners. In Liliane Haegeman (ed.) \textit{The New Comparative Syntax}, pp. 95-123. London and New York: Longman.
\end{styleStandard}

\section[Giusti, Giuliana. 2002. The Functional Structure of Noun Phrases. A Bare Phrase Structure \ \ Approach. In Guglielmo Cinque (ed.) Functional Structure in DP and IP: The \ \ Cartography of Syntactic Structures. vol.1, pp. 54{}-90. Oxford: Oxford University Press.]{\textmd{Giusti, Giuliana. 2002. The Functional Structure of Noun Phrases. A Bare Phrase Structure \ \ Approach. In Guglielmo Cinque (ed.) }\textmd{\textit{Functional Structure in DP and IP: The \ \ Cartography of Syntactic Structures}}\textmd{. vol.1, pp. 54-90. Oxford: Oxford University Press.}}
\begin{styleStandard}
Giusti, Giuliana. 2015. \textit{Nominal Syntax at the Interfaces: A Comparative Analysis of}
\end{styleStandard}

\begin{styleStandard}
\textit{Languages with Articles}. Newcastle: Cambridge Scholars Publishing.
\end{styleStandard}

\begin{styleStandard}
Giusti, Giuliana, and Rossella Iovino. 2016.~Latin as a Split-DP Language. \textit{Studia \ \ Linguistica} \ \ 70(3): 221-249~
\end{styleStandard}

\begin{styleStandard}
Glaser, Elvria. 1993. Syntaktische Strategien zum Ausdruck von Indefinitheit und Partitivität im \ \ Deutschen (Standardsprache und Dialekt). In Werner Abraham und Josef Bayer (eds) \ \ \textit{Dialektsyntax}, pp. 99-116. Opladen: Westdeutscher Verlag.
\end{styleStandard}

\begin{styleStandard}
Greenberg, Joseph. 1978. How does a language acquire gender markers. In Joseph Greenberg\ \ \ \ (ed.), \textit{Universals of human language}, Vol. 3, 47-82. Stanford: Stanford University\ \ \ \ \ \ Press.
\end{styleStandard}

\begin{styleStandard}
Grice, H. Paul. 1975. Logic and conversation. In Peter Cole and Jerry Morgan (eds.) \textit{Syntax and \ \ Semantics. Vol. 3: Speech Acts}, 43-58. New York: Academic Press.
\end{styleStandard}

\begin{styleStandard}
Grice, H. Paul. 1978. Further notes on logic and conversation. In Peter Cole (ed.) \textit{Syntax and \ \ Semantics. Vol. 9: Pragmatics}, 113-28. New York: Academic Press.
\end{styleStandard}

\begin{styleJBReferenceList}
Griesbach, Heinz \& Dora Schulz. 1965. \textit{Grammatik der deutschen Sprache}. (3\textsuperscript{rd} ed.) München: \ \ Max Hueber Verlag.
\end{styleJBReferenceList}

\begin{styleStandard}
Grimm, Jacob. 1870. \textit{Deutsche Grammatik}. Vol 1. (2\textsuperscript{nd} ed). Berlin. 
\end{styleStandard}

\begin{styleStandard}
Grimm, Scott. 2012. \textit{Number and Individuation. }Doctoral Dissertation,\textbf{\textit{ }}Stanford University.
\end{styleStandard}

\begin{styleJBReferenceList}
Grimshaw, Jane. 1991. Extended Projections. Ms., Brandeis University.
\end{styleJBReferenceList}

\begin{styleStandard}
Grohmann, Kleanthes K. \& Liliane Haegeman. 2003. \href{http://www.punksinscience.org/kleanthes/papers/scl19_gh.pdf}{\textstyleInternetlink{Resuming Reflexives}}. \textit{Proceedings of the \ \ 19th Scandinavian Conference on Linguistics. Nordlyd} 31(1): 46-62.
\end{styleStandard}

\begin{styleStandard}
Gunkel, Lutz, Adriano Murelli, Susan Schlotthauer, Bernd Wiese, and Gisela Zifonun with \ \ collaboration by Christine Günther and Ursula Hoberg. 2017. Grammatik des Deutschen \ \ im europäischen Vergleich. Das Nominal (Volume 2). [Schriften des Instituts für \ \ Deutsche Sprache 14]. Berlin, Boston: De Gruyter. 
\end{styleStandard}

\begin{styleStandard}
Hachem, Mirjam. 2015. \textit{Multifunctionality: The Internal and External Syntax of D-and W-Items \ \ in German and Dutch}. Utrecht: LOT.
\end{styleStandard}

\begin{styleFooter}
Haegeman, Liliane \& Terje Lohndal. 2010. Negative Concord and (Multiple) Agree: A Case \ \ Study of West Flemish. \textit{Linguistic Inquiry} 41 (2): 181-211.
\end{styleFooter}

\begin{styleStandard}
Haider, Hubert. 1988. Die Structure der deutschen NP. \textit{Zeitschrift für Sprachwissenschaft} 7 (1): 32-59.
\end{styleStandard}

\begin{styleStandard}
Haider, Hubert. 1992. Die Struktur der Nominalphrase. Lexikalische und funktionale Strukturen. In Ludger Hoffmann (ed.) \textit{Deutsche Syntax. Ansichten und Aussichten. Institut für deutsche Sprache Jahrbuch 1991}, pp. 304-333. Berlin: Walter de Gruyter.
\end{styleStandard}

\begin{styleStandard}
Haider, Hubert. 1993. \textit{Deutsche Syntax – Generativ}. Tübingen: Narr.
\end{styleStandard}

\begin{styleStandard}
Halle, Morris. 1997. Distributed Morphology: Impoverishment and Fission. \textit{MIT Working \ \ Papers in Linguistics} 30: 425-449.
\end{styleStandard}

\begin{styleStandard}
Halle, Morris \& Alec Marantz. 1993. Distributed Morphology and the Pieces of Inflection. In \ \ Kenneth Hale \& Samuel K. Keyser (eds.)\textit{ The View from Building 20. Essays in Honor of \ \ Sylvain Bromberger}, pp. 111-176. Cambridge: MIT Press.
\end{styleStandard}

\begin{styleStandard}
Halle, Morris, and Alec Marantz. 1994. Some key features of Distributed Morphology. \textit{MIT \ \ Working Papers in Linguistics} 21: 275-288.
\end{styleStandard}

\begin{styleStandard}
Hallman, Peter. 2004. NP-interpretation and the structure of predicates. \textit{Language} 80(4): 707-47.
\end{styleStandard}

\begin{styleStandard}
Harbert, Wayne. 2007. \textit{The Germanic Languages}. Cambridge: Cambridge University Press.
\end{styleStandard}

\begin{styleStandard}
Harley, Heidi and Rolf Noyer. 1999. Distributed Morphology. \textit{Glot International} 4(4): 3-9. \ \ \url{https://babel.ucsc.edu/~hank/mrg.readings/harley&noyer.pdf}
\end{styleStandard}

\begin{styleStandard}
Harris, James. 1994. The syntax-phonology mapping in Catalan and Spanish clitics. \ \ In~\href{http://broca.mit.edu/mitwpl.web/WPLs.html}{\textstyleInternetlink{\textit{MITWPL}}}\textit{~21: Papers on phonology and morphology}, ed. Andrew Carnie and Heidi \ \ Harley, 321-353.~Cambridge: MITWPL.
\end{styleStandard}

\begin{styleStandard}
Haspelmath, Martin. 2019. Indexing and flagging, and head and dependent marking. \textit{Te Reo} \ \ 62(1). 93–115.
\end{styleStandard}

\begin{styleStandard}
Haspelmath, Martin \& Andrea D. Sims. 2010. \textit{Understanding Morphology}. (2\textsuperscript{nd} ed.) London: \ \ Hodder Education.
\end{styleStandard}

\begin{styleStandard}
Hawkins, John A. 1978. \textit{Definiteness and indefiniteness: A study in reference and \ \ grammaticality prediction}. London: Croom Helm.
\end{styleStandard}

\begin{styleStandard}
Hawkins, John A. 1994. \textit{A performance theory of order and constituency}. Cambridge: \ \ Cambridge University Press.
\end{styleStandard}

\begin{styleStandard}
Hawkins, John A. 2004. \textit{Efficiency and Complexity in Grammars}. Oxford: Oxford University \ \ Press.
\end{styleStandard}

\begin{styleStandard}
Hazout, Ilan. 2004. The Syntax of Existential Constructions. \textit{Linguistic Inquiry} 35(3): 393-430.
\end{styleStandard}

\begin{styleStandard}
Heck, Fabian, Gereon Müller \& Jochen Trommer. 2008. A phase-based approach to \ \ Scandinavian definiteness marking. In Charles B. Chang and Hannah J. Haynie (eds.) \ \ \textit{Proceedings of WCCFL 26}, pp. 226-233. Somerville, MA: Cascadilla Proceedings \ \ Project.
\end{styleStandard}

\begin{styleStandard}
Heim, Irene \& Angelika Kratzer. 1998. \textit{Semantics in Generative Grammar.} Cambridge, Mass.: Blackwell.
\end{styleStandard}

\begin{styleTextbody}
Helbig, Gerhard \& Joachim Buscha. 2001. \textit{Deutsche Grammatik. Ein Handbuch für den \ \ Ausländerunterricht}. Berlin und München: Langenscheidt.
\end{styleTextbody}

\begin{styleStandard}
Heusinger, Klaus von. 2011. Specificity, Referentiality and Discourse Prominence: German \ \ Indefinite Demonstratives. In Ingo Reich (ed.)\textit{ Proceedings of Sinn \& Bedeutung }15, pp. \ \ 9-30.
\end{styleStandard}

\begin{styleStandard}
Heycock, Caroline and Roberto Zamparelli. 2005. Friends and colleagues: Plurality, \ \ coordination, and the structure of DP. \textit{Natural Language Semantics} 13: 201–270.
\end{styleStandard}

\begin{styleStandard}
Higginbotham, James. 1987. Indefiniteness and Predication. In Eric J. Reuland \& Alice G.B. ter \ \ Meulen (eds.) \textit{The Representation of (In)definiteness}, pp. 43-70. Cambridge: MIT Press.
\end{styleStandard}

\begin{styleStandard}
Hoekstra, Teun. 1984. \textit{Transitivity: Grammatical Relations in Government-Binding Theory}. \ \ Dordrecht: Foris.
\end{styleStandard}

\begin{styleStandard}
Höhn, Georg F.K. 2020. The third person gap in adnominal pronoun constructions. \textit{Glossa: a \ \ journal of general linguistics} 5(1): 69. 1-43.
\end{styleStandard}

\begin{styleStandard}
Hole, Daniel \& Gerson Klumpp. 2000. Definite type and indefinite token: the article \textit{son} in \ \ colloquial German. \textit{Linguistische Berichte} 182: 231-244.
\end{styleStandard}

\begin{styleStandard}
Hoof, Hanneke van. 2006. Split Topicalization. In Martin Everaert \& Henk van Riemsdijk (eds.) \ \ \textit{The Blackwell companion to syntax}, Vol. IV, pp. 410-465. Oxford: Blackwell.
\end{styleStandard}

\begin{styleStandard}
Hurford, James R. 2003. The interaction between numerals and nouns. In Frans Plank (ed.) \textit{Noun \ \ Phrase Structure in the Languages of Europe }[Empirical Approaches to Language \ \ Typology, EUROTYP 20-7], pp. 561-620. Berlin: Mouton de Gruyter.
\end{styleStandard}

\begin{styleStandard}
Ihsane, Tabea \& Genoveva Puskás. 2001. Specific is not definite. \textit{Generative Grammar in \ \ Geneva} 2: 39-54.
\end{styleStandard}

\begin{styleStandard}
Iordăchioaia, Gianina. 2020. D and N are different nominalizers. \textit{Glossa: A Journal of General \ \ Linguistics} 5(1): 53. 1-25.
\end{styleStandard}

\begin{styleStandard}
Jackendoff, Ray S. 1977. \textit{X’ Syntax: A Study of Phrase Structure}. Cambridge: MIT Press.
\end{styleStandard}

\begin{styleStandard}
Jacobs. Joachim. 1980. Lexical decomposition in Montague Grammar. \textit{Theoretical Linguistics} 7: \ \ 121-136.
\end{styleStandard}

\begin{styleStandard}
Jacobs, Neil G. 2005. \textit{Yiddish. A linguistic introduction}. Cambridge: Cambridge University \ \ Press.
\end{styleStandard}

\begin{styleStandard}
Jäger, Agnes, \& Penka, Doris. 2012. Development of sentential negation in the history of \ \ German. In Peter Ackema, Rhona Alcorn, Caroline Heycock, Dany Jaspers, Jeroen van \ \ Craenenbroeck, and Guido Vanden Wyngaerd (eds.) \textit{Comparative Germanic Syntax: The \ \ State of the Art }[Linguistik Aktuell/Linguistics Today\textit{ }191], pp 199-222. Amsterdam: \ \ John Benjamins.
\end{styleStandard}

\begin{styleStandard}
Janda, Richard D. \& Brian D. Joseph. 1992. Pseudo-Agglutinativity in Modern Greek Verb-\ \ Inflection and “Elsewhere”. In \textit{Proceedings from the 28th Regional Meeting of the \ \ Chicago Linguistic Society}. Vol. 1, pp. 251-266.
\end{styleStandard}

\begin{styleStandard}
Julien, Marit. 2002. Determiners and Word Order in Scandinavian DPs. \textit{Studia Linguistica} 56: 264-315.
\end{styleStandard}

\begin{styleStandard}
Julien, Marit. 2005a. \textit{Nominal Phrases from a Scandinavian Perspective}. Amsterdam: John \ \ Benjamins.
\end{styleStandard}

\begin{styleStandard}
Julien, Marit. 2005b. Possessor licensing, definiteness and case in Scandinavian. In Marcel den Dikken \& Christina Tortora (eds.) \textit{The Function of Function Words and Functional Categories}, pp. 217-49. Amsterdam: John Benjamins Publishing Company.
\end{styleStandard}

\begin{styleStandard}
Julien, Marit. 2016. Possessive predicational vocatives in Scandinavian. \textit{The Journal of \ \ Comparative Germanic Linguistics} 19: 75-109.
\end{styleStandard}

\begin{styleStandard}
Karnowski, Paweł and Jürgen Pafel. 2004. A topological schema for noun phrases in German. In \ \ Gereon Müller, Lutz Gunkel, and Gisela Zifonun (eds.) \textit{Explorations in nominal \ \ inflection}, pp. 161-188. Berlin: Mouton de Gruyter.
\end{styleStandard}

\begin{styleStandard}
Katzir, Roni, and Tal Siloni. 2014. Agreement and definiteness in Germanic DPs. In Anna \ \ Bondaruk, Gréte Dalmi, and Alexander Grosu (eds.) \textit{Advances in the Syntax of DPs: \ \ Structure, agreement, and case}. [Linguistik Aktuell/Linguistics Today 217], 267-293. \ \ Amsterdam: John Benjamins.
\end{styleStandard}

\begin{styleStandard}
Keller, Rudi. 2004. Sprachwandel: Faszination Sprache – Herausforderung Übersetzung. Paper presented at \textit{Kongress des Bundesverbandes der Dolmetscher und Übersetzer e.V. 2000} [Conference of the Federal Union of Interpreters and Translators]. (available at: https://www.phil-fak.uni-duesseldorf.de/uploads/media/Sprachwandel.pdf)
\end{styleStandard}

\begin{styleStandard}
Kester, Ellen-Petra. 1996a. Adjectival inflection and the licensing of empty categories in DP. \textit{Journal of Linguistics} 32: 57-78.
\end{styleStandard}

\begin{styleStandard}
Kester, Ellen-Petra. 1996b. The Nature of Adjectival Inflection. Doctoral Dissertation, Utrecht University.
\end{styleStandard}

\begin{styleStandard}
Klein, Thomas. 2007. Von der semantischen zur morphologischen Steuerung. In Hans Fix (ed.), \ \ \textit{Beiträge zur Morphologie: Germanisch, Baltisch, Ostseefinnisch}. pp. 193-225. \ \ Amsterdam: John Benjamins Publishing Company.
\end{styleStandard}

\begin{styleStandard}
Klima, Edward S. 1964. Negation in English. In Jerry Fodor and Jerrold Katz (eds.) \textit{The \ \ structure of language: Readings in the philosophy of language}, pp. 246-323. Englewood \ \ Cliffs, NJ: Prentice Hall.
\end{styleStandard}

\begin{styleStandard}
Klockmann, Heidi. 2020. The article \textit{a(n)} in English quantifying expressions: A default marker \ \ of cardinality. \textit{Glossa: A Journal of General Linguistics} 5 (1): 85. 1-31.
\end{styleStandard}

\begin{styleStandard}
Kobele, Gregory M. \& Malte Zimmermann. 2012. Quantification in German. In E.L. Keenan \& D. Paperno (eds.) \textit{Handbook of Quantifiers in Natural Languages} [Studies in Linguistics and Philosophy 90], pp. 227-283. Springer.
\end{styleStandard}

\begin{styleStandard}
Kratzer, Angelika. 1995. Stage-level and Individual-level Predicates. In Gregory N. Carlson \& Francis Jeffry Pelletier (eds.) \textit{The Generic Book}, pp. 125-175. Chicago: The University of Chicago Press.
\end{styleStandard}

\begin{styleStandard}
Krause, Cornelia. 1999. Two Notes on Prenominal Possessors in German. \textit{MIT Working Papers in Linguistics} 33: 191-217.
\end{styleStandard}

\begin{styleStandard}
Krifka, Manfred. 2009. Case syncretism in German feminines: Typological, functional and \ \ structural aspects. \textstyledisplayname{Patrick O. Steinkrüger}\textstylecontributor{~}\textstylemetadataandcontributorsfont{and~}\textstyledisplayname{Manfred \textit{Krifka}}\textit{ }(eds.)\textit{ On Inflection}, pp. \ \ 141-171. Berlin: de Gruyter Mouton.
\end{styleStandard}

\begin{styleStandard}
Krischke, Wolfgang. 2012. Des Menschens Genitive: normabweichende Genitiv-\ \ \ \ \ \ \ \ Varianten bei schwachen Maskulina. \textit{Linguistik online} 53: 55–84.
\end{styleStandard}

\begin{styleStandard}
Kupferman, Lucien. 1991. Structure événementielle de l’alternance un/Ø devant les noms \ \ humains attributs. \textit{Langages} 101: 52-75.
\end{styleStandard}

\begin{styleStandard}
Lasnik, Howard. 2000. \textit{Syntactic Structures }revisited\textit{. Contemporary Lectures on Classic Transformational Theory.} Cambridge, Mass.: MIT Press.
\end{styleStandard}

\begin{styleStandard}
Lasnik, Howard \& Juan Uriagereka with Cedric Boeckx. 2005. \textit{A course in minimalist syntax. \ \ Foundations and Prospects}. Malden, Mass.: Blackwell.
\end{styleStandard}

\begin{styleStandard}
Lawrenz, Birgit. 1993. Apposition. Begriffsbestimmung und syntaktischer Status. Tübingen: \ \ Gunter Narr Verlag.
\end{styleStandard}

\begin{styleStandard}
Leiss, Elisabeth. 1994. Genus und Sexus. Kristische Anmerkungen zur Sexualisierung von\ \ \ \ Grammatik. \textit{Linguistische Berichte }152: 281-300.
\end{styleStandard}

\begin{styleStandard}
Leiss, Elisabeth. 1997. Genus im Althochdeutschen. In Elvira Glaser \& Michael Schäfer (eds.)\ \ \ \ \textit{Grammatica ianua artium. Festschrift für Rolf Bergmann}, pp. 33-48. Heidelberg: Winter.
\end{styleStandard}

\begin{styleStandard}
Lekakou, Marika, and Kriszta Szendrői. 2007. Eliding the Noun in Close Apposition, or
\end{styleStandard}

\begin{styleStandard}
Greek polydefinites Revisited. \textit{University College London Working Papers in Linguistics \ \ }19: 129-154.
\end{styleStandard}

\begin{styleStandard}
Lekakou, Marika, and Kriszta Szendrői. 2012. Polydefinites in Greek: Ellipsis, close
\end{styleStandard}

\begin{styleStandard}
apposition and expletive determiners. \textit{Journal of Linguistics }48(1): 107-149.
\end{styleStandard}

\begin{styleStandard}
Leu, Thomas. 2005. Something invisible in English. \textit{UPenn Working Papers in Linguistics} 11 \ \ (1): 143-55.
\end{styleStandard}

\begin{styleStandard}
Leu, Thomas. 2007. These HERE demonstratives. \textit{U. Penn Working Papers in Linguistics} 13 (1): \ \ 141-154.
\end{styleStandard}

\begin{styleSubtitle}
\textmd{\textup{Leu, Thomas. 2008a. The Internal Syntax of Determiners. Doctoral Dissertation, New York \ \ University.}}
\end{styleSubtitle}

\begin{styleStandard}
Leu, Thomas. 2008b. \textit{What for} Internally. Syntax 11(1): 1-25.
\end{styleStandard}

\begin{styleStandard}
Leu, Thomas. 2015. \textit{The Architecture of Determiners}. New York: Oxford University Press.
\end{styleStandard}

\begin{styleStandard}
Lieber, Rochelle. 1988. Phrasal Compounds in English and the Morphology-Syntax Interface. \ \ Papers from the Parasession on Agreement in Grammatical Theory. \textit{Chicago Linguistic \ \ Society} 24: 202-22.
\end{styleStandard}

\begin{styleStandard}
Lindauer, Thomas. 1995. \textit{Genitivattribute. Eine morphologische Untersuchung zum deutschen DP/NP-System}. Tübingen: Max Niemeyer Verlag.
\end{styleStandard}

\begin{styleStandard}
Link, Godehard. 1998. Ten years of research on plurals – where do we stand? In Fritz Hamm \& \ \ Erhard Hinrichs (eds.) \textit{Plurality and quantification. }[Studies in Linguistics and \ \ Philosophy 69], pp. 19-54. Dordrecht: Kluwer.\newline
Lobeck, Anne. 1995. Ellipsis: Functional heads, licensing, and identification. Oxford: Oxford \ \ University Press.
\end{styleStandard}

\begin{styleStandard}
Löbel, Elisabeth. 1989. Q as a Functional Category. In Christa Bhatt, Elisabeth Löbel \& Claudia \ \ Schmidt (eds.) \textit{Syntactic Phrase Structure Phenomena in Noun Phrases and Sentences}, \ \ pp. 133-58. Amsterdam: John Benjamins Publishing Company.
\end{styleStandard}

\begin{styleStandard}
Löbel, Elisabeth. 1990a. D and Q als funtionale Kategorien in der Nominalphrase. \textit{Linguistische\ \ \ \ Berichte} 127: 232-264.
\end{styleStandard}

\begin{styleStandard}
Löbel, Elisabeth. 1990b. Typologische Aspekte funktionaler Kategorien in der Nominalphrase. \ \ \textit{Zeitschrift für Sprachwissenschaft} 9: 135-169.
\end{styleStandard}

\begin{styleStandard}
Löbel, Elisabeth. 1991. Apposition und das Problem der Kasuszuweisung und \ \ Adjazenzbedingung in der Nominalphrase des Deutschen. In Gisbert Fanselow \& Sascha \ \ Felix (eds.) \textit{Strukturen und Merkmale syntaktischer Kategorien}, pp. 1-32. Tübingen: \ \ Gunter Narr Verlag.
\end{styleStandard}

\begin{styleStandard}
Lockwood, W. B. 1968. \textit{Historical German Syntax. }Oxford: Clarendon.
\end{styleStandard}

\begin{styleStandard}
Lockwood, W. B. 1995. \textit{Lehrbuch der modernen jiddischen Sprache}. Hamburg: Helmut Buske \ \ Verlag.
\end{styleStandard}

\begin{styleStandard}
Lohrmann, Susanne. 2010. The Structure of the DP and its Reflex in Scandinavian. Ph.D. \ \ dissertation, University of Stuttgart.
\end{styleStandard}

\begin{styleStandard}
Lohrmann, Susanne. 2011. A unified structure for Scandinavian DPs. In Petra Sleeman \& Harry \ \ Perridon (eds.) \textit{The Noun Phrase in Romance and Germanic: Structure, variation, and \ \ change}, pp. 111-125. Amsterdam: John Benjamins.
\end{styleStandard}

\begin{styleStandard}
Longobardi, Giuseppe. 1994. Reference and Proper Names: A Theory of N-Movement in Syntax \ \ and Logical Form. \textit{Linguistic Inquiry} 25 (4): 609-665.
\end{styleStandard}

\begin{styleSubtitle}
\textmd{\textup{Longobardi, Giuseppe. 2001. The Structure of DPs: Some Principles, Parameters, and \ \ Problems. In Mark Baltin \& Chris Collins (eds.) }}\textmd{The Handbook of Contemporary \ \ Syntactic Theory}\textmd{\textup{, pp. 562-603. Malden, MA: Blackwell Publishing Ltd.}}
\end{styleSubtitle}

\begin{styleStandard}
Lowe, John J. 2016. English possessive ’\textit{s}: clitic \textit{and} affix. \textit{Natural Language \& Linguistic \ \ Theory} 34: 157-195.
\end{styleStandard}

\begin{styleStandard}
Lühr, Rosemarie. 1991. Die deutsche Determinansphrase aus historischer Sicht. Zur Flexion \ \ von \textit{der}, \textit{die}, \textit{das} als Demonstrativpronomen, Relativpronomen und Artikel. \textit{Beiträge zur \ \ Geschichte der deutschen Sprache und Literatur }113(2): 195-211
\end{styleStandard}

\begin{styleStandard}
Lyons, Christopher. 1999. \textit{Definiteness}. Cambridge: Cambridge University Press.
\end{styleStandard}

\begin{styleStandard}
Maling, Joan \& Rex A. Sprouse. 1995. Structural Case, specifier-head relations, and the case of \ \ predicate NPs. In Hubert Haider, Susan Olsen \& Sten Vikner (eds) \textit{Studies in \ \ Comparative Germanic Syntax}, pp. 167-186\textit{.} Dordrecht: Kluwer.
\end{styleStandard}

\begin{styleStandard}
Marantz, Alec. 1997. No Escape from Syntax: Don’t Try Morphological Analysis in the Privacy \ \ of Your Own Lexicon. In A. Dimitriadis et al (eds.) Proceedings of the 21\textsuperscript{st} Penn \ \ Linguistics Colloquium. \textit{UPenn Working Papers in Linguistics} 4(2), pp. 201-225.
\end{styleStandard}

\begin{styleStandard}
Matushansky, Ora. 2002. A Beauty of a Construction. In Line Mikkelsen \& Christopher Potts \ \ (eds.) \textit{Proceedings of the 21}\textit{\textsuperscript{st}}\textit{ West Coast Conference on Formal Linguistics}, pp. 264-\ \ 277. Somerville, MA: Cascadilla.
\end{styleStandard}

\begin{styleStandard}
Matushanksy, Ora. 2006. Head Movement in Linguistic Theory. \textit{Linguistic Inquiry} 37(1): 69-\ \ 109.
\end{styleStandard}

\begin{styleStandard}
Matushansky, Ora. 2008. On the linguistic complexity of proper names. \textit{Linguistics and \ \ Philosophy }21: 573–627.
\end{styleStandard}

\begin{styleStandard}
Matushansky, Ora, and Benjamin Spector. 2005. Tinker, tailor, soldier, spy. In Emar Maier, \ \ Corien Bary \& Janneke Huitink (eds.) \textit{Proceedings of Sinn \& Bedeutung 9}, pp. 241-255. \ \ Nijmegen: NCS.
\end{styleStandard}

\begin{styleStandard}
Merchant, Jason. 1996. Object scrambling and quantifier float in German. \textit{NELS} 26: 179-193.
\end{styleStandard}

\begin{styleStandard}
Milner, Jean-Claude, and Judith Milner. 1972. \textit{La morphologie du groupe nominal en allemand}. \ \ DRLAV 2. Université de Paris VIII.
\end{styleStandard}

\begin{styleStandard}
Milsark, Gary. 1974. Existential Sentences in English. Doctoral Dissertation, MIT.
\end{styleStandard}

\begin{styleStandard}
Müller, Gereon. 2002a. Remarks on nominal inflection in German. In Ingrid Kaufmann \& \ \ Barbara Stiebels (eds.) \textit{More than Words. A Festschrift for Dieter Wunderlich}. Berlin: \ \ Akademie Verlag. pp. 113-145.
\end{styleStandard}

\begin{styleStandard}
Müller, Gereon. 2002b. Zwei Theorien der pronominalen Flexion im Deutschen. \textit{Deutsche \ \ Sprache} 30(4): 328-365.
\end{styleStandard}

\section[Müller, Michael. 1986. Zur Verbindbarkeit von Determinantien und Quantoren. In Heinz Vater \ \ (ed.) Zur Syntax der Determinantien, pp. 33{}-55. Tübingen: Gunter Narr Verlag. ]{\textmd{Müller, Michael. 1986. Zur Verbindbarkeit von Determinantien und Quantoren. In Heinz Vater \ \ (ed.) }\textmd{\textit{Zur Syntax der Determinantien}}\textmd{, pp. 33-55. Tübingen: Gunter Narr Verlag. }}
\begin{styleStandard}
Munn, Alan \& Cristina Schmitt. 2005. Number and indefinites. \textit{Lingua} 115: 821-855.
\end{styleStandard}

\begin{styleStandard}
Muñoz, Patrick. 2019. The proprial article and the semantics of names. \textit{Semantics and \ \ Pragmatics }12(6). 1–32. 
\end{styleStandard}

\begin{styleStandard}
Murphy, Andrew. 2018. Pronominal inflection and NP ellipsis in German. \textit{The Journal of \ \ Comparative Germanic Linguistics} 21(3): 327-379.
\end{styleStandard}

\begin{styleStandard}
Neeleman, Ad and Hans van de Koot. 2006. Syntactic Haplology. In Martin Everaert and Henk \ \ van Riemsdijk (eds.) \textit{The Blackwell Companion to Syntax}, Vol IV, pp. 685-710. Oxford: \ \ Blackwell Publishers.
\end{styleStandard}

\begin{styleStandard}
Neeleman, Ad, Hans van de Koot \& Jenny Doetjes. 2004. Degree expressions. \textit{The Linguistic \ \ \ \ Review} 21: 1-66.
\end{styleStandard}

\begin{styleStandard}
Nevins, Andrew. 2007. The representation of third person and its consequences for person-case \ \ effects. \textit{Natural Language \& Linguistic Theory} 25(2): 273–313. 
\end{styleStandard}

\begin{styleStandard}
Norris, Mark. 2014. A Theory of Nominal Concord. Doctoral Dissertation, University of \ \ California, Santa Cruz.
\end{styleStandard}

\begin{styleStandard}
Nübling, Damaris. 2011. Unter großem persönlich\textit{em} oder persönlich\textit{en} Einsatz? Der sprachliche \ \ Zweifelsfall adjektivischer Parallel- vs. Wechselflexion als Beispiel für aktuellen \ \ grammatischen Wandel. In Köpcke, Klaus Michael, and Arne Ziegler (eds.) \textit{Grammatik – \ \ Lehren, Lernen, Verstehen: Zugänge zur Grammatik des Gegenwartsdeutschen}, 175-195. \ \ Berlin/New York: de Gruyter.
\end{styleStandard}

\begin{styleStandard}
Nübling, Damaris, Fabian Fahlbusch, and Rita Heuser. 2015. \textit{Namen. Eine Einführung in die \ \ Onomastik}. (2nd ed.) Tübingen: Narr.
\end{styleStandard}

\begin{styleStandard}
Nunes, Jairo. 2001. Sideward Movement. \textit{Linguistic Inquiry} 32 (2): 303-344.
\end{styleStandard}

\begin{styleStandard}
Nykiel, Jerzy. 2015. The reduced definite article \textit{th’} in late Middle English and beyond: An \ \ insight from the definiteness cycle. \textit{Journal of Germanic Linguistics} 27(2): 105-144.
\end{styleStandard}

\begin{styleSubtitle}
\textmd{\textup{Olsen, Susan. 1987. Zum “substantivierten” Adjektiv im Deutschen: Deutsch als eine pro-Drop-Sprache. }}\textmd{Studium Linguistik}\textmd{\textup{ 21: 1-35.}}
\end{styleSubtitle}

\begin{styleSubtitle}
\textmd{\textup{Olsen, Susan. 1989a. AGR(eement) in the German Noun Phrase. In Christa Bhatt, Elisabeth Löbel \& Claudia Schmidt (eds.) }}\textmd{Syntactic Phrase Structure Phenomena in Noun Phrases and Sentences}\textmd{\textup{, pp. 39-49. Amsterdam: Benjamins.}}
\end{styleSubtitle}

\begin{styleSubtitle}
\textmd{\textup{Olsen, Susan. 1989b. Das Possessivum: Pronomen, Determinans oder Adjektiv? }}\textmd{Linguistische Berichte}\textmd{\textup{ 120: 133-153.}}
\end{styleSubtitle}

\begin{styleSubtitle}
\textmd{\textup{Olsen, Susan. 1991a. AGR(eement) und Flexion in der deutschen Nominalphrase. In Gisbert Fanselow \& Sascha Felix (eds.) }}\textmd{Strukturen und Merkmale syntaktischer Kategorien}\textmd{\textup{, pp. 51-70. Tübingen: Gunter Narr Verlag.}}
\end{styleSubtitle}

\begin{styleSubtitle}
\textmd{\textup{Olsen, Susan. 1991b. Die deutsche Nominalphrase as “Determinansphrase”. In Susan Olsen \& Gisbert Fanselow (eds.) }}\textmd{DET, COMP und INFL. Zur Syntax funktionaler Kategorien und grammatischer Funktionen}\textmd{\textup{, pp. 35-56. Tübingen: Niemeyer.}}
\end{styleSubtitle}

\begin{styleStandard}
Olsvanger, Immanuel. (ed). 1947. \textit{Röyte pomerantsen} (Red oranges). New York: Schocken \ \ Books.
\end{styleStandard}

\begin{styleFooter}
Oomen, Ingelore. 1977. \textit{Determination bei generischen, definiten und indefiniten \ \ Beschreibungen im Deutschen}. Tübingen: Niemeyer.
\end{styleFooter}

\begin{styleFooter}
Ott, Dennis. 2011a. Local Instability. The Syntax of Split Topics. Doctoral Dissertation, Harvard \ \ University.
\end{styleFooter}

\begin{styleFooter}
Ott, Dennis. 2011b. Diminutive-formation in German. \textit{The Journal of Comparative Germanic \ \ Linguistics} 14 (1): 1-46.
\end{styleFooter}

\begin{styleFooter}
Ott, Dennis \& Andreea Nicolae. 2010. The Syntax and Semantics of Genus-species Splits in \ \ German. Ms, Havard University. [available at: \url{http://ling.auf.net/lingBuzz/001071}]
\end{styleFooter}

\begin{styleStandard}
Pafel, Jürgen. 1994. Zur syntaktischen Struktur nominaler Quantoren. \textit{Zeitschrift für \ \ Sprachwissenschaft} 13(2): 236-275.
\end{styleStandard}

\begin{styleFooter}
Pafel, Jürgen. 1995. Kinds of Extraction from Noun Phrases. In Uli Lutz \& Jürgen Pafel (eds.) \ \ \textit{On Extraction and Extraposition in German}, pp. 145-177. Amsterdam: John Benjamins. 
\end{styleFooter}

\begin{styleStandard}
Pafel, Jürgen. 2005. \textit{Quantifier Scope in German}. [Linguistik Aktuell/Linguistic Today-Series \ \ 84]. Amsterdam: John Benjamins Publishing Company.
\end{styleStandard}

\begin{styleStandard}
Panagiotidis, Phoevos. 2000. Demonstrative determiners and operators: The case of Greek. \ \ \textit{Lingua} 110: 717-742.
\end{styleStandard}

\begin{styleStandard}
Panagiotidis, Phoevos. 2002. \textit{Pronouns, Clitics and Empty Nouns. ‘Pronominality’ and \ \ licensing in syntax}. Amsterdam: John Benjamins.
\end{styleStandard}

\begin{styleStandard}
Panagiotidis, Phoevos. 2003. Empty Nouns. \textit{Natural Language \& Linguistic Theory} 21(2): 381-\ \ 432.
\end{styleStandard}

\begin{styleStandard}
Partee, Barbara H. 1973. Some transformational extensions of Montague Grammar. In Barbara \ \ H. Partee (ed.) \textit{Montague Grammar}, pp. 51-76. New York: Academic Press.
\end{styleStandard}

\begin{styleStandard}
Paul, Hermann, Peter Wiehl, and Siegfried Grosse. 1989. \textit{Mittelhochdeutsche Grammatik}. (23rd \ \ ed.). Tübingen: Max Niemeyer Verlag.
\end{styleStandard}

\begin{styleStandard}
Perlmutter, David M. 1970. On the article in English. In Manfred Bierwisch \& Karl Erich Heidolph (eds.) \textit{Progress in Linguistics. A Collection of Papers}, pp. 233-248. The Hague: Mouton.
\end{styleStandard}

\begin{styleSubtitle}
\textmd{\textup{Pesetsky, David. 1978. Category Switching and So-called Pronouns. }}\textmd{Chicago Linguistic Society \ \ }\textmd{\textup{14: 350-361.}}
\end{styleSubtitle}

\begin{styleStandard}
Pesetsky, David \& Esther Torrego. 2007. The syntax of valuation and the interpretability of \ \ features. In Simin Karimi, Vida Samiian \& Wendy K. Wilkins (eds.) \textit{Phrasal and clausal }\textit{\ \ architecture: Syntactic derivation and interpretation. In honor of Joseph E. Emonds}, 262-\ \ 294. Amsterdam: John Benjamins Publishing Company.
\end{styleStandard}

\begin{styleStandard}
Peter, Klaus. 2013. Steuerungsfaktoren für Parallel- und Wechselflexion in Adjektivreihungen.\ \ \ \ \textit{Jahrbuch für Germanistische Sprachgeschichte} 4(1): 186-204.
\end{styleStandard}

\begin{styleStandard}
Petrova, Svetlana. 2024. On the distribution of the strong and weak adjectival inflection in Old \ \ High German: A corpus investigation. In Kristin Bech \& Alexander Pfaff (Eds.), \textit{Noun \ \ phrases in} \textit{early Germanic languages}, pp. 181 - 218. Berlin: Language Science Press.
\end{styleStandard}

\begin{styleStandard}
Pétursson, Magnús. 1992. \textit{Lehrbuch der Isländischen Sprache. Mit Übungen und Lösungen}. \ \ Hamburg: Helmut Buske Verlag.
\end{styleStandard}

\begin{styleSubtitle}
\textmd{\textup{Pfaff, Alexander. 2015. Adjectival and Genitival Modification in Definite Noun Phrases in\ \ \ \ Icelandic. A Tale of Outsiders and Inside Jobs. Doctoral dissertation, University of \ \ Tromsø.}}
\end{styleSubtitle}

\begin{styleSubtitle}
\textmd{\textup{Pfaff, Alexander. 2017.~Adjectival inflection as diagnostic for structural~position: inside and \ \ outside the Icelandic definiteness~domain.~}}\textmd{Journal of Comparative German Linguistics}\textmd{\textup{ \ \ 20: 283–322.}}
\end{styleSubtitle}

\begin{styleStandard}
Pfaff, Alexander. 2020. How to become an adjective when you're not strong (enough)? \textit{Nordlyd\ \ \ \ }44(1). 19-34.
\end{styleStandard}

\begin{styleStandard}
Pike, Kenneth L. 1963. Theoretical Implications of Matrix Permutation in Fore (New Guinea). \ \ \textit{Anthropological Linguistics} 5: 1-23.
\end{styleStandard}

\begin{styleStandard}
Pike, Kenneth L. 1965. Non-Linear Order and Anti-Redundancy in German Morphological \ \ Matrices. \textit{Zeitschrift für Mundartforschung} 31: 193-221.
\end{styleStandard}

\begin{styleStandard}
Plank, Frans. 2003. Double Articulation. In Frans Plank (ed.) \textit{Noun Phrase Structure in the \ \ Languages of Europe }[Empirical Approaches to Language Typology, EUROTYP 20-7], \ \ pp. 337-395. Berlin: Mouton de Gruyter.
\end{styleStandard}

\begin{styleSubtitle}
\textmd{\textup{Postal, Paul M. 1966. On the so-called pronouns in English, in F. Dinneen (ed.) }}\textmd{Nineteenth \ \ Monograph on Language and Linguistics}\textmd{\textup{, Georgetown University Press, Washington, \ \ D.C., pp. 201-24.}}
\end{styleSubtitle}

\begin{styleFooter}
Presslich, Marion. 2000. \textit{Partitivität und Indefinitheit: die Entstehung und Entwicklung des \ \ \ \ \ \ indefiniten Artikels in den germanischen und romanischen Sprachen am Beispiel des \ \ \ \ \ \ Deutschen, Niederländischen, Französischen und Italienischen}. Frankfurt a. Main: Lang.
\end{styleFooter}

\begin{styleSubtitle}
\textmd{\textup{Prokosch, Eduard. 1939. }}\textmd{A Comparative Germanic Grammar}\textmd{\textup{. Baltimore: The Linguistic Society \ \ of America. }}
\end{styleSubtitle}

\begin{styleStandard}
Pysz, Agnieszka. 2006. The structural location of adnominal adjectives: Prospects for Old \ \ English. \textit{Skase Journal of Theoretical Linguistics} 3(3): 59-85.
\end{styleStandard}

\begin{styleStandard}
Rankin, Jamie \& Larry D. Wells. 2016. \textit{Handbuch zur deutschen Grammatik. Wiederholen und Anwenden}. (6th ed.) Boston, MA: Cengage Learning.
\end{styleStandard}

\begin{styleStandard}
Rauh, Gisa. 2003. Warum wir Linguisten “euch Linguisten”, aber nicht “sie Linguisten” \ \ akzeptieren können. \textit{Linguistische Berichte} 196: 389-424.
\end{styleStandard}

\begin{styleStandard}
Rauh, Gisa. 2004. Warum “Linguist” in “ich/du Linguist” kein Schimpfwort sein muß. \ \ \textit{Linguistische Berichte} 197: 77-105.
\end{styleStandard}

\begin{styleStandard}
Rauth, Philipp, and Augustin Speyer. 2021. Adverbial reinforcers of demonstratives in dialectal \ \ German. \textit{Glossa: a journal of general linguistics} 6(1:4). 1-24.
\end{styleStandard}

\begin{styleStandard}
Reershemius, Gertrud. 1997. \textit{Biographisches Erzählen auf Jiddisch. Grammatische und \ \ diskursanalytische Untersuchungen}. Tübingen: Niemeyer.
\end{styleStandard}

\begin{styleStandard}
Rehn, Alexandra. 2019. \textit{Adjectives and the Syntax of German (ic) DPs. }Doctoral dissertation, \ \ University of Konstanz.
\end{styleStandard}

\begin{styleStandard}
Riemsdijk, Henk van. 1989. Movement and Regeneration. In Paola Benincà (ed.) \textit{Dialect Variation and the Theory of Grammar}, pp. 105-136. Dordrecht: Foris Publications.
\end{styleStandard}

\begin{styleStandard}
Riemsdijk, Henk van. 1998a. Head Movement and Adjacency. \textit{Natural Language and Linguistic Theory }16: 633-678.
\end{styleStandard}

\begin{styleStandard}
Riemsdijk, Henk van. 1998b. Categorial Feature Magnetism: The endocentricity and distribution \ \ of projections. \textit{The Journal of Comparative Germanic Linguists} 2: 1-48.
\end{styleStandard}

\begin{styleStandard}
Riemsdijk, Henk van. 2005. Silent nouns and the spurious indefinite article in Dutch. In Mila \ \ Vulchanova and Tor A. Åfarli (eds.) \textit{Grammar \& Beyond: Essays in honor of Lars \ \ Hellan}, pp. 163-178. Oslo: Novus Press.
\end{styleStandard}

\begin{styleFooter}
Ritter, Elizabeth. 1991. Two functional categories in noun phrases: evidence from Modern \ \ Hebrew. In Susan D. Rothstein (ed.) \textit{Perspectives on Phrase Structure: Heads and \ \ Licensing}. [Syntax and Semantics 25], pp. 37-62. San Diego: Academic Press.
\end{styleFooter}

\begin{styleSubtitle}
\textmd{\textup{Rizzi, Luigi. 1986. Null Objects in Italian and the Theory of }}\textmd{pro}\textmd{\textup{. }}\textmd{Linguistic Inquiry}\textmd{\textup{ 17(3): 501-\ \ 557.}}
\end{styleSubtitle}

\begin{styleStandard}
Roehrs, Dorian. 2002. The Boundary between Morphology and Syntax: Determiners \textit{Move} Into \ \ the Determiner Phrase. \textit{IULC Working Papers Online} 2: 1-46.
\end{styleStandard}

\begin{styleStandard}
Roehrs, Dorian. 2005. Pronouns are Determiners After All. In Marcel den Dikken \& Christina \ \ Tortora (eds.) \textit{The Function of Function Words and Functional Categories}, pp. 251-285. \ \ Amsterdam: John Benjamins.
\end{styleStandard}

\begin{styleStandard}
Roehrs, Dorian. 2006a. The Morpho-Syntax of the Germanic Noun Phrase: Determiners MOVE \ \ into the Determiner Phrase. Doctoral Dissertation, Indiana University.
\end{styleStandard}

\begin{styleStandard}
Roehrs, Dorian. 2006b. Pronominal Noun Phrases, Number Specifications, and Null Nouns. In \ \ Jutta M. Hartmann \& László Molnárfi (eds.) \textit{Comparative Studies in Germanic Syntax: \ \ From Afrikaans to Zurich German}, pp. 143-180. Amsterdam: John Benjamins.
\end{styleStandard}

\begin{styleStandard}
Roehrs, Dorian. 2008. Something Inner- and Cross-linguistically Different. \textit{The Journal of \ \ Comparative Germanic Linguistics} 11 (1): 1-42.
\end{styleStandard}

\begin{styleStandard}
Roehrs, Dorian. 2009a. \textit{Demonstratives and definite articles as nominal auxiliaries}. Amsterdam: \ \ John Benjamins.
\end{styleStandard}

\begin{styleStandard}
Roehrs, Dorian. 2009b. Inflectional Parallelism with German Adjectives. \textit{Interdisciplinary \ \ Journal for Germanic Linguistics and Semiotic Analysis }14 (2): 289-326
\end{styleStandard}

\begin{styleStandard}
Roehrs, Dorian. 2010. Demonstrative-reinforcer Constructions. \textit{The Journal of Comparative \ \ Germanic Linguistics }13 (3): 225-268.
\end{styleStandard}

\begin{styleStandard}
Roehrs, Dorian. 2011. Split NPs. \textit{Groninger Arbeiten zur germanistischen Linguistik} 53 (1), \ \ edited by C. Jan-Wouter Zwart, 79-113.
\end{styleStandard}

\begin{styleStandard}
Roehrs, Dorian. 2012. Complex Determiners: A Case Study of German \textit{ein jeder}.\ \ \ \ \textit{Interdisciplinary Journal for Germanic Linguistics and Semiotic Analysis} 17 (1): 1-56.
\end{styleStandard}

\begin{styleStandard}
Roehrs, Dorian. 2013a. The Inner Makeup of Definite Determiners: The Case of Germanic. \ \ \ \ \textit{Journal of Germanic Linguistics} 25 (4): 295-411.
\end{styleStandard}

\begin{styleStandard}
Roehrs, Dorian. 2013b. Possessives as Extended Projections. \textit{Working Papers in Scandinavian \ \ Linguistics} 91, edited by Christer Platzack, 37-112.
\end{styleStandard}

\begin{styleStandard}
Roehrs, Dorian. 2015. Inflections on Pre-nominal Adjectives in Germanic: Main Types, \ \ Subtypes, and Subset Relations. \textit{The Journal of Comparative Germanic Linguistics }18 \ \ (3): 213-271.
\end{styleStandard}

\begin{styleStandard}
Roehrs, Dorian. 2019. Three Typological Differences between the North and the West Germanic \ \ DP. \textit{Journal of Germanic Linguistics} 31 (4): 363-408.
\end{styleStandard}

\begin{styleStandard}
Roehrs, Dorian. 2020a. The Left Periphery of the German Noun Phrase. \textit{Studia Linguistica} 78 \ \ (1): 98-138.
\end{styleStandard}

\begin{styleStandard}
Roehrs, Dorian. 2020b. The Structure of Noun (NP) and Determiner Phrases (DP). In Michael T. \ \ Putnam \& B. Richard Page (eds.) \textit{The} \textit{Cambridge Handbook of Germanic Linguistics}, pp. \ \ 537-563. Cambridge: Cambridge University Press.
\end{styleStandard}

\begin{styleStandard}
Roehrs, Dorian. 2021. Semantically Vacuous Elements in German: Adjectival Inflections and the \ \ Article \textit{ein}. Unpublished book manuscript, version 2 (available on Lingbuzz: \ \ https://ling.auf.net/lingbuzz/002457)
\end{styleStandard}

\begin{styleStandard}
Roehrs, Dorian. 2022. Prenominal Possessives in Yiddish: \textit{mayn khaver} vs. \textit{mayner a khaver}. \ \ \textit{Linguistics: An Interdisciplinary Journal of the Language Sciences} 60 (1).
\end{styleStandard}

\begin{styleStandard}
Roehrs, Dorian \& Marit Julien. 2014. Adjectives in German and Norwegian: Differences in weak \ \ and strong inflections. In Petra Sleeman, Freek Van de Velde, \& Harry Perridon (eds.) \ \ \textit{Adjectives in Germanic and Romance}. [Linguistik Aktuell/Linguistic Today-Series 212], \ \ pp. 245-261. Amsterdam: John Benjamins Publishing Company.
\end{styleStandard}

\begin{styleStandard}
Rubin, Edward J. 1996. The Transparent Syntax and Semantics of Modifiers. \textit{Proceedings of the \ \ 15}\textit{\textsuperscript{th}}\textit{ WCCFL}, pp. 429-439.
\end{styleStandard}

\begin{styleStandard}
Russ, Charles V. J. (ed.). 1989. \textit{The Dialects of Modern German. A linguistic survey}. Stanford, \ \ CA: Stanford University Press.
\end{styleStandard}

\begin{styleStandard}
Sahel, Said. 2021. Parallel- vs. Wechselflexion im DAT SG MASC/NEUTR – Ein \ \ Erklärungsansatz zur Persistenz der Variation. \textit{Zeitschrift für germanistische Linguistik} \ \ 49(1): 3-32.
\end{styleStandard}

\begin{styleStandard}
Salmons, Joseph. 2021. \textit{Sound Change}. Edinburgh: Edinburgh University Press.
\end{styleStandard}

\begin{styleStandard}
Salzmann, Martin. 2020. The NP vs. DP debate. Why previous arguments are inconclusive and \ \ what a good argument could look like. Evidence from agreement with hybrid nouns. \ \ \textit{Glossa: a journal of general linguistics} 5(1). 83. 1-46.
\end{styleStandard}

\begin{styleStandard}
Salzmann, Martin. 2022. The NP vs. DP-debate and notions of headedness. In Ulrike Freywald, \ \ Horst J. Simon, and Stefan Müller (eds.) \textit{Headedness and/or grammatical anarchy?} 55-\ \ 71. Berlin: Language \ \ Science Press.
\end{styleStandard}

\begin{styleStandard}
Sapp, Christopher \& Dorian Roehrs. 2016. Head-to-Modifier Reanalysis: The Rise of the \ \ Adjectival Quantifier \textit{viel} and the Loss of Genitive Case Assignment. \textit{Journal of \ \ Germanic Linguistics} 28 (2): 89-166.
\end{styleStandard}

\begin{styleStandard}
Sauerland, Uli. 1996. The Late Insertion of Germanic Inflection. Ms., MIT.
\end{styleStandard}

\begin{styleStandard}
Schirmunski, Viktor. M. 2010. \textit{Deutsche Mundartkunde. Vergleichende Laut- und Formenlehre \ \ der deutschen Mundarten} (original version from 1962 edited by Larissa Naiditsch). \ \ Frankfurt am Main: Peter Lang.
\end{styleStandard}

\begin{styleStandard}
Schlenker, Philippe. 1999. La flexion de l’adjectif en allemand: la morphologie de haut en bas. \ \ \textit{Recherches Linguistiques de Vincennes} 28: 115-132. \ \ [http://rlv.revues.org/document1216.html]
\end{styleStandard}

\begin{styleNewTimesRoman}
Schoorlemmer, Erik. 2009. Agreement, Dominance and Doubling. The Morphosyntax of DP. \ \ Doctoral Dissertation, University of Leiden, the Netherlands [published by Utrecht: \ \ LOT].
\end{styleNewTimesRoman}

\begin{styleNewTimesRoman}
Schoorlemmer, Erik. 2012. Definiteness marking in Germanic: morphological variations on the \ \ same syntactic theme. \textit{The Journal of Comparative Germanic Linguistics} 15(2): 107-156.
\end{styleNewTimesRoman}

\begin{styleStandard}
Schrodt, Richard. 2004. \textit{Althochdeutsche Grammatik II}. Tübingen: Max Niemeyer Verlag.
\end{styleStandard}

\begin{styleSubtitle}
\textmd{\textup{Schwartz, Bonnie D. \& Sten Vikner. 1996. The verb always leaves IP in V2 clauses. In Adriana \ \ Belletti \& Luigi Rizzi (eds.) }}\textmd{Parameters and Functional Heads: Essays in Comparative \ \ Syntax}\textmd{\textup{, pp. 11-62. New York: Oxford University Press.}}
\end{styleSubtitle}

\begin{styleStandard}
Sigurðsson, Halldór Ármann. 1989. Verbal Syntax and Case in Icelandic. In a Comparative GB \ \ Approach. Doctoral Dissertation, University of Lund.
\end{styleStandard}

\begin{styleStandard}
Sigurðsson, Halldór Ármann. 2006. The Icelandic Noun Phrase: Central traits. \textit{Arkiv för nordisk \ \ filologi} 121: 193-236.
\end{styleStandard}

\begin{styleStandard}
Sigurðsson, Halldór Ármann, and Jim Wood. 2020. “We Olaf”: Pro[(x-)XP] constructions in \ \ Icelandic and beyond. \textit{Glossa: A Journal of General Linguistics }5(1): 16. 1-16).
\end{styleStandard}

\begin{styleStandard}
Starke, Michal. 2009. Nanosyntax. A short primer to a new approach to language. \textit{Nordlyd} 36(1): \ \ 1–6.
\end{styleStandard}

\begin{styleStandard}
Steinmetz, Donald. 2001. The Great Gender Shift and the attrition of neuter nouns in West \ \ Germanic: The example of German. In Irmengard Rauch \& Gerald F. Carr (eds.) \textit{New \ \ Insights in Germanic II}, pp. 201-224. Frankfurt am Main: Verlag Peter Lang.
\end{styleStandard}

\begin{styleStandard}
Sternefeld, Wolfgang. 2004. Feature Checking, Case, and Agreement in German DPs. In Gereon \ \ Müller, Lutz Gunkel, and Gisela Zifonun (eds.) \textit{Explorations in Nominal Inflection}, pp. \ \ 269-299. Berlin: Mouton de Gruyter.
\end{styleStandard}

\begin{styleStandard}
Sternefeld, Wolfgang. 2008. \textit{Syntax. Eine morphologisch motivierte generative Beschreibung \ \ des Deutschen}. Vol I. Tübingen: Stauffenberg.
\end{styleStandard}

\begin{styleStandard}
Stowell, Tim. 1989. Subjects, Specifiers, and X-Bar Theory. In Mark R. Baltin \& Anthony S. Kroch (eds.) \textit{Alternative Conceptions of Phrase Structure}, pp. 232-262. Chicago: The University of Chicago Press.
\end{styleStandard}

\begin{styleStandard}
Stowell, Tim. 1991. Determiners in NP and DP. In Katherine Leffel \& Denis Bouchard (eds.) \textit{Views on Phrase Structure}, pp. 37-56, Kluwer, Netherlands.
\end{styleStandard}

\begin{styleStandard}
Studler, Rebekka. 2011. \textit{Artikelparadigmen: Form, Funktion und syntaktisch-semantische \ \ Analyse von definiten Determinierern im Schweizerdeutschen. }Doctoral Dissertation,\textbf{\textit{ \ \ }}University of Zurich.
\end{styleStandard}

\begin{styleStandard}
Stump, Gregory T. 2001. \textit{Inflectional Morphology. A Theory of Paradigm Structure}. Cambridge: \ \ Cambridge University Press.
\end{styleStandard}

\begin{styleStandard}
Svenonius, Peter. 1993. The Structural Location of the Attributive Adjective. In Erin Duncan, \ \ Donka Farkas \& Philip Spaelti (eds.) \textit{The Proceedings of the Twelfth WCCFL}, pp. 439-\ \ 454. Stanford Linguistics Association.
\end{styleStandard}

\begin{styleStandard}
Swart, Henriëtte de, Yoad Winter, and Joost Zwarts. 2005. Bare Predicate Nominals in Dutch. In \ \ Emar Maier, Corien Bary \& Janneke Huitink (eds.) \textit{Proceedings of Sinn \& Bedeutung 9}, \ \ pp. 446-460. Nijmegen: NCS.
\end{styleStandard}

\begin{styleStandard}
Swart, Henriëtte de, Yoad Winter, and Joost Zwarts. 2007. Bare Nominals and Reference to \ \ Capacities. \textit{Natural Language \& Linguistic Theory} 25 (1): 195-222.
\end{styleStandard}

\begin{styleStandard}
Szabolsci, Anna. 1994. The Noun Phrase. In Ferenc Kiefer \& Katalin É. Kiss (eds.) \textit{Syntax and \ \ Semantics 27. The Syntactic Structure of Hungarian}, pp. 179-274, San Diego: Academic \ \ Press.
\end{styleStandard}

\begin{styleStandard}
Tappe, Hans-Thilo. 1989. A Note on Split Topicalization in German. In Christa Bhatt, Elisabeth \ \ Löbel \& Claudia Schmidt (eds.) \textit{Syntactic Phrase Structure Phenomena in Noun Phrases \ \ and Sentences}, pp. 159-179. Amsterdam: John Benjamins Publishing Company.
\end{styleStandard}

\begin{styleStandard}
Taraldsen, Knut Tarald. 1990. D-projections and N-projections in Norwegian. In Joan Mascaró \ \ \& Marina Nespor (eds.) \textit{Grammar in Progress. Glow Essays for Henk van Riemsdijk}, pp. \ \ 419-431. Dordrecht: Foris Publications.
\end{styleStandard}

\begin{styleStandard}
Thráinsson, Höskuldur. 1994. Icelandic. In Ekkehard König and Johan van der Auwera (eds.) \ \ \textit{The Germanic Languages}, pp. 142-89. New York: Routledge.
\end{styleStandard}

\begin{styleStandard}
Thráinsson, Höskuldur. 2007. \textit{The Syntax of Icelandic}. Cambridge: Cambridge University Press.
\end{styleStandard}

\begin{styleStandard}
Trommer, Jochen. 2005. Die formale Repräsentation von Markiertheit in der Flexion des \ \ Deutschen. Paper presented at Linguistischer Arbeitskreis, Köln 2005.
\end{styleStandard}

\begin{styleStandard}
Vater, Heinz. 1982. Der „unbestimmte“ Artikel als Quantor. In Werner Welte (ed.) \textit{Sprachtheorie und Angewandte Linguistik: Festschrift für Alfred Wollmann zum 60. Geburtstag}, pp. 67-74. Tübingen: Narr.
\end{styleStandard}

\begin{styleStandard}
Vater, Heinz. 1984. Determinantien und Quantoren im Deutschen. \textit{Zeitschrift für Sprachwissenschaft} 3(1): 19-42.
\end{styleStandard}

\begin{styleStandard}
Vater, Heinz. 1985. Determinantien und Pronomina. In Angelika Redder (ed.) \textit{Deutsche Grammatik II}. \textit{Osnabrücker Beiträge zur Sprachtheorie} 30, pp. 107-126. Osnabrück: Universität Osnabrück.
\end{styleStandard}

\begin{styleStandard}
Vater, Heinz. 1991. Determinatien in der DP. In Susan Olsen \& Gisbert Fanselow (eds.) \textit{DET, COMP und INFL. Zur Syntax funktionaler Kategorien und grammatischer Funktionen}, pp. 15-34. Tübingen: Niemeyer.
\end{styleStandard}

\begin{styleStandard}
Vater, Heinz. 2000. “Pronominantien” – oder: Pronomina sind Determinantien. In Rolf Thieroff, \ \ Matthias Tamrat, Nanna Fuhrhop \& Oliver Teuber (eds.) \textit{Deutsche Grammatik in Theorie \ \ und Praxis}, pp. 185-99. Tübingen: Max Niemeyer Verlag.
\end{styleStandard}

\begin{styleStandard}
Vater, Heinz. 2002. The word class ‘Article.’ In D. Alan Cruse, Franz Hundsnurscher, Michael \ \ Job \& Peter Rolf Lutzeier (eds.) \textit{Lexikologie: Ein internationales Handbuch zur Natur \ \ und Struktur von Wörtern und Wortschätzen}, (Vol. 1), pp. 621-28. Berlin: de Gruyter.
\end{styleStandard}

\begin{styleFooter}
Velde, Freek van de. 2011. Anaphoric adjectives becoming determiners. In Petra Sleeman \& \ \ Harry Perridon (eds.) \textit{The Noun Phrase in Romance and Germanic: Structure, variation, \ \ and change}, pp. 241-256. Amsterdam: John Benjamins.
\end{styleFooter}

\begin{styleStandard}
Vergnaud, Jean-Roger \& Maria Luisa Zubizaretta. 1992. The Definite Determiner and the Inalienable Constructions in French and English. \textit{Linguistic Inquiry} 23(4): 595-652.
\end{styleStandard}

\begin{styleStandard}
Vogel, Petra M. 2006. {\textquotedbl}Ich hab da nen kleines Problem!{\textquotedbl} Zur neuen Kurzform des indefiniten \ \ Artikels im Deutschen. \textit{Zeitschrift für Dialektologie und Linguistik}. 73(2): 176-193.
\end{styleStandard}

\begin{styleStandard}
Watanabe, Akira. 2006. Functional Projections of Nominals in Japanese: Syntax of Classifiers. \ \ \textit{Natural Language and Linguistic Theory} 24(1): 241-306.
\end{styleStandard}

\begin{styleStandard}
Wegener, Heide. 1995. \textit{Die Nominalflexion des Deutschen – verstanden als Lerngegenstand}. \ \ Tübingen: Niemeyer.
\end{styleStandard}

\begin{styleStandard}
Weinreich. Uriel. 1999. \textit{College Yiddish}. (6\textsuperscript{th} ed.) New York City: YIVO Institute for Jewish Research.
\end{styleStandard}

\begin{styleStandard}
Wesener, Thomas. 1999. Production strategies in German spontaneous speech: definite and \ \ indefinite articles. In John J. Ohala, Yoko Hasegawa, Manjari Ohala, Daniel Granville, \ \ and Ashlee C. Bailey (eds.) \textit{14}\textit{\textsuperscript{th}}\textit{~International Congress of Phonetic Sciences}, 687-690. \ \ San Francisco, CA. (http://www.internationalphoneticassociation.org/icphs/icphs1999)
\end{styleStandard}

\begin{styleStandard}
Wiese, Bernd. 1996. Iconicity and Syncretism. On Pronominal Inflection in Modern German. In \ \ Sackmann, Robin (ed.) \textit{Theoretical Linguistics and Grammatical Description. Papers in \ \ Honour of Hans-Heinrich Lieb. On the occasion of his 60th birthday}. [Current Issues in \ \ Linguistic Theory 138], 323-344. Amsterdam/Philadelphia: Benjamins.
\end{styleStandard}

\begin{styleStandard}
Wiese, Bernd. 1999. Unterspezifizierte Paradigmen: Form und Funktion in der pronominalen \ \ Deklination. \textit{Linguistik Online 4, 3/99}.
\end{styleStandard}

\begin{styleStandard}
Wiese, Bernd. 2009. Variation in der Flexionsmorphologie: Starke und schwache \ \ Adjektivflexion nach Pronominaladjektiven. In Konopka, Marek \& Bruno Strecker \ \ (eds.)~\href{https://pub.ids-mannheim.de/laufend/jahrbuch/jb2008.html}{\textstyleInternetlink{\textit{Deutsche Grammatik – Regeln, Normen, Sprachgebrauch}}}. [\href{https://pub.ids-mannheim.de/laufend/jahrbuch/}{\textstyleInternetlink{Jahrbuch des Instituts \ \ für deutsche Sprache}}~2008], 166-194. Berlin/New York: de Gruyter.
\end{styleStandard}

\begin{styleFooter}
Wiese, Heike \& Joan Maling. 2005. Beers, kaffi, and schnaps: Different grammatical options for \ \ Restaurant Talk coercions in three Germanic languages. \textit{Journal of Germanic Linguistics} \ \ 17(1): 1-38.
\end{styleFooter}

\begin{styleFooter}
Wiese, Richard. 1988. The Proper Treatment of Inflection in the German Article System. \textit{Wiener \ \ Linguistische Gazette}, Beiheft [Supplement] 7: 32-4.
\end{styleFooter}

\begin{styleFooter}
Wiese, Richard. 1996a. \textit{The Phonology of German}. Oxford: Oxford University Press.
\end{styleFooter}

\begin{styleStandard}
Wiese, Richard. 1996b. Phrasal Compounds and the Theory of Word Syntax. \textit{Linguistic Inquiry} \ \ 27(1): 183-93. 
\end{styleStandard}

\begin{styleTextbodyindent}
Williams, Edwin S. 1982. Another Argument that Passive is Transformational. \textit{Linguistic \ \ Inquiry} 13(1): 160-3.
\end{styleTextbodyindent}

\begin{styleSubtitle}
\textmd{\textup{Wiltschko, Martina. 1998. On the Syntax and Semantics of (Relative) Pronouns and \ \ Determiners. }}\textmd{Journal of Comparative Germanic Linguistics}\textmd{\textup{ 2: 143-181.}}
\end{styleSubtitle}

\begin{styleStandard}
Winter, Yoad. 2005. On some problems of (in)definiteness within flexible semantics. \textit{Lingua} \ \ 115: 767-786.
\end{styleStandard}

\begin{styleStandard}
Wood, Johanna L. 2007. Demonstratives and possessives. In Elisabeth Stark, Elisabeth Leiss, \ \ and Werner Abraham (eds) \textit{Nominal determination. Typology, context constraints, and \ \ historical emergence}, pp. 339-361. Amsterdam: John Benjamins.
\end{styleStandard}

\begin{styleStandard}
Wood, Johanna L. \& Sten Vikner. 2011. Noun phrase structure and movement. A cross-linguistic \ \ comparison of \textit{such}/\textit{sådan}/\textit{solch} and \textit{so}/\textit{så}/\textit{so}. In Petra Sleeman \& Harry Perridon (eds.) \ \ \textit{The Noun Phrase in Romance and Germanic: Structure, variation, and change}, pp. 89-\ \ 109. Amsterdam: John Benjamins.
\end{styleStandard}

\begin{styleStandard}
Wood, Johanna \& Sten Vikner. 2013. What’s to the left of the indefinite article? – Et sådan et \ \ spørgsmål er svært at svare på. In \textit{Gode ord er bedre end guld}. Festskrift til Henrik \ \ Jørgensen, Simon Borchmann, Inger Schoonderbeek Hansen, Tina Thode Hougaard, Ole \ \ Togeby \& Peter Widell (eds.), 515-540. Aarhus: Aarhus University. 
\end{styleStandard}

\begin{styleStandard}
Wunderlich, Dieter. 1997. Der unterspezifizierte Artikel. In C. Dürscheid, K. H. Ramers \& M.\ \ \ \ Schwarz (eds.) \textit{Sprache im Fokus}, 47-55. Niemeyer, Tübingen,.
\end{styleStandard}

\begin{styleTextbodyindent}
Zamparelli, Roberto. 2000. \textit{Layers in the Determiner Phrase}. New York: Garland Publishing.
\end{styleTextbodyindent}

\begin{styleStandard}
Zamparelli, Roberto. 2005. Introduction: some questions about (in)definiteness. \textit{Lingua} 115: \ \ 759-766.
\end{styleStandard}

\begin{styleStandard}
Zamparelli, Roberto. 2008. Bare predicate nominals in Romance languages. In Henrik Høeg \ \ Müller \& Alex Klinge (eds.) \textit{Essays on Nominal Determination. From morphology to \ \ discourse management}, pp. 101-130. Amsterdam: John Benjamins.
\end{styleStandard}

\begin{styleStandard}
Zeijlstra, Hedde. 2011. On the syntactically complex status of negative indefinites. \textit{The Journal \ \ of Comparative Germanic Linguistics} 14 (2): 111-138.
\end{styleStandard}

\begin{styleStandard}
Zhang, Niina Ning. 2012. Countability and numeral classifiers in Mandarin Chinese. In Diane \ \ Massam (ed.), \textit{Count and mass across languages},\textit{ }pp. 220-237. Oxford: Oxford \ \ University Press.
\end{styleStandard}

\begin{styleStandard}
Zifonun, Gisela, Ludger Hoffmann \& Bruno Strecker. 1997. \textit{Grammatik der deutschen Sprache}. \ \ Berlin: Mouton de Gruyer.
\end{styleStandard}

\begin{styleStandard}
Zimmermann, Malte. 2011. On the functional architecture of DP and the feature content of \ \ pronominal quantifiers in Low German. \textit{The Journal of Comparative Germanic \ \ Linguistics} 14 (3): 203-40.
\end{styleStandard}

\begin{styleStandard}
Zwart, C. Jan-Wouter. 1997. \textit{Morphosyntax of Verb Movement. A Minimalist Approach to the \ \ Syntax of Dutch}. Dortrecht: Kluwer Academic Publishers.
\end{styleStandard}

\end{document}
