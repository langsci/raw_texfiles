\chapter{Conclusion}\label{conclusion}
\largerpage
The empirical study and results reported in the previous chapter provide answers to the questions developed throughout the thesis. In this concluding chapter, the theoretical foundations for the study, the results of the \isi{eye tracking} recordings and the questionnaire, the proposed set of basic \isi{placement} strategies, and the proposed workflow are revisited. The limitations of the study are discussed and follow-up studies and topics for future research suggested.

\section{Summary}\label{sec:9.1}

First of all, an overview of audiovisual translation with a focus in subtitling was given. The various kinds of subtitles were introduced and common challenges and shortcomings discussed – such as the paradox of aiming for invisibility. This includes the need for viewers to split their \isi{visual attention} between the processing of the image and reading the titles as well as other challenges. Conventional guidelines and traditional solutions for subtitles helped to define the gap integrated titles might be able to fill. Most of these guidelines handle time and space constraints and content translation. Problems with contrast, collisions with other text elements or relevant image areas, as well as interference with the \isi{overall image composition} seem to be widely accepted as unpleasant, but unavoidable features of subtitles (\chapref{overview}).

In a holistic approach, all kinds of text elements that can occur in films were analysed and discussed in order to define what \isi{graphical translation} strategies are already used and what could be applied in integrated titles. Out of a corpus of 100 recent and influential films, 52 were analysed to give an overview of \isi{graphical translation} strategies and their frequency. Additionally, \isi{typographic identity} was introduced as an overall concept emerging from the combination of the used typefaces, colours, and effects throughout a film. As subtitling proved to be the main strategy in translating text elements such as captions and displays, it has to be taken into account that this translation strategy can cause interferences with the \isi{shot composition}, the \isi{typographic identity} of the film, the focus of the audience, and – in the worst case – collisions with other relevant text elements or image areas (\sectref{sec:2.6}).

In search of an alternative subtitling concept that can avoid, or at least decrease these negative effects, more adaptive concepts were presented. Some guidelines for SDH seem already quite thoughtful, including colour-coded \isi{speaker identification} and sometimes horizontal displacement. Terms such as “\isi{abusive} subtitles” \citep{nornes1999}, “hybrid subtitles” (\citealt{Diaz_cintas2006}), “speaker-following subtitles” \citep{park2008}, “\isi{aesthetic} subtitles” \citep{Foerster2010}, “creative subtitles” (\citealt{mcclarty2012}), and “dynamic subtitles” (\citealt{Armstrong2014}) show the wide range of possible alternative concepts and approaches. It was decided to used the term “integrated titles” in order to include all these concepts, but also hint on the \isi{main focus} to ‘integrate’ the titles into the film’s image concept and \isi{layout}.

As the creation of the integrated titles should not be purely based on translation studies, film studies, \isi{communication design}, \isi{usability} studies, and computer sciences were included. Film studies offer a basis on \isi{image composition} and story telling, \isi{communication design} helps to define \isi{aesthetics} and creativity, \isi{usability} studies offer knowledge on \isi{user experience} and \isi{interface design}, and computer sciences can help with automation and software design. This combination of fields of study led to the creation of a set of required skills (basically the understandment of filmmaker’s intentions and the \isi{layout} and design of a film), and desirable characteristics of good integrated titles – \isi{intuitiveness}, usefulness, and satisfaction. While the design and \isi{typographic identity} of titles can be quite individual depending on the film, consistency is demanded in all mentioned characteristics. Therefore, the \isi{placement} strategies should be consistent and comprehensible (\chapref{integrated}).

To form a basis for the \isi{placement} strategies for integrated titles, two groups of commercially used integrated titles in films were analysed: Titles for the hearing-impaired audience with only horizontal variation (‘partially integrated titles’), and integrated titles targeted mainly at hearing audiences with the subcategories ‘partial translation’ and ‘complete translation’. The overview on the origins of the various integrated titles showed that the interference of professionals from the film business without any background in translation and subtitling brought this new course, and the analysis of the frequency of \isi{placement} strategies and shortcomings demonstrated that the positions differ much less between films than suspected: noise transcriptions are placed below their source or indicate a source’s position outside the frame. The same goes for a combined title of \isi{noise transcription} and dialogue. Titles for off-screen speakers are placed below a focus or \isi{focal point}, next to it or in \isi{indication of speaking direction}. One or more speakers offer placements below or next to the speaker, in \isi{speaking direction} and below or next to the focus in the image. \textit{John Wick} was the only film where the integrated titles were recreated with the same positions and \isi{layout} – in all other films, the titles were either deleted in the German image track, replaced by conventional subtitles, or accompanied by additional subtitles. While they exhibited a few incidents of unintended \isi{simultaneity}, made at times use of the weak top-centre position, collided with other relevant elements or areas, and were sometimes placed too far from the \isi{main focus}, the main shortcoming was inconsistency (\chapref{placement}).

Based on this theoretical framework and the two major analyses, the first workflow for the creation of integrated titles and a set of basic \isi{placement} strategies were developed and presented. It combines on one side the \isi{translation process}, including corrections and changes throughout the whole process, and on the other side the steps required for the creation of integrated titles: the analysis of the overall image concept and intention of the film, the operationalisation of \isi{placement} and \isi{layout} strategies, and the final application (\chapref{workflow}).

As the main study makes use of \isi{eye tracking}, an overview of relevant eye movements and functions was given as well as recent \isi{eye tracking} research in general, on reading behavior, and subtitling. A number of studies were presented that influenced the design of the present study and the phrasing of the research questions (\chapref{eyetracking}).

Based on the experiences gained from the pilot study in 2012, the main study made use of a Tobii TX300 \isi{eye tracker} and Adobe Premiere Pro was used to create the integrated titles. The first step was to analyse the film – this included an interview with the filmmaker, Pablo Romero-Fresco. The film was then translated into German and both traditional subtitles and the integrated titles were created, including \isi{layout} and \isi{placement}. Of the 45 participants, 31~were German native speakers and 14 English native speakers. The English native speakers watched the original version of the film material \textit{Joining the Dots}, providing information on the undistracted \isi{natural focus} without subtitles. This \isi{gaze} data did not only help in the \isi{placement} of the integrated titles but could also later be compared to the \isi{gaze} behavior of the German participants. Of these, 15 saw it with traditional subtitles and 16~German participants watched the film with the integrated titles. The participants who watched the version with the integrated titles were then asked to fill in a questionnaire on their experience (\chapref{method}).

The results of the main study are based on a number of eye movement measures and the short questionnaire. Concerning the eye movements, the \isi{mean fixation duration} in the \isi{title area} and image, the correspondence to the \isi{natural focus} of the natives, and the time to \isi{first fixation} (\isi{reaction time}) were analysed. All hypotheses were confirmed: The \isi{fixation duration} on the integrated titles decreased by about 14.4\,\% compared to traditional subtitles, and IT participants focused on the \isi{title area} on average 47.5\,\% compared to 51.6\,\% for the TS participants. The IT participants seemed to be more motivated to focus on the image and to leave the \isi{title area} directly after the reading process. Only about 16.5\,\% of all recorded times to \isi{first fixation} were 0, indication that the IT participants were less likely to focus on the \isi{title area} before the \isi{title display} (compared to 28.7\,\% for the TS participants). This also shows that the \isi{split attention} of the IT participants was stronger towards the image, resulting in more time for exploration and \isi{detail perception}. Based on a random sample, the focus points of the German natives were compared to those of the English natives. The IT participants were not only more likely to fixate the same areas (83.3\,\% compared to 75.3\,\% for TS participants) but also fixated on average almost all titles (98.2\,\% compare to 88.1\,\% for the TS participants). Thus, the focus of the IT participants resembled the \isi{natural focus} of the English natives stronger. As suspected, the \isi{reaction time} between \isi{title display} and \isi{first fixation} increased – from about 57~ms on average for the TS participants to on average 74~ms for the IT participants. This, however, can also be interpreted as a result of the longer \isi{image exploration} and stronger \isi{split attention} towards the image. Further possible approaches of interpretation that should be included in future studies are the effects of search efficiency, \isi{processing effort}, \isi{visual attention}, stress level, and cognitive load.

The IT participants rated the integrated titles, compared to their previous experiences with subtitles, as more \isi{aesthetic}, and they experienced a stronger \isi{information intake}. This supports the hypotheses that the \isi{placement} and adjusted \isi{layout} can have a positive effect on the audience and increase the \isi{information intake} and \isi{detail perception}. The majority of participants stated that they would like to use integrated titles in the future.


All in all, the evaluation of the \isi{eye tracking} data shows that integrated titles can decrease the \isi{reading time} and motivate the viewer to return to the \isi{natural focus} points faster. The \isi{title area} seems to be less likely to be fixated before the title actually fades in, and a random sample of ten scenes indicated that the focus of an audience using integrated titles is more likely to archive the \isi{natural gaze behaviour} of the native participants. The \isi{reaction time}, however, increased visibly.



Practical implications from this study arise for all areas of film translation. Film producers should be aware of the effects traditional subtitles can have on the film’s perception – especially in the light of top-grossing and award-winning Hollywood films making more profit in their translated versions than at home. The presented strategies and the proposed workflow offer new possibilities for filmmakers to have their work translated more respect- and artfully for the target audience. The number of recent projects on the creation of integrated titles, e.g. the integrated titles for the hearing-impaired audience for \textit{Notes on Blindness}, already show that the model is applicable in real-life settings.


\section{Limitations}\label{sec:9.2}

A number of limitations of the present study was already mentioned during the previous chapters or at least hinted at. The two film corpora used were far from perfect: The film corpus on text element translation did not include information on the analysed image track as this information is hard to come by – not even the client support of \textit{Amazon Prime} or \textit{Netflix} could clearly state the available image tracks for their films. As the example in \sectref{sec:2.6} concerning \textit{The Incredibles} showed, there can be quite some differences between the (original) English image track and the German image track. Furthermore, only 52 films of the 100 films in the corpus were analysed for this first study. A full analysis will provide more reliable data and a more comprehensive overview of the \isi{graphical translation} of text elements in film. The corpus on integrated titles for both hearing-impaired and hearing audiences can also hardly be seen as complete. As with the other corpus, there is no reliable information available on BDs that include horizontally placed titles for hearing-impaired audiences and there are no lists available on films that include integrated titles as translation of an \isi{additional language}. Therefore, this corpus can also offer an approximation of current \isi{placement} strategies in English films.



Concerning the main study, a number of small limitations have to be taken into account. The film material is quite specific being a rather static documentary with little action or motion. It should therefore not be seen as representative for all kinds of films and genres. The participants were mainly university students and are therefore also a highly specific group – results with more heterogeneous groups could be different. The questionnaire was only designed for and handed out to the IT participants, therefore not offering any comparable results to the TS participants. Additionally, it was phrased very positively and thus possibly influenced the answers of the participants. For future studies, a questionnaire should be designed in a way that it can be handed out to both groups and question the participants concerning more specific topics such as \isi{enjoyment}, transportation, and \isi{usability}.


\newpage 
The main limitations in this study, however, were caused by the used software and lacking programming skills to analyse the recorded \isi{eye tracking} data any other way. While \textit{Tobii Studio} definitely improved compared to the pilot study and \isi{scene selection} was possible – also with an immense expenditure of time – some measures could not be collected in a useful way and had therefore to be left out in the analysis. Mainly, this affected the analysis of \isi{saccade length}, as the data just was not completely clear on this topic and an unclear \isi{scene selection} had a too strong influence on it.



Concerning the creation of the integrated titles, software issues were also the main limitation. \textit{Premiere Pro} is, so far, just not intended to create integrated titles. While there are plug-ins and tools to import traditional subtitle files such as .srt, these cannot be edited as much as titles created by hand. Therefore, each title has to be created by hand, including copying the content, the start time, the end time of the title, colours, fonts, and effects. As long as there is no automation of at least the copying process from traditional subtitle file to individual titles in \textit{Premiere Pro}, the creation of integrated titles remains a massively time consuming process compared to the creation of traditional subtitles.



While the workflow was tested in real-life settings, no objective study of the process or the strategies has been undertaken yet.


\section{Outlook}\label{sec:9.3}

The insights into graphical aspects and effects as well as the results of this study are applicable to a range of other areas and future studies. First of all, the corpus on the \isi{graphical translation} of text elements can be analysed completely and the information on available image tracks added. This would offer not only a better overview of the handling of text elements but also reveal differences in translation and provide a broader basis for future research. Combined with the concept of \isi{typographic identity}, adjustments and improvements to current processes can be motivated. Additionally, a corpus on television series could reflect a more contemporary picture of translation, design, and \isi{layout} of text elements (see \textit{Sherlock} or \textit{House of Cards} [USA 2013-]).

Acknowledging today’s possibilities in the areas of \isi{machine translation}, post-editing, and automatic \isi{placement}, combinations of these modes can be researched and both the graphical and \isi{translation process} analysed. The basis for first studies could be the proposed workflow, the presented automation algorithms, and \isi{machine translation} specialised on subtitle contents.

\newpage
As mentioned in \sectref{sec:9.2}, the film genre of the used material was quite specific. Further studies can investigate whether there are genres that are especially suitable (or especially unsuitable) for integrated titles. The use of integrated titles in \textit{Sherlock Holmes – A Game of Shadows} in \citet{Kruger????b} indicated that even quick-paced scenes and dialogues can be handled with integrated titles.

Concerning the data analysis of eye movements when watching a film with integrated titles, there are more interesting measures future studies can investigate: the time frame after the \isi{title display}, and the overall eye movements during whole scenes (before, during, and after the \isi{title display}) pose interesting research opportunities. A differentiation between consecutive titles and stand-alone titles, the analysis of the individual \isi{placement}, \isi{saccade length} between \isi{main focus} and \isi{title area} as well as revisits and \isi{layout} aspects such as \isi{typeface}, contrast, and effects also offer more than enough interesting topics not only for \isi{eye tracking} studies.

Furthermore, the accessibility of integrated titles targeted at hearing-impaired audiences poses an interesting topic. Titles such as those created for \textit{Notes on Blindness} can be analysed regarding possible improvement, the audience’s \isi{enjoyment}, and \isi{information intake}. This includes for example the \isi{speaker identification} through colour that could be analysed based on \isi{AOIs} on title and speaker as well as questionnaires. Noise indication and hints on volume and general indication of noise sources could also deliver interesting \isi{eye tracking} data.

All in all, these first \isi{placement} strategies and workflow, possibly together with suitable software, can become a basic tool set for subtitle professionals and filmmakers that want to create integrated titles. This process might as well become part of the film production again instead of taking place outside and far away from the initial decision makers. The goal is to motivate a more respectful handling of both the film and the translation while considering aspects such as \isi{image composition}, content, and \isi{typographic identity}. This would allow for the titles to actually become a part of the film’s identity instead of interfering with it, providing a more authentic and accessible, and possibly more enjoyable, film experience for everyone.
