\addchap{Introduction}

While translation has long been an integral part of film production, beginning with the intertitles and title cards for silent films, it slowly shifted towards the \isi{post-production process} where it remains until today. It therefore has become highly unlikely that any consultation takes place between translator and filmmakers, pre-existing text elements are often not editable, and the potential for translation errors and \isi{layout} challenges has continuously risen – regarding both the \isi{dubbing} and subtitling process. Historically a \isi{dubbing country},\footnote{For the history of \isi{dubbing} in Germany, refer to \url{http://www.sprechersprecher.de/blog/die-geschichte-der-film-synchronisation-in-deutschland} [2014-12-30, in German].} Germany is not well-known for subtitled productions. But while \isi{dubbing} is obviously predominant in Germany and other neighbouring countries with a similar language-related history and a sufficiently large target audience, more and more German viewers prefer the original versions of English film productions.\footnote{This is reflected in the increasing number of screenings of original versions in German cinemas (see for example \url{http://www.koeln.de/kino/ov-filme} [2014-12-16, in German] and \url{http://against-dubbing.com/de/ovkinos/} [2014-12-16, in German]), especially since the introduction of digital film, changing from 35~mm film to digital projection. This allowed for a considerably easier and more cost-efficient process for film distributors (see \url{http://www.dw.de/der-35mm-film-stirbt-aus-kino-wird-digital/a-17013764} [2014-12-16, in German]).} Fans of series such as \textit{Game of Thrones} (USA/UK 2011-)\footnote{For better \isi{readability}, film credits (year and country of origin) will only be stated the first time a film is mentioned. The full credits for all mentioned films are listed in Appendix B.} or \textit{The Big Bang Theory} (USA 2007-) yearn for each new episode and many do not want to wait for the German dubbed version. Combined with the desire for a more authentic film experience, many German viewers prefer original and subtitled versions of their favourite show.\footnote{This assumption is supported by the increasing number of German internet forums that centre around creating and providing fansubs – subtitles created by fans – for download: \url{subcentral.de} with approximately 134 new posts per day, as well as \url{subtitles.de}, \url{tv4user.de}, and \url{opensubtitles.org} [2014-12-30].}


Traditional subtitling, however, can be seen as a strong intrusion into the original \isi{image composition} that can not only disrupt but also destroy the director’s intended \isi{shot composition} and focal points. But is the carefully composed interplay of image and sound not what makes film “the most popular art form” (\citealt{mercado2010}:~35) of today’s entertainment landscape? Long eye movements between focal points and subtitles decrease the viewer’s \isi{information intake}, and especially German audiences, who are often not used to subtitles, seem to prefer to wait for the next subtitle instead of looking back up again. Furthermore, not only the \isi{placement}, but also the overall design of traditional subtitles can disturb the \isi{image composition} – for instance titles with a weak contrast, inappropriate \isi{typeface} or irritating colour system. So should it not, despite the \isi{translation process}, be possible to preserve both image and sound as far as possible? Especially given today’s numerous artistic and technical possibilities and the huge amount of work that goes into the visual aspects of a film, taking into account not only special effects, but also typefaces, opening credits and text-image compositions. A further development of existing subtitling guidelines would not only express respect towards the original film version but also the translator’s work.
 
\largerpage
Nowadays, audiovisual translation, i.e. subtitling, is only part of the main production process in the case of multilingual films. And a few of these mainly English production use individually placed titles to translation an additional language in the film – for instance \textit{Man on Fire} (USA/UK 2004), \textit{Heroes} (USA 2006-2010), and \textit{Slumdog Millionaire} (UK/FR/USA 2008). These new concepts of subtitles are seen as “radical, popular re-conception of the subtitle's status” \citep{Kofoed2011} that “have been redirected from their customarily subsidiary, external position to become a central aspect of the filmic~mise-en-scene” (\citealt{Kofoed2011}). The idea therefore explored in this thesis is to retain the audio track – as usually done in traditional subtitling – but to place the “sub-”titles in a way that is based on the original image composition. These titles are to be integrated into the image in such a way as to create the best possible contrast and indicate speaking direction and speaker position. There are already several terms that attempt to grasp these new subtitling concepts and designs: the term of “abusive subtitles” (\citealt{nornes1999}:~17ff.) describes the experimental use of subtitles in regard to both graphical and linguistic aspects while the terms “hybrid” (\citealt{Diaz_cintas2006}:~51) and “creative” (\citealt{mcclarty2012}:~133ff.) subtitles focus on the overall presentation and are presented in opposition to traditionally placed and designed subtitles. Even though they do cover most of the differences between traditional subtitles and more recent concepts, these terms do not seem to apply to the titles used in the pilot study in \citet{Fox2012} and the present study as they might still refer to subtitles being automatically placed in the bottom (or top) area of the screen. Therefore, the term “integrated titles” (\citealt{Fox2012}:~1ff.) was used, referring to titles being integrated\footnote{Inspired by Bayram and Bayraktar who described “text information [that is placed] directly into the picture” (\citeyear{Bayram2012}:~82) as “integrated formats” (\citeyear{Bayram2012}:~82).} into the shot composition.
Nowadays, audiovisual translation, i.e. subtitling, is only part of the main production process in the case of multilingual films. And a few of these mainly English production use individually placed titles to translate an \isi{additional language} in the film – for instance \textit{Man on Fire} (USA/UK 2004), \textit{Heroes} (USA 2006-2010), and \textit{Slumdog Millionaire} (UK/FR/USA 2008). These new concepts of subtitles are seen as “radical, popular re-conception of the subtitle's status” \citep{Kofoed2011} that “have been redirected from their customarily subsidiary, external position to become a central aspect of the filmic~mise-en-scene” (\citealt{Kofoed2011}). The idea therefore explored in this thesis is to retain the audio track – as usually done in traditional subtitling – but to place the “sub-”titles in a way that is based on the original \isi{image composition}. These titles are to be integrated into the image in such a way as to create the best possible contrast and indicate speaker direction and position. There are already several terms that attempt to grasp these new subtitling concepts and designs: the term  “\isi{abusive} subtitles” (\citealt{nornes1999}:~17ff.) describes the experimental use of subtitles in regard to both graphical and linguistic aspects while the terms “hybrid” (\citealt{Diaz_cintas2006}:~51) and “creative” (\citealt{mcclarty2012}:~133ff.) subtitles focus on the overall presentation and are presented in opposition to traditionally placed and designed subtitles. Even though they do cover most of the differences between traditional subtitles and more recent concepts, these terms do not seem to apply to the titles used in the pilot study in \citet{Fox2012} and the present study as they might still refer to subtitles being automatically placed in the bottom (or top) area of the screen. Therefore, the term “integrated titles” (\citealt{Fox2012}:~1ff.) was used, referring to titles being integrated\footnote{Inspired by Bayram and Bayraktar who described “text information [that is placed] directly into the picture” (\citeyear{Bayram2012}:~82) as “integrated formats” (\citeyear{Bayram2012}:~82).} into the \isi{shot composition}.


So far, there has been no \isi{eye tracking} analysis of German integrated titles except for the first basic pilot study in \citet{Fox2012}. Additionally, no \isi{eye tracking} study has been conducted on the \isi{aesthetics} and perception of subtitles combined with attempts to draft a new, updated set of guidelines for more recent subtitling concepts.



The present \isi{eye tracking} study addressed whether the individual \isi{placement} and design of (sub)titles can increase the viewer’s \isi{reading time}, the time spent exploring the image (rather than waiting for the next title), shorten the eye movements between titles and focus points as well as improve the overall \isi{viewing experience}. Additional thought was given to indicate \isi{speech direction} and rate as well as \isi{speaker position}. Pablo Romero-Fresco gave his permission to use his short documentary \textit{Joining the Dots} (UK 2012) and agreed to discuss his \isi{image system} and shot compositions as a first step to creating the integrated titles. A total of 14~English native speakers watched the film without subtitles to define the \isi{natural focus}\footnote{The term 'natural' is used to describe the baseline data collected from the English native speakers as their eye movements were not affected by subtitles.} points and provide reference data. Fifteen native speakers of German with little or no knowledge of English watched the film with traditional subtitles and 16~additional German native speakers with little or no knowledge of English watched the film with integrated titles. The \isi{gaze behaviour} of the German participants was analysed in regard to reaction times, reading times, and the general \isi{visual attention distribution}.


\largerpage
Expected results are decreased reading times and a more \isi{natural gaze behaviour} with integrated titles. It is to be assumed that the \isi{reaction time} for integrated titles is slightly longer than for traditional titles. Due to the individual \isi{placement} of the titles, the distance between \isi{focal point} and title is on average smaller and the viewer would therefore gain more time to explore the image and focus on the focal points. Overall, expectations are that integrated titles will have a positive effect on both the \isi{aesthetic} \isi{viewing experience} of the audience and the \isi{split attention} between image and title, as integrated titles appear to motivate the viewer to return to the main \isi{focal point} faster and spend more time exploring the image in between titles. Based on the study described in the present thesis, the advantages and disadvantages of integrated titles will be discussed and the question answered whether these titles can improve the \isi{viewing experience}. The study aims to prove that consciously placed titles do a better job of maintaining the original \isi{image composition} and reduce the necessary eye movements between title and \isi{main focus} thus giving the viewer more time to explore the image. Additionally, integrated titles should have a positive effect on the \isi{aesthetic} \isi{viewing experience}.

\largerpage
The relevance of this research is seen not only in the increasing use of integrated titles in English film productions but in the fact that “even though these translation and accessibility services only account for 0.1\,\% – 1\,\% of the budget of an average film production \citep{Lambourne2012}, over half of the revenue of, for example, both top-grossing and award-winning Hollywood films comes from foreign territories” (\citealt{romero-fresco2013}:~202). Therefore, it is only in the interest of film producers to take a critical look at the perception of the translated version of their film and studies on more content- and image-related ways of audiovisual translation might be helpful in motivating this shift. And as a huge amount of work is invested nowadays in creating breath-taking effects from fonts to the \isi{image composition} to the special effects, both of technical and artistic nature, why not also invest in the translation of the film and uphold the same standard and respect through the last steps of the production process. Implementing efforts concerning audiovisual translation strategies and practices would mean a step closer to the original film and be indicative of the requisite respect to both the film material and the translation.

\chapref{audiovisual} of this thesis provides an overview of the fundamentals of audiovisual translation in general and subtitling in particular. It focuses on the relevant theoretical aspects, the different kinds of subtitling as well as traditional guidelines, specific challenges, and conventional solution strategies.

\chapref{overview} discusses text elements in film and their \isi{graphical translation}. Concerning film \isi{aesthetics} and design, the way the text elements are treated is one of the more obvious indicators for changes made during the \isi{translation process}. Not only do subtitles change films due to their additive character, but also the translation of already existing text elements such as captions, displays, inserts etc. can influence the reception of a film. Thus, the individual text elements and conventional \isi{graphical translation} strategies are presented. Furthermore, a first concept of \isi{typographic film identity} is introduced.

\chapref{integrated} then introduces the concept of integrated titles. This includes a historical overview, a discussion of existing similar terms, and the reasoning for the used term of ‘integrated titles’. Further fields of study such as \isi{communication design} and film studies are used to create a concept of required characteristics of integrated titles and the skills needed to create them. As there are already some examples of similar concepts, these are analysed with regard to the used \isi{placement} strategies in \chapref{placement}.

\chapref{workflow} puts together the theoretical framework discussed and created in the previous chapters and introduces a first workflow for the creation of integrated titles. The proposed three main steps of film material analysis, creation of \isi{placement} and \isi{layout} strategies, and the application process are presented.

\chapref{eyetracking} introduces \isi{eye tracking} as the main tool in this study. The general relevant functions of the eye are explained and, based on \isi{eye tracking} studies in reading, \isi{usability research}, and conventional subtitling, relevant eye movement measures are defined. Furthermore, relevant studies that served as inspiration for the experiment design are presented.

\chapref{method} then gives an overview over the method and experiment design. This includes the used \isi{eye tracking} setup, software, insights from the pilot study, and the resulting hypotheses. The main study is introduced in \chapref{sec:7.5} including the participant groups, film material, and individual adjustments to conventional subtitling strategies.

\largerpage
\chapref{results}, the recorded \isi{eye tracking} data and questionnaire data is presented and discussed. \chapref{conclusion} then presents the conclusion, discusses the limitations of the study, and gives an outlook on future work and studies.

