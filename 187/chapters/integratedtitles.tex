\chapter{Integrated titles}\label{integrated}

So why look for an alternative or improvement? What do subtitle concepts such as ‘integrated titles’ have to offer? The previous chapters illustrated that, while there has been much research and focus on content-related challenges, there is still much to learn concerning the reception and \isi{aesthetics} of subtitles, and text elements in film overall. Therefore, this chapter introduces integrated titles and insights from associated fields of study.

Technically speaking, the creation of individually placed titles has been possible for quite a while now. While the changeover from analogue to digital processing and the rise of non-linear video editing in the 90s made it possible not only for professionals but also for amateurs at home to edit film material, there were already ways of adding text into film before that (see \sectref{sec:1.1}). Nowadays, video editing software that allows to place text elements in videos (e.g. \textit{Adobe Premiere Pro} as example for a professional tool) or subtitle programmes such as \textit{Aegisub} that can create subtitle files with individual positions are sufficient to create integrated titles. For more elaborate purposes such as 3D titles that are placed in the depths of the image, software programmes such as \textit{Adobe After Effects} are more suitable.

While guidelines focused on SDH seem quite thoughtful – at least in terms of \isi{speaker identification} and \isi{horizontal placement} – traditional or “conventional” (\citealt{Foerster2010}:~81) subtitles are sometimes still seen as “a blemish on the film screen” (\citealt{Diaz_cintas2007}:~82) and “intrusion into the visual space of film” (\citealt{thompson2000}:~1) that have the “potential to ‘drown’ the images [and] instead of watching images, the audience starts literally to see only texts” (\citealt{sinha2004}:~174). This perception shows that, even though the content is at least as often under criticism – the authentic and \isi{aesthetic experience} of the audience is a relevant part of subtitling. The created split \isi{visual attention}, the distraction of inexperienced viewers particularly, and the traditionally thoughtless design (and often \isi{layout}) of subtitles call on the paradox of the demanded invisibility of translation, and therefore subtitles. In addition to well discussed challenges such as the afore-mentioned time and space constraints and the transfer of \isi{humour} and cultural-specific elements, the position and design of traditional subtitles can be seen as one of the major drawbacks. Especially viewers from countries that do not make major use of subtitling – mainly English speaking countries and those that have a long tradition of \isi{dubbing} films – might find it hard to focus on both reading the subtitles and perceiving the visual aspects of the plot. British director Danny Boyle, appreciated for films such as \textit{Trainspotting} (UK 1996) and \textit{Slumdog Millionaire}, described the problem by stating that “you don’t watch the film – you read the film and you scan occasionally to the actors” \citep{Beckman2008}. And he does not stand alone with this opinion. Rawsthorn describes conventional subtitles as “limply at the bottom of the screen” (\citeyear{rawsthorn2007}) and only legible “if you’re lucky” (\citealt{rawsthorn2007}). But “mostly, they’re not” (\citealt{rawsthorn2007}). She goes even further in saying that
\begin{quote}
subtitles are almost always badly designed. Illegible typefaces drift on- and off-screen at the wrong moments, lurking so low that the bottoms of the letters are chopped off, and obstructing the audience's view of gripping twists in the plot, or especially beautiful scenes. It doesn't seem to matter how good – or bad – the film is, the size of its budget, the quality of the cinematography, sets, costumes or titles, because the subtitles are still dire. Every other area of movie \isi{aesthetics} has a proud design history, except subtitling. (\citealt{rawsthorn2007})
\end{quote}
Stuart Comer, the curator of film at the Tate museum in London, gives one of the main reasons for these shortcomings: “Subtitling often takes place after the film is completed. It is not necessarily done by the director, and there is less quality control. That's why it can seem thoughtless” \citep{rawsthorn2007}. While it is understandable that before digital technology took over, the complex and time-consuming process of adding subtitles to film did not spark much interest in adding additional effort and costs to the design process, today’s processes are much more flexible. However, few filmmakers and producers have adjusted to this not so new reality. “Purely utilitarian” \citep{Vit2005} subtitles are predominant and keep “interfering with the beautifully framed and considered shot of the director” (\citeyear{Vit2005}). The demand for the cheapest possible solution might seem to be an understandable explanation – but how much might a slightly higher investment into subtitling actually weigh, especially compared to the horrendous costs of \isi{dubbing}? Combined with Romero-Fresco’s observation that “half of the revenue of […] both top-grossing and award-winning Hollywood films comes from foreign territories” (\citealt{romero-fresco2013}:~202) while “translation and accessibility services only account for 0.1\,\% – 1\,\% of the budget of an average film production \citep{Lambourne2012}” (\citeyear{romero-fresco2013}:~202), it can only be in the interest of film producers to take a critical look at the perception of the translated version of their film.

SDH guidelines already propose a more logical \isi{placement} of titles and incorporate colour and \isi{layout} into the relevant aspects that affect viewer’s perception. Various researchers, however, defined less traditional concepts of subtitling based on a quite contrary form of subtitling – fansubs. Of these, Nornes appears to be one of the first to discuss subtitles not only based on content but also \isi{layout} (\citeyear{nornes1999}). He recognises the criticism concerning the graphical intrusion of subtitles into the film image:
\begin{quote}
Spectators often find cinema’s powerful sense of mimesis muddied by subtitles, even by skillful ones. The original, foreign, object – its sights and its sounds – is available to all, but it is easily obscured by the graphic text through which we necessarily approach it. Thus, the opacity or awkwardness of subtitles easily inspires rage. (\citeyear{nornes1999}:~17)
\end{quote}
As Nornes considers it “likely that no one ever has come away from a foreign film admiring the translation” (\citeyear{nornes1999}:~17), he emphasises the need for exploring new methods such as what he calls “\isi{abusive}” subtitles (\citeyear{nornes1999}:~17). Abusive subtitles include not only “textual abuse” (\citealt{nornes1999}:~18) but also “graphical abuse” (\citeyear{nornes1999}:~18) that means “experimentation with […] visual qualities” (\citeyear{nornes1999}:~18). Nornes sees the roots of these experiments within Japanese cinema: Japanese subtitles can be placed “both horizontally and vertically” (\citeyear{nornes1999}:~21) and are therefore prone to \isi{abusive} application. Not only can subtitles therefore be modified in a way to provide a “graphic representation of the materiality of the speech” (\citealt{nornes1999}:~25, concerning the Japanese subtitles for the film \textit{M} [USA 1951]) but also following the \isi{image composition} of the film:
\begin{quote}
Furthermore, Japanese subtitlers routinely placed their titles in different areas of the scene depending on the cinematographer’s position. It was thought that the position of the words should complement mise-en-scène and movement. At the same time, there are indications that subtitle positioning depended upon narrative as well. One story from critic Yodogawa Nagaharu describes a dreamy Hollywood love scene where the subtitles appeared \textit{between} the two lovers. Of course! (\citealt{nornes1999}:~25)
\end{quote}
Even back then, Nornes realised that the possibilities for modifications concerning e.g. \isi{placement}, size, and colour of subtitles were not a problem of technical constraints and “nothing was preventing them [students in an exemplary class] from indulging in the most outrageous innovations” (\citeyear{nornes1999}:~31) but rather professionals and students being held back “by the inertia of convention and the ideology of corruption” (\citeyear{nornes1999}:~31). The advantage of early Japanese \isi{anime} fandom was the subtitler’s position “outside of the mainstream translation industry” (\citeyear{nornes1999}:~31) and following their “instinct” (\citeyear{nornes1999}:~31): They used “different colored subtitles”, additional definitions, footnotes, cultural explanations, and “different fonts, sizes, and colors to correspond to material aspects of language, from voice to dialect to written text within the frame” and they “freely insert their ‘subtitles’ all over the screen” (\citeyear{nornes1999}:~32). Nornes therefore defines this “abuse” as “directed at convention, even at spectators and their expectations” (\citeyear{nornes1999}:~32).

Díaz Cintas and Muñoz Sánchez seem to follow this definition when they discuss what they call “hybrid” subtitles (\citeyear{Diaz_cintas2006}:~51). While subtitles are supposed to “pass unnoticed to the viewer” (\citeyear{Diaz_cintas2006}:~47) and “be as invisible as possible” (\citeyear{Diaz_cintas2006}:~47), “some of these new conventions are nowadays making an appearance in the commercial versions of some programmes (\citealt{Diaz_cintas2005})” (\citeyear{Diaz_cintas2006}:~47). Ferrer Simó (\citeyear{Ferrer_simo2005}) provides the following “comprehensive list of key features that define fansubs” (\citeyear{Diaz_cintas2006}:~47):
\begin{quote}
Use of different fonts throughout the same programme.

Use of colours to identify different actors.

Use of subtitles of more than two lines (up to four lines).

Use of notes at the top of the screen.

Use of glosses in the body of the subtitles.

The position of subtitles varies on the screen (scenetiming).

Karaoke subtitling for opening and ending songs.

Adding of information regarding fansubbers.

Translation of opening and closing credits.
\end{quote}
While the use of colours for \isi{speaker identification}, subtitles longer than two lines, additional information, and varying positions are similar to the SDH guidelines presented earlier, fansubs distinguish themselves through several additional features such as varying fonts, glosses, and karaoke subtitles. They are “a hybrid resorting to conventions used both in subtitling for the hearing as well as in subtitling for the \isi{deaf} and the hard-of-hearing” (\citeyear{Diaz_cintas2006}:~51). Thus, this more thoughtful approach has its roots in Japanese subtitling that influenced fansubs for Japanese \isi{anime} films and seems to combine features of both traditional subtitles and SDH.

While the terms ‘\isi{abusive}’ and ‘hybrid’ focus on subtitles in the context of Japanese productions and Nornes’ ‘\isi{abusive} subtitles’ are often seen as mainly “making translations linguistically visible” (\citealt{Foerster2010}:~86), researchers such as \citet{Foerster2010} and \citet{mcclarty2012} approach the topic from a more aesthetics- and creativity-based perspective. Foerster criticises the conventional guidelines’ aim of invisibility and the resulting “register and […] design for subtitles that never call attention to themselves” (\citealt{Foerster2010}:~82), defining subtitles “solely as a means of understanding what is being said on screen” (\citeyear{Foerster2010}:~82). Mentioning examples such as \textit{Monty Python and the Holy Grail} (UK 1975), \textit{Annie Hall} (USA 1977), and “Desperanto” (segment in \textit{Montreal Stories}, CAN 1991) and discussing the subtitles of \textit{Nochnoy Dozor}\footnote{Released in the USA by Twentieth Century Fox and Fox Searchlight Pictures in 2006 with English titles.} (“Night Watch”, RUS 2004; see \sectref{sec:4.2.1}), she defines “\isi{aesthetic} subtitling” (\citeyear{Foerster2010}:~85) as a practice that “draws attention to the subtitles via \isi{aesthetic} means exploring semiotic possibilities, which include the semantic dimension without being restricted by it” (\citeyear{Foerster2010}:~85) and is “predominantly designed graphically to support or match the \isi{aesthetics} of the audiovisual text and consequently develop an \isi{aesthetic} of their own” (\citeyear{Foerster2010}:~85). 

McClarty follows a more film studies-based approach when she speaks of “creative subtitles” (\citeyear{mcclarty2012}:~138). Like Foerster, she sees “subtitling practitioners [as] mere norm-obeying machines” (\citeyear{Foerster2010}:~135) that “continue to have their hands tied by the constraints of the field and the norms of the profession” (\citeyear{Foerster2010}:~135) while failing to “acknowledge the insights that could be gained by referring to audiovisual translation’s parallel discipline: film studies” (\citeyear{Foerster2010}:~135). While McClarty sees the roots of creative film titles in the “artistry involved in the design of intertitles during cinema’s silent era” (\citeyear{mcclarty2012}:~136) that “completed the scene” (\citeyear{mcclarty2012}:~136), she also acknowledges the failure of Japanese \isi{anime} distributors to fulfil the “specific demands of that fan community” (\citeyear{mcclarty2012}:~137) and thereby providing the motivation for fans to create their own subtitles – or, fansubs. By using their own concept of typefaces, colours and placements they “forced the hand of \isi{anime} distributors, leading them to adapt their subtitling styles” (\citeyear{mcclarty2012}:~137). McClarty sees the solution in a multidisciplinary approach similar to the recent turn to more practical approaches in theatre translation:
\begin{quote}
Regardless of how timely, beneficial or aesthetically pleasing a creative subtitling strategy may be, referring only to ideas from within translation studies and audiovisual translation will fail to produce a new form of subtitling that is truly innovative. (\citealt{mcclarty2012}:~138)
\end{quote}
Instead of ‘abuse’, she sees the need of a ‘creative’ approach that does not simply “describe a \isi{subtitling practice} that differs from the norm but [that denotes] an approach that looks outward from its own discipline as well as its own culture” (\citealt{mcclarty2012}:~138) and aims for “difference rather than sameness” (\citealt{mcclarty2012}: 140). McClarty emphasises that not the creativity of a translator or subtitle practitioner leads to innovative titles such as those visible in \textit{Slumdog Millionaire}, \textit{La Antena} (AR 2007) or \textit{Austin Powers in Goldmember} (USA 2002), but the “imagination of film directors and editors” (\citeyear{mcclarty2012}: 140) that did not only want to provide a translation but also create additional effects of comedic or artistic nature (\citeyear{mcclarty2012}:~140--142). Thus, subtitles can not only convey content but also sound (\citealt{mcclarty2012}:~146) or speaker location (\citealt{mcclarty2012}:~148). The goal should not be a new set of norms but a “creative response to individual qualities within and between films” (\citeyear{mcclarty2012}:~148.), created by a “translator-\isi{title designer}” (\citealt{mcclarty2012}:~149) that aims for both linguistically and aesthetically pleasing titles. These should be produced within the postproduction phase of a film’s making and in cooperation with the film maker, editors, and title designers.

Another term that can be found in recent studies that is more strongly focused on \isi{usability} and automation is that of “dynamic subtitles” (\citealt{Armstrong2014},  \citealt{Brown2015}). The concept is based on placing the titles close to the speaker and allowing for an easier \isi{speaker identification} and more time for the \isi{image exploration}. The two \isi{eye tracking} studies dealing with this concept are presented in \sectref{sec:6.3}.

In similar fashion to these approaches, the studies by \citet{park2008}, \citet{Hong2010} and \citet{Hu2013} focus not only on the improved perception but also on the automation of the process. They use terms such as “speaker-following subtitles” (\citealt{Hong2010}; \citealt{Hu2013}) that relate to the automatic speaker recognition systems of their software. These studies are presented in \sectref{sec:3.4}.

While the presented terms are still based on the concept of subtitles that are placed below or at the bottom of the image, even Nornes back in 1999 put the “sub” in parentheses “because they were not always at the bottom of the frame” (\citealt{nornes1999}:~23). As Bayram and Bayraktar speak of “text information” presented in “integrated formats” (\citeyear{Bayram2012}:~82) when combining text and image, the term of “integrated titles” was chosen for the pilot study in 2012 and kept for the following studies as even Armstrong and Brooks mention the enhancement of subtitles through “integrating them with the moving image” (\citeyear{Armstrong2014}). Even though this approach could also be deemed ‘creative’ or ‘\isi{aesthetic}’ and the title \isi{placement} ‘dynamic’ and ‘speaker-following’, using the term ‘integrated’ seems to include both these concepts while emphasising the relationship between image and title. The various terms, however, show the wide bandwidth of possible approaches that cannot be defined by tight norms or strict guidelines.

As \citet{Foerster2010}, \citet{mcclarty2012}, and others stated, not only the analysis (cf. \citealt{Chaume2004}) but also the creation of subtitles should not purely be based on translation studies. Film studies can offer insight into \isi{image composition} and storytelling, communication and graphic design can offer definitions of \isi{aesthetics} and creativity, \isi{usability} studies provide basics of \isi{user experience} design as well as \isi{interface design}, and computer sciences provide insight into potential automation processes and software design. Therefore, the following sections discuss these fields of study in regard to integrated titles.

\section{Communication design: Aesthetics and layout}\label{sec:3.1}

In the previous \sectref{sec:1.2.3}, various challenges regarding the \isi{layout} of traditional subtitles were presented and discussed. While the main criticism was that subtitles with a conventional \isi{layout} can cover interesting or \isi{plot-relevant image} regions and objects, further points of criticism should be considered as well, such as the loss of \isi{aesthetic} value and the disruption of the intended \isi{image composition}. Aspects such as \isi{aesthetics}, creativity, and \isi{layout} were mentioned – but how are these similar sounding concepts defined and how are they relevant for title reception? First of all, they can be put into an order following relevance: Graphical aspects such as \isi{readability}, \isi{legibility}, and contrast are highly relevant to the overall experience and information transfer while characteristics of good design and \isi{aesthetic} concepts are not always that specific to be easily and directly applied. Readability is affected by the “way in which words and blocks of type are arranged” \citep{Loyd2013} while \isi{legibility} “refers to how a \isi{typeface} is designed and how well one individual character can be distinguished from another” \citep{Loyd2013}. Therefore, \isi{readability} is tightly connected to rule-based decisions such as the position of the \isi{line break} (see \sectref{sec:1.2.3}), the position of the subtitle in general (see \sectref{sec:1.2.3} and \chapref{placement}), and the decision between one- and two-liners. Legibility, on the other side, refers to the choice of \isi{typeface}, the character spacing and kerning as well as the \isi{line spacing}, and the size and style of the \isi{typeface}. Based on these characteristics, subtitles can be easy to read – or not. While there are detailed discussions and analyses of \isi{readability} and \isi{legibility} of printed typefaces, there is not that much material on subtitles yet. Projects such as “Screenfont”\footnote{See \url{http://www.screenfont.ca} [2016--03--25].}, however, provide insight into the challenges and shortcomings of typefaces used for subtitles:


\begin{quote}
Because of careful hinting and special adaptation for computer displays, these screenfonts are generally more pleasant and easier to read than many other fonts. But their purpose is to display static pages of text. […] They don’t appear and disappear like pop-on captions or subtitles, nor do they scroll or paint on like certain other captions. You don’t have a single nominal chance to read the text that uses those fonts before that text disappears. \citep{screenfont2016}
\end{quote}

While these problems might become less relevant with higher resolutions and the rise of the BD, it should be taken into account that a bad choice of \isi{typeface}, style, and size can render any subtitle unreadable. Organisations such as the Irish National Disability Authority (NDA) list several more criteria of readable subtitles (\citealt{nda????}):

\begin{quote}
\begin{itemize}
\item Rationale
\item Standard readable language
\item Start and end at natural, logical points
\item Carefully chosen linebreaks
\item Adequate \isi{reading time}
\item Clear visual presentation
\item Position of subtitles that avoids obscuring important content
\end{itemize}
\end{quote}


In the section on ‘clear visual presentation’, the Centre for Excellence in Universal Design (CEUD) of the NDA puts emphasis on the importance of subtitles staying within the so-called ‘\isi{safe area}’ that is never cut regardless of the video size or playback device. Furthermore, suitable typefaces and a strong contrast as well as legible colour combinations are demanded. While the most legible colour combinations are seldom suitable for most image concepts (blue/white, red/white, cyan/blue and blue/cyan), the recommendation of using “colours with a \isi{saturation index} of less than 85\,\% to avoid distortion and flicker” (\citealt{nda????}) can easily be followed when working with colours (and also black and white typefaces). Interestingly, the CEUD also demands to not obscure the speaker’s mouth, burnt-in subtitles or “any important activity” (\citealt{nda????}). As the recommended colour combinations might not suit most films, calculating the contrast might be a useful tool in ensuring \isi{legibility}. Due to the way the eyes work (especially the rods and cones and our sensitivity to light), luminance and contrast are most relevant in distinguishing colours and therefore objects. While “contrast in light and dark is the most effective” \citep{Bradley2016}, there’s also complementary colours, temperature, and saturation. These can be retrieved from tables or through calculation.\footnote{See i.e. calculation of luminance and contrast ratio (\url{http://vanseodesign.com/web-design/luminance-contrast-ratio-accessibility/} [2016--08--01]) and tools for colour contrast check (\url{http://www.snook.ca /technical/colour\_contrast/colour.html\#fg=33FF33,bg=333333} [2016--08--01] and \url{http://openweb.eu.org/articles/text-contrasts}) [2016--08--01].} Even though \isi{readability} and \isi{legibility} are discussed concerning design and \isi{aesthetics}, they are tightly connected to \isi{usability} and accessibility as well.\footnote{Cf. \url{https://www.nngroup.com/articles/low-contrast/} [2016--08--01] on low contrast and accessibility issues.} While there is no perfect set of required characteristics of subtitle text, some aspects are clear: A bold, \isi{sans-serif} \isi{typeface} that provides a strong contrast due to its colour and possibly an outline and offers good \isi{legibility} due to a spacing that allows a clear distinction of each letter is likely to be a good fit for most films.\footnote{Further and more detailed discussions can be found here: \url{http://webdesign.tutsplus.com/articles/typographic-readability-and-legibility--webdesign-12211} [2016--04--18].}

Aesthetics, on the other hand and by itself, is difficult to define. Most aspects that would be used to define it are normally entirely subjective and can vary between different viewers. This is why in his book on the “\isi{aesthetic experience} and literary hermeneutics” Hans Jobert Jauß demands that “\isi{aesthetic} judgment” should be based on “both instances of effect and of reception”\footnote{In German, the passage from \textit{Ästhetische Erfahrung und literarische Hermeneutik} reads \textit{“}[das] ästhetische Urteil [aus den] beiden Instanzen von Wirkung und Rezeption [zu] bilden“.} (\citeyear{Jaus1991}:~9). Thus, \isi{aesthetics} is not only about the superficial effect of a piece of art, or a film in this case, but also about how it is interpreted. And if it can be interpreted, then there is usually also an intention behind it, an intended effect of the film. If we regard any film, including translated film, as an artistic synthesis, then we can therefore critically examine the intention behind the target language adaptation. The current appearance of subtitles basically suggests that the intention behind the adaptation normally includes little more than the goal of providing access through translation, combined with considerations of time and cost efficiency. In the worst case, the audiovisual translation of (older) films is subject to censorship.\footnote{See, for example, the German dubbed version of the American film \textit{Die Hard} (USA 1988), where the nationality of the originally German alleged terrorists was altered (see \url{http://www.schnittberichte.com/schnittbericht.php?ID=1105} [2016--07--29]). \textit{Casablanca} (USA 1942), as well, was subject to denazification (see \url{http://www.dradio.de/dkultur/sendungen/zeitreisen/918145/} [2016--07--29].} Furthermore, there are only few examples based on more advanced \isi{aesthetic} considerations. A problem in this regard could be, of course, the recipients’ “expectation and experience” mentioned by Jauß (\citeyear{Jaus1991}:~13). Since users of subtitled films are usually not accustomed to or do not know any other way of subtitling, they do not make demands regarding the \isi{aesthetics} of the adaptation. And if the target group makes no demands, the industry apparently rarely feels compelled to act.



So if we accept film as a work of art, even if it is such an ‘ephemeral’ one as a television series, and apply these basic design rules, then what are the probable consequences for subtitling?


\begin{itemize}
\item Film should not be subject to any negative influences due to translation and should not be altered to its detriment as a consequence of translation.
\item Concerning \isi{layout}, audiovisual translation should either follow the original as closely as possible or should compensate losses caused by the transfer in another way.
\item The value of the translated film should not be reduced by an aesthetically unambitious presentation.
\item Film and translation should not be created and judged separately from each other.
\end{itemize}

Based on these assumptions, translators can only deliver a quality translation if they have full access to the source film as well as the inserted texts (e.g. captions) and the texts to be inserted, so that they can connect the non-verbal \isi{visual channel} with the other channels. To show that this mode of translation is indeed possible, and that film can be translated graphically in an appealing and respectful manner, this study also analyses the \isi{aesthetics} of the newly created integrated titles. As the \isi{aesthetic experience} cannot be measured via eye movements recorded with an \isi{eye tracker}, a questionnaire was designed according to the recommendations of the WPGS (\textit{Wirtschaftsspsychologische Gesellschaft}, Society for Behavioural Economics and Economic Psychology). It included a number of questions on information processing and the \isi{aesthetic experience}. The study is presented in \chapref{method}, the results and the discussion in \sectref{sec:8.2}.



Gottlieb, however, judges the \isi{aesthetics} of subtitles based on \isi{subtitle design} – referring to line breaks etc. – and comfortable \isi{reading speed} as well as subtitle duration (\citeyear{Gottlieb2012}:~53). He therefore defines subtitle \isi{aesthetics} through their \isi{usability}, and criticises the low standards that can sometimes be observed:

\begin{quote}
Although in most countries subtitling companies that produce bona fide subtitles allow subtitlers enough time to cue subtitles in advance, not all countries focus on oral synchrony — or visual, for that matter. Worse still, in news programs and other direct transmissions in any country, live and semi–live subtitles are, inevitably, often grossly out of sync with the speaker on screen. (\citealt{Gottlieb2012}:~41)
\end{quote}

As discussed earlier, \isi{interlingual} and \isi{intralingual} subtitles created for \isi{deaf} and hard-of-hearing audiences provide several additional features that allow an easier \isi{speaker identification}, e.g. through colour, and “although color identification may indeed be less relevant to a hearing audience, certain dramatic film sequences with overlapping speech may benefit from a (modest) use of colors, so that two or three main characters may be distinguished throughout fast verbal exchanges” (\citealt{Gottlieb2012}:~68). In addition, “letter size […] and alignment are suited for the purpose” (\citealt{neves2007}:~94--95) and varying screen sizes and target groups should be taken into account (\citealt{Gottlieb2012}:~68). However, “viewers are not as aware of ‘quality’ in subtitle content and design as researchers and practicing subtitlers would like them to be” (\citealt{Gottlieb2012}:~69) and the common minimum should be the “synchronous cueing and harmonious \isi{subtitle design}” (\citeyear{Gottlieb2012}:~69). Gottlieb reasons that subtitling “has not yet proven to be reader-\textit{un}friendly” (\citeyear{Gottlieb2012}:~69):
\begin{quote}
1.  Sentence-level synchrony between dialogue and subtitles is secured, helping viewers identify speakers and comprehend the original dialogue.

2.  Subtitles will appear as an integral part of the polysemiotic totality of the production, thus creating semiotic cohesion (\citealt{Diaz_cintas2007}:~49--52) and the illusion in readers that they can follow the original dialogue.

3.  Each subtitle is a self-contained entity, making sense even in isolation from the previous and following subtitles. This, however, should not lead to staccato-like sequences of isolated subtitles: grammatical and logical cohesion is indispensable.

4.  Subtitles do not cover important pictorial information.

5.  Subtitles do not give away important information to the tar-get audience that the original audience has not yet obtained through the dialogue.

6.  Well-designed subtitles pave the way for a ‘fuller’ translation of the spoken lines. Reader-friendly segmentation, including effective line breaks, allows for more positive intersemiotic feedback and may result in higher reading speeds. In other words, well-designed subtitles mean less condensation — and minimal loss of information. (\citealt{Gottlieb2012}:~69)
\end{quote}
In combination, the definition of \isi{aesthetic} subtitling seems to be based on being pleasing and useful to the audience. Similarly, creativity can be defined as an “achievement [that] emerges from the awareness of a problem and presents something new that in a certain time within one culture is accepted as sensible by experts” (\citealt{preiser1976}:~5) and a creative product as “a new product that is regarded usable or satisfactory at a certain time by a certain group” \citep{stein1953}. Foerster summarises these needs in a similar fashion: 
\begin{quote}
The evaluation of a creative solution is thus based on two pillars: the awareness of a problem, as innovations have to meet certain parameters and norms to eliminate the discontent and to be regarded as satisfactory by experts (Kußmaul 2000:~17). (\citeyear{Foerster2010}:~95)
\end{quote}
This combination of creative and \isi{aesthetic} elements with features of the common subtitling guidelines might then “form the essence of an approach that is professional and yet of artistic value” (\citeyear{Foerster2010}:~95) that transports the plot to the audience in the best possible way. As the demanded usefulness of these new approaches is emphasised as much as the possible improvements in \isi{aesthetics} and creativity, a look at \isi{usability} studies and \isi{user experience} design might provide further insight into what would be a good future approach for innovative subtitling strategies.

\section{Usability studies: Interface and user experience design}\label{sec:3.2}

Although there is no specific set of \isi{usability} guidelines for subtitles, Mosconi and Porta argue that those concerning web accessibility and \isi{usability} “can be applied generally to them” (\citeyear{mosconi2012}:~106). They define \isi{usability} in the context of subtitles as pursuing a concept that works as “intuitively and efficiently as possible” (\citeyear{mosconi2012}:~106), based on the definition of \isi{usability} being a “quality attribute that assesses how easy user interfaces are to use” \citep{nielsen2003}. Nielsen himself defines five quality components of \isi{usability} — \isi{learnability}, efficiency, \isi{memorability}, errors resp. error management, and satisfaction:
\begin{quote}
According to this idea, it will be easy for users to accomplish basic tasks the first time they come across a new usable interface (\textit{learnability}). Once the way the usable interface functions has been digested and learnt by new users, new tasks will be performed quickly (\textit{efficiency}). When users return to a given usable interface after a period of not using it, it will be easy for them to re-establish proficiency (\textit{memorability}). As far as errors are concerned, within usable interfaces their frequency and seriousness are generally acceptable and it is easy to recover from them. Lastly, also satisfaction is important: using a usable interface should be a pleasant task. (\citeyear{mosconi2012}:~114)
\end{quote}
Part II of the “Ergonomics of Human System Interaction” (ISO 9241--11, 1988), for example, defines \isi{usability} as “the extent to which a product can be used by specified users to achieve specific goals with effectiveness, efficiency and satisfaction in a specific context of use” (\citealt{mosconi2012}:~115). The importance of not only usefulness but also satisfaction can be related to the goals of the demanded \isi{aesthetic} approach in the previous chapter and are also reflected in basic design guidelines. As highlighted by \citet{Foerster2010}, not only the \isi{aesthetics} and usefulness are relevant, but also consistency – additional effects and individual \isi{placement} within a film must follow comprehensible rules or make it a rule to break those. So, while web \isi{usability} typically deals with command wording, \isi{layout}, position of page elements, site ‘\isi{intuitiveness}’, and other user-oriented criteria (cf. \citealt{Cherim2007}), it can be applied to subtitling, e.g. in terms of a consistent \isi{layout} and \isi{readability} (\citealt{mosconi2012}:~107). In combination with the rules of good design by \citet{shneiderman1998}, demands for good subtitling – or good integrated titles – can be split into four major aspects:

\begin{itemize}
\item Consistency
\item Readability/Legibility
\item Usefulness
\item Satisfaction
\end{itemize}

These aspects should be obeyed when making decisions concerning \isi{subtitle layout}, timing, length etc. Consistency can refer to the position of line breaks, the length of titles, the timing, rhythm, and contents, but also \isi{layout} decisions such as typefaces, contrast-increasing features, and \isi{individual title placement}, e.g. below speakers in order to provide \isi{speaker identification}. Readability and \isi{legibility} were already explained in the context of graphic design earlier in this chapter and the need for these aspects is obvious. The usefulness is a combination of a suitable translation, consistency, \isi{readability} and \isi{legibility}. In turn combined with a \isi{layout} and \isi{design concept} that not only pleases the viewer’s need for information but also provides an \isi{aesthetic experience}, satisfaction can be reached.

Usefulness and the \isi{aesthetic experience} can also be based on the closely related film studies as these can offer insight into \isi{image composition} and provide an understanding of the filmmaker’s goals in an overall film as well as specific scenes – overall, allowing the subtitle creator to ‘read’ the film the way it was intended.

\section{Film studies}\label{sec:3.3}

Obviously, film studies are very closely related to audiovisual translation. When watching a film, “viewers make sense of visual and acoustic sign systems that are complemented by an acoustic channel and presented to them on a screen (\citealt{Diaz_cintas2007}:~45). Therefore, “subtitles have to become part of this semiotic system” (\citeyear{Diaz_cintas2007}:~45) and “must interact with and rely on all the film’s different channels” (\citeyear{Diaz_cintas2007}:~45). But in order to being able to “follow the movement of the camera” (\citealt{Diaz_cintas2007}:~53) or “ignore the different camera positions” (\citeyear{Diaz_cintas2007}:~53), a subtitle creator has to understand the use of cameras, shots and cuts to begin with. As Foerster phrases it: A “more creative space means more responsibility and the obligation to fully understand traditional practice in order to be able to divert from it successfully” (\citeyear{Foerster2010}:~96). Thus, an understanding of both the underlying film and the traditional subtitling strategies is needed. Only then a decision can be made whether to adhere to these strategies and guidelines or to break the rules.

Walter Murch, a well-known (sound) editor who worked on films such as \textit{The Godfarther:~Part~II} (USA 1974) and \textit{Apocalypse Now} (USA 1979), gives an overview over film composition in his book “In the Blink of an Eye: A Perspective on Film Editing” (\citeyear{murch2001}). He mentions the “rule of six” (\citeyear{murch2001}:~18) that rates the six most important aspects of filmmaking:

\begin{itemize}
\item Emotion (51\,\%)
\item Story (23\,\%)
\item Rhythm (10\,\%)
\item Eye-trace (7\,\%)
\item Planarity (5\,\%)
\item Three-dimensional space of action (4\,\%)
\end{itemize}

While the mediation of ‘emotion’ and ‘story’ make up the biggest part in this distribution, the point of ‘eye-trace’ with after all 7\,\% is striking and a significant argument for integrated titles. ‘Planarity’ and the ‘three-dimensional space of action’ describe the creation of a three-dimension space with a two-dimensional product and the position and relation of people in a room to one another (\citeyear{murch2001}:~18). McClarty sees the importance of conveying emotion as a reason for titles to “respond to the emotion of the moment” (\citeyear{mcclarty2012}:~145) in “both linguistic and graphic forms” (\citeyear{mcclarty2012}:~145), including “colours, styles or special effects” (\citeyear{mcclarty2012}:~145). She also understands “Murch’s criterion of ‘eye-trace’ [as] significant when considering a creative \isi{subtitling practice}, as it seems to pose an argument for subtitles being raised into the heart of the on-screen action” (\citeyear{mcclarty2012}:~146). A subtitle professional should therefore understand from the film the importance of these aspects for the respective filmmaker – one might focus on emotion, another one on story or eye-trace. Creative decisions should be made accordingly.


Similar to subtitling guidelines, “the rules of \isi{cinematic composition} are not written in stone” (\citealt{mercado2010}:~XIV) and an “integrated approach [is needed] to understanding and applying the rules of \isi{cinematic composition} […] that takes into account the technical and narrative aspects that make shots […] so powerful” (\citeyear{mercado2010}:~XIV). However, a decision-making model or workflow based on more commonly found shots and knowledge on the technical and visual conventions could provide the foundation to create integrated titles that suite the respective film and its \isi{image system}. Image systems refer “to the use of recurrent images and compositions in a film to add layers of meaning to narrative” (\citealt{mercado2010}:~21). As this can be a “powerful tool to introduce themes, motifs, and symbolic imagery” (\citeyear{mercado2010}:~21), it allows the audience to make “connections not only within, but also between shots” (\citeyear{mercado2010}:~21). Similarly, integrated titles should follow a continuative system that provides \isi{recognition value} to the audience and, much like image systems, “work best when they support and add meaning to, and not become, the point of [your] film” (\citeyear{mercado2010}:~21). Some principles of composition and technical concepts of filmmaking can support decision-making, for example when placing titles individually:


\begin{itemize}
\item Rule of thirds
\item Hitchcock’s rule
\item Balanced/Unbalanced compositions
\item High/Low angles
\item Open/Closed frames
\item Focal points
\end{itemize}

The “rule of thirds” (\citealt{mercado2010}:~7) indicates the frequent “looking room” (\citeyear{mercado2010}:~7) that prevents a static feel and creates a “dynamic composition” (\citeyear{mercado2010}:~7) and sometimes even “walking room” (\citeyear{mercado2010}:~7), thereby offering space for title \isi{placement} – or, in the case of ‘waking room’, preventing a \isi{placement}. Another rule from filmmaking that can be applied to title \isi{layout} is “Hitchcock’s rule” (\citealt{mercado2010}:~7ff.) that says that “the size of an object in the frame should be directly related to its importance in the story at that moment” (\citeyear{mercado2010}:~7ff.). Balanced and unbalanced compositions can both define suitable areas for titles and lend their definition to creative titles:
\begin{quote}
Every object included in a frame carries with it a \isi{visual weight}. The size, color, brightness, and \isi{placement} of an object can affect the audience’s perception of its relative \isi{visual weight}, making it possible to create compositions that feel balanced when the \isi{visual weight} of the objects in the frame is evenly distributed, or unbalanced when the \isi{visual weight} is concentrated in only one area of the frame. (\citealt{mercado2010}:~8)
\end{quote}
So if a balanced composition provides an “evenly” or “symmetrically distribution” and “convey[s] order, uniformity, and predetermination” (\citeyear{mercado2010}:~8), titles integrated into this scene should do the same. The same goes for an unbalanced composition that is often “associated with chaos, uneasiness, and tension” (\citeyear{mercado2010}:~8). Similarly, high and low angles can – always depending on the situation – communicate either confidence and control or weakness and passiveness. The use of closed or open frames can directly be connected to the decision whether or not – and in what way – to indicate a speaker’s position and the direction of someone’s speech: Closed frames can be created in a way that they “do not acknowledge or require the existence of off-screen space to convey their narrative meaning” (\citealt{mercado2010}:~10) and open frames “do not contain all the necessary information to understand their narrative meaning” (\citeyear{mercado2010}:~10). Focal points help defining the “center of interest in a composition, the area where the viewer’s \isi{gaze} will gravitate to because of the arrangement of all the visual elements in the frame” (\citealt{mercado2010}:~11) – thus, any title \isi{placement} in a scene with one or more focal points should take place accordingly. These focal points can be created with the afore-mentioned rules and additional technical aspects such as the used lenses and the use of depth. \citet{mercado2010} and \citet{Kenworthy2011} mention various common shots that can be distinguished in film. These can in turn be split into two basic groups – one being rather content-related and one illustrating the ‘size’ of the frame. The following sizes of shots are commonly distinguished:

\begin{itemize}
\item Extreme close up
\item Close up
\item Medium close up
\item Medium shot
\item Medium long shot
\item Long shot
\item Extreme long shot
\end{itemize}

While the extreme close up and close up isolate “single, visual details from the rest of the scene” (\citealt{mercado2010}:~29), medium and long shots create relationships between characters and the space around them. Close ups should be kept free of “any visual elements that might distract the audience from the main subject” (\citealt{mercado2010}:~31) while medium shots “can contain a lot of visual detail” (\citealt{mercado2010}:~47) and usually need to stay longer on the screen to allow processing. These shots can make use of Hitchcock’s rule, create several layers of depth or a narrative between characters through their respective \isi{placement} in the frame (\citealt{mercado2010}:~60). A content-based categorisation of shots would at least include the following groups:

\begin{itemize}
\item Establishing shot
\item Over the shoulder shot
\item Subjective shot
\item Two shot
\item Group shot\footnote{Many more shots such as macro and zoom shots, canted and tilt shots, emblematic shots, abstract shots, dolly shots, crane shots etc. can be found in recent films (see \citealt{mercado2010}).}
\end{itemize}

Understanding the usually three layers of an ‘over the should shot’, the technical features of a ‘canted shot’ that makes vertical lines “run diagonally across the frame” (\citealt{mercado2010}:~101) or the relationships created through \isi{placement}, scale or depth can allow for a considerate \isi{placement} of text elements. As they become part of the image, they will “be interpreted by an audience as being there for a specific purpose that is directly related and necessary to understand the story they are watching” (\citealt{mercado2010}:~2) and have to respect the “direct connection between what takes place in the story and the use of a particular composition” (\citealt{mercado2010}:~3).


From a translation studies perspective, \citet{Chaume2004} proposed a “film studies-based approach for the analysis of audiovisual texts” (\citealt{mcclarty2012}:~138), based on ten “signifying codes of cinematic language” (\citealt{Chaume2004}:~16). This multidisciplinary approach aims at translators knowing the “functioning of these codes” (\citealt{mcclarty2012}:~138):


\begin{enumerate}
 \item The \textsc{linguistic code} as translators are faced with a text “written to be spoken as if not written” and that “has to appear oral and spontaneous” (\citealt{Chaume2004}:~19).
 \item The \textsc{paralinguistic code} concerning the representation of paralinguistic signs such as silence or a pause.

 \item The \textsc{musical code} and the \textsc{special effects code}, e.g. highlighting music lyrics by using italics.

 \item The \textsc{sound arrangement code} that allows a differentiation of diegetic and non-diegetic sound, e.g. italics for \isi{narrative text}.

 \item  Various \textsc{iconographic codes} that, for example, acknowledge the challenge of the visual \isi{feedback effect}.

 \item  Various \textsc{photographic codes} that relate to scenes that might make it necessary to use different orthography in the subtitles, e.g. italics in dark scenes where the speaker is unclear.

 \item  The \textsc{planning code} that refers to types of shots, e.g. the application of Hitchcock’s rule with the close up of a plot-relevant poster.
 \item Various \textsc{mobility codes} such as “proxemics signs”, “kinetic signs”, and “phonetic articulation” (\citealt{Chaume2004}:~21) that relate to distances between characters and characters and the camera, movement and relating a subtitle to a visibly speaking character.
 \item Various \textsc{graphic codes}.
 \item Various \textsc{syntactic codes} of editing, as the “process of shot associations chosen by the director can have repercussions on the translation” (\citealt{Chaume2004}:~22), as well as “audiovisual punctuation marks” (\citeyear{Chaume2004}:~22), e.g. a fade to black resulting in a subtitle change. 
\end{enumerate}
  

Thus, Chaume bases his analysis not only on translation theory and discourse analysis but also film studies and communication studies, and demands that an audiovisual translator is “capable of transmitting not only the information contained in each narrative [visual and verbal] and each code […] but the meaning that erupts as a result of this interaction” (\citealt{Chaume2004}:~23) and adds value \citep{Chion1993} or additional meaning (\citealt{Fowler1986}:~69).

Desiderata that can be derived from this design- and aesthetics-based perspective are discussed in \sectref{sec:3.5}. However, as one of the main arguments against new approaches based on \isi{communication design} and film studies seems to be the additional workload for a subtitle creator, possible automation of the \isi{placement} processes will be discussed in the following section.

\section{Computer science: Automatic subtitle placement}\label{sec:3.4}

In recent years, various automation solutions for \isi{individual title placement} were developed and tested. One of the first studies that can be found was done by \citet{park2008}, describing a “framework for displaying synchronized text around the speaker” (\citealt{mosconi2012}:~129) with the aim to develop a “new media player, namely MoNaPlayer, which displays subtitles near by speakers” (\citealt{park2008}:~166) that reuses existing subtitle formats (\citeyear{park2008}:~166). Their concept includes a module that automatically transforms existing information from the subtitle file and an “automatic subtitles localization through \isi{speaker identification}” (\citeyear{park2008}:~168) and fulfils three tasks:

\begin{enumerate}
 \item[(a)]  Extracting descriptive context (e.g., text, styling, timing model, and etc.) and contents from existing timed text formats,
 \item[(b)]  Analyzing video source for subtitles region detection and \isi{speaker identification}, and
 \item[(c)]  Transforming the timed text to a new spatio-temporal text and generating SMIL\footnote{SMIL stands for “Synchronized Multimedia Integration Language”, cf. \url{https://www.w3.org/TR/1998/REC-smil-19980615/} [2016--07--30].} document in order to integrate the video and the subtitles (\citeyear{park2008}:~168).
\end{enumerate} 

Additionally, Park et al. defined three cases of \isi{speaker identification}: a speaker visible in the screen, a speaker that “may not appear” or “can not [sic!] be identified exactly due to particular reasons” (\citeyear{park2008}:~169), and “open captions, recorded in video itself, for explaining the scene or situation even though there are no sound and speakers” (\citeyear{park2008}:~169). After evaluating their system, Park et al. found several limitations in their first version of the system: the system could only detect speakers facing the camera, had problems “detecting subtitle regions as a consequence of multiple moving objects” (\citealt{park2008}:~171), produced subtitles overlapping with the speaker’s face, and did not offer a method for “detecting reusable regions” (\citeyear{park2008}:~171). Overall, their system represents a first encouraging approach that they propose as a future solution targeted at “foreign-language and visual learners, for beginning readers, and for a videoconferencing” (\citeyear{park2008}:~171).

\largerpage
Similar to the approach by Park et al., Hong and colleagues wanted to illustrate that “existing captioning techniques are far from satisfactory in assisting the hearing impaired audience to enjoy videos” \citep{Hong2010}. Opposed to the traditional “static captioning” (\citeyear{Hong2010}), they propose the concept of “dynamic captioning” (\citeyear{Hong2010}) as an “assistive approach” (\citeyear{Hong2010}) for hearing-impaired audiences. They identified the following problems with traditional subtitles: Confusion of the speaking characters and the tracking of captioning (meaning the speaking pace), and the lost of volume information (\citeyear{Hong2010}). Hong et al. refer to a study by Gulliver and Ghinea (\citeyear{Gulliver2003}) that concluded that “the conventional captioning approach can hardly add significant information” and that “information from other sources such as visual content and video text will be significantly reduced” (\citeyear{Gulliver2003}). They therefore developed a script-based “automatic approach to intelligently present captions” (\citeyear{Gulliver2003}) that puts the titles in “suitable regions, aligns them with speech and also illustrates the variation of voice volume” (\citeyear{Gulliver2003}). The system includes face detection, tracking and grouping, script-face mapping (including lip motion analysis with a recognition accuracy above 80\,\%), non-intrusive detection, and voice volume analysis. To test their system, Hong et al. conducted a user study with 60 mostly prelingually \isi{deaf} participants with sign language as first or preferred language that watched video material in three modes – no \isi{caption}, static captions, and dynamic captions. In their user study, they asked the participants questions concerning the content, \isi{enjoyment} and naturalness, and their preference. They could not find a difference between static and dynamic mode concerning the content but “remarkably higher [scores] for the questions that are related to video text or visual content” (\citeyear{Gulliver2003}). However, the dynamic captions “remarkably outperform” (\citeyear{Gulliver2003}) the other two modes concerning \isi{enjoyment} and preference – 53 out of 60 participants choose dynamic captioning over static captions. They mainly focused on the “technical part of dynamic captioning and [cared] less about user interface, such as the visualization of volume variation […] and the style of script highlight” (\citeyear{Gulliver2003}). However, they stated user \isi{interface design} to be “crucial for real-world application” (\citeyear{Gulliver2003}) and that it should be considered in future approaches to automation.

Based on the system by \citet{Hong2010}, Hu and colleagues (\citeyear{Hu2013}) attempted to improve the automatic \isi{placement}. They reason that humans can “only read text clearly in a narrow vision span” (\citeyear{Hu2013})  of “approximately 6 degrees of arc, which yields a region with a diameter of 5.23~cm when viewed from 50~cm away” (\citeyear{Hu2013}; cf. \citealt{rayner1975}, \citealt{Just1987}, \citealt{mcconkie1989}). Therefore, viewers have to “constantly move their eye \isi{gaze} between the main viewing area and the bottom of the screen, leading to a high level of \isi{eyestrain}” \citep{Hu2013}. To improve the existing system by \citet{Hong2010}, they present a “new \isi{speaker detection} algorithm to accurately detect the speakers based on visual and audio information” and an “efficient optimization algorithm […] to place the subtitle based on a number of factors” \citep{Hu2013}. While it might seem obvious to consult strategies of \isi{text placement} in comics (see, for example, \citealt{Kurlander1996}, \citealt{Chun2006}, \citealt{Groensteen2007}), Hu et al. argue that \isi{placement} strategies for single, static frames “cannot easily be extended to video subtitles because they cannot ensure cross-frame coherence” (\citeyear{Hu2013}). Therefore, their \isi{placement} is based on \isi{speaker detection} that works through face tracking combined with lip motion and center contribution, as “a speaker is more likely than non-speakers to be located towards the center of the screen” (\citeyear{Hu2013}), as well as audiovisual \isi{synchronicity}. The \isi{subtitle placement} was based on the following rules:

\sloppy
\begin{enumerate}
\item   In each frame, the subtitle should be close enough to its speaker so not to cause any confusion that it is spoken by another person; 
\item   Across consecutive frames, the overall distance between all subtitle placements should be small to reduce \isi{eyestrain}; 
\item   Subtitles should not be placed near screen boundaries so not to detract the viewer from the central viewing area;
\item   Subtitles should not occlude any important visual contents (e.g. faces) (\citeyear{Hu2013})
\end{enumerate}
\fussy 

Based on these rules, Hu et al. only considered eight positions around the speaker – above left, above, above right, below left, below, below right, left, and right. The ultimate position was then “computed based on the speaker’s location and size of the speaker’s face and the length and \isi{font} size of the subtitle” (\citeyear{Hu2013}). In order to alleviate eye-strain, “subtitles of subsequent speaking video segments [were] to follow the positions of the preceding subtitles” (\citeyear{Hu2013}) and “not be placed at the screen boundaries in order to allow the viewer to focus more on the central viewing part of the screen” (\citeyear{Hu2013}). Additionally, subtitles should follow moving speakers – splitting subtitles into shorter segments that would follow the speaker’s path “rather than letting the subtitles float with the speaker” (\citeyear{Hu2013}). If a cut would render a subtitle in an unfavourable position or the speaker disappears, it should be placed “at a default position (such as the bottom of the screen)” \citep{Hu2013}. Compared to \citet{Hong2010}, Hu et al. found an improvement by 7.5--17\,\% which they reasoned to be based on a better \isi{placement} algorithm that was not solely based on a saliency map but also placed subtitles far away from non-speakers. In their final evaluation, they showed 11 videos of about 2.3~minutes length in one static and three slightly differing dynamic versions. The 219 participants scored the videos on a scale 1--10 on overall \isi{viewing experience} and \isi{eyestrain} level. Hu et al. found the \isi{speaker detection} of their system to be “more accurate than the method by \citet{Hong2010}” \citep{Hu2013} and achieved an accuracy of over 90\,\% for \isi{speaker detection} for their so-called “speaker-following subtitles” (\citeyear{Hu2013}) they also see suitable for 2D and 3D games – with the limitation of identifying off-screen speakers.

While this can only be a basic overview of some of the recent studies concerning automatic \isi{speaker detection} and \isi{text placement} in audiovisual material, it is visible that automation is indeed possible and audience feedback is already quite positive. However, these systems have only been tested with questionnaires and it is unclear whether traditional subtitling guidelines were adhered and how far. Furthermore, not much thought went into \isi{layout}, \isi{interface design}, and \isi{usability} of the created subtitles. No analysis of existing integrated titles has taken place so far, the reception of ‘speaker-following subtitles’ has not yet been analysed with \isi{eye tracking} or EEG, and the studies apparently were not combined with film studies, translation studies or any other connected field of study.

\section{Summary}\label{sec:3.5}

This chapter has provided a basic overview of integrated titles and associated fields of study. In addition to the objective shortcomings determined in \sectref{sec:1.3}, emotional reactions and the audience’s view have to be taken into account as well. These are neutral at best, and as Rawsthorn puts it, even when quality, budget size, set, costumes, and cinematography are at their best, “subtitles are still dire” (\citeyear{rawsthorn2007}). As the subtitling takes place after the film’s post-production, there is less or even no control through the filmmakers and producers. Furthermore, development is held back “by the inertia of convention and the ideology of corruption” (\citealt{nornes1999}:~31). Technologically speaking, there have not been limitations that would prevent the creation of more creative titles since the digitalisation of film. This has, however, not led to a paradigm change in either research or the industry and the statement by Nornes still holds true. Therefore, it seems like research has to lead the way this time and show how these recent developments can improve reception and \isi{enjoyment} as well as accessibility for audiovisual products – with the aim to motivate further change and developments in the industry.

The few creative examples in commercial films come from the “imagination of film directions and editors” (\citealt{mcclarty2012}:~140), that, similarly to the intentions behind additional text elements discussed in \chapref{overview}, intend to not only translate but also add or support tone and atmosphere. Beyond traditional translation skills in the area of audiovisual translation, the following skills are therefore required for the creation of integrated titles:

\begin{itemize}
\item Understanding of filmmakers’ intentions and basic film studies to interpret atmosphere and tone:
  \begin{itemize}
  \item Shot compositions
  \item Tools and rules
  \item Emotion and story (e.g. for \isi{layout}, but also timing and content translation)
  \item Rhythm (e.g. for title timing)
  \item Use of three-dimensional space (\isi{speaker identification}, \isi{indication of speaking direction}, atmosphere)
  \item Image systems: titles should follow a continuative system that provides \isi{recognition value} and supports the film without becoming the \isi{main focus}
  \item Eye-trace (title \isi{placement})
  \end{itemize}
\item Ability to read \isi{typographic identity} of a film; understand design and \isi{layout} choices
\end{itemize}

Based on the criteria from \isi{communication design} studies, \isi{usability} studies, film studies, and automation processes, the following characteristics are required from good integrated titles:

\sloppy
\begin{itemize}
\item Intuitiveness (\isi{learnability}, efficiency, and \isi{memorability})
\item Usefulness 
  \begin{itemize}
  \item Suitable translation (e.g. preventing \isi{negative acoustic feedback} effects)
  \item Consistency (following comprehensible rules and avoid irritation, frustration or amusement when not intended)
  \item Readability and \isi{legibility}
  \item Reduced \isi{eyestrain} (small distance between consecutive titles)
  \item Close to action and close to speaker (\isi{speaker identification})
  \end{itemize}
\item Satisfaction, based on the combination of pleasant \isi{layout} and \isi{comprehensible design concept}: 
  \begin{itemize}
  \item Titles are within \isi{safe area}
  \item Suitable \isi{typeface}
  \item Legible colour combinations
  \item Saturation index <85\,\%
  \item Not obscuring speaker’s mouth, other text elements or important activity
  \end{itemize}
\end{itemize}
\fussy 

Provided with full access to existing text elements and this basic set of additional skills, the creation of integrated titles should be possible. However, especially the sources on \isi{usability} demand comprehensible rule sets for titles (at least within a film). As the \isi{layout} and design should be an individual decision process for each film, what remains are the \isi{placement} strategies. Therefore, the following chapter gives an overview of existing commercial integrated titles and the most frequent \isi{placement} strategies in mainstream films.

