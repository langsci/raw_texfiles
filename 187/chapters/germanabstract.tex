\section{German Abstract}

Während die Synchronisation in Deutschland eigentlich dominiert, ist gerade im Bereich der Serien eine starke Bewegung hin zur Originalversion zu spüren – und somit oft auch zur Untertitelung. Die in der Regel automatisch über eine Software eingefügte Untertitelung kann jedoch deutlich die Bildkomposition stören und lenkt dann vom eigentlichen Bildgeschehen ab. Darüber hinaus steht meist ein einziges, umfassendes Konzept und damit verbundene Arbeit hinter einer Filmproduktion, welches durch die in der Regel rücksichtslos platzierten Untertitel oft ge-, wenn nicht sogar zerstört wird. Daraus ergeben sich die folgenden Leitfragen: Können rücksichtsvoll platzierte, integrierte Titel das Filmerlebnis qualitativ verbessern? Hat der Zuschauer dadurch mehr Zeit für das Bild und erhöhen sich somit die Informationsaufnahme und der Unterhaltungswert?

Anhand einiger Szenen aus der britischen BBC-Serie \textit{Being Human} wurde in einer Pilotstudie \citep{Fox2012} bereits grundlegend gezeigt, wie integrierte Titel den Informationsfluss steigern können und dabei die Bildkomposition erhalten sowie Rücksicht auf die Ästhetik des Filmbildes genommen wird. In einem dreistufigen Experiment wurden die Vor- und Nachteile integrierter Titel anhand von Augenbewegungsdaten von mehr als 45 Probanden analysiert. Im Zuge der vorliegenden Dissertation wurde dieses Experiment in deutlich größerem Rahmen anhand der der Dokumentation \textit{Joining the Dots} von Pablo Romero-Fresco erneut durchgeführt. Hier konnten die integrierten Titel in Abstimmung mit dem Produzenten platziert werden und erneut sowohl der natürliche Fokus von englischen Muttersprachlern sowie der abweichende Fokus von deutschen Zuschauern, die auf Untertitel angewiesen sind, bestimmt werden. Die Daten zeigen, dass eine Reduktion der notwendigen Sakkaden (Augenbewegungen zwischen den Fokuspunkten) dem Zuschauer deutlich mehr Zeit zur Betrachtung des Bildes gibt und die Titel leichter mit dem Geschehen der Handlung verknüpft werden können. Darüber hinaus wird die Kombination aus Filmmaterial und Titel deutlich ästhetischer und als mehr originalgetreu wahrgenommen. Auf Basis der Eyetrackingdaten, die sowohl eine niedrigere Lesezeit als auch eine kaum auffällig höhere Reaktionszeit bei integrierten Titeln nahelegen, wurden erste modulare Richtlinien sowie ein Workflow zur Erstellung integrierter Titel und dem Erhalt typografischer Filmidentität formuliert.
