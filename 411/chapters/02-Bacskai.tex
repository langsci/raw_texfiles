\documentclass[output=paper]{langscibook} 
\ChapterDOI{10.5281/zenodo.10123629}

\author{Julia Bacskai-Atkari\affiliation{University of Amsterdam;University of Potsdam}}

\title[Doubling in South Slavic relative clauses]
      {Doubling in South Slavic relative clauses and the predictability of morphosyntactic features}
% replace the above with your title
\abstract{The paper investigates the morphosyntactic properties of relative markers in South Slavic. In Slavic languages, like in many other European languages, relative clauses can be introduced by two kinds of relative markers: (i) relative complementisers, which are invariant in their form, and (ii) relative pronouns, which are inflected (for case, number, and gender, depending on the language). Slavic languages regularly use wh-based complementisers and/or pronouns. Crucially, the two cannot co-occur: this ban is not grounded in the syntactic structure per se, but it derives from the feature incompatibility of two wh-based relative markers, which are regularly equipped with an uninterpretable relative feature. The only exception is Macedonian: in this case, however, there is independent evidence for the complementiser to have different features, suggesting that while morphological properties are good predictors for the relevant syntactic constraints, they are not deterministic.

\keywords{demonstrative pronoun, feature checking, finiteness, inflection, interrogative clause, relative clause}
}

\lsConditionalSetupForPaper{}

\begin{document}

%%% uncomment the following line if you are a single author or all authors have the same affiliation
\SetupAffiliations{mark style=none}

\maketitle

% Just comment out the input below when you're ready to go.

\section{Introduction}
There are various elements that can overtly mark and introduce relative clauses; two examples from English are given in \REF{bacsk:ex:english} below:

\ea \label{bacsk:ex:english}
\ea This is the problem \textit{which} we should solve first. \label{bacsk:ex:which}
\ex This is the problem \textit{that} we should solve first. \label{bacsk:ex:that}
\z
\z

\noindent On the one hand, there are differences in the etymology (cf. \citealt{hoppertraugott1993}, \citealt{heinekuteva2002}): relative markers can be interrogative-based, like \textit{which} in \REF{bacsk:ex:which} above (also: \textit{who(m)}, \textit{whose} etc.), or demonstrative-based, like \textit{that} in \REF{bacsk:ex:that} above.

On the other hand, there are differences in the position of these elements: relative markers can be relative pronouns, like the interrogative-based English pronouns \textit{which}, \textit{who(m)} etc. and the demonstrative-based German pronouns \textit{der}/\textit{die}/\textit{das} etc., or they can be relative complementisers, like the demonstrative-based English \textit{that} and the interrogative-based South German \textit{wo} (cf. \citealt{bayer1984}, \citealt{salzmann2006, salzmann2017}, \citealt{brandnerbraeuning2013}, \citealt{weiss2013}). Given the positional differences, it is not surprising that doubling patterns consisting of an overt relative operator and an overt relative complementiser are attested, as illustrated in \REF{bacsk:ex:englishdoubling}:

\ea \% This is the problem \textit{which that} we should solve first. \label{bacsk:ex:englishdoubling}
\z

\noindent As indicated (\%), this pattern is not accepted in all varieties of English (it is, for instance, excluded from the standard variety).

Regarding Germanic, \citet{bacskaiatkari2020jcgl} made the observation that while overt relative pronouns and overt relative complementisers can be combined, these combinations appear to be restricted by the etymology, in that only asymmetric combinations are attested as genuine \textsc{rel}$+$\textsc{rel} combinations; that is, as combinations where both elements are attested as relative markers on their own as well.\footnote{As will be discussed in \sectref{bacsk:sec:germanic}, this is not merely the result of what items are available. Both in English and in German, wh-based pronouns are available; in addition, both of these languages have varieties where wh-based complementisers are attested. Nevertheless, \textsc{wh}$+$\textsc{wh} combinations are not attested in these varieties either.} This observation raises several questions. First, it should be clarified how strong the generalisation is cross-linguistically: in this article, I am going to examine Slavic data in this respect, as Slavic languages are known to have the various kinds of relative markers mentioned above. Consider the following examples from Bosnian-Croatian-Serbian (henceforth BCS):

\ea \label{bacsk:ex:sc}
\ea \gll čovjek \textit{što} puši \label{bacsk:ex:scstofirst}\\
man that smokes\\
\glt `a/the man that smokes/is smoking' \hfill (\citealt[27]{gracaninyuksek2013})\\
\ex \gll čovjek \textit{koji} puši \label{bacsk:ex:sckojifirst}\\
man which.\textsc{nom} smokes\\
\glt `a/the man who smokes/is smoking' \hfill (\citealt[26]{gracaninyuksek2013})
\z
\z

\noindent The relative clause is introduced by the complementiser \textit{što} in \REF{bacsk:ex:scstofirst} and by the relative pronoun \textit{koji} (inflected for case) in \REF{bacsk:ex:sckojifirst}. Both of these elements are wh-based: as will be discussed in \sectref{bacsk:sec:data}, this is the regular Slavic pattern (see \citealt{auderset2020} for typological insights). The relevance of this pattern for testing the validity of the above-mentioned hypothesis is clear: while Germanic languages tend to have asymmetric patterns due to the availability of demonstrative-based relative markers, the wh-based Slavic patterns may provide us insights into whether the lack of \textsc{wh}$+$\textsc{wh} patterns is systematic or rather coincidental in nature.

Second, the question arises how apparently excluded combinations can be analysed synchronically: while pointing to the etymology may be satisfactory for descriptive purposes, it is highly unlikely that it can be taken as a grammatical constraint per se. In this article, I will argue that the etymological differences correspond to differences formulated in terms of morphosyntactic features.

Third, related to this, the question arises what independent evidence we have for the featural properties of individual elements. Without such independent evidence, simply translating etymological differences into features would again amount to mere descriptive adequacy. The present paper argues that the combinations are restricted by the distribution of [rel] features that are ultimately determined by the etymology, but can show subsequent deviations.

The paper is structured as follows. In \sectref{bacsk:sec:germanic}, I am going to briefly discuss the observations for Germanic. In \sectref{bacsk:sec:data}, I will present the data from (South) Slavic, and I will provide an analysis for the doubling patterns in \sectref{bacsk:sec:doubling}.

\section{Germanic} \label{bacsk:sec:germanic}
In Germanic languages, we can observe doubly-filled COMP effects involving an overt pronoun and an overt complementiser; these can be assigned the schematic structure shown in \figref{bacsk:fig:structure}.\footnote{I adopt a single CP analysis for doubling in relative clauses, following \citet{bacskaiatkari2020jcgl}; under this view, there are no designated projections for left-peripheral elements, unlike in cartographic approaches (going back to \citealt{rizzi1997}). Note also that while doubling is attested in these languages, it is altogether not very frequent (unlike in embedded interrogatives, where doubly-filled COMP effects are widely attested). \citet{bacskaiatkari2022jb} attributes this to discourse factors: the relative pronoun is essentially redundant (at least when the relative complementiser is overt).}

\begin{figure} 
\caption{The structure of doubly-filled COMP}
\label{bacsk:fig:structure}
\begin{forest}
[CP
	[which]
	[C$'$
		[C
			[that]
		]
		[TP]
	]
]
\end{forest}
\end{figure}

The combination of a wh-pronoun and a d-complementiser can be observed in non-standard varieties of English (see \citealt{vangelderen2009}) and marginally also in Swedish, as shown by the data in \REF{bacsk:ex:engswe}.

\ea \label{bacsk:ex:engswe}
\ea It's down to the community \textit{in which that} the people live. \label{bacsk:ex:inwhichthat}\\\hfill (\citealt[59]{vangelderen2013})\\
\ex \gll Detta	är studenten \textit{vilken} \textit{som} bjöd in	Mary.\\
this is	the.student	which that invited in	Mary\\
\glt `This is the student who invited Mary.'\\\hfill (\citealt[247]{bacskaiatkaribaudisch2018})
\z
\z

\noindent The combination of a d-pronoun and a wh-complementiser can be observed in South German dialects (\citealt{brandnerbraeuning2013}, \citealt{weiss2013}, \citealt{fleischer2017}), illustrated for Hessian and for (North) Bavarian in \REF{bacsk:ex:hessian} and in \REF{bacsk:ex:dfcbavarian}, respectively:\footnote{In these varieties, the wh-based complementisers also regularly introduce relative clauses on their own. The complementiser \textit{wo} has a wider distribution geographically; note that it is not used as a declarative complementiser or as a mere finiteness marker.}

\ea
\ea \gll Des Geld, \textit{des} \textit{wo} ich verdiene, des geheert mir. \label{bacsk:ex:hessian}\\
the.\textsc{n} money that.\textsc{n} \textsc{rel} I earn.\textsc{1sg} that.\textsc{n} belongs I.\textsc{dat}\\
\glt `The money that I earn belongs to me.' \hfill (\citealt{fleischer2017})\\
\ex \gll Mei Häusl (\ldots), \textit{dös} \textit{wos} dorten unten (\ldots) steht \label{bacsk:ex:dfcbavarian}\\
my house.\textsc{dim} {} that.\textsc{n} \textsc{rel} there below {} stands\\
\glt `My little house, which stands down there' \hfill (\citealt[780]{weiss2013})
\z
\z

\noindent Given the differences between elements related to position and etymology, there are four logically possible configurations; out of these, only two are attested as genuine \textsc{rel}$+$\textsc{rel} combinations (that is, where both members are independently and productively attested as relative markers). This is shown in \tabref{bacsk:tab:markers}.

\begin{table}
\caption{Combinations of genuine relative markers}
\label{bacsk:tab:markers}
 \begin{tabularx}{.8\textwidth}{lcc}
  \lsptoprule
            & d-complementiser & wh-complementiser\\
  \midrule
  d-pronoun  &   --/?? & $+$\\
  wh-pronoun  &   $+$ & --\\
  \lspbottomrule
 \end{tabularx}
\end{table}

While the asymmetric combinations are straightforward, the \textsc{d}$+$\textsc{d} combination is at least questionable. On the surface, this kind of combination is attested in Waasland Dutch (\citealt{boef2013}), as shown in \REF{bacsk:ex:dutch}.

\ea \gll Dat is de man \textit{die} \textit{dat} het verhaal verteld heeft. \label{bacsk:ex:dutch}\\
that is the man who that the story told has\\
\glt `That is the man who has done it.' \hfill (\citealt[93]{boef2008})
\z

\noindent In this case, however, it is very probable that the combination cannot be considered as genuine \textsc{rel}$+$\textsc{rel}. In Dutch, relative clauses introduced by a single \textit{dat} (as a complementiser) are found in Vlaams-Brabant Dutch (\citealt{boef2013}) and thus not in the same area where the doubling pattern is attested: in the doubling pattern in \REF{bacsk:ex:dutch}, then, the complementiser marks finiteness, not [rel].\footnote{The availability of \textit{dat} as a finiteness marker is also independently motivated: it is also attested in embedded constituent questions across Dutch dialects, that is, in environments where it cannot be a declarative complementiser (see \citealt{schallertdroegepheiff2018} for a recent discussion). Another potential counterexample to the generalisation in \tabref{bacsk:tab:markers} comes from Old English (see \citealt{vangelderen2009}), as illustrated below:

\ea \gll ac gif we asmeagaþ þa eadmodlican dæda \textit{þa} \textit{þe} he worhte, þonne ne þincþ us þæt nan wundor\\
but if we consider those humble deeds that that he wrought then not seems us that no wonder\\
\glt `But if we consider the humble deeds which he wrought, that will seem no wonder to us.' \hfill (\textit{Blickling Homilies} 33; \citealt[364]{watanabe2009}, citing \citealt{allen1980})
\z

\noindent In Old English, we find the above doubling pattern as an intermediate stage in the process of reanalysis of one of the d-pronouns (\textit{that}) into a complementiser, removing the original complementiser \textit{þe} (\citealt{vangelderen2009}): this suggests that \textit{þe} was possibly only a finiteness marker, or that the pronoun was initially still a demonstrative but not [rel]. This (and the Waasland Dutch pattern) crucially differs from the present-day English pattern, where \textit{that}-relatives are common and productive: in other words, there is no reason to assume that patterns like \REF{bacsk:ex:inwhichthat} would involve a mere finiteness marker.}

In other words, there is no strong evidence for the existence of genuine \textsc{d}$+$\textsc{d} doubling. More importantly, no combinations of the form ``wh-pronoun $+$ wh-complementiser'' are attested (even though they would be logically possible in certain varieties, such as in English with the complementiser \textit{what} and in South German with the complementisers \textit{wo} and \textit{was}).

\section{The data} \label{bacsk:sec:data}
\subsection{Relative markers in South Slavic} \label{bacsk:sec:relativemarkers}
South Slavic languages are particularly interesting regarding the above generalisation, since these languages regularly use wh-based elements (cf. \citealt[36]{kljajevic2012}, \citealt{auderset2020}) as relative markers. In addition, both major strategies (that is, pronouns versus complementisers) are attested in (South) Slavic languages.

Consider again the examples from BCS in \REF{bacsk:ex:sc}, repeated here for the sake of convenience as \REF{bacsk:ex:screpeat}:

\ea \label{bacsk:ex:screpeat}
\ea \gll čovjek \textit{što} puši \label{bacsk:ex:scsto}\\
man that smokes\\
\glt `a/the man that smokes/is smoking' \hfill (\citealt[27]{gracaninyuksek2013})\\
\ex \gll čovjek \textit{koji} puši \label{bacsk:ex:sckoji}\\
man which.\textsc{m.nom} smokes\\
\glt `a/the man who smokes/is smoking' \hfill (\citealt[26]{gracaninyuksek2013})
\z
\z

\noindent In \REF{bacsk:ex:scsto}, the relative clause is introduced by the complementiser \textit{što}; in \REF{bacsk:ex:sckoji}, it is introduced by the relative pronoun \textit{koji}, which is, unlike the complementiser, inflected for case. This becomes evident if we compare the elements above, which occur in subject relative clauses, to their counterparts in direct object relative clauses, as shown in \REF{bacsk:ex:scobjectsto} and \REF{bacsk:ex:scobjectkojeg}, and in indirect object relative clauses, as shown in \REF{bacsk:ex:scindobjectsto} and \REF{bacsk:ex:scindobjectkojem}:

\ea \label{bacsk:ex:doio}
\ea \gll čovjek \textit{što} ga Jan vidi \label{bacsk:ex:scobjectsto}\\
man that \textsc{3sg.acc.cl} Jan sees\\
\glt `a/the man who Jan sees' \hfill (\citealt[27]{gracaninyuksek2013})\\
\ex \gll čovjek \textit{kojeg} Jan vidi \label{bacsk:ex:scobjectkojeg}\\
man which.\textsc{m.acc} Jan sees\\
\glt `a/the man who Jan sees' \hfill (\citealt[27]{gracaninyuksek2013})\\
\ex \gll čovjek \textit{što} mu Jan pokazuje put \label{bacsk:ex:scindobjectsto}\\
man.\textsc{nom} that \textsc{3sg.dat.cl} Jan.\textsc{nom} shows way.\textsc{acc}\\
\glt `a/the man to whom Jan shows/is showing the way'\\\hfill (\citealt[27]{gracaninyuksek2013})
\ex \gll čovjek \textit{kojem} Jan pokazuje put \label{bacsk:ex:scindobjectkojem}\\
man.\textsc{nom} which.\textsc{m.dat} Jan.\textsc{nom} shows way.\textsc{acc}\\
\glt `a/the man to whom Jan shows/is showing the way'\\\hfill (\citealt[27]{gracaninyuksek2013})
\z
\z

\noindent As can be seen, while \textit{što} does not change its form, the relative pronoun is inflected for accusative and dative case.\footnote{Note also another difference between the two strategies in \REF{bacsk:ex:doio}, which cannot be seen in \REF{bacsk:ex:screpeat}: the direct object and the indirect object relative clauses with \textit{što} contain a resumptive pronoun (\textit{ga} and \textit{mu}, respectively), while this is not the case in the counterparts containing the relative pronoun. Resumptive pronouns are used to lexicalise the gap in certain languages: since in this respect they are similar to relative pronouns, it is actually expected that they should not co-occur with the relative pronoun while they can (and in the given cases, must, see \citealt[27]{gracaninyuksek2013}) surface when the relative clause is introduced by a complementiser. In this respect, the presence/absence of resumptive pronouns in \REF{bacsk:ex:doio} is yet another indicator for the structural difference between the relative markers under scrutiny. Note that the absence of resumptive pronouns in subject relative clauses is also expected: resumptive pronouns are more likely to occur in functions that are lower in the Noun Phrase Accessibility Hierarchy, and subjects constitute the highest function, so that the use of resumptive pronouns in this function is extremely rare cross-linguistically (\citealt{keenancomrie1977}).} Importantly, relative operators are phonologically identical to their interrogative counterparts (also inflected for case, number and gender); \textit{što} is phonologically identical to the most unmarked interrogative form (nominative/accusative; the dative would be \textit{čèmu}). The interrogative patterns are illustrated in \REF{bacsk:ex:int} below:

\ea \label{bacsk:ex:int}
\ea \gll \textit{Što} je Marija videla?\\
what.\textsc{acc} \textsc{aux} Mary seen\\
\glt `What did Mary see?' \hfill (\citealt[77]{halpern1995})\\
\ex \gll \textit{Koji} čovek je voleo Mariju?\\
which.\textsc{m.nom} man \textsc{aux} seen Mary.\textsc{acc}\\
\glt `Which man saw Mary?' \hfill (\citealt[78]{halpern1995})\\
\ex \gll \textit{Koju} \v{z}abu je lane liznulo?\\
which.\textsc{f.acc} frog.\textsc{acc} \textsc{aux} fawn lick.\textsc{ptcp}\\
\glt `Which frog did the fawn lick?' \hfill (\citealt[34]{kljajevic2012})
\z
\z

\noindent The syntactic positions of the relevant elements are illustrated in Figures~\ref{bacsk:fig:sctreekoji} and~\ref{bacsk:fig:sctreesto}. We can observe the same variation between complementisers and pronouns in Macedonian, as shown in \REF{bacsk:ex:macedonianbasic}.

\begin{figure}
\begin{floatrow}
\captionsetup{margin=.05\linewidth}
\ffigbox{\begin{forest}
[CP
	[koji/kojeg/\ldots{}]
	[C$'$
		[C [$\emptyset$]]
		[TP]
	]
]
\end{forest}}
{\caption{The position of relative pronouns in Slavic}\label{bacsk:fig:sctreekoji}}

\ffigbox{\begin{forest}
[CP
	[Op]
	[C$'$
		[C [što]]
		[TP]
	]
]
\end{forest}}
{\caption{The position of relative complementisers in Slavic}
\label{bacsk:fig:sctreesto}}
\end{floatrow}
\end{figure}

\ea \label{bacsk:ex:macedonianbasic}
\ea \gll Covekot \textit{koj} vleze e moj sosed.\\
man.the.\textsc{m.sg} who.\textsc{m.sg} come.\textsc{aor.3sg} is my.\textsc{m.sg} neighbour\\
\glt `The man who came in is my neighbour.' \hfill (\citealt[232]{buzarovska2009})\\
\ex \gll Covekot \textit{što} go sretnavme e moj sosed.\\
man.the.\textsc{m.sg} that \textsc{3sg.acc.cl} meet.\textsc{aor.1pl} is my.\textsc{m.sg} neighbour\\
\glt `The man whom we met is my neighbour.' \hfill (\citealt[232]{buzarovska2009})
\z
\z

\noindent Again, both elements are interrogative-based. This is illustrated in \REF{bacsk:ex:intmac} below:

\ea \label{bacsk:ex:intmac}
\ea \gll \textit{\v{S}to} jade deteto?\\
what eats child.the\\
\glt `What does the child eat?' \hfill (\citealt[134]{lazarovanikovska2003})\\
\ex \gll \textit{Koj} te potseti?\\
who.\textsc{cl} \textsc{2sg.acc.cl} reminded.\textsc{3sg.perf.prs}\\
\glt `Who reminded you?' \hfill (\citealt{tomic2006})
\z
\z

\noindent Slovene also makes use of both strategies, as illustrated in \REF{bacsk:ex:slovenedata}:

\ea \label{bacsk:ex:slovenedata}
\ea \gll Poznam človeka, \textit{katerega} so iskali.\\
know.\textsc{1sg} man.\textsc{acc} which.\textsc{acc} \textsc{aux.3pl} looked.for\\
\glt `I know the man who they were looking for.' \hfill (\citealt[10]{hladnik2010})\\
\ex \gll Poznam človeka, \textit{ki} so ga iskali.\\
know.\textsc{1sg} man.\textsc{acc} that \textsc{aux.3pl} \textsc{m.acc.cl} looked.for\\
\glt `I know the man that they were looking for.' \hfill (\citealt[10]{hladnik2010})
\z
\z

\noindent The relative pronoun is inflected and it is obviously a wh-based element (\citealt[225]{mitrovic2016}); the complementiser \textit{ki} lacks an interrogative counterpart in the modern language (\citealt[225]{mitrovic2016}) but it derives from Proto-Indo-European *\textit{kʷís} `who, what' and Slovene \textit{ki} developed into an interrogative complementiser after the 14th century (\citealt[225]{mitrovic2016}). As \citet[38]{hladnik2010} notes, citing \citet{cazinkic2001}, \textit{ki} is often perceived to be a reduced form of the relative pronoun, which is etymologically wrong. Further, prescriptive rules favour the pronoun strategy over the complementiser strategy (\citealt[38]{hladnik2010}): this is in fact reminiscent of the situation in West Germanic.

\subsection{A note on Bulgarian} \label{bacsk:sec:bulgarian}
Bulgarian represents a special case within South Slavic regarding relative markers. Both strategies (the pronoun strategy and the complementiser strategy) can be observed in Bulgarian, with the colloquial complementiser \textit{deto} (\citealt{rudin2014}) and with regular relative pronouns, as shown by the corpus examples taken from \citet{buzarovska2009} in \REF{bacsk:ex:bulgariancorp}:

\ea \label{bacsk:ex:bulgariancorp}
\ea \gll Imaše xora, \textit{koito} ne viždaxa ništo pred sebe si.\\
have.\textsc{imperf.3sg} people who.\textsc{pl} not see.\textsc{imperf.3sg} nothing before own \textsc{cl}\\
\glt `There were people who saw nothing in front of them.'\\ \hfill(\citealt[249]{buzarovska2009})\\
\ex \gll Da bjaxa mi kazali, \v{c}e ima xora, \textit{deto} bjagat ot dobroto kato zajci ot kopoj\ldots{}\\
\textsc{sm} be.\textsc{pl.imperf} \textsc{1sg.dat.cl} told.\textsc{pl.part} that has people that run.\textsc{3pl} from good.the.\textsc{n.sg} like rabbits from hound.\textsc{m.sg}\\
\glt `If I had been told that there are people who run away from good like rabbits from a hound\ldots{}' \hfill (\citealt[249]{buzarovska2009})
\z
\z

\noindent The relative operator is evidently wh-based; as for \textit{deto}, it also goes back to an interrogative operator (\citealt[234]{buzarovska2009}; see \citealt[1241]{krapova2010} for a more detailed analysis) and, as mentioned above, it counts as colloquial, reminiscent of the prescriptive preferences for relative pronouns in Slovene and in West Germanic.

Note that the situation in Bulgarian is in fact somewhat more complex, as wh-pronouns in relative pronouns are apparently complex: \textit{koito} consists of the wh-base \textit{koj} and the element -\textit{to} (this pattern is productive, e.g. \textit{kakvo-to} `what' or \textit{kolko-to} `how much'), whereby the status of -\textit{to} is subject to much debate, as discussed by \citet{rudin2014} in detail. The most important question in this respect is whether the combination is primarily syntactic (involving distinct syntactic positions) or morphological (involving a single syntactic node). Unlike \textit{\v{s}to}, -\textit{to} is not available as a complementiser in other constructions and it does not resemble a wh-element either (\citealt[322]{rudin2014}). \citet{rudin2009} analyses this element as a specifically relative complementiser: in this case, Bulgarian would in fact show doubling, but note that as -\textit{to} is not a wh-based element, this does not go against the generalisation under scrutiny here, i.e. that \textsc{wh}$+$\textsc{wh} combinations are regularly not attested; further, -\textit{to} is not available as a relative marker on its own, so that a genuine \textsc{rel}$+$\textsc{rel} doubling pattern would not arise either. \citet[324]{rudin2014} remarks that the complementiser approach faces problems with complex wh-phrases such as \textit{kolkoto goljam} `how big', where -\textit{to} appears to be incorporated into the wh-phrase. According to \citet{rudin2014}, a further problem lies in the fact that the complementiser account would predict more parallelism with \textit{\v{s}to}, which is problematic as e.g. \textit{\v{s}to} in Macedonian is banned from comparatives but Bulgarian -\textit{to} is not. This is, however, not a strong counterargument: as argued by the present paper, relative complementisers may show different behaviour (and distribution) due to their different featural properties; further, relative complementisers appearing in comparatives show considerable variation, and \textit{\v{s}to} is in fact available in comparatives in BCS (see \citealt{bacskaiatkari2016alh} for discussion). Other analyses include treating -\textit{to} as a definiteness marker (e.g. \citealt{izvorski2000}; see \citealt{rudin2009} and \citealt[322--323]{rudin2014} for counterarguments) or as a morphological marker of relative pronouns (\citealt{hauge1999}, see \citealt[325]{rudin2014} for some concerns): in these cases, however, no complex left periphery is involved and these accounts would again not be problematic for the issues discussed in the present paper. For this reason, Bulgarian -\textit{to} will not be discussed in \sectref{bacsk:sec:doubling}.

\subsection{Interim summary and outlook} \label{bacsk:sec:interim}
In sum, it is evident that South Slavic languages by default show variation between the relative complementiser strategy and the relative operator strategy. It is worth mentioning that this kind of variation is not restricted to South Slavic but can be more generally observed across Slavic languages, though the exact distribution and acceptability patterns differ.

In West Slavic, the standard option seems to be the use of relative pronouns, but once non-standard varieties are also taken into account, we can also find relative complementisers in these languages, i.e. Czech and Polish \textit{co} and Slovak \textit{čo} (\citealt{simik2008}, \citealt{guz2017}, \citealt{minlos2012}).

In East Slavic, both relative pronouns and relative complementisers are attested: while Russian \textit{čto} is a markedly colloquial option (\citealt{meyer2017}), Ukrainian and Belarusian \textit{\v{s}\v{c}o} seems to be more widespread (\citealt{danylenko2018}).

In other words, the variation between the relative complementiser strategy and the relative operator strategy is not restricted to South Slavic languages but can be found more generally in Slavic languages. The complementiser strategy is overall more restricted; South Slavic seems to offer the best testing ground for potential \textsc{wh}$+$\textsc{wh} combinations. For this reason, I am going to restrict myself to the discussion of South Slavic data in the discussion to follow.

\section{Doubling} \label{bacsk:sec:doubling}
\subsection{A note on features} \label{bacsk:sec:note}
I adopt standard minimalist assumptions regarding formal features, going back to \citet{chomsky1995}; see also \citet{zeijlstra2014}. According to this, the kind of features that can participate in morphosyntactic operations are called formal features: this set of features intersects with semantic features. Interpretable formal features are in the intersection; uninterpretable features are pure formal features (they cannot be interpreted at LF) and need to be checked off (or, in more recent terms, valued); this can be done via a matching interpretable feature. Note that the presence of any uninterpretable feature, [u-F], on a certain element implies only that the particular feature is not interpretable on that given element in LF, and it does not imply in any way that the given element would lack other semantic features (or meaning).
\subsection{The analysis of doubling patterns} \label{bacsk:sec:analysisdoubling}
As mentioned in \sectref{bacsk:sec:germanic}, doubling patterns appear to be asymmetric; this observation led \citet{bacskaiatkari2020jcgl} to the hypothesis that the observed differences may be due to differences in the interpretability of [rel] features. According to this, we should have the following distribution: d-pronouns and d-complementisers are [i-rel] and wh-pronouns and wh-complementisers are [u-rel].\footnote{One might wonder why this should be so: so far, this hypothesis gives the right empirical predictions, yet it would be desirable to detect more general properties behind the particular feature distribution. As far as Germanic is concerned, it is evident that demonstrative-based elements constitute the older strategy (see \citealt[467]{ringetaylor2014} for Old English \textit{þe} and \citealt[46]{axeltober2017} for Old High German \textit{the}): wh-based elements were introduced later into headed relative clauses, via analogy (from free relatives and interrogatives). Apart from this, note that the source elements differ in terms of definiteness features: demonstratives are definite, while the wh-base itself is indefinite (see \citealt{watanabe2009}, who also shows that the indefinite wh-base in English was also quantificational, turning the clause into a complete proposition, which was incompatible with headed relatives). Relative pronouns are co-referential with the head noun under a matching analysis (cf. \citealt[55--179]{salzmann2017}) and definite pronouns are thus natural candidates as anaphors. Indeed, the reanalysis of demonstrative markers into C-elements is traditionally considered to have evolved from paratactic structures involving a genuine demonstrative pronoun, since such examples are indeed possible and attested unlike with interrogative pronouns (but see \citealt{axeltober2017} for a critical evaluation of this as the sole trigger of the relevant changes). In this sense, it is possible that the features [i-rel] and [u-rel] are ultimately related to the definite versus indefinite distinction, respectively. Future research will have to determine whether this idea is on the right track and, if so, how the diachronic feature inheritance can be modelled.}

At any rate, the asymmetric patterns ensure proper feature checking, as shown in Figures~\ref{bacsk:fig:treewhd} and~\ref{bacsk:fig:treedwh}. In both configurations, the uninterpretable feature is properly checked off by its interpretable counterpart. By contrast, symmetric patterns are essentially problematic for feature checking. In the case of two [i-rel] features, the movement of the operator is not motivated; in the case of two [u-rel] features, the uninterpretable feature cannot be checked off. Relative complementisers regularly encode finiteness, [fin].

\begin{figure}
\begin{floatrow}
\captionsetup{margin=.05\linewidth}
\ffigbox
{\begin{forest}
[CP
	[which\textsubscript{{[}u-rel{]}}]
	[C$'$
		[C\textsubscript{{[}i-rel{]},{[}fin{]}} [that\textsubscript{{[}i-rel{]},{[}fin{]}}]]
		[TP]
	]
]
\end{forest}}
{\caption{Features in \textsc{wh}$+$\textsc{d}}\label{bacsk:fig:treewhd}}

\ffigbox
{\begin{forest}
[CP
	[der\textsubscript{{[}i-rel{]}}]
	[C$'$
		[C\textsubscript{{[}u-rel{]},{[}fin{]}} [wo\textsubscript{{[}u-rel{]},{[}fin{]}}]]
		[TP]
	]
]
\end{forest}}
{\caption{Features in \textsc{d}$+$\textsc{wh}} 
\label{bacsk:fig:treedwh}}
\end{floatrow}
\end{figure}

Regarding\largerpage[2] the former, we observed in \sectref{bacsk:sec:data} that some d-pronoun $+$ d-com\-ple\-men\-tiser combinations seem to exist, even though they were classed as not genuine. In the case of Waasland Dutch, the complementiser \textit{dat} marks finiteness, and is thus underspecified for [rel]. In the case of Old English, \textit{þe} was in the process of losing its [i-rel] specification, ultimately changing into being underspecified for [rel] and marking finiteness only, similarly to the Waasland Dutch combination.\footnote{Note that this does not make two projections necessary (i.e., one for clause type and one for finiteness, as in cartographic approaches like that of \citealt{rizzi1997} or \citealt{baltin2010}), as also shown by \citet{bacskaiatkari2020jcgl} for embedded interrogatives. Intervening elements (which are often used as arguments for designated projections in cartographic approaches) are not attested in Germanic between clause-type markers (including finiteness markers).} This suggests that \textsc{d}$+$\textsc{d} patterns can be accounted for in this model: an underspecified complementiser is used to lexicalise the complementiser and the abstract [u-rel] feature can be checked off regularly by the pronoun, as illustrated for Waasland Dutch in \figref{bacsk:fig:treedutch}.{\interfootnotelinepenalty=10000\footnote{The mismatch between the underlying syntactic feature bundle and the inserted vocabulary item is in line with the core property of Distributed Morphology called Underspecification, according to which the inserted Vocabulary Items (the phonological expressions of abstract words) are not necessarily fully specified for the particular syntactic positions where they are inserted (see \citealt[401--405]{mcginnisarchibald2016} for a summary; see \citealt{hallemarantz1994}, \citealt{harleynoyer1999}). This is a basic property of Late Insertion and it does not go against inclusiveness (\citealt[225]{chomsky1995}).}}\pagebreak

\begin{figure} 
\caption{Doubling in Waasland Dutch} 
\label{bacsk:fig:treedutch}
\begin{forest}
[CP
	[die\textsubscript{{[}i-rel{]}}]
	[C$'$
		[C\textsubscript{{[}u-rel{]},{[}fin{]}} [dat\textsubscript{{[}fin{]}}]]
		[TP]
	]
]
\end{forest}
\end{figure}

Note that there is independent evidence for the d-complementiser as underspecified for [rel]: the same complementiser appears in declaratives, where there are no head nouns. One might wonder why lexicalising the (finite) C position is necessary: this seems to be a general tendency in Germanic (\citealt{bacskaiatkari2018slavic, bacskaiatkari2020jcgl}) and it is not of further interest in this paper.

Crucially, the more problematic \textsc{wh}$+$\textsc{wh} patterns are not attested in Germanic. However, South Slavic shows variation here: while such combinations are not attested in BCS (\citealt[292]{goodluckstojanovic1996}) and Slovene (\citealt[12--13]{hladnik2010}), this pattern appears to be possible in Macedonian (\citealt[320]{rudin2014}). This is illustrated by the following example:

\ea \gll čovekot \textit{koj-što} zboruva \label{bacsk:ex:macedonian}\\
the.man who-that talks\\
\glt `the man who is talking' \hfill (\citealt[316]{rudin2014})
\z

\noindent The pattern in \REF{bacsk:ex:macedonian} seems to be productive: it is attested with all relative pronouns. The only exception is when the pronoun also has the form \textit{što}, so that the sequence *\textit{što što} is ungrammatical (\citealt[320]{rudin2014}, citing \citealt{kramer1999}). This may well be a phonological constraint (and as such it is not direct evidence against the pronominal status of the second \textit{što} element): as shown by \citet{boskovic2002}, similar constraints can be observed in multiple wh-fronting in Slavic languages.

Importantly, both relative markers in \REF{bacsk:ex:macedonian} are clearly interrogative-based, as their surface-identical counterparts are available as interrogative operators, as shown in \REF{bacsk:ex:intmac} above and in \REF{bacsk:ex:macrudin} below:

\ea\label{bacsk:ex:macrudin}
\ea \gll \textit{Koj} zboruva?\\
who talks\\
\glt `Who is talking?' \hfill (\citealt[315]{rudin2014})\\
\ex \gll \textit{Što} sakaš?\\
what want.\textsc{2sg}\\
\glt `What do you want?' \hfill (\citealt[320]{rudin2014})
\z
\z

\noindent The data thus suggest that \REF{bacsk:ex:macedonian} apparently has a \textsc{wh}$+$\textsc{wh} pattern, which seems to contradict the hypothesis mentioned above. In order to determine to what extent \REF{bacsk:ex:macedonian} actually poses a problem for the theory, the distribution of the complementiser should be examined further. In Macedonian, \textit{što} is also available as a declarative complementiser (\citealt{rudin2014}), as demonstrated in \REF{bacsk:ex:stomac}:

\ea \gll Se raduvam, \textit{što} ve gledam. \label{bacsk:ex:stomac}\\
\textsc{refl} rejoice.\textsc{1sg} that you.\textsc{pl.acc} see.\textsc{1sg}\\
\glt `I am happy that I see you.' \hfill (\citealt[419]{tomic2006})
\z

\noindent This differs from the wh-based complementisers in Germanic, which may also be the reason for the differences regarding the doubling patterns in relative clauses. Regarding the status of \textit{što} in relative clauses, \citet[320]{rudin2014} provides strong arguments that it should definitely taken to be a complementiser (contrary to \citealt{tomic2012}). First, the doubly-filled COMP patterns such as \REF{bacsk:ex:macedonian} indicate that it cannot be a pronoun, as it appears in addition to the relative pronoun:\footnote{Unlike interrogative pronouns, which can co-occur in a single clause, there can only be a single relative pronoun in a relative clause: the head noun is co-referential with the relative pronoun, which can be base-generated only in a single position. See also \citet[320]{rudin2014}.} note that the word order constraint follows from the internal structure of the CP (\citealt{bacskaiatkari2018slavic, bacskaiatkari2020jcgl}). Second, there is independent evidence for \textit{što} being a complementiser otherwise, see \REF{bacsk:ex:stomac} above. Third, prepositions cannot take relative \textit{što} as a complement (the same applies to English \textit{that}).\footnote{This is shown by the following example:

\ea[*]{\gll studentkata, \textit{za} \textit{što} zboruvame\\
student about that speak.\textsc{1pl}\\
\glt Intended: `the student about whom we speak' \hfill (\citealt[320]{rudin2014})}
\z

\noindent \citet[320]{rudin2014}, citing \citet{tomic2012} and \citet{kramer1999}, confirms that such patterns are impossible in relative clauses. Note that this of course does not imply anything about the interrogative pronoun \textit{što} in questions.}

Based on these observations, the structure in itself is not problematic, as it appears to demonstrate the same underlying syntax as the doubling patterns mentioned above and it can be derived from the structures in Figures~\ref{bacsk:fig:sctreekoji} and~\ref{bacsk:fig:sctreesto} in a straightforward way, as shown in \figref{bacsk:fig:treekojsto}.

\begin{figure}
\caption{Doubling in Macedonian}
\label{bacsk:fig:treekojsto}
\begin{forest}
[CP
	[koj]
	[C$'$
		[C [što]]
		[TP]
	]
]
\end{forest}
\end{figure}

The question is rather what the feature specification of \textit{što} is. In essence, there are two possibilities: (i) underspecification for [rel], just like \textit{dat} in Waasland Dutch, or (ii) specification as [i-rel]. 

Regarding the first hypothesis, we can establish the following. Underspecification in itself is plausible under a late insertion approach (\citealt{hallemarantz1993}; see also the discussion in this section above), inasmuch as the abstract underlying head is lexicalised by a partial match (see \figref{bacsk:fig:treedutch} for Waasland Dutch). This assumption is less problematic if the abstract head is [u-rel] than when it is [i-rel], since uninterpretable features are deleted anyway after check-off, so that Vocabulary Insertion taking place in the morphological component (after Spell-Out) does not actually see [u-rel]. The same argumentation does not follow automatically for [i-rel], though: leaving the C position in Macedonian as underspecified or as [u-rel] would require the relative pronoun to be specified as [i-rel], but there is no independent evidence for Macedonian wh-operators to be different from the general properties of wh-based relative markers, that is, creating an exception for wh-based relative pronouns in Macedonian as [i-rel] would be ad hoc.\footnote{This crucially differs from the Dutch scenario, where the d-pronoun can regularly be assumed to have an [i-rel] specification, in line with the general hypothesis.} In principle, this possibility cannot be excluded but making such an assumption without independent evidence would be merely descriptive at this stage.

On the other hand, however, we have independent evidence for \textit{što} having different properties from the Germanic pattern. In the hypothesis formulated in (ii) above, \textit{što} is [i-rel], which actually implies a difference from the Germanic pattern. There are two points of interest here. First, doubling patterns in Germanic are primarily attested in embedded interrogatives and much less in relative clauses (\citealt{bacskaiatkari2022jb}), due to the lexicalisation preference on C: the same does not apply to Slavic. In other words, while both language groups may show doubling patterns in relative clauses, the underlying reasons are likely to be different, and thus it cannot be expected that the two groups show parallel behaviour in all respects. Second, regarding the status of \textit{što}, it should be noted that such relative declarative complementisers in South Slavic introduce factives and not all kinds of declarative clauses, unlike what we can observe in Germanic.

Consider the following examples from BCS:

\ea
\ea \gll Jesam ti rekao \textit{da} je Marija ori\v{s}la na odmor? \label{bacsk:ex:nonfact}\\
\textsc{aux.1sg} you.\textsc{dat} told that \textsc{aux.3sg} Marija gone on vacation\\
\glt `Did I tell you that Marija went on vacation?' \hfill (\citealt[341]{arsenijevic2020})\\
\ex \gll Jesam ti rekao \textit{\v{s}to} je Marija ori\v{s}la na odmor? \label{bacsk:ex:factive}\\
\textsc{aux.1sg} you.\textsc{dat} told that \textsc{aux.3sg} Marija gone on vacation\\
\glt `Did I tell you that Marija went on vacation?' (it is a fact that she did)\\\hfill (\citealt[341]{arsenijevic2020})
\z
\z

\noindent In \REF{bacsk:ex:nonfact}, the embedded clause is non-factive: it may or may not be true that Marija went on vacation. In \REF{bacsk:ex:factive}, however, the embedded clause is factive: this is the context where \textit{što} can appear. As \citet{arsenijevic2020} argues, \textit{što}-declaratives have referential properties and are thus similar to relative clauses (see \citealt[1266]{krapova2010} for Bulgarian and Macedonian and \citealt{buzarovska2009} and \citealt[69]{browne1986} for Macedonian; see also \citealt{aboh2005} for factives being a special kind of relative clause).\footnote{This may be related to the fact that \textit{što}-relatives in BCS are used in relative clauses where the head noun is familiar (see \citealt[341--342]{arsenijevic2020}). Note that the familiarity of the referent (as expressed by the head noun) does not equal definiteness on the relative pronoun, as familiarity and definiteness are distinct (though not unrelated) properties. Consider the following example:

\ea I saw a/the shopkeeper who was wearing a kilt. \label{bacsk:ex:kilt}
\z

\noindent In \REF{bacsk:ex:kilt}, the head noun is either indefinite or definite: this does not affect the relative marker (the pronoun \textit{who}).} However, notice that there is no head noun and no relative operator movement in such configurations: this indicates that \textit{što} cannot be [u-rel] in these constructions, as there would be no element to check off this feature. In other words, while the interrogative element can be assumed to have a regular [u-rel] feature, this feature is lost in factive declaratives.\footnote{Note that the similarities between (factive) declaratives and relative clauses do not make the two constructions equal. In particular, they differ in terms of operator movement, as shown by \citet{arsenijevic2009lingua}. In (headed) relative clauses, the matrix correlate (the head noun) is co-referential with the relative pronoun, which is interpreted in the relativisation site (the base position) and in the CP-domain (the landing site): such elements undergo movement. By contrast, while \citet{arsenijevic2009lingua} assumes that there is also a matrix correlate in (factive) declaratives, the co-referential nominal element in the subordinate clause has its relativisation site at the top of their structure, that is, in the projection that specifies the illocutionary force of the clause. In other words, this configuration involves a higher projection site and no relative operator movement; consequently, the feature checking relation discussed in the present article does not apply.} This leads to the configuration shown in \figref{bacsk:fig:treekojstofeat}.

\begin{figure} 
\caption{Features involved in doubling in Macedonian}
\label{bacsk:fig:treekojstofeat}
\begin{forest}
[CP
	[koj\textsubscript{{[}u-rel{]}}]
	[C$'$
		[C\textsubscript{{[}i-rel{]},{[}fin{]}} [što\textsubscript{{[}i-rel{]},{[}fin{]}}]]
		[TP]
	]
]
\end{forest}
\end{figure}

Note that the loss of [u-rel] does not make [i-rel] automatically available on the inserted lexical items as an inherent property: in particular, there is no \textsc{wh}$+$\textsc{wh} doubling in ordinary relative clauses in BCS, so there is no reason to assume that BCS \textit{što} in ordinary relative clauses would be [i-rel]. By contrast, we can observe \textsc{wh}$+$\textsc{wh} doubling in ordinary relative clauses in Macedonian, indicating that Macedonian \textit{što} is available as [i-rel]. In this way, we can set up an implicational hierarchy: wh-based declaratives are a prerequisite for \textsc{wh}$+$\textsc{wh} doubling in ordinary relative clauses but not vice versa. That is, the existence of wh-based declaratives does not imply the existence of \textsc{wh}$+$\textsc{wh} doubling in ordinary relative clauses.

\section{Conclusion}
In this paper, I examined doubling in South Slavic relative clauses, concentrating on the effects of the morphological inventory: crucially, both wh-pronouns and wh-complementisers are available in these languages. The typological predictions based on Germanic and Slavic are the following: (i) genuine \textsc{wh}$+$\textsc{wh} combinations are not attested, and (ii) the only exception is Macedonian, where the wh-complementiser \textit{što} has different properties (as supported by independent evidence), indicating that further (featural) reanalysis is possible. This indicates that while morphological properties are decisive for most patterns, they do not prohibit further grammaticalisation even in languages where the original wh-element is still available. In this sense, morphological properties are not deterministic, as morphosyntactic features may deviate from the original, predictable patterns.


%For a start: Do not forget to give your Overleaf project (this paper) a recognizable name. This one could be called, for instance, Simik et al: OSL template. You can change the name of the project by hovering over the gray title at the top of this page and clicking on the pencil icon.

\section{Introduction}\label{sim:sec:intro}

Language Science Press is a project run for linguists, but also by linguists. You are part of that and we rely on your collaboration to get at the desired result. Publishing with LangSci Press might mean a bit more work for the author (and for the volume editor), esp. for the less experienced ones, but it also gives you much more control of the process and it is rewarding to see the quality result.

Please follow the instructions below closely, it will save the volume editors, the series editors, and you alike a lot of time.

\sloppy This stylesheet is a further specification of three more general sources: (i) the Leipzig glossing rules \citep{leipzig-glossing-rules}, (ii) the generic style rules for linguistics (\url{https://www.eva.mpg.de/fileadmin/content_files/staff/haspelmt/pdf/GenericStyleRules.pdf}), and (iii) the Language Science Press guidelines \citep{Nordhoff.Muller2021}.\footnote{Notice the way in-text numbered lists should be written -- using small Roman numbers enclosed in brackets.} It is advisable to go through these before you start writing. Most of the general rules are not repeated here.\footnote{Do not worry about the colors of references and links. They are there to make the editorial process easier and will disappear prior to official publication.}

Please spend some time reading through these and the more general instructions. Your 30 minutes on this is likely to save you and us hours of additional work. Do not hesitate to contact the editors if you have any questions.

\section{Illustrating OSL commands and conventions}\label{sim:sec:osl-comm}

Below I illustrate the use of a number of commands defined in langsci-osl.tex (see the styles folder).

\subsection{Typesetting semantics}\label{sim:sec:sem}

See below for some examples of how to typeset semantic formulas. The examples also show the use of the sib-command (= ``semantic interpretation brackets''). Notice also the the use of the dummy curly brackets in \REF{sim:ex:quant}. They ensure that the spacing around the equation symbol is correct. 

\ea \ea \sib{dog}$^g=\textsc{dog}=\lambda x[\textsc{dog}(x)]$\label{sim:ex:dog}
\ex \sib{Some dog bit every boy}${}=\exists x[\textsc{dog}(x)\wedge\forall y[\textsc{boy}(y)\rightarrow \textsc{bit}(x,y)]]$\label{sim:ex:quant}
\z\z

\noindent Use noindent after example environments (but not after floats like tables or figures).

And here's a macro for semantic type brackets: The expression \textit{dog} is of type $\stb{e,t}$. Don't forget to place the whole type formula into a math-environment. An example of a more complex type, such as the one of \textit{every}: $\stb{s,\stb{\stb{e,t},\stb{e,t}}}$. You can of course also use the type in a subscript: dog$_{\stb{e,t}}$

We distinguish between metalinguistic constants that are translations of object language, which are typeset using small caps, see \REF{sim:ex:dog}, and logical constants. See the contrast in \REF{sim:ex:speaker}, where \textsc{speaker} (= serif) in \REF{sim:ex:speaker-a} is the denotation of the word \textit{speaker}, and \cnst{speaker} (= sans-serif) in \REF{sim:ex:speaker-b} is the function that maps the context $c$ to the speaker in that context.\footnote{Notice that both types of small caps are automatically turned into text-style, even if used in a math-environment. This enables you to use math throughout.}$^,$\footnote{Notice also that the notation entails the ``direct translation'' system from natural language to metalanguage, as entertained e.g. in \citet{Heim.Kratzer1998}. Feel free to devise your own notation when relying on the ``indirect translation'' system (see, e.g., \citealt{Coppock.Champollion2022}).}

\ea\label{sim:ex:speaker}
\ea \sib{The speaker is drunk}$^{g,c}=\textsc{drunk}\big(\iota x\,\textsc{speaker}(x)\big)$\label{sim:ex:speaker-a}
\ex \sib{I am drunk}$^{g,c}=\textsc{drunk}\big(\cnst{speaker}(c)\big)$\label{sim:ex:speaker-b}
\z\z

\noindent Notice that with more complex formulas, you can use bigger brackets indicating scope, cf. $($ vs. $\big($, as used in \REF{sim:ex:speaker}. Also notice the use of backslash plus comma, which produces additional space in math-environment.

\subsection{Examples and the minsp command}

Try to keep examples simple. But if you need to pack more information into an example or include more alternatives, you can resort to various brackets or slashes. For that, you will find the minsp-command useful. It works as follows:

\ea\label{sim:ex:german-verbs}\gll Hans \minsp{\{} schläft / schlief / \minsp{*} schlafen\}.\\
Hans {} sleeps {} slept {} {} sleep.\textsc{inf}\\
\glt `Hans \{sleeps / slept\}.'
\z

\noindent If you use the command, glosses will be aligned with the corresponding object language elements correctly. Notice also that brackets etc. do not receive their own gloss. Simply use closed curly brackets as the placeholder.

The minsp-command is not needed for grammaticality judgments used for the whole sentence. For that, use the native langsci-gb4e method instead, as illustrated below:

\ea[*]{\gll Das sein ungrammatisch.\\
that be.\textsc{inf} ungrammatical\\
\glt Intended: `This is ungrammatical.'\hfill (German)\label{sim:ex:ungram}}
\z

\noindent Also notice that translations should never be ungrammatical. If the original is ungrammatical, provide the intended interpretation in idiomatic English.

If you want to indicate the language and/or the source of the example, place this on the right margin of the translation line. Schematic information about relevant linguistic properties of the examples should be placed on the line of the example, as indicated below.

\ea\label{sim:ex:bailyn}\gll \minsp{[} Ėtu knigu] čitaet Ivan \minsp{(} často).\\
{} this book.{\ACC} read.{\PRS.3\SG} Ivan.{\NOM} {} often\\\hfill O-V-S-Adv
\glt `Ivan reads this book (often).'\hfill (Russian; \citealt[4]{Bailyn2004})
\z

\noindent Finally, notice that you can use the gloss macros for typing grammatical glosses, defined in langsci-lgr.sty. Place curly brackets around them.

\subsection{Citation commands and macros}

You can make your life easier if you use the following citation commands and macros (see code):

\begin{itemize}
    \item \citealt{Bailyn2004}: no brackets
    \item \citet{Bailyn2004}: year in brackets
    \item \citep{Bailyn2004}: everything in brackets
    \item \citepossalt{Bailyn2004}: possessive
    \item \citeposst{Bailyn2004}: possessive with year in brackets
\end{itemize}

\section{Trees}\label{s:tree}

Use the forest package for trees and place trees in a figure environment. \figref{sim:fig:CP} shows a simple example.\footnote{See \citet{VandenWyngaerd2017} for a simple and useful quickstart guide for the forest package.} Notice that figure (and table) environments are so-called floating environments. {\LaTeX} determines the position of the figure/table on the page, so it can appear elsewhere than where it appears in the code. This is not a bug, it is a property. Also for this reason, do not refer to figures/tables by using phrases like ``the table below''. Always use tabref/figref. If your terminal nodes represent object language, then these should essentially correspond to glosses, not to the original. For this reason, we recommend including an explicit example which corresponds to the tree, in this particular case \REF{sim:ex:czech-for-tree}.

\ea\label{sim:ex:czech-for-tree}\gll Co se řidič snažil dělat?\\
what {\REFL} driver try.{\PTCP.\SG.\MASC} do.{\INF}\\
\glt `What did the driver try to do?'
\z

\begin{figure}[ht]
% the [ht] option means that you prefer the placement of the figure HERE (=h) and if HERE is not possible, you prefer the TOP (=t) of a page
% \centering
    \begin{forest}
    for tree={s sep=1cm, inner sep=0, l=0}
    [CP
        [DP
            [what, roof, name=what]
        ]
        [C$'$
            [C
                [\textsc{refl}]
            ]
            [TP
                [DP
                    [driver, roof]
                ]
                [T$'$
                    [T [{[past]}]]
                    [VP
                        [V
                            [tried]
                        ]
                        [VP, s sep=2.2cm
                            [V
                                [do.\textsc{inf}]
                            ]
                            [t\textsubscript{what}, name=trace-what]
                        ]
                    ]
                ]
            ]
        ]
    ]
    \draw[->,overlay] (trace-what) to[out=south west, in=south, looseness=1.1] (what);
    % the overlay option avoids making the bounding box of the tree too large
    % the looseness option defines the looseness of the arrow (default = 1)
    \end{forest}
    \vspace{3ex} % extra vspace is added here because the arrow goes too deep to the caption; avoid such manual tweaking as much as possible; here it's necessary
    \caption{Proposed syntactic representation of \REF{sim:ex:czech-for-tree}}
    \label{sim:fig:CP}
\end{figure}

Do not use noindent after figures or tables (as you do after examples). Cases like these (where the noindent ends up missing) will be handled by the editors prior to publication.

\section{Italics, boldface, small caps, underlining, quotes}

See \citet{Nordhoff.Muller2021} for that. In short:

\begin{itemize}
    \item No boldface anywhere.
    \item No underlining anywhere (unless for very specific and well-defined technical notation; consult with editors).
    \item Small caps used for (i) introducing terms that are important for the paper (small-cap the term just ones, at a place where it is characterized/defined); (ii) metalinguistic translations of object-language expressions in semantic formulas (see \sectref{sim:sec:sem}); (iii) selected technical notions.
    \item Italics for object-language within text; exceptionally for emphasis/contrast.
    \item Single quotes: for translations/interpretations
    \item Double quotes: scare quotes; quotations of chunks of text.
\end{itemize}

\section{Cross-referencing}

Label examples, sections, tables, figures, possibly footnotes (by using the label macro). The name of the label is up to you, but it is good practice to follow this template: article-code:reference-type:unique-label. E.g. sim:ex:german would be a proper name for a reference within this paper (sim = short for the author(s); ex = example reference; german = unique name of that example).

\section{Syntactic notation}

Syntactic categories (N, D, V, etc.) are written with initial capital letters. This also holds for categories named with multiple letters, e.g. Foc, Top, Num, etc. Stick to this convention also when coming up with ad hoc categories, e.g. Cl (for clitic or classifier).

An exception from this rule are ``little'' categories, which are written with italics: \textit{v}, \textit{n}, \textit{v}P, etc.

Bar-levels must be typeset with bars/primes, not with an apostrophe. An easy way to do that is to use mathmode for the bar: C$'$, Foc$'$, etc.

Specifiers should be written this way: SpecCP, Spec\textit{v}P.

Features should be surrounded by square brackets, e.g., [past]. If you use plus and minus, be sure that these actually are plus and minus, and not e.g. a hyphen. Mathmode can help with that: [$+$sg], [$-$sg], [$\pm$sg]. See \sectref{sim:sec:hyphens-etc} for related information.

\section{Footnotes}

Absolutely avoid long footnotes. A footnote should not be longer than, say, {20\%} of the page. If you feel like you need a long footnote, make an explicit digression in the main body of the text.

Footnotes should always be placed at the end of whole sentences. Formulate the footnote in such a way that this is possible. Footnotes should always go after punctuation marks, never before. Do not place footnotes after individual words. Do not place footnotes in examples, tables, etc. If you have an urge to do that, place the footnote to the text that explains the example, table, etc.

Footnotes should always be formulated as full, self-standing sentences.

\section{Tables}

For your tables use the table plus tabularx environments. The tabularx environment lets you (and requires you in fact) to specify the width of the table and defines the X column (left-alignment) and the Y column (right-alignment). All X/Y columns will have the same width and together they will fill out the width of the rest of the table -- counting out all non-X/Y columns.

Always include a meaningful caption. The caption is designed to appear on top of the table, no matter where you place it in the code. Do not try to tweak with this. Tables are delimited with lsptoprule at the top and lspbottomrule at the bottom. The header is delimited from the rest with midrule. Vertical lines in tables are banned. An example is provided in \tabref{sim:tab:frequencies}. See \citet{Nordhoff.Muller2021} for more information. If you are typesetting a very complex table or your table is too large to fit the page, do not hesitate to ask the editors for help.

\begin{table}
\caption{Frequencies of word classes}
\label{sim:tab:frequencies}
 \begin{tabularx}{.77\textwidth}{lYYYY} %.77 indicates that the table will take up 77% of the textwidth
  \lsptoprule
            & nouns & verbs  & adjectives & adverbs\\
  \midrule
  absolute  &   12  &    34  &    23      & 13\\
  relative  &   3.1 &   8.9  &    5.7     & 3.2\\
  \lspbottomrule
 \end{tabularx}
\end{table}

\section{Figures}

Figures must have a good quality. If you use pictorial figures, consult the editors early on to see if the quality and format of your figure is sufficient. If you use simple barplots, you can use the barplot environment (defined in langsci-osl.sty). See \figref{sim:fig:barplot} for an example. The barplot environment has 5 arguments: 1. x-axis desription, 2. y-axis description, 3. width (relative to textwidth), 4. x-tick descriptions, 5. x-ticks plus y-values.

\begin{figure}
    \centering
    \barplot{Type of meal}{Times selected}{0.6}{Bread,Soup,Pizza}%
    {
    (Bread,61)
    (Soup,12)
    (Pizza,8)
    }
    \caption{A barplot example}
    \label{sim:fig:barplot}
\end{figure}

The barplot environment builds on the tikzpicture plus axis environments of the pgfplots package. It can be customized in various ways. \figref{sim:fig:complex-barplot} shows a more complex example.

\begin{figure}
  \begin{tikzpicture}
    \begin{axis}[
	xlabel={Level of \textsc{uniq/max}},  
	ylabel={Proportion of $\textsf{subj}\prec\textsf{pred}$}, 
	axis lines*=left, 
        width  = .6\textwidth,
	height = 5cm,
    	nodes near coords, 
    % 	nodes near coords style={text=black},
    	every node near coord/.append style={font=\tiny},
        nodes near coords align={vertical},
	ymin=0,
	ymax=1,
	ytick distance=.2,
	xtick=data,
	ylabel near ticks,
	x tick label style={font=\sffamily},
	ybar=5pt,
	legend pos=outer north east,
	enlarge x limits=0.3,
	symbolic x coords={+u/m, \textminus u/m},
	]
	\addplot[fill=red!30,draw=none] coordinates {
	    (+u/m,0.91)
        (\textminus u/m,0.84)
	};
	\addplot[fill=red,draw=none] coordinates {
	    (+u/m,0.80)
        (\textminus u/m,0.87)
	};
	\legend{\textsf{sg}, \textsf{pl}}
    \end{axis} 
  \end{tikzpicture} 
    \caption{Results divided by \textsc{number}}
    \label{sim:fig:complex-barplot}
\end{figure}

\section{Hyphens, dashes, minuses, math/logical operators}\label{sim:sec:hyphens-etc}

Be careful to distinguish between hyphens (-), dashes (--), and the minus sign ($-$). For in-text appositions, use only en-dashes -- as done here -- with spaces around. Do not use em-dashes (---). Using mathmode is a reliable way of getting the minus sign.

All equations (and typically also semantic formulas, see \sectref{sim:sec:sem}) should be typeset using mathmode. Notice that mathmode not only gets the math signs ``right'', but also has a dedicated spacing. For that reason, never write things like p$<$0.05, p $<$ 0.05, or p$<0.05$, but rather $p<0.05$. In case you need a two-place math or logical operator (like $\wedge$) but for some reason do not have one of the arguments represented overtly, you can use a ``dummy'' argument (curly brackets) to simulate the presence of the other one. Notice the difference between $\wedge p$ and ${}\wedge p$.

In case you need to use normal text within mathmode, use the text command. Here is an example: $\text{frequency}=.8$. This way, you get the math spacing right.

\section{Abbreviations}

The final abbreviations section should include all glosses. It should not include other ad hoc abbreviations (those should be defined upon first use) and also not standard abbreviations like NP, VP, etc.


\section{Bibliography}

Place your bibliography into localbibliography.bib. Important: Only place there the entries which you actually cite! You can make use of our OSL bibliography, which we keep clean and tidy and update it after the publication of each new volume. Contact the editors of your volume if you do not have the bib file yet. If you find the entry you need, just copy-paste it in your localbibliography.bib. The bibliography also shows many good examples of what a good bibliographic entry should look like.

See \citet{Nordhoff.Muller2021} for general information on bibliography. Some important things to keep in mind:

\begin{itemize}
    \item Journals should be cited as they are officially called (notice the difference between and, \&, capitalization, etc.).
    \item Journal publications should always include the volume number, the issue number (field ``number''), and DOI or stable URL (see below on that).
    \item Papers in collections or proceedings must include the editors of the volume (field ``editor''), the place of publication (field ``address'') and publisher.
    \item The proceedings number is part of the title of the proceedings. Do not place it into the ``volume'' field. The ``volume'' field with book/proceedings publications is reserved for the volume of that single book (e.g. NELS 40 proceedings might have vol. 1 and vol. 2).
    \item Avoid citing manuscripts as much as possible. If you need to cite them, try to provide a stable URL.
    \item Avoid citing presentations or talks. If you absolutely must cite them, be careful not to refer the reader to them by using ``see...''. The reader can't see them.
    \item If you cite a manuscript, presentation, or some other hard-to-define source, use the either the ``misc'' or ``unpublished'' entry type. The former is appropriate if the text cited corresponds to a book (the title will be printed in italics); the latter is appropriate if the text cited corresponds to an article or presentation (the title will be printed normally). Within both entries, use the ``howpublished'' field for any relevant information (such as ``Manuscript, University of \dots''). And the ``url'' field for the URL.
\end{itemize}

We require the authors to provide DOIs or URLs wherever possible, though not without limitations. The following rules apply:

\begin{itemize}
    \item If the publication has a DOI, use that. Use the ``doi'' field and write just the DOI, not the whole URL.
    \item If the publication has no DOI, but it has a stable URL (as e.g. JSTOR, but possibly also lingbuzz), use that. Place it in the ``url'' field, using the full address (https: etc.).
    \item Never use DOI and URL at the same time.
    \item If the official publication has no official DOI or stable URL (related to the official publication), do not replace these with other links. Do not refer to published works with lingbuzz links, for instance, as these typically lead to the unpublished (preprint) version. (There are exceptions where lingbuzz or semanticsarchive are the official publication venue, in which case these can of course be used.) Never use URLs leading to personal websites.
    \item If a paper has no DOI/URL, but the book does, do not use the book URL. Just use nothing.
\end{itemize}

\section*{Abbreviations}

\begin{multicols}{2}
\begin{tabbing}
\textsc{imperf}\hspace{.5em}\=imperfective\kill
\textsc{1} \> first person\\
\textsc{2} \> second person\\
\textsc{3} \> third person\\
\textsc{acc} \> accusative\\
\textsc{aor} \> aorist\\
\textsc{aux} \> auxiliary\\
\textsc{cl} \> clitic\\
\textsc{dat} \> dative\\
\textsc{dim} \> diminutive\\
\textsc{f} \> feminine\\
\textsc{imperf} \> imperfective\\
%\textsc{inf} \> infinitive\\
\textsc{m} \> masculine\\
\textsc{n} \> neutral\\
\textsc{nom} \> nominative\\
\textsc{part} \> particle\\
\textsc{perf} \> perfective\\
\textsc{pl} \> plural\\
\textsc{prs} \> present tense\\
\textsc{ptcp} \> participle\\
\textsc{refl} \> reflexive\\
\textsc{rel} \> relative\\
\textsc{sg} \> singular\\
\textsc{sm} \> subject marker
\end{tabbing}
\end{multicols}

\section*{Acknowledgements}
This research was funded by the Deutsche Forschungsgemeinschaft (DFG, German Research Foundation), as part of my project ``Asymmetries in relative clauses in West Germanic'' (BA 5201/2) carried out at the University of Konstanz. I owe many thanks to Mariia Privizentseva and Berit Gehrke for their helpful questions at the FDSL-14 conference, as well as to the reviewers for their constructive suggestions.

\printbibliography[heading=subbibliography,notkeyword=this]

\end{document}
