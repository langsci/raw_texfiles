\documentclass[output=paper,colorlinks,citecolor=brown]{langscibook}
\ChapterDOI{10.5281/zenodo.10123639}

\author{Irenäus Kulik\affiliation{Friedrich Schiller University Jena}\orcid{0000-0002-4178-5137}}
\SetupAffiliations{mark style=none}

\title{Clitic climbing without restructuring in Czech and Polish}
\abstract{Clitic climbing, i.e. the realization of one or more clitics in a syntactic constituent hierarchically higher than the clitics’ licensing predicate, has been accounted for in terms of a restructuring approach. The embedded infinitive the clitics are extracted from has been assumed to be structurally deficient -- that is, a bare VP. Due to the lack of projections above the lexical V-head, clitics escape the infinitival domain to get their morphosyntactic features licensed in the matrix clause. However, the predictions of the restructuring approach do not withstand a corpus linguistic examination and are falsified by empirical data of Czech and Polish. Clitic climbing cannot be adequately accounted for by syntax proper and alternative accounts have to be taken into consideration seriously. It will be proposed to exploit information structure as a feasible explanatory account of clitic climbing.

\keywords{clitic climbing, restructuring, syntax, information structure, Czech, Polish}
}

\lsConditionalSetupForPaper{}

\begin{document}
\maketitle

\section{Introduction}\label{kul:sec:intro}

\textsc{Clitic climbing} (CC)  is the realization of a pronominal or reflexive clitic in a syntactic constituent hierarchically higher than the licensing predicate. \citet[66]{Junghanns2002a} schematizes CC as in \REF{kul:ex:alphabeta-a}, whereby a constituent {\textalpha} embeds a constituent {\textbeta}. CC is analyzed as movement of the \textsc{clitic} (CL) from {\textbeta} to {\textalpha}. \REF{kul:ex:alphabeta-b} paraphrases the scheme in a theory-neutral way.\footnote{I will attempt to phrase my arguments in a theory-neutral way, in order not to impose a specific approach on the reader. For instance, I will not opt for a particular analysis of long-distance dependencies in terms of e.g. movement or copy-and-delete.} The gap \textit{e} co-indexed with CL captures the fact that CL is linearized in {\textalpha}, but subcategorized for by the verbal predicate in {\textbeta}. In principle, an arbitrary number \textit{n} of phrase boundaries may intervene between {\textalpha} and {\textbeta} (with \{$\mathbb{N}\textsubscript{0}$\} denoting the set of natural numbers including zero).

\ea\label{kul:ex:alphabeta}
\ea[]{[{\textsubscript{\textalpha}}~…~CL~…~[\textsubscript{\textbeta}~…~t\textsubscript{CL}~…~]]}\label{kul:ex:alphabeta-a}

\ex[]{[{\textsubscript{\textalpha}}~CL\textsubscript{j}~[\textsubscript{n}~[\textsubscript{\textbeta}~\textit{e}\textsubscript{j}~]]],~n${}\in\{\mathbb{N}\textsubscript{0}\}$}\label{kul:ex:alphabeta-b}
\z
\z

\noindent There is an extensive body of research literature on CC in Romance, which signi\-ficantly inspired research on Slavic. In the Italian example \REF{kul:ex:martina-a}, the direct object clitic \textit{lo} `him.{\ACC}' follows the embedded infinitive \textit{legger}(\textit{e}) ‘read’ it is argument of.\footnote{Clitics will be highlighted in italics for ease of reference. I adopt the integer-index-convention from \citet{Hana2007}, \citet{Rosen2014}, and \citet{KolakovicFritz2022} to indicate the structural hierarchy between the verbal heads as well as the subcategorization relations between a verbal predicate and its dependents.} This is the local or in situ realization of the clitic. CC is found in \REF{kul:ex:martina-b} with the object clitic being realized before the finite verb of the matrix phrase, which will be also referred to as non-local placement.

\ea\label{kul:ex:martina}
\ea[]{
\gll Martina vuole\textsubscript{1} legger\textsubscript{2}=\textit{lo}\textsubscript{2}.\\ 
Martina.{\NOM} want.{\PRS.3\SG} read.{\INF}=him.{\ACC}\\ \hfill (in situ/local)
\glt `Martina wants to read it.'}\label{kul:ex:martina-a}
\ex[]{
\gll Martina \textit{lo}\textsubscript{2} vuole\textsubscript{1} leggere\textsubscript{2}.\\
Martina.{\NOM} him.{\ACC} want.{\PRS.3\SG} read.{\INF}\\ \hfill(CC/non-local) \\
\glt `Martina wants to read it.'   \hfill (Italian; \citealt[163--164]{SpencerLuís2012})}\label{kul:ex:martina-b}
\z
\z

\noindent The remainder of the paper is organized as follows: \sectref{kul:sec:status-clitics} briefly comments on the syntactic status of clitic pronouns in West Slavic. \sectref{kul:sec:climbing-westslav} provides a concise overview of basic clitic climbing properties in West Slavic. \sectref{kul:sec:climbing-restructure} is the core of the paper and tests the correlates of restructuring empirically, focusing on accusative case licensing \sectref{kul:sec:case-licensing}, the absence of an underlying subject in the infinitive phrase \sectref{kul:sec:missing-subjects}, the dependence of the infinitive’s temporal reference upon the matrix verb’s tense information \sectref{kul:sec:temp-reference}, and the all-or-nothing quality of clitic climbing \sectref{kul:sec:all-nothing}. \sectref{kul:sec:cc-inform-structure} addresses the role of information structure for clitic climbing. Concluding remarks are given in \sectref{kul:sec:conclusion}.

\section{The status of West Slavic clitics}\label{kul:sec:status-clitics}

It is common to distinguish Polish and Czech clitics along the lines of \citeposst{Zwicky1977} simple/special-clitic dichotomy. The second position clitics in Czech are considered special clitics, whereas Polish clitics being distributed rather freely are simple clitics. Given this distinction, cliticization in these two closely-related languages is an ideal test field for theorizing about clitic phenomena and the micro-typology of Slavic cliticization. However, the often-made statement that Polish clitics are typologically peculiar in comparison to clitics in other West and South Slavic languages turns out to be controversial on closer inspection. On the one hand, \citet{Rappaport1988}, \citet{Dziwirek1998}, \citet{Kupsc2000}, \citet{BorsleyRivero1994}, \citet{Franks2009, Franks2010}, and \citet{FranksKing2000} treat Polish as a language without second position clitics -- hence, not possessing special clitics. On the other hand, \citet[725]{Rothstein1993}, \citet[62]{Urbanczyk1976}, \citet[§4.8 Footnote~23]{Veselovska1995}, and \citet[85]{Dimitrova-Vulchanova1999} consider Polish to be a second position clitic language essentially. \citet[390]{Spencer1991} regards Polish as a special clitic language. From a different angle, Czech exhibits positional deviations from second position cliticization with third, fourth, and fifth position placement being attested (see \citealt[103--112]{Hana2007}, \citealt[177--178]{Junghanns2021}). Therefore, Czech and Polish clitics will be treated alike throughout the paper. I restrict myself to the set of short pronominals, adopting a traditional terminology from Slavic studies here.\footnote{The notion rests upon the formal distinction of “short” (e.g. Cz. \textit{mu}, \textit{ho}) vs. “long” pronouns (e.g. Cz. \textit{jemu}, \textit{jeho}). Note that terminology differs between authors. The short pronouns are referred to as \textit{konstantní přiklonky} ‘constant clitics’ in Czech linguistics (see \citealt{Travnicek1959}, \citealt{Rosen2001}, \citealt{Hana2007}), \citet{AvgustinovaOliva1997} propose the term \textit{pure clitics}, and \citet{Junghanns2002b} coins \textit{lexikalische Klitika} ‘lexical clitics’.} The respective sets for Czech and Polish are provided in \tabref{kul:tab:unambiguous-pronominals} (see \citealt{Fried1994}, \citealt{AvgustinovaOliva1997}, \citealt{Rosen2001}, \citealt{Junghanns2002b}, \citealt{Petkevic2009} for Czech, \citealt{Kupsc2000} for Polish).\footnote{Short dative reflexive \textit{se} occurs in colloquial Polish, but remains unconsidered in most analyses (e.g. \citealt{Spencer1991}, \citealt{Kupsc2000}). \citet[137]{Rubadeau1996} claims that ``Polish […] does not have a clitic form of the dative reflexive.'' On the other hand, \citet[58]{Urbanczyk1976} discusses \textit{se} in his outline of Polish dialects. \citet[150]{FranksKing2000} list \textit{se} among the Polish clitics, but note that it “is used only in the spoken language” (cf. also \citealt[702]{Rothstein1993}). \citet{AguadoDogil1989} explicitly take \textit{se} into consideration.} I follow the spirit of \citet{Dotlačil2007} and -- most recently -- \citet{Adam2019} in refraining from hypothesizing about the exact syntactic status of the short pronouns, e.g. whether they are syntactic phrases or heads, or whether they are weak rather than clitic pronouns.\footnote{An anonymous reviewer pointed out that Czech short pronouns are true second position clitics, whereas Polish short pronouns are weak pronouns. This point of view is reminiscent of \citeposst{CardinalettiStarke1999} tripartite typology of pronouns. Since there is no general consensus on this matter, it appears that the typology of clitics in Slavic still needs further investigation. For an alternative view, see \citet{JungMigdalski2022}, who propose an extension of \citeposst{CardinalettiStarke1999} approach to a four-way classification.}

%Table 1%

\iffalse

\begin{table}
\begin{tabularx}{\textwidth}{p{0.7cm}QQQQQQQ}
\lsptoprule
Czech & tě & ho & mi & ti & mu & se & si\\
Polish & cię & go & mi & ci & mu & się & se \\
       & ‘you\textsubscript{\textsc{acc}}’ & ‘him\textsubscript{\textsc{acc}}’ & ‘me\textsubscript{\textsc{dat}}’ & ‘you\textsubscript{\textsc{dat}}’ & ‘him\textsubscript{\textsc{dat}}’ & ‘refl\textsubscript{\textsc{acc/gen}}' & `refl\textsubscript{\textsc{dat}}’ \\
\lspbottomrule
\end{tabularx}
\caption{Unambiguous short pronominals in Czech and Polish}
\label{kul:tab:unambiguous-pronominals}
\end{table}

\fi 

\begin{table}
\caption{Unambiguous short pronominals in Czech and Polish}
\label{kul:tab:unambiguous-pronominals}
\begin{tabularx}{.4\textwidth}
{lQQQ} %.77 indicates that the table will take up 77% of the textwidth
  \lsptoprule
 English & Czech & Polish\\
  \midrule
you\textsubscript{\textsc{acc/gen}} & \textit{tě} & \textit{cię}\\
him\textsubscript{\textsc{acc}} & \textit{ho} & \textit{go} \\
me\textsubscript{\textsc{dat}} & \textit{mi} & \textit{mi}\\
you\textsubscript{\textsc{dat}} & \textit{ti} & \textit{ci}\\
him\textsubscript{\textsc{dat}} & \textit{mu} & \textit{mu}\\
refl\textsubscript{\textsc{acc(/gen)}} & \textit{se} & \textit{się}\\
refl\textsubscript{\textsc{dat}} & \textit{si}& \textit{se} \\
  \lspbottomrule
 \end{tabularx}
\end{table}

\section{Clitic climbing in West Slavic}\label{kul:sec:climbing-westslav}

As in the Italian example \REF{kul:ex:martina}, CC occurs from embedded infinitives in Czech and Polish.\footnote{An anonymous reviewer pointed out that CC is possible from a subset of morphologically finite \textit{da}-clauses in Serbian, i.e. from subjunctive-like \textit{da}-clauses (see \citealt{Progovac1993, Progovac1996} for the relevant distinction of indicative and subjunctive \textit{da}-clauses). It is not necessary to rely on the indicative-subjunctive distinction to account for CC in West Slavic, which is best captured by the conditions of the embedded verb’s infinitive-hood and the absence of a subordinator.} Several scholars point out that infinitive-hood is a necessary, but not a sufficient condition for CC (cf. \citealt[69]{Junghanns2002a} on Czech, \citealt[58]{Kupsc2000} on Polish, \citealt[221--222]{Golden2003} on Slovene). The infinitival domain must not be introduced by a \textsc{subordinator}.\footnote{The term \textsc{subordinator} is meant to broadly cover elements introducing subordinate clauses of different kinds, i.e. (i) complementizers introducing argument clauses, (ii) subordinate conjunctions introducing adjunct clauses, and (iii) relative pronouns and adverbs introducing relative clauses.}  Note that the clitic, which is subject to climbing, is not necessarily an argument, e.g. {\REFL} of a reflexive tantum or in impersonal constructions. Therefore, I adopt the term \textsc{dependent} from dependency grammar as a general notion for clitics licensed by a verbal head. CC occurs in a variety of syntactic constructions, i.e. raising, subject and object control, and the \textsc{accusative with infinitive} (ACI, from Latin \textit{accusativus cum infinitivo}) (see \citealt{Junghanns2002a}, \citealt{Golden2008}, \citealt{Kupsc2000}).\footnote{The accusative with infinitive is known as exceptional case marking in generative grammar.} Note that clitic climbing is ungrammatical in object control constructions in Romance, but not in Slavic (cf. 
\citealt[315]{Golden2008}). Note also that standard Polish and its vernacular do not possess the ACI construction (see \citealt[33]{PrzepiórkowskiRosen2005}, \citealt[96]{Kupsc2000}). The lack of ACI is a general property of Polish syntax, but it is not a particular feature of the Polish clitic system. The ACI is attested in several diatopic varieties of Polish (see \citealt[56]{Urbanczyk1976}). It has been observed that CC is not obligatory and clitics may be realized in situ as in Italian \REF{kul:ex:martina-a}. In the same way, both a- and b-examples are grammatical in Czech \REF{kul:ex:asi-ho} and Polish \REF{kul:ex:jan-go}. The question arises then, why CC does come into being and what are the conditions for the local vs. non-local realization of the clitics.

\ea\label{kul:ex:asi-ho}
\ea[]{
\gll Asi \textit{ho}\textsubscript{2} chtěla\textsubscript{1} usušit\textsubscript{2} pomalu.\\
perhaps him.{\ACC} want.{\PST.\SG.}\textsc{f} dry.{\INF} slowly\\
\glt `Perhaps she wanted to dry it slowly.’
\label{kul:ex:asi-ho-a}}
\ex[]{
\gll Asi chtěla\textsubscript{1} usušit\textsubscript{2} \textit{ho}\textsubscript{2} pomalu.\\
perhaps want.{\PST.\SG.}\textsc{f} dry.{\INF} him.{\ACC} slowly \\
\glt `Perhaps she wanted to dry it slowly.’ \hfill (Czech; \citealt[82]{Junghanns2002a}) \label{kul:ex:asi-ho-b}}
\z
\z

\ea\label{kul:ex:jan-go}
\ea[]{
\gll Jan \textit{go}\textsubscript{2} chciał\textsubscript{1} obudzić\textsubscript{2} o szóstej.\\
Jan.{\NOM} him.{\ACC} want.{\PST.\SG.}\textsc{m} wake-up.{\INF} at six\\
\glt `Jan wanted to wake him up at six o’clock.’ \label{kul:ex:jan-go-a}}
\ex[]{
\gll Jan chciał\textsubscript{1} obudzić\textsubscript{2} \textit{go}\textsubscript{2} o szóstej.\\
Jan.{\NOM} want.{\PST.\SG.}\textsc{m} wake-up.{\INF} him.{\ACC} at six\\
\glt `Jan wanted to wake him up at six o’clock.’  \hfill (Polish; \citealt[60]{Kupsc2000}) \label{kul:ex:jan-go-b}}
\z
\z

\begin{sloppypar}
\noindent It has been proposed to account for CC in Slavic in terms of a restructuring approach by \citet{Rezac2005} for Czech and \citet{Aljović2004} for Bosnian-Croatian-Montenegrin-Serbian (BCMS) (cf. \citealt{Rizzi1982} on restructuring in Italian and \citealt{Wurmbrand2001} for a general analysis of restructuring properties on the basis of German and Japanese). According to such an approach the optionality of CC is only an alleged one. While clitics must remain in situ in true bi-clausal structures, they are forced to climb under restructuring, which is underlyingly mono-clausal due to the structural deficiency of the embedded so-called \textsc{restructuring infinitive} (RI). Being bare VPs, RIs lack the \textit{v}P- and TP-shell.\footnote{Note that Wurmbrand (\citeyear{Wurmbrand2001} and subsequent work) proposes a multi-way distinction of restructuring, which is not limited to binary parametrization. I cannot take these proposals into consideration here due to space limitations and must leave them for future discussion.} Several correlates have been put forward to support the restructuring analysis: (a)~RIs are unable to license accusative case, (b)~RIs do not have an underlying subject ($=\text{PRO}$), (c)~RIs do not constitute a binding domain for principle B, (d)~either all clitics climb as a consequence of the infinitive’s structural deficiency or none, (e)~RIs are temporally dependent upon the matrix verb’s tense. Criteria~(a)–(d) are taken from \citet{Rezac2005}, criterion~(e) is taken from \citet{Todorovic2012}.
\end{sloppypar}

In what follows I will test the hypothesis that CC is dependent upon restructuring by assessing the above-mentioned correlates empirically towards corpus data from the \textit{Český národní korpus} ‘Czech national corpus’ (ČNK) and the \textit{Narodowy Korpus Języka Polskiego} ‘National corpus of Polish’ (NKJP) respectively. In particular, the Czech data are drawn from the subcorpus SYN version~8 (see \citealt{KřenZasina2019}, \citealt{HnátkováSkoumalová2014}).\footnote{\url{https://www.korpus.cz/}} For Polish, I searched the full NKJP corpus through the Poliqarp search engine (see \citealt{PrzepiórkowskiLewandowska-Tomaszczyk2012}).\footnote{\url{http://www.nkjp.pl/}}

\section{Clitic climbing and correlates of restructuring}\label{kul:sec:climbing-restructure}

\subsection{Case licensing}\label{kul:sec:case-licensing}

Due to the lack of \textit{v}P/TP, RIs are unable to license accusative case. Clitics climb in order to receive case in the matrix phrase then.\footnote{It is irrelevant for the purpose of the present study how case licensing is technically implemented.} \citet{Lenertova2004} and \citet{Dotlačil2004} recognize independently for Czech that CC into passivized matrix domains contradicts this argument. It is generally known from Burzio’s generalization that passivized verbs are unable to license accusative case (cf. \citealt{Burzio1986}). Data like \REF{kul:ex:privezl-puk} and \REF{kul:ex:pavel} contradict the case-based argument, as the case of the clitic \textit{ho} ‘him.{\ACC}’ cannot be licensed by the passivized matrix verb, but only by the embedded infinitive. \citeposst{Lenertova2004} and \citeposst{Dotlačil2004} arguments are corroborated by examples \REF{kul:ex:kdo-byl} and \REF{kul:ex:ze-juz-nigdy} for Czech and Polish respectively.

\ea\label{kul:ex:privezl-puk}
\gll '\minsp{(} Přivezl puk za švýcarskou branku,) ale tam \textit{ho}\textsubscript{3} byl\textsubscript{1} donucen\textsubscript{2} předat\textsubscript{3} Lubinovi.\\
{} bring.{\PST.\SG.}\textsc{m} puck behind Swiss goal but there him.{\ACC} be.{\PST.\SG.}\textsc{m} forced.{\PASS.\SG.}\textsc{m} give.{\INF} Lubin.{\DAT}\\
\glt `(He brought the puck behind the Swiss goal,) but there he was forced to give it to Lubina.’ \hfill (Czech; \citealt[159]{Lenertova2004})
\z

\ea\label{kul:ex:pavel}
\gll Pavel \textit{ho}\textsubscript{3} byl\textsubscript{1} nucen\textsubscript{2} zničit\textsubscript{3}.\\
Pavel.{\NOM} him.{\ACC} be.{\PST.\SG.}\textsc{m} force.{\PASS.\SG.}\textsc{m} destroy.{\INF}\\
\glt `Pavel was forced to destroy it.’ \hfill (Czech; \citealt[88]{Dotlačil2004})
\z

%\protectedex{
\ea\label{kul:ex:kdo-byl}
\ea[]{
\gll […], kdo by \textit{ho}\textsubscript{3} byl\textsubscript{1} oprávněn\textsubscript{2} zbavit\textsubscript{3} zodpovědnosti za osud Ruska.\\
{} who \textsc{cond} him.{\ACC} be.{\PST.\SG.}\textsc{m} entitle.{\PASS.\SG.}\textsc{m} relieve.{\INF} responsibility.{\GEN.\SG} for fate Russia.{\GEN.\SG}\\ 
\glt `…, who would have been entitled to relieve him of his responsibility for the fate of Russia.’ \label{kul:ex:kdo-byl-a}
}
\ex[]{
\gll […], kteří \textit{ho}\textsubscript{3} byli\textsubscript{1} připraveni\textsubscript{2} zatknout\textsubscript{3}. \\
{} who him.{\ACC} be.{\PST.\PL.}\textsc{man} prepare.{\PASS.\PL.}\textsc{man} arrest.{\INF}\\
\glt `…, who were prepared to arrest him.’ \hfill (Czech; ČNK) \label{kul:ex:kdo-byl-b}}
\z
\z%}


\ea\label{kul:ex:ze-juz-nigdy}
\ea[]{
\gll […] że już nigdy nie będę\textsubscript{1} \textit{cię}\textsubscript{3} zmuszona\textsubscript{2} oglądać\textsubscript{3}.\\
{} that already never \textsc{neg} be.{\FUT.1\SG} you.{\ACC} force.{\PASS.\SG.}\textsc{f} look.{\INF}\\ \label{kul:ex:ze-juz-nigdy-a}
\glt `… that I will never be forced to look at you, again.’
}
\ex[]{
\gll bo z powodu drżenia twoich rąk będę\textsubscript{1} \textit{cię}\textsubscript{3} zmuszony\textsubscript{2} wrzucić\textsubscript{3} do KF\\
because from reason tremor your hands be.{\FUT.1\SG} you.{\ACC} force.{\PASS.\SG.}\textsc{m} throw.{\INF} to KF\\
\glt `because of your hands’ tremor I will be forced to throw you to the KF [$=\text{kill file}$]’ \hfill (Polish; NKJP) \label{kul:ex:ze-juz-nigdy-b}
}
\z
\z

\subsection{Missing subjects}\label{kul:sec:missing-subjects}

The lack of \textit{v}P yields RIs without having an underlying subject (PRO). \citet[114]{Rezac2005} states that RIs “will not constitute a binding domain of their own, and coreference between a pronominal argument of the infinitive and any argument of the upstairs verb should be blocked.” He provides the minimal pair in \REF{kul:ex:anna-mu-a}--\REF{kul:ex:anna-mu-b}.\footnote{An anonymous reviewer pointed out that \textit{políbit ji nashledanou} is an unusual calque on the basis of English ‘to kiss someone goodbye’. This does not have an impact on CC, however.} In \REF{kul:ex:anna-mu-a} the embedded clitic \textit{ji} ‘her.{\ACC}’ is co-referential with either the matrix subject \textit{Anna} (index \textit{a}) or a distinct discourse referent beyond the sentence-level (index \textit{b}). In \REF{kul:ex:anna-mu-b} the clitic has climbed due to restructuring. As a consequence, there is no clause boundary between the matrix and subordinate domain, thus co-reference between \textit{Anna} and \textit{ji} is excluded. \citet{Rezac2005} accounts for \REF{kul:ex:anna-mu-b} by a violation of binding principle~B, according to which “[a] pronominal is free [i.e. unbound] in its governing category [i.e. clause]” \citep[188]{Chomsky1981}. However, principle~B is inconclusive. It determines semantic co-reference by syntactic non-co-membership, which does not reveal anything about clause boun\-daries here. Co-reference between \textit{Anna} and \textit{ji} is still excluded by principle~B in presence of a clause boundary, for both the subject and the clitic are members of the matrix domain, cf. \REF{kul:ex:anna-mu-c}. Note, furthermore, that the matrix object \textit{mu} ‘him.{\DAT}’ still co-refers with the kisser of the embedded kissing-event despite CC of \textit{ji}.


\ea\label{kul:ex:anna-mu}
\ea[]{
\gll \minsp{[} Anna\textsubscript{a} \textit{mu}\textsubscript{c} dovolila \minsp{[} PRO\textsubscript{c} políbit \textit{ji}\textsubscript{a/b} nashledanou]].\\
{} Anna.{\NOM} him.{\DAT} allow.{\PST.\SG.}\textsc{f} {} {} kiss.{\INF} her.{\ACC} good-bye\\
\glt `Ana permitted him to kiss her good-bye.’\label{kul:ex:anna-mu-a}
}
\ex[]{
\gll \minsp{[} Anna\textsubscript{a} \textit{mu} \textit{ji}\textsubscript{*a/b} dovolila políbit nashledanou].\\
{} Anna.{\NOM} him.{\DAT} her.{\ACC} allow.{\PST.\SG.}\textsc{f} kiss.{\INF}  good-bye\\
\glt `Ana permitted him to kiss her good-bye.’\label{kul:ex:anna-mu-b}
}
\ex[]{
\gll \minsp{[} Anna\textsubscript{a} \textit{mu}\textsubscript{c} \textit{ji}\textsubscript{*a/b} dovolila \minsp{[} (PRO\textsubscript{c}) políbit nashledanou]].\\
{} Anna.{\NOM} him.{\DAT} her.{\ACC} allow.{\PST.\SG.}\textsc{f} {} {} kiss.{\INF}  good-bye\\
\label{kul:ex:anna-mu-c}
\glt `Ana permitted him to kiss her good-bye.’ \hfill (Czech; \citealt[114]{Rezac2005})
}
\z
\z

\noindent \citet[114--115]{Rezac2005} further states that neither matrix argument binds subject-oriented anaphora \textit{svým} ‘one’s.{\POSS.\PL.\DAT}’ in \REF{kul:ex:pavel-prikazal-b} in contrast to \REF{kul:ex:pavel-prikazal-a}. As the clitic \textit{je} ‘them.{\ACC}’ has climbed, restructuring must have occurred and PRO is missing thence. However, co-reference between the matrix subject and the anaphorical possessive pronoun should be still expected in a restructuring context. In fact, \citet{Dotlačil2007} and \citet{Skoumalova2005} judge \REF{kul:ex:pavel-prikazal-b} grammatical with both interpretations, such that embedded \textit{svým} is bound by either matrix argument (\textit{Pavel, Janovi}) despite CC. These judgements are corroborated by the corpus data in \REF{kul:ex:stavitel-mu} and \REF{kul:ex:a-prave}. First, the matrix subject \textit{stavitel} ‘constructor’ binds the possessive anaphor \textit{své} ‘one.{\POSS}’ after CC in \REF{kul:ex:stavitel-mu} as expected. Second, and even more intriguing, example \REF{kul:ex:a-prave} shows that the matrix object clitic \textit{mu} ‘him.{\DAT}’ binds the embedded possessive anaphor \textit{své} in spite of the climbed embedded clitic \textit{ho} ‘him.{\ACC}’. While the binding relations in \REF{kul:ex:stavitel-mu} are expected under standard assumptions in any mono-clausal domain, the binding facts in \REF{kul:ex:a-prave} are best analyzed by assuming an underlying subject in the embedded infinitive (i.e. PRO under standard generative assumptions).\footnote{I do not adopt \citeposst{Hornstein1999} proposal in abandoning the raising-control distinction, which has originally been the main motivation for the assumption of PRO (cf. \citealt{PrzepiórkowskiRosen2005} for a similar account in HPSG and \citealt{CulicoverJackendoff2001}, \citealt{Landau2003} for a critique).}


\ea\label{kul:ex:pavel-prikazal}
\ea[]{
\gll Pavel\textsubscript{a} přikázal\textsubscript{1} Janovi\textsubscript{b} dát\textsubscript{2} \textit{je}\textsubscript{2} svým\textsubscript{a/b} přátelům.\\
Pavel.{\NOM} order.{\PST.\SG.}\textsc{m} Jan.{\DAT} give.{\INF} them {\POSS} friends.{\DAT}\\
\glt `Pavel\textsubscript{a} ordered Jan\textsubscript{b} to give them to his\textsubscript{a/b} friends.’\label{kul:ex:pavel-prikazal-a}
}
\ex[*]{
\gll Pavel\textsubscript{a} \textit{je}\textsubscript{2} Janovi\textsubscript{b} přikázal\textsubscript{1} dát\textsubscript{2} svým\textsubscript{a/b} přátelům.\\
Pavel.{\NOM} them Jan.{\DAT} order.{\PST.\SG.}\textsc{m} give.{\INF} {\POSS} friends.{\DAT}\\
\glt Intended: `Pavel\textsubscript{a} ordered Jan\textsubscript{b} to give them to his\textsubscript{a/b} friends.’ \label{kul:ex:pavel-prikazal-b} \\ \hfill (Czech; \citealt[114--115]{Rezac2005})
}
\z
\z

\ea\label{kul:ex:stavitel-mu}
\gll Stavitel\textsubscript{a} \textit{mu}\textsubscript{2} nechtěl\textsubscript{1} vnucovat\textsubscript{2} své\textsubscript{a} mínění.\\
constructor.{\NOM} him.{\DAT} \textsc{neg}.want.{\PST.\SG.}\textsc{m} impose.{\INF} {\POSS} opinion \\
\glt […]\\`The constructor didn’t want to impose his opinion on him …’ \\ \hfill (Czech; ČNK)
\z

\ea\label{kul:ex:a-prave}
\gll \minsp{(} A právě \minsp{[\textsubscript{\textsc{np}}} \minsp{“} ten myš”\minsp{]\textsubscript{a}} se sourozenci\textsubscript{b} zalíbil natolik, že mě požádal,) abych \textit{mu}\textsubscript{1/b} \textit{ho}\textsubscript{2/a} dovolil\textsubscript{1} použít\textsubscript{2} v jedné své\textsubscript{b} písničce.\\
{} and exactly {} {} this.{\SG.}\textsc{m} mouse {\REFL} sibling.{\SG.\DAT} please.{\PST.\SG.}\textsc{m} so.much that me.{\ACC} ask.{\PST.\SG.}\textsc{m} so-that him.{\DAT} him.{\ACC} allow.{\PST.\SG.}\textsc{m} use.{\INF} in one {\POSS} song\\
\glt `(And [my] sibling liked exactly “this he-mouse” so much that he asked me,) if I would allow him to use it in one of his songs.’ \hfill (Czech; ČNK)
\z

\noindent Another argument that challenges the predicted binding correlations of restructuring has been put forth by \citet[316]{Golden2008} for Slovene. She observed that certain object control constructions are semantically ambiguous, although the embedded clitic has climbed \REF{kul:ex:slovene}. That is, an object may be interpreted as either being subcategorized for by the matrix verb, whereby the object controls the embedded PRO-subject, or by the embedded infinitive and no control occurs. The ambiguity remains in case of CC, although PRO should be absent due to restructuring, such that the control reading should not be available. Correspondingly, the ambiguity of Czech \REF{kul:ex:stryc-mu} and Polish \REF{kul:ex:kaza-mu} calls for an analogue of a PRO-analysis for the infinitive.\footnote{An anonymous reviewer pointed out that a PRO-less analysis is available following work by Gennaro Chierchia (see \citealt{Chierchia1984}). As the consequences of this approach are not clear to me at this moment, I will leave it for future research.}

\ea\label{kul:ex:slovene}
\ea[]{
\gll Janez \textit{ji}\textsubscript{1/2} \textit{jih}\textsubscript{2} je dovolil\textsubscript{1} kupiti\textsubscript{2}.\\
Janez.{\NOM} her.{\DAT} them.{\ACC} {\AUX.3\SG} allow.{\PST.\SG.}\textsc{m} buy.{\INF}\\ \label{kul:ex:slovene-a}
\glt
(i) `Janez allowed her to buy them.’\\
(ii) `Janez allowed (someone) to buy them/it for her/them.’ \\ \hfill (Slovene; \citealt[316]{Golden2008})
}
\ex[]{
\gll Jaz \textit{sem} \textit{ji}\textsubscript{1/2} \textit{ga}\textsubscript{2} dovolil\textsubscript{1} poslati\textsubscript{2} po pošti.\\
I.{\NOM} {\AUX.1\SG} her.{\DAT} him.{\ACC} allow.{\PST.\SG.}\textsc{m} send.{\INF} by mail \label{kul:ex:slovene-b} \\
\glt (i) `I allowed her to send it by mail.’\\
(ii) `I allowed (somebody) to send it to her by mail.’ \\ \hfill (Slovene; \citealt[312]{Golden2008})
}
\z
\z

\ea\label{kul:ex:stryc-mu}
\gll [...], strýc \textit{mu}\textsubscript{1/2} \textit{ho}\textsubscript{2} nedovolí\textsubscript{1} přečíst\textsubscript{2}.\\
{} uncle him.{\DAT} him.{\ACC} \textsc{neg}.allow.{\PRS.3\SG} read.{\INF} \\
\glt (i) `The uncle doesn’t allow him to read it.’ 
\glt (ii) `The uncle doesn’t allow (someone) to read it to him.’ \hfill (Czech; ČNK)
\z

\ea\label{kul:ex:kaza-mu}
\gll […] każą\textsubscript{1} \textit{mu}\textsubscript{1/2} \textit{go}\textsubscript{2} rozebrać\textsubscript{2}.\\
{} order.{\PRS.3\PL} him.{\DAT} him.{\ACC} deconstruct.{\INF} \\
\glt (i) `… they order him to deconstruct it.’ 
\glt (ii) `… they order (someone) to deconstruct it for him.’ \hfill (Polish; NKJP)
\z

\subsection{Temporal reference}\label{kul:sec:temp-reference}

It has been argued that the RI’s temporal reference is dependent upon the one presupposed by the matrix verb. RIs are ungrammatical with a temporal adverb which refers to a time frame deviating from the matrix verb’s one. \citeposst{Wurmbrand2001} German example \REF{kul:ex:hans-hat-a} provides a grammatical utterance without restructuring. The main verb encodes the past tense (morphosyntactically encoded by the analytical perfect form), but the embedded infinitive refers to the future by the time adverb \textit{morgen} ‘tomorrow’. On the other hand, the presence of the time adverb is ungrammatical in a restructuring context like \REF{kul:ex:hans-hat-b}. Example \REF{kul:ex:on-zeli-da} from \citet{Aljović2004} suggests that the same holds for Slavic. The presence of the time adverb \textit{sutra} ‘tomorrow’ in the embedded clause is grammatical in the BCMS example \REF{kul:ex:on-zeli-da-a}, as long as the pronominal clitic \textit{ga} ‘him.{\ACC}’ is in situ. When restructuring occurs and the clitic climbs, then the realization of the time adverb yields the utterance ungrammatical \REF{kul:ex:on-zeli-b}.

\ea\label{kul:ex:hans-hat}
\ea[]{
\gll Hans hat beschlossen \minsp{(} morgen) zu verreisen.\\
Hans have.{\PRS.3\SG} decide.{\PTCP} {} tomorrow to travel.{\INF}\\ 
\glt `John decided to go on a trip (tomorrow).’\label{kul:ex:hans-hat-a}
}
\ex[]{ 
\gll Hans hat versucht \minsp{(*} morgen) zu verreisen.\\
Hans have.{\PRS.3\SG} try.{\PTCP} {} tomorrow to travel.{\INF} \\
\glt `John tried to go on a trip.’\hfill (German; \citealt[73]{Wurmbrand2001}) \label{kul:ex:hans-hat-b}
}
\z
\z

\ea\label{kul:ex:on-zeli-da}
\ea[]{
\gll On želi\textsubscript{1} da \textit{ga}\textsubscript{2} \minsp{(} sutra) Jovanu predstavi\textsubscript{2}.\\
he want.{\PRS.3\SG} that him.{\ACC} {} tomorrow Jovan.{\DAT} introduce.{\PRS.3\SG} \\
\glt `He wants to introduce him to John tomorrow.’
\label{kul:ex:on-zeli-da-a}
}
\ex[*]{ 
\gll On \textit{ga}\textsubscript{2} želi\textsubscript{1} da \minsp{(} sutra) Jovanu predstavi\textsubscript{2}.\\
he him.{\ACC} want.{\PRS.3\SG} that {} tomorrow Jovan.{\DAT} introduce.{\PRS.3\SG} \\
\glt Intended: `He wants to introduce him to John tomorrow.’\\\hfill (BCMS; \citealt[193]{Aljović2004}) \label{kul:ex:on-zeli-b}
}
\z
\z

\begin{sloppypar}
\noindent \citet{Lenertova2004} notes that the aforementioned argument does not hold for Czech, where CC co-occurs with the embedded infinitive’s independent temporal reference. The clitic \textit{ho} ‘him.{\ACC}’ in \REF{kul:ex:misto-toho} has climbed to the matrix domain headed by the past tense verb \textit{rozhodl} ‘decide.{\PST.\SG.}\textsc{m}’. However, the realization of the temporal adverb \textit{příště} ‘next time’ or adverbial PP \textit{na moment} ‘for a moment’ within the infinitive’s domain is grammatical. Lenertová’s observation is corroborated for Czech \REF{kul:ex:pritom-ho} and Polish \REF{kul:ex:ja-sie} by corpus data. Note that climbing is grammatical irrespective of whether the time adverb(ial)s (ADV) intervene between matrix verb and embedded infinitive or not. This fits \citeposstpg{Junghanns2002a}{66} observation that the cascade of verbs, which constitutes an environment for CC, does not form a verb cluster (Germ. \textit{Verb}[\textit{al}]\textit{komplex}) in Czech, i.e. they do not need to be contiguous (cf. also \citealt[313]{Golden2008}).\largerpage
\end{sloppypar}

\ea\label{kul:ex:misto-toho}
\gll Místo toho \textit{se}\textsubscript{1} \textit{ho}\textsubscript{2} rozhodl\textsubscript{1} [\textsubscript{\textsc{adv}} na moment] /\hspace{1cm} [\textsubscript{\textsc{adv}} příště] ignorovat\textsubscript{2}.\\
instead-of this {\REFL} him.{\ACC} decide.{\PST.\SG.}\textsc{m} {} on moment {} {} next-time ignore.{\INF}\\
\glt `Instead, he decided to ignore him for a moment\slash next time.’ \\\hfill (Czech; \citealt[157]{Lenertova2004})
\ex\label{kul:ex:pritom-ho}
\ea[]{
\gll Přitom \textit{ho}\textsubscript{2} chtěla\textsubscript{1} odstartovat\textsubscript{2} [\textsubscript{\textsc{adv}} příští sobotu] při příležitosti oslav 700 let od udělení městských práv Sokolovu.\\
but-in-fact him.{\ACC} want.{\PST.\SG.}\textsc{f} launch.{\INF} {} next Saturday at occasion celebration 700 years from awarding city rights Sokolov \label{kul:ex:pritom-ho-a} \\
\glt `But in fact, [the town’s administration] wanted to launch it on the occasion of the 700\textsuperscript{th} anniversary of Sokolov receiving its town charter.’
}
\ex[]{ 
\gll Lidé, kteří \textit{se}\textsubscript{2} chtěli\textsubscript{1} [\textsubscript{\textsc{adv}} zítra večer] bavit\textsubscript{2} při filmu Borat mají smůlu. \\
people who {\REFL} want.{\PST.\PL}.\textsc{man} {} tomorrow evening entertain.{\INF} at film Borat have.{\PRS.3\PL} bad-luck \\
\glt `Those people, who wanted to enjoy the Borat movie tomorrow evening, have bad luck.’ \hfill (Czech; ČNK) \label{kul:ex:pritom-ho-b}
}
\z
\ex\label{kul:ex:ja-sie}
\ea[]{
\gll Ja \textit{się}\textsubscript{2} postanowił-em nie podrapać\textsubscript{2} [\textsubscript{\textsc{adv}} jutro o 12.15 ] […].\\
I.{\NOM} {\REFL} decide.{\PST.\SG.}\textsc{m}-\textsc{m}{.1\SG} \textsc{neg} scratch.{\INF} {} tomorrow at 12.15 \label{kul:ex:ja-sie-a} \\
\glt `I decided not to scratch myself tomorrow at 12:15{…}.’
}
\ex[]{ 
\gll ja \textit{mu}\textsubscript{2} zdecydował-em\textsubscript{1} \textit{się}\textsubscript{1} odpowiadać\textsubscript{2} [o ile na jakieś posty będzie warto] [\textsubscript{\textsc{adv}} po 24 godzinach] \\
I.{\NOM} him.{\DAT} decide.{\PST.\SG.}\textsc{m}-\textsc{m}{.1\SG} {\REFL} reply.{\INF} [at how.much on some posts be.{\FUT.3\SG} worth] {} after 24 hours. \\
\glt `I decided to respond to him [as far as some posts will be worth it] after 24~hours.’ \hfill (Polish; NKJP) \label{kul:ex:ja-sie-b}
}
\z
\z

\subsection{All or nothing}\label{kul:sec:all-nothing}

CC has been deemed an “all-or-nothing phenomenon” \citep[111]{Rezac2005}, whereby either all embedded clitics climb or none (see \citealt[194]{Aljović2004} for a similar position regarding BCMS). Due to the RI’s structural deficiency, the clitics escape the infinitival domain to satisfy their formal requirements in the matrix phrase, where they are placed in the respective clitic cluster. \textsc{Diaclisis} of co-dependents poses a problem for such an approach then.\footnote{I adopt the term \textsc{diaclisis} for split-clitic-constructions from \citet[34]{KolakovicFritz2022}, who took the notion from \citeposst{Janse1998} discussion of clitics in Cappadocian Greek.} In the empirically attested Serbian example \REF{kul:ex:i-pocelo}, both the pronominal clitic \textit{mi} ‘me.{\DAT}’ and the reflexive clitic \textit{se} ‘{\REFL}’ are subcategorized for by the embedded verb \textit{vrti} ‘spin.{\PRS.3\SG}’ of the \textit{da}-clause.\footnote{An anonymous reviewer pointed out that the availability of diaclisis in BCMS was already recognized by Sandra Stjepanović (see \citealt{Stjepanovic1998, Stjepanovic1999, Stjepanovic2004}). The data remained controversial, as the positive judgement of diaclisis has not been generally accepted (cf. \citealt[192 Footnote~3]{Aljović2004}, \citealt[335]{FranksKing2000}).} However, it is only the pronominal clitic that climbs, and the reflexive remains in situ (see \citealt[307]{KolakovicFritz2022}). As restructuring is supposed to affect all embedded clitics equally, the approach is unable to predict differences in the distribution of co-dependent clitics. This is also true for diaclisis in Czech \REF{kul:ex:lehce-sie}  and Polish \REF{kul:ex:my-sie}.\footnote{An anonymous reviewer suggested that an all-or-nothing analysis might be available for Czech, if one assumes a verb-adjacent placement pattern with both pronouns leaning on the matrix verb. Such an approach is debatable. It remains unclear why Czech special clitics do not build a cluster (cf. also Footnote~\ref{kul:foot:TheChechData}). Bulgarian and Macedonian clitics are verb-adjacent and obey clitic clustering (but do not have clitic climbing, cf. \citealt[241]{FranksKing2000}). In the scenario suggested, the reflexive needs to procliticize, while the pronominal encliticizes on the same host. Thus, the prosodic orientation appears to be rather arbitrary, contra \citet{Toman1996}, who argues that, while short pronouns encliticize by default in Czech, procliticization is a function of the phonological environment, i.e. the lack of a prop due to a prosodic break (see also \citealt[178--179]{Junghanns2021}).} Again, both reflexive and pronominal clitics are co-dependents of the embedded infinitives. Only the reflexive clitic occupies the clausal second position, whereas the pronominal clitic appears further to the right and does not build a cluster with the reflexive. Uwe Junghanns (p.c.) pointed out that it is impossible to determine a priori whether the pronominal clitics in \REF{kul:ex:lehce-sie}--\REF{kul:ex:my-sie} have in fact climbed or whether they are still positioned within the infinitival phrase (see also \citealt[67--68]{Junghanns2002a}).\footnote{The contrast is schematised in (i) and (ii).
\begin{enumerate}
    \item[(i)] [\textsubscript{\textalpha}  V\textsubscript{\textalpha} CL\textsubscript{\textbeta} [\textsubscript{\textbeta} V\textsubscript{\textbeta}]] (CC/non-local)
    \item[(ii)] [\textsubscript{\textalpha} V\textsubscript{\textalpha} [\textsubscript{\textbeta} CL\textsubscript{\textbeta} V\textsubscript{\textbeta}]] (in situ/local)
\end{enumerate}
         
\noindent Proclisis to the following infinitive would be indicative of clitic in situ placement. Proclisis is available in Czech (see \citealt{Toman1996}) and Polish (see \citealt[62--64]{Kraska-Szlenk1995}) for the set of clitics relevant here.} While this behavior would appear unsurprising for Polish given the rather peculiar status its clitic system is assigned, the occurrence of the same pattern in Czech is unexpected. The Czech data are problematic for \citeposst{Bošković2001} PF-filtering approach to clitic clustering, according to which clitics in a second position clitic language are placed according to two parameters: first, initial positioning in an intonation phrase (\textiota\textRho), second, being suffixed to a prop. Bošković accounts for diaclisis in Polish by assuming that Polish clitics do not possess the second position requirement of being \textiota\textRho-initial.\footnote{\label{kul:foot:TheChechData}The Czech facts cannot be captured, since the approach predicts that split clitics are placed in distinct \textiota\textRho s. However, there is only one \textiota\textRho\, for the relevant clause in Czech \REF{kul:ex:lehce-sie}, cf. (i)–(ii). I thank Martina Berrocal (p.c.) for the judgement (\# marks a pause).
\begin{enumerate}
    \item[(i)]  [\textsubscript{\textiota\textRho} Lehce \textit{si} (*\#) uměla \textit{ho} představit]
    \item[(ii)] [\textsubscript{\textiota\textRho} kdekdo \textit{se} (*\#) začal \textit{mi} smáti]
\end{enumerate}
} 
Note that the diaclitic distribution is independent of the argument status of the clitics. The reflexives in the a-examples are true arguments of the embedded infinitives, whereas they are not in the b-examples. Both Cz. \textit{smát(i) se} ‘laugh’ and Pol. \textit{bać się} ‘be afraid’ are reflexiva tantum, i.e. the appearance of the reflexive is lexically specified. This finding supplements \citeposstpg{Lenertova2004}{138--139} observation that Czech cli\-tics do not need to cluster together, as she found that conditional and refle\-xive/pronominal clitics may occur non-contiguously.

\ea\label{kul:ex:i-pocelo}
\gll […] i počelo\textsubscript{1} \textit{mi}\textsubscript{2} \textit{je} \minsp{[} da \textit{se}\textsubscript{2} vrti\textsubscript{2} u glavi].\\
{} and start.{\PST.\SG.}\textsc{n} me.{\DAT} {\AUX.3\SG} {} that {\REFL} spin.{\PRS.3\SG} in head\\
\glt `… and I started to feel dizzy.’  \hfill (Serbian; \citealt[307]{KolakovicFritz2022})
\z

\ea\label{kul:ex:lehce-sie}
\ea[]{
\gll Lehce \textit{si}\textsubscript{2} uměla\textsubscript{1} \textit{ho}\textsubscript{2} představit\textsubscript{2}, […]\\
easily {\REFL} be-able.{\PST.\SG.}\textsc{f} him.{\ACC} imagine.{\INF}\label{kul:ex:lehce-sie-a} \\
\glt `She could easily imagine him, …’
}
\ex[]{ 
\gll […] kdekdo \textit{se}\textsubscript{2} začal\textsubscript{1} \textit{mi}\textsubscript{2} smáti\textsubscript{2}. \\
{} almost-everybody {\REFL} start.{\PST.\SG.}\textsc{m} me.{\DAT} laugh.{\INF} \\
\glt `… almost everybody started to laugh at me.’ \hfill (Czech; ČNK) \label{kul:ex:lehce-sie-b}
}
\z
\z 

\ea\label{kul:ex:my-sie}
\ea[]{
\gll My \textit{się}\textsubscript{2} musimy\textsubscript{1} \textit{go}\textsubscript{2} nauczyć\textsubscript{2}.\\
we.{\NOM} {\REFL} must.{\PRS.1\SG} him.{\ACC} teach.{\INF} \label{kul:ex:my-sie-a} \\
\glt `We have to learn it.’
}
\ex[]{ 
\gll Już \textit{się}\textsubscript{2} zaczęli\textsubscript{1} \textit{go}\textsubscript{2} bać\textsubscript{2}, […] \\
already {\REFL} start.{\PST.\PL.}\textsc{map} him.{\ACC} fear.{\INF} \\
\glt `They already started to be afraid of him, …’ \hfill (Polish; NKJP) \label{kul:ex:my-sie-b}
}
\z
\z

\section{Clitic climbing and information structure}\label{kul:sec:cc-inform-structure}

The previous sections showed that a purely syntactic account in terms of restructuring cannot cope with CC in Czech and Polish. The corpus data provided in \sectref{kul:sec:climbing-restructure} contradict the predictions of the approach. Therefore, I agree with \citet[87]{Dotlačil2004} in that CC does not occur because of restructuring and that both should be regarded as independent phenomena.

If restructuring is not responsible for CC, then the question arises, what is. Which alternatives are available, if one refrains from accounting for CC by syntax proper? Proposals in terms of phonology, morphology, and syntax-prosody interaction have been put forward, but were also criticized. For instance, \citet[287--291, 293--305]{FranksKing2000} critically review purely phonological and purely syntactic accounts of cliticization (incl. CC) and conclude that both types of approach face several problems in accounting for clitic phenomena (cf. also the discussion in \citealt[36--80]{Bošković2001}). However, I do not pursue a mixed syntax-PF account like \citeposstpg{FranksKing2000}{§11--12} and \citeposst{Franks2010} PF-filtering approach. Another mixed account is \citeposst{Halpern1995} Prosodic Inversion, which is critically reviewed in detail by \citet[11--36]{Bošković2001}. On the other hand, \citeposst{Bošković2001} own intonational-phrase-based proposal has been criticized by \citet[150--151]{Lenertova2004} and \citet[passim]{Golden2008} (see also Footnote~\ref{kul:foot:TheChechData}). An explanatory account of CC has to shed light on the actuation or causation of CC vs. clitic in situ positioning.\footnote{{I refer to the traditional notion of explanation based on causality, not to \citeposstpg{Chomsky1965}{25--26} concept of explanatory adequacy. Instead, I allude to the actuation problem coined by \citet{WeinreichHerzog1968} towards the background of historical linguistics and adopt it for the field of synchronic grammar research.}} I propose to take a candidate into consideration that repeatedly appears in the literature on clitics, but has been mostly neglected: information structure. \citet[206]{Stjepanovic2004} considers the possibility that CC is an instance of object shift, which has been reported to rely on information structural notions in Northern Germanic: only objects having background status are shifted.

\citet[82--83]{Junghanns2002a} takes the farthest step towards information structure I am aware of and proposes that information structure is the actual reason for CC, whereas syntax merely restricts which domains clitics can escape. Consequently, a clitic climbs, if it belongs to the background of the whole sentence, else it remains in situ. The else-case covers utterances in which the clitic is part of a topicalized or focused constituent. The clitic itself does not bear topic or focus, but is an element of a domain specified as either [$+\text{Topic}$] or [$+\text{Focus}$].

Accounting for the ban of CC across CP, Dotlačil (\citeyear{Dotlačil2004}: 93, 98, \citeyear{Dotlačil2007}: 89) suggests that clitics cannot escape CPs, because they cannot bear the discourse functions of topic or focus. He observes that topicalized or focused constituents are able to escape CPs in Czech. If clitics are hosted in such a topicalized or focused domain, they can cross a CP as a part of the respective constituent. However, no CC occurs, as the clitics remain in their licensing domain.

I adopt \citeposst{Junghanns2002a} proposal and paraphrase it tentatively with the notation in \REF{kul:ex:background-topic-focus}, which reads as follows: for every $x$, if $x$ has property CL ($=\text{is a clitic}$) and $x$ is element of the information structural background, then $x$ is realized in domain {\textalpha} and co-indexed with a gap \textit{e} in domain {\textbeta}. Implication \REF{kul:ex:background-topic-focus-b} specifies the else-case with the clitic in situ realization. Note that the asymmetry between {\textalpha} and  {\textbeta} is purely syntactic and refers to dominance or hierarchic order, but does not tell us anything about linear order.

\ea\label{kul:ex:background-topic-focus}
\ea[]{$\forall x\text{(CL}(x) \wedge x \in \text{\textsc{[Background]}} \rightarrow [\textsubscript{{\textalpha}} x\textsubscript{j} [\textsubscript{β} e\textsubscript{j}]])$}\label{kul:ex:background-topic-focus-a}
\ex[]{$\forall x\text{(CL}(x) \wedge x \in \text{\{\textsc{[Topic], [Focus]}\}} \rightarrow [\textsubscript{{\textalpha}} … [\textsubscript{β} x ]])$}\label{kul:ex:background-topic-focus-b}
\z
\z

\section{Conclusion}\label{kul:sec:conclusion}

The paper empirically tested the hypothesis that clitic climbing in Czech and Polish is contingent upon a mono-clausal restructuring environment. In particular, I reviewed the predicted correlates of the proposal that clitics escape defective infinitival complements, which are bare VPs. Utilizing data from both the Czech National Corpus and the National Corpus of Polish, it has been shown that (i) accusative clitics climb to passivized domains incapable of accusative case licensing, (ii) binding phenomena and ambiguities in climbing constructions call for an underlying subject (PRO) analysis of the infinitival domain, (iii) embedded infinitives possess temporal reference independent of the finite matrix verb, and (iv) co-dependent clitics do not behave uniformly with respect to climbing and end up non-contiguously. In sum, the respective predictions of the restructuring approach have been falsified. Clitic climbing thus cannot be regarded as being restricted to mono-clausal structures, but occurs in what is considered an underlyingly bi-clausal structure. This state of affairs yields the approach ineligible for clitic climbing in Czech and Po\-lish. More generally, as syntax proper does not provide us with an explanatory account for the very existence of clitic climbing, alternatives have to be taken into consideration seriously. Following \citet{Junghanns2002a}, I referred to information structure, whereby clitics climb, if they are elements of the background of the entire sentence, but remain in situ, if they are elements of a topicalized or focused constituent. Admittedly, the present study neither addresses how to account for diaclisis in terms of information structure nor how topic and focus domains are determined in order to capture clitic in situ positioning. This has to be dealt with in future research. What is more, the resemblance between Czech and Polish clitic distributions suggests that the typological peculiarity of Polish is not well-grounded. I propose to revisit and refine the micro-typology of Slavic cliticization on a sound empirical basis.

\section*{Abbreviations}

\begin{multicols}{2}
\begin{tabbing}
\textsc{nkjp}\hspace{.5em}\= National Corpus of Polish\kill
\textsc{1} \> first person\\
\textsc{2} \> second person\\
\textsc{3} \> third person\\
\textsc{acc} \> accusative\\
\textsc{aci} \> accusative with infinitive\\
\textsc{adv} \> adverb(ial)\\
\textsc{aux} \> auxiliary\\
\textsc{cc} \> clitic climbing\\
\textsc{čnk} \> Czech National Corpus\\
\textsc{dat} \> dative\\
\textsc{f} \> feminine\\
\textsc{fut} \> future tense\\
\textsc{gen} \> genitive\\
\textsc{inf} \> infinitive\\
\textsc{m} \> masculine\\
\textsc{n} \> neuter\\
\textsc{neg} \> negation\\
\textsc{nkjp} \> National Corpus of Polish\\
\textsc{nom} \> nominative\\
\textsc{pass} \> passive\\
\textsc{man} \> masculine-animate\\
\textsc{map} \> masculine-personal\\
\textsc{pl} \> plural\\
\textsc{poss} \> possessive\\
\textsc{prs} \> present tense\\
\textsc{pst} \> past tense\\
\textsc{ptcp} \> participle\\
\textsc{refl} \> reflexive\\
\textsc{ri} \> restructuring infinitive\\
\textsc{sg} \> singular
\end{tabbing}
\end{multicols}

\section*{Acknowledgments}
I would like to thank three anonymous reviewers and the audiences of FDSL-14 (Leipzig, 2–4 Jun, 2021) and SLS-16 (Urbana-Champaign, 3–5 Sep, 2021) for their valuable comments and suggestions, especially Uwe Junghanns, Hagen Pitsch, Luka Szucsich, Steven Franks, Peter Kosta, and Jacek Witkoś. I also owe thanks to Ruprecht von Waldenfels for discussion and to Björn Hansen, Edyta Jurkiewicz-Rohrbacher, and Zrinka Kolaković for an inspiring exchange of ideas. I kindly thank Susanne Wurmbrand for providing me with information on recent deve\-lopments in the study of restructuring phenomena. Finally, I am much indebted to Martina Berrocal, who kindly helped me with the Czech data and provided native speaker judgements. All improvements are thanks to the aforementioned individuals, while all shortcomings remain my sole responsibility.

\printbibliography[heading=subbibliography,notkeyword=this]

\end{document}
