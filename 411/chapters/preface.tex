\addchap{Preface}
\begin{refsection}
\textit{Advances in formal Slavic linguistics 2021} offers a collection of thirteen high quality articles on Slavic linguistics. The volume covers all branches of Slavic languages and features synchronic as well as diachronic analyses. It contains both empirically oriented work, underpinned by experimental methods or corpora analyses, and more theoretically based contributions. It comprises a wide array of topics, such as degree achievements, clitic climbing in Czech and Polish, typology of Slavic \textit{l}-participles, aspectual markers in Russian and Czech, doubling in South Slavic relative clauses, congruence and case-agreement in close apposition in Russian, cataphora in Slovenian, Russian and Polish participles, prefixation and telicity in Serbo-Croatian, Bulgarian adjectives, negative questions in Russian and German and imperfectivity in discourse.\largerpage[2.5]

Early versions of the papers included in this volume were presented at the conference on Formal Description of Slavic Languages 14 or at the satellite Workshop on Secondary Imperfectives in Slavic, which were held in Leipzig on June 2--5, 2021 -- the year referred to in the title of the volume. Originally, the conference was set to December 2020 but due to the Covid pandemic it had to be postponed and could only take place in the hybrid format in June 2021.

Three quarters of the submitted abstracts made it into the 36 presentations of the conference. Each article underwent an extensive reviewing process in line with the usual standards (double-blind peer reviewing). The conference also featured 5 invited talks. The 13 papers in the present volume were developed from these contributions in the course of a further thorough reviewing process. Neither the original conference nor the present volume would have been possible without the readiness of so many experts to devote their time and thoughts to the critical evaluation and helpful commenting of their colleagues’ research papers. We would like to thank both the 38 anonymous reviewers for the present volume, and the more than 80 reviewers of the original conference abstracts.

This book would have also been impossible without our student assistants, Anastasiya Koretskykh and Julius Lambert. We also wish to acknowledge the extensive technical support of the whole Language Science Press editorial team, particularly Radek Šimík and Berit Gehrke.\bigskip\\
Petr Biskup, Marcel Börner, Olav Mueller-Reichau \& Iuliia Shcherbina\\
Leipzig, 21 July 2023
\end{refsection}

