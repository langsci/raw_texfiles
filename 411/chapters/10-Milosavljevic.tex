\documentclass[output=paper,colorlinks,citecolor=brown]{langscibook}
\ChapterDOI{10.5281/zenodo.10123645}

\author{Stefan Milosavljević\affiliation{University of Graz}\orcid{0000-0003-2305-2519}}
% replace the above with you and your coauthors
% rules for affiliation: If there's an official English version, use that (find out on the official website of the university); if not, use the original
% orcid doesn't appear printed; it's metainformation used for later indexing

%%% uncomment the following line if you are a single author or all authors have the same affiliation
\SetupAffiliations{mark style=none}


% replace the above with your paper title
%%% provide a shorter version of your title in case it doesn't fit a single line in the running head
\title[The sequence of similar events interpretation in Slavic]{Simple imperfective verbs, the sequence of similar events interpretation, and Slavic aspectual composition}
\abstract{The paper examines the so-called sequence of similar events (SSE) interpretation in Serbo-Croatian (SC), which emerges with telic predicates expressed by imperfective verbs in the presence of bare plural objects. I show that this is an interpretation that, just as in English, allows the use of both durative adverbials (DurAds) and time-span adverbials (TSAds) at the same time. I argue that TSAds, as standardly assumed, modify a telic event predicate, while DurAds merge once the predicate has been made homogeneous/atelic by the plural operator (contra \citeauthor{MacDonald_2008}'s \citeyear{MacDonald_2008} claim that DurAds combine with telic predicates in such cases). The fact that the SSE interpretation is available in SC (or Slavic more generally) for imperfective verbs -- including simple ones -- suggests that in Slavic there is a syntactic projection responsible for telicity analogous to that in English, and telicity of a verbal predicate can be triggered by the quantity properties of its internal arguments. 


\keywords{simple imperfective verbs, sequence of similar events interpretation,  telicity, Serbo-Croatian, Slavic, English}
}

\lsConditionalSetupForPaper{}

\begin{document}
\maketitle


\section{Introduction} 
The temporal modification test (TMT) is one of the most standard diagnostics for (a)telicity, according to which durative adverbials (DurAds), often referred to as \textit{for}-adverbials, modify atelic predicates, whereas time-span adverbials (TSAds), widely known as \textit{in}-adverbials, modify telic predicates -- but not vice versa, as in \REF{mil:ex:TMT} from English.\footnote{The TMT is probably the most widely used test for telicity since it is employed regardless of the exact way telicity is approached -- in terms of the event-argument homomorphism (e.g. \citealt{Dowty1991, Krifka1992, Krifka1992}), the result state component (e.g. \citealt{Pustejovsky1995}), atomicity (e.g. \citealt{Rothstein2008a, Rothstein2008b}), non-homogeneity/quantity (e.g. \citealt{Borer_2005}), scale features (e.g. \citealt{Hayetal1999}); for an overview of different approaches to telicity see, e.g., \citet{ArsenijevicMarin2013}.} 

\ea \label{mil:ex:TMT}
\ea John ran \textit{for an hour} / *\textit{in an hour}. \hfill \textsc{atelic}
\ex John wrote a letter \textit{in an hour} / *\textit{for an hour}.       \hfill \textsc{telic}
\z
\z

\noindent According to \citet{MacDonald_2008}, these two adverbials can be combined in English under the so-called sequence of similar events (SSE) interpretation, illustrated in \REF{mil:ex:in+for}. The SSE interpretation, as analyzed in \citet{MacDonald_2008}, emerges when a predicate is telic, with bare plurals (BPs) contributing an indefinite number of objects that can participate in each of the iterated subevents. The BP bears the feature [$+$q] (akin to the $+$SQA feature in \citealt{Verkuyl_1972, Verkuyl1999}, standing for the specified quantity), and telicity emerges due to the so-called object-to-event mapping (OTEM) (in the sense of \citealt{Verkuyl_1972}). The contribution of DurAds, under such a view, amounts to assigning an indefinite number of repetitions to the telic event.

\ea \label{mil:ex:in+for}
The guy drank cans of beer \textit{in ten seconds} \textit{for an hour straight}.
\z 

\noindent Given that DurAds and TSAds are expected to be in complementary distribution, as the same predicate cannot be both telic and atelic at the same time, their combination is (at least at first glance) unexpected. \citet[36]{MacDonald_2008} claims that in such cases (i.e. under the SSE reading), it is possible to combine DurAds with telic predicates,\footnote{\citet[33]{MacDonald_2008} also refers to other works arguing that DurAds are compatible with telic predicates under the iterative interpretation \citep{Alsina1999,Jackendoff1996, Schmitt1996, Tenny1987, VandenWyngaerd2001}.} rejecting the widely accepted generalization that DurAds require atelicity, also known as the homogeneity requirement (see \citealt{Borer_2005,Csirmaz_2009,LandmanRothstein2010,LandmanRothstein2012a,LandmanRothstein2012b}, a.o).

\citet{MacDonald_2008} claims that the SSE interpretation is available in English, but not in Russian (/Slavic).\footnote{MacDonald analyzes only Russian data, but many of his claims about Russian hold for Serbo-Croatian, which is why I generalize some of his claims in the present paper.} His argumentation, based on the analysis of Russian simple imperfective verbs and (prefixed) perfective verbs, proceeds in the following way: the SSE interpretation requires telic predicates, in Russian only perfective verbs are telic, but perfectives are incompatible with the SSE interpretation. Imperfectives, on the other hand, are always atelic, and bare plurals, when combined with an imperfective verb, have a vague denotation associated with the mass noun interpretation (a group interpretation in which different parts of all the objects are affected at the same time), hence they never induce the SSE interpretation \citep [147]{MacDonald_2008}. He takes this (purported) difference between SSE in English and Russian as one of the main arguments for the claim that aspectual composition in these languages is radically different. Except for the SSE interpretation, aspectual composition in Russian differs from that in English in the unavailability of OTEM, i.e. in Russian an NP cannot affect the aspectual interpretation, and, consequently, this language lacks the syntactic projection responsible for inner aspect. Namely, in Russian, as stated by \citeauthor{MacDonald_2008}, inner aspectual properties are determined through the event features, and only perfective verbs are equipped with the feature specifying the endpoint of the event, which triggers telicity. This feature is determined in the lexicon, before entering the narrow syntax, and can be either brought about by the lexical prefixes through the lexical derivational process, or lexically specified (in the case of simple perfective verbs). In English, on the other hand, there is an AspP between the \textit{v}P and the \textit{v}P\footnote{The \textit{v}P hosts the external argument in \citeauthor{MacDonald_2008}'s approach.} with which (features of) NPs interact. An NP yielding telicity has the quantity feature [$+$q], while an NP that fails to induce telicity has the [$-$q] feature. The relation between the NP and the AspP is established via Agree, leading to telicity if the NP is [$+$q], or atelicity if the NP is [$-$q] \citep{MacDonald_2008, MacDonald2010, MacDonald2012}.
The idea that in Slavic, unlike in English, internal arguments of the verb (incremental direct objects and/or goal PPs) do not contribute to telicity is a fairly standard one (see e.g. \citealt{Łazorczyk2010, Rothstein2016, FleischhauerGabrovska2019}, among many others). Instead, it is a common view that telicity is triggered by prefixation (\citealt{ Borer_2005, Nossalik2007, Łazorczyk2010, Svenonius2004a, Svenonius2004b, Slabakova2005, Arsenijević2007b, Ramchand2008, FleischhauerGabrovska2019}, a.o.).

In this paper, I adopt the view of telicity as computed based on the quantity properties along the lines of \citet{Borer_2005}: a predicate is telic (= Quantity) if it is non-homogeneous, i.e. if it is quantized or non-cumulative. The Quantity is assigned in the projection specifying the value of inner aspect -- Asp\textsubscript{Q} in \citet{Borer_2005}, or Q(uantification)P(hrase) in \citet{Arsenijević2006book, Arsenijević2007a, Arsenijević2013}. I assume that the presence of TSAds signals that a predicate is telic (non-homogeneous\slash bounded), i.e. that the QP is activated, following standard analyses (e.g. \citealt{Krifka_1998, Borer_2005, Arsenijević2006book, MacDonald_2008, Mittwoch2010, Mittwoch2013, Mittwoch2019}, among many others). 

Based primarily on data from Serbo-Croatian (SC), I show that the SSE interpretation is available in Slavic, and emerges in the presence of bare plurals when a telic predicate is expressed by an imperfective verb, as in \REF{mil:ex:SSE-intro}.\footnote{Unless explicitly indicated otherwise, all Slavic examples in the paper are from SC.} I offer an analysis according to which the QP in SC/Slavic can be triggered by the quantity properties of internal arguments, and the SSE interpretation emerges once a telic predicate has been made homogeneous/atelic by the (covert) plural operator (in the sense of \citealt{vanGeenhoven2004, vanGeenhoven2005, Arsenijević2006eah}). Under such an approach, TSADs, as expected, modify the QP, while DurAds combine with a homogeneous (plural) predicate. This is in line with the standard view that DurAds always combine with atelic/homogeneous structures (e.g. \citealt{Borer_2005, Csirmaz_2009, Mittwoch2010, LandmanRothstein2010, LandmanRothstein2012a, LandmanRothstein2012b}), and contra \citeauthor{MacDonald_2008}'s claim that in the case of SSE interpretation DurAds are compatible with telic predicates. The proposed analysis also implies that both Slavic and English employ a syntactic projection responsible for telicity (contra \citealt{MacDonald_2008}), i.e. aspectual composition in these languages is not radically different in this regard. Within the proposed system, prefixes are argued to be specifiers of singularity which combine with telic predicates rather than introducing telicity/perfectivity by themselves, as commonly assumed.

\ea \label{mil:ex:SSE-intro}
\gll Mika   je  pet minuta pio / iz-pi-ja-o limenke piva za deset sekundi. \\
     Mika \textsc{aux} five minute.\textsc{gen.pl} drink.\textsc{ptcp.m} {} \textsc{pref}-drink-\textsc{si-ptcp.m} can.\textsc{acc.pl} beer.\textsc{gen.sg} in\textsubscript{za} ten second.\textsc{gen.pl} \\
\glt `Mika drank cans of beer in ten seconds for an hour.'
\z

\noindent The paper is organized as follows. In \sectref{mil:sec:SlavicImperfectivityTelicity}, I briefly introduce and discuss the relationship between (im)perfective verbs and telicity in Slavic. In \sectref{mil:sec:SSE}, I analyze the SSE interpretation in SC. \sectref{mil:sec:Residual-Issues} addresses the broader picture, in particular how prefixed and biaspectual verbs fit into the proposed model of Slavic aspectual composition. \sectref{mil:sec:Conclusion} concludes the paper.


\section{Slavic (im)perfectivity vs. telicity} \label{mil:sec:SlavicImperfectivityTelicity}

The question of how telicity is assigned in Slavic is tightly related to the ongoing debate on the relationship between (im)perfectivity and telicity in this group of languages. As is well-known, in Slavic languages, verbs are traditionally divided into two classes: imperfective verbs (IVs) and perfective verbs (PVs).\footnote{In all examples from SC the superscripts \textsuperscript{I} and \textsuperscript{P} stand for IVs and PVs, respectively.} A typical way in which aspect morphology is expressed in SC is illustrated in \REF{mil:ex:AspectualMorphologyOfSCverb}. The verb in \REF{mil:ex:AspectualMorphologyOfSCverb-a} consists of just a root, a theme vowel and an inflectional ending. Most such verbs are imperfective and can be perfectivized by prefixation, as in \REF{mil:ex:AspectualMorphologyOfSCverb-b}. The prefixed verb can be imperfectivized by a secondary imperfectivizing suffix, as in \REF{mil:ex:AspectualMorphologyOfSCverb-c}. Finally, an imperfective verb derived in this way can be made perfective again by prefixation, as illustrated in \REF{mil:ex:AspectualMorphologyOfSCverb-d}. (The same holds, mutatis mutandis, for other Slavic languages.)

\ea \label{mil:ex:AspectualMorphologyOfSCverb}
\ea[]{
\gll vrš-i-ti\textsuperscript{I}\\
perform-\textsc{tv}-\textsc{inf}
\\
\glt `to perform' \label{mil:ex:AspectualMorphologyOfSCverb-a}
}
\ex[]{
\gll iz-vrš-i-ti\textsuperscript{P}\\ \textsc{pref}-perform-\textsc{tv}-\textsc{inf}\\
\glt `to perform/execute' \label{mil:ex:AspectualMorphologyOfSCverb-b}
}
\ex[]{
\gll iz-vrš-ava-ti\textsuperscript{I} \\
\textsc{pref}-perform-\textsc{si}-\textsc{inf} \\
\glt `to perform/execute'} \label{mil:ex:AspectualMorphologyOfSCverb-c}
\ex[]{
\gll po-iz-vrš-ava-ti\textsuperscript{P}\\
\textsc{pref}-\textsc{pref}-perform-\textsc{si}-\textsc{inf}\\
\glt `to perform/execute all'} \label{mil:ex:AspectualMorphologyOfSCverb-d}
\z
\z

\noindent According to one of the most standard tests, if a verb can be used as a complement of a phasal verb, it is imperfective; otherwise, it is perfective, as in \REF{mil:ex:imperfectivityTest}; see \citet{Borik_2006, Łazorczyk2010, Zinova2021} for discussion of different tests. This will be the main diagnostics applied in this paper as well.

\ea \label{mil:ex:imperfectivityTest}
\gll Jovan je počeo da peva\textsuperscript{I} / \minsp{*} od-peva\textsuperscript{P} pesmu.\\
     Jovan 
     \textsc{aux}
     begin.\textsc{tv.ptcp.m}
     \textsc{comp}
     sing.\textsc{prs.3sg}
    {} {}
     \textsc{pref}-sing.\textsc{prs.3sg}
     song.\textsc{acc.sg}\\
\glt `Jovan began to sing a song.'
\z

\noindent However, the exact status of PVs and IVs is largely debated. Probably the most common view is that they are grammaticalized forms of the (perfective and imperfective) grammatical (viewpoint/outer) aspect in Slavic (cf. e.g. \citealt{Pereltsvaig2005, Borik_2006, Ramchand2008, Rothstein2016, Minor2022}). \citet{Łazorczyk2010} and \citet{Tatevosov2011, Tatevosov2015Slavic} argue for separating grammatical aspect from the verb, since it can only emerge once the clausal architecture is fully established (given that the viewpoint depends on the interaction between the event time and the reference time).\footnote{One of the classical definitions is that of \citet[16]{Comrie1976}, for whom ``perfectivity indicates the view of a situation as a single whole, without distinction of the various separate phases that make up that situation; while the imperfective pays essential attention to the internal structure of the situation''. According to a more formal definition, inspired by work of \citet{Reichenbach1947}, imperfective viewpoint arises when the Reference Time interval is included in the Event time interval (hence, we look at the event ``from the inside''), whereas perfective viewpoint stands for the Event Time interval being contained within the Reference Time interval (hence the event is seen ``from the outside'') (cf. \citealt{Klein1994, Bhatt_Pancheva2005, Łazorczyk2010}); for recent overviews, see \citet{Arche2014a, Arche2014b, Rothstein2016}.} I adopt \citeauthor{Łazorczyk2010}'s and \citeauthor{Tatevosov2011}'s view on divorcing the Slavic verb (aspectual morphology included) from grammatical aspect, and in the remainder of the paper I will not go into a deeper discussion of how grammatical aspect is to be analyzed.

\citet{Łazorczyk2010} argues that IVs and PVs in Slavic are better accounted for in terms of telicity (PVs) vs. atelicity (IVs).\footnote{Such a view is assumed in \citet{MacDonald_2008} as well. \citet[]{Borer_2005} also analyzes Slavic perfectivity as Quantity/telicity and simple IVs as atelic, but she treats secondary imperfectives as species of outer aspect (in the sense of \citealt[]{Verkuyl_1972}).} Typically, indeed, Slavic IVs and PVs are used as counterparts of English atelic vs. telic predicates, as shown in \REF{mil:ex:SCcounterparts}, with SC equivalents of English examples from \REF{mil:ex:TMT} above.\largerpage

\ea \label{mil:ex:SCcounterparts}
\ea[]{
\gll Džon je trčao\textsuperscript{I} sat vremena / \minsp{*} za sat vremena.\\
     John \textsc{aux} run.\textsc{tv.ptcp.m} hour.\textsc{acc.sg} time.\textsc{gen.sg} {} {} in\textsubscript{za} hour.\textsc{acc.sg} time.\textsc{gen.sg}\\
\glt `John ran for an hour/ *in an hour.'
}
\ex[]{
\gll Džon je na-pisao\textsuperscript{P} pismo za sat vremena / \minsp{*} sat vremena.\\
     John 
     \textsc{aux}
     \textsc{pref}-write.\textsc{tv.ptcp.m} letter.\textsc{acc.sg} in\textsubscript{za} hour.\textsc{acc.sg} time.\textsc{gen.sg} {} {} hour.\textsc{acc.sg} time.\textsc{gen.sg}\\
\glt `John wrote a letter in an hour/ *for an hour.'}
\z
\z

\noindent In some contexts, however, IVs are compatible with TSAds, e.g. in habitual and general-factual uses, illustrated in (\ref{mil:ex:TSADsIterative}--\ref{mil:ex:general-factual}) from SC. There are also some PVs that combine with DurAds, e.g. those with the delimitative prefix \textit{po-}, as in \REF{mil:ex:delimitative}. Strictly relying on the TMT, IVs in \REF{mil:ex:TSADsIterative} and \REF{mil:ex:general-factual} could be treated as telic, while PVs like those in \REF{mil:ex:delimitative} should be atelic. These types of contexts have led some researches to claim that (im)perfectivity and telicity are independent systems in Slavic (e.g. \citealt{Borik_2006, Gehrke2008, Gehrke2008b, Ramchand2008, Stanojević2012, FleischhauerGabrovska2019}). In the remainder of the paper, I focus on IVs in telic environments.\footnote{See \citet{Milosavljević2022} for a detailed analysis of perfectives with the delimitative prefix \textit{po-} as telic predicates.} Some authors argue that telicity in such cases is possible only with secondary imperfectives,  claiming that it is the prefix that is responsible for telicity of imperfective verbs (e.g. \citealt{Stanojević2012, FleischhauerGabrovska2019}). Yet, examples like \REF{mil:ex:general-factual} show that telic readings emerge also in the absence of prefixes (and see \citealt{Pereltsvaig2000, Szucsich2000, Szucsich2001, BraginskyRothstein2008}, \citetv{chapters/01-Arsenijevic} for similar kinds of examples).

\ea \label{mil:ex:TSADsIterative}
\gll Pera	je	uvek	iz-pad-a-o\textsuperscript{I}	iz	igre	za	par	minuta.\\ Pera	\textsc{aux}	always	out-fall-\textsc{si-ptcp.m}		from	game.\textsc{gen.sg}	in\textsubscript{za}	couple	minute.\textsc{gen.pl}
     \\
\glt `Pera has always been out of the game in a couple of minutes.'
\ex \label{mil:ex:general-factual}
\gll Žika	se	već	peo\textsuperscript{I}	na	to	brdo	za	pola	sata.\\ Žika	\textsc{refl}	already	climb.\textsc{ptcp.m}	on	that	hill.\textsc{acc.sg}	in\textsubscript{za}	half	hour.\textsc{gen.sg}
     \\
\glt `Žika (has) already climbed that hill in half an hour.'
\ex \label{mil:ex:delimitative}
\gll Mika je juče po-sedeo\textsuperscript{P}	kod nas par sati.\\ Mika	\textsc{aux}	yesterday	\textsc{del}-sit.\textsc{tv.ptcp.m}	at us couple hour.\textsc{gen.pl}
     \\
\glt `Mika stayed at our place for two hours yesterday.’
\z

\noindent One way to account for the diversity of readings IVs are associated with is to assume that they are unspecified for telicity, rather than atelic (as  in \citealt[]{Łazorczyk2010}). In other words, what is traditionally referred to as an imperfective verb is just a verbalized structure unspecified for both telicity and grammatical aspect. This stance is similar in spirit to the proposal of \citet[]{Arsenijević2018} according to which Slavic IVs are unmarked for grammatical aspect, i.e. ambiguous between imperfective and perfective aspect. In this paper, I focus on simple forms, but the analysis can be extended to secondary imperfectives straightforwardly once secondary imperfectivizing suffixes are analyzed as re-verbalizing morphemes (\citealt{Arsenijević2018}), i.e. sequences of theme vowels (\citealt{SimonovićArsMil2021}).

\section{The sequence of similar events interpretation} \label{mil:sec:SSE}

Having removed the obstacle presented by the view that IVs are incompatible with telicity, we are in a position to revisit the claim that the SSE interpretation is not available in Slavic. Examples with both simple (\ref{mil:ex:gubiti-partije-šaha}--\ref{mil:ex:praviti-torte}) and prefixed verbs \REF{mil:ex:ispijati-limenke-piva} show that the SSE interpretation can arise in SC as well. Just as in the case described in \citet{MacDonald_2008}, in all such examples, there is an indefinite number of telic events iterated within the time interval specified by DurAds. Actually, using IVs is the only available way to express the SSE interpretation in SC in the presence of DurAds, since perfective forms cannot be combined with DurAds in such contexts.\footnote{An anonymous reviewer suggests that the SSE reading with simple IVs is only marginally acceptable in Russian (i.e. possibly admissible in some contexts), and that it slightly improves when a secondary imperfective is used. The reviewer points out that a more natural way to express the SSE interpretation in Russian is when the argument introduces distributivity and not just plurality, as in \REF{mil:ex:R1-Russian-distributivity}. (The progressive form as a translation of the IV \textit{čitat} `read' is provided by the reviewer. \citealt{MacDonald_2008} consistently uses simple past forms for such readings in English.) 
\ea \label{mil:ex:R1-Russian-distributivity}
\gll Nedelju čital\textsuperscript{I} po vypusku za čas.
\\ week.\textsc{acc.sg} read.\textsc{ptcp.m} on issue.\textsc{dat.sg} in\textsubscript{za} hour.\textsc{acc.sg}
     \\
\glt `For a week I was reading issues in an hour.'
\z 
Crucially for the purposes of the present paper, these examples once again show that there is no ban on using simple IVs to express telic predicates. This further suggests that the differences in the degree of acceptability of bare plurals with the SSE interpretation between SC and Russian (and possibly other Slavic languages) are not to be sought in the impossibility of IVs to express telic predicates, as analyzed in \citet{MacDonald_2008}.
} 

\ea \label{mil:ex:gubiti-partije-šaha}
\gll Kandidat je dva sata gubio\textsuperscript{I} \minsp{[} partije šaha] od velemajstora za manje od dva minuta.
\\ candidate \textsc{aux} two hour.\textsc{pcl} lose.\textsc{tv.ptcp.m} {} game.\textsc{acc.pl} chess.\textsc{gen.sg} from grandmaster.\textsc{gen.sg} in\textsubscript{za} less than two minute.\textsc{pcl}
     \\
\glt `The candidate lost [chess games] to the grandmaster in less than two minutes for two hours.'
\ex \label{mil:ex:praviti-torte}
\gll Ana je za Luninu svadbu ceo dan pravila\textsuperscript{I} torte za manje od pola sata.
\\ Ana \textsc{aux} for Luna.\textsc{poss} wedding.\textsc{acc.sg} whole day.\textsc{acc.sg} make.\textsc{tv.ptcp.f} cake.\textsc{acc.pl} in\textsubscript{za} less than half hour.\textsc{gen.sg}
     \\
\glt ‘For Luna’s wedding, Ana made cakes in less than half an hour the whole day.’
\ex\label{mil:ex:ispijati-limenke-piva}
\gll Pera	je	dva	minuta		iz-pi-ja-o\textsuperscript{I}	limenke	piva  za	deset	sekundi.
\\ Pera		\textsc{aux}	two	minute.\textsc{pcl}	out-drink-\textsc{si-ptcp.m}	can.\textsc{acc.pl}		beer.\textsc{gen.sg}  in\textsubscript{za}	ten	second.\textsc{gen.pl}
     \\
\glt  ‘Pera drank cans of beer in ten seconds for two minutes.’
\z

\noindent\largerpage{} Before moving on to the exact analysis of the SSE interpretation in SC, a few clarification points are in order. The availability of the SSE interpretation in examples like (\ref{mil:ex:gubiti-partije-šaha}--\ref{mil:ex:ispijati-limenke-piva}) does not mean that other interpretations of IVs with BPs are impossible. For instance, there are at least three possible interpretations of example \REF{mil:ex:SSE-intro-repetead}: (i) the SSE interpretation, with the distributive interpretation of the BP (one song per event); (ii) the iterative interpretation in which Mika recites a set of songs repeatedly, but each set is different; (iii) the iterative interpretation in which Mika recites the same set of songs repeatedly. In both (ii) and (iii) the BP is interpreted collectively. (Later in this section, we will see how the difference between the distributive and the collective interpretation of the BP reflects its different syntactic status.) The third type of interpretation is an instance of the multiple event interpretation which \citet[41]{MacDonald_2008} labels the “sequence of identical events interpretation” (= SIE interpretation), since the same object is implicated in each of the iterated subevents (i.e. the BP is interpreted specifically\slash definitely). The SIE interpretation is also available with singular specific objects, as in \REF{mil:ex:SIE-SC-1-nesto}. 

\ea \label{mil:ex:SSE-intro-repetead}
\gll Mika   je  recitovao   pesme sat vremena. \\ Mika \textsc{aux} recite.\textsc{tv.ptcp.m} song.\textsc{acc.pl} hour.\textsc{acc.sg} time.\textsc{gen.sg}  \\
\glt `Mika recited songs for an hour.'
\ex \label{mil:ex:SIE-SC-1-nesto}
\gll Mika   je  recitovao   pesmu   sat vremena. \\
     Mika \textsc{aux} recite.\textsc{tv.ptcp.m} song.\textsc{acc.sg} hour.\textsc{acc.sg} time.\textsc{gen.sg.} \\ 
\glt `Mika recited a song for an hour.'
\z

\noindent Finally, outside of multiple event interpretations discussed above, BPs in SC, just as in English, may receive a vague denotation which \citet[46, 147]{MacDonald_2008} refers to as a M(ass)N(oun) interpretation. This is illustrated by the SC example \REF{mil:ex:BPs-massInterpretation} similar to those discussed for English in \citet[46]{MacDonald_2008}. Under the MN interpretation of \REF{mil:ex:BPs-massInterpretation}, it does not have to be the case that Mika made multiple dragons -- actually, \REF{mil:ex:BPs-massInterpretation} would still be true if he worked on making only one dragon without ever finishing it. As stated in \citet[46]{MacDonald_2008}, predicates in examples like \REF{mil:ex:BPs-massInterpretation} are interpreted as activities, with a taste of a habitual interpretation.

\ea \label{mil:ex:BPs-massInterpretation}
\gll Mika je u slobodno vreme pravio papirne zmajeve.
\\ Mika \textsc{aux} in free time.\textsc{acc.sg} make.\textsc{tv.ptcp.m} paper.\textsc{poss} dragon.\textsc{acc.pl} 
     \\
\glt  ‘Mika made paper dragons (in his free time).’
\z


\subsection{The SSE interpretation and plural telic predicates} \label{mil:sec:SSE-interpretation}\largerpage

In this subsection, I propose an analysis of telicity in SC as computed on the basis of the quantity properties of internal arguments, which straightforwardly captures the possibility to get the SSE interpretation with simple IVs, i.e. in the absence of prefixes. For the sake of simplicity, I focus on the derivation from the point at which the \textit{v}P is instantiated. I use the \textit{v}P as a verbalizing projection (i.e. devoid of external arguments, cf. \citealt{Harley2013}), assuming that theme vowels in Slavic are verbalizers (with \citealt{Svenonius2004a, Biskup2019, Kovačevićetal2021, MilosavljevićArsenijević2021}). I will primarily use examples with measuring-out direct objects -- traditional incremental themes, since these are the most typical cases where the aspectual role of internal arguments can be observed, and they are the main kind of examples used by \citet{MacDonald_2008} to illustrate the SSE interpretation in English.\largerpage

When the verb merges with an incremental theme object equipped with the [$+$q] feature (in the sense of \citealt{Verkuyl_1972, MacDonald_2008}), the projected \textit{v}P is culminative, i.e. it denotes a culminative predicate, as in Figure \ref{mil:ex:VP-culminative}. Otherwise, the \textit{v}P is non-culminative, see Figure \ref{mil:ex:VP-non-culminative}. Examples of culminative predicates include \textit{praviti tortu} `make a cake', \textit{gubiti meč} `lose the match', \textit{peti se na brdo} `climb the hill', whereas non-culminative \textit{v}Ps are those without a bounded internal argument, e.g. \textit{jesti šećer} `eat sugar' (with an object interpreted as mass), or typical intransitive activities such as \textit{trčati} `run', \textit{spavati} `sleep'. Many of non-culminative predicates can easily be turned into culminative ones, providing the [$+$q] internal argument is composed with a given verb, e.g. \textit{trčati maraton} `run a marathon' or \textit{spavati popodnevnu dremku} `sleep an afternoon nap'.\footnote{Culminativity in this sense is close in spirit to telicity at the level of \textit{v}P (as the locus of the telic description) in \citet{Arsenijević2006book}, the lexical aspect in the sense of \citet{Rothstein2016}, or completability in \citet{Janda2011}.}


\begin{figure} 
\caption{Culminative \textit{v}P}
\label{mil:ex:VP-culminative}
\begin{forest}
[\textit{v}P\textsubscript{[$+$Cul]}
[NP\textsubscript{[$+$q]}]
    [\textit{v}$'$
    [\textit{v}]
    ]
]
\end{forest} 
\end{figure}


%\ea \label{mil:ex:VP-culminative}
%\begin{forest} 
%[vP\textsubscript{[+Cul]}
%[NP\textsubscript{[+q]}]
 %   [v'
  %  [v]
   % ]
%]
%\end{forest} 
%\z

\begin{figure} 
\begin{floatrow}
\captionsetup{margin=.05\linewidth}
\ffigbox
{\begin{forest}
[\textit{v}P\textsubscript{[$-$Cul]}
[(NP\textsubscript{[$-$q]})]
    [\textit{v}$'$
    [\textit{v}]
    ]
]
\end{forest}}
{\caption{Non-culminative \textit{v}P}\label{mil:ex:VP-non-culminative}}%
%\ea \label{mil:ex:QP-tree}
\ffigbox{\begin{forest} 
[QP
    [NP\textsubscript{[$+$q]}, name=target]
    [Q$'$
        [Q]
            [\textit{v}P
            [t\textsubscript{NP\textsubscript{[$+$q]}}, name=source]
                [\textit{v}$'$
                [\textit{v}]
                ]
            ]
    ]
]
\draw[->] (source) to[out=south west,in=south] (target);
\end{forest}}
{\caption{QP}\label{mil:fig:QP-tree}}
\end{floatrow}
\end{figure}

Culminative \textit{v}Ps give rise to telicity (i.e. the projection of the QP) by default. This is achieved by the movement of the accusative object from its base-gen\-er\-ated position (Spec\textit{v}P) to the specifier position of the QP, where it checks the [$+$q] feature (in the sense of \citealt{Pereltsvaig1999, Pereltsvaig2000,}, see also \citealt{Travis2005}), as illustrated in \figref{mil:fig:QP-tree}. (Culminative predicates fail to trigger the projection of QP in progressive contexts, as briefly discussed in \sectref{mil:sec:CulminativePredicatesDurads}.)



Once the QP is projected, the derivation can proceed in two ways, both of which lead to the projection of the Num(eral)P, a phrase responsible for number in the verbal domain: the QP composes with the plural operator, yielding a plural telic predicate, or it composes with the prefix, giving rise to a singular telic predicate. The former option is how the SSE interpretation arises, and it is addressed in detail in the remainder of this subsection. The singular telicity is briefly analyzed in \sectref{mil:sec:CulminativePredicatesPrefixes}, since it sheds light on the overall system of the computation of telicity in Slavic. 

The structure of the plural telic predicate is shown in \figref{mil:fig:NumP-plural-tree}. Here I build on the insights of \citet{vanGeenhoven2004, vanGeenhoven2005} and \citet{Arsenijević2006eah}, who propose that distributive multiple event interpretations (referred to as SSE and SIE interpretations in this paper, following \citealt{MacDonald_2008}) are instances of verbal plurality, or (silent) pluractionality, which is a verbal counterpart of nominal plurality.\footnote{For related ideas, see also \citet{Landman2000, Rothstein2004, Rothstein2008a}, and references therein.} Although many languages, including English and SC, do not make use of the overt plural marking directly on the verb, there are languages with such a morphological makeup, e.g. West Greenlandic, discussed in \citet{vanGeenhoven2004}. 

%\ea \label{mil:ex:NumP-plural-tree}
\begin{figure}
    \centering
\begin{forest} 
[NumP
[Num$'$
[Num\textsubscript{[$+$pl]}]
[QP
    [NP\textsubscript{[$+$q]}, name=target]
    [Q$'$
        [Q]
            [\textit{v}P
            [t\textsubscript{NP\textsubscript{[$+$q]}}, name=source]
                [\textit{v}$'$
                [\textit{v}]
                ]
            ]
    ]]
    ]
]
\draw[->] (source) to[out=south west,in=south] (target);
\end{forest} 
\caption{Plural NumP}
    \label{mil:fig:NumP-plural-tree}
\end{figure}

I adopt \citeposst{Arsenijević2006eah} analysis according to which in the case of the SSE interpretation the plural gets lexicalized on the noun. The relation between the plurality head and the plural marking on the noun is established via a binding relation. This is possible because the object NP, being unspecific, does not esta\-blish its referential properties outside the eventuality it is bound by, including the number specification (see \citealt{Arsenijević2006eah} for technical details).

Under this approach, the plurality is responsible for the homogenizing effects, enabling DurAds to combine with such a predicate. As also pointed out by \citet[142--143]{vanGeenhoven2004}, plural (pluractional in her terminology) predicates are like mass nouns (i.e. cumulative and divisive), which makes them unbounded, i.e. non-homogeneous/atelic. 

There are several advantages of the proposed analysis of the SSE interpretation. Let me start by comparing \citeauthor{MacDonald_2008}'s and the approach proposed here. According to \citet[50]{MacDonald_2008}, BPs (i) must be [$+$q] in order to trigger telicity, and they (ii) introduce an indefinite number of objects, while DurAds (i) combine with a telic predicate, and (ii) contribute an indefinite number of repetitions of the telic event since they force the event to continue for the amount of time they specify. This division of labor between BPs and DurAds in contributing the SSE interpretation implies that this type of multiple event interpretation is not available in the absence of DurAds, contrary to the fact: DurAds only make it more prominent, i.e. pragmatically salient. In my approach, just as in MacDonald's, the internal argument contributes the [$+$q] feature, but it is the plural in the verbal domain that is responsible for the multiple events interpretation, bringing about the homogeneity effects in this way. DurAds then provide the time interval within which these multiple events occur. While I remain agnostic with respect to the exact way DurAds should be represented in this case,\footnote{A plausible candidate would be an aspectual projection responsible for repetitivity immediately above the NumP, i.e. the AspP\textsubscript{repetitive} in the sense of \citet[]{Cinque1999}.} the crucial point is that they do not compose with a telic predicate, rather -- they enter the derivation once the plural homogeneous predicate has been formed. Consequently, my proposal preserves the standard analyses of both TSAds (which modify telic predicates) and DurAds (which compose with atelic predicates). In addition, the proposal preserves the view that a bounded internal argument contributes the [$+$q] feature and that the bare plural makes a predicate homogeneous, with the difference that in this case the plurality applies directly in the verbal domain.

The proposed analysis straightforwardly captures the difference between the distributive and collective interpretation of BPs in contexts sketched in \REF{mil:ex:SSE-intro-repetead} above: they are instances of the event plurality and the object plurality, respectively. Namely, under the collective interpretation, the plural is interpreted on the noun, and the plurality operator scopes over it, which delivers interpretations according to which multiple objects are affected within every counting unit of a plural event. In addition, we will see in \sectref{mil:sec:CulminativePredicatesPrefixes} that only BPs which reflect the NP plurality can be used in the scope of prefixes -- just as expected if prefixes, as assigners of singularity, are in complementary distribution with plural operators.

An anonymous reviewer raises the question of how the proposed analysis of the verb plurality as lexicalized on the noun under the SSE interpretation captures the fact that there are plurality interpretations dissociated from plural morphology on the noun, e.g. the SIE interpretation in the sense of \citet[41]{MacDonald_2008}; recall that this is a multiple events interpretation in which the same object is implicated in each of the iterated subevents, illustrated in \REF{mil:ex:SIE-SC-1} from SC.

\ea \label{mil:ex:SIE-SC-1}
\gll Mika   je  recitovao   pesmu   sat vremena \minsp{(} za pet minuta). \\
     Mika \textsc{aux} recite.\textsc{tv.ptcp.m} song.\textsc{acc.sg} hour.\textsc{acc.sg} time.\textsc{gen.sg} {} in\textsubscript{za} five minute.\textsc{gen.pl} \\
\glt `Mika recited a song for an hour (in five minutes).'
\z

\noindent I propose that in this case the aspectual composition proceeds in the same way as under the SSE interpretation: telicity is triggered by the specified quantity brought about by the internal argument, and the telic \textit{v}P (= QP) is then pluralized by the (covert) plural operator. Unlike in the case of SSE interpretation, in the SIE contexts the plural fails to be lexicalized on the noun since in this case the object NP is specific, i.e. it establishes referential properties independently of the eventuality, including its own number specification (cf. \citealt{Arsenijević2006eah}). 

\subsection{Culminative \textit{v}Ps and ``failed'' telicity} \label{mil:sec:CulminativePredicatesDurads}

The default pattern sketched in \figref{mil:fig:QP-tree} -- culminative \textit{v}Ps yielding telicity -- fails to be established only if the progressive-like kind of operator intervenes, yielding a stative interpretation in the sense of \citet{Ramchand2018} (see also \citealt{Parsons1990}). \citet[58--59]{Ramchand2018} proposes an \textit{ing}P projection above the \textit{v}P, still within the first phase (i.e. within the domain of event description) for English progressive constructions, thus moving away from the standard analyses of the progressive as an instantiation of grammatical aspect (see also \citealt{RamchandSvenonius2014, Ramchand2017}). In analogy with this proposal, examples with culminative \textit{v}Ps that have the interpretation analogous to the English progressive (as in \REF{mil:ex:progressive-Sneško}) can be accounted for by assuming a (null) progressive operator immediately above the \textit{v}P, as in \figref{mil:fig:StateP-tree}, which blocks the projection of the QP. 

\ea \label{mil:ex:progressive-Sneško}
\gll Maja je juče dva sata \minsp{(*} za dva sata) pravila\textsuperscript{I} sneška, kad je sneg odjednom počeo\textsuperscript{P} da se topi\textsuperscript{I} i prekinuo\textsuperscript{P} njen poduhvat. 
\\ Maja \textsc{aux} yesterday two hour.\textsc{pcl} {} in\textsubscript{za} two hour.\textsc{pcl} make.\textsc{tv.ptcp.f} snowman.\textsc{acc.sg} when \textsc{aux} snow suddenly begin.\textsc{tv.ptcp.m} \textsc{comp} \textsc{refl} melt.\textsc{prs.3sg} and interrupt.\textsc{tv.ptcp.m} her endeavor.\textsc{acc.sg}
     \\
\glt  ‘Yesterday, Maja had been making a snowman for two hours when the snow suddenly began to melt and interrupted her endeavor.’
\z

\noindent Hence, \textit{v}Ps like \textit{praviti sneška} ‘make a snowman’ in the progressive contexts are culminative, but they are not telic, since the projection of the QP fails. I assume with \citet[58]{Ramchand2018} that for every event description P, the progressive (operator) introduces an Identifying State as ``a stative eventuality that manifests sufficient cognitive/perceptual identifiers of the event property P'', which is why I label such a projection StateP in \figref{mil:fig:StateP-tree}. The proposed view straightforwardly explains why culminative predicates in SC in examples like \REF{mil:ex:progressive-Sneško} can be used with DurAds, but cannot be modified by TSAds: TSAds require the projection of the QP, which fails in this case. DurAds, on the other hand, are felicitous, since in progressive contexts they can be analyzed as scoping over the progressive operator, modifying the Identifying State of a snowman building event, as also pointed out by an anonymous reviewer.

%\ea \label{mil:ex:StateP-tree}
\begin{figure}[t]
    \centering
\begin{forest} 
[StateP
[State$'$
        [State\textsubscript{[Prog]}]
            [\textit{v}P
            [NP\textsubscript{[$+$q]}, name=source]
                [\textit{v}$'$
                [\textit{v}]
                ]
            ]
        ]
]
\end{forest}
\caption{StateP}
    \label{mil:fig:StateP-tree}
\end{figure}

\section{Broadening the picture: Singular telic predicates} \label{mil:sec:Residual-Issues}

The analysis presented in \sectref{mil:sec:SSE} enables accounting for telicity in Slavic and Germanic languages in a unified way: telicity \textit{can} be triggered by the properties of internal arguments. In other words, it is not the case that in Germanic languages properties of internal arguments are crucial in computing telicity, whereas in Slavic they have no effect whatsoever, as standardly assumed (see e.g. \citealt{MacDonald_2008, Łazorczyk2010, Rothstein2016}). It should be emphasized, however, that the proposed analysis does not imply that internal arguments with a specified quantity are the only way to assign telicity: e.g. it can be triggered by some measure adverbials (cf. e.g. \citealt{Pereltsvaig2000} for Russian, \citealt{Milosavljević2022} for SC). This again is similar with what we find in Germanic languages, where various types of adverbials can trigger the projection of QP (see e.g. \citealt{Borer_2005}). However, the role of internal arguments in affecting telicity in SC described in the previous section was constrained only to plural contexts, which, at first glance, contrasts with the state of affairs we find in English.\footnote{I assume that other syntactic contexts in which IVs are used in telic environments (e.g. habitual and general-factual uses) include a (potential) repetition of the same (telic) event type/kind (see \citealt[]{Milosavljević2019}), hence they are also based on the plurality of telic \textit{v}Ps (but see \citetv{chapters/01-Arsenijevic} for a different view). However, the exact analysis of these cases goes beyond the scope of the present paper. For a unified treatment of habitual and general-factual readings of imperfectives in Russian, see \citet{Minor2019}. For accounts of the general-factual meaning that employ the notion of event kind, see \citet[]{Mehlig2013, MuellerReichau2013, Mueller-Reichau2015}.} I propose that internal arguments retain their role in aspectual composition in Slavic in singular contexts as well. This is achieved by analyzing Slavic prefixes as scoping over the QP triggered by internal arguments, as proposed in \sectref{mil:sec:CulminativePredicatesPrefixes}. Another context where singular telicity emerges in the absence of prefixes productively is with biaspectual verbs, which will be briefly discussed in \sectref{mil:sec:BiaspectualVerbs}.

\subsection{Prefixes and singular telic predicates} \label{mil:sec:CulminativePredicatesPrefixes}\largerpage

Prefixless incremental theme verbs discussed in previous subsections usually have prefixed variants, and such pairs are typically referred to as aspectual pairs, which have the same meaning and differ only with respect to the aspectual value. Some aspectual pairs from SC are provided in \REF{mil:ex:aspectual-pairs}. 

\ea \label{mil:ex:aspectual-pairs}
\ea[]{
\gll graditi\textsuperscript{I} kuću /  sa-graditi\textsuperscript{P} kuću\\
build.\textsc{tv.inf} house.\textsc{acc.sg} {} with-build.\textsc{tv.inf} house.\textsc{acc.sg}\\
\glt `build a house'
}
\ex[]{
\gll praviti\textsuperscript{I} tortu /  na-praviti\textsuperscript{P} tortu\\
make.\textsc{tv.inf} cake.\textsc{acc.sg} {} on-make.\textsc{tv.inf} cake.\textsc{acc.sg}\\
\glt `make a cake'
}
\ex[]{
\gll gubiti\textsuperscript{I} meč /  iz-gubiti\textsuperscript{P} meč\\
lose.\textsc{tv.inf} match.\textsc{acc.sg} {} out-lose.\textsc{tv.inf} match.\textsc{acc.sg}\\
\glt `lose a match'
}
\ex[]{
\gll čitati\textsuperscript{I} knjigu /  pro-čitati\textsuperscript{P} knjigu\\
read.\textsc{tv.inf} book.\textsc{acc.sg} {} through-read.\textsc{tv.inf} book.\textsc{acc.sg}\\
\glt `read a book'}
\z
\z

\noindent These prefixes are often labeled as purely perfectivizing prefixes (PPPs) and are typically analyzed as semantically empty.\footnote{I use the term PPPs descriptively here -- it does not necessarily mean that these prefixes are devoid of meaning; for detailed semantic analyses of prefixes traditionally claimed to be semantically empty, see e.g.  \citet{EndresenSokolova2012, JandaLyashevskaya2013, Miljković2021}.} In this subsection, I propose that PPPs compose with telic predicates, and that they are specifiers of the projection responsible for number in the verbal domain, where they specify a telic verbal predicate for singularity (via specifier-head agreement in the sense of \citealt{Borer_2005}), as shown in \figref{mil:fig:NumP-tree}. I opt for an analysis of prefixes as specifiers rather than heads building on \citeposst{Milosavljević_in_prep} proposal that the semelfactive suffix \textit{-nu} is an exponent of the head of this projection, and the two morphemes can be combined (e.g. \textit{od-gur-nu-ti} [\textsc{pref}-push-\textsc{sem}-\textsc{inf}] `push away').\footnote{See \citet[]{Svenonius2008} for additional arguments in favor of the analysis of prefixes as specifiers.}


%\ea \label{mil:ex:NumP-tree}
\begin{figure}
    \centering
\begin{forest} 
[NumP
[Prefix]
[Num$'$
[Num\textsubscript{[$+$sg]}]
[QP
    [NP\textsubscript{[$+$q]}, name=target]
    [Q$'$
        [Q]
            [\textit{v}P
            [t\textsubscript{NP\textsubscript{[$+$q]}}, name=source]
                [\textit{v}$'$
                [\textit{v}]
                ]
            ]
    ]]
]
]
\draw[->] (source) to[out=south west,in=south] (target);
\end{forest} 
\caption{Singular NumP}
    \label{mil:fig:NumP-tree}
\end{figure}

Let me situate this proposal against some common analyses in the literature.
As is well known, the object of PVs gets an obligatorily bounded interpretation. On the common view, such an interpretation is usually analyzed as brought about either by the prefix or the perfective aspect, a process inverse to what we see in English: instead of the object determining the interpretation of the verbal predicate, the verbal predicate determines the properties of the object (see \citealt[]{Szucsich2001, Szucsich2002, Łazorczyk2010, MacDonald_2008, Rothstein2016}). IVs, on the other hand, do not impose restrictions on the interpretation of the direct object, i.e. it may be both unbounded and bounded and can be optional with same verbs, e.g. \textit{pisati (pismo)} `to write (a letter)' or \textit{čitati (knjigu)} `to read (a book)'. To account for this difference in the status of objects of PVs and IVs, \citet[]{Basilico2008}, for instance, proposes that they are introduced by different heads at different points in the syntactic derivation: the direct object of PVs is introduced by the (affixed) Root, while the direct object of IVs is introduced by the \textit{v} categorizing head. 

My approach to prefixation is closer to an alternative view, suggested in \citet[50]{Krifka1992} and \citet[102]{Verkuyl1999}. For these authors, prefixes, as perfective operators, require the \textit{v}P they combine with to be quantized/terminative (which is possible only if the object NP is bounded). According to \citet[126--127]{Verkuyl1999}, until the \textsc{asp}-node, which hosts a prefix, merges, the derivation of the verb has not yet been completed, and the bounded object, though necessary, is not itself sufficient to bring about the terminative/bounded \textit{v}P. Only after the perfective prefix is added, the perfective terminative (= telic) \textit{v}P arises. Hence, in this approach, although the prefix merges with a terminative/quantized/telic \textit{v}P, such a \textit{v}P is always realized only in perfective contexts, after the prefix has been merged.

The view according to which prefixes scope over bounded/telic predicates has several advantages. First, it recognizes the role of internal arguments in affecting telicity in both English and Slavic, without a need for specifying the inverse operation for the latter group of languages. Second, the object NP of IVs and PVs need not to be analyzed as generated in different ways (as in \citealt{Basilico2008}), since, as we have seen, its obligatory nature with PVs follows from the fact that the prefix picks out the \textit{v}P with a bounded NP object. In this way, PVs are actually aspectual counterparts of IVs with a bounded object.\footnote{E.g. it is not the case that the verbs \textit{pisati}\textsuperscript{I} and \textit{na-pisati}\textsuperscript{P} `to write' are themselves aspectual pairs, rather \textit{na-pisati}\textsuperscript{P} $+$ NP\textsubscript{[$+$q]} is a counterpart of \textit{pisati}\textsuperscript{I} $+$ NP\textsubscript{[$+$q]}.} Finally, if the QP has its telic aspectual status independently prior to merging with the prefix, we expect to find it in some other syntactic contexts as well. The SSE interpretation, analyzed in \sectref{mil:sec:SSE}, provides exactly the kind of context that employs the QP divorced from prefixes. Hence, while I share with \citet[]{Krifka1992} and \citet[]{Verkuyl1999} the view that prefixes scope above complex (telic) \textit{v}Ps, in my approach prefixation is not the only syntactic context that enables telic predicates to show up. 
Prefixes are specifiers of singularity, and as such they are in complementary distribution with plural telic predicates presented in \sectref{mil:sec:SSE}. For instance, BPs with prefixes in SC cannot give rise to the SSE interpretation, rather -- they always receive a collective interpretation. This is expected if the BP giving rise to the SSE interpretation reflects the plurality of events, while under the collective interpretation it reflects the plurality of objects. As expected, in the latter case the prefix is able to compose with a predicate whose object is expressed by a BP when the BP is bounded (which is usually contextually provided), as in \REF{mil:ex:BPs-collective}.

\ea \label{mil:ex:BPs-collective}
{
\gll Pera je na-pravio\textsuperscript{P} torte. \\ Pera aux \textsc{pref}-make.\textsc{tv.ptcp.m} cake.\textsc{acc.pl}
\\
\glt `Pera made the cakes.'}
\z

\noindent Except for their complementary distribution with plural predicates, I prefer the analysis of prefixes as markers of singularity rather than markers of perfectivity, as in \citet{Krifka1992} and \citet{Verkuyl1999} (see also \citealt{Slabakova2005}), because the prefix does not guarantee perfectivity. Namely, in many cases, the prefixed QP can undergo secondary imperfectivization (and the prefixed QP is realized as perfective only upon the inclusion of the reference time). Moreover, the view of prefixes as singulative morphemes also accords well with some recent approaches to prefixes as (morphemes of the same kind as) numeral classifiers (see \citealt{DickeyJanda2015}).\footnote{In this section, I focused on PPPs with incremental theme verbs. In  \citet[]{Milosavljević2022, Milosavljević_in_prep}, I argue that Slavic prefixes generally compose with telic predicates. In short, just as internal arguments are not the only way to trigger telicity, the proposal that prefixes combine with telic predicates does not mean that they must combine with telic predicates whose telicity is triggered by internal arguments. For instance, in \citet[]{Milosavljević2022, Milosavljević_in_prep} an analysis of the delimitative prefix \textit{po-} in Slavic is proposed according to which this prefix combines with the QP triggered by DurAds or some contextually provided quantity.}

\subsection{Biaspectual verbs and telicity} \label{mil:sec:BiaspectualVerbs}

Biaspectual verbs (BVs) are traditionally analyzed as verbs that can be either perfective or imperfective, depending on the syntactic context (see \citealt{Janda2007What, Kolaković2018, Zinova2021, Stary2017}, a.o.). In terms of the system presented in this paper, BVs can be used in both singular and plural telic environments, as in \REF{mil:ex:BV-singular} and \REF{mil:ex:BV-plural} from SC. Since they are simple, i.e. unprefixed forms, BVs can be taken as additional evidence that telicity in Slavic can emerge in the absence of prefixes. Some extensive corpus-based studies show that BVs are based on culminative \textit{v}Ps (see \citealt{Grickat1957/8, Janda2007What, Kolaković2018}), which also supports the view that telicity is based on culminativity, which is in turn based on the contribution of internal arguments, as proposed in \sectref{mil:sec:SSE}. 

\ea \label{mil:ex:BV-singular}
{
\gll Pera je malopre downlodovao film za 15 minuta. \\ 
     Pera \textsc{aux} {just.now} download.\textsc{tv.ptcp.m} movie\textsc{acc.sg} in\textsubscript{za} 15 minute.\textsc{gen.pl}
\\
\glt `Pera just downloaded a movie in 15 minutes.'}
\z

\ea \label{mil:ex:BV-plural}
{
\gll Pera je ceo dan downlodovao filmove za 15 minuta. \\ Pera \textsc{aux} whole day.\textsc{acc.sg} download.\textsc{tv.ptcp.m} movie.\textsc{acc.pl} in\textsubscript{za} 15 minute.\textsc{gen.pl}
\\
\glt `Pera downloaded movies in 15 minutes the whole day.'}
\z

\noindent While the plural telicity emerges when the QP is combined with the plural operator, it remains an open question how the singular reading emerges in the absence of prefixes (or the semelfactive suffix). A possible solution is to assume that singularity is triggered by a variable-like anaphoric element -- following the argumentation in \citet{Stanley2000, StanleySzabo2000}, a.o., that all effects of extra-linguistic context on the truth-condition are represented at LF.\footnote{An alternative option would be to assume a null prefix to account for singular telic uses or ``perfective'' uses of bi-aspectuals, as suggested in \citet[]{Grickat1957/8, Grickat1966/7, Łazorczyk2010}.}

\section{Conclusion} \label{mil:sec:Conclusion}

In this paper, I examined the so-called sequence of similar events interpretation in Serbo-Croatian, which emerges in the presence of bare plural objects when a telic predicate is expressed by an imperfective verb. I showed that this is an interpretation that, just as in English, allows the use of both durative adverbials and time-span adverbials at the same time. I proposed that, as standardly assumed, TSAds modify a telic event predicate, while DurAds in such cases merge once the predicate has been made homogeneous/atelic by the plural operator (contra \citeauthor{MacDonald_2008}'s \citeyear{MacDonald_2008} claim that DurAds combine with telic predicates in such cases). The fact that the SSE interpretation is possible in Serbo-Croatian (and at least some other Slavic languages), and is realized by employing imperfective verbs -- including simple ones (i.e. those without prefixes) -- suggests that in Slavic there is a syntactic domain responsible for telicity analogous to that in English (contra \citealt{MacDonald_2008}).

\section*{Abbreviations}
\begin{multicols}{2}
\begin{tabbing}
\textsc{DurAds}\hspace{.5em}\= durative adverbials \kill
\textsc{acc} \> accusative \\
\textsc{aux} \> auxiliary \\
\textsc{BP} \> bare plural \\
\textsc{BV} \> biaspectual verb \\
\textsc{comp} \>  complementizer \\
\textsc{dat} \> dative \\
\textsc{del} \> delimitative (prefix) \\
\textsc{DurAds} \> durative adverbials \\
\textsc{gen} \>  genitive \\
\textsc{f} \> feminine \\
\textsc{IV} \> imperfective verb \\
\textsc{loc} \> locative \\
\textsc{m} \> masculine \\
\textsc{OTEM} \> object-to-event mapping \\
\textsc{pl} \> plural \\
\textsc{pcl} \> paucal \\
\textsc{poss} \> possesive \\
\textsc{ptcp} \> participle \\
\textsc{pref} \> prefix \\
\textsc{PV} \> perfective verb \\
\textsc{refl} \>  reflexive \\
\textsc{SC} \> Serbo-Croatian \\
\textsc{sg} \> singular \\
\textsc{sem} \> semelfactive \\
\textsc{si} \> secondary imperfectivizing \\ \> (suffix) \\
\textsc{SIE} \> sequence of identical events \\
\textsc{SSE} \> sequence of similar events \\
\textsc{TSAds} \> time-span adverbials \\
\textsc{tv} \> theme vowel \\
\end{tabbing}
\end{multicols}

\section*{Acknowledgments}
The paper is a result of the research conducted within the project \textit{Hyperspacing the Verb: The interplay between prosody, morphology and semantics in the Western South Slavic verbal domain}, financed by the Austrian Science Fund FWF (grant I4215). I would like to thank Boban Arsenijević for valuable discussions of various versions of this paper. I also thank the two anonymous reviewers, as well as the editors of the volume for useful suggestions and comments. Needless to say, all remaining errors are my own.

\printbibliography[heading=subbibliography,notkeyword=this]

\end{document}
