\documentclass[output=paper]{langscibook}
\ChapterDOI{10.5281/zenodo.10123631}


\author{Petr Biskup\affiliation{Universität Leipzig}\orcid{0000-0003-2602-8856}}


\title{Aspect separated from aspectual markers in Russian and Czech}

\abstract{This article is concerned with the derivation of morphological aspect in Russian and Czech. It investigates four aspectual markers: prefixes, the secondary imperfective suffix, the semelfactive marker, and the habitual suffix. It argues that not only in Russian (see \citealt{Tatevosov2011,Tatevosov2015Slavic}) but also in Czech aspect interpretation is separated from prefixes and the secondary imperfective suffix. Moreover, it extends the separation to the semelfactive suffix and the habitual marker. Specific morphological aspect properties of Russian and Czech predicates are derived by an Agree analysis with minimality based on dominance relations in the complex verbal head.

\keywords{Agree, aspect, prefixes, habitual suffix, secondary imperfective, semelfactive suffix}}

\lsConditionalSetupForPaper{}

\begin{document}
\maketitle

% Just comment out the input below when you're ready to go.
% For a start: Do not forget to give your Overleaf project (this paper) a recognizable name. This one could be called, for instance, Simik et al: OSL template. You can change the name of the project by hovering over the gray title at the top of this page and clicking on the pencil icon.

\section{Introduction}\label{sim:sec:intro}

Language Science Press is a project run for linguists, but also by linguists. You are part of that and we rely on your collaboration to get at the desired result. Publishing with LangSci Press might mean a bit more work for the author (and for the volume editor), esp. for the less experienced ones, but it also gives you much more control of the process and it is rewarding to see the quality result.

Please follow the instructions below closely, it will save the volume editors, the series editors, and you alike a lot of time.

\sloppy This stylesheet is a further specification of three more general sources: (i) the Leipzig glossing rules \citep{leipzig-glossing-rules}, (ii) the generic style rules for linguistics (\url{https://www.eva.mpg.de/fileadmin/content_files/staff/haspelmt/pdf/GenericStyleRules.pdf}), and (iii) the Language Science Press guidelines \citep{Nordhoff.Muller2021}.\footnote{Notice the way in-text numbered lists should be written -- using small Roman numbers enclosed in brackets.} It is advisable to go through these before you start writing. Most of the general rules are not repeated here.\footnote{Do not worry about the colors of references and links. They are there to make the editorial process easier and will disappear prior to official publication.}

Please spend some time reading through these and the more general instructions. Your 30 minutes on this is likely to save you and us hours of additional work. Do not hesitate to contact the editors if you have any questions.

\section{Illustrating OSL commands and conventions}\label{sim:sec:osl-comm}

Below I illustrate the use of a number of commands defined in langsci-osl.tex (see the styles folder).

\subsection{Typesetting semantics}\label{sim:sec:sem}

See below for some examples of how to typeset semantic formulas. The examples also show the use of the sib-command (= ``semantic interpretation brackets''). Notice also the the use of the dummy curly brackets in \REF{sim:ex:quant}. They ensure that the spacing around the equation symbol is correct. 

\ea \ea \sib{dog}$^g=\textsc{dog}=\lambda x[\textsc{dog}(x)]$\label{sim:ex:dog}
\ex \sib{Some dog bit every boy}${}=\exists x[\textsc{dog}(x)\wedge\forall y[\textsc{boy}(y)\rightarrow \textsc{bit}(x,y)]]$\label{sim:ex:quant}
\z\z

\noindent Use noindent after example environments (but not after floats like tables or figures).

And here's a macro for semantic type brackets: The expression \textit{dog} is of type $\stb{e,t}$. Don't forget to place the whole type formula into a math-environment. An example of a more complex type, such as the one of \textit{every}: $\stb{s,\stb{\stb{e,t},\stb{e,t}}}$. You can of course also use the type in a subscript: dog$_{\stb{e,t}}$

We distinguish between metalinguistic constants that are translations of object language, which are typeset using small caps, see \REF{sim:ex:dog}, and logical constants. See the contrast in \REF{sim:ex:speaker}, where \textsc{speaker} (= serif) in \REF{sim:ex:speaker-a} is the denotation of the word \textit{speaker}, and \cnst{speaker} (= sans-serif) in \REF{sim:ex:speaker-b} is the function that maps the context $c$ to the speaker in that context.\footnote{Notice that both types of small caps are automatically turned into text-style, even if used in a math-environment. This enables you to use math throughout.}$^,$\footnote{Notice also that the notation entails the ``direct translation'' system from natural language to metalanguage, as entertained e.g. in \citet{Heim.Kratzer1998}. Feel free to devise your own notation when relying on the ``indirect translation'' system (see, e.g., \citealt{Coppock.Champollion2022}).}

\ea\label{sim:ex:speaker}
\ea \sib{The speaker is drunk}$^{g,c}=\textsc{drunk}\big(\iota x\,\textsc{speaker}(x)\big)$\label{sim:ex:speaker-a}
\ex \sib{I am drunk}$^{g,c}=\textsc{drunk}\big(\cnst{speaker}(c)\big)$\label{sim:ex:speaker-b}
\z\z

\noindent Notice that with more complex formulas, you can use bigger brackets indicating scope, cf. $($ vs. $\big($, as used in \REF{sim:ex:speaker}. Also notice the use of backslash plus comma, which produces additional space in math-environment.

\subsection{Examples and the minsp command}

Try to keep examples simple. But if you need to pack more information into an example or include more alternatives, you can resort to various brackets or slashes. For that, you will find the minsp-command useful. It works as follows:

\ea\label{sim:ex:german-verbs}\gll Hans \minsp{\{} schläft / schlief / \minsp{*} schlafen\}.\\
Hans {} sleeps {} slept {} {} sleep.\textsc{inf}\\
\glt `Hans \{sleeps / slept\}.'
\z

\noindent If you use the command, glosses will be aligned with the corresponding object language elements correctly. Notice also that brackets etc. do not receive their own gloss. Simply use closed curly brackets as the placeholder.

The minsp-command is not needed for grammaticality judgments used for the whole sentence. For that, use the native langsci-gb4e method instead, as illustrated below:

\ea[*]{\gll Das sein ungrammatisch.\\
that be.\textsc{inf} ungrammatical\\
\glt Intended: `This is ungrammatical.'\hfill (German)\label{sim:ex:ungram}}
\z

\noindent Also notice that translations should never be ungrammatical. If the original is ungrammatical, provide the intended interpretation in idiomatic English.

If you want to indicate the language and/or the source of the example, place this on the right margin of the translation line. Schematic information about relevant linguistic properties of the examples should be placed on the line of the example, as indicated below.

\ea\label{sim:ex:bailyn}\gll \minsp{[} Ėtu knigu] čitaet Ivan \minsp{(} často).\\
{} this book.{\ACC} read.{\PRS.3\SG} Ivan.{\NOM} {} often\\\hfill O-V-S-Adv
\glt `Ivan reads this book (often).'\hfill (Russian; \citealt[4]{Bailyn2004})
\z

\noindent Finally, notice that you can use the gloss macros for typing grammatical glosses, defined in langsci-lgr.sty. Place curly brackets around them.

\subsection{Citation commands and macros}

You can make your life easier if you use the following citation commands and macros (see code):

\begin{itemize}
    \item \citealt{Bailyn2004}: no brackets
    \item \citet{Bailyn2004}: year in brackets
    \item \citep{Bailyn2004}: everything in brackets
    \item \citepossalt{Bailyn2004}: possessive
    \item \citeposst{Bailyn2004}: possessive with year in brackets
\end{itemize}

\section{Trees}\label{s:tree}

Use the forest package for trees and place trees in a figure environment. \figref{sim:fig:CP} shows a simple example.\footnote{See \citet{VandenWyngaerd2017} for a simple and useful quickstart guide for the forest package.} Notice that figure (and table) environments are so-called floating environments. {\LaTeX} determines the position of the figure/table on the page, so it can appear elsewhere than where it appears in the code. This is not a bug, it is a property. Also for this reason, do not refer to figures/tables by using phrases like ``the table below''. Always use tabref/figref. If your terminal nodes represent object language, then these should essentially correspond to glosses, not to the original. For this reason, we recommend including an explicit example which corresponds to the tree, in this particular case \REF{sim:ex:czech-for-tree}.

\ea\label{sim:ex:czech-for-tree}\gll Co se řidič snažil dělat?\\
what {\REFL} driver try.{\PTCP.\SG.\MASC} do.{\INF}\\
\glt `What did the driver try to do?'
\z

\begin{figure}[ht]
% the [ht] option means that you prefer the placement of the figure HERE (=h) and if HERE is not possible, you prefer the TOP (=t) of a page
% \centering
    \begin{forest}
    for tree={s sep=1cm, inner sep=0, l=0}
    [CP
        [DP
            [what, roof, name=what]
        ]
        [C$'$
            [C
                [\textsc{refl}]
            ]
            [TP
                [DP
                    [driver, roof]
                ]
                [T$'$
                    [T [{[past]}]]
                    [VP
                        [V
                            [tried]
                        ]
                        [VP, s sep=2.2cm
                            [V
                                [do.\textsc{inf}]
                            ]
                            [t\textsubscript{what}, name=trace-what]
                        ]
                    ]
                ]
            ]
        ]
    ]
    \draw[->,overlay] (trace-what) to[out=south west, in=south, looseness=1.1] (what);
    % the overlay option avoids making the bounding box of the tree too large
    % the looseness option defines the looseness of the arrow (default = 1)
    \end{forest}
    \vspace{3ex} % extra vspace is added here because the arrow goes too deep to the caption; avoid such manual tweaking as much as possible; here it's necessary
    \caption{Proposed syntactic representation of \REF{sim:ex:czech-for-tree}}
    \label{sim:fig:CP}
\end{figure}

Do not use noindent after figures or tables (as you do after examples). Cases like these (where the noindent ends up missing) will be handled by the editors prior to publication.

\section{Italics, boldface, small caps, underlining, quotes}

See \citet{Nordhoff.Muller2021} for that. In short:

\begin{itemize}
    \item No boldface anywhere.
    \item No underlining anywhere (unless for very specific and well-defined technical notation; consult with editors).
    \item Small caps used for (i) introducing terms that are important for the paper (small-cap the term just ones, at a place where it is characterized/defined); (ii) metalinguistic translations of object-language expressions in semantic formulas (see \sectref{sim:sec:sem}); (iii) selected technical notions.
    \item Italics for object-language within text; exceptionally for emphasis/contrast.
    \item Single quotes: for translations/interpretations
    \item Double quotes: scare quotes; quotations of chunks of text.
\end{itemize}

\section{Cross-referencing}

Label examples, sections, tables, figures, possibly footnotes (by using the label macro). The name of the label is up to you, but it is good practice to follow this template: article-code:reference-type:unique-label. E.g. sim:ex:german would be a proper name for a reference within this paper (sim = short for the author(s); ex = example reference; german = unique name of that example).

\section{Syntactic notation}

Syntactic categories (N, D, V, etc.) are written with initial capital letters. This also holds for categories named with multiple letters, e.g. Foc, Top, Num, etc. Stick to this convention also when coming up with ad hoc categories, e.g. Cl (for clitic or classifier).

An exception from this rule are ``little'' categories, which are written with italics: \textit{v}, \textit{n}, \textit{v}P, etc.

Bar-levels must be typeset with bars/primes, not with an apostrophe. An easy way to do that is to use mathmode for the bar: C$'$, Foc$'$, etc.

Specifiers should be written this way: SpecCP, Spec\textit{v}P.

Features should be surrounded by square brackets, e.g., [past]. If you use plus and minus, be sure that these actually are plus and minus, and not e.g. a hyphen. Mathmode can help with that: [$+$sg], [$-$sg], [$\pm$sg]. See \sectref{sim:sec:hyphens-etc} for related information.

\section{Footnotes}

Absolutely avoid long footnotes. A footnote should not be longer than, say, {20\%} of the page. If you feel like you need a long footnote, make an explicit digression in the main body of the text.

Footnotes should always be placed at the end of whole sentences. Formulate the footnote in such a way that this is possible. Footnotes should always go after punctuation marks, never before. Do not place footnotes after individual words. Do not place footnotes in examples, tables, etc. If you have an urge to do that, place the footnote to the text that explains the example, table, etc.

Footnotes should always be formulated as full, self-standing sentences.

\section{Tables}

For your tables use the table plus tabularx environments. The tabularx environment lets you (and requires you in fact) to specify the width of the table and defines the X column (left-alignment) and the Y column (right-alignment). All X/Y columns will have the same width and together they will fill out the width of the rest of the table -- counting out all non-X/Y columns.

Always include a meaningful caption. The caption is designed to appear on top of the table, no matter where you place it in the code. Do not try to tweak with this. Tables are delimited with lsptoprule at the top and lspbottomrule at the bottom. The header is delimited from the rest with midrule. Vertical lines in tables are banned. An example is provided in \tabref{sim:tab:frequencies}. See \citet{Nordhoff.Muller2021} for more information. If you are typesetting a very complex table or your table is too large to fit the page, do not hesitate to ask the editors for help.

\begin{table}
\caption{Frequencies of word classes}
\label{sim:tab:frequencies}
 \begin{tabularx}{.77\textwidth}{lYYYY} %.77 indicates that the table will take up 77% of the textwidth
  \lsptoprule
            & nouns & verbs  & adjectives & adverbs\\
  \midrule
  absolute  &   12  &    34  &    23      & 13\\
  relative  &   3.1 &   8.9  &    5.7     & 3.2\\
  \lspbottomrule
 \end{tabularx}
\end{table}

\section{Figures}

Figures must have a good quality. If you use pictorial figures, consult the editors early on to see if the quality and format of your figure is sufficient. If you use simple barplots, you can use the barplot environment (defined in langsci-osl.sty). See \figref{sim:fig:barplot} for an example. The barplot environment has 5 arguments: 1. x-axis desription, 2. y-axis description, 3. width (relative to textwidth), 4. x-tick descriptions, 5. x-ticks plus y-values.

\begin{figure}
    \centering
    \barplot{Type of meal}{Times selected}{0.6}{Bread,Soup,Pizza}%
    {
    (Bread,61)
    (Soup,12)
    (Pizza,8)
    }
    \caption{A barplot example}
    \label{sim:fig:barplot}
\end{figure}

The barplot environment builds on the tikzpicture plus axis environments of the pgfplots package. It can be customized in various ways. \figref{sim:fig:complex-barplot} shows a more complex example.

\begin{figure}
  \begin{tikzpicture}
    \begin{axis}[
	xlabel={Level of \textsc{uniq/max}},  
	ylabel={Proportion of $\textsf{subj}\prec\textsf{pred}$}, 
	axis lines*=left, 
        width  = .6\textwidth,
	height = 5cm,
    	nodes near coords, 
    % 	nodes near coords style={text=black},
    	every node near coord/.append style={font=\tiny},
        nodes near coords align={vertical},
	ymin=0,
	ymax=1,
	ytick distance=.2,
	xtick=data,
	ylabel near ticks,
	x tick label style={font=\sffamily},
	ybar=5pt,
	legend pos=outer north east,
	enlarge x limits=0.3,
	symbolic x coords={+u/m, \textminus u/m},
	]
	\addplot[fill=red!30,draw=none] coordinates {
	    (+u/m,0.91)
        (\textminus u/m,0.84)
	};
	\addplot[fill=red,draw=none] coordinates {
	    (+u/m,0.80)
        (\textminus u/m,0.87)
	};
	\legend{\textsf{sg}, \textsf{pl}}
    \end{axis} 
  \end{tikzpicture} 
    \caption{Results divided by \textsc{number}}
    \label{sim:fig:complex-barplot}
\end{figure}

\section{Hyphens, dashes, minuses, math/logical operators}\label{sim:sec:hyphens-etc}

Be careful to distinguish between hyphens (-), dashes (--), and the minus sign ($-$). For in-text appositions, use only en-dashes -- as done here -- with spaces around. Do not use em-dashes (---). Using mathmode is a reliable way of getting the minus sign.

All equations (and typically also semantic formulas, see \sectref{sim:sec:sem}) should be typeset using mathmode. Notice that mathmode not only gets the math signs ``right'', but also has a dedicated spacing. For that reason, never write things like p$<$0.05, p $<$ 0.05, or p$<0.05$, but rather $p<0.05$. In case you need a two-place math or logical operator (like $\wedge$) but for some reason do not have one of the arguments represented overtly, you can use a ``dummy'' argument (curly brackets) to simulate the presence of the other one. Notice the difference between $\wedge p$ and ${}\wedge p$.

In case you need to use normal text within mathmode, use the text command. Here is an example: $\text{frequency}=.8$. This way, you get the math spacing right.

\section{Abbreviations}

The final abbreviations section should include all glosses. It should not include other ad hoc abbreviations (those should be defined upon first use) and also not standard abbreviations like NP, VP, etc.


\section{Bibliography}

Place your bibliography into localbibliography.bib. Important: Only place there the entries which you actually cite! You can make use of our OSL bibliography, which we keep clean and tidy and update it after the publication of each new volume. Contact the editors of your volume if you do not have the bib file yet. If you find the entry you need, just copy-paste it in your localbibliography.bib. The bibliography also shows many good examples of what a good bibliographic entry should look like.

See \citet{Nordhoff.Muller2021} for general information on bibliography. Some important things to keep in mind:

\begin{itemize}
    \item Journals should be cited as they are officially called (notice the difference between and, \&, capitalization, etc.).
    \item Journal publications should always include the volume number, the issue number (field ``number''), and DOI or stable URL (see below on that).
    \item Papers in collections or proceedings must include the editors of the volume (field ``editor''), the place of publication (field ``address'') and publisher.
    \item The proceedings number is part of the title of the proceedings. Do not place it into the ``volume'' field. The ``volume'' field with book/proceedings publications is reserved for the volume of that single book (e.g. NELS 40 proceedings might have vol. 1 and vol. 2).
    \item Avoid citing manuscripts as much as possible. If you need to cite them, try to provide a stable URL.
    \item Avoid citing presentations or talks. If you absolutely must cite them, be careful not to refer the reader to them by using ``see...''. The reader can't see them.
    \item If you cite a manuscript, presentation, or some other hard-to-define source, use the either the ``misc'' or ``unpublished'' entry type. The former is appropriate if the text cited corresponds to a book (the title will be printed in italics); the latter is appropriate if the text cited corresponds to an article or presentation (the title will be printed normally). Within both entries, use the ``howpublished'' field for any relevant information (such as ``Manuscript, University of \dots''). And the ``url'' field for the URL.
\end{itemize}

We require the authors to provide DOIs or URLs wherever possible, though not without limitations. The following rules apply:

\begin{itemize}
    \item If the publication has a DOI, use that. Use the ``doi'' field and write just the DOI, not the whole URL.
    \item If the publication has no DOI, but it has a stable URL (as e.g. JSTOR, but possibly also lingbuzz), use that. Place it in the ``url'' field, using the full address (https: etc.).
    \item Never use DOI and URL at the same time.
    \item If the official publication has no official DOI or stable URL (related to the official publication), do not replace these with other links. Do not refer to published works with lingbuzz links, for instance, as these typically lead to the unpublished (preprint) version. (There are exceptions where lingbuzz or semanticsarchive are the official publication venue, in which case these can of course be used.) Never use URLs leading to personal websites.
    \item If a paper has no DOI/URL, but the book does, do not use the book URL. Just use nothing.
\end{itemize}

\section{Introduction: Aspectual markers}\label{bis:sec:intro}

This section introduces four aspectual markers: prefixes, the secondary imperfective marker, the semelfactive suffix, and the habitual suffix. I call these morphemes aspectual markers since they are relevant to morphological aspect (they can change the perfective/imperfective value of the base predicate) and/or since they are relevant to aspect more generally, e.g. because of bringing about (a)teli\-city, habituality or new aktionsart properties. 

\subsection{Prefixes}\label{bis:sec:pref}

Lexical prefixes (also called internal, qualifying, resultative) as well as superlexical (external, modifying, aktionsart) prefixes almost always perfectivize the imperfective simplex verb (for discussion of the two types of prefixes, see e.g. \citealt{Isacenko1962,Petr1986,Lehmann1993,Schoorlemmer1995,BabkoMalaya1999,Svenonius2004b,Arsenijevic2006,Romanova2006,Gehrke2008b,Tatevosov2013,Szucsich2014,Biskup.Zybatow2015,Caha.Zikova2016,Biskup2019,KlimekJankowska.Błaszczak2021,KlimekJankowska.Błaszczak2022}). For the perfectivizing effect of lexical prefixes, see examples (\ref{bis:ex:kleit}) and (\ref{bis:ex:chovat}).\footnote{Lexical prefixes are glossed with a meaning of the corresponding preposition and superlexical prefixes are glossed with the appropriate aktionsart abbreviation.}

\ea\label{bis:ex:kleit}
\ea\label{bis:ex:kleitA} \gll kleiť\textsuperscript{IPF} \\ 
stick \\
\glt ‘to stick on’
\ex\label{bis:ex:nakleitB} \gll na-kleiť\textsuperscript{PF} \\  
on-stick \\
\glt ‘to stick on’\hfill (Russian)
\z

\ex\label{bis:ex:chovat}
\ea\label{bis:ex:chovatA} \gll chovat\textsuperscript{IPF} \\ 
raise \\
\glt ‘to raise’
\ex\label{bis:ex:vychovatB} \gll vy-chovat\textsuperscript{PF} \\ 
out-raise \\
\glt ‘to raise’\hfill (Czech)
\z\z

\noindent With respect to the perfectivizing effect of superlexical prefixes, consider examples (\ref{bis:ex:delat}) and (\ref{bis:ex:plest}).

\ea\label{bis:ex:delat}
\ea\label{bis:ex:delatA} \gll delať\textsuperscript{IPF} \\ 
do \\
\glt ‘to do’
\ex\label{bis:ex:nadelatB} \gll na-delať\textsuperscript{PF} \\  
\textsc{cum}-do \\
\glt ‘to do a lot’\hfill (Russian)
\z

\ex\label{bis:ex:plest}
\ea\label{bis:ex:plestA} \gll plést\textsuperscript{IPF} \\ 
knit \\
\glt ‘to knit’
\ex\label{bis:ex:doplestB} \gll do-plést\textsuperscript{PF} \\  
\textsc{comp}-knit \\
\glt ‘to complete knitting’\hfill (Czech)
\z\z

\noindent Both Russian and Czech also have simplex verbs that are perfective. If they combine with a lexical or a superlexical prefix, they remain perfective, as demonstrated by the Russian examples in (\ref{bis:ex:kupit}) and the Czech examples in (\ref{bis:ex:dodat}).

\ea\label{bis:ex:kupit}
\ea\label{bis:ex:vykupitA} \gll [vy-[kupiť]\textsuperscript{PF}]\textsuperscript{PF} \\ 
out-buy \\
\glt ‘to buy sb.’s freedom’
\ex\label{bis:ex:nakupitB} \gll [na-[kupiť]\textsuperscript{PF}]\textsuperscript{PF} \\  
\textsc{cum}-buy \\
\glt ‘to buy a lot’\hfill (Russian)
\z

\ex\label{bis:ex:dodat}
\ea\label{bis:ex:dodatA} \gll [do-[dat]\textsuperscript{PF}]\textsuperscript{PF} \\
to-give \\
\glt ‘to deliver’
\ex\label{bis:ex:doriciB} \gll [do-[říci]\textsuperscript{PF}]\textsuperscript{PF} \\  
\textsc{comp}-say \\
\glt ‘to say to the end’\hfill (Czech)
\z\z

\noindent Lexical and superlexical prefixes can co-occur, as shown by the following examples. Also in this case, the predicate remains perfective. In addition, it holds that the superlexical prefix must occur outside the lexical prefix, as demonstrated by the contrast between  examples (\ref{bis:ex:perevypolnitA}), (\ref{bis:ex:prevychovatA}) and examples (\ref{bis:ex:vyperepolnitB}) and (\ref{bis:ex:vyprechovatB}).

\ea\label{bis:ex:perevypolnit}
\ea[]{\gll [pere-[vy-polniť]\textsuperscript{PF}]\textsuperscript{PF} \\ 
\textsc{exc}-out-fulfill \\
\glt ‘to overfulfill’\label{bis:ex:perevypolnitA}}
\ex[*]{\gll [vy-[pere-polniť]\textsuperscript{PF}]\textsuperscript{PF} \\ 
out-\textsc{exc}-fulfill\\
\hfill (Russian) \label{bis:ex:vyperepolnitB}}
\z

\ex\label{bis:ex:prevychovat}
\ea[]{\gll [pře-[vy-chovat]\textsuperscript{PF}]\textsuperscript{PF} \\
\textsc{rep}-out-raise \\
\glt ‘to re-educate’\label{bis:ex:prevychovatA}}
\ex[*]{\gll [vy-[pře-chovat]\textsuperscript{PF}]\textsuperscript{PF} \\ 
out-\textsc{rep}-raise\\
\hfill (Czech) \label{bis:ex:vyprechovatB}}
\z\z

\subsection{The secondary imperfective marker}\label{bis:sec:si}
In this section, I consider the effect of the secondary imperfective suffix on the morphological aspect of the base predicate. Let us begin with Russian.

The secondary imperfective suffix derives an imperfective predicate from a perfective predicate, which can contain a lexical prefix, as in examples (\ref{bis:ex:zarabotat}) and (\ref{bis:ex:pomoc}).

\ea\label{bis:ex:zarabotat}
\ea\label{bis:ex:zarabotatA} \gll [za-[rabot-a]\textsuperscript{IPF}]\textsuperscript{PF}-ť \\ 
behind-work-\textsc{th}-\textsc{inf} \\
\glt ‘to earn’
\ex\label{bis:ex:zarabatyvatB} \gll [[za-[rabat]\textsuperscript{IPF}]\textsuperscript{PF}-yva]\textsuperscript{IPF}-ť \\  
behind-work-\textsc{si}-\textsc{inf} \\
\glt ‘to earn’\hfill (Russian)
\z

\ex\label{bis:ex:pomoc}
\ea\label{bis:ex:pomocA} \gll [po-[moč’]\textsuperscript{IPF}]\textsuperscript{PF} \\ 
along-can \\
\glt ‘to help’
\ex\label{bis:ex:pomagatB} \gll [[po-[mag]\textsuperscript{IPF}]\textsuperscript{PF}-a]\textsuperscript{IPF}-ť \\  
along-can-\textsc{si}-\textsc{inf} \\
\glt ‘to help’\hfill (Russian)
\z\z

\noindent The imperfectivizing suffix can also derive an imperfective predicate from a perfective stem with a superlexical prefix, as in (\ref{bis:ex:zarabotatSP}), or from a perfective stem without a prefix, as shown in (\ref{bis:ex:dat}).

\ea\label{bis:ex:zarabotatSP}
\ea\label{bis:ex:zarabotatSPA} \gll [za-[rabot-a]\textsuperscript{IPF}]\textsuperscript{PF}-ť \\ 
\textsc{inc}-work-\textsc{th}-\textsc{inf} \\
\glt ‘to start working’
\ex\label{bis:ex:zarabatyvatSPB} \gll [[za-[rabat]\textsuperscript{IPF}]\textsuperscript{PF}-yva]\textsuperscript{IPF}-ť \\  
\textsc{inc}-work-\textsc{si}-\textsc{inf} \\
\glt ‘to start working’\hfill (Russian)
\z

\ex\label{bis:ex:dat}
\ea\label{bis:ex:datA} \gll [d-a]\textsuperscript{PF}-ť \\ 
give-\textsc{th}-\textsc{inf} \\
\glt ‘to give’
\ex\label{bis:ex:davatB} \gll [[d-a]\textsuperscript{PF}-va]\textsuperscript{IPF}-ť \\  
give-\textsc{th}-\textsc{si}-\textsc{inf} \\ \\
\glt ‘to give’\hfill (Russian)
\z\z

\noindent Certain superlexical prefixes can also attach outside the imperfectivizing suffix (see e.g. \citealt{Ramchand2004,Gehrke2008b,Tatevosov2013,Szucsich2014,KlimekJankowska.Błaszczak2021,KlimekJankowska.Błaszczak2022}) and they perfectivize the predicate again, as illustrated in example (\ref{bis:ex:vytalkivat}).

\ea\label{bis:ex:vytalkivat}
\ea\label{bis:ex:vytalkivatA} \gll 
[[vy-[talk]\textsuperscript{IPF}]\textsuperscript{PF}-iva]\textsuperscript{IPF}-ť \\
out-push-\textsc{si}-\textsc{inf} \\
\glt ‘to push out’
\ex\label{bis:ex:povytalkivatB} \gll [po-[[vy-[talk]\textsuperscript{IPF}]\textsuperscript{PF}-iva]\textsuperscript{IPF}]\textsuperscript{PF}-ť \\  
\textsc{dist}-out-push-\textsc{si}-\textsc{inf} \\
\glt ‘to push out one after another’\hfill (Russian)
\z\z

\noindent Some superlexical prefixes can occur both inside the imperfectivizing suffix, as the inceptive \textit{za-} in (\ref{bis:ex:zarabotatSP}), and outside the secondary imperfective marker, as the inceptive \textit{za-} in the following example.

\ea\label{bis:ex:otkryvat}
\ea\label{bis:ex:otkryvatA} \gll 
[[ot-[kry]\textsuperscript{IPF}]\textsuperscript{PF}-va]\textsuperscript{IPF}-ť \\
away-cover-\textsc{si}-\textsc{inf} \\
\glt ‘to open’
\ex\label{bis:ex:zaotkryvatB} \gll [za-[[ot-[kry]\textsuperscript{IPF}]\textsuperscript{PF}-va]\textsuperscript{IPF}]\textsuperscript{PF}-ť \\  
\textsc{inc}-away-cover-\textsc{si}-\textsc{inf} \\
\glt ‘to start opening’\hfill (Russian)
\z\z

\noindent Standardly, the secondary imperfective suffix is taken to have three forms: \textit{-yva-}\slash\textit{-iva-}, as in (\ref{bis:ex:zarabatyvatB}), (\ref{bis:ex:zarabatyvatSPB}) and (\ref{bis:ex:vytalkivat}), \textit{-va-}, as in (\ref{bis:ex:davatB}) and (\ref{bis:ex:otkryvat}), and \textit{-a-}/\textit{-ja-}, as in (\ref{bis:ex:pomagatB}); see e.g. \citet{Vinogradov.etal1952}, but there are also alternative analyses like \citet{Isacenko1962} and \citet{Matushansky2009}. A closer look at the data under discussion reveals that \textit{v} is present in \textit{-va-} because of blocking hiatus; compare examples (\ref{bis:ex:dat}) and (\ref{bis:ex:otkryvat}) with example (\ref{bis:ex:pomagatB}). 			

In Czech, an analogous pattern is observed: the secondary imperfective suffix derives an imperfective verb from a perfective stem and the base predicate can contain either a lexical prefix or a superlexical prefix. Examples (\ref{bis:ex:zabijetB}) and (\ref{bis:ex:vyprosovatB}) show an imperfective predicate derived from a lexically prefixed verb. 

\ea\label{bis:ex:zabit}
\ea\label{bis:ex:zabitA} \gll [za-[bí]\textsuperscript{IPF}]\textsuperscript{PF}-t \\ 
behind-beat-\textsc{inf} \\
\glt ‘to kill’
\ex\label{bis:ex:zabijetB} \gll [[za-[bí]\textsuperscript{IPF}]\textsuperscript{PF}-je]\textsuperscript{IPF}-t \\  
behind-beat-\textsc{si}-\textsc{inf} \\
\glt ‘to kill’\hfill (Czech)
\z

\ex\label{bis:ex:vyprosit}
\ea\label{bis:ex:vyprositA} \gll [vy-[pros-i]\textsuperscript{IPF}]\textsuperscript{PF}-t \\ 
out-beg-\textsc{th}-\textsc{inf} \\
\glt ‘to beg’
\ex\label{bis:ex:vyprosovatB} \gll [[vy-[proš]\textsuperscript{IPF}]\textsuperscript{PF}-ova]\textsuperscript{IPF}-t \\  
out-beg-\textsc{si}-\textsc{inf} \\
\glt ‘to beg’\hfill (Czech)
\z\z

\noindent In contrast, example (\ref{bis:ex:dopletatB}) demonstrates an imperfective predicate derived from a superlexically prefixed predicate. 

\ea\label{bis:ex:doplest}
\ea\label{bis:ex:doplestA} \gll [do-[plés]\textsuperscript{IPF}]\textsuperscript{PF}-t \\ 
\textsc{comp}-knit-\textsc{inf} \\
\glt ‘to complete knitting’
\ex\label{bis:ex:dopletatB} \gll [[do-[plét]\textsuperscript{IPF}]\textsuperscript{PF}-a]\textsuperscript{IPF}-t \\  
\textsc{comp}-knit-\textsc{si}-\textsc{inf} \\
\glt ‘to complete knitting’\hfill (Czech)
\z\z

\noindent The imperfectivizing suffix can also derive an imperfective predicate from an unprefixed perfective verb, as illustrated in examples (\ref{bis:ex:datCZ}) and (\ref{bis:ex:vratit}).

\ea\label{bis:ex:datCZ}
\ea\label{bis:ex:datCZA} \gll [d-á]\textsuperscript{PF}-t \\ 
give-\textsc{th}-\textsc{inf} \\
\glt ‘to give’
\ex\label{bis:ex:davatCZB} \gll [[d-á]\textsuperscript{PF}-va]\textsuperscript{IPF}-t \\  
give-\textsc{th}-\textsc{si}-\textsc{inf} \\ \\
\glt ‘to give’\hfill (Czech)
\z

\ex\label{bis:ex:vratit}
\ea\label{bis:ex:vratitA} \gll [vrát-i]\textsuperscript{PF}-t \\ 
return-\textsc{th}-\textsc{inf} \\
\glt ‘to return’
\ex\label{bis:ex:vracetB} \gll [[vrac]\textsuperscript{PF}-e]\textsuperscript{IPF}-t \\  
return-\textsc{si}-\textsc{inf} \\ \\
\glt ‘to return’\hfill (Czech)
\z\z

\noindent In Czech, too, certain superlexical prefixes attach to the stem after the imperfectivizing suffix. Hence, they perfectivize the secondary imperfective predicate, as illustrated in the following example, based on example (\ref{bis:ex:zabit}).

\ea\label{bis:ex:zabijet}
\ea\label{bis:ex:zabijetA} \gll [[za-[bí]\textsuperscript{IPF}]\textsuperscript{PF}-je]\textsuperscript{IPF}-t \\  
behind-beat-\textsc{si}-\textsc{inf} \\
\glt ‘to kill’
\ex\label{bis:ex:pozabijetB} \gll [po-[[za-[bí]\textsuperscript{IPF}]\textsuperscript{PF}-je]\textsuperscript{IPF}]\textsuperscript{PF}-t \\  
\textsc{dist}-behind-beat-\textsc{si}-\textsc{inf} \\
\glt ‘to kill one after another’\hfill (Czech)
\z\z

\noindent Some superlexical prefixes can attach to the verb both before the imperfectivizing suffix, as in (\ref{bis:ex:doplest}), and after the imperfectivizing marker, as in (\ref{bis:ex:dovypletatC}). Both examples contain an occurrence of the completive prefix \textit{do-}.\footnote{In this respect, Czech differs from Russian, which only allows completive \textit{do}- in the lower position (see  \citepossalt{Tatevosov2008} discussion of intermediate prefixes).}

\ea\label{bis:ex:vyplest}
\ea\label{bis:ex:vyplestA} \gll [vy-[plés]\textsuperscript{IPF}]\textsuperscript{PF}-t \\ 
out-string-\textsc{inf} \\
\glt ‘to string’
\ex\label{bis:ex:vypletatB} \gll [[vy-[plét]\textsuperscript{IPF}]\textsuperscript{PF}-a]\textsuperscript{IPF}-t \\  
out-string-\textsc{si}-\textsc{inf} \\
\glt ‘to string’
\ex\label{bis:ex:dovypletatC} \gll [do-[[vy-[plét]\textsuperscript{IPF}]\textsuperscript{PF}-a]\textsuperscript{IPF}]\textsuperscript{PF}-t \\  
\textsc{comp}-out-string-\textsc{si}-\textsc{inf} \\
\glt ‘to complete stringing’\hfill (Czech)
\z\z

\noindent It is obvious from the examples that there are three secondary imperfective markers in Czech: -(\textit{v})\textit{a}-, present in (\ref{bis:ex:doplest}), (\ref{bis:ex:datCZ}) and (\ref{bis:ex:vyplest}), -\textit{ova}-, occurring in (\ref{bis:ex:vyprosit}), and the suffix -(\textit{j})\textit{e}-, which is present in (\ref{bis:ex:zabit}) and (\ref{bis:ex:vratit}) and which is not productive (see \citealt{Petr1986}). The examples also suggest that \textit{v} in -\textit{va}- and \textit{j} in -\textit{je}- block hiatus; compare (\ref{bis:ex:datCZ}) with (\ref{bis:ex:dopletatB}) and (\ref{bis:ex:zabijetB}) with (\ref{bis:ex:vracetB}). In fact, the pattern could be simplified if we decomposed -\textit{ova}- and the Russian -\textit{yva}-/-\textit{iva}-. They follow the general Slavic -V\textit{va}- pattern, with a vowel, -\textit{v}- blocking hiatus and (the iterative) -\textit{a}- (see e.g. \citealt{Kuznecov1953} and \citealt{Lunt2001}). For ease of exposition, I will treat the imperfectivizing markers as a whole in what follows.

Thus, the relevant part of the linearized structure with aspectual markers and their aspectual effects looks like (\ref{bis:ex:SIstr}). \textit{LP} stands for lexical prefixes, \textit{SP} for superlexical prefixes and \textit{SI} for the secondary imperfective suffix. 

\ea\label{bis:ex:SIstr}	
[SP\textsubscript{higher}[[SP\textsubscript{lower}[LP[√root]\textsuperscript{PF/IPF}]\textsuperscript{PF}]\textsuperscript{PF}SI]\textsuperscript{IPF}]\textsuperscript{PF} 
\z

\noindent Recall that some superlexical prefixes merge lower and others higher than the imperfectivizing suffix (and some of them can merge in a lower as well as in a higher position).

\subsection{The semelfactive marker}\label{bis:sec:seml}
The semelfactive suffix consists of -\textit{n}- and some vowel in Slavic (the original form was *-\textit{nVn}-; see \citealt{Wiemer.Serzant2017}). It selects a root with a punctual or instantaneous property and derives a perfective stem, as illustrated in the Russian example (\ref{bis:ex:krik}) and the Czech example (\ref{bis:ex:bod}).\footnote{Some Russian verbs take the expressive, extended marker \textit{-anu-} (and some both \textit{-nu-} and \textit{-anu-}); see e.g. \citet{Isacenko1962} and  \citet{Svedova1980}.}

\ea\label{bis:ex:krik}
\ea\label{bis:ex:krikA} \gll 
krik \\ 
shout \\
\glt ‘shout’
\ex\label{bis:ex:kriknutB} \gll krik-nu-ť\textsuperscript{PF} \\  
shout-\textsc{seml}-\textsc{inf} \\
\glt ‘to shout out’\hfill (Russian)
\z

\ex\label{bis:ex:bod}
\ea\label{bis:ex:bodA} \gll 
bod \\ 
point \\
\glt ‘point’
\ex\label{bis:ex:bodnoutB} \gll bod-nou-t\textsuperscript{PF} \\  
point-\textsc{seml}-\textsc{inf} \\
\glt ‘to stab’\hfill (Czech)
\z\z

\noindent The semelfactive marker differs from the suffix -\textit{nV}- present in other verbs like degree achievements. The degree achievement -\textit{nV}- selects a root denoting a property and does not have a perfectivizing effect on the verb (see \citealt{TaraldsenMedova.Wiland2019} for the relation and differences between the two -\textit{nV}- suffixes).

Since the semelfactive suffix attaches directly to the root and verbalizes it, as shown by the contrasts in (\ref{bis:ex:krik}) and (\ref{bis:ex:bod}), I assume that it spells out the verbalizing head \textit{v}. If correct, then we expect the semelfactive suffix to be in complementary distribution with other themes representing the verbalizing \textit{v}. This prediction is borne out, as demonstrated below. The  examples in (\ref{bis:ex:kricatA}) and (\ref{bis:ex:bodatA}) show a grammatical combination of the root and a theme vowel, whereas the examples in (\ref{bis:ex:kriknuvatB})--(\ref{bis:ex:kricanutC}) and (\ref{bis:ex:bodanoutB})--(\ref{bis:ex:bodnouvatC}) -- based on grammatical forms (\ref{bis:ex:kriknutB}) and (\ref{bis:ex:bodnoutB}) – demonstrate that the co-occurrence of the theme vowel and the semelfactive suffix leads to ungrammaticality in both orders.\footnote{A reviewer suggests analyzing the marker \textit{-nu-} as a sequence of the semelfactive marker (with the perfective feature) and the theme vowel, which would have the advantage that all theme vowels would be analyzed identically: as verbalizers without aspectual features. The disadvantage, however, is that then the verbalizer (the theme vowel) would not be adjacent to the root, contrary to the standard assumption. In addition, the elements behave like a unit, e.g. with respect to elision; cf. the following Czech alternatives in the past tense: \textit{tiskl}/\textit{tisknul} ‘printed’.}\footnote{To avoid hiatus, I insert /v/ between the semelfactive suffix and the theme vowel in (\ref{bis:ex:kriknuvatB}) and (\ref{bis:ex:bodnouvatC}), a strategy known from secondary imperfectives.} 

\ea\label{bis:ex:kricat}
\ea[]{\gll 
krič-a-ť \\ 
shout-\textsc{th}-\textsc{inf} \\
\glt ‘to shout’\label{bis:ex:kricatA}}
\ex[*]{\gll krik-nu-va-ť \\ 
shout-\textsc{seml}-\textsc{th}-\textsc{inf} \\
\glt Intended: ‘to shout out’\label{bis:ex:kriknuvatB}}
\ex[*]{\gll krič-a-nu-ť \\ 
shout-\textsc{th}-\textsc{seml}-\textsc{inf} \\
\glt Intended: ‘to shout out’\label{bis:ex:kricanutC}
\hfill (Russian)}
\z

\ex\label{bis:ex:bodat}
\ea[]{\gll 
bod-a-t \\ 
point-\textsc{th}-\textsc{inf} \\
\glt ‘to stab’\label{bis:ex:bodatA}}
\ex[*]{\gll bod-a-nou-t \\ 
point-\textsc{th}-\textsc{seml}-\textsc{inf} \\
\glt Intended: ‘to stab’\label{bis:ex:bodanoutB}}
\ex[*]{\gll bod-nou-va-t \\ 
point-\textsc{seml}-\textsc{th}-\textsc{inf} \\
\glt Intended: ‘to stab’\label{bis:ex:bodnouvatC}
\hfill (Czech)}
\z\z

\noindent Given that the semelfactive marker represents the verbalizing head \textit{v}, the complementary distribution of this suffix and the secondary imperfective marker – shown in (\ref{bis:ex:kriknuvat}) and (\ref{bis:ex:bodnouvat}) – cannot be based on structural blocking, as proposed e.g. by \citet{Markman2008Slavic} for Russian.

\ea[*]{\gll krik-nu-va-ť \\ 
shout-\textsc{seml}-\textsc{si}-\textsc{inf}\\
\glt Intended: ‘to shout out’\label{bis:ex:kriknuvat}
\hfill (Russian)}
\z

\ea[*]{\gll bod-nou-va-t\\ 
point-\textsc{seml}-\textsc{si}-\textsc{inf}\\
\glt Intended: ‘to stab’\label{bis:ex:bodnouvat}
\hfill (Czech)}
\z

\noindent The reason for ungrammaticality of cases like (\ref{bis:ex:kriknuvat}) and (\ref{bis:ex:bodnouvat}) can be rather semantic. For instance, \citet{Jablonska2007} argues that semelfactives – being instantaneous – do not have a process part in their event structure, on which the progressive operator of secondary imperfectives could operate. Another possibility is to assume that the secondary imperfective suffix spells out an atelicizer/eventizer, which combines with complex events, i.e. accomplishments ($λR.λe.∃s[R(e)(s)]$, see \citealt{Lazorczyk2010} and \citealt{Tatevosov2015Slavic}). It is obvious that semelfactives are not of the appropriate eventive type; they do not introduce a change of state (e.g. \citealt{Smith1991}) and they are taken to be achievements by \citet{Vendler1957}.\footnote{The second reasoning could also explain the incompatibility of the degree achievement -\textit{n}(\textit{V})- with the imperfectivizing suffix in cases like (\ref{bis:ex:sochnuvatB}). Alternatively, one may suggest that the ungrammatical status of (\ref{bis:ex:sochnuvatB}) has an economy reason because degree achievement verbs like \textit{sochnuť} in (\ref{bis:ex:sochnutA}) are imperfective (without the imperfectivizing suffix).
\ea
\ea[]{ 
\gll 
soch-nu-ť \\ 
dry-\textsc{da}-\textsc{inf} \\
\glt ‘to dry’
\label{bis:ex:sochnutA}
}
\ex[*]{ 
\gll soch-nu-va-ť \\ 
dry-\textsc{da}-\textsc{si}-\textsc{inf} \\
\glt Intended: ‘to dry’\label{bis:ex:sochnuvatB}
\hfill (Russian)
}
\z
\z
}

There is also a possibility to exclude cases like (\ref{bis:ex:kriknuvat}) and (\ref{bis:ex:bodnouvat}) by morphological blocking, where the existence of the simpler imperfective forms \textit{kričať} in (\ref{bis:ex:kricatA}) and \textit{bodat} in (\ref{bis:ex:bodatA}) prevents the use of the more complex forms (\ref{bis:ex:kriknuvat}) and (\ref{bis:ex:bodnouvat}). The advantage of the second and the third possibility is that in contrast to the argument by \citet{Jablonska2007} they can also answer the question of why (\ref{bis:ex:kriknuvat}) and (\ref{bis:ex:bodnouvat}) are not possible with the iterative (non-progressive) reading of the imperfectivizing suffix.\footnote{As pointed out by a reviewer, the claim that the complementary distribution of the semelfactive suffix and the secondary imperfective marker is not based on structural blocking is also supported by the fact that in languages like South-East Serbo-Croatian, the two markers are combined quite productively, as in \textit{tak-n-uje-m} ‘I touch repeatedly’.}			

As to structural properties of the semelfactive -\textit{n}(\textit{V})-, it needs to be placed outside lexical prefixes, as demonstrated in (\ref{bis:ex:SEMLstr}), with SEML representing the verbalizing head \textit{v}. 

\ea\label{bis:ex:SEMLstr}	
[SP\textsubscript{higher}[[SP\textsubscript{lower}[\textsubscript{\textit{v}} SEML [LP[√root]\textsuperscript{PF/IPF}]\textsuperscript{PF}]\textsuperscript{PF}]\textsuperscript{PF}SI]\textsuperscript{IPF}]\textsuperscript{PF} 
\z

\noindent The rationale behind is that root nominalizations can contain lexical prefixes but cannot include the semelfactive -\textit{n}(\textit{V})-. As shown in (\ref{bis:ex:podkop}) for Russian and in (\ref{bis:ex:prikop}) for Czech, root nominalizations can contain lexical prefixes but can include neither lower superlexicals nor higher superlexical prefixes (see also \citealt{Caha.Zikova2016} for Czech data). The Russian \textit{podkop} can only have the meaning ‘tunnel’; the attenuative superlexical interpretation of \textit{pod-} is not available in this case. Similarly in the Czech (\ref{bis:ex:prikop}), \textit{příkop} can only mean ‘ditch’ and the prefix \textit{pří-} cannot have the attenuative interpretation. 

\ea\label{bis:ex:podkop}
\ea[]{\gll 
pod-kop \\ 
under-dig \\
\glt ‘tunnel’\label{bis:ex:podkopA}}
\ex[*]{\gll pod-kop \\ 
\textsc{att}-dig \\
\glt Intended: ‘little kick’
\label{bis:ex:podkopB}
\hfill (Russian)}
\z

\ex\label{bis:ex:prikop}
\ea[]{\gll 
pří-kop \\ 
at-dig \\
\glt ‘ditch’\label{bis:ex:prikopA}}
\ex[*]{\gll pří-kop \\ 
\textsc{att}-dig \\
\glt Intended: ‘little kick’
\label{bis:ex:prikopB}
\hfill (Czech)}
\z\z

\noindent This means that the boundary of root nominalizations must be placed between the projection containing lexical prefixes and the projections with lower superlexicals (and the projection with the semelfactive suffix) in (\ref{bis:ex:SEMLstr}).

There is, however, an interesting distinction between Russian and Czech with respect to nominalizations and the semelfactive suffix. While in Czech the suffix can be a part of stem nominalizations, in Russian it is not possible; consider the contrast between (\ref{bis:ex:kopnutie}) and (\ref{bis:ex:kopnuti}).

\ea[*]{\gll kop-nu-t-i-e \\ 
dig-\textsc{seml}-\textsc{n/t}-\textsc{nmlz}-\textsc{nom.sg}\\
\glt Intended: ‘a dig/kick’
\label{bis:ex:kopnutie}
\hfill (Russian)}
\ex[]{\gll kop-nu-t-í \\ 
dig-\textsc{seml}-\textsc{n/t}-\textsc{nmlz.nom.sg}\\
\glt ‘a dig/kick’
\label{bis:ex:kopnuti}
\hfill (Czech)} 
\z

\noindent This can be related to the fact that in contrast to Czech nominalizations, Russian stem nominalizations are structurally less complex and do not contain the aspectual projection, as discussed in the next section. 			

As illustrated in (\ref{bis:ex:krik}) and (\ref{bis:ex:bod}), the semelfactive suffix perfectivizes the stem, as do prefixes. If both elements co-occur, then unsurprisingly the predicate remains perfective, irrespective of whether the prefix is lexical or superlexical. For a lexical prefix, consider the Russian example in (\ref{bis:ex:vskriknut}) and for a superlexical prefix consider the Czech example (\ref{bis:ex:naprasknout}), with an attenuative reading.

\ea[]{\gll [vs-[krik-nu-ť]\textsuperscript{PF}]\textsuperscript{PF} \\ 
up-shout-\textsc{seml}-\textsc{inf} \\
\glt ‘to give a scream’
\label{bis:ex:vskriknut}
\hfill (Russian)}
\ex[]{\gll [na-[prask-nou-t]\textsuperscript{PF}]\textsuperscript{PF} \\ 
\textsc{att}-crack-\textsc{seml}-\textsc{inf} \\
\glt ‘to crack partially’
\label{bis:ex:naprasknout}
\hfill (Czech)} 
\z

\noindent Generally, it is difficult to find examples of semelfactive predicates with a superlexical prefix. This results from the fact that semelfactive predicates refer to bounded singleton events that are punctual, which clashes with the fact that superlexical prefixes typically modify the spatiotemporal path of the event expressed by the base predicate. Moreover, the perfective aspect of semelfactive verbs pose a problem for the imperfective selection properties of some superlexical prefixes.

As the comparison of (\ref{bis:ex:pobodatA}) and (\ref{bis:ex:pobodnoutB}) shows, the semelfactive -\textit{n}(\textit{V})- is responsible for the ungrammatical status of the verb prefixed by the delimitative prefix \textit{po-}.

\ea\label{bis:ex:pobodat}
\ea[]{\gll 
po-bod-a-t \\ 
\textsc{del}-point-\textsc{th}-\textsc{inf} \\
\glt ‘to stab to a certain extent several times’\label{bis:ex:pobodatA}}
\ex[*]{\gll po-bod-nou-t \\ 
\textsc{del}-point-\textsc{seml}-\textsc{inf} \\
\glt Intended: ‘to stab in a short time frame’
\label{bis:ex:pobodnoutB}
\hfill (Czech)}
\z
\z

\noindent Building on the data, I propose the following meaning for the semelfactive -\textit{n}(\textit{V})-.

\ea 
\sib{SEML}${}=\lambda P\lambda e[P(e)\wedge \cnst{atom}(e)\wedgeμ(e) = 1]$
\label{bis:ex:seml}
\z

\noindent It derives predicates with a single occurrence of the event (via the measure function μ: cardinality) described by the stem and the event is atomic. That is, there is no proper part of the event (it is punctual), which means that the predicate is not divisive, which in turn means that it is quantized (see \citealt{Borer2005}). Because of the minimal (atomic) property of the semelfactive -\textit{n}(\textit{V})-, there is no path in the event that could be accessible to the delimitative \textit{po-} in cases like (\ref{bis:ex:pobodnoutB}).\footnote{The minimal property is a (language) idealization; in the real world, there can be some trajectory involved e.g. in the stab movement (cf. \citealt{Rothstein2004}).} The ungrammaticality of (\ref{bis:ex:pobodnoutB}) cannot be based on unsatisfied selection properties of the prefix \textit{po-} if delimitative \textit{po-} and attenuative \textit{po-} form a natural class. Specifically, the attenuative prefix can also adjoin to perfective predicates in Czech, as in [\textit{po-}[\textit{otevřít}]\textsuperscript{PF}]\textsuperscript{PF} ‘to open a little’.

The single occurrence property of the semelfactive -\textit{n}(\textit{V})- in (\ref{bis:ex:seml}) is responsible for the fact that the iterative reading is not available in cases like \textit{kriknuť} ‘to shout out’ and \textit{bodnout} ‘to stab’ in (\ref{bis:ex:kriknutB}) and (\ref{bis:ex:bodnoutB}), respectively. In contrast, predicates with the identical root but without the semelfactive -\textit{n}(\textit{V})- like \textit{kričať} ‘to shout’ and \textit{bodat} ‘to stab’ in (\ref{bis:ex:kricatA}) and (\ref{bis:ex:bodatA}) allow the iterative interpretation.\footnote{The single occurrence property can also be defined in terms of a maximality operator; see \citet{Egg2018}.}

\subsection{The habitual marker}\label{bis:sec:hab}

Russian habitual forms like (\ref{bis:ex:pisyvatB}) – derived from (\ref{bis:ex:pisatA}) – are classified as colloquial or archaic and it is often claimed that they only occur in the past tense (see \citealt{Isacenko1962,Zaliznjak.Smelev1997,Paducheva2015}, but see also \citealt{Tatevosov2013}).\footnote{I use the term \textsc{habitual} but various terms can be found in the literature: ``iterative'', ``frequentative'' and ``generic''.}

\ea\label{bis:ex:pisat}
\ea\label{bis:ex:pisatA} \gll 
pis-a-ť\textsuperscript{IPF} \\ 
write-\textsc{th}-\textsc{inf} \\
\glt ‘to write’
\ex\label{bis:ex:pisyvatB} \gll pis-yva-ť\textsuperscript{IPF} \\ 
write-\textsc{hab}-\textsc{inf} \\
\glt ‘to write repeatedly’\hfill (Russian)
\z
\z

\noindent In contrast, Czech derives analogous imperfective forms quite productively (\citealt{Filip1993,Filip.Carlson1997,Esvan2007,Nubler2017}, but see also \citealt{Berger2009}); consider example (\ref{bis:ex:psat}). Certain authors even consider forms like (\ref{bis:ex:psavatB}) to be an instantiation of a ‘third aspect’ (see e.g. \citealt{Kopecny1962}).\footnote{Against expectations, Polish is even more restricted than Russian with respect to habitual forms like \textit{pis-ywa-ć} ‘to write repeatedly’. There are only a few verbs (see \citealt{Grzegorczykowa.etal1984} and \citealt{Lazinski2020}).}

\ea\label{bis:ex:psat}
\ea\label{bis:ex:psatA} \gll 
ps-á-t\textsuperscript{IPF} \\ 
write-\textsc{th}-\textsc{inf} \\
\glt ‘to write’
\ex\label{bis:ex:psavatB} \gll ps-á-va-t\textsuperscript{IPF} \\ 
write-\textsc{th}-\textsc{hab}-\textsc{inf} \\
\glt ‘to write repeatedly’\hfill (Czech)
\z
\z

\noindent The examples above show that in both languages, the habitual suffix derives an imperfective verb from an imperfective base.

In Czech, there are also reduplicative forms like (\ref{bis:ex:psavavat}), which are usually described as expressive predicates denoting a longer (or temporally distant, see \citealt{Filip1993}) habitual event. They are imperfective, too.

\ea\label{bis:ex:psavavat}
\gll 
ps-á-vá-va-t\textsuperscript{IPF} \\ 
write-\textsc{th}-\textsc{hab}-\textsc{hab}-\textsc{inf} \\
\glt ‘to write repeatedly for a long time/long ago’\hfill (Czech)
\z

\noindent In contrast to Russian, it is also possible to derive a habitual predicate from a secondary imperfective verb in Czech, as shown by the pair in (\ref{bis:ex:vypisovat}). The derived verb is again imperfective. 

\ea\label{bis:ex:vypisovat}
\ea\label{bis:ex:vypisovatA} \gll 
vy-pis-ova-t\textsuperscript{IPF} \\ 
out-write-\textsc{si}-\textsc{inf} \\
\glt ‘to excerpt’
\ex\label{bis:ex:vypisovavatB} \gll vy-pis-ová-va-t\textsuperscript{IPF} \\
out-write-\textsc{si}-\textsc{hab}-\textsc{inf} \\
\glt ‘to excerpt repeatedly’\hfill (Czech)
\z
\z

\noindent Examples (\ref{bis:ex:psavatB}) and (\ref{bis:ex:vypisovavatB}) show that the habitual marker is outside the theme and the imperfectivizing suffix, respectively. Building on the structural proposal in (\ref{bis:ex:SEMLstr}), that means that the habitual suffix must also be higher than lexical prefixes and lower superlexical prefixes.

In fact, the habitual marker is even higher than higher superlexical prefixes and the aspectual projection. The argument goes as follows. It has been argued that Russian \textit{nie}-nominals are aspectless (see \citealt{Svedova1980,Schoorlemmer1995,Gehrke2008b,Tatevosov2011,Tatevosov2020}); hence phasal verbs can combine with prefixed nominals derived from a perfective stem like in (\ref{bis:ex:nacal}). 

\ea\label{bis:ex:nacal}
\gll 
načal 		na-pis-a-n-i-e \\ 
started	on-write-\textsc{th}-\textsc{n/t}-\textsc{nmlz}-\textsc{acc.sg} \\
\glt ‘started writing’  \hfill (Russian; based on \citealt{Tatevosov2011}: ex. (18))
\z

\noindent On the contrary, Czech stem nominalizations have the morphological aspect (e.g. \citealt{Prochazkova2006}). For this reason, the phasal verb is compatible with the imperfective nominals in (\ref{bis:ex:zacalA}) and (\ref{bis:ex:zacaloA}) but is not compatible with the perfective nominals in (\ref{bis:ex:zacalB}) and (\ref{bis:ex:zacaloB}).\largerpage

\ea\label{bis:ex:zacal}
\ea[]{
\gll 
začal vy-pis-ová-n-í \\ 
started	out-write-\textsc{si}-\textsc{n/t}-\textsc{nmlz.acc.sg} \\
\glt ‘he started writing out’\label{bis:ex:zacalA}}
\ex[*]{\gll 
začal vy-ps-á-n-í \\ 
started	out-write-\textsc{th}-\textsc{n/t}-\textsc{nmlz.acc.sg} \\
\glt Intended: ‘he started writing out’
\label{bis:ex:zacalB}
\hfill (Czech)}
\z
\ex\label{bis:ex:zacalo}
\ea[]{
\gll 
začalo na-kup-ová-n-í \\ 
started	on-buy-\textsc{si}-\textsc{n/t}-\textsc{nmlz.nom.sg} \\
\glt ‘buying started’\label{bis:ex:zacaloA}}
\ex[*]{\gll 
začalo na-koup-e-n-í \\ 
started	on-buy-\textsc{th}-\textsc{n/t}-\textsc{nmlz.nom.sg} \\
\glt Intended: ‘buying started’
\label{bis:ex:zacaloB}
\hfill (Czech)}
\z
\z

\noindent Czech stem nominalizations can be prefixed with higher superlexical prefixes like the cumulative \textit{na-} in example (\ref{bis:ex:nahazeniA}), in contrast to Russian \textit{-nie} nominals, which only allow superlexicals in the lower position (see \citealt{Tatevosov2011}). Note that the prefix \textit{na-} is indeed cumulative because the prefixed predicate can take a plural object like in \textit{naházení židlí na něco} ‘throwing chairs on sth.’ but cannot combine with a quantized singular object like in \textit{naházení židle na něco} ‘throwing a chair on sth.’. Crucially, stem nominalizations cannot contain the habitual suffix, as demonstrated in (\ref{bis:ex:psavaniB}). 

\ea\label{bis:ex:nahazeni}
\ea[]{
\gll 
na-ház-e-n-í \\ 
\textsc{cum}-throw-\textsc{th}-\textsc{n/t}-\textsc{nmlz.nom.sg} \\
\glt ‘throwing a lot of sth.’ \label{bis:ex:nahazeniA}}
\ex[*]{\gll 
ps-á-vá-n-í \\ 
write-\textsc{th}-\textsc{hab}-\textsc{n/t}-\textsc{nmlz.nom.sg} \\
\glt Intended: ‘repeated writing’
\label{bis:ex:psavaniB}
\hfill (Czech)}
\z
\z

\noindent This means that stem nominalizations include the structure in (\ref{bis:ex:SEMLstr}). Their structure includes higher superlexical prefixes but also the aspectual projection in Czech, which hosts the perfective or the imperfective operator responsible for the morphological aspect interpretation.\footnote{In the case of the perfective operator, the event time is included in the reference time, as in (\ref{bis:ex:pfA}), and with the imperfective operator, the reference time is included in the event time, as shown in (\ref{bis:ex:ipfB}) (both taken from \citealt{Paslawska.Stechow2003}: 322).
\ea
\ea{ 
PERFECTIVE${}=\lambda P\lambda t\exists e.τ(e) ⊆ t\wedge P(e)$
\label{bis:ex:pfA}
}
\ex{ 
IMPERFECTIVE${}=\lambda P\lambda t\exists e.t ⊆ τ(e)\wedge P(e)$
\label{bis:ex:ipfB}
}
\z
\z
For predicates with a result state introduced by a prefix, one can add the state variable and the trace function mapping the state to its time, as in (\ref{bis:ex:state}) (taken from \citealt{Biskup2019}: 43).
\ea\label{bis:ex:state}
PERFECTIVE${}=\lambda R\lambda t\exists s\exists e[R(s)(e)\wedgeτ(e) ⊆ t\wedgeτ(e) ⊃⊂ τ(s)]$
\z
The presence of the appropriate operator is tested with the standard diagnostics for perfectivity and imperfectivity, i.e. (in)compatibility with the auxiliary ‘to be’, (im)possibility of the future interpretation of the present form, (in)compatibility with phase verbs and the formation of participles. Note that I follow the two-component approach to aspect and distinguish the morphological (grammatical, outer) aspect from the lexical (situation, inner) aspect.\label{fn:biskup:12}} At the same time, the data suggest that the habitual suffix is higher than superlexical prefixes and the aspectual projection.

The high position of the habitual affix finds support in the fact that the marker can scope over quantificational adverbs, which are very high in the clausal structure; consider the following example.  

\ea\label{bis:ex:zdovolene}
\gll 
Z dovolené ps-á-va-l velmi zřídka. \\ 
from vacation write-\textsc{th}-\textsc{hab}-\textsc{part.m.sg} very rarely\\
\glt ‘It was almost always the case that when he was on vacation, he sent a letter very rarely.’
\hfill (Czech)
\z

\noindent I assume for the time being that the meaning of the habitual marker is ‘to tend to' or ‘almost always’, as shown in the translation in (\ref{bis:ex:zdovolene}). The rationale behind is that the meaning of always is too strong. Given that sentence (\ref{bis:ex:clovek}) is anomalous, the meaning of the habitual marker cannot be ‘always’. That would derive a fully acceptable sentence.

\ea[*]{\label{bis:ex:clovek}
\gll 
Člověk 	bý-vá-∅ smrtelný. \\ 
man be-\textsc{hab}-\textsc{3.sg} mortal\\
\glt ‘Man is almost always mortal.’
\hfill (Czech)}
\z

\noindent Given the high structural position of the habitual marker, the question arises why it is not compatible with the semelfactive \textit{-n}(\textit{V})-, as illustrated in (\ref{bis:ex:kriknuvatHAB}) and (\ref{bis:ex:bodnouvatHAB}). The answer is not complicated. The habitual suffix selects an imperfective predicate but the semelfactive affix derives perfective verbs.

\ea[*]{\gll krik-nu-va-ť \\ 
shout-\textsc{seml}-\textsc{hab}-\textsc{inf}\\
\glt Intended: ‘to shout out repeatedly’
\label{bis:ex:kriknuvatHAB}
\hfill (Russian)}
\ex[*]{\gll bod-nou-va-t\\ 
point-\textsc{seml}-\textsc{hab}-\textsc{inf}\\
\glt Intended: ‘to stab repeatedly’
\label{bis:ex:bodnouvatHAB}
\hfill (Czech)}
\z

\noindent In both languages, the habitual suffixes are identical to the secondary imperfective suffixes. Russian mostly uses the marker \textit{-yva-}/\textit{-iva-}, as in (\ref{bis:ex:pisyvatB}), but the markers \textit{-va-} and \textit{-a-}/\textit{-ja-} can also be found; consider verbs in (\ref{bis:ex:pet}) and (\ref{bis:ex:videt}). These examples again suggest that \textit{-va-} and \textit{-a-} are phonologically conditioned allomorphs.

\ea\label{bis:ex:pet}
\ea\label{bis:ex:petA} \gll 
pe-ť\textsuperscript{IPF} \\ 
sing-\textsc{inf} \\
\glt ‘to sing’
\ex\label{bis:ex:pevatB} \gll pe-va-ť\textsuperscript{IPF} \\
sing-\textsc{hab}-\textsc{inf} \\
\glt ‘to sing repeatedly’\hfill (Russian)
\z
\ex\label{bis:ex:videt}
\ea\label{bis:ex:videtA} \gll 
vid-e-ť\textsuperscript{IPF} \\ 
see-\textsc{th}-\textsc{inf} \\
\glt ‘to see’
\ex\label{bis:ex:vidatB} \gll vid-a-ť\textsuperscript{IPF} \\
see-\textsc{hab}-\textsc{inf} \\
\glt ‘to see repeatedly’\hfill (Russian)
\z
\z

\noindent In Czech, habitual suffixes form a subset of the secondary imperfective markers. Beside \textit{-va-}, there is also its allomorph \textit{-a-}, as in (\ref{bis:ex:jist}), and the marker \textit{-e-}, which is not productive (see \citealt{Petr1986}).

\ea\label{bis:ex:jist}
\ea\label{bis:ex:jistA} \gll 
jís-t\textsuperscript{IPF} \\ 
eat-\textsc{inf} \\
\glt ‘to eat’
\ex\label{bis:ex:jidatB} \gll jíd-a-t\textsuperscript{IPF} \\
eat-\textsc{hab}-\textsc{inf} \\
\glt ‘to eat repeatedly’\hfill (Czech)
\z
\z

\noindent In what follows, I argue that – albeit homophonous – the habitual markers are not secondary imperfective suffixes. First, there are morphological aspect differences. While the imperfectivizing suffix derives an imperfective predicate from a \textit{perfective} verb, the habitual suffix derives an imperfective predicate from an \textit{imperfective} base.

There are also interpretational differences. Secondary imperfective verbs can have the progressive interpretation, the iterative interpretation, the factual and the habitual/generic interpretation. In contrast, predicates with the habitual suffix can only have the habitual/generic interpretation, as demonstrated by the (\textit{repeatedly}) translations in this section. An analogous distinction is observed in cases with iterative adverbs, as in (\ref{bis:ex:zdovolene3}). In sentence (\ref{bis:ex:zdovolene3A}), two interpretations are available: The first, cardinality interpretation has three iterated events of writing during one vacation. The second one is the habitual quantificational interpretation, which is probably stronger than the habitual interpretation of predicates with the overt habitual marker. In contrast, with the habitual suffix, as in (\ref{bis:ex:zdovolene3B}), only the habitual interpretation is available, with \textit{z dovolené psával} going to the restrictor and \textit{třikrát} to the nucleus of the habitual quantifier \textsc{almost always} (or of the standard generic operator).

\ea\label{bis:ex:zdovolene3}
\ea\label{bis:ex:zdovolene3A}
\gll 
Z dovolené ps-a-l třikrát. \\ 
from vacation write-\textsc{th}-\textsc{part.m.sg} three.times\\
\glt ‘From vacation, he sent a letter three times.’
\glt ‘From vacation, he tended to send a letter three times.’
\ex\label{bis:ex:zdovolene3B}
\gll 
Z dovolené ps-á-va-l třikrát. \\ 
from vacation write-\textsc{th}-\textsc{hab}-\textsc{part.m.sg} three.times\\
\glt ‘It was almost always the case that when he was on vacation, he sent a letter three times.’
\hfill (Czech)
\z
\z

\noindent The next argument is based on differences in nominalizations. As already shown by the ungrammatical form *\textit{psávání} in (\ref{bis:ex:psavaniB}), the habitual marker cannot be included in stem nominalizations. However, the secondary imperfective suffix can be a part of such nominalizations, as illustrated in (\ref{bis:ex:vypisovaniB}) (and simplex verbs can also be nominalized, as shown in (\ref{bis:ex:psaniA})).

\ea\label{bis:ex:psani}
\ea\label{bis:ex:psaniA} \gll 
ps-a-n-í \\ 
write-\textsc{th}-\textsc{n/t}-\textsc{nom.sg} \\
\glt ‘writing’
\ex\label{bis:ex:vypisovaniB} \gll vy-pis-ová-n-í \\
out-write-\textsc{si}-\textsc{n/t}-\textsc{nom.sg}\\
\glt ‘excerpting’’\hfill (Czech)
\z
\z

\begin{sloppypar}
\noindent As to phonological properties of the secondary imperfective suffix and the habitual marker, there are many similarities. Both affixes can induce a vowel change, most typically the change from the phoneme /o/ to /a/, which is a relic of the Proto-Indo-European vowel gradation (lengthening, see e.g. \citealt{Nandris.Auty1969}). For the Russian imperfectivizing suffix, consider (\ref{bis:ex:sprosit}) and for the habitual marker, see (\ref{bis:ex:chodit}).\footnote{In the perfective form in (\ref{bis:ex:sprositA}), the phoneme /o/ is reduced and surfaces as the phone [ɐ] given its positioning in the first pretonic syllable.}
\end{sloppypar}

\ea\label{bis:ex:sprosit}
\ea\label{bis:ex:sprositA} \gll 
s-pr\textit{o}s-í-ť\textsuperscript{PF} \\ 
with-ask-\textsc{th}-\textsc{inf} \\
\glt ‘to ask’
\ex\label{bis:ex:sprasivatB} \gll s-pr\textit{á}š-iva-ť\textsuperscript{IPF} \\
with-ask-\textsc{si}-\textsc{inf} \\
\glt ‘to ask’\hfill (Russian)
\z
\z

\ea\label{bis:ex:chodit}
\ea\label{bis:ex:choditA} \gll 
ch\textit{o}d-í-ť\textsuperscript{IPF} \\ 
walk-\textsc{th}-\textsc{inf} \\
\glt ‘to walk’
\ex\label{bis:ex:chazivatB} \gll ch\textit{á}ž-iva-ť\textsuperscript{IPF} \\
walk-\textsc{hab}-\textsc{inf} \\
\glt ‘to walk repeatedly’\hfill (Russian)
\z
\z

\noindent The examples also show that both aspectual morphemes can shift the accent to the root and that the underlying front theme vowel can palatalize the root consonant in the derived forms in (\ref{bis:ex:sprasivatB}) and (\ref{bis:ex:chazivatB}).

Lengthening processes are observed in Czech, too. In (\ref{bis:ex:vydelat}) the imperfectivizing marker -(\textit{v})\textit{a}- lengthens the preceding theme vowel. Similarly, in (\ref{bis:ex:choditCZ}) the habitual marker -(\textit{v})\textit{a}- lengthens the preceding theme  \textit{-i-}. This lengthening also applies in reduplicated form, as already shown in (\ref{bis:ex:psavatB}) and (\ref{bis:ex:psavavat}) by the habitual form \textit{ps-á-va-t} and the reduplicated \textit{ps-á-vá-va-t}, respectively.

\ea\label{bis:ex:vydelat}
\ea\label{bis:ex:vydelatA} \gll 
vy-děl-\textit{a}-t\textsuperscript{PF} \\ 
out-make-\textsc{th}-\textsc{inf} \\
\glt ‘to earn’
\ex\label{bis:ex:vydelavatB} \gll vy-děl-\textit{á}-va-t\textsuperscript{IPF} \\
out-make-\textsc{th}-\textsc{si}-\textsc{inf} \\
\glt ‘to earn’\hfill (Czech)
\z
\z

\ea\label{bis:ex:choditCZ}
\ea\label{bis:ex:choditCZA} \gll 
chod-\textit{i}-t\textsuperscript{IPF} \\ 
walk-\textsc{th}-\textsc{inf} \\
\glt ‘to walk’
\ex\label{bis:ex:chodivatB} \gll chod-\textit{í}-va-t\textsuperscript{IPF} \\
walk-\textsc{th}-\textsc{hab}-\textsc{inf} \\
\glt ‘to walk repeatedly’\hfill (Czech)
\z
\z

\noindent However, there are differences between phonological effects of the two markers. The habitual marker lengthens the preceding vowel but does not induce transitive palatalization in contrast to the secondary imperfective suffix. Consider the following examples, with the root \textit{pros}, which is palatalized by the theme  \textit{-i-} in (\ref{bis:ex:vyprositHABA})--(\ref{bis:ex:vyprosovatHABB}) but is not affected in (\ref{bis:ex:prositC})--(\ref{bis:ex:prosivatD}).

\ea\label{bis:ex:vyprositHAB}
\ea\label{bis:ex:vyprositHABA} \gll vy-\textit{pros}-i-t\textsuperscript{PF} \\ 
out-beg-\textsc{th}-\textsc{inf} \\
\glt ‘to beg’
\ex\label{bis:ex:vyprosovatHABB} \gll vy-\textit{proš}-ova-t\textsuperscript{IPF} \\  
out-beg-\textsc{si}-\textsc{inf} \\
\glt ‘to beg’
\ex\label{bis:ex:prositC} \gll \textit{pros}-i-t\textsuperscript{IPF} \\ 
beg-\textsc{th}-\textsc{inf} \\
\glt ‘to beg’
\ex\label{bis:ex:prosivatD} \gll \textit{pros}-í-va-t\textsuperscript{IPF} \\  
beg-\textsc{si}-\textsc{hab}-\textsc{inf} \\
\glt ‘to beg repeatedly’\hfill (Czech)
\z\z

\noindent This different behavior possibly results from a specific templatic properties of secondary imperfective verbs in Czech, which must weigh three morae without the prefix (see \citealt{Scheer2003,Caha.Scheer2008,Caha.Zikova2016} for templatic properties of Czech verbal forms). In fact, this is what we expect if the imperfectivizing suffix and the habitual marker are two different elements representing distinct pieces of structure that enter into relations with differently complex constituents.

Moreover, the Czech habitual marker does not induce the vowel gradation in the root (with transitive palatalization) in contrast to the imperfectivizing marker. Compare \textit{chod-í-va-t} ‘to walk repeatedly’ from (\ref{bis:ex:chodivatB}) with the Russian \textit{cháž-iva-ť} ‘to walk repeatedly’ in (\ref{bis:ex:chazivatB}) and with (\ref{bis:ex:vytvorit}), which contains the /o/-/a/ alternation induced by the imperfective suffix.

\ea\label{bis:ex:vytvorit}
\ea\label{bis:ex:vytvoritA} \gll 
vy-\textit{tvoř}-i-t\textsuperscript{PF} \\ 
out-make-\textsc{th}-\textsc{inf} \\
\glt ‘to make’
\ex\label{bis:ex:vytvaretB} \gll vy-\textit{tvář}-e-t\textsuperscript{IPF} \\
out-make-\textsc{si}-\textsc{inf} \\
\glt ‘to make’\hfill (Czech)
\z
\z

\noindent Given the differences just discussed, I conclude that the imperfectivizing suffix and the habitual suffix are not identical elements. Yet, there can be one underspecified vocabulary item that spells out both elements, as shown in (\ref{bis:ex:yva}). 

\ea\label{bis:ex:yva}	
\textit{-yva-} ↔ [ipf]
\z

\noindent According to this rule, \textit{-yva-} (which represents allomorphs of the habitual and the imperfectivizing suffix) is inserted into a morphosyntactic context specified as imperfective. That is, \textit{-yva-} can realize the habitual and the imperfectivizing head, which both have the imperfective feature (for more discussion, see \sectref{bis:sec:der}). The syntactic, semantic and phonological differences between the two suffixes then result from the fact that they represent distinct pieces of the morphosyntactic structure and consequently enter into relations with different elements.

To conclude this section, the linearized structure with the four aspectual markers and their morphological aspect effects looks like (\ref{bis:ex:HABstr}). 

% \ea\label{bis:ex:HABstr}	
% [[[SP\textsubscript{higher}[[SP\textsubscript{lower}[\textsubscript{\textit{v}} SEML [LP[√root]\textsuperscript{PF/IPF}]\textsuperscript{PF}]\textsuperscript{PF}]\textsuperscript{PF}SI]\textsuperscript{IPF}]\textsuperscript{PF}Asp] HAB]\textsuperscript{IPF}
% \z

\ea\label{bis:ex:HABstr}	
[[[SP\textsubscript{higher} [[SP\textsubscript{lower} [\textsubscript{\textit{v}} SEML [LP [√root]\textsuperscript{PF/IPF}]\textsuperscript{PF}]\textsuperscript{PF}]\textsuperscript{PF} SI]\textsuperscript{IPF}]\textsuperscript{PF} Asp] HAB]\textsuperscript{IPF}\medskip\\
\z

\noindent Note that it is an overall picture that does not take into account selection properties and particular incompatibilities of the markers. 

\section{Aspect separated from the four aspectual markers}\label{bis:sec:asp} 
We have seen that the aspectual interpretation is determined by several elements, which can have opposite aspectual effects (perfective versus imperfective). The discussion of the four markers and their morphological aspect effects showed that the morphological aspect value of a predicate can change in the course of its derivation. That is, each new aspect marker adds a new aspect layer to the preceding derivation that covers the preceding aspect values. Recall that we have seen that the morphological aspect is determined by the last attached aspectual morpheme. I will call it \textit{Morphological Aspect Generalization} (MAG); consider (\ref{bis:ex:mag}).

\ea\label{bis:ex:mag}	
\textit{Morphological Aspect Generalization}

The morphological aspect is determined by the last attached aspectual morpheme.
\z

\noindent I also showed that in certain cases aspectual markers do not change the morphological aspect interpretation. These facts are not new; see e.g. \citet{Karcevski1927}, \citet{Isacenko1962}, \citet{Zinova.Filip2015} and \citet{Tatevosov2020}. Given these facts, we need a mechanism that can \textit{inspect} all the relevant aspectual morphemes and can determine which of them is the final one.

The ideal candidate is the operation Agree. Given that it can establish a relation between the probe and the goal at a distance, it is suitable for cases where the interpretation is separated from the element that triggers it.

\citet{Tatevosov2011} argues that prefixes are not morphological exponents of the perfective aspect. His argument is based on the fact that Russian stem nominalizations are aspectless although they are formed from prefixed stems. In other words, if prefixes were not dissociated from the perfective meaning, Russian \textit{-nie} nominals would have to be interpreted as perfective. According to \citet{Pazelskaya.Tatevosov2008} and \citet{Tatevosov2011}, Russian stem nominalizations include the projection with the secondary imperfective suffix at the most. As discussed in \sectref{bis:sec:hab}, Czech stem nominalizations also contain higher superlexical prefixes and the aspectual projection. Thus, the structures of the two languages differ in the presence/absence of higher superlexicals and the aspectual projection (i.e. the presence/absence of the aspectual interpretation), as shown in my notation in (\ref{bis:ex:nomR}) and (\ref{bis:ex:nomCZ}).

\ea\label{bis:ex:nomR}	
[[[[SP\textsubscript{lower} [\textsubscript{\textit{v}} SEML [LP [√root]\textsuperscript{PF/IPF}]\textsuperscript{PF}]\textsuperscript{PF}]\textsuperscript{PF} SI]\textsuperscript{IPF} N/T] \textit{n}]
\hfill (Russian)
\z
\ea\label{bis:ex:nomCZ}	
[[[[SP\textsubscript{higher} [[SP\textsubscript{lower} [\textsubscript{\textit{v}} SEML [LP [√root]\textsuperscript{PF/IPF}]\textsuperscript{PF}]\textsuperscript{PF}]\textsuperscript{PF} SI]\textsuperscript{IPF}]\textsuperscript{PF} Asp] N/T] \textit{n}]
\hfill (Czech)
\z

\noindent Now I will extend the separation argument to the semelfactive marker. Since Russian nominalizations generally disallow the presence of the semelfactive -\textit{n}(\textit{V})- and Czech stem nominalizations (with or without SEML) always have the morphological aspect, we cannot construct a direct argument with aspectless nominals containing the semelfactive -\textit{n}(\textit{V})-.\footnote{The question of exactly how the presence of Asp licenses the presence of the semelfactive marker in Czech, I leave for future research.} Recall that I argued in \sectref{bis:sec:seml} that the semelfactive suffix spells out the verbalizing head \textit{v}, as do other theme elements; consider (\ref{bis:ex:nomR}) and (\ref{bis:ex:nomCZ}) again. Given this and the fact that the aspectual projection occurs outside the projection with the imperfectivizing suffix (and also higher than projections with the \textit{-n-}/\textit{-t-} suffix and the nominalizing suffix in Russian, as shown in (\ref{bis:ex:nomR})), it is obvious that the semelfactive marker is separated from the perfective aspect. Below I will show that the semelfactive marker is also separated from the aspectual projection by the projection of Voice, which introduces the agent argument.

Note that it would not be reasonable to postulate another aspectual projection with the perfective interpretation specific to the semelfactive -\textit{n}(\textit{V})- because of the reason of language economy and because of universality of the clausal hierarchy. Moreover, given that the perfectivity effect of the semelfactive -\textit{n}(\textit{V})- is real – see the periphrastic future test in (\ref{bis:ex:budet}) and (\ref{bis:ex:bude}) – the analysis of the semelfactive marker cannot be based only on its inner aspect properties.

\ea\label{bis:ex:budet}
\ea[]{
\gll 
budet krič-a-ť\textsuperscript{IPF} \\
will shout-\textsc{th}-\textsc{inf} \\
\glt ‘it/(s)he will shout’ \label{bis:ex:budetA}}
\ex[*]{\gll 
budet krik-nu-ť\textsuperscript{PF} \\
will shout-\textsc{seml}-\textsc{inf} \\
\glt Intended: ‘it/(s)he will shout out’
\label{bis:ex:budetB}
\hfill (Russian)}
\z
\z

\ea\label{bis:ex:bude}
\ea[]{
\gll 
bude bod-a-t\textsuperscript{IPF} \\
will point-\textsc{th}-\textsc{inf} \\
\glt ‘it/(s)he will stab’ \label{bis:ex:budeA}}
\ex[*]{\gll 
bude bod-nou-t\textsuperscript{PF} \\
will point-\textsc{seml}-\textsc{inf} \\
\glt Intended: ‘it/(s)he will stab’
\label{bis:ex:budeB}
\hfill (Czech)}
\z
\z

\noindent \citet{Romanova2004}, \citet{Tatevosov2015Slavic} and \citet{Mueller-Reichau2020} argue for Russian that the imperfectivizing suffix merges inside the verbal domain. Thus, the secondary imperfective marker, too, is dissociated from its interpretation because the aspectual head responsible for the imperfective interpretation is located in a higher position above \textit{v}P. According to \citet{Biskup2020} -- who uses a scope argument like the one in \citet{Tatevosov2015Slavic} -- scope facts with the Czech cumulative \textit{na-} also suggest that the position of the imperfectivizing suffix is below the projection with the agentive argument. The same point can be done with the distributive prefix \textit{po-}.\largerpage[1]

Concretely, cumulative \textit{na-} and distributive \textit{po-} can quantify over an object, as shown by the grammatical plural (non-quantized) object in (\ref{bis:ex:posbiratA}). The ungrammaticality of the quantized, singular object \textit{jablko} ‘apple’ shows that the prefix \textit{na-} is indeed cumulative and the prefix \textit{po-} distributive. In contrast, the prefixes cannot quantify over an agentive subject, as demonstrated in (\ref{bis:ex:posbiratB}), where the plural subject is ungrammatical. Only if the object is plural, non-quantized, the sentence is grammatical, as demonstrated in (\ref{bis:ex:posbiratC}). This goes hand in hand with the fact that when we want to quantify over the agentive subject, the argument structure (including case properties) of the verb needs to be manipulated and the reflexive element must be added in the case of the cumulative \textit{na-}, as shown in (\ref{bis:ex:posbiratD}).\footnote{Also compare the following examples with ‘self’ and the cumulative/saturative \textit{na-}, which can quantify over the subject.
\ea
\ea[]{ 
\gll 
na-begat’-sja \\ 
on-run-self \\
\glt ‘to have one’s fill of running’
\hfill (Russian)
\label{bis:ex:nabegatsjaA}
}
\ex[]{ 
\gll na-běhat se \\ 
on-run self \\
\glt ‘to have one’s fill of running’
\label{bis:ex:nabehatseB}
\hfill (Czech)
}
\z
\z
}

\ea\label{bis:ex:posbirat}
\ea[]{
\gll 
po-/na-s-bír-a-t\textsuperscript{PF} \minsp{\{} jablka	/  \minsp{*} jablko\} \\ 
\textsc{dist-}/\textsc{cum}-with-take-\textsc{si}-\textsc{inf}	{} apples {} {} apple \\
\glt distributive: 	‘to pick apples/*apple one after another’\\
cumulative: 	‘to pick amount of apples/*apple’\label{bis:ex:posbiratA}}
\ex[*]{
\gll 
Sousedi po-/na-sbírali jablko. \\ 
neighbors	\textsc{dist-}/\textsc{cum}-picked	apple \\
\glt Intended distributive: ‘Neighbors one after another picked an apple.’\\
Intended cumulative: ‘Amount of neighbors picked an apple.’
\label{bis:ex:posbiratB}}
\ex[]{\gll
Sousedi po-/na-sbírali jablka. \\ 
neighbors	\textsc{dist-}/\textsc{cum}-picked	apples \\
\glt distributive: ‘Neighbors picked apples one after another.’\\
cumulative: ‘Neighbors picked amount of apples.’\label{bis:ex:posbiratC}}
\ex[]{\gll
Sousedi se nasbírali jablek do sytosti. \\
neighbors self picked apples.\textsc{gen.pl} to one’s.fill \\
\glt ‘Neighbors had their fill of picking apples.’\hfill (Czech)\label{bis:ex:posbiratD}}

\z
\z

\noindent Given that the perfective \textit{nasbírat} is derived by attaching the cumulative \textit{na-} and the distributive \textit{po-} to the stem after the secondary imperfective suffix, the example suggests that higher superlexical prefixes like the cumulative \textit{na-} and the distributive \textit{po-} merge below the head introducing the agent and above the imperfectivizing suffix in Czech. Consequently, in the light of the fact that the aspectual projection is above the projection introducing the agent (e.g. \citealt{Babko-Malaya2003,Filip2005,Blaszczak.Klimek-Jankowska2012,Gribanova2015}), it is possible to conclude that the imperfective interpretation is separated from the imperfectivizing suffix.

At the same time, if it is correct that higher superlexical prefixes merge below the projection with the agent (VoiceP), we also have an argument for separating prefixes from the perfective interpretation occurring in the aspectual projection.

The following examples show that stem nominalizations like the Russian \textit{-nie} nominals and the Czech \textit{-ní} nominals can have an agent. The nominals can co-occur with an agent-oriented modifier, as in (\ref{bis:ex:umyslennojeA}) and (\ref{bis:ex:umyslneA}), and can be modified by an agentive \textit{by}-phase, as shown in (\ref{bis:ex:soversenieB}) and (\ref{bis:ex:spachaniB}). 

\ea\label{bis:ex:umyslennoje}
\ea\label{bis:ex:umyslennojeA}
\gll 
umyšlennoe	prestuplenie \\
deliberate		delict \\
\glt ‘a wilful delict’ 
\ex\label{bis:ex:soversenieB}
\gll 
soveršenie   		prestuplenija 	licom… \\
perpetration		delict.\textsc{gen.sg} person.\textsc{instr.sg} \\
\glt ‘a perpetration of the delict by a person’	
\hfill (Russian)
\z
\z

\ea\label{bis:ex:umyslne}
\ea\label{bis:ex:umyslneA}
\gll 
úmyslné		poškození \\
deliberate	damage \\
\glt ‘a malicious damage’ 
\ex\label{bis:ex:spachaniB}
\gll 
spáchání   			trestného 				činu 				osobou… \\
perpetration		criminal.\textsc{gen.sg} act.\textsc{gen.sg} person.\textsc{instr.sg} \\
\glt ‘a perpetration of the delict by a person’	
\hfill (Czech)
\z
\z

\noindent Now let us combine it with the fact that Russian stem nominalizations are aspectless (as discussed in \sectref{bis:sec:hab}). Applying the containment argument again, we conclude that (at least in Russian) the aspectual projection is indeed above VoiceP, as shown in (\ref{bis:ex:ASPstr}). 

\ea\label{bis:ex:ASPstr}	
[[[SP\textsubscript{higher} [[SP\textsubscript{lower} [\textsubscript{\textit{v}} SEML [LP [√root]\textsuperscript{PF/IPF}]\textsuperscript{PF}]\textsuperscript{PF}]\textsuperscript{PF}SI]\textsuperscript{IPF}]\textsuperscript{PF} Voice] Asp]
\medskip\\
\z

\noindent \citet{Kwapiszewski2021} argues for the position of the secondary imperfective suffix below Voice and in this way also for separating the imperfectivizing suffix from the morphological aspect in Polish. He builds on \citet{Baker.Vinokurova2009} and draws a parallelism between English nominals in \textit{-er} and Polish agent/instrument \textit{-acz}/\textit{-arka} nominals. He shows that Polish \textit{-acz}/\textit{-arka} nominalizations can contain the imperfectivizing suffix but do not embed the Voice projection since they do not allow the relevant modifiers.

The same argument can be done for the Czech counterpart: \textit{-č} nominals (Russian does not have this form of nominals). The animate as well as the inanimate nominal contain the imperfectivizing suffix but do not allow agent-oriented modifiers, as demonstrated in (\ref{bis:ex:umyslny}).

\ea\label{bis:ex:umyslny}
\ea\label{bis:ex:umyslnyA}
\gll 
\minsp{(*} úmyslný)	vy-jedn-a-va-č (*, aby zabránil válce) \\
{} deliberate out-one-\textsc{th}-\textsc{si}-\textsc{nmlz} {} so.that	prevent war \\
\glt Intended: 'someone who (deliberately) negotiates (in order to avoid a war)' 
\ex\label{bis:ex:ovladacB}
\gll 
o-vlad-a-č \minsp{(*} osobou)	 \minsp{(*} s 			cílem 	měnit 	programy) \\
about-rule-\textsc{si}-\textsc{nmlz} {} person.\textsc{instr.sg} {} with	goal		switch	channels \\
\glt Intended: ‘a remote control (used by a person) (for switching TV channels)’	
\hfill (Czech)
\z
\z

\noindent Thus, in Czech, too, such nominalizations include the projection with the secondary imperfective suffix but are structurally smaller than VoiceP and by transitivity, also smaller than AspP. Beside separating the imperfective suffix from the imperfective interpretation, it also argues for the claim that prefixes are separated from the perfective interpretation in the aspectual projection. Because of the presence of the imperfectivizing suffix, at least lexical and lower superlexical prefixes are expected to be able to occur in this type of nominalizations. This seems to be correct, given the prefixed examples in (\ref{bis:ex:umyslny}).

If \citet{Baker.Vinokurova2009} are correct in that agentive nominalizing morphemes like \textit{-er} are nominal versions of the Voice head (having meanings similar to morphemes of Voice heads) that combine with the same complements as Voice does, then the order of the morphemes itself can be taken to mean that the projection of Voice is higher than the projection of the secondary imperfective suffix. The point is that the imperfectivizing suffix is always closer to the root than the agentive nominalizing morpheme.

It is possible to extend this reasoning to other agent nominalizations, e.g. to nominals ending in \textit{-tel’} in Russian, \textit{-tel} in Czech and \textit{-ciel} in Polish and to Russian nominals with the suffixes -(\textit{l’})\textit{ščik} and \textit{-čik}, which are counterparts of the Czech \textit{-č} discussed above. Such agent nominalizations can contain the imperfectivizing suffix and the suffix is always closer to the root than the agentive morpheme, independently of whether the nominal is inanimate (instrument), as in (\ref{bis:ex:peregruzatelA}), or animate, as in examples (\ref{bis:ex:rassevalscikB}) and (\ref{bis:ex:osetrovatel}).

\ea\label{bis:ex:peregruzatel}
\ea\label{bis:ex:peregruzatelA}
\gll 
pere-gruž-a-tel’ \\
over-load-\textsc{si}-\textsc{nmlz} \\
\glt ‘a loader’ 
\ex\label{bis:ex:rassevalscikB}
\gll 
ras-se-va-l’ščik \\
apart-sow-\textsc{si}-\textsc{nmlz} \\
\glt ‘a sorter’	
\hfill (Russian)
\z
\z

\ea\label{bis:ex:osetrovatel}
\gll 
o-šetř-ova-tel \\
about-spare-\textsc{si}-\textsc{nmlz} \\
\glt ‘a keeper’ 
\hfill (Czech)
\z

\noindent The consequences for dissociating prefixes and the secondary imperfective suffix from the corresponding morphological aspect interpretation are identical to those in the case of \textit{-acz}/\textit{-arka} and \textit{-č} nominalizations discussed above.

The current analysis with AspP above VoiceP, as discussed wrt. (\ref{bis:ex:ASPstr}), goes against analyses like \citet[571, 585]{Zdziebko2017}, who argues that in Polish, the agentive VoiceP is placed above the aspectual projection(s). According to a reviewer, data like (\ref{bis:ex:ta}) suggest that in Polish, VoiceP is also higher than HabP since the habitual \textit{-yw-} is inside the passive \textit{-n-}.

\ea\label{bis:ex:ta}
\gll 
Ta 		melodia 	jest / była 	grywana 		w 	wielu 	rozgłośniach radiowych. \\ 
this 	melody 	is {} was 		played.\textsc{hab}	in 	many stations radio. \\
\glt `This 	melody 	is/was 		played	in 	many 	radio stations.'
\hfill (Polish)
\z

\noindent However, I assume that \textit{-n-} in fact projects a participial phrase, as in \citet{Biskup2016} and \citet[Chapter 4]{Biskup2019}. PartP then includes HabP. An argument for HabP above VoiceP could be based on the fact that stem nominalizations can be agentive but cannot contain the habitual morpheme, like the Russian *\textit{pisyvanie} ‘writing’ and the Czech *\textit{psávání} ‘writing’ in (\ref{bis:ex:psavaniB}). Since Polish habitual nominalizations like \textit{pisywanie} ‘writing’ are grammatical, they can also contain HabP.\footnote{In addition, given the reasoning in \sectref{bis:sec:hab} that HabP is above AspP, the ‘be’ auxiliary in constructions like (\ref{bis:ex:jan}) cannot be placed in AspP, contrary to \citet{Blaszczak.Klimek-Jankowska2012} and \citet{Blaszczak.etal2014}. As to the Russian habitual \textit{igryvať} ‘to play repeatedly’, it is standardly claimed that such forms are colloquial and used only in the past tense (see \sectref{bis:sec:hab} again).
\ea\label{bis:ex:jan}
\ea[]{ 
\gll 
Jan 	będzie 	grywać 		w 	różnych 		lokalach 	w 	Londynie.  \\
Jan 	will 		play.\textsc{hab} 	in 	different 	pubs 			in 	London. \\
\hfill (Polish)
\label{bis:ex:janA}
}
\ex[]{ 
\gll
Jan	bude		hrávat			v		různých		hospodách	v		Londýně. \\
Jan 	will 		play.\textsc{hab} 	in 	different 	pubs 			in 	London. \\
\glt Both: ‘Jan 	will 		play in London in various pubs.’
\hfill (Czech)
\label{bis:ex:janB}
}
\z
\z
}

Since the nominalizations under discussion typically refer to an instrument or an agent repeatedly performing the event expressed by the verb stem (they often contain the imperfectivizing morpheme, as in (\ref{bis:ex:umyslny}--\ref{bis:ex:osetrovatel})), they are incompatible with the semelfactive suffix. Specifically, they conflict with the \textit{cardinality one} property of the semelfactive morpheme, as defined in (\ref{bis:ex:seml}).

The next structural prediction is that the nominalizations under discussion cannot include the habitual marker for it is located above the aspectual projection. This prediction seems to be correct since e.g. the Czech National Corpus, SYN 8 \citep{CNK} contains no agent nominalization that have the habitual marker and ends in \textit{-vatel}.

Let us now consider the separation of the morphological aspect interpretation from the habitual marker. The habitual suffix is special. First, in contrast to the other aspectual markers, it occurs above the aspectual projection, as argued in \sectref{bis:sec:hab}. Second, in contrast to the other markers, it does not reverse the morphological aspect value of the predicate to which it adjoins. Because of the second property, it in actuality does not have to be in a syntactic relation with the aspectual head. It suffices when it imposes the imperfective requirement on its complement. Moreover, given this selection property and the specific quantificational meaning of the marker, the habitual suffix can be treated as semantically independent from the aspectual head, which encodes the inclusiveness relation between the event time and the reference time.\footnote{For the specific aspectual operators, see footnote~\ref{fn:biskup:12}.} Furthermore, since there are forms with the morphological aspect interpretation that exclude the habitual marker – recall the Czech stem nominalizations from \sectref{bis:sec:hab} –, I conclude that the habitual marker can be separated from the aspectual phrase as well.

\section{Deriving the morphological aspect value}\label{bis:sec:der} 

As stated in the beginning of the preceding section, the operation Agree is very suitable for cases where a certain interpretation is separated from the element bringing out the interpretational effect. In our case, it is about perfective versus imperfective effects triggered by the four aspectual markers. For this reason, we need an interpretable unvalued aspect feature on the aspectual head and valued features on the aspectual markers. The feature on the aspectual markers (either perfective or imperfective) can value the unvalued feature on the head Asp and in this way, it can bring about the appropriate inclusiveness relation between the event time and the reference time.

In the current proposal, I follow the Agree analysis by \citet{Biskup2020} and assume that the secondary imperfective marker has an uninterpretable aspect feature with the imperfective value (recall the imperfectivizing effect of this suffix from \sectref{bis:sec:si}). In contrast, since prefixes perfectivize the base predicate, as we saw in \sectref{bis:sec:pref}, they bear an uninterpretable aspect feature with the perfective value. The same also holds for the semelfactive marker because it also has the perfective effect, as discussed in \sectref{bis:sec:seml}. With respect to the habitual head, I argued in the preceding section that it has an imperfective selection feature and that it does not have to enter into an Agree relation with the aspectual head. However, the habitual head bears the imperfective aspect feature, which ensures that the marker \textit{-yva-} can spell out it in accordance with the rule (\ref{bis:ex:yva}).

If we make the standard assumption that lexical prefixes merge in the complement position of the root (e.g. \citealt{Ramchand2004,Svenonius2004b,Gehrke2008b,Biskup2019}), then the (non-linearized) hierarchy with the four aspect markers and their aspect features looks like (\ref{bis:ex:ASPFstr}).

\ea\label{bis:ex:ASPFstr}	
[\textsubscript{HabP }HAB\textsubscript{ipf} [\textsubscript{AspP} Asp\textsubscript{asp-F:[ ]}[\textsubscript{VoiceP} Voice [\textsubscript{SPP} SP\textsubscript{pf} [\textsubscript{SIP} SI\textsubscript{ipf} [\textsubscript{SPP} SP\textsubscript{pf} [\textsubscript{\textit{v}P} SEML\textsubscript{pf} [\textsubscript{√P} √ [\textsubscript{PP} LP\textsubscript{pf}]]]]]]]]]
\medskip\\
\z

\noindent Assuming that morphemes are structurally heads, lexical prefixes head a prepositional phrase, the semelfactive marker heads the \textit{v}P projection, superlexical prefixes head their own projection SPP and the habitual suffix heads the habitual projection.  Superlexical projections can be iterated and occur either lower or higher than the projection of the imperfectivizing morpheme SIP.

The Agree analysis can successfully deal with the generalization MAG, that is, with the fact that the morphological aspect value is determined by the last attached aspectual morpheme. Specifically, using the standard operation of downward Agree, the last – structurally, the highest – aspectual marker can be determined on the basis of minimality, i.e. the structural distance from the probing aspectual head. The aspect feature of the closest marker will then value the unvalued aspect feature of the aspect head. Since only downward Agree is used, the habitual marker – occurring in a higher structural position – is not visible for the probing aspectual head. This, however, does not pose a problem because the marker cannot change the morphological aspect value, as already discussed above.

If it is correct that the verb moves to the head Asp, as argued by \citet{Gribanova2013,Gribanova2015} for Russian, we receive the syntactic structure in \figref{bis:ex:synstr}. Concretely, when the unvalued feature of the aspectual head probes, the complex verbal head is located in Voice. To determine the closeness of aspectual affixes and their features, I employ the concept of dominance. It is the head to which the moving element adjoins that projects, as demonstrated in the abstract structure in \figref{bis:ex:synstr}. Since this head dominates the adjoined head, its features (among others, its valued aspect feature) are closer to the c-commanding aspectual head than the features of the adjoined head.

\begin{figure} 
\caption{The derivation of the perfective morphological aspect}
\label{bis:ex:synstr}
\begin{forest}
[Asp$'$
    [Asp\textsubscript{{[}asp-F: pf{]}},name=asphead]
    [VoiceP
        [DP]
        [Voice$'$
            [Voice
                [SP\textsubscript{higher[pf]}, name=SPhead
                    [SI\textsubscript{[ipf]}
                      [SP\textsubscript{lower[pf]}
                        [\textit{v}
                            [√
                                  [P\textsubscript{[pf]}]
                                [√] 
                            ]
                            [\textit{v}]
                        ]
                        [SP\textsubscript{lower[pf]}]
                      ]
                      [SI\textsubscript{[ipf]}]
                    ]
                    [SP\textsubscript{higher[pf]}]
                ]
                [Voice]
            ]
            [SP\textsubscript{higher}P [t\textsubscript{SP\textsubscript{higher}}]
                [SIP
                    [t\textsubscript{SI}]
                    [SP\textsubscript{lower}P
                        [t\textsubscript{SP\textsubscript{lower}}]
                    [\textit{v}P
                    [t\textsubscript{\textit{v}}]
                        [√P
                            [t\textsubscript{√}]
                           [PP
                           [t\textsubscript{P}, roof]
                            ]
                        ]
                    ]
                    ]
                ]
            ]
        ]
    ]
]
\draw[Circle-Circle] (asphead) to[out=220,in=180] node[pos=0.5,left, xshift=-3mm]{Agree} (SPhead);
\end{forest}
\end{figure}

The complex Voice head in \figref{bis:ex:synstr} contains the following markers with their aspect features: a lexical prefix (the preposition), a lower superlexical prefix, the secondary imperfective suffix and a higher superlexical prefix. Therefore, the structure can represent predicates like the Russian \textit{po-pere-za-pis-yva-ť} ‘to re-record for a while’. The delimitative prefix \textit{po-} merges in the higher superlexical position and the repetitive \textit{pere-} merges in the lower superlexical position, i.e. below the secondary imperfective suffix \textit{-yva-}. The lexical prefix \textit{za-} is represented by the preposition in \figref{bis:ex:synstr}. What is crucial here, is that the delimitative \textit{po-} projects its perfective feature and dominates the SI constituent headed by \textit{-yva-} with its imperfective aspect feature. Hence, it is the perfective feature of the delimitative \textit{po-} that is the closest aspect feature and values the unvalued aspect feature on Asp. Consequently, the predicate is interpreted as perfective.

Nothing changes on the result, if the lower superlexical prefix is missing like in the perfective Russian example \textit{po-vy-talk-iva-ť} ‘to push out one after another’ from \sectref{bis:sec:si}. The distributive \textit{po-}, with its perfective aspect feature, spells out the higher SP in \figref{bis:ex:synstr} and it is again the closest element to the aspectual head.

In contrast, if a single superlexical prefix merges in the lower SP position like in the Czech predicate \textit{do-plét-a-t} ‘to complete knitting’ in (\ref{bis:ex:dopletatB}), the imperfective feature of the imperfectivizing suffix will be the closest aspect feature to Asp. Consequently, the imperfective operator will be used for the aspectual head.

It is obvious from the discussion that there can be aspectual markers with valued, uninterpretable aspect features that do not enter into an Agree relation (recall also the habitual head, which is not c-commanded by the probing Asp and bears a valued, uninterpretable imperfective feature). To cope with this issue, I assume that for the semantic interface, only unvalued features (but not uninterpretable features) are offending. Concretely, the uninterpretable property of a feature just signals that the feature should not be interpreted at the semantic interface (cf. \citealt{Zeijlstra2009}). In other words, the interpretable versus uninterpretable property can indicate where (i.e. which occurrence of) the feature should be interpreted in the structure.

In the case of predicates containing a lexical prefix and the imperfectivizing suffix like the Russian \textit{za-rabat-yva-ť} ‘to earn’ in (\ref{bis:ex:zarabatyvatB}) and the Czech \textit{vy-proš-ova-t} ‘to beg’ in (\ref{bis:ex:vyprosovatB}), we also receive the imperfective aspect because the mother SI node, with its imperfective feature, unambiguously dominates the P element (lexical prefix); consider the structure in \figref{bis:ex:synstr} again.

If only a lexical prefix attaches to the predicate, as in \textit{na-kle-i-ť} ‘to stick on’ in (\ref{bis:ex:nakleitB}) and \textit{vy-chov-a-t} ‘to raise’ in (\ref{bis:ex:vychovatB}), the aspectual head probes the whole way down in the complex Voice head and finally finds the only available aspect feature on P. This brings about the perfective interpretation. Obviously, the same result is obtained if a superlexical prefix is added to the lexical one, as in the Russian \textit{pere-vy-poln-i-ť} ‘to overfulfill’ in (\ref{bis:ex:perevypolnitA}) and the Czech \textit{pře-vy-chov-a-t} ‘to re-educate’ in (\ref{bis:ex:prevychovatA}). There, however, it is the perfective feature of the superlexical prefix that values the aspectual head.

Since lexical prefixes merge in the complement of the root and then adjoin to it, it must be the root that projects its features in the complex verbal head. From this and the fact that lexical prefixes perfectivize the base predicate, it follows that the root cannot have an imperfective aspect feature. For this reason, I assume that the morphological aspect of simplex verbs is derived by a default mechanism. Specifically, if the probing aspectual head does not find an aspect feature in its c-command domain, it will receive the imperfective aspect value when it is sent to the interfaces (see \citealt{Preminger2014} for the claim that the operation Agree can fail). Note that this proposal is in line with the standard approach to Slavic aspect, which takes imperfectivity to be the default aspect value (see e.g. \citealt{Jakobson1932,Jakobson1956,Comrie1976,Nubler.etal2017}). As to the root of the exceptional perfective simplex predicates like the Rusian and Czech \textit{kupiť}/\textit{koupit} ‘to buy’ and \textit{dať}/\textit{dát} ‘to give’, it bears a perfective feature, which is found by the probing aspectual head. Concerning bi-aspectual verbs, I assume that their root can optionally have the perfective feature (in addition to applying the default mechanism resulting in imperfectivity) until the aspect value of the predicate is settled.

With respect to the semelfactive marker, it was shown in \sectref{bis:sec:seml} that the suffix combines with prefixes but does not co-occur with the secondary imperfective suffix and the habitual marker. Given that the semelfactive marker also bears an aspect feature and spells out the verbalizing head \textit{v}, its perfective feature will value the aspect feature of Asp in the case of lexically prefixed predicates like the Russian \textit{vs-krik-nu-ť} ‘to give a scream’ in (\ref{bis:ex:vskriknut}) and, of course, in the case of unprefixed semelfactive verbs like \textit{krik-nu-ť} ‘to shout out’ in (\ref{bis:ex:kriknutB}), which were discussed in \sectref{bis:sec:seml}.

On the contrary, in the case of superlexically prefixed semelfactive verbs like the Czech \textit{na-prask-nou-t} ‘to crack partially’ in (\ref{bis:ex:naprasknout}), it will be the perfective feature of the superlexical prefix that values the aspectual head (independently of whether it is a lower or a higher superlexical prefix) since any SP projected by a superlexical prefix always dominates \textit{v}.

As discussed in sections \sectref{bis:sec:seml} and \sectref{bis:sec:hab}, Russian and Czech stem nominalizations differ in the complexity of their structure, specifically, in the presence or absence of higher superlexical prefixes and the aspectual projection. In the case of Czech \textit{-ní} nominals – which can contain higher superlexicals and have the morphological aspect – the morphological aspect value on the aspectual head will be derived as described above. In the case of Russian \textit{-nie} nominals there is no Agree operation because they are aspectless and include the projection with the imperfectivizing marker at the most, plus the projection with the suffix \textit{-n-}/\textit{-t-} and the nominalizing projection \textit{n}P; see (\ref{bis:ex:nomR}) again.  Here, the assumption that the uninterpretability of features just signals whether or not the appropriate (instance of the) feature should be interpreted at the semantic interface is applicable. This reasoning applies to all forms that lack the aspectual projection but contain an aspectual marker with an aspect feature, e.g. to the root nominalizations discussed in \sectref{bis:sec:seml}, which can include a lexical prefix.

The proposal in \figref{bis:ex:synstr} derives the correct order for all morphemes except superlexical prefixes. Given that prefixes display a peculiar behavior more generally, I assume that they also have weak prosodic properties which force them to linearize to the left (see e.g. \citealt{Caha.Zikova2016}, who argue for a proclitic character of short verbal prefixes in Czech, and \citealt{Biskup.etal2011}, who discuss differences between prefixed verbs and particle verbs in German and argue that in prefixed verbs the prepositional phonological word is weak in contrast to particle verbs).

\section{Conclusions}\label{bis:sec:concl} 
I have argued that the four aspectual morphemes (prefixes, the secondary imperfective suffix, the semelfactive marker and the habitual suffix) are not exponents of the morphological aspect in Russian and Czech; they just work as a trigger of the corresponding aspectual interpretation. However, this is not to say that the aspectual markers are meaningless. They have their own meaning, which can be inner aspectual, as proposed e.g. for the semelfactive suffix in \sectref{bis:sec:seml}. I have shown that the morphological aspect value is determined by the last attached aspectual marker. The aspect value, I have derived by means of the operation Agree, using the concept of closeness based on dominance relations in the moved verbal head. The last-attached aspectual marker is the closest element with a valued aspect feature.


\section*{Abbreviations}

\begin{tabularx}{.5\textwidth}{@{}lQ}
\textsc{acc}&accusative\\
\textsc{att}&attenuative\\
\textsc{comp}&completive\\
\textsc{cum}&cumulative\\
\textsc{da}&degree achievement\\
\textsc{del}&delimitative\\
\textsc{dist}&distributive\\
\textsc{exc}&excessive\\
\textsc{hab}&habitual\\
\textsc{inc}&inceptive\\
\textsc{inf}&infinitive\\
\end{tabularx}%
\begin{tabularx}{.5\textwidth}{lQ@{}}
\textsc{ipf}&imperfective\\
\textsc{lp}&lexical prefix\\
\textsc{nmlz}&nominalizing affix\\
\textsc{nom}&nominative\\
\textsc{part}&participle\\
\textsc{pf}&perfective\\
\textsc{rep}&repetitive\\
\textsc{seml}&semelfactive\\
\textsc{si}&secondary imperfective\\
\textsc{sp}&superlexical prefix\\
\textsc{th}&theme (vowel)\\
% &\\ % this dummy row achieves correct vertical alignment of both tables
\end{tabularx}

\section*{Acknowledgments}
Funded by the Deutsche Forschungsgemeinschaft (DFG, German Research Foundation) – Project-ID 498343796. I would also like to thank reviewers and the audience of the FDSL-14 conference for their helpful comments.\largerpage[2]


\printbibliography[heading=subbibliography,notkeyword=this]

\end{document}

