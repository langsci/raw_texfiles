\documentclass[output=paper,colorlinks,citecolor=brown]{langscibook}
\ChapterDOI{10.5281/zenodo.10123641}

\author{Ljudmila Geist\orcid{0000-0001-7907-4958}\affiliation{University of Stuttgart} and Sophie Repp\orcid{0000-0003-1575-4553}\affiliation{University of Cologne}}
% replace the above with you and your coauthors
% rules for affiliation: If there's an official English version, use that (find out on the official website of the university); if not, use the original
% orcid doesn't appear printed; it's metainformation used for later indexing

%%% uncomment the following line if you are a single author or all authors have the same affiliation
% \SetupAffiliations{mark style=none}

%% in case the running head with authors exceeds one line (which is the case in this example document), use one of the following methods to turn it into a single line; otherwise comment the line below out with % and ignore it
% \lehead{Šimík, Gehrke, Lenertová, Meyer, Szucsich \& Zaleska}
% \lehead{Radek Šimík et al.}

\title{Responding to negative biased questions in Russian}
% replace the above with your paper title
%%% provide a shorter version of your title in case it doesn't fit a single line in the running head
% in this form: \title[short title]{full title}
\abstract{The paper investigates polar responses to biased questions with outer vs. inner negation and the particle \textit{razve} ‘really' in Russian. We present experimental evidence from two acceptability judgment studies and show that the two question types have slightly different answer patterns. We argue that the meaning previously suggested for the particles \textit{da/net} ‘yes/no' must be revised. We propose an analysis of our results which combines a proposal for outer vs. inner negation in terms of the illocutionary operator \textsc{falsum} vs. propositional negation (\citealt{Repp2006}, \citeyear{Repp2009}), and a proposal for response particles in terms of propositional anaphors that realize certain polarity features (\citealt{RoelofsenFarkas2015}). We argue that the set of polarity features hitherto assumed should be extended to features that are sensitive to the type of antecedent that polar responses react to: assertion or question.

\keywords{question, question bias, negation, response particle, propositional anaphor, acceptability judgments}
}

\lsConditionalSetupForPaper{}

\begin{document}
\maketitle

\section{Introduction}\label{geist-repp:sec:intro}

\sloppy Response particles like \textit{yes} and \textit{no} have been assumed to fulfil two functions: they may affirm or reject the truth of a previous utterance (truth-based function), or they may signal the polarity of the response (polarity-based function). The difference becomes relevant in responses to assertions or questions with a negation. For instance, in reaction to the assertion \textit{Nina didn’t sneeze}, a particle like \textit{yes} in principle may signal that the assertion is true, i.e. signal agreement with \textit{Nina didn't sneeze}, but it may also signal that the response is positive, i.e. that \textit{Nina sneezed}. Languages differ with respect to which of these functions the individual response particles preferably realize -- or in how far these functions are combined. There has been much research on cross-linguistic as well as inter-individual variation on this issue in recent years, and earlier assumptions that there might be a division into \textsc{truth-based languages} and \textsc{polarity-based languages} (\citealt{Pope1976}, \citealt{Jones1999}) have been called into question (e.g., \citealt{Krifka2013}, \citealt{GoodhueWagner2018}, \citealt{Gonzalez-FuenteTubauEspinalPrieto2015}, \citealt{KramerRawlins2011}, \citealt{Holmberg2013}, \citeyear{Holmberg2015}, \citealt{MeijerClausReppKrifka2015}, \citealt{RoelofsenFarkas2015}, \citealt{LiGonzalez-FuenteEspinal2016}, \citealt{ClausMeijerReppKrifka2017}, \citealt{FarkasRoelofsen2019}, \citealt{ReppMeijerScherf2019}, \citealt{LoosSteinbachRepp2020}).

Response particles are generally thought to be anaphoric devices. They have been analysed as propositional anaphors (\citealt{Krifka2013}, \citealt{RoelofsenFarkas2015}, \citealt{FarkasRoelofsen2019}), and as remnants of an elliptic clause (\citealt{KramerRawlins2011}, \citealt{Holmberg2013}, \citeyear{Holmberg2015}). As propositional anaphors they refer to a salient proposition in the previous utterance. While assertions normally are assumed to introduce one proposition (unless they contain a negation), questions are usually assumed to introduce a set of two propositions (e.g., \citealt{Hamblin1973}). For instance, the Russian polar question \textit{Nina čichnula?} ‘Did Nina sneeze?’ introduces the positive proposition \textit{p}, \textit{Nina sneezed}, and the negative proposition $\bar{p}$, \textit{Nina did not sneeze}. In principle, response particles may take up either proposition as antecedent but since anaphors are sensitive to the salience of potential antecedents, and since it has been argued that the particular form of a question may influence the salience of the two propositions, the issue arises which proposition a particle picks up. 

Formal aspects potentially influencing the salience of \textit{p} or $\bar{p}$ include for instance the presence vs. absence of a negative marker (e.g., \citealt{RoelofsenGool2010}, \citealt{RoelofsenFarkas2015}), the form and position of the negative marker, and the presence of certain particles. These formal means mark certain contextual and speaker-related biases, which may correspond to \textit{p} or $\bar{p}$ (e.g., \citealt{Ladd1981}, \citealt{BuringGunlogson2000}, \citealt{RomeroHan2004}, \citealt{Repp2009}, \citealt{Sudo2013}, \citealt{Seeliger2015}, \citeyear{Seeliger2019}, \citealt{Gyuris2017}, \citealt{SeeligerRepp2018}, \citealt{ArnholdRomero2021}, \citealt{ReppGeistToAppear}). To illustrate, a question like \textit{Didn't Nina sneeze?} may be used to double-check the truth of \textit{p} \textit{(Nina sneezed)} because the speaker had assumed that \textit{p} is true -- this might make \textit{p} salient. The same question may also be used to double-check the truth of $\bar{p}$ \textit{(Nina didn't sneeze)} because this is what the evidence suggests -- this might make $\bar{p}$ salient. Most accounts of question bias assume different analyses for the negation in these two question uses (or meanings): as \textsc{outer negation} and \textsc{inner negation}, respectively, so that a question with outer negation (\textsc{ON-question}) double-checks a positive proposition, and a question with inner negation (\textsc{IN-question}) checks a negative proposition. Hence, it is to be expected that \textit{yes} and \textit{no} as well as their correlates in other languages pick up different propositions when answering ON- vs. IN-questions. 

In this paper we investigate the meaning and use of the response particles \textit{da/net} ‘yes/no' in Russian in responses to biased ON/IN-questions in Russian. We present quantitative evidence from two acceptability judgment experiments. The goal of our investigation is to improve our understanding of bias in questions on the one hand, and of the meaning and use of response particles, on the other hand. In Russian, polar questions typically have a declarative syntax, and are distinguished from assertions by prosody. To indicate question bias, interrogative particles may be used. The two readings of polar questions as ON- vs. IN-questions are attested, albeit not necessarily by this terminology (e.g., \citealt{BaranovKobozeva1983}, \citealt{Brown.Franks1995}, \citealt{Brown1999}, \citealt[307]{Kobozeva2004}, \citealt{Meyer2004}, \citealt{Satunovskij2005}). As for the meaning and use of response particles, Russian has been argued to combine truth-based and polarity-based strategies (\citealt{Gonzalez-FuenteTubauEspinalPrieto2015}, \citealt{Esipova2021}). Most previous investigations on this issue focus on lexical, prosodic and (co-speech) gestural answering strategies in responses to positive and negative antecedents without considering a potential difference between ON/IN-question readings. However, work by \citet{Restan1972}, \citet{Meyer2004} and, most recently, the experimental work by \citet{Pancenko2021} on \textit{da/net} in responses to negative questions suggests that the ON/IN-difference plays a role for the acceptability of the Russian response particles. 

The paper is structured as follows. \sectref{geist-repp-polarQuestionBias} discusses the notion of question bias in relation to ON/IN-readings both in general and for Russian. \sectref{geist-repp:sec:Response-particles} discusses the analysis of response particles in one of the anaphora accounts (\citealt{RoelofsenFarkas2015}, \citealt{FarkasRoelofsen2019}). \sectref{geist-repp:sec:acceptability-judgment-experiments} presents the two acceptability studies. \sectref{geist-repp:sec:Discussion-Conclusion} discusses the results and provides a theoretical evaluation.

\section{Polar question bias and negation}\label{geist-repp-polarQuestionBias}

\subsection{Background}\label{geist-repp:sec:background}

As mentioned above, negative polar questions may express certain contextual and speaker-related biases. Two dimensions have proven helpful in the analysis of these biases (\citealt{Sudo2013}, \citealt{GaertnerGyuris2017}): (i) \textsc{epistemic bias} (roughly: prior speaker belief or speaker knowledge) and (ii) \textsc{evidential bias} (current situational evidence, including propositions implied by the addressee).\footnote{Epistemic bias has also been associated with the speaker’s desires or expectations (\citealt{Sudo2013}). We are not considering these meaning aspects here.} For instance, in the context description in \REF{geist-repp:ex:sarah-tom} we learn about a belief of the person asking the question, Sarah. Sarah believes that the proposition \textit{p}, \textit{Ms Miller has booked the tickets}, is true. This belief implies that the departure time for the flights under discussion cannot be changed. Tom’s suggestion to take an earlier flight ($=\text{the evidence}$) therefore is incompatible with Sarah’s belief: the evidence suggests that $\bar{p}$  is true. To resolve this conflict between the evidential and the epistemic bias, Sarah asks a negative polar question. 

\eanoraggedright\label{geist-repp:ex:sarah-tom}
Sarah and Tom are preparing a business trip to Milan. Ms Miller, their secretary, is helping them. Just before they go home, Sarah and Tom are talking about the business trip. Sarah assumes that Ms Miller has organized everything and the departure time of the flights is fixed.\\[6pt]
    Tom: \tabto{1.7cm}Maybe we should take an earlier flight.\\[3pt]
    Sarah: \tabto{1.7cm}Hasn’t Ms Miller booked the tickets?
\z

\noindent As mentioned above, a question like Sarah’s may double-check the epistemic bias or the evidential bias.{\interfootnotelinepenalty=10000\footnote{For English, this ambiguity only is obligatorily present with so-called preposed negation, i.e. with the negation marker cliticized to the auxiliary like in \REF{geist-repp:ex:sarah-tom}. Questions with non-preposed negation, i.e. \textit{Has Ms Miller not booked the tickets?}, do not necessarily have the implicature that the speaker had a previous belief: they can be asked in neutral contexts (\citealt{RomeroHan2004}). We are not considering the difference between preposed and non-preposed negation here as we did not manipulate the position of the Russian negation-plus-verb complex in our experimental materials.}} \citet{Ladd1981} argued that the presence of a positive polarity item (PPI) vs. a negative polarity item (NPI) disambiguates the two readings. We are showing this for the PPI \textit{already} and the weak NPI \textit{yet} in \REF{geist-repp:ex:miller-a}--\REF{geist-repp:ex:miller-b}, since we used the Russian counterparts of these elements in our experiments. \REF{geist-repp:ex:miller-a} contains \textit{already}, \REF{geist-repp:ex:miller-b} contains \textit{yet}. Both questions are negative but in \REF{geist-repp:ex:miller-a} the negation does not seem to anti-license the PPI, which is why it is called \textsc{outer negation}. The negation licensing the NPI in \REF{geist-repp:ex:miller-b} is \textsc{inner negation} (\citealt{RomeroHan2004}). The idea behind this terminology is that outer negation is ``too far out'' to anti-license the PPI, whereas inner negation is close enough to license the NPI (\citealt{Ladd1981}). \tabref{geist-repp:tab:Characteristics-ON/IN-questions} summarises the main characteristics of ON/IN-questions.\largerpage

\ea\label{geist-repp:ex:miller}
\ea[]{
Hasn’t Ms Miller \textit{already} booked the tickets?\label{geist-repp:ex:miller-a}}
\ex[]{
Hasn’t Ms Miller booked the tickets \textit{yet}? \label{geist-repp:ex:miller-b}
}
\z
\z

\noindent The difference between the two negations has been analysed in various ways, for instance in terms of scope relations between the negation and an epistemic conversational operator (\citealt{RomeroHan2004}), as illocutionary vs. propositional negation (\citealt{Repp2006}, \citeyear{Repp2009}, \citeyear{Repp2013}; also \citealt{Romero2015}), or in terms of scope relations between speech act operators (\citealt{Krifka2015}); see \citet{Romero2020} for a review. We are following here the analysis proposed by Repp (\citeyear{Repp2006}, \citeyear{Repp2009}, \citeyear{Repp2013}). 

\begin{table}[t]
\caption{Characteristics of ON/IN-questions}
\label{geist-repp:tab:Characteristics-ON/IN-questions}
 % \begin{tabularx}{\textwidth}{p{3.09cm}YYYYY} 
 \begin{tabularx}{\textwidth}{p{2.5cm}XXXXX}
  \lsptoprule
    Form  & Polarity item & Epistemic bias  & Evidential bias & ``Function'' & Negation\\
  \midrule
  \textit{Hasn't Ms Miller already booked\newline the tickets?}\smallskip &   PPI  &    \textit{p}  &    $\bar{p}$ or none & double-checks \textit{p}\medskip  & outer\\
  \textit{Hasn’t Ms Miller booked the\newline tickets yet?}  &   NPI &   \textit{p}  &    $\bar{p}$  & double-checks $\bar{p}$ & inner\\
  \lspbottomrule
 \end{tabularx}
\end{table}

Repp assumes that outer negation corresponds to the illocutionary (or common ground managing) operator \textsc{falsum}. \textsc{falsum} expresses that the speaker is sure that the proposition in its scope should not be added to the common ground. Being an illocutionary operator, \textsc{falsum} always scopes over a (positive) proposition (unless there are several negation markers), but it scopes under the question operator so that a question with \textsc{falsum} asks whether or not the speaker is sure that a given proposition should not be added to the common ground. Thus, in this analysis a biased question is not a set of two propositions but a set of two semantic-pragmatic objects including an illocutionary operator, see \REF{geist-repp:ex:on-question} for the proposed logical form (LF) of ON-questions and their meaning. For inner negation, Repp builds on \citet{RomeroHan2004}, who assume that preposed negation obligatorily introduces a conversational epistemic operator \textsc{verum} (based on \citepossalt{Hoehle1988}, \citeyear{Hoehle1992} \textsc{verum} focus). \textsc{verum} expresses that the speaker is sure that the proposition in its scope should be added to the common ground.\footnote{\citet{RomeroHan2004} propose a \textsc{verum} analysis for both ON- and IN-questions. They assume that in ON-questions, \textsc{verum}, which itself is in the scope of negation, scopes over a positive proposition: [Q~[\neg\,\textsc{verum}~\textit{p}]]. In IN-questions, \textsc{verum} scopes over a negative proposition: [Q~[\textsc{verum}~$\bar{p}$]]. Repp (\citeyear{Repp2006}, \citeyear{Repp2009}, \citeyear{Repp2013}) departs from this proposal inter alia because an analysis in terms of \textsc{verum} in some contexts produces meanings that are ``too weak''. For instance, for rejections like \textit{She \textsc{did}n’t buy the tickets}, \citet{RomeroHan2004} also assume a \textsc{verum} analysis. However, [\neg\,\textsc{verum} [\textit{she bought the tickets}]] means that the speaker is not sure that the proposition \textit{she bought the tickets} should be added to the common ground, contrary to the intuition of what this rejection expresses, namely that the speaker is sure that this proposition should not be added to the common ground. Also see \citet{Romero2015} for an analysis of negative polar questions that uses both \textsc{verum} and \textsc{falsum}.}  Repp assumes that \textsc{verum}, like \textsc{falsum}, is an illocutionary operator and takes scope over a proposition. In IN-questions, \textsc{verum} scopes over a negative proposition because the negation in these questions is propositional negation, see \REF{geist-repp:ex:in-question} for the corresponding LF. A question with \textsc{verum} asks whether or not the speaker is sure that a given negative proposition should be added to the common ground. Note that the occurrence of PPIs in ON-questions and of NPIs in IN-questions is predicted by this account because only in the latter is there propositional negation, which by hypothesis is required to license NPIs.

\ea\label{geist-repp:ex:ON/IN-Question}
\ea[]{
ON-question: $[\text{Q}~[\textsc{falsum}\,p]]=\{\textsc{falsum}\,p\text{, }\neg\,\textsc{falsum}\,p\}$ \label{geist-repp:ex:on-question}
}
\ex[]{
IN-question: $[\text{Q}~[\textsc{verum}\,\bar{p}]]=\{\textsc{verum}\,\bar{p}, \neg\,\textsc{verum}\,\bar{p}\}$ \label{geist-repp:ex:in-question}
}
\z
\z

\noindent Repp’s account predicts that in responses to ON- vs. IN-questions, different propositions are made available for anaphoric uptake: \textit{p} and $\bar{p}$, respectively. Evidence that this might indeed be the case comes from acceptability rating studies in German. \citet{ClausFruehaufMeijerKrifka2016} and \citet{ReppClausFruehauf} show that ON-questions are answered as if they were positive questions. This is expected if the negation in ON-questions is not propositional. Responses to IN-questions do not show this pattern. In our study, we will test whether the predictions of Repp’s account for ON- vs. IN-questions can be confirmed for Russian.  

\subsection{Question bias and negation in Russian}\label{geist-repp:sec:question-bias-negation}

As already mentioned, Russian polar questions by default have the form of assertive declarative sentences: subject-verb-object order without subject-auxiliary inversion. Questionhood is marked by intonation: whereas in (out-of-the-blue) assertions the default nuclear accent is on the object of the clause, in (out-of-the-blue) interrogatives it is on the verb (\citealt{Bryzgunova1975}, \citealt{Ladd1996}). The accent in interrogatives is described as a steep rise $\text{L}+\text{H*}$ with peak delay into the postnuclear syllable, which may be followed by a secondary L*~target (\citealt{MeyerMleinek2006}; cf. \citealt{Bryzgunova1980}). 

Russian has interrogative particles that indicate different question biases: \textit{razve}, \textit{neuželi}, \textit{li}, \textit{ved’}, \textit{že}, among others (e.g., \citealt[387f.]{Svedova2005}). Here we discuss the particle \textit{razve} ‘really’, which we used in our experiments. \textit{Razve} is used in situations where there is an evidential bias for the proposition denoted by the clause that is used as question, and an epistemic bias for the complement of this proposition (\citealt{ReppGeistToAppear}). For instance in \REF{geist-repp:ex:ivan-otpusk}, A’s utterance implies that Ivan is married (evidential bias for \textit{p}). The occurrence of \textit{razve} in B’s question (\textit{p}?) indicates that~B originally had the belief that Ivan is not married (epistemic bias for $\bar{p}$). The use of \textit{razve} in B’s question indicates moderate surprise or doubt concerning the evidence in view of B’s original belief (\citealt{Apresjan1980}, \citealt{Rathmayr1985}, \citealt{Baranov1986}, \citealt{Kirschbaum2001}, \citealt{Matko2014}), and signals that B wishes to double-check the evidential bias \textit{p} (\textit{he is married}). 

\ea\label{geist-repp:ex:ivan-otpusk}
\begin{xlist}[M:]
\exi{A:} 
    \gll Ivan	ezdil	v 	otpusk	so 	svoej	ženoj.\\
    Ivan went in holiday with his.own wife \\
    \glt ‘Ivan was on holiday together with his wife.’ \\
\exi{B:}\label{geist-repp:ex:ivan-otpusk-B} 
    \gll A razve	on	ženat?\\
    but \textsc{part} he married \\
    \glt ‘But is he really married?’\hfill (\citealt[5]{Zaliznjak2020})
\end{xlist}
\z

\noindent\textit{Razve} can also occur in negative questions. Negation in Russian is expressed by the preverbal particle \textit{ne}. \citet{ReppGeistToAppear} present experimental evidence which indicates that negative questions ($\bar{p}$?) with \textit{razve} are more acceptable when they occur in biased contexts, i.e. in contexts where there is evidence for $\bar{p}$ and the speaker had a previous belief for \textit{p}, than when they occur in neutral contexts. Negative questions without \textit{razve} display the opposite pattern.

As already mentioned, there are descriptions of ON- and IN-question readings in the literature on Russian (\citealt{Restan1972}, \citealt{BaranovKobozeva1983}, \citealt{Brown.Franks1995}, \citealt{Brown1999}, \citealt{Meyer2004}, \citealt{Kobozeva2004}, \citealt{Satunovskij2005}, \citealt{Pancenko2021}, \citealt{ReppGeistToAppear}).  Whether or not the position of the negation-verb complex (clause-initial or not) contributes to the different readings is controversial (\citealt{Brown.Franks1995}, \citealt{Meyer2004}). \citet{ReppGeistToAppear} discuss data from the Russian National Corpus (\url{ruscorpora.ru}; \citealt{Rachilina2008}) with the negation-verb complex in non-initial position which show that both ON- and IN-readings are available in questions with \textit{razve} (see \REF{geist-repp:ex:skazuGlavnoe} and \REF{geist-repp:ex:razveNiRazuNeSkazal} further below). \citet{ReppGeistToAppear} assume that Russian \textit{eščë}, the approximate counterpart of the English NPI \textit{yet}, indicates the inner negation reading, and Russian \textit{uže}, the approximate counterpart of the English PPI \textit{already}, indicates the outer negation reading.\footnote{There are other diagnostics in Russian to distinguish the two readings. For instance, \citet{Pancenko2021} provides experimental evidence showing that ON is marked by the combination of the particle \textit{li} with \textit{ne} (\textit{ne}...\textit{li}). \citet{Meyer2004}, following \citet{Restan1972}, argues that certain modal particles and sentence adverbs, for instance \textit{že} ‘$\approx\text{but}$’,  \textit{ved’} ‘$\approx\text{but}$’, \textit{konečno} ‘of course’ and \textit{stalo byt’} ‘apparently’, may only occur in IN-questions and not in ON-questions. See \citet{Brown.Franks1995} and \citet{Meyer2004} for other morphosyntactic cues. The role of intonation is uncertain (\citealt{Meyer2004}, \citealt{Pancenko2021}).} 

As just suggested, \textit{eščё} and \textit{uže} cannot be fully identified with \textit{yet} and \textit{already}: \textit{eščё} and \textit{uže} have many different uses (\citealt{Boguslavskij1996}). The polarity-sensitive uses that we are interested in here are attested in combination with a verb in perfective aspect. In this context, \textit{eščё} patterns with the English NPI \textit{yet} and needs licensing by negation, \REF{geist-repp:ex:John.has.left.1}, whereas \textit{uže} patterns with \textit{already} and is excluded under sentence negation, \REF{geist-repp:ex:John.has.left.2}.

\eanoraggedright\label{geist-repp:ex:John.has.left.1}
John has left \{already / *yet\}.	\hfill \textit{positive clause} \\
Ivan uechal  \{uže / *eščë\}.
\ex\label{geist-repp:ex:John.has.left.2}
John has not left \{*already / yet\}. \hfill \textit{negative clause}  \\
Ivan ne uechal \{*uže  / eščë\}.
\z

\noindent The polarity sensitivity of \textit{eščё} and \textit{uže} furthermore shows up in combination with other NPIs and PPIs. \textit{Eščë} may co-occur with strong NPIs like the negative pronoun \textit{nikuda} ‘nowhere’, \REF{geist-repp:ex:Ivan.ezdil.1}, but cannot co-occur with PPIs like the intensifier \textit{gorazdo} ‘considerably’, \REF{geist-repp:ex:Ivan.ezdil.2} (cf. \citealt{vanderWouden1997} for intensifiers as PPIs). For \textit{uže} it is the other way round.

\ea\label{geist-repp:ex:Ivan.ezdil.1}
\gll Ivan \minsp{\{*} eščë nikuda\textsubscript{NPI} / \minsp{\textsuperscript{\textsc{ok}}} uže gorazdo\textsubscript{PPI} bystree\} uechal.\\
Ivan {} yet nowhere {} {} already considerably faster left\\
\glt Intended: ‘Ivan hasn’t left anywhere yet.’ /\\‘Ivan has left already considerably faster.’ \\
\ex\label{geist-repp:ex:Ivan.ezdil.2}
\gll Ivan \minsp{\{\textsuperscript{\textsc{ok}}} eščë nikuda\textsubscript{NPI} / \minsp{*} uže gorazdo\textsubscript{PPI} bystree\} ne uechal.\\
Ivan {} yet nowhere {} {} already considerably faster not left\\
\glt‘Ivan hasn’t left anywhere yet.’ /\\Intended: ‘Ivan has left already considerably faster.’\\
\z

\noindent Turning now  to the occurrence of \textit{eščё} and \textit{uže} in negative questions in the Russian National Corpus, as discussed by \citet{ReppGeistToAppear}, consider \REF{geist-repp:ex:skazuGlavnoe} and \REF{geist-repp:ex:razveNiRazuNeSkazal}. In \REF{geist-repp:ex:skazuGlavnoe} speaker~B has an epistemic bias for the positive proposition \textit{p} (\textit{A~has already told me the main thing}). However, A’s utterance provides evidence for $\bar{p}$. To resolve the conflict, B~asks a question containing the NPI \textit{eščë}, double-checking $\bar{p}$, the evidential bias.


\ea\label{geist-repp:ex:skazuGlavnoe}
\begin{xlist}[M:]
    \exi{A:} \gll Sejčas	ja	tebe	skažu	glavnoe.\\
now	I	you	tell	main.thing\\
\glt ‘Now I am telling you the main thing.’\\
    \exi{B:}
\gll Razve	eščë	ne	skazal?\\
\textsc{part}	yet	not	said\\
\glt ‘Haven't you told it to me yet?’ \hfill \textit{IN-question}\\
\hfill [A. I. Spasovskiy, “Bolšaja kniga peremen / Volga” 2010]
\end{xlist}
\z

\noindent The assumption that a negative \textit{razve}-question containing \textit{eščë} is indeed an IN-question is supported by the observation that the strong NPI \textit{ni razu} ‘not once’ can occur in such a question:

\ea\label{geist-repp:ex:razveNiRazuNeSkazal}{
\gll Razve	eščë ni razu ne	skazal?\\
\textsc{part} yet {\NEG} once	not	said\\
\glt ‘Haven’t you ever told me?’
}
\z

\noindent Example \REF{geist-repp:ex:vytashitIzProshlogo} shows that \textit{uže} can occur in a \textit{razve}-question, indicating that \textit{razve}-questions can be ON-questions. The \textit{razve}-question in \REF{geist-repp:ex:vytashitIzProshlogo} conveys the same biases as the \textit{razve}-question in \REF{geist-repp:ex:skazuGlavnoe}: an epistemic bias for \textit{p} (\textit{You have dragged me out of the past already}), and an evidential bias for $\bar{p}$. To resolve the conflict, the speaker asks the question. Here it is the epistemic bias that is checked, as is indicated by the presence of the PPI \textit{uže}. The question is an ON-question.

\ea\label{geist-repp:ex:vytashitIzProshlogo}
\begin{xlist}[M:]
    \exi{A:} 
\gll Čestnoe	slovo,	ne	znaju,	kak	vytaščit’	tebja	iz	prošlogo.\\
honest	word	not	know	how	drag	you	out.of	past\\
\glt ‘Frankly, I don't know how to drag you out of the past.’\\
    \exi{B:}
\gll Razve	ty	uže	ne	vytaščila	menja	iz	prošlogo?\\
\textsc{part}	you	already	not	dragged	me	out.of	past\\
\glt ‘Haven't you dragged me out of the past already?’  \hfill \textit{ON-question}\\
\hfill [Alexander Bogdan, Gennadi Praškewič. “Čelovek Č” 2001]
\end{xlist}
\z

\noindent As is shown in \REF{geist-repp:ex:*RazveTyUzeNiRazu}, the outer negation in the \textit{razve}-question in \REF{geist-repp:ex:vytashitIzProshlogo} anti-licenses the strong NPI \textit{ni razu}, which supports the assumption that the question in \REF{geist-repp:ex:vytashitIzProshlogo} indeed is an ON-question. 

\ea[*]{\gll Razve	ty	uže	ni razu	ne	vytaščila	menja	iz	prošlogo?\\
\textsc{part}	you	already {\NEG} once	not	dragged	me	out.of	past\\
\glt Intended: ‘Haven't you dragged me out of the past once already?’}\label{geist-repp:ex:*RazveTyUzeNiRazu}
\z

\noindent We conclude that a negative \textit{razve}-question $\bar{p}$? comes with an epistemic bias for \textit{p} and an evidential bias for $\bar{p}$. The question may double-check, and~-- by hypothesis~-- make salient, different propositions. Which proposition is double-checked and made salient may be disambiguated by polarity-sensitive items like \textit{eščë} and \textit{uže}. 

 

\section{Response particles}\label{geist-repp:sec:Response-particles}

There are various analyses of response particles, which fall into two major types: anaphora and ellipsis analyses. We already mentioned in \sectref{geist-repp:sec:intro} that response particles have been analysed as propositional anaphors, i.e. they take up a salient proposition in the discourse context (\citealt{Krifka2013}, \citealt{RoelofsenFarkas2015}, \citealt{FarkasRoelofsen2019}). Ellipsis accounts treat response particles as remnants of elided response clauses (\citealt{KramerRawlins2011}, \citealt{Holmberg2013}, \citeyear{Holmberg2015}). All these accounts aim at explaining the gradual differences in the acceptability and use of response particles that have been observed in recent years. For reasons of space, we only discuss one of the anaphora accounts here, namely \citeauthor{RoelofsenFarkas2015}’s feature model (\citealt{RoelofsenFarkas2015}, \citealt{FarkasRoelofsen2019}). 

\subsection{The feature model}\label{geist-rep:sec:The-feature-model}

\citet{RoelofsenFarkas2015} assume that response particles like English \textit{yes} and \textit{no} realize two types of semantic presuppositional features, which are formal instantiations of the two functions that response particles were argued to have in earlier literature: to indicate the polarity of the response or the truth of the antecedent (e.g., \citealt{Pope1976}, \citealt{Jones1999}). Accordingly, the first type of feature are \textsc{absolute polarity} features, which presuppose that the polarity of the response is positive (feature $[+]$) or negative (feature $[-]$). The second type are \textsc{relative polarity} features, which presuppose that the response has the same or the opposite polarity of the antecedent (the features [\textsc{agree}] and [\textsc{reverse}]). 

In the feature model, language-specific \textsc{feature-particle mappings} indicate which particle may realize which feature. For instance, English maps $[+]$ and [\textsc{agree}] onto \textit{yes}, and $[-]$ and [\textsc{reverse}] onto \textit{no}. Some languages map feature combinations onto a dedicated particle, like German does for [$+$,~\textsc{reverse}], which maps onto \textit{doch}. The feature-particle mapping for English in comparison to German as suggested by \citet{RoelofsenFarkas2015} is given in \tabref{geist-repp:tab:agree-reverse-EN-DE}.

\begin{table}
\caption{The feature-particle mapping for English and German}
\label{geist-repp:tab:agree-reverse-EN-DE}
 \begin{tabularx}{.8\textwidth}{lXX}
  \lsptoprule
  \multicolumn{2}{l}{English:}\\
  &$[+]$ and $[\textsc{agree}]\rightarrow\textit{yes}$ & $[-]$ and $[\textsc{reverse}]\rightarrow\textit{no}$\\
  \multicolumn{2}{l}{German:}\\
  &$[+]$ and $[\textsc{agree}]\rightarrow\textit{ja}$ & $[-]$ and $[\textsc{reverse}]\rightarrow\textit{nein}$\\
  && $[+,\textsc{reverse}]\rightarrow\textit{doch}$\\
  \lspbottomrule
 \end{tabularx}
\end{table}

In responses to positive assertions and questions, the absolute and relative polarity of response particles coincide, but in responses to negative questions and assertions these two functions come apart. This is illustrated in \REF{geist-repp:ex:MsMiller-tickets}, where the feature combination of the whole response is given in square brackets and the feature realized by the respective particle is marked by a frame. In \REF{geist-repp:ex:MsMiller-tickets-a}--\REF{geist-repp:ex:MsMiller-tickets-b} the absolute polarity feature to be realized is $[-]$ because the response clause contains negation, and the relative polarity feature to be realized is [\textsc{agree}] because the polarity of the response is the same as the polarity of the antecedent. In \REF{geist-repp:ex:MsMiller-tickets-a} \textit{no} realizes $[-]$, and in \REF{geist-repp:ex:MsMiller-tickets-b} \textit{yes} realizes [\textsc{agree}].    


\ea\label{geist-repp:ex:MsMiller-tickets}
Antecedent:	\\
Ms Miller hasn't booked the tickets. / Hasn’t Ms Miller booked the tickets?\medskip\\
Response:
\ea[]{
No, she hasn't. \hfill[\,\efbox{\rule[0mm]{0mm}{2mm}$-$}\,,~\textsc{agree}]\label{geist-repp:ex:MsMiller-tickets-a}
}
\ex[]{
Yes, she hasn't. \hfill[$-$,~\efbox{\textsc{agree}}\,] \label{geist-repp:ex:MsMiller-tickets-b}
}
\ex[]{
No, she has. \hfill[$+$,~\efbox{\textsc{reverse}}\,]  \label{geist-repp:ex:MsMiller-tickets-c}
}
\ex[]{
Yes, she has. \hfill[\,\efbox{$+$}\,,~\textsc{reverse}]  \label{geist-repp:ex:MsMiller-tickets-d}
}
\z
\z

\noindent The pattern shown in \REF{geist-repp:ex:MsMiller-tickets} reflects the feature-particle mapping for English but it does not represent the actual preference patterns for \textit{yes} and \textit{no} in English in the various discourse contexts. In other words, although both particles may in principle realize both types of features, there are clear differences in (graded) acceptability and use (\citealt{RoelofsenFarkas2015}, \citealt{ReppMeijerScherf2019}). To account for such observations, \citet{FarkasRoelofsen2019} model the realization of features in a stochastic optimality-theoretic (OT) framework. In this model, different constraint weightings are used to explain language-specific answering patterns and gradual preference patterns. \tabref{geist-repp:tab:Optimality-theoretic-constraints} lists the constraints.

\begin{table}
\caption{OT constraints in the feature model (\citealt{FarkasRoelofsen2019})}
\label{geist-repp:tab:Optimality-theoretic-constraints}
 \begin{tabularx}{\textwidth}{lQ}
  \lsptoprule
\textsc{maximize marked}:& Maximize the realization of marked polarity features or feature combinations.\\
\textsc{expressiveness}: & Maximize the expression of feature content.\\
\textsc{maximize relative}:  & Maximize the realization of relative polarity features.\\
\textsc{maximize absolute}: & Maximize the realization of absolute polarity features.\\
\lspbottomrule
 \end{tabularx}
\end{table}

The constraint \textsc{maximize marked} is a typical OT markedness constraint and thus is thought to be generally operative in response systems. It favours the realization of marked features or feature combinations. The features $[-]$ and [\textsc{reverse}] are thought to be marked: negation $[-]$ is assumed to be hard to process, and disagreeing in discourse [\textsc{reverse}] is a dispreferred discourse move. The feature combination [$+,\textsc{reverse}$] also is considered to be marked. In a language where the constraint \textsc{maximize marked} has a particularly high weight, marked features have a particularly high realization need and a particle that realizes a marked feature (combination) will be preferred over other particles. 

The constraint \textsc{expressiveness} is relevant if there is a preference in a language for particles expressing more rather than less features. For instance, for German, \textsc{expressiveness} is assumed to have a high weight, which explains why the particle \textit{doch}, which realizes [$+,\textsc{reverse}$], is more accepted in [$+,\textsc{reverse}$] responses than particles realizing only one of the features $[+]$ and [\textsc{reverse}]. \textsc{expressiveness} can be viewed as an instance of the general principle \textsc{Maximize presupposition!} (\citealt{Heim1991}): the polarity features are presuppositional.

The constraints \textsc{maximize relative} and \textsc{maximize absolute}, by which relative and absolute polarity features, respectively, have a high realization need, are response-specific constraints, and arguably cannot be linked to more general principles. However, given that languages do display different general tendencies to express truth vs. polarity (see \sectref{geist-repp:sec:intro}), it seems warranted to assume these constraints.

To see how these constraints can be used to explain gradual preferences for response particles, consider how \citet{ReppMeijerScherf2019} explain findings from an acceptability judgment experiment testing \textit{yes} and \textit{no} responses to negative assertions in English. \citeauthor{ReppMeijerScherf2019} suggest that the relative weight of two of the above constraints is relevant to account for the data (the other constraints have low weights), see \REF{geist-repp:ex:absolute-marked-features}, where \ding{227} stands for ‘has greater weight than’.

\ea\label{geist-repp:ex:absolute-marked-features}
\textsc{realize absolute features} \ding{227} \textsc{realize marked features}
\z

\noindent The acceptability patterns found by \citeauthor{ReppMeijerScherf2019} are shown in \REF{geist-repp:ex:MsMiller-tickets2}.\footnote{We are glossing over the inter-individual differences found by \citet{ReppMeijerScherf2019}.}  As before, a frame indicates the feature that is realized. In addition, marked features are highlighted in grey. \REF{geist-repp:ex:MsMiller-tickets2} shows that in agreeing responses, \REF{geist-repp:ex:MsMiller-tickets2-a}, \textit{no} was much more acceptable ($\gg$) than \textit{yes}. In these responses, \textit{no} realizes absolute, marked $[-]$, and \textit{yes} realizes relative, unmarked [\textsc{agree}]. In rejecting responses, \REF{geist-repp:ex:MsMiller-tickets2-b}, \textit{yes} was more acceptable ($>$) than \textit{no} but the difference was not so extreme. In rejecting responses, \textit{yes} realizes absolute, unmarked $[+]$, and \textit{no} realizes relative, marked [\textsc{reverse}]. Thus, in both agreeing and rejecting responses, the particle realizing the absolute feature was more acceptable than the particle realizing the relative feature. However, only in agreeing responses the particle realizing the marked feature was more acceptable than the particle realizing the unmarked feature. This pattern can be explained with the weighting indicated in \REF{geist-repp:ex:absolute-marked-features}: realizing absolute features has more weight in English than realizing marked features.  

\ea\label{geist-repp:ex:MsMiller-tickets2}
Antecedent:	\\
Ms Miller hasn't booked the tickets.\medskip\\
Response:
\ea[]{
$\text{No, she hasn't. [\,\efbox[backgroundcolor=lightgray]{\rule[0mm]{0mm}{2mm}$-$}\,,~\textsc{agree}]}\gg\text{*Yes, she hasn't. [\efbox[backgroundcolor=lightgray,leftline=false,rightline=false,topline =false,bottomline=false]{\rule[0mm]{0mm}{2mm}$-$}\,,~\efbox{\textsc{agree}}\,]}$\label{geist-repp:ex:MsMiller-tickets2-a}
}
\ex[]{
$\text{Yes, she has. [\,\efbox{$+$}\,,~\efbox[backgroundcolor=lightgray,,leftline=false,rightline=false,topline =false,bottomline=false]{\textsc{reverse}}]}>\text{No, she has. [$+$,\,\efbox[backgroundcolor=lightgray]{\textsc{reverse}}\,]}$\label{geist-repp:ex:MsMiller-tickets2-b}
}
\z
\z

\subsection{Russian response particles in the feature model}\label{geist-repp:sec:Russian-response-particles-feature-model}

Russian has two response particles: \textit{da} and \textit{net}. In two recent feature model analyses (\citealt{Esipova2021}, \citealt{Gonzalez-FuenteTubauEspinalPrieto2015}), which do not distinguish between ON- and IN-questions, Russian has been proposed to differ from English in its feature-particle mapping. Like English \textit{no}, Russian \textit{net} may realize the absolute feature $[-]$ or the relative feature [\textsc{reverse}]. Unlike English \textit{yes}, however, Russian \textit{da} may only realize the relative feature [\textsc{agree}]. Thus, the proposed feature-particle mapping is the one given in \REF{geist-repp:ex:russian-agree-da-net}, and the corresponding acceptability pattern is illustrated in \REF{geist-repp:ex:Nina-ne-sdala-ekzamen} from \citet{Esipova2021}.\footnote{\citet{Esipova2021} assumes the same pattern for questions and assertions as antecedents. However, she does not specify the bias profile or the ON/IN-readings of the questions. The non-preposed position of the negation in the English translation given by Esipova might be taken to hint at a ‘bias-free’ reading, which like the IN-negation reading arguably makes $\bar{p}$ salient, but Esipova is not explicit on this issue.}

\ea\label{geist-repp:ex:russian-agree-da-net}
Russian: $[\textsc{agree}]\rightarrow\textit{da}$	\hfil $[-]$ and $[\textsc{reverse}]\rightarrow\textit{net}$
\z

\ea\label{geist-repp:ex:Nina-ne-sdala-ekzamen}
Antecedent:\\
\gll  Nina	ne	sdala	ekzamen \{?,.\}\\
Nina not passed	exam\\
\glt  ‘\{\,Did Nina not pass the exam?, Nina did not pass the exam.\,\}’\medskip\\
Response:
\ea[]{
\gll Net, ne sdala.\\
  no not passed\\
\glt ‘No, she didn’t.’ \hfill [\,\efbox{\rule[0mm]{0mm}{2mm}$-$}\,,~\textsc{agree}]\label{geist-repp:ex:Nina-ne-sdala-ekzamen-a}
}
\ex[]{
\gll Da, ne sdala. \\
yes not passed\\
\glt ‘Yes, she didn’t.’ \hfill[$-$,~\efbox{\textsc{agree}}\,] \label{geist-repp:ex:Nina-ne-sdala-ekzamen-b}
}
\ex[]{
\gll Net, sdala. \\
no passed\\
\glt ‘No, she did.’ \hfill[$+$,~\efbox{\textsc{reverse}}\,]  \label{geist-repp:ex:Nina-ne-sdala-ekzamen-c}
}
\ex[*]{
\gll Da, sdala.\\
yes passed\\
\glt Intended: ‘Yes she did.’ \hfill[\,\efbox{$+$}\,,~\textsc{reverse}]  \label{geist-repp:ex:Nina-ne-sdala-ekzamen-d} \\ \hfill \citet[3f.]{Esipova2021}
}
\z
\z

\noindent\citet{Meyer2004} (following \citealt{Restan1972}, \citealt{Brown.Franks1995}) distinguishes between “purely informative” negative questions (questions without a bias) as antecedents, and questions with a negative implicature (the speaker expects a negative answer). For the former type of question, Meyer suggests that only the responses given in \REF{geist-repp:ex:Nina-ne-sdala-ekzamen-a} and \REF{geist-repp:ex:Nina-ne-sdala-ekzamen-d} are acceptable. Thus, the pattern is clearly different from the one given by \citet{Esipova2021} in \REF{geist-repp:ex:Nina-ne-sdala-ekzamen}. According to Meyer, \textit{da} and \textit{net} undoubtedly indicate absolute polarity as responses to such questions, i.e. $[+]$ and $[-]$. However, \citet{ReppGeist2022} report experimental evidence on responses to unbiased questions in rich discourse contexts which does not confirm Meyer’s claims: \textit{da} was clearly degraded in responses to such questions whereas \textit{net} was rated as acceptable -- both independently of the polarity of the response. For questions with a negative implicature -- which is a category that does not fit our description of biases -- \citet{Meyer2004} proposes the same pattern as the one given by Esipova in \REF{geist-repp:ex:Nina-ne-sdala-ekzamen-a}--\REF{geist-repp:ex:Nina-ne-sdala-ekzamen-d}. He also highlights that the pattern would be the same with assertions as antecedents, thus corroborating Esipova’s suggestion. However, since the question type is not specified by Esipova, a comparison is difficult. Overall, this empirical picture leaves open many questions and needs careful empirical investigation, especially in rich discourse contexts so that the exact question meaning can be controlled. For our investigation, we will work with the hypothesis that \textit{da} can only realize [\textsc{agree}] (\citealt{Esipova2021}, \citealt{Gonzalez-FuenteTubauEspinalPrieto2015}).

For sake of completeness, it should be noted here that in addition to particles, Russian uses lexico-syntactic response strategies. For instance, \citet{Gonzalez-FuenteTubauEspinalPrieto2015} identify the echoic answering strategy, where the speaker may repeat the verb without a particle, for instance to mark a rejection like \REF{geist-repp:ex:Nina-ne-sdala-ekzamen-d}. We restrict our investigation to the response particles \textit{da} and \textit{net}.

\section{Acceptability judgment experiments}\label{geist-repp:sec:acceptability-judgment-experiments}

In this section we are presenting the acceptability judgment experiments that we conducted to explore the feature-particle mapping for Russian \textit{da} and \textit{net} as summarized in \REF{geist-repp:ex:russian-agree-da-net}, for responses to biased ON/IN-questions, where the two types of negation are signalled by the polarity-sensitive items \textit{uže} ‘already’ and \textit{eščë} ‘yet’. Specifically, we explored the predictions that can be made on the basis of Repp’s (\citeyear{Repp2006}, \citeyear{Repp2009}, \citeyear{Repp2013}) analysis of such questions in English and German. Recall that according to this analysis, ON-questions vs. IN-questions make different propositions available for anaphoric uptake, which predicts that the type of negation will influence the felicity of \textit{da/net} for expressing that \textit{p} or $\bar{p}$ is true. We hypothesized that in responses to ON-questions, which check the epistemic bias for \textit{p} and according to Repp have the LF~[Q~[\textsc{falsum}~\textit{p}]], the positive proposition \textit{p} is taken up by \textit{da/net}. In responses to IN-questions, which check the evidential bias $\bar{p}$ and have the LF~[Q~[\textsc{verum}~$\bar{p}$]], it is the negative proposition $\bar{p}$ which is taken up by \textit{da/net}. 

\tabref{geist-repp:tab:Predictions-for-feature-realization-preferences-RU-ON/IN-questions} summarizes our specific predictions. For responses expressing that \textit{p} ($=\text{the epistemic bias}$) is true, we predict that after ON-questions only \textit{da} is felicitous because only \textit{da} can realize one of the features that potentially can be realized in such discourses (\,[\textsc{agree}] and $[+]$\,): \textit{da} realizes [\textsc{agree}], which presupposes that antecedent polarity and response polarity are the same. After IN-questions, we predict that only \textit{net} is felicitous: it realizes [\textsc{reverse}], which presupposes that antecedent polarity and response polarity are the opposite. For responses expressing that $\bar{p}$ (the evidential bias) is true, we predict that after ON-questions, only \textit{net} is felicitous: \textit{net} indicates the negative polarity of the response, and it indicates that the polarities of antecedent and response are the opposite. After IN-questions, \textit{net} should be felicitous because it expresses negative response polarity, and \textit{da} should be felicitous because it signals that antecedent and response polarity are the same. However, \textit{net} should be preferred over \textit{da} by \textsc{maximize marked features} because \textit{net} realizes a marked feature whereas \textit{da} does not.

\begin{table}
\caption{Predictions for feature realization preferences in responses to Russian ON/IN-questions}
\label{geist-repp:tab:Predictions-for-feature-realization-preferences-RU-ON/IN-questions}
 \begin{tabular}{cll} 
\lsptoprule
 & \multicolumn{2}{c}{Antecedent} \\\cmidrule(lr){2-3}
\multirow{3}{6em}{State of affairs = polarity of response}  &  ON-question   & IN-question\\
& \textit{Hasn’t ... already...?} & \textit{Hasn’t ... yet...?}\\
& [Q~[\textsc{falsum} \textit{p}]] & [Q~[\textsc{verum} $\bar{p}$]]\\\midrule
$p$ & [$+,\efbox{\textsc{agree}}\,]\rightarrow\textit{da}$ & [$+,\,\efbox{\textsc{reverse}}\,]\rightarrow\textit{net}$\\\addlinespace
$\bar{p}$ & [$\,\efbox{\rule[0mm]{0mm}{2mm}$-$}\,,\textsc{reverse}]\rightarrow\textit{net}$  & [$\,\efbox{\rule[0mm]{0mm}{2mm}$-$}\,,\textsc{agree}]\rightarrow\textit{net}$ \\
& [$-,\,\efbox{\textsc{reverse}}\,]\rightarrow\textit{net}$ & [$-,\efbox{\textsc{agree}}\,]\rightarrow\textit{da}$ \\
& & $net>da$\\
 \lspbottomrule
 \end{tabular}
\end{table}

We note here that although ON/IN-questions by their structure are assumed to introduce only one propositional discourse referent, the context might make additional propositions available. ON-questions double-check the epistemic bias for \textit{p} for a reason: there is evidence for $\bar{p}$ in the context. Therefore, it might be the case that $\bar{p}$ is salient to some extent. Similarly, IN-questions double-check the evidential bias $\bar{p}$ for a reason: the speaker believed \textit{p} to be true. So \textit{p} might be perceived to be salient to some extent. This interplay is not reflected in the LF of the questions and raises the interesting issue of the discourse status of the ``unchecked'' biases. We will come back to this issue in \sectref{geist-repp:sec:Discussion-Conclusion}.

\subsection{Method}\label{geist-repp:sec:method}

In our acceptability judgment experiments, we presented participants with question-answer dialogues embedded in contexts which make clear what the contextual evidence, the speaker’s previous beliefs, and the actual state of affairs (SoA) are. Experiment~1 tested responses to ON-questions, and Experiment~2 tested responses to IN-questions. We describe the two experiments together because of the great overlap in materials and method.

The materials of our study were based on those used in the experiments reported in \citet{ClausMeijerReppKrifka2017} (also see \citealt{MeijerClausReppKrifka2015}). \citeauthor{ClausMeijerReppKrifka2017} investigated responses to assertions in German, so we translated and localized the materials, and we adapted the contexts to license the question biases. The experimental items were descriptions of short scenarios including a question-answer dialogue between two interlocutors, Dima and Katja. The question was an ON-question (Experiment~1) or an IN-question (Experiment~2), and the answer consisted of a response particle (\textit{da}, \textit{net}) and an answer clause. 

Both experiments had a $2\times2$ design with the factors \textsc{state of affairs (soa)} and \textsc{particle}. \REF{geit-repp:ex:sample-item} is a sample item.  Each item started with a description of a situation, which informed the reader about the general setting, including information on whether or not a certain SoA obtained or not ($=\text{factor \textsc{soa}}$). In \REF{geit-repp:ex:sample-item} the SoA concerned whether Marina \mbox{Petrovna} had booked tickets for a flight or not. For mnemonic reasons, we are using the strings \textsc{done} and \textsc{not done} to indicate whether the relevant SoA obtains (\textit{p} is true), or not ($\bar{p}$ is true). The SoA was what the question-answer dialogue was about. The description of the situation further contained information about the knowledge states and assumptions of the interlocutors and the existing contextual evidence (epistemic and evidential bias). The person asking the question, Katja, always believed that \textit{p} is true (epistemic bias for \textit{p}), and the contextual evidence always suggested that $\bar{p}$ might be true. Thus, there was a conflict between the epistemic and the evidential bias, which produces doubt or surprise in Katja. To dispel her doubt, Katja asks a question. In Experiment~1 the question contained the PPI \textit{uže} ‘already’ and thus by hypothesis was an ON-question checking the epistemic bias. In Experiment~2 the question contained the NPI \textit{eščë} ‘yet’ and thus by hypothesis was an IN-question double-checking the evidential bias. Dima's response consisted of a response particle (factor \textsc{particle}: \textit{da}, \textit{net}) and a response clause (where the subject was elided), which -- depending on the question -- contained \textit{uže} or \textit{eščë}. The response clause was always truthful: it reflected the actual state of affairs.

\eanoraggedright\label{geit-repp:ex:sample-item}
\textsc{Sample item}\\
Dima i Katja gotovjatsja k komandirovke v Milan. Im pomogaet ich sekretar’– Marina Petrovna Mironova. \textit{‘Dima and Katja are preparing a business trip to Milan. Marina Petrovna Mironova, their secretary, is helping them’.}\\
[12pt]
\textsc{soa done}: Segodnja utrom Dima razgovarival s Mariej Petrovnoj i uznal, čto ona uže zabronirovala aviabilety. \textit{‘Dima talked to Marina Petrovna this morning and learned that she has already booked the tickets’.}\\
[6pt]
\textsc{soa not done}: Segodnja utrom Dima razgovarival s Mariej Petrovnoj i uznal, čto ona budet bronirivat’ aviabilety na sledujuščej nedele. \textit{‘This morning Dima talked to Marina Petrovna and learned that she would book the tickets next week’.}\\
[12pt]	
Nezadolgo do okončanija rabočego dnja Dima i Katja obsuždajut predstojaščuju komandirovku. Katja uverena v tom, čto Marina Petrovna uže vsё organizovala i vremja vyleta uže izvestno. Poėtomu ona udivljaetsja, kogda Dima predlagaet letet’ bolee rannim rejsom. \textit{‘Just before they go home, Dima and Katja are talking about the business trip. Katja assumes that Marina Petrovna has organized everything and that the departure time is fixed. So she is a little surprised when Dima suggests taking an earlier flight’.}\\[12pt]
Katja:\\
\mbox{}\hfill \textit{ON-question, Experiment 1}\\ 
\gll Razve	Marina	Petrovna	uže 	ne	zabronirovala	aviabilety? \\
\textsc{part}	Marina	Petrovna	already	not	booked	flight.tickets\\ 
\glt ‘Hasn't Ms Miller already booked the tickets?’\\
\pagebreak
\mbox{}\hfill \textit{IN-question, Experiment 2}\\
\gll Razve	Marina	Petrovna	eščë 	ne	zabronirovala	aviabilety?\\ 
\textsc{part}	Marina	Petrovna	yet	not	booked	flight.tickets\\ 
\glt ‘Hasn't Ms Miller booked the tickets yet?’ \\
\vspace{12pt}
Dima:\\ 
\vspace{6pt}
\gll Net/ Da, uže zabronirovala.\\
no yes already booked\\
\glt ‘No/Yes, she has already booked the tickets.’ \hfill \textit{Experiments~1,~2}\\
\vspace{6pt}
\gll Net/ Da, eščё ne zabronirovala.\\
no yes yet not booked\\
\glt ‘No/Yes, she has not booked the tickets yet.’ \hfill \textit{Experiments~1,~2}\\
\z

\noindent Each experiment contained 24~lexicalizations in the four conditions just described. In addition to the experimental items, there were 24~lexicalizations which were very similar to the scenarios in the experimental items except that the question was positive and there was no bias. Otherwise they had the same $2\times2$ design. The fillers served mainly as control items and we will not discuss them here. The 48~lexicalizations were distributed over four lists in a Latin square design so that each list contained 24~experimental and 24~filler items. In addition, there were two practice items on each list.

The task of the participants was to judge the naturalness of the answer as a response to the question in view of the information described in the scenario. The judgment was given on a seven-point-scale with one scale end labelled \textit{očen’ estestvenno} ‘very natural’ and the other scale end \textit{očen’ stranno} ‘very strange’. For the statistical analysis, these end points were transformed to the numbers~7 and~1, respectively, with the other scale points sitting in between. In addition to giving the acceptability judgment, participants verified a statement about the context, which was to ensure that they read the scenarios carefully. The verification statement was shown to the participants after they had read the test item and given the acceptability judgment.

The experiments were run as a web experiment on SoSci Survey (\url{soscisurvey.de}; \citealt{Leiner2021}). For Experiment~1, 36~participants (28~female, 8~male; mean age:~35.3; age range: 29--54) with Russian as their native language were recruited via Prolific (\url{www.prolific.co}). For Experiment~2, 39~participants (30~female, 8~male, 1~unspecified; mean age:~37.5; age range: 20--56) were recruited. Before taking part in the experiment, they gave informed consent. Due to the recruiting strategy via Prolific, we had not originally planned to conduct cross-experimental comparisons because we did not expect the same participants to take part in both experiments, which were conducted two weeks apart. As it turned out, 29~participants took part in both experiments. We decided to pool the data for these participants from both experiments for the statistical analysis because this allowed a direct comparison between the two question types. We discarded the data of the other participants. 

To tackle the problem which recruiting participants via prolific brings about -- the danger that most of the participants might be heritage speakers with potentially low levels of proficiency in Russian -- we collected sociodemographic data of our participants. Of the 29~participants that took part in both experiments, 18~were born in Russia, 3~in~Estonia, 3~in~Latvia, 3~in~the Ukraine, 1~in Moldavia, and 1~in~Mongolia.\footnote{We assigned participants that had indicated the Soviet Union as birth place to the respective post-Soviet countries. Russian is a widespread native language in all the above-mentioned countries, except Mongolia. None of participants born in Estonia, and Moldavia indicated that they speak Estonian or Moldavian. One person from Latvia speaks Latvian regularly but only several times per month. The person from Mongolia, and the Latvian person just mentioned were excluded from the statistics for poor performance on the control items (see \sectref{geist-repp:sec:results}) along with one other person.} Almost all had also spent the longest part of their lives in these countries, except for two people born in Russia, who had spent most time in the Ukraine and in the UK, respectively, and one person from the Ukraine and one from Moldavia, who both had spent most time in the UK. We take these numbers to indicate that our participants are proficient Russian speakers, although we note that the age of one of the people having spent most time in the UK indicates a pre-adult move to the UK. We note that 26~participants reported to speak English on a daily basis, for one this was the case for French, and for one for Ukrainian. There were several other languages that were used less frequently.

\subsection{Results}\label{geist-repp:sec:results}\largerpage
All 29~participants reached at least 80~percent correctness for the verification task so no participant was excluded on that criterion. The data from three participants were excluded from the analysis because they had not chosen the expected side of the naturalness scale in more than ten percent of the filler items, where the judgment for the use of \textit{da} or \textit{net} is unequivocal. This left 1248~data points for analysis. The analysis was conducted by fitting a cumulative link mixed model for ordinal data (R~package ordinal, \citealt{Christensen2019}). \textsc{question type} ($=\text{Experiment}$), \textsc{soa} and \textsc{particle} were fixed factors. They were sum-coded. Initially, participant and lexicalization were random factors. However, since the random effects of lexicalization produced models that were a singular fit, the final model only contained random intercepts and slopes for the experimental factors and their interaction per participant and not per lexicalization.

\figref{geist-repp:fig:proportionsOfRating} shows the results in terms of proportions of rating levels broken down for the experimental conditions including the median ratings per condition. \tabref{geist-repp:tab:model-estimates-pooled-data} shows the model estimates. There were main effects of \textsc{question type} (experiment) and of \textsc{particle}, and an interaction of \textsc{particle} and \textsc{soa}.

% Figure1

\begin{figure}
%\includegraphics[width=\textwidth]{figures/Figure1-ProportionsOfRating.png}
\includegraphics[width=.8\textwidth]{figures/Figure1-ProportionsOfRating.pdf}
\caption{Proportions of rating levels for responses to ON/IN-questions. Numbers on the bars are the medians per condition}
\label{geist-repp:fig:proportionsOfRating}
\end{figure}

\begin{table}
\caption{Model estimates for the pooled data of both experiments}
\label{geist-repp:tab:model-estimates-pooled-data}
 \begin{tabular}{l  S[table-format=-1.2] S[table-format=1.2] S[table-format=-1.2] S[table-format=<1.3]@{}l}
  \lsptoprule
                                                                             & {Estimate} & {SE}  & {$z$} & {$p$} & \\
     \midrule
          \textsc{question type}                                             &   0.62  &  0.25  &   2.52  & 0.012&*\\
          \textsc{soa}                                                       &  -0.02  &  0.16  & -0.11 & 0.912\\
          \textsc{particle}                                                  &  1.04   &  0.29  & 3.53 & <0.001&***\\
          $\textsc{question type}\times\textsc{soa}$                         &  0.02   &  0.15  & 0.10 & 0.921 \\
          $\textsc{question type}\times\textsc{particle}$                    &  0.28   &  0.15  & 1.84 & 0.065\\
          $\textsc{particle}\times\textsc{soa}$                              &  2.58   &  0.40  & 6.53 & <0.001&***\\
          $\textsc{question type}\times\textsc{soa}\times\textsc{particle}$  &  -0.16  &  0.17  & -0.92 & 0.357\\
  \lspbottomrule
 \end{tabular}
\end{table}

Overall, the particles were judged to be more natural after IN-questions, and \textit{net} was more natural than \textit{da}. We resolved the interaction $\text{\textsc{particle}}\times\text{\textsc{soa}}$ by subsetting the data for each SoA. In the \textsc{done} context, \textit{da} received higher ratings than \textit{net} ($b=-1.43$, $\text{SE}=0.51$, $z=-2.81$, $p=0.005$). In the \textsc{not done} contexts, \textit{net} received higher ratings than \textit{da} ($b=3.95$, $\text{SE}=0.46$, $z=8.51$, $p<0.001$). Since \textsc{question type} did not interact reliably with the other two factors, we take the effect of question type to be present in both SoAs and for both particles. Looking at the medians, however, the effect becomes particularly visible for \textit{net} in the \textsc{done} contexts: After IN-questions \textit{net} has a median in the scale part towards naturalness ($\text{median}=5.5$) whereas after ON-questions \textit{net} has a median that is in the scale part towards unnaturalness ($\text{median}=3$). For \textit{da} in \textsc{not done} contexts we observe only differences in the scale part toward unnaturalness: \textit{da} is judged to be more unnatural after ON-questions ($\text{median}=2$) than after IN-questions ($\text{median}=3$).

Since previous research has found considerable inter-individual variation in the acceptability of response particles in various languages (\citealt{ClausMeijerReppKrifka2017}, \citealt{ReppMeijerScherf2019}), we investigated this issue for our data. Figures~\ref{geist-repp:fig:InterIndividualVariationOnQuestion} and~\ref{geist-repp:fig:InterIndividualVariationInQuestion} show the variation for ON-questions and for IN-questions respectively. The figures indicate that the variation is fairly similar. In \textsc{done} contexts, the majority of participants judge \textit{da} as natural (median~6 or~7), and as more natural than \textit{net}. There are a few participants, however, who judge \textit{net} more natural than \textit{da}, and some who find neither particle natural after ON-questions (median below~6). In \textsc{not done} contexts, almost all participants find \textit{net} natural whereas for \textit{da} naturalness ratings vary considerably.

% Figure2

\begin{figure}
%\includegraphics[width=\textwidth]{figures/Figure2-InterIndividualVariationOnQuestion.png}
\includegraphics[width=.8\textwidth]{figures/Figure2-InterIndividualVariationOnQuestion.pdf} 
\caption{Inter-individual variation in responses to ON-questions. Dot size represents the number of participants with the same combination of median rating for \textit{da} and median rating for \textit{net} for the respective SoA. Dots in the orange box represent participants for whom \textit{net} had a median of at least~6 and \textit{da} had a median of maximum~2, i.e. for whom the difference between the particles was very pronounced. Dots in the green box represent participants for whom \textit{da} had a median of at least~6 and \textit{net} had a median of maximum~2. Dots in the blue box represent participants for whom both \textit{da} and \textit{net} had a median of at least~6.}
\label{geist-repp:fig:InterIndividualVariationOnQuestion}
\end{figure}


% Figure3

\begin{figure}
\includegraphics[width=.8\textwidth]{figures/Figure3-InterIndividualVariationInQuestions.pdf}
\caption{Inter-individual variation in responses to IN-questions. For the coding system, see caption of \figref{geist-repp:fig:InterIndividualVariationOnQuestion}}
\label{geist-repp:fig:InterIndividualVariationInQuestion}
\end{figure}

To better assess the difference between the two question types, we plotted the inter-individual variation in a way that allows us to directly compare participants’ medians across question types, see \figref{geist-repp:fig:MedianRatingsPerParticipant}. \figref{geist-repp:fig:MedianRatingsPerParticipant} has two facets which indicate differences between the question types: For \textit{da} in the \textsc{not done} context, many dots are quite far away from the (perfect correlation) diagonal in both directions, which suggests that the speakers’ judgments for the two question types differ in scale direction. For \textit{net} in the \textsc{done} context, the dots are above the diagonal, which indicates generally higher ratings after IN-questions. Hence, we assume that there is a real difference for many speakers between the two question types here.

% Figure4

\begin{figure}
\includegraphics[width=.9\textwidth]{figures/Figure4-MedianRatingsPerParticipant.pdf}
\caption{Median ratings per participant for ON- vs. IN-questions. Dot size represents the number of participants with the same combination of median rating for ON-questions and for IN-questions. Dots on the diagonal line represent participants that had the same ratings for both question types. Dots in the grey bars represent ratings of~6 or~7 for IN-questions (horizontal bar) or ON-questions (vertical bar) or both (overlap of bars).}
\label{geist-repp:fig:MedianRatingsPerParticipant}
\end{figure}


\section{Discussion and Conclusion}\label{geist-repp:sec:Discussion-Conclusion}

\tabref{geist-repp:tab:ResultsAndPredictions} summarizes the results of our experiments in comparison to our predictions. Confirmed predictions are marked with \ding{51}. Unpredicted results are marked with \ding{55}. The table shows that many of our expectations were confirmed. Especially for ON-questions, our hypotheses seem to be on the right track: what is checked by an ON-question is a positive proposition \textit{p}, and \textit{p} is the proposition that serves as the antecedent for \textit{da} and \textit{net}. Accounts assuming an LF where ON-questions contain only a positive proposition can explain these findings. For IN-questions, we obtained several unexpected results, especially concerning \textit{da}. We will discuss these in detail in what follows.\largerpage[-1]\pagebreak

\begin{table}
\caption{Results and predictions}
\label{geist-repp:tab:ResultsAndPredictions}
\fittable{\begin{tabular}{lllll} 
\lsptoprule
& \multicolumn{4}{c}{Antecedent} \\\cmidrule(lr){2-5}
SoA  & ON-question &  & IN-question &\\
& \textit{Hasn’t ... already...?} & &\textit{Hasn’t ... yet...?} &\\
& \multicolumn{2}{l}{[Q~[\textsc{falsum}~\textit{p}]]} & \multicolumn{2}{l}{[Q~[\textsc{verum}~$\bar{p}$]]}\\\midrule
$p$ (\textsc{done}) & [$+,\efbox{\textsc{agree}}\,]\rightarrow\textit{da}$ &  \ding{51} & [$+,\,\efbox{\textsc{reverse}}\,]\rightarrow\textit{net}$  & \ding{51} \\
& & & & \ding{55}\,$da>net$ \\
\addlinespace
$\bar{p}$ (\textsc{not} \textsc{done}) & $[\,\efbox{\rule[0mm]{0mm}{2mm}$-$}\,,\textsc{reverse}]\rightarrow\textit{net}$ & \multirow{2}{*}{\ding{51}} & $[\,\efbox{\rule[0mm]{0mm}{2mm}$-$}\,,\textsc{agree}]\rightarrow\textit{net}$ & \ding{51}\\
& $[-,\,\efbox{\textsc{reverse}}\,]\rightarrow\textit{net}$ & & $[-,\efbox{\textsc{agree}}\,]\rightarrow\textit{da}$ & \ding{55} \textsuperscript{??}\textit{da}\\
& & & $net>da$ & \ding{51}\\
\lspbottomrule
\end{tabular}}
\end{table}

The high acceptability of \textit{da} in responses to IN-questions in \textsc{done} contexts ($\text{median}=7$) is completely unexpected. Recall that \textit{da} by hypothesis only realizes [\textsc{agree}], and an IN-question by hypothesis only makes the negative proposition $\bar{p}$ available. Since the response is supposed to express that \textit{p} is true, the presupposition of [\textsc{agree}] is not met. We conclude from this finding that either \textit{razve}-questions with \textit{eščë} do not have the LF proposed for IN-questions by \citet{RomeroHan2004} and Repp (\citeyear{Repp2006}, \citeyear{Repp2009}), or the hypothesis for \textit{da} that we developed on the basis of \citet{Esipova2021} and \citet{Gonzalez-FuenteTubauEspinalPrieto2015} is wrong. A third avenue for explaining the result is re-investigating the salience of the various propositions and the role of the particle \textit{razve}. We will discuss these three options for the \textsc{done} contexts and also consider the repercussions for the other contexts. 

Regarding the potential conclusion that IN-questions do not have the assumed LF, there is a finding in our experiments that in our view speaks against it: \textit{net} is fairly acceptable after IN-questions in \textsc{done} contexts ($\text{median}=5.5$), in contrast to ON-questions ($\text{median}=3$). Indeed, the median for \textit{net} is on the acceptable scale end for IN-questions, which is not the case for ON-questions. This finding suggests that an IN-question does make $\bar{p}$ available, which can serve as the antecedent that is required for the presupposition of [\textsc{reverse}] in a \textsc{done} context: [\textsc{reverse}] is the feature that is realized by \textit{net}.\footnote{Note that the high acceptability of \textit{net} in a \textsc{done} context does not parallel \citeposst{Meyer2004} empirical claims about unbiased questions: in Meyer’s example, \textit{net} is unacceptable in this context.}

Regarding a different feature-particle mapping for \textit{da}, we will consider two options: one makes the mapping more general, the other makes it more specific. Starting with the more general one, we could assume that instead of $[\textsc{agree}]\rightarrow \textit{da}$, the mapping is $[+],[\textsc{agree}]\rightarrow\textit{da}$, i.e. \textit{da} may realize [\textsc{agree}] as well as $[+]$, just like English \textit{yes}. This could explain the high ratings in the \textsc{done} context in IN-questions in the following way. If in Russian the constraint \textsc{realize absolute features} has a considerably higher weight than \textsc{maximize marked features} and than \textsc{realize relative features}, the observed preference for \textit{da} over \textit{net} in \textsc{done} contexts is explained: \textit{da} realizes absolute, unmarked $[+]$, \textit{net} realizes relative, marked [\textsc{reverse}]. This assumption could also explain the low ratings for \textit{da} after IN-questions in \textsc{not done} contexts ($\text{median}=3$), where \textit{da} realizes relative, unmarked [\textsc{agree}], whereas \textit{net} ($\text{median}=7$) realizes absolute, marked $[-]$. However, there also is a problem. Recall from \sectref{geist-repp:sec:Russian-response-particles-feature-model} that \citet{Esipova2021} claims that \textit{da} cannot be used in [+, \textsc{reverse}] contexts after negative assertions, see \REF{geist-repp:ex:Nina-ne-sdala-ekzamen-d} above. This claim is fully confirmed by experimental findings in \citet{ReppGeist2022}. So assuming that \textit{da} can realize $[+]$ seems to be on the wrong track because of substantial empirical differences between IN-questions and negative assertions as antecedents. We will return to this issue further below. 

The more specific feature-particle mapping that is a promising candidate to explain our findings is: $[+,\textsc{agree}]\rightarrow\textit{da}$. Here, we would have to assume that the presupposition of [\textsc{agree}] is fulfilled in IN-questions by the presence of the (less salient) epistemic bias \textit{p}, which -- recall our discussion in \sectref{geist-repp:sec:background} -- is an integral part of biased ON/IN-questions although this is not reflected in the LF of IN-questions. If \textit{da} realizes [$+,\textsc{agree}$], a high weighting of \textsc{expressiveness} will ensure the preference of \textit{da} over \textit{net} because \textit{da} realizes more features than \textit{net} does. This more specific feature-particle mapping would also be able to explain why \textit{da} is quite unacceptable ($\text{median}=3$) as a response to IN-questions in \textsc{not done} contexts: \textit{da} cannot express [\textsc{agree}] if the response clause is a negative proposition. However, the more specific feature-particle mapping also faces the problem that there is a difference with previous findings for assertions. Recall from \sectref{geist-repp:sec:Russian-response-particles-feature-model} that \citet{Esipova2021} claims that \textit{da} is acceptable in \textsc{not done} contexts if the antecedent is a negative assertion, see \REF{geist-repp:ex:Nina-ne-sdala-ekzamen-b} above -- the answer with the features [$-$, \efbox{\textsc{agree}}\,]. \citet{ReppGeist2022} present experimental evidence supporting this claim, at least to some extent.

Regarding the salience assumptions, we could also take a more drastic step and assume that the epistemic bias \textit{p} is made very salient by the interrogative particle \textit{razve}, so that \textit{p} is more salient than the evidential bias $\bar{p}$, which is part of the LF of IN-questions. On this assumption, we would not have to alter the feature-particle mapping [\textsc{agree}] for \textit{da} because after IN-questions in \textsc{done} contexts \textit{da} just picks up the more salient proposition \textit{p} and therefore is more acceptable than \textit{net} (median~7 vs.~5.5). After IN-questions in \textsc{not done} contexts, \textit{da} is expected to be unacceptable because signalling the same polarity of epistemic bias and response does not express the intended meaning $\bar{p}$. To test the relative salience of the biases in \textit{razve}-questions, follow-up studies with other interrogative particles are needed. Note, however, that the sketched salience account essentially assumes the same salience differences between \textit{p} and $\bar{p}$ in IN- and ON-questions, so that subtle differences between the question types -- for instance in responses with \textit{net} -- cannot be explained.

An anonymous reviewer suggests that by using \textit{da} the speaker indicates agreement with the interlocutor’s epistemic bias independently of salience considerations. This proposal could indeed explain the patterns for ON- and IN-questions for \textit{da}, because for \textit{da} the difference does not seem to matter (a lot). It would also be compatible with the observation that \textit{da} can be used to signal agreement with a negative assertion (\citealt{Esipova2021}, \citealt{ReppGeist2022}), because asserting $\bar{p}$  plausibly presupposes having a bias for $\bar{p}$. Finally, this proposal would also be compatible with the observation in \citet{ReppGeist2022} that \textit{da} is clearly degraded in responses to unbiased negative questions, independently of the response polarity (see \sectref{geist-repp:sec:Russian-response-particles-feature-model}). However, intuitively, \textit{da} seems to be the appropriate answer to a positive question with \textit{razve}, like~B in \REF{geist-repp:ex:ivan-otpusk} in \sectref{geist-repp:sec:question-bias-negation}, if the response polarity is positive:

\ea\label{geist-repp:ex:he-married}
\begin{xlist}[M:]
    \exi{B:} \gll A 	razve	on	ženat?\\
but 	\textsc{part}	he	married \\
\glt ‘But is he really married?’ \\
    \exi{A:} \gll Da,	 on	ženat.\\
yes	he	married\\
\glt ‘Yes, he is married.’ \\ 
\end{xlist}
\z

\noindent As laid out in \sectref{geist-repp:sec:question-bias-negation}, the epistemic bias of B in this example is $\bar{p}$. A does not agree with this bias, but with the evidential bias. The evidential bias is the bias that arguably is made salient by the question. 

In the final part of this discussion, we will sketch a way to reconcile the observed differences between questions and assertions as antecedents. We think that these differences can only be explained on the assumption that \textit{da} is ambiguous, and that the ambiguity must involve a presupposition regarding the type of antecedent. At present we cannot decide between the mappings that we discussed to account for our results for ON/IN-questions, $[+,\textsc{agree}]\rightarrow\textit{da}$, or $[+]$, $[\textsc{agree}]\rightarrow\textit{da}$. The former has the advantage that it is more parsimonious in the overall setup because there will be less ambiguity, but the choice is an empirical question that must be addressed in future research. 

Our new proposal is that \textit{da} also can realize a feature that we will call [\textsc{accept}]. \REF{geist-repp:ex:ACCEPT-presupposes} gives the presupposition of [\textsc{accept}] in abbreviated form. It contains an illocutionary component: the conversational table (\citealt{FarkasBruce2010}). 

\eanoraggedright\label{geist-repp:ex:ACCEPT-presupposes}
[\textsc{accept}] presupposes the existence of a single proposition on the conversational table, which has the same polarity as the response clause. 
\z

\noindent \REF{geist-repp:ex:ACCEPT-presupposes} shows that [\textsc{accept}] is sensitive to how many propositions there are on the table. We have no space to discuss this here but we assume that questions place a set of propositions on the table, which might be more or less salient, and it is up to the addressee to decide which proposition enters the common ground (if any). Assertions place only one proposition on the table. \citet{RoelofsenFarkas2015} emphasize that for any anaphor, including response particles, there must be a \textit{unique salient} antecedent in the context. The presupposition in \REF{geist-repp:ex:ACCEPT-presupposes} is stricter than that: it allows only one proposition on the table at all, irrespective of the non-salience of potential other propositions. Assuming that a constraint like \textsc{Maximize presupposition!} (\citealt{Heim1991}) is generally operative, [\textsc{accept}] will be the feature that is relevant in responses to assertions. In responses to questions there will be a presupposition failure for [\textsc{accept}], so that (one of) the other feature-particle mapping(s) for \textit{da} applies (depending on the answers regarding the future research questions above, $[+,\textsc{agree}]\rightarrow\textit{da}$ or $[+],[\textsc{agree}]\rightarrow\textit{da}$).

We are not the first to suggest that questions and assertions receive different responses. \citet{Holmberg2015} has made suggestions along these lines for English. Similarly, \citet{ReppClausFruehauf} propose for German that \textit{nein} ‘no’ is used to express a counterpart of [\textsc{accept}] in responses to assertions, namely [\textsc{reject}]. The observed differences require much more quantitative empirical research, also because there is substantial inter-individual variation, as we could also verify for Russian. 

Overall, our investigation has shown that the answer patterns for Russian \textit{da/net} differ depending on whether the antecedent is an IN-question or an ON-question. We have also discussed some differences with assertions, which, however, were not the focus of the present study. On the basis of our findings, we assume that \textit{da} and \textit{net} are sensitive to the interpretation of the negation in biased questions with \textit{razve}, as it is indicated by the polarity-sensitive items \textit{uže} and \textit{eščë}. The account of inner vs. outer negation in terms of propositional negation vs. the illocutionary operator \textsc{falsum} goes some way to explaining the answer patterns for these questions. However, we also saw that we might have to make additional assumptions concerning the salience of a bias that is not double-checked. This is an issue that needs further attention in future research as it poses interesting empirical and theoretical challenges. Specifically, we need to find out more about potential differences in salience between epistemic bias and evidential bias. After all, the evidential bias for $\bar{p}$ does not seem to play a role for responses to ON-questions. Furthermore, we need a model that integrates the biases in a more explicit way, which explains how they become part of the discourse representation.

\section*{Acknowledgments}

The authors are in alphabetical order. They wish to thank Olav Mueller-Reichau and two anonymous reviewers for their very helpful comments. We are also grateful to Radek Šimík for technical support and to the audiences at the conference FDSL-14 (Leipzig University, 2.06.2021), at the Slavistics Colloquium at the Humboldt-Universität zu Berlin, 17.02.2021, and at the conference \textit{Biased Questions: Experimental Results \& Theoretical Modelling} (Leibniz-Zentrum Allgemeine Sprachwissenschaft, 4.02.2021) for discussion and comments. This research was funded by the German Research Council DFG in the priority program XPrag.de (SPP 1727), project \textit{Affirmative and rejecting responses to assertions and polar questions~2} (Repp).

\printbibliography[heading=subbibliography,notkeyword=this]

\end{document}
