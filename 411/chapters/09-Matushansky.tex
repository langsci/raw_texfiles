\documentclass[output=paper,colorlinks,citecolor=brown]{langscibook}
\ChapterDOI{10.5281/zenodo.10123643}

\author{Ora Matushansky\affiliation{SFL (CNRS/Université Paris-8/UPL); ILS (Utrecht University)}\orcid{0000-0002-1420-7091}}
\SetupAffiliations{mark style=none}


\title{Phi-congruence and case agreement in close apposition in Russian}
\abstract{I demonstrate that case-marking on the proper name in close apposition in Russian depends on two factors: the semantic sort of the proper name (where object-denoting proper names differ from place-denoting proper names, i.e., toponyms) and within the latter category on the lexical-semantic class of the toponym: major landmarks, such as cities and countries, special landmarks (rivers, streets, etc.) and the rest. While animate proper names necessarily agree in case with their sortals and inanimate ones obligatorily appear in the nominative case, case agreement with toponyms is conditioned by phi-congruence: cities and countries require number congruence, special landmarks need gender congruence and for the residue only phi-congruent adjectival toponyms may agree in case. I suggest that the phi-congruence condition should be analyzed as semantic agreement and hypothesize that toponyms differ from object-denoting proper names in that the former may have interpretable phi-features despite being inanimate.

\keywords{case, semantic agreement, close apposition, Russian, phi-features}
}

\lsConditionalSetupForPaper{}

\begin{document}
\maketitle


\section{Introduction}\label{mat:sec:intro}

Proper names in Russian are divided into three categories in function of their case-marking in close apposition: those that must agree in case with the sortal (animates), those that can agree (toponyms) and those that cannot agree (the rest). Within the second category the possibility of case agreement is conditioned by \textsc{phi-congruence}: the values of certain phi-features of the toponym must match those of the sortal. The question arises how to model these facts and what they tell us about the nature of agreement.

I begin with the presentation of the phenomenon of close apposition in general (\sectref{mat:sec:close-apposition}). In \sectref{mat:sec:case-marking-proper-names} I will describe what is known about the empirical landscape of case-agreement in close apposition in Russian, focusing on toponyms and the impact of phi-congruence. \sectref{mat:sec:excluded-hypotheses} will deal with and dismiss several possible analyses of these facts. \sectref{mat:sec:toponyms-as-semantic-sort} is dedicated to a sketch of a proposal, linking case agreement to semantic agreement. \sectref{mat:sec:Conclusion} concludes.

\section{Close apposition}\label{mat:sec:close-apposition}

Appositions can be defined as a single constituent containing more than one NP yet only one referent. The obvious difference between close apposition in \REF{mat:ex:element-material-actor} and loose apposition in \REF{mat:ex:Jackendoff-Callas} is that in \REF{mat:ex:element-material-actor} there is no intervening pause and in \REF{mat:ex:Jackendoff-Callas} the proper name or kind name alone refers to the same individual as the NP combining with it:

\ea\label{mat:ex:element-material-actor}
\ea[]{the element engoopium}\label{mat:ex:element-material-actor-a}
\ex[]{the material polyacrynilate\label{mat:ex:element-material-actor-b}
}
\ex[]{the actor John Gielgud \label{mat:ex:element-material-actor-c} \hfill (\citealt{Jackendoff1984})}
\z
\z

\ea\label{mat:ex:Jackendoff-Callas}
\renewcommand{\labelenumi}{\alph{enumi}.}
\begin{enumerate}
\item This element, engoopium, was invented by Ray Jackendoff. 
\item The prima/Maria Callas, the best Carmen ever, outsings everyone in this role.
\end{enumerate}
\z

\noindent In the type of close apposition exemplified in \REF{mat:ex:element-material-actor} the first noun (henceforth, \textsc{the sortal}) is the syntactic head (\citealt{Jackendoff1984}, \citealt{Lasersohn1986}, \citealt{McCawley1996}, \citeyear{McCawley1998}; contra \citealt{Haugen1953}, \citealt{BurtonRoberts1975}, \citealt{Noailly1991}, \citealt{Keizer2005}), as shown by the fact that agreement is determined by the phi-features of the sortal rather than by those of the proper name (or the second noun), as in the Russian example \REF{mat:ex:krejser-avrora}, and that the case assigned to the NP as a whole must surface on the sortal (and may, on the proper name), as in \REF{mat:ex:ulica-Jakimanka}: \footnote{We set aside here several other types of close apposition, such as \textit{Francis Bacon the philosopher} (restrictive, picking out one of the possible name bearers), \textit{Karl Marx the Jew} (singling out a particular guise or aspect of an individual) or other marginal instances where NP\textsubscript{2} is headed by a common noun and contains an overt determiner, since those of them that can be reliably translated into Russian are all animate and exhibit obligatory case agreement.}\footnote{\label{mat:foot:TheFirstGeneralization}The first generalization seems to be contradicted by animate proper names, where the sortal may be masculine while the proper name (and the referent) is feminine, as in \textit{doktor Liza}, which triggers feminine agreement. This contradiction is only apparent, as human-denoting nouns in Russian can agree semantically (\citealt{Corbett1979}, see also Footnote~\ref{mat:foot:OneMightObject}).} 

\ea\label{mat:ex:krejser-avrora}
\gll Krejser	\minsp{“} Avrora”	ne	\minsp{\{} streljal / \minsp{*} streljala\}. \\
cruiser.{\MASC} {} Aurora.{\FEM}	{\NEG}	{} fired.{\MASC.\SG} {} {} fired.{\FEM.\SG}\\  
\glt `The cruiser Aurora was not firing.' \hfill agreement
\z

\ea\label{mat:ex:ulica-Jakimanka}
\gll na	\minsp{\{} ulice / \minsp{*} ulica\}	\minsp{\{} Jakimanke / Jakimanka\}\\
in	{} street.{\FEM.\SG.\LOC} {} {} street.{\FEM.\SG.\NOM}	{} Yakimanka.{\FEM.\SG.\LOC} {} Yakimanka.{\FEM.\SG.\NOM}\\ 
\glt `on the Yakimanka street' \hfill case
\z

\noindent The fact that the sortal NP may contain a complement (\REF{mat:ex:former-president-rejgan}, after \citealt[473]{McCawley1998}) shows that the proper name, which is clearly not a semantic argument of the sortal anyway, must be treated as a modifier (\figref{mat:fig:ronald-reagan}).

\ea\label{mat:ex:former-president-rejgan}
\gll byvšij	prezident	SŠA	i	gollivudskaja	kinozvezda Ronal'd Rejgan\\
former	president	USA.{\GEN}	and	Hollywood.{\ADJ} movie~star	Ronald Reagan\\
\glt `the former president of the US and Hollywood star Ronald Reagan' 
\z

\begin{figure}[ht]
    \centering
\begin{forest}
[DP
[D\textsuperscript{0}[\textit{the}]]
[NP\textsubscript{1}
[NP\textsubscript{1}
    [\textit{former president of the US and Hollywood star},roof]
]
[NP\textsubscript{2}
    [\textit{Ronald Reagan},roof]
    ]
]
]
\end{forest}
    \caption{Close apposition: internal structure}
    \label{mat:fig:ronald-reagan}
\end{figure}

Case-marking in close apposition \REF{mat:ex:ulica-Jakimanka} not only offers insight into its internal structure, but also suggests that agreement can occur between two noun phrases, as the choice of case can be conditioned by phi-congruence of the two nouns: having the same values for the gender and number features on the proper name as on the sortal may be a necessary condition for having an agreeing case on the proper name.

\section{Case-marking in close apposition with proper names}\label{mat:sec:case-marking-proper-names}

As \REF{mat:ex:ulica-Jakimanka} shows, close apposition permits two options for the proper name: the proper name can bear either the same case as the sortal or the default nominative case. The availability of either option depends on the lexical-semantic class of the proper name. Three broad groups can be established:

\ea\label{mat:ex:animate-referents}animate referents: obligatory case agreement\\
\ea[]{\label{mat:ex:animate-referents-a}
\gll o	russkom	poėte	\minsp{\{} Bloke /\hspace{2cm} \minsp{*} Blok\}\\
about	Russian.{\MASC.\SG}.{\LOC}	poet.{\MASC.\SG}.{\LOC} {} Blok.{\MASC.\SG.\LOC} {} {} Blok.{\MASC.\SG}.{\NOM}\\
\glt `about the Russian poet Blok' \hfill [$+\text{animate}$]}
\ex[]{
\gll o	russkom	poėte	\minsp{\{} Cvetaevoj /\hspace{1.3cm} \minsp{*} Cvetaeva\}\\
about	Russian.{\MASC.\SG}.{\LOC}	poet.{\MASC.\SG}.{\LOC}	{} Tsvetaeva.{\FEM.\SG.}{\LOC} {} {} Tsvetaeva.{\FEM.\SG.\NOM}\\
\glt `about the Russian poet Tsvetaeva'\label{mat:ex:animate-referents-b}
}
\z
\ex\label{mat:ex:non-toponymic-proper-names}non-toponymic proper names: forbidden case agreement\\
\ea[]{\label{mat:ex:non-toponymic-proper-names-a}
\gll s	familiej	\minsp{\{} Blok / \minsp{*} Blokom\}\\
with	surname.{\INS} {}	Blok.{\NOM} {} {} Blok.{\INS}\\
\glt `with the surname Blok'}
\ex[]{
\gll o	krejsere	\minsp{\{} \minsp{“} Moskva” / \minsp{*} \minsp{“} Moskve”\}\\
about	cruiser.{\MASC.\SG}.{\LOC}	{}	{} Moscow.{\FEM.\SG}.{\NOM} {} {} {} Moscow.{\FEM.\SG.\LOC}\\
\glt `about the cruiser Moscow' \label{mat:ex:non-toponymic-proper-names-b}
}
\z
\ex\label{mat:ex:toponyms-case-agreement}toponyms: case agreement restricted by phi-feature congruence\\
\ea[]{\label{mat:ex:toponyms-case-agreement-a}
\gll na	ulice	\minsp{\{} Jakimanka / Jakimanke\}\\
in	street.{\FEM.\SG.}{\LOC} {}	Yakimanka.{\FEM.\SG.}{\NOM} {} Yakimanka.{\FEM.\SG.\LOC}\\
\glt `on the Yakimanka street' \hfill   \ding{51} phi-congruent }
\ex[]{
\gll na	ulice	\minsp{\{} Balčug / \minsp{*} Balčuge\}\\
in	street.{\FEM.\SG.}{\LOC}	{} Balčug.{\MASC.\SG}.{\NOM} {} {} Balčug.{\MASC.\SG.\LOC} \\
\glt `on the Balčug street' \hfill  \ding{55} phi-congruent \label{mat:ex:toponyms-case-agreement-b}
}
\z
\z


\noindent Close apposition is also possible with kind names, as in \REF{mat:ex:element-material-actor-b}--\REF{mat:ex:element-material-actor-c} and \REF{mat:ex:uroven-gormona}. The eight native speakers I asked split fifty-fifty as to which variant they accept and no one has accepted both, so kind names seem to pattern either as city/country names or as non-toponymic names.\footnote{\label{mat:foot:CloseAppositionAlsoPossible}Month names allow only the odd reversed construction in \REF{mat:ex:v-marte-mecjace} with obligatory agreement. Sortals are not used with days of the week or event names (like \textit{WWII}), perhaps because these proper names refer unambiguously and so a sortal is pragmatically infelicitous. Finally, holiday names allow the appositive oblique, as in \REF{mat:ex:prazdnik-pasxi}, as do the names of galaxies, constellations, and certain others, see \citet{Logvinova2018,LogvinovaInPress}.

\ea\label{mat:ex:v-marte-mecjace}
\gll v	marte	mesjace\\
in	March.{\LOC}	month.{\LOC}\\
\glt `in the month of March'\\
\ex\label{mat:ex:prazdnik-pasxi}
\gll 	prazdnik	Pasxi\\
holiday	Easter.{\GEN} \\
\glt `the Holiday of Easter'\\
\z}
 

\ea\label{mat:ex:uroven-gormona} 
\gll uroven'	gormona	\minsp{\{} \minsp{\%} kortizol / \minsp{\%} kortizola\}\\
level	hormone.{\GEN} {}	{} cortisol.{\NOM} {} {} cortisol.{\GEN}\\
\glt `the level of the hormone cortisol'\\
\z


\noindent While with animate referents \REF{mat:ex:animate-referents} non-agreeing case on the proper name is disallowed in close apposition, with non-toponymic proper names \REF{mat:ex:non-toponymic-proper-names} nominative is required on the proper name. Finally, for the third category, which only contains toponyms, both options are possible and, as \REF{mat:ex:toponyms-case-agreement} shows, the availability of the agreeing option is conditioned by their phi-features.

The focus of this paper is on case agreement for toponyms, which has been shown to depend on phi-congruence, i.e., on whether the sortal and the proper name match in phi-features. In addition to prescriptivist works like \citet{RozentalKabanova1998}, two corpus studies, \citet{GraudinaKatlinskaja1976} and \citet{Logvinova2018}, show that within the broad category of toponyms different lexical-semantic classes can be distinguished in function of whether they require matching only in number or also in gender. After having examined the empirical picture provided by these works and discussed which deviations from these patterns are possible and why,\footnote{Most of the generalizations below come from \citet{GraudinaKatlinskaja1976} and \citet{RozentalKabanova1998} and are verified by \citet{Logvinova2018, LogvinovaInPress}. Deviations from and extensions of the patterns described there have been cross-checked in the National Russian Language Corpus \citep{RNC}, on Google, and with some native speakers.} I will argue (\sectref{mat:sec:excluded-hypotheses}) that the first hypotheses that come to mind cannot account for them and then advance an approach based on the assumption that phi-congruence enables semantic agreement (\sectref{mat:sec:toponyms-as-semantic-sort}).

\subsection{Number congruence and optional case agreement}\label{mat:sec:number-congruence-case-agreement}

The most permissive category of toponyms are proper names introduced by the sortals \textit{gorod} ‘city, town’ (M), \textit{stolica} ‘capital’ (F) and \textit{strana} ‘country’ (F) (although not the coextensional \textit{gosudarstvo} ‘state’ (N)), where agreeing and non-agreeing cases can be in free variation with no obvious difference in interpretation. Yet the phi-feature specification of the proper name is relevant for case-agreement, as can be seen from morphologically plural proper names. While both masculine and feminine city and country names generally allow case agreement~\REF{mat:ex:moskva-tallin-francija-kitaj}, plural ones, as in~\REF{mat:ex:gagry-velikieLuki}, do not (\citealt[141]{GraudinaKatlinskaja1976}, \citealt[281]{RozentalKabanova1998}, confirmed by \citealt[25--28]{Logvinova2018}, \citeyear{LogvinovaInPress}; the same is true for Ukrainian, see \citealt{Gorpinic1987}):\footnote{\label{mat:foot:ForTheSakeOfSimplicity}For the sake of simplicity I avoid neuter toponyms, as these tend to behave as indeclinables, appearing in the nominative even without a sortal (\citealt[138--140]{GraudinaKatlinskaja1976}). The neuter sortals \textit{selo} ‘village’ and \textit{gosudarstvo} ‘state’ avoid case agreement even with phi-congruent toponyms, though the former allows it with phi-congruent adjectival proper nouns (there exist no adjectival state names).} 

\ea\label{mat:ex:moskva-tallin-francija-kitaj}
\ea[]{
\gll v	gorode	\minsp{\{} Moskva / Moskve\}\\
in	city.{\MASC.\SG}.{\LOC}	{} Moscow.{\FEM.\SG.}{\NOM} {} Moscow.{\FEM.\SG.\LOC}\\
\glt `in the city of Moscow' \hfill singular sortal, feminine PN \label{mat:ex:moskva-tallin-francija-kitaj-a}}
\ex[]{
\gll v	gorode	\minsp{\{} Tallinn / Tallinne\}\\
in	city.{\MASC.\SG}.{\LOC}	{} Tallinn.{\MASC.\SG}.{\NOM} {} Tallinn.{\MASC.\SG.\LOC}\\
\glt `in the city of Tallinn' \hfill singular sortal, masculine PN \label{mat:ex:moskva-tallin-francija-kitaj-b}
}
\ex[]{
\gll o	strane	\minsp{\{} Francija / Francii\} \\
about	country.{\FEM.\SG.}{\LOC}	{} France.{\FEM.\SG.}{\NOM} {} France.{\FEM.\SG.\LOC}\\
\glt `about the great country France' \hfill feminine sortal, feminine PN \label{mat:ex:moskva-tallin-francija-kitaj-c}
}
\ex[]{
\gll o	strane	\minsp{\{} Kitaj / Kitaje\}\\
about	country.{\FEM.\SG.}{\LOC}	{} China.{\MASC.\SG}.{\NOM} {} China.{\MASC.\SG.\LOC}\\
\glt `about the great country China' \hfill feminine sortal, masculine PN \label{mat:ex:moskva-tallin-francija-kitaj-d}
}
\z
\z

\ea\label{mat:ex:gagry-velikieLuki}
\ea[]{
\gll v	gorode	\minsp{\{} Gagry / \minsp{*} Gagrax\}\\
in	city.{\MASC.\SG}.{\LOC}	{} Gagra.{\PL}.{\NOM} {} {} Gagra.{\PL.\LOC}\\
\glt `in the city of Gagra' \hfill singular sortal, plural PN}
\ex[]{
\gll v	gorode	\minsp{\{} Velikie Luki / \minsp{*} Velikix Lukax\}\\
in	city.{\MASC.\SG}.{\LOC}	{} Velikie Luki.{\PL}.{\NOM} {} {} Velikie Luki.{\PL.\LOC} \\
\glt `in the city of Velikie Luki' \hfill singular sortal, complex plural PN
}
\z
\z

\noindent A caveat should be introduced here. The non-agreeing pattern is an innovation in the history of Russian and is anecdotally taken to have arisen as a response to the logistical challenges of WWI, when the use of the nominative form after a sortal could distinguish one location from another. Prior to that time the preference was for case agreement between the sortal and the proper name, and this pattern is still attested even for number-incongruent proper names. While \REF{mat:ex:jocasta-pljaziGagry-a} can be taken to result from artificial archaization, \REF{mat:ex:jocasta-pljaziGagry-b}, forming a near-minimal pair with \REF{mat:ex:gagry-velikieLuki}, is taken from a recent article about paragliding, indicating that the language change is still in progress.

\ea\label{mat:ex:jocasta-pljaziGagry}
\ea[]{
\gll V	gorode	Fivax	pravili	car'	Laj	i	carica	Iokasta.{\footnotemark}\\
in	city.{\MASC.\SG}.{\LOC}	Thebes.{\PL}.{\LOC}	ruled	king	Laius	and	queen	Jocasta\\
\footnotetext{\url{https://lit.wikireading.ru/hbGcTPBY34}}
\glt `King Laius and Queen Jocasta ruled in the city of Thebes.' }\label{mat:ex:jocasta-pljaziGagry-a}
\ex[]{
\gll V	majskie	prazdniki	on	paril	nad	pljažami	v	kurortnom	gorode	Gagrax.{\footnotemark}\\
in	May	holidays	he	soared	above	beaches	in	resort.{\MASC.\SG.\LOC}	town.{\MASC.\SG.\LOC}	Gagra.{\PL}.{\LOC}\\
\footnotetext{\url{http://www.paraplanerism.ru/kolomenskoe.php}}
\glt `During May holidays he soared above the beaches in the resort town of Gagra.' \label{mat:ex:jocasta-pljaziGagry-b}
}
\z
\z

\noindent Examples \REF{mat:ex:jocasta-pljaziGagry} were not ungrammatical for some of the native speakers I consulted, including those who, when asked earlier about \REF{mat:ex:gagry-velikieLuki}, had rejected the agreeing variant.

There also exists a more restrictive group of speakers, who reject agreeing case on a country or city name that is not gender-congruent with the sortal (see also \citealt[281]{RozentalKabanova1998}). This is in fact the pattern described by \citet{RozentalKabanova1998} for toponymic sortals other than the masculine \textit{gorod} ‘city, town’ and the feminine \textit{strana} ‘country’. In most of the current usage, however, as shown by the statistical data in \citet[43]{Logvinova2018}, case agreement with the masculine sortal \textit{gorod} ‘city, town’ is not affected by gender. Interestingly, however, there is one context where the gender factor seems active for this category:

\ea\label{mat:ex:dva-goroda}
\ea[]{
\gll A	vy	znaete,	čto	v	Rossii	est'	dva	goroda	\minsp{\{} Pavlovska / \minsp{*} Pavlovsk\}?\\
and	you.{\PL}	know.2{\PL}	that	in	Russia	is	two	city.{\GEN}	{} Pavlovsk.{\GEN} {} {} Pavlovsk.{\NOM}\\
\glt `Are you aware that there are two cities named Pavlovsk in Russia?' }
\ex[]{
\gll A	vy	znaete,	čto	v	Štatax	est'	dva	goroda	\minsp{\{} Moskva / \minsp{*} Moskvy\}?\\
and	you.{\PL}	know.2{\PL}	that	in	States	is	two	city.{\GEN}	{} Moscow.{\NOM} {} {} Moscow.{\GEN} \\
\glt `Are you aware that there are two cities named Moscow in the States?'
}
\z
\z

\noindent The fact that in exactly the same environment case agreement is grammatical for a masculine toponym and ungrammatical, for a feminine one, strongly suggests that case agreement is not correlated with a major difference in meaning.

\subsection{Gender congruence as a condition on case agreement}\label{mat:sec:gender-congruence-condition}

For the sortals \textit{derevnja} ‘village’, \textit{selo} ‘village’, \textit{posëlok} ‘village’, \textit{reka} ‘river’, \textit{xutor} ‘farm’ and \textit{ulica} ‘street’ (the exact list varies from source to source, and \citealt{Logvinova2018} claims that in contemporary Russian \textit{reka} ‘river’ and \textit{gora} ‘mountain’ reflect this tendency), the toponym in apposition does not agree in case unless congruent with the sortal both in number and in gender (\citealt[281]{RozentalKabanova1998}, \citealt[140]{GraudinaKatlinskaja1976}):

\ea\label{mat:ex:Jakimanka-Balcug-CistyePrudy}
\ea[]{
\gll na	ulice	\minsp{\{} Jakimanka / Jakimanke\}\\
in	street.{\FEM.\SG.}{\LOC}	{} Yakimanka.{\FEM.\SG.}{\NOM} {} Yakimanka.{\FEM.\SG.\LOC} \\
\glt `on the Yakimanka street' \hfill \ding{51} phi-congruent \label{mat:ex:Jakimanka-Balcug-CistyePrudy-a}}
\ex[]{
\gll na	ulice	\minsp{\{} Balčug / \minsp{*} Balčuge\}\\
in	street.{\FEM.\SG.}{\LOC} {}	Balčug.{\MASC.\SG}.{\NOM} {} {} Balčug.{\MASC.\SG.\LOC} \\
\glt `on the Balčug street' \hfill \ding{55} phi-congruent \label{mat:ex:Jakimanka-Balcug-CistyePrudy-b}
}
\ex[]{
\gll na	ulice	\minsp{\{} Čistye Prudy / \minsp{*} Čistyx Prudax\}\\
in	street.{\FEM.\SG.}{\LOC} {}	Čistye Prudy.{\MASC}.{\PL}.{\NOM} {} {} Čistye Prudy.{\MASC.\PL.\LOC} \\
\glt `on the Čistye Prudy street' \hfill \ding{55} phi-congruent \label{mat:ex:Jakimanka-Balcug-CistyePrudy-c}
}
\z
\z

\noindent The lack of agreement in \REF{mat:ex:Jakimanka-Balcug-CistyePrudy-c}, containing a masculine plural proper name with a feminine singular sortal, could be due to number incongruence, gender incongruence or both (the lack of familiar plural street names precludes the construction of a gender-congruent example). Moreover, the toponym in \REF{mat:ex:Jakimanka-Balcug-CistyePrudy-c} is also internally complex, which, oddly enough, introduces an additional factor to be discussed in \sectref{mat:sec:complex-toponyms}. Since gender is not syntactically active in the plural in Russian, the question arises if number congruence in this category of toponyms should be analyzed as a separate factor, which it is for the toponyms discussed in the previous section, or as merely reflecting the syntactic inactivity of gender in the plural.

\subsection{Case agreement with number-congruent pluralia tantum toponyms}\label{mat:sec:case-agreement-number-congruent-pluralia}

While \citet{GraudinaKatlinskaja1976}, \citet{RozentalKabanova1998} and other prescriptive sources agree that morphologically plural toponyms disallow agreement, such is not the case when the sortal itself is plural, as with archipelagos \REF{mat:ex:maldivy-alpy-a} or mountain chains \REF{mat:ex:maldivy-alpy-b}:

\ea\label{mat:ex:maldivy-alpy}
\ea[]{
\gll Kak	žit'	na	rajskix	ostrovax	Mal'divax	za	suščie	groši?{\footnotemark}\\
how	live.{\INF}	on	Paradise.{\ADJ}	islands.{\LOC}	Maldives.{\LOC}	for	real	pennies\\
\footnotetext{\url{https://arissston.livejournal.com/140512.html}}
\glt `How to live in the island paradise of the Maldives for peanuts?' }\label{mat:ex:maldivy-alpy-a}
\ex[]{
\gll gorami	Al'pami{\footnotemark}\\
mountains.{\PL}.{\INS}	Alps.{\PL}.{\INS}\\
\footnotetext{\url{https://limon.kg/ru/news:67260}}
\glt `with the Alps' \label{mat:ex:maldivy-alpy-b}
}
\z
\z

\noindent Confirming this observation, \citet{Logvinova2018} also points out that case agreement is possible when a plural sortal is followed by a conjunction of singular toponyms:

\ea\label{mat:ex:balakov-saratov}
\gll v	gorodax	Balakove	i	Saratove \\
in	city.{\MASC}.{\PL}.{\LOC}	Balakov.{\MASC}.{\LOC}	and	Saratov.{\MASC}.{\LOC}\\
\glt 'in the cities of Balakov and Saratov' \hfill (\citealt{Logvinova2018}) 
\z

\noindent As the proper name here is a conjunction of two singular toponyms and is therefore plural only by virtue of its semantics, it cannot be argued that number congruence as a precondition for case agreement is ensured by the proper name agreeing with the sortal.

\subsection{Case agreement with phi-congruent adjectival proper names only}\label{mat:sec:case-agreement-phiCongruent}

For the remaining categories of toponyms case agreement in close apposition is possible only with morphologically adjectival toponyms on the condition of both gender and number congruence with their sortals:

\ea\label{mat:ex:Bologoe-Moskva-Tixoreckaja}
\ea[]{
\gll do	stancii	\minsp{\{} Bologoe / \minsp{*} Bologogo\}\\
until	station.{\FEM.\SG.}{\GEN} {} Bologoe.{\NEUT.\SG}.{\NOM} {} {} Bologoe.{\NEUT.\SG.\GEN} \\
\glt `until the station Bologoe' \hfill \ding{55} phi-congruent, \ding{51} adjective \label{mat:ex:Bologoe-Moskva-Tixoreckaja-a}}
\ex[]{
\gll do	stancii	\minsp{\{} Moskva / \minsp{*} Moskvy\}\\
until	station.{\FEM.\SG.}{\GEN}	{} Moscow.{\FEM.\SG.}{\NOM} {} {} Moscow.{\FEM.\SG.\GEN} \\
\glt `until the station Moscow' \hfill \ding{51} phi-congruent, \ding{55} adjective \label{mat:ex:Bologoe-Moskva-Tixoreckaja-b}
}
\ex[]{
\gll do	stancii	\minsp{\{} Tixoreckaja / Tixoreckoj\}\\
until	station.{\FEM.\SG.}{\GEN} {}	Tixoreckaja.{\FEM.\SG.}{\NOM} {} Tixoreckaja.{\FEM.\SG.\GEN} \\
\glt `until the station Tixoreckaja' \hfill \ding{51} phi-congruent, \ding{51} adjective \label{mat:ex:Bologoe-Moskva-Tixoreckaja-c}
}
\z
\z

\begin{sloppypar}
\noindent An incomplete list of such sortals includes ports, lakes, bays, volcanoes, hills (especially the Far Eastern \textit{sopka}), mountains, planets, and railway stations. Prescriptive grammars may insist that case agreement is impossible with such proper names or include in this list islands, republics, etc., but this is because adjectival toponyms are rarely considered. Thus, toponyms preceded by the sortals \textit{aul} ‘a village in the Caucasus and Central Asia’ and \textit{kišlak} ‘a village in Central Asia’ are often claimed to never agree for case, but this is because the names of such villages are extremely unlikely to be morphologically adjectival: when an adjectival toponym is used, case agreement becomes possible:
\end{sloppypar}

\ea\label{mat:ex:aul-Severnyj}
\gll v	kišlake / aule	\minsp{\{} Severnom / Severnyj\} \\
in	kishlak.{\MASC.\SG}.{\LOC} {} aul.{\MASC.\SG}.{\LOC}	{} Northern.{\MASC.\SG}.{\LOC} {} Northern.{\MASC.\SG.\NOM}\\
\glt `in the kishlak / aul Severnyj'
\z 

\noindent The observation (\citealt[143]{GraudinaKatlinskaja1976}, confirmed by Logvinova) that foreign toponyms do not agree in case when combining with such sortals as \textit{štat} ‘state’, \textit{respublika} ‘republic’, etc., is explained by the non-existence of morphologically adjectival foreign proper names.

As far as I could ascertain, adjectival toponyms can always agree in case with their sortal if they are phi-congruent. In this they differ from proper names of other entities, which do not allow this option:

\ea\label{mat:ex:minonosec-roman}
\ea[]{
\gll na	minonosce	\minsp{\{} \minsp{“} Blestjaščij” / \minsp{*} \minsp{“} Blestjaščem”\}\\
on	torpedo.boat.{\MASC.\SG}.{\LOC}	{} {} Shining.{\MASC.\SG}.{\NOM} {} {} {} Shining.{\MASC.\SG.\LOC} \\
\glt `on the torpedo boat The Shining' \label{mat:ex:minonosec-roman-a}}
\ex[]{
\gll o	romane	\minsp{\{} \minsp{“} 
 Nepobedimyj” / \minsp{*} \minsp{“} Nepobedimom”\}\\
about	novel.{\MASC.\SG}.{\LOC}	{} {} Invincible.{\MASC.\SG}.{\NOM} {} {} {} Invincible.{\MASC.\SG.\LOC} \\
\glt `about the novel The Invincible' \label{mat:ex:minonosec-roman-b}
}
\z
\z

\noindent The contrast between adjectival and nominal toponyms strongly suggests that the latter do not contain an implicit sortal (which would have made them nominal).

\subsection{Complex toponyms}\label{mat:sec:complex-toponyms}

One more important characterization of close apposition in Russian is that complex toponyms appear to be more restrictive than simplex toponyms. As noted in \citet[142]{GraudinaKatlinskaja1976}, syntactically complex city and country names differ from syntactically simple ones in that the former agree in case only on the condition of gender congruence, just like street names, \REF{mat:ex:BelayaCerkov-PetropavlovskKamcatskij}:\footnote{\citet{Gorpinic1987} asserts that in Ukrainian complex toponyms in close apposition do not agree in case, but a quick informal check has shown that such is not the case for at least some native speakers.}

\ea\label{mat:ex:BelayaCerkov-PetropavlovskKamcatskij}
\ea[]{
\gll v	gorode	\minsp{\{} Belaja Cerkov' / \minsp{*} Beloj Cerkvi\}\\
in	city.{\MASC.\SG}.{\LOC}	{} White Church.{\FEM.\SG.}{\NOM} {} {} White Church.{\FEM.\SG.\LOC} \\
\glt `in the city of Belaya Cerkov (lit. White Church)' \label{mat:ex:BelayaCerkov-PetropavlovskKamcatskij-a} \hfill \ding{55} phi-congruent}
\ex[]{
\gll v	gorode	\minsp{\{} Petropavlovsk-Kamčatskij / Petropavlovske-Kamčatskom\} \\
in	city.{\MASC.\SG}.{\LOC}	{} Petropavlovsk-Kamčatka.{\ADJ}.{\MASC.\SG}.{\NOM} {} Petropavlovsk-Kamčatka.{\ADJ}.{\MASC.\SG.\LOC} \\
\glt `in the city of Petropavlovsk-Kamchatskij (lit. Petropavlovsk of Kamchatka)' \label{mat:ex:BelayaCerkov-PetropavlovskKamcatskij-b}
}
\z
\z

\noindent As before, Internet searches locate some instances of case agreement for \REF{mat:ex:BelayaCerkov-PetropavlovskKamcatskij-a} that probably reflects an earlier stage of the linguistic change in progress, where\-as the native speakers that I consulted conform to the generalization in \citet{GraudinaKatlinskaja1976}: only phi-congruent complex city names can agree in case, exhibiting the more restricted pattern associated with street names (\sectref{mat:sec:case-agreement-number-congruent-pluralia}). \citet{Logvinova2018} supports this generalization showing that complex masculine city names (the word \textit{gorod} ‘city, town’ is masculine) are less likely to agree than simplex masculine city names of comparable frequency.\footnote{\citet[149]{GraudinaKatlinskaja1976} also claims that while agreeing adjectival modifiers have this effect, PP modifiers do not. \citet{Logvinova2018} does not examine such cases and I have not been able to verify this claim or disprove it.}

A similar effect is reported for internally complex street names, such as \textit{Novaja Zarja} ‘the New Dawn’. While street names are generally asserted to require gender congruence (as in \REF{mat:ex:Jakimanka-Balcug-CistyePrudy} in \sectref{mat:sec:gender-congruence-condition}), some prescriptivists claim that complex feminine street names behave like masculine street names and disallow case-agreement (recall that the sortal \textit{ulica} ‘street’ is feminine), resulting in the pattern in \REF{mat:ex:NovajaZarja-MalajaBronnaja-a}.\footnote{E.g., \url{https://newslab.ru/article/465957}} Others only draw a distinction between feminine street names (which agree in case) and masculine ones (which do not).\footnote{E.g., \url{http://new.gramota.ru/spravka/buro/search-answer?s=295848}} Importantly, complex adjectival street names allow case agreement \REF{mat:ex:NovajaZarja-MalajaBronnaja-b}.

\ea\label{mat:ex:NovajaZarja-MalajaBronnaja}
\ea[]{
\gll na	ulicu	\minsp{\{} Novaja Zarja / \minsp{*} Novuju Zarju\} \\
on	street.{\ACC}	{} New Dawn.{\NOM} {} {} New  Dawn.{\ACC}\\
\glt `on(to) the street New Dawn' \label{mat:ex:NovajaZarja-MalajaBronnaja-a}}
\ex[]{
\gll na	ulicu	\minsp{\{} Malaja Bronnaja / Maluju Bronnuju\} \\
on	street.{\ACC}	{} Small Hauberk.{\ADJ}.{\NOM} {} Small Hauberk.{\ADJ}.{\ACC}\\
\glt `on(to) the Lesser Hauberk street' \label{mat:ex:NovajaZarja-MalajaBronnaja-b}
}
\ex[]{
\gll na	Maluju Bronnuju 	ulicu\\
on	Small Hauberk.{\ADJ}.{\ACC} 	street.{\ACC}\\
\glt `on(to) the Lesser Hauberk street' \label{mat:ex:NovajaZarja-MalajaBronnaja-c}
}
\z
\z

\noindent Even though adjectives do not modify adjectives and \textit{malaja} ‘small’ in \REF{mat:ex:NovajaZarja-MalajaBronnaja-b} is originally a restrictive modifier (the Small Hauberk street, as opposed to the bigger one), it seems unlikely that \REF{mat:ex:NovajaZarja-MalajaBronnaja-b} contains a null head noun, or it would behave the same as \REF{mat:ex:NovajaZarja-MalajaBronnaja-a}. One more possibility is that \REF{mat:ex:NovajaZarja-MalajaBronnaja-b} is derived by inversion from \REF{mat:ex:NovajaZarja-MalajaBronnaja-c}, where the sortal forms part of the toponym, yet inversion is generally impossible with toponyms (\ref{mat:ex:Sennaja}--\ref{mat:ex:Nevskij-prospekt}), except in poetry:

\ea\label{mat:ex:Sennaja}
\ea[]{
\gll na	Sennoj	ploščadi\\
on	hay.{\ADJ}.{\FEM.\SG.}{\LOC}	Square.{\FEM}.{\LOC} \\
\glt `on Hay Square' \label{mat:ex:Sennaja-a} }
\ex[*]{
\gll na	ploščadi 	Sennoj\\
on	Square.{\FEM}.{\LOC} 	hay.{\ADJ}.{\FEM.\SG.}{\LOC}\\
\label{mat:ex:Sennaja-b}
}
\z
\ex\label{mat:ex:Nevskij-prospekt}
\ea[]{
\gll na	Nevskom	\minsp{(} prospekte)\\
on 	Nevsky.{\MASC.\SG}.{\LOC} {}	 avenue.{\MASC}\\
\glt `on the Nevsky (Prospekt)' \label{mat:ex:Nevskij-prospekt-a}}
\ex[*]{
\gll na	prospekte	\minsp{(} Nevskij / Nevskom) \\
on	avenue.{\MASC}.{\LOC} {}	Nevsky.{\MASC.\SG}.{\NOM} {} Nevsky.{\MASC.\SG.\LOC}\\
\label{mat:ex:Nevskij-prospekt-b}
}
\z
\z

\noindent The fact that complex adjectival toponyms do not behave as nominal ones provides additional support for the lack of an implicit sortal in adjectival toponyms, which the contrast between adjectival and nominal toponyms has already suggested.

\subsection{Intermediate summary}\label{mat:sec:intermediate-summary}
\begin{sloppypar}
The behavior of toponyms clearly shows that case agreement depends on phi-congruence and that the strictness of this condition is determined by the lexical-semantic class of the proper name: while animate proper names require case agreement and non-toponymic inanimate ones disallow it, toponyms permit case agreement on variable conditions of phi-congruence: while for cities and countries number congruence is a sufficient condition for case agreement, street names require gender congruence in addition, and other toponyms can agree in case only if they are adjectival, as shown below. 
\end{sloppypar}

\begin{table}
\caption{Case agreement with proper names}
\label{mat:tab:Case-agreement-proper-names}
 % \begin{tabularx}{\textwidth}{L{2.5cm}YYYYY} 
\begin{tabularx}{\textwidth}{l@{}C@{~}C@{~}C@{~}C@{~}C} 
  \lsptoprule
   & no case  & \minsp{$+$} adjectival & gender  & number & no congruence\\
  \midrule
  animates  &   \ding{55}  &    \ding{51}  &    \ding{51}     & \ding{51} & \ding{51}\\
  cities, countries, rivers…  &   \ding{51} &   \ding{51}  &    \ding{51}     & \ding{51} & \ding{55}\\
  streets, villages…  &   \ding{51} &   \ding{51}  &    \ding{51}     & \ding{55} & \ding{55}\\
  other toponyms  &   \ding{51} &   \ding{51}  &    \ding{55}     & \ding{55} & \ding{55}\\
  non-toponymic inanimates  &   \ding{51} &   \ding{55}  &    \ding{55}     & \ding{55} & \ding{55}\\
  \lspbottomrule
 \end{tabularx}
\end{table}

For some speakers certain lexical-semantic classes seem to be more restrictive than described by the existing sources and “shifted downwards” in the table, and the same appears to be the case for internally complex toponyms, though the facts are yet far from clear.

Several facts should be accounted for, which excludes some analyses that appear plausible at a first glance:

\begin{itemize}\sloppy
    \item animate sortals require a case-agreeing proper name
    \item case agreement is impossible with inanimate non-toponymic proper names
    \item without an overt sortal all proper names are appropriately case-marked
    \item it is the sortal that determines how the entire NP agrees
    \item the same proper noun (e.g., \textit{Moskva} ‘Moscow’ in \REF{mat:ex:non-toponymic-proper-names-b} and in \REF{mat:ex:moskva-tallin-francija-kitaj-a}) may behave differently with different sortals
    \item it does not seem that agreeing toponyms permit some interpretation or usage that non-agreeing ones do not
    \item internally complex toponyms may yield different congruence restrictions, though the entire empirical picture is yet unclear
    \item at a prior stage of the language toponyms did not require phi-congruence for case agreement
    \item with cardinals, sorted city names require gender congruence \REF{mat:ex:dva-goroda}
\end{itemize}

The distinction between toponyms and other proper names suggests that the lexical-semantic class of a proper name is reflected in its syntax in a principled way.

\section{Excluded hypotheses}\label{mat:sec:excluded-hypotheses}

The empirical generalizations established above provide the desiderata for an explanation that exclude several immediately obvious and not-so-obvious hypotheses.

\subsection{Semantic type distinction}\label{mat:sec:semantic-type-distinction}

A question that needs to be addressed by any theory of close apposition is the semantic type of the proper name (or kind name, for that matter). Two options are available: a predicate and an individual.

The standard approach to proper names is to regard them as individual constants: in argument positions the name \textit{Alice} denotes the individual \textit{a}. However, since, as first pointed out in this context by \citet{Sloat1969}, proper nouns can also appear in positions where such a denotation is impossible \REF{mat:ex:smith}, an additional denotation for them is needed, where they denote predicates.\largerpage[2]

\ea\label{mat:ex:smith}
    \ea *Some/\ding{51} sóme Smith/man stopped by. 
    \ex Some/sóme Smiths/men stopped by.
    \ex Smiths/men must breathe.
    \ex The clever Smith/man stopped by. 
    \ex The Smith/man who is clever stopped by. 
    \ex A clever Smith/man stopped by. 
    \ex The Smiths/men stopped by.
    \ex The *Smith/\ding{51} man stopped by.
    \ex Smith/*man stopped by. \hfill (\citealt{Sloat1969})
    \z
\z

\noindent The predicative approach to proper names (see \citealt{Matushansky2008}, \citealt{Gray2015}, and \citealt{fara2015} for recent takes and references) argues that the denotation in \REF{mat:ex:Alice-a} can and must be derived as a referential definite description built on the basis of the predicative denotation presented in a simplified form in \REF{mat:ex:Alice-b}. 

\ea\label{mat:ex:Alice}
\ea[]{\sib{\text{Alice}}${}=\text{a}$ \label{mat:ex:Alice-a}
}
\ex[]{\sib{\text{Alice}}${}=\lambda x \in D_e\, .\,  x\, \text{is called /ælıs/}$
\label{mat:ex:Alice-b}
}
\z
\z
% \ex[]{ $\iota\text{x}\in\text{D}_e$ .  $\text{x is called /ælıs/}$
% \label{mat:ex:Alice-c}


\noindent Yet for our purposes it is sufficient that the toponym in close apposition can in principle be referential or predicative.\footnote{It is tempting to appeal to the lack of the article in \textit{the river Rhine} as an argument for treating the toponym as non-referential. However, in the next language over, Dutch, the article is present: 
\ea
\gll de	rivier	de	Rijn\\
the	river	the	Rhine\\
\glt `the river Rhine' \hfill (Dutch, \citealt{vanRiemsdijk1998})\\
\z
} Can case agreement be taken as an argument for the simultaneous availability of both options and used to differentiate between the two?

Several reasons can be provided why this approach should not be taken. Firstly, the fact that animate proper names require case agreement, while inanimate non-toponymic proper names disallow it is hard to square with different denotations: we do not expect animacy to interact in this way with the semantic type. Secondly, if case-agreeing toponyms are referential and non-agreeing ones are predicative (or \textit{vice versa}), we expect that there is some context of use that the non-agreeing close apposition in \REF{mat:ex:Bologoe-Moskva-Tixoreckaja-a}--\REF{mat:ex:Bologoe-Moskva-Tixoreckaja-b} lacks and the agreeing close apposition in \REF{mat:ex:Bologoe-Moskva-Tixoreckaja-c} has, which does not seem to be the case. While a more detailed survey might reveal such a difference, no research so far has indicated that there is some meaning or use that \REF{mat:ex:vgorodeMoskve-vgorodeGagry-a} might have while \REF{mat:ex:vgorodeMoskve-vgorodeGagry-b} would lack it, nor is there any obvious interpretational distinction for the agreeing vs. non-agreeing options for one and the same toponym in \REF{mat:ex:Bologoe-Moskva-Tixoreckaja-c} or for the gender-distinct toponyms in exactly the same environment in \REF{mat:ex:dva-goroda}.

\ea\label{mat:ex:vgorodeMoskve-vgorodeGagry}
\ea[]{
\gll v	gorode	Moskve\\
in	city.{\MASC.\SG}.{\LOC}	Moscow.{\FEM.\SG.}{\LOC} \\
\glt `in the city of Moscow' \hfill singular sortal, feminine PN \label{mat:ex:vgorodeMoskve-vgorodeGagry-a}}
\ex[]{
\gll v	gorode	Gagry \\
in	city.{\MASC.\SG}.{\LOC}	Gagra.{\PL}.{\NOM} \\
\glt `in the city of Gagra' \hfill singular sortal, plural PN
\label{mat:ex:vgorodeMoskve-vgorodeGagry-b}
}
\z
\z

\noindent It can be argued that a predicative proper noun, as in \REF{mat:ex:Alice-b}, can be combined with the definite article (or the corresponding type-shift, the iota-operator) to give rise to a definite NP with an interpretation that is virtually indistinguishable from \REF{mat:ex:Alice-a}, as in \REF{mat:ex:Alice-c}. While the predicative approach to proper names argues that this is in fact how their referential use is derived, the referential approach may rely on the ambiguity in \REF{mat:ex:Alice-a}--\REF{mat:ex:Alice-b} to derive the two syntactic options: the proper name \REF{mat:ex:Alice-a} and the definite DP \REF{mat:ex:Alice-c}.

\ea{ $\iota x \in D_e\, .\,  x\, \text{is called /ælıs/}$
\label{mat:ex:Alice-c}}
\z

\noindent While at first blush such an analysis could be taken as an argument in favor of the referential approach to proper names, two problems arise as a result. Firstly, in general, if both options are available in principle, how do we know which one we are dealing with in \textit{Alice is here?} Secondly, specifically to the empirical issue at hand, why should one of the two options be unavailable for animate proper names (which require case agreement in close apposition) and the other, for inanimate non-toponymic proper names (which require nominative) and why should gender features, as in \REF{mat:ex:dva-goroda}, be relevant? The same two issues arise for any view that derives the variation in case agreement from a difference in the interpretation, and the theory to be discussed now is no exception.

\subsection{Quotation}\label{mat:sec:quotation}

The semantic distinction between mention and use looks like a plausible explanation for the two different syntactic options. It is an immediately obvious hypothesis that case invariability involves quotation, and even the objection raised at the end of \sectref{mat:sec:semantic-type-distinction} might be overcome: maybe quotations are obligatorily inanimate and cannot function as anthroponyms or zoonyms, thus explaining why animate proper names require case agreement.

Two issues remain, however. Firstly, it is still an open question why inanimate non-toponyms disallow case agreement. Secondly, if the interpretation of the proper name is not the same in agreeing vs. non-agreeing cases, some difference in use is expected. There are, however, no cases where a phi-congruent and hence agreeing toponym is possible and another toponym, which does not permit agreement due to phi-incongruence, is excluded. In other words, the fact that a certain toponym cannot agree with a given sortal does not preclude its appearance in any context where an agreeing toponym with the same sortal can appear, which strongly suggests no difference in semantics for case-agreeing and invariant toponyms.

\subsection{The sortal as the locus of variation}\label{mat:sec:sortalAs-locus-of-variation}

Although case agreement variation for toponyms is usually described in terms of lexical-semantic classes, it is tempting to hypothesize that it is not the toponyms that are responsible for it, but their sortals, e.g., that some nouns can enter the derivation underspecified for some phi-features. The advantage of this approach is that it can explain why the same proper nouns (e.g., \textit{Moskva} ‘Moscow’ in \REF{mat:ex:non-toponymic-proper-names-b} and in \REF{mat:ex:moskva-tallin-francija-kitaj-a}) behave differently by suggesting that it is not in the proper noun but in the sortal where the difference lies.\footnote{\label{mat:foot:Logvinova2018documents}\citet{Logvinova2018} documents a difference in the behavior of the same toponyms with the feminine \textit{strana} ‘country’ (case agreement conditioned by number congruence) as opposed to the neuter \textit{gosudarstvo} ‘state’ (no case agreement). While the question is open whether the (sorted) toponyms denote the same entity, the syntax could still be the same, as the observed difference would also follow from the gender of the sortal: there were no neuter country names in the data set. Furthermore, as discussed in Footnote~\ref{mat:foot:ForTheSakeOfSimplicity}, neuter toponyms resist case-marking even without a sortal.} The flip side is the prediction that different sortals applying to the same set of proper names are not expected to behave the same. Testing this prediction is difficult: the same behavior for different sortals can easily be attributed to coincidence. In fact, the feminine \textit{stolica} ‘capital’, which combines with a subset of the toponyms that the masculine \textit{gorod} ‘city, town’ can combine with, also requires only number congruence, whereas the difference between the coextensional \textit{strana} ‘country’ and \textit{gosudarstvo} ‘state’ (see Footnote~\ref{mat:foot:Logvinova2018documents}) can be due to the difference in their gender, so confirming or disproving this prediction is impossible.

Another problem with this hypothesis is that it cannot explain why phi-con\-gru\-ent adjectival toponyms can always agree in case with their sortal, and why non-toponymic proper names never do: if the source of the relevant phi-features is the proper name, adjectival and nominal proper names should not differ, and the same is true for toponyms vs. non-toponyms. One more problem is motivation: these sortals do not exhibit any obvious semantic or syntactic peculiarities in any other contexts (which, however, is also true for the toponyms themselves). Finally, the very mechanism of “agreement as valuation” is ill-suited for dealing with phi-congruence, as we will now see.

\subsection{Phi-congruence as valuation}\label{mat:sec:phi-congruence-as-valuation}

Two mechanisms are provided by the current syntactic theory for comparing the phi-features of two constituents: agreement and semantic matching. As it is generally assumed that gender features of inanimate nouns are not interpretable, the feminine of \textit{Jakimanka} and that of \textit{ulica} ‘street’ in \REF{mat:ex:Jakimanka-Balcug-CistyePrudy-a} cannot be matched by ensuring that their presuppositions match: they do not introduce any.\footnote{This assessment will be reexamined in \sectref{mat:sec:toponyms-as-semantic-sort}.} Syntactic agreement remains then the only option.

While number can reasonably be argued to not be inherent to a noun, gender arguably is. It is possible, however, that the gender feature is introduced on a special functional head (e.g., \textit{n}, see \citealt{Kihm2005}, \citealt{Lowenstamm2007}, \citealt{Acquaviva2009}, \citealt{Percus2011}, and \citealt{Kramer2015}, among others) and some additional (and independently needed) mechanism ensures that it correlates properly with the semantics of the noun (for animates) and its declension class. How can we then implement the fact that some sortals, e.g., \textit{ulica} ‘street’, can agree with the toponym?

Suppose that \textit{ulica} ‘street’ can combine directly with the toponym and the gender-introducing functional head \textit{n} (be it categorizing or not) enters the derivation afterwards.

\begin{figure}
    \centering
    \begin{subfigure}[b]{0.5\textwidth}
         \centering
         \begin{forest}
            [\textit{n}P\textsubscript{2}
            [\textit{n}\textsubscript{2}[{[F]}]]
            [\textit{n}P\textsubscript{1}
            [sortal
              [\textit{ulica}\textsubscript{\textgamma},roof]
            ]
            [\textit{n}P\textsubscript{1-name} 
              [\textit{Jakimanka}\textsubscript{\FEM},roof]
            ]
            ]
            ]
\end{forest}
         \caption{}
         \label{mat:fig:ulicaJakimankaTree}
     \end{subfigure}%
     \begin{subfigure}[b]{0.5\textwidth}
         \centering
         \begin{forest}
            [\textit{n}P\textsubscript{2}
            [\textit{n}\textsubscript{2}[{[F]}]]
            [\textit{n}P\textsubscript{1}
            [sortal         
               [\textit{ulica}\textsubscript{\textgamma},roof]
               ]
            [\textit{n}P\textsubscript{1-name}
            [\textit{Balčug}\textsubscript{\MASC},roof]
            ]
            ]
            ]
\end{forest}
         \caption{}
         \label{mat:fig:ulicaBalchugTree}
     \end{subfigure}
    \caption{Gender congruence: sortal \textit{n} as a head external to close apposition}
    \label{mat:fig:ulicaJakimankaTreeAndulicaBalchugTree}
\end{figure}

Setting aside many technical details, consider \figref{mat:fig:ulicaBalchugTree}, where the gender values of the sortal and of the proper name do not match. The proper name is masculine (a valued feature), so \textit{ulica} ‘street’ should also be assigned masculine, contrary to its declension class, which assigns it to feminine (and the gender feature of the resulting complex NP (\textit{n}P\textsubscript{2}) should also be feminine). Nouns whose gender does not match their declension class, such as semantically feminine nouns ending in a consonant \REF{mat:ex:madam-KarmenIvanovna}, do not decline in Russian.\largerpage[2]

\ea\label{mat:ex:madam-KarmenIvanovna}
\ea[]{
\gll k	ėtoj	\minsp{\{} madam / \minsp{*} madame /\hspace{1.5cm} \minsp{*} madamu\} \\
towards	this.{\DAT}	{} madam.{\DAT}\textsubscript{\textsc{indecl}} {} {} madam.{\DAT}\textsubscript{a-\textsc{decl}} {} {} madam.{\DAT}\textsubscript{c-\textsc{decl}}\\
\glt `towards this madam' \label{mat:ex:madam-KarmenIvanovna-a}}
\ex[]{
\gll s	\minsp{\{} Karmen / \minsp{*} Karmenoj / \minsp{*} Karmenom\} Ivanovnoj\\
with	{} Carmen.{\DAT}\textsubscript{\textsc{indecl}} {} {} Carmen.{\DAT}\textsubscript{a-\textsc{decl}}{} {} {} Carmen.{\DAT}\textsubscript{c-\textsc{decl}}	Ivanovna.{\DAT}\textsubscript{a-\textsc{decl}} \\
\glt `with Carmen Ivanovna'
\label{mat:ex:madam-KarmenIvanovna-b}
}
\z
\z

\noindent At the \textit{n}P\textsubscript{1} level the prediction is that \textit{ulica} ‘street’ would not agree. This is a wrong result, so let us suppose that the feminine feature of \textit{n}\textsubscript{2} somehow overrides the masculine obtained from \textit{n}P\textsubscript{1-name}, both on the sortal and on the proper name. Feminine gender specification contradicts the morphological properties of the toponym, so the structure in \figref{mat:fig:ulicaBalchugTree} would result in a non-agreeing form, as desired.\footnote{The fact that phi-congruent toponyms may still not agree in case requires an additional richer structure, where the sortal is specified for gender and the toponym, not having agreed with it, does not count as part of the same NP for the purposes of case-assignment (or more likely, concord).}

This approach, however, cannot be extended to toponyms agreeing in case on the condition of number congruence. Firstly, number is generally associated with the presence of plural semantics, i.e., a *-operator or a cardinal (or both, this depends on the adopted approach to cardinals). In the case of number-congruent pluralia tantum toponyms, like in \REF{mat:ex:maldivy-alpy}, where both the sortal and the toponym bear plural morphology, there seems to be no reasonable way in which one of them could be unvalued.\footnote{I note here that in the singular the feminine noun \textit{gora} ‘mountain’ allows case agreement on the condition of gender congruence, though to a lesser degree than \textit{strana} ‘country’ or \textit{reka} ‘river’ (\citealt[22]{Logvinova2018}).} To see this, consider \figref{mat:fig:goryAlpyTree}.

\begin{figure}
    \centering
    \begin{subfigure}[b]{0.5\textwidth}
         \centering
         \begin{forest}
            [NumP\textsubscript{1}
                [Num\textsuperscript{0}[{[PL]}]]
            [\textit{n}P\textsubscript{1}
            [\textit{n}P\textsubscript{1}
              [\textit{gory}\textsubscript{\#},roof]
            ]
            [NumP\textsubscript{2}
              [\textit{Al'py}\textsubscript{\PL},roof]
            ]
            ]
            ]
\end{forest}
         \caption{}
         \label{mat:fig:goryAlpyTree-a}
     \end{subfigure}%
     \begin{subfigure}[b]{0.5\textwidth}
         \centering
         \begin{forest}
            [NumP\textsubscript{1}
                [NumP\textsubscript{1}[Num\textsuperscript{0}]
                [\textit{n}P\textsubscript{1}        
                   [\textit{gory}\textsubscript{\PL},roof]]
                ]
                [NumP\textsubscript{2}[Num\textsuperscript{0}]
                [\textit{n}P\textsubscript{2}
                     [\textit{Al'py}\textsubscript{\PL},roof]
                  ]
                  ]
            ]
\end{forest}
         \caption{}
         \label{mat:fig:goryAlpyTree-b}
     \end{subfigure}
    \caption{Number congruence: the position of Num in close apposition}
    \label{mat:fig:goryAlpyTree}
\end{figure}

The toponym \textit{Al'py} ‘the Alps’ in \figref{mat:fig:goryAlpyTree} corresponding to \REF{mat:ex:maldivy-alpy-b} is plural, on both morphological and semantic grounds, so its number feature is valued. Consider first \figref{mat:fig:goryAlpyTree-a}, where the number feature of the sortal is unvalued and so can in principle agree with the valued number feature of the toponym. However, the semantics of \figref{mat:fig:goryAlpyTree-a} is incorrect: if \textit{Al'py} ‘the Alps’ is referential here, then the higher \textit{n}P\textsubscript{1} node denotes the set of singular mountains that is the Alps, i.e., the empty set. If \textit{Al'py} is predicative, then the higher \textit{n}P\textsubscript{1} node denotes a set of mountains each of which either is called (the) Alps or is a plurality called (the) Alps, which is equally incorrect.

Consider now \figref{mat:fig:goryAlpyTree-b} as the structure for \REF{mat:ex:maldivy-alpy-b}, assuming that Num\textsuperscript{0} of the sortal is the source of the plural semantics (if it isn’t, the same problem arises as in \figref{mat:fig:goryAlpyTree-a}). The semantics is now correct, but the number feature of the sortal cannot be unvalued.

Two more options are available in principle. One (\figref{mat:fig:goryAlpyTree2-c}) is to assume that the unvalued number feature is on the toponym, contrary to what has been assumed before (and despite the fact that it is a \textit{plurale tantum}). The second (\figref{mat:fig:goryAlpyTree2-d}) is to treat number features as unvalued on both the sortal and the toponym.

\begin{figure}
    \centering
    \begin{subfigure}[b]{0.5\textwidth}
         \centering
         \begin{forest}
            [NumP\textsubscript{1}
                [NumP\textsubscript{1}[Num\textsuperscript{0}]
                [\textit{n}P\textsubscript{1}[\textit{gory}\textsubscript{\PL},roof]]
                ]
            [\textit{n}P\textsubscript{2}[\textit{Al'py}\textsubscript{\#},roof]]
            ]
\end{forest}
         \caption{}
         \label{mat:fig:goryAlpyTree2-c}
     \end{subfigure}%
     \begin{subfigure}[b]{0.5\textwidth}
         \centering
         \begin{forest}
            [NumP\textsubscript{1}
                [Num\textsuperscript{0}[{[PL]}]]
                [\textit{n}P\textsubscript{1}
                     [\textit{n}P\textsubscript{1} 
                        [\textit{gory}\textsubscript{\#},roof]]
                     [\textit{n}P\textsubscript{2}[\textit{Al'py}\textsubscript{\#},roof]
                  ]]
                  ]
\end{forest}
         \caption{}
         \label{mat:fig:goryAlpyTree2-d}
     \end{subfigure}
    \caption{Number congruence: uninterpretable number is on the name}
    \label{mat:fig:goryAlpyTree2}
\end{figure}

Even setting aside their syntactic plausibility, both options fail with the conjoined singulars in \REF{mat:ex:balakov-saratov} where the toponym cannot be reasonably regarded as having unvalued number: a non-intersective conjunction of two singulars (be it a sum of two individuals or a set-product of two predicates) can under no assumptions be non-plural semantically.

We conclude that case agreement with a phi-congruent plural sortal poses an unsurmountable obstacle to treating the phi-congruence condition in toponymic close apposition as valuation.

\subsection{Intermediate summary}\label{mat:sec:intermediate-summary2}

We have examined four theories that can be advanced to explain the phenomenon of varying case agreement in close apposition in Russian. Two of them suggest a semantic difference between agreeing proper names (assumed to be referential) and non-agreeing proper names (which are attributed predicative semantics (or maybe indirectly referential semantics) or the semantics of quotation). The other two address the syntactic side of the problem: the locus of the unvalued features that should drive case agreement and the applicability of the theory of agreement as feature valuation to close apposition.

The failure of syntactic theories is due to the fact that phi-congruence is established between interpretable features that can be simultaneously valued on the sortal and on the proper name. On the semantic side one problem is that the immediately obvious potential solutions do not take into consideration the difference between lexical-semantic classes of proper names, and another, that there is no independent evidence for a semantic distinction.

What follows is a sketch of a solution based on two assumptions: (a) that agreement in close-apposition is semantic and as such, based on feature-value matching rather than valuation and (b) that the semantic sort of toponyms is different from that of other proper names, so they can be singled out on semantic grounds. 

\section{Toponyms as a semantic sort}\label{mat:sec:toponyms-as-semantic-sort}

One of the main facts to be accounted for is the distinction between animate proper names (which obligatorily agree in case), toponyms (which may do so) and inanimate non-toponyms (which cannot do so).

As case agreement is clearly dependent on phi-congruence, it is natural to hypothesize that a proper name counts as part of the same NP as the sortal if it agrees with it in some feature. Case agreement then becomes something of a free-rider in the sense that case-assignment to the proper name forming part of the same NP as the sortal (which is what agreement enables) can be viewed as concord: multiple realizations of the case assigned to the entire NP. Without further elaboration of this hypothesis, I further suggest that different lexical-semantic classes of proper names underlyingly have different semantic phi-feature specifications and attempt to motivate these distinctions by independent factors.

\subsection{The role of animacy}\label{mat:sec:role-of-animacy}

Being a subtype of nouns, proper names have valued formal phi-features determined by their semantics and their declension class. Since formal gender (for inanimate nouns) and formal number (for pluralia tantum) can be inherently valued and fail to agree, the only remaining option for agreement in close apposition are semantic phi-features. The first such feature is obviously animacy.

I will not decide here how this feature value is set. Three possibilities can be envisaged: from the sortal, from the denotation of the proper name itself (if it is referential) or from the denotation of the entire appositive noun phrase. What is crucial is that semantically, animacy is a privative feature, so inanimate nouns lack it. This means that a proper name can semantically agree for animacy only with animate sortals, which would explain why only animate proper names agree in close apposition.\footnote{\label{mat:foot:OneMightObject}One might object that animate proper names also have semantic gender, which they need not share with the sortal (Footnote~\ref{mat:foot:TheFirstGeneralization}). A counterargument to this objection is that a human-denoting NP in Russian may acquire semantic gender that overrides its formal gender (\citealt{Crockett1976}, \citealt{Corbett1979}, \citealt{Rothstein1980}, \citealt{Nikunlassi2000}, \citealt{Asarina2008}, \citealt{Pesetsky2013}, etc.):

\ea
\gll U	nas	byla	ocen'	xorošaja	zubnoj	vrač. \\
with	us	was.{\FEM.\SG.}	very	good.{\FEM.\SG.}	dental.{\MASC.\SG}	doctor.{\MASC}\\
\glt `We had a very good dentist.' \hfill (\citealt{Crockett1976})
\z

\noindent In other words, sortals whose gender is different from that of the anthroponym can also agree on the basis of the gender of the referent outside close apposition, so arguably either do not possess underlying semantic gender or can acquire the gender of their referent by an independently motivated mechanism and then presumably agree with the proper name.
}

The question is now why toponyms do not behave as other inanimate proper names.

\subsection{Locative nominals as a lexical-semantic class}\label{mat:sec:locative-nominals-as-lexical-semantic-class}

There is mounting evidence that the syntax of nouns denoting places is different from that of nouns denoting other entities. Thus \citet{Haspelmath2019} shows that cross-linguistically nouns denoting places are less marked in locative environments than regular object-denoting nouns and \citet{Matushansky2019} argues that crosslinguistic use of toponyms and a few common nouns as locative adverbials with zero or special marking indicates denotation in the special locative domain (variants of which have been independently postulated to account for the semantics of spatial prepositions, see \citealt{Bierwisch1988}, \citealt{Wunderlich1991}, \citealt{ZwartsWinter2000}, \citealt{Kracht2002}, \citealt{Bateman2010}, etc.). Evidence for a special status of locative place names in Martinican Creole can also be found in \citealt{ZribiHertzLoic2014, ZribiHertzLoic2017, ZribiHertzLoic2018}. In Russian itself, support for this view comes from the so-called \textsc{locative-II}: the special form of the Russian locative case that certain nouns take when appearing with the prepositions \textit{v} ‘in’ or \textit{na} ‘on’ denoting the default locative relations with these nouns \REF{mat:ex:vodaVtazu-nadpisNataze}.\footnote{The distribution of the “second prepositional case” (locative II) is very complicated, as discussed in \citet{Plungjan2002}, \citet{Brown2007} and \citet{Itkin2016} (see \citealt{Nesset2004} for its use in temporal expressions).} Other nouns (including other location nouns) do not show this distinction:

\ea\label{mat:ex:vodaVtazu-nadpisNataze}
\ea[]{
\gll voda	v	taz-u\\
water	in	hand.basin-{\LOC}\textsubscript{II}\\
\glt `water in the hand-basin' \label{mat:ex:vodaVtazu-nadpisNataze-a}\hfill default locative meaning}
\ex[]{
\gll nadpis'	na	taz-e\\
writing	on	hand.basin-{\LOC}\\
\glt `writing on the hand-basin'
\label{mat:ex:vodaVtazu-nadpisNataze-b} \hfill non-default locative meaning\\
\hfill (\citealt{Plungjan2002})
}
\z
\z

\noindent The fact that adjectival modification of nouns in locative II is allowed shows that they cannot denote in the loci domain (since loci, be they regions, sets of points, or sets of vectors, do not have the same domain structure as objects and cannot be modified by the same modifiers). Yet locative II provides evidence for a crucial underlying distinction between object nouns and place nouns, and I propose that toponyms can be distinguished from other proper names on precisely these grounds (even though toponyms are never marked with locative II in Russian). Moreover, since locative-II nouns denote not only places, but also objects (i.e., any such noun can enter the derivation with either sort), we expect that the non-agreeing option will be possible in the latter denotation.

The question is how this distinction translates into optional case agreement on the condition of phi-congruence.

\subsection{Number features of toponyms}\label{mat:sec:number-features-of-toponyms}\largerpage

Importantly, Russian toponyms are not syntactically uniform. Their behavior with respect to case agreement separates them into three classes (cf. \tabref{mat:tab:Case-agreement-proper-names}): 

\begin{itemize}
    \item countries and cities: number congruence is required for case agreement
    \item rivers, villages, etc.: number and gender congruence is required
    \item others: only agreeing adjectival toponyms agree
\end{itemize}

I stipulate that, unlike other proper names, toponyms by virtue of their semantic sort cannot be mass. This generates the semantic feature of number, which is denotation-based. For most toponyms this would mean singular, but it is overridden by the formal plural with a plurale tantum toponym. It is only when the sortal is plural as well that no conflict arises.

The question is then what to do with gender.

\subsection{Semantic agreement and referentiality}\label{mat:sec:semantic-agreement-referentiality}\largerpage[2]

The appeal to semantic agreement raises the question of whether case-agreeing proper names are referential since semantic agreement is known to rely on the properties of the denotatum. Importantly, case agreement is known to be facilitated if the toponym is familiar \citep{GraudinaKatlinskaja1976, RozentalKabanova1998, Logvinova2018, LogvinovaInPress}.\footnote{\citet[][56]{LogvinovaInPress} provides evidence from city names that higher frequency of a toponym increases the frequency of case agreement. As previously described (\citealt{GraudinaKatlinskaja1976}, \citealt{RozentalKabanova1998}), plural and two-word toponyms are less likely to agree in case. She also observes that unexpectedly, adjectival city names are less likely to agree in case.} While it seems plausible therefore that case agreement in close apposition correlates with the referentiality of the toponym, testing this hypothesis with native speakers does not support this conclusion:

\ea\label{mat:ex:obsledovanieEkaterinburg-Pavlovsk}
\ea[]{
\gll obsledovanie	domašnix	xozjajstv	žitelej	goroda	Ekaterinburga a	takže	naxodjaščixsja	na	territorii	Sverdlovskoj	oblasti	gorodov Pervoural'ska	i	Kamensk-Ural'skogo \\
examination	home	economy.{\GEN}	residents.{\GEN}	city.{\GEN}	Ekaterinburg.{\GEN} and	also	located.{\PL}.{\GEN}	on	territory	Sverdlovsk.{\ADJ}	region	cities.{\GEN} Pervouralsk.{\GEN}	and	Kamensk-Ural'sky.{\GEN}\\
\glt `an examination of the housekeeping of the residents of the city of Ekaterinburg as well as of 	the towns of Pervouralsk and Kamensk-Uralsky, located in the Sverdlovsk region' \hfill (RNC) \label{mat:ex:obsledovanieEkaterinburg-Pavlovsk-a}}
\ex[]{
\gll Krome	goroda	Pavlovska	pod	Piterom,	est'	eščë	odin --	pod Voronežem.\\
besides	city.{\GEN}	Pavlovsk.{\GEN}	under	Piter	is	also	one	{} under Voronezh\\
\glt `Besides the town of Pavlovsk near St. Petersburg, there is one more near 	Voronezh.'
\label{mat:ex:obsledovanieEkaterinburg-Pavlovsk-b}
}
\z
\z

\noindent The RNC example \REF{mat:ex:obsledovanieEkaterinburg-Pavlovsk-a} strongly implies that the hearer is not familiar with the two towns in question, yet case agreement is grammatical there. More convincingly, perhaps, the toponym \textit{Pavlovsk} cannot be referential in example \REF{mat:ex:obsledovanieEkaterinburg-Pavlovsk-b} because two places with such a name exist in the context, and the same is true in \REF{mat:ex:dva-goroda}.

Nonetheless as the presupposition of countability applies to all toponyms it seems reasonable to view semantic features here as derived from the denotation. The situation is more complex where it comes to gender.

\subsection{Inanimate gender as a formal feature}\label{mat:sec:inanimate-gender-as-formal}\largerpage

To account for case agreement on the condition of gender congruence (\sectref{mat:sec:gender-congruence-condition}) I propose that, contrary to what happens to inanimates in general, gender features of toponyms may be interpretable. Independent evidence for this comes from indeclinable toponyms and common nouns. While inanimate nouns in Russian are generally assigned gender on the basis of their declension class, the gender of indeclinable toponyms is often the same as the gender of their hypernym (\citealt{RozentalKabanova1998}, \citealt{Doleschal1996}, \citealt{Murphy2000}, \citealt{Matushansky2022}, a.o.), which strongly suggests that inanimate gender can also be interpretable at LF.\footnote{Indeclinable common nouns can also be assigned semantic gender on the basis of their hypernym (see \citealt{Wang2014}, \citealt{Baranova2016}, \citealt{ChuprinkoMagomedovaSlioussarSubmitted}, a.o.), both in online computation and prescriptively.} If, as corpus searches reveal, along with the neuter expected for inanimates the indeclinable \textit{Zimbabve} ‘Zimbabwe’ can be feminine (because \textit{strana} ‘country’ is feminine) and \textit{Bol'šoj Zimbabve} ‘Great Zimbabwe’ can be masculine (because \textit{gorod} ‘city’ is), nothing excludes that morphologically declinable toponyms can also have semantic gender. If their gender is systematically determined by their declension class (as can be seen from their agreement outside close apposition), then for case agreement this semantic/formal gender of a toponym would have to match the gender of the sortal along the same lines as discussed for animacy and number.

The hypothesis that formal gender features can be semantically interpretable (as is needed to explain toponyms requiring gender congruence for case agreement (\sectref{mat:sec:gender-congruence-condition})) entails that gender features of toponyms requiring number congruence only (\sectref{mat:sec:number-congruence-case-agreement}) should also be interpretable. Where does the difference come from?

By our prior reasoning toponyms are non-mass, so semantic agreement in number is possible for all toponyms and seems to be required for case agreement. To explain the role of gender it is necessary to assume that when gender is semantically interpretable, semantic agreement just for number is insufficient. The question then arises why gender is interpretable for some toponyms (\sectref{mat:sec:gender-congruence-condition}) but not for others (\sectref{mat:sec:number-congruence-case-agreement}) and how come it suddenly becomes so for the latter in cases like \REF{mat:ex:dva-goroda}.

The crucial property of \REF{mat:ex:dva-goroda} is obviously the paucal cardinal. The cardinal assigns a formal plural (or paucal) value to the number features of the sortal and the toponym, which both are morphologically singular and, following \citet{IoninMatushansky2006,IoninMatushansky2018}, semantically atomic, even though the denotatum is semantically plural. Furthermore, the toponym, being in the scope of the cardinal, is not referential. Which of these factors (number mismatches or non-referentiality) can explain the more restricted character of toponyms discussed in \sectref{mat:sec:gender-congruence-condition} remains an open question.

\subsection{Adjectival toponyms}\label{mat:sec:Adjectival-toponyms}

To conclude the proposed sketch of a solution, it is necessary to explain why case agreement with a phi-congruent adjectival toponym is possible for any sortal. The core intuition should rely on the fact that adjectives normally do not have any underlying phi-features at all. As metalinguistic as it sounds, it seems reasonable that adjectival toponyms come with a strong intuition of what the sources of their valued phi-features are, i.e., with some presupposition about their sortals. While it is unlikely that the hyperonym is syntactically represented, it can function as the source of semantic phi-features, enabling the toponym to establish semantic agreement with its sortal.

\section{Conclusion}\label{mat:sec:Conclusion}

We have seen that Russian proper names fall into three categories in function of how they behave with respect to case agreement in close apposition. Proper names of human and other animate entities necessarily agree in case with the sortal. Names of inanimate entities that are not locations conversely never agree in case with the sortal. Finally, toponyms fall into the intermediate category: they may fail to agree in case with the sortal or allow case agreement on the condition of congruence in number (\sectref{mat:sec:number-congruence-case-agreement}) or in number and gender (\sectref{mat:sec:gender-congruence-condition}). While we have not looked at kind names in detail, they seem to pattern either as city/country names or as non-toponymic names (Footnote~\ref{mat:foot:CloseAppositionAlsoPossible}).

I propose that the crucial distinction between toponyms and other inanimate proper names is that toponyms may introduce interpretable phi-features in close apposition. The advantages of this hypothesis are that, on the one hand, it does not need to assume that any semantic factors distinguish between toponyms agreeing and not agreeing in case, and on the other, that the introduction of interpretable phi-features can be naturally linked to frequency: more frequent toponyms would be more clearly identified with some presuppositions.

Many questions remain. For the time being we have no principled explanation for why there are these three classes of toponyms, or why internal syntactic complexity of proper names influences case agreement.\footnote{One possible answer might be that internally complex toponyms are simply less frequent, but this hypothesis requires independent confirmation.} We have not explored adjectival proper names in sufficient detail and only sketched a possible solution for the apparently obligatory gender congruence with cardinals. Likewise, we have not addressed the fact that close apposition may involve restrictive or non-restrictive interpretation of the sortal and did not make clear how agreement (or congruence) in phi-features can enable agreement in case (which is, after all, a purely syntactic operation).

The entire phenomenon of phi-congruence in case agreement in toponymic close apposition, which we have encoded by hypothesizing that inanimate proper names may acquire semantic gender features, might instead be regarded as an argument in favor of treating agreement as matching rather than valuation. Irrespective of the eventual implementation, the issue of phi-congruence in case-agreement raises a number of problems for standard approaches to both proper names and agreement.


\section*{Abbreviations}

\begin{tabularx}{.5\textwidth}[t]{@{}lQ}
\textsc{2}&second person\\
a-\textsc{decl}&the declension class of nouns ending in a in the nominative\\
\textsc{acc}&accusative\\
\textsc{adj}&adjective\\
\textsc{c-decl}& the declension class of masculine nouns ending in a consonant in the nominative\\
\textsc{dat}&dative\\
\textsc{f}&feminine\\
\end{tabularx}%
\begin{tabularx}{.5\textwidth}[t]{lQ@{}}
\textsc{gen}&genitive\\
\textsc{inf}&infinitive\\
\textsc{ins}&instrumental\\
\textsc{loc}&locative\\
\textsc{loc-ii}&locative-II\\
\textsc{m}&masculine\\
\textsc{n}&neuter\\
\textsc{neg}&negation\\
\textsc{nom}&nominative\\
\textsc{pl}&plural\\
\textsc{sl}&singular\\
&\\ % this dummy row achieves correct vertical alignment of both tables
\end{tabularx}

\section*{Acknowledgments}
Precursors, parts and pieces of this work have been presented and fruitfully discussed at many venues, including~MIT (2011), Groningen University (2012), Radboud University of Nijmegen (2012), Moscow State Humanitarian University \& Academy of Sciences Linguistic Institute (2013), UMR~7023 in Paris (2013) and several others. I am also very grateful to the three FDSL-14 reviewers, whose detailed and insightful comments have led me to completely restructure the paper, casting it as a challenge to a far greater extent than before.

\printbibliography[heading=subbibliography,notkeyword=this]

\end{document}
