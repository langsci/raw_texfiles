\documentclass[output=paper]{langscibook} 
\ChapterDOI{10.5281/zenodo.10123627}

\author{Boban Arsenijević\affiliation{University of Graz}}
\title{Specification of telicity in Serbo-Croatian, without null prefixes}

\abstract{The paper reconsiders the claim that null prefixes must be posited in order to maintain the generalization that telicity is necessarily marked by an affix in Slavic languages (\cite{Lazor.2010}). Two classes of verbs apparently showing telic behavior without overt aspectual affixes are investigated on the empirical material from Serbo-Croatian: simple telic perfectives, and simple imperfectives compatible with the \textit{za}-phrase (SC modifiers with the preposition \textit{za} `for' are equivalent to the English \textit{in-X-time} expression, and SC modifiers without a preposition to the English \textit{for-X-time} expression). It is argued that the former are indeed telic verbs without an aspectual affix, but that these verbs are idiomatically stored rather than being compositionally interpreted, and hence are irrelevant for the generalization. The latter are argued to be genuinely atelic. Their compatibility with the \textit{za}-phrases are not evidence for telicity: \textit{za}-phrases are not exclusively compatible with telic eventualities. This view is supported by a number of semantic and morphological similarities and differences between the verb classes involved, and quantitative evidence from corpus research. At least for Serbo-Croatian, then, \posscitet{Lazor.2010} generalization that telicity never occurs without affixes can be maintained without postulating null prefixes.

\keywords{null prefixes, telicity, aspectual pairs, simple verbs, Serbo-Croatian}\vspace*{-1.25\baselineskip}
}

\lsConditionalSetupForPaper{}

\begin{document}

%%% provide a shorter version of your title in case it doesn't fit a single line in the running head
% \shorttitlerunninghead{your short title}

%%% uncomment the following line if you are a single author or all authors have the same affiliation
\SetupAffiliations{mark style=none}

\maketitle

\section{Introduction}\label{ars:sec:intro}\largerpage

Slavic verbal aspectual morphology is a hallmark of both Slavic linguistics and general research of aspect, and probably needs no introduction -- especially in a volume from a Slavic conference encompassing a workshop on secondary imperfectives. I therefore give only a very brief introduction to Slavic aspectual morphology, and then skip to the actual topics of the article. 

The prototypical morphologically simple Slavic verb (inflection morphology aside) is broadly assumed to be imperfective and atelic, as in \REF{ars:ex:SlavAsp-a}.\footnote{In the paper I qualify verbs as telic or atelic (i.e. unspecified), while it is actually the entire VP that can be telic or atelic and not the verb alone. In Slavic, however, a set of verbs is restricted to fitting in telic VPs only, and therefore describing them as telic is not incorrect. Other verbs are unspecified for telicity, as argued in \sectref{ars:sec:Background}. Note also that the nature of and criteria for attesting telicity are highly debated issues, both in general linguistic theory and in its application to Slavic languages. In the current paper, I do not go deeper into this discussion, but rather stick to the tests which display consistency when implemented on the Slavic linguistic material.}\footnote{In this paper, where relevant, verbs are specified for belonging to the traditional classes of perfective or imperfective verbs by the last item in their glosses. This item is added after a period, and does not correspond to any morpheme in the original example, which is meant to specify that this specification applies to the entire verb, and not to the last glossed morpheme.
 
Throughout the paper, I also use the standard marking of the grammaticality status of the example: ? for slightly degraded, ?? for strongly degraded, * for ungrammatical and {\%} for examples grammatical in some varieties, i.e. for some speakers. The sign {\#} is used for examples which are pragmatically or semantically ill-formed.

The relevant verbs in the examples are glossed following an assumed morphological analysis, i.e. decomposed into morphemes represented by their default morphs in order to keep a consistent coding of morphemes across examples (one exception is allomorphy triggered by imperfectivization, where the exact allomorphs are given to keep visible the illustrated morphological operation). Due to phonological alternations, some of the morphemes in some of the given examples surface with different morphs.

All the examples in the paper are constructed by the author, who is a native speaker of the Ekavian standard Serbo-Croatian and the Torlakian dialect. For each constructed example, it has been verified in the corpus that the structural pattern used is attested in the relevant syntactic and semantic environment.} It derives a perfective telic verb by taking a lexical prefix -- one that corresponds to the predicate of result, as in \REF{ars:ex:SlavAsp-b}, or the semelfactive suffix that imposes arbitrary bounds, as in \REF{ars:ex:SlavAsp-b1}. The verb emerging as a perfectivized version of a simple imperfective can be imperfectivized again by a suffix, resulting in what is traditionally referred to as a secondary imperfective, as in \REF{ars:ex:SlavAsp-c}. Finally, both simple and secondary imperfectives can be perfectivized by a superlexical prefix: a prefix which does not express the result (in the narrow conventional sense as in \cite{Ramchand.2004, Svenonius2004b}; but see \cite{Arsenij.2007, Arsenij.2007b, Žaucer.2009} for a resultative analysis of superlexical prefixes) and expresses a meaning related to the quantity of the event, as in \REF{ars:ex:SlavAsp-d}--\REF{ars:ex:SlavAsp-e}, respectively.\largerpage

\ea\label{ars:ex:SlavAsp}
	\begin{xlist} 
		
	    \ex  \gll Pio je čaj.\\ 
drink.\textsc{ptcp.ipfv} \textsc{aux} tea\\
\glt `He was drinking tea.'\label{ars:ex:SlavAsp-a} 
				
		\ex \gll Od-pio je čaj.\\
from-drink.\textsc{ptcp.pfv} \textsc{aux} tea\\
\glt `He took a sip from the tea.'\label{ars:ex:SlavAsp-b}

		\ex \gll Pi-nu-o je čaj.\\
drink-\textsc{suff-ptcp.pfv} \textsc{aux} tea\\
\glt `He took a sip from the tea.'\label{ars:ex:SlavAsp-b1}

		\ex \gll Od-pi-ja-o je čaj.\\
from-drink\textsc{-suff-ptcp.ipfv} \textsc{aux} tea\\
\glt `He was taking a sip\slash sips from the tea.' \label{ars:ex:SlavAsp-c}

		\ex \gll  Po-pi-o je čaj.\\
over-drink-\textsc{ptcp.pfv} \textsc{aux} tea\\
\glt `He drank all the tea.'\label{ars:ex:SlavAsp-d}

		\ex \gll Iz-od-pi-ja-o je čaj.\\
out-from-drink\textsc{-suff-ptcp.pfv} \textsc{aux} tea\\
\glt `He took sips of the tea to its exhaustion.'\label{ars:ex:SlavAsp-e}
		
	\end{xlist}
\z 

\noindent A large number of observations, generalizations and problems have been reported and discussed in the rich literature in this field. This paper tackles one narrow, but core question in this domain: Is telicity universally marked by affixes in Slavic? In order to answer this question, I discuss several related issues, most importantly the relevant opposition behind the traditional division of Slavic verbs into perfectives and imperfectives, including the structural representation and semantic content of the relevant asymmetry and the relation between the members of the so-called aspectual pairs. A remark is due regarding aspectual pairs, as their reality represents another unresolved issue in Slavic linguistics. I take two verbs to be an aspectual pair if one of them is perfective and the other imperfective, there is an independently attested morphological operation that derives one from the other, and, abstracting away from aspect, they mean the same. For polysemous verbs, it suffices that there is at least one meaning of the perfective and one of the imperfective verb such that the condition of semantic equivalence abstracting away from aspect applies to their combination. The availability of non-shared interpretations poses no problem for this relation. The criterion used to establish that two verbs form a pair is that a sentence can be constructed following the general pattern illustrated for two verbs in \REF{ars:ex:paircrit}, such that the imperfective verb fits the first verbal slot and the imperfective the second.\largerpage[2]

\ea\label{ars:ex:paircrit}
	\begin{xlist} 

		\ex \gll Marija satima jede sendviče iznova i iznova, i upravo je pojela poslednji.\\
        M hours eat.\textsc{pres.3.sg.ipfv} sandwiches again and again and just \textsc{aux.3.sg.pfv} eat.\textsc{ptcp.f.sg.pfv} last\\
        \glt `Marija has been eating sandwiches again and again for hours, and she just ate the last one.'\label{ars:ex:paircrita}

		\ex \gll Jovan satima dotrčava kući iznova i iznova, i upravo je dotrčao poslednji put.\\
        J hours run.to.\textsc{pres.3.sg.ipfv} home again and again and just \textsc{aux.3.sg.pfv} run.to.\textsc{ptcp.f.sg.pfv} last time.\\
        \glt `Jovan has been coming home running again and again for hours, and he just came home running for the last time.'\label{ars:ex:paircritb}
		
	\end{xlist}
\z 

\noindent I provide arguments from Serbo-Croatian (SC; all the examples in the paper are from SC unless otherwise indicated) supporting the statements in \REF{ars:ex:Emp-quest}.

\eanoraggedright\label{ars:ex:Emp-quest}
\eanoraggedright The strong generalization made by \citet{Lazor.2010}, that telicity is universally reflected in affixal material, taking affixes as the feature content of some relevant syntactic heads rather than the morphs surfacing on the verb, holds in SC without the need to postulate null prefixes. \label{ars:ex:Emp-quest-a}
\ex Morphologically simple verbs passing all or some tests as telic are either idiomatically stored and thus irrelevant for the generalization above, or are rather unrestricted for telicity (i.e. atelic in the traditional view) with telic interpretations emerging from pragmatics. \label{ars:ex:Emp-quest-b}
\z\z

\noindent The paper is organized as follows. \sectref{ars:sec:The empirical base} introduces the database that I use to inspect the relevant quantitative properties of the relevant verb classes, and \sectref{ars:sec:Background} presents the relevant existing views of Slavic verbal aspect. \sectref{ars:sec:theoretical} presents the structural model at the syntax-semantics interface assumed to underlie the aspectual morphology and semantics in Slavic languages. In \sectref{ars:sec:s-simplepref}, I discuss the affixless perfectives and argue that they are all idiomatic, i.e. non-compositional, and hence irrelevant for the generalization about affixal marking of aspect. \sectref{ars:sec:s-simpleimperf} gives a general overview of the four classes of traditional imperfectives regarding telicity, with special attention for secondary imperfectives and simple imperfectives passing some tests as telic. The latter class is then scrutinized in \sectref{ars:sec:Simple} with respect to the issue of null prefixes, and it is argued that these verbs do not support the introduction of null prefixes either. \sectref{ars:sec:conc} concludes.  

\subsection{The empirical base}\label{ars:sec:The empirical base}\largerpage

Besides the common sources of empirical data, including previous literature, corpora and grammaticality judgments, the research reported includes quantitative insights from the Database of the Western Slavic verbal system (\cite{Arsetal.2021}). The database consists of 5300 SC and 3000 Slovenian verbal lemmata retrieved from the srWac, hrWac and bsWac corpora for SC (\cite{Ljube.2011}) and from the Slovenian National Corpus for Slovenian (\cite{SloCorp}). The verbs are selected based on frequency: the 3000 most frequent lemmata from each of the corpora are included and annotated. As srWaC, hrWaC and bsWaC are corpora of different SC varieties, the SC database combines all three sets of 3000 verbs from the three corpora. Different morphophonological shapes that the same verbs had in two or all three varieties (e.g., Ekavian, Ijekavian, Ikavian versions or those emerging from using different integration suffixes to adopt borrowed verbs or to imperfectivize native ones) were introduced as separate entries, and annotated as variants of one verb. Each verb is annotated for a fixed set of over 40 different properties, including frequency, lexical and grammatical aspect as verified by the chosen tests, argument structure (taking accusative, genitive, dative, PP, clausal arguments; reflexivity), the characteristic morphemes (the root, prefixes, suffixes), their special properties (e.g. root-allomorphy), prosodic characteristics (position of the high tone, long syllables), theme vowels and others.

In the present investigation, the database was used to determine the quantitative properties of significance for the research such as the relative sizes of various relevant classes of verbs or their frequencies.

\subsection{The background: The asymmetry underlying the opposition between the traditional Slavic perfective and imperfective verbs}\label{ars:sec:Background}\largerpage

As the central question of the paper concerns verbal aspect and affixation, the aim in this section is to highlight some of the relevant notions and introduce the views that are particularly important for the discussion to come, as a bridge to a more precise formulation of the research goals. The relation between lexical and grammatical aspect in Slavic languages and the role of prefixation have received numerous accounts, and still continue to evade an overarching analysis (\cite{Borer.2005, Borik.2006, Ramchand.2004, Arsenij.2006}, among others).\footnote{For a definition of notions like lexical and grammatical aspect, i.e. (a)telicity and (im)per\-fec\-tiv\-i\-ty, as well as quantization and homogeneity, incrementality, etc., see \textcitetv{chapters/10-Milosavljevic}.} Regarding the nature of the morphologically marked opposition between the two classes of verbs in Slavic languages traditionally referred to as perfectives and imperfectives, \citet{Lazor.2010} argues that Slavic verbs are only marked for the lexical aspect, and that the grammatical aspect is not marked up until the structural level of inflection, i.e. it may only be marked by specific verb forms. \citet{Arsenij.2018a} divides Slavic verbs into those that are marked as perfective and those that are unmarked, hence ambiguous, but with an imperfective bias emerging via antipresupposition: that the speaker has not used a verb specified as perfective implies that the speaker did not want to convey a perfective meaning, but its disjunctive alternative, i.e. the imperfective one. \citet{Milosav.dis} argues for a hybrid between these views: as in \citet{Lazor.2010}, verbs in Slavic are only marked for lexical aspect, and as in \citet{Arsenij.2018a}, they can be either strictly telic (the traditional perfective verbs) or unspecified for telicity (traditional imperfectives). Like in \citet{Arsenij.2018a}, the atelic bias of the traditional imperfectives stems from antipresupposition, but is often additionally supported by the aktionsart (it is more difficult to impose a telic interpretation on verbs denoting states than on verbs denoting processes, which are in turn more difficult than verbs denoting culminating events, such as secondary imperfectives). Like in \citet{Lazor.2010}, grammatical aspect is specified at a higher structural level, strongly dependent on the value of lexical aspect (see e.g. \cite{Borik.2006} for discussion). The analysis I develop here builds on \citeauthor{Milosav.dis}'s view. In what follows, I spell out the exact telic and atelic interpretations between which traditional imperfective verbs are ambiguous (a more fine-grained discussion is offered in \sectref{ars:sec:impftel}). 

Based on the presented view, in the rest of the paper, I use the term \textsc{aspectually unspecified (AU) verbs} for the traditional imperfective verbs, and \textsc{aspectually singular (AS) verbs} for the traditional perfective verbs in Slavic. AU verbs normally head verbal expressions that pass tests as atelic, i.e. homogeneous predicates (following \cite{Bennet.1972, Verkuyl.1972, Bach.1986, Krifka.1989} and others, in assuming that properties of quantity mereologically modelled as quantization and homogeneity underlie the notions of telicity and atelicity, respectively). The predicate describing a state in \REF{ars:ex:ambiga}, or one describing a process, as in \REF{ars:ex:ambigb}, indeed by default show atelic behavior. AS verbs normally head verbal predicates that display telic behavior and describe events involving a phase transition (which makes them quantized), as in \REF{ars:ex:ambigc}. Finally, there are also AU verbs which describe eventualities that involve a phase transition, as in \REF{ars:ex:ambigd}. I refer to this as the secondary imperfective pattern since it most frequently occurs with traditional secondary imperfectives (verbs derived by imperfectivizing a perfective verb, in the adopted terminology: secondary AU verb), but, crucially for the present discussion, there are other classes (apparently) displaying this pattern too. Verbal expressions headed by these verbs normally pass tests both as telic and as atelic, and can be assigned four different readings.\largerpage

\ea\label{ars:ex:ambig}
		\begin{xlist}
	
	    \ex  \gll Marija je spava-la \minsp{(??} za) dva sata.\\ 
        M \textsc{aux} sleep\textsc{-ptcp.ipfv} {} for two hours \\ 
        \glt (Intended:) `Marija slept for/in two hours.'\label{ars:ex:ambiga}
	    
	    \ex  \gll Marija je ras-la \minsp{(??} za) 15 godina.\\ 
        M \textsc{aux} grow\textsc{-ptcp.ipfv} {} for 15 years \\ 
        \glt (Intended:) `Marija grew for\slash in 15 years.'\label{ars:ex:ambigb}
        
	    \ex  \gll Marija se u-spava-la \minsp{*(} za) dva sata.\\ 
        M \textsc{refl} in-sleep\textsc{-ptcp.pfv} {} for two hours \\
        \glt (Intended:) `Marija fell asleep in\slash for two hours.'\label{ars:ex:ambigc}

	    \ex  \gll Marija se u-spavlj-iva-la \minsp{(} za) dva sata.\\ 
        M \textsc{refl} in-sleep\textsc{-suff-ptcp.ipfv} {} for two hours \\
        \glt `Marija was falling asleep in\slash for two hours.'\label{ars:ex:ambigd}
        \ea process\slash preparatory stage: (Intended:) `Marija was working on getting herself to sleep for two hours\slash in two hours.'
        \ex phase transition (slow motion): (Intended:) `Marija was falling asleep for two hours\slash in two hours.'
        \ex an unbounded series of iterations: `A series of iterations of events of Maria falling asleep (in two hours) was going on (for two hours).'
        \ex the general-factual reading: (Intended:) `At least once in the past, Maria fell asleep for two hours\slash in two hours.'
        \z

       	\end{xlist}
\z

\noindent While expressions headed by AS verbs are strictly telic, those headed by imperfective verbs display atelic behavior, but are not restricted to it. As soon as a possible source of quantization is introduced into the predicate describing the event -- in terms of any kind of overtly, or contextually specified bounds -- the predicate begins to display the secondary imperfective pattern, including passing the temporal duration modification test as telic (for a detailed discussion see \citealt{chapters/10-Milosavljevic, Milosav.dis}). This is illustrated in \REF{ars:ex:boundimp}, where the latent source of quantization is a measure phrase as in \REF{ars:ex:boundimpa} and \REF{ars:ex:boundimpb}, i.e. a goal phrase as in \REF{ars:ex:boundimpc}. I argue in this paper that these predicates have aspectually unspecified interpretations. The set of eventualities matching their extension includes pragmatically salient subsets which satisfy telic predicates (i.e. subsets consisting solely of bounded events). The latent sources of telicity in \REF{ars:ex:boundimp} merely support the pragmatic strengthening of the interpretation in the sense of \citet{Horn.1989} to one of these subsumed telic meanings.\largerpage


\ea\label{ars:ex:boundimp}
		\begin{xlist}
	
	    \ex  \gll Marija je spavala svoju dozu \minsp{(} za) dva sata.\\ 
        M \textsc{aux} slept\textsc{.ptcp.ipfv} her dose {} for two hours \\ 
        \glt `Marija had her dose of sleep for\slash in two hours.'\label{ars:ex:boundimpa}
	    
	    \ex  \gll Marija je rasla dva centimetra \minsp{(} za) godinu dana.\\ 
        M \textsc{aux} grew\textsc{.ptcp.ipfv} two centimeters {} for year days  \\ 
        \glt `Marija grew two centimeters for\slash in a year.'\label{ars:ex:boundimpb}

	    \ex  \gll Marija je putovala do Lajkovca \minsp{(} za) dva sata.\\ 
        M \textsc{aux} travelled\textsc{.ptcp.ipfv} to Lajkovac {} for two hours  \\ 
        \glt `Marija (has) travelled to Lajkovac in/for two hours.'\label{ars:ex:boundimpc}

       	\end{xlist}
\z

\noindent As indicated by the examples in \REF{ars:ex:SlavAsp}, \REF{ars:ex:ambig} and \REF{ars:ex:boundimp}, the simplest verbal predicates are unspecified for aspect, and there are various ways to assign them a telic interpretation. I argue in this paper that there are two degrees of strength of this assignment. Consider the verbal expression headed by a simple verb in \REF{ars:ex:QPbounda}, which I analyze as unspecified for telicity with a strong bias for an atelic interpretation due to antipresupposition (the availability of a direct telic counterpart indicates that telicity was not intended). On the one hand, this predicate can be imposed telicity by prefixation, as in \REF{ars:ex:QPboundb} where a lexical prefix contributes a result, or in \REF{ars:ex:QPboundb1}, where a superlexical prefix specifies a bounded quantity. Alternatively, the suffix \textit{-nu} may strongly impose telicity by specifying a quantity smaller than some contextually provided standard, as in \REF{ars:ex:QPboundc}. Both strong ways of imposing telicity make the verb perfective in the traditional sense. 

On the other hand, a quantized incremental theme as in \REF{ars:ex:QPboundd} or a result (i.e. goal) specification, as in  \REF{ars:ex:QPboundd1}, when the verb licenses one, may impose an interpretation which makes prominent a subset of events from the extension of the predicate, which itself matches a telic characteristic predicate. The example in \REF{ars:ex:QPboundd} makes prominent the set of eventualities measured out and thus telicized by the bounds of the daily dose of planking, and that in \REF{ars:ex:QPboundd1} the set of eventualities telicized by a pair of a presupposed initial point and the explicated final point (muscle cramps). Finally, quantization may come from a measure phrase, as in \REF{ars:ex:QPbounde} (see also \citealt{Pereltsvaig.2000, Szucsich.2001, chapters/10-Milosavljevic, Milosav.dis} for a discussion of temporal adverbials imposing telicity). In this latter set of cases, the verb remains AU, and the overall interpretation preserves its default atelic status. In \sectref{ars:sec:theoretical}, I argue that these bounds only provide a specification of atoms for the lexical component of the predicate, but do not necessarily include the contribution of the syntactic head responsible for telicity.\largerpage[2]

\ea\label{ars:ex:QPbound}
		\begin{xlist}
	
	    \ex  \gll Marija je radila plenking \minsp{(??} za) dva sata.\\ 
        M \textsc{aux} do\textsc{.ptcp.ipfv} planking {} for two hours \\ 
        \glt `Marija did planking for two hours.'\label{ars:ex:QPbounda}
        
        \ex  \gll Marija je do-radila plenking \minsp{??(} za) dva minuta.\\ 
        M \textsc{aux} to-do\textsc{.ptcp.pfv} planking {} for two minutes \\ 
        \glt `Marija finished her planking in two minutes.'\label{ars:ex:QPboundb}
	    
        \ex  \gll Marija je od-radila plenking \minsp{??(} za) dva minuta.\\ 
        M \textsc{aux} from-do\textsc{.ptcp.pfv} planking {} for two minutes \\ 
        \glt `Marija did her planking in two minutes.'\label{ars:ex:QPboundb1}
	
	    \ex  \gll Marija je rad-nu-la plenking \minsp{(} za) dve sekunde.\\ 
        M \textsc{aux} do\textsc{-suff-ptcp.pfv} planking {} for two seconds \\ \label{ars:ex:QPboundc}
        \ea Without \textit{za}: `Marija did two seconds of planking.' 
        \ex With \textit{za}: `Marija did a little bit of planking in two seconds.'
        \z
       \ex  \gll Marija je radila  svoj dnevni plenking \minsp{(} za) dva sata.\\ 
        M \textsc{aux} do\textsc{.ptcp.ipfv} her daily planking {} for two hours \\ 
        \glt `Marija did her daily portion of planking for\slash in two hours.'\label{ars:ex:QPboundd}
        
       \ex  \gll Marija je radila  plenking do grča mišića \minsp{(} za) dva sata.\\ 
        M \textsc{aux} do\textsc{.ptcp.ipfv} planking to spasm muscles {} for two hours \\ 
        \glt `Marija did planking until her muscles cramped for\slash in two hours.'\label{ars:ex:QPboundd1}

	    \ex  \gll Marija je radila plenking pet minuta za sat vremena.\\ 
        M \textsc{aux} do\textsc{.ptcp.ipfv} planking five minutes for hour time \\ 
        \glt `Marija has done (at least once) an aggregate of five minutes of planking in one hour.'\label{ars:ex:QPbounde}

       	\end{xlist}
\z

\noindent Based on observations of this type, where prefixes and the semelfactive suffix correspond to obligatory telic interpretations, and other sources of quantization to rather latent telicity, the literature in the area of Slavic verbal aspect establishes a strong link between telicity and verbal prefixes. \citet{Fleisch.2019} argue that the only way to derive telic verbal predicates in Slavic is prefixation, and \citet{Lazor.2010} goes as far as claiming that the mapping is bijective: there is no telic verb without the suffix -\textit{nu} or a prefix, nor is there any instance of -\textit{nu} or a prefix that does not introduce telicity. For telic expressions showing no visible telicizing affixes, she postulates a null prefix. Expressions involving a morphologically simple verb with a latent quantization, and more generally all the expressions with an iterative interpretation, which can only be defined in the background of a telic predicate, raise the question whether the simple verbs heading them too involve a null prefix, whose contribution gets overwritten by a structural layer which re-imposes unspecification, or the attested interpretations are pragmatically promoted for truly simple verbs.\largerpage[2]

A related question concerns AU verbs which are prefixed. \citet{Lazor.2010} argues that these prefixes introduce telicity, which is then neutralized by an atelicizing operation (typically, secondary imperfectivization by a suffix). Considering that these verbs too pass tests both as telic and as atelic, a prominent analytic option is that the embedded telic structure is available for the tests of telicity. This would mean that the full predicate is atelic, but tests may also target its compositional components, and gives a reductionist advantage to one of the two analyses invoked above -- the one which assumes a null prefix also for the latently quantized simple AU verbs. The reductionist advantage lies in the fact that all (latently) quantized predicates can be generalized to involve a prefix, rather  than having to define particular subclasses, some of which do and some do not involve a prefix. In light of the main goal of this paper, to scrutinize the arguments for null verbal prefixes in Slavic languages, this expands the empirical focus of the paper also to the simple AU verbs that may have progressive and iterative interpretations.

\section{The assumed theoretical view}\label{ars:sec:theoretical}

I present my view of the composition of verbal predicates using, for convenience, the framework of Distributed Morphology (DM, \citealt{Halle.1993}), but it could equally well be formulated in terms of Nanosyntax (\citealt{Starke.2010}) or another realizational framework, as nothing crucially depends on the specific properties of DM. I take roots to denote predicates which can take arguments (e.g. \citealt{Travis2012}). The structure consisting of the root and its arguments is uncategorized, but I label it as the $\surd\text{-phrase}$ ($\surd\text{P}$) for the purpose of reference in the text, without implying a syntactic projection. Once a root structure is categorized, its arguments may move up to positions introduced by functional projections. 

$\surd\text{P}$s can only merge with a category feature. The one relevant for the discussion is the verbal category. This is illustrated in \REF{ars:ex:rootp} and Figures \ref{ars:fig:1}--\ref{ars:fig:2}, where two \textit{v}Ps are schematically presented, one without and one with a specified goal. The subject leaves the $\surd\text{P}$ in both cases, but in \REF{ars:ex:rootpb}, i.e. \figref{ars:fig:2}, there is additionally a predicate of the small clause, which remains inside the \textit{v}P. The verbal category is assumed to be realized by the theme vowel (\textsc{th}) in SC, and since the examples represent just the \textit{v}P the inflection is completely missing. I remain agnostic regarding the way the root ends up forming a word with the categorizer (via head-movement, PF dislocation or in some other way) as it is not relevant for the topic of discussion.

\ea\label{ars:ex:rootp}
		\begin{xlist}
	
	    \ex  \gll ptica let-i-\\ 
        bird fly-\textsc{th}\\ \label{ars:ex:rootpa}
	    
	    \ex  \gll ptica let-i- na jug\\ 
        bird fly-\textsc{th} on south\\ \label{ars:ex:rootpb}

	    \end{xlist}
\z

      \begin{figure}
      \begin{floatrow}
      \captionsetup{margin=.05\linewidth}
      \ffigbox[.4\textwidth]{\begin{forest}
                [\textit{v}P [ptica `bird']
                [\textit{v}$'$
                [{\textit{-i ${[v]}$}}]
                [$\surd$P [$\surd{\text{let}}$ `fly'] 
                [\sout{ptica} `bird'
                ]]]]
         \end{forest}}{\caption{Syntactic representation of \REF{ars:ex:rootpa}\label{ars:fig:1}}}
      \ffigbox[.6\textwidth]{\begin{forest}
                [\textit{v}P [ptica `bird']
                [\textit{v}$'$
                [{\textit{-i ${[v]}$}}]
                [$\surd$P [$\surd{\text{let}}$ `fly'] 
                [SC [\sout{ptica} `bird'] 
                [PP [na `on'] 
                [jug `south'
                ]]]]]]
         \end{forest}}{\caption{Syntactic representation of \REF{ars:ex:rootpb}\label{ars:fig:2}}}  
     \end{floatrow}
     \end{figure}

I assume the verbal category to have a double contribution. It restricts the ontological class of the predicate to eventualities, and to kinds, by introducing a variable restricted to event kinds as the referential argument of the expression, and imposes division on the complement, thus acting as a grinder (\citealt{Pelletier.1975}). As a result, the \textit{v}P denotes a non-atomic join lattice (as opposed to \citeauthor{Chierchia1998}'s \citeyear{Chierchia1998} atomic join lattice for nominal kinds) satisfying the predicate in its complement (the base of the lattice consists of parts of events). The \textit{v}P in \REF{ars:ex:rootpa}, i.e. Figure~\ref{ars:fig:1}, thus denotes all the possible sums over the maximal set of events of birds flying and all their parts, and the \textit{v}P in \REF{ars:ex:rootpb}, i.e. Figure~\ref{ars:fig:2}, all the possible sums over the maximal set of events of birds flying south and all their parts.

Recall that as explicated in \sectref{ars:sec:Background}, I argue that the aspectual division characteristic of Slavic verbs traditionally described as one of perfectivity is rather an opposition between telic and unspecified verbs. Following \citet{Borer.2005} and \citet{Lazor.2010}, I assume that it structurally corresponds to the presence or absence of a functional projection immediately above the category projection \textit{v}P, which I label QP. Verbs with a QP above their \textit{v}P are telic, i.e. they fall in the traditional class of perfectives, and those without it are atelic, i.e. AU verbs.

Unlike \citet{Borer.2005} and \citet{Lazor.2010}, who take Q$^0$ to effect quantization, I take it (with \citealt{Milosav.dis} -- see his work for further arguments for this view and for references to relevant previous discussions) to impose a singular interpretation, i.e. to restrict the non-atomic join lattice to its base and impose atomicity on it. This basically corresponds to \posscitet{FilipRothstein.2005} maximality operator, except that in the current approach it applies to the base of the lattice rather than to the entire predicate (here it would mean the entire lattice, in which case the derived denotation would be the sum of the entire base).\footnote{One difference to \citet{FilipRothstein.2005} is that in their approach the verb includes in its denotation the atomicity crucial for the application of the maximality operator, while in the present approach atomicity is provided in a latent way by subconstituents of the VP, or simply by the context.} The meaning derived is the set of individual maximal event kinds satisfying the predicate denoted by the $\surd\text{P}$. On this view, a near equivalence can be established between telic verbal predicates (the denotations of AS verbs) and nominal singulars, as well as between atelic verbal predicates (denotations of AU verbs) and mass nouns. Consequently, the semantic effect of the respective head can be considered the same: whatever the way that singularity is imposed by singular number on nominals, is also the way it is imposed by the verbal counterpart (e.g. by blocking, or failing to provide, the sum operation needed to form the lattice).

The question emerges why telic predicates, i.e. predicates headed by AS verbs, tend to be used with the perfective viewpoint aspect, and atelic predicates, i.e. those whose expression involves AU verbs, with the imperfective viewpoint aspect. I assume that this mapping is pragmatically induced. A perfective viewpoint aspect presupposes boundedness; otherwise, it would be logically impossible to take a perspective on the eventuality from a time outside of its temporal trace, or to have the trace be contained in the reference time, which are the standard ways of modelling the perfective viewpoint. In light of the view that all events are presupposed to have initial bounds (\citealt{Arsenij.2006}), a final bound suffices for quantization. If whenever the viewpoint aspect is perfective, the event predicate satisfies telicity, then perfective viewpoint aspect will present a pragmatically stronger interpretation of telic verbal expressions, and will thus undergo strengthening in \posscitet{Horn.1989} sense whenever the context supports it. In result, quantized predicates, typically headed by AS verbs, will be the default way of describing eventualities viewed from the perfective perspective. On the other hand, if the reference time is properly included in the temporal interval of the eventuality, then within the reference time, it is impossible to epistemically verify the boundedness of the predicate (the ground for the imperfective paradox). Therefore, AU verbs are the default way of describing eventualities viewed in the imperfective perspective. That both these present pragmatic rather than semantic effects is evidenced by the fact that they can be cancelled: the general-factual use of AU verbs involves a perfective viewpoint, and instances of the imperfective paradox involve the use of AS verbs in interrupted progressive (hence imperfective viewpoint) contexts.

The feature representing the singular operator in QP needs to operate on a unit of counting, but is in itself underspecified for it. In the typical case, it receives this specification from the structurally closest compositional component of the $\surd\text{P}$ contributing the characteristic predicate of the atom to the aggregate predicate. I hence model this specification as a feature that is copied from the respective sub-predicate as the value of the singular atomizing feature in the head of QP. When such a predicate is absent from the structure, the singular feature receives the default value and the corresponding interpretation, where the unit of counting is the smallest eventuality satisfying the predicate for some contextually specified level of granulation. The singular feature with the default value is realized as the semelfactive suffix \textit{-nu}, as illustrated in \REF{ars:ex:QPa}, i.e. \figref{ars:fig:3}. 

When the singular feature takes a specific value and thus imposes atoms defined by the respective characteristic predicate as the unit of counting, valuation obtains via agreement: the singular feature probes into the c-commanded structure and agrees with the most local predicate specifying a possible unit of counting. Typically, this is a source, as in \REF{ars:ex:QPb}, i.e. \figref{ars:fig:4}, a goal, as in \REF{ars:ex:QPc}, i.e. \figref{ars:fig:5}, or a result predicate (of another kind). The contrast between, on the one hand \REF{ars:ex:QPa}--\REF{ars:ex:QPb}, where the inclusion of the goal in the event is not entailed, and \REF{ars:ex:QPc} on the other, where it is, is exactly predicted by the analysis: agreement with the goal, realized by a goal-oriented prefix on the verb, results in a restriction of  the counting units to event-atoms specified by reaching the goal, and hence it cannot be negated. Effectively, in this example, the agreement of Q$^0$ with the predicate of the small clause, i.e. its promotion from a regular sub-argument into the value of the feature singular, changes the interpretation of the small clause from the direction into the goal of the motion event. The absence of agreement or agreement with the source, as in the first two examples, allows for the negation of reaching the goal, since it leaves the small clause with the source interpretation.\largerpage

\ea\label{ars:ex:QP}
		\begin{xlist}
	
	    \ex  \gll Ptica je let-nu-la na jug, ali nije stigla.\\ 
        bird \textsc{aux} fly-\textsc{sem-ptcp} on south, but \textsc{neg.aux} arrived\\
        \glt `The bird flew south a little bit, but hasn't arrived.'\\ \label{ars:ex:QPa}

	    \ex  \gll Ptica je od-let-e-la na jug, ali nije stigla.\\ 
        bird \textsc{aux} from-fly-\textsc{th-ptcp} on south, but \textsc{neg.aux} arrived\\
        \glt `The bird flew away towards the south, but hasn't arrived.'\\ \label{ars:ex:QPb}
        
	    \ex  \gll Ptica je do-let-e-la na jug, \minsp{\#} ali nije stigla.\\ 
        bird \textsc{aux} to-fly-\textsc{th-ptcp} on south, {} but \textsc{neg.aux} arrived\\
        \glt `The bird came to the south flying, $\#$but \textsc{neg.aux} arrived.'\\ \label{ars:ex:QPc}

	    \end{xlist}
\z

       \begin{figure}[p]\small
      \caption{Syntactic representation of \REF{ars:ex:QPa}}
	     \begin{forest}
[Q$'$ [{-nu [\textit{atom}]}]
[\textit{v}P [ptica `bird']
[\textit{v}$'$
[{\textit{-e $[v]$}}]
[$\surd$P [$\surd{\text{let}}$ `fly'] 
[SC [\sout{ptica} `bird'] 
[PP [na `on'] 
[jug `south'
]]]]]]]
         \end{forest}\label{ars:fig:3}
         \end{figure}

       \begin{figure}[p]\small
      \caption{Syntactic representation of \REF{ars:ex:QPb}}
	     \begin{forest}
            [Q$'$ [{od- $[\textit{atom}:\textit{from}]$}]
            [\textit{v}P [ptica `bird']
            [\textit{v}$'$
            [{\textit{-e $[v]$}}]
            [$\surd$P [$\surd{\text{let}}$ `fly'] 
            [SC [\sout{ptica} `bird'] 
            [PathP [PP [od pro `from \textit{pro}'] [`bird']] 
            [PP [na `on'] 
            [jug `south'
            ]]]]]]]]
         \end{forest}\label{ars:fig:4}
         \end{figure}

       \begin{figure}\small
      \caption{Syntactic representation of \REF{ars:ex:QPc}}
	     \begin{forest}
            [Q$'$ [{do- $[\textit{atom}:\textit{to}]$}]
            [\textit{v}P [ptica `bird']
            [\textit{v}$'$
            [{\textit{-e $[v]$}}]
            [$\surd$P [$\surd{\text{let}}$ `fly'] 
            [SC [\sout{ptica} `bird'] 
            [PathP [PP [do pro `to \textit{pro}'] [`bird']] 
            [PP [na `on'] 
            [jug `south'
            ]]]]]]]]
         \end{forest}\label{ars:fig:5}
         \end{figure}

These examples show that telicity, i.e. singularity in the present account, depends on the syntactic marking and not on the lexical description. The same lexical description (i.e. the same $\surd\text{P}$ taking a path PP) derives an atelic predicate if no QP projects, as in \REF{ars:ex:rootpb}, i.e. Figure~\ref{ars:fig:2}. Its path component  (\textit{na jug} `to the south' in the examples above) is only a latent telicizer: it realizes this capacity only if a QP agrees with its predicate head. This is the reason why \citet{Quaglia.2021} describe what I label QP as the result-Voice phrase: the projection that introduces the result as an argument of the verb by agreeing with the predicate of a respective phrase, copying its content and realizing it as a clitic. QP fits better as a label as it also includes the option with an unvalued singular feature realized by the suffix -\textit{nu} as well as valuation by various adverbials (see \citetv{chapters/10-Milosavljevic}) or source prefixes. The fact that an atelic event predicate often stands in a superset relation to a (discourse-prominent) telic event predicate becomes relevant in \sectref{ars:sec:impftel}, where simple AU verbs are discussed whose interpretation involves a prominent role of a salient telic predicate, as I argue -- without involving a structural level specifying telicity.

Finally, I follow \citet{Arsenij.2018a} and \citet{SimMilAr2021} in analyzing secondary imperfectivization as reverbalization. \citet{SimMilAr2021} start from the observation that certain secondary imperfectives are derived by stacking an additional theme vowel on top of the existing one, and that all imperfectivizing suffixes can be analyzed into two of the independently attested theme vowels with a consonant in between which is plausibly realized as a glide. An analysis is developed where indeed secondary imperfectivization is always effected by either a single theme vowel or a sequence of two theme vowels. Considering that secondary imperfectivization targets AS verbs and assuming that theme vowels realize the category head, this implies that secondary imperfectivization amounts to deriving an unrestricted verb from a verb which is restricted to singularity. A new unrestricted verb is derived by merging the verbal structure with a new verbal category head, i.e. deriving a new verb from it. As in the present view, the category head \textit{v} grinds the predicate in the complement, the contribution of the QP is neutralized and the verb denotes an AU predicate again. This is represented in \REF{ars:ex:reverb}.


\ea\label{ars:ex:reverb}
    \gll Ptica je do-let-e-a-la na jug.\\ 
    bird \textsc{aux} to-fly-\textsc{th-th-ptcp} on south\\
    \glt `The bird was coming to the south flying.'\\ (\textit{do-let-e-a-la} is realized /doletala/ for reasons that I do not discuss.)\\ 
\z\largerpage

\begin{figure}
\caption{Syntactic representation of \REF{ars:ex:reverb}}
\begin{forest}
    [\textit{v}P [ptica `bird']
    [\textit{v}$'$ [{$[v]$}]
    [Q$'$ [{do- $[\textit{atom}:\textit{to}]$}]
    [\textit{v}P [\sout{ptica} `bird']
    [\textit{v}$'$
    [{\textit{-e $[v]$}}]
    [$\surd$P [$\surd{\text{let}}$ `fly'] 
    [SC [\sout{ptica} `bird'] 
    [PathP [PP [do pro `to \textit{pro}'] [`bird']] 
    [PP [na `on'] 
    [jug `south'
    ]]]]]]]]]]
\end{forest}\label{ars:fig:6}
\end{figure}

The theme vowel of the lower \textit{v}P cannot be fully realized and it obligatorily merges with the final segment of the root. While the theme $\langle \text{j(e), i}\rangle$% $\angle$(j)e, i$\rangle$
, as in the example in \REF{ars:ex:reverb}, i.e. \figref{ars:fig:6}, contracts without a trace, other themes, including \mbox{$\langle$i, i$\rangle$} as illustrated in \REF{ars:ex:palatal}, palatalize the final segment of the base, or display other phonological effects.{\interfootnotelinepenalty=10000\footnote{All theme vowels in SC have two allomorphs, surfacing in different subsets of verb forms. In the present paper, therefore, each theme vowel is represented as an ordered pair of the allomorph surfacing in the present tense and that surfacing in the infinitive, in that order.}}

\ea\label{ars:ex:palatal}

		\begin{xlist}
	
	    \ex \gll rod-i-ti / rađ-a-ti\\
        bear-\textsc{th-inf.pfv} {} bear-\textsc{th-inf.ipfv}\\
        \glt `give birth'
        
        \ex \gll set-i-ti / seć-a-ti\\ 
         remember-\textsc{th-inf.pfv}  {} remember-\textsc{th-inf.ipfv}\\
        \glt `remember' 
        
        \ex \gll u-prav-i-ti / u-pravlj-a-ti\\ 
        in-straight-\textsc{th-inf.pfv}  {} in-straight-\textsc{th-inf.ipfv}\\
        \glt `steer'

		\end{xlist}

\z

\noindent In the view presented, \posscitet{Lazor.2010} generalization about obligatory affixes in Slavic languages then translates as a requirement that the features copied to Q$^0$ by agreement be realized, whether or not the QP is embedded in a reverbalizing \textit{v}P. Note that the view that Q$^0$ specifies singularity rather than quantization does not bear on the particular issue of affixation. The model outlined crucially departs from \citet{Borer.2005} and \citet{Lazor.2010} in the reverbalization view of secondary imperfectivization. This issue is in particular relevant for the imperfective verbs compatible with the \textit{za}-phrase, because it raises the question whether these verbs are \textit{v}Ps without a QP, in which case their behavior in tests of telicity needs to be explained, but their affixless realization is expected, or they are \textit{v}Ps projected on top of a QP, in which case their lack of prefixes needs to be explained (e.g., in terms of null prefixes), but their telic behavior on certain tests is expected. In \sectref{ars:sec:Simple}, I argue for the former option.


\section{Simple telic perfectives}\label{ars:sec:s-simplepref}\largerpage

Every Slavic language has a class of simple telic perfective verbs -- i.e., verbs without prefixes or suffixes (other than inflection endings) that pass tests as telic. All these verbs describe achievements (or semelfactives), which makes them less compatible with durative adverbials. For this reason, I use the conjunction test (\citealt{Verkuyl.1972}) to illustrate their telicity in \REF{ars:ex:simplepref}, where neither of the verbs allows for a single event interpretation characteristic of atelic verbal predicates.\largerpage[1]

\ea\label{ars:ex:simplepref}
	\begin{xlist} 
		
	    \ex  \gll Jovan je stavio mleko u frižider sinoć i jutros.\\ 
J \textsc{aux} put\textsc{.ptcp.pfv} milk in fridge last.night and  this.morning \\ 
\glt `Jovan put the milk in the fridge last night and this morning.'\\\hfill(two events only) \label{ars:ex:simplepref-a}
	    \ex  \gll Marija je spasila psa iz reke sinoć i  jutros.\\ 
M \textsc{aux} save\textsc{.ptcp.pfv} dog from river  last.night and  this.morning \\ 
\glt `Marija saved the dog from the river last night and this morning.'\\\hfill (two events only) \label{ars:ex:simplepref-b}

	\end{xlist}
\z

\noindent \citet{Lazor.2010} discusses verbs of this type in Russian and Polish, and postulates phonologically null prefixes to maintain the strong generalization that telicity is universally marked by an affix. This theoretical move both complicates the system and raises some additional questions such as the conditions on null realization of prefixes (when does the same feature get an overt and when a null realization?), the grammatical status of null prefixes (what kind of empty category are they?), and their competition with overt prefixes. This calls for a thorough consideration of alternative analyses. 

Simple telic perfectives have been downplayed in the literature as an enumerable closed class, plausibly listed in the lexicon (e.g. \cite{Topor.2000}). If all these verbs are stored in the lexicon and idiomatic, then they do not pose a problem for the generalization that Slavic languages obligatorily mark singularity (i.e. telicity) by affixes, as the generalization only concerns compositionally derived telicity. \citet{Lazor.2010} gives an ambiguous view of the issue. In one place (p. 80), she compares simple perfectives with English irregular plurals, pointing out that both are small closed classes (hence likely listed). In another (pp. 28--29), however, she stipulates that null prefixes are productive, pointing out that in Russian simple loan verbs can easily be used as perfective, and that in Bulgarian there are also a larger number of simple perfectives. 

In Slavic languages, verbs are borrowed as biaspectuals. On the present approach, biaspectual verbs are irrelevant for the necessity of null prefixes, due to the fact that the meaning of AU verbs, identified with the homogeneous kind denotation of the \textit{v}P, includes the base of the lattice, i.e. the denotation of the singular predicate and in the absence of competition (i.e. of a restricted atomic minimal pair) can be used for singular denotations. However, the claim that simple AS verbs too are productive, as indicated for Bulgarian, indeed supports the introduction of null prefixes. 

As SC is similar to Bulgarian in having, at least at first glance, a larger number of simple AS verbs, I focus on establishing whether indeed this class can be considered productive, or it rather shows the quantitative properties of classes idiomatically listed in the lexicon. This question is best answered by a quantitative investigation into the size and frequency of the class of simple telic perfectives. A closed unproductive class fits a relatively small size (several dozens at most) and a high frequency. An open productive class makes the inverse prediction. 

Among the 5300 SC verbs collected in the database by \citet{Arsetal.2021}, 46 are annotated as telic simple verbs (throughout the quantitative report, by verbs, I refer to verbal lemmata in the corpus). This amounts to 5.5\% of all the simple verbs in the database, i.e. 0.85\% of all the verbs included. On a more thorough analysis, it turns out that the class is even smaller, since the original set of verbs includes geographic variants of the same verb as well as verbs which on a closer look display the semelfactive suffix in certain forms and\slash or varieties. After cleaning up these verbs, the number of simple telic perfectives is reduced to 29 (this number cannot be used to calculate the percentage as the rest of the base has not been cleaned from geographic variants, but indicates that such percentages would be significantly lower). 

Besides being small, the class also includes at least three verbs with a somewhat archaic feel (\textit{bataliti} `quit', \textit{turiti} `put', \textit{latiti se} `tackle'), and not a single borrowed verb or neologism. Its average frequency (105.15 tokens per million) is more than three times higher than the average for the database (32.05) -- another marker of low productivity (e.g., \cite[22--35]{Plag.2012}). All in all, the quantitative data are compatible with treating these verbs as idiomatic and thus orthogonal to \posscitet{Lazor.2010} generalization. The stem of these verbs (the component consisting of the root and the theme vowel) is likely lexically stored with the semantics matching a QP, without a QP being projected, compositionally interpreted and realized as a prefix.


\section{Imperfectives and telicity}\label{ars:sec:s-simpleimperf}\largerpage

Since secondary imperfectives are assumed in the present paper to be reverbalized telic event kinds, and hence each of these verbs embeds a structure which represents a telic event kind, the generalization investigated in the paper raises the question whether there are simple verbs with a semantics equivalent to secondary imperfectives (i.e. having progressive and iterative meanings). If there are such verbs, they too become relevant for the generalization that telicity is universally marked by an affix. The reason is that secondary imperfectives are taken to include a QP, and therefore morphologically simple verbs expressing the semantics of secondary imperfectives might also be a class in which the QP is present but not realized, contra \posscitet{Lazor.2010} generalization. This section identifies a class of verbs that at first sight match the described pattern, and discusses them in light of the generalization. 

Before focusing on simple imperfectives, a discussion is due of imperfectives more generally, and their behavior regarding aspectual pairs. This discussion is intended to show two things. The first is to identify the class of simple imperfectives indicated above and the second is to argue that the relevant, iterative interpretation of such simple imperfectives always also has a perfective realization by a verb involving an overt prefix. As it is well known that there is a strong correlation between the combination of meaning and argument structure on the one hand and the prefix on the other, this supports the analysis in terms of null prefixes, counterparts of those visible on the perfective pair. In the rest of this section I pursue a more detailed analysis of these issues, leading to the conclusion that simple imperfectives are never full equivalents of secondary imperfectives, and that they consequently do not involve null prefixes either.

\subsection{Four classes of imperfectives regarding aspectual pairs}\label{ars:sec:pairs}\largerpage[2]

The notion of aspectual pairs holds a prominent place in the theory of Slavic verbal aspect. An aspectual pair consists of two verbs with exactly the same meaning and argument structure, distinguished minimally in their aspect: one of them belongs to AS verbs and the other to imperfectives. The prototypical aspectual pair involves a perfective verb and its secondary (i.e. derived) imperfective, but as discussed below, pairs may also be argued to exist where the perfective seems to morphologically include the imperfective (\cite{Janda.2011}), as well as where both members appear to display the same degree of morphological complexity (these are the pairs whose perfective members are the simple perfectives from \sectref{ars:sec:s-simplepref}). The case where the perfective seems to derive from its imperfective pair by prefixation has been subject to debate with respect to the role of the prefix. As the verbal prefix in Slavic languages contributes conceptual content beyond its grammatical effect, the question is how the prefixed verb can still mean the same as its prefixless imperfective pair. In the Russian grammatical tradition, two different answers to this question have been proposed. On one, the prefix in such cases is void of any conceptual content (\cite{Vinogradov.1938, Šahmatov.1941, Švedova.1980}). On the other, referred to as the implication or overlap approach, the meaning of the prefix is included in the meaning of the verbal base; hence, it does not add any new content (\cite{Isačenko.1960, Timberlake.2004, Janda.2011}).

In a somewhat modified version of \posscitet{Maslov.1948} classification of imperfective verbs regarding their aspectual pairs, I divide them into four classes: (i) secondary imperfectives, illustrated in \tabref{ars:tab:prefixed-table},\footnote{The morphological analysis assumed includes the theme vowel of the base verb in its secondary imperfective, even though in some examples, including those used in these examples, it is not visible on the surface (e.g. by lengthening of the vowel). For arguments in favor of this analysis and reason for the lack of surface effects, see \citet{SimMilAr2021}.} (ii) simple imperfectives that have prefixed perfective pairs, while when the semelfactive suffix \textit{-nu} is added their meaning is changed beyond the aspectual contrast, illustrated in \tabref{ars:tab:simplepref-table} (the suffixed perfectives of these verbs are typically rare in use, need to be productively derived, and bear the flavor of a neologism), (iii) simple imperfectives that have perfective pairs with the semelfactive suffix \textit{-nu}, while all their prefixed counterparts display semantic shifts, as in \tabref{ars:tab:simplesuf-table} and (iv) simple imperfectives that have no proper aspectual partners -- as both the prefixed and the suffixed variant bear additional or shifted semantics as in \tabref{ars:tab:simplenone-table}. The difference between the last two classes is that the simple AU verbs forming an aspectual pair via suffixation denote cumulative atomized predicates, i.e. predicates describing iterations of a more or less clearly individuated atom (waving consists of atomic waves, banging of atomic bangs, nodding of atomic nods), while simple imperfectives without perfective partners have prototypical mass properties. The former then present another type of verbal expressions which build on atomic lexical descriptions but are not singular due to the lack of a QP (recall the discussion around example \REF{ars:ex:QP}). Of particular importance for the discussion are simple imperfectives with prefixed perfective partners.\largerpage

\begin{table}[H]
%\begin{tabular}{llll}
\begin{tabularx}{\textwidth}{lXXX}
\lsptoprule
\multicolumn{3}{l}{\textsc{imperfective} (class (a))}\\
%\textbf{class a)}    
& iz-bac-i-iva-ti    & u-trlj-a-ava-ti    & do-trč-a-ava-ti  \\
    & out-throw-\textsc{th-suf-inf} & in-rub-\textsc{th-suf-inf} & to-run-\textsc{th-suf-inf}  \\
    & `throw out'   & `rub in'      & `run to'    
    \medskip\\
\multicolumn{3}{l}{\textsc{perfective} (a minimal pair)}\\
%\textbf{perfective}    
& u-baciti    & pro-trljati    & do-trčati \\ 
%\textbf{min. pair}    
& in-throw & through-rub & to-run \\ 
    & `throw in'   & `rub a little'      & `run to'\\\lspbottomrule
\end{tabularx}
\caption{Aspectual pairs including a secondary imperfective} 
\label{ars:tab:prefixed-table}
\end{table}

\begin{table}[H]
\begin{tabularx}{\textwidth}{lXXX}
\lsptoprule
\multicolumn{3}{l}{\textsc{imperfective} (class (b))}\\
& ređati    & pržiti    & kriviti  \\
    & arrange & fry & blame  \\
    & `arrange'   & `fry'      & `blame'    \medskip\\
\multicolumn{3}{l}{\textsc{prefixed perfective} (a minimal pair)}\\
& po-ređati & iz-pržiti & o-kriviti    \\ 
    & over-arrange & out-fry & around-blame \\ 
    & `arrange'   & `fry'      & `blame'    \medskip\\
\multicolumn{3}{l}{\textsc{suffixed perfective} (not a minimal pair)}\\
& ređ-nu-ti & prž-nu-ti & kriv-nu-ti    \\ 
    & arrange-\textsc{suff-inf} & fry-\textsc{suff-inf} & blame-\textsc{suff-inf} \\ 
    & `arrange a bit'   & `fry a bit'      & `blame a bit'\\\lspbottomrule
\end{tabularx}
\caption{Simple imperfectives with prefixed perfective pairs}
\label{ars:tab:simplepref-table}
\end{table}\pagebreak

\begin{table}[h]
\begin{tabularx}{\textwidth}{lXXX}
\lsptoprule
\multicolumn{3}{l}{\textsc{imperfective} (class (c))}\\
& mahati    & lupati    & klimati\\ 
    & wave & bang & nod\\
    & `wave'    & `bang'    & `nod'\medskip\\
\multicolumn{3}{l}{\textsc{prefixed perfective} (not a minimal pair)}\\
& od-mahati & u-lupati & raz-klimati \\
& from-wave & in-bang & away-nod \\
    & `wave back'    & `whisk'    & `loosen'\medskip\\
\multicolumn{3}{l}{\textsc{suffixed perfective} (a minimal pair)}\\
& mah-nu-ti & lup-nu-ti & klim-nu-ti \\
    & wave-\textsc{suff-inf} & bang-\textsc{suff-inf} & nod-\textsc{suff-inf} \\
    & `wave'    & `bang'    & `nod'\\
\lspbottomrule
\end{tabularx}
\caption{Simple imperfectives with suffixed perfective pair}
\label{ars:tab:simplesuf-table}
\end{table}  

\begin{table}[h]
\begin{tabularx}{\textwidth}{lXXX}
\lsptoprule
\multicolumn{3}{l}{\textsc{imperfective} (class (d))}\\
& sedeti    & mrzeti    & smrdeti\\ 
    & sit & hate & stink\\
    & `sit'    & `hate'    & `stink'\medskip\\ 
\multicolumn{3}{l}{\textsc{prefixed perfective} (not a minimal pair)}\\
& pre-sedeti & za-mrzeti & u-smrdeti \\
& across-lead & for-hate & in-stink \\
    & `sit through'    & `start hating'    & `make stinky' \medskip\\
\multicolumn{3}{l}{\textsc{suffixed perfective} (not a minimal pair)}\\
& ?sed-nu-ti & ?mrz-nu-ti & ?smrd-nu-ti \\
   & \hspace{5pt}sit-\textsc{suff-inf} & \hspace{5pt}hate-\textsc{suff-inf} & \hspace{5pt}stink-\textsc{suff-inf} \\
    & `sit a bit'    & `hate a bit'    & `stink a bit'\\
    \lspbottomrule
\end{tabularx}
\caption{Simple imperfectives without perfective pair}
\label{ars:tab:simplenone-table}
\end{table}

\subsection{Simple imperfectives with perfective pairs: Their aspectual properties}\label{ars:sec:impftel}

Whether secondary imperfective verbs, derived from telic perfectives, are telic, atelic or both has been a matter of debate. One group of authors argue that all imperfectives are atelic (\cite{Borer.2005}, \cite{Macdon.2008}, \cite{Lazor.2010}), another treats them as possibly telic (\cite{Arsenij.2006}, \cite{Borik.2006}, \cite{Brag.2008}, \cite{Stanoj.2012}, \cite{Fleisch.2019}). I assume here, as discussed in \sectref{ars:sec:Background}, that secondary imperfectives are unspecified for telicity, but they are derived from telic predicates over event kinds. 

Consider the tests of telicity in \REF{ars:ex:test-SI}. On the temporal adverbial test, secondary imperfectives pass both the test for telicity and for atelicity. On the temporal conjunction test, they turn out to be atelic: they can combine with a conjunction of \textit{at-x-time} expressions with a single event interpretation. So why do different tests give different results (see also \cite{Mittwoch.2010, Mittwoch.2013} and \citetv{chapters/10-Milosavljevic} for a critical assessment of the tests of telicity)?

\ea\label{ars:ex:test-SI}
	\begin{xlist} 
		
	    \ex  \gll Marija je rasklapala pušku dva minuta.\\ 
                  M \textsc{aux} disassemble\textsc{.ptcp.ipfv} rifle two minutes\\ 
            \ea Process\slash preparatory stage: `Marija was removing parts of the rifle for two minutes (without necessarily reaching completion).'
            \ex Phase transition (slow motion): `Marija was completing her disassembling of the rifle for two minutes (completion is being reached).'
            \ex An unbounded series of iterations: `A series of iterations of events of Maria disassembling the rifle was going on for two minutes.'  
            \z\label{ars:ex:test-SIa}
            
           \ex  \gll Marija je rasklapala pušku za dva minuta.\\ 
            M \textsc{aux} disassemble\textsc{.ptcp.ipfv} rifle for two minutes\\ 
            \glt An unbounded series of iterations: `A series of iterations of events of Maria disassembling the rifle in two minutes was going on.'
            \label{ars:ex:test-SIb}
            
        \ex  \gll Marija je rasklapala pušku u pola pet i u pet.\\ 
                 M \textsc{aux} disassemble\textsc{.ptcp.ipfv} rifle in half five and in five\\ 
            \glt `Marija was disassembling a rifle at half past five and at five o'clock.' \label{ars:ex:test-SIc}

	\end{xlist}
\z

\noindent I argue, based on the discussion in \sectref{ars:sec:theoretical}, that the reason for ambiguity is that the tests target different structural levels, corresponding to different predicates. One level is the QP, and the other the reverbalizing \textit{v}P. The former is accessible to the temporal duration adverbial (and it has to be the one with \textit{za} `for', since only that one matches the QP), but not to the conjunction of temporal adverbials locating the epistemic evaluation time (i.e. reference time), because the epistemic evaluation time is only specified for the reverbalized structure. The reverbalizing \textit{v}P is hence accessible to both kinds of temporal adverbials, but without another QP on top of the reverbalizing \textit{v}P, the temporal duration adverbial has to be the bare one (on the adopted view of aspect, \textit{za}-adverbials must be taken to require restriction to singularity). On closer scrutiny, hence, secondary imperfectives are AU, but they embed a telic event kind, which yields an illusion of telicity with adverbials for duration.\largerpage

In the present paper, I do not dwell on this aspect of the proposal, but turn to its consequences for the main argument of the paper.{\interfootnotelinepenalty=10000\footnote{Prompted by a suggestion by the editors, I provide one quick argument in favor of this view. The analysis predicts that when occurring together as modifiers of secondary imperfectives, being embedded deeper than the bare ones, \textit{za}-adverbials are harder to move higher in the structure than bare temporal duration adverbials. Indeed, for instance, fronting for focalization (with the focal stress indicated in (i) below by the capital letters) is more readily accessible to the bare adverbials, than to the \textit{za}-adverbials (the latter only works as a correction).
\ea\label{ars:ex:adv-foc}
	\begin{xlist} 
		
	    \ex  \gll DVA SAta je Marija rasklapala pušku za dva minuta.\\ 
                  two hours \textsc{aux} M disassemble\textsc{.ptcp.ipfv} rifle for two minutes\\ 
            \glt `It was for two hours that Marija was disassembling the rifle in two minutes.' \label{ars:ex:adv-foca}
            
           \ex \gll Za DVA miNUta je Marija rasklapala pušku dva sata.\\ 
                  for two minutes \textsc{aux} M disassemble\textsc{.ptcp.ipfv} rifle two hours\\ 
            \glt `It was in two minutes that Marija was disassembling the rifle for two hours.' \label{ars:ex:adv-focb}
            
	\end{xlist}
\z
}} Secondary imperfectives all include an affix realizing the QP, in line with \posscitet{Lazor.2010} generalization. However, if there are simple imperfectives which are semantically equivalent to secondary imperfectives, then they embed a QP but do not realize it morphologically. This directs our attention to the simple imperfectives with prefixed perfective counterparts. These verbs have available the same readings as secondary imperfectives: the two progressive interpretations (zooming in onto the process subevent or onto the phase transition) and the iterative one, as illustrated in \REF{ars:ex:test-simple-a}. Moreover, they combine with the \textit{in}-phrase, which is interpreted as a measure of the temporal interval of a related telic event predicate (describing the repeating unit in the iterative interpretation), as well as with the \textit{for}-phrase, which is interpreted as a measure of the temporal interval of the event denoted by the derived imperfective, as in the reading in \REF{ars:ex:test-simple-aa-iii}. The conjunction test verifies atelicity, as shown in \REF{ars:ex:test-simple-ac}.

\ea\label{ars:ex:test-simple-a}
	\begin{xlist} 
		
	    \ex  \gll Marija je punila pušku dva minuta.\\ 
                  M \textsc{aux} charge\textsc{.ptcp.ipfv} rifle two minutes\\ 
            \ea Process\slash preparatory stage: `Marija was putting bullets in the rifle for two minutes.'
            \ex Phase transition (slow motion): `Marija was on the verge of finishing charging the rifle for two minutes.'
            \ex An unbounded series of iterations: `A series of iterations of events of Maria charging the rifle was going on for two minutes.'\label{ars:ex:test-simple-aa-iii}  
            \z\label{ars:ex:test-simple-aa}
            
           \ex  \gll Marija je punila pušku za dva minuta.\\ 
            M \textsc{aux} charge\textsc{.ptcp.ipfv} rifle for two minutes\\ 
            \glt An unbounded series of iterations: `A series of iterations of events of Maria charging the rifle in two minutes was going on.'
            \label{ars:ex:test-simple-ab}
            
        \ex  \gll Marija je punila pušku u pola pet i u pet.\\ 
                 M \textsc{aux} charge\textsc{.ptcp.ipfv} rifle in half five and in five\\ 
            \glt `Marija was charging a rifle at half past five and at five o'clock.' \label{ars:ex:test-simple-ac}

	\end{xlist}
\z

\noindent The remaining two bigger classes of simple imperfectives, those with suffixed perfective counterparts and those without any, have only one interpretation, as illustrated in \REF{ars:ex:test-simple-b} and \REF{ars:ex:test-simple-c}, which makes them uninteresting for the current investigation, except as evidence that other patterns exist.\largerpage

\ea\label{ars:ex:test-simple-b}
	\begin{xlist} 
		
	    \ex  \gll Marija je mahala \minsp{(*} za) dva minuta.\\ 
                  M \textsc{aux} wave\textsc{.ptcp.ipfv} {} for two minutes\\ 
            \glt (Intended:) `Marija was waving for\slash in two minutes.' \label{ars:ex:test-simple-ba}
        \ex  \gll Marija je mahala u pola pet i u pet.\\ 
                 M \textsc{aux} wave\textsc{.ptcp.ipfv} in half five and in five\\ 
            \glt `Marija was waving at half past five and at five o'clock.' \label{ars:ex:test-simple-bb}

	\end{xlist}
\ex\label{ars:ex:test-simple-c}
	\begin{xlist} 
		
	    \ex  \gll Marija je spavala \minsp{(*} za) dva minuta.\\ 
                  M \textsc{aux} sleep\textsc{.ptcp.ipfv} {} for two minutes\\ 
            \glt (Intended:) `Marija was sleeping for\slash in two minutes.' \label{ars:ex:test-simple-ca}
        \ex  \gll Marija je spavala u pola pet i u pet.\\ 
                 M \textsc{aux} sleep\textsc{.ptcp.ipfv} in half five and in five\\ 
            \glt `Marija was sleeping at half past five and at five o'clock.' \label{ars:ex:test-simple-cb}

	\end{xlist}
\z

\noindent To sum up, two classes of simple imperfectives, those illustrated in \REF{ars:ex:test-simple-b} and \REF{ars:ex:test-simple-c}, are plain atelic predicates, even though one of them involves atomic lexical descriptions. This indicates that they lack the QP, which is in line with their lack of aspectual affixation. The third class, simple imperfectives with prefixed perfective pairs, are more similar to secondary imperfectives, in involving an atomic lexical description, having both iterative and progressive interpretations and being compatible with the \textit{za}-phrase (taken to attest telicity). The relevant question is whether these verbs involve a QP as the above grouping suggests, and therefore require the postulation of null prefixes, or they describe eventualities that lend themselves well to the singular interpretation, but do not have it structurally realized, and hence structurally correspond to a primary \textit{v}P rather than to a reverbalizing one. In the latter case, the consequence is that no verbs in SC involve the compositional contribution of a QP without overtly realizing it through affixation.

Before providing a deeper analysis, it is important to establish whether simple imperfectives patterning with secondary imperfectives are productive or wheth\-er they too can be considered a listed idiomatic class and therefore irrelevant for the discussion.

\subsection{Quantitative insights}\label{ars:sec:Quantins}

The database of SC verbs (\cite{Arsetal.2021}) includes 1886 derived AU verbal lemmata, more than double the number of simple ones, of which there are 720. Exactly 800 of the derived AU verbs are secondary imperfectives, i.e. verbs derived from perfectives by an imperfectivizing suffix. All of them form aspectual pairs with their perfective bases, following the pattern in \tabref{ars:tab:prefixed-table}. The remaining derived AU verbs fall into three classes: those that are not part of a minimal aspectual pair (758 verbs), those that have an aspectual partner formed by an additional prefix (40 verbs) and those derived from nouns, adjectives and  borrowings, typically with a biaspectual interpretation (288 verbs, which due to their non-verbal base, are not listed in the tables below).

Among the 720 simple imperfectives in the database, 344 have prefixed and 39 suffixed aspectual partners, 17 are derived from simple perfectives by adding a theme vowel and 320 do not form aspectual pairs. The quantitative data are summarized in \tabref{ars:tab:quanti}.

%\begin{table}
%\begin{tabularx}{\textwidth}{QQQQQQ}
%\lsptoprule
% & prefixed pair & suffixed pair & pair has different theme vowels & no pair & secondary  imperfective \\ 
% \midrule
%Derived & 40 &\slash &\slash & 758 & 800 \\
%Simple & 344 & 39 & 17 & 320 & na \\ 
%\lspbottomrule
%\end{tabularx} \caption{Classes of imperfectives and their sizes, summarized} \label{ars:tab:quanti}
%\end{table}

\begin{table}
\begin{tabularx}{.75\textwidth}{lYY}
\lsptoprule
&Derived&Simple\\\midrule
 Prefixed partner &40&344\\
 Suffixed partner &\slash &39\\
 Partner has different theme vowels &\slash &17\\
 No pair &758&320\\
 Secondary imperfective&800&na \\ 
\lspbottomrule
\end{tabularx} \caption{Classes of imperfectives and their sizes, summarized} \label{ars:tab:quanti}
\end{table}


I argued in \sectref{ars:sec:s-simplepref} that simple perfectives are lexically listed and idiomatic. The 17 simple imperfectives deriving from them by reverbalization are then also not problematic for the generalization that Q$^0$ must be realized by an affix. However, if it turns out that they indeed embed structures deriving singular (i.e. telic) predicates, the 344 simple imperfectives with prefixed perfective pairs are less likely to be listed. This accounts for less than 8\% of all the verbs in the database and 47.78\% of all the simple imperfectives, a fraction unlikely to be listed as idiomatic. The average frequency of this class is 22.33 tokens per million, which is lower than the average for the database, at 32.05. This too is compatible with treating the class as productive.

\section{Simple imperfectives with prefixed aspectual pairs are truly simple}\label{ars:sec:Simple}

\citet{Lazor.2010} generalizes that telicity (in my approach: singularity) must be marked in QP by a prefix or by the semelfactive suffix -\textit{nu}. Such verbs may then be reverbalized, thus becoming AU again. Reverbalization too must be morphologically marked in Slavic, and as argued by \citet{SimMilAr2021}, this marking consists of (sequences of) theme vowels. This is illustrated for prefixed perfectives in \REF{ars:ex:reverbalizaa} for a single theme and in \REF{ars:ex:reverbalizab} for a sequence, as well as in \REF{ars:ex:reverbalizac} for simple perfectives.

\ea\label{ars:ex:reverbaliza}
	\begin{xlist} 
		
	    \ex  \gll u-vid-e-ti $>$ {u-vid-e-a-ti, /uviđati/}\\ 
                  in-see-\textsc{th-inf.pfv} {} in-see-\textsc{th-th-inf.ipfv} \\ 
            \glt `realize/see' \label{ars:ex:reverbalizaa}
            
        \ex  \gll po-plav-i-ti $>$ {po-plav-i-i-a-ti,  /poplavljivati/}\\ 
                  over-flood-\textsc{th-inf.pfv} {} over-flood-\textsc{th-th-th-inf.ipfv} \\ 
            \glt `flood' \label{ars:ex:reverbalizab}
            
        \ex  \gll stav-i-ti $>$ {stav-i-a-ti, /stavljati/}\\ 
                  put-\textsc{th-inf.pfv} {} put-\textsc{th-th-inf.ipfv}\\ 
            \glt `put' \label{ars:ex:reverbalizac}

	\end{xlist}
\z

\noindent The prediction for imperfective verbs lacking prefixes or the semelfactive suffix is thus that if they are underlying secondary imperfectives they will have at least two theme vowels each, and if they are truly simple they will have exactly one.

\ea\label{ars:ex:simples}
	\begin{xlist} 
		
	    \ex  \gll {{cvat-$\emptyset$-ti, /cvasti/}}\\ 
                  bloom-\textsc{th-inf.ipfv} \\ 
            \glt `bloom' \label{ars:ex:simplesa}
            
        \ex  \gll vežb-a-ti\\ 
                  exercise-\textsc{th-inf.ipfv}\\ 
            \glt `exercise' \label{ars:ex:simplesb}
            
        \ex  \gll kvar-i-ti\\ 
                  spoil-\textsc{th-inf}\\ 
            \glt `spoil' \label{ars:ex:simplesc}

	\end{xlist}
\z

\noindent Empirical data support the latter approach. The verbs in question give no ground for identifying more than one theme vowel. This is most obvious for verbs with the theme $\langle \emptyset\text{, e}\rangle$, illustrated in \REF{ars:ex:simplesa} (there are 34 such verbs among the simple imperfectives compatible with the \textit{za}-phrase, which include 720 verbs). Since the theme $\langle \emptyset\text{, e}\rangle$ never occurs in reverbalizing sequences, in verbs of this class, an additional theme $\langle\text{a, a}\rangle$ (the only theme able to reverbalize alone) or a sequence of themes occurring as a reverbalizer would be clearly visible. Examples with other themes are given in \REF{ars:ex:simplesb} and \REF{ars:ex:simplesc}.

This view is further supported by the 17 imperfective aspectual partners reported as simple above in \sectref{ars:sec:Quantins}. Under the analysis adopted here from \citet{SimMilAr2021}, these verbs actually need to be treated as derived by secondary imperfectivization. The reason is that they can be convincingly argued to involve a thematic vowel on top of that realized on the perfective pair, as in those contexts in which the lower theme is expected to surface, it indeed does. This is illustrated in \REF{ars:ex:simplenot}% (the theme vowels which are, due to further derivation, reduced to features -- as shown in detail in \citealt{SimMilAr2021} -- are given in superscript)
, where in \REF{ars:ex:simplenota}, the lower theme is null, hence invisible, in \REF{ars:ex:simplenotb} the final consonant of the root fully absorbs the front theme vowel, but in the contexts like \REF{ars:ex:simplenotc}, where the contraction results in a phonological change, the change is attested on the surface.\footnote{\citet{SimMilAr2021} analyze certain morphological realizations of the verbal category to involve a floating high vowel, which is realized only when it resolves the hiatus and else is silent. These are represented in the examples in \REF{ars:ex:simplenot} and in the following by a superscript.} This strengthens the assumption that in SC, secondary imperfectivization is never null, and that the imperfectives that do not show any traces of it are indeed simple.

\ea\label{ars:ex:simplenot}
	\begin{xlist} 
		
	    \ex  \gll pad-$\emptyset$-ti $>$ {pad-$^\emptyset$-a-ti, /padati/}\\ 
                  fall-\textsc{th-inf.pfv} {} fall-\textsc{th-th-inf.ipfv}\\ 
            \glt `fall' \label{ars:ex:simplenota}
            
	    \ex  \gll bac-i-ti $>$ {bac-$^i$-a-ti, /bacati/}\\ 
                  throw-\textsc{th-inf.pfv} {} throw-\textsc{th-th-inf.ipfv}\\ 
            \glt `throw' \label{ars:ex:simplenotb}
            
	    \ex  \gll stav-i-ti $>$ {stav-$^i$-a-ti, /stavljati/}\\ 
                  put-\textsc{th-inf.pfv} {} put-\textsc{th-th-inf.ipfv}\\ 
            \glt `put' \label{ars:ex:simplenotc}

	\end{xlist}
\z

\noindent Semantic evidence also goes in this direction. I report two relevant observations. The first is that both secondary imperfectives and simple imperfectives compatible with the \textit{za}-phrase fail to license the progressive interpretation in combination with the \textit{za}-phrase. The narrow iterative interpretation with a series of events in one reference time is also unavailable: the only iterative reading available is the general-factual use distributed over a plural reference time (an instantiation of the event kind satisfying both the temporal adverbial and the verbal predicate has taken place in each from the set of relevant reference times). This is illustrated in \REF{ars:ex:za}. The sentence with a perfective verb in \REF{ars:ex:zaa} has an interpretation which involves reference to an event in the past, while the two sentences with imperfective verbs, the one with a simple imperfective in \REF{ars:ex:zab} and the one with a secondary imperfective in \REF{ars:ex:zac}, can only be used if the question under discussion is whether events of Jovan running (in)to the school in ten minutes, i.e. Jovan interrogating Marija in ten seconds, have taken place in each of a set of discourse-given or accommodated reference times, but not to actually refer to a series of such events. The failure to refer to an individual event is also reflected in the fact that the latter two sentences cannot have the progressive interpretation (Jovan was in the process of running (in)to the school in ten minutes and Jovan was in the process of interrogating Marija in ten seconds, respectively). This asymmetry is triggered by the \textit{za}-phrase, as without it, all three sentences can have the progressive interpretation, in addition to other options (see \REF{ars:ex:putema}).

\ea\label{ars:ex:za}
	\begin{xlist} 
		
\ex  \gll Jovan je po-je-$\emptyset$-o kolač za deset	sekundi.\\
J \textsc{aux} over-eat-\textsc{th-ptcp.pfv} cake for ten seconds\\
\glt `Jovan completed an event of eating a cake and it took ten seconds.’\label{ars:ex:zaa}

\ex  \gll Jovan je trč-a-o u školu za deset	minuta.\\
J \textsc{aux} run-\textsc{th-ptcp.ipfv} in school for ten minutes\\
\glt `Jovan used to get to school running in ten minutes.’\label{ars:ex:zab}

\ex  \gll Jovan je iz-pit-i-$^u$a-o Mariju za deset	sekundi.\\
J \textsc{aux} out-ask-\textsc{th-th-ptcp.ipfv} M for ten seconds\\
\glt `Jovan used to complete interrogations of Marija in ten seconds.’\label{ars:ex:zac}
\end{xlist}
\z
	
\noindent Secondary imperfectives uncontroversially embed telic structures, i.e. QPs. At first glance, the observed parallel seems to support the view that simple imperfectives compatible with the \textit{za}-phrase embed a QP too, i.e. that they are secondary imperfectives which fail to show the morphological signature of reverbalization, and should be modeled in terms of null prefixes. The failure of such verbs modified by the \textit{za}-phrase to refer to a single event is then due to the \textit{za}-phrase occurring at the level of the QP, below the reverbalizing \textit{v}P. The latter derives a kind, and therefore the \textit{za}-phrase can only be interpreted at the kind level.

However, also the alternative, that simple imperfectives compatible with the \textit{za}-phrase are primary \textit{v}Ps (i.e. verbalized $\surd\text{P}$s), has the potential to account for this interpretation. Assume that the predicate denoted by the $\surd\text{P}$ is modified by the \textit{za}-phrase. It thus contributes to the predicate that is verbalized in the same fashion as the goal PP, i.e. before the meaning is homogenized by the category head. After categorization, an AU predicate is derived denoting a sum of events of running to school in ten minutes, all their parts, and all the sums thereof. Since AU predicates are weaker than AS predicates, under multiple reference times, the interpretation gets pragmatically strengthened \citep{Horn.1989} to the AS interpretation, i.e. to including one maximal event per reference time. 

\begin{figure}
\caption{Syntactic representation of \REF{ars:ex:zab}}
\begin{forest}
    [\textit{v}P [Jovan]
    [\textit{v}$'$
    [{\textit{-a $[v]$}}]
    [$\surd$P [za 10 minuta `in 10 minutes'] 
    [$\surd$P [$\surd{\text{trč}}$ `run'] 
    [SC [\sout{Jovan}] 
    [PP [u `in'] 
    [školu `school'
    ]]]]]]]
\end{forest}
\end{figure}

This view raises two questions. One is, if the \textit{za}-phrase can modify the $\surd\text{P}$, then how is it excluded from other atelic verbal predicates, i.e. how does it derive the behavior that has qualified it as a test for telicity? The \textit{za}-phrase requires that the modified predicate specifies a possible atom, not necessarily that it is singular. This is exactly what characterizes the simple imperfectives that resemble the secondary ones. One of the other two classes are verbs denoting states (the pattern in \tabref{ars:tab:simplenone-table}), and their roots clearly specify no atoms. The other includes event predicates which are inherently atomic, but do not specify or likely lead to a result (the pattern in \tabref{ars:tab:simplesuf-table}). These verbs do not combine with the \textit{za}-phrase because their atoms are conceptualized to take a point in time, and hence resist this type of modification just like semelfactives do. This is confirmed by the fact that when an appropriate context is set, which implies a prolonged duration, modification is actually possible. This is illustrated in \REF{ars:ex:simsem}.\footnote{\REF{ars:ex:simsema} is acceptable if the \textit{za}-phrase measures the epistemic evaluation time: for a second, Jovan was waving, but this interpretation is orthogonal to the issue.}

\ea\label{ars:ex:simsem}\judgewidth{??}
	\begin{xlist} 
		
\ex[??]{\gll Jovan je mah-a-o za sekund.\\
J \textsc{aux} wav-\textsc{th-ptcp.ipfv} for second\\
\glt `Jovan used to wave in a second.’\label{ars:ex:simsema}}

\ex[??]{\gll Jovan je mah-nu-o za sekund.\\
J \textsc{aux} wav-\textsc{th-ptcp.pfv} for second\\
\glt `Jovan waved once in a second.’\label{ars:ex:simsemb}}

\ex[]{\textit{Context:} The task was to wave a big flag as fast as possible, while always making full waves from one horizontal direction of the flag to the opposite. Fastest full waves were recorded and the wavers were ranked. Jovan was the fastest.\smallskip\\
     \gll Jovan je mah-a/nu-o za sekund.\\
J \textsc{aux} wav-\textsc{th-ptcp.ipfv\slash pfv} for second\\
\glt `Jovan managed to wave in a second (on at least\slash exactly one occasion).’\label{ars:ex:simsemc}}

	\end{xlist}
\z

\noindent Classes c) and d) above are hence excluded on different grounds, either due to not licensing atomic conceptualization, or due to specifying atoms whose temporal trace cannot be non-trivially measured.

The other question is how these verbs when combined with the \textit{za}-phrase receive the interpretation of a general-factual imperfective distributed over a plural reference time. The issue is even more striking in light of the observation that this combination cannot have a progressive interpretation (denoting the process stage of an ongoing event of, e.g., running to school in ten minutes). I argue that the same explanation holds for simple imperfectives that applies to the secondary ones, which show the same pattern of behavior. Namely, on the progressive interpretation, the sentences in \REF{ars:ex:zab} and \REF{ars:ex:zac} exemplify the imperfective paradox, as at the epistemic evaluation time it can only be verified that the event of Jovan running to school is taking place, but not how long it will take to completion, or even that it will be completed. The progressive readings are degraded exactly because the speaker cannot know the duration of an event before its completion (i.e. the speaker cannot describe an incomplete event in terms of an event kind resorting to the temporal duration of completed events). They are hence not grammatically unavailable, but rather pragmatically blocked. 

The blocking above looks like the imperfective paradox, where too the issue is that an event is described including a result, yet at the time of epistemic evaluation it is impossible to evaluate whether the result obtains. The difference is likely in the fact that the result in the relevant cases is pragmatically established as a plausible defining property of a natural class, while the duration expressed by the \textit{za}-phrase, with an infinite range of possible measures -- each standing for a different natural class, is not. As pointed by Olav Mueller Reichau (p.c.), for predicates including the \textit{za}-phrase that do match an established natural kind, such as e.g. God's creation in seven days, the progressive interpretation becomes available.

The second observation that supports the universal simple analysis of simple imperfectives concerns the status of the result, i.e. goal predicate. Recall that in \sectref{ars:sec:theoretical}, I have shown that the semantic specification of the result at the level of the $\surd\text{P}$ does not suffice to derive singularity (i.e. telicity), and moreover that without the agreement of the Q$^0$ with the result predicate, the result predicate is not bound by the speech act predicate (i.e. it is not asserted in assertions). Furthermore, it was shown that semantic effects of result agreement are preserved after reverbalization (i.e. secondary imperfectivization), in spite of the grinding effect of reverbalization, arguably due to the pragmatic competition with the respective simple imperfective. 

A similar asymmetry can be observed between simple imperfectives compatible with the \textit{za}-phrase and secondary imperfectives. Consider the examples in \REF{ars:ex:putem}.

\ea\label{ars:ex:putem}
	\begin{xlist} 
		
\ex  \gll Jovan je trč-$^u$a-o u školu \minsp{(} dužim putem).\\
J \textsc{aux} run-\textsc{th-ptcp.ipfv} in school {} longer way\\
\glt `Jovan was running to school (the longer way).’\label{ars:ex:putema}

\ex  \gll Jovan je u-trč-$^u$a-$^u$a-o u školu \minsp{(\#} dužim	putem).\\
J \textsc{aux} in-run-\textsc{th-th-ptcp.ipfv} in school {} longer way\\
\glt `Jovan was entering the school running ($\#$the longer way).’\label{ars:ex:putemb}

	\end{xlist}
\z

\noindent Without the path modifier, example \REF{ars:ex:putema} with a simple imperfective can mean the same as \REF{ars:ex:putemb}, which includes a secondary imperfective. This again at first glance supports the null prefix analysis, under the assumption of full compositionality. However, with the adverbial modifying the path, the sentence with a secondary imperfective is pragmatically ill-formed, while the one with a simple imperfective is fine. This is the case because the secondary imperfective on the progressive interpretation tends to refer to the narrow phase transition to the result state (i.e. from Jovan being outside the school to him being inside the school), and the path of this transition is conceptualized as a point in space, which cannot be modified for length (even on the slow motion\slash temporal zooming in interpretation licensing the progressive). The simple imperfective rather refers to the preparatory stage, i.e. to the motion event leading to the phase transition.\footnote{Secondary imperfectives involving goal/result- and source-oriented prefixes show the effect of shrinking to the point of phase transition. Those with path-oriented prefixes do not, as illustrated in \REF{ars:ex:noshrink}.

\ea\label{ars:ex:noshrink}

\gll Marija je uz-trč-$^u$a-$^u$a-la uz Rtanj \minsp{(} dužim	putem).\\
M \textsc{aux} up\_along-run-\textsc{th-th-ptcp.ipfv} up\_along Rtanj {} longer way\\
\glt `Marija was running up the mountain Rtanj (the longer way).’\label{ars:ex:noshrinka}

\z}

Irrespective of the analysis of aspectual morphology, the asymmetry in \REF{ars:ex:putem} argues for different syntactic structures and types of meanings for simple and secondary imperfectives. Simple imperfectives have no QP, and secondary imperfectives embed one. The latter fact restricts their denotation to sums of parts of events involving the specified result (in the given case, to parts of the event of switching from being outside to being inside the school). In light of the analysis proposed in \sectref{ars:sec:Background}, where the aspectual semantic restrictions of AS and AU verbs are largely pragmatically determined, with an important role played by the contrasts between aspectual pairs, the fact that verbs of both classes have prefixed perfective aspectual partners even more clearly implies that their compositional semantics, and hence also their structures, are different. A plausible difference suggested by their morphology is that secondary imperfectives do involve a QP and a reverbalizing secondary \textit{v}P, while simple imperfectives never do.

Evidence provided in this section thus supports the view in which simple imperfectives compatible with the \textit{za}-phrase are not telic and do not embed a telic structure. Consequently, they do not require the positing of null prefixes.

\section{Conclusion}\label{ars:sec:conc}
The starting point of the investigation was the strong generalization from \citet{Lazor.2010} that in Slavic languages telicity is necessarily marked by an affix, and that affixless verbs which show telic behavior involve null prefixes. The main question tackled by the paper was whether the strong generalization can be maintained without the introduction of null prefixes, i.e. whether the empirical data renders null prefixes necessary to maintain the hypothesis. On the material from SC, I argued that neither of the affixless verb classes showing (aspects of) telic behavior involve null prefixes. More precisely: proper simple perfectives are all idiomatized and stored in the lexicon with a non-compositional telic meaning. Affixless imperfectives compatible with the \textit{za}-phrase do not show true telic behavior, and do not embed the structure corresponding to a telic eventuality. This simplifies the model by eliminating null prefixes, while still preserving the strong generalization about affixes and telicity. I presented morphological and semantic asymmetries, as well as quantitative corpus-based evidence in support of this view.

\section*{Abbreviations}

\begin{multicols}{2}
\begin{tabbing}
\textsc{ptcp}\hspace{.5em}\= participle\kill
\textsc{$\surd{}$} \> root\\
\textsc{AS} \> aspectually singular\\
\textsc{AU} \> aspectually unspecified\\
\textsc{aux} \> auxiliary\\
\textsc{dat} \> dative\\
\textsc{gen} \> genitive\\
\textsc{inf} \> infinitive\\
\textsc{ipfv} \> imperfective\\
\textsc{pfv} \> perfective\\
\textsc{ptcp} \> participle\\
\textsc{sg} \> singular\\
\textsc{th} \> theme vowel
\end{tabbing}
\end{multicols}

\section*{Acknowledgements}
The paper owes much to the comments from and discussion with Stefan Milosavljević. I am also grateful to two anonymous reviewers and an editor of this volume, Olav Mueller Reichau, for valuable comments. These interactions made me dramatically change the arguments in the paper, as well as add a large amount of new empirical material. All errors are of course only mine. This research is part of the project \textit{Hyperspacing the verb}, funded by FWF (I4215-G30).

\printbibliography[heading=subbibliography,notkeyword=this]

\end{document}
