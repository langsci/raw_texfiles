\documentclass[output=paper]{langscibook} 
\ChapterDOI{10.5281/zenodo.10123635}
\author{Berit Gehrke\affiliation{Humboldt-Universität zu Berlin}}

\title{“True” imperfectivity in discourse}
% replace the above with your title
\abstract{By taking into account the broader discourse structure, I show that a standard imperfective (\textsc{ipfv}) semantics can also account for cases in Russian where \textsc{ipfv} forms describe actually completed events,
thereby refuting an analysis of such forms as ``fake'' \textsc{ipfv}s with a perfective (\textsc{pfv}) semantics. The proposed account captures the general intuition that the use of the \textsc{ipfv} is conditioned by a particular discourse structure, in which the event described is already part of the common ground, and the \textsc{ipfv} sentence elaborates on this event, zooming in on a narrower reference time. The proposal also has repercussion for definitions of the \textsc{pfv} and encourages us to take a closer look also at the role of \textsc{pfv} beyond the sentential level.%Abstract goes here and should not have more than 150 words.

\keywords{Russian aspect, imperfective, perfective, discourse, general-factual, presupposition}}

\lsConditionalSetupForPaper{}

\begin{document}
%%% uncomment the following line if you are a single author or all authors have the same affiliation
\SetupAffiliations{mark style=none}
\maketitle

\section{Introduction}\label{gehr:sec:intro}

Cross-linguistically, the perfective aspect (\textsc{pfv}) is assumed to involve the event time (or situation time) being included in the reference time (or topic/assertion time), while with the imperfective aspect (\textsc{ipfv}), the reference time is taken to be included in the event time (e.g. \citealt{klein95} for Russian). This results in an external (\textsc{pfv}) or internal (\textsc{ipfv}) perspective on a given event, or in \textsc{pfv} and \textsc{ipfv} predicates denoting whole or partial events (e.g. \citealt{filip99, altshuler14} for Russian). In addition, there is a common intuition that completed events involve \textsc{pfv} semantics. The notion of a ``completed event'' in this context is usually just an intuitive notion and never properly defined. Nevertheless, this intuition is commonly thought to be problematic for Russian, in which \textsc{ipfv} forms appear in descriptions of (intuitively) completed events, most famously in the so-called general-factual use. This has led \citet{gronn15} to claim that the Russian \textsc{ipfv} is a ``fake'' \textsc{ipfv} in these contexts %. In particular, he argues
and to propose that \textsc{ipfv} forms in these contexts have a \textsc{pfv} semantics, thereby giving up on the otherwise attractive idea that (here: Russian) \textsc{ipfv} forms have a uniform \textsc{ipfv} semantics. 

In this paper, I will argue that there is no ``fake'' \textsc{ipfv} in Russian but that a uniform semantics for \textsc{ipfv} forms succeeds if we take into account the discourse structure in which these forms occur. \sectref{gehr:sec:aspect} provides background information on Russian aspect, characterises general-factual uses of the \textsc{ipfv}, and discusses prominent accounts of the semantics of \textsc{ipfv} that also aim at dealing with general-factuals. In \sectref{gehr:sec:eventcompl}, I will call into question the analytical move to take the intuition of event completion at the sentence level as a basis for analysing \textsc{ipfv} forms as involving \textsc{pfv} semantics; I will show that event non-completion is neither a necessary nor sufficient condition for the use of \textsc{ipfv} forms, and moreover, that event completion is not a necessary or sufficient condition for the use of \textsc{pfv} forms, either. In \sectref{gehr:sec:discourse}, I will demonstrate how we can still work with a ``proper'' \textsc{ipfv} semantics for the given \textsc{ipfv} forms when we take into account the discourse structure in which these forms occur. \sectref{gehr:sec:concl} concludes.

\section{Grammatical aspect in Russian}
\label{gehr:sec:aspect}

This section provides background information on grammatical aspect in Russian, the canonical and non-canonical readings of the \textsc{ipfv}, in particular factual ones, and outlines recent proposals with a focus on how they deal with factual \textsc{ipfv}s. 

\subsection{Background on Russian aspect morphology}
\label{gehr:sect:backgroundaspect}

Like all Slavic languages, Russian has a grammatical category aspect. This means that a given verb form is either \textsc{ipfv} or \textsc{pfv}. Identical lexical meaning can be expressed by \textsc{ipfv} and \textsc{pfv} verb forms, and there is the common assumption that many verb(form)s come in aspectual pairs. The received view is that one type of aspectual pair is derived from simple \textsc{ipfv}s by so-called ``empty'' prefixes; see \REF{gehr:ex:appref}.

\ea\label{gehr:ex:appref} 
\ea[]{\textsc{ipfv} \textit{pit'} 	$>$ \textsc{pfv} \textit{vy-pit'} 		\hfill `to drink'}
\ex[]{\textsc{ipfv} \textit{risovat'} 	$>$ \textsc{pfv} \textit{na-risovat'} 	\hfill `to draw'}
\z	
\z

\noindent Another type of aspectual pair involves a suffix deriving an \textsc{ipfv} from a \textsc{pfv}; see \REF{gehr:ex:apsuf}.\pagebreak

\ea\label{gehr:ex:apsuf}
\ea[]{\textsc{pfv} \textit{pro-dat'} 	$>$ \textsc{ipfv} \textit{pro-da-va-t'} 	\hfill `to sell' (lit. through-give)}\label{gehr:ex:apsufa}			
\ex[]{\textsc{pfv} \textit{ot-kryt'} 	$>$ \textsc{ipfv} \textit{ot-kry-va-t'} \hfill `to dis-cover, open' (lit. from-cover)}\label{gehr:ex:apsufb}
\ex[]{\textsc{pfv} \textit{dat'} 	$>$ \textsc{ipfv} \textit{da-va-t'} 	\hfill `to give'}\label{gehr:ex:apsufc} 
\z	
\z

\noindent Given that such suffixes most often attach to already prefixed verbs (but not always, see \REF{gehr:ex:apsufc}), %such 
the derivations involved are descriptively labeled \textsc{secondary imperfectives} (\textsc{si}). There are other types of aspectual pairs, which I set aside for now, namely suppletive pairs that -- at least from a synchronic point of view -- are not morphologically transparent. I will also set aside (im)perfectiva tantum, which do not appear in aspectual pairs (arguably due to the lexical semantics of the predicates involved) \citep[see, e.g.,][]{isacenko62}, as well as biaspectual verbs, for which the aspectual semantics is determined by context \citep[see, e.g.,][]{janda07}.

We can already see from these few examples that there is no uniform morpho\-logy for \textsc{(i)pfv}s in Russian: \textsc{ipfv}s can appear without any aspectual affixes, such as those in \REF{gehr:ex:appref} (\textsc{simple} \textsc{ipfv}s) or they can appear with a suffix and often also a prefix, such as those in \REF{gehr:ex:apsuf} (\textsc{si}s); \textsc{pfv}s can contain a prefix, such as those in \REF{gehr:ex:appref}, \REF{gehr:ex:apsufa}, and \REF{gehr:ex:apsufb}, or they can lack aspectual affixes altogether, such as the one in \REF{gehr:ex:apsufc}. Nevertheless, native speakers clearly have an intuition what it means for a given verb form to be \textsc{ipfv} or \textsc{pfv}, and there are also diagnostics for \textsc{(i)pfv} forms. For example, only \textsc{ipfv} verb forms can derive a periphrastic future tense form (the future auxiliary in combination with the \textsc{ipfv} infinitive) \REF{gehr:ex:diagnosticsa}; phase verbs like \textit{begin, start, continue, stop, finish} only combine with \textsc{ipfv} infinitives \REF{gehr:ex:diagnosticsb}. 

\ea\label{gehr:ex:diagnostics}
\ea[]{\gll Ja budu \minsp{\{*} pročitat' / čitat'\} knigu.\\
I will.\textsc{1sg} {} read.\textsc{pfv} {} read.\textsc{ipfv} book.\textsc{acc}\\
\glt `I will read a/the book.'}\label{gehr:ex:diagnosticsa}			
\ex[]{\gll Ja načinaju \minsp{\{*} pročitat' / čitat'\} knigu.\\
I start.\textsc{ipfv.prs.1sg} {} read.\textsc{pfv.inf} {} read.\textsc{ipfv.inf} book.\textsc{acc}\\
\glt `I am starting to read a/the book.'}\label{gehr:ex:diagnosticsb}
\z	
\z

\noindent The diagnostics are illustrated in \REF{gehr:ex:diagnostics} only for aspectual pairs with simple \textsc{ipfv}s and prefixed \textsc{pfv}s, but what is said here extends to other aspectual pairs as well \citep[see, e.g., the discussion in][]{isacenko62, borik02}.  

\subsection{Canonical and non-canonical readings of the Russian \textsc{ipfv}}

There are two ``canonical'' readings (or two groups of readings) that Russian \textsc{ipfv} forms give rise to; these readings are canonical because such readings are commonly attested for \textsc{ipfv} forms cross-linguistically \citep[see, e.g.,][]{deo09}. The first canonical \textsc{ipfv} reading is a process/durativity reading, which for example is the reading expressed by the English Progressive, an instance of \textsc{ipfv}. This reading is illustrated for Russian in the main clause of \REF{gehr:ex:process}. 

\ea[]{\gll Kogda ja vošla, moj brat čital knigu.\\
when I in.went.\textsc{pfv} my.\textsc{nom} brother.\textsc{nom} read.\textsc{ipfv} book.\textsc{acc}\\
\glt 	`When I came in, my brother was reading a book.'}\label{gehr:ex:process}			
\z

\noindent The second canonical reading is that of iterativity/habituality, illustrated in \REF{gehr:ex:hab}.  

\ea[]{\gll Ona každyj den' otkryvaet okno.\\
she every day opens.\textsc{si} window.\textsc{acc}\\
\glt `She opens the window every day.'}\label{gehr:ex:hab}			
\z 

\noindent This is not a reading that the English Progressive expresses primarily but it is a reading that \textsc{ipfv} forms in some other languages with grammatical aspect can give rise to. In Russian, whenever an event happened more than once (or potentially more than once), that is, whenever the reference does not involve a single event, the \textsc{ipfv} has to be used.\footnote{\label{gehr:fn:vivid}A notable exception to this rule is the so-called vivid-exemplifying use of a \textsc{pfv} present tense form in habitual contexts that are clearly marked as such \citep[see][]{zaliznjaksmelev}. I will set such cases aside.} 

There are also non-canonical \textsc{ipfv} readings in Russian, i.e. readings that \textsc{ipfv} forms give rise to that are not common \textsc{ipfv} readings cross-linguistically, and outside of Slavic they might not even be attested. One family of such readings falls under the label \textsc{general-factual} \citep[\textit{obščefaktičeskoe}, after][]{maslov59}, where \textsc{ipfv} forms can appear in contexts with typical \textsc{pfv} meanings, namely when referring to bounded ``completed'' events.\footnote{However, the traditional literature also discerns subtypes of the general-factual with intuitively non-completed events; I will come back to this in \sectref{gehr:sec:OFnocompl}.} The literature on Russian aspect distinguishes at least two subtypes of the general-factual \textsc{ipfv}, the existential type \citep[][]{paduceva96, gronndiss} and what Grønn calls the presuppositional type \citep[``actional'' in][]{paduceva96}. 

The \textsc{existential ipfv} is illustrated in \REF{gehr:ex:existential} \citep[corpus example from][]{gronndiss}.

\ea[]{\gll 	Ne bylo somnenij, čto ja prežde vstrečal ee.\\
not was.\textsc{3sg.n} doubt.\textsc{gen.pl} that I before met.\textsc{si} her\\
\glt `There was no doubt that I had met her before.'}\label{gehr:ex:existential} \z 

\noindent In this example, the speaker asserts that he had a meeting with a female person in the past, and meetings in the past intuitively involve completed events that actually happened (at some time in the past). Nevertheless, we find an \textsc{ipfv} form here to describe such a meeting. More generally, the existential \textsc{ipfv} can be paraphrased as `There has been/is/etc. (at least) one event of this type.' \citep[following the idea that existential \textsc{ipfv}s involve event types or kinds; see][]{mehlig01, mehlig13, gehrkemueller}. So in this case the paraphrase would be `There was at least one event of the type ``meet her''.' 

In this paper, I will not discuss the existential \textsc{ipfv} in detail, but I assume that the reason why an \textsc{ipfv} form is used in existential contexts has to do with the fact that the event is not necessarily a single event and that we are dealing with potential iterativity \citep[labeled \textit{kratnost'} `(lit.) multiple-ness' in][]{paduceva96}. As stated at the beginning of this section, iterativity is one of the canonical readings of the Russian \textsc{ipfv}, so an account of the existential \textsc{ipfv} can build on an account for why the \textsc{ipfv} appears in iterative contexts \citep[e.g. in terms of unbounded event plurality, as in][]{ferreira05, altshuler14}. This also means that a semantic account of the \textsc{pfv} in Russian somehow has to build in a restriction to single events, rather than just the external perspective on an event.   

The \textsc{presuppositional ipfv} is illustrated in \REF{gehr:ex:presuppositional} \citep[from][]{glovinskaja82}.

\ea[]{\gll Zimnij Dvorec stroil Rastrelli.\\
winter.\textsc{adj.acc} palace.\textsc{acc} built.\textsc{ipfv} Rastrelli.\textsc{nom}\\
\glt `It was Rastrelli who built the Winter Palace.'}\label{gehr:ex:presuppositional}
\z 

\noindent The presuppositional \textsc{ipfv} (at least with telic predicates) is probably the most noteworthy mismatch between event completion and aspect usage in Russian. In our example at hand we are dealing with a single event that happened in the past, namely the building of the Winter Palace in Saint Petersburg (which hosts the Hermitage). It is a known fact that this event took place only once and that it was completed, because we can see the result in front of us. It is also known when this event happened. Nevertheless, an \textsc{ipfv} verb form is used to describe this event. 

The presuppositional \textsc{ipfv} is used when it is already clear from the context that the event in question exists (this is why \citeauthor{gronndiss} labels it presuppositional), and the sentence in which the \textsc{ipfv} form appears provides further information about this event. A suitable paraphrase is therefore `The (already mentioned or contextually retrievable) event was/is/etc. such and such.' In our example, this means that context presupposes the existence of the event `build Winter Palace', and the new information is that the architect of the building was Rastrelli. This use of the \textsc{ipfv} often goes hand in hand with a particular information structure, which is also evident in our example (and in the English translation I provided, a cleft construction): What is presupposed or backgrounded appears sentence-initially (the building of the Winter Palace) and the new information in focus is Rastrelli, in sentence-final position, resulting in a non-canonical OVS order. 

In the following, I will outline the conditions under which this use of the \textsc{ipfv} arises, building on \citet{gronndiss} \citep[who, in turn, heavily builds on empirical generalisations in the Russian literature, e.g.][]{glovinskaja82, paduceva96}.  

\subsection{Presuppositional \textsc{ipfv}s: \citet{gronndiss}}

Let us look at another example from \citet{gronndiss} to discuss empirical generalisations about presuppositional \textsc{ipfv}s, namely the chess example in \REF{gehr:ex:chess}.

\ea[]{\gll Sdelav ėtot xod {[...]}, ja \minsp{[} predložil nič'ju]\textsubscript{antecedent}. [...] Navernjaka, černye deržatsja [...], no mne ne xotelos' načinat' sčetnuju igru, \minsp{[} poėtomu]\textsubscript{F} ja i \minsp{[} predlagal nič'ju]\textsubscript{anaphora}.\\
made.\textsc{pfv.ap} this.\textsc{acc} move.\textsc{acc} {} I {} offered.\textsc{pfv} draw.\textsc{acc} {} probably blacks.\textsc{nom} hold-back.\textsc{ipfv} {} but I.\textsc{dat} not wanted.\textsc{ipfv.refl} begin.\textsc{ipfv} calculating.\textsc{acc} game.\textsc{acc} {} therefore I and/also {} offered.\textsc{si} draw.\textsc{acc}\\
\glt `Having played this move, I offered a draw. Black can probably hold on, but I didn't want to get involved in heavy calculations, and for this reason, I offered a draw.' \hfill \citep[after][207; my glosses]{gronndiss}}\label{gehr:ex:chess}
\z 

\noindent In this example, the first sentence introduces a new event in the \textsc{pfv} (\textit{predložil nič'ju} `offered a draw'). The following discourse elaborates on the reason for offering a draw, and the last part of it states that for this reason (\textit{poėtomu}) the draw was offered. This second mentioning of the event (offering a draw) is now described with an \textsc{ipfv} verb form (\textit{predlagal}, the aspectual partner of \textit{predložil}), and this is an instance of the presuppositional \textsc{ipfv}. The verb in this case is deaccentuated \citep[see also][]{paduceva96}, focus (indicated by the subscript F) is on some other constituent, in this case on \textit{poėtomu} `for this reason'. \citeauthor{gronndiss} argues that the deaccentuation of the verb leads to the event given by the verb being backgrounded and to its prior instantiation being presupposed.

Following \citet{geurtssandt97}, \citet{gronndiss} treats presuppositions as anaphora that are either directly bound in the discourse, as in \REF{gehr:ex:chess} (the antecedent for the \textsc{ipfv} \textit{predlagal} is the \textsc{pfv} \textit{predložil} in the first sentence of the example), or contextually derivable, as in \REF{gehr:ex:departure}. 

\ea[]{\gll Dlja bol'šinstva znakomyx vaš \minsp{[} ot"ezd]\textsubscript{(pseudo-)antecedent} stal polnoj  neožidannost'ju ... Vy\hspace{3.3cm} \minsp{[} uezžali]\textsubscript{anaphora} v Ameriku \minsp{[} ot čego-to, k čemu-to ili že prosto voznamerilis' spokojno provesti tam buduščuju starost']\textsubscript{F}?\\
for majority acquaintants.\textsc{gen} your.\textsc{nom} {} departure.\textsc{nom} became.\textsc{pfv} full.\textsc{instr} unexpectedness.\textsc{instr} {} you.\textsc{nom} {} away.drove.\textsc{si} in America.\textsc{acc} {} from what-\textsc{to} to what-\textsc{to} or \textsc{prt} simply decided.\textsc{pfv} calmly spend.\textsc{inf.pfv} there future.\textsc{adj.acc} old-age.\textsc{acc}\\
\glt `For most of your friends your departure to America came as a total surprise ... Did you leave for America for a particular reason or with a certain goal, or did you simply decide to spend your retirement calmly over there?'	\hfill \citep[after][207f.; my glosses]{gronndiss}}\label{gehr:ex:departure}
\z 

\noindent In this example we do not have a direct finite \textsc{pfv} antecedent to the presuppositional \textsc{ipfv} \textit{uezžali} `departed'; instead, a nominalisation based on a related verb, \textit{ot"ezd} `departure', serves as what Grønn labels pseudo-antecedent in the previous discourse. Again, the presuppositional \textsc{ipfv} verb form is deaccentuated and focus lies on the questions for the reasons for the departure. 

To illustrate \citeauthor{gronndiss}'s account of the presuppositional \textsc{ipfv} let us look at his analysis of \REF{gehr:ex:loveletter} \citep[attributed to][]{forsyth70}.

\ea[]{\gll	V ėtoj porternoj ja [...] napisal pervoe ljubovnoe pis'mo. Pisal \minsp{[} karandašom]\textsubscript{F}.\\
	in this tavern I {} wrote.\textsc{pfv} first.\textsc{acc} love.\textsc{adj.acc} letter.\textsc{acc} 			 wrote.\textsc{ipfv} {} pencil.\textsc{instr}\\
\glt `In this tavern I wrote my first love letter. I wrote it with pencil.'}\label{gehr:ex:loveletter}
\z 

\noindent \citeauthor{gronndiss}'s DRT analysis of the VP of the second sentence of \REF{gehr:ex:loveletter} is given in \REF{gehr:ex:gronnanalysis}.\footnote{DRT is the abbreviation of Discourse Representation Theory \citep[see][]{kampreyle}. \citeauthor{gronndiss} employs a linear notation for Discourse Representation Structures (DRSs), where discourse referents are written on the left-hand side, before $|$ (in a traditional DRS they appear at the top of the DRS), and the conditions on these discourse referents are listed to the right of $|$, separated by commas (which in a different notation can be translated as conjunctions).\label{gehr:fn:DRT}}

\ea[]{%$[$VP$]$: 
$\lambda e[x\,|\,\cnst{instrument}(e, x), \textsc{pencil}(x)]_{[ \hspace{0.3em} |\,\textsc{write}(e)]}$}\label{gehr:ex:gronnanalysis}
\z 

\noindent \citeauthor{gronndiss} argues that the VP is divided into background and focus \citep[following][]{krifka01}, where backgrounded material is turned into a presupposition, following the Background/Presupposition Rule in \citet{geurtssandt97}. In \citeauthor{gronndiss}'s DRT analysis, backgrounded material is subscripted in the DRS, so in this example the writing event itself is backgrounded and presupposed in the discourse. This VP gets further embedded under Aspect and Tense, which is where my proposal will differ from \citeauthor{gronndiss}'s proposal, but up to this point I will follow his account of presuppositional \textsc{ipfv}s. 

What is the semantics of the \textsc{(i)pfv} then? In the following, I will discuss various proposals in light of how they deal with existential and presuppositional \textsc{ipfv}s. 

\subsection{The semantics of Russian aspect: Some proposals}
\label{gehr:sec:prev}

As outlined in the introduction, common approaches to the semantics of Russian aspect treat it as a relation between reference/assertion time and some other temporal interval \citep[e.g.][]{klein95, schoorlemmer95, borik02, paslawskastechow, gronndiss, gronn15, ramchandlingua, tatevosov11, tatevosov15} or as an event predicate modifier, in the opposition of total vs. partial events \citep[e.g.][]{filip99,altshuler14}. The most common approach is to provide a positive definition only of the \textsc{pfv} and to treat the \textsc{ipfv} as (semantically) ``unmarked'' ($-$\textsc{pfv} or $\pm$\textsc{pfv}), but some approaches also provide a positive definition of the \textsc{ipfv}. One of the main motivations for treating the \textsc{ipfv} as unmarked is precisely the general-factual \textsc{ipfv}. Most agree that \textsc{pfv} forms always express a uniform \textsc{pfv} meaning, for example that the event time is included in the reference time. There is more disagreement with respect to the question whether \textsc{ipfv} forms come with a uniform \textsc{ipfv} meaning. Setting aside explicitly modal definitions of the \textsc{ipfv}, such as \citet{arregui+14}, who argue that different \textsc{ipfv} readings come about due to different modal bases, let me outline four representative types of proposals. 

\citet{borik02} argues that the meaning of the \textsc{ipfv} is the negation of the positive definition of the \textsc{pfv}, as illustrated in \REF{gehr:ex:borikPFIPF}.  

\ea\label{gehr:ex:borikPFIPF} 
\ea[]{$S \cap R = \emptyset\, \&\, E \subseteq R$ \hfill \textsc{pfv}}\label{gehr:ex:borikPF}
\ex[]{$\neg (S \cap R = \emptyset\, \&\, E \subseteq R)$ %\\
	%\hspace{2em} 
	$= S \cap R \neq \emptyset \vee E \not\subseteq R$\hfill \textsc{ipfv}}\label{gehr:ex:borikIPF}
\z	
\z

\noindent The \textsc{pfv} is defined as a conjunction of two conditions that have to be met \REF{gehr:ex:borikPF}: The speech time $S$ must not overlap with the reference time $R$, and the event time $E$ is included in the reference time. Negating this conjunction leads to a disjunction for the \textsc{ipfv} in \REF{gehr:ex:borikIPF}: Speech time and reference time overlap, or the event time is not included in the reference time. This disjunction captures what \citeauthor{borik02} labels the ``progressive'' reading of the \textsc{ipfv} (when the event time is not included in the reference time) as well as what she labels the ``present perfect'' reading, which is essentially the existential \textsc{ipfv} reading outlined in the previous section (speech time and reference time overlap). \citeauthor{borik02} explicitly sets habitual and iterative rea\-dings of the \textsc{ipfv} aside, but we could assume that they can be incorporated along the lines of other proposals in the literature. What is problematic for her account, though, is that it leaves the presuppositional \textsc{ipfv} unaccounted for.   

\citet{gronndiss} and \citet{altshuler14} provide weak positive definitions for the \textsc{ipfv} that get pragmatically/contextually strengthened in different directions. Buil\-ding on \citet{klein95}, \citet{gronndiss} argues that the \textsc{ipfv} involves the event time overlapping with the reference time ($e \bigcirc t$). This weak semantics gets pragmatically strengthened to a ``proper'' \textsc{ipfv} (the reference time is included in the event time), or to an actual \textsc{pfv} semantics (the event time is included in the reference time), which, he argues, happens in the case of factual \textsc{ipfv}s. \citeauthor{gronndiss} takes into account the role of information structure to characterise the contexts in which strengthening happens in one or the other direction. 

\citet{altshuler14} provides the definition of the \textsc{ipfv} in \REF{gehr:ex:altshulerIPF}, according to which the \textsc{ipfv} denotes an event $e'$ that is a stage of an event $e$ that exists in world $w$ (where the current world of $e'$ is $w*$) and that has the property $P$.\footnote{I render \citeposst{altshuler14} original formalisations, which use indirect translation. Otherwise, I use direct translations in this paper, and where not directly relevant I omit worlds and assignment functions.}

\ea\label{gehr:ex:altshulerIPF} 
\textsc{ipfv} $\leadsto \lambda P \lambda e' \exists e \exists w [\cnst{stage}(e', e, w^*, w, P)]$
\z

\noindent A stage of an event is defined as in \REF{gehr:ex:altshulerPART}, building on \citeposst{landman92} definition of the English Progressive.\footnote{Note that with respect to the condition in \REF{gehr:ex:altshulerIPFd}, \citet{altshuler14} deviates from \citet{landman92} and defines the English Progressive as a proper part relation, as he views this to be the crucial difference between Russian (part-of-relation) and English (proper-part relation). \citeauthor{landman92}, on the other hand, employed the weaker part-relation for the Progressive.}

\ea \sib{$\cnst{stage}(e', e, w^*, w, P)$}$^{M, g}$ = 1 iff (a)--(d) hold:\label{gehr:ex:altshulerPART}
\ea[]{the history of $g(w)$ is the same as the history of $g(w^*)$ up to and including $\tau(g(e'))$}\label{gehr:ex:altshulerIPFa}
\ex[]{$g(w)$ is a reasonable option for $g(e')$ in $g(w*)$}
\ex[]{\sib{$P$}$^{M, g}(e,w)$ = 1}\label{gehr:ex:altshulerIPFc}
\ex[]{$g(e') \subseteq g(e)$}\label{gehr:ex:altshulerIPFd}
\z	
\z

%\ea $\cnst{stage}(e', e, w^*, w, P) = 1$ (relative to an assignment $g$) iff (a)--(d) hold:

\noindent This is essentially an account of \textsc{ipfv} events as denoting partial events, and to capture what it means for an event to be a partial event (and notably also to capture the imperfective paradox), the definitions of stages and histories of events in \REF{gehr:ex:altshulerIPFa}--\REF{gehr:ex:altshulerIPFc} are needed. For our purposes, however, the essential part of the definition is given in \REF{gehr:ex:altshulerIPFd}, according to which the event description in question is part of or equals the whole event. \citeauthor{altshuler14} argues that this can get pragmatically strengthened to a proper part meaning for the ongoing \textsc{ipfv} ($g(e') \subset g(e)$), or it can get strengthened to $g(e') = g(e)$, which essentially says that the partial event is identical to the whole event. In particular this last type of strengthening gives rise to the presuppositional \textsc{ipfv} reading. \citeauthor{altshuler14} does not address existential \textsc{ipfv}s \citep[but see][]{altshuler12}, but again this use arguably follows from a full account of habituality and iterativity. He argues that the use of \textsc{ipfv} for habitual event descriptions is captured by assuming a theory of plural events, following \citet{ferreira05}.     

Finally, \citet{gronn15} departs from his earlier work and proposes that \textsc{ipfv} forms can express both \textsc{ipfv} (the reference time is included in the event time) and \textsc{pfv} semantics (the event time is included in the reference time), as in \REF{gehr:ex:gronn15PFIPF}.

\ea\label{gehr:ex:gronn15PFIPF} 
\ea[]{\sib{\textsc{pfv}} = $\lambda t \lambda e [e \subseteq t]$}
\ex[]{\sib{\textsc{ipfv}\textsubscript{ongoing}} = $\lambda t \lambda e%\,.\, 
[t \subseteq e]$}
\ex[]{\sib{\textsc{ipfv}\textsubscript{factual}} = $\lambda t \lambda e [e \subseteq t]$ \hspace{12em} ``Fake'' \textsc{ipfv}}\label{gehr:ex:fakeipf}
\z	
\z

\noindent \citeauthor{gronn15} calls the \textsc{ipfv} that has the same semantics as the \textsc{pfv} in \REF{gehr:ex:fakeipf} a ``fake'' \textsc{ipfv}. The existence of \textsc{ipfv}\textsubscript{factual} alongside the \textsc{pfv}, he argues, leads to an aspectual competition. In the default case the \textsc{pfv} appears but in certain contexts, he argues, the \textsc{ipfv}\textsubscript{factual} wins the competition. This gives rise to the presuppositional \textsc{ipfv} in cases where narrative progression is to be avoided (under the assumption that the \textsc{pfv} always leads to narrative progression). The existential \textsc{ipfv} appears when the reference time is too large for the perfective semantics to be informative. 

\citeposst{gronn15} account essentially gives up on the idea that the Russian \textsc{ipfv} can have a uniform semantics. \citeposst{altshuler14} account provides a weak semantics for the \textsc{ipfv}. Both delegate the role of distinguishing between different \textsc{ipfv} readings to pragmatics and to the context. In this paper, I will equally take into account the role of context, but I will explore how far we can take a strong, positive definition of the \textsc{ipfv} while still accounting for the occurrence of the presuppositional \textsc{ipfv}. In particular, I will argue that we can stick to a ``proper'' \textsc{ipfv} semantics, as opposed to a weak semantics or even a \textsc{pfv} semantics, if we take the discourse and information structural cues into account. First, however, I will show that taking the intuitive notion of event completion as a crucial indicator for the right formal account of the semantics of aspect in Russian is misleading.  

\section{The focus on event completion is misleading}
\label{gehr:sec:eventcompl}

As discussed in the previous section, the fact that intuitively completed events can be described by \textsc{ipfv} forms has led to semantic accounts of the \textsc{ipfv} that give it a rather weak semantics \citep[][]{gronndiss,altshuler14} or even argue that it can express both \textsc{pfv} and \textsc{ipfv} meanings \citep[][]{gronn15}. In this section, I will show that event non-completion is indeed neither a necessary nor a sufficient condition for an \textsc{ipfv} form to arise, just as we would expect from an account like \citeauthor{gronn15}'s, which takes the intuitive notion of event completion as its star\-ting point. We have already discussed factual \textsc{ipfv}s in the previous section, and further contexts to be addressed here involve chains of foregrounded events in habitual contexts and in the historical present, as well as the ``annulled result'' reading, which is sometimes considered a subtype of the factual \textsc{ipfv}. However, I will also show that event completion (as an intuitive notion) is neither a necessary nor a sufficient condition for a \textsc{pfv} form to arise. This is the case with \textsc{pfv} forms with the prefixes \textit{po-} and \textit{pro-}, as well as with the last event in a unique chain of foregrounded events. 

If event completion is taken as a key notion or intuition behind the definition of the \textsc{pfv}, these examples are problematic. Instead, I will argue that the intuitive notion of event completion is not useful, at least not at the sentence level, since at this level we are interested in the particular description of events and make assertions that hold during particular reference time intervals, without making any claims about the actual events being completed or not. If we compare this with the nominal domain, we can also have complete entities, for example chairs and tables, but we can also choose to describe only parts of these in a particular sentence. The intuitive notion of event completion can still be relevant at the discourse level, however, and this is precisely what I will argue for in this paper. A main conclusion from this section will be that the discourse structure plays a crucial role in the choice of aspect in Russian \citep[see also][]{altshuler12}.\largerpage[1]

I will first discuss the use of \textsc{ipfv}s with completed events, then move on to the use of \textsc{pfv}s with non-completed events. At the end of the section, I will point out that general-factual readings also arise in the absence of intuitively completed events, which shows that giving factual \textsc{ipfv}s a \textsc{pfv} semantics will not work in these cases. What all these examples aim to show is that in contexts in which the \textsc{ipfv} occurs despite the intuitition that the event is completed, other than the factual \textsc{ipfv}, there is an explanation for the use of the \textsc{ipfv} that still falls within a ``proper'' \textsc{ipfv} semantics. It is only for factual \textsc{ipfv}s that authors like \citet{gronn15} depart from such a semantics. This conclusion will serve as a point of departure for \sectref{gehr:sec:discourse}, in which I will argue that also these can be accounted for with a ``proper'' \textsc{ipfv} semantics. 

\subsection{\textsc{ipfv} with completed events}

Let us take a look at \REF{gehr:ex:bulg} \citep[discussed in][]{gehrke02,gehrke22}.

\ea[]{\gll Ona prixodila ko mne každyj den', a ždat' ee ja načinal s utra. [...]  
Za desjat' minut ja sadilsja k okoncu i načinal prislušivat'sja, ne stuknet li vetxaja kalitka.\\
she.\textsc{nom} to.went.\textsc{si} to me every day and wait.\textsc{inf.si} her.\textsc{gen} I began.\textsc{si} from morning.\textsc{gen} {}  
	within ten minutes I down.sat.\textsc{si} to window and began.\textsc{si} listen.\textsc{inf.si} not clatters.\textsc{pres.pfv} \textsc{prt} old.\textsc{nom} gate.\textsc{nom}\\
\glt 	`She came to me every day, and I started waiting for her from morning onwards. Within ten minutes [of her arrival] I sat next to the window and started listening whether the gate clatters.' \\ 
\hfill  (from Bulgakov, \textit{Master i Margarita})}\label{gehr:ex:bulg}	
\z

\noindent The whole passage in \REF{gehr:ex:bulg} is explicitly marked as habitual by \textit{každyj den'} `every day' in the first sentence. There are four foregrounded events (\textit{prixodila} `arrived', \textit{načinal ždat'} `started to wait', \textit{sadilsja} `sat down', \textit{načinal prislušivat'sja} `started to listen'), out of which at least two (the first and the third) are intuitively completed, before the other two start. Nevertheless, these verb forms are \textsc{ipfv} (\textsc{si}s) and the \textsc{pfv} would even be infelicitous in this context.\footnote{Note that other Slavic languages might be different in this respect. For example, in a Czech translation of \REF{gehr:ex:bulg}, the third form is translated with a \textsc{pfv} verb (habituality in this language does not require \textsc{ipfv}), and this might indicate that event completion does play a bigger role here. For further discussion of differences in aspect usage between Russian and Czech see \citet{gehrke02,gehrke22%, gehrke03
}; for a description of cross-Slavic differences in general, see \citet{dickey00}.} However, these \textsc{ipfv}s are generally not treated as cases of ``fake'' \textsc{ipfv} because the common explanation for the occurrence of the \textsc{ipfv} here is that habituality requires \textsc{ipfv} forms. I do not want to dispute this explanation, I just want to point out that event completion does not play a crucial role here for the choice of aspectual form.  

Similarly, event completion does not seem to play a role in passages in the historical present. The historical present is a stylistic device in narratives, and in these contexts Russian cannot use \textsc{pfv} forms (with the caveat mentioned in fn. \ref{gehr:fn:vivid}). One such example is given in \REF{gehr:ex:HP}.

\ea[]{\gll  [...] les končilsja, neskol'ko kazakov vyezžajut iz nego na poljanu, i vot, vyskakivaet prjamo k nim moj Karagez; vse kinulis' za nim s krikom [...]\\
{} forest.\textsc{nom} end.\textsc{pfv.pst} some cossacks out.ride.\textsc{si.prs} out it on field and there out.jump.\textsc{si.prs} directly to him my.\textsc{nom} Karagez.\textsc{nom} all.\textsc{nom.pl} rush.\textsc{pfv.pst} after him with shout\\
\glt  `The forest ended, a few cossacks are riding out of it into the field, and there my Karagez jumps out directly towards them. They all rushed after him with a shout.'\\\hfill \citep[from Lermontov, \textit{Geroj našego vremeni}; discussed in][25]{galton76}}\label{gehr:ex:HP} 
\z

\noindent In this example there is again a chain of completed events, in particular the riding out of the forest (\textit{vyezžajut}) and the jumping out (\textit{vyskakivaet}), as a reaction to the first event, but these are nevertheless described with \textsc{ipfv} forms. Again, nobody calls these forms ``fake'' \textsc{ipfv}s, instead an alternative explanation is provided for why the historical present is incompatible with a \textsc{pfv} semantics (e.g. that a true present tense semantics is incompatible with the event time being part of the reference time).\footnote{See \citet{anand+19} for a recent account of the historical present, which is incompatible also with the Progressive in English even in contexts where an ongoing event is described.}  

Finally, let us look at the example in \REF{gehr:ex:anres} \citep[after][311]{smith91}, which illustrates the use of the \textsc{ipfv} where the result is ``annulled''.

\ea\label{gehr:ex:anres}
%\ea[]
{\gll K vam kto-to prixodil.\\
    to you someone to.went.\textsc{si}\\
\glt `Someone came to you.' (The person is not there anymore.)}\label{gehr:ex:prixodil}		
\z

\noindent In this example there is an intuitively completed event, and the \textsc{ipfv} is used to signal that the result state of this event (someone being there) does not hold anymore at the time of utterance. While \citet{gronndiss,gronn15} subsumes cases like these under the notion of factual \textsc{ipfv}s and therefore would also treat them as ``fake'' \textsc{ipfv}s,\footnote{Treatments of such cases as a type of general-factual \textsc{ipfv} can also be found in the Slavistic traditional literature; e.g. \citet{paduceva96} calls this meaning \textit{dvunapravlennoe obščefaktičeskoe} `bi-directed general-factual', especially with motion verbs, as in \REF{gehr:ex:prixodil}.} it is again clear that the role that these \textsc{ipfv}s play in discourse is crucial and we might want to look at an alternative explanation for the use of the \textsc{ipfv} in such contexts in Russian. 

\subsection{\textsc{pfv} with non-completed events}

Let me then move on to \textsc{pfv} forms that can be used to describe non-completed events. It is well-known that in chains of foregrounded single events Russian requires \textsc{pfv} verb forms for reference time movement \citep[in the sense of][]{kampreyle} \citep[see also][]{borik02}. This is also true for the last event in the chain, even if this event is not necessarily completed, as illustrated in \REF{gehr:ex:za}.

\ea[]{\gll No v tot \v{z}e mig vspomnil svoj dom i gor'ko \minsp{\{} zaplakal / \minsp{*} plakal\}.\\
but in this \textsc{prt} moment remembered.\textsc{pfv} his.\textsc{refl.acc} house.\textsc{acc} and bitterly {} \textsc{za}.cried.\textsc{pfv} {} {} cried.\textsc{ipfv}\\
\glt `But at that moment he remembered his home and wept bitterly.'\\\hfill (grammatical version from \url{http://skazbook.ru/vodyanoi})}\label{gehr:ex:za}
\z 

\noindent In this example the crying starts right after the remembering, but the crying itself does not necessarily have to be completed. In all likelihood we are just witnessing the beginning of the crying here. While some authors try to reason that the actual event described is precisely the onset and not the crying itself and that this warrants the use of the \textsc{pfv} \citep[see, for instance,][]{ramchandlingua}, descriptions and intuitions about such ingressive events suggest that the event in focus is the crying itself, including its process, not so much its onset, and that intuitively this event is not or at least does not have to be completed. Nevertheless the \textsc{pfv} is and has to be used. Furthermore, the example in \REF{gehr:ex:zasim} (discussed in \citealt[][224]{dickey00} and attributed to \citealt{svedtrof83}) shows that several such \textsc{pfv} verbs with the ingressive prefix \textit{za-} in a row can be intepreted as ``actions beginning simultaneously''.

\ea[]{\gll	Fljagin vyšel: Čto tut načalos'! Zagudeli, zavorčali, zakričali.\\
	Fljagin.\textsc{nom} out.went.\textsc{pfv} what.\textsc{nom} then began.\textsc{pfv} \textsc{za}.hooted.\textsc{pfv.pl} \textsc{za}.grumbled.\textsc{pfv.pl} \textsc{za}.shouted.\textsc{pfv.pl}\\
\glt	`Fljagin went out. And what began then! They started hooting, grumbling and shouting.'}\label{gehr:ex:zasim}
\z

\noindent What all these examples show is that event (non-)completion is not (necessarily) decisive for the choice of \textsc{(i)pfv} in a given sentence and should therefore not play the central role in formal semantic accounts of \textsc{(i)pfv}, at least not at the sentence level. Instead we need to pay closer attention to the discourse structure and to the role that \textsc{(i)pfv} forms play in discourse. 

\subsection{General-factual \textsc{ipfv} without completed events}
\label{gehr:sec:OFnocompl}

Finally, merely treating factual \textsc{ipfv}s as ``fake'' \textsc{ipfv}s with a \textsc{pfv} semantics is missing an important insight from the Russian traditional linguistic literature \citep[e.g.][]{glovinskaja81, paduceva96}. In particular, this literature discusses different subtypes of factual \textsc{ipfv}s, including some that appear with intuitively ``incomplete'' events. For example, \citet{paduceva96} differentiates between resultative factual uses (the cases of existential \textsc{ipfv}s we have discussed so far), bi-directed factual uses (of the type in \REF{gehr:ex:anres}), as well as non-resultative (\textit{nerezul'tativnoe}) and atelic (\textit{nepredel'noe}) factual \textsc{ipfv}s.\footnote{Recall that she treats presuppositional \textsc{ipfv}s as distinct from other factual \textsc{ipfv}s, under the label \textit{akcional'noe} `actional'.} The latter two are illustrated in \REF{gehr:ex:OFincompl}.

\ea\label{gehr:ex:OFincompl}
\ea[]{\gll Ja ugovarival ee vernut'sja.\\
I convinced.\textsc{si} her return.\textsc{inf.pfv}\\
\glt `I convinced (tried to convince) her to return.' \hfill \citep[][22]{paduceva96}}\label{gehr:ex:nonresOF}
\ex[]{\gll Ja vas ljubil.\\
I you.\textsc{acc} loved.\textsc{ipfv}\\
\glt `I loved you.' \hfill \citep[][32]{paduceva96}}\label{gehr:ex:atelOF}
\z 
\z 

\noindent In the non-resultative factual \textsc{ipfv} in \REF{gehr:ex:nonresOF} it remains open whether the speaker succeeded in convincing the person referred to by `her', which could be made explicit by adding `tried to' to the translation. The atelic factual \textsc{ipfv} in \REF{gehr:ex:atelOF}, in turn, is the famous first line of a poem by Puškin, which continues with \textit{ljubov' ešče, byt' možet, v duše moej ugasla ne sovsem} `it is possible that in my soul this love is not yet completely extinguished', and this continuation makes explicit the effect of the atelic factual \textsc{ipfv}: it remains open whether the state described still holds at the moment of utterance. Both types share with the ``resultative'' factual \textsc{ipfv} (which for Padučeva involves existential \textsc{ipfv}s) that the time in the past at which these events or states held is not specific and that the relation to the current time of utterance is unclear; the first example furthermore involves potential iterativity.  

These examples are usually ignored in the formal literature, because the more extraordinary situation seems to be where a (presumably) single ``completed'' event is referred to with an \textsc{ipfv} form. However, they still constitute a different \textsc{ipfv} ``reading'' than process or habituality, and we would want to know more about these readings rather than just treating one subset of factual \textsc{ipfv}s as ``fake'', thereby ignoring these other cases that share important similarities. Calling factual \textsc{ipfv}s ``fake'' \textsc{ipfv}s and giving them the same semantics as \textsc{pfv} is missing the point. 

How can we account for the semantics of factual \textsc{ipfv}s then? %While I will leave existential \textsc{ipfv}s aside and assume that an account for the use of \textsc{ipfv} in habitual and iterative contexts and the requirement of a single event for the \textsc{pfv} will play a role here, t
The following section will provide an explicit account of presuppositional \textsc{ipfv}s that employs a standard \textsc{ipfv} semantics and takes into account information structural cues and the discourse.\footnote{As stated before, I will leave existential \textsc{ipfv}s aside and assume that an account for the use of \textsc{ipfv} in habitual and iterative contexts and the requirement of a single event for the \textsc{pfv} will play a role here; see \citet{gehrke22} for further discussion.} Event completion will be shown not to play a role at the sentence level, but at the discourse level the intuition of event completion will still be captured.

\section{A discourse semantic account of presuppositional \textsc{ipfv}s}
\label{gehr:sec:discourse}

As the previous section showed, Russian aspectual forms play a crucial role in discourse \citep[see also][]{altshuler12}, which can easily be overlooked if one simply stays at the sentential level. Following \citet{gronndiss}, I assume that presuppositional \textsc{ipfv}s are anaphorically linked to a previously introduced event in the ideal case, or that the presupposition that the event is already given in the context has to be accommodated. %Combining this with discourse semantic accounts, such as \citet{kampreyle} and \citet{lascaridesasher}, this means that a presuppositional \textsc{ipfv} introduces an eventive discourse referent that is identified with another eventive discourse referent already introduced in previous discourse, in parallel to the treatment of individual pronouns and definite descriptions in the nominal domain in, e.g., DRT. 
In particular, I propose that a presuppositional \textsc{ipfv} introduces an eventive discourse referent that is identified with another eventive\linebreak discourse referent already introduced in previous discourse. This proposal directly builds on the treatment of individual pronouns and definite descriptions in the nominal domain in discourse semantic accounts, such as \citet{kampreyle} and \citet{lascaridesasher}. In terms of discourse relations that hold between events, in the case of presuppositional \textsc{ipfv}s we are intuitively dealing with \textsc{Elaboration}. In \citeauthor{lascaridesasher}'s system of rhetorical relations between events described in two clauses $\alpha$ and $\beta$, where the former precedes the latter, Elaboration holds when $\beta$'s event is part of $\alpha$'s. So at this point \citeposst{altshuler14} partitive semantics is more promising than \citeposst{gronndiss} weak \textsc{ipfv} semantics as mere temporal overlap or even \citeposst{gronn15} \textsc{pfv} semantics. \citeauthor{altshuler14} himself suggests in his discussion of the example in \REF{gehr:ex:loveletter} \citep[(97) in][769]{altshuler14} that Elaboration is the discourse relation involved and that pragmatic strengthening of the part relation to an equal-relation leads to both events being identical. In this paper, I propose to go a step further and work with a proper part semantics from the start, thereby abandoning the need for pragmatic strengthening. Instead, I will argue that event identity follows from the information structural cues, along the lines of what was proposed in \citet{gronndiss}.

\subsection{First attempt}

As an empirical point of departure for illustrating how a proper part semantics coupled with standard discourse semantic assumptions will account for the presuppositional \textsc{ipfv}, I will use data from a corpus study with Olga Borik \citep{borikgehrkefdsl}. In this study we show that \textsc{ipfv} past passive participles (PPPs) in Russian, which are often claimed not to exist (at least from a synchronic point of view), are attested in corpora, and that they can be given a compositional semantics and are not just frozen forms. The corpus study results indicate important restrictions though: First, there are no secondary \textsc{ipfv} PPPs, and second -- more importantly for our purposes -- there are no \textsc{ipfv} PPPs with a process meaning. Our hypothesis was that \textsc{ipfv} PPPs are always factual, and we particularly focussed on presuppositional \textsc{ipfv} PPPs, like the one in \REF{gehr:ex:platy} \citep[from][]{borikgehrkefdsl}.

\ea[]{\gll	Čto kasaetjsa platy deneg, to plačeny byli naličnymi šest' tysjač rublej [...]\\
	what concerns payment money.\textsc{gen} so paid.\textsc{ipfv} were in.cash six.\textsc{nom} thousand Rubles\\
\glt	`What concerns the payment: 6000 Rubles were paid in cash.'}\label{gehr:ex:platy}
\z 

\noindent In this example, the payment event is first introduced by a nominalisation (\textit{plata} `payment'), and the \textsc{ipfv} PPP in the main clause links back to this already introduced event. The marked word order and the most natural way to read this example also indicate a marked information structure: the paying event appears in the beginning of the sentence and is backgrounded, focus lies on the sentence-final subject and (possibly also) on the modifier (`6000 Rubles (in cash)'). 

Let us work with a proper part semantics for the \textsc{ipfv} and build on independently motivated and received assumptions about discourse semantics. A first attempt, employing a linear notation of DRT (recall fn. \ref{gehr:fn:DRT}) but leaving the division into background/presupposed and focused material implicit, is in \REF{gehr:ex:try1}.

\ea[] {$[e_1, e_2, t, n, x\,|\,\textsc{payment}(e_1), \textsc{pay}(e_2), e_2 = e_1,$\\ $\cnst{theme}(e_2, x), \textsc{6,000R}(x), \textsc{in-cash}(e_2), t \subset \tau(e_2), t < n]$}\label{gehr:ex:try1}
\z 

\noindent The DRS keeps track of various discourse referents and conditions on these, as follows. \textit{Plata} `payment' is an event nominal that introduces the event discourse referent $e_1$. Since it is a non-finite (i.e. tenseless) verb form, I assume that there is no reference time and no temporal trace related to it; I will get back to this.\footnote{The temporal trace of an event is represented as $\tau(e)$, following \citet{krifka98}.} The event described by the \textsc{ipfv} PPP is represented by $e_2$, and this event description is treated like a definite description that is anaphorically linked to $e_1$ ($e_2 = e_1$), along the lines of the DRT treatment of definite descriptions in the nominal domain.\footnote{I assume that, due to the information structure involved, a prior step involves \citeposst{gronndiss} account for the VP domain, as outlined in \sectref{gehr:sec:aspect}; in this section I already take this step for granted and outline the following step in which information structural cues have already been resolved.} The new information in focus is about $e_2$, and since $e_2$ is identical to $e_1$ it is also about $e_1$: the theme of $e_2$ is `6,000 Rubles' and this was paid `in cash' (treated as an event modifier). Following \citet{kampreyle}, the semantic contribution of past tense is that it introduces  a reference time interval $t$ that is before now ($t < n$). The crucial condition now is that we analyse \textsc{ipfv} with a proper part semantics, which I treat as a temporal relation: the reference time interval $t$ is properly included in the run time of $e_2$ ($t \subset \tau(e_2)$).

If we still wanted to capture the intuition that the actual paying event was completed, at least in the overall discourse, this analysis does not succeed, because the antecedent (or pseudo-antecedent) for the factual \textsc{ipfv} is not a finite verb form but a nominalisation. In the next section, I will make a second attempt, in order to see if we can remedy this potentially intuitive shortcoming.  

\subsection{Second attempt}

If we wanted to directly capture the intuition that in the overall discourse the event referred to by the nominalisation is completed, we would have to reconstruct a \textsc{pfv} semantics for the nominalisation, along the lines of \REF{gehr:ex:try2}.  

\ea[] {$[e_1, e_2, t_1, t_2, n, x\,|\,\textsc{payment}(e_1), \textsc{pay}(e_2), \cnst{theme}(e_2, x),$\\ $\textsc{6,000R}(x), \textsc{in-cash}(e_2), e_2 = e_1, \tau(e_1) \subset t_1, t_2 \subset \tau(e_2), t_2 < n]$}\label{gehr:ex:try2}
\z 

\noindent What is new now is that we add a new discourse referent $t_1$ to the DRS, which serves as a reference time for $e_1$ (the event discourse referent introduced by the nominalisation). We furthermore reconstruct a \textsc{pfv} semantics for this nominalisation, since this would represent our intuition that the event is completed: the run time of $e_1$ is properly included in the reference time $t_1$ ($\tau(e_1) \subset t_1$). 

However, we now face new problems. Since nominalisations are non-finite, $t_1$ is not related to $n$; intuitively it is before $n$, but this would be a second reconstruction. Furthermore, without this reconstruction, we do not know how $t_1$ and $t_2$ are related (with it, it will work as in \sectref{gehr:sec:acc}). More generally, we do not know whether we want to associate nominalisations with temporal traces to begin with -- this might at most make sense for complex event nominals \citep[in the sense of][]{grimshaw90} but not necessarily for nominalisations in general. It is also not clear why we would associate nominalisations with a particular aspect semantics; intuitively we want a \textsc{pfv} semantics here because intuitively the event is completed. However, Russian nominalisations do not come in aspectual pairs, which could be taken as evidence for nominalisations lacking a functional projection associated with Aspect (AspP), as argued, for instance, by \citet{schoorlemmer95}. So why associate them with \textsc{(i)pfv} semantics at all? 

I do not think our first two attempts at a formalisation should make us want to give up on the idea that we can have an \textsc{ipfv} semantics for factual \textsc{ipfv}s in a given sentence, while still capturing the overall intuition at the discourse level that the actual event was completed. I think it rather shows that in the cases where we have to accommodate a discourse referent, as in the case with nominalisations (if we follow \citeauthor{gronndiss}'s \citeyear{gronndiss} reasoning), we will also have to accommodate more information that is otherwise contributed by tense and aspect. A full-fledged theory of accommodation would have to address this, but I will not attempt to do this in this short contribution.\footnote{Olav Mueller-Reichau (p.c.) suggests that the completedness intuition might be captured by assuming that presupposed entities are whole entities (unless there is evidence to the contrary), because they are listed as items on file cards.} Instead, in the following, I will explore what happens if the discourse does contain a \textsc{pfv} antecedent that explicitly provides the antecedent for the factual \textsc{ipfv}. 

\subsection{The account: The zooming-in function of presuppositional \textsc{ipfv}s}
\label{gehr:sec:acc}

In order to work with an example with a finite \textsc{pfv} antecedent for the presuppositional \textsc{ipfv}, I constructed an example that is not attested in the corpus, unlike \REF{gehr:ex:platy}, but which is still a fully acceptable discourse, namely \REF{gehr:ex:zaplatili}.\footnote{This is not to say that there are no such examples in the corpus, it is just that presuppositional \textsc{ipfv}s quite often require accommodation rather than true antecedents, so I wanted to address the general issue of how do deal with accommodation. An example from the corpus with a \textsc{pfv} antecent and an analysis that works just like \REF{gehr:ex:zaplatilib} is the following.

\ea 
\ea \gll I tak napisano, čto mnogie rasplakalis' -- krovju duši pisano.\\
and so written.\textsc{n.sg.pfv} that many.\textsc{nom} started.crying.\textsc{pfv} {} blood.\textsc{instr} soul.\textsc{gen} written.\textsc{n.sg.ipfv}\\
\glt `It was written so that many started to cry, it was written with the blood of the soul.'
\ex $[e_1, e_2, t_1, t_2, n, x\,|\,\textsc{write}(e_1), \tau(e_1) \subset t_1, t_1 < n, \textsc{write}(e_2),$\\ $\textsc{blood-of-soul}(x), \cnst{instrument}(e_2, x), e_2 = e_1, t_2 \subset \tau(e_2), t_2 < n]$
\z 
\z 
}

\ea\label{gehr:ex:zaplatili} 	
\ea[]{\gll	Zaplatili. Plačeny byli naličnymi šest' tysjač rublej.\\
	paid.\textsc{3pl.pfv} paid.\textsc{ipfv} were in-cash six.\textsc{nom} thousand Rubles\\
\glt	`They paid. It was paid 6,000 Rubles in cash.'}
\ex[]{	$[e_1, e_2, t_1, t_2, n, x\,|\,\textsc{pay}(e_1), \tau(e_1) \subset t_1, t_1 < n, \textsc{pay}(e_2), \cnst{theme}(e_2, x),$\\ $\textsc{6,000R}(x), \textsc{in cash}(e_2), e_2 = e_1, t_2 \subset \tau(e_2), t_2 < n]$}\label{gehr:ex:zaplatilib}
\z 
\z 

\noindent Under the analysis in \REF{gehr:ex:zaplatilib}, there is a paying event $e_1$, introduced by the \textsc{pfv} verb form in the first sentence: its run time, $\tau(e_1)$, is properly included in the reference time $t_1$ (the semantics of \textsc{pfv}), which is before n(ow) (the semantics of past tense). The analysis for the second sentence does not differ from the second attempt: The presuppositional \textsc{ipfv} PPP introduces a second paying event $e_2$, which is anaphorically linked to $e_1$, i.e. $e_2 = e_1$. The new information about this event is that its theme is 6,000 Rubles and it was paid in cash. The \textsc{ipfv} semantics specifies that there is a second reference time, $t_2$, which is properly included in the run time of the event, $\tau(e_2)$, and past tense indicates that this reference time is before the time of utterance. 

At this point, a proponent of the ``fake'' \textsc{ipfv} analysis might object and say that the \textsc{ipfv} semantics for $e_2$ in the second sentence still does not directly capture that the paying event was completed. This is indeed true, but only at the sentence level. However, it follows from the discourse structure as a whole: Event completion information is already given in the first sentence about $e_1$ (its run time falls within the first reference time $t_1$). Since $e_2$ equals $e_1$, the actual event of paying remains completed. Furthermore, the second reference time, $t_2$, is properly included in the run time of $e_2$, and therefore it is also properly included in the run time of $e_1$ (since $e_2$ is identical to $e_1$). By transitivity, $t_2$ must also be properly included in the first reference time, $t_1$. The effect of the presuppositional \textsc{ipfv}, then, is that it is used to zoom in on a narrower reference time within a bigger reference time; the link between the two reference times $t_1$ and $t_2$ is only indirect, via the events involved, but it can still be made. The assertion that the sentence with the presuppositional \textsc{ipfv} makes, then, is only for part of the bigger reference time and only for part of the actual event, and this is what is captured by the \textsc{ipfv} semantics. This is precisely what we expect if the event description provided by the presuppositional \textsc{ipfv} merely elaborates on the first event. 

There are at least two advantages of this proposal over \citeposst{gronn15} ``fake'' \textsc{ipfv} account. First, it can easily be extended to atelic and non-resultative subtypes of the presuppositional \textsc{ipfv}, which are well discussed in the descriptive literature (recall the discussion in \sectref{gehr:sec:OFnocompl}). For \citeauthor{gronn15} such subtypes would not involve ``fake'' \textsc{ipfv}s (with a \textsc{pfv} semantics) and would thus not be analysed along the same lines, even though some of these (the presuppositional ones) share the same information structural properties and anaphoric link to previously introduced events (these events are just not completed, in this intuitive sense). Second, we maintain a uniform semantics for \textsc{ipfv} verb forms. 

The gist of the proposal treats presuppositional \textsc{ipfv}s as a special case of the ongoing reading of \textsc{ipfv}s, since both involve the reference time being properly included in the run time of the event. The ongoing reading is analysed as a proper-part-relation by \citet{altshuler14} as well, but under his account both readings (presuppositional and ongoing) are arrived at only after pragmatically strengthening the weaker partitive semantics he proposes for the \textsc{ipfv}. The two readings end up with a different strengthened semantics since for him the result of pragmatic strengthening with presuppositional \textsc{ipfv}s is identity of the two events (recall the discussion in \sectref{gehr:sec:prev}). In contrast, the current proposal starts out with the stronger \textsc{ipfv} semantics, which is the same as under the ongoing rea\-ding; identity of the two events follows from the information structural cues that build an anaphoric link to the previously introduced (or accommodated) event, just like what we find with definites in the nominal domain. Thus, by taking the information structural cues already identified by \citet{gronndiss} as a point of departure to spell out a discourse semantic account that integrates independently proposed assumptions about definites and anaphoric relations in discourse, event identity is the result of the discourse structure and not of pragmatic strengthening of the \textsc{ipfv} semantics.  

\section{Conclusion}
\label{gehr:sec:concl}

In this paper I argued that an analysis of factual \textsc{ipfv}s as ``fake'' \textsc{ipfv}s, assigning them a \textsc{pfv} semantics, is misguided by the strong focus on event completion. I claimed that taking the intuitive notion of ``completed'' events as a central ingredient of the semantic definition of the \textsc{(i)pfv} aspect at the sentential level is misleading because there are numerous mismatches between \textsc{(i)pfv} forms and (in)complete events in the actual world. Rather, since we are primarily concerned with the way we describe a given event (with aspectual forms) in a given sentence and such descriptions can also involve descriptions of parts of events, the intuition of event completion could also be delegated to the level of the discourse. I argued that by taking into account the discourse structure it is possible to provide a strong \textsc{ipfv} semantics for presuppositional \textsc{ipfv}s, which therefore turn out to be ``true'' \textsc{ipfv}s: they elaborate on a part of a previously introduced event.

There are remaining issues for future research. For one, I have not addressed other subtypes of the factual \textsc{ipfv}, such as the existential \textsc{ipfv} or the annulled result cases (if these are indeed subcases). However, I am confident that a full-fledged account of habituality and iterativity, coupled with the single event requirement for \textsc{pfv}s and possibly further discourse semantic considerations, will work for existential \textsc{ipfv}s. Annulled results also point to a discourse function that needs to be explored further. A second area for further investigation arises because the proposed analysis crucially builds on there being a finite \textsc{pfv} antecedent. What do we do with non-finite antecedents (e.g. nominalisations) which -- at least in Russian -- do not come in a particular aspect? And finally, how do we handle accommodation, which is similar to bridging in the nominal domain \citep[see discussion in][]{borikgehrkefdsl}? 

\section*{Abbreviations}

\begin{tabularx}{.5\textwidth}[t]{@{}lX@{}}
\textsc{1}&first person\\
\textsc{3}&third person\\
\textsc{acc}&accusative case\\
\textsc{adj}&adjective\\
\textsc{ap}&adverbial participle\\
\textsc{dat}&dative case\\
\textsc{gen}&genitive case\\
\textsc{ipfv}&imperfective\\
\textsc{inf}&infinitive\\
\textsc{instr}&instrumental case\\
\textsc{f}&focus\\
\textsc{n}&neuter\\
\end{tabularx}%
\begin{tabularx}{.5\textwidth}[t]{@{}lX@{}}
\textsc{nom}&nominative case\\
\textsc{pfv}&perfective\\
\textsc{pl}&plural\\
\textsc{prs}&present tense\\
\textsc{prt}&particle\\
\textsc{pst}&past tense\\
\textsc{refl}&reflexive\\
\textsc{to}&specific indefinite marker \textit{-to}\\
\textsc{sg}&singular\\
\textsc{si}&secondary imperfective\\
\textsc{za}&inchoative/ingressive prefix \textit{za-}\\
&\\ % this dummy row achieves correct vertical alignment of both tables 
\end{tabularx}

\section*{Acknowledgements}
Thanks to Olav Mueller-Reichau and two anonymous reviewers for comments on an earlier version of this paper. All remaining errors are, of course, my own. 

\printbibliography[heading=subbibliography,notkeyword=this]
\end{document}
