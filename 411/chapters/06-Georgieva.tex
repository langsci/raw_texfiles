\documentclass[output=paper]{langscibook} 
\ChapterDOI{10.5281/zenodo.10123637}

\author{Ekaterina Georgieva\affiliation{HUN-REN Hungarian Research Centre for Linguistics}}
\title{Inflectionless adjectives in Bulgarian as a case of nominal predication}
\abstract{This paper deals with the so-called inflectionless adjectives in Bulgarian. Several new empirical observations are made regarding the syntactic distribution, the restrictions on definiteness, and the exclamatory flavour of the noun phrases in which these adjectives occur. The main proposal is that these lexical items are predicates of (nominal) small clauses and that the construction in question does not seem to be limited to these exceptional adjectives. It is argued that both the attributive type and the comparative type of nominal predication are attested in Bulgarian, on a par with English small clauses like \emph{an idiot doctor} and \emph{an idiot of a man}. I outline a syntactic account of these two types of nominal predication, according to which the two types correspond to different structures. I also propose that the semantic and syntactic properties of inflectionless adjectives are best accounted for if we assume that they combine with a null noun.

\keywords{inflectionless adjectives, nominal predication, small clause, null noun, definiteness, Bulgarian}
}
\lsConditionalSetupForPaper{}

\begin{document}
\SetupAffiliations{mark style=none}
\maketitle

\noindent In this paper I discuss a small class of nominal modifiers in Bulgarian, previously referred to as “inflectionless adjectives” (see \citealt{Halpern1995,SpencerLuis2012,Nicolova2017,Adamson2019PhD,Adamson2020,Adamson2022}, a.o.). Adjectives in Bulgarian inflect for gender and number, but a small group of loanword adjectives, some of which are borrowings from Turkish, do not. A non-exhaustive list is given in \REF{ge-ex-list-adj} (based on \citealt[178]{Nicolova2017} and  \citealt{Adamson2019PhD} with some additions). 

\ea  \label{ge-ex-list-adj}
\emph{serbez} `bold, insolent', \emph{ursuz} `crabby, mean', \emph{erbap} `capable, skillful, cocky', \emph{sert} `assertive, testy, strong, quick-tempered', \emph{\v{c}e\v{s}it} `weird, crank', \emph{inat} `stubborn, obstinate' (also used as a noun, with the meaning `stubbornness, obstinacy')%\footnote{\tiny This item can bear gender morphology; it also has a nouny use, with the meaning `stubbornness'.}
, \emph{kofti} `bad, shitty', \emph{kurnaz} `bold, cocky', \emph{pi\v{s}kin} `experienced, spirited', \emph{mukajat} `determined, proactive', \emph{pi\v{s}man} `fake, feigned, sham'; and also \emph{\v{s}ik} `chic', \emph{ekstra} `perfect', \emph{seksi} `sexy', \emph{super} `super', \emph{pop} `pop', \emph{d\v{z}az} `jazz' (not discussed here)
\z

\noindent It should be emphasized that the items in \REF{ge-ex-list-adj} do not likely form a unified group. In this paper, I will focus on the Turkish borrowings (with the exception of \emph{inat} `stubborn' because of its clearly noun-like use) and will show that certain semantic and syntactic properties of the noun phrases containing these items have been left unnoticed: first, the noun phrases with some of these items show restrictions on definiteness, and second, they also have a limited syntactic distribution. The inflectionless adjectives have been discussed in the literature mostly in connection with the placement of the definiteness marker, which is exceptional in comparison to what we find with inflecting adjectives \citep[see][]{Halpern1995,SpencerLuis2012,Adamson2019PhD,Adamson2020}. Although the present paper does not aim to focus on the placement of the definiteness marker, the new empirical data will refine the claims made in the literature about the use of the definiteness marker with inflectionless adjectives, in particular with respect to the interspeaker variation discussed in the earlier studies.

In this paper, I propose that the inflectionless adjective and the noun form a predication structure comparable to well-known cases of nominal predication, e.g., English \emph{an idiot of a doctor} (see \citealt{Napoli1989,Kayne1994,denDikkenLiptak1997,HulkTellier2000,DoetjesRooryck2003,Casillas2003,denDikken2006,VillalbaBartra-Kaufman2010}, a.o.). These noun phrases have been referred to as \textsc{the qualitative binominal noun phrase} (QBNP) in \citet{denDikken2006} and \citet{VillalbaBartra-Kaufman2010}; as ``qualitative construction'' in \citet{DoetjesRooryck2003}, and as ``\emph{N1/A de N2} affective construction'' in \citet{Casillas2003}. Throughout the paper I use the terms nominal predication, nominal small clause and QBNP interchangeably. 

I will point out similarities between the noun phrases with inflectionless adjectives such as \REF{ge-ex-excl1-a} and the corresponding nominal predication constructions in the languages discussed in the above-mentioned sources. Even more importantly, I will show that nominal predication in Bulgarian is not limited to the closed class of borrowed lexical items listed in \REF{ge-ex-list-adj}, as (non-borrowed) nouns can also be used as the first part of the nominal predication, as shown in  \REF{ge-ex-stain1}.\footnote{Possibly, the rather poorly-understood class of compounds with comparative semantics (e.g., \emph{gaitan ve\v{z}di} `woollen.braid eyebrow.\Pl{}, well-shaped eyebrows') and the so-called `appositive compounds' also belong here \citep[see][]{Bagasheva2017}. But since most of these examples are quite archaic, I leave them out from the present discussion and focus on more productive patterns like the one illustrated in \REF{ge-ex-stain1}.} 

\ea \label{ge-ex-excl1-a}
\gll Eh, kakva ursuz \v{z}ena!\\
\Prt{} what.kind crabby woman\\
\glt `What a crabby woman!'
\z

\ea \label{ge-ex-stain1}
\ea
\gll [\ldots{}] e leke \v{c}ovek [\ldots{}] \\ 
{} be.\Prs{}.\Tsg{} stain person {}\\
\glt `[X] is a rotter of a man (lit.\ a stain person) [\ldots{}]' \hfill [Google search] 
\ex
\gll v\u{a}j sega tuj leke \v{c}ovek na Bolen\\
\Prt{} now this stain person of Bolen\\
\glt `wow, (and) now this rotter of a man of Bolen's!' \hfill [Google search] 
\z 
\z

\noindent What is common between these examples is that semantically they express an (often negative) evaluation of the referent of the NP. I will also point out certain structural similarities between them, such as the restrictions on definiteness as well as their exclamative flavour. Thus, the phenomenon under consideration cannot possibly be explained with the exceptional properties of the inflectionless adjectives. 

The paper is organised as follows: in \sectref{ge-sec-prev}, I first summarize the previous claims made in the literature regarding inflectionless adjectives. In \sectref{ge-sec-data}, I present novel observations regarding the noun phrases containing inflectionless adjectives. Then in \sectref{ge-sec-theorbackgr}, I present an overview of the theoretical analyses of nominal small clauses, based on which I outline a possible analysis of the Bulgarian data in \sectref{ge-sec-analysis}. In \sectref{ge-sec-concl}, I conclude and raise some further questions for future research. 



%--------------section--------------	
\section{Previous approaches to inflectionless adjectives}\label{ge-sec-prev}

As already mentioned in the introduction, inflectionless adjectives have been discussed in the literature mostly in connection with the placement of the definiteness marker in Bulgarian. %Generally, there are two strands of analysis of this marker: some treat it as a syntactic process \citep{Franks2001}, while others propose a postsyntactic account (in various implementations) \citep[see][]{EmbickNoyer2001,DostGribanova2006,Harizanov2014PhD,Adamson2019PhD}. 
As far as empirical data are concerned, the baseline for the placement of the definiteness marker (\textsc{def}) is that it attaches to the noun 
\REF{ge-ex-n-def}, but if the noun is preceded by adjectival modifiers, then it is placed on the (first) adjective \REF{ge-ex-adj-def} \citep[for more details see][]{Halpern1995,Franks2001,EmbickNoyer2001,DostGribanova2006,Harizanov2014PhD,Harizanov2018,HarizanovGribanova2015,Adamson2019PhD}.

\ea 
\ea \label{ge-ex-n-def} 
\gll kniga-ta\\
book.\F{}-\Def{}\\
\glt `the book'
\ex \label{ge-ex-adj-def} 
\gll nova-ta kniga\\
new.\F{}-\Def{} book.\F{}\\
\glt `the new book'
\z
\z
	
\noindent The adjectives in \REF{ge-ex-list-adj} are exceptional with respect to the placement of \Def{}: the definiteness marker cannot attach to them \REF{ge-ex-skipa}, unlike what is observed with regular adjectives \REF{ge-ex-adj-def}. Instead, it skips the adjective and attaches to the noun as in \REF{ge-ex-skipb} (\citealt{Adamson2019PhD} refers to this as `skipping'). Importantly, the skipping variant in \REF{ge-ex-skipb} is grammatical only for some speakers; for others, \Def{} cannot be used with inflectionless adjectives at all \citep[see][]{SpencerLuis2012}. 

\ea\judgewidth{\%}
\ea[*]{\gll sert-\u{a}t m\u{a}\v{z} \\
assertive-\Def{} man.\M{} \\
\glt `the assertive man' \label{ge-ex-skipa}}
\ex[\%]{\gll sert m\u{a}\v{z}-\u{a}t \\
assertive man.\M{}-\Def{} \\
\glt `the assertive man' \label{ge-ex-skipb}}
\z 
\z 

\noindent Thus, inflectionless adjectives have been brought into the discussion of \Def{} as they raise two questions: (i) why \Def{} cannot attach to them, but appears on the noun instead \citep{Adamson2019PhD,Adamson2020}; (ii) why \Def{} cannot be used at all (for some speakers) \citep[see][]{Halpern1995}. These issues will be addressed from a new perspective in \sectref{ge-sec-data}.

\citet[165, fn.~22]{Halpern1995} proposes that inflectionless adjectives form neo\-lo\-gistic compounds with the noun. Under a compound analysis, the placement of the definiteness marker on the noun, i.e., the head of the compound, is not surprising. On the other hand, \citet{Adamson2019PhD} argues these are not compounds since the adjectives can be intensified \REF{ge-ex-intens}, can stand in the comparative form \REF{ge-ex-cmpr}, and need not be adjacent to the noun \REF{ge-ex-adjac}.%\footnote{Example \REF{ge-ex-adjac} is very degraded to me; I discuss the possibility of other adjectival modifiers interleaving the inflectionless adjectives and the noun in Section~\textcolor{blue}{XXXXXX}.}. 
	
\ea[]{\label{ge-ex-intens}%
\gll mnogo serbez dete\\
very bold child.\N{}\\
\glt `a very bold child'}
\ex[]{\label{ge-ex-cmpr}%
\gll po-serbez\\
\Cmpr{}-bold\\ %\Cmpr{}
\glt `bolder'}
\ex[?]{\gll erbap b\u{a}lgarsko dete\\
stubborn Bulgarian.\N{} child.\N{}\\ 
\glt `a stubborn [sic: capable] Bulgarian child' \hfill \citep[94]{Adamson2019PhD} \label{ge-ex-adjac}} % ex.~(82)
\z 

%But compounds in Bulgarian are stressed .... \citep{Nicolova2017}, e.g. xxx, \emph{sert m\u{a}\v{z}-\u{a}t }

\noindent All these diagnostics are taken by \citet[94]{Adamson2019PhD} to indicate that we are dealing with adjectives (or adjective-like modifiers) and, in his view, to falsify the neologistic compound analysis proposed by \citet{Halpern1995}. \citet{Adamson2019PhD} proposes that \textsc{def} moves postsyntactically to the head of the closest phrase that bears nominal features. Adjectives undergo node-sprouting, as a result of which $a$Infl elaborates the adjective (the M(orphological)W(or)d $a$, to be precise). As this operation precedes the (postsyntactic) Lowering of D \citep[see][]{EmbickNoyer2001}, the definiteness marker ends up on the adjective, as in \REF{ge-ex-adj-def}. In order to account for the inflectionless adjectives, \citet[96]{Adamson2019PhD} proposes that the adjectival heads combining with certain loanword roots bear the diacritic feature [$\alpha$] and the node-sprouting rule gives no results in the presence of this feature. This is how these $a$Ps are rendered inflectionless. Since the definiteness marker is sensitive to the nominal features present, two possible scenarios arise in the case of the inflectionless adjectives: (i) \textsc{def} attempts to attach to the inflectionless adjective and the derivation crashes, (ii) \textsc{def} skips the adjective and attaches to the noun instead. The two scenarios are meant to capture the interspeaker variation (recall that \REF{ge-ex-skipb} is acceptable only for some speakers, according to Adamson's data). Without going into further detail, I would like to note that this analysis refers to a list of vocabulary items, i.e., it relies on the properties of inflectionless adjectives as specified in the lexicon. As already pointed out in the introduction, the construction in question also occurs with non-loan nouns (cf.\ \REF{ge-ex-stain1}), this phenomenon cannot possibly be fully derived from the exceptional features of loanwords. 

Furthermore, although I agree with \citet{Adamson2019PhD} in his criticism of \citet{Halpern1995}, there are some remarks to be made here. First, example \REF{ge-ex-intens} indeed proves that we are not dealing with compounds, as the possibility of using adverbials like \emph{mnogo} `very, a lot' suggests that a degree phrase (DegP) is present, and more generally, that we are dealing with phrasal modifiers. However, \emph{mnogo} `very, a lot' also appears with verb phrases in Bulgarian. Additionally, the comparative clitic in Bulgarian can attach to nouns and verb phrases (e.g., \emph{p\`{o} m\u{a}\v{z}} `\Cmpr{} man, more of a man/a real man' and \emph{p\`{o} obi\v{c}am} `\Cmpr{} love.\Prs{}.\Fsg{}, I love more/prefer'). Thus, the data in \REF{ge-ex-intens} and \REF{ge-ex-cmpr} do not convincingly prove the adjectival status of these modifiers. As for example \REF{ge-ex-adjac}, the judgments indicate degraded acceptability. I discuss the possibility of other adjectival modifiers interleaved between the inflectionless adjective and the noun in \sectref{ge-sec-analysis}.

Putting the issues about the placement of \textsc{def} aside, I will show that inflectionless adjectives differ from genuine adjectives both semantically and syntactically with respect to definiteness and syntactic distribution. %Importantly, the account of these noun phrases proposed here is not limited to these morphologically exceptional lexical items. 
In the next section, I will present novel empirical data regarding these loans and the noun phrases they appear in.



%--------------section--------------
\section{New empirical data}\label{ge-sec-data}

In this section, I will discuss new data on inflectionless adjectives on the basis of which the following generalisations emerge: (i) 
the noun phrases with inflectionless adjectives split into two groups with respect to definiteness: some of them are compatible with a definite reading, while others are not, %most noun phrases containing inflectionless adjectives show restrictions on definiteness, 
(ii) these noun phrases show a limited syntactic distribution, and (iii) they have a strong exclamative flavour. The data presented here were tested with three native speakers, including myself; additionally, corpus examples from the Bulgarian National Corpus [BulNC] are also included.\footnote{The corpus contains 1.2 billion words and is available online at: \url{http://search.dcl.bas.bg/}. The searches were carried out in May 2021.}
%\href{http://search.dcl.bas.bg/}{Bulgarian National Corpus [BulNC]}.

Let us begin with the use of the definiteness marker. As said in the previous section, this suffix cannot attach to the adjective (for all speakers), but according to the literature, for some speakers, it can attach to the noun as in \REF{ge-ex-skipdef} (``skipping''). 

\judgewidth{\%}
\ea [\%]{
\gll Erbap \v{z}ena-ta \minsp{(} se obadi). \\
skillful woman.\F{}-\Def{} {} \Refl{} call.\Pst{}.\Tsg{} \\
\glt `The skillful woman called.' \hfill \citep{Adamson2020}
\label{ge-ex-skipdef}}
\z 

\noindent Firstly, I will argue below that inflectionless adjectives fall into two groups: with some of them, `skipping' is perfectly fine, while with others it is not. It will be shown that \emph{erbap} `capable, skillful, cocky' belongs to the latter group. Secondly, I will also demonstrate that the acceptability of the definiteness marker in this group of inflectionless adjectives depends on the definiteness of the noun phrase. This sheds new light on the interspeaker variation reported in the previous literature. Before I proceed with the investigation of the restrictions on definiteness in these noun phrases, let me make an important methodological remark. Most of the examples in literature \citep{Adamson2019PhD,Adamson2020,SpencerLuis2012,Halpern1995} are not full sentences, but simply Adj+N combinations and the definiteness of the noun phrases is not controlled for. This might have been the reason why certain speakers have accepted the examples, perhaps having in mind one particular reading, while others have rejected them, and this might have given the false appearance of interspeaker variation being present. In order to control for definiteness, I tested the noun phrases with inflectionless adjectives in full sentences. An additional problem is the exclamative flavour of the noun phrases with inflectionless adjectives (see below), which can be also controlled for by using full sentences. 

In order to investigate the use of the definiteness marker and the definiteness of these noun phrases, I collected corpus data from BulNC. Based on these data, it can be shown that noun phrases with inflectionless adjectives may contain a zero article, an indefinite article (`one') or a demonstrative. As for the use of the definiteness marker, it is not attested in the corpus with the inflectionless adjectives listed in \REF{ge-ex-list1}, but it is attested with the ones given in \REF{ge-ex-list2}.

\ea
\ea \label{ge-ex-list1} \emph{serbez} `bold, insolent', \emph{ursuz} `crabby, mean', \emph{erbap} `capable, skillful, cocky', \emph{sert} `assertive, testy, strong, quick-tempered', \emph{\v{c}e\v{s}it} `weird, crank', \emph{inat} `stubborn, obstinate', \emph{pi\v{s}kin} `experienced, spirited'\footnote{There are no hits for \emph{mukajat} `determined, proactive' in the corpus, with any determiner, but based on my native speaker intuitions it belongs to the group in \REF{ge-ex-list1}.} %\hfill [definite \ding{55}]
\ex \label{ge-ex-list2} \emph{kofti} `bad, shitty', \emph{pi\v{s}man} `fake, feigned, sham' %\hfill [definite \ding{51}]
\z
\z

\noindent Let us take a closer look at the items in \REF{ge-ex-list1} based on native-speaker intuitions. I tested what readings these inflectionless adjectives allow for with various types of determiners in order to obtain a more fine-grained picture of the types of definiteness possible with them and to verify whether the fact that they are not attested with the definiteness marker in the corpus is merely accidental. Below I summarize the judgments.\largerpage

First, noun phrases with the inflectionless adjectives in \REF{ge-ex-list1} can have an indefinite article (`one'), yielding an indefinite non-specific reading \REF{ge-ex-indef}. An indefinite specific reading as in \REF{ge-ex-indef2} is also possible, but it requires a proper context like the relative clause given in parenthesis in the translation line in order to facilitate the specific reading of the noun phrase.\footnote{Compare \REF{ge-ex-indef2} with the minimally different \REF{ge-ex-indef3}, which, in an out-of-the-blue context, can be used only humorously, i.e., as an epithet. Note that \REF{ge-ex-indef3} is also different from \REF{ge-ex-indef4}, which shows that noun phrases without inflectionless adjectives can readily have a specific reading without requiring much contextualization.

\ea\judgewidth{?/\#}
\ea[?/\#]{ 
\gll T\u{a}rsja edna ursuz \v{z}ena.\\
look.for.\Prs{}.\Fsg{} one.\F{} crabby woman.\F{} \\
\glt `I am looking for a crabby woman.' 
\label{ge-ex-indef3}
}
\ex[]{
\gll T\u{a}rsja edna \v{z}ena.\\
look.for.\Prs{}.\Fsg{} one.\F{} woman.\F{} \\
\glt `I am looking for one (specific) woman.' \label{ge-ex-indef4}
}
\z
\z

}
\judgewidth{(?)}
\ea[]{\label{ge-ex-indef}
\gll Obadi mi se edna ursuz \v{z}ena.\\
call.\Pst{}.\Tsg{} to.me \Refl{} one.\F{} crabby woman.\F{} \\
\glt `A crabby woman called me.' \hfill [\emph{one}; indefinite non-specific \ding{51}]}
\ex[(?)]{ 
\gll Hodih da t\u{a}rsja edna ursuz \v{z}ena.\\
go.\Pst{}.\Fsg{} \Cmpr{} look.for.\Prs{}.\Fsg{} one.\F{} crabby woman.\F{} \\
\glt `I went looking for a crabby woman (e.g., who had called me the day before).' \hfill [\emph{one}; indefinite specific \ding{51}] 
\label{ge-ex-indef2}
}
\z

\noindent Turning to the definiteness marker, we see that its use is highly degraded when the noun phrase has a definite reading as in \REF{ge-ex-def}, but it is acceptable if the noun phrase is interpreted generically as in \REF{ge-ex-gen}.\footnote{\REF{ge-ex-def} was modelled after \REF{ge-ex-skipdef}, which is very degraded to the speakers I have consulted.}$^,$\footnote{The \textsc{def} marker in Bulgarian can be used with generic noun phrases, both in the singular and in the plural (see \citealt[165]{Nicolova2017}).}

\judgewidth{??}
\ea[??]{
\gll Ursuz \v{z}ena-ta pak mi se obadi. \\
crabby woman.\F{}-\Def{} again to.me \Refl{} call.\Pst{}.\Tsg{} \\
\glt `The crabby woman called me again.'\footnote{This sentence was tested in the following context: `Yesterday I talked to Maria about the project, and she really annoyed me. Today...'} \hfill [\Def{}, definite \ding{55}] \label{ge-ex-def}
}
\ex[]{ \label{ge-ex-gen}
\gll Ursuz \v{z}ena-ta se poznava po pogled-a. \\
crabby woman.\F{}-\Def{} \Refl{} recognize.\Prs{}.\Tsg{} by gaze.\M{}-\Def{} \\
\glt `You can recognize a crabby woman by her gaze.' \hfill [\Def{}, generic \ding{51}]}
\z

\noindent The contrast between \REF{ge-ex-def} and \REF{ge-ex-gen} is important because \citet{Halpern1995}, \citet{SpencerLuis2012}, and \citet{Adamson2019PhD} have claimed that \textsc{def} is either ungrammatical altogether or that it is grammatical but only for some speakers. But what we observe is that first, there is a contrast between the items in \REF{ge-ex-list1} and \REF{ge-ex-list2} and second, as far as the group in \REF{ge-ex-list1} is concerned, the grammaticality of \textsc{def} is not a real case of interspeaker variation, as examples \REF{ge-ex-def} and \REF{ge-ex-gen} are given different judgements by the same speakers. Rather, the acceptability depends on the type of definiteness of the noun phrase. 

\citet[129]{SpencerLuis2012} argue that the impossibility of using \textsc{def} with inflectionless adjectives cannot be explained with restrictions on definiteness since demonstratives are licit: 

\ea \label{ge-ex-dem}
\gll Tazi ursuz \v{z}ena pak mi se obadi. \\
this.\F{} crabby woman.\F{} again to.me \Refl{} call.\Pst{}.\Tsg{} \\
\glt `This crabby woman called me again.' \hfill [\Dem{} \ding{51}]
\z 

\noindent Indeed, such examples are fully acceptable, even for the speakers who reject \REF{ge-ex-def}, and examples with demonstratives are also attested in the corpus. However, one remark has to be made. It is well-known that demonstratives differ from definite articles in both deictic and anaphoric contexts (\citealt{Lyons1999,Wolter2006PhD}; a.o.). Additionally, demonstratives may have several discourse\slash pragmatic functions. For instance, they may have an indefinite specific reading as in \REF{ge-ex-thisindef}; according to \citet{Ionin2006}, in this case specificity is to be explained with noteworthiness. The indefinite use is often subsumed under a broader category, namely, the so-called emotive use of demonstratives, as illustrated in \REF{ge-ex-emotive} and \REF{ge-ex-emotive2} \citep[see][]{Lakoff1974,Wolter2006PhD,PottsSchwarz2010}.

%a discourse, pragmatic function, this is the so-called `emotive/affective' use of demonstratives \citep[see][]{Lakoff1974}, illustrated in \REF{ge-ex-emotive}. This use of demonstrative has three subtypes: (i) the speaker alludes to something, already mentioned, but not part of the discourse proper; (ii) replacing indefinites; (iii) expressing  ``solidarity''.

\ea 
\ea \label{ge-ex-thisindef}
Mary wants to see \emph{this} new movie; I don't know which movie it is, but she's been all excited about seeing it for weeks now. \hfill \citep{Ionin2006}
\ex \label{ge-ex-emotive} 
\emph{that} mother of John \hfill \citep{Lakoff1974}
\ex \label{ge-ex-emotive2} 
How's \emph{that} throat? \hfill \citep{Lakoff1974}
\z
\z

\noindent Importantly, the demonstrative in \REF{ge-ex-dem} cannot be interpreted deictically, i.e., the sentence cannot be uttered felicitously when pointing at someone. According to my intuitions, the referent is interpreted as specific and it must be salient in the discourse (at least on part of the speaker), but it does not need to be unique as with definites. Thus, \REF{ge-ex-dem} can also be uttered felicitously if the speaker has several crabby women in mind, but wishes to mention only one of them. 

So far I have demonstrated that the grammaticality of the inflectionless adjectives given in \REF{ge-ex-list1} depends on definiteness. These were also unattested with \Def{} in the corpus. Based on the corpus data, however, we saw that there are two `outliers', namely, \emph{kofti} `bad, shitty' and \emph{pi\v{s}man} `fake, feigned, sham' in \REF{ge-ex-list2}: with these lexical items, the skipping examples are perfectly fine, even with a definite reading. Two corpus examples are given below: \REF{ge-ex-kofti1} is most likely to be interpreted generically (as it combines with a mass noun), but \REF{ge-ex-kofti2} clearly has a definite reading: the NP has a unique referent, previously mentioned in the discourse. According to my native speaker intuitions, these items do not posit the restrictions on definiteness we observed for the ones in \REF{ge-ex-list1}.

\ea \label{ge-ex-kofti1}
\gll [\ldots{}] az se naso\v{c}ih k\u{a}m \v{s}tand-a s kofti hrana-ta [\ldots{}] \\
{} I \Refl{} direct.\Pst{}.\Fsg{} to stall.\M{}-\Def{} with bad food.\F{}-\Def{} {}\\
\glt `I headed towards the junk food section.' \hfill [BulNC]
\z 

\ea \label{ge-ex-kofti2}
\gll Kofti kopele-to, radist-\u{a}t, izpratil s\u{a}ob\v{s}tenie-to na anglijski, be\v{s}e povi\v{s}en v star\v{s}ina. \\
bad bastard.\N{}-\Def{} radio.operator.\M{}-\Def{} send.\Ptcp{} message.\N{}-\Def{} on English be.\Pst{}.\Tsg{} promote.\Ptcp{} in sergeant.major\\
\glt `That idiot bastard, the radio operator, who (had) sent the message in English, was promoted to sergeant major.' \hfill [BulNC]
\z

% Кофти копелето, радистът, изпратил съобщението на английски, беше повишен в старшина  \hfill [BulNC]

\noindent The second important observation is that
the noun phrases with inflectionless adjectives have a limited syntactic distribution.\footnote{For the inflectionless adjectives in \REF{ge-ex-list1} it comes as no surprise that they have a fairly limited distribution in argument position, as they are incompatible with a definite reading.} Based on corpus data, it seems that these noun phrases tend to occur in the following syntactic environments: (i) as predicates of copular clauses \REF{ge-ex-pred}, (ii) in exclamations \REF{ge-ex-excl0}, (iii) with predicates like \emph{d\u{a}r\v{z}i se} `behave, act (like)', \emph{izgle\v{z}da} `look like, seem', \emph{izliza} `turn out (to be)', \emph{okazva se} `turn out (to be)', \emph{minava (za)} `be considered (as)', \emph{ostava si} `remain, to continue to be' \REF{ge-ex-secpred0}. These predicates normally select for a predicative complement (a small clause), and more generally, the syntactic environments in (i)--(iii) are similar to each other, as they all express a predication relation, and thus can be subsumed under one more general type, namely, predication.

\ea \label{ge-ex-pred}
\gll  Marija e mnogo ursuz / kofti \v{c}ovek.\\
Maria be.\Prs{}.\Tsg{} very crabby {} bad person.\M{}\\
\glt `Maria is a very crabby / bad person.'
\z

\ea \label{ge-ex-excl0}
\ea \label{ge-ex-excl1}
\gll Eh, kak\u{a}v ursuz / kofti \v{c}ovek!\\
\Prt{} what.kind.\M{} crabby {} bad person.\M{} \\
\glt `What a crabby / bad person!'
\ex \label{ge-ex-ecl2}
\gll ursuz / kofti \v{c}ovek\\
crabby {} bad person.\M{}\\
\glt `(a) crabby / bad person' \emph{or}\\
`What a crabby / bad person!'
\z
\z

\ea \label{ge-ex-secpred0}
\ea \label{ge-ex-secpred1}
\gll Izleze erbap \v{z}ena tja.\\
turn.out.\Pst{}.\Tsg{} capable woman.\F{} she\\
\glt `She turned out to be a capable woman.' \hfill [BulNC]
\ex \label{ge-ex-secpred2}
\gll Tja izleze kofti \v{c}ovek.\\
she turn.out.\Pst{}.\Tsg{} bad person.\M{} \\
\glt `She turned out to be a bad person.' 
\z 
\z

%Такива, додето са под влиянието на наркотика, се държат много серт
%серт човек изглежда
%излезе ербап жена тя
%но той се оказа инат човек
%уж минавах за ербап момиче
%остана си курназ


\noindent In \sectref{ge-sec-analysis}, I will argue that inflectionless adjectives stand in a predication relation with the noun, with the two groups of them, \REF{ge-ex-list1} an \REF{ge-ex-list2}, exemplifying two different predication structures within the noun phrase, the comparative and the attibutive one, respectively. I will argue that inflectionless adjectives actually combine with a null noun and that this noun phrase functions as the predicate of the nominal small clause. This analysis is supported by the fact that these nominal small clauses are attested not only with the exceptional items traditionally referred to as inflectionless adjectives, but also with two noun phrases containing non-loan lexical items as in \REF{ge-ex-def-stain} and \REF{ge-ex-def-gen-stain}. Observe also that these two examples show similar restrictions regarding definiteness: the definiteness marker is highly degraded when used with a definite reading, but it is acceptable with a generic reading, as in \REF{ge-ex-def-gen-stain} (compare with \REF{ge-ex-def} and \REF{ge-ex-gen}, respectively). These constructions also have an exclamative flavour. 

\ea[??]{
\gll Leke \v{c}ovek-\u{a}t pak post\u{a}pi u\v{z}asno.\\
stain person.\M{}-\Def{} again behave.\Pst{}.\Tsg{} awfully\\
\glt `That scoundrel/rotter of a man behaved awfully again.' \hfill [definite \ding{55}] \label{ge-ex-def-stain}
}
\ex[]{ \label{ge-ex-def-gen-stain}
\gll Leke \v{c}ovek-\u{a}t se poznava po post\u{a}pk-i-te.\\
stain person.\M{}-\Def{} \Refl{} recognize.\Prs{}.\Tsg{} by deed-\Pl{}-\Def{}.\Pl{}\\
\glt `You can recognize a scoundrel/rotter of a man by his deeds.'\\\hfill\hbox{[generic \ding{51}]}}
\z 

\noindent Let us recap the main empirical points presented in this section. First, it was shown that inflectionless adjectives fall into two groups: some of them show restrictions on definiteness, as they are compatible only with the generic use of \Def{}, but not with the definite one. The other group of inflectionless adjectives do not show such restrictions. These facts further qualified the claims about the use of \Def{} with these lexical items made in the existing literature. 
Second, the noun phrases with inflectionless adjectives also show a limited syntactic distribution, being mostly used in predicative contexts. Thirdly, these noun phrases have a strong exclamative flavour. Finally, it was shown that nominal small clauses are also possible with non-loan items; moreover, those show a parallel behaviour with respect to the use of the definiteness marker.



%--------------section--------------
\section{Background on nominal predication}\label{ge-sec-theorbackgr}

In a nutshell, my proposal regarding inflectionless adjectives will be that they stand in a predicational relationship with the noun. I argue that these constructions are comparable to well-studied cases of nominal predication (see \citealt{Napoli1989,HulkTellier2000,DoetjesRooryck2003,Casillas2003,VillalbaBartra-Kaufman2010} among others on Romance languages, \citealt{denDikken2006} on English and Dutch, \citealt{denDikkenLiptak1997} on Hungarian). Below, I will first provide a summary of the main types of nominal predication and their properties, focusing mostly on English and Spanish, based on the existing literature. I will also summarize the main analytical solutions proposed. %Then, I will outline an analysis of nominal predication in Bulgarian.


%\subsection{Background on nominal predication}

%In this section, I will summarize the main semantic and syntactic properties of nominal predication and its subtypes, and the possible analyses of these constructions, largely following \Citet{denDikken2006} with some additional remarks. 
\Citet{denDikken2006} argues that predication structures in the noun phrase come in two guises: \textsc{attributive nominal predication} and \textsc{comparative nominal predication}, as illustrated below:

\ea 
\ea 
an idiot doctor, an idiot of a doctor \hfill [attributive]
\ex a jewel of a village, an idiot of a man \hfill [comparative]
\z
\z

\noindent \Citet[161]{denDikken2006} points out that the two types of nominal predication are not simply semantic variants to each other, as evidenced by the structural differences between the two types in Italian (examples from \citealt{Napoli1989}). In the attributive type, which has the meaning that the referent of the complex noun phrase is an ignoramus in his capacity as a doctor, the second noun is bare \REF{ge-ex-it1}. In the comparative type, on the other hand, the second noun bears a definite determiner \REF{ge-ex-it2}. The meaning of the latter type is that the referent of the complex noun phrase is ignorant as an individual (and just happens to be a doctor by profession). In a similar vein, although the English examples like \emph{an idiot of a doctor} are ambiguous between the two readings, it can be shown that the two noun phrases participating in the nominal predication are obligatorily connected by \emph{of~a} in the comparative nominal predication but not in the attributive one, as evidenced by \REF{ge-ex-engcopdrop} \citep[164]{denDikken2006}.

\ea
\ea \label{ge-ex-it1}
\gll quell’ ignorante di dottore  \\
that ignoramus of doctor \\
\glt `that ignoramus (of a) doctor' \hfill [attributive] 
\ex \label{ge-ex-it2}
\gll quell’ ignorante del dottore \\
that ignoramus of-the doctor \\ 
\glt `that ignoramus of a doctor'  \hfill [comparative]
\z
\z

\ea \label{ge-ex-engcopdrop}
\ea 
That idiot (of a) doctor prescribed me the wrong medicine. \hfill  \\ \phantom{} \hfill [attributive]
\ex That idiot \#(of a) doctor just wrecked my car. \hfill [comparative]
\z
\z

\noindent What is common between the two types is that in structural terms both are small clauses, i.e., they express a predicational relationship between a subject and a predicate. The difference between them is that they correspond to different syntactic structures according to \citet{denDikken2006}. In the attributive type, the predicate is in the specifier of the small clause (which is a R(elator)P in his terms), see \figref{ge-tree-attr}. The comparative type, on the other hand, is different: the predicate is base-generated in the complement position of the small clause, but subsequently undergoes predicate inversion, which derives the surface order, see \figref{ge-tree-cmpr}. 

%\ea
%\ea \phantom{}[$_\text{RP}$ [$_\text{NP}$ Pred] [$_{\text{R}'}$ Relator [$_\text{NP}$ Subj] ]] \label{ge-br-attr}
%\ex \phantom{}[$_\text{FP}$ Pred$_j$ [$_{\text{F}'}$ Linker+Relator$_i$ [$_\text{RP}$ Subj [$_{\text{R}'}$ \emph{t}$_i$ \emph{t}$_j$ ]]]] \label{ge-br-cmpr}
%\z
%\z

%\newpage

%\columnsep=-3cm

%\begin{multicols}{2}
%\ea 
%\ea Attributive QBNPs \label{ge-tree-attr}\\\vspace{-0.5ex}
%\begin{forest}
%[RP [NP [Pred] ] [R$'$ [Relator] [NP [Subj] ]]]
%\end{forest}        \columnbreak{}
%\ex Comparative QBNPs \label{ge-tree-cmpr}\\\vspace{-0.5ex}
%\begin{forest}
%[FP [NP [Pred$_j$] ] [F$'$ [Linker+Relator$_i$] [RP [NP [Subj] ] [R$'$ [t$_i$] [t$_j$] ]]]]
%\end{forest}
%\z
%\z
%\end{multicols}


\begin{figure}[ht]
    \centering
    %\begin{subfigure}{.48\textwidth}
        %\centering
        %\scalebox{0.8}{
        \begin{forest}
            for tree={s sep=1cm, inner sep=0, l=0}
            [RP [NP [Pred] ] [R$'$ [Relator] [NP [Subj] ]]]
        \end{forest}%}
        \caption{Attributive QBNPs} \label{ge-tree-attr}
    %\end{subfigure}
\end{figure}

\begin{figure}[ht]
    \centering
    %\begin{subfigure}{.48\textwidth}
        %\centering
        %\scalebox{0.8}{
        \begin{forest}
            for tree={s sep=1cm, inner sep=0, l=0}
            [FP [NP [Pred$_j$] ] [F$'$ [Linker+Relator$_i$] [RP [NP [Subj] ] [R$'$ [t$_i$] [t$_j$] ]]]]
        \end{forest}%}
        \caption{Comparative QBNPs} \label{ge-tree-cmpr}
    %\end{subfigure}
    %\caption{Syntactic structures of QBNPs}
\end{figure}


The trees in \figref{ge-tree-attr} and \ref{ge-tree-cmpr} illustrate the main structural difference between the two types of nominal predication: the attributive type is ``born'' as an inverse predication structure as in \figref{ge-tree-attr}, while in the comparative type the predicate acquires its surface position via movement as in \figref{ge-tree-cmpr}. A further difference concerns the size of the subject and predicate noun phrases (labeled as NP for convenience) and the functional heads connecting them. 

Let us take a closer look at the attributive type. \Citet[166--168]{denDikken2006} argues that structurally, it has two subtypes: \emph{an idiot doctor} and \emph{an idiot of/as a doctor}. Both are  small clauses with the structure as in \figref{ge-tree-attr}, i.e., the predicate is base-generated in the specifier position. In the former case, the subject and the predicate are bare NPs. The small clause is then embedded under a nominal layer (NumP), which derives the external nominal distribution. Attributive nominal predications like \emph{an idiot of a doctor}, on the other hand, contain larger nominal projections as their subparts: both the subject and the predicate are NumPs (as they have an indefinite article) and the small clause is topped off by (a zero) D, giving rise to nominal external syntax. \Citet[166--168]{denDikken2006} argues that \emph{of} in the attributive nominal predication is a nominal copula that lexicalizes the Relator head. 

Turning to the comparative nominal predication, \citet[175--181]{denDikken2006} proposes that its predicate starts out in the complement position of the small clause, but undergoes predicate inversion, as the result of which the surface order is derived (see \figref{ge-tree-cmpr}). The mechanics of this predicate inversion is the following. First, the Relator moves up to the small clause-external F$^0$. As a result of this phase-extending movement, the predicate is allowed to move to the specifier of FP; that being a case of A-movement. \Citet{denDikken2006} argues that the predicate inversion is triggered by the need for licensing an empty head. This empty head, SIMILAR, is part of the predicate; this is how the semantics of comparison is encoded. The comparative type of nominal predication is different from the attributive type with respect to the functional heads connecting the two noun phrases. \Citet{denDikken2006} argues that in comparative QBNPs the Relator head is spelled out by the spurious indefinite article (based on evidence from Dutch and Hungarian). In addition, comparative QBNPs also feature a Linker spelling out F$^0$, namely, the nominal copula \emph{of}. The nominal copula \emph{of} is argued to be similar to the obligatory copula in copular inversion constructions (e.g., \emph{I consider the best candidate *(to be) John}). The nominal small clause acquires its outwardly noun-like distribution by virtue of being topped off by a NumP layer that harbours the (indefinite) outer determiner in examples like \emph{a jewel of a village}. Moreover, in \citeauthor{denDikken2006}'s (\citeyear{denDikken2006}) account, both the subject and the predicate are NumPs rather than bare NPs. Hence, comparative nominal predications that would correspond to attributive ones of the type \emph{an idiot doctor} are not possible in English (\emph{a jewel village} is a case of N\nobreakdash-N compounding rather than of comparative nominal predication, see \citealt[163--164, 173]{denDikken2006}).


Spanish utilizes two types of nominal predication constructions (see \citealt{VillalbaBartra-Kaufman2010} for an in-depth discussion). The first one is the so-called \emph{lo-de} construction \REF{ge-ex-sp-lode}: the subject of the small clause (\emph{la casa} `the house') is preceded by an adjective in the neuter and the neuter article \emph{lo}. The second type is the qualitative binominal noun phrase (QBNP) in which two noun phrases participate as the subject and the predicate of the small clause \REF{ge-ex-sp-qbnp}. Both \REF{ge-ex-sp-lode} and \REF{ge-ex-sp-qbnp} are analysed by \citet{VillalbaBartra-Kaufman2010} as small clauses, having the underlying structure of \REF{ge-br-sp-lode} and \REF{ge-br-sp-qbnp}, respectively. (The small clause is labeled as XP in their study.) 

\ea
\ea \label{ge-ex-sp-lode}
\gll lo caro de la casa \\
\Def{}.\N{} expensive.\N{} of \Def{}.\F{} car.\F{} \\
\glt `the (high degree of) expensiveness of the house' \hfill [\emph{lo-de} construction]
\ex \label{ge-ex-sp-qbnp}
\gll el idiota del alcalde \\
\Def{}.\M{} idiot.\M{} of.\Def{}.\M{} mayor.\M{} \\
\glt `that idiot of a mayor' \hfill [Spanish QBNP]
\z 
\z 

\ea
\ea \label{ge-br-sp-lode}
[$_\text{XP}$ [$_{\text{DP}}$ \emph{la casa} ] [$_{\text{X}'}$  X \ldots{} [$_\text{AP}$ \emph{car-} ]]] \hfill = \REF{ge-ex-sp-lode}
\ex \label{ge-br-sp-qbnp}
[$_\text{XP}$ [$_{\text{DP}}$ \emph{el alcalde} ] [$_{\text{X}'}$  X [$_\text{DP}$ \emph{idiota} ]]] \hfill = \REF{ge-ex-sp-qbnp}
\z
\z

\noindent The \emph{lo-de} construction requires some more explanation. First, the predicate of the small clause is argued to have a more complex structure than what was shown in \REF{ge-br-sp-lode}. As the predicate semantically expresses a high degree quantification, it is argued to contain a DegP on top of the adjectival phrase. Furthermore, the DegP also contains a silent DEGREE head.\footnote{This is in a way similar to Den Dikken's proposal that the predicate of comparative nominal predication constructions contains an additional component, i.e., the SIMILAR head, though according to \citet{VillalbaBartra-Kaufman2010}, only the \emph{lo-de} construction contains a DegP, while Spanish QBNPs do not: in them the evaluative property of the predicate is lexically encoded.} Finally, the specifier of this phrase hosts a comparative operator. Thus, the structure of the predicate in \emph{lo-de} constructions is as in \REF{ge-br-degp}.

\ea \label{ge-br-degp}
[$_\text{DegP}$ Op [$_{\text{Deg}'}$ DEGREE [$_\text{AP}$ Adj ]]] %\hfill [predicate of \emph{lo-de} construction, \REF{ge-ex-sp-lode}]
\z

\noindent With these assumptions about the structure of the small clause in \REF{ge-br-sp-lode} in mind, let us proceed to how the surface order of \REF{ge-ex-sp-lode} is derived. This is argued to be the result of three subsequent steps of movement. 
Firstly, similarly to \citet{denDikken2006}, \citet{VillalbaBartra-Kaufman2010} also assume that the DEGREE head must move to F$^0$ where it is lexicalized as \emph{de} `of'. Then, the whole DegP moves to the specifier of the DP-internal FocP, yielding an information-structural partition of the nominal predication construction where the predicate is a focus and the subject is a background topic. The final step is that the operator hosted in SpecDegP moves to SpecDP. \citet{VillalbaBartra-Kaufman2010} argue that the exclamatory flavor of the construction arises from the combination of a degree quantificational structure with the definiteness of the Det head: the null degree operator is argued to function like a wh-element. The three movement steps are shown in \figref{ge-tree-lode-mvt}. 

\begin{figure}[ht]
\begin{forest}
for tree={s sep=1cm, inner sep=0, l=0}
[DP [Op, name=op] [DP$'$ [D [\Def{}.\N{}]] [FocP [DegP, name=degp [\emph{t}$_\text{Op}$, name=optr] [Deg$'$ [\emph{t}$_\text{DEGREE}$] [AP [Adj] ]]] [Foc$'$  [DEGREE+X+Foc, name=degx [\emph{of}] ] [XP [Subj] [X$'$ [\emph{t}$_\text{DEGREE+X}$, name=degxtr] [\emph{t}$_\text{DegP}$, name=degptr] ]]]]]]]
\draw[->, dashed] (degxtr) to [out=-170, in=270] node[circle, draw, solid, fill=lightgray, inner sep=0.7pt] {1} (degx.200);
\draw[->, dashed] (degptr.south) .. controls (-1,-6.5) and (-2.5,-5) .. node[circle, draw, solid, fill=lightgray, inner sep=0.7pt] {2} (degp.west);
%\draw[->, dashed] (degptr.south) .. controls (-2,-9) and (-2,-5) .. node[circle, draw, solid, fill=lightgray, inner sep=0.7pt] {2} (degp.west);
%\draw[->, dashed] (degptr) to [out=-120, in=-150] (degp);
\draw[->, dashed] (optr) .. controls (-1.5,-4) and (-3,-3) .. node[circle, draw, solid, fill=lightgray, inner sep=0.7pt] {3} (op.west);
\end{forest}
\caption{The Spanish \emph{lo-de} construction}
\label{ge-tree-lode-mvt}
\end{figure}


Thus, \citeapos{VillalbaBartra-Kaufman2010} account is similar to \citeauthor{denDikken2006}'s (\citeyear{denDikken2006}) analysis as it assumes that the predicate of the nominal small clause undergoes movement, but it crucially argues that this is an A$'$-dependency, tied in with the information-structural properties of the construction. It should be mentioned that there are also other approaches that assume A$'$-movement of the predicate, the difference between them being the landing site of the moved predicate: SpecDP \citep{Kayne1994}, SpecCP \citep{DoetjesRooryck2003}, or a DP-internal SpecFoc position \citep{VillalbaBartra-Kaufman2010}. % DoetjesRooryck2003 p. 284 In all of these cases, the NP which has been moved into the Specifier of C0 de/that determines the agreement properties of the DP as a whole.
It is also noteworthy that the idea that the exclamative flavor of the \emph{lo-de} construction is linked to the movement of a null operator is also found in other works: for example, \citet{HulkTellier2000} propose for French QBNPs that the head of the small clause moves because of an affective operator in its predicate. 

In sum, it has been shown that nominal predication has two types: attributive and comparative. The latter involves movement of the predicate, but the existing analyses differ as to whether this is a case of A- or A$'$-movement. Having summarized the main semantic and syntactic properties of nominal predication in English and Spanish, let us turn to the Bulgarian data. 



\section{Towards an analysis of nominal predication in Bulgarian} \label{ge-sec-analysis}

In this section, I lay out an analysis of noun phrases with inflectionless adjectives in Bulgarian in terms of nominal predication. %In other words, I argue that these inflectionless adjectives stand in a predicational relationship with the noun, i.e., the subject of the small clause. This means that, unlike genuine adjectives, they are not modifiers, but predicates. 
Importantly, I suggest that the same analysis can be extended to cover small clauses containing two noun phrases as their subparts, thus, the construction in question is not limited to inflectionless adjectives. In the course of this section, I will make the following claims regarding the semantics and the structure of these nominal small clauses: 

\eanoraggedright \label{ge-claims}
\eanoraggedright There is a subject--predicate relationship between the two elements in the complex noun phrase, i.e., we are dealing with a nominal small clause. \label{ge-claim-sc}
\ex Two types of nominal predication are to be distinguished: an attributive and a comparative one. \label{ge-claim-attrcmpr}
\ex The attributive one is an inverse predication structure in the sense of \citet{denDikken2006}; the comparative one involves movement of the predicate. \label{ge-claim-struct}
\ex In both types, the noun phrases in the nominal small clauses are bare NPs. \label{ge-claim-size}
\ex In both types, the predicate is a noun phrase: the inflectionless adjective modifies a null noun. \label{ge-claim-nulln}
\z
\z

\noindent In what follows, I will first provide evidence for the subject-predicate relation and for the existence of two types of nominal predication (\sectref{ge-sec:3.2.1}). Then I will argue that the predicate of the small clause is a noun phrase in which the inflectionless adjective modifies a null noun (\sectref{ge-sec:3.2.2}). Finally, I will discuss the structure of the noun phrases containing nominal predication (\sectref{ge-sec:3.2.3}). 


\subsection{Two types of nominal predication in Bulgarian} \label{ge-sec:3.2.1}

I argue that the two elements in noun phrases with inflectionless adjectives stand in a subject-predicate relationship. Support for this comes from the entailments in \REF{ge-ex-entail} \citep[based on][]{VillalbaBartra-Kaufman2010}. The continuations given in brackets are perceived as contradictions rather than implicature cancellations.\footnote{In contrast, garden-variety adjectives can be cancelled: the continuation in \REF{ge-ex-cancel} does not feel like a contradiction but rather as an implicature cancellation.

\ea \label{ge-ex-cancel}
\gll Tja e krasiva \v{z}ena. \\
she be.\Prs{}.\Tsg{} beautiful.\F{} woman.\F{} \\
\glt `She is a beaufitul woman (but she's actually not (that) beautiful).'
\z}

\ea
\ea \label{ge-ex-entail}
\gll ursuz / kofti \v{c}ovek \\ %$\Rightarrow$ \v{C}ovek-\u{a}t e ursuz / kofti. \\
crabby \phantom{/} bad person.\M{} \\ %{} person.\M{}-\Def{}.\M{} be.\Prs{}.\Tsg{} crabby \phantom{/} bad \\
\glt `(a) crabby/bad man' \\ $\Rightarrow$ `The man is crabby/bad (\#but he's actually not crabby/bad).' 
\ex 
\gll leke \v{c}ovek \\ %$\Rightarrow$ \v{C}ovek-\u{a}t e leke. \\
stain.\N{} person.\M{}  \\ %{} person.\M{}-\Def{}.\M{} be.\Prs{}.\Tsg{} stain.\N{} \\
\glt `(a) rotter of a man' \\ $\Rightarrow$ `The man is a rotter (\#but he's actually not a rotter).'
\z 
\z

\noindent Additionally, as argued by \citet{VillalbaBartra-Kaufman2010} for Spanish, the subject-predicate relation is also constrained lexico-semantically, as the Spanish \emph{lo-de} construction cannot contain stage-level predicates, but only individual-level predicates. The lexical items participating in the Bulgarian construction illustrated in \REF{ge-ex-entail} are a closed class, thus, we cannot make a compelling argument based on this parallel. But still, it can be observed that all items in \REF{ge-ex-list1} and \REF{ge-ex-list2} are individual-level predicates. 

Furthermore, I argue that the two types of nominal small clauses distinguished by \citet{denDikken2006}, namely, the attributive one and the comparative one, are also attested in Bulgarian. Specifically, I propose that the lexical items in \REF{ge-ex-list1} participate in comparative small clauses, while the ones in \REF{ge-ex-list2} are used in attributive nominal predication. The two types can be distinguished semantically when combined with profession-denoting nouns. Comparative nominal small clauses like %the inflectionless adjectives clearly ascribe a permanent property to the subject of the nominal predication. 
\REF{ge-ex-dem-pron} are more naturally interpreted as `X is a crabby person in general', rather than `X is crabby (only) in his capacity of a standard bearer'. On the other hand, in attributive small clauses like \REF{ge-ex-attrsem}, the meaning is such that `X is bad in his capacity of policeman/driver'. 

\ea \label{ge-ex-dem-pron}
\gll Da ne be\v{s}e toja tvoj ursuz {bajraktar [\ldots{}]}\\
\Comp{} \Neg{} be.\Pst{}.\Tsg{} this.\M{} your.\M{} crabby standard.bearer.\M{}\\
\glt `If it wasn't this crabby standard bearer of yours  [\ldots{}]' \hfill [BulNC]
\ex \label{ge-ex-attrsem}
\ea
\gll kofti policaj \\
bad policeman \\
\glt `bad policeman (e.g., corrupt)'
\ex pi\v{s}man \v{s}ofjor \\
fake driver \\
\glt `bad driver (e.g., not having a driving license)'
\z
\z

\noindent This is also confirmed by the following contradiction test \citep[see][170]{denDikken2006}: the English comparative nominal predication is infelicitous in such a context, while the attributive one is perfectly fine \REF{ge-ex-eng-contra}. %\footnote{Similarly, \citet[826]{VillalbaBartra-Kaufman2010} also show  that this context leads to contradiction in the case of the Spanish \emph{lo-de} construction.} 
The Bulgarian examples in \REF{ge-ex-bg-contra} are parallel to the English ones. This test provides further support for the proposal that both types of nominal small clauses are attested in Bulgarian.

\judgewidth{\#}
\ea \label{ge-ex-eng-contra}
\ea[]{That idiot of a doctor is not an idiot (as a person). \hfill [attributive]}
\ex[\#]{That idiot of a man is not an idiot.  \hfill [comparative] } 
\z
\ex \label{ge-ex-bg-contra}
\ea[]{
\gll Tozi kofti policaj.\M{} (vs\u{a}\v{s}tnost) ne e kofti kato \v{c}ovek. \\
this.\M{} bad policeman \phantom{(}actually \Neg{} be.\Prs{}.\Tsg{} bad as person.\M{} \\
\glt `This bad policeman is actually not bad as a person.'
}
\ex[\#]{
\gll Tazi ursuz \v{z}ena (vs\u{a}\v{s}tnost) ne e ursuz. \\
this.\F{} crabby woman.\F{} \phantom{(}actually \Neg{} be.\Prs{}.\Tsg{} crabby \\
\glt `This crabby woman is actually not crabby.' \label{ge-ex-entailBG}
}
\z 
\z

\noindent Having defended the claims in \REF{ge-claim-sc} and \REF{ge-claim-attrcmpr}, namely, that we are dealing with nominal small clauses and that these small clauses fall into either the attributive or the comparative type, let us move to their structure. 

I adopt the main insight of \citeauthor{denDikken2006}'s (\citeyear{denDikken2006}) analysis: attributive nominal small clauses are inverse predication structures in which the predicate is base-generated in the specifier position, while comparative nominal small clauses involve movement of the predicate to a higher position in order to derive the surface order. Thus, I am assuming the structures in \REF{ge-br-bgattr1} and \REF{ge-br-bgcmpr1}, respectively. (The small clause is labeled as XP, that being the most theory-neutral term, instead of R(elator)P as in \citealt{denDikken2006}.)

\ea Nominal predication in Bulgarian (1st version)
\ea \label{ge-br-bgattr1}
[\ldots{} [$_\text{XP}$ Predicate [$_{\text{X}'}$ X Subj ]]] \hfill [attributive]
\ex \label{ge-br-bgcmpr1}
[\ldots{} [$_\text{FP}$ Predicate$_{i}$ [$_{\text{F}'}$ F [$_\text{XP}$ Subj [$_{\text{X}'}$ X t$_{i}$ ]]]]] \hfill [comparative]
\z 
\z

\noindent The proposed structures account for the semantic differences between the two types: in \REF{ge-br-bgattr1}, the predicate is given an attributive interpretation, as it ascribes a(n additional) property to the referent of the noun phrase, while in \REF{ge-br-bgcmpr1}, it draws a comparison in such a way that the subject is understood to intrinsically show the property denoted by the predicate and to be identifiable with it. 

In the next two subsections, I will discuss the structure of these nominal small clauses in greater detail. 


\subsection{Inflectionless adjectives combine with a null noun} \label{ge-sec:3.2.2}

I propose that the predicates of nominal small clauses in Bulgarian are noun phrases. Furthermore, I argue that inflectionless adjectives in Bulgarian combine with a null noun. Thus, the QBNPs containing them are actually binominal. This is supported by the fact that nominal predication in Bulgarian is also possible when the predicate of the small clause is a (non-loan) noun as in \REF{ge-ex-def-gen-stain}. I propose the following underlying structure for the predicate of both the attributive and the comparative types of nominal predication with inflectionless adjectives \REF{ge-br-bg-predSC}. For concreteness, I assume that the null noun is semantically roughly equivalent to `kind/type/quality'.

\ea \label{ge-br-bg-predSC}
[$_\text{NP}$ Adj [$_\text{NP}$ NOUN ]]
\z 

\noindent This proposal not only allows us to unify examples like \REF{ge-ex-def-gen-stain} and the ones containing inflectionless adjectives but also to explain several properties of inflectionless adjectives. For example, the fact that they allow for degree modification like \emph{mnogo} `very', but are outwardly nominal follows from this. In this sense, they are similar to well-known cases of adjectives combining with a null noun like %the so-called `human null construction', e.g.,
\emph{the rich/the poor}, which also allow for adverbial modification of the adjective (cf.\ \emph{the very poor}). This has been taken to suggest that the adjective modifies a null noun \citep[see][]{Kester1996,GiannakidouStavro1999}. 

Evidence for positing a null noun comes from the use of the inflectionless adjectives in copular clauses. In Secion~\ref{ge-sec-data}, I showed that QBNPs are often predicates of copular clauses (example \REF{ge-ex-pred} is repeated in \REF{ge-ex-predx2} for the reader's convenience).

\ea \label{ge-ex-predx2}
\gll  Marija e mnogo ursuz / kofti \v{c}ovek.\\
Maria be.\Prs{}.\Tsg{} very crabby \phantom{/} bad person.\M{}\\
\glt `Maria is a very crabby / bad person.'
\z

\noindent In addition to this, inflectionless adjectives also have what may look like a stand-alone use as predicates of copular clauses. This is illustrated in \REF{ge-ex-pred2inan} with \emph{kofti} `bad, shitty' (cf.\ \REF{ge-ex-list2}), but it is also possible with the adjectives in \REF{ge-ex-list1}.

\ea \label{ge-ex-pred2inan}
\gll Prognoza-ta za vreme-to e mnogo kofti.\\
forecast.\F{}-\Def{} for weather.\N{}-\Def{} be.\Prs{}.\Tsg{} very bad\\
\glt `The weather forecast is very bad.'
\z

\noindent One way to approach the example in \REF{ge-ex-pred2inan} is to say that this is indeed a standalone use of the adjective, without positing a null noun -- which would not be too surprising given that adjectives in Bulgarian can be used as predicates of copular clauses. However, this analysis is insufficient to account for the data in \REF{ge-ex-pred0}. In \REF{ge-ex-pred1}, we see that the adjective cannot have a standalone use: the noun \emph{\v{c}ovek} `person' following it cannot be omitted. At this point, one might wonder if the standalone use is only possible with inanimates as in \REF{ge-ex-pred2inan}, but impossible with animates (humans) as in \REF{ge-ex-pred1}. This analysis, however, is immediately falsified when we look at \REF{ge-ex-pred2}: the adjective can have a standalone use, even though it refers to an animate (human) subject. 

\ea \label{ge-ex-pred0}
\ea \label{ge-ex-pred1}
\gll  Marija e mnogo kofti \#(\v{c}ovek).\\
Maria be.\Prs{}.\Tsg{} very bad \phantom{\#(}person.\M{}\\
\glt `Maria is very bad.'
\ex \label{ge-ex-pred2}
\gll  U\v{c}itelka-ta e mnogo kofti.\\
teacher.\F{}-\Def{} be.\Prs{}.\Tsg{} very bad \\
\glt `The (female) teacher is very bad.'
\z
\z

\noindent Another way to approach these data would be to say that the adjective \emph{kofti} `bad, shitty' cannot occur in predicative position, as it is well-known that certain adjectives cannot occur as predicates, e.g., \emph{biv\v{s}} `former, ex' in Bulgarian. This could explain \REF{ge-ex-pred1}, but not the contrast with \REF{ge-ex-pred2inan} and \REF{ge-ex-pred2}. Thus, the explanation cannot possibly be related to animacy or to the attributive/predicative use of the adjective itself.

In order to account for the triplet of data in (\ref{ge-ex-pred2inan}--\ref{ge-ex-pred0}), I propose the following. The adjective combines with a null noun that has the meaning `kind\slash type\slash quality'. Thus, in \REF{ge-ex-pred2inan} and \REF{ge-ex-pred2}, we are not dealing with a standalone use of the adjective; rather, there is a null nominal modified by it. This also provides an explanation of why \REF{ge-ex-pred1} is infelicitous: the sentence underlyingly corresponds to `\#Maria is (of) bad KIND/TYPE/QUALITY', which is semantically anomalous. The sentence improves if the noun \emph{\v{c}ovek} `person' is present; in this case, I propose that we are dealing with a nominal small clause of the attributive type. That is, the subject of the small clause is \emph{\v{c}ovek} `person' and the predicate is the noun phrase with the null noun modified by the adjective. The meaning of the attributive QBNP corresponds to `Maria is (of) bad KIND/TYPE/QUALITY as a person', which is semantically perfectly fine.\footnote{At this point, the reader might have started to wonder whether \emph{kofti \v{c}ovek} `bad person' should be classified as an attributive QBNP, i.e., `bad \emph{as a} person', as argued above, or as a comparative one like \emph{ursuz \v{c}ovek} `crabby person'. As was shown above, attributive QBNPs are fine in contradiction contexts (cf.\ \REF{ge-ex-eng-contra} and \REF{ge-ex-bg-contra}) and this holds for \emph{kofti \v{c}ovek} `bad person' in \REF{ge-ex-bg-contra2}, thus verifying that we are dealing with an attributive QBNP:

\ea \label{ge-ex-bg-contra2}
\gll Marija e kofti \v{c}ovek, no e dob\u{a}r u\v{c}itel. \\
Maria be.\Prs{}.\Tsg{} bad person but be.\Prs{}.\Tsg{} good teacher \\
\glt `Marija is bad as a person, but is a good teacher.'
\z}

Thus, I propose that inflectionless adjectives \emph{always} compose with a null noun, this giving rise to their ``standalone'' use, which is, as I argue, in fact a noun phrase with a null noun. This noun phrase can appear in predicative position in copular clauses as in \REF{ge-ex-pred2inan} and \REF{ge-ex-pred0}.\largerpage

Before I proceed further with the details of my analysis, let me discuss an alternative that has been proposed in the literature. \citet[100--103]{Adamson2019PhD} mentions the standalone noun-like use of inflectionless adjectives and discusses two subtypes of this: (a) cases in which the adjective is used as a noun, e.g., \emph{inat} `stubbornness', for which he claims that the (acategorial) root is directly nominalized by $n$, and (b) cases in which the adjective appears as an appositive to a proper noun \REF{ge-ex-appos}. In the latter case, he proposes that the (acategorial) root is first categorized by $a$ (thus, degree modifiers will be possible), and then a nominal layer with a [+human] $n$ is added to further nominalize it: $n \succ a \succ \sqrt{\text{INAT}}$.

\ea \label{ge-ex-appos}
\gll Ivan \minsp{(\{} mnogo / po-\}) inat-\u{a}t\\
Ivan {} very {} \Cmpr{} stubborn-\Def{}\\
\glt `Ivan, the (very/ more) stubborn' \hfill \citep[102]{Adamson2019PhD}
\z

\noindent The case I discuss above is similar to the second scenario in the sense that the adjective still preserves its properties with respect to modification. But as we see in \REF{ge-ex-pred2inan}, the referent need not be a human, so it is unlikely that the adjective is nominalized by a [+human] $n$. Besides, in my opinion, the triplet in (\ref{ge-ex-pred2inan}--\ref{ge-ex-pred0}) cannot be easily explained in an nominalization analysis. Finally, the example in \REF{ge-ex-appos} is very degraded for me with the adverbial\slash degree modification; the perfectly grammatical variant is when \emph{inat} is modified by the adjective \emph{golemijat} `big.\Def{}', which would be an example of the `direct nominalization' strategy ($n \succ \sqrt{\text{ROOT}}$). This casts doubts whether $n \succ a \succ \sqrt{\text{ROOT}}$ is possible with inflectionless adjectives at all.

Thus, a nominalization analysis cannot sufficiently explain the properties of inflectionless adjectives, which I argue to be derivable from the presence of a null noun. In addition to postulating a null noun that combines with the adjective, I also propose that this noun phrase can be used as the predicate of a nominal small clause of either the attributive or the comparative type. In \sectref{ge-sec-prev}, I showed that inflectionless adjectives split into two groups with respect to the use of the definiteness marker (cf.\ \ref{ge-ex-list1} and \ref{ge-ex-list2}), and in \sectref{ge-sec:3.2.1}, I argued that these two groups correspond to either the attributive or the comparative type of nominal predication. I tentatively submit that it depends on the lexical properties of the adjective whether the noun phrase that contains it (=\ref{ge-br-bg-predSC}) can be used in an attributive or in a comparative QBNP. 

Furthermore, on the assumption that the null noun the inflectionless adjective combines with is morphosyntactically deficient, i.e., lacking gender features, we would not expect the adjective to show concord with it. As for number features, I assume that the null noun is in the singular (or is specified as [$\minus$Pl]). This might seem to be a circular way of explaining why these loanword adjectives do not show inflection, but the deficiency of the null noun could in principle be relevant if we  approach the question from yet another angle, namely, why non-loan adjectives do not combine with it: because they require a noun that they can show concord with. Thus, we predict them not to be able to combine with this null noun. The third consequence of the morphosyntactic deficiency of the null noun will become clear when we take a closer at the structure of the noun phrases containing QBNPs.  

\subsection{The structure of noun phrases with nominal predication} \label{ge-sec:3.2.3}

Based on the last two subsections we have arrived at the following structure for nominal small clauses in Bulgarian:

\ea Nominal predication in Bulgarian (2nd version)
\ea \label{ge-br-bgattr2}
[\ldots{} [$_\text{XP}$ [$_\text{NP}$ Adj [$_\text{NP}$ NOUN ]] [$_{\text{X}'}$ X Subj ]]] \hfill [attributive]
\ex \label{ge-br-bgcmpr2}
[\ldots{} [$_\text{FP}$ [$_\text{NP}$ Adj [$_\text{NP}$ NOUN ]]$_{i}$ [$_{\text{F}'}$ F [$_\text{XP}$ Subj [$_{\text{X}'}$ X t$_{i}$ ]]]]] \hfill [comparative]
\z 
\z

\noindent The structures in \REF{ge-br-bgattr2} and \REF{ge-br-bgcmpr2} raise the following questions: (i) what is the internal structure (and size) of the subject of the small clause; (ii) what functional layers top off the small clause (informally marked by the ellipsis dots above); and (iii) what is the landing site of the predicate in the case of the comparative nominal predication (labeled above as SpecFP). 

The first two questions are somewhat interrelated and can be answered if we compare the Bulgarian examples with English QBNPs. Recall that in the case of English attributive QBNPs like \emph{an idiot doctor}, \citet{denDikken2006} argues that the subject and the predicate are bare NPs. I argue that nominal predication in Bulgarian is strikingly similar in this respect: in both the attributive and the comparative types, I propose that the nominal predication consists of bare NPs. This is supported by the fact that interleaved adjectival modifiers are highly degraded, as shown in \REF{ge-ex-adj-intrl} (\emph{pace} \citealt{Adamson2019PhD}, cf.\ example \REF{ge-ex-adjac} above).

\ea \label{ge-ex-adj-intrl}
\gll B\u{a}lgarsko-to serbez dete / ??serbez b\u{a}lgarsko-to dete se poznava po pogled-a.\\
Bulgarian.\N{}-\Def{} bold child.\N{} {} \phantom{??}bold Bulgarian.\N{}-\Def{} child.\N{} \Refl{} recognize.\Prs{}.\Tsg{} by gaze.\M{}-\Def{} \\
\glt `You can recognize the bold Bulgarian child by his\slash her gaze  (lit. The bold Bulgarian child is recognized by his\slash her gaze).' 
\z 

\noindent The example in \REF{ge-ex-adj-intrl} is peculiar if we assume that the inflectionless adjective is a modifier like the adjective `Bulgarian': in fact, nationality-denoting adjectives usually precede quality-denoting ones in Bulgarian, thus, the grammatical word order in \REF{ge-ex-adj-intrl} is unexpected if we are dealing with regular adjectival modifiers. However, the word order restrictions fall out naturally if we assume that the inflectionless adjective and the noun form a small clause and that this small clause consists of bare NPs. 

This argument can be further strengthened when we look at recursion in nominal predication. The example in \REF{ge-ex-recurs} is interesting for several reasons. First, it shows that there are two inflectionless adjectives involved. This might at first sight be taken to contradict the claim made above that the subject of the small clause must be a bare NP and cannot be modified by an adjective (as in \REF{ge-ex-adj-intrl}). But in \REF{ge-ex-recurs}, \emph{pi\v{s}man} `fake, sham' is interleaved between \emph{ursuz} `crabby' and the noun. I take this to support the nominal predication analysis from yet another angle: \emph{pi\v{s}man} `fake, sham' is not a regular adjectival modifier, but participates in an attributive QBNP. Then, the attributive QBNP acts as the subject of the comparative QBNP. \REF{ge-ex-recurs} also shows that the the reverse order of the adjectives is degraded. (According to my intuitions, the meaning would be the same, which is in fact predicted, since it is lexically determined whether the adjective participates in the attributive or the comparative type of nominal predication, as I argued above, cf.\ the lists in \REF{ge-ex-list1} and \REF{ge-ex-list2}). The meaning and the word order of \REF{ge-ex-recurs} fit very nicely with the observations about recursion in QBNPs: N$_2$ in N$_1$\nobreakdash-\emph{of}\nobreakdash-N$_2$\nobreakdash-\emph{of}\nobreakdash-N$_3$ is forced into an attributive reading, and comparative QBNPs are not recursive, cf.\ \emph{*that beauty of a jewel of a village}.\footnote{I thank Marcel den Dikken for the discussion of recursion and the English example.} The example in \REF{ge-ex-recurs} complies with this, as the outer QBNP is of the comparative type and the inner one is of the attributive type. %The low POSITION of the small clause (below $n$P) is also not unreconcilable with the idea to analyse the inflectionless adjective and the noun as a compound (as \citealt{Halpern1995} did).

\ea \label{ge-ex-recurs}
%\ea 
\gll tozi ursuz pi\v{s}man \v{s}ofjor / ??tozi pi\v{s}man ursuz \v{s}ofjor\\
this.\M{} crabby fake driver.\M{} \phantom{/} \phantom{??}this.\M{} fake crabby driver.\M{} \\
\glt `this grump of an idiot driver' 
%\ex[??]{
%\gll tozi ursuz pi\v{s}man \v{s}ofjor \\
%this.\M{} crabby fake driver \\
%}
%\z
\z 

\noindent Furthermore, I propose that in order to acquire a nominal external distribution, the small clauses are embedded under a nominal layer. This is in unison with the proposals made for languages like English, Dutch, Spanish, and Hungarian. But I would like to propose that nominal predication in Bulgarian takes place very low in the structure, at the $n$P level, which is in sharp difference with QBNPs in English and Spanish. In Bulgarian, further nominal layers can be built up on top on the $n$P (to harbour numerals, possessive pronouns, and demonstratives, cf.\ example \REF{ge-ex-dem-pron}). With these two assumptions in mind, i.e., that the small clause contains bare NPs and is being topped off by $n$P, we can account for the ungrammaticality of interleaved adjectival modifiers. Since QBNPs are formed at the $n$P level, their referent is understood to be a single individual (thus, in a way, it is not surprising that previous accounts, e.g., \citealt{Halpern1995}, have treated them as nominal compounds). 

This analysis also allows us to make an interesting typological observation regarding the structure of nominal small clauses. It is noted by \citet[168]{denDikken2006} that apart from “bare” attributive nominal small clauses in English like \emph{an idiot doctor}, he is not aware of this type of constructions from other languages. Thus, in a way, the Bulgarian nominal small clauses fill a gap in the typology of nominal predication. Attributive QBNPs like \emph{an idiot doctor} are also “bare” in the sense that they do not contain an overt copula\slash linking element between the subject and the predicate of the small clause, in contrast with \emph{an idiot of a doctor} and \emph{a~ jewel of a village}. Similarly, nominal predication in Bulgarian shows no linking element, and thus conforms to the observation that the size of the subject and the predicate of the nominal small clause correlates with the presence of an overt linking element.
%Den Dikken p. 168 One issue that comes up once we broaden the scope of the discussion to include languages other than English is that attributive QBNPs lacking a nominal copula between the two noun phrases are quite rare: while English has them, they are ungrammatical, for instance, in Dutch (see (17a)10). Nor am I familiar with ‘‘bare’’ attributive QBNPs from other languages.

Finally, we need to discuss the landing site of the predicate in comparative QBNPs. I propose that it is precisely Spec$n$P that the predicate moves to. Recall from \sectref{ge-sec-theorbackgr} that there are several proposals on market regarding the position targeted by the movement predicate. One option would be SpecDP, as proposed by \citet{Kayne1994} for the predicate of English QBNPs; similarly, in \citeapos{VillalbaBartra-Kaufman2010} account, the operator hosted in SpecDegP in the Spanish \emph{lo-de} construction lands in SpecDP (while the predicate moves to a DP-internal SpecFoc position). Additionally, movement to SpecDP has been also proposed for structures like \emph{how tall a man} \citep[see][]{Hendrick1990}. However, movement to SpecDP faces some difficulties in the case of the Bulgarian nominal small clauses. The main problem comes from the order of nominal modifiers. It can be seen from \REF{ge-ex-dem-pron}, repeated below as \REF{ge-ex-dem-pron-2}, that comparative QBNPs can be preceded by demonstratives and pronominal possessive adjectives. 

\ea \label{ge-ex-dem-pron-2}
\gll Da ne be\v{s}e toja tvoj ursuz {bajraktar [\ldots{}]}\\
\Comp{} \Neg{} be.\Pst{}.\Tsg{} this.\M{} your.\M{} crabby standard.bearer.\M{}\\
\glt `If it wasn't this crabby standard bearer of yours  [\ldots{}]' \hfill [BulNC]
\z 

\begin{sloppypar}
\noindent As demonstratives in Bulgarian are said to be always in SpecDP (either being base-generated there or being obligatorily moved there, cf.\ \citealt{DValch-Giusti1995}), the predicate of the small clause cannot possibly move to the very same position. As the linear order in \REF{ge-ex-dem-pron-2} is Dem~>~PronPoss~>~NP$_\text{pred}$~>~NP$_\text{subj}$, this suggests that the landing site of the predicate must be lower than the functional projections that harbour demonstratives and pronominal possessors. I propose that Spec$n$P is an appropriate landing site for the moved predicate. Building on the intuition in \citet{denDikken2006}, I argue that the movement has interpretive effects: the subject NP is compared to the predicate NP. Furthermore, it has been proposed for the Romance languages that the movement is related to the exclamative flavour and/or the information-structural partition of the nominal small clause \citep[see][]{HulkTellier2000,VillalbaBartra-Kaufman2010}. It has been shown in \sectref{ge-sec-data} that Bulgarian QBNPs also have an exclamative flavour and express emphasis on the predicate of the small clause. I would like to tentatively propose that the movement to Spec$n$P also derives this property of the construction on the assumption that Spec$n$P can function as a low focus projection in the nominal domain. As Bulgarian is generally considered to be a split-DP language (for example, \citealt{DValch-Giusti1998} postulate a TopP on top of DP), it is not implausible to assume that noun phrases in Bulgarian contain a functional projection below D that can serve as the landing position of the predicate of the small clause. Thus, the structures of attributive and comparative QBNPs in Bulgarian are as shown in Figures~\ref{ge-tree-bg-attr1} and~\ref{ge-tree-bg-cmpr1}, respectively.
\end{sloppypar}

\begin{figure}[ht]
     \centering
    %\begin{subfigure}{.48\textwidth}
        %\centering
        %\scalebox{0.7}{
            \begin{forest}
                for tree={s sep=1cm, inner sep=0, l=0}
                [\ldots{} [\phantom{XP}] [nP [\phantom{XP}] [XP [NP$_\text{pred}$ [Adj [ \emph{kofti} \\ `bad' ]] [NP [NOUN] ]] [X$'$ [X] [NP$_\text{subj}$ [ \emph{\v{c}ovek} \\ `person'] ]]]]]
            \end{forest}%}
        \caption{Attributive QBNPs in Bulgarian}
        \label{ge-tree-bg-attr1}
    %\end{subfigure}
\end{figure}

\begin{figure}[ht]
    \centering
    %\begin{subfigure}{.48\textwidth}
        %\centering
        %\scalebox{0.7}{
            \begin{forest}
                for tree={s sep=1cm, inner sep=0, l=0}
                [\ldots{} [\phantom{XP}] [nP [NP$_\text{pred}$, name=np [Adj [ \emph{ursuz} \\ `crabby' ]] [NP [NOUN] ]] [n$'$ [n] [XP [NP$_\text{subj}$ [ \emph{\v{c}ovek} \\ `person' ]] [X$'$ [X] [ \emph{t}$_{\text{NP}_\text{pred}}$, name=tr ] ]]]]]
            \end{forest}%}
        \caption{Comparative QBNPs in Bulgarian}        
        \label{ge-tree-bg-cmpr1}
    %\end{subfigure}
    %\caption{Syntactic structures of QBNPs in Bulgarian}        
\end{figure}

Finally, I would like to briefly address the use of the definiteness marker in nominal small clauses, as this was the main question discussed in the literature on inflectionless adjectives, and different proposals have been made for why the definiteness marker attaches to the noun rather than to the adjective. In my view, one of the welcome consequences of the structures in Figures~\ref{ge-tree-bg-attr1} and \ref{ge-tree-bg-cmpr1} is that the adjective is ``buried'' inside the noun phrase of the null noun, that is, it is not a modifier of the subject NP. Since it is not in the structural position that regular inflecting adjectives occupy in Bulgarian, it will not be visible for the definiteness marker to attach to it. Similarly, the null noun itself would also be invisible (either because of lacking a phonological representation or because of its morphosyntactic deficiency). In \sectref{ge-sec-data}, I refined the claims made in earlier studies, and the two most important conclusions were that the definiteness marker is fine with attributive QBNPs, while with comparative ones, it is limited to generic readings.%\footnote{As far as definiteness is concerned, it is noteworthy that similar observations have been made for English, too. \citet[181--182; 300: fn.\ 25]{denDikken2006} mentions that the use of \emph{the} as the outer determiner is not perfectly acceptable for all speakers and that it is more common for British English, but he also reports a couple of examples from American English, such as \emph{I'm having the devil of a job} \Citep[300: fn.\ 25]{denDikken2006}. %It does not appear to be straightforward from this example that the noun phrase has a definite reading. Note also that the majority of English examples reported in the literature contain a demonstrative as their outer determiner, though I have not been able to find information on whether these demonstratives can be interpreted deictically or rather have discourse-pragmatic functions (see the discussion in \sectref{ge-sec-data}).} 
I would like to submit that the movement of the predicate NP in comparative QBNPs results in the unavailability of definite readings with the definiteness marker. I would tentatively propose that this is due to the fact that the predicate NP is indefinite, which thus precludes the definite reading of \Def{} for the whole $n$P after the predicate has moved to Spec$n$P. Thus, the presence of a null noun in the structure can explain why inflectionless adjectives are ``skipped'' by the definiteness marker and also why comparative QBNPs disallow definite readings of \Def{}.

%(see Den Dikken 2006) -- \textcolor{red}{p. 181--182, p. 198 and fn. 50 p. 300, also fn. 25 p. 293}; \\



%--------------section--------------
\section{Conclusion}\label{ge-sec-concl}

In this paper I took a look at the closed set of inflectionless adjectives in Bulgarian from a different angle than the one advocated in the existing literature. I highlighted several empirical facts that have been left unnoticed so far: the limited syntactic distribution of these noun phrases and their exclamative flavour. I also refined the claims on the grammaticality of the definiteness marker with the noun phrases containing these adjectives. First, I pointed out that there are two groups of inflectionless adjectives: with one of them the definiteness marker is fully acceptable, while with the other it is grammatical only with a generic interpretation. These new findings also refined the claims about the interspeaker variation: according to my data, there is no interspeaker variation, as the grammaticality depends on the type of inflectionless adjective and the definiteness of the noun phrase. My main proposal was that inflectionless adjectives are predicates in nominal small clauses, and I also emphasized the fact that this kind of nominal predication is also attested with (non-loan) nouns as well. I outlined an account in terms of nominal predication, and proposed that both the attributive and the comparative types of nominal predication are used in Bulgarian. I suggested that these two types of nominal small clauses have a different structure: the attributive one is an inverse predication structure, whereas the comparative one involves predicate movement. Furthermore, I proposed that the inflectionless adjectives combine with a null noun. This allowed us to give a unified analysis of the nominal small clauses featuring inflectionless adjectives with those in which the predicate is a (non-loan) noun. These were touched upon rather superficially, only for the purpose of comparison with the small clauses with inflectionless adjectives, and further research is needed to reveal the scope of QBNPs in Bulgarian with respect to both its semantic properties and syntactic distribution.

%\printglossaries

\section*{Abbreviations}
\begin{multicols}{3}
\begin{tabbing}
\textsc{comp}\hspace{.75em}\= complementizer\kill
\textsc{1} \> first person\\ 
\textsc{3} \> third person\\
\textsc{cmpr} \> comparative\\
\textsc{comp} \> complementizer\\
\textsc{def} \> definite\\
\textsc{dem} \> demonstrative\\
\textsc{f} \> feminine\\
\textsc{m} \> masculine\\
\textsc{n} \> neuter\\
\textsc{neg} \> negation\\
\textsc{pl} \> plural\\
\textsc{prs} \> present tense\\
\textsc{prt} \> particle\\
\textsc{pst} \> past tense\\
\textsc{ptcp} \> participle\\
\textsc{refl} \> reflexive\\
\textsc{sg} \> singular\\
\end{tabbing}
\end{multicols}

\section*{Acknowledgements}
\begin{sloppypar}
I would like to thank the audiences of SLS5 and FDSL14 as well as Irina Burukina, Lena Borise, Éva Dékány, Katalin É. Kiss, and Marcel den Dikken for their insightful feedback. The manuscript greatly benefited from the helpful comments and suggestions of the two anonymous reviewers. All disclaimers apply. This research has been supported by the research projects ``Nominal Structures in Uralic Languages'' (NKFIH FK 125206) and ``Implications of endangered Uralic languages for syntactic theory and the history of Hungarian'' (NKFIH KKP 129921).
\end{sloppypar}

\printbibliography[heading=subbibliography,notkeyword=this]

\end{document}
