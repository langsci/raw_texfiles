\documentclass[output=paper]{langscibook} 
\ChapterDOI{10.5281/zenodo.10123651}

\author{Björn Wiemer\affiliation{Johannes Gutenberg University Mainz} and Joanna Wrzesień-Kwiatkowska\affiliation{Johannes Gutenberg University Mainz} and Alexander Rostovtsev-Popiel \affiliation{Johannes Gutenberg University Mainz}}
% replace the above with you and your coauthors

% rules for affiliation: If there's an official English version, use that (find out on the official website of the university); if not, use the original

%% in case the running head with authors exceeds one line (which is the case in this example document), use one of the following methods to turn it into a single line; otherwise comment the line below out with % and ignore it
\lehead{Wiemer, Wrzesień-Kwiatkowska \& Rostovtsev-Popiel}

\title[On the grammatical integration of \emph{n/t}-participles
of imperfective stems]{On the grammatical integration of \emph{n/t}-participles
of imperfective stems in Polish and Russian}

% replace the above with your title

\abstract{The article presents a critical discussion of recent work on the semantics of lexical prefixes and of the aspect meanings of \textit{n/t}-participles of imperfective stems in contemporary Russian and Polish, and on the role of all these formations in the voice system of both languages. On this background, a corpus-based study on the development of the aspect functions of these participles for imperfective and perfective stems in Russian and Polish from~1730 until today is discussed, including their syntactic distribution (predicative, appositive, attributive use) and the role of secondary imperfective stems. Special attention is paid to coarse measures of productivity and the changing relation between type and token frequency. This study can be considered the first usage-based investigation from a diachronic perspective in Slavic linguistics, which, to a large extent, is made possible thanks to a database of aspect triplets.

\keywords{Russian, Polish, aspect, voice, \textit{n/t}-participles, diachronic morphology, corpora}
}

\lsConditionalSetupForPaper{}

\begin{document}


%%% uncomment the following line if you are a single author or all authors have the same affiliation
\SetupAffiliations{mark style=none}

\maketitle

\section{Introduction}\label{wiem:sec:intro}
\begin{sloppypar}
The aspect system of Slavic languages is based on a binary distinction between perfective (pfv.) and imperfective (ipfv.) stems. These stems are related not only lexically, but also, on average, morphologically on the basis of productive and commonplace derivational patterns (see \sectref{wiem:sec:triplets}). The stems gain their pfv. and ipfv. status, respectively, from their complementary distribution over sets of contexts, or conditions, which can be defined grammatically (e.g., co-occurrence restrictions with tense markers, with phasal verbs or with modal auxiliaries) or pragmatically (e.g., triggering of presuppositions, type of illocution); cf. \citet{Wiemer2008}, \citet[243--255]{Wiemer.Seržant2017}, \citet[\S\S2--3]{Wiemer.Wrzesień-Kwiatkowska.Wyroślak2020}. All finite and non-finite verb forms are derived from these stems, including participles marked with an \textit{n/t}-suffix (e.g., Russ. \textit{obrabota-n-a} ‘worked out’, Pol. \textit{podję-t-y} ‘taken up’). These participles are employed in every Slavic language, however Slavic languages differ as for the degree to which these participles are restricted by aspect and how integrated they are into the voice system (and its intersections with perfects).\footnote{Cf. \citet{WiemerGiger2005}, \citet{Wiemer2017}, \citet{Arkadiev.Wiemer2020}.} Russian, in particular, demonstrates considerable restrictions on \textit{n/t}-participles from ipfv. stems, and one wonders which role they might play in passive constructions. Polish, to the contrary, has tightly integrated \textit{n/t}-participles of either aspect into its voice system. In fact, after the late 18\textsuperscript{th} century (if not earlier) the role played by ipfv. \textit{n/t}-participles in Russian and Polish has developed in radically different ways.
\end{sloppypar}

This investigation is an attempt at opening a window into this divergent development. Simultaneously, it demonstrates how usage-based accounts should complement formal semantic approaches, mainly because such accounts concentrate not on model-theoretic assumptions, but on distributional patterns. Although the focus of this study is on diachrony, namely the time from 1730 up to now, we will first survey some recent findings and claims about ipfv. participles, particularly in Russian (\sectref{wiem:sec:recent-accounts}). This will lead us to some questions (\sectref{wiem:sec:questions}) and provide a point of departure for a corpus-based study on the functional development and productivity of \textit{n/t}-participles in Russian and Polish. The study is connected to a database of aspect triplets (\sectref{wiem:sec:further-premises}). After a discussion of findings (\sectref{wiem:sec:findings}) some conclusions will be drawn (\sectref{wiem:sec:conclusion}). An Appendix accessible under \url{https://zenodo.org/record/6602167#.YqBH0OzP2Un} contains tables with more detailed information on the statistical figures referred to below. The glosses will consistently distinguish between unprefixed, or simplex, ipfv. stems (IPFV1) and ipfv. stems derived via suffixation from a prefixed pfv. stem (``secondary imperfectives'': IPFV2).\footnote{The glossing also indicates zero-marked categories, without any additional marking in brackets.} Examples cited from corpus samples will only be provided with a general indication of the source and the year or time interval. Examples without an indication of source are constructed by an informed native speaker.

\section{On the status of ipfv. participles}\label{wiem:sec:status-participles}

Our considerations set out from two recent accounts of ipfv. \textit{n/t}-participles in Russian and their involved theoretical repercussions.

\subsection{Recent accounts concerning Russian ipfv. \textit{n/t}-participles}\label{wiem:sec:recent-accounts}

\begin{sloppypar}
On the basis of a query from the Russian National Corpus (RNC; \url{https://ruscorpora.ru}), \citet{Borik.Gehrke2018} found that, in Russian, \textit{n/t}-participles of ipfv. verbs cannot be dismissed as rare or haphazard, nor are they \textit{in toto} to be characterized as lexicalized forms (adjectives).\footnote{In practice, \textit{verb} equals \textit{stem}, if not indicated otherwise.} Instead, their meanings are often compositional and they do occur in constructions that can only be analyzed as true, i.e. event-oriented passives. Compare the following examples, with the (a)-examples containing ipfv. \textit{n/t}-participles, the (b)-examples their active equivalents (the (a)-examples are cited after \citealt[pp. 61, 66, 65 respectively]{Borik.Gehrke2018}):\footnote{Due to the extended length of some (corpus) examples, we do not always gloss the whole example. In case only a proper subpart of an example is glossed (typically a clause), that part is surrounded by square brackets and is matched by a bracketed part in the translation.}
\end{sloppypar}

\ea\label{wiem:ex:strich}
\ea[] {\gll {My} {oba} {\textit{{by-l-i}}} \textit{{striže-n-y}} {nagolo}.\\
we.\textsc{nom} both.\textsc{m}.\textsc{nom} be-\textsc{pst}-\textsc{pl} cut.hair.\textsc{ipfv1-pp-pl} naked.\textsc{adv}\\
\glt ‘We both had our hair \textit{cut off}.’\label{wiem:ex:strich-a}
}

\ex[] {\gll
{Nas} {obo-ix} \textit{{strig-l-i}} {nagolo}.\\
we.\textsc{acc} both-\textsc{m.acc} cut.hair.\textsc{ipfv1-pst-pl} naked.\textsc{adv}\\
\glt ‘They \textit{cut} our hair \textit{off}.’\label{wiem:ex:strich-b}
}
\z
\ex\label{wiem:ex:uchit}
\ea[]{ \gll
\minsp{[} {Ne} {raz} {ja} {\textit{by-l}} {\textit{uče-n}}], {molču} {i} {znaju}.\\ 
{} \textsc{neg} once I.\textsc{nom} be-\textsc{pst-sg.m} teach.\textsc{ipfv1-pp-sg.m}\\
\glt ‘[I was \textit{taught} more than once], I keep silent and know.’ \label{wiem:ex:uchit-a}
}

\ex[]{ \gll
{Ne} {raz} {menja} {\textit{uči-l-i}}.\\
\textsc{neg} once I.\textsc{acc} teach.\textsc{ipfv1-pst-pl}\\
\glt ‘They have \textit{taught} me more than once.’ \label{wiem:ex:uchit-b}
}
\z
\ex\label{wiem:ex:pisat}
\ea[]{ \gll
{Pisa-n-o} {ėto} {by-l-o} {Dostoevsk-im} {v} {1871} {god-u}.\\ 
write.\textsc{ipfv1-pp-sg.n} this be-\textsc{pst-sg.n} \textsc{pn-ins} in \textsc{} year-\textsc{loc}\\
\glt ‘This \textit{was} \textit{written} by Dostoevskij in 1871.’ \label{wiem:ex:pisat-a}
}

\ex[]{ \gll
\textit{{Pisa-l}} {ėto} {Dostoevsk-ij} {v} {1871} {god-u.}\\
write.\textsc{ipfv1-pst-sg.m} this \textsc{pn-nom} in \textsc{} year-\textsc{loc}\\
\glt ‘Dostoevskij \textit{wrote} this in 1871.’ \label{wiem:ex:pisat-b}
}
\z
\z

\noindent An event-oriented use (or eventive orientation) can hardly be denied for \REF{wiem:ex:uchit} and \REF{wiem:ex:pisat}.\footnote{\label{wiem:foot:OneCouldSpeak}One could speak of ``eventive focus''  as well. However, the term ``focus'' occurs in two different, though related senses. It either refers to the asserted part of an event structure (as in (\ref{wiem:ex:uchit}--\ref{wiem:ex:pisat})), or it refers to the comment as part of the information structure of an utterance. It is hardly possible to distinguish these two senses by short circumscriptions or different synonyms, and their relation becomes clear when we realize that time adverbials which gain informational focus can be (and often are) employed as means to test the asserted part of some tense-aspect marker: such diagnostics relies on the “harmony” between both kinds of focus in an utterance. Hopefully, in the remainder the respective context will disambiguate the intended sense.} In \REF{wiem:ex:uchit} this orientation is supported by an adverbial which marks the situation as repeated; a resultant state is only implied, it is indicated in the subsequent clause
(‘$\text{people taught me}\leadsto\text{I know (because of that)}$’).
% (‘$\text{people taught me}>\text{I know (because of that)}$’).
In \REF{wiem:ex:pisat} the event is presupposed, while the communicative focus is on the adverbial (‘in 1871’) which puts this event in a larger time frame. By contrast, an eventive orientation is more difficult to get for \REF{wiem:ex:strich}, rather it refers to the state that results after the hair cut was completed.\largerpage[1]

\citet{Borik.Gehrke2018} point out that Russian ipfv. \textit{n/t}-participles seem to be restricted to \textsc{general-factual} meanings; in general, progressive readings are practically unattested (cf. also \cites[57--58]{Knjazev1989}[489]{Knjazev2007}). Since the general-factual (GF) meaning is considered dominant, one wonders how GF relates to event-oriented uses of ipfv. \textit{n/t}-participles. We should be aware that the label ``general-factual'' unites at least two rather different main functions, called \textsc{presuppositional} and \textsc{existential}, and that in the discussion about ipfv. \textit{n/t}-participles the presuppositional GF clearly dominates.\footnote{For detailed analyses of GF cf. \citet{Groenn2004}, \citet{Mehlig2011}, \citet{Dickey2015}, \citet{Mueller-Reichau2018}.} Thus, \citeauthor{Borik.Gehrke2018} demonstrate that ipfv. \textit{n/t}-participles are comparable to definite descriptions, since they anaphorically refer to known, or presupposed, situations (eventualities), and these can be events; see \REF{wiem:ex:plata}. The same can be said for GF in the active voice; see \REF{wiem:ex:pismo}. The parts in curly brackets contain the notional antecedents of the “anaphoric” verb forms (in italics).\largerpage

\ea\label{wiem:ex:plata}{
{čto} {kasaetsja} \{{platy} {deneg}\},\\
\gll \minsp{[} {to} {\textit{plač-e-ny}} \textit{{by-l-i}} {naličnymi} {šest'} {tysjač} {rublej}].\\ 
{} \textsc{ptc} pay.\textsc{ipfv1-pp-pl} be-\textsc{pst-pl} cash six thousands rubels\\
\glt ‘As for the payment, [six thousand rubles \textit{were paid} in cash].’\\\hfill
(cited from \citealt[70]{Borik.Gehrke2018})}
\ex\label{wiem:ex:pismo}{
{V} {ėtoj} {porternoj} \{{ja} {napisal} {pervoe} {ljubovnoe} {pis'mo}\}.\\ \gll \minsp{[} \textit{{Pisa-l}} {karandaš-om}.]\\
{} write.\textsc{ipfv1-pst-sg.m} pencil.\textsc{m-ins.sg}\\
\glt ‘In this tavern, I wrote my first love letter. [I \textit{wrote} it with a pencil].’\\\hfill
(cited from \citealt[64]{Borik.Gehrke2018})}
\z

\noindent Here, we should however take into account that anaphoric use (with event modifiers) is characteristic only of presuppositional GF (see (\ref{wiem:ex:plata}--\ref{wiem:ex:kniga})), not of the existential (or verifying) type, to which \REF{wiem:ex:uchit} comes close. Apart from that, at least in contemporary Russian, ipfv. \textit{n/t}-participles in existential GF are very rare unless they occur under negation. Without negation they sound archaic (see \REF{wiem:ex:pirogi}, constructed); \REF{wiem:ex:korol} is one of the few examples without negation found in the RNC. Note that, apart from existential readings, in such cases \textit{n/t}-participles from ipfv. stems (as well as their finite forms) may also be interpreted as referring to a repeated event:\largerpage

\ea\label{wiem:ex:kniga} {\gll
{Knig-a} {\textit{pečata-n-a}} {pri} {Petre} {Pervom}.\\
book.\textsc{f-nom.sg} print.\textsc{ipfv1-pp-sg.f} at Peter First \\
\glt‘The book was \textit{printed} under Peter the Great.’ \\  $\rightarrow$ presuppositional, narrow scope:  temporal location possible 
}
% \z

%\ea\label{wiem:ex:kniga}{
%\begin{tabularx}{\textwidth}{@{}QQ@{}}
%Kniga pečatana\textsuperscript{IPFV1} pri Petre Pervom. &  \rightarrow presuppositional, narrow scope:\\
%‘The book was printed under Peter the Great.’ &  temporal location possible \end{tabularx}%
%}
%\z

\ex\label{wiem:ex:pirogi} {\gll
{U} {vas} \textit{{byl-i}} \textit{{peče-n-y}} {pirogi}?\\
at we.\textsc{gen} be-\textsc{pst-pl} bake.\textsc{ipfv1-pp-pl} pie-\textsc{nom.pl}\\
\glt ‘Did you bake pies?'\\
(more lit.: `Did you have pies \textit{baked}?') ($>$ can also refer to habits) \\
$\rightarrow$ existential, verum focus (wide scope): no temporal location 
}
% \z

%\ea\label{wiem:ex:pirogi}{
%\begin{tabularx}{\textwidth}{@{}QQ@{}}
%U vas byli pečeny\textsuperscript{IPFV1} pirogi? & \rightarrow existential, verum focus (wide scope):\\
%‘Did you bake pies?‘  & no temporal location\\
%(more lit. Did you have pies baked?) & $({}>\text{can also refer to habits})$ \end{tabularx}%
%}
%\z

\ex\label{wiem:ex:korol}
{U} {odnogo} {korolja} {byl} {šut}. {V} {junosti} {pošučival} {na} {svoj} {strax} {i} {risk} {na} {ploščadjax} {i}\\
{\gll \minsp{[} \textit{{by-l}} \textit{{poro-t}}], {vsledstvie} {čego} {poumnel}.\\ 
{} be-\textsc{pst.sg.m}	flog.\textsc{ipfv-pp.sg.m} {}\\
\glt 
‘One king had a jester. In his youth, he (the jester) joked at his own peril and risk in the squares and [\textit{was flogged}], as a result of which he grew wiser.’\hfill (Russian; RNC; 2000)\\$\rightarrow$ existential, wide scope: no temporal location
}
\z

\noindent Contrary to the presuppositional type, in existential GF the eventuality constitutes the informational focus (indicated by stress, e.g. on \textit{byli} in \REF{wiem:ex:pirogi}), i.e. the part which is unknown and which can be asked about (see Footnote~\ref{wiem:foot:OneCouldSpeak}). Jointly with this, it does not matter whether this eventuality took place once or more than once.\footnote{This brings the existential GF close to the experiential function of perfects known from typological research (cf. \citealt{Arkadiev.Wiemer2020}).} The fact that the concrete temporal location is not at stake explains the just mentioned “oscillation” with habitual readings. Therefore, what existential and presuppositional GF unites is the downgrading of the eventuality denoted by the VP, although this happens for diametrically opposed reasons. Thus, with presuppositional GF, downgrading concerns the information structure (since the eventuality is already known to have taken place within a specific reference interval), whereas with existential GF, downgrading concerns time location (there is no discrete interval for which it may be claimed true that the eventuality occurred, or this is irrelevant). What follows from this is that even if the predicate refers to a distinct single event, this (or any other) actionality feature is assigned background status. However, since presuppositional GF is associated to time-located events, this location can be targeted, e.g. by temporal adverbials (see \REF{wiem:ex:pisat} and \REF{wiem:ex:kniga}).

Similarly, \citet[][59]{Borik.Gehrke2018} characterize cases like \REF{wiem:ex:strich} as adjectival participles: “unlike with verbal passives, the underlying event in adjectival passives lacks spatiotemporal location or referential event participants, and only the state associated with the adjectival participle can be located temporally”. Verbal participles, in turn, can have “spatiotemporal event modifiers, referential \mbox{by-}\slash\mbox{with-}phrases, and similar such expressions” which highlight the event (\citealt[59]{Borik.Gehrke2018}). However, adjectival participles need not be lexicalized.\largerpage[1]

\citeauthor{Borik.Gehrke2018} also point out that Russian ipfv. \textit{n/t}-participles in true passives derive from a restricted set of verbs, most of them related to speech acts or with incremental objects. That is, in comparison to compositional pfv. \textit{n/t}-participles, their overall type and token frequency in passives seems to be low, after all. Moreover, only simplex (IPFV1) stems are used, while ipfv. stems derived via suffixation from a prefixed pfv. stem (IPFV2) are absent in modern Russian. While these claims are largely supported by our findings, we will show that some of them require qualification when we look at them from a usage-based perspective (see \sectref{wiem:sec:findings}).

In turn, \citet[288--292]{Tatevosov2015Akcional} employs the behavior of Russian \textit{n/t}-par\-ti\-ci\-ples of IPFV1 stems as support for his claim that lexical (or ``inner'') prefixes add resultative subevents, while IPFV1 stems are void of this component. Compare the following examples with their logical structures, in which the subscripts A and S indicate an action and a state, respectively:

\ea\label{wiem:ex:vanja}
\ea[]{ \gll
{Vanj-a} {pisa-l} \minsp{(} stat'j-u).  \\
\textsc{pn.m-nom.sg} write.\textsc{ipfv1-pst-sg.m} {} article.\textsc{f-acc.sg}\\
\glt ‘Vanja wrote\slash was writing (an article).’\\
\sib{pisa} $= \lambda y \lambda x \lambda e \,[\textsc{{write}\textsubscript{A}}(e) \wedge \cnst{{initiator}}(x)(e) \wedge \cnst{{theme}} (y)(e)]$}\label{wiem:ex:vanja-a}

\ex[]{ \gll
Vanj-a na-pisa-l \minsp{*(} stat'j-u).  \\
\textsc{pn.m-nom.sg} \textsc{pvb}-write.\textsc{pfv-pst-sg.m} {} article.\textsc{f-acc.sg} \\
\glt ‘Vanja wrote\slash has written an article.’\\
\sib{napisa} $=\lambda y \lambda x \lambda e \lambda s [\textsc{{write}\textsubscript{A}}(e) \wedge \cnst{{initiator}}(x)(e) \wedge \cnst{{theme}}(y)(e) \wedge \cnst{{causing}}(s)(e) \wedge\textsc{{write}\textsubscript{s}(s)} \wedge\cnst{{arg}}(y)(s)]$
}\label{wiem:ex:vanja-b}
\z
\z

\noindent Notably, Tatevosov treats as lexical prefixes not only those which modify, or change, the lexical meaning of the IPFV1 stem (as in Russ. \textit{rabotat’ (*den’gi)} (intended:) ‘work (money)’ {\rightarrow} \textit{za-rabotat’ *(den’gi)} ‘earn money’), but also so-called natural prefixes (\citealt{Janda2007Aspectual}), whose function overlaps with a meaning component implied by the simplex (e.g., Russ. \textit{varit’ {\rightarrow} s-varit’ *(sup)} ‘cook (soup)’, \textit{delit’ {\rightarrow} raz-delit’ *(gruppu)} ‘divide (group)’). Natural prefixes are a precondition for the rise of aspect triplets (see \sectref{wiem:sec:further-premises}).

Since \textit{n/t}-participles are derived from these stems, they should also show behavior that ensues from the presence vs. absence of a resultative subevent. In fact, IPFV1 \textit{n/t}-participles in passives usually require modifiers that relate to the event, not a subsequent state (as confirmed by \citealt{Borik.Gehrke2018}, see above); compare Russ.\textit{ Pis’mo} \textit{pisano} *(\textit{na} \textit{tonkoj} \textit{bumage}) ‘The letter is written *(on thin paper)’ vs. \textit{Pis’mo} \textit{napisano} (\textit{i} \textit{ležit} \textit{na} \textit{stole}) ‘The letter has been [lit. is] written (and is lying on the table)’. Tatevosov takes this as evidence that IPFV1 stems, and with them their \textit{n/t}-participles, lack a resultative subevent.\footnote{Here we need not take stance as for Tatevosov’s subsequent claim that (pfv.) aspect is assigned above \textit{v}P and not a property of the verb stem (cf. \citealt[107--110]{Wiemer2019} for discussion).} Simultaneously, he points out that ipfv. \textit{n/t}-participles are unable to denote not only ongoing processes, but even habitual situations. Thus, the only reading “left” for them is general-factual meanings (\citealt[291]{Tatevosov2015Akcional}).\largerpage[1.5]

\begin{sloppypar}
The conclusion concerning GF is congruent with the analysis by \citet{Borik.Gehrke2018}, but it raises the question why certain ipfv. \textit{n/t}-participles in Russian prefer stative readings; see \REF{wiem:ex:strich} and the following example:
\end{sloppypar}

\ea\label{wiem:ex:pol}{\gll
{Pol} {\textit{by-l}} \textit{{mošče-n}} {širok-imi} {serovat-ymi} {kamnj-ami}.\\
floor.\textsc{m-nom.sg} be-\textsc{pst-sg.m} pave.\textsc{ipfv1-pp-sg.m} wide-\textsc{ins.pl} grayish-\textsc{ins.pl} stone-\textsc{ins.pl}\\
\glt ‘The floor \textit{was} \textit{paved} with wide grayish stones.’\hfill \citep[292]{Tatevosov2015Akcional}
}
\z

\noindent Tatevosov declares \textit{moščen} `paved' to be an adjectivized participle, so that the explanation would be the same as by \citet{Borik.Gehrke2018} for adjectival passives (see above), in particular we understand why this form yields the same aspectual semantics as does its pfv. counterpart (\textit{vymoščen}), but, contrary to the latter, cannot be used with a focus on the event itself. Note that this holds independently from the distinction between existential and presuppositional GF. Thus, \REF{wiem:ex:pol} could be uttered in continuation, e.g., of \textit{Oni vošli v~ogromnyj zal} ‘They entered a huge hall’, on which \REF{wiem:ex:pol} would add information concerning a salient part of the newly introduced referent \textit{zal} ‘hall’; but this information only would refer to the state of the floor, not to an event of paving it. Consider also cases like \textit{Rab byl porot}\textsuperscript{IPFV1}\slash\textit{vyporot}\textsuperscript{PFV} \textit{triždy} ‘The slave was flogged three times’. However, there are other cases of ipfv. \textit{n/t}-participles which cannot be explained away as adjectives (or adjectival passives); see the examples and discussion in \sectref{wiem:sec:findings}. Again, the question arises as to how an eventive orientation relates to GF.

\textcitetv{chapters/05-Gehrke} argues that (finite or participial) forms of ipfv. stems may be used with reference to concrete single events if these events have been mentioned in, or can be inferred from, the immediately preceding discourse. Following Gehrke's suggestion, we should realize that anaphoric relations to events sometimes need support by metonymic relations (between parts of events that have been mentioned and those which have not) to be inferred. In addition to Gehrke, one wonders whether it is necessary to assume an eventive component in the semantic description of forms of ipfv. stems. However, Gehrke rightly criticizes formal accounts of GF for having put too strong an emphasis on event completion; completion is neither a necessary nor a sufficient condition for choice of pfv. aspect, nor for the exclusion of ipfv. aspect. 

As a matter of fact, (non-)completion is not a constitutive property of (im)per\-fec\-tive aspect; instead, the crucial criterion is (non-)boundedness, or whether an eventuality is presented as limited or not (cf. \citealt{Lehmann1999}, \citealt{Wiemer2017}, \citealt{Wiemer.Seržant2017}, \citealt{Breu2021}, among many others). We therefore support both points made by Gehrke. However, again, her argument is based on the presuppositional type of GF, leaving open how it might work for other usage conditions of ipfv. aspect. The bulk of examples from our study that may be classified as GF do not represent the presuppositional type, and there is much leeway in categorizing these examples anyway (see \sectref{wiem:sec:further-premises}--\sectref{wiem:sec:findings}).

Finally, there is one issue left concerning \citeposst{Tatevosov2015Akcional} analysis. Namely, why should forms of stems for which resultative subevents are lacking be incapable of denoting ongoing processes (progressive meaning) or habitual situations, which are functions typically associated to ipfv. aspect? First, a resultative subevent presupposes a change of state, and this is entailed by pfv. stems which contribute to a telic meaning of the clause (see \REF{wiem:ex:vanja}%the examples above
). There is thus no inherent reason why the lack of a resultative subevent should block the denotation of an activity for which boundaries are absent, or defocused (and, thus, progressive meaning). Second, pluractional meanings, like the habitual one, are insensitive to actionality distinctions ($\pm$ telic; process, event, state).\footnote{Cf. \citet[118]{Tatevosov2016} and the literature referred to in Footnote~\ref{wiem:foot:WeFollowSystem}.} Compare \textit{He used to sleep after dinner} ({\rightarrow} habitual state), \textit{During our discussions she used to remark that P} ({\rightarrow} habitual event), \textit{Whenever I met them in the club, they used to be discussing the latest soccer game} ({\rightarrow} habitual process). This becomes particularly obvious in Russian, where ipfv. stems -- both IPFV1 and IPFV2 -- are usually employed as “placeholders” of their pfv. counterparts in event readings that focus on the attainment of a goal ($=$ right boundary). Compare habitual readings of events which, each time they occurred, reached their culmination point:

\ea\label{wiem:ex:direktor}{\gll
{Po} {utram} {direktor} {\textit{vyzvaniva-l}} {vsex} {zamov} {i} {\textit{raspredelja}-l} {meždu} {nimi} {zadači} {na} {den}’.\\
on mornings director.\textsc{m-sg.m} call.out.\textsc{ipfv2-pst-sg.m} all deputies and distribute.\textsc{ipfv2-pst-sg.m} among them tasks on day\\
\glt ‘In the morning, the director \textit{used to call} all the deputies and \textit{distribute} among them tasks for the day.’
}
\ex\label{wiem:ex:jabloki}{
{Kogda} {mne} {bylo} {pjat’} {let},\\
\gll \minsp{[} {ja} {každyj} {den’} {na} {zavtrak} \textit{{s"eda-l}} {tri} {jabloka}].\\
{} \textsc{I.nom} each day on breakfest eat.\textsc{ipfv2-pst-sg.m} three apples\\
\glt‘When I was five years old, [I \textit{ate} three apples every day for breakfast].’
}
\ex\label{wiem:ex:obed}{\gll
{Na} {vyxodnye} {on} {vsem} {k~času} \textit{{gotovi-l}} {obed}.\\
on free.days he.\textsc{nom} everybody-\textsc{dat.pl} to~hour prepare.\textsc{ipfv1-pst-sg.m} dinner-\textsc{acc}\\
\glt ‘On weekends, he \textit{cooked} dinner for everyone by one o’clock.’
}
\z

\noindent This ability to function as grammatical equivalents of pfv. stems in denoting completed events can only be explained if we assume that ipfv. stems can acquire properties of their pfv. counterparts. The question is to which extent this carries over to their participles. Therefore, even if it turned out true that ipfv. \textit{n/t}-participles are incapable of denoting habitual situations, this could hardly be explained from model-theoretic assumptions and other premises accepted by Tatevosov. After all, it should be checked to which extent this claim is empirically adequate.

\subsection{Questions}\label{wiem:sec:questions}

The claims presented by \citet{Borik.Gehrke2018} and \citet{Tatevosov2015Akcional} generate some questions. First of all, Tatevosov’s morpheme-centric generative analysis would imply that the additional subevent remains with IPFV2 stems, since these suffixed stems are derived from prefixed (pfv.) stems. Tatevosov does not consider IPFV2, certainly because in contemporary standard Russian \textit{n/t}-participles of IPFV2 stems practically do not exist. Thus, one wonders which consequences are to follow for \textit{n/t}-participles of IPFV2 stems in passives if they do occur. First of all, shouldn’t habitual readings be compatible, if not preferred, for the reasons indicated in \sectref{wiem:sec:recent-accounts}?

This can be tested for Polish, where, as in other West Slavic languages, \textit{n/t}-participles of ipfv. stems are commonplace (\citealt[135--138]{Wiemer2017}). Polish has completely integrated \textit{n/t}-participles of both IPFV1 and IPFV2 stems into the aspect system and its interface with voice (\citealt{Lehmann1992}, \citealt{Wiemer1996}, \citealt{Górski2008}). However, we do not know much about their productivity and function range, first of all, in a diachronic perspective. Moreover, one may ask to which extent the ability of ipfv. \textit{n/t}-participles to function as full-fledged members on the aspect-voice interface correlates with the overall frequencies (for all grammatical forms) of their stems (see \sectref{wiem:sec:correlation}).

Furthermore, GF is insensitive to actionality features as well, but if GF relates to an event this may entail a result (provided the event is telic); see examples in \sectref{wiem:sec:recent-accounts}. The point is not whether a result has ensued, but whether it is treated as the asserted part of the message or as presupposed information. That is, in accordance with \citet{Borik.Gehrke2018} and \textcitetv{chapters/05-Gehrke}, GF should be evaluated not so much with respect to the internal structure of events, but in terms of information structure, and (deictic or relative) time location may, but need not, become an issue (see \REF{wiem:ex:pisat}, \REF{wiem:ex:kniga}). Moreover, the lexical (and natural) prefixes which, according to Tatevosov, “bind” a resultative subevent, can be taken as means that establish this subevent as an undeniable part of the verb’s meaning. However, this does not imply that the respective IPFV1 stems exclude a resultative subevent. Provided they occur (as VP heads) in suitable clausal contexts, they may just be able to defocus such a subevent; that is, they are labile in this respect. Otherwise, how would we explain the employment of ipfv. stems (finite forms or \textit{n/t}-participles) in GF which evidently refer to an event (e.g., Pol. \textit{Już byłem o to pytany}\textsuperscript{IPFV1} ‘I have already been asked about that’, Russ. \textit{Menja ob ėtom uže sprašivali}\textsuperscript{IPFV2} ‘They already asked me about that’), and their employment as replacements of pfv. verbs in the denotation of events, e.g., in the narrative present? Apart from that, it is justified to ask for stative readings of ipfv. \textit{n/t}-participles in the contemporary and earlier stages. Here we should keep in mind that stative readings do not automatically indicate that participles have lexicalized as adjectives (see \citealt{Borik.Gehrke2018} in \sectref{wiem:sec:recent-accounts}).

\section{Further premises and the data used for the study}\label{wiem:sec:further-premises}

\sloppy Related empirical questions are addressed in the following. We present findings concerning the aspectual behavior of \textit{n/t}-participles from IPFV1 and IPFV2 stems, primarily on the basis of a comprehensive database containing potential aspect triplets (e.g., Pol. \textit{tworzyć}\textsuperscript{IPFV1} -- \textit{stworzyć}\textsuperscript{PFV} -- \textit{stwarzać}\textsuperscript{IPFV2} ‘create’, Rus. \textit{paxat’}\textsuperscript{IPFV1} -- \textit{vspaxat’}\textsuperscript{PFV} -- \textit{vspaxivat’}\textsuperscript{IPFV2} ‘plough’), which covers the period 1750--2018 in Russian and Polish. Triplets have the advantage that the meanings and behavior of IPFV1 and IPFV2 can be compared directly. We first comment on triplets and our database (\sectref{wiem:sec:triplets}) before we turn to the sampling procedure (\sectref{wiem:sec:sampling-procedure}) and the annotation schema (\sectref{wiem:sec:annotation}).

\subsection{Triplets}\label{wiem:sec:triplets}

In connection with the \textit{DiAsPol}-project, a database of aspect triplets for the period 1750--2018 has been created for Polish, Czech, and Russian.\footnote{See \url{https://www.diaspol.uw.edu.pl/eng/}. A detailed description of the database is underway.} Aspect triplets (or, more strictly, ``bi-imperfective aspect triplets'', see \citealt[235--236]{Zaliznjak.Mikaėljan.Šmelev2015},  henceforth simply \textsc{triplets}) are built on a constellation in which two ipfv. stems lexically correspond to the same cognate pfv. stem: one ipfv. stem is derived from the pfv. stem by a suffix (= IPFV2), the other is an unprefixed ipfv. stem (= IPFV1, or simplex) and itself the morphological basis for the pfv. stem. 
Compare, for instance, the following illustrations for Russian and Polish.\footnote{Suffixation may consist either in an addition (as with \{va\} in the case of \textit{na-gre-va-t’}), or in a replacement (as with $\{i\}>\{a\}$ in the case of \textit{roz-dziel-a-ć}), or
in a replacement that resulted from the coalescence of two more elementary segments (as with $\{iva\}<\{i\text{-}va\}$ in the cases of \textit{pri-gotavl-iva-t’} and 
\textit{na-kaz-ywa-ć}).
These distinctions are irrelevant for the present concern.}

% % Table for the Russian examples 14a and 14b

\ea\label{wiem:tab:gret-gotovit}%
\begin{tabular}[t]{@{}ll@{~~}l@{~~}l@{~~}l@{~~}l@{~~}ll@{}}
& & IPFV1 & & PFV & &  IPFV2 &\\
& a.&\textit{gre-t’} & \rightarrow &  \textit{na-gre-t’} & \rightarrow & \textit{na-gre-va-t’} & ‘warm up’\\
& b.&\textit{gotov-i-t’} & \rightarrow & \textit{pri-gotov-i-t’} & \rightarrow & \textit{pri-gotavl-iva-t’} & ‘prepare (meal)’\\
\end{tabular}

% Table for the Polish examples 15a and 15b
\ex\label{wiem:tab:divide-order}%
\begin{tabular}[t]{@{}ll@{~~}l@{~~}l@{~~}l@{~~}l@{~~}ll@{}}
& &IPFV1 & & PFV & &  IPFV2 &\\
& a.&\textit{dzieli-ć} & \rightarrow & \textit{roz-dziel-i-ć} & \rightarrow & \textit{roz-dziel-a-ć} & ‘divide; separate’\\
& b.&\textit{kaz-a-ć} & \rightarrow & \textit{na-kaz-a-ć} & \rightarrow  & \textit{na-kaz-ywa-ć} & ‘order’\\
\end{tabular}
\z


\noindent Triplets result from an overlay of the two most productive patterns by which aspect pairs are created in Slavic languages, namely (for Russian):

% Table for the Russian examples 16 and 17
\ea\label{wiem:tab:write-rewrite}%
\begin{tabular}[t]{@{}ll@{~~}l@{~~}l@{~~}l@{~~}l@{~~}ll@{}}
&& IPFV1 && PFV & & IPFV2 &\\
&a.& \textit{pis-a-t’} & \rightarrow &  \textit{na-pis-a-t’}  &  & & ‘write’\\
&b.& &  & \textit{pere-pis-a-t’} & \rightarrow & \textit{pere-pis-yva-t’} & ‘rewrite’\\
\end{tabular}
\z

\noindent Another precondition is that the derivation IPFV1 {\rightarrow} PFV involves a natural prefix, so that no lexical change (or modification) obtains (see \sectref{wiem:sec:recent-accounts}). In triplets, both IPFV1 and IPFV2 function as lexical replacements of PFV in grammatically or pragmatically defined contexts, well-known to Slavic aspectology, and IPFV1 and IPFV2 may also replace one another. Admittedly, in many cases only IPFV2 is considered an exact “lexical copy” of its pfv. counterpart, which also shares the argument requirements of the latter, while IPFV1 stems betray less strict requirements (usually for objects) and are lexically more diffuse than the remaining pair of prefixed PFV and IPFV2. However, in many cases there is no IPFV2 -- or it is derived only occasionally and not considered part of the standard language -- and IPFV1 alone “fulfills the duties” of the PFV’s lexical copy, as in the case of \textit{pisat’}~-- \textit{napisat’}. 

\tabref{wiem:tab:sizes-aspect} provides the number of items of which the Russian and the Polish triplet database is composed.

\begin{table}
\begin{tabular}{lccr}
\lsptoprule
Polish & 1,773 triplets & $-$ & 1,386 (IPFV1), 1,773 (PFV), 1,807 (IPFV2)\\
Russian & 1,275 triplets & $-$ & 837 (IPFV1), 1,275 (PFV), 1,461 (IPFV2)\\
\lspbottomrule
\end{tabular}
\caption{Sizes of aspect triplet database}
\label{wiem:tab:sizes-aspect}
\end{table}

There are less IPFV1 than PFV stems because many IPFV1 stems enter into more than one triplet (with different prefixes), and the number of IPFV2 stems is larger than for PFV stems since there happen to be suffix variants. In this study, the latter are neglected, and the number of triplets per period coincides with the number of PFV stems.\largerpage[1]

\subsection{Sampling procedure}\label{wiem:sec:sampling-procedure}

For the participle study we established five subperiods: 1730--1800, 1801--1850, 1890--1918, 1945--1980, 1990--2020. The size of the available corpora (see References) and the periods differed, partially quite considerably (see \tabref{wiem:tab:sizes-subcorpora}). For each period, we drew random samples à 100~tokens of \textit{n/t}-participles of ipfv. and pfv. stems. Not always was this mark reached because of the corpus size, and some of the Russian samples were slightly larger (see Appendix, Part I). A~sample contained no more than 15~items of the same stem; on this account, the randomizing procedure could be violated.

\begin{table}
\begin{tabularx}{\textwidth}{lY@{~~}Y@{~~}Y@{~~}Y@{~~}Y}
\lsptoprule
& 1730--1800 & 1801--1850 & 1890--1918 & 1945--1980 & 1990--2020\\
\midrule
Polish & 3,478,168 & 1,837,844 & 776,277 & 9,100,510 & 284,820,841\\
Russian & 3,227,827 & 14,716,129 & 46,943,219 & 51,398,237 & 114,020,580\\
\lspbottomrule
\end{tabularx}
\caption{Sizes of subcorpora (in tokens of expressions) for periods}
\label{wiem:tab:sizes-subcorpora}
\end{table}

The sampling procedure did not distinguish between the syntactic function, thus predicative participles, e.g.  \REF{wiem:ex:associated}, had the same chance to get into a sample as did participles used as NP-modifiers (``attributive'') or in appositions (``semi-predicative''). The percentage of predicative use per sample varies a lot, but it rarely approaches, or exceeds, \qty{50}{percent} (see Appendix, Part I). However, we will not dwell on biases in syntactic use, apart from correlations between syntactic use and aspect functions (see \sectref{wiem:sec:relation-functions}).\largerpage[1]

Appositive use subsumes cases in which the participle constitutes a clause on its own. Of course, the distinction from, respectively, predicative and attributive use is often troublesome, but, as a rule, appositive use can be distinguished on the following criteria: (a) the participle follows its head noun; (b) it does not denote any stable (inherent) feature (as do characterizing adjectives and participles, usually lexicalized). Compare \REF{wiem:ex:applied} as opposed to \REF{wiem:ex:built}:

\ea\label{wiem:ex:associated}{\gll
{Otóż} {t-a} {reform-a} {nie} {będzi-e} \textit{{kojarzo-n-a}} {z} {nik-im}.\\
\textsc{ptc} this-\textsc{f.nom.sg} reform.\textsc{f.-nom.sg} \textsc{neg} \textsc{fut-3sg} associate.\textsc{ipfv1-pp-f.nom.sg} with nobody-\textsc{ins}\\
\glt ‘Well, this reform will not be \textit{associated} with anyone.’\hfill
$\rightarrow$ predicative
}
\ex\label{wiem:ex:applied}{\gll
{Kar-ą} {\textit{stosowa-n-ą}} {wobec} {żołnierzy} {jest} {także} {areszt} {wojskowy}.\\
punishment.\textsc{f-ins.sg} apply.\textsc{ipfv1-pp-f.ins.sg} towards soldiers be.\textsc{prs.3sg} also arrest.\textsc{m-nom.sg} military.\textsc{m.nom.sg}\\
\glt ‘Military arrest is also a penalty \textit{applied} to soldiers.’ \hfill
$\rightarrow$ appositive
}
\ex\label{wiem:ex:built}{\gll
{Ogląda-my} {wpływ} {\textit{budowa-n-ego}} {przez} {dziesięciolecia} {genius} {loci} {na} {now-ych} {mieszkańc-ów}.\\
watch.\textsc{ipfv-prs.1pl} influence.\textsc{m.acc.sg} build.\textsc{ipfv1-pp-n.gen.sg} over decades genius loci on new-\textsc{acc.sg} inhabitant-\textsc{acc.pl}\\
\glt ‘We see the impact of the genius loci \textit{built} over decades on new inhabitants.’ \hfill
$\rightarrow$ attributive\\\hfill (Polish)
}
\z

\noindent For Polish and Russian, samples were drawn from the respective national corpora (Polish National Corpus/PNC -- \url{http://nkjp.pl/}; RNC -- \url{https://ruscorpora.ru/}). The triplet database served as a means to restrict corpus queries by which sets of IPFV1 stems could be compared to sets of IPFV2 stems, i.e. two sets of ipfv. stems with basically identical lexical meaning (see \sectref{wiem:sec:triplets}). In fact, as for Russian, the triplet database turned out to be the only means to get a handle on certain frequency data from the RNC (see \sectref{wiem:sec:lex-diversity}). We did not, however, compare specific IPFV1 stems to their individual IPFV2 counterparts (or vice versa). This would require a series of case studies that we did not intend to perform.

For Polish, the annotation schema and interface of the PNC allowed us to work with a more diversified array of samples than this was possible for the RNC. Thus, for Polish separate series of samples (à five periods) were drawn from the PNC for IPFV1, PFV and IPFV2 stems that were contained in our database, according to the following guidelines:

\begin{enumerate}[label=(\roman*)]
    \item à 25~stems with the highest token frequencies of \textit{n/t}-participles (or less if the list contained less than 25~stems in the most frequent group);\footnote{The most frequent groups were established on the basis of salient frequency cuts (individually for each sample). In many samples there were not enough stems with a participle $\text{frequency of}>5$.}
    \item 25~stems selected by chance from among stems with a frequency of 1--5 \textit{n/t}-participle tokens. 
\end{enumerate}

\noindent Below these groups are named ``freq(uent)'' and ``infreq(uent)''. In addition, for each stem type and each period we composed (iii) a random sample of \textit{n/t}-participles from just any possible stem (regardless of the frequency of its forms); this sample series served as control. 

As for Russian, samples did not distinguish for different frequency ranges of \textit{n/t}-participles from different stems, since no sufficiently reliable figures required for such a distinction could be obtained from the RNC. We therefore just created (i) random samples of \textit{n/t}-participles of IPFV1 and PFV stems for all periods based on our database and (ii) analogous random samples independently from our database. In addition, we (iii) “skimmed through” all \textit{n/t}-participles we could find for IPFV2 stems (46~tokens, of which 41 belong to the period 1730--1800). These were not considered for inferential statistics.

Therefore, we can compare Russian and Polish \textit{n/t}-participles of IPFV1 and PFV stems. However, while the Polish data, in addition, gives a chance to estimate whether the database-driven samples show any bias in comparison to entirely random samples of \textit{n/t}-participles, in Russian we can only control for possible biases of \textit{n/t}-participles of IPFV1 and PFV stems from the database.\largerpage[-2]

\subsection{Annotation: Aspect functions and syntactic functions}\label{wiem:sec:annotation}

All samples were annotated manually for aspect functions, the aforementioned syntactic functions and polarity. Russian examples were additionally annotated for nominal vs pronominal (``short'' vs. ``long'') form. Their proportions can be inferred from the table in Part III of the Appendix; however, their distribution will not be discussed in this article. As concerns aspect functions and syntactic functions, we used the tag ``\_d"~($=$ doubtful) for the closest acceptable value, since many cases turned out difficult to categorize. Overall, the Russian data contain considerably more cases of doubt in the assignment of aspect function than the Polish data, and a particularly large share of such cases (in both languages) falls on the general-factual function, which often is difficult to distinguish from a habitual or stative function, as defined below (see Parts IV--V of the Appendix). However, even on an account of doubtful cases the general-factual function in Russian begins to predominate over these two other functions only in the later periods (see \sectref{wiem:sec:gen-factVS-hab}), and there is no reason to assume that, on average, there was a bias in favor of any of these more frequent functions. In semantic annotation, such decision problems are well known, and we consider it important to mark ambiguity or problematic cases in the original data, as they supply valuable information on the “edges” of categorial distinctions. However, this issue will not be addressed here, either; instead, problematic cases were integrated into counts according to the value which we accepted as the closest one, and we comment on such problems in passing below. All annotations were thoroughly double-checked by an informed and trained native speaker and by B.~Wiemer.

As concerns aspect functions of pfv. participles, we made a distinction between eventive (a.k.a. actional) and stative (i.e. resultative) use, since these are two crucial meanings distinguished for passives. For ipfv. participles, since they are more interesting in terms of inner-Slavic differentiation, a more diversified array of values was assumed. All of them are widely applied in aspectology and are briefly commented on here, with illustrations from our samples.

\subsubsection{Progressive (PROG)}

Situations can consist of phases. A progressive reading focuses on any internal phase(s), so that boundaries are defocused.\footnote{In \citeposst{Klein1994} terms this means that Topic Time is included in Time of Situation, whereas the eventive meaning implies a limitation and, thus, amounts to the inclusion of Time of Situation in Topic Time.} Note that, in our Russian samples, almost all cases of progressive reading raised doubts, and they need rather strong contextual support \REF{wiem:ex:burnt}.


\ea\label{wiem:ex:loaded-pol}
{Poczekać kilkanaście sekund. W tym czasie pokaże się pasek postępu},\\
\gll \minsp{[} {na} {któr-ym}  {będ-ą} {widoczn-e } \textit{{ładowa-n-e}} {element-y}].\\ 
{} on \textsc{rel-m.loc.sg} \textsc{fut-3pl} visible-\textsc{m.nom.pl} load.\textsc{ipfv1-pp-m.nom.pl} element.\textsc{m-nom.pl}\\
\glt ‘Wait a dozen or so seconds. During this time, the progress bar will show, [on which the items being \textit{loaded} will be visible].’ \\
\hfill(Polish; PNC; 1990--1920)
\z

\ea\label{wiem:ex:burnt}
{Ona volokom pritaščila na odejale drova iz saraja, namjala bumažnyx komkov} (...). {Spički lomalis’ i gasli, potom okazalos’, čto net tjagi},\\
\gll \minsp{[} {ėto} {uže} {kogda} {po} {komnate} {popolz-l-i} {plast-y} {syr-ogo} {dym-a} {i} {von’} {\textit{žže-n-oj}} {bumag-i}].\\
{} \textsc{ptc} already when along room crawl.\textsc{pfv-pst-pl} layer-\textsc{nom.pl} damp-\textsc{gen.sg.m} smoke.\textsc{m-gen.sg} and stink.\textsc{f-nom.sg} burn.\textsc{ipfv1-pp-f.gen.sg} paper.\textsc{f-gen.sg}\\
\glt‘She dragged firewood from the shed on a blanket, crumpled paper wads (...). The matches broke and went out, then it turned out that there was no draft, [this was already when layers of damp smoke and the stink of burnt paper (i.e. the paper being burnt) crawled around the room].’ \\ 
\hfill (Russian; RNC; 1890--1918)
\z




\subsubsection{General-factual (GF)}

See the discussion in \sectref{wiem:sec:recent-accounts}.

\subsubsection{Iterative (ITER)}

Here this term strictly refers to predicates that denote the repetition of an event on a single occasion (i.e. within a larger episode). This repetition can have a restricted count (e.g., \textit{He knocked at the door five times}), the count may be unspecific (\textit{He knocked at the door several times}), or it may be unrestricted (e.g., \textit{He constantly knocked at the door}). In the latter case, it may become difficult to delimit iterative from progressive meaning.

Properly iterative use of participles in passives is extremely rare. In our samples we spotted only a handful of doubtful cases that might also be analyzed as progressive \REF{wiem:ex:knocked} or general-factual \REF{wiem:ex:tri-said}.

\ea\label{wiem:ex:knocked} 
\gll {Wśród} {stukotu} {\textit{obija-n-ych}} {garnk-ów} {i} {talerz-y} {klientk-i} {dobiera-ł-y} {pokrywk-i} {do} {rondl-i}  {albo} {talerzyk-i} {do} {filiżanek}.\\
among clatter knock.on.\textsc{ipfv2-pp-m.gen.pl} pot.\textsc{m-gen.pl} and plate.\textsc{m-gen.pl} customer.\textsc{f-nom.pl} choose.\textsc{ipfv2-pst-pl.nvir} lid.\textsc{f-acc.pl} 
to saucepant.\textsc{m-gen.pl} or plate.\textsc{m-acc.pl} to cup.\textsc{f-gen.pl}\\
\glt ‘Amid the clatter of \textit{knocked} pots and plates, customers chose pot lids or plates for cups.’\hfill (Polish; PNC; 1990--2020)
\z

\ea\label{wiem:ex:tri-said}
-- {Ne mel’teši, Mixalyč,} -- {proburčal Balandin.}\\
\gll -- \minsp{[} {Triždy} {už} \textit{{govore-n-o}}].\\
{} {} thrice already say.\textsc{ipfv1-pp-sg.n}\\
{Čego opjat’ nakačivaeš’}?\\
\glt ‘-- Don’t flicker, Mikhalych, Balandin muttered. \\-- [It \textit{has been said} already three times]. Why are you pumping up again?’\\
\hfill (Russian; RNC; 2004)
\z 

\subsubsection{Habitual (HAB)}

Habitual meanings occupy central stage in typologies of event-external pluractionality.\footnote{\label{wiem:foot:WeFollowSystem}We follow systematic classifications and their foundation, as in \citet{Cusic1981}, \citet{Xrakovskij1997}, \citet{Mattiola2019}, and, first of all, \citet{Šluinskij2005,Šluinskij2006}.} They mark unlimited repetitions without an account of external boundaries. What is “counted” is not subintervals within one episode, but the episodes themselves, and this count is unspecific.\footnote{Notably, habitual situations need not be regular; in fact, more often than not they are irregular. This applies also to the meaning of \textsc{always} and \textsc{never}, which support habitual readings.} Habitual readings as such are insensitive to actionality distinctions, i.e. to whatever is represented as repeated in an unspecified number of occurrences.\largerpage[-1]

\ea\label{wiem:ex:played}{{No nam užasno nravilos’ slušat’, i} \\
\gll \minsp{[} {ja} {do} {six} {por} {ne} {mog-u} {ravnodušno} {slyša-t’} \textit{{igra-nn-ye}} {e-ju} {p’esk-i}].\\
{} I.\textsc{nom} until this moment \textsc{neg} can-\textsc{prs.1sg} indifferently hear.\textsc{ipfv1-inf} play.\textsc{ipfv1-pp-pl} \textsc{3sg.f-ins} piece-\textsc{nom.pl}\\
\glt ‘But we really enjoyed listening, and [I still cannot indifferently hear the pieces played by her (i.e. which she used to play, which becomes evident from the broader context)].’\hfill (Russian; RNC; 1952--1971)
}
\z

\ea\label{wiem:ex:washed}{
{Ona byla perevedena v konservatorskuju studiju} --\\
\gll \minsp{[} {k} {načinajušč-im} {ščenk-am} {s} {dran-ymi} {nosk-ami} {i} {redko} {\textit{my-t-ymi}} {griv-ami} {do} {pleč}].\\
{} to beginning-\textsc{dat.pl} puppy-\textsc{dat.pl} with tattered-\textsc{ins.pl} sock-\textsc{ins.pl} and seldom wash.\textsc{ipdv1-pp-ins.pl} mane-\textsc{ins.pl} up.to shoulder-\textsc{gen.pl}\\
\glt ‘She was transferred to a conservatory studio -- [to beginner puppies with tattered socks and rarely washed shoulder-length manes].’ \\
\hfill (Russian; RNC; 2003)
}
\z

\ea\label{wiem:ex:provided}{
{Oprócz tego MOP rozwinęła szeroką działalność naukowo-badawczą i wydawniczą (\dots)},\\
\gll \minsp{[} {rosn-ą} {też} {rozmiar-y} {pomoc-y} {techniczn-ej} \textit{{świadczo-n-ej}} {przez} {organizacj-ę} {kraj-om} {Trzeci-ego} {Świat-a}].\\
{} grow.\textsc{ipfv1-prs.3pl} also size.\textsc{m-nom.pl} help.\textsc{f-gen.sg} technical-\textsc{f.gen.sg} provide.\textsc{ipfv1-pp-f.gen.sg} through organization.\textsc{f-acc.sg} country.\textsc{m-dat.pl} third-\textsc{gen.sg.m} world.\textsc{m-gen.sg}\\
\glt ‘In addition, the ILO has developed extensive research and publishing activities (…), [and the size of technical assistance \textit{provided} by the organization to Third World countries is also growing].’\\ 
\hfill (Polish; PNC; 1945--1980)
}
\z

\subsubsection{Stative (STAT)}\largerpage[1.75]

Stative meanings capture situations without any boundaries and without any (sub)intervals. The latter property distinguishes states from habitual situations.\footnote{Examples are often ambiguous between a stative and a habitual reading for the reason that they do not contain a clear indication of interval properties. While we cannot dwell on this issue here, it does not much affect the statistical figures presented below, since HAB and STAT are anyway the most frequent functions of ipfv. participles (see \sectref{wiem:sec:relation-functions}) and there is no reason why, in such ambiguous cases, the annotation might have been biased toward either HAB or STAT.\label{wiem:foot:ExamplesAreOften}} However, states may change. A particular case is resultative states.


\ea\label{wiem:ex:caused}{ 
\gll \minsp{[} {List} {swój} {napisa-ł-a} \textit{{powodowa-n-a}} {żal-em}], \\
{} letter.\textsc{m-acc.sg} \textsc{possref-m.acc.sg} write.\textsc{pfv-pst-f.sg} cause.\textsc{ipfv1-pp-f.nom.sg} regret.\textsc{m-ins.sg} \\
\glt
{że w moim przewodniku po Puszczy kampinoskiej, nie znalazła wzmianki o wsi swojego dzieciństwa i młodości}.\\
‘[She wrote her letter out of regret (lit. \textit{caused} by regret)] that in my guide to the Kampinoski Forest she found no mention of the village of her childhood and youth.’ \hfill (Polish; PNC; 1990--2020)
}
\ex\label{wiem:ex:covered}{ \gll
{Ėt-ot} {svoeobrazn-yj} {gazov-yj} {ballon} {soedinja-et-sja} {so} {stvol-om}, \textit{{perekry-t-ym}} {diafragm-oj}.\\
this-\textsc{m.nom.sg} peculiar-\textsc{m.nom.sg} gas-\textsc{m.nom.sg} cylinder.\textsc{m-nom.sg} connect.\textsc{ipfv2-prs.3sg-refl} with barrel.\textsc{m-ins.sg} occlude.\textsc{pfv-pp-ins.sg} diaphragm.\textsc{f-ins.sg} \\
\glt ‘This kind of gas cylinder is connected to the barrel, \textit{occluded} with a diaphragm.’ \hfill (Russian; RNC; 1974)
}
\ex\label{wiem:ex:cleaned}{ \gll
\minsp{[} {Kovrov-ye} {dorožk-i} {kazenno} {unyl-ogo} {cvet-a} \\
{} carpet-\textsc{nom.pl} pathway-\textsc{nom.pl} state-owned dull-\textsc{gen.sg} colour.\textsc{m-gen.sg}\\
\gll {\textit{ne}} {\textit{čišče-n-y}} {i} {sbi-t-y}],\\
\textsc{neg} clean.\textsc{ipfv1-pp-pl} and knock.down.\textsc{pfv-pp-pl}\\
{tam i sjam vidny zatoptannye okurki.}\\
\glt ‘[The carpets of the official dull color are \textit{not cleaned} and knocked down], here and there trampled cigarette butts are visible.’ \hfill (Russian; RNC; 2004)
}
\ex\label{wiem:ex:divided}{ \gll
{Jug} {moskovsk-oj} {ojkumen-y} \textit{{dele-n}} {na} {gorn-yj} {jugo-zapad} {i} {ravninn-yj} {jugo-vostok}.\\
south.\textsc{m-nom.sg} Moscow-\textsc{f.gen.sg} ecumene.\textsc{f-gen.sg} divide.\textsc{ipfv1-pp-sg.m} on mountainous-\textsc{m.acc.sg} south-west.\textsc{m-acc.sg} and flat-\textsc{m.acc.sg} south-east.\textsc{m-acc.sg} \\
\glt ‘The south of the Moscow ecumene is \textit{divided} into mountainous southwest and flat southeast.’ \hfill (Russian; RNC; 2005)
}
\ex\label{wiem:ex:loaded-ru}{ \gll
{Tak-oj} {malen’k-ij}, {a} {tašč-it} {na} {buksir-e} {dv-e} {ogromn-ye} {barž-i}, \textit{gruže-nn-ye} {tes-om}.\\
such-\textsc{m.nom.sg} small-\textsc{m.nom.sg} but drag-\textsc{prs.3sg} on tow.\textsc{m-loc} two-\textsc{f.acc} huge-\textsc{acc.pl} barge-\textsc{acc.pl} load.\textsc{ipfv1-pp-acc.pl} batten.\textsc{m-ins.sg}\\
\glt ‘So small, and drags in tow two huge barges, \textit{loaded} with boards.’ \\
\hfill (Russian; RNC; 1959)
}
\z

\section{Findings}\label{wiem:sec:findings}

We start from an account of lexical diversity (\sectref{wiem:sec:lex-diversity}), then turn to frequency relations between \textit{n/t}-participles and the general amount of grammatical forms (\sectref{wiem:sec:correlation}) over to the relation between aspect and syntactic functions (\sectref{wiem:sec:relation-functions}), before we dwell on some more specific issues (\sectref{wiem:sec:specific-issues}).

\subsection{Lexical diversity}\label{wiem:sec:lex-diversity}

In order to get an idea of how well-represented are grammatical means (or constructions) in a language (at some stage), it seems useful to assess their productivity, i.e. their spread among the stock of lexical units to which they apply. We will call this spread ‘lexical diversity’. Since Slavic aspect is based on oppositions between stems (\sectref{wiem:sec:intro}), we may count IPFV1, PFV and IPFV2 stems as lexical units to which participle suffixes, as grammatical means, apply. Our data allows for three, rather crude, ways to approximate lexical diversity (LD).

The first approach rests on type/token ratios, i.e. on coefficients between the number of different stems ($=$ types) and the number of tokens in each sample. The value of type/token ratios varies between 0 and 1; the higher the value, the more diversified the number of stems which made it into the sample. \tabref{wiem:tab:type-tok-rus} provides the figures for the Russian samples.

\begin{table}[ht]
\begin{tabularx}{\textwidth}{lYYYYc}
\lsptoprule
\multicolumn{6} {c} {n/t} \\\cmidrule(lr){2-6}
& \multicolumn{2} {c} {IPFV1} & \multicolumn{2} {c} {PFV} & IPFV2\\\cmidrule(lr){2-3}\cmidrule(lr){4-5}\cmidrule(lr){6-6}
& triplets & control & triplets & control & all that could be found\\\midrule
1730--1780 & 0.21 & 0.40 & 0.24 & 0.69 & 0.67 (48~tokens)\\
1801--1850 & 0.19 & 0.38 & 0.24 & 0.78 & only 4~tokens\\
1890--1918 & 0.17 & 0.41 & 0.23 & 0.81 & only 1~token\\
1945--1980 & 0.18 & 0.42 & 0.20 & 0.86 & $-$\\
1990--2020 & 0.19 & 0.50 & 0.21 & 0.81 & $-$\\
\lspbottomrule
\end{tabularx}
\caption{Type/token ratio of stems with participles -- Russian}
\label{wiem:tab:type-tok-rus}
\end{table}

Through all periods, the LDs are considerably higher in the control samples (which is natural since the choice of stems was not restricted by the triplet database), and among the control samples they are higher for PFV stems. Moreover, the figures are stable over time, except a steady increase for the PFV control samples, with a slight decrease in the last period. Moreover, we see that \textit{n/t}-participles were derived from IPFV2 stems still in the 18\textsuperscript{th} century and that, despite being infrequent, for that period their type/token ratio was comparable to the ratio of pfv. \textit{n/t}-participles. Afterwards they drastically declined and virtually disappeared altogether (see further \sectref{wiem:sec:rus-particip-ipf2}).\largerpage[-1]

\tabref{wiem:tab:type-tok-pol} shows the figures of the Polish samples.

\begin{table}[h]
\begin{tabularx}{\textwidth}{l@{~~}Y@{~}Y@{~}Y@{~~~}Y@{~}Y@{~}Y@{~~~}Y@{~}Y@{~}Y}
\lsptoprule
\multicolumn{10} {c} {n/t} \\\cmidrule(lr){2-10}
& \multicolumn{3} {c} {IPFV1} & \multicolumn{3} {c} {IPFV2} & \multicolumn{3} {c} {PFV}\\\cmidrule(lr){2-4}\cmidrule(lr){5-7}\cmidrule(lr){8-10}
& freq & infreq & ctrl & freq & infreq & ctrl & freq & infreq & ctrl\\\midrule
1730--1780 & 0.23 & 0.63 & 0.58 & 0.23 & 1.00 & 0.62 & 0.25 & 0.65 & 0.83\\
1801--1850 & 0.22 & 0.73 & 0.74 & 0.33 & 1.00 & 0.69 & 0.24 & 0.71 & 0.92\\
1890--1918 & 0.23 & 0.43 & 0.60 & 0.23 & 0.80 & 0.64 & 0.26 & 0.43 & 0.95\\
1945--1980 & 0.23 & 0.50 & 0.64 & 0.29 & 0.50 & 0.64 & 0.22 & 0.53 & 0.83\\
1990--2020 & 0.25 & 0.59 & 0.39 & 0.19 & 0.38 & 0.57 & 0.24 & 0.31 & 0.66\\
\lspbottomrule
\end{tabularx}
\caption{Type/token ratio of stems with participles -- Polish}
\label{wiem:tab:type-tok-pol}
\end{table}

Among the freq-samples (for items restricted by the database) the type/token ratios are rather stable, and consistently so, in the range 19--33, mostly around 25--26, whereas in the infreq-samples the ratios are considerably higher, but they also vary over larger ranges, with higher values in the earlier periods particularly for PFV and IPFV2 stems. With the infreq-samples, IPFV2 stems even reach the possible maximum type/token ratio of~1, but it then drops drastically, and toward the last period the ratio of IPFV1 stems outruns the ratio of IPFV2 stems. As for the control groups, the ratios of PFV stems are consistently higher than for ipfv. stems. A decrease of the ratios can be observed for all control samples in the last period, quite drastically for IPFV1 and PFV stems, while the ratios of IPFV2 keep their level more or less over all periods.

The samples (already small as such) differ a lot as for their size. We therefore additionally calculated Herdan’s Index, a type-token measure which is less sensitive to different sizes between samples, since it is based on the natural logarithms of raw figures (cf. \citealt[523]{Panas2011}). However, this measure yielded the same relations between the coefficients as did the simple type\slash token measure presented in Tables~\ref{wiem:tab:type-tok-rus} and~\ref{wiem:tab:type-tok-pol} (see Appendix, Part II).

Regardless, we should be aware that type/token ratios need not have decreased as such. Simply, the subcorpora of the later periods have a larger size, and in drawing random samples, frequent units have a better chance of making it into the sample more often. On the other hand, larger corpora also supply better chances for rare phenomena (e.g., stems with participles) to get into the sample. To draw more reliable conclusions about productivity, hapax phenomena would be more telling (cf. \citealt{Baayen2009}). However, “fishing” them out would require an entirely different access to corpus data (which was not available).\largerpage[-1]

A way of getting an idea of productivity which comes closer to looking for hapaxes is the following. We can approach the LD of \textit{n/t}-participles by asking for the proportion of stems which have at least one such participle form in relation to the overall amount of stems (= lexical units) in the corpus. This amounts to asking for the type frequency of \textit{n/t}-participles in the corpus. We could not do that for the whole population of stems in the corpora, but we calculated these proportions (per period) for IPFV1, PFV and IPFV2 stems that are included in the triplet database. \tabref{wiem:tab:type-freq-stem} provides the relevant figures.

\begin{table}[p]
\begin{tabularx}{\textwidth}{l@{~~}l@{~~}Y@{~~}Y@{~~}Y@{~~}Y@{~~}Y}
\lsptoprule
&& 1730--1780 & 1801--1850 & 1890--1918 & 1945--1980 & 1990--2020\\
\midrule
\multicolumn{2}{l}{Polish}\\
&\textit{n/t} number of IPFV1 ($=\qty{100}{\percent}$)& 798 & 694 & 1,018 & 958 & 1,275\\
&stems with at least one& 268 & 144& 315 & 280 & 633\\
&participle token &(\qty{33.6}{\percent}) &(\qty{20.7}{\percent})&(\qty{30.9}{\percent}) & (\qty{29.2}{\percent}) & (\qty{49.6}{\percent})\\\addlinespace
&number of PFV ($=\qty{100}{\percent}$)& 851 & 832 & 1,274 & 1,316 & 1,721\\
&stems with at least one& 480 & 405 & 749 & 648 & 1,138\\
&participle token &(\qty{56.4}{\percent}) &(\qty{48.7}{\percent}) &(\qty{58.8}{\percent}) &(\qty{49.2}{\percent}) &(\qty{66.1}{\percent})\\
\addlinespace
&number of IPFV2 ($=\qty{100}{\percent}$)& 372 & 391 & 661 & 565 & 1,009\\
&stems with at least one& 46 & 41 & 132 & 92 & 395\\
&participle token &(\qty{12.4}{\percent}) &(\qty{10.5}{\percent}) &(\qty{20.0}{\percent}) &(\qty{16.3}{\percent}) &(\qty{39.1}{\percent})\\
\midrule
\multicolumn{2}{l}{Russian}\\
&\textit{n/t} number of IPFV1 ($=\qty{100}{\percent}$)& 537 & 692 & 762 & 804 & 810\\
&stems with at least one& 126 & 185 & 244 & 261 & 290 \\
&participle token &(\qty{23.5}{\percent}) &(\qty{26.7}{\percent}) &(\qty{32.0}{\percent}) &(\qty{32.5}{\percent}) &(\qty{35.8}{\percent})\\
\addlinespace
&number of PFV ($=\qty{100}{\percent}$)& 664 & 1,028 & 1,170 & 1,207 & 1,438\\
&stems with at least one& 333 & 549 & 657 & 718 & 897\\
&participle token &(\qty{50.2}{\percent}) &(\qty{53.4}{\percent}) &(\qty{56.2}{\percent}) &(\qty{59.5}{\percent}) &(\qty{62.4}{\percent})\\\addlinespace
&number of IPFV2 ($=\qty{100}{\percent}$) & 389 & 606 & 793 & $-$ & $-$\\
&stems with at least one& 41& 4& 1& $-$ & $-$\\
&participle token &(\qty{10.5}{\percent}) &(\qty{0.7}{\percent}) &(\qty{0.1}{\percent})\\
\lspbottomrule
\end{tabularx}
\caption{Type frequencies of stems (from database) with at least one \textit{n/t}-participle}
\label{wiem:tab:type-freq-stem}
\end{table}


As for Polish, there is no clear tendency, but PFV stems are most and IPFV2 are least productive. Russian PFV stems are more productive than IPFV1 stems with \textit{n/t}-participles as well, and their figures are comparable to those of Polish PFV stems. However, in Russian both PFV and IPFV1 stems reveal a slight, but steady increase of productivity over time. Remember, however, that these observations are exclusively based on triplets.

A third approach toward LD is by calculating the mean proportions of participle token frequencies in relation to the token frequencies of the remainder of grammatical forms for each stem type (IPFV1, PFV, IPFV2). As a shortcut, these values may be dubbed ``proportional frequencies''. Again, this calculation was possible only for the units included in the triplet database.

Tables~\ref{wiem:tab:means-proportion-pol} and \ref{wiem:tab:means-proportion-ru} present the relevant figures. We start with Polish.

\begin{table}[ht]
\begin{tabularx}{\textwidth}{lS@{}S@{}S@{}S@{}S}
\lsptoprule
&\multicolumn{1}{c}{1730--1780} & \multicolumn{1}{c}{1801--1850} & \multicolumn{1}{c}{1890--1918} & \multicolumn{1}{c}{1945--1980} & \multicolumn{1}{c}{1990--2020}\\
\midrule
IPFV1 \textit{n/t}& 0.09 & 0.052 & 0.07 & 0.075 & 0.07\\
PFV \textit{n/t}& 0.26 & 0.23 & 0.26 & 0.23 & 0.23\\
IPFV2 \textit{n/t}& 0.03 & 0.027 & 0.03 & 0.032 & 0.038\\
\lspbottomrule
\end{tabularx}
\caption{Means of proportions between \textit{n/t}-participles and the rest of forms (Polish)}
\label{wiem:tab:means-proportion-pol}
\end{table}

\tabref{wiem:tab:means-proportion-pol} testifies to a solid time stability of the mean token frequency of Polish \textit{n/t}-participles for PFV and IPFV2 stems, only slightly less for IPFV1 stems. With PFV stems \textit{n/t}-participles are considerably more frequent than for ipfv. stems. Moreover, IPFV2 stems employ \textit{n/t}-participles about 2--3~times less frequently than do IPFV1 stems. However, Polish IPFV2 \textit{n/t}-participles were by magnitudes more frequent than in Russian already in the 18\textsuperscript{th} century, and their mean frequency has remained stable over time.

\tabref{wiem:tab:means-proportion-ru} supplies the figures for the Russian samples.\footnote{Missing values indicate that corpus hits only contained citations from earlier periods.}

\begin{table}[ht]
\begin{tabularx}{\textwidth}{lS@{}S@{}S@{}S@{}S}
\lsptoprule
&\multicolumn{1}{c}{1730--1780} & \multicolumn{1}{c}{1801--1850} & \multicolumn{1}{c}{1890--1918} & \multicolumn{1}{c}{1945--1980} & \multicolumn{1}{c}{1990--2020}\\
\midrule
IPFV1 \textit{n/t}& 0.12 & 0.09 & 0.09 & 0.10 & 0.09\\
PFV \textit{n/t}& 0.31 & 0.39 & 0.40 & 0.45 & 0.26\\
IPFV2 \textit{n/t}& 0.009 & 0.006 & 0.007 & \multicolumn{1}{c}{$-$} & \multicolumn{1}{c}{$-$}\\
\lspbottomrule
\end{tabularx}
\caption{Means of proportions between \textit{n/t}-participles and the rest of forms (Russian)}
\label{wiem:tab:means-proportion-ru}
% * Corpus hits only contained citations from earlier periods.
\end{table}

We notice a reliable time stability for \textit{n/t}-participles of IPFV1 stems, while such participles of IPFV2 stems were extremely rare already since 1730 and practically ceased to occur after 1918 (and probably earlier); this is why \tabref{wiem:tab:type-freq-stem} lacks figures for the last two periods. However, as Tables~\ref{wiem:tab:type-tok-rus} and \ref{wiem:tab:type-tok-pol} show, in the first period (1730--1780) the type/token ratio of IPFV2 \textit{n/t}-participles was quite high: with 0.67 it was comparable to the ratio of Polish IPFV2 \textit{n/t}-participles of that period (0.62) (\tabref{wiem:tab:type-tok-pol}), simultaneously it was practically as high as of Russian PFV \textit{n/t}-participles of that period and much higher than for IPFV1 \textit{n/t}-participles of any period (\tabref{wiem:tab:type-tok-rus}). We will return to this issue in \sectref{wiem:sec:rus-particip-ipf2}.

As for PFV stems, the means fluctuate over a slightly larger range, also in comparison to the Polish sample series, even though the figures for both PFV and IPFV1 stems in Russian are consistently higher than for the equivalent Polish stems. Simultaneously, to repeat, Polish \textit{n/t}-participles of IPFV2 stems were by magnitudes more frequent than in Russian already in the 18\textsuperscript{th} century, and their mean frequency has remained stable over time.

In general, a comparison of Tables~\ref{wiem:tab:means-proportion-pol} and \ref{wiem:tab:means-proportion-ru} reveals that Russian has been employing \textit{n/t}-participles for passives slightly more frequently than Polish, even with IPFV1 stems. This might seem surprising, however what needs to be checked is the relation of token frequency to lexical diversity. A comparison of Tables~\ref{wiem:tab:type-tok-rus} and \ref{wiem:tab:type-tok-pol} shows that the type/token ratios for IPFV1 stems and, in particular, for PFV stems in Polish are on average higher than in Russian. To the extent that type/token ratios can be understood as a rough indicator of productivity, Russian shows lower productivity than Polish, whereas in terms of proportional frequencies (Tables~\ref{wiem:tab:means-proportion-pol} and \ref{wiem:tab:means-proportion-ru}) the relation is inverse: \textit{n/t}-participles of IPFV1 and PFV stems on average occurred more often in Russian than in Polish.\largerpage[2]

\subsection{Correlation between general token frequency and frequency of \textit{n/t}-participles (Polish)}\label{wiem:sec:correlation}

Let us now look whether, on average, the token frequency of \textit{n/t}-participles for IPFV1, PFV, IPFV2 stems depends on the overall token frequency of forms occurring for these stems in the corpus. The more linear this relation, the stronger the correlation (Pearson’s ρ; cf. \citealt[116--126]{Levshina2015}). We have been able to calculate this correlation only for the Polish samples based on the triplet database.\footnote{The database supplies a large, but manageable amount of verb stems, and the Polish corpora provide annotations which can be used directly for getting the relevant frequency data. The RNC does not provide such annotations and search tools, so that we could not avail ourselves of the relevant frequencies for these correlations.} The results are obvious (see \tabref{wiem:tab:correlation-pol}). First, this correlation is strongest for PFV stems, consistently over all periods; it drops after the first period, but then remains more or less at the same level. Second, for IPFV2 stems the correlation grows almost steadily, and it outruns the correlations for IPFV1 stems in the last three periods. We may take this as an indication of an increasing integration of IPFV2 stems -- and jointly with them of their \textit{n/t}-participles -- into the grammatical system.

\begin{table}[ht]
\begin{tabularx}{\textwidth}{lS@{}S@{}S@{}S@{}S}
\lsptoprule
Pearson’s~ρ & \multicolumn{1}{c}{1730--1780} & \multicolumn{1}{c}{1801--1850} & \multicolumn{1}{c}{1890--1918} & \multicolumn{1}{c}{1945--1980} & \multicolumn{1}{c}{1990--2020}\\
\midrule
IPFV1 & 0.61 & 0.52 & 0.50 & 0.44 & 0.59\\
PFV & 0.86 & 0.73 & 0.71 & 0.60 & 0.73\\
IPFV2 & 0.35 & 0.43 & 0.52 & 0.50 & 0.65\\
\lspbottomrule
\end{tabularx}
\caption{Correlation between token frequencies of \textit{n/t}-participles and all forms (Polish)}
\label{wiem:tab:correlation-pol}
\end{table}

\subsection{Relation between aspect functions and syntactic functions}\label{wiem:sec:relation-functions}

We tested on significance ($\chi^2$ or Fisher’s exact test) and strength of association, or effect size (Cramer’s V, with values ranging between 0~and~1).\footnote{Tests on significance inform about the likelihood that the same correlation would be obtained from other samples; this likelihood is customarily indicated by a \textit{p}-value (the smaller this value, the lesser the probability that the given result has been obtained by chance). In turn, strength of association measures the correlation itself. The latter does not depend on the sample size, whereas significance increases with sample size (cf. \citealt[129--130]{Levshina2015}).} Here only the main results are communicated.\largerpage[2]

As concerns \textit{n/t}-participles of PFV stems, in either language a consistent dominance of the stative function regardless of syntactic status can be observed. In most samples over all periods, this dominance reaches a significance level between $p<0.05\,(\text{*})$ and $p<0.001\,(\text{***})$. Higher \textit{p}-values (i.e. less significance) are due to the eventive function, which shows a bias toward predicative use. Nonetheless, coefficients of eventive:stative function are only rarely higher than~0.5, i.e. the share of eventive \textit{n/t}-participles in a sample is only rarely larger than \qty{33}{\percent}. This happens in the last period of the Polish freq-sample (20:25), for the last period of the Polish control sample (15:11), and in the last period of the Russian control sample (16:25). That is, the relative frequency of eventive use slightly increases over time, but, except for the last period in the Polish control samples, it never prevails over the stative one. Moreover, for the Russian samples Cramer’s V only rarely rises above 0.3, whereas for most Polish samples its value is between 0.4 and 0.7; that is, the correlation is in general stronger in Polish.

As concerns ipfv. \textit{n/t}-participles, we only compared the most frequent aspect functions, STAT and HAB, against the remainder of aspect functions. That is, GF is not the predominant function in either of the languages. This also applies to their predicative use; in Polish, the habitual function proves to be particularly dominant in the predicative use of IPFV2 \textit{n/t}-participles. Admittedly, beside examples with an undoubtedly stative reading (see \REF{wiem:ex:isntpaid}), a larger number of examples turned out difficult to categorize, as e.g. \REF{wiem:ex:isspent}, which might be assigned habitual function, or \REF{wiem:ex:notmauled}, which might be interpreted as GF. See examples for Russian:\largerpage[-2]

\ea\label{wiem:ex:isntpaid}{{(\dots) našu večno bedstvovavšuju prijatel’nicu Ninu Alovert,}\\
\gll \minsp{[} {u} {kotor-oj} {do} {s-ix} {por} {za} {telefon} {\textit{ne}} {\textit{plače-n-o} (\dots)]}.\\
{} by \textsc{rel}-\textsc{f.gen.sg} up.to this-\textsc{gen.pl} time-\textsc{gen.pl} for telephone.\textsc{m-acc.sg} \textsc{neg} pay.\textsc{ipfv1-pp-sg.n}\\
\glt ‘our ever-poor friend Nina Alovert, [who still hasn’t paid (lit. on which it \textit{isn’t paid}) for her phone]’\hfill (Russian; RNC; 1998)
}
\ex\label{wiem:ex:isspent}{\gll
\minsp{[} {Skol’ko} {sil} {na} {«Orfej-a»} {\textit{trače-n-o}}],\\
{} how.much power-\textsc{gen.pl} on Orpheus.\textsc{m-acc.sg} spend.\textsc{ipfv1-pp-sg.n}\\
{\\a čego-to ėtoj tragedii vsegda ne dostavalo!}\\
\glt ‘[How much effort had been spent (lit. \textit{is spent}) on “Orpheus”], but something was always lacking in this tragedy!’\hfill (Russian; RNC; 2010)
}
\ex\label{wiem:ex:notmauled}{{Pavlo vskočil, paren’ molodoj, krov’ svežaja,}\\
\gll \minsp{[} {lagerj-ami} {ešče} {\textit{ne}} {\textit{trepa-n}]},\\
{} camp-\textsc{ins.pl} yet \textsc{neg} maul.\textsc{ipfv1-pp-sg.m}\\
{na galuškax ukrainskix rjažka ot"edennaja.}\\
\glt‘Pavlo sprang to his feet, a young lad with fresh blood, [\textit{not} yet \textit{mauled} by the camps], used to stuff his face with Ukrainian dumplings.’ \\ 
\hfill (Russian; RNC; 1961)
}
\z

\noindent For Polish, analogous cases could be adduced.

On the other hand, there is also a considerable number of examples for which GF has been assigned only with doubts, as in \REF{wiem:ex:notcleaned}--\REF{wiem:ex:said}:

\ea\label{wiem:ex:notcleaned}{{Žal’ tol’ko,}\hfill GF or STAT?\\ 
\gll \minsp{[} {čto} {ona} {davno} {\textit{ne}} {\textit{čišče-n-a}}].\\
{} \textsc{comp} she.\textsc{nom.sg} long.ago \textsc{neg} clean.\textsc{ipfv1-pp-sg.f}\\
\glt ‘It is only a pity [that it (the scapula) has \textit{not} been (lit. is not) \textit{cleaned} for a long time].’\hfill (Russian; RNC; 2011)
}
\z

\ea\label{wiem:ex:said}{{A naši ženščiny vse ravno byli v nas,}\hfill GF or HAB?\\
\gll \minsp{[} {i} {skol’ko} {\textit{by-l-o}} {o} {nix} {\textit{govore-n-o}}].\\
{} and how.much be-\textsc{pst-sg.n} about \textsc{3.loc.pl} say.\textsc{ipfv1-pp-sg.n}\\
\glt ‘And our women were all the same in us, [and how much \textit{was said} about them].’\hfill (Russian; RNC; 1990--1996)
}
\z

\noindent There is no reason to assume that such hesitations in the assignment of aspect functions have skewed their general distribution in the samples toward any of these functions (GF vs HAB or STAT); see also Footnote~\ref{wiem:foot:ExamplesAreOften}.

Regardless, as concerns \textit{n/t}-participles of IPFV1 stems, results on different levels of significance ($\text{from}{}<0.05\text{*~to}{}<0.001\text{***}$) for Polish are found in most periods of the freq- and the control samples, with Cramer’s V between~0.24 and~0.47 (except in the first period of the control sample: $\text{V}=0.53$). In general, STAT is more closely associated to attributive use, but in the control samples we also observe an extreme preference of HAB in predicative use; compare \REF{wiem:ex:produced}--\REF{wiem:ex:judged}.

\ea\label{wiem:ex:produced}{\gll
{W} {świecie} {już} {od dawna} {są} {\textit{produkowa-n-e}} {różn-ego} {rodzaj-u} {ciągadł-a} {ciśnieniow-e} {o} {podobn-ej} {zasadzi-e} {działani-a.}\\
in world already since.long be.\textsc{prs.3pl} produce.\textsc{ipfv1-pp-nom.pl} various-\textsc{m.gen.sg} kind.\textsc{m-gen.sg} die.\textsc{n-nom.pl} of.pressure-\textsc{n.nom.pl} of similar-\textsc{gen.sg.f} principle.\textsc{f-gen.sg} action.\textsc{n-gen.sg}\\
\glt ‘Various types of pressure dies with a similar principle of operation have been \textit{produced} in the world for a long time.’\hfill (Polish; PNC; 1945--1980)
}
\z

\ea\label{wiem:ex:judged}{{Ci zaś, którzy jak dzieci na wielkanocnych plackach odłubują tylko rodzynki życia i zjadają,}\\
\gll \minsp{[} {c-i} {wiecznie} {za} {dziec-i} {będ-ą} {mia-n-i} {i} {\textit{sądze-n-i}} {jako} {dziec-i.}]\\
{} this-\textsc{vir.nom.pl} eternally for child.\textsc{n-acc.pl} \textsc{fut-3pl} have.\textsc{ipfv1-pp-vir.nom.pl} and judge.\textsc{ipfv1-pp-vir-nom.pl} as child.\textsc{n-nom.pl}\\
\glt ‘And those who, like children on Easter cakes, pick only the raisins of life and eat them, [those will eternally be considered children and be \textit{judged} as children].’\hfill (Polish; PNC; 1801-1850)
}
\z

\noindent The samples of Russian IPFV1 \textit{n/t}-participles produced highly significant \textit{p}-values for almost all samples, except the control samples, among which there is no clear tendency for the stative function (nor more remarkable values of Cramer’s V). For the triplet-based samples, we observe a clear association between stative function and attributive use only in the last two periods; see \REF{wiem:ex:whitewashed}.

\ea\label{wiem:ex:whitewashed}
\gll
{Grubo} {\textit{bele-nn-ye}} {sten-y} {ne} {kaza-l-i-s’} {sliškom} {goly (\dots).}\\
roughly whiten.\textsc{ipfv1-pp-pl} wall-\textsc{nom.pl} 
\textsc{neg} seem.\textsc{ipfv-pst-pl-refl} too.much naked-\textsc{nom.pl}\\
\glt‘The roughly \textit{whitewashed} walls did not seem too bare.’ \\
\hfill (Russian; RNC; 1996--1997)
\z

\noindent The samples of Polish IPFV2 \textit{n/t}-participles yielded results on significance levels ${}<0.5\,(\text{*})~\text{to}<0.001\,(\text{***})$ in all freq-samples and in all control samples but the last, i.e. this period does not show a clear distribution. In general, STAT tends towards attributive and appositive use (see (\ref{wiem:ex:sliced}--\ref{wiem:ex:wettened})), but HAB does not show any clear preference, except for the last period in the freq-samples. The infreq-samples yielded no clear results.\largerpage[-1]


\ea\label{wiem:ex:sliced}
{Po ukończeniu obrządku,}\\
\gll \minsp{[} {z} {jednej} {strony} {gospodarz}, {a} {z} {drugiej} {gospodyn-i} {dom-u} {traktowa-l-i} {wszystk-ich} {z} {kolei} {\textit{pokraja-n-em}} {jaj-em}] \\
{} from one side housekeeper.\textsc{m-nom.sg}
and from other housekeeper.\textsc{f-nom.sg} house.\textsc{m-gen.sg}
treat.\textsc{ipfv1-pst-pl} everybody-\textsc{acc.pl} from turn
slice.\textsc{ipfv2-pp-n.ins.sg} egg.\textsc{n-ins.sg} \\
{i nie opuszczali najmniejszeg dziecka.} \\
\glt ‘After completing the rite, [the housekeeper, from one side, and his wife, from the other, treated everybody with a \textit{sliced} egg], and they did not omit even the smallest child.’\hfill (Polish; PNC; 1801--1850)
\ex\label{wiem:ex:wettened}
\gll {(...)} {zaraz} {zatamowa-ł-a} {krew} {zapomocą} {bibuł-y}, {\textit{umacza-n-ej}} {w} {jakimś} {płynie} {gryzącym}.\\
{} immediately block.\textsc{pfv-pst-sg.f} blood.\textsc{acc} with.aid blotting.paper.\textsc{f-gen} wetten.\textsc{ipfv2-pp-f.gen.sg} in some liquid biting\\
\glt ‘(\dots) she immediately blocked the blood with blotting paper, dipped (lit. \textit{wettened}) in some acrid liquid.’\hfill (Polish; PNC; 1890--1918)
\z

\subsection{Specific issues}\label{wiem:sec:specific-issues}

In the remainder, we will discuss aspect functions of ipfv. \textit{n/t}-participles in both languages (\sectref{wiem:sec:gen-factVS-hab}) and the fate of \textit{n/t}-participles of IPFV2 stems in Russian (\sectref{wiem:sec:rus-particip-ipf2}).

\subsubsection{General-factual vs. habitual and progressive function of ipfv. \textit{n/t}-participles}\label{wiem:sec:gen-factVS-hab}

As stated in \sectref{wiem:sec:relation-functions}, GF has not turned out a predominant aspect function of ipfv. \textit{n/t}-participles for most of the samples over the periods, and this also holds for Russian. This may surprise given the prominence ascribed to this function in research dealing with ipfv. \textit{n/t}-participles in contemporary Russian, which has concentrated on predicative use (see \sectref{wiem:sec:recent-accounts}). Here we discuss GF together with PROG on the background of HAB, and we start with Russian.

As \tabref{wiem:tab:hab-gf-prog-rus} shows, the token frequency of progressive readings has always been low, and toward the present period it approaches zero. Only in one sample of the 1\textsuperscript{st} and of the 2\textsuperscript{nd} period did PROG prevail over GF, the predominance of the latter increases toward the current period. GF also increases in comparison to HAB, but only during the last two periods has it taken dominance over HAB. Thus, its salience mentioned in the aforementioned research appears to be recent.\footnote{We use ``na'' if none of the functions is attested. If only one of the compared functions is attested, GF, HAB or PROG, respectively.}\largerpage[-2]

\begin{table}\small
\begin{tabular}{lcccccccccc}
\lsptoprule
& \multicolumn{2} {c} {1730--1780} & \multicolumn{2} {c} {1801--1850} & \multicolumn{2} {c} {1890--1918} & \multicolumn{2} {c} {1945--1980} & \multicolumn{2} {c} {1990--2020} \\\cmidrule(lr){2-3}\cmidrule(lr){4-5}\cmidrule(lr){6-7}\cmidrule(lr){8-9}\cmidrule(lr){10-11}
& tripl & ctrl & tripl & ctrl & tripl & ctrl & tripl & ctrl & tripl & ctrl\\\midrule
HAB & 26 & 48 & 19 & 51 & 13 & 32 & 1 & 15 & 0 & 24\\
GF & 22 & 2 & 8 & 11 & 9 & 17 & 12 & 40 & 25 & 23\\
PROG & 5 & 5 & 10 & 7 & 3 & 5 & 1 & 6 & 0 & 0\\
coeff.& 4.4 & 0.4 & 0.8 & 1.6 & 3.0 & 3.4 & 12.0 & 6.7 & GF & GF\\
GF/PROG\\
coeff.& 5.2 & 9.6 & 1.9 & 7.3 & 4.3 & 6.4 & 1.0 & 2.5 & na & HAB\\
 HAB/PROG\\
$\sum\text{sample}$ & 108 & 97 & 119 & 100 & 100 & 100 & 142 & 90 & 109 & 100\\
\lspbottomrule
\end{tabular}
\caption{HAB, GF, PROG for Russian IPFV1 \textit{n/t}-participles}
\label{wiem:tab:hab-gf-prog-rus}
% \begin{minipage}{12cm}
% \small  \textbf{Remarks}\\
% - ‘na’, if none of the functions is attested.\\
% - If only one of the compared functions is attested, GF, HAB or PROG, respectively.
% \end{minipage}
\end{table}

As concerns PROG, even the few examples found in 1945--1980 raise doubts, as their temporal reference is not entirely clear and they can also be assigned GF, perhaps even ITER (see \ref{wiem:ex:pronounced}--\ref{wiem:ex:stewed}). In other cases, one can argue for assigning STAT, also for earlier periods, as in \REF{wiem:ex:alarmed}:

\ea\label{wiem:ex:pronounced}{{Posle ėtogo professora Universiteta v vide protesta ustroili obed v čest’ Mendeleeva,}\\
\gll \minsp{[} vo vremja {kotor-ogo} {\textit{govore-n-y}} {\textit{by-l-i}} {sootvetstvujušč-ie} {reč-i}].\\
{} at time which-\textsc{m.gen.sg} say.\textsc{ipfv1-pp-pl} be-\textsc{pst-pl} appropriate-\textsc{nom.pl} speech-\textsc{nom.pl}\\
\glt ‘After that, as a protest, the professors of the University organized a dinner in honor of Mendeleev, [during which appropriate speeches \textit{were pronounced}].’\hfill (Russian; RNC; 1968)
}
\ex\label{wiem:ex:stewed}{{S utra do noči stol lomilsja ot edy i vina -- ot lobii, sacivi, žarenoj ryby loko, (\dots)} \\
\gll \minsp{[} {glinjan-yx} {goršočk-ov} {s} {\textit{tuše-nn-ym}} {v} {ostr-yx} {prjanostj-ax} {mjas-om}].\\
{} clay-\textsc{gen.pl} pot.\textsc{m}-\textsc{gen.pl} with stew.\textsc{ipfv1-pp-ins.sg} in pungent-\textsc{loc.pl} spice-\textsc{loc.pl} meat.\textsc{m-ins.sg}\\
\glt ‘From morning to night, the table was full of food and wine -- from lobia, satsivi, fried loco fish (…), [clay pots with meat \textit{stewed} in hot spices].’ \\
\hfill (Russian; RNC; 1963)
}
\ex\label{wiem:ex:alarmed}{\gll
\minsp{[} {Siloslav}, {bud-uči} {\textit{trevož-en}} {i-mi}],\\
{} \textsc{pn.m-nom.sg} be-\textsc{cvb} alarm.\textsc{ipfv1-pp.sg.m} \textsc{3-ins.pl}\\ 
{govoril volšebnice, čto ne možet probyt’ tut ni odnoj minuty.}\\
\glt ‘[Siloslav, \textit{alarmed} by them], told the sorceress that he could not stay here for a single minute.’\hfill (Russian; RNC; 1766--1768)
}
\z

\noindent A few clear examples of progressive use can be found in the samples of the first two periods; compare \REF{wiem:ex:led}--\REF{wiem:ex:felt}:

\ea\label{wiem:ex:led}{\gll
\minsp{[} {Kogda} {ja} {\textit{by-l}} {\textit{vede-n}} {na} {kazn’}],\\
{} when \textsc{1sg.nom} be-\textsc{pst-sg.m} lead-\textsc{pp-sg.m} to execution.\textsc{f-acc.sg}\\
{to šestvie moe bylo takim obrazom.}\\
\glt ‘[When I \textit{was (being) led} to execution], my procession was like this.’\\
\hfill (Russian; RNC; 1766--1768)
}
\ex\label{wiem:ex:felt}{{Ne mogu opisat’ vamъ radosti,}\\ 
\gll \minsp{[} {\textit{čuvstvova-nn-oj}} {mn-oju} {vъ} {s-iju} {minut-u}].\\
{} feel.\textsc{ipfv1-pp-f.gen.sg} \textsc{1sg-ins} in this-\textsc{f.acc.sg} minute.\textsc{f-acc.sg}\\
\glt ‘I cannot describe to you the joy [I felt (lit. {\textit{felt}} by me) at this moment].’  \\
\hfill (Russian; RNC; 1812)
}
\z

\noindent Examples from the period 1890--1918 are more difficult to classify. In \REF{wiem:ex:led-large}, for instance, the classification as progressive or iterative (which itself is extremely rare) depends on whether the focus is on a series of intervals within a discrete larger episode ($\rightarrow$ iterative) or on the continuity of attempts ($\rightarrow$ progressive); the problem is not just whether we are dealing with subintervals, but whether accentuating such intervals is the proper “point” -- a question that can at best be solved on the basis of a larger discourse fragment.

\ea\label{wiem:ex:led-large}{{K severo-vostoku ot Černovic protivnik otčajannymi kontr-atakami,}\\
\gll \minsp{[} {\textit{vede-nn-ymi}} {bol’š-imi} {sil-ami}], \\
{} carry.out.\textsc{ipfv1-pp-ins.pl} big-\textsc{ins.pl} power-\textsc{ins.pl}\\
{pytalsja zaderžat’ naše nastuplenie.}\\
\glt ‘To the northeast of Chernivtsi, the enemy tried to delay our advance with deperate counter-attacks [\textit{carried out} by large forces].’ \\
\hfill (Russian; RNC; 1916)
}
\z

\noindent Now let us turn to Polish, for which we first discuss the relation between GF and PROG, then between HAB and PROG. 

\tabref{wiem:tab:gf-prog-coef-pol} provides coefficients between GF and PROG; values higher than~1 testify to a predominance of GF, values between~1 and~0 indicate a predominance of PROG. If only one of the two functions is attested, the cell indicates GF or PROG, respectively.\footnote{We use ``na'' if none of the functions is attested.} As \tabref{wiem:tab:gf-prog-coef-pol} shows, no clear tendencies can be inferred from a pairwise comparison of IPFV1 and IPFV2 stems over the periods.

\begin{table}[ht]
\begin{tabularx}{\textwidth}{lSSSSS}
\lsptoprule
coefficient& \multicolumn{1}{c}{1730--1780} &  \multicolumn{1}{c}{1801--1850} & \multicolumn{1}{c}{1890--1918} & \multicolumn{1}{c}{1945--1980} & \multicolumn{1}{c}{1990--2020}\\
\midrule
freq\\
~~~ IPFV1& 1.7 & 0.4 & 0.4 & 0.1 & 1.1\\
~~~ IPFV2& 1.0 & \multicolumn{1}{c}{GF} & \multicolumn{1}{c}{PROG} & 0.1 & 3.0\\
infreq\\
~~~ IPFV1& 6.0 & 9.0 & 2.0 & 0.4 & 1.0\\
~~~ IPFV2& \multicolumn{1}{c}{GF} & \multicolumn{1}{c}{na} & \multicolumn{1}{c}{PROG} & 0.2 & 0.4\\
control\\
~~~ IPFV1& GF & 0.7 & 0.4 & 0.8 & 0.3\\
~~~ IPFV2& 3.7 & 1.0 & 0.1 & 0.1 & 0.6\\
\lspbottomrule
\end{tabularx}
\caption{GF\slash PROG coefficients for Polish ipfv. \textit{n/t}-participles}
\label{wiem:tab:gf-prog-coef-pol}
% \begin{minipage}{12cm}
% \small \textbf{Remark}\\
% - ‘na’, if none of the functions is attested.
% \end{minipage}
\end{table}

Despite the lack of a clear tendency of PROG in relation to GF, and in contrast to Russian, ipfv. \textit{n/t}-participles do not “lose” PROG, but retain it. It even seems to slightly increase by the modern period, at least in the control and the infreq-samples, in which PROG now seems a bit more prominent for \textit{n/t}-participles of IPFV2 stems than of IPFV1 stems. Good examples of PROG are difficult to find for the earliest periods (maybe because of data scarcity). All these findings hold true for either stem type; see \REF{wiem:ex:driven}--\REF{wiem:ex:wrapped}.

The following four examples show the employment of \textsc{ipfv1} stems ((\ref{wiem:ex:driven})--(\ref{wiem:ex:realized})) and of \textsc{ipfv2} stems ((\ref{wiem:ex:operated})--(\ref{wiem:ex:wrapped})) in progressive function.

\ea\label{wiem:ex:driven}
\gll {Puści-ł} {żagiel} {i,} {\textit{gna-n-y}} {pochyleni-em} {się} {statk-u,} {pobieg-ł} {mimowolnie} {drobn-ym} {kroki-em} {na} {tył}.\\
let.\textsc{pfv-pst-m.sg} sail-\textsc{acc.sg} and drive.\textsc{ipfv1-pp-m.nom.sg} stoop.\textsc{n-ins.sg} \textsc{refl} ship.\textsc{m-gen.sg} run.\textsc{pfv-pst-m.sg} involuntarily small-\textsc{ins.sg.m} step.\textsc{m-ins.sg} on rear-\textsc{acc}\\
\glt ‘He let go off the sail and, \textit{driven} by the stoop of the ship, involuntarily ran in small steps to the rear.’\hfill (Polish; PNC; 1890--1918)
\ex\label{wiem:ex:realized}{\gll
{Największ-a} {inwestycj-a} {obecnie} {\textit{realizowa-n-a}} {w} {park-u} {to} {budow-a} {aqua-parku „Fala”.}\\
largest-\textsc{f.nom.sg} investment.\textsc{f-nom.sg} presently perform.\textsc{ipfv1-pp-f.nom.sg} in park-\textsc{loc} \textsc{ptc} construction.\textsc{f-nom.sg} aquapark\\
\glt ‘The largest investment currently underway [lit. \textit{(being) realized}] in the park is the construction of the “Fala” aquapark.’\hfill (Polish; PNC; 1990--2020)
}\largerpage
\ex\label{wiem:ex:operated}
\gll {Wesz-l-i,} {i} {w} {głęb-i} {stodoł-y} {ujrze-l-i} {wialni-ę do czyszczeni-a zboż-a,} {\textit{obsługiwa-n-ą}} {przez} {kilk-u} {robotnik-ów}.\\
enter.\textsc{pfv-pst-pl} and in depth.\textsc{f-loc.sg} barn.\textsc{f-gen.sg} spot.\textsc{pfv-pst-pl} cleaning.plant.\textsc{f.acc} operate.\textsc{ipfv2-pp-f.acc.sg} through some-\textsc{acc} laborer.\textsc{m-acc.pl}\\
\glt ‘They entered and, at the far end of the barn, saw a grain cleaning plant \textit{operated} by some laborers.’\hfill (Polish; PNC; 1890--1918)
\ex\label{wiem:ex:wrapped}
\gll {Zauważyłem} {go} {wychodząc} {na} {pokład,} {gdy} {znika-ł} {za} {ruf-ą,} {\textit{zawija-n-y}} {już} {w} {zwoj-e} {mokr-ej} {mgł-y}.\\
noticed him exiting on deck when disappear.\textsc{ipfv-pst-sg.m} behind stern.\textsc{f-acc.sg} wrap.\textsc{ipfv2-pp-m.nom.sg} already in coul.\textsc{m-acc.pl} wet-\textsc{gen.sg.f} fog.\textsc{f-gen.sg}\\
\glt ‘I spotted it as I stepped onto the deck, when it was about to disappear after the stern, \textit{wrapped} in coils of wet fog.’\hfill (Polish; PNC; 1990--2020)
\z

\noindent These observations are indicative that the absence vs. presence of the prefix in the stem does not have a conceivable impact on the average aspectual behavior of the ipfv. \textit{n/t}-participle. First of all, since both IPFV1 and IPFV2 \textit{n/t}-participles occur in progressive function, either stem type can defocus a culmination point entailed by telicity, whether it be induced by the lexical meaning of the stem or only by the prefix. Moreover, IPFV1 stems can convey an idea of unlimited repetition of culmination points (i.e. $\text{HAB}+\text{telic}$) as well. See examples from the samples:

\ea\label{wiem:ex:widely-applied}{{Nie rozumieli dlaczego znów mają płacić za elektroniczne,}\\
\gll \minsp{[} {skoro} {wcześniej} {płaci-l-i} {już} {za} {wyparkow-e,} {tańsz-e} {w} {obsłudze} {i} {powszechnie} {\textit{stosowa-n-e}} {w} {cał-ym} {kraj-u}].\\
{} because earlier pay.\textsc{ipfv1-pst-pl} already for evaporator-\textsc{m.acc.pl} cheaper-\textsc{m.sacc.pl} in service and commonly apply.\textsc{ipfv1-pp-m.acc.pl} in whole-\textsc{loc.sg.m} country.\textsc{m-loc.sg}\\
\glt ‘They did not understand why they had to pay for electronics again, [since they had already paid for evaporators, cheaper to operate and widely \textit{applied} throughout the country].’\hfill (Polish; PNC; 1990--2020)
}
\z

\pagebreak
\ea\label{wiem:ex:service-provided}
{{Wprowadzenie takiego systemu i dalsze udoskonalanie go }\\
\gll \minsp{[} {pozwala} {poprawi-ć} {jakość} {\textit{świadczo-n-ych}} {usług}]\\
{} allow.\textsc{ipfv-prs.3sg} improve.\textsc{pfv-inf} quality.\textsc{f-acc.sg} provide.\textsc{ipfv1-pp-f.gen.pl} service.\textsc{f-gen.pl} \\{i utrzymać na odpowiednim poziomie koszty eksploatacji.}\\
\glt ‘The introduction of such a system and further improvement of it [allows to improve the quality of services \textit{provided}] and to keep operating costs at an appropriate level.’\hfill (Polish; PNC; 1990--2020)
}
\z

\ea\label{wiem:ex:made}{\gll
{W} {Poznani-u} {\textit{realizowa-n-e}} {są} {częściowo} {dostaw-y} {w} {butelk-ach} 0,5 l.\\
in Poznań.\textsc{m-loc.sg} perform.\textsc{ipfv1-pp-f.nom.pl} be.\textsc{prs.3pl} partially delivery.\textsc{f-nom.pl} in bottle-\textsc{loc.pl}\\
\glt ‘In Poznań, some deliveries are made in 0.5 l bottles.’ \\
\hfill (Polish; PNC; 1945--1980)
}
\z

\noindent Finally, a look at the coefficients between HAB and PROG (see \tabref{wiem:tab:hab-prog-coef-pol}) reveals that, although, again, there is no clear tendency for any of the sample groups, HAB dominates over PROG in most samples ($\text{coefficient}>1$), and more consistently so than between PROG and GF (see \tabref{wiem:tab:gf-prog-coef-pol}). The degree of this dominance varies a lot, but it is often very high, particularly for IPFV2 \textit{n/t}-participles. To the contrary, if PROG dominates ($\text{coefficient}<1$), it is for IPFV1 stems.\footnote{HAB if no PROG is attested in the sample.}

\begin{table}
\begin{tabularx}{\textwidth}{lSSSSS}
\lsptoprule
coefficient& \multicolumn{1}{c}{1730--1780} &  \multicolumn{1}{c}{1801--1850} & \multicolumn{1}{c}{1890--1918} & \multicolumn{1}{c}{1945--1980} & \multicolumn{1}{c}{1990--2020}\\
\midrule
freq\\
~~~ IPFV1& 8.0 & 2.5 & 0.8 & 0.8 & 3.1\\
~~~ IPFV2& 13.5 & \multicolumn{1}{c}{HAB} & 1.7 & 1.6 & 76.0\\
{infreq}\\
~~~ IPFV1& 6.0 & 8.0 & 2.0 & 1.1 & 7.5\\
~~~ IPFV2& \multicolumn{1}{c}{HAB} & \multicolumn{1}{c}{HAB} & 7.5 & 1.7 & 7.4\\
{control}\\
~~~ IPFV1& \multicolumn{1}{c}{HAB} & 1.5 & 0.5 & 0.9 & 2.2\\
~~~ IPFV2& 13.7 & 4.5 & 3.2 & 2.8 & 4.8\\
\lspbottomrule
\end{tabularx}
\caption{HAB\slash PROG coefficients for Polish ipfv. \textit{n/t}-participles}
\label{wiem:tab:hab-prog-coef-pol}
% \begin{minipage}{12cm}
% \small \textbf{Remark}\\
% HAB if no PROG is attested in the sample.
% \end{minipage}
\end{table}

\subsubsection{Russian \textit{n/t}-participles of IPFV2 stems}\label{wiem:sec:rus-particip-ipf2}\largerpage[1.5]

As pointed out in \sectref{wiem:sec:lex-diversity}, although we found that Russian \textit{n/t}-participles of IPFV2 stems (altogether 48~tokens) were considerably less frequent than their equivalents in Polish already in the 18\textsuperscript{th} century and obviously ceased to be in active use by the beginning 20\textsuperscript{th} century, their type/token ratio was remarkably high during 1730--1780. With~0.67 it was comparable to the LD of Polish IPFV2 \textit{n/t}-participles in the same period (0.62) (see \tabref{wiem:tab:type-tok-pol}); simultaneously, it was practically as high as of Russian PFV \textit{n/t}-participles of that period (0.69) and much higher than for IPFV1 \textit{n/t}-participles of any period (see \tabref{wiem:tab:type-tok-rus}). From this we might infer that, despite their rarity, \textit{n/t}-participles of IPFV2 stems showed a broader spread in the lexicon (= number of verb stems on which they occurred). In fact, these participles appear to have been more productive than any Russian ipfv. \textit{n/t}-participles after the end of the 18\textsuperscript{th} century, before their frequency dropped abruptly and shortly after became zero.

Moreover, among the 41~instances we found in the RNC for 1730--1780, the habitual function prevails (16~instances), while there are 10~cases with stative function, and most of the 9~GF uses are doubtful; there are also 2~debatable cases of progressive use. See 
\REF{wiem:ex:laid}--\REF{wiem:ex:discomforted}.

(\ref{wiem:ex:laid}) exemplifies the habitual function.

\ea\label{wiem:ex:laid}
{Na sej konecъ postroeny byli vně zemljanago Kammer-Kolležskago vala, po vsěmъ bol’šimъ 	dorogamъ anbary i torgovyja města,}\\
\gll \minsp{[} {gdě} {s"estn-ye} {pripas-y} {\textit{skladyva-n-y}}],\\
{} where edible-\textsc{nom-pl} supply-\textsc{nom.pl} lay.\textsc{ipfv2-pp-pl}\\
{i gdě by vъ slučaě 	nadobnosti dolžny byli priěžžajuščie na torgi krest’jane črezъ ogradu vsemu činit’ prodažu (…)}\\
\glt ‘In this end, along all the main roads, (a number of) barns and trading places were built outside of the Kamer-Kollezhsky rampart, [where comestibles \textit{used to be laid}] and where peasants who would come to trade, in case of necessity, had to carry out sales over a fence.’\\
\hfill (Russian; RNC; 1775)
\z

\noindent Example (\ref{wiem:ex:painted-carved}) illustrates the stative function.

\ea\label{wiem:ex:painted-carved}
{Ja sobral iz naxodjasčixsja v zemle razvalin nekoliko izrascov zelenyx, byvšix v stroenii,}\\
\gll \minsp{[} {meždo} {kotor-ymi} {dv-a} {cel-ye} {šestiugol’nik-a} {\textit{rozpisyva-n-y}} {zolot-om,} {drug-ie,} {na} {kotor-yx} {\textit{vrezyva-n-y}} {liter-y} {bel-ye,} {in-ye} {\textit{sostavliva-n-y}} {iz} {kusk-ov} {razn-ago} {cvet-a} {\textit{poliva-nn-yx}}] {i viditsja bez uzora.}\\
{} between which-\textsc{ins.pl} two-\textsc{m.nom.pl} whole-\textsc{nom.pl} hexagon.\textsc{m-gen.sg} paint.\textsc{ipfv2-pp-pl} gold.\textsc{n-ins.sg} other-\textsc{nom.pl} on which-\textsc{loc.pl} carve.into.\textsc{ipfv2-pp-pl} letter.\textsc{f-nom.pl} white-\textsc{nom.pl} other-\textsc{nom.pl} compile.\textsc{ipfv2-pp-pl} from piece-\textsc{gen.pl} distinct-\textsc{gen.sg} colour.\textsc{m-gen.sg} permeat.\textsc{ipfv2-pp-gen.pl}\\
\glt ‘I collected several green tiles from the underground ruins, that had been in the building, [among which there were two intact hexagons \textit{painted} with gold, some with white letters carved into them, some \textit{compiled} from pieces \textit{permeated} with distinct colours] and appear patternless.’ \\ \hfill (Russian; RNC; 1741)
\z

\noindent The following is a clear illustration of the existential type of the general-factual function.

\ea\label{wiem:ex:aired-out}
{No vъ čislě tovarov, šersti i xlopčatoj bumagi, kotoraja ne vъ dělě, otkudabъ vezena ni byla, xotja by i svidětel’stvo imeli,}\\
\gll \minsp{[} {čto} {\textit{provětriva-n-o}}],\\ 
{} \textsc{comp} air.\textsc{ipfv2-pp-sg.n}\\
{ne propuskat’.}\\
\glt ‘But of the wares, do not let pass wool and unmanufactured cotton, regardless of whence it be brought, (and) even if (the carriers) present a certificate (stating) [that (the item) \textit{has been aired out]}.’ \\
\hfill (Russian; RNC; 1771)
\z

\noindent As for (\ref{wiem:ex:aneled}), it is difficult to tell whether it shows the general-factual or the progressive reading.

\ea\label{wiem:ex:aneled}
{Prežde vsego vspomni, čto, kogda ty ešče byv mladencem, vyšel iz spasitel’noj kupeli:}\\ 
\gll \minsp{[} {togda} {svjaščenn-ym} {mir-om} {ušes-a} {tvo-i} {\textit{by-l-i}} {\textit{pomazyva-n-y}}],\\
{} then holy-\textsc{m.ins.sg} myrrh.\textsc{m-ins.sg} ear.\textsc{n-nom.pl} thy-\textsc{nom.pl} be-\textsc{pst-pl} anel.\textsc{ipfv2-pp-pl}\\
{s proiznošeniem six slov: Vo uslyšanie very.}\\
\glt ‘First of all, recall that when thou wert an infant, thou camest from the redeeming laver: [then \textit{were} thyne ears \textit{aneled} with holy myrrh], along with the pronunciation of the following words: For the (true) understanding of faith.’\hfill (Russian; RNC; 1777)
\z

\noindent (\ref{wiem:ex:discomforted}) most likely exemplifies the progressive function, referring to a telic process.\largerpage[1.5]

\ea\label{wiem:ex:discomforted}
\gll \minsp{[} {No} {naposledok} {on-ym} {igumen-om} {i} {bojar-y} {tak} {\textit{by-l}} {\textit{obezpokoeva-n}}],\\
{} but lastly.\textsc{adv} that-\textsc{m.ins.sg} hegumen.\textsc{m-ins.sg} and boyar.\textsc{m}-\textsc{ins.pl} that.way be-\textsc{pst-sg.m} discomfort.\textsc{ipfv2-pp-sg.m}\\
{čto, poslušav ix, kljatvu, dannuju Jaroslavu, prestupil i (…) do smerti o tom sožalel i nikogda bez plača i vozdyxanija ne vospominal.}\\
\glt ‘[But lastly, (he) \textit{was} so much \textit{discomforted} by that hegumen and the boyars] that, having listened to them, (he) violated the oath (that he had) made to Jaroslav and (…) regretted that till his death and never recalled (that) without lamentation and sighing.’\hfill (Russian; RNC; 1750)
\z

\noindent Among the few instances encountered in the second and third period, practically all are problematic in the assignment of aspect functions. For instance, in the following example it is hardly possible to decide whether we are dealing with GF or HAB:

\ea{{Xotja zemleopisatel’noj ėkspedicii i ne nadležit otnjud’ vxodit’ v meževanie zemel’, no v opisanijax svoix dolžna ona označit',}\\
\gll \minsp{[} {k-em}, {kak-ie} {zeml-i} {obrabatyvaj-u-tsja} {ili} {zapušče-nn-ye} {k-em,} {prežde} {\textit{obrabotyva-n-y}} {\textit{by-l-i}}].\\
{} who-\textsc{ins} what.kind.of-\textsc{nom.pl} land-\textsc{nom.pl} cultivate.\textsc{ipfv2-prs.3pl-refl} or neglect.\textsc{pfv-pp-pl} who-\textsc{ins} previosuly.\textsc{adv} cultivate.\textsc{ipfv2-pp-pl} be-\textsc{pst-pl}\\
\glt ‘Although the geographical expedition ought not at all to conduct boundary surveys, even so it has to specify in its descriptions [what lands are cultivated and by whom, or by whom the ones that are left untilled \textit{had} previously \textit{been cultivated}].’\hfill (Russian; RNC; 1822)
}
\z

\noindent At any rate, the history of IPFV2 \textit{n/t}-participles in Russian provides an example of a category which showed productivity in the lexicon (high type-frequency), although it was rare on token level, before it “died out”. With regard to aspect functions (as far as less than 50~examples in our samples can be indicative), the predominance of HAB seems to confirm what one would predict from \citeposst{Tatevosov2015Akcional} analysis of verb stems with lexical prefixes. However, we also find \textit{n/t}-participles from IPFV2 stems which hardly refer to culmination points, either because of the actionality of the stem (as atelic \textit{osmeivano} `ridiculed' in \REF{wiem:ex:ridiculed}) or because it otherwise is difficult to “get” (\textit{vozpitovany} `raised' in \REF{wiem:ex:gathered}).\footnote{\REF{wiem:ex:ridiculed} also illustrates that \textit{n/t}- and \textit{m}-participles of IPFV2 stems could occur in coordination. However, here the role of \textit{m}-participles (in Russian) is not considered.} 

\ea\label{wiem:ex:ridiculed}{{(…) a drugie po zavisti ko mne, tret'i po trusosti (…) raznymi sposobami davali mne v tom prepinanija, tak čto togda počti v každom tex gospod dome to moe, čtob v polki vmesto anglijskix iz rossijskix sukon mundiry delat’,}\\ 
\gll \minsp{[} {predprijati-e} {\textit{osmeiva-n-o}} {i} {\textit{xudo}} {\textit{tolkova-n-o}} {\textit{by-l-o}}].\\
{} undertaking\textsc{.n-nom.sg} ridicule.\textsc{ipfv2-pp-sg.n} and illy.\textsc{adv} interpret.\textsc{ipfv1-pp-sg.n} be-\textsc{pst-sg.n}\\
\glt ‘(…) and others out of envy toward me and yet others out of their cowardice (…) created impediments for me in that in almost every house of those lords, [my undertaking with regard to manufacturing full dress uniforms for the army (lit. regiments) out of Russian broadcloths instead of English ones \textit{was} then \textit{ridiculed} and \textit{ill-interpreted}].’ \\
\hfill (Russian; RNC; 1766--1777)
}
\z

\ea\label{wiem:ex:gathered}{{O sem potrebno by vnjatnee razsmotret’, ibo onye sut’ dvojakie,}\\ 
\gll \minsp{[} {odn-i} {bogougodn-omu} {ustav-u} {Petr-a} {Velik-ago} {\textit{by-l-i}} {\textit{zbira-n-y}}, {\textit{vozpitova-n-y}}, {i} {\textit{obuča-em-i}} {v} {sirotsk-ix} {dom-ex}].\\
{} one-\textsc{nom.pl} charitable-\textsc{m.dat.sg} charter\textsc{.m-dat.sg} \textsc{pn.m-gen.sg} great-\textsc{m.gen.sg} be-\textsc{pst-pl} gather.\textsc{ipfv2-pp-pl} raise.\textsc{ipfv2-pp-pl} and teach.\textsc{ipfv2-prs.pp-pl} in orphan-\textsc{loc.pl} house-\textsc{loc.pl}\\
\glt ‘This (matter) requires a more articulate consideration, for those are double-natured, [the solitary \textit{were gathered}, \textit{raised}, and \textit{taught} in orphanages in accordance with Peter the Great’s charitable charter].’ \\\hfill (Russian; RNC; 1733)
}
\z

\section{Conclusions and outlook}\label{wiem:sec:conclusion}

We may draw some conclusions. These have to be cautious, at least as for their empirical basis, since this is probably the first corpus-based pilot study on the development of participles and their role in the aspect-voice system of Russian and Polish.

As concerns aspectual semantics and aspect functions, some principled remarks appear appropriate. First, the general-factual function (GF) associated to ipfv. aspect should be primarily assessed in terms of information structure (presupposed vs asserted information) rather than in temporal semantics. Second, the status of resultative subevents for IPFV1 stems in a telic setting (on clause level) is labile. A resultative subevent is asserted by suitable (lexical, including natural) prefixes, but it need not be absent (and can be “activated”) if such prefixes are lacking. Actually, this is what happens with IPFV1 stems when they “replace” their PFV counterparts (with natural prefixes), e.g. in the narrative present tense or in pluractional functions.

Third, since IPFV2 stems inherit the resultative subevent from “their” PFV stem, there is no point in restricting their use from habitual readings -- \textit{pace} \citet{Tatevosov2015Akcional}, who does not seem to notice the consequences of his reasoning concerning the role of lexical (including natural) prefixes. Functions of external pluractionality are themselves insensitive to actionality distinctions, as is GF. Fourth, since, following Tatevosov’s reasoning, IPFV1 stems are void of (non-cancellable) resultative subevents, there is also no reason why their \textit{n/t}-participles should block, or avoid, habitual or progressive readings. If such readings are indeed avoided (or inacceptable) for ipfv. \textit{n/t}-participles in contemporary Russian, the reason can exactly not be sought in their semantic structure; instead, the reason should be connected to the way these participles are integrated (or not) into the grammatical system at the interface between aspect and voice. This can be clearly seen from the largely different development of these participles in Polish, including also IPFV2 stems. The relatively infrequent occurrence of ipfv. \textit{n/t}-participles in progressive function is nothing particular of constructions with these participles, but a general feature of aspect use in Slavic languages: many ipfv. verbs do not allow for progressive function (cf. \citealt{Lehmann1998} for Russian), i.e. their type frequency is limited, and this applies also to token frequency (\citealt{Wiemer.Wrzesień-Kwiatkowska.Wyroślak2020}, and M. Łaziński, p.c., for Polish).

In addition, from among the findings of our corpus-based study we may point out the following. Although participles of IPFV2 stems show a certain preference for habitual situations, this does not entail a focus on achieved culmination points (as \citeauthor{Tatevosov2015Akcional}'s \citeyear{Tatevosov2015Akcional} reasoning would indeed predict). The token frequency of \textit{n/t}-participles in habitual function was particularly high in the early periods, both for IPFV1 and for IPFV2 stems. But while in Russian IPFV2 \textit{n/t}-participles became obsolete, their Polish equivalents have been integrated tightly into the aspect-voice interface. This can be seen from their productivity indicators and from their more even distribution over aspect functions (in parallel to IPFV1 \textit{n/t}-participles). Russian IPFV2 \textit{n/t}-participles, before they disappeared, were mainly used in habitual and stative meaning, not in GF. Furthermore, type frequency (lexical diversity) does not seem to depend much on token frequency. For instance, despite restricted usage in 18th-19th c. Russian (and their subsequent disappearance), the type/token ratio of IPFV2 \textit{n/t}-participles in the 18th century was not lower than for their Polish equivalents.

As concerns ipfv. \textit{n/t}-participles in general-factual use, the corpus data indicate that in Russian this function has become prominent only in the last two periods (i.e. since 1945). Moreover, even in this recent time, presuppositional GF does not seem to dominate over negated ipfv. \textit{n/t}-participles which mark existential GF. This “polarity split” for subtypes of GF needs further research. By contrast, Polish \textit{n/t}-participles of ipfv. stems do not show any bias toward GF; in general, aspect functions are rather evenly distributed over \textit{n/t}-participles of both IPFV1 and IPFV2 stems. This testifies to their tight integration into the aspect-voice interface in the language.


Admittedly, these conclusions rest on a restricted amount of data, albeit largely assembled via random sampling. Therefore, our findings should be tested against a larger amount of data, as well as for other Slavic languages in which ipfv. stems demonstrate productive derivation of \textit{n/t}-participles. An analogous caveat is justified concerning productivity. We have applied different, and rather rough measures of productivity (lexical diversity) with partially different vantage points. This has, in part, led to superficially contradictory results (see \sectref{wiem:sec:lex-diversity}). Type/token ratios give us only a bird’s-eye view (see Tables~\ref{wiem:tab:type-tok-rus} and~\ref{wiem:tab:type-tok-pol}), while \tabref{wiem:tab:type-freq-stem} supplies more detailed information on types and Tables~\ref{wiem:tab:means-proportion-pol} and \ref{wiem:tab:means-proportion-ru} are oriented toward tokens. Type\slash token ratios are a basic (and certainly insufficient) measure of productivity beside measurements oriented toward phenomena that come close to hapax legomena (cf. \citealt{Baayen2009}). However, our study was not concerned particularly with such phenomena; it was mainly oriented toward the productivity of grammatical patterns between different types of verb stems. A database of aspect triplets has proved helpful in getting a handle on corpora whose annotation often appears insufficient for determining grammatical patterns related to aspect and voice. New approaches toward productivity in diachronic corpus studies, such as permutation testing (cf. \citealt{Säily.Suomela2017}), should be checked as for their suitability in cases like the one presented here.


% \section*{Abbreviations}

% \begin{tabularx}{.5\textwidth}{@{}lX@{}}
% \textsc{pfv}&perfective\\
% \textsc{ipfv}&imperfective\\
% \textsc{gf}&general-factual\\
% \textsc{vp}&verb phrase\\
% \textsc{prog}&progressive\\
% \textsc{iter}&iterative\\
% \textsc{hab}&habitual\\
% \textsc{stat}&stative\\
% \textsc{ld}&lexical diversity\\
% \end{tabularx}%
% \begin{tabularx}{.5\textwidth}{@{}lX@{}}

% &\\ % this dummy row achieves correct vertical alignment of both tables
% \end{tabularx}

\section*{Acknowledgements}\largerpage
Research connected to this article was performed in the \textit{DiAsPol}-project (\url{http://www.diaspol.uw.edu.pl/eng/}), for which we gratefully acknowledge funding by the DFG (WI~1286/19-1) and the NCN in the joint \textit{Beethoven~II} program.

We are obliged to Oleg Bulatovs’kyj for his inventive support as a programmist, without which a wealth of data would have remained inaccessible to us, and to Michał Dudas for annotating and double-checking the Polish samples. We furthermore thank two anonymous reviewers and Olav Mueller-Reichau for their (mostly complementary) comments on an earlier version, which proved most helpful. The usual disclaimers apply.

\printbibliography[heading=subbibliography,notkeyword=this]

\end{document}
