\documentclass[output=paper,colorlinks,citecolor=brown]{langscibook}
\ChapterDOI{10.5281/zenodo.10123649}

\author{Matic Pavlič\affiliation{University of Ljubljana} and Arthur Stepanov\affiliation{University of Nova Gorica}}


\title[Number mismatch effect and processing cataphora in a \textit{pro}-drop language]{Number mismatch effect and processing cataphora in a \textit{pro}-drop language: The case of Slovenian}
% replace the above with your paper title
%%% provide a shorter version of your title in case it doesn't fit a single line in the running head
% in this form: \title[short title]{full title}
\abstract{\sloppy Cataphora (also known as backward anaphora) is a type of pronominals that precede their antecedents linearly. Previous research on processing cataphora has explored the idea that cataphoric pronouns trigger a forward-looking active search for an antecedent to establish a coreference relation similar to a filler-gap dependency between a fronted wh-phrase and its base-generated syntactic position (\citealt{cowart1987}). \citet{vanGompelLiversedge2003} have shown that in an active search, the parser establishes a cataphoric coreference before considering pronominal phi-features. This results in a gender mismatch effect: sentences with incongruent incoming NP antecedents were more difficult to read than their congruent counterparts, as evidenced by slower reading times and eye movement regressions. In this paper, we report the results of a self-paced reading experiment in which the active search hypothesis is further tested by examining online cataphora resolution with respect to the number feature in Slovenian, a pro-drop language with a rich nominal and verbal morphology.

\keywords{cataphora, \textit{pro}, feature mismatch effect, forward-looking active 	search, self-paced reading, Slovenian, psycholinguistics}
}

\lsConditionalSetupForPaper{}

\begin{document}
\maketitle
\section{Introduction}\label{ste-pav:sec:intro}
The interpretation of pronominal expressions such as the Slovenian reflexive \textit{svoj} ‘self’s’ in \REF{ste-pav:ex:svoje-sorodnike} depends on their relation to referring expressions in the context in which they are used.

\ea\label{ste-pav:ex:svoje-sorodnike}
\gll Svoje sorodnike kliče po več krat na dan.\\
\textsc{self’s} relatives.{\PL} calls.{\SG} at	several times on day \\
\glt ‘He calls his relatives several times a day.’\hfill (Slovenian)
\z

\noindent If we restrict ourselves to the syntactic context, a dependency relation is established between the base-generated position of a pronominal expression and a referring expression: The latter acts as an antecedent and c-commands the former according to Principles A~(reflexives) and B~(pronouns) of the Binding Theory (\citealt{chomsky1981, reinhart1983}). The linear order of the two expressions may be changed so that the pronominal element is spelled out after the antecedent, as in the case of forward anaphora, or before the antecedent, as in the case of backward anaphora or cataphora, but it should not itself be in the position of c-commanding the antecedent according to Principle~C (the referring expressions should not be bound at any time).

Due to the relative word order of the two expressions, anaphora and cataphora resolutions differ fundamentally in real time sentence comprehension (\citealt{Lust1986, reinhart1986, Blackwell2003, tsimpliFiliaci2004, kennisonBowers2009, LoboSilva2016}). In the case of anaphora, the parser encounters a pronominal expression and simultaneously considers all possible candidates previously integrated into the incoming sentence and stored in working memory. This mechanism is closely related to the processes of memory retrieval (\citealt{chow2014}). In the case of a cataphora, the parser does not find an available antecedent in the previous syntactic context. Therefore, the parser expects to find it in the incoming sentence material and considers each subsequent noun phrase as a potential antecedent. Crucially, the ``active'' or ``impatient'' parser does not wait until all potential antecedents are stored in working memory but evaluates them one by one as they are integrated into the structure. More specifically, the parser attempts to associate the cataphora with the first potential antecedent as soon as the structural requirements of its c-commanding relation to the cataphora are confirmed~-- but before it considers the interpretative requirements (phi-feature matching).

This analysis was first introduced in the seminal work of \citet{cowart1987} who observed a strong preference for linking cataphora to the first possible noun phrase encountered by the parser. \citet{sturt2003} and \citet{vanGompelLiversedge2003} supported this finding by measuring the gaze direction and reading time of sentences such as \REF{ste-pav:ex:when-appeared}. 

\ea \label{ste-pav:ex:when-appeared} 
\ea When he\textsubscript{i} appeared, the king\textsubscript{i} immediately greeted the boys very warmly. \label{ste-pav:ex:when-appeared-a}
\ex When they\textsubscript{i} appeared, the boys\textsubscript{i} immediately greeted the king very warmly. \label{ste-pav:ex:when-appeared-b}
\ex When they\textsubscript{i} appeared, the king immediately greeted the boys\textsubscript{i} very warmly. \label{ste-pav:ex:when-appeared-c}
\ex When he\textsubscript{i} appeared, the boys immediately greeted the king\textsubscript{i} very warmly. \label{ste-pav:ex:when-appeared-d}
\z\z

\noindent In \REF{ste-pav:ex:when-appeared-a} and \REF{ste-pav:ex:when-appeared-b}, the first potential NP antecedent bears cataphora-congruent phi-features, in this case, number. The parser associates the cataphora \textit{he/they} with the referring expression, as marked in the example by matching indices. \Citet{vanGompelLiversedge2003} used examples such as \REF{ste-pav:ex:when-appeared-a} and \REF{ste-pav:ex:when-appeared-b} as a baseline for analysing the reading times and gaze directions of examples such as \REF{ste-pav:ex:when-appeared-c} and \REF{ste-pav:ex:when-appeared-d}. For the latter, they found an effect when the number value of the first potential NP antecedent (or gender, in other experiments) did not match that of a preceding cataphor. Van Gompel and Liversedge refer to this as a mismatch effect and claim that it reflects the parser's unsuccessful attempt to establish a referential dependency between the antecedent and the cataphora. They add that the gender or number mismatch effect can only occur when the parser attempts to establish a referential dependency before comparing the features of the NP with those of the cataphora.

\citet{kazanina2005} and \citet{kazaninaLauPhillips2007} replicated the gender mismatch effect with a paradigm of self-paced reading and explored it in more detail. They attributed the slowdown in reading time to the parser’s search for an antecedent, which involves predictive processes. One of these predictive processes is the active search mechanism, which was originally used to interpret filler-gap dependencies (\citealt{crain1985, stowe1986, frazier1989, frazierFlores1989, garnsey1989, kaanHolcomb2000, stepanovStateva2015}). Wh-dependencies are established between the fronted wh-phrase and its base-generated position. The search for a gap begins as soon as a wh-phrase is processed. This was demonstrated in online experiments by filling the gap position with an overt element (which prevented the parser from interpreting the wh-phrase in that position), resulting in longer processing times compared to a sentence without an overt element in the gap position (\citealt{crain1985, stowe1986, lee2004}). Thus, the active search mechanism assumes that the parser expects a gap as soon as a wh-phrase is encountered (\citealt{frazier1989}). In the case of pronoun interpretation, the active search mechanism predicts that a search for an antecedent will be initiated as soon as a pronoun is encountered to resolve the interpretation of the pronoun (\citealt{frazier1989, kazaninaLauPhillips2007, kazaninaPhillips2010}). Although pronouns may have an antecedent outside the sentence in which they occur, the active search mechanism states that searching for an antecedent within the sentence is the default strategy when there is no preceding context.

Moreover, \citet{kazaninaLauPhillips2007}, \citet{aoshima2009}, \citet{kazaninaPhillips2010}, and \citet{yoshidaSturt2014} show that the gender mismatch effect is absent in syntactic contexts where the incoming NP is not available for coreference because it cannot be bound by a c-commanding expression (Principle~C of Binding Theory; \citealt{chomsky1981, reinhart1983}). For several types of structures containing cataphora and potential NP antecedents to which Principle~C applies, the researchers found no differences in reading time between the gender-congruent and gender-incongruent conditions. In examples \REF{ste-pav:ex:she-was-taking-a} and \REF{ste-pav:ex:she-was-taking-b}, the potential NP antecedent is within the c-command domain of a cataphoric pronoun. Therefore, the parser discards NP as a potential antecedent without looking at the phi-features of NP \textit{Kathryn}~(\textsc{F}) or \textit{Russell}~(\textsc{M}) and without comparing them to the cataphoric features~(F).

\ea \label{ste-pav:ex:she-was-taking} 
\ea[*]{She\textsubscript{i} was taking classes full-time while Kathryn\textsubscript{i} was working two jobs to pay the bills. \label{ste-pav:ex:she-was-taking-a}}
\ex[*]{She\textsubscript{j} was taking classes full-time while Russell\textsubscript{j} was working two jobs to pay the bills. \hfill (\citealt{kazaninaLauPhillips2007}) \label{ste-pav:ex:she-was-taking-b}}
\z\z

\noindent Cataphoric coreference has also been investigated using the event-related potential (ERP) technique. According to previous literature on gender error processing in this domain, frontal positivity within the P600 time window (reflecting syntactic repair) and late anterior negativity (reflecting additional load on working memory) were expected. These effects have been associated with less preferred continuations of syntactically ambiguous sentences (\citealt{osterhoutHolcomb1992, barber2004, gouvea2010}) and agreement errors (\citealt{hagoort1999, osterhoutMobley1995}). It has also been argued that the P600 signals difficulties or errors in integrating syntactic dependencies, which are predicted according to context (\citealt{federmeier2007, DeLong2011}, \citeyear{DeLong2014, vanPetten2006}). ERP results in reading comprehension of cataphoric dependencies in Dutch (\citealt{pablosCheng2015, pablosCheng2018}) showed that gender incongruence leads to P600 only in positions where the binding principles are satisfied. There was no ERP effect in the incongruent NP antecedent that would violate Principle~C if coreference with cataphora had been established. According to the researchers, the negativity in these cases reflects the cancellation of the tentative antecedent and not the gender incongruence between cataphora and antecedent.

\section{Cataphora in \textit{pro} contexts}\label{ste-pav:sec:cataphora-pro-contexts}

In addition to syntactic constraints such as Binding Theory, pronominal resolution (both anaphoric and cataphoric) is also determined by several different language-specific factors, such as the complementary/shared distribution of null and overt personal pronouns (\citealt{boschEtal2003, wilsonKeller2009}, and the references below). 

The interpretative properties of null \textit{pro} in contexts of anaphora and cataphora compared to those of overt pronouns have mainly been studied in Chinese (\citealt{yulong2007, zhiyi2019}), Italian (\citealt{carminati2002}, \citeyear{carminati2005, fedele2014}) and Spanish material (\citealt{alonso2002}). \citet{carminati2002} shows that the shared syntactic distribution of null anaphora and overt anaphora causes the parser to preferentially associate null anaphora with more prominent antecedents and to associate overt anaphora with less prominent antecedents. Prominence here is understood in structural terms, e.g.\, the antecedent in the highest specifier projection (SpecIP) is considered more prominent than the antecedent in the lower projections. In Italian, the subordinate pronominal expression refers to the main clause subject \textit{Mario} when the former is realised as a null pronoun \textit{pro} \REF{ste-pav:ex:mario-telefonato-giovanni-a}. On the other hand, the subordinate pronominal expression refers to the indirect object \textit{Giovanni} of the main clause when the former is realised as the overt pronoun \textit{lui} \REF{ste-pav:ex:mario-telefonato-giovanni-b}.\footnote{Note that \citet{belletti2007} found the opposite result: Null pronouns tend to prefer the object in anaphoric configurations, while overt pronouns seem to prefer the extra-sentential referent.}\largerpage[-1]

% \protectedex{
\ea \label{ste-pav:ex:mario-telefonato-giovanni}
\ea[]{
\gll Mario\textsubscript{i}	ha	telefonato	a	Giovanni\textsubscript{j}	quando	\textit{pro}\textsubscript{i}	aveva	appenafinite	di mangiare.\\
Mario	has	telephoned	to	Giovanni	when	{} had	just-finished	to eat\\
\glt ‘Mario called John, when he just finished eating.’
} \label{ste-pav:ex:mario-telefonato-giovanni-a}
\ex[]{
\gll Mario\textsubscript{i}	ha	telefonato	a	Giovanni\textsubscript{j}	quando	lui\textsubscript{j}	aveva	appenafinite	di mangiare.\\
Mario	has	telephoned	to	Giovanni	when	he	had	just-finished	to eat\\
\glt ‘Mario called John, when the latter just finished eating.’ \\ \hfill (Italian; \citealt{carminati2002})
} \label{ste-pav:ex:mario-telefonato-giovanni-b}
\z
\z
% }

\noindent In offline experiments assessing interpretations, \citet{fedele2014} extended this study to cases of cataphora. These authors found that (i) null cataphors tend to prefer subject antecedents compared to overt cataphors, and (ii) overt cataphors tend to prefer object antecedents compared to null cataphors, such that the null pronominal expression \textit{pro} in \REF{ste-pav:ex:mentre-parla-viaggio-a} refers to the subject NP \textit{Maria}, while the pronominal expression \textit{lei} in \REF{ste-pav:ex:mentre-parla-viaggio-b} refers to the object NP \textit{Rita}.

\ea \label{ste-pav:ex:mentre-parla-viaggio}
\ea[]{
\gll Mentre	\textit{pro}\textsubscript{i}	parla	del	viaggio	a	Londra	Maria\textsubscript{i}	abbraccia	Rita\textsubscript{j}.\\
while	{}	speaks	of-the	trip	to	London	Maria	hugs	Rita\\
\glt ‘While speaking about the trip to London, Maria hugs Rita.’
} \label{ste-pav:ex:mentre-parla-viaggio-a}
\ex[]{
\gll Mentre	lei\textsubscript{j}	parla	del	viaggio	a	Londra	Maria\textsubscript{i}	abbraccia	Rita\textsubscript{j}\\
while	she	speaks	of-the	trip	to	London	Maria	hugs	Rita\\
\glt ‘While she speaks about-the trip to London Maria hugs Rita.’  \\ \hfill (Italian; \citealt{fedele2014})
} \label{ste-pav:ex:mentre-parla-viaggio-b}
\z
\z

\noindent These results recall \citeauthor{carminati2002}'s (\citeyear{carminati2002}, \citeyear{carminati2005}) Position of Antecedent Hypothesis (PAH) for anaphora contexts. According to PAH, null pronouns refer to a structurally prominent antecedent in a SpecIP position, and overt pronouns refer to an antecedent lower in the clause structure. The contrast in \REF{ste-pav:ex:mentre-parla-viaggio} is broadly consistent with PAH if the latter is construed in terms of preferences rather than absolute expectations. Importantly, when \textit{pro} is part of the main clause \REF{ste-pav:ex:pro-parla-viaggio}, it acquires an extra-linguistic ‘someone else’ interpretation, suggesting that speakers are guided by Principle~C of Binding Theory which overrides the intra-sentential referential bias.

\ea\label{ste-pav:ex:pro-parla-viaggio}
\gll \textit{pro}\textsubscript{k}	parla	del	viaggio	a	Londra	mentre	Maria\textsubscript{i}	abbraccia	Rita\textsubscript{j}.\\
{}	speaks	of-the	trip	to	London	while	Maria	hugs	Rita \\
\glt ‘Somebody speaks about the trip to London, while Maria hugs Rita.’ \\ \hfill (\citealt{fedele2014})
\z

\noindent \citet{alonso2002} test the PAH in Spanish and find that it also holds for the Spanish null subject \textit{pro}, both within sentences and across sentences. Moreover, the interpretation of the Spanish \textit{pro} is sensitive to antecedents that are discourse topics and thus interact closely with the topic-focus system (see also \citealt{yulong2007, zhiyi2019}).

\section{Research hypothesis}\label{ste-pav:sec:research-hypothesis}

In the present work we extend research on the interpretative properties of the null subject \textit{pro} in the context of cataphora to Slovenian, another \textit{pro}-drop language. In Slovenian, there is no overt pronoun available in configurations like in Italian \REF{ste-pav:ex:mentre-parla-viaggio-b} above. This is evident in example \REF{ste-pav:ex:ko-pripoveduje-b}, where the overt personal pronoun \textit{ona} cannot co-refer with either the subject NP \textit{Marija} or the object NP \textit{Rita}.

\ea \label{ste-pav:ex:ko-pripoveduje}
\ea[]{
\gll Ko	\textit{pro}\textsubscript{i}	pripoveduje	o	Londonu	Marija\textsubscript{i}	objema	Rito\textsubscript{j}.\\
when	{}	speaks	of	London	Marija	hugs	Rita\\
\glt ‘While speaking/she speaks about London, Marija hugs Rita.’
} \label{ste-pav:ex:ko-pripoveduje-a}
\ex[]{
\gll Ko	ona\textsubscript{k}	pripoveduje	o	Londonu	Marija\textsubscript{i}	objema	Rito\textsubscript{j}.\\
when	she	speaks	of	London	Marija	hugs	Rita\\
\glt ‘Somebody speaks about London, while Marija hugs Rita.’  \\ \hfill (Slovenian)
} \label{ste-pav:ex:ko-pripoveduje-b}
\z
\z

\noindent This is in line with the Avoid Pronoun Principle (\citealt{chomsky1981}), according to which a null variant is preferred to an overt pronoun whenever possible, provided that a language has a null and an overt subject pronoun in the given syntactic environment. Consequently, Slovenian provides a good testing ground for online comprehension of null cataphors, as language-specific factors such as PAH (see the previous section) and the contrast between null and overt pronouns (\citealt{boschEtal2003, wilsonKeller2009}) do not clash with syntactic constraints. Moreover, Slovenian has a rich verbal inflection as well as overtly realised gender and number features on NP. The topic of cataphora processing has so far received little attention in the Slovenian psycholinguistic literature. These considerations were crucial for the focus of the present study.

\begin{sloppypar}
Previous studies of \textit{pro}-cataphora processing have mostly relied on global or offline evaluation metrics, such as comprehension questions. It is not clear whether the active search mechanism postulated for overt cataphoric pronouns that provide unambiguous cues works in a similar way for silent pronouns such as \textit{pro}. Our main interest was therefore in better understanding the mechanism of establishing a cataphoric dependency in the absence of an overt pronoun cue. Specifically, building on the earlier experimental findings on the subject-oriented nature of the null cataphoric \textit{pro}, we asked whether null \textit{pro} triggers the parser’s active search mechanism that links the \textit{pro} to the subject of the main clause, as described above. Our second goal was to investigate the mechanism of active search in \textit{pro}-cataphora at the level of specific phi-features by exploiting the rich Slovenian morphology. Specifically, we were interested in the mismatch effect in the context of number. \citet{carminati2002, carminati2005} argues that number is a better pronoun disambiguator than gender in ambiguous anaphora contexts in the feature hierarchy (\citealt{Greenberg1963, silverstein1985}).\footnote{Studies such as that of \citet{mancini2014} support the idea that features do not in fact behave in the same way. In self-paced reading of online processing of subject-verb agreement in Italian, where both person and number agreement factors were manipulated, results showed a greater processing penalty for violations of person agreement compared to number agreement. This was interpreted as evidence for separate access to the two features. On the other hand, \citet{vanGompelLiversedge2003}, among others, report a generally similar pattern of processing the number and gender features.} \citet{vanGompelLiversedge2003} showed that number initiates a mismatch effect in a similar fashion to gender, for overt pronoun contexts. To our knowledge, the number feature has not yet been studied in the domain of the mismatch effect in cataphors in the absence of an overt cue. If the incongruence or mismatch effect holds in Slovenian with null \textit{pro}, one can also ask how it is distributed in the time course of reading the respective sentence and how different values of the number feature may modulate this effect, given that, on null hypothesis, the incongruent conditions are expected to manifest a similar performance pattern.
\end{sloppypar}

\section{Experiment}\label{ste-pav:sec:experiment}

We conducted an online self-paced reading experiment in Slovenian \textit{pro}\babelhyphen{hard}cataphora sentences with \textit{when}-subordinate clauses. In this experiment we explored the number congruency effect associated with integration of \textit{pro} with the subject of the main clause and whether this effect is sensitive to the actual number feature of a silent subject pronoun \textit{pro} in the function of a cataphor. In a self-paced reading task, the informant reads individual sentences on a computer screen, with stimulus sentences presented word by word in moving window mode (\citealt{JustWoolley1982}). When the informant presses a predefined key, the first word is displayed. The next time he presses the key, the first word disappears and the next appears. The informant continues in this way until the end of the sentence. Since the informant sees only one word at a time, he must retain the incoming information in his short-term memory. Since the participant does not receive a direct cue to the pronominal reference in the case of the silent \textit{pro}, but must infer it from a more indirect cue, participial agreement, when he encounters the subject of the main clause, the subsequent active search procedure of ``looking forward'' presumably contributes to the load on short-term memory. The self-paced reading paradigm was chosen because it allows us to test the difference between two lexically identical sentences that differ in their functional elements and/or their phi-features. The method thus allows a direct comparison between two related syntactic structures, e.g.\, between \textit{when}-subordinate clause with congruent and incongruent number feature on the second (i.e., main) clause subject. When reading a sentence in self-paced mode (i.e., word by word), readers show longer reading times in the region, which causes additional mental load due to syntactic repairs, less preferred readings, agreement errors, difficulties in integrating syntactic dependencies, etc.

\subsection{Materials}\label{ste-pav:sec:materials}

Our chosen sentences consisted of a main clause and a subordinate clause which linearly preceded the former. The main clause was a transitive clause with a time adverbial phrase and all arguments overtly expressed by referring expressions. The subordinate clause was a copular clause headed by a connector \textit{when}, \textit{if} or \textit{because} (evenly distributed across items). The subordinate (and preceding) clause always included the null subject \textit{pro}, the auxiliary verb \textit{be} as copula, and an attributive adjective (or an adjective-like present participle). 

The target material was arranged in a $2\times2$ design crossing factors \textsc{Congruency} (congruent, incongruent) and \textsc{Number} (sg/pl) on the first (i.e., subordinate) adjective and its accompanying auxiliary verb. This resulted in four conditions in the manipulation: subordinate null subject in singular $+$ main clause overt subject in singular (\REF{ste-pav:ex:relatives-call-uncle-a}; congruent), subordinate null subject in singular $+$ main clause overt subject in plural (\REF{ste-pav:ex:relatives-call-uncle-b}; incongruent), subordinate null subject in plural $+$ main clause overt subject in singular (\REF{ste-pav:ex:relatives-call-uncle-c}; incongruent), and subordinate null subject in plural $+$ main clause overt subject in plural (\REF{ste-pav:ex:relatives-call-uncle-d}; congruent). All target sentences are grammatical in the normal everyday language.
\largerpage[2]

\ea \label{ste-pav:ex:relatives-call-uncle}
\ea{
\gll Ko	je	osamljen,	stric	kliče	sorodnike	po 	več krat	na	dan.\\
when	is.{\AUX.\SG}	lonely.{\SG}	uncle.{\SG}	calls.{\SG}	relatives.{\PL}	at several	times	a	day\\
\glt ‘When he is lonely, the uncle calls relatives several times a day.’ \\ \hfill [$+\text{congr,}$ $+\text{sg}$]
} \label{ste-pav:ex:relatives-call-uncle-a}
\ex{
\gll Ko	je	osamljen,	sorodniki	kličejo	strica	po	več krat	na	dan.\\
when	is.{\AUX.\SG}	lonely.{\SG}	relatives.{\PL}	call.{\PL}	uncle.{\SG}	at several	times	a	day\\
\glt ‘When he is lonely, the relatives call the uncle several times a day.’ \\ \hfill [$-\text{congr,}$ $+\text{sg}$] 
} \label{ste-pav:ex:relatives-call-uncle-b}
\ex{
\gll Ko	so	osamljeni,	stric	kliče	sorodnike	po	več krat	na	dan.\\
when	are.{\AUX.\PL}	lonely.{\PL}	uncle.{\SG}	calls.{\SG}	relatives.{\PL}	at several	times	a	day\\
\glt ‘When they are lonely, the uncle calls the relatives several times a day.’ \hfill [$-\text{congr,}$ $-\text{sg}$] 
} \label{ste-pav:ex:relatives-call-uncle-c}
\ex{
\gll Ko	so	osamljeni,	sorodniki	kličejo	strica	po	več krat	na	dan.\\
when	are.{\AUX.\PL}	lonely.{\PL}	relatives.{\PL}	call.{\PL}	uncle.{\SG}	at several	times	a	day\\
\glt ‘When they are lonely, the relatives call the uncle several times a day.’\\\hfill [$+\text{congr,}$ $-\text{sg}$] 
} \label{ste-pav:ex:relatives-call-uncle-d}
\z
\z

\noindent For each condition, 6~sentences were formed, giving a total of 24~target item sets. They were counterbalanced so that each participant saw only one lexical version of a given item per condition. In addition, 48~filler sentences (32~of which represented conditions from an unrelated experimental manipulation) were added. The total number of stimulus sentences was thus~72. Each sentence (including the filler sentences) was followed by a \textit{yes-no} comprehension question that tested the understanding of the event described in the stimulus sentence. For instance, a sentence from a set like the one in \REF{ste-pav:ex:relatives-call-uncle} could be followed by a question such as ‘Does the uncle visit the relatives?’ expected to be answered with a ‘no’ (for all the sentences in a set). The proportion of correct ‘yes’ and ‘no’ responses to the comprehension questions was evenly distributed across conditions. Per word reading times was the only dependent variable in this manipulation.

\subsection{Participants}\label{ste-pav:sec:participants}

Thirty-three self-reported adult native speakers of Slovenian (21~female,  mean age${}=36.69$, $\text{SD}=14.27$, $\text{median age}=31$) participated in the experiment voluntarily (providing online informed consent), anonymously, and without material compensation. All participants had normal or corrected-to-normal vision and reported no neurological disorders. One participant was excluded because they did not meet the 66.6\%~(two-thirds) accuracy threshold for \textit{yes-no} comprehension questions, pre-set in advance. This left the data from 32~participants for further analysis.

\subsection{Procedure}\label{ste-pav:sec:procedure}\largerpage

Participants were instructed to read the sentences at a natural pace and to make sure that they understood what they were reading. If an incorrect answer was given to a comprehension question, they received feedback. If a correct answer was given, they received no feedback. No answer within 7~seconds was counted as an incorrect answer. Concentration and correct comprehension were checked with a \textit{yes-no} question that followed each sentence and referred to its content. Before the main experiment, subjects read 4 practice sentences to familiarise themselves with the task. The experiment was programmed on the web-based Ibex Farm platform (by Alex Drummond; \url{https://adrummond.net/ibexfarm}). The order of stimulus presentation was pseudo-randomised for each participant by the experimental software, and it was ensured that at least 1 filler sentence was between two target items. The entire experimental session lasted 20--25~minutes. Participants performed the task at a location of their choice without coming to the lab. They were specifically instructed to ensure that external disturbances were kept to a minimum while performing the task.

\subsection{Data analysis}\label{ste-pav:sec:data-analysis}

Only the sentences followed by a correctly answered comprehension question were selected for analysis, which constituted \qty{84.2}{\percent} of the total data. For all analyses, the last two regions (second part of time adverbial phrase) of the sentence were removed. Reading times shorter than \qty{90}{ms} or longer than \qty{3000}{ms} were trimmed as unlikely to have been generated by relevant linguistic processes. This affected approximately \qty{0.2}{\percent} of the total data. Outliers were then identified and excluded from further analysis. The criterion was 3~standard deviations from the mean RT for a given condition and region, for each participant (excluding 79~measurements or additional \qty{1.5}{\percent} of the total data).

To analyse the reading time data, we constructed linear mixed-effects models (\citealt{bates2015}). This allowed us to model individual RTs based on manipulated fixed factors, namely \textsc{Congruency} and \textsc{Number}, while accounting for random variance in the form of participant and item. We used a maximal or near maximal random effects structure adding random slopes for \textsc{Congruency} and \textsc{Number} up to model convergence (\citealt{matuschek2017}). Analyses were conducted using the \textit{lme4} package in~R version~4.0.2 (\citealt{teamR2020}). We report $\chi^2$ and $p$-values for main effects based on the likelihood ratio test, which compares a model containing the fixed effect of interest to a model that is identical in all respects except the fixed effect of interest, using the $\chi^2$~distribution. $P$-values for pairwise comparisons with Tukey adjustment were obtained using the \textit{multcomp} package in~R.

\subsection{Results}\label{ste-pav:sec:results}

The time course of reading sentences in all four conditions is shown in \figref{ste-pav:fig:TimeCourse}. Overall, reading times were higher in the non-congruent conditions (a subordinate null subject in singular followed by a plural main-clause subject (Npl) and a subordinate null subject in plural followed by a singular main-clause subject (Nsg)) than in the congruent conditions. Total reading times with standard errors per condition are shown in \tabref{ste-pav:tab:contrasts-congruence}.

%Figure1
\begin{figure}
    \includegraphics[width=\textwidth]{figures/TimeCourse.jpg}
    \caption{Time course of self-paced reading (the last two regions not shown)}
    \label{ste-pav:fig:TimeCourse}
\end{figure}

%Table1
\begin{table}
\begin{tabularx}{\textwidth}{ll@{~~}c@{~~}c@{~~}lll}
\lsptoprule
\multicolumn{2}{l}{Main effect (CONGR)} & Conditions & Total RT (ms) & SE & \multicolumn{2}{l}{Main effect (NUM)} \\
\midrule
 &\cellcolor{black!10}{} & \cellcolor{black!10}pl-Npl & 4751 & 220 & \cellcolor{black!10}PL & \\
$\chi^2(2)=11.102$ &\multirow{-2}{*}{\cellcolor{black!10}$+$congr}  & \cellcolor{black!10}sg-Nsg & 4849 & 286 & \cellcolor{black!5} &  $\chi^2(2)=6.461$\\
$p=0.0008$*** & \cellcolor{black!5} & \cellcolor{black!5}sg-Npl & 5254 & 299 &\multirow{-2}{*}{\cellcolor{black!5}SG} &$p=0.01$**\\
 & \multirow{-2}{*}{\cellcolor{black!5}$-$congr} & \cellcolor{black!5}pl-Nsg & 4976 & 242 & \cellcolor{black!10}PL\smallskip &  \\
\multicolumn{7}{c}{Interaction CONGR*NUM: $\chi^2(1)=5.0685, p=0.024$*} \\
\lspbottomrule
\end{tabularx}
\caption{Contrasts across the Congruence and Number factors, total reading times}
\label{ste-pav:tab:contrasts-congruence}
\end{table}

As \tabref{ste-pav:tab:contrasts-congruence} shows, there are overall main effects of \textsc{Congruency} as well as \textsc{Number}. Moreover, \textsc{Congruency} interacted with \textsc{Number}: there was no difference in reading times between sentences with singular and plural subordinate null subjects in the congruent conditions but sentences with singular subordinate null subjects were read more slowly (about \qty{40}{ms} per word) than those with plural subordinate null subjects in the incongruent conditions. The main sites of slow-down were primarily the post-antecedent regions, i.e., the verb phrase following the second (i.e., main) clause subject.

Per region analyses revealed that there were no main effects or interactions up to the main verb region ($p$s${}>0.10$). At the main verb region (cf.\ \textit{kliče}), there is no main effect of congruence, but there is a marginal main effect of \textsc{Number} ($\chi^2(2)=4.85$, $p=0.08$) indicating that conditions with the singular subordinate null subjects are read about \qty{50}{ms} slower at this region, and there is a marginal interaction of the two factors ($\chi^2(1)=3.39$, $p=0.065$); pairwise comparisons indicate that the sg-Npl condition stands out in terms of higher reading times compared to the other conditions, although the contrasts do not quite reach significance ($ps>0.10$). Furthermore, at the direct object region there is again no main effect of congruence, but there is a significant main effect of \textsc{Number} ($\chi^2(2)=17.32$, $p<0.001$) and there is an interaction between the two factors ($\chi^2(1)=11.52$, $p<0.001$). Pairwise comparisons show that this region takes longer to read in the incongruent sg-Npl condition than in the congruent condition sg-Nsg ($β=161.84$, $\text{SE}=33.4$, $t=4.844$, $p<0.001$), as well as in the other incongruent condition, namely, pl-Nsg ($β=138.67$, $\text{SE}=33.6$, $t=4.125$, $p<0.001$). In contrast, there is no difference in reading this region in the other incongruent condition, pl-Nsg, in comparison to the corresponding congruent condition pl-Npl ($β=1.08$, $\text{SE}=33.5$, $t=0.032$, $p>0.10$). The main effect of \textsc{Number} marginally persists up to the next region (the preposition in \figref{ste-pav:fig:TimeCourse}, ($\chi^2(2)=4.52$, $p=0.10$); no other effects were observed in this and the final regions.

\section{Discussion and conclusions}\label{ste-pav:sec:discussion-conclusions}\largerpage

The congruency or mismatch effect observed in our experiment suggests that Slovenian speakers are sensitive to the interpretational properties of the pronoun, despite its silent character: the parser initiates an active search mechanism in the case of the null subject \textit{pro}, just as it would in the case of the overt pronoun. Our results are largely consistent with those of the eye-tracking experiment in which \citeauthor{vanGompelLiversedge2003} (\citeyear{vanGompelLiversedge2003}, Experiment 3) tested the number-mismatch effect in overt cataphora contexts (cf.\ \ref{ste-pav:ex:when-appeared} above). Van Gompel and Liversedge reported significantly prolonged first-pass reading times (i.e., the sum of all fixation durations from the first fixation within a region to a fixation outside the region) in cases of incongruence or number mismatch (cf.\ \ref{ste-pav:ex:when-appeared-b}), compared to congruent cases (cf.\ \ref{ste-pav:ex:when-appeared-a}) in (i) the region immediately following the main subject NP (which in their case is an adverb that does not occur in \REF{ste-pav:ex:when-appeared}), (ii) on the main verb (cf.\ `visited' in \REF{ste-pav:ex:when-appeared-b}, difference only by item), and they also reported significant first-pass regressions (i.e., the percentage of leftward eye movements crossing the left boundary of the region initiated immediately after a first-pass fixation in the region) on the direct object; the effects decay after this region. In our study, the per region dynamics is very similar for online reading: the divergence starts at the verb (there was no preverbal element in our stimuli, such as an adverb in the above study), goes steeply up to the direct object, and decays in the final regions. But there are also at least two important differences between our results and those of Van Gompel and Liversedge.

First, in our study, the main effect of congruency was observed only for total reading times across critical regions, but not for per-region measurements, whereas the authors cited above report this main effect for three critical post-subject regions. This suggests that our congruence effect is less ``pronounced'' than that in Van Gompel and Liversedge’s study, as the difference per region is sufficient to sum up to a global-level effect, but insufficient to independently mark individual regions. If this difference proves robust, it may indeed point to an important aspect in which the processing of a \textit{pro}-cataphora differs from that of overt pronominal cataphors in previous studies. An obvious caveat is that the experimental methodology of the two studies is different. Whether the contrast remains when the methodology is made consistent needs to be investigated further.\largerpage

Another important difference between the two studies is that in our study congruency interacted with number consistently across the (post-subject) regions of interest, whereas in Van Gompel and Liversedge's study no interaction between congruency and number was observed in any post-subject region; each of the two factors affected eye movement measures independently. Our study also revealed the main source of this interaction, namely the sg-Npl condition. The fact that the two factors interacted consistently, with no main effect of congruency in specific regions, may suggest that the parser has an increased sensitivity to the number feature in the context of the active search mechanism activated by the silent \textit{pro}. Recall that there is no overt pronominal cue to the number feature, so the parser must infer the number feature based only on the inflection of the subordinate copula plus adjective. Van Gompel and Liversedge (\citeyear{vanGompelLiversedge2003}) argue, based on their results, that the use of morphological information occurs only after coreference relations have been computed (see also \citealt{cowart1987, kazaninaLauPhillips2007, kazaninaPhillips2010}). Our results in the sg-Npl condition are broadly consistent with this conjecture. However, the performance of our speakers in the other incongruent condition, namely pl-Nsg, casts doubt on it: In this condition, neither a congruency nor a mismatch effect was observed, a priori suggesting that speakers use the morphological information about number early enough. This divergent pattern calls for an explanation. Here we offer some initial thoughts on a possible line.{\interfootnotelinepenalty=10000\footnote{An anonymous reviewer suggests that the mismatch in the pl-Nsg condition could be tolerated because of an additional available parse compatible with a split antecedent reading, as in, e.g.\ \textit{When pro\textsubscript{$i+j$} are lonely, the uncle\textsubscript{i} calls the relatives\textsubscript{j} several times a day.} While a reasonable possibility for Slovenian, it does not sit easily with the results of \citeposst{vanGompelLiversedge2003} original English experiment (with an overt cataphora) whereby no contrast was reported between the two non-matching conditions (cf.\ \ref{ste-pav:ex:when-appeared-c} vs. \ref{ste-pav:ex:when-appeared-d}).}}

One possibility is that the parser behaves differently when it tries to establish a coreference in incongruent contexts with a singular and a plural \textit{pro} in Slovenian: While the active search mechanism accesses the singular value of \textit{pro} from the beginning (and before the coreference is established), an alternative, more global parsing strategy could work with the plural \textit{pro} in accordance with the schedule a la Van Gompel and Liversedge. This alternative is clearly unattractive, as it seems to overstate the relationship between parsing strategies and lexically encoded information about \textit{pro} nominals, such as morphological features. Another alternative, which we consider more feasible and promising, is that an additional factor plays a role in modulating the active parsing scenario, which is sensitive to specific number features.

We hypothesise that this additional factor is grammatical in nature and has to do with the way various number features are semantically encoded in the grammar module, the latter playing an active role in driving sentence processing. Informally speaking, this encoding has to do with markedness of certain feature values. In contrast to the commonly held view in theoretical and experimental research that the singular is unmarked while the plural is a marked form (cf.\ \citealt{bock1993}), there is a growing consensus in the recent semantics literature on the opposite view according to which the singular is endowed with an additional property in its lexical entry, namely the singularity presupposition (the presupposition that the cardinality of the set in question is exactly~1). In this sense, the singular is semantically more ``loaded'' than the plural and can therefore be regarded as having a more marked value (\citealt{sauerland2003, sauerland.etal2005, spector2007}).\footnote{The arguments for this view come from the domains of using the plural under the scope of negation, downward-entailing operators and the like. See the references in the text for more discussion.}

Establishing a coreference in real time involves matching a previously activated feature value between \textit{pro} and its antecedent (the latter term is of course not very appropriate in the cataphora context). In the case of an incongruent sg-Npl condition, this matching needs to include the singularity presupposition of \textit{pro}: Since the (plural) main clause subject lacks this property, the matching cannot be complete and the mismatch effect occurs. In contrast, in the incongruent pl-Nsg condition, there is no element of presupposition checking in the process of plural \textit{pro} establishing coreference. A possible mismatch effect is thus excluded. Note that this scenario rests on the assumption that the feature matching procedure is asymmetric. This naturally follows from the ``forward-looking'' character of cataphoric dependency formation: the singularity presupposition is triggered by the element initiating the dependency, that is, \textit{pro}. The antecedent just needs to match this property, not the other way around. 

This of course raises a question as to why a similar avoidance effect does not occur in the corresponding English constructions of Van Gompel and Liversedge (cf.\ \ref{ste-pav:ex:when-appeared}), where the incongruence or mismatch effect occurs in both directions. We believe the answer has to do with the morphological realization of the corresponding pronoun (overt vs. null). Null \textit{pro} is generally considered underspecified compared to overt pronominal and may instantiate less morphosyntactic structure than the latter (\citealt{cardinaletti1999}). It is possible that in English checking the morphological plural feature is additionally required as part of establishing coreference given that the parser activates it by reading the overt cue (the pronoun itself), whereas in Slovenian this additional process is not necessarily initiated due to the phonologically silent character of \textit{pro}. In other words, besides the semantic part of establishing coreference, the English coreference formation includes a morphological part, whereas the Slovenian dependency processing does not. This would explain the divergent way the incongruent conditions are processed in Slovenian in the general context of the active search mechanism and highlight another difference between \textit{pro}-drop and non-\textit{pro}-drop languages in terms of cataphora resolution. Moreover, this provides an interesting starting point for further research, possibly involving other \textit{pro}-drop languages and\slash or feature continua.

The above line of argument underscores the role of the morphological component in establishing coreference. Within a model of syntactic parsing of the weak interactive type (e.g.\ \citealt{altmann1988}) the processor tries to compute a coreference relation between the cataphoric pronoun and the first available antecedent. In the pl-Nsg incongruent condition, the unmarked character of \textit{pro} does not prevent establishing this coreference relation in either English or Slovenian but the additional morphological processing routine results in a mismatch that blocks this relation in the former, but not the latter. According to this model, processing difficulty in this condition occurs because the syntactic component of the processor allows for the coreference but the morphological information on the overt pronoun in English is inconsistent with it. In the sg-Npl incongruent condition, the marked character of singular \textit{pro} triggers the mismatch effect in both languages regardless of the morphological realization. Alternatively, within the modular model of processing, the coreference relation happens during the first state of analysis on the basis of only syntactic information. Disruption due to an additional morphological process in English, but not in Slovenian, happens at the second, post-syntactic stage when the processor recognizes the initial analysis as inconsistent with morphological information on the pronoun and therefore has to revise the initially postulated coreference relation.

Overall, the patterns of results observed in this study demonstrated that the null \textit{pro}, postulated on the basis of agreement information in the auxiliary+ad\-jec\-tive complex, initiates an active search for an incoming NP as a target antecedent. Establishing a cataphoric coreference with null \textit{pro} proceeds similarly in many respects to the corresponding process with overt pronoun, with some important differences in terms of the construction of an online representation of the coreference that bears on the overt\slash null morphological distinction.

\section*{Acknowledgments}
\begin{sloppypar}
Arthur Stepanov’s research was financially supported by the ARRS research project no.~J6-1805. The authors thank Penka Stateva for useful discussions of this material.
\end{sloppypar}


\printbibliography[heading=subbibliography,notkeyword=this]

\end{document}
