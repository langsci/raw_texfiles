\documentclass[output=paper]{langscibook} 
\ChapterDOI{10.5281/zenodo.10123647}


\title[Slavic \textsc{l-}periphrases]{Slavic \textsc{l-}periphrases: Linguistic change and variation}
\author{Hagen Pitsch\affiliation{Georg-August-Universität Göttingen}}
\SetupAffiliations{mark style=none}
\abstract{The present paper addresses the variation in \textsc{l-}periphrases looking at a broad range of modern Slavic languages. Based on a thorough description, a typological division between \textsc{auxiliary languages} and \textsc{particle languages} is proposed. The difference between them is then motivated by sketching diachronic scenarios of linguistic change and subsequently given syntactic analyses. In sum, the paper reveals a remarkable variation that has so far been widely disregarded from a theoretical point of view. 

\keywords{past, perfect, preterit, conditional, tense, mood, linguistic change}
}

\lsConditionalSetupForPaper{}
\begin{document}
\maketitle

\section{Introduction}

A common thread of present-day Slavic languages is that they use \textsc{\textsc{l-}periphrases} to express specific tenses and moods, namely the future, the preterit (perfect or generalized past), and the conditional; see \REF{pitsch:ex:fut}, \REF{pitsch:ex:perf}, and \REF{pitsch:ex:cond}, respectively.\footnote{Unless otherwise indicated, examples are constructed by myself.}\textsuperscript{,}\footnote{Kashubian has several stem variants for its future auxiliary. Besides \textit{md-} illustrated in \REF{pitsch:ex:fut_Kashubian}, the stem can be \textit{będ-}, \textit{bãd-}, or \textit{bd-} \citep[see][776--777]{Stone1993}.}\textsuperscript{,}\footnote{BCMS has \textsc{l-}future forms only in temporal and conditional clauses like \REF{pitsch:ex:fut_BCMS}. If perfective, they are interpreted as a future perfect, otherwise as a simple future.}
    
\ea \textsc{l-}future
\ea\gll Md-ã pisa-ł-a.\\
\textsc{fut-1sg} write\textsc{-l-sg.f}\\ 
\glt `I shall be writing.' \hfill (Kashubian) \label{pitsch:ex:fut_Kashubian}
\ex\gll Będ-ą prosi-ł-y o pokój na {wyższych piętrach}.\\
\textsc{fut-3pl} ask\textsc{-l-pl.f} for room.\textsc{acc} on {higher floors.\textsc{loc}}\\
\glt `They shall ask for room on the higher floors.' \hfill (Polish)
\ex\gll Kad bude-mo govori-l-i s Marijom, sve će biti jasno.\\
when \textsc{fut-1pl} speak\textsc{-l-pl.m} with Maria.\textsc{ins} all \textsc{fut.3sg} be.\textsc{inf} clear\\ 
\glt `When we speak with Marija (in the future), everything will be clear.' \\ \hfill \hfill (BCMS; \citealt[331]{Browne1993}) \label{pitsch:ex:fut_BCMS}
\ex\gll Prosi-l-a bo-š za dopust.\\
ask\textsc{-l-sg.f} \textsc{fut-2sg} for vacation.\textsc{acc}\\ 
\glt `You shall apply for leave.' \hfill (Slovene)
\z \label{pitsch:ex:fut}
\z
    
\ea \textsc{l-}preterit
\ea\gll Ima-l-a je razgovor sa psihologom.\\
have\textsc{-l-sg.f} be.\textsc{3sg} talk.\textsc{acc} with psychologist.\textsc{ins}\\ 
\glt `She had a talk with her psychologist.' \hfill (BCMS)
\ex\gll Wona je dźěła-ł-a jako bibliotekarka.\\
she be.\textsc{3sg} work\textsc{-l-sg.f} as librarian.\textsc{sg.f}\\ 
\glt `She has been working as a librarian.' \hfill (Upper Sorbian)
\ex\gll Ma-l {veľké šťastie}.\\
have\textsc{-l.sg.m} {big luck.\textsc{acc}}\\ 
\glt `He had enormous luck.' \hfill (Slovak)
\ex\gll Koly ty narody-l-a-s'?\\
when you give-birth\textsc{-l-sg.f-refl}\\ 
\glt `When were you born?' \hfill (Ukrainian)
\z \label{pitsch:ex:perf}
\z

\ea \textsc{l-}conditional
\ea\gll Ima-l-a bi-h sigurno napad panike.\\
have\textsc{-l-sg.f} \textsc{cond-1sg} certainly attack.\textsc{acc} panic.\textsc{gen}\\ 
\glt `I would certainly have a panic attack.' \hfill (BCMS)
\ex\gll Da bi se v žlici vode utopi-l!\\
\textsc{part} \textsc{cond} \textsc{refl} in spoon.\textsc{loc} water.\textsc{gen} drown\textsc{-l.sg.m}\\ 
\glt `May you drown in a spoonful of water!' \hfill \hfill (Slovene; \citealt[431]{Priestly1993})
\ex\gll Ma-l-a by som ísť do postele.\\
have\textsc{-l-sg.f} \textsc{cond} be.\textsc{1sg} go.\textsc{inf} to bed.\textsc{gen}\\ 
\glt `I should go to bed.' \hfill (Slovak)
\ex\gll Ja \minsp{\{} b\} c'oho ne skaza-v \minsp{\{} by\}.\\
I {} \textsc{cond} this.\textsc{gen} \textsc{neg} say\textsc{-l.sg.m} {} \textsc{cond}\\ 
\glt `I would not have said that.' \hfill \hfill (Ukrainian; \citealt[158]{AmirBabenko2007})
\z \label{pitsch:ex:cond}
\z

\noindent The general format of \textsc{l-}periphrases is given in \REF{pitsch:ex:definition}.\footnote{As the auxiliary unit (AU) is absent in a subset of cases, ``periphrasis'' seems to be partly inadequate to characterize the verb forms in question. Later on, however, I will show that, syntactically, all cases are indeed bipartite/analytic.}

\ea (AU) V\textsubscript{L} \label{pitsch:ex:definition} 
\z

\noindent In \REF{pitsch:ex:definition}, ``AU'' and ``V\textsubscript{L}'' stand for \textsc{auxiliary unit} and \textsc{verbal \textsc{l-}form}, respectively. I prefer ``AU'' over the more familiar notion ``auxiliary'' due to its being more neutral: Saying ``auxiliary'', one usually thinks of an inflected verb form. While AUs in Slavic \textsc{l-}periphrases can indeed be inflected verb forms -- and always are in \textsc{l-}futures --, they may also be noninflected, in which case they are commonly called \textsc{particles}. This is why in \REF{pitsch:ex:aux}, which shows the general morphological makeup of Slavic AUs, I put the agreement categories in brackets.

\ea AU: stem(-\textsc{person/number}) \label{pitsch:ex:aux} \z
    
\noindent Finally, \REF{pitsch:ex:lform} depicts the general morphological structure of V\textsubscript{L}s.
 
\ea V\textsubscript{L}: stem\textsc{-l-}\textsc{number(/gender)} \label{pitsch:ex:lform} \z

\noindent The variation between absent and present and -- if present -- inflected and noninflected AUs is the main issue of the present paper. It aims at (i) giving a detailed description of this cross-Slavic variation, (ii) reconstructing it from a theoretical point of view, (iii) integrating the perspective of linguistic change, and (iv) putting forward a syntax-based formalization of the auxiliary/particle distinction as manifested in Slavic \textsc{l-}periphrases, most notably the \textsc{l-}preterit and the \textsc{l-}conditional.\footnote{The motivation for having such a formalization is that the linguistic notions of auxiliary and particle, while ubiquitous in the literature, are still very vague.}

To that end, \sectref{pitsch:sec:description} gives a detailed description of the relevant \textsc{l-}periphrases.
%\footnote{This paper addresses the following Slavic languages: Belarusian, Bosnian-Croatian-Monte\-negrin-Serbian (BCMS), Bulgarian, Kashubian, Czech, Macedonian, Lower Sorbian, Russian, Slovene, Slovak, Ukrainian, and Upper Sorbian.} 
In \sectref{pitsch:sec:change}, I sketch a set of diachronic scenarios of linguistic change which are likely to have given rise to the present-day situation. Finally, \sectref{pitsch:sec:syntax} presents my claims as to the syntax of Slavic \textsc{l-}periphrases. \sectref{pitsch:sec:summary} summarizes the paper.

%%% %%% %%% %%%  %%% %%% %%% %%% 

\section{Description}\label{pitsch:sec:description}

This part describes the cross-Slavic variation in the \textsc{l-}preterit and \textsc{l-}conditional, leaving aside the \textsc{l-}future due to the fact that it does not display any variation in the languages that have it.

%%% %%% %%% %%%  

\subsection{\textsc{l-}preterit}\label{pitsch:sec:perfect}

\subsubsection{The general picture}

All present-day Slavic languages exhibit verb forms related to the Late Proto-Slavonic present-perfect periphrasis, which consisted of a present-tense form of the \textsc{be}-auxiliary (showing person and number; e.g., Old Church Slavic \textit{jesm\u{\i}} `am', \textit{jesi} `are', etc.) and the main verb V\textsubscript{L}. While these forms retain their original present-perfect meaning in Bulgarian and the standard varieties of BCMS, Macedonian, and Sorbian, they have developed into a general(ized) past in the remaining languages/varieties. To conflate these notions, I use the term \textsc{l-preterit}.

On the other hand, the modern languages show considerable variation concerning the shape of the AU: Some have clitic \textsc{be}-auxiliaries inflected for person and number throughout the paradigm; see \tabref{pitsch:tab:perfect_full}.\footnote{As to Kashubian, \citet[776]{Stone1993} notes that the variant AU $+$ V\textsubscript{L} (``Kashubian-A'') is widely used in the literature and characteristic in the spoken language of the older generation, while elsewhere, the preterit consists of V\textsubscript{L} only (``Kashubian-B'' in \tabref{pitsch:tab:perfect_none}). See also \citet{Menzel2013} and \citet{Bartelik2015}. Note that descriptions vary. Thus, \citet[268]{Lubas2002} and \citet[174]{Breza2009} make no mention of the AU-less variant.} A smaller subset of languages has inflected auxiliaries everywhere in the paradigm except for the 3rd person; see \tabref{pitsch:tab:perfect_3rd} (page~\pageref{pitsch:tab:perfect_3rd}). Finally, East Slavic languages and Kashubian spoken by younger speakers lack AUs altogether; see \tabref{pitsch:tab:perfect_none}.

\tabref{pitsch:tab:perfect_overview} provides an overview. It shows that the variation cuts across the traditional division between South and West Slavic, while the East Slavic languages behave uniformly. Kashubian comes in two varieties: Kashubian-A (literary language and older speakers) aligns with ``minor'' West Slavic languages, whereas Kashubian-B (younger speakers) resembles East Slavic. In \sectref{pitsch:sec:EastSlavic} and \sectref{pitsch:sec:Kashubian}, respectively, I describe the diachronic background underlying the absence of AUs in the \textsc{l-}preterit in East Slavic and Kashubian-B. On the other hand, Polish AUs stand out from AUs in the remaining languages in that they seem to be suffixes. Again, some diachronic background is supplied not only to track the changes underlying the present-day situation but also to arrive at assumptions about the syntax of the relevant AUs. That background is presented in \sectref{pitsch:sec:Polish}.

% Tabelle 1

\begin{table}
\begin{tabular}{llllll}
\lsptoprule
& & \multicolumn{2}{c}{\textsc{sg}} & \multicolumn{2}{c}{\textsc{pl}} \\\cmidrule(lr){3-4}\cmidrule(lr){5-6}
& & AU & V\textsubscript{L} & AU & V\textsubscript{L} \\
\midrule
BCMS & 1 & \textit{sam} & \textit{pisala} & \textit{smo} & \textit{pisale} \\ 
`write'    & 2 & \textit{si} & \textit{pisala} & \textit{ste} & \textit{pisale} \\
    & 3 & \textit{je} & \textit{pisala} & \textit{su} & \textit{pisale} \\\addlinespace
Bulgarian & 1 & \textit{s\u{a}m} & \textit{\v{c}ela} & \textit{sme} & \textit{\v{c}eli} \\ 
`read'            & 2 & \textit{si} & \textit{\v{c}ela} & \textit{ste} & \textit{\v{c}eli} \\ 
            & 3 & \textit{e} & \textit{\v{c}ela} & \textit{sa} & \textit{\v{c}eli} \\\addlinespace
Slovene & 1 & \textit{sem} & \textit{pohvalila} & \textit{smo} & \textit{pohvalile} \\ 
`praise'            & 2 & \textit{si} & \textit{pohvalila} & \textit{ste} & \textit{pohvalile} \\ 
            & 3 & \textit{je} & \textit{pohvalila} & \textit{so} & \textit{pohvalile} \\\addlinespace
Lower Sorbian & 1 & \textit{som} & \textit{słyšała} & \textit{smy} & \textit{słyšali} \\ 
`hear'                & 2 & \textit{sy} & \textit{słyšała} & \textit{sćo} & \textit{słyšali} \\ 
                & 3 & \textit{jo} & \textit{słyšała} & \textit{su} & \textit{słyšali} \\\addlinespace
Upper Sorbian & 1 & \textit{sym} & \textit{dźěłała} & \textit{smy} & \textit{dźěłali} \\ 
`work'            & 2 & \textit{sy} & \textit{dźěłała} & \textit{sće} & \textit{dźěłali} \\
            & 3 & \textit{je} & \textit{dźěłała} & \textit{su} & \textit{dźěłali} \\\addlinespace
Kashubian-A & 1 & \textit{jem} & \textit{robiła} & \textit{jesmë} & \textit{robiłë} \\ 
`make, work'            & 2 & \textit{jes} & \textit{robiła} & \textit{jesta} & \textit{robiłë} \\ 
            & 3 & \textit{je} & \textit{robiła} & \textit{są} & \textit{robiłë} \\ 
\lspbottomrule
\end{tabular}
    \caption{\textsc{l-}preterit with inflected AU throughout}
    \label{pitsch:tab:perfect_full}
\end{table}

% Tabelle 2

\begin{table}
\begin{tabular}{llllll}
\lsptoprule
& & \multicolumn{2}{c}{\textsc{sg}} & \multicolumn{2}{c}{\textsc{pl}} \\\cmidrule(lr){3-4}\cmidrule(lr){5-6}
& & AU & V\textsubscript{L} & AU & V\textsubscript{L} \\
\midrule
Macedonian & 1 & \textit{sum} & \textit{molela} & \textit{sme} & \textit{molele} \\ 
`ask'    & 2 & \textit{si} & \textit{molela} & \textit{ste} & \textit{molele} \\ 
    & 3 & & \textit{molela} & & \textit{molele} \\\addlinespace
Czech & 1 & \textit{jsem} & \textit{udělala} & \textit{jsme} & \textit{udělaly} \\ 
`make'            & 2 & \textit{jsi} & \textit{udělala} & \textit{jste} & \textit{udělaly} \\ 
            & 3 & & \textit{udělala} & & \textit{udělaly} \\\addlinespace
Slovak & 1 & \textit{som} & \textit{volala} & \textit{sme} & \textit{volali} \\ 
`call'            & 2 & \textit{si} & \textit{volala} & \textit{ste} & \textit{volali} \\ 
            & 3 & & \textit{volala} & & \textit{volali} \\\addlinespace
Polish & 1 & \textit{-m} & \textit{prosiła} & \textit{-śmy} & \textit{prosiły} \\ 
`ask'            & 2 & \textit{-ś} & \textit{prosiła} & \textit{-ście} & \textit{prosiły} \\ 
            & 3 & & \textit{prosiła} & & \textit{prosiły} \\ 
\lspbottomrule
\end{tabular}
\caption{\textsc{l-}preterit without inflected AU in the 3rd person}
\label{pitsch:tab:perfect_3rd}
\end{table}

% Tabelle 3

\begin{table}
\begin{tabular}{llll}
\lsptoprule
& & \textsc{sg} & \textsc{pl} \\ 
& & V\textsubscript{L} & V\textsubscript{L} \\
\midrule
Belarusian & 1 & \textit{čytala} & \textit{čytali} \\ 
`read'    & 2 & \textit{čytala} & \textit{čytali} \\ 
    & 3 & \textit{čytala} & \textit{čytali} \\\addlinespace
Russian & 1 & \textit{skazala} & \textit{skazali} \\ 
`say'            & 2 & \textit{skazala} & \textit{skazali} \\ 
            & 3 & \textit{skazala} & \textit{skazali} \\\addlinespace
Ukrainian & 1 & \textit{bula} & \textit{buly} \\ 
`be'            & 2 & \textit{bula} & \textit{buly} \\ 
            & 3 & \textit{bula} & \textit{buly} \\\addlinespace
Kashubian-B & 1 & \textit{robiła} & \textit{robiłë} \\ 
`make, work'            & 2 & \textit{robiła} & \textit{robiłë} \\ 
            & 3 & \textit{robiła} & \textit{robiłë} \\ 
\lspbottomrule
\end{tabular}
    \caption{\textsc{l-}preterit without AU throughout}
    \label{pitsch:tab:perfect_none}
\end{table}

% Tabelle 4 (Überblick Perfekt/Präteritum)

\begin{table}
\begin{tabular}{lccc}
\lsptoprule
& \multicolumn{3}{c}{AU in {\dots} person} \\
& 1st & 2nd & 3rd \\
\midrule
BCMS & $\bullet$ & $\bullet$ & $\bullet$ \\ 
Slovene & $\bullet$ & $\bullet$ & $\bullet$ \\
Bulgarian & $\bullet$ & $\bullet$ & $\bullet$ \\ 
Macedonian & $\bullet$ & $\bullet$ & \\\addlinespace
Czech & $\bullet$ & $\bullet$ & \\
Slovak & $\bullet$ & $\bullet$ & \\
Polish & $\bullet$ & $\bullet$ & \\
Lower Sorbian & $\bullet$ & $\bullet$ & $\bullet$ \\
Upper Sorbian & $\bullet$ & $\bullet$ & $\bullet$ \\
Kashubian-A & $\bullet$ & $\bullet$ & $\bullet$ \\ 
Kashubian-B & & & \\\addlinespace
Belarusian & & & \\
Russian & & & \\
Ukrainian & & & \\
\lspbottomrule
\end{tabular}
\caption{Cross-Slavic variation in the \textsc{l-}preterit}
\label{pitsch:tab:perfect_overview}
\end{table}

%%% %%% %%% %%% 

\subsubsection{Auxiliary loss in East Slavic}\label{pitsch:sec:EastSlavic}

Beginning with the 11th century, Old East Slavic gradually lost the present-tense paradigm of \textit{byti} `be' (see, a.o., \cites{Issatchenko1940}[391]{Ivanov1964}[298]{BorkovskijKuznecov1965}{Sokolova2017}). First of all, this process affected the third-person forms (\textsc{3sg} \textit{jestʹ}, \textsc{3pl} \textit{sjatʹ}), the remaining forms (\textsc{1sg} \textit{esmʹ}, \textsc{2sg} \textit{esi}, \textsc{1pl} \textit{esme}, \textsc{2pl} \textit{este}) following suit.\footnote{The same sequence of changes can be traced for Old Polish (see \citealt[127--128]{Decaux1955}, \citealt[41]{Migdalski2006}).} As a consequence, the present-perfect paradigm, formerly periphrastic, lost the AU without substitution, turning it effectively into a synthetic form consisting exclusively of V\textsubscript{L}. The scheme in \REF{pitsch:ex:OldRussianChange_Perfect} depicts this change, using the 2nd singular preterit of the verb \textit{čitati} `read' as an illustration.\largerpage[-2]

\ea 
\gll AU $+$ V\textsubscript{L} $\longrightarrow$ V\textsubscript{L} \\
\textit{jesi} {} \textit{čitala} {} \textit{čitala} \\
\label{pitsch:ex:OldRussianChange_Perfect}
\z

\noindent The absence/loss of the AU had further implications: For one thing, the by now solitary V\textsubscript{L}, once a participle, acquired the role of the finite verb. Nonetheless, it retained its original nominal agreement (number and gender), thus leaving person unexpressed in the verbal domain. This in turn added significance to (the use of) overt personal pronouns to avoid ambiguity \citep[see][193]{Issatchenko1940}.\footnote{It is not clear from the literature whether the loss of the \textit{byti}-forms fostered the more frequent use of personal pronouns or whether it was the other way around. Fortunately, this issue is of minor importance for the present investigation.} 

AU-lessness is characteristic of all present-day East Slavic languages. To capture it for Russian, \citet{Junghanns1995} claims a new agreement pattern in the past tense with the person feature underspecified; see \REF{pitsch:ex:Junghanns_a} as opposed to the ``canonical'' non-past pattern in \REF{pitsch:ex:Junghanns_b}.

\ea
\ea
{[}$-$\textsc{past}] $\rightarrow$ [\textalpha\!  \textsc{person}, \textbeta\! \textsc{number}, $\varnothing$\! \textsc{gender}]\label{pitsch:ex:Junghanns_a}
\ex {[}$+$\textsc{past}] $\rightarrow$ [$\varnothing$\! \textsc{person}, \textbeta\! \textsc{number}, \textgamma\! \textsc{gender}] \label{pitsch:ex:Junghanns_b}\\\hfill \citep[see][174]{Junghanns1995}
\z
\z

\noindent Due to the loss of the present-tense paradigm of \textit{byti}, present-day East Slavic languages are also ``copula-less''.\footnote{\citet[192]{Issatchenko1940} applies Leonard Bloomfield's term \textsc{equational predications} to the resulting copula structures.} In emphatic (verum and contrastive focus; see \citealt[127]{Geist2007}) contexts, however, the former \textsc{3sg} form -- Belarusian \textit{ëscʹ}, Russian \textit{estʹ}, Ukrainian \textit{je} -- survives but has lost its agreement specification and thus occurs in all persons and numbers. This leads \citet[192]{Issatchenko1940} to the statement that Russian \textit{estʹ} ``has lost its verbal character; it has become an impersonal particle.''\footnote{Especially speakers from the Western Ukraine may employ \textit{je} in place of the zero copula in all persons and numbers. Elsewhere, the zero copula is the default choice.} Moreover, equational and identificational clauses involve a (de-pronominal) particle: Belarusian \textit{hėta}, Russian \textit{ėto}, Ukrainian \textit{ce}. Crucially, all these particles can by no means function as AUs in the \textsc{l-}preterit. These observations will be taken up in \sectref{pitsch:sec:syntax}.

%%% %%% %%% %%% 

\subsubsection{Auxiliary drop in Kashubian-B}\label{pitsch:sec:Kashubian}\largerpage

There are two ways to form the \textsc{l-}preterit in Kashubian: Either V\textsubscript{L} is combined with an inflected \textsc{be}-auxiliary as schematized in \REF{pitsch:ex:definition_KashubianA} or V\textsubscript{L} is used alone as in \REF{pitsch:ex:definition_KashubianB} \citep[see][130--134]{BrezaTreder1981}.\footnote{According to \citet[776]{Stone1993}, pattern \REF{pitsch:ex:definition_KashubianA} is characteristic of older speakers, while younger speakers prefer \REF{pitsch:ex:definition_KashubianB}. See \citet{Menzel2013} for a corpus-based discussion. Crucially, there is no Polish-like variant of the preterit with reduced (``suffixal'') agreement markers (see \sectref{pitsch:sec:Polish}).}

\ea
\ea AU V\textsubscript{L} \hfill (Kashubian-A) \label{pitsch:ex:definition_KashubianA}
\ex V\textsubscript{L} \hfill (Kashubian-B) \label{pitsch:ex:definition_KashubianB} 
\z\z


\noindent \citet[100]{Rittel1970} assumes that the situation in \REF{pitsch:ex:definition_KashubianB} was fostered by the increased use of personal pronouns (allegedly induced by language contact with German; see also \citealt{Nomachi2014}), which resembles the development described for East Slavic languages in \sectref{pitsch:sec:EastSlavic}. It is fair to assume that the co-existence of the two patterns documents a linguistic change in progress which parallels the change in Old East Slavic sketched in \REF{pitsch:ex:OldRussianChange_Perfect}. An analogous scheme for Kashubian-B is given in \REF{pitsch:ex:Kashubian_Perfect} using the \textsc{1pl} preterit of the verb \textit{robic} `make, work' as an illustration.

\ea 
\gll AU $+$ V\textsubscript{L} $\longrightarrow$ V\textsubscript{L} \\
\textit{jesmë} {} \textit{robiłë} {} \textit{robiłë} \\
\label{pitsch:ex:Kashubian_Perfect}
\z

\noindent Summarizing so far, East Slavic languages and Kashubian-B share the AU-less type of \textsc{l-}preterit due to the loss or drop, respectively, of the \textsc{be}-auxiliary. Their current \textsc{l-}preterit consists exclusively of V\textsubscript{L} and lacks overt person agreement.

The next section shows that the diachronic reshaping of the present-tense \textsc{be}-paradigm can give rise to yet another, rather peculiar, situation.

%%% %%% %%% %%% 

\subsubsection{Auxiliary reduction in Polish}\label{pitsch:sec:Polish}

Polish reshaped the present-tense forms of its \textsc{be}-verb far more profoundly than the remaining Slavic languages. To put it informally, Polish reduced the inherited present-tense forms of \textit{być} `be' to such an extent that their modern reflexes function as mere agreement markers. While this state of affairs is well-investigated (see, a.o., \cites{Decaux1955}{Rittel1970}{Andersen1987}{PiskorzAbrahamLeiss2013}), the actual nature of the ``new'' agreement markers is still a subject of controversy (see, a.o., \cites{BooijRubach1987}{BorsleyRivero1994}{Embick1995}{FranksBanski1999}; an overview and discussion is provided in \cite[5--9]{Abramowicz2008}).

The relevant changes seem to have started in the 14th century (\cites[103]{Rittel1970}[41]{Migdalski2006}). While most Slavic languages reduced the present-tense forms of their \textsc{be}-verb -- especially when used as an auxiliary -- to clitics, their reduction went even further in Polish. This process gave rise to two coexisting sets of present-tense \textsc{be}-forms in Old Polish dubbed \textsc{orthotonic} and \textsc{atonic}, respectively, by \citet{Andersen1987}. Modern Polish retains only the latter. \tabref{pitsch:tab:OldPolish} (from \citealt[41]{Migdalski2006}; see also \cites[99--103]{Rittel1970}[24]{Andersen1987}[3]{Embick1995}) summarizes the diachronic development.

 % Tabelle 5 (Altpolnisch)

\begin{table}
\begin{tabularx}{7.7cm}{XXlll}
\lsptoprule
& & \multicolumn{2}{c}{16th century} & today \\
& & orthotonic & \multicolumn{2}{c}{atonic} \\ 
\midrule
\multirow{3}{*}{\textsc{sg}} & \textsc{1} & \textit{jeśm} & \textit{-(e)śm}/\textit{-(e)m} & \textit{-(e)m} \\ 
& \textsc{2} & \textit{jeś} & \textit{-(e)ś} & \textit{-(e)ś} \\ 
& \textsc{3} & \textit{je/jest/jeść} & --- & --- \\\addlinespace
\multirow{3}{*}{\textsc{pl}} & \textsc{1} & \textit{jesm(y)} & \textit{-(e)smy} & \textit{-(e)śmy} \\ 
& \textsc{2} & \textit{jeść} & \textit{-(e)śće} & \textit{-(e)ście} \\ 
& \textsc{3} & \textit{są} & --- & ---\\ 
\lspbottomrule
\end{tabularx}
\caption{Diachronic development of Polish present-tense \textit{być}-forms}
\label{pitsch:tab:OldPolish}
\end{table}

\citet{Embick1995} emphasizes that the modern atonic forms are restricted to the \textsc{l-}preterit (and \textsc{l-}conditional; see \sectref{pitsch:sec:conditional}), which is illustrated in \tabref{pitsch:tab:PolishPast}.

The lost orthotonic forms were compensated for by a completely new present-tense paradigm for \textit{być} based on the former third-singular form \textit{jest} suffixed with the ``new'' atonic agreement markers from \tabref{pitsch:tab:OldPolish}; see \tabref{pitsch:tab:PolishFullVerb}.\footnote{As shown in the rightmost column, some Polish dialects employ the original third-plural form \textit{są-} as plural stem \citep[42]{Migdalski2006}.}\largerpage[-1]


\begin{table}
\begin{floatrow}
\captionsetup{margin=.005\linewidth}
\ttabbox{\begin{tabular}{lll}
\lsptoprule
& \textsc{sg} & \textsc{pl} \\ 
\midrule
1 & \textit{prosiła-m} & \textit{prosiły-śmy} \\ 
2 & \textit{prosiła-ś} & \textit{prosiły-ście} \\ 
3 & \textit{prosiła} & \textit{prosiły} \\ 
\lspbottomrule
\end{tabular}}
{\caption{The Polish \textsc{l-}preterit}\label{pitsch:tab:PolishPast}}
\ttabbox{\begin{tabular}{llll}
\lsptoprule
& \textsc{sg} & \textsc{pl} & (dialectal)\\ 
\midrule
1 & \textit{jest-em} & \textit{jest-eśmy} & (\textit{są-śmy}) \\ 
2 & \textit{jest-eś} & \textit{jest-eście} & (\textit{są-śće}) \\ 
3 & \textit{jest} & \multicolumn{2}{c}{\textit{są}} \\ 
\lspbottomrule
\end{tabular}}
{\caption{Modern Polish present-tense \textit{być}-forms (full verb)}
\label{pitsch:tab:PolishFullVerb}}
\end{floatrow}
\end{table}

Moreover, the atonic agreement markers occur on \textit{powinien} `should, ought', a former predicative adjective that developed into a modal quasi-verb; see \REF{pitsch:ex:Polish_powinien}. Rarely, they can also fulfil the function of the copula as in \REF{pitsch:ex:Polish_ShortCopula}.

\ea
\ea\gll Nie powinn-a-m \minsp{(} by-ł-a) jechać.\\
\textsc{neg} obliged\textsc{-sg.f-1sg} {} be\textsc{-l-sg.f} go.\textsc{inf}\\ \hfill (Polish)
\glt `I should not (have) go(ne).' \label{pitsch:ex:Polish_powinien}
\ex\gll Zmęczony-m i głodny(-m).\\
tired-\textsc{1sg} and hungry.\textsc{1sg}\\ 
\glt `I am tired and hungry.' \hfill \citep[234]{Migdalski2006} \label{pitsch:ex:Polish_ShortCopula}
\z\z

\noindent \citet[275--276]{Migdalski2006} claims that the third-person forms of the full verb \textit{być} -- i.e. \textit{jest} and \textit{są} -- do not specify any person feature. Moreover, judging from their combinatorial potential, \textit{jest} is completely underspecified ([\textalpha\! \textsc{number}]), whereas \textit{są} is marked as plural ([\textsc{pl}]).\footnote{This becomes apparent by the fact that \textit{jest} can combine with any person and number marker, whereas \textit{są} is restricted to the plural as shown in \tabref{pitsch:tab:PolishFullVerb}.} From these facts, \citet[275]{Migdalski2006} concludes that \textit{jest} and \textit{są} are in a lower syntactic position as compared to the other forms of the paradigm, and that they have to raise in the structure to adjoin to the relevant person/number marker (\textit{-m}, \textit{-ś}, etc.). Only in the 3rd person do they always remain in situ, as there is no (overt) agreement marker to adjoin to. Finally, considering data like \REF{pitsch:ex:Polish_ShortCopula}, it seems fair to assume that \textit{jest}/\textit{są} may also be absent or left unpronounced under specific circumstances.

In \sectref{pitsch:sec:EastSlavic}, I referred to \citet[192]{Issatchenko1940} who claims that Russian \textit{estʹ} ``has become an impersonal particle.'' I suggest that the facts about Polish \textit{jest} and \textit{są} just mentioned point in the same direction, though Polish seems to be in an intermediate stage: While in isolation, \textit{jest} seems to have lost its verbal character (agreement) just like Russian \textit{estʹ}, it can still be ``upgraded'' into a (composite) verb by merging it with an agreement marker. On the other hand, \textit{są} retains number agreement anyway. The parallels and differences allow the determination of the syntactic positions of the elements in question in \sectref{pitsch:sec:syntax}.

To sum up thus far, Polish reduced its original present-tense \textsc{be}-forms to atonic agreement markers which occur in the \textsc{l-}preterit but also in the ``new'' present-tense paradigm of \textit{być}, and which are likely to be located in a relatively high (functional) syntactic position.

%%% %%% %%% %%%  %%% %%% %%% %%% 

\subsection{\textsc{l-}conditional}\label{pitsch:sec:conditional}

\subsubsection{The general picture}\largerpage[-1]

Unlike the \textsc{l-}preterit, the \textsc{l-}conditional has an AU in all Slavic languages. However, variation obtains in the presence or absence, respectively, of person/number agreement on it. Moreover, if there is agreement, there is variation as to its shape.

Languages with inflected conditional AUs are illustrated in Tables \ref{pitsch:tab:conditional_synthetic} and \ref{pitsch:tab:conditional_analytic}: The AUs in \tabref{pitsch:tab:conditional_synthetic} are clearly synthetic. Most of these AUs are inherited from Late Proto-Slavic, which employed aorist \textsc{be}-forms as auxiliaries in the periphrastic conditional.\footnote{``BCMS-A'' and ``Czech-A'' stand for the standard (written) varieties of these languages. As to Čakavian, see \citet[35]{Panzer1967}, \citet[248--249]{Nehring2002}, and \citet[17--27]{Lisac2009}. Note that the Čakavian forms display analogy-based present-tense endings (\textsc{1sg} \textit{-n} is the regular reflex of \textit{-m}; some dialects feature a \textsc{3pl} \textit{biju}).} On the other hand, the AUs in \tabref{pitsch:tab:conditional_analytic} are apparently analytic, as they seem to contain a noninflected particle \textit{bi/by} accompanied by one of the clitic \textsc{be}-auxiliaries familiar from the \textsc{l-}preterit (see \tabref{pitsch:tab:perfect_3rd}).\footnote{As to colloquial/dialectal Czech (``Czech-B''), see \citet[310]{Toman1980} and \citet[92]{FranksKing2000}. The writing of the Czech-B AUs in one word is \citeauthor{Toman1980}'s.}\textsuperscript{,}\footnote{\label{fn:Macedonian}``Macedonian\textsuperscript{+}'' marks the special case when speakers use \textit{bi} plus a present-tense form of \textit{sum} `be' to disambiguate or emphasize the grammatical person \citep[see][110--111]{Kramer1986}. Elsewhere, \textit{bi} alone is used (see \tabref{pitsch:tab:conditional_particle}).}

Polish occurs in \tabref{pitsch:tab:conditional_analytic} since its characteristic agreement markers are, at least diachronically, reduced \textsc{be}-auxiliaries (see \sectref{pitsch:sec:Polish}). The same applies to the variants of Kashubian (see \sectref{pitsch:sec:Kashubian}).\footnote{\citet[778]{Stone1993} refers to \citet[134]{BrezaTreder1981} when stating that \textit{bë} ``may or may not acquire a personal ending''. \citet[392--393]{Dulicenko2005} adds that the ``inflected'' variants of the AU (\textit{bë-m}, \textit{bë-s}, etc.), which I dub ``Kashubian-A1'', are influenced by Polish, and that the ``Kashubian-A2'' AU-type is an archaism. Given this, the ``Kashubian-B'' variants in \tabref{pitsch:tab:conditional_particle} are the modern standard.}

\tabref{pitsch:tab:conditional_particle} shows those languages or varieties that have a noninflected AU.\footnote{``BCMS-B'' stands for colloquial/dialectal varieties (see \cites[39]{Panzer1967}[105]{Kramer1986}[253]{Browne2004}[276]{Xrakovskij2009}). As to Burgenland Croatian, see \citet[240]{Tornow2002}.}

% Tabelle 8

\begin{table}[p]
\begin{tabular}{llllll}
\lsptoprule
& & \multicolumn{2}{c}{\textsc{sg}} & \multicolumn{2}{c}{\textsc{pl}} \\ \cmidrule(lr){3-4}\cmidrule(lr){5-6}
& & AU & V\textsubscript{L} & AU & V\textsubscript{L} \\
\midrule
BCMS-A & 1 & \textit{bih} & \textit{pisala} & \textit{bismo} & \textit{pisale} \\ 
`write'    & 2 & \textit{bi} & \textit{pisala} & \textit{biste} & \textit{pisale} \\ 
& 3 & \textit{bi} & \textit{pisala} & \textit{bi} & \textit{pisale} \\\addlinespace
Čakavian & 1 & \textit{bin} & \textit{bila} & \textit{bimo} & \textit{bili} \\ 
`be' & 2 & \textit{biš} & \textit{bila} & \textit{bite} & \textit{bili} \\
 & 3 & \textit{bi} & \textit{bila} & \textit{bi} & \textit{bili} \\\addlinespace
Bulgarian & 1 & \textit{bix} & \textit{\v{c}ela} & \textit{bixme} & \textit{\v{c}eli} \\ 
`read'    & 2 & \textit{bi} & \textit{\v{c}ela} & \textit{bixte} & \textit{\v{c}eli} \\ 
            & 3 & \textit{bi} & \textit{\v{c}ela} & \textit{bixa} & \textit{\v{c}eli} \\\addlinespace
Czech-A & 1 & \textit{bych} & \textit{udělala} & \textit{bychom} & \textit{udělaly} \\ 
`make' & 2 & \textit{bys} & \textit{udělala} & \textit{byste} & \textit{udělaly} \\
 & 3 & \textit{by} & \textit{udělala} & \textit{by} & \textit{udělaly} \\\addlinespace
Upper Sorbian & 1 & \textit{bych} & \textit{dźěłała} & \textit{bychmy} & \textit{dźěłali} \\ 
`work'   & 2 & \textit{by} & \textit{dźěłała} & \textit{by\v{s}će} & \textit{dźěłali} \\ 
         & 3 & \textit{by} & \textit{dźěłała} & \textit{bychu} & \textit{dźěłali} \\
\lspbottomrule
\end{tabular}
    \caption{\textsc{l-}conditional with inflected synthetic AU}
    \label{pitsch:tab:conditional_synthetic}
\end{table}

% Tabelle 9

\begin{table}[p]
\begin{tabular}{llllll}
\lsptoprule
& & \multicolumn{2}{c}{\textsc{sg}} & \multicolumn{2}{c}{\textsc{pl}} \\\cmidrule(lr){3-4}\cmidrule(lr){5-6}
& & AU & V\textsubscript{L} & AU & V\textsubscript{L} \\
\midrule
Czech-B & 1 & \textit{bysem} & \textit{udělala} & \textit{bysme} & \textit{udělaly} \\ 
`make' & 2 & \textit{bysi} & \textit{udělala} & \textit{byste} & \textit{udělaly} \\ 
 & 3 & \textit{by} & \textit{udělala} & \textit{by} & \textit{udělaly} \\\addlinespace
Slovak & 1 & \textit{by som} & \textit{volala} & \textit{by sme} & \textit{volali} \\ 
`call'        & 2 & \textit{by si} & \textit{volala} & \textit{by ste} & \textit{volali} \\ 
            & 3 & \textit{by} & \textit{volala} & \textit{by} & \textit{volali} \\\addlinespace
Polish & 1 & \textit{by-m} & \textit{prosiła} & \textit{by-śmy} & \textit{prosiły} \\ 
`ask'        & 2 & \textit{by-ś} & \textit{prosiła} & \textit{by-ście} & \textit{prosiły} \\ 
            & 3 & \textit{by} & \textit{prosiła} & \textit{by} & \textit{prosiły} \\\addlinespace
Kashubian-A1 & 1 & \textit{bë-m} & \textit{miała} & \textit{bë-smë} & \textit{miałë} \\ 
`have'        & 2 & \textit{bë-s} & \textit{miała} & \textit{bë-sta} & \textit{miałë} \\ 
            & 3 & \textit{bë} & \textit{miała} & \textit{bë} & \textit{miałë} \\\addlinespace
Kashubian-A2 & 1 & \textit{bë jem} & \textit{miała} & \textit{bë jesmë} & \textit{miałë} \\ 
`have'        & 2 & \textit{bë jes} & \textit{miała} & \textit{bë jesta} & \textit{miałë} \\ 
            & 3 & \textit{bë je} & \textit{miała} & \textit{bë są} & \textit{miałë} \\\addlinespace
Macedonian\textsuperscript{+} & 1 & \textit{bi sum} & \textit{molela} & \textit{bi sme} & \textit{molele} \\ 
`ask' & 2 & \textit{bi si} & \textit{molela} & \textit{bi ste} & \textit{molele} \\ 
 & 3 & \textit{bi} & \textit{molela} & \textit{bi} & \textit{molele} \\
\lspbottomrule
\end{tabular}
    \caption{\textsc{l-}conditional with inflected analytic AU}
    \label{pitsch:tab:conditional_analytic}
\end{table}
\clearpage
% Tabelle 10

\begin{table}[t]
\begin{tabular}{llllll}
\lsptoprule
& & AU & \textsc{sg} & \textsc{pl} \\ 
\midrule 
BCMS-B & 1--3 & \textit{bi} & \textit{pisala} & \textit{pisale} \\
Burgenland Croatian & 1--3 & \textit{bi} & \textit{gledala} & \textit{gledale} \\
Slovene & 1--3 & \textit{bi} & \textit{pohvalila} & \textit{pohvalile} \\
Macedonian & 1--3 & \textit{bi} & \textit{molela} & \textit{molele} \\\addlinespace
Kashubian-B & 1--3 & \textit{b(ë)} & \textit{miała} & \textit{miałë} \\
Lower Sorbian & 1--3 & \textit{by} & \textit{słyšała} & \textit{słyšali} \\\addlinespace
Belarusian & 1--3 & \textit{b(y)} & \textit{čytala} & \textit{čytali} \\
Russian & 1--3 & \textit{b(y)}  & \textit{skazala} & \textit{skazali} \\
Ukrainan & 1--3 & \textit{b(y)} & \textit{bula} & \textit{buly} \\ 
\lspbottomrule
\end{tabular}
    \caption{\textsc{l-}conditional with noninflected AU}
    \label{pitsch:tab:conditional_particle}
\end{table}

%%% %%% %%% %%% 

\subsection{\citeposst{Garde1964} observation}

In his paper on the Slavic conditional, \citet[88]{Garde1964} makes an interesting note: Only Polish and the East Slavic languages have a particle in the conditional, and it is only these languages that can use more than only V\textsubscript{L} in conditional clauses. While \citeauthor{Garde1964} does not provide any evidence supporting his former claim, the latter one is valid and needs to be extended to Kashubian. The examples in \REF{pitsch:ex:ImpersCond_Polish}--\REF{pitsch:ex:ImpersCond_Russian} illustrate some alternative verb forms in the conditional periphrasis of the relevant languages.

\ea
\ea\gll \ldots, \minsp{(} że-)by przeczyta-ć książkę. \\
{} {} that-\textsc{cond} read-\textsc{inf} book.\textsc{acc} \\ 
\glt `\ldots, (in order) to read the book.'
\ex\gll Włączo-no by radio. \\
turn.on-\textsc{imps} \textsc{cond} radio.\textsc{acc} \\
\glt `One would switch on the radio.' \hfill \hfill (Polish; \citealt[253]{Migdalski2006})
\z
\label{pitsch:ex:ImpersCond_Polish}
\z

\ea\gll \ldots, że-bë mie-c {jednã klasã} wëżi. \\
{} that-\textsc{cond} have-\textsc{inf} {one class}.\textsc{acc} more \\
\glt `\ldots, (in order) to have one more class.'\\ \hfill \hfill (Kashubian; \url{www.odroda.kaszubia.com/01-07/edukacja.htm})
\label{pitsch:ex:ImpersCond_Kashubian}
\z

\ea
\ea\gll Pospa-t' by! \\
sleep-\textsc{inf} \textsc{cond} \\ \hfill (Russian) 
\glt `If I could only sleep a little while!' \hfill \citep[346]{Isacenko1962}
\ex\gll \ldots, čto-by spa-t'. \\
{} that-\textsc{cond} sleep-\textsc{inf} \\  
\glt `\ldots, (in order) to sleep.' 
\ex\gll Ne skaž-i \minsp{(} by) on mne ėtogo vo-vremja, \ldots \\
\textsc{neg} say-\textsc{imp} {} \textsc{cond} he me.\textsc{dat} this.\textsc{acc} in-time \\
\glt `If he had not told me that in time, \ldots' \hfill \citep[22]{Panzer1967}
\z
\label{pitsch:ex:ImpersCond_Russian}
\z

\noindent In addition to infinitives and imperatives, Russian combines \textit{by} with the present tense, participles, adverbs, and even nominals (see \cites[195]{Issatchenko1940}[21--23]{Panzer1967}). 

As indicated by round brackets in \tabref{pitsch:tab:conditional_particle}, East Slavic and Kashubian \citep[see][26]{Panzer1967} exhibit a reduced particle variant \textit{b}. The same holds for Polish, albeit in colloquial (presumably dialectal) contexts; see \REF{pitsch:ex:reduction_Polish}.\footnote{Examples \REF{pitsch:ex:reduction_Polish_a} and \REF{pitsch:ex:reduction_Polish_b} are taken from the National Corpus of Polish (\url{http://nkjp.pl/}).}\textsuperscript{,}\footnote{The colloquial character of \REF{pitsch:ex:reduction_Polish_a} also manifests in the absence of the appropriate agreement marker \textit{-m}. Only thanks to its absence can the particle undergo phonological reduction.}

    \largerpage

\ea
\ea\gll Prosi mnie raz, że-b ja z nim nad rzekę poszed-ł.\\
ask.\textsc{3sg} me.\textsc{acc} once that-\textsc{cond} I with him.\textsc{ins} above river.\textsc{acc} go-\textsc{l.sg.m} \\ 
\glt `Once he asks me to go to the river with him.' \\ \hfill (H. Auderska: \textit{Babie lato}, 1974) \label{pitsch:ex:reduction_Polish_a}
\ex\gll {[}D]o końca walczyliśmy, że-b awansować do Ligi Mistrzów. \\
to end.\textsc{gen} fight.\textsc{l.1pl} that-\textsc{cond} ascend.\textsc{inf} to league.\textsc{gen} champion.\textsc{gen.pl} \\
\glt `We fought to the end to ascend to the Champions League.' \\ \hfill (W. Batko: \textit{Dramat pod Akropolem}, 2005) \label{pitsch:ex:reduction_Polish_b}
\ex \gll Wróciwszy wczoraj z zakupów usiadła ja na kanapie z kubkiem melisy w ręku, że-b się uspokoić. \\
having.returned yesterday from shopping.\textsc{gen.pl} sit.\textsc{l.sg.f} I on sofa.\textsc{loc} with cup.\textsc{ins} melissa.\textsc{gen} in hand.\textsc{loc} that-\textsc{cond} \textsc{refl} calm.down.\textsc{inf} \\
\glt `When I returned from shopping yesterday, I sat down on the sofa with a cup of melissa tea in my hands to calm down.' \\ \hfill (Polish; \href{https://gazetaolsztynska.pl/mragowo/527914,Panorama-Helutki-Grzecznosci-za-wiele-FELIETON.html}{gazetaolsztynska.pl}, 2021 [accessed 4/2022])
\z
\label{pitsch:ex:reduction_Polish}
\z

\noindent Crucially, it is precisely the languages (plus Kashubian) that \citet{Garde1964} claims to possess a conditional (inflexible) particle which allow the phonological reduction of that very particle. I wish to propose that \citeposst{Garde1964} intuition is perfectly right, and that there is a fundamental difference between \textsc{particle languages} -- Kashubian, Polish, Belarusian, Russian, and Ukrainian, all of which have inflexible (or even absent) mood/tense markers and allow infinitives in the conditional periphrasis -- and \textsc{auxiliary languages}, which have auxiliary verbs specified for person and number. The latter holds true for all remaining languages, even if they display a particle from a descriptive point of view (BCMS-B, Burgenland Croatian, Slovene, Macedonian, and Lower Sorbian).

There is another phenomenon to be taken into consideration: AU-doubling in the \textsc{l-}conditional.

%%% %%% %%% %%%  %%% %%% %%% %%% 

\subsection{AU-doubling in the \textsc{l-}conditional}\label{pitsch:sec:doubling}

In a number of Slavic languages, the conditional AU can occur twice in the same clause. While this is well-documented for Russian and older stages of Polish and Polish in early acquisition, there is only little data available on the remaining languages. This is likely to be due to the fact that AU-doubling is a phenomenon characteristic of substandard speech and considered incorrect by most grammars. For Russian, \citet{Xrakovskij2009} mentions the examples in \REF{pitsch:ex:doubling_Russian}.\footnote{\citet[331]{Hansen2010} notes that a random sample taken from the National Corpus of Russian indicates that \textit{by}-doubling occurs ``quite frequently'' in Russian despite its being not accepted by the norms of the standard language.}

\ea
\ea\gll Ja by pogulja-l by segodnja večerom. \\
I \textsc{cond} take.a.walk-\textsc{l.sg.m} \textsc{cond} today evening.\textsc{ins} \\
\glt `I would like to take a walk tonight.'  \label{pitsch:ex:doubling_Russian_a}
\ex\gll Čto-by ja tebja by zdes' bol'še ne vide-l. \\
that.\textsc{cond} I you.\textsc{acc} \textsc{cond} here more \textsc{neg} see-\textsc{l.sg.m} \\
\glt `So that I would not see you here again.' \\ \hfill (Russian; \citealt[277]{Xrakovskij2009}) \label{pitsch:ex:doubling_Russian_b}
\z
\label{pitsch:ex:doubling_Russian}
\z

\noindent \citet{Rittel1973} gives \REF{pitsch:ex:doubling_Kashubian} and \REF{pitsch:ex:doubling_Masovian} from Kashubian and Masovian, respectively.

\ea\gll
jag by úna by odeš-ŭ-a \\
how \textsc{cond} she \textsc{cond} walk.away\textsc{-l-sg.f} \\ 
\glt `as though she should have gone away' \hfill (Kashubian)
\label{pitsch:ex:doubling_Kashubian}
\z

\ea\gll že-by učy-ł-by s'e xoźić \\
that-\textsc{cond} learn\textsc{-l.sg.m-cond} \textsc{refl} walk.\textsc{inf} \\ 
\glt `in order for him to learn to walk' \hfill (Masovian; \citealt[146]{Rittel1973})
\label{pitsch:ex:doubling_Masovian}
\z

\noindent \textit{By}-doubling is also found in colloquial Polish as shown in \REF{pitsch:ex:doubling_Polish_coll}.

\ea
\ea\gll {\ldots} to dziś by-m by-ł-by szejkiem! \\
{} then today \textsc{cond-1sg} be\textsc{-l.sg.m-cond} sheikh.\textsc{ins} \\
\glt `{\ldots} then today I would be a sheikh' \hfill (\href{https://www.wykop.pl/link/2244300/comment/24557794}{wykop.pl}, accessed 4/2022)
\label{pitsch:ex:doubling_Polish_coll_a}
\ex\gll nie sądzę, by-śmy by-l-i-by tak blisko siebie i tak związani jak my, gdyby nie ten czas \\
\textsc{neg} think.\textsc{1sg} \textsc{cond-1pl} be\textsc{-l-pl.m-cond} so close \textsc{refl.acc} and so connected.\textsc{pl.m} as we if \textsc{neg} this time \\
\glt `I don't think we would be as close to each other and as connected as we are if it were not for this time' \\ \hfill (coll. Polish; \href{https://pl.spiceend.com/little-couplesneak-peek}{pl.spiceend.com}, accessed 4/2022)
\label{pitsch:ex:doubling_Polish_coll_b}
\z
\label{pitsch:ex:doubling_Polish_coll}
\z

\noindent \citet[624]{Smoczynska1985} notes that children acquiring Polish as their first language quite regularly double the conditional AU; see \REF{pitsch:ex:doubling_Polish_acq}.

\ea\gll A moja mamusia też by mia-ł-a-by {ładne włoski}. \\ 
and my mum also \textsc{cond.3sg} have\textsc{-l-sg.f-cond.3sg} {pretty hair}.\textsc{acc.pl} \\ 
\glt `My mum would also like to have pretty hair.' \\ \hfill (Polish; \citealt[from][119]{Blaszczyk2018})
\label{pitsch:ex:doubling_Polish_acq}
\z

\noindent Especially in subjunctive clauses, the phenomenon has also been observed in Lithuanian-Polish bilinguals; see \REF{pitsch:ex:doubling_Polish_biling}.\largerpage[-1]

\ea 
\ea\gll Teraz to-by na pewno zaintersowani by by-l-i. \\
now \textsc{part-cond} for sure interested.\textsc{pl.m} \textsc{cond} be\textsc{-l-pl.m} \\  
\glt `Now they would certainly be interested.'  
\ex\gll \ldots, że-by my nie widzie-l-i-b co oni gadają. \\
{} that-\textsc{cond} we \textsc{neg} see\textsc{-l-pl.m-cond} what.\textsc{acc} they chatter.\textsc{3pl} \\   
\glt `\ldots, so that we do not know what they are chattering.' \\ \hfill (Polish; \citealt[58]{Smulkowa1999}, from \citealt[132]{Blaszczyk2018})
\z
\label{pitsch:ex:doubling_Polish_biling}
\z 

\noindent According to \citet[132]{Blaszczyk2018}, many similar examples can be found in \citet{GrekPabisowaMaryniakowa1999}, who describe the linguistic peculiarities of the dialects of the former Polish Eastern Borderlands. \citet{Zielinska2002} does not regard such examples as the result of interference/contact but as local variants. She adds that they might well be considered archaisms, as doubling already occurs in Old Polish as documented in \REF{pitsch:ex:doubling_OldPolish}.

\ea\gll iże-by by by-ł-y wysłuchany {twoje prośby} \\
that-\textsc{cond} \textsc{cond} be\textsc{-l-pl.f} heard.\textsc{pl.f} {your pleas}.\textsc{acc.pl} \\
\glt `so that your pleas might be heard' \hfill (Old Polish; \textit{Historia  Aleksandra}, 1510) \\ \hfill (\citealt[113]{Rittel1975}, from \citealt[133]{Blaszczyk2018})
\label{pitsch:ex:doubling_OldPolish}
\z

\noindent There must clearly be more research as to the extent of AU-doubling in Slavic but the data allow for the following generalizations: First, AU-doubling is not a recent phenomenon. Second, it seems to be restricted to colloquial and dialectal varieties as well as speech produced in the course of early language acquisition. Third, it seems to prevail in East Slavic (first of all Russian), Polish and Kashubian.\footnote{Luka Szucsich (p.c.) reports that \textit{bi}-doubling seems to be possible in Burgenland Croatian. The question calls for further (corpus-based) research.} Potentially, AU-doubling might turn out to be another piece of evidence for the special status of the languages and varieties of the ``North-Eastern group'' as regards their conditional AU. In \sectref{pitsch:sec:doubling_analysis}, I sketch a syntactic analysis to account for the phenomenon.

    \largerpage[-1]

%%% %%% %%% %%%  

\subsection{Putting the pieces together}

Bringing together the pieces of information provided thus far -- (i) the absence or presence of person/number agreement in the \textsc{l-}preterit and the \textsc{l-}conditional, (ii) the non-/availability of other forms than V\textsubscript{L} in the conditional, and (iii) the possibility of particle reduction in those languages that (seem to) have one -- gives us the overall picture in \tabref{pitsch:tab:overview}.\footnote{I omit AU-doubling. According to the data, it is possible in the same languages that allow for particle reduction. From the varieties of Kashubian, I list only Kashubian-B, as it seems to represent present-day Kashubian (regarding the conditional). ``$\circ$'' signifies the possible lack of agreement in the Macedonian conditional in unmarked contexts (see footnote \ref{fn:Macedonian}).}

\begin{table}
\begin{tabularx}{12cm}{Xcccc}
\lsptoprule
& \multicolumn{2}{c}{agreement on AU in the \dots} & more & particle \\
& \textsc{l-}preterit & \textsc{l-}conditional & than V\textsubscript{L} & reduction \\
\midrule
BCMS-A & $\bullet$ & $\bullet$ & & \\ 
BCMS-B & $\bullet$ & & & \\ 
Burgenland Croatian & $\bullet$ & & & \\ 
Čakavian & $\bullet$ & $\bullet$ & & \\ 
Slovene & $\bullet$ & & & \\ 
Bulgarian & $\bullet$ & $\bullet$ & & \\ 
Macedonian & $\bullet$ & $\circ$ & & \\\addlinespace
Polish & $\bullet$ & $\bullet$ & $\bullet$ & $\bullet$ \\ 
Kashubian(-B) & & & $\bullet$ & $\bullet$ \\ 
Czech & $\bullet$ & $\bullet$ & & \\ 
Slovak & $\bullet$ & $\bullet$ & & \\ 
Lower Sorbian & $\bullet$ & & & \\ 
Upper Sorbian & $\bullet$ & $\bullet$ & & \\\addlinespace
Belarusian & & & $\bullet$ & $\bullet$\\ 
Russian & & & $\bullet$ & $\bullet$\\ 
Ukrainian & & & $\bullet$ & $\bullet$\\ 
\lspbottomrule
\end{tabularx}
    \caption{\textsc{l-}periphrases in comparison}
    \label{pitsch:tab:overview}
\end{table}

The overall picture reveals a number of facts:

First, the variation in agreement in the \textsc{l-}preterit is not coextensive with the one in the \textsc{l-}conditional: Whilst in the preterit, only Kashubian and East Slavic do not express person/\hspace{0mm}number agreement, this holds for far more languages in the conditional. Thus, it seems that the conditional AU is more prone to linguistic change than the AU in the \textsc{l-}preterit.\footnote{Tentatively, this is due to the more ``regular'' shape of the conditional AU with the stem \textit{bë/bi/by-} throughout its whole paradigm. By contrast, the preterit AU lacks a similarly consistent base. Reducing the conditional AU to its stem by dropping the agreement ending (and thus boiling it down to its essential grammatical meaning) seems thus more natural than in the case of the preterit AU (which can at best be dropped altogether).}

Second, there is no obvious correlation between the absence of person/number agreement in the \textsc{l-}conditional and the availability of verb forms other than V\textsubscript{L}.

Third, there is a robust correlation between the availability of verb forms other than V\textsubscript{L} in the conditional and the possibility of phonologically reducing the AU.

The availability of verb forms other than V\textsubscript{L} as well as of particle reduction clearly distinguish Kashubian, Polish, and the East Slavic languages. Crucially, in all of them diachronic change lead to the loss or reshaping of the present-tense paradigm of the (former) \textsc{be}-auxiliary (see §§\ref{pitsch:sec:EastSlavic}--\ref{pitsch:sec:Polish}). 

In what follows, I will sketch a number of scenarios of language change to explain the present-day situation in the Slavic languages. In doing so, I will identify four groups of languages with a distinct development each.

%%% %%% %%% %%%  %%% %%% %%% %%% 
%%% %%% %%% %%%  %%% %%% %%% %%% 

\section{Linguistic change}\label{pitsch:sec:change}

The modern shape of the \textsc{l-}preterit and the \textsc{l-}conditional in Slavic allows reflections about what happened to the relevant periphrases in preceding centuries and has thus given rise to the current state of affairs. Four distinct diachronic scenarios emerge. 

%%% %%% %%% %%%  

\subsection{``Old symmetry''}\label{pitsch:sec:old_symmetry}

The first scenario concerns the following South and West Slavic languages:

\begin{multicols}{3}
\begin{itemize}
    \item BCMS-A
    \item Bulgarian
    \item Čakavian
    \item Macedonian\textsuperscript{+}
    \item Czech
    \item Slovak
    \item Upper Sorbian
\end{itemize}
\end{multicols}
    
\noindent All of them retain the Late Proto-Slavic shape of the \textsc{l-}preterit and \textsc{l-}conditional, especially of the relevant AUs, i.e., they use inflected auxiliary verbs expressing person/number agreement. What is more, they do not allow verb forms other than V\textsubscript{L} in the conditional, and they do not reduce their conditional AU.

No particular language change took place in these languages apart from occasional replacement of the old aorist inflections on the conditional AU with present-tense markers. In this respect, it is possible to discern two subgroups: 

\begin{enumerate}
\item Čakavian has replaced the aorist inflections with present-tense suffixes and thus retains synthetic auxiliary verbs (e.g., \textsc{1sg} \textit{bi-n} [$<$ \textit{bi-m}], \textsc{2sg} \textit{bi-š}, etc.). 
\item Czech-B, Macedonian\textsuperscript{+}, and Slovak have substituted the old aorist markers with the present-tense forms of their respective \textsc{be}-auxiliary (e.g., Mace\-donian\textsuperscript{+} \textit{bi sum}, \textit{bi si}, etc.; Slovak \textit{by som}, \textit{by si}, etc.). Minor analogies of the same type took place in BCMS-A (\textsc{1pl} \textit{bi-smo}) and Czech (\textsc{2sg} \textit{by-s}). I claim that these new ``analytic'' auxiliary forms are really (still) synthetic, i.e. that the \textsc{be}-forms substituting the old aorist inflections are suffixes, not clitics. They have been carried over from the \textsc{l-}preterit by analogy but changed their morphosyntactic status. Thus, for instance, BCMS-A \textsc{1pl} \textit{bi-smo} and Slovak \textit{by sme} are clearly parallel formations -- irrespective of orthographic conventions.
\end{enumerate}

%%% %%% %%% %%%  

\subsection{``Asymmetry''}\label{pitsch:sec:asymmetry}

The second scenario concerns the following languages:

\begin{itemize}
    \item BCMS-B
    \item Burgendland Croatian
    \item Macedonian
    \item Slovene
    \item Lower Sorbian
\end{itemize}

\noindent There is an asymmetry in that these languages feature an inflected auxiliary verb in the \textsc{l-}preterit but a noninflected AU (\textit{bi} or \textit{by}) in the \textsc{l-}conditional. But like the varieties described in \sectref{pitsch:sec:old_symmetry}, they exclude any verb forms other than V\textsubscript{L} from the conditional and lack reduced variants of their conditional AU.

A straightforward way to explain these facts goes as follows: The conditional AU is merely a ``pseudo-particle'', i.e. we are actually (still) dealing with an inflected auxiliary verb. This verb, however, has dropped its agreement marking at the surface, which means that it is underspecified for person and number. In other words, /bi/ should be analyzed as being associacted with person/number agreement features as sketched in \REF{pitsch:ex:underspecified}.\footnote{The agreement features might also be located in a silent agreement suffix attached to the stem.}

\ea /bi/\textsubscript{[\textalpha\! \textsc{person},\textbeta\! \textsc{number}]} 
\label{pitsch:ex:underspecified}
\z

\noindent If this is on the right track, the languages and varieties in question form a larger class with the ones addressed in \sectref{pitsch:sec:old_symmetry}, the reason being that both groups retain -- even if covertly -- synthetic auxiliary verbs that encode person/number agreement.

Possible causes for the loss of overt agreement are phonological reduction (drop) of inflectional endings or/and paradigm leveling (intraparadigmatic analogy). Both mechanisms seem to have been involved, for instance, in the development from BCMS-A to BCMS-B; see \tabref{pitsch:tab:drop}.

\begin{table}
\begin{tabularx}{7cm}{llXX}
\lsptoprule
& & \textsc{sg} & \textsc{pl} \\ 
\midrule
BCMS-A & 1 & \textit{bih} & \textit{bismo} \\ 
& 2 & \textit{bi} & \textit{biste} \\ 
    & 3 & \textit{bi} & \textit{bi} \\\addlinespace
\multicolumn{4}{c}{$\downarrow$ phonological drop (\textsc{1sg}) $\downarrow$} \\\addlinespace
 & 1 & \textit{bi\sout{h}} & \textit{bismo} \\ 
  & 2 & \textit{bi} & \textit{biste} \\ 
    & 3 & \textit{bi} & \textit{bi} \\\addlinespace
\multicolumn{4}{c}{$\downarrow$ paradigm leveling (\textsc{1/2pl}) $\downarrow$} \\\addlinespace
BCMS-B & 1 & \textit{bi} & \textit{bi\sout{smo}} \\ 
 & 2 & \textit{bi} & \textit{bi\sout{ste}} \\ 
    & 3 & \textit{bi} & \textit{bi} \\ 
\lspbottomrule
\end{tabularx}
    \caption{Loss of overt agreement encoding in BCMS}
    \label{pitsch:tab:drop}
\end{table}

Occasionally, language contact is identified as another possible source of overt agreement loss. Thus, for instance, \citet[24]{Panzer1967} suggests that Lower Sorbian dropped the person/number suffixes on its conditional AU due to the increased use of personal subject pronouns (induced by language contact with German). A similar explanation is put forward by \citet[100]{Rittel1970} to derive the present-day state of the \textsc{l-}periphrases in Kashubian-B.

Evidence in favor of analyzing the conditional AUs in question as underspecified auxiliary verbs comes from Macedonian: In cases where speakers need to disambiguate the person feature (``Macedonian\textsuperscript{+}''; see footnote \ref{fn:Macedonian}), \textit{bi} co-occurs with what looks like clitic \textsc{be}-forms as used in the \textsc{l-}preterit, hence \textsc{1sg} \textit{bi sum}, \textsc{2sg} \textit{bi si}, etc. in place of solitary \textit{bi}. I wish to claim that these elements do not differ from, e.g., Slovak \textit{by som}, \textit{by si}, etc. (see \sectref{pitsch:sec:old_symmetry}) -- i.e. they are suffixes. However, different from Slovak, the Macedonian suffixes can be left unpronounced when there is no need to express the person feature on the AU. Thus, when Macedonian \textit{bi} occurs without person/number agreement, it resembles \REF{pitsch:ex:underspecified}. Incidentally, it does not seem too bold a claim that the step from Macedonian\textsuperscript{+} to Macedonian represents phonological drop (\tabref{pitsch:tab:drop}) and, thus, linguistic change in progress.

\largerpage[-1]

%%% %%% %%% %%%  

\subsection{``New symmetry''}\label{pitsch:sec:new_symmetry}

The third scenario concerns the following languages:

\begin{itemize}
    \item Belarusian
    \item Russian
    \item Ukrainian
    \item Kashubian-B
\end{itemize}
    
\noindent As said in \sectref{pitsch:sec:EastSlavic} and \sectref{pitsch:sec:Kashubian}, respectively, present-day East Slavic and Kashubian-B lack AUs in the \textsc{l-}preterit and the \textsc{l-}conditional for individual diachronic reasons. Apart from that, they employ verb forms other than only V\textsubscript{L} in the conditional, and they also allow the reduction of their conditional AU (\textit{by/bë} $\rightarrow$ \textit{b}).\footnote{Whereas the two variants are in complementary phonological distribution in Belarusian and Ukrainian, their choice depends primarily on stylistic factors in Kashubian-B and Russian.}

For Old East Slavic, historical grammars commonly note the significant effect the changes sketched in \sectref{pitsch:sec:EastSlavic} had on the East Slavic verbal system. Thus, for instance, \citet[193]{Issatchenko1940} writes that ``[t]his change, which at first affected only the verb \textit{byti}, shook the whole verbal system.'' In the same vein, \citet[395]{Ivanov1964} states that the essence of the relevant changes consisted in the loss of (agreement on) the former AU, which in turn caused a shift of the ``center of the tense/mood form'' to V\textsubscript{L}. 

What the authors refer to is a shift in agreement marking and finiteness: While before the changes, Old East Slavic \textsc{l-}periphrases uniformly contained a finite auxiliary verb and a nonfinite \textsc{l-}participle, the changes turned the former into a particle encoding tense/mood (but not agreement), and the latter into a form associated with a complete set of agreement features. Initially, the change affected only the \textsc{l-}preterit, effectively deleting the auxiliary due to the loss of the present-tense paradigm of \textit{byti}. As a consequence, speakers now recognized V\textsubscript{L} as the only (finite) verb, associating it with a ``hidden'' (underspecified) person feature \citep[88]{Junghanns1995}; see \REF{pitsch:ex:Junghanns_b} in \sectref{pitsch:sec:EastSlavic}. 

Only after the \textsc{l-}preterit had thus turned into a synthetic form, the change spread to the \textsc{l-}conditional: By analogy, speakers now also perceived V\textsubscript{L} in the \textsc{l-}conditional as finite. As a clause can only contain one finite verb, a finite auxiliary became redundant. This paved the way for dropping person/number agreement on the conditional AU, which thus turned into a mere mood particle. This chain of events is schematized in \tabref{pitsch:tab:change_EastSlavic}, using the \textsc{1sg} of \textit{čitati} `read' as an illustration.

\begin{table}
\begin{tabular}{lllcll}
\lsptoprule
 & \multicolumn{2}{c}{I} & & \multicolumn{2}{c}{II} \\\addlinespace
\multirow{2}{*}{\textsc{l-}preterit:} & \textit{jesmĭ} & \textit{čitala} & $\rightarrow$ & $\varnothing$ & \textit{čitala} \\
 & {\small finite} & {\small nonfinite} & & {\small \textsc{part} } & {\small finite} \\\addlinespace
 & \multicolumn{5}{c}{$\downarrow$} \\\addlinespace
\multirow{2}{*}{\textsc{l-}conditional:} & \textit{bychŭ} & \textit{čitala} & $\rightarrow$ & \textit{by} & \textit{čitala} \\
 & {\small finite} & {\small nonfinite} & & {\small \textsc{part} } & {\small finite} \\
 \lspbottomrule
\end{tabular}
\caption{Diachronic change in East Slavic}
\label{pitsch:tab:change_EastSlavic}
\end{table}

\noindent I suggest that by and large the same took place in Kashubian -- though at a later time --, giving rise to the situation in present-day Kashubian-B. In \sectref{pitsch:sec:syntax}, I put forward a syntactic account to explain the availability of verb forms other than V\textsubscript{L} in the conditional. This account builds upon the presence of a particle in the relevant languages, i.e., of a tense/mood operator in the functional domain of the clause. With one important addition, the analysis also captures present-day Polish, which I turn to in the following section. 

%%% %%% %%% %%%  

\subsection{``Demolition and reconstruction''}\label{pitsch:sec:reconstruction}

As outlined in \sectref{pitsch:sec:Polish}, the present-tense forms of Polish \textit{być} `be', inherited from Late Proto-Slavic, completely vanished due to their reduction to atonic forms and concomitant repurposing as person/number markers, which compensated for the lost auxiliaries in the \textsc{l-}preterit. These very markers have subsequently also been used to form an utterly new present-tense paradigm for the copula \textit{być} (\textit{jest-em}, \textit{jest-eś}, etc.). Finally, they also occur on the Modern Polish conditional AU as shown in \REF{pitsch:ex:Polish_Cond}.

\ea
\ea\gll Ja by-m pisa-ł-a. \\
I \textsc{cond-1sg} write\textsc{-l-sg.f} \\ \hfill (Polish)
\glt `I would be writing.' 
\ex
\gll {\ldots} że=by-m ja pisa-ł-a. \\
{} that=\textsc{cond-1sg} I write\textsc{-l-sg.f} \\
\glt `{\ldots} that I would be writing.' 
\z
\label{pitsch:ex:Polish_Cond}
\z

\noindent However, different from the \textsc{l-}preterit, the atonic agreement markers are syntactically immobile once they show up on conditional \textit{by}. This raises the question if the members of the paradigm of the Polish conditional AU are not simply synthetic forms with agreement endings that merely ``imitate'' the atonic markers from the \textsc{l-}preterit by analogy. In \sectref{pitsch:sec:syntax}, I will argue against this view and claim that the monolithic nature of \textit{by-m}, \textit{by-ś}, etc. is due to the fact that the Polish atonic markers are generated in the specifier of the functional head I\textsuperscript{0} (occupied by \textit{by}) and subsequently ``m-merge'' \citep{Matushansky2006,Pietraszko2021} with it. It follows that, ultimately, both form a single and inseparable unit.

Additional evidence for treating Polish \textit{by} as a particle that is initially separate from the atonic agreement markers comes from language acquisition (a.o., \cites{Smoczynska1985}{Blaszczyk2018}{DogilAguado1989}); see \REF{pitsch:ex:LanguageAcquisition}.

\ea
\ea\gll pisał-em-by \\
write.\textsc{l.m.sg-1sg-cond} \\ \hfill (Polish)
\ex\gll Ja by pisał-em \\
I \textsc{cond} write.\textsc{l.m.sg-1sg} \\
\glt `I would be writing'  \hfill (\citealt[640]{Smoczynska1985}, from \citealt[118]{Blaszczyk2018}) 
\z
\label{pitsch:ex:LanguageAcquisition}
\z

\noindent The data show that children frequently ``mix up'' the canonical positions of \textit{by} and the agreement markers, respectively. Apparently, they do so by analogy with the \textsc{l-}preterit, where the latter mostly attach directly to V\textsubscript{L}. However, in \sectref{pitsch:sec:syntax}, I will try to show that \citet{Embick1995} is right in claiming that the direct attachment of the agreement markers to V\textsubscript{L} is an illusion. Underlyingly, the \textsc{l-}preterit involves a silent past-tense operator in I$^0$, and it is this operator which the agreement marker adjoins to.

My scenario for Polish is thus the following: The demolition of the inherited present-tense \textsc{be}-paradigm led to a situation where Old Polish was very close to East Slavic and Kashubian-B (\sectref{pitsch:sec:new_symmetry}): It had effectively lost the inflected auxiliary verb in the \textsc{l-}preterit and would at some later point in time face the same situation in the conditional. But unlike East Slavic and Kashubian-B, Polish did not entirely dispose of the old \textsc{be}-forms but re-utilized them as agreement markers. Combining insights of \citet{Embick1995}, \citet{Matushansky2006}, and \citet{Pietraszko2018,Pietraszko2021}, I argue that these markers are clitic heads generated in SpecTP from where they adjoin to I$^0$, which is silent in the \textsc{l-}preterit (realis mood, past tense) but overt (\textit{by}) in the \textsc{l-}conditional (irrealis mood).

%%% %%% %%% %%%  %%% %%% %%% %%% 
%%% %%% %%% %%%  %%% %%% %%% %%% 

\section{Towards a syntactic analysis}\label{pitsch:sec:syntax}

Based on the preceding observations, I wish to argue that there are two major classes of Slavic languages with regard to \textsc{l-}periphrases. The difference between them concerns the category of their AUs.

\subsection{The framework}\label{pitsch:sec:syntax_frame}

With modifications, I rely on the framework developed in \citet{Pietraszko2018,Pietraszko2021} who argues that, in periphrases, T$^0$ ($=$ I$^0$) has an uninterpretable (i.e. selectional) feature [uV] which cannot be checked against the interpretable (categorial) feature [iV] of V$^0$ due to an intervening functional projection, namely AspP.\footnote{According to \citeauthor{Pietraszko2018}, this constellation underlies, e.g., English progressive tenses.} As a consequence, an auxiliary verb (Aux) with its own [iV] is generated in the specifier of I$^0$ where it satisfies the selectional requirement; see \figref{Pitsch:fig:Pietraszko_initial}.\footnote{Circle-ended lines mark Agree relations, checked features are struck out. \citeauthor{Pietraszko2021} uses the framework of Bare Phrase Structure, so in her tree the auxiliary is generated next to I$^0$, which equals SpecIP under X-bar assumptions.}

\begin{figure}
    \centering
\begin{forest}
[IP
    [Aux\textsubscript{$\lbrack$iV$\rbrack$},name=auxhead]
        [I$'$
        [I\textsubscript{$\lbrack$\sout{uV}$\rbrack$},name=ihead]
            [AspP
                [Asp\textsubscript{$\lbrack$\sout{uV}$\rbrack$},name=asphead] 
                    [VP
                    [V\textsubscript{$\lbrack$iV$\rbrack$},name=vhead,roof] 
            ]
        ]
    ]
]
\draw[Circle-Circle] (auxhead) to[out=south,in=west] (ihead);
\draw[Circle-Circle] (asphead) to[out=south,in=west] (vhead);
\end{forest} 
    \caption{Configuration giving rise to periphrasis \citep[see][11]{Pietraszko2021}}
    \label{Pitsch:fig:Pietraszko_initial}
\end{figure}

Unlike \citeauthor{Pietraszko2018}, I claim that the crucial (type of) feature in Slavic \textsc{l-}peri\-phrases is not [V] but rather [\textphi], i.e. verb-subject agreement. This modification is motivated by the fact that, no matter whether or not AspP is assumed in the syntax of Slavic languages, a constellation like \figref{Pitsch:fig:Pietraszko_initial} is unlikely to arise: If AspP is projected, it is so in general, hence each and every Slavic clause should be periphrastic. On the other hand, if AspP is not assumed (because viewpoint aspect is taken to be a lexical rather than a grammatical category), it should again be absent in general, which eliminates \citeposst{Pietraszko2021} structural motivation for periphrasis.

Verb-subject agreement is a more plausible candidate: If the verb in V$^0$ comes from the lexicon equipped with a complete set of \textphi-features (\textphi$^+$), there is no need to project any auxiliary, which gives rise to a synthetic structure. On the other hand, if V$^0$ is occupied by a verb with an incomplete set of \textphi-features (\textphi$^-$), the missing features have to be supplied by an auxiliary. Crucially, for a \textphi-set to be incomplete, one of the following conditions has to be complied: Either the set lacks a person feature (participles) or it is completely empty (infinitives).\footnote{See \citet{Pitsch2015} for a formal account of the finite/nonfinite distinction in Slavic resting on a prominent role of grammatical person.}

Depending on the class a language belongs to, it either has or has not available ``true'' auxiliary verbs (in Aux$^0$ or/and I$^0$) that come with \textphi$^+$. If it has, I$^0$ owns or receives (via percolation; see \citealt{Pietraszko2018}) \textphi$^+$ and can thus enter into an Agree relation with the subject. If it has not, one of two scenarios are possible: In Polish and Kashubian-A1, \textphi$^+$ is generated in SpecIP in the form of an atonic agreement marker and subsequently fused (via m-merger; see \citealt{Matushansky2006}) with I$^0$. On the other hand, in East Slavic and Kashubian-B, V\textsubscript{L} comes from the lexicon with a complete set of \textphi-features (see \sectref{pitsch:sec:EastSlavic}), so I$^0$ can establish an Agree relation with the subject without the intervention of an auxiliary or agreement marker. 

What the ``North-Eastern group'' of Slavic languages have in common is that I$^0$ is a mere particle \citep{Garde1964}, which is due to the diachronic reduction or loss, respectively, of the present-tense paradigm of `be'. All remaining languages retain ``true'' auxiliary verbs. I address both these classes in the following sections.

%%% %%% %%% %%% 

\subsection{Auxiliary languages}

The first class is constituted by the languages discussed in §§\ref{pitsch:sec:old_symmetry}--\ref{pitsch:sec:asymmetry}, i.e. BCMS (both varieties), Bulgarian, Burgenland Croatian, Čakavian, Macedonian (both varieties), Slovene, Czech, Lower Sorbian, Slovak, and Upper Sorbian. All retain auxiliary verbs specified for person and number, hence \textphi$^+$. On the other hand, the participle in V$^0$ only specifies number and possibly also gender, hence \textphi$^-$.

Crucially, I claim that it is the verbiness of auxiliaries that allows them to select V\textsubscript{L} in V$^0$, which is therefore the only verb form available in \textsc{l-}periphrases.

According to \citet{Pietraszko2018,Pietraszko2021}, verbs carry [iV], while I$^0$ has [uV], which is checked against the closest [iV] (see \cites{Svenonius1994}{Chomsky1995}{Julien2002}{Adger2003}{Cowper2010}). Additionally, I argue that auxiliary verbs carry both [uV] and [iV], so they select (a verb in V$^0$) and are selected (by I$^0$) at the same time.\footnote{The feature [uV] of the auxiliary merely requires a verbal category in its complement domain. In addition, the auxiliary comes with a feature requiring that this verb be a V\textsubscript{L}.} In a subset of periphrases, said auxiliary verbs are generated as the head of an AuxP between IP and VP as shown in \REF{pitsch:ex:AuxPsyntax}.

\ea IP $>$ AuxP $>$ VP 
\label{pitsch:ex:AuxPsyntax}
\z

\noindent This spine underlies, for instance, the \textsc{l-}preterit in BCMS with the \textsc{3sg} \textit{je} `is' (see \citealt[838]{Tomic1996}) as well as the BCMS and Polish \textsc{l-}future (see \cites[331]{Browne1993}[275]{Migdalski2006}). The auxiliary in Aux$^0$ selects V\textsubscript{L} in V$^0$ and adds a person feature (\textphi$^+$). By contrast, V\textsubscript{L} is \textphi-incomplete (\textphi$^-$). Following \citet{Pietraszko2018}, the \textphi-probe undergoes feature percolation under V-checking, i.e., from V$^0$ (number/gender) and Aux$^0$ (adding person) to I$^0$. Only in its percolated position does the probe become active and enters in an Agree relation with a subject; see \REF{tree:select_1} and illustrations from BCMS in \REF{pitsch:ex:select_1.1} and \REF{pitsch:ex:select_1.2} $=$ \REF{pitsch:ex:fut_BCMS}.\footnote{The dashed arrow indicates the selection of V\textsubscript{L} by Aux$^0$. Inactive \textphi-features are gray.}

\ea 
\ea
\begin{forest}
for tree={s sep=15mm, inner sep=0, l=0}
[IP  
    [I\textsubscript{[\sout{uV},\textphi$^+$]},name=ihead]
        [AuxP 
            [Aux\textsubscript{[iV,\sout{uV},\textcolor{gray}{\textphi$^+$}]},name=auxhead] 
                [VP 
                    [V\textsubscript{L[iV,\textcolor{gray}{\textphi$^-$}]},name=vhead,roof] 
]]]
\draw[Circle-Circle] (ihead.210) to [out=south,in=south west,looseness=1.25] (auxhead.250); 
\draw[-Circle]       (auxhead.245) to [out=south,in=south west,looseness=1.25] (vhead.210);
% \draw[Circle-Circle] (auxhead.250) 
\draw[->,>=stealth',dashed] (auxhead) to[out=300,in=west] (vhead);
\end{forest}  
\label{tree:select_1}
\ex
\gll {[}\textsubscript{IP} $\varnothing$ [\textsubscript{AuxP} Ivana je [\textsubscript{VP} govori-l-a ]]] \\
{} \textsc{pst} {} I. \textsc{aux-3sg} {} speak\textsc{-l-sg.f} {} \\
\glt `Ivana (has) spoke(n)' 
\label{pitsch:ex:select_1.1}
\ex
\gll Kad [\textsubscript{IP} $\varnothing$ [\textsubscript{AuxP} \textit{pro} bude-mo [\textsubscript{VP} govori-l-i \ldots ]]] \\
when {} \textsc{fut} {} \textsc{1pl} \textsc{aux-1pl} {} speak\textsc{-l-pl.m} {} {} \\
\glt `When we will speak \ldots' \hfill (BCMS)
\label{pitsch:ex:select_1.2}
\z
\z

\noindent It is crucial that the auxiliary in Aux$^0$ selects (thanks to its verbiness) V\textsubscript{L} in V$^0$. As a consequence, any other verb form in V$^0$ is excluded.\footnote{The Polish \textsc{l-}future may also contain an infinitive in V$^0$. Arguably, the \textit{będ-}auxiliary has a (more) flexible selectional frame.}

However, besides Aux$^0$, auxiliary verbs may also reside in I$^0$. According to \citet[838]{Tomic1996}, this holds for so-called weak pronouns in BCMS (all except \textsc{3sg} \textit{je}, i.e. \textit{sam}, \textit{si}, etc.). \citet[275]{Migdalski2006} makes a similar claim for Polish (see \sectref{pitsch:sec:FullVerbs}). By and large the same is likely to be true for Bulgarian and Macedonian. In Czech and Slovak, the placement of the negation \textit{ne} relative to the forms of the \textsc{be}-auxiliary and V\textsubscript{L} provides evidence that auxiliaries are generally merged in I$^0$. By contrast, the full-verb (copular) forms of \textit{být/byť} `be' (which also figure in the participial passive) are best analyzed as being generated in Aux$^0$, whereas ordinary full verbs -- including V\textsubscript{L} -- are in V$^0$. 

The structure with auxiliaries generated directly in I$^0$ is shown in \REF{tree:select_2}, with a Czech illustration in \REF{pitsch:ex:select_2}.\footnote{Possibly, the subject pronoun \textit{já} in \REF{pitsch:ex:select_2} does not merely go to SpecIP but adjoins to IP to be interpreted as contrastive or verum focus \citep[see][]{JunghannsZybatow2009}.} A complete set of \textphi-features is present in I$^0$ since it is occupied by the auxiliary (here: \textit{jsem}). Feature percolation is thus confined to a possible gender feature on V$^0$ and may in fact rather amount to an Agree relation between the two \textphi-sets in V$^0$ and I$^0$. Quite like in \REF{tree:select_1}, the auxiliary selects V\textsubscript{L} in V$^0$. 

\ea 
\ea
\begin{forest}
for tree={s sep=15mm, inner sep=0, l=0}
[IP  
    [I\textsubscript{[\sout{uV},\textphi$^+$]},name=ihead]
                [VP 
                    [V\textsubscript{L[iV,\textcolor{gray}{\textphi$^-$}]},name=vhead,roof] 
]]
\draw[Circle-Circle] (ihead.250) to[out=south,in=west] (vhead.185);
\draw[->,>=stealth',dashed] (ihead.290) to[out=south,in=west] (vhead.170);
\end{forest}  
\label{tree:select_2}
\ex
\gll {[}\textsubscript{IP} Já jsem [\textsubscript{VP} $\langle$\sout{já}$\rangle$ pracova-l-a ]]. \\
{} \textsc{1sg} \textsc{pst.1sg} {} {} work\textsc{-l-sg.f} {} \\ 
\glt `It is me who (has) worked.' \hfill (Czech)
\label{pitsch:ex:select_2}
\z
\z

\noindent One way or the other, auxiliary languages have verbal auxiliaries with a complete \textphi-set that select V\textsubscript{L} in the main verb slot, which is why other verb forms (like the infinitive) are unavailable in this position. It is therefore that ``impersonal'' conditionals/subjunctives are not attested.

%%% %%% %%% %%% 

\subsection{Particle languages}\label{pitsch:sec:ParticleLanguages}

The second class is constituted by the ``North-Eastern group'', i.e. Kashubian, Polish as well as Belarusian, Russian, and Ukrainian. These languages have a particle both in the \textsc{l-}preterit and in the \textsc{l-}conditional. This situation is the result of the diachronic reshaping or loss, respectively, of the present-tense paradigm of `be'.

Using \citeposst{Tomic2000} terminology, the relevant particles are \textsc{operators}, as they are in a high functional position -- I$^0$ -- from where they supply the proposition as a whole with their tense/mood semantics. They have developed from former auxiliaries which lost their ``verbal character'' \citep{Issatchenko1940}. In other words, they do not specify person/number agreement anymore and may even be silent in some cases (as the East Slavic \textsc{l-}preterit).

%%% %%%

    \largerpage[-1]

\subsubsection{East Slavic and Kashubian-B}

For the East Slavic languages and Kashubian-B, diachronic changes had at least two crucial consequences: 

\begin{enumerate}
\item Since there was no other way left to encode agreement, V\textsubscript{L}, hitherto a participle specified only for number and gender, was reinterpreted as a fully-fledged (finite) form \citep[see][757]{Tseng2009}, i.e., it was additionally associated with an underspecified person feature (\citealt[174]{Junghanns1995}; see \sectref{pitsch:sec:EastSlavic}). In other words, V\textsubscript{L} enters the syntactic derivation equipped with a complete set of \textphi-features (\textphi$^+$).
\item Not being a verbal category, the particle in I$^0$ fails to select a specific form in V$^0$. As a consequence, V\textsubscript{L} is not the only choice, at least in the conditional/\hspace{0pt}subjunctive.\footnote{The past tense always and exclusively contains V\textsubscript{L}. I suspect that this is due to the fact that no other verb form could possibly reflect the presence of the silent past-tense operator in I$^0$ (note that the languages in question have long-since lost past-tense aorist and imperfect forms).} On feature checking, \textphi$^+$ percolates from V$^0$ to I$^0$, allowing the latter to establish an Agree relation with the subject.
\end{enumerate}

The corresponding syntactic structure with V$^0$ being occupied by a V\textsubscript{L} is given in \REF{tree:select_3}. Two Ukrainian examples are shown in \REF{pitsch:ex:select_3_1} (past tense) and \REF{pitsch:ex:select_3_2} (conditional), repeated from \REF{pitsch:ex:perf} and \REF{pitsch:ex:cond}, respectively.\footnote{In \REF{pitsch:ex:select_3_1}, I stay agnostic about the base and target positions of the wh-word. Arguably, in \REF{pitsch:ex:select_3_2}, the subject pronoun moves further to adjoin to IP (contrastive focus), while the direct object \textit{c'oho} has moved to SpecNegP.} In the glosses, I indicate that V\textsubscript{L} is equipped with an implicit person feature matching the subject.

\ea 
\ea
\begin{forest}
for tree={s sep=15mm, inner sep=0, l=0}
[IP  
    [I\textsubscript{[\sout{uV},\textphi$^+$]},name=ihead]
                [VP 
                    [V\textsubscript{[iV,\textcolor{gray}{\textphi$^+$}]},name=vhead,roof] 
]]
\draw[Circle-Circle] (ihead) to[out=south,in=west] (vhead);
\end{forest}  
\label{tree:select_3}
\ex
\gll Koly [\textsubscript{IP} ty $\varnothing$ [\textsubscript{VP} $\langle$\sout{ty}$\rangle$ narody-l-a-s' ]]?\\
when {} \textsc{2sg} \textsc{pst} {} {} give-birth\textsc{-l-[2]sg.f-refl} {} \\ 
\glt `When were you born?'
\label{pitsch:ex:select_3_1}
\ex 
\gll {[}\textsubscript{IP} Ja b [\textsubscript{NegP} c'oho ne [\textsubscript{VP} $\langle$\sout{ja}$\rangle$ skaza-v $\langle$\sout{c'oho}$\rangle$ ]]].\\
{} \textsc{1sg} \textsc{cond} {} this.\textsc{gen} \textsc{neg} {} {} say\textsc{-l.[1]sg.m} {} \\
\glt `I would not have said that.' \hfill (Ukrainian)
\label{pitsch:ex:select_3_2}
\z
\z

\noindent But V$^0$ can also be occupied by an infinitive. As infinitives lack \textphi-features, there is nothing to percolate to I$^0$, thus the only possible subject is \textphi-less PRO. The resulting syntactic structure in \REF{tree:select_4} is what we find in irrealis conditionals (subjunctives) like  \REF{pitsch:ex:select_4}.\footnote{In \sectref{pitsch:sec:CRaining}, I will argue that \textit{by} goes from I$^0$ to C$^0$ and fuses with it.}\textsuperscript{,}\footnote{\citet{Willis2000} argues that Russian \textit{by} is generated in C$^0$ as a result of grammaticalization. He claims that it was originally merged in I$^0$, from where it frequently moved to C$^0$ in Old East Slavic. Speakers then reanalyzed its derived position as underlying. I am hesitant to agree, mainly due to \textit{by}-doubling (\sectref{pitsch:sec:doubling_analysis}). A theoretical possibility is that there are two homophonous instances: \textit{by}\textsubscript{I} (conditional mood) and \textit{by}\textsubscript{C} (subjunctive clause type). The same might be true for Polish \textit{by}, which can introduce subjunctive clauses even without a complementizer (an advocate for a \textit{by}\textsubscript{C} is \citealt[108]{Jedrzejowski2020}). Still, I prefer a movement/copying analysis with only one \textit{by} (like \citealt[259]{Migdalski2006}).}

\ea 
\ea
\begin{forest}
for tree={s sep=15mm, inner sep=0, l=0}
[IP  
    [I\textsubscript{[\sout{uV}]},name=ihead]
                [VP 
                    [V\textsubscript{INF}\textsubscript{[iV]},name=vhead,roof] 
]]
\draw[Circle-Circle] (ihead) to[out=south,in=west,looseness=1.2] (vhead);
\end{forest}  
\label{tree:select_4}
\ex
\gll {[}\textsubscript{CP} čto [\textsubscript{IP} by [\textsubscript{VP} PRO rabota-t' ]]] \\
{} that {} \textsc{cond} {} {} work-\textsc{inf} {} \\ 
\glt `(in order) to work' \hfill (Russian)
\label{pitsch:ex:select_4}
\z
\z

%%% %%%

\subsubsection{Polish and Kashubian-A1}

\noindent Polish and Kashubian-A1 possess atonic person/number agreement markers. I adopt the view that these markers are clitics. \citet{Embick1995} proposes that they are generated as adjuncts to I$^0$. Unlike in East Slavic and Kashubian-B, this ensures that person/number agreement is encoded in a position distinct from V$^0$, so V\textsubscript{L} is not in need of an underspecified person feature -- it is a participle proper. In other words, Polish and Kashubian-A1 have substituted their former auxiliary verbs (once in Aux$^0$) with composite items in I$^0$. These items consist of a particle ($\varnothing$ in the past tense, \textit{by/bë} in the conditional) plus an agreement marker; see \figref{Pitsch:fig:Embick}.

\begin{figure}
\centering
\begin{forest}
    [I 
        [I\\$\lbrace \varnothing$/\textit{by}$\rbrace$][AGR]] 
\end{forest} 
 \caption{Analytic I$^0$ according to \citet{Embick1995}}
 \label{Pitsch:fig:Embick}
\end{figure}
 
However, \citeposst{Embick1995} analysis has one crucial theoretical disadvantage: It assumes that two heads are base-generated as adjuncts to each other, which involves the danger of overgeneralization. To avoid this problem, I again follow \citet{Pietraszko2018,Pietraszko2021} who argues against the common claim that auxiliaries in periphrases are necessarily generated as heads within the clausal spine (i.e. Aux$^0$ or I$^0$). They can also be generated in SpecIP. To implement this alternative, \citeauthor{Pietraszko2021} adopts \citeposst{Matushansky2006} idea of ``m-merger'': A head is merged in the specifier of a functional head (here: I$^0$) and subsequently adjoins to that head to form an inseparable unit; see \figref{Pitsch:fig:Pietraszko_a} and \figref{Pitsch:fig:Pietraszko_b}, respectively.

\begin{figure}
     \centering
     \begin{subfigure}[b]{0.45\textwidth}
         \centering
\begin{forest}
[IP
    [AGR\textsubscript{$\lbrack$\textphi$^+\rbrack$}]
        [I$'$
        [I\textsubscript{$\lbrack$uV$\rbrack$}\\$\lbrace \varnothing$/\textit{by}$\rbrace$]
        [\ldots]
    ]
]
\end{forest} 
    \caption{Base generation in SpecIP}
    \label{Pitsch:fig:Pietraszko_a}
     \end{subfigure}
     \hfill
     \begin{subfigure}[b]{0.45\textwidth}
         \centering
\begin{forest}
[IP
    [$\langle$\sout{AGR}$\rangle$,name=Spec]
        [I$'$
            [I\textsubscript{$\lbrack$uV,\textphi$^+\rbrack$}
                [I\\$\lbrace \varnothing$/\textit{by}$\rbrace$][AGR,name=Head]
            ]
        [\ldots]
        ]
]
\end{forest} 
    \caption{Subsequent adjunction to I$^0$}
    \label{Pitsch:fig:Pietraszko_b}
     \end{subfigure}
     \caption{M-merger of an agreement marker (AGR)}
     \label{Pitsch:fig:Pietraszko}
\end{figure}

This analysis eliminates the danger of overgeneralization inherent to \citeposst{Embick1995} approach as there is a clearly defined motivation for merging the agreement marker in SpecIP: It compensates for the missing person feature in V$^0$. At the same time, the analysis yields the same syntactic configuration as in \figref{Pitsch:fig:Embick} -- i.e. \figref{Pitsch:fig:Pietraszko_b} -- and thus preserves its advantages.\footnote{See \citet{Abramowicz2008} for a survey of the advantages of \citeposst{Embick1995} analysis as compared to alternative approaches. Note that \citeposst{Pietraszko2018} approach is not restricted to the X-bar framework but also works under Bare Phrase and Labeling Algorithm assumptions.}

As a result of the adjunction in \figref{Pitsch:fig:Pietraszko_b}, the particle in I$^0$ and the agreement marker fuse into a complex I$^0$ specified with [uV] and [\textphi$^+$]. This gives a constellation very much similar to \REF{tree:select_2}. Put differently, Polish and Kashubian-A1 ""reconstruct" an analytic auxiliary verb in I$^0$.

Another advantage of the view that the atonic agreement markers are syntactic heads is that they can also be absent. If there is no agreement marker generated in SpecIP, I$^0$ stays a mere tense/mood particle and the clause lacks \textphi-features. Like in East Slavic and Kashubian-B, this makes it possible to have an infinitive or impersonal \textit{no/to}-form in V$^0$ and, consequently, a PRO subject as shown for Russian in \REF{tree:select_4} and \REF{pitsch:ex:select_4}.

%%% %%%

\subsubsection{Full verb `be' in East Slavic and Polish}\label{pitsch:sec:FullVerbs}

The present proposal should also be able to deal with the full verb (copular) forms of `be' in Polish and East Slavic.

As to Polish, I agree with \citet{Migdalski2006} that \textit{jest} and \textit{są} do not specify any person feature (also see \citealt{Tomic1996} on BCMS \textit{je}). Like \citet[275]{Migdalski2006}, I analyze them as heading an AuxP; see \REF{pitsch:ex:CopPol}.\footnote{In copular clauses, V$^0$ is silent but introduces a situation argument as well as argument slots for the predicate nominal and the subject. This silent head corresponds to \citeposst{Bowers1993} Pr(ed)$^0$ (see also \citealt{Bailyn2001,Bailyn2012, Markman2008Case}), \citeposst{Citko2008} \textpi$^0$, or \citeposst{denDikken2006} Rel$^0$.}

\ea
\gll I$^0$ $>$ {} Aux$^0$ $>$ V$^0$ $>$ XP \\
\textsc{mood/tense/agr} {} \minsp{\{} \textit{jest/są}\} {} {zero copula} {} {predicate nominal} \\
\label{pitsch:ex:CopPol}
\z

\noindent Except for the 3rd person, \textit{jest} (only in dialects also \textit{są}) usually raises to I$^0$ -- a silent present-tense operator -- to adjoin to the agreement marker m-merged with that operator; see \REF{pitsch:ex:CopPol_a}. However, though nowadays rarely, \textit{jest/są} can also stay \textit{in situ}; see \REF{pitsch:ex:CopPol_b}. In the 3rd person, \textit{jest/są} always stay \textit{in situ}; see \REF{pitsch:ex:CopPol_3PS}. Finally, Aux$^0$ may be absent as in \REF{pitsch:ex:CopPol_Silent}.

\ea 
\ea {[}\textsubscript{IP} ja [\textsubscript{I} jest+[\textsubscript{I} $\varnothing+$(e)m]] [\textsubscript{AuxP} $\langle$\sout{jest}$\rangle$ [\textsubscript{VP} $\langle$\sout{ja}$\rangle$ $\varnothing$\textsubscript{Cop} [\textsubscript{AP} głodny ]]]] \label{pitsch:ex:CopPol_a} 
\ex {[}\textsubscript{IP} ja [\textsubscript{I} $\varnothing+$m] [\textsubscript{AuxP} jest [\textsubscript{VP} $\langle$\sout{ja}$\rangle$ $\varnothing$\textsubscript{Cop} [\textsubscript{AP} głodny ]]]] \label{pitsch:ex:CopPol_b}
\z
`I am hungry'
\z

\ea {[}\textsubscript{IP} Anna $\varnothing$\textsubscript{I} [\textsubscript{AuxP} jest [\textsubscript{VP} $\langle$\sout{Anna}$\rangle$ $\varnothing$\textsubscript{Cop} [\textsubscript{AP} głodna ]]]]\\`Anna is hungry'
\label{pitsch:ex:CopPol_3PS}
\z

\ea {[}\textsubscript{IP} ja [\textsubscript{I} $\varnothing+$m] [\textsubscript{VP} $\langle$\sout{ja}$\rangle$ $\varnothing$\textsubscript{Cop} [\textsubscript{AP} głodny ]]]\\`I am hungry' \label{pitsch:ex:CopPol_Silent}
\z

\noindent The situation is different in East Slavic: I follow \citet{Issatchenko1940} in that Russian \textit{estʹ} has become a particle, and argue that this translates into a shift from Aux$^0$ to I$^0$. In other words, Belarusian \textit{ëscʹ}, Russian \textit{estʹ}, and Ukrainian \textit{je}, respectively, are the overt variant of an otherwise silent I-head encoding the present tense. Their being overt nicely matches the fact that they are, unlike Polish \textit{jest}, emphatic (verum or contrastive focus; see \citealt[127]{Geist2007}); see \REF{pitsch:ex:CopEast}.\footnote{Taking into consideration that ``usual'' verbs are emphasized by means of contrastive intonation, the existence of an overt present-tense I$^0$ specifically for copular clauses is likely to be due to the lack of overt present-tense \textsc{be}-forms in East Slavic.}

\ea {[}\textsubscript{IP} Anna estʹ\textsubscript{I} [\textsubscript{VP} $\langle$\sout{Anna}$\rangle$ $\varnothing$\textsubscript{Cop} [\textsubscript{AP} golodna ]]]\\`Anna IS hungry'
\label{pitsch:ex:CopEast}
\z

%%% %%% %%% %%% 

\subsection{The doubling issue}\label{pitsch:sec:doubling_analysis}

The analysis proposed in \sectref{pitsch:sec:syntax} covers all ``standard'' examples including those with reduced particles.\footnote{The reason particles can be reduced is that their final segments do not encode grammatical information and can thus be dropped on phonological grounds.} However, the phenomenon of particle doubling in conditional clauses described in \sectref{pitsch:sec:doubling} still calls for a syntactic explanation. 

For the time being, the data suggest that said doubling is characteristic of the ``North-Eastern group'', i.e. Kashubian, Polish, and the East Slavic languages. Therefore, I suspect that there is a connection between particle doubling and the syntactic peculiarities of the relevant languages. More precisely, I suggest that it is the existence of a conditional particle that enables its reduplication.

Any syntactic analysis designed to capture the doubling phenomenon has to ensure that there can be two instances of the conditional particle in the same clause. Furthermore, the second instance must be semantically vacuous, as doubling affects only the surface form, not meaning and interpretation (there is no doubling of the irrealis semantics in the sense of decreased probability, counterfactuality, or the like). 

%%% %%%

\subsubsection{Multiple copies}

The most straightforward way to achieve these goals is provided by the \textsc{Copy Theory of Movement} \citep[see, a.o.,][]{Chomsky1993,Nunes1995,CorverNunes2007}. According to this theory, the syntactic trace left behind of a moved element (``\textalpha'') is a copy of that very element; see \REF{pitsch:ex:copy}.

\ea {[}\textsubscript{XP} {\textalpha} [\textsubscript{YP} {\textalpha} ]] 
\label{pitsch:ex:copy}
\z 

\noindent As a rule, only one copy is pronounced. The choice is mostly considered a matter of phonology (PF). Thus, either the lower or the higher copy of {\textalpha} is deleted at PF; see \REF{pitsch:ex:copy_2.1} and \REF{pitsch:ex:copy_2.2}, respectively.

\ea
\ea {[}\textsubscript{XP} {\textalpha} [\textsubscript{YP} \sout{\textalpha} ]] \label{pitsch:ex:copy_2.1}
\ex {[}\textsubscript{XP} \sout{\textalpha} [\textsubscript{YP} {\textalpha} ]] \label{pitsch:ex:copy_2.2}
\z
\z 

\noindent However, there is evidence that more than one copy of {\textalpha} can be pronounced within the same clause (see, e.g., \citealt{BoskovicNunes2007} on so-called wh-copying constructions in, i.a., German, Afrikaans, and Romani). Slavic doubling data such as \REF{pitsch:ex:doubling_Russian} $=$ \REF{pitsch:ex:doubling_Russian_rep} show (i) that \textit{by} is indeed copied, (ii) that both copies are within the same clause, and (iii) that both copies are pronounced. 

\ea
\ea\gll Ja by pogulja-l by segodnja večerom. \\
I \textsc{cond} take.a.walk-\textsc{l.sg.m} \textsc{cond} today evening.\textsc{ins} \\ \hfill (Russian)
\glt `I would like to take a walk tonight' \label{pitsch:ex:doubling_Russian_rep_a}
\ex\gll Čto-by ja tebja by zdes' bol'še ne vide-l. \\
that.\textsc{cond} I you.\textsc{acc} \textsc{cond} here more \textsc{neg} see-\textsc{l.sg.m} \\
\glt `So that I would not see you here again.' \label{pitsch:ex:doubling_Russian_rep_b}
\z
\label{pitsch:ex:doubling_Russian_rep}
\z

\noindent It is noteworthy that the number of copies of the particle does not exceed two. A straightforward way to account for this fact is that \textit{by} in \REF{pitsch:ex:doubling_Russian_rep} occupies two distinct syntactic positions, and that there are no more than two such positions available to host its copies.

%%% %%%

\subsubsection{One particle in I$^0$ and C$^0$}\label{pitsch:sec:CRaining}

From the analysis in \sectref{pitsch:sec:ParticleLanguages}, it follows that one of these positions must be I$^0$, the basic position of the conditional particle. The second position must be higher in the tree, which makes C$^0$ a strong candidate. The fact that in the majority of doubling examples -- see \REF{pitsch:ex:doubling_Russian_rep_b} -- the higher copy of the particle is adjacent to a complementizer, confirms this location.\footnote{See also \citet[413]{Szucsich2009} who argues that it is only through ``subjunctive raising'' of \textit{by} from I$^0$ to C$^0$ that the irrealis feature of \textit{by} becomes visible to the embedding matrix verb.}

Additional evidence for C$^0$ as the target position of the conditional particle comes from examples without doubling. The particle occurs in two alternative positions in all languages under discussion. An example is the Russian minimal pair in \REF{pitsch:ex:Russian_alternative}.\footnote{It is an open question which of the two positions is more frequent. Based on a ``somewhat restricted data sample'' \citep[330]{Hansen2010}, \citet[60]{Hacking1998} spots a tendency for Russian \textit{by} to occur immediately after the subordinating conjunction in the protasis but immediately after the verb in the apodosis of conditional sentences. \citet{Hansen2010} notes that it is not difficult to detect counterexamples and that a more refined corpus-based empirical investigation is necessary to verify the distribution of \textit{by}.}

\ea
\ea 
\gll Ja vypi-l by stakan moloka.\\
I drink.up-\textsc{l.sg.m} \textsc{cond} glas milk.\textsc{gen}\\ 
\glt `I would like a glass of milk.' \hfill \citep[277]{Xrakovskij2009}
\ex
\gll Ja by vypi-l stakan moloka.\\
I \textsc{cond} drink.up-\textsc{l.sg.m} glas milk.\textsc{gen}\\
\glt `As for me, I would like a glass of milk.' \hfill (Russian)
\label{pitsch:ex:Russian_alternative_b}
\z
\label{pitsch:ex:Russian_alternative}
\z

\noindent Moreover, constituents that appear clause-initially and left of the conditional particle are interpreted as topic or focus (see, a.o., \cites[327]{Willis2000}[230--231]{Migdalski2006}), which holds for sentences with and without doubling; see the Russian examples in \REF{pitsch:ex:Russian_alternative_b} and \REF{pitsch:ex:doubling_Russian_rep_a}, respectively.

Finally, Polish provides additional evidence for \textit{by} in C$^0$: As shown in \REF{pitsch:ex:PolJedrz}, in subjunctive clauses the conditional particle (with or without agreement) occurs in a sentence-initial position either adjacent to a complementizer or alone. (It must not occur adjacent to V\textsubscript{L}.)

\ea 
\gll Każda matka chce, \minsp{(} że-)by jej syn chodzi-ł do przedszkola. \\
every mother want.3sg {} that-\textsc{cond} her son go-\textsc{l.sg.m} to kindergarten.\textsc{gen}\\
\glt `Every mother wants her son to go to the kindergarten.' \\ \hfill (Polish; \citealt[109]{Jedrzejowski2020})
\label{pitsch:ex:PolJedrz}
\z

\noindent A doubling example with ``solitary'' subjunctive \textit{by} is \REF{pitsch:ex:doubling_Polish_coll} $=$ \REF{pitsch:ex:doubling_Polish_coll_rep}.\footnote{Note that the agreement marker in \REF{pitsch:ex:doubling_Polish_coll_rep} occurs only on the higher copy. See \sectref{pitsch:sec:Fusion}.}

\ea
\gll nie sądzę, by-śmy by-l-i-by tak blisko siebie \ldots \\
\textsc{neg} think.\textsc{1sg} \textsc{cond-1pl} be\textsc{-l-pl.m-cond} so close \textsc{refl.acc} {} \\ 
\glt `I don't think we would be as close to each other \ldots' \hfill (Polish) \label{pitsch:ex:doubling_Polish_coll_rep}
\z

\noindent To summarize, it is a plausible claim that the conditional particle in particle languages is base-generated in I$^0$ and can subsequently be copied to C$^0$.

%%% %%%

\subsubsection{Fusion and doubling}\label{pitsch:sec:Fusion}

Following \citet{Nunes2004}, cases of simultaneous pronunciation of multiple copies are always due a morphological reanalysis of one of the copies as part of a bigger unit (``word''). He argues that this reanalysis corresponds to a syntactic operation combining two terminal nodes into one, i.e. \textsc{fusion} (see \cites{HalleMarantz1993}[§3.1]{MunozPerez2018}). Crucially, although fusion is the prerequisite for multiple-copy pronunciation, there is no mutual dependence: Fusion can well take place without only one overt copy.

I believe that \citeauthor{Nunes2004}' claim is in accordance with the Slavic data: Thus, in \REF{pitsch:ex:doubling_Russian_rep_a}, the higher copy of Russian \textit{by} is likely to fuse with C$^0$ which allows the pronunciation of both \textit{by}-copies. Presumably, the fact that \textit{by} is copied from I$^0$ to C$^0$ in the first place is linked to the information-structural status of the subject \textit{ja} `I': Its interpretation as topic depends on its being in a sentence-initial position and left of \textit{by}, so \textit{ja} itself has to go to SpecCP, while \textit{by} is copied to C$^0$; see \REF{pitsch:ex:doubling_Russian_rep_analysis}.\footnote{The verb \textit{poguljal} adjoins to I$^0$ for information-structural reasons, namely to leave the adverbial \textit{segodnja večerom} stranded in a clause-final position (information focus). The verb (meaning) itself is thus presented as (presupposed) background information. Additionally, the verb functions as a phonological host for the enclitic lower copy of \textit{by} in I$^0$.}

\ea 
{[}\textsubscript{CP} ja $\varnothing$\textsubscript{C}$+$by [\textsubscript{IP} poguljal$\textsubscript{V}+$by\textsubscript{I} [\textsubscript{VP} segodnja večerom [\textsubscript{VP} \sout{ja} \sout{poguljal} ]]]]
\label{pitsch:ex:doubling_Russian_rep_analysis}
\z

\noindent The same can be stated about \REF{pitsch:ex:Russian_alternative_b}, with the exception that here the lower copy of \textit{by} is deleted at PF, which complies with what prescriptive grammars require.

Fusion of \textit{by} with C$^0$ takes also place in \REF{pitsch:ex:doubling_Russian_rep_b}, and again it enables the doubling of \textit{by}; see \REF{pitsch:ex:doubling_Russian_rep_b_analysis}. Since C$^0$ hosts the complementizer \textit{čto} `that', the result is the complex C$^0$ \textit{čtoby}, which ``is sometimes treated as an independent lexeme and sometimes as a syntactic combination of two lexemes'' \citep[329]{Hansen2010}. I wish to claim that both views are justified: Before the fusion, there are two lexemes. After it, they have become one element.\footnote{Both the subject \textit{ja} and the object \textit{tebja} adjoin to IP to be backgrounded, so the (negated) verb is focused. I ignore the internal structure of the IP (NegP/VP) and V-to-Neg movement in \REF{pitsch:ex:doubling_Russian_rep_b_analysis}.}

\ea
{[}\textsubscript{CP} čto\textsubscript{C}$+$by [\textsubscript{IP} ja [\textsubscript{IP} tebja [\textsubscript{IP} by\textsubscript{I} zdes' bol'še ne \sout{ja} videl \sout{tebja} ]]]]
\label{pitsch:ex:doubling_Russian_rep_b_analysis}
\z

\noindent As mentioned above, Polish subjunctive clauses can be introduced with or without a complementizer. In other words, \textit{by} alone may, in addition to its basic conditional meaning, assume the function of a complementizer in subjunctive clauses. I suggest that both variants -- with and without a ``true'' complementizer -- have the same underlying syntax, the only difference being that C$^0$ is overtly filled in the former but silent in the latter case; see \REF{pitsch:ex:PolJedrz_analysis_a} and \REF{pitsch:ex:PolJedrz_analysis_b}, respectively.

\ea
\ea {[}\textsubscript{CP} że\textsubscript{C}$+$by [\textsubscript{IP} \sout{by} [\textsubscript{VP} jej syn chodził do przedszkola ]]] \label{pitsch:ex:PolJedrz_analysis_a}
\ex {[}\textsubscript{CP} $\varnothing$\textsubscript{C}$+$by [\textsubscript{IP} \sout{by} [\textsubscript{VP} jej syn chodził do przedszkola ]]]
\label{pitsch:ex:PolJedrz_analysis_b}
\z
\z

\noindent In both variants does fusion of C$^0$ with \textit{by} take place, yielding a complex C-head encoding subjunctive mood \citep[251]{Migdalski2006}. Thus, in a sense, \citet[109]{Jedrzejowski2020} is right in claiming that in \REF{pitsch:ex:PolJedrz}, ``[i]t is [\dots] \textit{by} which introduces the embedded clause and marks its illocutionary force as well its subordinate status.'' Crucially, however, the latter is due to the silent C-head fused with \textit{by}.

There is another issue that calls for an explanation: In Polish doubling examples such as \REF{pitsch:ex:doubling_Polish_coll_rep} with the verb in the first or second person, person/number agreement occurs only once, namely on the higher copy of \textit{by}. If \textit{by} is copied from I$^0$ to C$^0$ and subsequently fuses with it, allowing both copies to be pronounced, why does only the higher copy encode agreement? Following the Copy Theory of Movement, the way to account for this pattern is to say that, while \textit{by} is pronounced in both positions, the agreement marker is deleted in the lower one. This is shown in \REF{pitsch:ex:doubling_Polish_coll_b_analysis}. 

\ea
{[}\textsubscript{CP} $\varnothing$\textsubscript{C}$+$byśmy [\textsubscript{IP} byli$\textsubscript{V}+$by\sout{śmy}\textsubscript{I} [\textsubscript{VP} \textit{pro} \sout{byli} tak blisko siebie ]]]
\label{pitsch:ex:doubling_Polish_coll_b_analysis}
\z

\noindent As to the reason for the deletion of the agreement marker, I propose that it follows from economy: There is simply no need to pronounce it twice. Note that, in \REF{pitsch:ex:doubling_Polish_coll_b_analysis}, \textit{by} is pronounced in I$^0$ to reveal the movement (and concomitant backgrounding) of the verb \textit{byli} from V$^0$ to I$^0$. There is no need, however, to also pronounce the agreement marker in I$^0$ since the particle alone is perfectly sufficient to accomplish the task.

%%% %%% %%% %%%  %%% %%% %%% %%% 
%%% %%% %%% %%%  %%% %%% %%% %%% 

\section{Summary}\label{pitsch:sec:summary}

This paper provides evidence for a typological division of the Slavic languages into auxiliary languages and particle languages based on the kind of auxiliary unit used in the \textsc{l-}preterit and \textsc{l-}conditional. Where the members of the former group have inflected auxiliary verbs that encode person/number agreement, the latter have noninflected particles lacking any agreement whatsoever.

The group of particle languages is constituted by Polish, Kashubian, and East Slavic. In the East Slavic languages and Kashubian-B, the particle is generated in I$^0$ where it encodes the irrealis mood. Crucially, it does not select any specific verb form in V$^0$ which allows this position to be filled not only with an \textsc{l-}form but also with other forms, most prominently the infinitive. 

Polish and Kashubian-A1 are similar in that they, too, have a particle in I$^0$. However, they stand out within the Slavic branch due to the availability of mobile inflections. In the present paper, these markers are analyzed as syntactic heads generated in SpecIP and subsequently m-merged with I$^0$, thus yielding a complex inflectional unit encoding tense/mood and agreement \citep[see][]{Embick1995}. 

Put differently, Polish and Kashubian-A1 are able to furnish their tense/mood particle in I$^0$ with person/number agreement, whereas East Slavic and Kashubian-B are not. From this it follows that present-day East Slavic and Kashubian-B have \textsc{l-}forms associated with an underspecified person feature \citep[see][]{Junghanns1995}, while Polish and Kashubian-A1 -- on a par with the remaining Slavic languages~-- have \textsc{l-}participles (number and gender only). 

Moreover, the claim that Polish agreement markers are syntactic heads that are initially generated independently of the particle in I$^0$ provides a straightforward explanation for why Polish allows, besides \textit{l-}participles, infinitives and \textit{no/to-}forms: The agreement marker may simply not be part of the numeration. If this is the case, the structure is impersonal (lack of person/number agreement).

Crucially, the analysis put forward explains the observation that the conditional in auxiliary languages is limited to verbal \textsc{l-}forms, whereas it is not in particle languages: Auxiliaries retain their ``verbal character'' \citep{Issatchenko1940} including the capacity to select specific verb forms in their complement position. By contrast, particles are no verbal categories anymore, which is why there is no selection, hence the wider range of possible forms in V$^0$. 

Finally, the phenomenon of particle doubling attest in colloquial particle languages receives a syntactic explanation: They can be copied from I$^0$ to C$^0$, and both copies can be pronounced under specific circumstances (mostly related to information structure).

The present paper shows that there is a remarkable cross-Slavic variation, which is especially true of the auxiliary unit in the conditional periphrasis: While some languages either retain the inherited suffixes or replaced them with present-tense inflections, others developed a pseudo-particle (an underspecified auxiliary verb with a silent agreement suffix), while still others use analogy-based suffixes which look like the clitic \textsc{be}-auxiliaries from the \textsc{l-}preterit (``pseudo-clitics''). Despite these differences, all relevant languages possess inflected auxiliary verbs, which distinguishes them from particle languages. An overview is given in \tabref{pitsch:tab:classification}. 

\begin{table}
\begin{tabular}{lllll}
\lsptoprule
\multicolumn{3}{c}{auxiliary languages with} & \multicolumn{2}{c}{particle languages with} \\\cmidrule(lr){1-3}\cmidrule(lr){4-5}
inflectional& pseudo-       & silent        & no            & mobile \\
suffixes    & clitics       &  suffixes     & agreement     & agreement \\
\midrule
BCMS-A      & Czech-B       & BL Croatian   & Kashubian-B   & Kashubian-A  \\
Bulgarian   & Macedonian\textsuperscript{+}   & Macedonian    & Belarusian    & Polish \\
Čakavian    & Slovak        & Slovene       & Russian       & \\
Czech-A     &               & L. Sorbian    & Ukrainian     & \\
U. Sorbian  &               &               &               & \\
\lspbottomrule
\end{tabular}
    \caption{Auxiliary and particle languages}
    \label{pitsch:tab:classification}
\end{table}

Theoretically, the present paper argues in favor of a formalization of the auxiliary/particle distinction in morphosyntactic terms: Whereas the former are verbal categories generated in Aux$^0$ or I$^0$, the latter lost their verbal character and are particles (i.e. tense/mood operators) generated in I$^0$.

Overall, the paper reveals that the variation in auxiliary units in Slavic peri\-phrases raises a bulk of empirical and theoretical questions. For some, I hope to have provided convincing proposals.

%%% %%% %%% %%%  %%% %%% %%% %%% 
%%% %%% %%% %%%  %%% %%% %%% %%% 

\section*{Abbreviations}

\begin{tabularx}{.5\textwidth}{@{}lX@{}}
\textsc{1/2/3}&first/second/third person\\
\textsc{acc}&accusative\\
\textsc{AU}&auxiliary unit\\
\textsc{cond}&conditional\\
\textsc{dat}&dative\\
\textsc{f}&feminine\\
\textsc{fut}&future\\
\textsc{gen}&genitive\\
\textsc{imp}&imperative\\
\textsc{imps}&impersonal\\
\textsc{inf}&infinitive\\
\textsc{ins}&instrumental\\
\textsc{irr}&irrealis\\
\end{tabularx}%
\begin{tabularx}{.5\textwidth}{@{}lX@{}}
\textsc{l}&\textit{-l-}suffix\\
\textsc{loc}&locative\\
\textsc{m}&masculine\\
\textsc{n}&neuter\\
\textsc{neg}&negation\\
\textsc{nom}&nominative\\
\textsc{part}&particle\\
\textsc{pst}&past\\
\textsc{pl}&plural\\
\textsc{refl}&reflexive marker\\
\textsc{sg}&singular\\
\textsc{V\textsubscript{L}}&verbal \textsc{l-}form\\
&\\
\end{tabularx}

%%% %%% %%% %%%  %%% %%% %%% %%% 
%%% %%% %%% %%%  %%% %%% %%% %%% 

\section*{Acknowledgements}
I am grateful to the audiences of SLS-15 (Bloomington, IN), FDSL-14 (Leipzig), the Slavic Linguistics Colloquium (specifically Berit Gehrke), and the Slavic Linguistics Colloquium in Göttingen for their valuable comments. I also wish to express my thanks to two anonymous reviewers and the editors of the present volume. Special thanks are due to Danuta Rytel-Schwarz for turning my attention to the possible relevance of the peculiar developments in Polish as well as to Krzysztof Migdalski for many valuable remarks.

%%% %%% %%% %%%  %%% %%% %%% %%% 
%%% %%% %%% %%%  %%% %%% %%% %%% 

\printbibliography[heading=subbibliography,notkeyword=this]

\end{document}
