\documentclass[output=paper]{langscibook} 

\ChapterDOI{10.5281/zenodo.10123633}
\author{Mojmír Dočekal\affiliation{Masaryk University} and Lucia Vlášková\affiliation{Masaryk University} and Maria Onoeva\affiliation{Charles University}}
\title{Degree achievements from a Slavic perspective}
\abstract{The evaluative behaviour of degree achievements (e.g., \textit{cool, widen, lengthen, dry}) has been a~puzzling problem for many linguists. The currently standard theory \citep{kennedy2008measure} treats them as degree expressions based on different types of scales, which in turn influence the resulting evaluative or non-evaluative interpretation. While it may account for English, this theory faces empirical problems when confronted with cross-linguistic data. In this paper, we present an experiment on Russian exploring if verbal prefixes influence the (non-)evaluative interpretation of degree achievements. It follows from the results that prefixation is at least as important as the underlying scales for the cases we studied, which empirically challenges the scalar theory.

\keywords{degree achievements, evaluativity, prefix, Russian, Slavic, experimental evidence}}

\lsConditionalSetupForPaper{}

\begin{document}
\maketitle


\section{Introduction}\label{DA:sec:Introduction}

The current paper describes the relationship of evaluativity inferences of adjectival degree achievements with Slavic verbal morphology, namely verbal prefixes. We also report the results of an experiment testing  said relationship in Russian degree achievements.

Degree achievements such as \textit{increase} or \textit{age} are typically analysed as verbs where the argument undergoes a~positive scalar change, e.g., in the sentence \textit{The river widened}, the degree of the river's width undergoes a~positive change ($=$~increases) along some relevant dimension ($=$~width). A~large group of degree achievements consists of verbs derived from gradable adjectives, such as English \textit{widen}\textsubscript{V} from \textit{wide}\textsubscript{A} or \textit{empty}\textsubscript{V} from \textit{empty}\textsubscript{A}. These deadjectival degree achievements will be the focus of the current paper and the experiment on Russian.

It is common to analyse gradable adjectives via the formal semantics notion of an underlying scale. The underlying scales can differ with regard to their openness. A~scale is open when there are no endpoints specified; this leads to relative gradable adjectives, where the standard of comparison that is needed to license the positive form of such an adjective is supplied via the context of utterance. Hence, as an example, it will take different absolute lengths to be considered \textit{a~long desk} and \textit{a~long boat}. 

On the other hand, a~scale with at least one endpoint gives rise to an absolute gradable adjective: the upper-bounded adjectives have the maximum endpoint specified, the lower-bounded ones have the minimum, and closed-scale adjectives have both endpoints. The standard of comparison used in positive forms is then taken to be the specified endpoint of the given scale. Therefore, context does not play the same role as in the relative adjectives and there should be no difference in the degrees of dryness in \textit{a~dry desk} and \textit{a~dry boat}.

This division is supported by the different patterning of modifiers with different types of adjectives, as shown below. We take \textit{almost} and \textit{slightly} as examples of modifiers that are licensed only in particular situations: (i) for \textit{almost}, the scale in question has to have the maximum endpoint specified, hence the acceptability with upper-bounded and closed-scale adjectives; whereas (ii) for \textit{slightly}, it is the other way around -- the scale needs a minimum endpoint, as in lower-bounded and closed-scale adjectives. Naturally, the scale with no specified endpoints does not accept either of the mentioned modifiers.   

\begin{enumerate}
  \item relative adjectives: \textit{*almost long}, *\textit{slightly tall}
  \item absolute adjectives
  \begin{enumerate}
    \item[2.1] upper-bounded: \textit{almost dry}, \textit{*slightly clean}
    \item[2.2] lower-bounded: \textit{*almost dirty}, \textit{slightly wet}   
    \item[2.3] closed-scale: \textit{almost opaque}, \textit{slightly transparent}
  \end{enumerate}
\end{enumerate}

The scale typology will be important also while discussing degree achievements. Below, we will argue, following \citet{kennedy2008measure} among others, that the underlying scale of the adjective remains in the meaning of the derived degree achievement and influences its telicity and evaluativity behaviour.\largerpage

This paper is structured as follows: \sectref{DA:subsec:Telicity and evaluativity behaviour of degree achievements} and \sectref{DA:subsec:Accounts of degree achievements} present an overview of the degree achievement research, as well as the currently standard scalar theory. In \sectref{DA:sec:Slavic degree achievements}, we turn to Slavic degree achievements with a focus on their prefixation pattern. \sectref{DA:sec:Experiment} reports the experiment testing the evaluative inferences of Russian degree achievements. Finally, \sectref{DA:sec:Summary} summarises the article.

\subsection{Telicity and evaluativity behaviour of degree achievements}\label{DA:subsec:Telicity and evaluativity behaviour of degree achievements}\largerpage

Degree achievements are a~puzzling group with regard to their telicity and evaluativity behaviour, as was first noted by \citet{dowty1979word}. Moreover, these two notions have been often confused in the previous literature due to the misunderstandings in the terminology. This section aims to delineate the two aspects and clarify the terminology used in this paper.

We understand \textsc{telicity} as a property of verb phrases that denote an action or an event with a specific endpoint. Let us first look at the telicity pattern in motion verbs as a basis of the later comparison to degree achievements.

According to the standard telicity test of the acceptability of the adverbial phrase \textit{for/in an hour}, the predicate \textit{walked} in (\ref{DA:ex:John walked-a}) is atelic (having no specific endpoint). However, when an argument is added, e.g., \textit{to the pub}, the whole predicate \textit{walked to the pub} in (\ref{DA:ex:John walked-b}) becomes telic and licenses the adverbial \textit{in an hour}. The telic event is maximal in a sense that it reaches its goal, so the VP \textit{walked to the pub} describes such events where its agent ends in the pub. Thus, motion verbs can change their telicity according to the supplied arguments.

\ea\label{DA:ex:John walked}
\ea John walked \{for/*in\} an hour.\hfill \textsc{atelic}\label{DA:ex:John walked-a}
\ex John walked to the pub \{*for/in\} an hour.\hfill \textsc{telic}\label{DA:ex:John walked-b}
\z\z

\noindent On the other hand, as shown by (\ref{DA:ex:tea_cooled}), English degree achievement \textit{cool} is ambiguous between the atelic interpretation (plausibly, the sentence would be true in such a~situation where some decrease of the temperature in the tea occurred) and the telic interpretation (the most probable scenarios verifying the telic reading would be such where the tea reached the room's temperature). Moreover, the ambiguity seems not to be related to change of the argument like in (\ref{DA:ex:John walked}).

\ea\label{DA:ex:tea_cooled}The tea cooled \{for/in\} one hour.
\z

\noindent Furthermore, what we refer to as \textsc{evaluativity} (following \cite{brasoveanu2018evaluativity}), is a property of (deadjectival) degree achievements whose corresponding adjectives instantiate a~degree above a~particular standard. In other words, a degree achievement is evaluative if it implies its base adjective in its positive form, as in (\ref{DA:ex:tea_inference_eval}); and non-evaluative, if the implication does not hold, as in (\ref{DA:ex:tea_inference_noneval}).\footnote{Other terms used in the literature have been \textit{positive} and \textit{telic} for \textsc{evaluative} readings; and \textit{comparative} and \textit{atelic} for \textsc{non-evaluative} readings. However, it is important to differentiate between telicity and evaluativity, hence the separate terminology in this paper.}

%An important note is due about the terminology used in this paper. Different linguists have used different terms to talk about the same phenomenon, which we refer to as \textsc{evaluativity}. Following \citet{brasoveanu2018evaluativity}, we use the evaluative criterion in the following manner: the particular degree achievement is evaluative if its corresponding adjective instantiates a~degree above a~particular standard. In other words, a degree achievement is evaluative if it implies its base adjective in its positive form, as in (\ref{DA:ex:tea_inference_eval}); and non-evaluative, if the implication does not hold, as in (\ref{DA:ex:tea_inference_noneval}). Other terms used in the literature have been \textsc{positive} and \textsc{telic} for \textsc{evaluative} readings; and \textsc{comparative} and \textsc{atelic} for \textsc{non-evaluative} readings. In English, the telic predicates indicated by the adverbial \textit{in an hour} usually correspond with the evaluative interpretation, and vice versa, the atelic predicates indicated by \textit{for an hour} are usually non-evaluative. However, as \sectref{DA:sec:Slavic degree achievements} will show, Slavic degree achievements can differentiate between the two notions on a more visible level.

\ea\label{DA:ex:tea_inference}
\ea\label{DA:ex:tea_inference_eval} The tea cooled in an hour. $\rightsquigarrow$ The tea is cool. \hfill \textsc{evaluative}
\ex\label{DA:ex:tea_inference_noneval} The tea cooled for an hour. {$\not\rightsquigarrow$} The tea is cool. \hfill \textsc{non-evaluative}
\z\z

\noindent In English, the telic predicates indicated by the adverbial \textit{in an hour} usually correspond with the evaluative interpretation, and vice versa, the atelic predicates indicated by \textit{for an hour} are usually non-evaluative, although this is not always the case. However, as \sectref{DA:sec:Slavic degree achievements} will show, Slavic degree achievements can differentiate between the two notions on a more visible level. Nevertheless, the situation in Slavic languages is complicated by the fact that in degree achievements two notions of maximalization coincide and also interact: degree maximalization (called evaluativity in our article) and event maximalization (as described by \citealt{krifka_manfred_thematic_1992,Filip2008} a.o.). More about it in \sectref{DA:sec:Slavic degree achievements}.

\subsection{Accounts of degree achievements}\label{DA:subsec:Accounts of degree achievements}

The pattern presented above lead some researchers (most notably \citealt{abusch_verbs_1986}) to claim that all degree achievements are ambiguous between the evaluative and non-evaluative reading. But this is empirically incorrect, as was noticed by other linguists soon thereafter. Consider first the upper-bounded degree achievements \textit{quieten}, \textit{darken} and \textit{ripen} in (\ref{DA:ex:quitened_darkened_ripened}) from \citet[ex. 36--38]{kearns_telic_2007}. If all degree achievements were ambiguous between the evaluative and non-evaluative interpretation, the non-evaluative interpretation should warrant the acceptability of the continuation \textit{but it wasn't A} ($=$~the base adjective). The usual conclusion drawn from data like this is that English upper-bounded degree achievements strongly prefer the evaluative reading (see \citealt{hay_scalar_1999,kennedy2008measure} a.o.) and that the ambiguity behaviour of English degree achievements is more an exception than a~rule.

\ea\label{DA:ex:quitened_darkened_ripened}
\ea The room quietened in a~few minutes \#but it wasn’t quiet.
\ex The sky darkened in an hour \#but it wasn’t dark.
\ex The fruit ripened in five days \#but it wasn’t ripe.
\z\z

\noindent The same point can be concluded from the lower-bounded degree achievements, since they seem to prefer the non-evaluative reading, which we illustrate with (\ref{DA:ex:wet_coca}) from \citet{davies2009}, where the most salient interpretation is that the hands are only partially wet. The general conclusion, then, seems to be that again, for lower-bounded degree achievements, the ambiguity treatment is empirically wrong.

\ea\label{DA:ex:wet_coca} Wet your hands with warm water and mix the dough with your hands.\\\hfill (COCA)
\z

\noindent Finally, turning to relative degree achievements, despite examples like (\ref{DA:ex:tea_cooled}), they seem to strongly incline to the non-evaluative interpretation, as shown by the examples from \citet[ex. 6]{kennedy2008measure} repeated here as (\ref{DA:ex:widened_deepened}). 

\ea\label{DA:ex:widened_deepened}
\ea The gap between the boats widened \{for/*in\} a~few minutes.
\ex The recession deepened \{for/*in\} several years.
\z\z

\noindent To conclude, the current default theory of degree achievements \citep{hay_scalar_1999,kennedy2008measure,kennedy_2012}, which is constructed in a~way that is naturally reflecting the reported English contrasts, could be succinctly summarised as follows: (i) relative degree achievements tend to be interpreted as non-evaluative; (ii) lower-bounded degree achievements are, by default, interpreted as non-eval\-u\-a\-tive; (iii) upper-bounded degree achievements receive mostly evaluative interpretations; (iv) closed-scale degree achievements lead, by default, to the evaluative interpretation.  This more or less summarizes the empirical landscape of English degree achievements but it is an open question how much the scalar theory is adequate for cross-linguistic data.\footnote{During the revision of our article there appeared a new work which very nicely covers degree achievements from the cross-linguistic picture, \citet{martinez_vera_degree_2021}. Thanks to one of the anonymous reviewers for pointing us the article. In a future work we would be more than happy to integrate our findings with \citet{martinez_vera_degree_2021} but since \citet{martinez_vera_degree_2021} and our work build upon slightly different theoretical assumptions, such an integration would be non-trivial and alas is beyond the scope of this article.}  


Let us now introduce the mechanics of the standard scalar approach. It is based on analysing an adjective as a~measure function of the type $\stb{e,d}$, returning the degree of an object on a~scale along the relevant dimension. The measure function is then type shifted to a~property of objects with the morphologically null element \textbf{pos} (first introduced by \citealt{Kennedy1997ComparisonAP}), which also supplies the contextual standard needed for the interpretation of relative adjectives.

Turning now to degree achievements, the scalar approach models them as a~measure of change function, as seen in (\ref{DA:ex:Measure_of_change}). It is built on top of the ``regular'' measure function and returns the degree of change on the appropriate scale that the particular object underwent during the event. The core of its meaning is a~difference function \textbf{m}$_\Delta$, which returns the difference between the degree at the initial and the final phase of the event ($_\Delta$-notation signals the difference function). Note also the difference scale \textit{m}$^\uparrow$, ranging from the standard of comparison to the correlate, which represents a~common ground between the analyses of degree achievements and comparative forms of adjectives. This means, that the difference scale \textit{wide}$^\uparrow$ in (\ref{DA:ex:wide}), for example, would be the common meaning core of \textit{widen} and a~comparative form \textit{wider than}: $^\uparrow$ measures object on a scale provided -- scale of width in (\ref{DA:ex:wide}). Finally, we follow \citet{henderson_quantizing_2013} in extending \citeposst{kennedy2008measure} notation, which allows the verbal measure function to access its arguments via theta-roles, as reflected in (\ref{DA:ex:Measure_of_change}) and (\ref{DA:ex:wide}) -- $\Theta$ are then substituted for the individual theta-roles in the particular sentences.

\ea\label{DA:ex:Measure_of_change} \emph{Measure of change}\\
For any measure function \textbf{m},
\(\mathbf{m}^\Theta_\Delta=\lambda e\big[\mathbf{m}_{m^\uparrow\big(\Theta(e)\big)\big(\mathit{init}(e)\big)}\big(\Theta(e)\big)\big(\mathit{fin}(e)\big)\big]\)
\z

\ea\label{DA:ex:wide} \(\lambda e\big[\textsc{wide}_{\textsc{wide}^\uparrow\big(\Theta(e)\big)\big(\mathit{init}(e)\big)}\big(\Theta(e)\big)\big(\mathit{fin}(e)\big)\big]\)
\z

\noindent The measure of change function in (\ref{DA:ex:wide}) is then type-shifted into a property of events, again via the morphologically null element \textbf{pos}. The application is exemplified by (\ref{DA:ex:shadow_lengthened}) for relative degree achievements, and (\ref{DA:ex:shirt_dried}) for absolute ones. The standard of comparison (\textbf{stnd}) is supplied on the basis of the Interpretive Economy principle in (\ref{DA:ex:interpretive_economy}) from \citet{kennedy2008measure}. In absolute degree achievements, maximising the contributions of the elements means using the lexicalised endpoint of the underlying scale as the standard of comparison, e.g., the maximum endpoint of the upper-bounded \textit{dry} in (\ref{DA:ex:shirt_dried}): the truth conditions then specify that there was an event $e$ and the agens of the event dries over the course of $e$ in a way which exceeds the standard for drying. On the other hand, the relative \textit{long} in (\ref{DA:ex:shadow_lengthened}) has an open scale without endpoints, so the \textbf{stnd} needs to be supplied via context and the event $e$ exceeds any contextually provided degree $d$.

\ea\label{DA:ex:shadow_lengthened} The shadow of the tree lengthened.\\ \sib{\REF{DA:ex:shadow_lengthened}}${}={}$\(\exists e\big[\textsc{long}^{ag}_{\Delta}(e) \geq \textbf{stnd}(\textsc{long}_\Delta) \wedge ag(e) = \sigma x.^*\textsc{shadow}(x)\big]\)
\z

\ea\label{DA:ex:shirt_dried} The shirt dried. \\ \sib{\REF{DA:ex:shirt_dried}}${}={}$\(\exists e\big[\textsc{dry}^{ag}_{\Delta}(e) \geq \textbf{stnd}(\textsc{dry}_\Delta) \wedge ag(e) =\sigma x.^*\textsc{shirt}(x)\big]\)
\z

\eanoraggedright\label{DA:ex:interpretive_economy}\emph{Interpretive Economy} \citep[ex. 18]{kennedy2008measure}\\
Maximize the contribution of the conventional meanings of the elements
of a~sentence to the computation of its truth conditions.
\z

\section{Degree achievements in Slavic\label{DA:sec:Slavic degree achievements}}

Let us now turn to the data in focus: the Slavic degree achievements. The first important observation comes from the morphosyntactic realisation of Slavic degree achievements that is different from English. The majority of Slavic degree achievements seems to be perfective, prefixed verbs. This is supported by the data obtained from the national corpora of Czech \citep{korpus}, Slovak \citep{korpusSK} and Russian \citep{korpusRU}. For each language, we elicited three representative degree achievements and three other (transitive, unergative, and unaccusative) verbs and compared the proportions of prefixed vs unprefixed tokens within them. We ran the Fisher's test and concluded from the results (Czech: $p < \num{2.2e-16},\allowbreak \text{OR} \approx 10.6$; Slovak: $p < \num{2.2e-16},\allowbreak \text{OR} \approx 9.5$; Russian: $p < \num{2.2e-16},\allowbreak \text{OR} \approx 10.9$) that throughout these Slavic languages, the degree achievements are approximately 10 times more probable to be prefixed than the other verb types. A~full account of Slavic degree achievements would, of course, have to integrate the grammatical aspect as well, and compare imperfective vs perfective degree achievements, but statistics like this provide a~good argument to start analysing Slavic degree achievements from perfective, prefixed verbs, as is the case of the current paper.

In the rest of the article, we focus on the prefixed Slavic degree achievements (for reasons mentioned above) but let us make some preliminary notes concerning the interaction of grammatical aspect with the scalar component of Slavic degree achievements. We acknowledge that such notes are nothing more than first steps in a full story which would integrate event and degree maximalization and that our notes cannot show appropriate respect to the enormous Slavic aspectual literature. But be it as it may, we follow \citet{Filip2008} in her treatment of imperfective degree achievements as non-maximal. And that seems to hold even if the degree achievements are derived from upper-bounded scales. Consider (\ref{DA:ex:pradlo_schlo}) with the imperfective degree achievement \textit{schnout} `to dry' with the lexical scale based on the upper-bounded scale of the adjective \textit{suchý} `dry'. In this case, the non-maximal (atelic) interpretation of the imperfective aspect leads to the non-evaluative interpretation of the degree achievement and since the same is true for secondary imperfective version of the same verb, it seems probable that the decisive factor for imperfective degree achievements is the grammatical aspect which can override the scalar information. If we would apply the evaluativity test introduced in \sectref{DA:subsec:Telicity and evaluativity behaviour of degree achievements}, it would yield the non-evaluativity (truth of (\ref{DA:ex:pradlo_schlo}) does not imply truth of the base adjective \textit{suchý} `dry' in the positive form). This is also the claim of \citet{Filip2008} which (we believe) points in the right direction for the Slavic imperfective degree achievements but of course calls for a proper empirical verification.

\ea\label{DA:ex:pradlo_schlo} 
\gll Prádlo schlo dvě hodiny. \\ 
laundry was.drying two hours \\
\glt ‘The laundry was drying for two hours.’ \hfill \textsc{atelic/non-evaluative}
\z 

\noindent Turning now to the perfective degree achievements data, let us compare the prototypical English example presented in (\ref{DA:ex:tea_cooled}) with its Czech counterpart in (\ref{DA:ex:caj_vychladl}). The different readings of the English degree achievement \textit{cool} would be unambiguously expressed -- depending on the particular prefix -- by the following Czech predicates: the prefix \textit{vy-} in (\ref{DA:ex:caj_vychladlA}) yields the evaluative reading, which would be true in a~situation where the tea reached, e.g., the room temperature or the temperature suitable for drinking. On the other hand, the prefix \textit{o-} in (\ref{DA:ex:caj_vychladlB}) distinctly signalises the non-evaluative reading, which would be verified by any decrease of the tea's temperature. Moreover, the native speakers of Czech would infer that a~Czech sentence corresponding to English \textit{The tea was cool} would follow only from (\ref{DA:ex:caj_vychladlA}), not (\ref{DA:ex:caj_vychladlB}).

\ea\label{DA:ex:caj_vychladl}\ea\label{DA:ex:caj_vychladlA} \gll Čaj vy-chladl za hodinu. \\ 
tea from-cooled in hour \\
\glt ‘The tea cooled completely in an hour.’ \hfill \textsc{evaluative}
\ex\label{DA:ex:caj_vychladlB} \gll Čaj o-chladl za hodinu. \\ 
tea around-cooled in hour \\
\glt ‘The tea cooled slightly in an hour.’ \hfill \textsc{non-evaluative}
\z\z 

\noindent Notice, however, that the adverbial test we used in (\ref{DA:ex:caj_vychladl}) classifies both sentences as telic, which corresponds with the fact that both prefixed verbs are perfective (here, we follow the standard approach to the relationship between the grammatical and the lexical aspect in Slavic languages, exemplified by \citealt{brecht1985form} a.o.). Despite the fact that both \textit{vy-chladl} `cooled completely' in (\ref{DA:ex:caj_vychladlA}) and \textit{o-chladl} `cooled slightly' in (\ref{DA:ex:caj_vychladlB}) are classified as telic, we can clearly see that in Czech (and generally in Slavic languages), the verbal morphology distinguishes the (non-)eval-\linebreak[4]uative interpretation according to the prefix that is used. Notice as well that the evaluativity classification fits nicely with the standard theory's \citep{kennedy2008measure,kennedy_2012} emphasis on the core of the adjectival meaning that unites degree achievements and their corresponding adjectives.

As discussed above, Slavic languages allow disambiguation of such degree achievements like English \textit{cool} via different prefixes. But even more importantly, in some cases, the prefixes can override their default interpretation. By way of example, the English upper-bounded degree achievement \textit{dry} is predicted by the standard theory to be evaluative by default. But the Czech perfective example in (\ref{DA:ex:drevo}) shows that depending on the nature of the prefix, the degree achievement \{\textit{o-}/\textit{vy-}\}\textit{schnout} `dry' can be interpreted either as non-evaluative in (\ref{DA:ex:drevo_oschlo}) or evaluative in (\ref{DA:ex:drevo_vyschlo}). This pattern is general: for all types of degree achievements, absolute or relative, we can construct both evaluative and non-evaluative versions by various prefixes.\footnote{Inspired by \citet{zwarts2005prepositional}, we categorise verbal prefixes according to their cumulativity (bounded\slash unbounded nature) into evaluative and non-evaluative (in his terminology, telic vs atelic, respectively).} So, next to the imperfective degree achievements, as in (\ref{DA:ex:pradlo_schlo}), non-evaluative perfective degrree achievements can be found too. The data and theory that aims at explaining this Slavic degree achievement pattern can be found in \citet{DocekalVlaskova2021}.

\ea\label{DA:ex:drevo}\ea\label{DA:ex:drevo_oschlo} \gll Dřevo o-schlo, ale pořád bylo většinou vlhké. \\
wood around-dried but still was mostly wet \\
\glt ‘The wood dried slightly, but it was still mostly wet.’ 
\ex\label{DA:ex:drevo_vyschlo} \gll Dřevo vy-schlo, \minsp{\#} ale pořád bylo většinou vlhké. \\
wood from-dried {} but was still mostly wet  \\
\glt ‘The wood dried completely, \#but it was still mostly wet.’
\z\z 

\noindent In this article, we focus on the empirical properties of Slavic degree achievements and test them experimentally, but let us note that, semantically, Slavic prefixation of degree achievements resembles the English degree modifiers like \textit{completely} or \textit{partially}. As \citet{kennedy2008measure} notice while discussing their example (29) repeated bellow as (\ref{DA:ex:basin_filled}), such degree modifiers can override the default interpretation of closed-scale degree achievements like \textit{fill}. The default interpretation is supported with the degree modifier \textit{completely} in (\ref{DA:ex:basin_filled-a}) but coerced to the non-evaluative interpretation with \textit{partially} in (\ref{DA:ex:basin_filled-b}). 

\ea\label{DA:ex:basin_filled}\ea\label{DA:ex:basin_filled-a} The basin filled completely in 10 minutes.
\ex\label{DA:ex:basin_filled-b} The basin filled partially ??in 10 minutes.
\z\z

\noindent In this respect, Slavic prefixes and English degree modifiers resemble each other semantically, but there are still some important differences: the first is the near obligatory presence of prefixes on Slavic verbs (as noted above); the second concerns the relative degree achievements. Consider (\ref{DA:ex:caj_vychladlA}) again: the degree achievement is constructed on the open scale, so how can we even attain the evaluative interpretation, when there is no clear scalar boundary to be reached? One reasonable way to understand this theoretically is to propose that at least some degree achievements are variable with respect to their scales and in case like (\ref{DA:ex:caj_vychladl}) they allow both relative and bounded scale. While in English the difference between the scales would be left for the context, Slavic languages can signal the nature of the scale morphologically. And because of that, Slavic relative degree achievements can get the evaluative interpretation with the right kind of prefixes. Again here we seem to be following \citet{Filip2008} when she claims that perfective degree achievements (at least with bounded scales) are always maximal in terms of the event structure and by default also evaluative, but their evaluativity can be contextually overridden. Our experimental research can be then understood as a~search for morphological clues determining the factors which \citet{Filip2008} claims to be contextual. 

To summarise this section: once we move beyond the territory of English degree achievements and focus on Slavic, we seem to see two sources of the \mbox{(non-)eval}\-uative interpretation: (i) the scalar lexical information inherited from the source adjectives; (ii) the degree modifiers and their contribution to the evaluative profile of the degree achievement. And this leads us to the research question behind our experiment, formulated in (\ref{DA:ex:research_question}). It is clear that both factors (nature of the scale and the prefixation type) play a~role, but only a~controlled experiment can give us hints about their relative strength.

\eanoraggedright\label{DA:ex:research_question} What are the factors of the evaluative interpretation in the case of Slavic degree achievements?   
\z

\section{Experiment}\label{DA:sec:Experiment}

In order to find out what the factors of the evaluative interpretation of Slavic degree achievements are, prefixes or adjectival scales, we conducted an experiment on Russian. We reformulated the research question above into three sub-questions in (\ref{DA:ex:research_question_3}).

\eanoraggedright\label{DA:ex:research_question_3}\eanoraggedright How much does the lexical semantics of Russian degree achievements influence their evaluativity?
\ex How much does the prefix of Russian degree achievements affect their evaluativity? 
\ex Which of the two factors is stronger (at least in terms of statistics)?
\z\z

\noindent The measuring of the experimental results can give us at least partial answers to the questions above. The most interesting question is the third one: such a~question is also not answerable by native speakers' intuition, that can otherwise give reasonable hints in case of the two previous sub-questions.

This section is structured as follows: we briefly describe the design of the experiment in \sectref{DA:subsec:design}, present its outcomes in \sectref{DA:subsec:results} and analyse the results in \sectref{DA:subsec:Discussion}. The experiment was carried out as a~part of a~Master's thesis of one of the authors. Therefore, the following section borrows from \citet{Onoeva2021thesis}. 

\subsection{Design}\label{DA:subsec:design}

The experiment was completed by 165 native speakers, but the data of three of them were excluded due to low reliability discovered via their filler ratings. The experiment was a~coherence acceptability task. The subjects evaluated how justified is a~reasoning from indirect speech containing a~degree achievement to a~sentence containing an adjective in a~positive form on a~Likert scale from 1 `completely unacceptable' to 5 `completely acceptable'. The design was $2\times 2$ with 4 conditions. Each participant saw 8 items and 8 fillers. A~total of 16 stimuli was randomised for each participant. L-Rex platform by \citet{lrex} was chosen for hosting.  

The degree achievements tested in the experiment are present in Table \ref{DA:tab:items}. The absolute/relative adjectival distinction was used, thus, we divided the degree achievements into two groups. Then, we found the evaluative and non-evaluative prefixes for each verb. Whether the prefixes contribute total or partial reading was decided based on the judgements of the author of the experiment who is a~native speaker of Russian. We were looking for the verbs which allow both types of prefixes, otherwise they were not suitable. 


\begin{table}
    \centering
    \begin{tabular}{lll}
         \lsptoprule
         adjective &  eval. DAs & non-eval. DAs \\\midrule
         \multicolumn{3}{l}{\textit{relative}} \\
         \textit{gorjačij} `hot' & \textit{razo-greť} & \textit{po-greť} \\
         \textit{nizkij} `low' & \textit{s-niziťsja} & \textit{po-niziťsja} \\
         \textit{bednyj} `poor' & \textit{o-bedneť} & \textit{po-bedneť} \\
         \textit{korotkij} `short' & \textit{u-korotiť} & \textit{pod-korotiť} \\
         \addlinespace
         \multicolumn{3}{l}{\textit{absolute}} \\
         \textit{suxoj} `dry' & \textit{vy-soxnuť} & \textit{pod-soxnuť} \\
         \textit{polnyj} `full' & \textit{na-polniť} & \textit{po-polniť} \\
         \textit{mokryj} `wet' & \textit{vy-močiť} & \textit{po-močiť} \\
         \textit{čistyj} `clean' & \textit{vy-čistiť} & \textit{po-čistiť} \\
         \lspbottomrule
    \end{tabular}
    \caption{The lists of the adjectives and DAs used in the experiment}
    \label{DA:tab:items}
\end{table}

The items always consisted of two sentences: the first one in indirect speech with a~degree achievement, \REF{DA:ex:item1a} and \REF{DA:ex:item2a}, the second one with its core adjective in a~positive form, \REF{DA:ex:item1b} and \REF{DA:ex:item2b}. As mentioned above, the absolute/relative adjectival distinction was used. In \REF{DA:ex:item1}, there is an example of the verb derived from an absolute adjective \textit{suchoj} `dry', while in \REF{DA:ex:item2}, \textit{gorjačij} `hot' is relative. %Then, for each verb, two prefixes were found coercing the degree achievements to either total or partial meaning, i.e., they were intuitively divided into evaluative and non-evaluative. In the given examples, 
When it comes to the prefixes, these are \textit{vy-} `out' and \textit{razo-} `from' contributing the total reading in the given examples, then \textit{pod-} `under' and \textit{po-} `along, on' providing only the partial one. 

%We were looking for the verbs which allow both types of prefixes, otherwise they were not suitable. 

\ea\label{DA:ex:item1} 
\ea\label{DA:ex:item1a} 
\gll Detektiv Smit s mesta prestuplenija soobščil svoemu kollege detektivu Džonsonu, čto rubaška na sušilke \{vy-soxla, pod-soxla\}. \\ 
Detective Smith from scene crime reported his colleague detective Johnson that shirt on drying-rack out-dried  under-dried {} \\
\glt `Detective Smith reported to his colleague detective Johnson from a~crime scene that a~shirt dried on a~drying rack.'
\ex\label{DA:ex:item1b} \gll Detektiv Džonson rešil, čto rubaška byla suxaja. \\
Detective Johnson concluded that shirt was dry \\
\glt `Detective Johnson concluded that the shirt was dry.'
\z \z

\ea\label{DA:ex:item2}
\ea\label{DA:ex:item2a}\gll Detektiv Smit s mesta prestuplenija soobščil svojemu kollege detektivu Džonsonu, čto ubityj prjamo pered smerťju \{razo-grel, po-grel\} edu. \\
Detective Smith from scene crime report his colleague detective Johnson that murdered just before death from-hot on-hot food \\
\glt `Detective Smith reported to his colleague detective Johnson from a~crime scene that the~murdered man warmed food right before his death.'
\ex\label{DA:ex:item2b} \gll Detektiv Džonson rešil, čto eda v moment prestuplenija byla gorjačaja. \\
Detective Johnson concluded that food in moment crime was hot \\
\glt `Detective Johnson concluded that food was warm at the time of the crime.'
\z \z

\noindent We tested whether the subjects interpret the meaning of a~particular degree achievement as evaluative (then the continuation with the positive form of an adjective should be acceptable for them) or as non-evaluative (in which case the continuation should be rejected). In other words, we used the evaluative criterion discussed above in form of a~coherence acceptability task. Generally, the expectation was that the speakers will accept the evaluative prefix with the absolute degree achievements better than other types of degree achievements. 

We used the same structure with a~verb in the first sentence and a corresponding adjective or past participle in the second for the fillers. They were also divided into two sets: good (4--5 ratings expected) and bad (1--2 ratings expected). The verbs in the good fillers were always perfective, e.g., \textit{postroiť} `to built' or \textit{vypiť} `to drink out', therefore, the participants could conclude that the second sentence was completely acceptable, while in the bad set, all the verbs were imperfective, e.g., \textit{čitať} `to read' or \textit{pisať} `to write', so they should be unacceptable in the given contexts. 

\subsection{Results}\label{DA:subsec:results}

It was expected that the degree achievements with the evaluative prefixes should be accepted more, as they denote the finite state reading which should be equal to the meaning of the corresponding adjectives in their positive form. However, from the~descriptive statistics of the experiment presented in \tabref{DA:tab:meanmedian} and \figref{DA:fig:SEgraph}, it follows that this was not always the case. 

\begin{table}
\caption{Measures of central tendency}
\label{DA:tab:meanmedian}
 \begin{tabular}{l ccc}
  \lsptoprule
  item                       & mean & median & variation \\ 
  \midrule
  absolute + non-evaluative  & 3.01 & 3 & 1.61\\
  absolute + evaluative      & 3.96 & 4 & 1.36\\
  relative + non-evaluative  & 2.80 & 3 & 1.69\\
  relative + evaluative      & 2.90 & 3 & 1.93\\
  \lspbottomrule
 \end{tabular}
\end{table}

\begin{figure}[p]
    \includegraphics[width=\textwidth]{figures/SEgraph.eps}
    \caption{Standard error graph}
    \label{DA:fig:SEgraph}
\end{figure}

\begin{figure}[p]
    \includegraphics[width=\textwidth]{figures/new-allDAs.eps}
    \caption{Box plot graph for each degree achievement}
    \label{DA:fig:allitems}
\end{figure}

The degree achievements derived from the absolute adjectives with the evaluative prefixes were accepted better in comparison with the non-evaluative ones, whereas there is no big difference in acceptability of the relative degree achievements. With the aim of checking what happened inside the classes and to get a~detailed view, we also looked at each item separately, see \figref{DA:fig:allitems}.

The absolute degree achievements (left facet) fall under the expected pattern: the verbs with the non-evaluative prefixes have lower acceptability rates than the verbs with the evaluative ones, which are favoured in general. Nevertheless, the non-evaluative variants of \textit{čistyj} `clean' and \textit{mokryj} `wet' climbed higher than the other two and have the same medians as their evaluative counterparts. 

When it comes to the relative class (right facet), it is clear that \textit{gorjačij} `hot' was placed on top of it. Even though its evaluative variant \textit{razogreť} `heat up' was definitely liked better, non-evaluative \textit{pogreť} `heat up' also has the median rating~4. The degree achievements based on \textit{nizkij} `low' and \textit{bednyj} `poor' correspond to the expected pattern, but their acceptability was lower in general. A~curious thing happened to \textit{korotkij} `short': both verbs were rated relatively low, but according to the mean ratings, what we considered to be the non-evaluative variant \textit{podkorotiť} `shorten', was slightly better accepted than evaluative \textit{ukorotiť} `shorten'. 

We analysed the data in a~mixed-effects linear model with subject and item intercept+slope random effects via the \textsc{lme4} package \citep{bates2015} in R \citep{r-core}. The explanatory variables were conditions \textsc{DAClass} (values: relative, absolute), \textsc{prefix} (values: evaluative, non-evaluative) and their interaction. The dependent variable was the subject's rating. The reference levels were absolute and non-evaluative for the conditions \textsc{DAClass} and \textsc{prefix}, respectively. 

The strongest effect recorded was a positive effect of the evaluative prefixes: $t = 11.437$, $p < 0.001$. Next, we found a~negative effect of the relative degree achievement class ($t = -2.318$, $p <0.05$) and a~negative interaction of the relative degree achievement class by the evaluative prefixes: $t = -6.652$, $p < 0.001$. The coefficients are reported in \tabref{DA:tab:model}.

\begin{table}
\caption{Linear mixed model}
\label{DA:tab:model}
 \begin{tabular}{l S[table-format=-1.5] S[table-format=1.5] S[table-format=-1.3] S[table-format=<1.3]}
  \lsptoprule
   & {Est.} &  {SE} & {$t$} & {$p$}\\
   \midrule
  (Intercept)                          & 2.96587  & 0.21794 & 13.609 & <0.001  \\
  \textsc{DAClassrelative}             & -0.20988 & 0.09056 & -2.318 & 0.02 \\
  \textsc{prefixeval}                  & 1.04047  & 0.09097 & 11.437 & <0.001 \\
  \textsc{DAClassrelative:prefixteval} & -0.85185 & 0.12807 & -6.652 & <0.001 \\
  \lspbottomrule
 \end{tabular}
\end{table}

\subsection{Discussion}\label{DA:subsec:Discussion}

Now we can answer the research questions, for convenience repeated in (\ref{DA:ex:research_question_3_2}):

\eanoraggedright\label{DA:ex:research_question_3_2}\eanoraggedright How much does the lexical semantics of Russian degree achievements influence their evaluativity?
\ex How much does the prefix of Russian degree achievements affect their evaluativity? 
\ex Which of the two factors is stronger?
\z\z

\noindent The descriptive statistics and the model give some answers to both first and second question. Firstly, the negative effect of the relative degree achievement (\textsc{DAClassrelative}) class shows that in Russian, the lexical semantics of the degree achievements clearly affect their non-evaluative interpretation. The subjects judged the inference to the positive form of the corresponding adjective as less acceptable in items with relative degree achievements (which is already predicted by the standard theory). 

Secondly: the strongest effect (the positive effect of the evaluative prefix: \textsc{prefixeval}) seems to show that at least in the material we tested the nature of the prefix was a stronger factor than the nature of the scale (see also \cite{DocekalVlaskova2021}). But of course it is not straighforward to translate strength of the statistic effects into the linguistic theory, so we do not want to jump to too hasty a conclusion. Nevertheless, the most intriguing is the last question: simply comparing the strength of the main effects indicates that prefixation (at least for the verbs we tested) is the more important factor. But the interaction between the two factors also shows that the picture is not that clear: the negative interaction seems to be a~reflex of the observed pattern in judgements -- the evaluative prefix (which improves the acceptance with absolute degree achievements) plays a~significantly smaller role in the case of relative degree achievements. 

Why do the speakers have problems accessing the evaluative interpretation with relative degree achievements is a~very important question, and the standard theory gives an answer: it is because relative degree achievements do not have scalar boundaries. But the answer faces some difficulties when we look at the absolute degree achievements where the prefix clearly plays the most important role and overrides the lexical information. Theoretical conclusions which can be drawn from the results of our experiment are divergent. One possibility would be to claim that some degree achievements are able to be linked with both relative and bounded scales and the nature of the prefix then determines the scale: if the degree achievement is prefixed with a non-evaluative prefix and it can be associated with both relative and upper-bounded or closed scale, the degree achievement would choose the relative scale (and the reverse pattern for the evaluative prefix).\footnote{Thanks to the one of our anonymous reviewers for pointing out the importance of this possibility.} But there is still the interaction effect which (simply put) tells us that it is easier (for subjects) to un-maximize the absolute degree achievements via some non-evaluative prefix but the reverse strategy, to maximize relative degree achievements, is much harder. At this stage of work we simply report this asymmetry and offer some ideas above, but a real theoretical description of what is going on is left for a future work.

\section{Summary}\label{DA:sec:Summary}

In our article, we summarised the Slavic degree achievements data, which pose an empirical problem for the standard theory. More importantly, we reported the results of our experiment, which basically gives us some preliminary answers to the research questions in (\ref{DA:ex:research_question_3})/(\ref{DA:ex:research_question_3_2}). Namely, the evaluative profile of Slavic degree achievements is related both to the lexical semantics (the nature of the scale, as predicted by the standard theory) and to the prefixes which modify the degree achievements. The nature of the prefix is, as it appears from the experiment, the more important factor, at least for the the absolute degree achievements. For the relative degree achievements the effect is palpable, too, but its impact is smaller. 

But of course, as usually in the problem solving cycle, the answers we got from the experiment just mean starting another cycle of research questions, experiments and their analysis. Let us list some of the open questions which naturally appear: (i) Why do absolute and relative degree achievements show different sensitivity to prefixes? (ii) Is there some semantic (or other) criterion that distinguishes the evaluative prefixes from the non-evaluative ones? (iii) Why are some degree achievements perfectly fine without any prefix attached, while the others require it to be felicitous?

One possible answer to the first open question is the following: the relative degree achievements allow the evaluative interpretation (signalled via prefixation) only if they allow scalar variability as suggested above. This hypothesis can be tested in an experiment measuring both scalar variability of a~particular relative degree achievement and its openness for evaluative prefixation. The second question is more theoretic in nature and some possible answers to it are given in \citet{DocekalVlaskova2021}, but see also \citet{Filip2008} or \citet{martinez_vera_degree_2021} for a more general perspective; again, the differing theoretical routes are good candidates for experimental testing. The third question is a~more general one without a~clear answer, but a possible route here would be experimentally targeting Slavic imperfective degree achievements and their evaluativity behaviour. In the end, it seems that we ended up with more open questions than we started with, but that is (hopefully) a~promise for a~fruitful future work. 

\section*{Acknowledgements}
The authors would like to thank the organisers and the audience of the 14th European Conference on Formal Description of Slavic Languages (FDSL 14) in Leipzig, Germany at the Slavic Department of Leipzig University for the opportunity to present the research and a~stimulating debate. 

We would also like to thank all the Russian speakers who took the time and participated in the experiment. 

\printbibliography[heading=subbibliography,notkeyword=this]

\end{document}
