\chapter{Results: metadiscursive activity in nineteenth-century U.S. newspapers}
\label{bkm:Ref523404731}\hypertarget{Toc63021232}{}\section{Metadiscourses on phonological forms}
\label{bkm:Ref13472461}\hypertarget{Toc63021233}{}\subsection{Frequency and temporal and regional distribution of newspaper articles}
\label{bkm:Ref6214186}\hypertarget{Toc63021234}{}
The primary aim of the quantitative analysis at this point is to determine the extent to which the linguistic forms which are the focus of the present study are part of metadiscursive activities in newspaper articles. As a first step, I therefore determined the overall frequency of occurrence of articles containing the search terms whose spelling represents a particular phonological form. The result of this analysis, which is shown in \tabref{tab:key:9}, shows that all search terms appear in a number of articles, with \emph{dawnce} occurring least often (in 73 articles) and \emph{bettah} occurring most often (in 374 articles). This already indicates that the pronunciation respellings chosen for this analysis were not just the product of one or two random choices by a few individuals, but they had become reliable indicators of the phonological forms and were used by writers because they expected their readers to recognize which pronunciation they aimed to represent. As can be seen in \tabref{tab:key:9}, I not only identified the number of articles containing the search terms (token frequency), but I also counted how many different articles containing each search term appeared (type frequency) because of the common practice to reprint the same article in different newspapers.\footnote{If the same article published in the same newspaper appeared in both databases (because the two databases overlap to a small extent), I considered it a duplicate and counted it as only one token.}  Measuring these frequencies provides insights into two important questions: the first one asking how widely the search terms circulated in discourses on language in newspapers and the second one asking the more difficult question of which inferences can be drawn with regard to the salience of the phonological forms represented by the search terms in metadiscursive activities. The first question can be answered by using the token frequencies (and if not indicated otherwise, all analyses conducted in the present study will be based on the token frequencies). \tabref{tab:key:9} shows that they vary considerably, with \emph{bettah} occurring in over five times more articles than \emph{dawnce}. This suggests that \emph{bettah} and more generally the representation of non-rhoticity had a wider circulation than that of the other search terms and phonological forms. Articles containing \emph{deah} AND \emph{fellah} as well as \emph{noospaper}/\emph{s} appear less than half as often as articles containing \emph{bettah}, and those containing \emph{hinglish} and \textsc{twousers} appear less than a third as often. While it is possible to argue that a higher circulation also contributes to a higher salience of the pronunciation respellings and the phonological forms that they represent, other factors have to be considered as well to answer the second question.


The first factor that obviously needs to be taken into consideration is the frequency of the lexical item that is used to represent the phonological form by means of a non-standard spelling. If a lexical item is generally more frequent (in standard spelling), it can be expected to occur more frequently in non-standard spelling as well, which means that a higher number of articles containing the search term could also be due to the higher frequency of the lexical item and not due to a higher salience of the phonological form to be represented. Consequently, I have extracted the number of articles containing the lexical item in standard spelling (ASS) and calculated the respective ratios to articles containing the lexical item in non-standard spelling (ANSS), which are shown in \tabref{tab:key:10}. When comparing the ratio of \textsc{twousers} to \emph{trousers} with the ratio of \emph{hinglish} to \emph{English}, for example, it is striking that even though the absolute frequency of ANSSs is almost the same, the ratio of ANSSs to ASSs is considerably different, suggesting that the lexical item \emph{trousers} is much more often respelled to suggest an alternative pronunciation than the item \emph{English}. The ratio of \emph{deah} AND \emph{fellah} to \emph{dear} AND \emph{fellow} is the lowest of all search terms, while the ratio of \emph{bettah} to \emph{better} is the second highest. Therefore, even though both forms indicate a non-rhotic pronunciation, this pronunciation is represented much more often in \emph{dear} AND \emph{fellow} than in \emph{better}. This finding is important in two respects: First, it indicates that the token frequencies of ANSSs are influenced by the lexical items and it is therefore hardly possible to deduce which of the phonological forms is more salient in metadiscursive activities based on the limited number of search terms. (Note that salience at this point is defined as having the quality of being highly frequent.) It is, for example, possible that if a more frequent lexical item containing the \textsc{bath}{}-vowel had been chosen for the analysis, the search would have yielded a significantly higher number of articles. Nevertheless, as stated above, the frequencies show that the pronunciation respellings drawing attention to phonological forms were not isolated cases but occurred consistently enough to constitute a regular pattern which is worth further investigation. Secondly, the ratios in \tabref{tab:key:10} serve as an indicator of how closely the lexical item is linked to the phonological form(s) represented by the spelling. Most notably, \emph{trousers} and the combination of \emph{dear} and \emph{fellow} are relatively often represented as being pronounced without the post-vocalic /r/ and, additionally or alternatively, with a labiodental approximant in the case of \emph{trousers}. This suggests that in these cases the indexical links are not just found on the phonological but also on the lexical level. A close connection between a phonological form and a lexical item can contribute to the salience of both forms, lexical and phonological, in other ways than by sheer frequency, as I will show in the second part of the analysis in \sectref{bkm:Ref534386498}.

The second factor that can be considered when assessing the salience of the phonological forms in metadiscursive activities in quantitative terms is the relation between the token and the type frequency of the articles. A low TTR-value indicates that the diversity of articles containing a search term is lower, which in turn means that there is a higher number of articles which were reprinted one or more times. \tabref{tab:key:9} shows that the search terms clearly fall into three groups with \emph{hinglish}, \emph{noospaper}/\emph{s} and \emph{dawnce} having a TTR-value of 71-73\%, \emph{deah} AND \emph{fellah} and \emph{bettah} having the same TTR-value of 66\% and lastly \textsc{twousers} having the lowest TTR-value of 53\%.

\begin{table}
\begin{tabularx}{\textwidth}{Qrrr}
\lsptoprule
Search term(s) & \parbox{3cm}{Number of articles in both databases (tokens)}
                    &  \parbox{3cm}{Number of articles in both databases (types)}
                        & \parbox{2cm}{~\newline Type-Token\newline Ratio (TTR)}\\
\midrule
hinglish &  107 &  78 &  73\%\\
noospaper/s &  162 &  117 &  72\%\\
dawnce &  73 &  52 &  71\%\\
\textsc{twousers} &  103 &  55 &  53\%\\
deah AND fellah &  157 &  104 &  66\%\\
bettah &  374 &  246 &  66\%\\
\lspbottomrule
\end{tabularx}
\caption{
Number of articles containing the search terms in both databases (tokens and types).
}
\label{tab:key:9}
\end{table}

How can these values be interpreted with regard to the salience of the phonological forms? It can be argued that a low TTR-value indicates that there is a higher number of popular articles. That is, articles which editors found so appealing that they decided to reprint them in their newspapers. This popularity can have several causes, of course, but since the pronunciation respellings are quite unusual, it is plausible that they at least affected the popularity of the articles in some way. It is striking that both search terms representing non-rhoticity have the exact same TTR-value even though the search term \emph{deah} AND \emph{fellah} appears in less than half the number of articles than \emph{bettah}, which can be taken as a confirmation that the TTR-value can serve as an indicator of the popularity of the phonological form and not just a specific search term (or something else entirely). If a search term appears in a high number of popular articles which have been reprinted several times, this could be interpreted as contributing to the salience of the phonological form – editors notice it and find that the article with the pronunciation respelling might appeal to the newspaper’s readers. It is, however, impossible to determine what exactly the editors found appealing, which again underlines the necessity of qualitative analyses to provide a full picture.

\begin{table}
\fittable{
\begin{tabular}{l@{}r}
\lsptoprule
Search term(s) & Ratio of ANSSs to ASSs (number of articles)\\
\midrule
deah AND fellah : dear AND fellow &  1:925\\
\textsc{twousers} \textsc{:} trousers &  1:1010\\
dawnce : dance &  1:6163\\
noospaper/s : newspaper/s &  1:9820\\
bettah : better &  1:10872\\
hinglish : English &  1:27667\\
\lspbottomrule
\end{tabular}
}
\caption{
Ratio of number of articles containing the search term(s) written in non-standard spelling (ANSSs) and in standard spelling (ASSs).
}
\label{tab:key:10}
\end{table}

Before turning to the qualitative part of the analysis, two important questions remain: When did articles containing the search terms first appear and how did the number of articles change over time? Even though the absolute token frequencies of the articles cannot be compared, it is possible to compare the temporal development of articles containing the search terms over time. \figref{fig:key:13} shows the frequency of articles (tokens) per million and decade for each search term and a clear pattern emerges. On the one hand, articles containing \emph{hinglish} and \emph{noospaper}/\emph{s} appear already early in the century and then keep appearing at a fairly steady rate. There are two obvious exceptions to this steadiness, which are the 1830s for \emph{hinglish} and the 1860s for \emph{noospaper}/\emph{s}, which will need to be investigated in the qualitative part. On the other hand, articles containing \emph{bettah}, \emph{deah} AND \emph{fellah}, \textsc{twousers} and \emph{dawnce} appear earliest in the 1840s and only at low frequencies until their occurrence rises dramatically in the 1880s. Most notably \emph{bettah} and \emph{deah} AND \emph{fellah} show a very similar development, which indicates that it is indeed the phonological forms and not the lexical items which draw more and more attention in metadiscursive activity. Articles containing \textsc{twousers} and \emph{dawnce} appear hardly at all before the 1880s. Based on the selected search terms, it therefore seems as if /h/-dropping and -insertion and yod-dropping played a role in metadiscursive activity throughout the nineteenth century, while the other phonological forms came into focus largely in the last decades of the nineteenth century.


As a means to ensure the reliability of the temporal development found in both databases combined, I compared the development in both databases separately. \figref{fig:key:14} shows the number of articles in the database AHN and \figref{fig:key:15} in the database NCNP. It is striking that the same general trends can also be observed in both databases separately, which indicates that the findings are fairly robust and that the databases are indeed comparable. Moreover, for three of the phonological forms I carried out additional searches in one of the databases (NCNP) to confirm that the observed patterns are related primarily to the phonological forms and not to the lexical items chosen to represent them. \figref{fig:key:16} shows the development of articles containing \emph{Hingland}, \emph{cawnt} and \emph{weally} (‘really’, the <w> signaling the realization of /r/ as a labiodental approximant) to the ones containing \emph{hinglish}, \emph{dawnce} and \textsc{twousers}.


\begin{figure}
\includegraphics[width=\textwidth]{figures/Paulsen-img13.pdf}
\caption{
The number of articles containing the search term(s) in the databases AHN and NCNP per million and decade
}
\label{fig:key:13}
\end{figure}


\begin{figure}
\includegraphics[width=\textwidth]{figures/Paulsen-img14.pdf}
\caption{
The number of articles containing the search term(s) in the database AHN per million and decade
}
\label{fig:key:14}
\end{figure}


\begin{figure}
\includegraphics[width=\textwidth]{figures/Paulsen-img15.pdf}
\caption{
The number of articles containing the search term(s) in the database NCNP per million and decade
}
\label{fig:key:15}
\end{figure}


\begin{figure}
\includegraphics[width=\textwidth]{figures/Paulsen-img16.pdf}
\caption{
The number of articles containing the search term(s) in the database NCNP per million and decade
}
\label{fig:key:16}
\end{figure}


How does the temporal pattern established above relate to the regional distribution of articles containing the search terms? \figref{fig:key:17a} and \figref{fig:key:17b} display how often the articles appeared in each state – the darker the color, the higher the number of articles in a particular state. The first general result is that none of the search terms is restricted to articles published only in one state or a specific region, but they are all more or less distributed across the whole country. A second observation is that the two search terms which appeared in articles throughout most of the century, \emph{hinglish} and \emph{noospaper/s}, are also similar with respect to the regional distribution. This might not be apparent at first sight because of the high number of articles containing \emph{noospaper/s} published in Wisconsin (21 articles), but for both search terms there is a relatively high number of articles published in the northeast of the country. For \emph{noospaper/s}, the second- and third\--highest number of articles appeared in Massachusetts (12) and New Hampshire (11). Maine (7) and New York (7) also occupy prominent positions on the map. For \emph{hinglish}, the four states with the highest number of articles are Massachusetts (10), Pennsylvania (10), Connecticut (8) and New York (8). In contrast, the other four search terms appear in a relatively small number of articles in the northeast. They are mostly found in articles published in the midwest, west and north of the country, with Colorado and Kansas occupying particularly prominent positions. There is only one eastern state with a relatively high number of articles containing all of the four search terms: Pennsylvania.


There are some dissimilarities between the four search terms as well. Wisconsin stands out among other states for having a high number of articles containing \emph{deah} AND \emph{fellah}, \emph{bettah} and \textsc{twousers} appeared, but the number of articles containing \emph{dawnce} is not particularly high. The high number of articles in Wisconsin is actually something that the three search terms have in common with the search term \emph{noospaper/s}. The search terms representing non-rhoticity also show a very similar distribution. They contrast with the other search terms in being part of a relatively high number of articles published in Missouri (24 articles containing \emph{bettah} and nine articles containing \emph{deah} AND \emph{fellah}). They differ, however, in that \emph{deah} AND \emph{fellah} occurs relatively often in the north (13 articles in North Dakota, 10 articles in Minnesota). The only other term with a relatively high number of articles in North Dakota is \emph{dawnce}. Regarding the southeastern region, another striking difference is that many articles containing \emph{deah} AND \emph{fellah} appear in Georgia (6), which is noticeable because \emph{noospaper}/\emph{s} is the only other search term with a relatively high number of articles in Georgia (8). \emph{Bettah}, on the other hand, appears relatively often in Massachusetts and is similar to both \emph{hinglish} and \emph{noospaper/s} in that respect. Finally, Texas and California also appear in the list of the ten states with the highest number of articles containing the search terms \emph{deah} AND \emph{fellah}, \emph{bettah}, \emph{dawnce} and \emph{hinglish}.


\begin{figure}
\includegraphics[width=0.65\textwidth]{figures/Paulsen-img17a.pdf}
\caption{
Number of articles containing \textit{hinglish}, \textit{twousers} or \textit{deah} AND \textit{fellah} in both databases (AHN and NCNP) per state
}
\label{fig:key:17a}
\end{figure}

\begin{figure}
\includegraphics[width=0.65\textwidth]{figures/Paulsen-img17b.pdf}
\caption{
Number of articles containing \textit{noospaper/s}, \textit{dawnce} or \textit{bettah} in both databases (AHN and NCNP) per state
}
\label{fig:key:17b}
\end{figure}
Although the regional distribution patterns described above are clearly based on a fairly low number of articles, it is still striking that the differences between the search terms with regard to the development over time are reflected in their distribution across the United States and that the distribution of the search terms representing non-rhoticity is fairly similar. If the distribution was just a reflection of the overall number of articles published in the states, most articles would have to appear in the northeast, which is clearly not the case for all search terms. This suggests that it is likely that the linguistic forms in general circulate more widely in metadiscourses in the respective areas and are highly salient based on their quantity: /h/-dropping and -insertion as well as yod-dropping seem to be part of metadiscursive activity in the northeast more often than in the rest of the country, while the other phonological forms are more often involved in metadiscourses in the northern and (mid)western parts of the United States. The lower the number of articles containing particular search terms in a state, the more difficult it is to ascertain whether they can be taken as an indicator of the extent to which the phonological form is part of metadiscursive activity in that state. Again, it is only in combination with a qualitative analysis that further light can be shed on the role of region in the enregisterment of these forms. The following section will therefore address the second research question asking which social values and social personae (characterological figures) are indexically linked to the different linguistic forms and the third research question asking how these indexical links are created and how the strategies employed contribute to the salience of the form–meaning links. Section \ref{bkm:Ref10890365} will then combine the insights of the quantitative and qualitative analyses to provide a comprehensive picture of how the enregisterment of the phonological forms proceeded.


\subsection{Indexical values and social personae: a qualitative analysis}
\label{bkm:Ref534386498}\hypertarget{Toc63021235}{}\label{bkm:Ref7775416}
The main aim of the qualitative analyses in this section is to identify how indexical links between the phonological forms and social values as well as social personae are created. This requires an analysis of the strategies which are employed to typify the linguistic form under investigation and other linguistic and non-linguistic signs by making them point to aspects of the context in which the forms are used. The articles I selected for the following qualitative analyses show the most common indexical links and illustrate different strategies which were used to establish these links. To find out whether the links are stable or changing, the articles were also taken from different points in time.


\subsubsection{\emph{hinglish}}
\hypertarget{Toc63021236}{}
In the databases, the first article containing the search term \emph{hinglish} was published in 1825 in three different newspapers: the \emph{Middlesex Gazette} (Middletown, Connecticut) on July 20, 1825\citesource{July201825}, the \emph{Boston Commercial Gazette} (Boston, Massachusetts) on July 21, 1825\citesource{July211825} and the \emph{Arkansas Weekly Gazette} (Little Rock, Ar\-kan\-sas) on September 20, 1825\citesource{September201825}. As indicated at the end of the article, the newspapers took the article from the \emph{Georgetown Metropolitan}, a newspaper published in Washington, D.C.


%%please move the includegraphics inside the {figure} environment
%%\includegraphics[width=\textwidth]{figures/Paulsen-img018.png}


\begin{figure}
\includegraphics[width=8cm]{figures/Paulsen-img18.png}
\caption{
An article representing the speech of an Englishman, published in the \emph{Middlesex Gazette} (Middletown, Connecticut) on July 20, 1825\citesource{July201825}, retrieved from America's Historical Newspapers
}
\label{Figure18}
\end{figure}


The article is a type of text which occurs very frequently in my sample of articles containing pronunciation respellings. I classify such texts as anecdotes because they contain typical characteristics of that genre. According to \citet[22]{Bauman2005}, an anecdote is “a short, humorous narrative, purporting to recount a true incident involving real people”. Furthermore, there is usually a “focus on a single scene and a tendency to limit attention to two actors” who engage with each other, so anecdotes “tend to be heavily dialogic in construction” \citep[22]{Bauman2005}. In the anecdote above, the truth-claim is established by a first-person narrator who claims that he or she has actually heard an Englishman complaining about Americans. Anecdotes usually focus on actors who can be seen as “representative […] regarding named individuals or groups” \citep[73]{Nicolaisen2011}, so the reader is invited to regard the opinions and behaviors of the Englishman described as being representative of the opinions and behaviors of English people in general. This particular example does not entail any dialogue, but it is an account of reported speech. The American reproduces the voice of the Englishman for the greatest part of the anecdote, thereby providing a representation of the Englishman’s accent. The last line of the text is not part of the anecdote. The introduction \emph{Quere} sets it apart from the main text and indicates that the following question is a reaction to the anecdote on the part of the editor publishing the article.


The first line of the text establishes that the main focus of the anecdote is language, particularly the language used by the Englishman, which the narrator labels as “a specimen of \emph{pure English}”, and the relationship between the English spoken by Englishmen and the English spoken by Americans. The value distinctions between ‘good’ and ‘bad’ as well as ‘pure’ and ‘impure’ play an important role, which is underlined typographically by italicizing the phrase \emph{pure English}. The evaluation of the Englishman’s speech as ‘pure’ alludes to discourses in England, where, as \citet{Mugglestone2003} shows, purity comes to be associated with speech that does not contain any regionally restricted forms (so-called \textit{provincialisms}). As early as 1687, Cooper described “the best dialect” of London speech as “the most pure and correct” (cited in \citealt[15]{Mugglestone2003}), which illustrates that links between a repertoire of forms and a place (London) become backgrounded in favor of links between the forms and social values like ‘purity’ and ‘correctness’, with the consequence that any use of forms which are associated with different places evokes the lesser values ‘impure’ and ‘incorrect’. Labeling the Englishman’s speech as ‘pure’ in an American context creates two effects: It emphasizes that the speech is purely English, without any American influence, and it also highlights the belief that a form of English speech is associated with correctness and purity, which implies that using any American forms would index impurity and incorrectness. The description of the Englishman’s attitude and behavior constructs him as a character who positions himself as linguistically superior: He “was abusing the Americans for many things, and among the rest for speaking such bad \emph{Hinglish} that he could not \emph{hunderstand} them”. By setting \emph{Hinglish} and \emph{hunderstand} apart from the rest of the sentence through italics, the author distinguishes the voice of the narrator and the voice of the Englishman. In doing so, he creates an indexical link between /h/-insertion and the Englishman – a link which is reinforced to a great extent by the following mix of direct and indirect representation of the Englishman’s speech because every word which would normally be spelled with an initial <h> is spelled without it and vice versa, which creates a picture of a person consistently dropping and inserting /h/ at the beginning of words. The sheer frequency of words representing /h/-dropping and -insertion, the highlighting of these words by means of italics and the almost complete absence of any other markedly differential linguistic forms make the phonological form and its link to Englishness (in contrast to Americanness) extremely salient. Establishing this link serves an important function: It reverses the relationship between English and American speech. The dropping and insertion of /h/ by the Englishman is clearly marked as deviant and non-standard through the spelling, indicating that the phonological form is ‘incorrect’. This makes it clear that the aim of the anecdote is to counter the prevailing ideology of English speech being superior to American speech by providing an example of English speech which is incorrect and not pure. The contrast between the label “pure English” and the Englishman’s actual speech, marked by the ‘incorrect’ insertion and dropping of /h/, is a source of irony and humor. The Englishman’s arrogance is expressed through his criticism not only of the language used in America but also of the climate (“it is too \emph{ot} in \emph{zummer}”), of the activities (“I \emph{ave} to \emph{habstain} from \emph{hall} the \emph{hinjoyments} of \emph{hexercise} on \emph{orseback} or \emph{hotherwise}”) and of the food (“\emph{Hindian} bread” and “\emph{am} and \emph{heggs} [...] with \emph{zumtimes} a fried \emph{en} instead of a chicken”). It culminates in the ridiculous claim that the “\emph{Hamerican orses}, \emph{hoxen} and \emph{ogs} are not equal to the \emph{Hinglish}”. The claims of superiority therefore extend beyond language, but it is primarily based on linguistic grounds that these claims are refuted and ridiculed. What is important is that the anecdote does not question the model itself, but it questions the ability of Englishmen to conform to this model while highlighting the Americans’ ability to do so.

The question “Was e heducated hat Hoxford?” in the last line adds to the ridicule. As the voice is that of an American asking about the Englishman’s education, the use of /h/-dropping and -insertion is marked as mockery, which underlines the irony of the question: It is hardly imaginable \-that an Englishman deviating from the model of pure speech through the insertion and dropping of /h/ could have attended this prestigious English university and received a good education. This creates an additional indexical link between the phonological form and a lack of education and, by implication, emphasizes the educatedness of Americans. Overall, the strength of the anecdote is that it creates an anecdotal exemplar of an incident – an incident that is to be understood as representative. This impression of representativeness is created through the contrast between the truth-claim and the unrealistic and exaggerated nature of the incident: It is very unlikely for so many words beginning with /h/ or a vowel to co-occur in such a short text. Nevertheless, the reader is called to believe that the general nature of the incident is true, namely that English people think of themselves and of their speech as superior and that they look down on Americans, on their speech and way of life, while they are “in fact” inferior.

The anecdote is a very common text type in the collection of articles containing \emph{hinglish} and anecdotes about similar incidents to the one described above appear repeatedly. An anecdote published on September 14, 1830\citesource{September141830}, in the \emph{Baltimore Gazette and Daily Advertiser} (Maryland) and reprinted in the \emph{Mechanics' Free Press} (Pennsylvania) on October 9, 1830\citesource{October91830}, as well as in \emph{The Arkansas Gazette} (Arkansas) on November 3, 1830\citesource{November31830}, also features an Englishman: “An English \emph{traveller}, who had never been out of the sound of the Bow-bells until he took it into his head to cross the “Hatlantic to take notes””. The encounter between the Englishman and (southern) Americans also includes funny misunderstandings caused by linguistic differences, culminating in the following reaction by the Englishman: “The cockney only half muttered to himself, “vell I do think they murder the king’s hinglish most hinumanly in Hamerica.” Again, /h/-insertion (and also /h/\--dropping) are linked to the speech of the Englishman, who is more specifically characterized as a Cockney, which indicates that there must be some familiarity with a stereotypical Cockney figure, including its linguistic repertoire, which also includes the second prominent linguistic form: the replacement of /w/ by /v/ (as in \emph{vell} ‘well’). The humor of the anecdote rests again on the irony created by the contrast between the Cockney’s assertion that the English in America is not as ‘good’ as in England while his own speech exhibits phonological forms which are associated with ‘incorrect’, ‘bad’, and ‘uneducated’ speech.

Another example of an anecdote focusing on /h/-dropping and -insertion to negotiate claims to (linguistic) superiority is the exchange published on July 22, 1856\citesource{July221856} between “a rather waggish New York Judge” and “an overfed John Bull” in a New York Central Railroad car, in which the latter criticizes American pronunciation by stating “It is most hastonishing, sir, to a Hinglish gentleman, to find the pronunciation of the Hinglish lendwidg so defective in this kentry”.\footnote{This anecdote was published in five newspapers in slightly different versions. The article cited above is the first one. It was published in \emph{The Boston Daily Atlas} in Boston, Massachusetts, on July 22, 1856\citesource{July221856}, under the headline “Orthoepy”. The focus on /h/-dropping and -insertion as a linguistic form marking the Englishman’s speech as incorrect and inferior is underlined by a change of headline in the second article containing the same anecdote, which was published four days later, on July 26, 1856\citesource{July261856}, in the \emph{Lake Superior Miner}, in Ontonagon, Michigan. It reads “Johnny Bull’s Idea of the “Hinglish” Language”.} The negative evaluation of /h/-dropping and -insertion as incorrect, which is the basis for the irony and humor created by this sentence, is reinforced by an explicit metalinguistic comment by the New York Judge, who replies “In this country we call a horse a \emph{horse}, but you call it ‘a \emph{nors}’, and you think that a man who don’t know what a \emph{nors} is must be a \emph{hass}!” In this article, the label “John Bull” is used to indicate that the Englishman is again a representative of Englishmen in general: John Bull is a figure which appeared in eighteenth-century England as a symbolic representation of the English people (see \citealt{Hunt2003} for details). As in the first anecdote, the two character traits highlighted are arrogance coupled with ignorance, a combination which creates a source of humor. The characterization of the American figure, the judge, as a wag, portrays Americans as smart, witty, humorous and superior without being arrogant. The final sentence concluding the anecdote asserts that the Englishman has been put in his place and accepted his (linguistic) inferiority: “A laugh, ‘like the neighing of all Tattersall’s’ at this sally, rang through the cars, and our Hinglishman suddenly ‘dried up,’ and never opened his lips until the train arrived, late at night, at Albany”.

A third anecdote, published in the \emph{Bangor Daily Whig \& Courier} (Maine) on May 23, 1857\citesource{May231857} (a reprint from the \emph{Boston Herald}), describes the encounter between an Englishman (called Thomas Brown) and an American judge during court proceedings. The Englishman is described as being originally from Liverpool and as having recently come to America with “a confused idea of the [American] nation” because he expects England to be still in control of the country. He is characterized as ignorant and naive, which was the cause of the events leading to the trial: He was exploited and made drunk by some Bostonian “wags” and had to be taken in by the police. The narrator emphasizes that the Englishman was treated in a very friendly manner by the American police - a treatment which is in stark contrast to the behavior of the Englishman during the ensuing dialogue with the judge during the legal proceedings. The Englishman's loud and arrogant manner is highlighted by contrasting it with the calm and self-confident nature of the judge:

\begin{ipquote}
{“My Lud,” cried Thomas, fixing his eye on the Judge, “I'm an Hinglish gentleman, and wishes to no why the hofficer took me in charge last night. My Lud, if I writes to his Highness, minister at Vashington and tells him of this affair, I vould not be responsible for the rage of my Lud Palmeston. I'm 'is friend, and he knows 'ow to protect the 'onor of us Britons.”
“We have no Luds in this country,” said the Judge mildly; “if you wish to say whether you are guilty or not, the Court is ready to hear you.”
“No Luds in this country, my Lud—no haristocracy, nothing but hingins and histers.—No vonder you 'ave books ritten abont [sic] your nation. Vy the last time I dined with my Lud Palmeston ve vere speaking of 'merica and he told me it was just the place for a man of my talent, and I leave a good business to come here to teach you, and vot is the konsequence?”

\centering[...]

“[...] I was told of a dozen different situvations, but vhen I vent to get 'em I vas laughed at. My Lud, I,ve [sic] been shamefully treated, and I vants justice, or the valls of this 'ere building will shake with Hingland's rage when she 'ears of it. Remember the fate of Sebastopol, and tremble for your 'omes.”}
The Judge intimated that he was not alarmed, as he knew that the revenue cutter was in port and could afford ample protection to the city against all the fleets which the English would send to revenge his cause. A fine of \$3 and cost was imposed, and as he was being taken down stairs he was begging the officers to let him have pen and paper for the purpose of “riting to the Hinglish Minister.”
\end{ipquote}


The anecdote creates a representative incident illustrating the struggle for superiority, with the better end for the Americans. The focus here is not explicitly on language but on perceived differences between the two countries regarding politics and society. The hierarchical English society with a system of nepotism and favoritism is contrasted humorously with the American egalitarian society with its just legal system, which puts the Englishman in his place. As in the anecdote above, the Englishman’s arrogance is countered by confidence, humor and wit, thus portraying the Americans not only as superior but also as more friendly and social. Language comes in more implicitly; that is, the speech of the Englishman is clearly marked as different from that of the Americans. The representation of the Englishman’s pronunciation focuses on two salient forms: /h/-dropping and -insertion and, as in the first anecdote, /v/-/w/ interchange. In addition, \emph{Lord} is consistently spelled <Lud>, which indicates a non-rhotic form, but as it is the only lexical item, it is not the case that non-rhoticity is associated with the Englishman more generally. Apart from the pronunciation respellings, there are also a few cases of eye dialect: \emph{ritten} ‘written’, \emph{riting} ‘writing’, \emph{no} ‘know’, \emph{konsequence} ‘consequence’. With regard to grammatical forms, the marking of the first-person singular by the suffix -\emph{s} is striking (note, however, the exception in \emph{I leave}). This illustrates that the Englishman is characterized as ignorant both explicitly and implicitly. The dialogue demonstrates that the Englishman knows nothing about American politics, society and the legal system, even though he claims to be a businessman with connections to the highest political circles, and this characterization is underlined implicitly through his use of linguistic forms indexing uneducatedness and ignorance.


Next to anecdotes, there are articles containing \emph{hinglish} in which the author discusses the differences between English and American speech explicitly. These articles are few but nevertheless very important as they confirm links created more implicitly in anecdotes. One such article deals with /h/-dropping and -insertion. It was published in the \emph{Cincinnati Daily Gazette} (Ohio) on October 18, 1873\citesource{October181873}, and carries the headline “The Hex-Hasperating Hinglish”. By using this headline, the author already creates a link between /h/-dropping and -insertion and Englishness and the characteristic of being exasperating and in the article he explains why this is the case. He reacts to a letter to the editor written by “an Englishman” to the \emph{New York Tribune} in which the Englishman denies “the statement that Americans in general have a more correct use of “h” than most of the educated Englishmen”. The author first cites the Englishman’s argument:

\begin{ipquote}
In England the proper use of the letter h constitutes a distinguishing line between the upper educated classes and that class composed of tradesmen, shopkeepers, mechanics, and laboring men, the latter never pronouncing it, while its omission by the former would exclude the offender from respectable society.
\end{ipquote}


The author’s response is full of irony: “Unhappily the h can not serve us; for common schooling and much intercourse have deplorably diffused the faculty of respectable speech. How happy is English society in having a dividing test which everyone aspirates!” He continues to argue that it is precisely that class of tradesmen, shopkeepers, mechanics and laboring men who are the “very substance of the kingdom” and while the Englishman arrogantly disregards them as not respectable, they are embraced by Americans, who recognize that every person plays an important part in the “great political, religious, benevolent, and social operations” of the American nation. In sum, the absence of /h/-dropping and -insertion is linked to the absence of social hierarchies in America, which is clearly evaluated positively. As a consequence, the argument by the Englishman is not deemed valid. As it is in fact the class whose members drop and insert /h/ which “embraces the very substance of the kingdom”, it is concluded in the article that the Englishmen generally speak less correctly. The author underlines this argument by claiming that /h/-dropping and -insertion is “a fault of speech which we verily believes is worse than all the Americans are guilty of”. This shows how salient the phonological form is in not only distinguishing English from American people but also as an index of American social and linguistic superiority. Furthermore, the articles show how metadiscursive activity in America is linked to metadiscourses in England. Through articles like these, Americans learn about associations between the phonological forms and social standing (‘lower class’), standardness (‘incorrect according to the standard’), degree of education (‘uneducated’, ‘ignorant’), and transform these evaluations in their context to fit their purpose of establishing superiority over England.


Another article which explicitly comments on differences focuses not so much on society or politics, but more on manners, behaviors and fashion. It is a letter written by Mr. Bailey, a “Danbury man abroad”, and printed in the \emph{Inter Ocean} (Chicago, Illinois) on October 31, 1874\citesource{October311874}. A small part of it was reprinted by the \emph{Lowell Daily Citizen and News} (Massachusetts) on January 30, 1875\citesource{January301875}. Mr. Bailey writes from England and compares English customs and fashion to his American (more specifically New England) home. The part of the letter reprinted in 1875 is headed “The Hinglish Hi-Glass”. The introductory sentence states “We exceed the English in building cars, but they completely distance us in wearing an eye-glass”. This not only describes a difference, but it also creates irony by comparing an enormous technological achievement to a simple fashion item. That the author evaluates this fashion item and the importance attributed to it as ridiculous becomes clear in the following paragraph:

\begin{ipquote}
It is worn only by the English exquisite, and he generally dons it as he asks a question, or on entering a room where there is anybody to see him. Sometimes it is suddenly put up without any apparent provocation. I imagine that it is worms. The wearer has a baggy costume, parts his hair in the middle, and has in his face an expression of mild idiocy, which is much strengthened by the glass.
\end{ipquote}


The writer creates a vivid picture of a typical English upper-class figure (“the English exquisite”) by describing his hair style, style of clothing and his inner qualities (or lack thereof). The focus on the eyeglass serves an important function: It illustrates the contrast between how the upper-class man wants to appear and the impression he actually makes on people. By emphasizing that “he generally dons it as he asks a question, or on entering a room where there is anybody to see him”, the author suggests that the eyeglass is used by the upper-class man to appear smart and educated. This is contrasted with the “expression of mild idiocy, which is much strengthened by the glass”, which suggests that in fact the opposite effect is created. This effect is strengthened by the link between /h/-dropping and -insertion and the lexical item \emph{eyeglass} (“hi-glass”) because the social meanings indexed by the phonological form (in particular stupidity) are linked to the object and to the person wearing it. Consequently, the man with the eyeglass appears vain, arrogant and idiotic (because he thinks more highly of himself than is actually justified). In this way, the form serves to characterize a group of people even though they are \emph{not} described as using the form themselves – usually /h/-dropping and -insertion is rather associated with the speech of lower classes, but through this indirect link the qualities of uneducatedness and ignorance, associated with the lower classes, are transferred to the upper classes. In this process, the negative evaluation of the form is in turn strengthened, particularly the qualities of being English, arrogant and ignorant.


That the social meaning indexed by /h/-dropping and -insertion is very well-established at least by the second half of the century can be shown by analyzing a particular text type: very short articles, which often consist of just one line, one paragraph or dialogue, and are often humorous in nature. They are normally published together in sections labeled for example “Multiple News Items” or “General and Personal”. The following article was published on July 10, 1878\citesource{July101878}, in the \emph{Daily Rocky Mountain News} (Colorado):

\begin{ipquote}
Punch puts these words into the mouth of an American lady when viewing Swiss scenery: “O, my! ain't it rustic!” Were she “Hinglish” she would say: “Ho, my hi's! Hisn't hit hawful 'igh!”
\end{ipquote}


Even in this very short text, English condescension towards American speech is countered by the representation of English speech as being marked by /h/-dropping and -insertion. The representation of the phonological form suffices to portray English speech as inferior, contrary to all claims made in England, as in this case by the magazine \emph{Punch}.


Another example of a short article shows how the negative evaluation of /h/-dropping and -insertion is used to mark other linguistic forms used by English people as ‘English’ and decidedly ‘non-American’ and to indicate that they carry no prestige in an American context at all. It appeared on April 19, 1881\citesource{April191881}, in the \emph{New Haven Evening Register} (Connecticut) and contains three sentences:

\begin{ipquote}
The London \emph{World} tells a story about an American banker, in which he is made to speak of his “top coat.” A reward of “4 pun, 6” is offered if the \emph{World} will put its finger on a straight haired American that speaks of his extensive outer garment as a “top coat.” That’s a “blarsted” Hinglishism.
\end{ipquote}


The lexical item marked as English and not American is \emph{top coat} (in America, \emph{overcoat} is used instead). This marking is done explicitly by calling it a “Hinglishism” (‘Englishism’), in analogy to the terms “Scotticism” or “Americanism” commonly used to mark words as being tied to a particular region and therefore deviant from ‘common’ and ‘proper’ English. Here, this view is reversed by marking the English form as deviant from common American English. The use of /h/-dropping and -insertion additionally links the lexical item \emph{top coat} to the social values ‘incorrect’ and ‘improper’, an evaluation which is reinforced by the adjective \emph{blarsted} (a respelling of \emph{blasted}, meaning ‘damned’), which is itself marked as an Englishism through the use of quotation marks. In addition, the spelling <blarsted> indicates a lower and back \textsc{bath} vowel, which is therefore constructed as yet another English form not evaluated as correct and desirable in an American context.\footnote{A lower and back \textsc{bath} vowel can be indicated by several spellings: <ar>, <ah> and <aw>.}


Even though /h/-dropping and -insertion is only one of several forms which are evaluated as ‘English’ and ‘not American’, it is nevertheless a very salient one. This salience is illustrated by the use of the form in articles which do not address linguistic or non-linguistic differences between England and America at all, but have a completely different topic. One such case is an article in which /h/-dropping and -insertion mainly serves to criticize an English person. It was published on May 27, 1882\citesource{May271882}, by the \emph{Salt Lake Tribune} as a reaction to an article written by an English journalist for the New York \emph{World}. The English journalist expressed a favorable attitude towards Mormons in his article, for example finding polygamy “neither unnatural, wicked nor licentious”. By labeling the journalist “Han Hinglish Hass” and using this designation as the title of the article, the author reacting to the article not only emphasizes that the author of the original article is English, but he also characterizes him as stupid, a trait evoked through the use of /h/-dropping and -insertion. From the very beginning, it is therefore clear that it is the aim of the article to counter the journalist’s view by questioning his character and his abilities, which is also done quite explicitly in the article itself by describing him as “a direct cross between a characterless knave and a pitiful fool” and as being “simply a cockney ass”.

A very extreme case of foregrounding the social meaning of the phonological form can be observed in instances where the social meaning indexed by the phonological form becomes part of or even replaces the semantic meaning of the lexical item. An example of this kind is an article published on October 20, 1892\citesource{October201892}, in \emph{The Galveston Daily News} (Texas), which reports the visit of Governor Jim Hogg to Farmersville, Texas, where he spoke to an audience of 2,000 people. The governor is quoted as saying:

\begin{ipquote}
They charge me with driving prosperity from Texas. That's Hinglish, you know. All know that we have our flush times and dull times just as we have hot and cold spells. The man who believes the people to be suckers is himself worse than a sucker - he is a mud cat.
\end{ipquote}


In this context, he uses the phrase \emph{That’s Hinglish, you know} to deny the charge that he drives prosperity from Texas, which makes the meaning of \emph{Hinglish} equivalent to the meaning ‘incorrect’ or ‘stupid’, which are exactly those social values which have come to be indexed by /h/-dropping and -insertion. The actual semantic meaning ‘English’ is backgrounded, if not completely replaced by the meanings indexed by the phonological form.


The cases discussed so far reveal how indexical values linked to a phonological form are used in the articles and how they are strengthened in each instance they are used. While there is always a potential of change inherent in each use, the social meaning of /h/-dropping and -insertion seems to be rather stable. One of its core indexical meaning is ‘Englishness’, and the stability of this meaning becomes clear in articles in which Americans are described as trying to imitate English manners, fashion and speech. In the article “The Danbury Man Abroad”, already discussed above, the author describes the following case:

\begin{ipquote}
There was one young man from Marlborough, Mass., stopping in London last summer, who devoted three whole months, but in vain, to make an eye-glass stay in his eye. I could always tell when he failed by hearing him howl and swear and kick the furniture. At the end of the three months he went home, as both his time and money were exhausted. When his room was cleaned, two full quarts of damaged eye-glasses were gathered up.
\end{ipquote}


This anecdote ridicules the attempts of Americans to imitate English fashion, in this case wearing an eyeglass, a fashion item that was linked to the use of /h/-insertion in articles cited above (“The Hinglish Hi-Glass”). The obvious exaggerations create humor and lead the readers to distance themselves from the American and his attempt to imitate something not desirable.


Whereas, in the case described above, the American imitates English fashion while he is in England, the following article describes this phenomenon happening in America as well. It represents yet another text type, a poem, which can be grouped with the short articles collected in rubrics like “All Sorts of Items”. The poem was published originally in the magazine \emph{Life} and was reprinted in the \emph{Daily Evening Bulletin} (San Francisco, California) on January 10, 1888\citesource{January101888}.

\begin{ipquote}
A THING TO BE EXPORTED.\\
Oh, why is the Anglo-American proud?–\\
    His style is imported, you know.\\
But why is his manner insuff’rably loud?–\\
    That’s also imported, you know.\\
\newpage
With “Lunnon made” raiment he cuts a great dash;\\
For everything “Hinglish” he shells out his cash;\\
No matter the value, to him all is trash\\
    That is not imported, you know.


His wines and cigars are the best to be had–\\
    That’s freshly imported, you know.\\
He makes it a point to adopt the latest “fad”\\
    That has been imported, you know.\\
With a little round window stuck into his eye,\\
    He ogles humanity as from on high,\\
An asinine figure to cut he doth try–\\
    The notion’s imported, you know.


It makes a plain Yankee excessively tired\\
    To see things imported, you know;\\
Placed up on a pedestal to be admired,\\
    Because they’re imported, you know.\\
And this Anglomaniac with his odd ways,\\
Who spends time and wealth on some imported craze,\\
Assuredly shou’d, for the rest of his days,\\
    Be quickly exported, you know.\\
\end{ipquote}


The poem contains a strong and explicit criticism of Americans who value English goods, fashion and manners more highly than American ones. The foreignness and non-nativeness of the described products and behaviors are emphasized through the regular repetition of “imported, you know” in the second, fourth and last line of every stanza, except in the very last line where the replacement of “imported” by “exported” creates a strong contrast highlighting the central message of the poem that these Anglo-Americans should leave the country. Even though linguistic differences are not addressed explicitly (except for the expression \emph{to cut a figure}, which is labeled “imported”), they are used to underline this criticism: By designating that what is imported “everything ‘Hinglish’”, the indexical values associated with /h/-dropping and -insertion are used to mark everything English as inferior to everything American. It alludes to the distinction between the values ‘inferior’ and ‘superior’ that also played a great role in the anecdotes discussed above. The American who puts imported English goods on a pedestal is labeled “Anglo-American” and “Anglomaniac” and he is characterized as feeling superior (“ogling humanity as from on high”), while, at the same time, he is also ridiculed and shown to be inferior, for example by describing him as wearing an eyeglass (“a little round window stuck into his eye”). By using English things and imitating English manners, the Anglo-American therefore also occupies the same role as the Englishmen in the anecdotes discussed above. Furthermore, a contrast is created between the apparent insanity and irrationality of the Anglo-American (his admiration of everything English is compared to a mania and a craze), and the ‘normality’ of the figure of the “plain Yankee”, who does not (need to) feel superior but who is simply sane, rational, and proud of American goods and ways of life and calls for an “exportation” of the “Anglomaniac”.


The last example I would like to discuss here also contains the figure of the Anglo-American, labeled “American Dude” (this figure will be discussed in more detail in later analyses). It is a short article containing a fictional dialogue between the dude and a “Tramp”. It was published on January 12, 1889\citesource{January121889}, in the \emph{Daily Inter Ocean} (Chicago, Illinois), and, like the poem above, taken from the magazine \emph{Life}.

\begin{ipquote}
\begin{center}
SWALLOWED IT WHOLE.
\end{center}
\textit{Life}: Tramp—Hi say, sir! Cahn’t you ’elp me a bit? Hi’m Hinglish meself, sir.

American Dude (pleased)—Aw—what’s that, me good fellow (takes out a bill), and—aw—why d’you think I’m English, y’know?

Tramp—Hoh, sir, henny one could see that! I beg parding; harn’t you the Duke of Southampton, sir—Your Grace I mean?

American Dude (sick with bliss)—There, there, me good fellow, take that to help you back to Lunnon (walks haughtily on.)
\end{ipquote}


In this scene, the tramp gets the American dude to give him money by flattering him: He pretends to identify him as an Englishman and establishes common ground by emphasizing his own Englishness (which might as well be put on). The representation of /h/-dropping and -insertion in this anecdote is particularly important because even though both characters want to signal their Englishness, only the tramp uses /h/-dropping and -insertion and he does so extensively. The American dude, on the other hand, is shown to use the filler \emph{aw} several times as well as the possessive pronoun \emph{me} and the discourse marker \emph{y’know}, which, as I will show in later analyses as well, mark him as an imitator of English speech. The character ridiculed in this article is clearly the American dude. He is described to be “sick with bliss” and haughty after he supposedly passed for an Englishman, and not just for any Englishman, but for a Duke. The allusion to sickness indicates that the American dude has lost his capacity to think straight, leading him to fall for the insincere flattery by the tramp who is just interested in getting money. The tramp is portrayed as smart and witty because he takes advantage of the dude’s vanity. Dropping and inserting /h/ enables him to index Englishness as well as a position of social inferiority, which he needs to compliment the American dude. The fact that the American is not portrayed as using the form indicates that it still serves as an index of British speech (in contrast to American speech) and a social position that the American does not aim at. His use of other linguistic forms to sound English and to position himself among the upper classes is ridiculed and criticized by showing how it leads him to be exploited by someone who is (supposedly) at the other end of the social ladder.


To summarize, the qualitative analyses of the collection of newspaper articles containing \emph{hinglish} show that /h/-dropping and -insertion is a very salient and stable index of Englishness and social and linguistic inferiority throughout the nineteenth century. It is used to counter perceived arrogant English claims of superiority and mark Englishmen as ignorant and unrefined. The social meaning was so well established that it could be shown to be part of or even replace the semantic meaning of the word (\emph{That’s Hinglish} meaning ‘That’s stupid’). Moreover, it was used to criticize the admiration of English goods, manners and speech on the part of a group of Americans and to ridicule their attempts to pass for Englishmen. In these attempts, the Americans never drop or insert /h/, which underlines that the form remains a stable index of Englishness and low social standing. The qualitative analyses also demonstrate that the forms occur in several text types and that there are several strategies by which the phonological forms are linked to their indexical values. A very effective strategy is to either describe characters and their speech as well as their behavior or to make use of these characters in anecdotes or short humorous dialogues and illustrate their speech and behaviors in concrete situations. In the last two text types, the reader is given the impression that he or she gets an immediate and truthful rendering of an event that actually happened (especially in the case of the anecdote which establishes a truth-claim) or that could have happened (in the case of the humorous dialogues). Regarding the effect of anecdotes, \citet[73--74]{Nicolaisen2011} writes:

\begin{quote}
Whether anecdotes told about […] individuals are intended to enhance or to denigrate them, the anecdotes relate incidents seen as typical of the individuals’ actions or other qualities. In performance, an anecdote is often used to underpin or confirm in its pointedness a characteristic previously ascribed to an individual (“To show you what I mean, let me tell you a story I heard about so-and-so”). Whether they are believed to be true or known to be apocryphal, anecdotes can be powerful rhetorical tools thinly disguised as narrative entertainment.
\end{quote}


This quotation illustrates how anecdotes contribute to the creation of characterological figures which are then used in shorter humorous dialogues and texts. These figures embody character traits seen as typical and it is shown how these traits co-occur with a linguistic repertoire and how this plays out in concrete situations. These figures therefore represent models of behavior and speech – usually both positive and negative ones. This supports \citegen[177]{Agha2007} argument that “characterological figures [...] motivate patterns of role alignment in interaction”. The readers of the article are invited to align themselves with the normal, intelligent and educated American, either in opposition to the arrogant and stupid Englishman (sometimes Cockney) or in opposition to the manic Anglo-American.


In the following section I will continue the qualitative analyses of newspaper articles by focusing on three search terms: \emph{dawnce, deah }AND\emph{ fellah \textup{and}}\textsc{ twousers}. I analyze them in one section because they not only show a very similar temporal development (appearing at the end of the nineteenth century, with an enormous increase in the 1880s), but also because the indexical values and characterological figures associated with the phonological forms are very similar. I will discuss \emph{bettah} in a separate section, however, because even though the search term represents non-rhoticity as well, there are additional social meanings linked to it which cannot be identified for \emph{deah} AND \emph{fellah}.

\subsubsection{\emph{dawnce, deah} AND\emph{ fellah,}\textsc{ twousers}}
\hypertarget{Toc63021237}{}
The first result of the qualitative analysis of articles containing \emph{dawnce}, \emph{deah} AND \emph{fellah} and \textsc{twousers} is that the linguistic forms are linked to Englishmen as well. The first article in the databases containing \emph{dawnce} is a short humorous paragraph describing the fate of an Englishman who had been traveling to America and had criticized many aspects of American life. Upon his return to England, he was supposedly arrested and “put in jail by his employers”. The paragraph was taken from the magazine \emph{The Knickerbocker} (from “the 5th number of the droll papers”) and published in the \emph{Boston Courier} (Massachusetts) on November 12, 1849\citesource{November121849}. It reads:

\begin{ipquote}
\textit{Summary}. A. Jarroldy, travellink Hinglishman, who found such faults with our otels, and could n't get any think to suit him, and druv a fine teem of osses while he was in this ked'ntry, and sported a white choke at our principal balls, has landed in Hingland, and been put in jail by his employers. Ha! ha! ha! “Oi loike the monner they dawnce in Frawnce!”
\end{ipquote}


There is an interesting division into two parts. First, the narrator tells the story by using pronunciation respellings indicating /h/-dropping and -insertion (\emph{Hinglishman}, \emph{otel}, \emph{osses}), non-rhoticity, which is restricted to the lexical item \emph{osses}, the realization of -\emph{ing} as /ɪŋk/ (\emph{travellink}, \emph{any think}) and a differential pronunciation of \emph{country} (three syllables and a higher \textsc{strut} vowel). There is also one case of eye dialect (<teem> ‘team’). In addition to the phonological forms, there is a grammatical form: the past tense form of \emph{drive} used by the narrator is \emph{druv}. This part of the text can be interpreted as the narrator’s voice which exhibits linguistic forms commonly associated with English speech to mock the Englishman. The characterization of the Englishman takes up established stereotypes: As in many articles analyzed above, the Englishman is depicted as arrogant and as claiming to be superior to Americans, which becomes visible through his repeated criticism of everything American (“who found such faults with our otels and could n’t get any think to suit him”). The humor rests on the contrast between the Englishman’s assumed superiority and his actual inferiority, which is pointed out in two ways: implicitly by representing his language and explicitly by describing his social downfall. The social values which I have shown to be indexically linked to /h/-dropping and -insertion mark him not only as English, but also as ignorant, uneducated, and unrefined. In America, he pretends to be a man of high social standing by attending balls, dressing elegantly and driving “a fine teem of osses”, but in England he is put into jail, which means that his use of language and his social position are aligned. That the Englishman’s social (and linguistic) inferiority is enjoyed by Americans is expressed through the laughter and schadenfreude (“Ha! ha! ha!”) on the part of the narrator. After this part, the quotation marks indicate a change of voice to that of the Englishman himself. In this last line of the story, new phonological forms are represented: the \textsc{price} vowel with a back and rounded onset (\emph{Oi}, \emph{loike}), a lowered and backed \textsc{bath} vowel (\emph{dawnce}, \emph{Frawnce}), and also a lowered and backed \textsc{trap} vowel (\emph{monner}). A likely interpretation of this exclamation is that the narrator aims at ironically contrasting the Englishman’s preference for fancy balls and upper-class activities with his speech, which clearly marks him as not belonging to these social circles. This interpretation suggests that the back realization of all of these vowels is linked to vulgar speech, which would fit the findings suggesting a negative social evaluation of the back \textsc{bath}-vowel in England in the early nineteenth century (see \sectref{subsection334}). As there are no other articles containing \emph{dawnce} which appear as early as the one cited above (the next ones were published in 1877), this interpretation cannot be supported by further evidence, but it remains tentative. The fact that the \textsc{bath}{}-vowel is only represented in the direct quotation could indicate that the author considers it a new linguistic form which is not well-established enough to be used in the mocking description of the Englishman in the first part. The repetition of the vowel in \emph{dawnce} and \emph{Frawnce} and their position at the very end of the article puts a lot of emphasis on the vowel and ensures that it is noticed by the reader. All in all, what can be deduced from the article is that the form was associated with British English speech and that it constitutes tentative evidence of an early stigmatization of a back vowel in \textsc{bath} (and also in \textsc{trap}).


Whereas this early article links the form to the figure of the arrogant but ultimately low-life Englishman that is typically associated with /h/-dropping and -insertion as well, there are different types of Englishmen linked to a back \textsc{bath} vowel and to the other phonological forms in later articles. One such figure is the English upper-class swell. It appears as a “young Oxford swell” in an article with the title “Pants”, published in the \emph{Inter Ocean} (Chicago, Illinois) on September 30, 1875\citesource{September301875}. The author of the article discusses the developments in pants fashion and writes:

\begin{ipquote}
A young Oxford swell once described his stock in trade after this fashion: “I have my walking twousers, my standing up and my sitting down twousers, my morning and my evening, my dining and my dancing twousers”. He was all twousers. We trouble to think what might have happened had he put on his standing trousers when he was sitting down, or his sitting when he was standing.
\end{ipquote}

It is noticeable that the representation of the speech of the Oxford swell focuses on one phonological form: the realization of pre-vocalic /r/ as [ʋ]. It becomes highly salient because it occurs in the word \emph{trousers}, which is used frequently as it refers to the object discussed in the article. Following the direct quotation, the author comments on the swell and by using the labiodental /r/ (“He was all twousers”), he not only puts emphasis on the form, but he also mockingly distances himself from its use and from people like the swell, who is portrayed as a person of a high social standing (he must have money to afford all these different kinds of trousers), who is educated (as the association with Oxford suggests), young and obviously very much concerned with dressing fashionably. The comment “We trouble to think what might have happened had he put on his standing trousers when he was sitting down, or his sitting when he was standing” is full of irony and illustrates that the American author finds this way of dressing ridiculous and impractical. As in the article “The Hinglish Hi-Glass”, published in the January of the same year and which I have analyzed above in relation to /h/-dropping and -insertion, the English upper-class is the subject of ridicule in this article. In “The Hinglish Hi-Glass”, I have shown that the negative social meanings indexed by /h/-dropping and -insertion are used to mark the upper-class figures as stupid, even though they are not shown to use the form themselves. This article portraying the Oxford swell confirms that /h/-dropping and -insertion is not part of the linguistic repertoire of upper-class English swells. In contrast to the article “The Hinglish Hi-Glass”, however, this article does specify a linguistic form that differentiates the repertoire of the swell from that of other (English) people: the labiodental /r/.

An article which links more linguistic forms to the repertoire of the English swell is a travelogue written by Prentice Mulford for the \emph{Daily Evening Bulletin} (San Francisco, California) and published on April 3, 1873\citesource{MulfordApril31873}. It is titled “Gossip from London” and contains accounts and comments on the author’s experiences when traveling in England. Among other things, he describes how he attended a literary reception in “England’s aristocratic halls” and encounters a particular type of young man in this context:

\begin{ipquote}
There were present many young gentlemen in full evening dress—tail coat, expansive bosom, white neck-tie, eye-glasses. They could say, “bah jove!” and “mah deah fellah”. They would give their eyeglasses the correct twirl. They stood up during the entire evening. As social perpendicularities, they were successes every one. I was delighted with them. I had read of this sort of thing. I had seen it portrayed on the stage but never before in actual reality. In these ran the genuine Simon-pure blood of the Dundreary’s.
\end{ipquote}


Again, the men are characterized as young and dressed fashionably. Dress and manner are linked through the description of the twirl given to the eyeglass and the emphasis on the “correct” twirl indicates that dress and manner are constitutive elements of this group of young men. The two examples given of his speech link not only non-rhoticity but also \textsc{price}{}-monophthongization (\emph{bah} ‘by’, \emph{mah} ‘my’) and the phrases \emph{by jove} and \emph{my dear fellow} to the figure of the young English swell. The author’s evaluation of this figure is conveyed through irony and humor. His description of them being “social perpendicularities” and “successes” seems to indicate the author’s admiration of their success in climbing the social ladder, but the exaggerated nature of this admiration (“I was delighted with them”) already hints at irony, which is supported by the contrast to the description of the rather funny looking appearance and manner of the swell and his emphasis on small gestures like twirling the eyeglass. The seemingly innocent comment “They stood up during the entire evening” can be read as an iconic representation of the swell’s uptight nature and his anxiety about the social position that he has reached, which does not allow him to relax and sit down. Against this background, the author’s delight in them can also be understood as his delight in laughing about them. The last lines of the quotation link his own experience to that of other Americans: They have read about such figures or seen them portrayed on stage. Lord Dundreary is a character in the play \emph{Our American Cousin}, written by the English playwright Tom Taylor and first staged in New York in 1858. \citet[206]{Adams2012} describes Lord Dundreary as a “buffoonish English aristocrat” who balances the main American character in the play: Asa Trenchard, a “loud and often boorish” Vermont backwoodsman. The play was very successful and Lord Dundreary “became one of the great comic turns of the latter half of the century” \citep[206]{Adams2012} and influenced the image that Americans had of English aristocrats. Mulford’s assertion that upper-class people behaving, dressing and speaking like Lord Dundreary are not just fictional constructs, but really exist in England, serves to consolidate this image.


Other English people who appear in the articles are also members of the upper-class. An article published in the \emph{Grand Forks Daily Herald} (North Dakota) on September 18, 1883\citesource{September181883}, is a comment on the rumor that the Hon. Lionel Sackville West, a British diplomat in the U.S., gave an interview to advertise the American Northwest to Americans in order to keep them out of Canada and protect the interests of the British aristocracy in Canada. The author of the article writes:

\begin{ipquote}
The probability is, however, that what he really did say to the reporter was: “Aw - my deah fellah - I weally cawn't do it, you naow, and theah you aw, you naow; if I weah to pewmit this pwecedent, it would be no end of twouble, and theah you, aw again, you naow. Vewy sorwy, that I weally cawn't oblige you - Jeames, kindly attend the  {}- aw - gentleman to the doah - good mawning”.
\end{ipquote}

In this article all phonological forms under investigation are represented: non-rhoticity (also in \emph{aw}, ‘are’, \emph{weah} ‘were’, \emph{theah} ‘there’, \emph{doah} ‘door’, \emph{pewmit} ‘permit’ and \emph{mawning} ‘morning’), labiodental /r/ (\emph{weally}, \emph{vewy}, \emph{pwecedent}, \emph{sorwy}) and the back \textsc{bath}{}-vowel (\emph{cawn’t}). In addition, \emph{aw} is also represented as a frequent filler item. Two further phonological forms are a wider \textsc{mouth}{}-vowel (\emph{naow} ‘now’) and a close, monophthongal \textsc{face}{}-vowel in the name \emph{James}. What is as noticeable as the phonological forms is the extensive use of discourse markers: \emph{there you are} and \emph{you know} occur multiple times. Because they occur so often, they disrupt the coherence of discourse rather than adding to it. In addition, the language used by the diplomat is highly formulaic – the only semantic content expressed is that he cannot do anything because this would create a problematic precedent. This creates the impression of indirectness and evasiveness, which can be considered typical of polite and diplomatic English speech, but which is also portrayed as unsatisfactory because of its low informational content and as annoying because of its incoherence. The exaggerated nature of the representation also shows that this type of speech is rather ridiculed and criticized in this context. The explicit comment that the quotation is based on “probability” and therefore imaginary indicates that the author aims to represent not just the speech of the Hon. Lionel Sackville West, but that of English upper-class diplomats and very likely upper-class men more generally. This has the effect of attributing universality to the form–meaning links because it indicates that there is a model available to create the image of incoherent, indirect upper-class English speech, which is full of polite, but ultimately meaningless phrases. Non-rhoticity, labiodental /r/ and a back realization of \textsc{bath} are evidently part of this model as well. Drawing on this model to create an instance of typical English upper-class speech, which is attributed to a real English upper-class diplomat, serves to transmit the model to a broad audience.

One anecdote that is told or alluded to several times also relates to a real person: a visitor from the other side of the Atlantic who is well-known in America, the author Oscar Wilde. It was published in two newspapers on February 16, 1882: in \emph{The Lynchburg Virginian} (Virginia, February 16, 1882\citesource{February161882b}) and in the \emph{Boston Daily Journal} (Massachusetts, February 16, 1882\citesource{February161882}).

\begin{ipquote}
One of Mr. Wilde's remarks made in Washington, has become a popular phrase in that city. The correspondent of the \textit{Courier-Journal} says that in response to an invitation to dance at the Bachelors' German, which he attended, escorted by Mrs. Robeson, he said: “I have dined and don't dawnce; those who dawnce don't dine”. And now this speech is repeated, with all manner of jeers and jokes, on all occasions.
\end{ipquote}

This shows how newspapers are not only part of speech-chain linkages, but they are also sources which describe how they work. In this case, the back \textsc{bath}{}-vowel is part of a phrase which is attributed to Oscar Wilde, who is perceived as an Englishman of a high social standing even though he was actually Irish.\footnote{In a newspaper article published on February 25, 1882, in the \emph{St. Louis Dispatch}, Wilde is described as looking “like a stout, well-fed, active young Englishman” and his “English accent” was described as “very noticeable”. The article was published in a collection of American newspaper articles containing interviews with Oscar Wilde \citep{Hofer2010}.} As the phrase is deemed funny it is often repeated, so that the form–meaning link ultimately becomes recognized by more and more people in America.

The articles describing the English swell discussed above demonstrate that /h/-dropping and -insertion is not a form associated with this figure. However, other English figures use /h/-dropping and -insertion in combination with non-rhotic forms and a back \textsc{bath} vowel, but they do not use labiodental realizations of /r/. One such figure is part of a popular anecdote taken from the magazine \emph{Texas Siftings} and published first in the \emph{Atchison Daily Globe} (Kansas) on March 26, 1892\citesource{March261892}, and then in the \emph{St. Paul Daily News} (Minnesota) on March 31, 1892\citesource{March311892b} and finally in the \emph{New Mexican} (Santa Fe, New Mexico) on April 30, 1892\citesource{April301892}.\footnote{\emph{Texas Siftings} was an illustrated weekly humor magazine, which was first published in Austin, but then later also in New York (by 1884) and in London (by 1887) \citep[412]{Kelsey2005}.}

\begin{ipquote}
\begin{center}
\textstyleStrong{He Wanted an ’Orse.}
\end{center}
An English visitor, stopping at a prominent New York hotel, sauntered up to the genial clerk during the recent cold snap and, adjusting his eyeglasses, said:

“My deah fellah, cawn’t you let me have a sledge?”

“A sledge?”

“Yas.”

“John,” said the clerk to the porter, “go to a blacksmith shop and get a sledge hammer for this gentleman.”

“No, my deah fellah, I don’t want a sledgehammer. I want one of those vehicles, you know—a sledge.”

“You mean a sleigh. Why, certainly. John, go around to the stables and get a sleigh. Put in a couple of buffalo{\kern0pt}es.”

“Biffalo{\kern0pt}es? But, my deah fellah, I carn’t drive a biffalo, ye know. Cawn’t ye let me ’ave an ’orse?”—Texas Siftings
\end{ipquote}


The humor of the anecdote rests mainly on two aspects: first, the misunderstanding caused by the difference between the word used for the vehicle in question, which is designated \emph{sledge} in England and \emph{sleigh} in the United States, and secondly the English visitor’s idea of getting around a city like New York City in a sleigh drawn by horses. The clerk responds to this idea with irony, which is designed to be apparent to the American reader, but which is apparently not understood by the English visitor. By asking the porter to get buffaloes from the stables, he plays on the stereotypical English obsession with activities involving horses, suggesting that Americans use buffaloes instead. The fact that the English visitor responds to this seriously makes him subject to ridicule – in the version published in the \emph{St. Paul Daily News}, the headline was changed to “How the English understand us” to emphasize the ignorance of English people who know so little about America that they believe that they use sleighs in cities and that they would use buffaloes to draw them instead of horses. This ignorance is also indexed linguistically by using /h/-dropping and -insertion, a feature which is highlighted in the version of the article cited above through its representation in the heading. That the English visitor is using non-rhotic forms and a back \textsc{bath}{}-vowel in addition to /h/-dropping and -insertion shows that the linguistic repertoire associated with English visitors has changed over time: English figures appearing in earlier articles do not use these forms (e.g. the Englishman abusing Americans in 1825, the English traveller in 1830, the “overfed John Bull” in the New York railroad car in 1856 and Thomas Brown in the Boston court in 1857), while English visitors in articles appearing towards the end of the century do.


Another example of the “new” English visitor figure is illustrated in a cartoon taken from the magazine \emph{Puck}, an American equivalent to the English magazine \emph{Punch}, and published in the \emph{St. Louis Republic} (Missouri) on May 1, 1892\citesource{May11892} (see \figref{fig:key:19}).


\begin{figure}
\includegraphics[height=.9\textheight]{figures/Paulsen-img19.png}
\caption{
A cartoon depicting two English tourists, published in the \emph{St. Louis Republic} (St. Louis, Missouri) on May 1, 1892, retrieved from America's Historical Newspapers
}
\label{fig:key:19}
\end{figure}

Its heading, “Proven”, refers to the aim of the two English tourists who are depicted in the two images: They want proof that assumptions circulating in England about Americans are true. The assumption tested here is expressed by the “First English Tourist” who says “Say, me deah fellah, they say an Hamerican halways ahnswers a question by ahsking one. Let’s try it and see for ourselves”. The “Second English Tourist” therefore suggests “Try the fellah at the windah, me deah boy”. The second image depicts the tourist talking to the “Official” at the ticket office asking him “Aw, me good fellah, what time does the next train leave?” to which the “Official” replies “Where to?” While this question might be taken by the Englishmen as a proof that their assumption is correct, it becomes clear to the reader that it cannot be taken as evidence of Americans exhibiting a general tendency of answering a question with a question because the question posed by the English tourist requires clarification, which makes the question by the ticket seller the only sensible reaction. This shows that while the Englishmen fail to prove assumptions about Americans, they do in fact prove assumptions about Englishmen travelling to America, namely that they are arrogant and ignorant. Their ignorance is also indexed by their use of /h/-dropping and -insertion (\emph{Hamerican}, \emph{halways}). As the English visitor figure who “wanted an ’orse”, the tourists in this cartoon also use non-rhotic forms (\emph{deah, fellah}, \emph{windah}) and a back vowel in \textsc{bath} (\emph{ahnswers}, \emph{ahsking}), which confirms that these forms have come to index Englishness.


The visual illustration links the form to other non-linguistic signs. First of all, the English tourists are depicted as wearing eyeglasses. The eyeglass has been described as an English item in several articles analyzed above, e.g. in “The Hinglish Hi-Glass” (January 30, 1875\citesource{January301875}) and in “The Gossip from London” (April 3, 1873\citesource{MulfordApril31873}). In these articles, the eyeglass has been linked to members of the upper class, the “English exquisite” and “young gentlemen” in “England’s aristocratic halls”. In this article and the article “He Wanted an ’Orse”, both published in 1892, the eyeglass is worn by an average English visitor, which indicates that the indexical link between the eyeglass and a high social position in society has weakened, while the link between the eyeglass and Englishness has become stronger, now extending to at least the upper-middle-class (English visitors to America must have some financial resources to afford travel). The fact that the English visitors use /h/-dropping and -insertion also indicates that they are not part of England’s social elite. The absence of the labiodental realization of /r/ in their speech suggests that this phonological form remains tied to the figure of the upper-class English swell. Further non-linguistic signs marking the English tourists as English are highlighted in the cartoon by depicting the tourists as being dressed completely alike. This makes it clear to the reader that the dress and the accessories chosen by the Englishmen are not based on personal preference, but on English ideas of what is fashionable. The style of the coat and the pants is identical, and a particular focus is put on the pant legs, which are turned up at the bottom (this will play a role in later analyses). The hats have the same shape and pattern, they both wear a small bag strapped across the body, and both tourists hold a cane – it is noticeable that they do not use the cane to get more stability in walking or standing, but that they hold it either in the middle or put it in the coat pocket, which suggests that the canes rather serve a decorative purpose. The cartoon therefore illustrates well how linguistic and non-linguistic signs are combined in creating and transmitting the stereotype of an English visitor to America.

The last article which focuses on the representation of English people that I analyze here is important because it illustrates that the back \textsc{bath} vowel is not only associated with English swells and ignorant English visitors but also with lower class English people. It therefore takes up the form–meaning link that occurred in the very first article containing \emph{dawnce}. It is a travelogue written by Kenneth Lamar and published in four newspapers contained in the databases: in the \emph{Worcester Daily Spy} (Massachusetts) on September 12, 1893\citesource{LamarSeptember121893},
in \emph{The North American} (Pennsylvania) on September 13, 1893\citesource{LamarSeptember131893},
in the \emph{Bismarck Daily Tribune} (North Dakota) on September 14, 1893\citesource{LamarSeptember141893}
and in the \emph{Atchison Daily Globe} (Kansas) on September 15, 1893\citesource{LamarSeptember151893}. The author describes the town Escanaba on the Upper Michigan Peninsula, which he labels “a town of aliens” – a label which is used as a heading for the article. The subheadings also indicate the main characteristics of the town: “American Soil, but Largely Controlled by British Capital”, “Dragooned by a Motley Crowd—A Halfway Unknown Region of the Union”. This indicates why the town attracts the interest of American newspapers: It raises the question of what makes a town American other than its geographical position. The author’s view becomes clear when he describes the people who live there as “a jabbering crowd of aliens, with only here and there an American” and a “motley mob of miners, lumbermen, dock wallopers and human drift in general”. He describes a particular encounter with a man, who he later explicitly calls an “Englishman”, while walking down to the docks:

\begin{ipquote}
“’Ello, pard!” cried a great beefy brute, coming toward me. “Hit’s a ’ard time we’re a-’awvin. Cawn’t ye set up an arf an arf?”\\
“What is an arf an arf?” I queried sharply.\\
“Now, listen to that, lads, will ye? Well, he’ll dawnce to our music ’fore he’s out o’ this, unless he let’s us ’awve a drink,” and he laughed a low, brutal laugh that made me shudder and look round for a possible policeman.
\end{ipquote}


This scene characterizes the Englishman as a dangerous, threatening figure and the impulse of the American to look for the police emphasizes his need for government authority in a place where people have apparently established their own rules, expressed by the Englishman telling him that “he’ll dawnce to our music ’fore he’s out o’ this, unless he let’s us ’awve a drink”. Linguistically, the two most prominent forms linked to this figure are /h/-dropping and -insertion and not only a back \textsc{bath} vowel but also a back \textsc{trap} vowel (in \emph{’awve} ‘have’ and \emph{’awving} ‘having’). They are foregrounded through the American’s inability to understand \emph{arf an arf} – even though it might be the case that he just does not know the lexical item (because he is not depicted as having difficulties understanding the Englishman in general). The differences in pronunciation are highlighted as well because he repeats the item in the question using the same phonological form as the Englishman. This shows that the back \textsc{bath} vowel can be linked to a whole range of English figures – what is important is that they are in opposition to American values and ideals. This is emphasized in this article by connecting the Englishman to a “wild, dismal, half way unknown region of the Union—a region mainly controlled by British capital—a region that has hardly a thing in common with the distinctive march forward of American ideas”. The conclusion by the author at the end of the article represents an urgent call for the Americanization of the region and even though it is not stated explicitly it becomes clear that this claim comprises language as well: “It is a duty we owe to our money and muscle to conquer and control it, or a few years hence it will be in utterly foreign hands”.


The second result of the analysis is that the search terms \emph{dawnce}, \emph{deah} AND \emph{fellah} and \textsc{twousers} are not only linked to English people, but that they also came to index a particular group of Americans: the Anglo-Americans that have already been subject to analysis in the section on /h/-dropping and -insertion. While Anglo-Americans never insert or drop /h/, they are shown to use a back \textsc{bath} vowel, non-rhotic forms, and a labiodental approximant as a realization of /r/. In the following analysis, I will take a closer look at different types of Anglo-Americans and the historical development of the figures by paying particular attention again to the strategies used to link language and social attributes of speakers.

The earliest articles featuring Anglo-American figures occur in the 1870s. The following short paragraph was published by the \emph{Trenton State Gazette} (New Jersey) on September 12, 1877\citesource{September121877}:

\begin{ipquote}
\textsc{A Trenton Dundreary at the Water Gap}—The following lament was uttered during a long rain at the Water Gap, by a Trenton Dundreary, who is suspected of painting and wearing corsets: He said, “Nawthing but wain–cawn’t have any fun. Maw maw won’t let me dawnce, and it is too wet to wamp in the woods, so I cawnt have any fun–cawnt wamp nor nawthin’.”\\---\textit{Water Gap Observer}
\end{ipquote}


By labeling the man a “Trenton Dundreary”, the author takes the English figure of Lord Dundreary and transposes it to an American context, claiming that there are men in America who behave and speak like the English Lord. The figure is portrayed as being as ridiculous as the English one – his childishness (asking his mother for permission, enjoying “ramping”, i.e. roaming wildly, in the woods) and his (suspected) effeminate traits (wearing corsets) emphasize his lack of masculinity. The two linguistic features highlighted here are the back \textsc{bath} vowel and labiodental /r/ (non-rhoticity is not represented).


The back \textsc{bath} vowel is also the most prominent feature in a dialogue that occurred in all sorts of variations and contexts in several articles. It appeared first in the \emph{Lynchburg Virginian} on February 1, 1881\citesource{February11881}, and was embedded in a short paragraph in the \emph{Worcester Daily Spy} (Massachusetts) on February 5, 1881\citesource{February51881b}, four days later:

\begin{ipquote}
When the society idiot asks, “Do you dawnce the lawncers?” the proper reply is said to be: “No, I don't dawnce the lawncers, but my sister Frawnces dawnces the lawncers and several fawncy dawnces.” The management of this sentence assures entrance into the highest circles.
\end{ipquote}


This paragraph shows that back realization of the \textsc{bath} vowel has come to index belonging to high social circles and that its use can even make access to these circles possible. By “management of this sentence” the commentator refers particularly to the pronunciation of the words, which is made obvious through the inclusion of nine words containing the vowel in the short exchange, one of which is not even a member of the \textsc{bath} set (\textit{fancy}). This makes the back realization of the vowel highly salient and ensures that people recognize the link between the vowel and the social position that it indexes. At the same time, however, the paragraph also contributes to the creation of further indexical links: By calling the person who is asking the question a “society idiot”, the form is evaluated negatively and associated with idiotic people in higher social circles and therefore questions whether it is even desirable to use the form to be part of that group of ‘idiotic’ people. In another article, the origin of this dialogue, which is described as very fashionable (“the very latest”) is attributed to an “Albany genius”, thus creating a link between the vowel and urban centers in the northeast of the United States.\footnote{The article appears twice in my data collection: The first one was published on February 7, 1881\citesource{February71881}, in the \emph{Daily Evening Bulletin} (San Francisco, California) and the second one on February 10, 1881\citesource{February101881}, in the \emph{St. Louis Globe-Democrat} (Missouri).} How such evaluations are transmitted can be illustrated by an February 5, 1881\citesource{February51881} article published in \emph{The Galveston Daily News} (Texas), which is a letter from a correspondent of the newspaper in Washington who quotes the short paragraph and comments “I mention this for the information of some of our young society folks in Texas, particularly at Tyler and Mineola”.


These early articles show that the figure of the English swell has been transferred to American figures with similar characteristics. While the articles illustrate and transmit the high social position indexically linked to a back \textsc{bath} vowel and labiodental /r/, they also criticize and ridicule people making use of these indexical links to achieve or ensure their social standing. This figure of the American swell is developed in much more detail in the following years and it is given a new label: “the dude”. In one of my initial exploratory searches, I came across an article which reveals the point in time when the label “dude” was coined and which type of person it designates. Its title is “The Dude” and it was originally published in the \emph{Brooklyn Eagle}, a New York newspaper, and reprinted in the \emph{Milwaukee Sentinel} (Wisconsin) on March 11, 1883\citesource{March111883}:

\begin{ipquote}
\begin{center}
\textstyleStrong{THE DUDE.}\\
\textstyleStrong{A Full Description of the Newly-Discovered Animal.}\\
\end{center}
A new word has been coined. It is d-u-d-e or d-o-o-d. The spelling do{\kern0pt}es not seem to be distinctly settled yet, but custom will soon regulate it. Just where the word came from nobody knows, but it has sprung into popularity within the last two weeks, so that now everybody is using it. It means a masher and yet it means something more than a masher. For instance, a masher may be young or old, or he may mash by virtue of his politeness, of his accomplishments, of his wealth, beauty, eyes, nose or fame; he may be a man of mature years, an old man, a young man or a boy. In speaking of mashers, one is never sure exactly what sort of a man is meant. There is a class of mashers in New York who will now have a definite place in the language of the town as dudes. A dude cannot be old; he must be young, and to be properly termed a dude he should be of a certain class who affect the Metropolitan theaters. The dude is from 19 to 28 years of age, wears trousers of extreme tightness, is hollow-chested, effeminate in his ways, apes the English and distinguishes himself among his fellow men as a lover of actresses. […] The word dude is a valuable addition to the slang of the day.
\end{ipquote}

The author of the article highlights many important characteristics of the dude: his age (young adult), the place where he lives (New York), his body (hollow-chested), his manner (effeminate), his behavior (an imitation of the English) and his clothing style (tight pants). His interest in theater and actresses as well as the description as hollow-chested also hints at a lack of sincerity and of substance – he is characterized as being more interested in a show than in real people and real life.

The article above does not link the dude figure to linguistic forms, but the articles containing the search terms \emph{deah} AND \emph{fellah}, \emph{dawnce} and \textsc{twousers} show that these terms and the phonological forms represented by the spellings are highly salient indexes of the American dude. An early example linking these forms to the dude is the following anecdote, published in the \emph{Rocky Mountain News} (Denver, Colorado) on September 17, 1883\citesource{September171883}, that is, half a year after the description of the dude in the article above.

\begin{ipquote}
\begin{center}
\textstyleStrong{FOUND AT LAST.}\\
\textstyleStrong{He Makes His Appearance and Creates a Lasting Impression.}
\end{center}
{Yesterday a dude walked to the Burlington ticket office. On his closely cropped, bullet head was perched a mammoth white beaver, the wide brim of which curled up like a scoop, and formed a roof for the protection of a large pair of pigeon-to{\kern0pt}ed ears. The tail of his delicate coat flirted around his suspender buttons, and entirely failed to cover the southern exposure of his tight-fitting pants, which were so small in the legs that they looked like two umbrella covers. Below the bottoms of the pants, which clasped the ankles like a Langtry glove, were a pair of feet which the Smithsonian Institute has been trying to find for a long time. They simply stood straight out from the rest of the dude as if they belonged to another family. While ticket clerks were wondering how the dude managed to slip his feet through his pants’ legs without using a sho{\kern0pt}e-horn, he leaned both elbows on the counter, and, with a gentle smile, said: “Cawn’t you sell me a sleeping caw ticket to Kansaw Citaw?”
Yes.” responded the ticket clerk.
“Aw! well do so, me deah fellaw, please. Aw! but I forgot to ask the price, you knaw.”
He was told the price of the ticket, and handing out the money he began drumming on the counter with his fingers and singing:

\centering 
We nevah speak as we pass by,\\
Although a teah drop–

“Aw, but I say, are you quite suah the sleepaw will go right though to Kansaw-Citaw?”}

{“Yes, it go{\kern0pt}es right there without fail.”

“Well, you knaw, that’s what I look out for when I take a sleepaw. I hate a beastly change, you knaw–

\centering
We nevah speak as–

“Ah, my deah fellah, tell me, there is no danger of a lay ovah?”}


{[He continues to ask questions and to sing, until the clerk becomes very annoyed. He then announces that he has to leave to get his things to the car.]}


As he glided down the street the people stopped, and looked, and wondered if they could buy the thing to play with.
\end{ipquote}


The incident narrated in this anecdote is that of ‘normal’, ‘average’ Americans encountering the dude for the first time. The setting is Burlington, which is probably Burlington, Colorado, as the newspaper is a Colorado one. The description of the dude in the beginning is similar to the one in the article before but much more humorous because it is full of similes and hyperboles (e.g. the pants looking like two umbrella covers). The conversation between the dude and the ticket clerk establishes a contrast between these figures. The dude is described as being annoying by asking silly questions and singing and drumming his fingers on the counter, whereas the ticket clerk is depicted as being calm and reasonable until the very end, when he reaches for a paperweight to throw at the dude. The dude is portrayed as an exotic figure which stands out in the normal, everyday life of a western city. The ticket clerk, on the other hand, represents a man of an average profession; he seems respectable, but in no way remarkable. The last sentence emphasizes not only the otherness of the dude, but also the fact that he is not taken seriously: He is perceived as more of a thing than a human and as something that people would play with (like a toy or a puppet) rather than interact with on an equal level. Linguistically, the form that stands out most in the dude’s repertoire and distinguishes it from the speech of the clerk is non-rhoticity. Even though it is not represented in every instance (e.g. \emph{forgot}, \emph{for}, \emph{there}, \emph{danger}), a spelling using <ah> or <aw> to indicate a non-rhotic form is found in nearly every sentence (e.g. \emph{caw}, \emph{nevah}, \emph{suah}, \emph{teah},\emph{ ovah}). A back vowel is represented not only in \textsc{bath} words (\emph{cawn’t}), but also in several instances where a back vowel would not be expected (e.g. \emph{Kansaw Citaw},\emph{ sleepaw},\emph{ fellaw}). In addition, the \textsc{goat} vowel is also represented by <aw> in \emph{knaw} (‘know’) which indicates a mid to mid-close monophthong, but in initial or medial position the vowel is not represented as being different. This creates the impression that the dude makes use of long back vowels rather inappropriately and too frequently in general, which signals hypercorrection and, connected to that, linguistic insecurity. Next to phonological forms, the phrases \emph{you know} and \emph{I say} also mark the dude’s speech as different from that of the ticket clerk.


That the speech of the dude is a deliberate imitation of English norms is explicitly described in an article written by George Salisbury for \emph{The Epoch} and reprinted in the \emph{Atchison Daily Champion} (Kansas) on October 8, 1887\citesource{SalisburyOctober81887}.

\begin{ipquote}
\begin{center}
AS AN ENGLISHMAN.\\
THE EFFORT YOUNG AMERICA MAKES TO POSE AS A BRITON.\\
A Long and Tedious Process of Preparation–The “English” Method of Speech, Oddities of Dress–A Distinctive Walk. The Eyeglass.
\end{center}
There is a large number of young men in these free states whose chief object in life is to be taken for Englishmen.

{The youth who wants to pass as an Englishman is obliged to put himself through a long and tedious process of preparation. He usually commences with a study of the “English” method of speech. The first task is to learn how to talk “away down the chest,” and the phrase chosen to experiment upon is, invariably, “By Jove.” When he can say this with the proper accent he next ventures upon “You don’t say so?” He then passes on to such sentences as “How awfully jolly. I cawn’t believe it, you know” and so on.

\centering {[...]}

If you live in the same house with him you can hear him} {up to a late hour of the night repeating over and over such words as “dawnce,” “cawn’t,” “pawth,” “chawnce,” “rathaw,” “fathaw” and “aw.” 

\centering {[...]}\\
}
\end{ipquote}

This article shows that rather than emphasizing the use of a low back vowel in a specific set of words, the dude’s speech is generally characterized as “to talk ‘away down the chest’”, which is a description of the impression that the use of the vowel creates for listeners rather than an accurate linguistic analysis by experts. This impression is foregrounded here, but non-rhoticity is also represented (in \emph{fathaw} and \emph{rathaw}) and connected to it (not least through the same spelling pronunciation). Again, a particular emphasis is also put on discourse markers (\emph{by Jove}, \emph{you know}) and lexical items (\emph{awfully}, \emph{jolly}). By focusing specifically on the process of acquiring the accent, the author stresses the unnaturalness of the pronunciation, which can only be attained through tedious practice and comes at the price of not sleeping at night. This alone makes it clear for the reader that it is not a desirable task, but the main reason why the phonological forms are evaluated negatively is that they are marked as English and not as American. What is constructed as being natural for an Englishman is at the same time constructed as not natural and impossible to acquire for an American and the article points out that young Americans who attempt to imitate English pronunciation make themselves subject to ridicule and derision. By describing a particular order which is followed in learning “the English method of speech”, the author suggests that the result is a uniform way of talking which comes at the expense of giving up the possibility of expressing one’s individuality.

In the late 1880s and 1890s, the stereotype of the American dude is strengthened continually through the appearance of this characterological figure in many humorous articles – comic dialogues, jokes, cartoons and anecdotes. I will give several examples here to show which traits of the dude were highlighted in these articles, which figures he was contrasted with and to what extent the figure was also subject to change.

The first article is an anecdote published in the \textit{Milwaukee Daily Journal} (Wisconsin) on June 29, 1889\citesource{June291889}. Its title and subtitle introduce the main characters as well as their social relationship and they also summarize the action: “Stunned. How a Montana Girl Paralyzed a New York Dude”. It consists only of direct speech, starting with that of a hostess of a social event in Boston, who introduces the New York dude, Mr. Chester de Montague, to the girl from Montana, Miss Sharpe. The remaining part of the anecdote consists of a conversation between these two characters: He is interested in her and asks her to dance with him, while she has trouble understanding him, regards his behavior as stupid and idiotic and rejects and insults him:

\begin{ipquote}
“Do you dawnce?”

“Do I what?”

“Dawnce – dawnce.”

“I – I – oh, do I \textit{dance}? Is that it?”

“Aw, yaas, yaas – to be sure.”

“Well, I dance sometimes, but I’m not going to dance any more tonight.”

“Oh, weally, \textit{weally}, me deah Miss Sharpe, you are too, too cruel. Mah I not have the honah of just one waltz – ah?”

“No, not to-night.”

“Naw, now, weally? Naw? You are vewy, vewy unkind – you weally are!”

“I guess you’ll live through it.”

“Beg pahdon?”

“What?”

“Oh – aw – I merely begged your pahdon.”

“What for?”

“Aw, weally, Miss Sharpe, you oughtn’t to guy a fellah so, you weally oughtn’t. Naw, be Jove; hanged if you ought.”

“Look here, young man, ain’t you a little off? And don’t this State make any provision for its idiots? And if it don’t, you come out to Montana and you’ll be taken care of without any expense to your friends. Good-bye!”
\end{ipquote}


 By contrasting the dude with a girl in this way, several attributes and character traits are highlighted. First of all, there is the gender difference: The dude is the male figure who tries to engage in a courting scenario with a female figure by following social conventions requiring the man to take the initiative and ask the girl to dance with him. In the course of the conversation it becomes clear, however, that the character traits typically associated with male and female figures are reversed in this case. The girl is very direct in her refusal to dance and appears tough, bold, practical and sane. The dude, on the other hand, is portrayed as indirect, hesitant, insecure and whiny. The anecdote therefore illustrates the effeminate behavior of the dude that has been explicitly mentioned in earlier articles and that is even more prominent through his juxtaposition to a girl (not a woman) who is more manly in her behavior than he is.


Secondly, the character traits are linked to different places. The dude comes from New York, a northeastern city, while the girl comes from Montana, a northwestern rural state. Their manner and behavior is therefore constructed as representative of people living in these places. The Montana girl is not interested much in social conventions, but appears natural, authentic and true to herself, while the dude is presented as affected and unnatural. This culminates in her assumption that he might be “a little off” and her practical suggestion to take him to Montana to become sane again.

Thirdly, the two figures are contrasted on a linguistic level. Their inability to understand each other puts an emphasis on the back vowel in \emph{dance} and the non-rhotic pronunciation of \emph{pardon} (which present difficulties to the Montana girl) and on the expression \emph{I guess you’ll live through it} (whose meaning is not clear to the dude). In general, the Montana girl’s use of \emph{ain’t} and the absence of third-person singular marking in \emph{it don’t} signal a lack of formal education, but her name (Miss Sharpe) indicates that she is intelligent and witty nevertheless. The use of these forms also underlines that she does not change her speech to appear educated, but that she remains authentic and down-to-earth. It is in fact the dude who, despite his education and high social position, is presented as inferior to the girl because of his affected pronunciation (the labiodental realization of /r/ is also indicated by spelling several times) and the discourse markers that he uses (\emph{by Jove}, \emph{hanged if you oughtn’t}), which make his speech appear incoherent and meaningless. The formulaic nature of his speech is reminiscent of the speech of the English diplomat represented in the article “Note and Comment” (September 18, 1883\citesource{September181883}) and in both cases it is ridiculed because of its emphasis on empty form over function (conveying meaning).

The anecdote illustrates very well how different types of performable signs are not only linked to different places, social personae and social relationships, but also to stereotypic pragmatic effects: The dude is stunned and paralyzed by the girl and therefore unsuccessful in his courtship attempts. Taking into consideration the representative nature of the anecdote, this specific effect hints at more general effects, suggesting that the manners and behavior of young men in eastern cities will lead to immobility and stand in the way of growth and progress, while the activity and practicality of westerners will advance the country’s development. This shows that the social meanings indexed by the linguistic forms are embedded in broader discourses creating stereotypical differences between easterners and westerners and I will discuss three articles which also draw on these differences and serve to reinforce them.

The first one is a cartoon which contrasts the dude with a farmer and therefore emphasizes the opposition between city and country life and people (see \figref{fig:key:20}).
\begin{figure}
\includegraphics[width=.8\textwidth]{figures/Paulsen-img20.png}
\caption{
A cartoon depicting a dude and a farmer, published in the \emph{Omaha World Herald} (Omaha, Nebraska) on September 16, 1894\citesource{September161894}, retrieved from America's Historical Newspapers
}
\label{fig:key:20}
\end{figure}
It originally appeared in the magazine \emph{Truth} and was published in two newspapers on September 16, 1894 (in Nebraska and in Massachusetts).\footnote{The dialogue without the cartoon appeared additionally in five newspapers in Colorado, Nebraska, Oregon and New Mexico.} It depicts the two figures standing across from each other on a field next to a farm, divided by a wooden fence. The dude is leaning on the fence because, as the caption explains, he wants to get over the fence, but he is worried that his “twowsahs” might get bagged at the knees in the process. The farmer on the other side stands a few feet away from the fence and suggests that the dude should simply take his pants off. The visual illustration of the two figures highlights their differences and illustrates their relationship in several ways. First, there is an obvious age difference, with the dude being much younger than the farmer. The farmer’s long beard, which is contrasted with the dude’s shaven face, underlines this difference in age and also in experience. Secondly, their style of clothing is very different: The dude is dressed elegantly, but also uncomfortably (e.g. the high and stiff collar) and the clothes are not suitable in a rural farming context. The farmer, on the other hand, is depicted as wearing clothes which are not fashionable, but comfortable and suitable for working on a farm. Thirdly, the difference is underlined by the accessories they bring to the scene: The farmer holds a rake in his hand, a useful tool for his work, while the dude holds a cane and wears an eyeglass, two items which, in this context, serve decorative purposes only. The cane and the eyeglass also serve to mark the figure as a dude because they are linked to English urban fashion, which is illustrated for example in the cartoon “Proven” published a year earlier, which depicts two English tourists wearing eyeglasses and holding canes in the same manner as the dude in this cartoon. Another important visual element is the fence between the two figures, a physical barrier that alludes to the social barrier that also exists between them. This impression of a barrier between the figures is reinforced by the farmer holding the rake upright in front of him, as if to make sure that the dude, who is leaning towards him, would not come near him. The textual elements add to the visual elements by providing a heading, labels for the figures, explicit descriptions of the emotional states of the figures and a dialogue. The heading “A City Boarder” marks the dude as a person living in a city and coming to the country temporarily as a boarder. His emotional state (“angrily”) and his question (“How the deuce can I get over this blamed fence without bagging me twowsahs at the knees?”) reveal not only his contempt for and impatience with the life in the country, but also his concern with dressing and speaking fashionably. The farmer’s answer reflects his emotional state: He “laconically” replies: “Take ’em off”. This reply in combination with his relaxed posture indicates that he is neither intimidated nor impressed by the dude, but down-to-earth and used to finding practical solutions to problems. All in all, the dude seems entirely out of place in the country and his extreme concern with fashion, his uptightness and his affected and impractical manner are presented as ridiculous in comparison to the concerns of a hard-working American farmer. Finally, his inability to get over the fence can also be interpreted as an inability to overcome the social barrier dividing him and the farmer and consequently his inability to acquire characteristics that are regarded highly in (especially rural) America.


A cartoon which draws on and contributes to the same stereotypical differences is headed “Trees and Trees” (see \figref{fig:key:21}) and was published in the \emph{Philadelphia Inquirer} (Pennsylvania) on September 19, 1897\citesource{September191897} and in \emph{The} \emph{Wheeling Register} (West Virginia) on September 22, 1897\citesource{September221897}.
\begin{figure}
\includegraphics[width=.8\textwidth]{figures/Paulsen-img21.png}
\caption{
A cartoon depicting a dude and a cowboy, published in the \emph{Philadelphia Inquirer} (Philadelphia, Pennsylvania) on September 19, 1897\citesource{September191897}, retrieved from America's Historical Newspapers
}
\label{fig:key:21}
\end{figure}
It juxtaposes the dude with the figure of the American cowboy. Even though the label \textit{dude} is not used here, the characterological figure is clearly recognizable due to his clothing and his language: The eyeglass, the cane, the hat and the pants which are extremely tight at the bottom suggest again a concern for English fashion. The cigarette in his mouth is also a characteristic feature of the dude which is mentioned in other articles. He is given the name “Tender Foot” and is correspondingly depicted as a small and weak figure that has to look up to the big and tough cowboy on his horse. The pun of the dialogue rests on the polysemy of the word \textit{tree}. While the dude tries to impress the cowboy with his aristocratic family tree, the cowboy counters it with a more literal meaning of \emph{family tree}: “Well, by thunder there’s our family tree. Father and my grandad were both hung on it”. This illustrates a difference in values that are regarded as important: on the one hand, fashion, refinement and a system of power and social hierarchy based on ancestry (as in aristocratic England), and on the other hand masculinity, toughness, and a system of power based on sheer physical characteristics. Linguistic differences are linked to these differences in values: The dude’s imitation of English values is also reflected in his speech through the use of non-rhoticity, possessive \emph{me} and the discourse marker \emph{don che know} (‘don’t you know’). The cowboy, on the other hand, uses the discourse marker \emph{by thunder}, which creates a contrast to the marker \emph{by Jove}, which is attributed to the dude’s linguistic repertoire in other articles. This marks the cowboy as calling on forces of nature rather than religion and links the figures to the extreme positions of culture versus nature.


How region is linked to the figure of the dude and contrasting figures, such as the cowboy, becomes clear in a cartoon which was published in the \emph{Grand Forks Daily Herald} (North Dakota), on April 17, 1898\citesource{April171898} (see \figref{fig:key:22}).\footnote{I found this article in a separate search, which aimed at finding cartoons or other visual illustrations of the dude. Even though it does not contain any of the search terms and is therefore not part of metadiscursive activities, it nevertheless illustrates that discourses on language are inextricably tied to other discourses – in this case to the figure of the dude and its opposite.} The figure shown is a man who appears to be strong, stout, sturdy and determined. His clothes and shoes are simple, but practical and even though they are worn out, they seem comfortable as well. His hat has a wide brim, which, together with his beard, protects the man from the harsh conditions of outdoor labor. He is depicted as being on the move with bags, boxes and a sleeping mat. The labels on these items show that he is from Dunkardville, Indiana, and is headed for North Dakota. They also serve to characterize the man explicitly as having a “good character”, “good habits” and as being hard-working and thrifty so that he has accumulated “hard earned savings”. The headline explicitly puts the figure in opposition to the dude: “He is No Dude”. The caption under the cartoon reads “But he is a good citizen, and he is welcomed to North Dakota—thousands of him”. The word \emph{but} indicates that the creator of the cartoon recognizes that some of the values associated with the dude are indeed positive – perhaps refinement, being part of higher social circles, being rich and having an interest in art and fashion. However, the cartoon points out the negative characteristics of the dude and portrays the opposing values as desirable, especially in North Dakota, a very rural state, and uses this positive characterization of rural workers to attract men to move there.


\begin{figure}
\includegraphics[width=.8\textwidth]{figures/Paulsen-img22.png}
\caption{
Cartoon “He is No Dude”, published in the \emph{Grand Forks Daily Herald} (Grand Forks, North Dakota) on April 17, 1898\citesource{April171898}, retrieved from Nineteenth-Century U.S. Newspapers
}
\label{fig:key:22}
\end{figure}

Although the contrast between east and west is very often tied to the contrast between urban and rural life, there is a comic dialogue in my collection of articles which illustrates that the differences between east and west extend to cities as well, in this case to New York and Chicago. It was published in the \emph{Atchison Daily Globe} on July 1, 1889\citesource{July11889}.


\begin{ipquote}
\begin{center}
\textstyleStrong{Only an Overgrown Village.}
\end{center}
In the window of a New York club:

Doolittle—How long have you been heah, Idlewild?

Idlewild—’Bout three houahs. And you?

Doolittle—I was heah an houah befaw you came in. Lots of girls pawsing. By the way, the fellaws tell me you have been to Chicago lately.

Idelwild—Ya-as, went out theah to look aftah some pwoperty left to me by my uncle. Queah place, Chicago. Some fellahs work out theah.

Doolittle—Naw.

Idlewild—Fact, deah chappie, weally. And you’ll nevah believe me when I tell you that all the time I was theah I nevah saw a solitarwy fellah sit in the club window and watch the girls foh moah than a half houah.

Doolittle—Ah, well; Chicago is only an overgrown village, after all.\\ —America.
\end{ipquote}


Doolittle and Idlewild are both telling names which link the dude to the character trait that is highlighted in this comic dialogue: laziness. The dudes are portrayed as spending their days sitting in clubs at the window, watching girls and finding this activity not only normal but typical of urban life, which leads them to conclude that Chicago can only be a village and a “queah place” because men actually “work out theah”. By ridiculing this attitude, the comic dialogue conveys a very positive evaluation of hard work, which connects the rural and the urban parts of the west and differentiates them from eastern cities like New York where idleness stands in the way of progress. This connects the article to the cartoon “He is no Dude”, which also places an emphasis on the fact that the savings by the rural worker are “hard earned”, which implies that he has worked for his money while the dude’s wealth is based on inheritance, as indicated in this dialogue when Idlewild tells Doolittle that the property he owns in Chicago was “left to me by my uncle”.


The laziness of the dude is also the source of humor in other comic dialogues and cartoons. The following article (see \figref{fig:key:23}) was taken from the magazine \emph{Life} and published on March 7, 1888\citesource{March71888}, in the \emph{Atchison Daily Globe} (Kansas). The two dudes are named Gus and Cholly, which are shortened and diminutive forms of the traditional names Augustus and Charles. These names were often used by emperors and kings throughout Europe so that they may evoke associations with royalty and a superior social position. However, using the nicknames instead of the full names rather creates links to familiarity, equality, closeness and, in the case of the diminutive form, even to inferiority and infantility.\footnote{Even though especially the name Charles was and still is a popular name that could be seen as ordinary and unremarkable, the choice of names can also be interpreted as symbolizing the contradiction inherent in the dude figure: He aspires to greatness but usually achieves the opposite.} Gus inviting Cholly to have a “glass of sodah” alludes to the dudes’ favorite activity: sitting in a club and having a non-alcoholic drink. Cholly’s reply is therefore surprising because the dude normally does not have important business to attend to. The humor rests on the description of the business that Cholly deems important: the purchase of stamps and a new pair of pants. That he buys the stamps for his mother makes him seem like a little boy and reinforces associations with weakness, dependence and unmanliness evoked by his name. That he needs to buy a new pair of pants alludes to his preoccupation with fashion and outer appearance, a characteristic illustrated by the visual elements of the cartoon. The two dudes look almost completely alike: They wear the same hats, they hold a cane in exactly the same manner and they wear the same type of shirt (with a high collar), overcoat and pants. The pants fashion seems to have changed: In earlier articles the pants have been described as extremely tight (like umbrella covers), while they are now depicted as rather wide. The overall message of the cartoon becomes clear through the heading “It’s Worry That Kills”, an ironic comment which makes the dude’s worries appear ridiculous.


\begin{figure}[t]
\includegraphics[height=0.5\textheight]{figures/Paulsen-img23.png}
\caption{
A cartoon depicting two dudes, published in the \emph{Atchison Daily Globe} (Atchison, Kansas) on March 7, 1888\citesource{March71888}, retrieved from Nineteenth-Century U.S. Newspapers
}
\label{fig:key:23}
\end{figure}

\largerpage

The dude’s preoccupation with dressing fashionably is also ridiculed in a comic dialogue published in the \textit{Emporia Daily Gazette} (Kansas) on January 31, 1893\citesource{January311893}, and which also highlights another characteristic of the dude: his unattractiveness to women.

\begin{ipquote}
\begin{center}
\textstyleStrong{Not Quite Desperate.}
\end{center}
Cholly (disconsolately)—Yaas, she wefused me, and she lawfed at me, too. If it wasn’t foh one thing, I’d drown myself.

Friend—You still hope?

Cholly—No, but the watah would take the cweases out of my twousers, you know.
\end{ipquote}


Even though the dude seems desperate after having been refused by a woman, his fear of not being dressed according to the latest (English) fashion, which in this case are creases at the front of the pants, keeps him from killing himself. By presenting the interest in fashion as a matter of life and death, the author of the dialogue creates the impression of absurdity, suggesting that the dude is stupid and cannot be taken seriously.


The stupidity of the dude is also targeted in a cartoon taken from the magazine \emph{Judge} and published in the \emph{Philadelphia Inquirer} (Pennsylvania) on July 9, 1893\citesource{July91893} (see \figref{fig:key:24}). The figures are immediately recognizable as dudes based on their dress and the canes they carry. The dude on the right is wearing clothes with a check pattern, which is described as being fashionable in England in other articles. The dude on the left is depicted as sucking his cane and this behavior in combination with the heading and the dialogue indicate that the title “lack of space” refers to the dude’s lack of brain capacity.

\begin{figure}
\includegraphics[width=.4\textwidth]{figures/Paulsen-img24.png}
\caption{
A cartoon depicting two dudes, one of them sucking his cane, published in the \emph{Philadelphia Inquirer} (Philadelphia, Pennsylvania) on July 9, 1893\citesource{July91893}, retrieved from America's Historical Newspapers
}
\label{fig:key:24}
\end{figure}

A characteristic which is absolutely crucial for the argument developed in the present study is the dude’s lack of authenticity and Americanness that is constructed in several articles, including the following comic dialogues. The first one was taken from the magazine \emph{Harper’s Bazar} and published in the \emph{Milwaukee Sentinel} (Wisconsin) on June 6, 1899\citesource{June61899b}.


\begin{ipquote}
\begin{center}
\textstyleStrong{Smuggled.}
\end{center}
Harper’s Bazar: Chappie had just returned from a visit to England.

“Now, my dear boy,” said his friend who met him on the pier, “keep your mouth shut. Don’t say a word to the custom house people.”

“Fawncy, now!” said Chappie. “And why, me deah fellah?”

“Because they’ll make you pay duty on that new English accent of yours.”

“Quite so!” said Chappie.

And he smuggled it in.
\end{ipquote}


The article explicitly discusses the dude’s linguistic behavior and marks his accent as English and therefore not American. The use of the smuggling metaphor evokes associations with illegal activities and sneakiness, which contributes to the negative evaluation of the accent. It also emphasizes that the accent is something foreign that has not been granted official acceptance in America.


The second example is an article which was taken from the \emph{Chicago News} and published in \emph{The Galveston Daily News} (Texas) on November 23, 1884\citesource{November231884}. Even though the figures speaking are not explicitly named, it becomes clear that an American dude is speaking to an Englishman. The irony of the dialogue is created by the dude denying the Englishman his English identity and authentic English behavior, while it becomes clear that he is in fact the person lacking authenticity and struggling with constructing a genuine identity. By adopting “English ways”, the dude becomes un-American; in addition, even these supposedly English ways are explicitly marked as not genuinely English, which makes the dude appear even less authentic. His attempts to be English fail as completely as his attempts to be accepted by American society.

\begin{ipquote}
\begin{center}
\textstyleStrong{He Was the Genuine Imported Article.}\\
{[Chicago News]}
\end{center}

“Aw, my dear fellah, I notice you cawwy youah cane by the handle.”

“Yes, sir; that is what the handle is made for.”

“But, you knaw, that is not the pwopah capah; it’s not English, you knaw.”

“I don’t care.”

“But tell me, deah fellah, why you do not assume English ways as the west of us do? You are so deuced odd, ye knaw.”

“I don’t know, unless it is that I am English and have lived in England all my life.”
\end{ipquote}

The third example is a comic dialogue which was taken from the \emph{Washington Star} and published in two newspapers contained in the databases: first in the \emph{Daily Inter Ocean} (Chicago, Illinois) on January 25, 1896\citesource{January251896}, and secondly in the \emph{Morning Oregonian} (Portland, Oregon) on February 10, 1896\citesource{February101896}. It addresses the question of what makes a person a true and authentic American in a humorous way:

\begin{ipquote}
“Chawles is what I call a twue patwiot,” remarked Willie Wibbles. “He’s Amerwican to the back bone.”

“How do you know?”

“He keeps his twousahs tuhned down now, whethah it is waining in London or not.”—Washington Star.
\end{ipquote}


The performable sign that is the main subject of the anecdote is the way that the bottom parts of the pants are worn. As I have shown above, the cartoon “Proven” (1892), which depicts two English tourists, links turned up pant legs to stereotypical Englishmen. The dialogue here makes use of this link by establishing that an American who turns up the bottom part of his pant legs is un-American and unpatriotic, while an American who keeps them turned down is an American “to the back bone”. The addition that Charles now wears them turned down “whethah it is waining in London or not” ridicules the imitation of English fashion on the grounds that it appears completely unreasonable and stupid to dress according to the weather in another country. By implication, speaking according to the fashion in another country is ridiculed as well and it marks the speaker, in this case Willie Wibbles, who is shown to use non-rhotic forms and labiodental realizations of /r/, as un-American and unpatriotic.


The question of what motivates the dude to imitate English manners and speech is also addressed humorously in the articles, for example in the following comic dialogue, published in the \textit{Milwaukee Journal} (Wisconsin) on October 12, 1894\citesource{October121894}.

\begin{ipquote}
\begin{center}
\textstyleStrong{The Social Test.}
\end{center}

Algernon (employed in extracting nourishment from his cane)—I say, Chawles, me boy, why do{\kern0pt}es a fellah have to suck his cane, don’t you know? Why cawn’t a fellah do without doing it, don’t you know?

Charles (similarly employed)—Deah me! cawn’t you see, me boy, why it is? It’s the social test, don’t you know!

Algernon—Cawn’t any fellah do what he likes with his cane?

Charles—P’whaps so, Algy, but they don’t count in society, me boy. It’s only intellect, culture and wefinement that count, don’t you know.

Algernon—Haw! why so, Chawles?

Charles—Pon me life, Algy, cawn’t you see, me deah boy! Society won’t tolerate those who haven’t bwains enough to suck their canes, don’t you know?

Algernon—Haw!—\textit{Truth}.
\end{ipquote}


Sucking the cane is presented here as a behavior that is regarded by the dude as indexical of “intellect, culture and wefinement”, qualities which are needed to be accepted as a member of high society. The dialogue illustrates that by imitating English speech (by using non-rhotic forms, a back \textsc{bath} vowel, labiodental /r/ and phrases like \emph{don’t you know}) and English manners (carrying a cane), American dudes seek to index these qualities, hoping to achieve a good position in society. At the same time, it ridicules this indexical relationship by showing that the dude not only regards wearing a cane as a “social test”, but also the activity of sucking it. Sucking the cane is reminiscient of childish behavior (e.g. children sucking their thumb) and as such in complete opposition to refined behavior (of adults). That the dude does not understand this marks him as stupid: The irony created by Charles’ statement “Society won’t tolerate those who haven’t bwains enough to suck their canes, don’t you know?” makes it clear to the readers that it is in fact the dude who lacks intellect and does not understand that the very activity of sucking the cane shows that he lacks the qualities of being part of upper-class society.


The following article sheds more light on the dude’s motivation and his position in society. It is a comic dialogue, taken from the magazine \emph{Truth} and published in \emph{The Duluth News Tribune} (Minnesota) on April 18, 1896\citesource{HallApril181896}.

\begin{ipquote}
\begin{center}
\textstyleStrong{HE GAVE UP.}\\
Why Chappie Was Baffled in His Efforts to be English.
\end{center}

Chappie—“Aw, there, deah chappie; I hardly expected to find you at the club today. What’s up?”

Algie—“Everything. I’ve given up. That’s what’s the matter.”

Chappie—“Given up? Good gwacious, deah boy, you don’t mean to say that you’re going to quit us?”

Algie—“That’s just it.”

Chappie—“Why, you’ve been the greatest monochromic-maniac of us all. What will we do for a leader without the white plume of Navarre and all that sort of thing we used to hear about at college?”

Algie—“Can’t help it; I’m done for, old fellah.”

Chappie—“Why, what do you mean?”

Algie—“Why, just this. Haven’t I bought all my clothes in London?”

Chappie—“Yes; that’s English, you know.”

Algie—“And not paid for them?”

Chappie—“Yes; that’s English y’know.”

Algie—“And turned up my trousers and played golf and yelled for the Valkyrie III and the Cambridge athletes and all that sort of thing?”

Chappie—“Yes, that was correct English, y’know.”

Algie—“Well, just at the end I have come to the limit of my resources.”

Chappie—“Aw, you don’t mean it, deah boy?”

Algie—“I do. I have discovered that I cannot marry a daughter of the Vanderbilts.”

Chappie—“Poor boy!”

Algie—“Yes—and I’ve got to remain poor. That’s just what’s the matter. —Tom Hall in \textit{Truth}.
\end{ipquote}


The dialogue shows clearly that “Algie” is a poor young man who aims at social advancement: By marrying the daughter of the Vanderbilts he hopes to gain access to the highest social circles in New York City. All his “efforts to be English” are part of his strategy to become accepted by high society (and ultimately chosen by the daughter of the Vanderbilts as a husband). That he is not alone in this behavior, but that the imitation of English ways is constitutive of the group of dudes is emphasized by Chappie asking him whether he is “going to quit us”. This reveals a development of the figure of the dude. Initially, the group of dudes comprised young men who were already part of the wealthy upper class, as I have shown in the analyses of articles above in which they are described as being interested in theaters and actresses, as not having to work because they inherited money and property (Doolittle and Idlewild) and as attending important social events like the New York dude meeting the Montana girl at a ball in Boston. Increasingly, the figure of the dude is extended to young men who are not part of these circles, but for whom the imitation of English speech and fashion becomes an instrument for social advancement. The negative evaluation of this behavior is shown in this article – not only is the goal of marrying the daughter of the Vanderbilts a completely unrealistic one for a poor man in the first place, but his realization that all his efforts were futile and his decision to give up trying to be English and quit the group of dudes show that social advancement through the imitation of English manners and speech is not possible. Algie’s ability to simply “give it up” is demonstrated in the dialogue through the absence of any linguistic forms indexing Englishness in his speech: He does not use non-rhotic forms, no back vowel in \textsc{bath} and no labiodental realizations of /r/ anymore. It also emphasizes again that the dude’s behavior and speech are inauthentic and affected. All in all, the dialogue therefore functions as an appeal to young American men to be authentic and achieve their goals by other means.


The negative effects of the “Anglomania” have also been described for the upper-class dudes. The following article combines an explicit description and comment on the development of the “Anglomania in New York” with an anecdote about a “young man about town of the name of Lamm” who talks to a friend about a “loan of twenty pounds” and how they draw money from the bank. It was originally published in the \emph{New York Tribune} only three years after the dude figure had been described first. It was reprinted in the \emph{Wheeling Register} (Virginia) on December 12, 1886\citesource{December121886}.

\begin{ipquote}
\begin{center}
ON THE DECREASE.\\
ANGLOMANIA IN NEW YORK SAID TO BE ON THE WANE.

One of the Most Curious Examples of Unrestorable Anglomaniac–A Unique Case–Making a Loan of “Twenty Pounds.”
\end{center}

Anglomania in New York is on the wane. The influence of the more American clubs, such as the Union League, the Lotos and the University, has been directed against it so steadily and with such effect that its manifestation is now mercilessly guyed. Nothing cures a weakness so quickly as ridicule. Among wealthy young club men, however, some advanced and hopeless cases remain. They have been proof against all the shafts of wit, against the ostracism of sensible men, against the contempt of women, against all the influences usually successful in restoring a mental equilibrium which some absurd charge of society has disturbed.

{Among the most curious examples of the unrestorable Anglomaniac is a young man about town of the name of Lamm. At least, if his name is not Lamm, it may as well be for all present purposes. He is wealthy, a college graduate, and a fellow of real intelligence. If his mind had not been wrecked by this unhappy mania, he would be an agreeable companion socially and perhaps useful in an honorable profession. But he is completely given over to Anglicisms.

\centering {[...]\\}
}
\end{ipquote}


This beginning of the article shows that the main metaphor used to discuss the imitation of English manners and fashion is illness. The admiration of “Anglicisms” is described as affecting the mental health of individuals and spreading uncontrollably so that quite some effort is required to contain and to cure it. The effects of the “illness” are first of all detrimental to the individual’s well-being: Even though the young men affected by the “illness” may possess many desirable characteristics (intelligence, high education, wealth), they become irrational, wrecked and weak and consequently subject to ridicule and contempt, which ultimately leads to their social isolation. In addition, the effects are detrimental to society because the equilibrium is disturbed and needs to be restored. This illness metaphor creates the impression that the individual is not in control and that a common effort by society as a whole is required to cure it to “restore” the individual to his original, healthy state, and the society to an equilibrium. That the anecdote serves the purpose of illustrating a representative case of an “unrestorable Anglomaniac” is made explicit when the narrator writes “At least, if his name is not Lamm, it may well be for the present purposes”. Even though the fictitiousness of the person and of the ensuing dialogue between him and his friend is made transparent, the message created through the anecdote adds to the efforts described explicitly before: to cure the Anglomania through ridicule.


The last article relating to the dude figure which I analyze here is interesting because it draws on the overwhelmingly negative stereotypical traits of the dude to ultimately subvert them and highlight positive characteristics of the figure. It was taken from the \emph{Omaha Bee} (Nebraska) and was reprinted in the \emph{St. Louis Globe-Democrat} on August 29, 1884\citesource{August291884b}.

\begin{ipquote}
\begin{center}
\textstyleStrong{A MUSCULAR DUDE.}\\
\textstyleStrong{He Wouldn’t Drink Whisky but Mopped Up the Floor with a Cowboy.}
\end{center}
“What’s that?”

The question came from a long-haired, big-hatted, leather-coated, wild-eyed specimen of Montana cowboy, who carried a belt full of revolvers and an odor of bovine impurity about him. He stood up before the bar in a Broadway saloon the other day, and as he spoke tipped his head in the direction of a pale-faced, hollow-eyed chap, who was leaning against the other end of the bar, tapping the pointed to{\kern0pt}e of his patent-leather sho{\kern0pt}e carelessly with a light-complexioned rattan cone [sic].

“That’s a dude,” replied the bartender.

\textstyleStrong{“A dude,”} repeated the cowboy, while a peculiar grin began to dispense itself over his sunburned features; “\textstyleStrong{a dude, is it? I’ve heard o’ them things, but I never got my lookers on one o’ them before; it’s a dandy, ain’t it? Make a good toothpick, wouldn’t it? Jest size ther critter up, will yer? Look at them ar legs o’ his; I’m a slum gullion ef I don’t spit terbacker juice all over ’em lookin’ glass sho{\kern0pt}es.}” With this the cowboy expectorated a deluge of nicotine humidity in the direction of the dude’s pedals that made them look as if they had just been pulled out of the manhole of a sewer. The dude slowly changed positions and lighted a cigarette, which he carelessly puffed without looking in the direction of the cowboy.

{[The cowboy tries to talk the dude into drinking whisky with him and when the dude politely declines, the cowboy insults the dude and attacks him. Surprisingly, the dude fights back.]}

{The collision lasted about three minutes, when the remnants of the cowboy were jammed down into a corner in an unconscious state. Readjusting his cravat, the dude laid a chunk of chewing gum down on the bar and said: \textstyleStrong{“Now, deah fellah, give me a glass of seltzah with a drop of lemon in it. That is weally the hardest work I’ve done since I played first base in the Yale nine.”}

\raggedleft
{[emphasis mine]}\\}
\end{ipquote}

As in anecdotes and cartoons analyzed above, the dude is contrasted with a cowboy figure. The setting is a Broadway saloon in New York City, that is, an urban place in the northeast. The cowboy is from Montana, a prototypically rural and northwestern state. He is obviously a visitor because he encounters a dude for the first time. However, his remark that he has heard of dudes before shows that the stereotype of the dude has already been transmitted even to remote regions. In contrast to the articles already analyzed, the cowboy is not evaluated positively here. He appears dangerous, crazy, wild, unpredictable (not the least because of his whisky drinking), uncultivated (chewing tobacco, spitting tobacco juice), animalistic and dirty (“a bovine impurity”) and eager to pick a fight with a dude who is depicted as calm, polite, controlled (because he is only drinking lemonade and water), elegant (smoking cigarettes, well-dressed, clean) and unimpressed by the cowboy’s aggressive behavior. The anecdote therefore stresses the negative characteristics of the cowboy and the positive ones of the dude and culminates in the unexpected reversal of inferiority and superiority. While the reader might expect and accept that the dude is more cultivated than the cowboy and thus more powerful on this dimension, the cowboy is clearly expected to be physically more powerful – an expectation which is violated here. This is already hinted at through the heading “A Muscular Dude” and it is confirmed when the dude beats the cowboy unconscious in the end and proves to be superior to him on all levels.

This anecdote serves an important function: It takes two stereotypical figures which mark extreme positions in society – the very end points on a continuum between nature and culture – and uses the fact that it is the cowboy figure which tends to be evaluated positively (as authentic, natural, masculine and physically powerful) to create humor. By reversing the roles and making the dude physically powerful, the anecdote subverts the stereotypes and emphasizes that physical power and cultural refinement are not opposites but can be embodied in one person. However, it is not the case that the dude is presented as a role model because other negative stereotypical character traits are still present. As in other anecdotes, the dude is depicted as unhealthy, feeling aloof and his refusal to drink any alcohol at all but only “seltzah with a drop of lemon” makes him seem childish and eccentric. Rather, the anecdote creates a positive evaluation of the middle-ground between the two poles by combining the positive characteristics of both sides (e.g. elegance, cultivatedness and calmness on the one hand, and authenticity, physical strength and a hard-working character on the other hand) and by avoiding the negative ones (e.g. lack of cultivation, lack of control, arrogance, affectedness, childishness). The representation of language supports this interpretation. The voices of the dude and the cowboy are clearly marked as different and both differ from the narrator’s voice, which serves as the representative of the ‘neutral’ middle position. The cowboy’s deviation from both the narrator and the dude on the grammatical and lexical level (his use of \emph{ain’t} and demonstrative \emph{them} as well as the use of \emph{lookers} ‘eyes’, \emph{critter} ‘creature’ and \emph{looking glass shoes}) indexes his lack of cultivation and education, especially in contrast with the narrator’s frequent use of Latinate words (e.g. \emph{expectorated}{,} \emph{deluge}{,} \emph{nicotine humidity}). But the three-way division becomes most noticeable on the phonological level, in particular with reference to the realization of /r/. The speech of the cowboy exhibits hyper-rhoticity (e.g. \emph{ther} ‘the’, \emph{yer} ‘you’, \emph{terbacker} ‘tobacco’), whereas the dude is shown to use non-rhotic forms (\emph{deah}, \emph{fellah}, \emph{seltzah}) and realizes pre-vocalic /r/ as a labiodental approximant (\emph{weally}).\footnote{For a discussion of hyper-rhoticity and non-rhoticity see \sectref{bkm:Ref530736302}.} Both hyper-rhoticity and non-rhoticity are therefore linked to the negative extreme positions, which creates an association between rhoticity and the positive middle position occupied by the ‘neutral’ narrator. Overall, this analysis illustrates again the general importance of the anecdote: It creates a representative instance of a physical fight symbolizing the ongoing fight over social and linguistic values and invites the readers to align themselves with the middle position between the dude and the cowboy.

Apart from the dude figure which represents the typical male Anglomaniac, there are also a few articles which present female figures that admire and imitate English manners and behavior. One of these article appeared before the first mention of the dude, on November 19, 1882\citesource{November191882}, in the \emph{Daily Republican-Sentinel} (Milwaukee, Wisconsin). It contains a description and a comment on “The Girls in Gotham” or, in other words, on “society girls” in New York City. In general, the author compliments the society girls on their “charmingly frank and earnest manner”, which he calls “a big improvement over the maidenly simper that formerly prevailed”. Despite all this praise, he also finds points of criticism, which he explicitly links to language use:

\begin{ipquote}
It is the craze for the English which do{\kern0pt}es her the greatest harm. In her struggle to get the English accent she lays herself open to ridicule. She is guilty of calling street-cars “trams”, and says such things as “I cawn’t dance any more”, or “I can’t dawnce any more”, combining the American and English in a most hybrid and enervating way.
\end{ipquote}


With regard to pronunciation, this article focuses on the back vowel in \textsc{bath}. By comparing the girls’ attempts to speak with an English accent to a struggle, the author emphasizes how difficult it is for the American girls to acquire the lexical distribution of the back vowel. The resulting “hybrid” pronunciation is evaluated negatively because it carries connotations of being neither authentically American nor English. Given that the author praises the girls for their unaffected manners, this reproach of lacking authenticity carries even more weight. Moreover, calling the girls \emph{guilty} indicates a violation of social norms which should not be accepted. Finally, the author clarifies that the reaction triggered by this way of talking is not positive: The word \emph{enervating} rather suggests that it puts a strain on other people.


A second article also links the back vowel in \textsc{bath} to American society girls. It was published in \emph{The Philadelphia Inquirer} (Pennsylvania) on September 14, 1895\citesource{September141895}, and contains a report on “an international cricket match between Oxford and Cambridge and the University of Pennsylvania”. The author not only describes the results of the game, but also aims to give a picture of the atmosphere of the event that took place at the University of Pennsylvania. He locates the people attending the event on the upper part of the social scale and mainly distinguishes three groups: English boys from Oxford and Cambridge, Philadelphian boys and American society girls. The group of society girls is characterized by means of an alliteration, which also highlights their relation to the other groups: “pretty, lisping lassies, who lavished lovely looks upon them [the “English lads”]”. While they do not neglect the Philadelphian boys, the English boys get most of their attention. This is illustrated by means of an anecdote about a group of four people, which is embedded in the report: an “Oxford boy”, “a Philadelphia layman” and two well-known society girls, who admire the former and neglect the latter. Language plays an important role in the girls’ attempt to get the attention of the English boy: The author reports that the girls pronounce \emph{dance} with a back vowel and use the phrase \emph{y’know},\emph{} and he comments that “you could almost catch the accent a square away” and that “accent was laid on as with a mortar trowel”. That the accent is English is made clear through the heading of the part of the report, which is “English, y’know”, and other explicit descriptions such as “the bonnie accent of old England was a foot thick everywhere”. The girls’ aim to impress the English boy through their imitation of an English accent is characterized as a “little affectation” which “was charming indeed–sometimes”. These descriptions of the accent emphasize that it is not authentic and not American, and even though the author finds it charming, the restriction to “sometimes” hints at a critical stance. The anecdote can also be read as a reproach: that the superficial American girls, who fall for the vain Oxford boy’s game, neglect their own countrymen.

The lack of authenticity of the accent is also highlighted by means of a second anecdote that is told in the article, which involves the English boy and a group of “Wissahickon waifs”, creating a stark contrast between the upper class people and the young straying men from the Philadelphia neighborhood, who are at the very bottom of society. Even though they are poor, the boys are depicted as being full of spirit and humor and having a carefree attitude to life. They are fully themselves and their sincerity is for example indexed by the expression \emph{on the level} in Jimmy’s request to Rock to tell him whether the boy is “a real British mug”. That their sincerity and authenticity makes them superior to the Oxford boy is established through their mockery: The “Dun-raven” joke uses the homophony of the name \emph{Dunraven} (the name of an English Lord whose possible attendance is discussed in the article) and \emph{done raven} (\emph{raven} probably used in the sense “to prowl ravenously after prey” given in the \emph{OED} \citeyear{raven}) to play on the English boy’s preying on the American girls. This reverses their roles: Even though the boys are intruders on the cricket grounds, they construct the English boy as an intruder who attempts to take American girls away from American boys. The primary target of their mockery is the English boy’s accent and by calling it a \emph{haccent}, they evoke the negative social meanings associated with /h/-dropping and -insertion and not only establish their superiority, but also mock the society girls’ admiration of that accent. Their own linguistic repertoire is also marked as different: Lexical items like \emph{mud}, \emph{queer} and \emph{nob} are colloquial (not used by the author in the rest of the article), the second person plural pronoun \emph{you’s} marks a grammatical difference, and on the phonological level final -\emph{ing} is realized as an alveolar /ɪn/ (\emph{lookin’}, \emph{bloomin’}), a final /ən/ is hyper-corrected to /ɪŋ/ (\emph{parding}), the \textsc{dress} vowel is raised (\emph{git}), there is TH-stopping (\emph{dat}, \emph{de}) and there are hyper-rhotic forms (\emph{yer} ‘you’, \emph{onter} ‘onto’). This anecdote therefore creates a similar effect to the article “The Muscular Dude”: It defines extreme positions, both socially and linguistically. While the anecdote about the dude subverts the stereotypes of the strong cowboy and the weak dude, this anecdote challenges the prestige associated with English accents in upper class circles by opposing their affectedness to the authenticity, sincerity and humor of the lower-class Philadelphian boys. Nevertheless, as in the case of the dude and the cowboy, neither the Oxford swell nor the Philadelphian “waifs” are constructed as role models – they are rather used to delimit a middle ground and clarify which linguistic forms are \emph{not} part of ‘normal’ and ‘neutral’ middle-class speech.

\begin{ipquote}
\begin{center}
\textstyleStrong{THE BIG CRICKET GAME}\\
{[…]}
\end{center}
{Cricket is a “society game,” and while it do{\kern0pt}es not partake of the excitement of football nor the interest of the national game, yet it draws a patronage from exclusive circles and is recognized as the favored sport of upper tendom.

\centering {[…]} \textsc{English, y’know.} {[…]}

But the bonnie accent of old England was a foot thick everywhere. The dainty American girls, society belles and heiresses, some light, others dark, some short and plump and others tall and most divinely fair, all used it.} Their little affectation was charming indeed—sometimes. When play was off and during the intermission between the first and second half, the English lads were always the centres of admiring feminine groups. They talked in broad accents and were answered back in “y’knows” by pretty, lisping lassies, who lavished lovely looks upon them. Of course, the Philadelphia boys were not neglected, for there were plenty of bright and pretty girls, enough to go around. But one group was particularly noticeable, probably not against its will. There were \textstyleStrong{two girls, both society belles}, quite well known, of whom the public often reads, and there were two young men, one \textstyleStrong{an Oxford boy}, the other \textstyleStrong{a Philadelphia layman}. And in the patois of the street, the latter was like the driver of the hearse in the funeral cortege—he simply wasn’t in it. Everything went the Englishman’s way, and he held one of the girls’ parasol over his curly head with an air of nonchalance that seemed to say, “\textstyleStrong{Look at me; I’m a real English cricket swell.}” And they all did look, the girls anyway. Both belles, one of them is a noted blonde beauty, were telling the cricketer how much she’d like him to come to the dance. \textstyleStrong{She called it “dawnce,” and you could almost catch the accent a square away}. In fact, accent was laid on as with a mortar trowel. Then the other girl wanted him to come and “\textstyleStrong{dine with us, y’know},” while the \textstyleStrong{hero of it all} twirled the parasol, uttered laconisms and \textstyleStrong{looked superior}. The \textstyleStrong{poor Philadelphia boy looked miserable} and seemed to feel that for the nonce, at least, he was lost to mind, and fully realized that it was not his day.

Across from this group were two \textstyleStrong{Wissahickon waifs}, who had somehow eluded the vigilance of the police officers and the club’s employes and surreptitiously stolen into the grounds. Said one of them: \textstyleStrong{“Hi, Jimmy! Git onter his nobs, de Englisher. Wonder if he’s Dun-raven yet? Hah!”} The point of the joke was evidently relished by “Jimmy,” for he roared with exuberance of spirit. \textstyleStrong{“Say, on de level, Rocks,”} he replied. “Is dat a real British mug? My eye, wa[?] a lookin’ guy. See me queer ’im,” and he walked up to the group of four.

\textstyleStrong{“Say, mister,”} he said, with a scrape and a bow, \textstyleStrong{“will you’s parding me fer a minnit?”}

{“Well, what is it, Bobbie?” asked the cricketer.

\textstyleStrong{“I just wanted to ask yer wat yer’d take”}—here he got ready to run—“fer yer bloody, bloomin’ haccent,” and off he went with a mocking laugh. {[…]}

\raggedleft
{[emphasis mine]}\\}

\end{ipquote}

The last article that I analyze in this part involves a female figure who not only uses a back \textsc{bath} vowel but also non-rhotic forms and a labiodental realization of /r/. It is an anecdote which was published on December 29, 1885\citesource{December291885}, in the \emph{Rocky Mountain News} (Denver, Colorado) and which consists almost exclusively of a dialogue between two salesladies. The introductory sentence establishes the truth-claim explicitly and it is striking that the reporter who wrote the anecdote emphasizes that he did not influence the women’s conversation in any way (they were “wholly oblivious” to his presence) and that he did not intend to overhear them (the conversation was “forced” on him). This reinforces the impression that the dialogue has indeed taken place exactly in the way that it is rendered by the writer. The main topic of the dialogue is the language use of one of the salesladies who has changed her style after changing from the department of “plain goods” to the “ribbons”. The ribbons department seems to attract more upper-class customers and it is described by the saleslady as more cultured, which has motivated her to change her language style to the style used by her customers (described as “some of the highest-toned and dwessiest lady’s”) as well as by the other salesladies in the department (“neahly all of us in the ‘wibbons’ are at it togetha”). Her friend compliments her on her progress, which she deems remarkable because of her lack of education (which she points out repeatedly), but she is reluctant to change her speech herself. The ‘cultured’ speech style used by the saleslady is marked most prominently by non-rhoticity and the labiodental /r/ (e.g. in \emph{wibbons depawtment}), but the back vowel in \textsc{bath} is represented as well (e.g. in \emph{cawnt}).

The evaluation of the speech style conveyed by the anecdote is very negative. First of all, it is presented as unnatural and affected: The saleslady repeats several times that she needs to practice the style and that she has to take lessons. Moreover, already in the first sentence she corrects herself (“all that sort—soath of thing”), which implies that her true natural speech still comes through. Secondly, the saleslady who changed her speech is characterized as unintelligent and uneducated. It is a source of humor in the article that she wants to convince her friend to change her speech because she does not understand that her friend is too smart, sensible and authentic to fall for the new fashion. Statements made by the friend, like “You see being at the desk sort of dulls one”, are full of irony because the dialogue makes it clear that it is indeed her friend who is the dull person. The saleslady’s lack of education is also underlined by the use of eye dialect to represent her speech (e.g. in \emph{twubble} ‘trouble’) and by her ‘incorrect’ pronunciation of words like \emph{certainly} (\emph{certahainly}). This shows that the subheading “Remarkable Proficiency Manifested by Her for a Beginner” is full of irony because the article does not portray her in a positive light. Thirdly, the speech style is linked to pressure and force. The saleslady is explicitly labeled a “victim” through the subheading and the dialogue establishes that she experiences pressure from the upper-class customer (she is afraid of not being able to communicate with them) and from her fellow peers (the salesladies at the ribbons department who have decided that they have to pay a fine for every mistake). The force exerted on her is not direct – she voluntarily adopts the speech style and seems very proud of it – but the author of the anecdote conveys that the pressure works indirectly and he compliments people like the friend of the saleslady who resist the pressure and stay authentic and true to themselves.

The strategy used to convey the evaluation of the speech forms is very powerful. The author does not condemn “the latest lingo” explicitly; on the contrary, he emphasizes that he is only an observer without an agenda – someone who lets the “facts” speak for themselves. The basic messages conveyed by the salesladies’ conversation, namely that there are linguistic differences between saleswomen working in different departments, and that these differences correlate with the social class of the customers who shop at the department, are realistic. In fact, Labov’s Department Store Study, conducted in 1962 \citep{Labov2006}, presented evidence of this correlation by comparing the use of rhotic forms by salespeople working in three department stores in New York City, and this correlation was confirmed in two replica studies (\citealt{Fowler1986} and \citealt{Mather2012}). The striking difference is that in the anecdote, non-rhoticity is presented as the upper-class prestige variant that the saleswomen working in the ribbon apartment use to accommodate to their customers, while in the New York Department Stores in the twentieth and twenty-first centuries salespeople in the upper-class stores use a higher percentage of \emph{rhotic} forms compared to the other stores. (The implications of this difference will be discussed further in \sectref{bkm:Ref2158524}.) By presenting himself as a neutral observer giving a realistic account of the salesladies’ conversation, the author of the article appeals to the readers’ intelligence and common sense to form a judgement. He refrains from any explicit disapproval of the saleslady’s “latest lingo” because this might have provoked objections to his view, whereas the humorous nature of his account makes the issue appear light and not serious, so that anyone objecting to the argument presented would risk appearing like a humorless and unpleasant person.

\begin{ipquote}
\begin{center}
\textstyleStrong{CULCHAWED CLEHKS.}\\
\textstyleStrong{The Latest Lingo and How it is Acquired by Victims.}\\
A Saleslady Tells a Companion How She Got it So Well.\\
\textstyleStrong{Remarkable Proficiency Manifested by Her for a Beginner.}
\end{center}
The following conversation was forced on a News reporter yesterday in a street car by two young “salesladies,” who appeared to be wholly oblivious to the presence of the scribe:

“Oh, deah, I’m so tiahed of Quissmiss and holidays and all that sort—soath of thing, you knoah.”

“Why, you do that remarkably well for the time you are at it.”

“Doah whaat?”

“Oh, talk that way, of course. Why, you remember when you went into ‘ribbons’ three months ago you talked quite plainly considering your lack of education.”

“Oh, deah, you flattah me.”

“Not in the least, I assure you. You must have applied yourself very steadily.”

“Why, weally, I cawnt say os I have, but you see the ‘ribbons’ is vewy culchawed, an’ one gets on fostah theah than in ‘plain goods.’”

“I judge so.”

“But of coahs I didn’t get all my culchaw theah!”

“Did you take lessons?”

“Why, cetaihanly. I’ve been an’ taken pwivate lessons and I’ve joined a litwy club foh exesize.”

“This must benefit you greatly.”

“Oh! deah, yes; I could nevah get along without pwactice. You see it’s easy enough to get onto the way of doing it but the twubble comes in putting it in pwactice.”

“It must be very difficult.”

“Yes, without pwactice impossible, I should expect. But you see neahly all of us in the ‘wibbons’ are at it togetha and that helps us wondahfully. We’ve agreed upon a fine foh any one that makes a mistake. Won’t you guess whaat it is?”

{“I’m sure that I don’t think I would be able to guess. You see being at the desk sort of dulls one.”

\centering
{[…]}\\
{[The friend tells her that she is afraid of trying to pick up the style.]}

“Oh! deah no, you mustn’t be afwaid. Why yove had an education, and ought to be able to pick up weal fast.”}

{“I fear not. As near as I can judge from the people who talk in this stylish manner the absence of education is an assistance rather than a drawback.”

“Oh! deah, no, why some of the highest toned and dwessiest lady’s can talk it just as good and even bettah than us in the wibbons depawtment.”

\centering {[...]}\\}
\end{ipquote}

To summarize the analysis of articles containing \emph{dawnce}, \emph{deah} AND \emph{fellah} and \textsc{twousers}, I have shown that the phonological forms they represent are, first of all, linked to English people. The range of indexical meanings is greatest for the back \textsc{bath} vowel, which comprises links to English people of all social classes. The labiodental realization of /r/, on the contrary, is restricted to the figure of the upper-class English swell. Non-rhoticity can index the swell, but also more average English people, often embodied in the figure of the English visitor or tourist. Next to these English figures, there are also American figures associated with the forms: people who imitate English manners, fashion and behavior. They are either part of the upper class or they seek access to high society, which shows that the forms must have acquired the indexical meanings ‘refined’, ‘cultivated’ and ‘of high social position’. The most prominent figure is the American dude, who is originally an American version of the upper-class swell, but the figure is developed further to comprise also poor young men who hope to advance socially. The dude is usually portrayed as weak, effeminate, childish, affected, lazy, sickly, unintelligent and overly concerned with his outer appearance and with impressing females.

Female figures linked to the use of the phonological forms are the society girls (in both articles, however, only the back vowel in \textsc{bath} was represented) and the saleslady, who wants to become more ‘cultured’ and be able to interact with upper class ladies. The females’ language use is also evaluated as affected and the female figures were characterized as rather unintelligent, easy to impress and insecure in their eagerness to please others. The affectation and unnaturalness indexed by the forms is underlined by opposing these figures to American figures with a lower-class and/or a rural background, such as the cowboy, the Montana girl or the Philadelphian “waifs”. These appear as authentic, natural and down-to-earth, the cowboy also being tough, strong and hard-working, the Montana girl also being sane, practical and frank and the Philadelphian “waifs” also being witty, courageous and street-wise. The Montana cowboy and the Philadelphian “waifs” have been shown to also mark the other end of the extreme in two anecdotes: Rather than serving as figures to identify with, they are used to define the middle-ground, which is particularly well illustrated in the case of rhoticity, where the realization of post-vocalic /r/ marks the neutral middle position between hyper-rhoticity on the one extreme and non-rhoticity on the other. Rhoticity will also be the focus of the analysis in the next section, which deals with articles containing the search term \emph{bettah}.

\subsubsection{\emph{bettah}}
\hypertarget{Toc63021238}{}
As the search term \emph{bettah} represents non-rhoticity, it is not surprising that there are a number of articles linking \emph{bettah} to the same social values and social personae as \emph{deah} AND \emph{fellah}: Englishmen and Americans who imitate English fashion and speech, in particular the dude. However, there are three additional broad groups of people that become indexically linked to non-rhoticity in the articles containing \emph{bettah}, which is why I analyze these articles here separately. The three groups can roughly be categorized as Black Americans, white southern Americans and mountaineers. The fact that these three groups are not indexically linked to \emph{deah} AND \emph{fellah} suggests that the phonological and the lexical levels are connected here: The phrase \emph{dear fellow} is not constructed as being part of their linguistic repertoire.


Before discussing the three groups in more detail, however, I will analyze the first article containing \emph{bettah} in the databases because it constitutes a very good example of how non-rhotic forms were linked to British English speech. It was published relatively early, on April 17, 1844\citesource{April171844}, in the \emph{Weekly Ohio Statesman}, in an article which quoted a speech by John Teesdale, a British immigrant who was born in York, England, and who had arrived in Philadelphia with his parents in 1818 and became the editor of several newspapers, for example the \emph{Ohio State Journal} \citep[526--527]{Snodgrass2008}, which was the main competitor of the newspaper in which the article appeared. It is therefore not surprising that John Teesdale is portrayed in a very negative light. The representation of his speech is part of this negative characterization: His English origin is emphasized to discredit him and his political opinions. Linguistic forms represented in his speech not only comprise non-rhoticity, but also several others, as the following sentence shows: “A high Tawiff, sah, enhables owah manufactuahs to get bettah prices foh thah goods, and, sah, what is pekooliahly remarkable, it also enables the people to buy the same goods at low prices!” In this example sentence, non-rhoticity is represented in all instances except one (\emph{remarkable}) and there is one instance of a labiodental realization of /r/ (\emph{Tawiff}). There is also one instance of /h/-insertion (\emph{enhable}), but it is noticeable that the /h/ is not inserted in the second instance of \emph{enable} and it is not dropped in \emph{high}. Furthermore, there is also an instance of yod-dropping (\emph{pekooliahly}), a form which I will discuss in more detail in the next section. Given the results of the analysis of articles containing representations of /h/-dropping and -insertion, non-rhoticity and labiodental realizations of /r/, it is striking that the labiodental approximant [ʋ], which comes to be indexically linked to the English upper-class swell, co-occurs with /h/-dropping and -insertion here, which usually indexes that the speaker is uneducated and not part of the upper class. As the social meaning of /h/-dropping and -insertion was already fairly established in the 1840s, this suggests that labiodental /r/ (and probably non-rhoticity as well) signaled first and foremost that a speaker was English. Secondly, they probably signaled incorrectness (like /h/-dropping and -insertion) and were therefore used to question the competence and intelligence of the speaker and to ridicule his argument. This interpretation is strengthened by taking an article into consideration which was also published in the \emph{Ohio Statesman} (on May 21, 1845\citesource{May211845}) and in which John Teesdale’s English is discussed in more detail:\footnote{I found the article by searching for articles about John Teesdale in order to find more information about him. The fact that it also contains the search term \emph{hinglish}, but is not identified in the search for \emph{hinglish}, illustrates the methodological problem of the study that the search software very likely did not succeed in identifying \textit{all} instances of the search terms.}

\begin{ipquote}
\begin{center}
\textstyleStrong{’Orrible.}
\end{center}
The editor of the Journal, it grieves us to say, do{\kern0pt}es not like our mode of writing the English language, and, fancying himself a “schoolmaster abroad,” he has taken us to task for false grammar. Not being a “\textit{native} Hinglish cockney,” as is Mr. Teesdale, the English cannot be strictly called our mother tongue. Mr. Teesdale not only speaks the English language with fluency, but he speaks the \textit{language of England}, and that of her tory Ministers, on all questions at issue between Great Britain and the United States.
\end{ipquote}


The author of the article reacts to John Teesdale’s criticism of the language used in articles published in the \emph{Ohio Statesman} by pointing out that Teesdale’s arrogant claims to linguistic superiority are mainly based on his English origins and the underlying belief that British English is better and more correct than American English. The strategy used to criticize Teesdale is irony: The author pretends to admit that “the English cannot be strictly called our mother tongue”, but by calling Teesdale a “\emph{native} Hinglish cockney” (the word \emph{native} being emphasized through the use of italics), he implicitly conveys his real position, namely that even if someone was born in England, he or she did not necessarily speak English more correctly than an American. Calling Teesdale a “cockney” and linking him to /h/-dropping and -insertion (prominently in the heading \emph{’Orrible} and in the word \emph{Hinglish}) serves to underline the argument that he does not speak correctly and that he is neither a role model for Americans nor entitled to criticism. This suggests that non-rhoticity and labiodental /r/ were primarily markers of incorrect speech in the 1844 article as well, although the conclusion has to remain tentative and needs further support.


While the first article links \emph{bettah} to English speech, the majority of articles link it to the three groups mentioned above, to which I will now turn. The second article in the set of articles containing \emph{bettah} in the databases already connects the form to the group of Black Americans. It was published six years after the first article, on August 02, 1850\citesource{August021850}, in the \emph{Scotio Gazette}, which is also an Ohioan newspaper (of the city Chillicothe). The article reports on the state of health in the city and discusses the fear of cholera, which had presumably caused the death of several “colored persons”. The author concludes the report with a humorous anecdote involving a “real” person, “a black servant in our family”, whose sister was expected to die of cholera. She went to see her and reported, when she came back, that her sister was getting better despite having been infected either with cholera or with whooping cough.

\begin{ipquote}
\begin{center}
\textstyleStrong{Health of Chillicothe.}
\end{center}
Our city yet continues healthy—as we trust, by the favor of Providence, it may throughout the season. Reports to the contrary have spread about the country, which had their origin in the fact that nine colored persons, old and young, and of both sexes, have died within the last 48 hours, several of them suddenly. Of these, three died of choleratic symptoms, one of old age, and another of old corn whisky. We have no doubt the real Asiatic cholera carried off some of them—and there is much alarm among the colored population in consequence. White folks, too, had as well be careful of their diet, for there is no telling how soon the erratic disease may appear among us.

{This morning, a black servant in our family, was sent for, “to see her sister die of cholera.” She was absent till about dinner-time, when she returned, with a thicker

\centering
“Pout on her lip, and\\
Smile in her eye!”

On being inquired about her sister’s health, she replied:}

“I dozzant know wedder ’twas de kolra or de hoopin’-koff—but she’s gittin bettah!”
\end{ipquote}

The anecdote aims at ridiculing the Black female servant who is smiling, carefree and optimistic even in the face of serious and dangerous illnesses because she does not know any better. Her inability to distinguish cholera and whooping cough characterizes her as ignorant because whooping cough can be recognized easily by one of its symptoms: the severe cough which is indicated by the name of the disease. Her ignorance is also signaled linguistically through the use of eye dialect: spelling \emph{cholera} <kolra>, \emph{whooping cough} <hoopin’-koff> and \emph{doesn’t} <dozzant> creates the impression of illiteracy. Her speech is additionally marked as different on a grammatical level (first-person singular -\emph{s} in \emph{I doesn’t}) and on the phonological level (TH-stopping in \emph{wedder} and \emph{de}, a raised \textsc{dress} vowel in \emph{gittin’}, alveolar -\emph{ing} in \emph{hoopin’} and \emph{gittin’}, initial /h/ instead of /hw/ in \emph{hoopin’} and initial unstressed syllable-deletion in \emph{’twas}). She is also described as having a “thicker pout on her lip” (indicating that she has full lips) to emphasize her difference on a physical level. All in all, the anecdote functions as comic relief for the readers at the end of a serious article, relieving their tension at least temporarily after reading about a potentially serious threat to their lives, at the expense of the nameless Black American servant who becomes the object of ridicule.

There are several anecdotes and short paragraphs which link non-rhoticity to Black Americans and social characteristics like being poor, lacking education and having jobs in which they serve other people (mainly whites), like household servants, washerwomen, bootblacks, waiters and porters. Sometimes they are depicted as criminals (often thieves). The following three examples illustrate how these links are created.

One example is an anecdote with the heading “Rich”, which was published in the \emph{Atchison Daily Champion} (Kansas) on August 20, 1889\citesource{August201889}, but taken from the magazine \emph{Youth’s Companion}. It depicts the conversation between “a lady” and “her colored washer-woman” by using direct quotations to construct the voices of the two figures. The narrator describes what the anecdote serves to illustrate in the first line: “People have widely different ideas of what constitutes wealth”. In the conversation, the Black woman explains to her lady that her daughter has married and that this daughter and her husband are so rich that they do not have to worry about money. The humor rests on the sum of money that the woman assumes to be necessary to be rich like that, namely 139 dollars, and the readers’ knowledge that this sum does not constitute wealth. On the contrary, the washerwoman’s expectation that they will live a wealthy life emphasizes her own poverty. It also creates a stark contrast between her own way of life and that of the lady who she is talking to. This contrast is linked to the contrast between their voices. The lady’s voice takes up less space than the Black woman’s voice and does not exhibit any differences to the ‘neutral’ voice of the narrator. The constructed dialogue consists mostly of the Black woman’s speech, which is marked by a large number of differences on all linguistic levels. As in the article “Health in Chillicothe”, the Black female speaker is marked as non-rhotic, but non-rhoticity is not represented consistently (not in \emph{her} and \emph{dollars}). In contrast to the first article, the absence of /r/ extends here even to intervocalic contexts (\emph{mah’ied} ‘married). The counterpart to /r/-loss is also present in her speech as there are three cases of /r/-insertion or hyper-rhoticity (\emph{ter} ‘to’, \emph{erlong} ‘along’, \emph{par} ‘pa’). Like the Black servant in the first article, the washer-woman exhibits fricative-stopping (\emph{de}, \emph{deyselves}), but the stopping comprises not only voiced interdental fricatives here, but also voiced labiodental fricatives (\emph{hab} ‘have’, \emph{nebbah} ‘never’). A raised vowel in \emph{get} (\emph{git}) is also represented again here, as well as initial unstressed syllable deletion (\emph{’deed} ‘indeed’, \emph{’bout} ‘about’). A phonological form that did not occur in the first article but that is prominently represented in this one is the elision of the final consonant in syllable-final consonant clusters (\emph{ole} ‘old’, \emph{hol’} ‘hold’, \emph{han’s} ‘hands’,  \emph{las’} ‘last’, \emph{lef’} ‘left’, \emph{jess} ‘just’). The weakening and reduction of \emph{and} to \emph{en} or \emph{’n} is represented orthographically as well. Considering the large time gap of almost forty years between “The Health of Chillicothe” and this anecdote, it is striking that the repertoire of phonological forms linked to Black American speech has remained fairly stable over time.

On a grammatical level, there is first-person singular -\emph{s} marking, as in the first article, but also third-person plural -\emph{s} marking (\emph{I’se wu’ked}, \emph{dey has no need}). What is new in this article is the representation of perfective \emph{done}, once with a contracted form of \emph{be} or \emph{have} (\emph{she’s done mah’ied}) and once without an auxiliary verb (\emph{his par done died}). The third-person plural reflexive pronoun \emph{deyselves} is also linked to the Black washer-woman’s speech. Lexically, the use of the phrase \emph{I reckon} and the creole form \emph{gwine} ‘going’ are linked to the Black female figure. What is also highly relevant are the different labels used to designate her. The writer of the anecdote labels her a “woman”, while he calls the other person a “lady”, which reflects a difference in social status, the “lady” being superior to the “woman”. Furthermore, the Black woman is represented as labeling herself an “ole mammy”, which evokes the mammy stereotype, which is implicitly already present in the first article about the female Black servant in Chillicothe. In a detailed study of the stereotypic cultural representations of the mammy figure, \citet{WallaceSanders2008} describes their profound influence on American culture. Her analysis shows that the mammy figure was created in the 1820s and became increasingly popular and widely recognized already by the middle of the nineteenth century. She defines “the standard, most recognizable mammy character” as

\begin{quote}
a creative combination of extreme behavior and exaggerated features. Mam\-my's body is grotesquely marked by excess: she is usually extremely overweight, very tall, broad-shouldered; her skin is nearly black. She manages to be a jolly presence – she often sings or tells stories while she works – and a strict disciplinarian at the same time. First as slave, then as a free woman, the mammy is largely associated with the care of white children or depicted with noticeable attachment to white children. Her unprecedented devotion to her white family reflects her racial inferiority. \textit{Mammy} is often both her title and the only name she has ever been given. She may also be a cook or personal maid to her mistress – a classic southern belle – whom she infantilizes. Her clothes are typical of a domestic: headscarf and apron, but she is especially attracted to brightly colored, elaborately tied scarves. \citep[5--6]{WallaceSanders2008}
\end{quote}


In this anecdote, several of these characteristics are also present: She is a washerwoman and thus a hard worker, she is inferior to the “lady” and she is depicted as jolly and oblivious to her social and economic position. This is particularly highlighted in her last statement “Some folks jess seem ter be bawn lucky”, which is ironic because she refers to her daughter and her husband as the lucky ones, wile it is clear to the reader that all three of them are poor – not least because of their ethnicity, which they were given by birth (so they were in fact \emph{not} born lucky). The implicit statement conveyed by the anecdote is that Black Americans are not lucky because they are rich, but because they are too ignorant to realize how poor they are, and that because of their jolliness they do not need help or pity. This condescending attitude ultimately serves to evoke feelings of superiority on the part of the white readership, so that the superiority/inferiority distinction is also linked to rhoticity/non-rhoticity and the other linguistic differences as well.


\begin{ipquote}
\begin{center}
\textstyleStrong{Rich.}
\end{center}
People have widely different ideas of what constitutes wealth, as the following incident illustrates:

“I hear, Amanda, that your daughter is married,” said a lady to her colored washer-woman.

“Yes’m,” was the reply; “en I tell yo’ she’s done mah’ied bettah ’n her ole mammy did, en she’ll nebbah hab to wu’k like I’se wu’ked. No’m, all she’ll hab ter do ’ll be to set en hol’ her han’s.”

“Is her husband rich?”

“’Deed he is, ma’am; he had fifty dollars in bank de day he got mah’ied, en his par done died las’ week en lef’ him seventy-five mo’, en my Tilly she had fo’teen of her own, so I reckon dey has no need to worry deyselves ’bout dey’s gwine ter git erlong. Some folks jess seem ter be bawn lucky.”—Youth’s Companion
\end{ipquote}

The second example is a short paragraph consisting of a direct quotation of “Uncle Eben”. It must have been very popular because it was published in six different newspapers after its first occurrence in the \emph{Washington Star}: in the \emph{Morning Oregonian} (Portland, Oregon) on December 08, 1893\citesource{December081893}, in the \emph{Daily Picayune} (New Orleans, Louisiana) on December 14, 1893\citesource{December141893}, in the \emph{Idaho Statesman} (Boise, Idaho) on December 27, 1893\citesource{December271893b}, in the \emph{Atchison Daily Globe} (Kansas) on December 27, 1893\citesource{December271893}, in the \emph{Northern Christian Advocate} (Syracuse, New York) on January 3, 1894\citesource{January31894}, and in the \emph{Boston Investigator} (Massachusetts) on March 28, 1894\citesource{March281894}. The following version is the one that was published in the \emph{Idaho Statesman} and while the paragraph is largely the same as in the other newspapers it has been given the heading “Philosophy” here. This function of this heading is to increase the humor of the paragraph: The term \emph{philosophy} raises the expectation of a sophisticated mental exercise performed by an educated person with time and motivation to deal with the basic questions of human existence. This expectation is violated, however, by the quotation of the paragraph, which expresses a belief about life which is rather simple, namely that one should take care of basic needs (like nutrition) before aiming at improving one’s outward appearance. The simplicity of the belief is underlined by the metaphors used to illustrate it: A “cabbage undah yo’ wais’-coat” symbolizes the need for nutrition, while the “chrysanthemum in yer button-hole” symbolizes the care for outward appearance. The person uttering the statement is therefore portrayed in an ambivalent way: On the one hand, he is depicted as a character that is down-to-earth and not interested in vanity or luxury, but on the other hand, he is also ridiculed as a simpleton leading a way of life marked by poverty and the need to fulfill his basic needs. His Black ethnicity is marked through the use of the address term \emph{Uncle}, which was commonly given to male Black Americans instead of other address terms, such as \emph{Mister} \citep[10]{Harris2008}.

Regarding the representation of Uncle Eben’s voice, non-rhoticity (\emph{bettah}, \emph{undah}, \emph{yoh}) is again a prominent form, but, as in the anecdote “Rich”, some words are not marked as non-rhotic (\emph{yer} ‘your’, \emph{outward}) and hyper-rhotic forms occur as well (\emph{ter} ‘to’, \emph{er} ‘a’). The determiner \emph{your} is represented once as rhotic and once as non-rhotic, which creates the impression of inconsistency in Uncle Eben’s speech. Further phonological forms shared with the representation of Black speech in the anecdote “Rich” are final consonant cluster reduction (\emph{Doan} ‘don’t’, \emph{min} ‘mind’, \emph{wais’} ‘waist’) and the stopping of voiced interdental (\emph{dan} ‘than’) and labiodental fricatives (\emph{hab} ‘have’). A form that did not occur in the two articles discussed above is \emph{Hit}, a case of /h/-insertion, but as it only occurs in one function word, it is not a salient form here. On a grammatical level, subject-verb agreement is also marked as different in \emph{Hit am bettah}. Overall, the shortness of Uncle Eben’s philosophical statement in combination with the multitude of linguistic forms signaling difference from the ‘neutral’ speech forms used in surrounding newspaper articles underlines the uneducatedness of the figure and the simplicity of his approach to life and it creates the impression of otherness.

\begin{ipquote}
\begin{center}
\textstyleStrong{Philosophy.}
\end{center}
“Doan put yer min too much on outward decorations,” said Uncle Eben. “Hit am bettah ter hab er cabbage undah yoh wais’-coat dan er chrysanthemum in yer button-hole.”—Washington Star.
\end{ipquote}

The third example is the anecdote “An Unwelcome Fifteenth”, published originally by the \emph{Detroit Free Press} and printed in the \emph{Dallas Morning News} (Texas) on August 1 and August 2, 1899. The narrator describes a “couple of tourists” who talk to “an old negro” during their journey “in the rural district of the south”. The “negro” lives in “a small log cabin, out of which children of all sizes and age came swarming like bees from a hive” and he sells food to passing tourists. When he sends one of the children, called Judas Iscariot, to catch a chicken, the tourists ask him why he has given the boy this name. He answers

\begin{ipquote}
“Well, I’ll tell yo’, sah. Hit’s like dis: Yo’ see I’d had fo’teen chillum befo’ Judas Iscariot was bawn, an’ fo’teen chillum is a mighty big fam’ly fo’ a po’ man ter raise en keer fo’, thout habin’ no mo’, so when Judas Iscariot came erlong I gib ’im dat name caze you know de Bible hit say it’d be bettah fo’ Judas Iscariot if he’d nebbah been bawn.”
\end{ipquote}

This explanation highlights several supposed characteristics of Black people: their poverty, their big families, their religiousness and their lack of education and intelligence. The last aspects are foregrounded here, as the main aim of the anecdote is to ridicule the way that Black people understand religious aspects and how this influences their lives. To educated religious people, the main characteristic associated with Judas is that he was a traitor whose betrayal led to the crucifixion of Jesus, which is why they would not name a child Judas. Because of his lack of understanding the bible, the Black man only focused on the statement that it would have been better if Judas had never been born and used this as a basis for naming his child. His lack of intelligence and education therefore has negative consequences for the child. The comparison of the family’s cabin to a beehive evokes associations with animal-like behavior and adds to the impression that while Black people try to be civilized, for example by being religious, they fail to achieve it.

Linguistically, the forms that mark the Black man’s speech as different from that of the narrator are strikingly similar to those used in the anecdote “Rich” and in the short paragraph “Philosophy”, which were published ten and six years earlier. Non-rhoticity is combined with hyper-rhoticity, and in one instance (\emph{keer} ‘care’) non-rhoticity is not marked. In the same word, however, the spelling of the \textsc{square} vowel as <ee> in \emph{care} suggests a differential realization of the vowel as a high front monophthong. Furthermore, there are several instances of interdental and labiodental fricative stopping. Final consonant cluster reduction is not marked in the quotation above because there are no words containing final clusters, but it is marked in other parts of his speech represented in the anecdote. The third-person singular personal pronoun \emph{it} is pronounced \emph{hit}, as in Uncle Eben’s speech, and it becomes clear here that this does not represent a general process of /h/-insertion, but that the insertion is lexically restricted to the pronoun because if the author had wanted to mark the speaker as exhibiting /h/-insertion on a phonological level, it is very likely that he would have spelled the name Iscariot with an initial <H>. Furthermore, alveolar -\emph{ing} and initial unstressed syllable deletion are forms marking the Black man’s speech as well (\emph{thout habin’} ‘without having’). Additional forms, which have not been part of representations analyzed above, are the result of phonological reduction processes affecting lexical items: \emph{chillum} ‘children’ and \emph{fam’ly} ‘family’.

On a grammatical level, differences in subject-verb agreement are also represented, but this time it is the absence of -\emph{s} marking the third-person singular (\emph{it say}). There are also grammatical forms not found in the articles above. In one case, a past tense form is not marked (\emph{when Judas Iscariot came erlong I gib’ im dat name}). Furthermore, \emph{de Bible hit say} is a case of left dislocation, where a noun phrase is repeated and replaced by a pronoun in its second occurrence \citep[196]{Schneider2015}. On a pragmatic level, the address term \emph{sah} is used frequently by the Black man (at least once in every contribution to the dialogue) and he also uses the address term \emph{gemmen} ‘gentlemen’ twice. As the tourists are depicted as not using any terms of address when talking to the Black man, their higher social position is underlined by linguistic means. The “gentlemen” are not required to be polite to the poor, uneducated Black man, but they may simply use his help to fulfill their basic need for food.

The three examples analyzed so far not only illustrate typical characteristics associated with Black people, but they also show that in the last two decades of the nineteenth century a fairly stable repertoire of linguistic forms was used to represent their speech. Before turning to the analysis of other text types, which exhibit a number of different strategies to establish indexical form–meaning links, I will analyze one last anecdote here because it aims at illustrating how stereotypes are formed and that relying on them can have negative consequences.

The anecdote tells a story about a white man who jumps to conclusions too quickly. It is entitled “Too Hasty” and it must have been a very popular anecdote because it was published four times in 1891 (first in the \emph{Dallas Morning News} (Texas) on December 20, 1891\citesource{December201891}, then in the\emph{ Atchison Daily Globe} (Kansas) on December 22, 1891\citesource{December221891} (this version is cited below), then in \emph{The Galveston Daily News} (Texas) on December 24, 1891\citesource{December241891}, and finally in the \emph{Philadelphia Inquirer} (Pennsylvania) on December 27, 1891\citesource{December271891}) and three times in 1892 (in \emph{The Emporia Daily Gazette} (Kansas) on January 28, 1892\citesource{January281892}, and in \emph{The Milwaukee Journal} (Wisconsin) on February 29, 1892\citesource{February291892}, and in the \emph{Aberdeen Daily News} (South Dakota) on April 19, 1892\citesource{April191892}). It was originally published by the magazine \emph{Youth’s Companion}. The anecdote exhibits a structure that is typical of anecdotes: In the first sentence, the narrator summarizes the main message that the incident described in the anecdote is supposed to illustrate: “Jumping at conclusions often results in embarrassment to all concerned”. The narrator continues by introducing the main figure that is the focus of the anecdote: “a certain clerk in a Court street law office”. The following part is mainly characterized by dialogues between the clerk and Black American figures: first an “old Negro” and then “another sable head”. Both of them offer to clean the window, but the clerk sends them away. The third person entering the office is described as “a dark face surmounted by a rather rusty hat”. The clerk, who gives the incomer only a quick look, assumes that this is another Black American who wants a window-cleaning job and says “I suppose you want to wash windows, too, don’t you?” The humor of the anecdote is created in the last paragraph which shows that the clerk’s assumption was wrong and that the man is actually the father of “the senior member of the firm”. This is revealed through the senior member’s enthusiastic greeting, accompanied by the words “Why, my dear old father, this is the most delightful surprise of my life!” This sentence, representing the voice of the senior member of the firm, indicates that the man in question is a white man because it does not exhibit any of the forms marking the speech of the two Black men who asked for a window cleaning job before. The father must therefore also be white, and the darkness of his face was probably rather a result of the hat casting a shadow. The clerk’s hasty conclusion draws attention to the way that stereotypes come into existence: Based on two encounters with Black men who offer to clean the windows, the clerk draws the generalization that all Black men who enter the office want to get a window-cleaning job, so, in other words, he assumes a social regularity based on a recurrent co-occurrence of signs (ethnicity of the speaker and his job/social position). However, the main point made in the anecdote is that perception can be influenced by stereotypes. Based on his prior experience, the clerk immediately assumes that the dark face of the third incomer belongs to a Black man who wants to clean windows. His expectation prevents him from taking a closer look at the man and this causes great embarrassment. It is notable that the anecdote does not call into question stereotypical assumptions about Black people and the jobs they hold, but it warns about applying these assumptions to the wrong people (that is, people who are not Black).

Language plays an important role in differentiating and marking the ethnicity of the speakers. The window cleaners’ speech exhibits linguistic forms which are typically associated with Black speech. Non-rhoticity is salient again and highlighted, for example through the use of a parallel structure when the second window cleaner and the clerk exchange greetings: “Good aftahnoon, boss”. / “Good afternoon”. The repetition of the lexical item \emph{afternoon} draws attention to the phonological difference, that is, the window cleaner’s non-rhoticity. The use of the address term \emph{boss} by the window cleaner and the absence of any address term by the clerk signal their unequal social relationship. The Black man also uses incomplete sentences: There are no subjects in the sentences \emph{Want yoh windows cleaned?} and \emph{Do it cheap, boss}, where they would normally be required. The other Black speaker also uses an address term (\emph{cap’n}) and he exhibits TH-stopping (\emph{dey} ‘they’), hyper-rhoticity (\emph{kinder}), a longer, higher and fronter vowel in \emph{leetle} ‘little’ and third-person plural verb forms which are marked by an -\emph{s} (\emph{they looks}, \emph{they needs}). Both Black speakers also exhibit elisions (\emph{cap’n}, \emph{’em}, \emph{lemme}) and it is noticeable that they both use the word \emph{better} as part of a modal construction, which \citet{vanderAuwera2013} call a \textsc{better} construction. In their framework, the \textsc{better} constructions consist of one of the comparative modals \emph{had better}, \emph{’d better} or \emph{better} followed by a verb, and they usually express the speaker’s advice and more rarely the speaker’s hope. In the anecdote, both Black speakers use the same construction (the modal \emph{better} followed by the verb phrase \emph{lemme} + V and without an explicit subject) to advise the clerk that the windows need to be cleaned. This parallel structure draws attention to the modal construction and it therefore illustrates how based on the form \emph{bettah} indexical links are created both on the phonological and the grammatical level. In contrast to the Black speakers, the senior member’s linguistic repertoire does not contain any forms that are different from the narrator’s voice or the clerk’s voice, which is a clear indicator of his whiteness. The fact that his whiteness is not explicitly pointed out but implicitly conveyed through his use of language (and also through his high social position in the firm) illustrates how white speakers and their speech are marked as the ‘default’, whereas the Black speakers are constructed as the ‘other’.

\begin{ipquote}
\begin{center}
\textstyleStrong{Too Hasty.}
\end{center}
Jumping at conclusions often results in embarrassment to all concerned. Perhaps nobody knows this better than a certain clerk in a Court street law office. He was sitting at his desk, writing busily, the other afternoon, when the door opened and an old negro put in his head.

“Say, cap’n, don’t yoh want yoh windows washed? Dey looks kinder like dey needs it.”

“No, not today; they were washed only last week.”

“Bettah lemme touch ’em up a leetle, cap’n.”

“No, no,” replied the clerk, going on with his work; “come around in a couple of weeks.”

With another intimation that the windows were susceptible of considerable improvement, the ancient cleaner withdrew.

But window washers were evidently on full force that day; for five minutes had not elapsed when the door opened again and another sable head popped in.

“Good aftahnoon, boss.”

“Good afternoon.”

“Want yoh windows cleaned?”

“No, not today.”

“Do it cheap, boss; bettah lemme clean ’em.”

“No; just engaged a man.”

 Three minutes later the door opened again, and a dark face surmounted by a rather rusty hat peeped in.

“Well?” asked the clerk, looking hastily up, “I suppose you want to wash windows, too, don’t you?”

It was difficult to tell whose surprise was greater, the newcomer’s or the clerk’s, when the senior member of the firm hastened forward from his room, and grasping the stranger affectionately by the hand exclaimed, “Why, my dear old father, this is the most delightful surprise of my life!”—Youth’s Companion.
\end{ipquote}

Apart from anecdotes and short humorous paragraphs, there are other text types which link Black Americans to non-rhoticity and other linguistic forms and, in doing so, emphasize several social characteristics related to being Black and contribute to the construction of several stereotypes of Black Americans. At first, I will discuss two examples of reports which focus on a specific Black figure: the Black preacher.

The first report that I will analyze here is entitled “A Negro Revival”. It was published on May 4, 1875\citesource{May41875}, in the \emph{Georgia Weekly Telegraph} (Macon, Georgia). The author describes a religious meeting of a group of Black people and focuses on the characteristics of the Black preacher and his interaction with his audience. He first addresses the preacher’s outward appearance by calling him “shining” and using the image of a “glossy [...] varnished beaver” for comparison. This already creates the impression that the preacher is very interested in impressing his audience (he wants to shine), but that he is also a slick character whose inner qualities and competence do not hold up to the shiny outward appearance. This impression is supported further by the author’s depiction of the preacher’s sermon, which despite some exaggerations also seems to aim at giving a realistic impression of the preacher’s rhetorical style. He presents it as an almost theatrical performance by describing the preacher’s quality of voice as “low and reverential” in the beginning but changing over time, when he starts a “wailing chant, with a prolonged sound in a higher key on emphatic words and syllables”. It is thus on the level of intonation that the preacher and the audience are described to connect: They “unite with the preacher in a piteous moan, between words, gliding down from the dominant note to the minor third below, and dying through diminuendos into sobs and sighs”. This description creates the impression of an extremely emotional, but also unstable person, who has a strong effect on his audience, but not in that he provides strong, reliable guidance, but more in that he creates strong emotions. This impression fits the explicit characterization of the preacher in the sub-heading: He is labeled “A Colored Moody”. The author addresses other linguistic aspects as well and connects them to an explicit evaluation: He quotes the preacher as saying “Thou Knoweth” and “You knows” and calls the use of this agreement pattern “indulging in doubtful grammar”. Furthermore, the use of “Thou” and “You” to address the Deity is regarded critically by the author because the two forms are used “indiscriminately”. Consequently, in both cases, the linguistic competence of the preacher is questioned because even though he uses forms which are found in the bible and thus associated with formal and old-fashioned religious language (\emph{thou} and \emph{knoweth}), he is not consistent in the use of old and modern forms and chooses an agreement pattern in both cases that is evaluated as incorrect. Next to the description and evaluation of the preacher and the meeting in general, the author also includes a direct quotation of the preacher’s sermon, which he calls an “exhortation”, thereby constructing the voice of the preacher on several linguistic levels.

The sub-heading already contains two direct quotations which highlight important phonological forms: voiced interdental and labiodental fricative stopping (\emph{wid} ‘with’, \emph{de} ‘the’, \emph{dan} ‘than’, \emph{debble} ‘devil’), alveolar -\emph{ing} (\emph{foolin}) and non-rhoticity (\emph{whispah} ‘whisper’). However, it also shows that non-rhoticity is not consistently marked because \emph{Lord}, \emph{better} and \emph{holler} are not represented as being pronounced without post-vocalic /r/. While \emph{Lord} is also not represented as non-rhotic throughout the direct quotation, \emph{bettah} is marked as non-rhotic there (\emph{you bettah stay away}). The other phonological forms represented in the long direct quotation are initial unstressed syllable deletion (\emph{foah} ‘before’, \emph{thout} ‘without’, \emph{roun} ‘around’, \emph{’tirely} ‘entirely’), final consonant cluster reduction (\emph{bes’} ‘best’) and syllable reduction (\emph{foh’rd} ‘forward’), which are all forms that I have shown to be represented as part of Black speakers’ linguistic repertoire in later articles as well. Two forms which were not present in those articles are a lower \textsc{kit} vowel in \emph{if} (spelled <ef>) and the fronting of the voiceless interdental fricative in \emph{fru} ‘through’.

On a grammatical level, differences in subject-verb agreement are most noticeable. The second-person plural is marked by the inflectional suffix -\emph{s} and this is highlighted through the use of parallelism. Several short clauses starting with \emph{you} and followed by a verb occur right next to each other:

\begin{ipquote}
You come foah you’s ready. You starts too soon. You don’t repent; you’s no mounah. Your foolin’ wid de Lord. You comes struttin’ up to de altah; you flops down on your knees, an’ you peeps fru you fingahs dis way, an’ you cocks up you eahs to see who’s makin’ de bes’ pray’r. You’s ’tirely too peart for peniten’s. You’s no mounahs.
\end{ipquote}


The section not only draws attention to the inflectional second-person plural -\emph{s}, but it also highlights inconsistencies. \emph{You come} and \emph{you comes} both occur without any indication why the marking with -\emph{s} is absent in the first case. With regard to the second-person plural form of \emph{be}, it is usually \emph{is} (\emph{you’s ready}, \emph{you’s no mounah},\emph{ you’s ’tirely too peart}), but in one case it is also \emph{are} (\emph{your foolin’}). Even though \emph{be} is an auxiliary verb in the second case and not a main verb, it is not clear whether this is a pattern causing the difference in inflection. \emph{You don’t} in \emph{you don’t repent} also violates the pattern established in the rest of the speech, as the usual marking by an -\emph{s} is absent here as well. As in the first two articles analyzed here (“Health of Chillicothe” and “Rich”), the first-person singular is marked by -\emph{s} (\emph{I knows}). All in all, the deviating pattern of subject-verb agreement, emphasized through the repeated use of short clauses containing a second-person plural subject and verb and combined with inconsistencies in the pattern, adds to the impression of linguistic incompetence. A further strategy to underline this impression is the use of eye dialect in \emph{wan’t} ‘want’, \emph{peniten’s} ‘penitents’ and \emph{your} ‘you’re’. A form that can be observed here but which is not present in the articles above is \emph{a}{}-prefixing (\emph{a}+verb+\emph{ing}, see \citealt[96]{CukorAvila2001}), which occurs in the clause \emph{you sinnahs comes foh’rd an’ holds your head too high a-comin’}. As it never occurs in the narrator’s speech in any of the articles analyzed here, nor in the speech of the (white) people interacting with Black people, it comes to be associated with Black speech in this article. A grammatical form that links this article to the anecdote “Too Hasty” is the \textsc{better} construction: \emph{bettah} is used by the preacher as part of a modal construction to give advice to his audience: “you bettah stay away”.


The construction of the speaker as linguistically not competent and therefore also not educated is underlined further by the explicit description of his sermon as “less eloquent”, but “certainly very practical”. The use of a tool metaphor to describe the preacher’s speech underlines this aspect of practicality, and additionally creates an image of directness and forcefulness: “The preacher struck nails square on the head as he hammered away”. This is in stark contrast to eloquence and refinement and to an argumentation that is persuasive because of its rhetorical strength and not because of its simple and repetitive structure, emotionally appealing intonation and direct appeals. The repeated criticism leveled by the preacher at the Blacks attending the meeting, namely that they are not ready because they are “too peart” and “no mounahs” adds to the negative impression created of Black people in general. The reproach that they constantly look for models because they do not know how to pray paints a picture of Black people being insecure in religious matters (“you peeps fru you fingahs dis way, an’ you cocks up you eahs to see who’s makin’ de bes’ pray’r”) and also of being insincere because instead of establishing their own individual connection with God they appear to be mainly interested in seeing what their neighbors are doing and in comparing the quality of the prayers. The message that is conveyed by the report is that Black people might be interested in religion, but that they are unable to practice it and that they are insincere because their preachers value loud displays of emotions and a shiny appearance over quiet and sincere repentance.


\newpage
\begin{ipquote}
\begin{center}
\textstyleStrong{A NEGRO REVIVAL}\\
\textstyleStrong{A Colored Moody who Wants “No Foolin wid de Lord”—“Better Whispah to de Lord dan Holler at de Debble.”}
\end{center}
Mississippi Corr. [?] Commercial

We must give the reader a few specimens of a prayer and exhortation we heard in a revival meeting among the colored folks. A shining black preacher, glossy as a varnished beaver, gave us a characteristic article in this line. Beginning his prayer in a low and reverential voice, he addressed the Deity as “Thou” and “You” indiscriminately, and sometimes indulging in the doubtful grammar of “Thou Knoweth” and “You knows.” Soon his words were uttered as a kind of wailing chant, with a prolonged sound in a higher key on emphatic words and syllables. The peculiar intonation, especially when the congregation would catch the key from the plaintive sounds, and unite with the preacher in a piteous moan, between words, gliding down from the dominant note to the minor third below, and dying through diminuendos into sobs and sighs. The effect was at times thrilling. Some parts of an exhortation to which we listened, however, while less eloquent, were certainly very practical. The preacher struck nails square on the head as he hammered away. For instance:

“Now, brethren and sisters, we wan’t mounahs heah to-night. No foolin’. Ef you can’t mouhn for your sins, don’t come foolin’ roun’ dis altah. I knows ye. You’s tryin’ mighty ha’hd to be convarted ‘thout bein’ hurt. The Lord ’spises mockery. Sometimes you sinnahs comes foh’rd an’ holds your head too high a-comin’. You come foah you’s ready. You starts too soon. You don’t repent; you’s no mounah. Your foolin’ wid de Lord. You comes struttin’ up to de altah; you flops down on your knees, an’ you peeps fru you fingahs dis way, an’ you cocks up you eahs to see who’s makin’ de bes’ pray’r. You’s ’tirely too peart for peniten’s. You’s no mounahs. Ef you comes heah to fool, you bettah stay away. […]”
\end{ipquote}

The second example of a report focusing on a Black preacher figure was published on July 4, 1885\citesource{July41885}, so ten years later than the article “A Negro Revival”, under the heading “An Old Time ‘Fo’th’” in the \emph{Rocky Mountain News} (Denver, Colorado). As the title indicates, the author of the article reports on the Fourth of July celebrations by Black people and in this context also discusses and compares the patriotism of Black and white Americans. From the beginning, the superiority of white Americans is firmly established: Black men are described as imitators of white men and the comment that Black men are “naturally close observers” attributes their reliance on whites for guidance to their nature and thus to something that cannot change. Furthermore, Black men are described as being less patriotic because they care more about “powder, fire-crackers and yells” than about remembering the actual events of fighting (for example the battle of Bunker Hill). The comparison of Black men to small boys makes them seem childish, and the author points out several times that after the Civil War, Black people started celebrating “whenever an opportunity presented itself”, which characterizes them as rather lazy and more interested in entertainment and other pleasures like eating good food than in contributing to the progress of the nation. Against this background, the figure of the Black preacher is described in more detail: He is called “Uncle Ephraim” and characterized by the author as “ignorant as a field hand” but nevertheless “shrewd and cunning”. Furthermore, he is presented as a person who wants to appear eloquent and dignified, like “a Roman senator”. However, the direct quotation of the speech that he gives, in which he “utter[s] his patriotic thoughts mixed with religious sentiments”, is supposed to ridicule his efforts and prove that he is in fact neither eloquent nor able to speak in a dignified manner.

The most salient linguistic form is again non-rhoticity, for the reason that it is already represented in the heading, where the word \emph{fo’th} ‘fourth’ is separated from the rest of the heading through quotation marks to indicate a change of voice. As the text is about the Fourth of July celebrations and the (lack of) patriotism of Black people, this word prominently links non-rhoticity to the topic of the article and the group of people that are indexically linked to the form. However, non-rhoticity is again not represented consistently and several forms keep the <r> in the spelling (e.g.\emph{’marked}, \emph{fadder}, \emph{wedder}). Another phonological form which is particularly emphasized in this article is the deletion of initial unstressed syllables: The Black preacher says \emph{’dependence} instead of \emph{independence} and therefore changes the meaning of the word to its opposite, so that he unintentionally creates humor when he says “To-day [...] am de declamation of ’dependence”. This very sentence also highlights a grammatical difference (third-person singular \emph{am}) and a lexical difference: the use of \emph{declamation} instead of \emph{declaration}. The words differ in only one sound, which invites the likely interpretation that the preacher mixed up the two words. This mix-up signals the linguistic insecurity of the speaker when it comes to Latinate words, which are associated with linguistic and rhetorical competence. The preacher’s use of \emph{unneccessarious} instead of \emph{unnecessary} and \emph{dissolutionary} instead of \emph{revolutionary} underlines this impression of a lack of rhetorical skill and therefore creates humor through the contrast between the preacher’s intention to appear eloquent and his failure to achieve it. Apart from these linguistic forms, voiced fricative stopping is also represented consistently in his speech (e.g. in \emph{fadders}, \emph{de} and \emph{ob}) and in the song sung by everyone at the very end, final consonant cluster reduction and alveolar -\emph{ing} are highlighted because they occur at the end of the lines (\emph{eas’}, \emph{wes’}, \emph{mo’nin’}). There are several further grammatical forms marked in his speech: the demonstrative \emph{them} (\emph{dem heroes}), the past tense form of \emph{fight}, which is once \emph{fought} and once \emph{fit} (indicating not only a different form, but also inconsistency in use), and the past-participle of \emph{frost-bite}, which is \emph{frost-bited} instead of \emph{frost-bitten}. In the fifth line of the song sung in the end, an invariant \emph{be} (derived from \emph{will/would} deletion) occurs as well.

The final comment in the last paragraph emphasizes again the thoroughly negative attitude towards Black people: Their “scramble for the lunch baskets” makes them appear hungry like animals and their fighting also characterizes them as uncivilized. That the preacher himself was involved in the fighting is the final step in destroying the picture of eloquence and dignity that he was anxious to portray and in providing “proof” of Black inferiority.

\begin{ipquote}
\begin{center}
AN OLD TIME “FO’TH”\\
{[…]}
\end{center}

Whatever a white man do{\kern0pt}es a colored man must do. They are naturally close observers, and, being good imitators, never fail to do what they have seen done. It may be that the fires of patriotism do not burn so brightly in their breasts as in the bosoms of white American citizens. They may not grow as enthusiastic as the New Englander when the battle of Bunker Hill is alluded to, but the average colored citizen takes as great delight in a Fourth of July celebration as the small boy, who connects the occasion with powder, fire-crackers and yells.

{In the years immediately succeeding the late unpleasantness the negro{\kern0pt}es, recognizing that they were freemen

\centering
\textsc{began to celebrate the fourth,}

a day which there was never much fuss made over in the South. Christmas time being the holidays of holidays for whites as well as blacks. But after the war things were changed. The negro felt it his bounden duty to celebrate whenever an opportunity presented itself and celebrate he did. The Fourth in a great measure took the place of the barbecue, a place where the dusky brother was wont to disport himself and feast on “eatins’ as good as de white folks.”}

Then there was the colored preacher who, always anxious to display his eloquence and air his oratory, made the Fourth an occasion to utter his patriotic thoughts mixed with his religious sentiments, and a Fourth of July celebration in the hands of the colored people was generally a cross between a camp meeting and a corn shucking.

{[…] The orator of the county, the bright particular star in the black firmament, was Ephraim Jenkins,

\centering
\textsc{“Uncle Ephraim,” as he was called.}

“Uncle Ephraim” was a portly man, black as ebony and as ignorant as a field hand.} {But he was shrewd and cunning, had a retentive memory, a facile command of language, not the most choice, and could talk, or rather rhapsodize, for a day at a stretch.

\centering
{[…]}\\
\textsc{The orator of the day}

arose with all of the dignity of a Roman senator. “To-day,” he said, “am de declamation of ’dependence.” After having made that remark, he paused to see what effect the startling announcement had created, and during the pause, wiped the perspiration form his shiny brow. He then resumed: As I ’marked befoah, dis am de declamation of ’dependence and our fadders—yo’ fadders as well as my fadders—fought in de dissolutionary wah and marked de snow and ice of Valley Fo’ge wid blood. Some of my heah has had de same ’sperience in de wintah, de ground froze in de co’n field, de to{\kern0pt}es out’n de sho{\kern0pt}es and run down at de heel. Some ob yo’ I spec is frost bited yet.}

{It am unneccessarious fo’ me to tell de history ob dem hero{\kern0pt}es who fit at Bunkum hill and died, yes, died, fo’ me an’ fo’ yo’, like de blessed Lamb.

I’d like for de audience to sing befoh we ’sperse for ‘freshments “You may bury me in de eas’.”

\centering
{[…]}


“You may bury me in de eas’,\\
You may bury me in de wes’,\\
But I’ll heah dat trumpet soun’ in de mo’nin’.\\
My eahs may change to clay,\\
An’ my tongue be waste away,\\
But I’ll answer dat trumpet in de mo’nin,\\
In de mo’nin’, in de mo’nin’, in de mo’nin’ ob de Lawd.\\
An’ we’ll all be togedder in de mo’nin’.”


When the song had been sung the audience was dismissed by Brother Euripides Shands, a local light, and a scramble for the lunch baskets ensued. A portion of the day was spent as became patriots, but the balance of it was devoted to fighting, and “Uncle Ephraim,” the orator, was one of a number who were slashed with razors. He was cut by the Rev. Euripides Shands in a dispute over a sister.}
\end{ipquote}




A further text type linking Black Americans to a specific linguistic repertoire is the short story. The example I will discuss here is entitled “Dat Deceptious Mule”. It was written by A. T. Worden, a white author, and published on September 21, 1895\citesource{WordenSeptember211895} in the \emph{Yenowine’s News} (Milwaukee, Wisconsin). The difference to the anecdotes and the reports discussed above is that no claim is made that the story is based on real events or that it is to be regarded as a typical incident of events or people occurring in real life. Nevertheless, there are also striking similarities to the anecdote because the story consists almost entirely of direct speech, quoting the conversation between two Black men, one wanting to sell a mule and the other looking to buy one. For this reason, almost the whole text represents the two Black voices, interrupted only very few times by the narrator. Due to this large absence of a narrator, the two Black men are not explicitly described. However, they are characterized implicitly based on what they are saying and they are also portrayed in an accompanying illustration (\figref{fig:key:25}).
\begin{figure}
\includegraphics[width=.8\textwidth]{figures/Paulsen-img25.png}
\caption{
Two Black men, the seller of the mule and Elder Pokeberry, in the short story “Dat Deceptious Mule”, published in the \emph{Yenowine’s News} (Milwaukee, Wisconsin) on September 21, 1895\citesource{WordenSeptember211895}, retrieved from Nineteenth-Century U.S. Newspapers
}
\label{fig:key:25}
\end{figure}
The seller is presented as very optimistic and good-natured. He praises the mule in every possible way. He addresses the potential buyer as “Eldah Pokeberry”, which suggests that the man is older and has a social position of authority. The illustration confirms this: The seller is rather young and the way that he leans on the mule to his left and on a stick to his right suggests a laid-back and confident attitude. He is wearing practical, comfortable clothes and a cowboy hat. The Elder, on the contrary, is standing upright, folding his hands in front of him and holding a cane and a briefcase under his arm, which makes him seem genuinely interested and ready for doing business. He wears a frock coat and a top hat and therefore appears more elegant than the seller. Despite these differences, however, the striking similarity between the figures is the way that their ethnicity is caricatured: Their skin color is black and their facial features, big and protruding eyes, ears and lips, are exaggerated to an extent that it makes them subject to ridicule and derision. This fits the story, which also mocks both Black figures. The first part is quoted below and shows how the seller praises the mule to the Elder in a greatly exaggerated manner. He puts special emphasis on the friendliness of the mule, which supposedly makes it an ideal family mule. However, in the second part of the story, the mule suddenly starts kicking aggressively and proves to be the complete opposite of the description of the seller. The seller explains the behavior of the mule by saying that it “understan’s talk” and that it was set off by the Elder’s remark that “he kin cyarry double”. This unrealistic explanation makes it clear that the seller lied from the beginning about the characteristics of the mule to get the Elder to buy it. In the end, the seller changes his mind about selling the mule and says “Yes sir, good-day. You can’t buy dat mule. I’ll keep him in de fambly”. It is clear, however, that he just tries to save face because it is unlikely that the Elder is still interested in buying the mule after seeing how aggressively it behaves. The Black seller is therefore characterized as someone who is cunning and who cannot be trusted because his optimistic and good-natured manner is only put on to achieve a goal. Ultimately, it is not the mule that is deceptive, but it is the Black seller of the mule, which reinforces negative stereotypes about Black people.





On a linguistic level, the Black men’s voices are represented as very similar and the sheer number of forms marking their speech as different is again striking. Non-rhoticity is highlighted as a phonological form in the text because it is one of the few forms that are contrasted in the text with the ‘neutral’ voice of the narrator (\emph{Eldah} vs. \emph{Elder}).\footnote{In the other cases, the comparison is implicit, which means that the author of the story assumes that readers compare the linguistic repertoire to their own repertoire or that of other articles in the newspaper.} However, as in other articles discussed above, it is not represented consistently (it is not marked e.g. in \emph{dar} ‘there’, \emph{burnt} and \emph{air}). There are also instances of hyper-rhoticity in function words, e.g. in \emph{ter} ‘to’, \emph{er} ‘a’, \emph{der} ‘the’ and in the content word \emph{sorrer} ‘sorrow’ (in the second part of the story). Cases of elision are frequent as well: of final consonants (\emph{da} ‘that’, \emph{o’} ‘of’, \emph{le’} ‘let’), of initial unstressed syllables (\emph{’count} ‘account’, \emph{spose} ‘suppose’). Final consonant clusters are reduced, as in \emph{understan’}, \emph{wouldn’}, \emph{behine}, \emph{stan’}, \emph{tole} and \emph{jess} ‘just’. Stopping of voiced interdental and labiodental fricatives is marked, as is fronting of voiceless interdental fricatives (\emph{da}, \emph{dar}, \emph{de}, \emph{dis}, \emph{dem}, \emph{wid}, \emph{dat}, \emph{dey}, \emph{ob}, \emph{ober}, \emph{mouf}, \emph{fink}). Final -\emph{ing} is realized by [ɪn] (as e.g. in \emph{zoonin’}, \emph{standin’}, \emph{maunin’}, \emph{durin’}, \emph{alludin’}). The \textsc{dress} vowel is raised in \emph{git}, as in representations of Black speech in articles discussed above, but in this story the vowel is also raised in \emph{Giniral}. Three vowel forms which have not been marked in articles above are realizations of \textsc{choice} with a lower and fronter onset (\emph{appintments}, \emph{pint}), of \textsc{square} with a lower and backer onset or even a low monophthong (\emph{dar} ‘there’, \emph{har} ‘hair’) and of \textsc{price} with a monophthong, although this last form is restricted to the pronoun \emph{ma} ‘my’. A consonantal process that has not been represented in the articles discussed so far is the insertion of a palatal glide after /k/ in \emph{cyarry}.

Regarding grammatical forms, there are numerous differences as well. With regard to subject-verb agreement, the third-person singular form of \emph{be} is sometimes \emph{am} (\emph{dis am a fambly mule}), sometimes \emph{are} (\emph{dis ar de mule}) and sometimes \emph{is} (\emph{He’s lonesome, dat mule is}). The suffix -\emph{s} is used to mark first-person and second-person singular (\emph{I calls}, \emph{yo’ falls}) and third-person plural (\emph{De chillun cracks}). Third-person singular is not marked, e.g. in \emph{dat mule stan’ an’ sleep}. Past-tense forms are sometimes marked by inflection and sometimes they are not, e.g. in \emph{Dar was de Stiggins’ mule dat kick so fast dat de friction of his legs burnt de hair off an’ made de air wam all aroun’ dar}. There is a bare infinitive construction (\emph{dat tried kick}), sometimes there is no \textsc{do}{}-support (\emph{How much you want}), adverbs are not marked (\emph{win de money easy}), demonstrative \emph{them} (\emph{dem cohns}) and \emph{this here} (\emph{disher mule}) can be found, the negator \emph{ain’t} is used (\emph{I ain’t gwine to sell}) and negative concord occurs, that is the occurrence of multiple negative elements as in \emph{dey couldn’ git nobody to hold de drum}. Moreover, the genitive construction \emph{dat one ob Lawson’s} is used and there is an instance of perfective \emph{done} (\emph{you done heard about}). Another interesting grammatical form is the use of periphrastic \emph{do} in an affirmative sentence where it does not seem to be used for emphasis: \emph{I des tell yo’}.\footnote{\citet{Schneider2015} finds “the use of periphrastic \emph{did} with past time reference, in affirmative structures where neither an emphatic nor a habitual reading may apply” in BLUR (The Blues Lyrics Collected at the University of Regensburg), a corpus of African American Southern English containing blues lyrics from 1920 to 1969. He analyzes it as “an analytic, preverbal tense marker” and emphasizes the need for further research on this structure. In the structure \emph{I des tell yo’}, found in the present short story, it is possible that \emph{does} also functions as an analytic marker of present tense. However, as it occurs only once in the story, it is difficult to identify a pattern here.} Finally, the Black seller sometimes uses reduced sentences, for example \emph{Kick, not him}, which consists only of a verb and the apposition \emph{not him} to give information about the agent of the activity denoted by the verb. Reduced sentences like this evoke the value of simplicity, so that the speaker appears naive.

The combination of phonological and grammatical forms creates the impression of the Black speakers having a linguistic repertoire that deviates enormously from that of white speakers. The two Black speakers are, however, also marked as different from each other on a lexical level: The Elder Pokeberry uses words of Latinate origin and rather infrequent (\emph{curb}, \emph{volubility}) and formulates rather complicated and long phrases (\emph{befo’ yo’} \emph{falls into de habit ob exaggeration}). This creates the impression that he aims to appear educated and that he wishes to emphasize his high social status as an Elder. However, as other phonological and grammatical “deviations” are present in his speech as well, his attempt has a rather comic effect because of the apparent incongruity between the lexical level and the other levels of speech. This effect can be illustrated by considering just one sentence: \emph{yo’ bettah curb de volubility ob yo’ imagination befo’ yo’ falls into de habit ob exaggeration an’ slops ober} combines phonological forms (non-rhoticity, fricative stopping) and grammatical forms (second-person singular -\emph{s} and the \textsc{better} construction that also occurred in “Too Hasty” and “A Negro Revival”), which are associated with uneducated Black people, with infrequent lexical items of Latinate origin, which are associated with highly educated people. This particular combination also links the figure of the Elder in this story to the figure of the Black preacher in the report “An Old Time Fo’th” because they both occupy high social positions in the Black communities and try to adapt to this position linguistically by using specific lexical items. Ridiculing this by showing how the presence of other linguistic deviations makes this attempt absurd and impossible establishes a relationship of superiority of the white authors to the Black figures that they describe or invent.\footnote{In “An Old Time Fo’th”, the author is not explicitly named, but the text leaves no doubt that the perspective taken on Black people is not an inside but an outside one – formulations like “The negro felt it his bounden duty to celebrate [...]” show how the author distances himself from the group he describes. The author of this short story is A. T. Worden, a white man who lived in New York and had a career as a writer and as a Baptist minister. He is also the author of several poems using racist imagery and stereotypes \citep[62--63]{Young2008}.}

\begin{ipquote}
\begin{center}
\textstyleStrong{Dat Deceptious Mule.}
\end{center}
“I understan’ yo’ Eldah Pokeberry da yo’ wanter buy a nice peaceable mule ter ride to yo’ appintments an’ git dar on time an’ all in one piece. Dis ar de mule wot you bin zoonin’ for. I raised disher mule in ma own fambly. Dis am a fambly mule. Buttah wouldn’ melt in dis mule’s mouf. I calls him Giniral Grant ’count o’him being so quiet. Kick, not him. I des tell yo’ Eldah he had cohns on his behine feet from standin’ still so long an’ I used ter cut dem cohns ever maunin’ wid ma jack-knife an’ dat mule stan’ an’ sleep all de durin’ time. Dat wot I call a gentle mule, mon.”

“Do{\kern0pt}es you fink dat mule would be kind to de chillum?” asked the Elder.

“Now you’s alludin’ to de strong pint in dat mule. De chillun cracks pecan nuts on dat mule’s heels wid a brick. He’s lonesome, dat mule is, onless he got some chillun aroun’. Dar was de Stiggins’ mule dat kick so fast dat de friction of his legs burnt de hair off an’ made de air wam all aroun’ dar. He would go out in de lot an’ practice all alone an’ Stiggins he made a bet dat de mule could kick de double drag on er base drum to the time of De White Cockade but dey couldn’ git nobody to hold de drum else de mule win der money easy.”

“Ma frien’,” said the Elder slowly, “yo’ bettah curb de volubility ob yo’ imagination befo’ yo’ falls into de habit ob exaggeration an’ slops ober.” {[...]}
\end{ipquote}

All in all, this example shows that the short story as a text type is similar to the text types analyzed above because it also contains direct representations of speech and thus creates social personae whose behaviors and appearances are linked to a specific set of linguistic forms. It is a particular characteristic of the short story “Dat Deceptious Mule” that it consists almost completely of dialogue and that the narrative voice is reduced to indicating who is speaking (“said the Elder”). The events of the story are instead told through the dialogues and illustrated by the accompanying visual elements and onomatopoeic words (“Rip! ping!! splum!!! bang!!!!”). The dialogic character in particular creates a similarity to the text type of the anecdote. However, the short story is also different because it is marked as entirely fictional: It does not claim to describe encounters between speakers that really happened (as in anecdotes) or to give realistic accounts of speakers (as in reports citing actual speakers). Whereas this reduces the authenticity that readers are invited to attribute to the representation, it increases the readers’ readiness to accept exaggerations or unrealistic depictions (like the mule understanding human speech). Due to its greater length, the short story also has more room to create a fictional world and a story that happens within this world, which allows for more detailed characterizations and much longer representations of speech.

\begin{figure}[b]
\includegraphics[width=.8\textwidth]{figures/Paulsen-img26.png}
\caption{
A cartoon depicting two stereotypical Black figures and a mule, published in the \emph{Milwaukee Journal} (Milwaukee, Wisconsin) on February 12, 1897\citesource{February121897}, retrieved from Nineteenth-Century U.S. Newspapers
}
\label{fig:key:26}
\end{figure}


An article type which rests more heavily on illustrations than on text is the cartoon. I will discuss two examples here which illustrate how visual elements are linked to linguistic forms which are constructed as part of Black speakers’ voices. The first example is headed “Dangerous” and it was published in the \emph{Milwaukee Journal} (Wisconsin) on February 12, 1897\citesource{February121897} (see \figref{fig:key:26}). The heading, which is printed in big capital letters, raises the expectation that something dangerous is depicted below. This has the effect that the reader is invited to associate the two Black people in the illustration with the characteristic of being dangerous and it shows that the cartoon plays on the stereotype of the “brute”, a “barbaric black out to raise havoc” \citep[13]{Bogle1990}. The dialogue underneath the illustration, however, makes it clear that it is in fact the mule, which is rather in the background behind the left figure, that the adjective \emph{dangerous} relates to. The owner of the mule, Erastus, asks the woman, Mrs. Johnsing, what she thinks about the mule that he had just clipped. She answers that it is dangerous to clip a mule in winter time because the mule is likely to get kleptomania. The pun is that while it makes sense that Mrs. Johnsing assumes that the mule might develop an illness because it could get too cold in winter with its clipped mane, she uses the term \emph{kleptomania} for the possible illness because the phonological form of the word is reminiscent of the phrase \emph{clipped mane}. As kleptomania is in fact a psychological condition in which people have the impulse to steal items and cannot resist it, Mrs. Johnsing’s unintended suggestion that the mule could develop this condition leads to the absurd image of a mule being dangerous because it is likely to become a thief. After reading the dialogue, it therefore becomes clear to the reader that the heading “Dangerous” mocks the word choice by the Black figure and draws attention to the inability of Black speakers to use specialist vocabulary.

The ridicule of Black Americans is also underlined by the visual elements: As in the illustration accompanying the short story “Dat Deceptious Mule”, the depiction of the Black speakers focuses on their physiognomy and highlights the color of their skin and especially the eyes and the lips, which are so big that they appear unnatural and make the two Black people appear less human. The female Black person, Mrs. Johnsing, has a round figure and her simple dress and head scarf are practical clothes, which, in combination with the basket and her worry about the mule getting sick, clearly reference the mammy stereotype again. The man is round and stout and small and his white hair suggests that he is rather old. His clothes are simple as well, with patches on his pants indicating that he is not particularly wealthy.

The mocking portrayal of the Blacks as simple-minded is supported by the representation of their voices in direct quotations, which exhibits several typical features associated with Black speech. Non-rhoticity is marked in \emph{mawning}, \emph{bettah}, \emph{wintah}, \emph{fus’} and the form \emph{bettah} links non-rhoticity to the grammatical level again because as in three of the articles analyzed above it occurs in a modal construction used to give advice (\textit{Bettah look out}). Final consonant cluster reduction (\emph{fus’}), fronting of the voiceless interdental fricative (\emph{fink}, \emph{fing}), stopping of voiced labiodental fricatives (\emph{ob}), a raised \textsc{dress} vowel in \emph{get} (\emph{git}) are represented as well, but the phonological form emphasized the most in this cartoon is the use of alveolar \emph{-ing}. Mrs. Johnsing’s speech exhibits this form in \emph{clippin’}, but there is additional attention drawn to the form by giving her the name Mrs. Johnsing, which indicates a hypercorrect form: The final /ən/ in \emph{Johnson} is reinterpreted as a realization of an -\emph{ing} suffix and consequently changed to the supposedly more correct realization /ɪŋ/. This adds to the implicit characterization and mockery of the Black woman as trying to appear more educated than she actually is: She not only fails to use a specialist term for a sickness (a Greek loanword associated with a high degree of education), but she is also shown to fail in her attempt to use a ‘correct’ phonological form.

A prominent element of the story is therefore the mule – it also creates an intertextual relation between this cartoon and the short story “Dat Deceptious Mule”. \citet[332]{Ferris2009} writes:

\begin{quote}
The lives of African American workers and the mule are intimately linked in every period of southern history. As slaves before the Civil War and as tenant farmers afterward, blacks worked with mules; the ubiquitous team of mule and African American driver was essential to the southern economy.
\end{quote}


In the short story, this close relationship is used to create the impression of a similarity in character traits between the mule and Black men. The main twist of the story is that the animal has the human characteristic of understanding language and that it has feelings which can be hurt so that it starts kicking when Elder Pokeberry suggests that it can carry double. By attributing human characteristics to the mule, it is implicitly suggested that characteristics of the mule can apply to humans as well, more particularly to Black men, with whom they are in a close relationship. In the story, the mule is described as sturdy, patient and gentle, but it turns out that it has an aggressive character as well. In the cartoon “Dangerous”, the suggestion that the mule can develop kleptomania might be unintended and absurd, but if characteristics of the mule are understood as applying to Black men as well, it can be read as an implied suggestion by the creator of the cartoon that Black people are likely to become thieves, especially when they are stripped of their resources (a metaphorical reading of the clipping of the mane). Both the cartoon and the short story therefore contribute to the construction of a very negative image of Black Americans and to a process of othering, in which they are constructed as physiologically, psychologically, socially and not the least linguistically markedly different from white people.



\begin{figure}[b]
\includegraphics[width=.8\textwidth]{figures/Paulsen-img27.png}
\caption{
A cartoon depicting a Black and a white boy with their dogs, published in the \emph{Philadelphia Inquirer} (Philadelphia, Pennsylvania) on August 13, 1899\citesource{August131899}, retrieved from America's Historical Newspapers
}
\label{fig:key:27}
\end{figure}

The second example of a cartoon linking \emph{bettah} to Black Americans was published in the \emph{Philadelphia Inquirer} (Pennsylvania) on August 13, 1899\citesource{August131899} (see \figref{fig:key:27}). The cartoon does not have a heading and the caption under the illustration is very small, which has the effect of foregrounding the visual element because the readers’ attention is drawn to the illustration first, simply because it requires some effort to read the small print of the text underneath. The illustration shows a Black boy and a white boy who encounter each other while they walk their dogs in a rural area. What is most noticeable is the difference in size between the boys, but especially between the dogs: The Black boy is smaller than the white boy, and while the white boy’s dog is as big as the boy himself, the Black boy’s dog is tiny and thin. The second striking difference is a difference in posture: The white boy and his dog stand still and upright, while the Black boy’s dog is moving towards them and the Black boy is trying to hold him back, with his big eyes and slightly open mouth indicating that he is scared. A difference in social status in indicated through their difference in dress. The white boy is dressed more elegantly than the Black boy: He is wearing well-fitted knee-pants and black stockings and his hat is fancier than the practical and simple hat worn by the Black boy. The impression created by these visual elements only is that the Black boy is intimidated by the white boy and his dog and tries to get away. However, the humor of the cartoon is created by the textual element, which contradicts the impression created by the image. The text reads: “Yo’ had bettah get away quick, white boy, ’cause I can’t hold dis yer dawg by mine much longer”. As the voice addresses the conversational partner with “white boy”, it is clear that the Black boy is speaking, but contrary to the assumption that he fears the white boy and feels inferior to him, he is now presented as feeling superior and being in a position in which he can protect the white boy. Given the objective difference in size between the dogs, the assumption that the small dog poses a threat to the white boy and his big dog seems ridiculous, so that the cartoon mainly conveys that Black people, embodied here by the Black boy, have a self-confidence that is unjustified given the situation that they are in, and that they try to cover their actual inferiority with boastful talk.


Linguistically, the Black boy’s speech is marked by non-rhoticity in \emph{bettah}, but non-rhoticity is not marked in \emph{yer} and \emph{longer}, suggesting either an inconsistent use or an inconsistent representation of the form. Another phonological form is the insertion of /j/ and dropping of /h/ in \emph{yer} ‘here’ and, as found frequently in the articles above, the voiced interdental fricative is replaced by a stop (\emph{dis}) and the initial unstressed syllable is elided in \emph{’cause}. As in four of the articles analyzed above, \emph{bettah} is used again in a modal construction, but this time the modal is \emph{had better}, which according to \citet[131]{vanderAuwera2013} is the historically older form. Furthermore, there are three grammatical forms which were also present in “Dat Deceptious Mule”: a double demonstrative consisting of a demonstrative determiner and a locational adverb (\emph{dis yer dog}),\footnote{\citet[74]{Diessel1999} writes that such forms are a product of a grammaticalization process in which locational deictics which were used adnominally to intensify a demonstrative determiner (as in \emph{this guy here}) become part of the demonstrative form. Diessel’s examples are not taken from English (varieties), however, but from Afrikaans, Swedish and French. Nevertheless, the Swedish form \emph{det här hus-et} ‘the/this here house-the’ is parallel to the form associated with Black American English in the cartoon analyzed here (\emph{this here dog}).} an unmarked adverb (\emph{quick}) and an unusual genitive construction (the phrase \emph{by mine} is used to mark possession of the preceding noun). All in all, the large number of marked forms in the one sentence spoken by the boy underlines his inferiority to the white boy, who does not even need to reply to the Black boy to assert his superiority.



The two cartoons show how powerfully visual and textual elements interact in the process of stereotyping. The short representations of speech highlight particular linguistic forms and the voices created in this way are linked to images, which also highlight particular aspects of the outer appearance, manner and attitudes of the figures. Furthermore, the visual element draws the readers’ attention to the cartoon and makes it likely that they will read the dialogue underneath. Another reason for the popularity of the cartoon is that readers expected to be entertained by it and that it even appealed to readers who had no interest in (or who were not capable of) reading longer texts.


The last two text types that I will analyze here to show how linguistic forms, among them non-rhoticity, become indexically linked to Black speakers are a poem and a soliloquy (a dramatic monologue that usually occurs in a play but appears as an isolated text in the newspaper). Like anecdotes, short paragraphs, short stories and cartoons, these text types are included in newspapers primarily for entertainment and not for information or critical discussion of current affairs. The soliloquy is entitled “Cato’s Soliloquy” and it appeared as a part of the section “Denver Postscripts” in the \emph{Denver Evening Post} (Colorado) on July 2, 1899\citesource{July21899}. The title indicates an intertextual relationship with Joseph Addison’s tragedy \emph{Cato} (1712), which features a soliloquy in which Cato, the hero, ponders suicide. The soliloquy in Addison’s play in turn contains intertextual links to Shakespeare’s \emph{Hamlet}. In the poem which appeared in the newspaper, the relationship with \emph{Hamlet} is obvious as well because the opening line is almost the same as the one in Hamlet’s soliloquy. The statement “To be, or not to be, that is the question” is turned into “To steal, or not to steal? Dat am de question”. This line is crucial: It not only links the soliloquy to \emph{Hamlet}, but through this link it also establishes a reference point for comparison. On the one hand with regard to content: By replacing the verb \emph{be} with the verb \emph{steal}, the existential question of being, of life or death, is turned into the more profane question of stealing. On the other hand with regard to language: By indicating TH-stopping orthographically in \emph{Dat} and \emph{de} and by changing the third-person singular form of \textsc{be} to \emph{am}, differences in the linguistic repertoire between this soliloquy and Hamlet’s soliloquy are highlighted. Given that form–meaning links between these two linguistic forms and Black people had been circulating widely for more than twenty years, it is likely that these forms in the first line already evoke the voice of a Black speaker even before his ethnicity is made explicit in the fourth line when he talks about “de white folks’ moufs”. The content of the soliloquy expresses a negative view on Black Americans: The Black man ponders the question whether he should steal a watermelon from a patch owned by white people. His main argument against stealing is his conscience, which he regrets having, and he makes religion responsible for it: “’Tis conscience dat makes cowards ob us coons, / Especsh’ly dem dat’s jes’ done jined de church / An’ had deir brack souls washed as white as snow”. Whiteness is associated with innocence and moral and Christian behavior here and the Black man’s evaluation of this behavior as cowardice serves to underline the moral corruption of Black people, the need to educate them, and to introduce them to religion. This undertaking is presented as very difficult, however, as the humor of the soliloquy rests on an argument developed by the Black speaker which justifies stealing the watermelon precisely \emph{because} of religious reasons. He argues that the preacher said that he loved watermelons and since the preacher is “de agent ob de Lawd”, pleasing him by sharing the stolen watermelon with him becomes an act of pleasing God. He therefore “confiscates” the watermelon “in de Mastah’s name” and plans to “invitate” the minister to share the watermelon with him. What is thus constructed here is an inability of Black people to control their impulses and their behavior. Especially the depiction of their craving of food, emphasized through the use of onomatopoetic words (“Yum! Yum! [...] Tunk-tunk-tunk!”) and the verb \emph{devour} in the last line of the soliloquy, is used as a means to characterize them as animalistic and uncivilized. This uncontrollable craving is ultimately presented as the cause for stealing (and not actual hunger). The label used by the Black voice to designate himself and Black people like him is “coons”, which establishes a link to stereotype of the coon popularized at the end of the nineteenth century by so-called “coon songs”. \citet[15]{Cox2011}, in her study on the creation of the south in American popular culture, writes:

\begin{quote}
The term “coon” was not used to describe blacks until the 1880s and can be attributed to the popularity of coon songs, which became a trend in music publishing that lasted through the first decade of the twentieth-century. Whites associated southern blacks with eating raccoons, and in coon songs they also became known as chicken-thieving, watermelon-eating, razor-wielding oafs. These tunes were enormously popular in the decade of the 1890s, not surprisingly during the period of heightened racial violence nationally. In that decade, over 600 coon songs were produced as sheet music and performed in music halls and vaudeville shows around the country. In effect, coon songs expressed American racism and were important to popularizing black stereotypes.
\end{quote}

The Black man speaking in the soliloquy conforms to the stereotype of the coon and it clearly marks him as inferior to whites, whose superiority is also underlined here by the fact that the intertextual reference to Addison’s \emph{Cato} and Shakespeare’s \emph{Hamlet} can only be recognized by educated people who are familiar with British literary classics. The author, who is most likely white, draws on white cultural knowledge to exclude the objects of derision and ridicule from the group of people who can achieve a full understanding of the text and to emphasize the importance of such knowledge as a uniting factor for this group of educated, white Americans.

With regard to the linguistic forms represented as being part of the Black speaker’s repertoire, they are largely similar to the forms found in other articles, e.g. in “Dat Deceptious Mule”. What is again noticeable is the combination of non-rhoticity with the hyper-rhotic form \emph{shadder} ‘shadow’. As in the other articles, non-rhoticity is also not consistently marked here, which is particularly striking in the line \emph{Come to mah a’ms an’ lie again mah heart} because it is marked in \emph{arms} but not in \emph{heart}, which is in close proximity. Two phonological forms which, of all the articles containing \emph{bettah} analyzed so far, have only occurred in “Dat Deceptious Mule”, are the monophthongization of \textsc{price} in \emph{mah} ‘my’ (but not in other lexical items like \emph{moonlight} or \emph{night}) and the lower onset of \textsc{choice} (\emph{jined}). An unusual phonological form, which is not found in any of the articles above, is the replacement of /l/ by /r/ in \emph{brack} ‘black’ and \emph{bressed} ‘blessed’. What is most striking on the lexical level is the phonological form of \emph{melon}, which is represented by \emph{millyun}, a spelling which indicates a raised \textsc{dress} vowel and the insertion of a palatal glide before the unstressed vowel. As the form occurs several times in the soliloquy and as the spelling differs considerably from \emph{melon}, it serves to emphasize the linguistic differences of the Black speaker to white speakers. Another noticeable difference is the malapropism \emph{invitate} ‘invite’, which is reminiscent of malapropisms found in other articles (e.g. \emph{unneccessarious} in “An Old Time Fo’th”) and serves to mark the speaker as uneducated.

On the grammatical level, the soliloquy exhibits several forms next to \emph{am} being highlighted in the first line as a third-person singular form of \textsc{be}. First of all, third-person singular \emph{is} occurs as well (\emph{inside him is dat conscience}), and \emph{am} is also used to mark third-person plural (\emph{all de fruits ob earth am jes’ de Lawd’s}). With regard to marking agreement on other verbs, there is first-, second- and third-person singular -\emph{s} (\emph{I gibs}, \emph{you’s a beauty}, \emph{’Tis conscience dat makes}), but unmarked first- and third-person singular forms occur as well (\emph{if I take}, \emph{Ol’ Mastah Petah tell me}). Secondly, \emph{bettah} occurs again as part of a modal construction (\emph{he bettah make a sneak}). Thirdly, negation with \emph{ain’t} and negative concord is used (\emph{dat ain’t no place fo’ niggahs}). Fourthly, there is a case of perfective \emph{done} (\emph{dem dat’s jes’ done jined}). And lastly, \emph{a}{}-prefixing (\emph{stop a hunchin’}) and the reflexive pronoun \emph{heself} increase the impression of a text which constructs the voice of a Black speaker as deviating enormously from the ‘neutral’ voice of the surrounding newspaper articles.

Overall, what makes this text stand out is the contrast between the expectations raised by the title and the form of the soliloquy and its actual content and form. The plays \emph{Cato} and \emph{Hamlet}, which are an important part of the literary canon of the time and therefore constitute an element of white cultural knowledge, are used as points of reference, understood only by educated white speakers who have the necessary cultural knowledge. Those white speakers are the implied audience because they are able to understand the humor and to confirm their superiority over Black people based on the text linking linguistic differences to much more fundamental differences in educatedness, degree of civilization and morality.


\begin{ipquote}
CATO’S SOLILOQUY.\\
To steal or not to steal? Dat am de question;\\
Whedder to let dat watermillyun lie\\
An’ slumbeh in de moonlight in de patch\\
An’ sabe its sweetness foh de white folks’ moufs,\\
Or tell mah conscience to behabe itself\\
An’ stop a hunchin’ at me dis-a-way,\\
An’ sneak ’roun’ froo de shadder ob de trees\\
Wha’ not an eye can see me but de Lawd’s,\\
An’ separate dat millyun from de vine.\\
De Lawd! Ah! dar’s de rub, fo’ mebbe He\\
Would chalk it down again dis niggah, so\\
Dat when de trumpet soun’s an’ I arise\\
On golden wings an’ light beside de gate\\
Ol’ Mahstah Petah tell me to go ‘long,\\
An’ smack me on de trousehs wif his boot,\\
An’ say dat ain’t no place fo’ niggahs dat\\
Doan’ propagate deir own ripe millyuns ’stead\\
Ob swipin’ dem from other folkses’ patch.\\
’Tis conscience dat makes cowards ob us coons,\\
Especsh’ly dem dat’s jes’ done jined de church\\
An’ had deir brack souls washed as white as snow.\\
Dar lies dat millyun in de moonlight, an’\\
Hyur stands dis niggah hidin’ ‘hin’ de fence,\\
An’ hyur inside him is dat conscience dat’s\\
A tellin’ him he bettah make a sneak\\
Away from hyur an’ git down on his knees\\
An’ ask de blessed Lawd to pahdon him\\
Fo’ eben thinkin’ ‘bout fo’bidden fruit.\\
But stay! Las’ night I hea’ de preacher say\\
Jes’ afteh prayehs dat nex’ to de Lawd\\
He loved a big fat watermillyun wif\\
A heart as red an’ temptin’ as de lips\\
Dat o’naments de face ob coal brack wench.\\
He am de agent ob de Lawd, an’ if\\
I take dat millyun an’ divide wif him\\
I gibs it to de Lawd, an’ wha’s de sin\\
Returnin’ him jes’ what He grow heself,\\
Fo’ all de fruits ob earth am jes’ de Lawd’s.\\
Come hyur, ol’ millyun, go along wif me!\\
I confiscates yo’ in de Mastah’s name!\\
Yum! yum! but you’s a beauty! Tunk-tunk-tunk!\\
An’ mighty ripe, too, ’cordin’ to de soun’.\\
Come to mah a’ms an’ lie again mah heart\\
Twell I can tote yo’ home, an’ den I’ll go\\
An’ invitate de ministeh to come\\
An’ ask a blessin’ on yo’, an’ den help\\
Devou’ yo’ in de bressed Mahstah’s name!
\end{ipquote}

The last article analyzed here that links non-rhoticity to Black speakers is the poem entitled “When de Co’n Pone’s Hot” and it is different from all other articles analyzed so far because it was written by a Black poet, Paul Laurence Dunbar. In the two databases there are ten articles which contain a version of the poem. The first one was published in the \emph{Philadelphia Inquirer} on November 12, 1893\citesource{November121893}. It is indicated that it has been published before in the \emph{Chicago Record}, but this newspaper is not part of the databases used for the present study. The following seven articles were published three years later, in 1896, and in four of them the poem is used as a part of the same advertisement (published in the \emph{State}, in Columbia, South Carolina). Two articles were published in 1898, but they do not contain the whole poem, but only parts of it, and in one case, this part is quoted within a longer report. The following analysis will focus on three versions of the complete poem, the first one published in 1893, the second one in the \emph{Penny Press} (Minneapolis, Minnesota) on July 25, 1896\citesource{July251896}, and the third one in the \emph{Omaha World Herald} (Nebraska) on October 11, 1896\citesource{October111896} (see \tabref{tab:key:11}). The third version is introduced by the following comment:

\begin{ipquote}
\begin{center}
NEGRO PO{\kern0pt}ET IS BORNE\\
\textstyleStrong{Mr. Paul Lawrence Dunbar Sings of “de Co’n Pone”}
\end{center}
Mr. Paul Lawrence Dunbar has been until recently an elevator boy in Dayton, O. While engaged in the ups and downs of life in that capacity he has cultivated his poetical talents so successfully that his verse has found frequent admission into leading magazines. At last a little collection of these verses reached William Dean Howels [sic], and Mr. Dunbar’s star at once became ascendant. He is said to be a full-blooded negro, the son of slave parents, and his best work is in the dialect of his race.
\end{ipquote}


\begin{table}[t]
\tiny
\begin{tabularx}{\textwidth}{XXX}
\lsptoprule

WHEN DE CO’N PONE’S HOT & WHEN DE CO’N PONE’S HOT & Here is one of Mr. Dunbar’s dialect poems entitled, “When de Co’n Pone’s Hot:”\\

Dey is times in life when \textbf{nacher}

Seems to slip a cog an’ go,

Jes’ a-rattlin’ down creation,

Lak \textbf{a} ocean’s \textbf{ober}flow;

When \textbf{the} worl\textbf{d} jes’ sta\textbf{’}ts a-spinnin’

Lak a picanniny’s top,

An’ yo’ cup \textbf{ob} joy is brimmin’

Twel\textbf{l} it seems about to slop

An’ yo feel jes’ \textbf{lak} a \textbf{racer},

Dat is trainin’ \textbf{fo’} to trot—

When yo’ mammy ses de blessin’

An’ de co’n pone’s hot.

When you se\textbf{’} down at the table,

\textbf{Sort} o’ weary lak an’ sad,

An’ \textbf{youse} jes’ a \textbf{lettle} \textbf{tired}

An’ perhaps a \textbf{leatle} mad—

How yo’ gloom tu’ns into gladness,

How yo’ joy \textbf{dribes} out de doubt—

When de \textbf{oben} do’ is opened,

An’ de smell comes po’in’ out;

Why, de ’lectric light\textbf{s} \textbf{ob} heaven

Seems to settle on de spot,

When yo’ mammy ses de blessin’

An’ de co’n pone’s hot.

When de cabbage pot is steamin’

An’ de bacon good an’ fat,

When de chittlin\textbf{’s} is a \textbf{sputtr’in’}

So’s to show yo’ \textbf{whar} dey’s at—

\textbf{Tek} away yo’ sody biscuit\textbf{s},

\textbf{Tek} away yo’ cake an\textbf{d} pie,

\textbf{Fo’} de glory time is comin’,

An’ \textbf{its} ’pr\textbf{oa}ching \textbf{bery} nigh,

An’ yo’ want to jump an’ hollah,

\textbf{Do’} \textbf{yo’} know \textbf{’yo} bettah not,

When \textbf{yo’} mammy ses de blessin’

An’ de co’n pone’s hot.

I \textbf{hab} \textbf{heerd} \textbf{ob} lots \textbf{ob} sermons,

I \textbf{hab} \textbf{heerd} \textbf{ob} lots \textbf{ob} prayers,

An’ I’ve listened to some singin’

Dat \textbf{hab} \textbf{took} me up de stairs

\textbf{Ob} de \textbf{g}lory lan\textbf{d} an’ set me

Jes’ below de Mast\textbf{ah}’s \textbf{throne}

An’ \textbf{hab} lef my \textbf{heart} a-singin’

In a happy aftah tone.

But dem \textbf{words} so \textbf{softly} murmured

Seem\textbf{s} to \textbf{touch} de softes’ spot,

When \textbf{yo’} mammy ses de blessin’

An’ de co’n pone’s hot.

—Chicago Record.

This version was published in the \textit{Philadelphia Inquirer} (Philadelphia, Pennsylvania) on November 12, 1893\citesource{November121893}. &

Dey is times in life when \textbf{Nature}

Seems to slip a cog an’ go,

Jes’ a rattlin’ down creation,

Lak \textbf{an} ocean’s \textbf{over}flow;

When \textbf{de} worl\textbf{’} jes’ sta\textbf{h}ts a-spinnin’

Lak a picanniny’s top,

An’ yo’ cup \textbf{o’} joy is brimmin’

\textbf{’}Twel it seems about to slop

An’ yo’ feel jes’ \textbf{like} a \textbf{racah},

Dat is trainin’ \textbf{fur} to trot—

When yo’ mammy ses de blessin’

An’ de co’n pone’s hot.

When you se\textbf{t} down at \textbf{de} table,

Kin’ o’ weary lak an’ sad,

An’ \textbf{yo’se} jes’ a \textbf{little} \textbf{tiahed}

An’ perhaps a \textbf{little} mad;

How yo’ gloom tu’ns into gladness,

How yo’ joy \textbf{drives} out de doubt

When de \textbf{oven} do’ is opened,

An’ de smell comes po’in’ out;

Why, de ’lectric light \textbf{o’} heaven

Seems to settle on de spot,

When yo’ mammy ses de blessin’

An’ de co’n pone’s hot.

When de cabbage pot is steamin’

An’ de bacon good an’ fat,

When de chittlin\textbf{s} is a \textbf{sputter’n’}

So’s to show yo’ \textbf{whah} dey’s at;

\textbf{Take} away yo’ sody biscuit,

\textbf{Take} away yo’ cake an\textbf{d} pie,

\textbf{Fur} de glory time is comin’,

An’ \textbf{its} ’pr\textbf{o}ching \textbf{very} nigh,

An’ \textbf{you} want to jump an’ hollah,

\textbf{Do} \textbf{you} know \textbf{yo’d} bettah not,

When \textbf{yo’} mammy ses de blessin’

An’ de co’n pone’s hot.

I \textbf{have} \textbf{heard} \textbf{o’} lots \textbf{o’} sermons,

An’ I’ve \textbf{heard} \textbf{o’} lots \textbf{’} prayers;

An’ I’ve listened to some singin’

Dat \textbf{has} \textbf{tuck} me up de stairs

\textbf{Of} de \textbf{G}lory-\textbf{L}an\textbf{’} an’ set me

Jes’ below de Mast\textbf{ah}’s \textbf{th’one}

An’ \textbf{have} lef’ my \textbf{hawt} a singin’

In a happy aftah tone.

But dem \textbf{words} so \textbf{sweetly} murmured

Seem to \textbf{tech} de softes’ spot,

When \textbf{yo’} mammy ses de blessin’

An’ de co’n pone’s hot.

—Lawrence Democrat.

This version was published in the \textit{Penny Press} (Minneapolis, Minnesota) on July 25, 1896\citesource{July251896}. &

Dey is times in life when \textbf{nature}

Seems to slip a cog an’ go,

Jes’ a-rattlin’ down creation,

Lak \textbf{an} ocean’s \textbf{over}flow;

When \textbf{de} worl’ jes’ sta\textbf{h}ts a-spinnin’

Lak a picanniny’s top,

An’ yo’ cup \textbf{o’} joy is brimmin’

\textbf{’}Twel it seems about to slop

An’ yo’ feel jes’ \textbf{lak} a \textbf{racah},

Dat is trainin’ \textbf{fu’} to trot—

When yo’ mammy ses de blessin’

An’ de co’n pone’s hot.

When you se\textbf{t} down at \textbf{de} table,

Kin’ o’ weary lak an’ sad,

An’ \textbf{you’se} jes’ a \textbf{little} \textbf{tiahed}

An’ perhaps a \textbf{little} mad;

How yo’ gloom tu’ns into gladness,

How yo’ joy \textbf{drives} out de doubt

When de \textbf{oven} do’ is opened,

An’ de smell comes po’in’ out;

Why, de ’lectric light \textbf{o’} heaven

Seems to settle on de spot,

When yo’ mammy ses de blessin’

An’ de co’n pone’s hot.

When de cabbage pot is steamin’

An’ de bacon good an’ fat,

When de chittlin\textbf{’s} is \textbf{a-sputter’n’}

So’s to show yo’ \textbf{whah} dey’s at;

\textbf{Take} away yo’ sody biscuit,

\textbf{Take} away yo’ cake an\textbf{’} pie,

\textbf{Fu’} de glory time is comin’,

An’ \textbf{it’s} ’pr\textbf{o}ching \textbf{very} nigh,

An’ \textbf{yo’} want to jump an’ hollah,

\textbf{Do} \textbf{you} know \textbf{you’d} bettah not,

When \textbf{you’} mammy ses de blessin’

An’ de co’n pone’s hot.

I \textbf{have} \textbf{heerd} \textbf{o’} lots \textbf{o’} sermons,

An’ I’ve \textbf{heerd} \textbf{o’} lots \textbf{o’} prayers;

An’ I’ve listened to some singin’

Dat \textbf{has} \textbf{tuck} me up de stairs

\textbf{Of} de \textbf{G}lory Lan\textbf{’} an’ set me

Jes’ below de Mahst\textbf{er}’s \textbf{th’one}

An’ \textbf{have} lef’ my \textbf{haht} a singin’

In a happy aftah-tone.

But dem \textbf{wu’s} so \textbf{sweetly} murmured

Seem to \textbf{tech} de softes’ spot,

When \textbf{my} mammy ses de blessin’

An’ de co’n pone’s hot.

---

This version was published in the \textit{Omaha World Herald} (Omaha, Nebraska) on October 11, 1896\citesource{October111896}.\\
\lspbottomrule
\end{tabularx}
\caption{
A comparison of three versions of Paul Laurence Dunbar’s “When de Co’n Pone’s Hot”
}
\label{tab:key:11}
\end{table}


This introduction describes the increasing popularity of Dunbar as a poet, which was particularly helped by a positive review by the white and well-known author William Dean Howells, and his ethnicity is explicitly emphasized through the heading, where he is labeled a “negro poet”, and in the text, where he is called “a full-blooded negro”. The introduction also draws attention to the linguistic repertoire used in the poem by mentioning that it is an example of a poem “in the dialect of his race”. This statement already links the forms of the poem to Black speakers. Added to this is a positive evaluation of the representation of Black speech in the poem because his dialect poems are judged to be “his best work”. This evaluation takes up Howells’ opinion expressed in the review of Dunbar’s collection of poems \emph{Majors and Minors} (published in 1895). The review appeared in the magazine \emph{Harper’s Weekly} on June 27, 1896 \citep[80]{Nettels1988}. The introduction therefore shows that the poems published in 1896 were related to a well-known and highly praised poet, whereas the same poet had not been famous when the first version of the poem appeared in 1893.

The content of the poem is the same in all three versions, except for the possessive pronoun in the second to last line, which is \emph{my} in the third version and \emph{yo’} in the first and in the second version. The poem describes the calming and soothing effect of corn pone (a type of corn bread). The stanzas are about feelings of instability (“When de worl’ jes’ stahts a-spinning”), about being held back (“An yo’ feel jes’ lak a racah / Dat is trainin’ fu’ to trot”), about being tired and sad (“When you set down at de table, / Kin’ o’ weary lak an’ sad”) and about having to contain emotions (“An’ yo’ want to jump and hollah, / Do you know you’d bettah not?”). But all these negative feelings go away “When yo’ mammy ses de blessin’ / An’ de co’n pone’s hot”, two lines which are repeated at the end of each of the four stanzas. The smell of corn pone causes “joy” that “drives out de doubt”, and in the last stanza, the mammy’s blessing is even described as more touching and important than all sermons, prayers and singing: “But dem wu’s so sweetly murmured / Seem to tech de softes’ spot”. Celebrating corn pone means celebrating one of the basic elements of southern foodways.\footnote{The term \textit{foodways} is defined by \citet[97]{Edge2009} as “the study of what we eat, as well as how and why and under what circumstances we eat it”. Studying foodways thus encompasses several aspects, among them food events, food processes and “aesthetic realms that touch upon the world of food (country songs about food, quilts raffled at community fish fries, literary references to eating)”.} \citet[98]{Edge2009} finds that

\begin{quote}
Corn – and particularly corn ground into cornmeal – makes up the second half of the southern food pantheon, with cornbread clearly standing as the region’s staff of life. Known by various names – including spiderbread, pone, suppone, hot-water cornbread, dog bread, cracklin’ cornbread, and ho{\kern0pt}ecake – cornbread is the most elemental of southern foodstuffs. It serves as a totem of identity as well as a marker of class standing.
\end{quote}


Even though corn pone is associated with poverty and the lower classes, it is celebrated in the poem precisely because of its important function as an essential element of nutrition and its connection to the fulfilment of other basic needs: of being cared for by a mother and the comfort provided by religious faith. The poem therefore illustrates the claim that corn pone “serves as a totem of identity” for poor southern (in this case Black) people.



The spelling of \emph{corn pone}, in which the <r> is replaced by an apostrophe, prominently links the indexical values associated with the lexical item to non-rhoticity. As in other articles analyzed above, this puts particular emphasis on non-rhoticity as a form associated with Black American speakers. In one case, /r/ is also marked as absent in intervocalic position (\emph{po’in} ‘pouring’), like in \emph{mah’ied} in the anecdote “Rich”. What can also be observed here is that non-rhoticity is not represented consistently. However, the existence of different versions of the same poem makes it possible to compare the representations of non-rhoticity and in \tabref{tab:key:12} I have listed those forms which have been represented differently in the versions (representations of rhotic forms are marked in grey). What can be observed is a major difference between the first version and the 1896 versions. Non-rhoticity is already represented in the 1893 version, but the degree of non-rhoticity has increased in 1896. This suggests that an effort has been made to represent non-rhoticity more consistently, and that it was probably Dunbar himself who made the changes when he included the poem in his collection in 1895. It is not possible, however, to determine with any certainty who decided on the exact form of the printed version (it is also possible that the editor of the newspaper changed the spelling), but it is not likely to be a coincidence that the degree of non-rhoticity increases after Dunbar became famous as a poet. It is striking that despite this increase of non-rhotic forms the two non-rhotic forms in the early version are represented as rhotic in one of the later versions (\emph{fo’}/\emph{fur} and \emph{Mastah}/\emph{Mahster}). This observation, combined with the presence of rhotic forms in all three versions, suggests that accuracy and consistency was not the goal in representing non-rhoticity. A further change observable in \tabref{tab:key:12} that relates to the phoneme /r/ is the elision of pre-vocalic /r/ following a consonant in both 1896 versions (\emph{throne} becomes \emph{th’one}).


\begin{table}
\begin{tabularx}{\textwidth}{XXl}
\lsptoprule
1893 & 1896 (July) & 1896 (October)\\
\midrule
racer & racah & racah\\
fo’ & fur & fu’\\
tired & tiahed & tiahed\\
whar & whah & whah\\
Mastah’s throne & Mastah’s th’one & Mahster’s th’one\\
heart & hawt & haht\\
words & words & wu’s\\
\lspbottomrule
\end{tabularx}
\caption{
Differences in the representation of non-rhoticity in Dunbar’s “When de Co’n Pone’s Hot”.
}
\label{tab:key:12}
\end{table}

With regard to other linguistic forms, interesting changes can be found as well. The most striking one is the complete absence of the stopping of voiced labiodental fricatives in the 1896 versions. For example, \emph{oberflow}, \emph{dribes} and \emph{oben} become \emph{overflow}, \emph{drives} and \emph{oven}. The consistency of this change suggests that for some reason Dunbar has chosen not to link this form to Black speakers anymore, when he published the poem in his collection \emph{Majors and Minors} (the version which was most likely the basis for the two 1896 versions). Other changes are more sporadic, sometimes because they are restricted to a few lexical items in the first place. For example, \emph{tek} becomes \emph{take}, suggesting that a monophthongal \textsc{face} vowel is not represented anymore, but \emph{tek} is also the only instance of such a monophthongal variant in the 1893 version. The word \emph{little} is spelled both <lettle> and <leatle> in the 1893 version, but this represented change of vowel quality is restricted to this lexical item and also not represented in the later versions. Instances of eye dialect have been changed in both directions: \emph{nacher} becomes \emph{Nature} and \emph{nature} in the later versions, but \emph{took} was also changed to \emph{tuck}. On a grammatical level, the third-person singular form \emph{hab} ‘have’ (\emph{dat hab took}) becomes \emph{has} (\emph{dat has tuck}) in the later versions and the third-person plural -s is not represented in the 1896 versions (\emph{But dem words so softly murmured} / \emph{Seems to touch de softes’ spot}). Another difference can be observed in the modal \textsc{better} construction: The modal is changed from \emph{bettah} (as found in most articles above) to\emph{’d bettah} (the contracted form of \emph{had bettah}, the modal used in the cartoon with the two boys and their dogs). The past participle \emph{heerd}, on the other hand, has been changed to \emph{heard} in only one of the later versions.

Linguistic forms that are represented to (roughly) the same extent in all three versions are forms which have also been found in articles analyzed above: Voiced TH-stopping, final consonant cluster reduction, final consonant deletion, third-person plural -\emph{s}. One interesting form is the \textsc{price} vowel because it is represented as containing a monophthong in \emph{like} (\emph{lak}) but not in any other words containing the vowel, not even in \emph{my}, which has been spelled \emph{ma} or \emph{mah} in “Dat Deceptious Mule” and in “Cato’s Soliloquy”. So on the one hand, attention is drawn to the vowel through the parallelism in the fourth and sixth lines of the poem, but it is also restricted to this context. In general, it is noticeable, however, that the number of deviant forms is lower in the poem than in other articles, even more so in the later versions because no cases of labiodental fricative stopping are represented. This becomes especially obvious when comparing the poem to the short story “Dat Deceptious Mule” or to “Cato’s Soliloquy”. This creates the impression that while the voice of a Black speaker is created through the representation of linguistic forms, the number of forms is kept below a level where it would invite mockery and ridicule simply because of the frequency of forms. Taking into account that the poet is Black and that he celebrates a basic element of southern (including Black) foodways, it is rather the case that the linguistic repertoire of Blacks is positively evaluated. The corn pone as much as the linguistic forms might be associated with poverty and a lower social standing, but for Black people they are also indexically linked to the comfort of home and family and thus a symbol of their identity. Especially in the third version with the introduction praising Dunbar’s qualities as a poet, a frame is established that leads the reader to expect a different set of values associated with Black speech than they normally find in anecdotes, reports, stories and cartoons. However, the fact that Dunbar labeled the section in \emph{Majors and Minors} that contains the poem “Humor and Dialect” indicates the dilemma that as soon as a Black voice is evoked through the representation of language, the otherness created through this representation invites at least humor but potentially also contempt.

The advertisement that appeared four times in the \emph{State} in December 1896 is an example which illustrates how the fame of the poem is exploited by a company to promote their product. \figref{fig:key:28} shows the advertisement (the left part appeared on top of the right part). The advertisement is framed by the company’s name, Lorick \& Lowrance, which is printed in small letters on top and in big bold letters at the bottom. Below the company’s name at the top of the advertisement, the title of the poem is printed in big bold letters as well, so that the readers’ attention is drawn to the advertisement. Below the poem, it is contextualized by the comment “‘De Co’n Pone’s Hot’ is very fine, no doubt, but nothing in comparison to our celebrated Cakes and Crackers”. The phrase \emph{Cakes and Crackers} is highlighted by being printed in big bold letters because they are the products advertised. The comment shows that the value of the products is established in relation to the corn pone celebrated in the poem. By first printing the poem, which constructs a high value of corn pone as a symbol of southern Black identity, and then reducing its value to “nothing” in comparison to their own product, the cakes and crackers seem even more valuable than without the comparison. The advertisement creates a stark contrast between corn pone as a home-made product, fulfilling the basic needs of poor Blacks (not only satisfying their hunger but also providing emotional comfort) and being at the heart of family and community gatherings, and cakes and crackers sold in large quantities to merchants by wholesale grocers, symbolizing thus the economic wealth of white people, both from the point of view of the consumers, who are able to afford buying fine cakes and crackers instead of having to bake corn pone themselves, and from the point of view of the grocers, who make money from selling their products to merchants, so that they reach a large number of people. Ultimately, the advertisement conveys that despite being celebrated by Black people, the corn pone is inferior to products like cakes and crackers and underlines Black people’s lower social and economic status.


\begin{figure}
\includegraphics[width=.5\textwidth]{figures/Paulsen-img28.png}
\caption{
Advertisement containing the poem “When de Co’n Pone’s Hot”, published in the \emph{State} (Columbia, South Carolina) on December 30, 1896\citesource{Lorick&LowranceDecember301896}, retrieved from America's Historical Newspapers
}
\label{fig:key:28}
\end{figure}

To conclude, all the articles analyzed above link non-rhoticity to Black speakers, usually in a southern context, although this is often not made explicit. Non-rhoticity as an index of Blackness predominates although it is noticeable that in the examples discussed above, direct representations of Black speech are only rarely contrasted with direct representation of white voices in the same article. In these exceptional cases, the white speakers are \emph{not} depicted as non-rhotic (the lady in the anecdote “Rich” and the senior member of the firm in the anecdote “Too Hasty”). This is interesting because non-rhoticity is also found in representations of white southern speakers in other articles. It suggests that if ethnicity is foregrounded, Black speakers are depicted as non-rhotic, while white speakers remain rhotic, but when belonging to the south is foregrounded, non-rhoticity primarily functions as an index of place and not of ethnicity, and it is thus found in representations of white speakers as well. The following analysis will show how these links to southernness were established and which other social values and linguistic forms also played a role in the process.


The first article linking non-rhoticity to white southerners was published on September 14, 1875\citesource{September141875}, in the \emph{Cincinnati Daily Gazette} (Ohio). It is a report on a shooting at a ball in Paris, Kentucky, where a young man of the Clay family opened fire on a Marshal, who had refused him access to the ball because of his “drunken state” and “disorderly temper”. In the article, non-rhoticity is associated with the southern upper-class: It is represented twice in the word \emph{sah} ‘sir’, which is in both cases combined with the mention of the Clay family and its description as “one of the finest families of Kentucky”. The fact that these combinations are set apart from the main text through quotation marks indicates a change of voice from that of the neutral reporter to that of another person – the reader is invited to infer that it is the voice of a member of the mentioned upper-class family. In the last line, the quotation marks indicate a voice belonging to someone attending the ball, someone who is part of the upper-class “society”. As non-rhoticity is the only phonological form represented in the quotation (in \emph{bettah} and \emph{heah}), it functions as a very salient index of southern upper-class speech in this article. It is also noticeable that \emph{bettah} is used in a modal construction here (\emph{bettah get out of heah}), as in most of the articles linking \emph{bettah} to Black American speakers, and that the adverb \emph{quick} is not morphologically marked. This shows that not only non-rhoticity but also the other forms represented here are not only associated with Black American speakers, but also with a group at the very opposite end of the social spectrum. What unites these articles is the negative attitude expressed towards these groups. In this article, the evaluation of the “sentiment of society” as “just and enlightened” is full of irony, because it is clear how unfairly the Marshal is treated. Even though he only defends himself against the attacker, he is the one who has to fear punishment – the implied reason is the high social position of the attacker. This young man is presented in a very negative light: Not only is he drunk and “in a disorderly temper”, but he is also aggressive, reckless, without any consideration for other people’s lives and, ultimately, also incapable of firing a gun accurately (which is not surprising, given his drunken state). The Marshal, on the other hand, is presented as very courageous because he opposes the young man despite his inferior social position and additionally, as very capable of doing his job because he manages to shoot his opponent with one or more bullets (without killing him). This contrast between the two figures only adds to the impression that the Marshal is treated very unfairly by the society people at the ball. All in all, the article combines a link between non-rhoticity and white southern upper-class speech with a negative evaluation of that same class.

\begin{ipquote}
They had an interesting little society incident over at Paris, Ky., Friday night at the ball which wound up at the Bourbon County Fair. A young man, one of “the Clay family, sah,” proposed to enter the ball-room in a drunken state and a disorderly temper. Being withstood by the City Marshal, he drew his revolver and opened a fusillade on the Marshal. Thereupon the Marshal drew his revolver and returned the fire. But the pistol of the scion of “one of the first families of Kentucky, sah,” only snapped every time, while the Marshall’s went off four times, and hit, too. The incident added much to the festivity of the ball, especially to the girls that went from the north side of the river. The condition of the scion of chivalry was thought critical. Of course, if he had shot the Marshal it would have been all right; but as it was the other way, the just and enlightened sentiment of society toward the Marshal found expression in such observations as: “Well, he’d bettah get out of heah d—m quick.”
\end{ipquote}

A negative evaluation of a southern upper-class figure is also expressed in an anecdote which was published in the \emph{Cleveland Herald} (Ohio) on January 6, 1884\citesource{January61884}. It had originally been published in the \emph{Chicago Inter-Ocean}. The main figure in the anecdote is Colonel Gutrippah, characterized in the sub-heading as a “Gentleman from Kentucky”. The title of Colonel indicates that the gentleman carries the title Kentucky Colonel, which is an honorary title conferred by the governor of Kentucky to men who have done an exceptional deed or service to the state.\footnote{The title “Kentucky Colonel” was originally a title given to members of the civilian state militia, but it was later granted to civilians as well \citep[493]{Kleber1992}.} As stated in \emph{The Kentucky Encyclopedia} \citep[493]{Kleber1992}, the Kentucky Colonel “has come to represent the daring, glamour, dignity, wit, charm, and attraction of outstanding men who have claimed the title—the stereotype of a southern gentleman”. Kentucky Colonels are therefore proud men, a characteristic which is used in the anecdote to create humor, as shown by the saying quoted in the sub-heading: “Pride Goeth Before a Fall”. This saying summarizes and evaluates the events told in the anecdote: The Colonel is very proud of his skating skills, but then proves to be a complete failure when he loses control, collides with a woman (a “fat lady weighing something like 250 pounds”), falls down on his back and ends up with the woman sitting on top of him. Even though he is laughed at, the end of the anecdote shows that he tries to restore his pride by directing the attention away from himself and towards the weight of the woman, comparing her to a horse (“I nevah thought, sah, that a woman could weigh as much as a hoss”). This illustrates that not only his pride is not justified, but also that he is not a gentleman at all because a gentleman would be expected to admit to his shortcomings and be respectful and polite to women.

The positive stereotype of the southern gentleman therefore becomes the subject of ridicule in the anecdote and it is especially the excessive and unjustified pride of the Kentucky Colonel that is criticized. His name, Gutrippah, already indicates the fictionality of the anecdote because it is a telling name which alludes to the characteristics of the figure (the colonel being a “good tripper on skates”, implying that he is not steady on skates and likely to stumble and fall). This obvious fictionality ensures that the readers understand that the criticism is not directed against an actual person but at the figure of the southern gentleman in general. The linguistic forms represented in his speech (his voice is demarcated through direct quotations) are also linked to this rather negative image of the southern gentleman. Non-rhoticity is by far the most salient linguistic form. It not only occurs frequently, but it is also highlighted in several ways. First of all, it is very visible through its representation in the name of the Colonel, which is part of the heading (<Gutrippah> being a pronunciation respelling of \emph{good tripper}). Secondly, it is emphasized through the repetitive use of the address term \emph{sah} ‘sir’ and through its occurrence in parallel structures which occur in close proximity, as in “They ice [...] is hahdah, smoothah, and bettah in every condemned mannah”. Thirdly, it is represented in the lexical item \emph{hoss} ‘horse’, which occupies a prominent position in the anecdote because it occurs at the very end, and because it is central in characterizing the Colonel as disrespectful (by comparing a woman to a horse). There are very few other phonological forms: The repeated spelling of \emph{the} as <they> and the spelling of \emph{Kentucky} as <Kaintucky> indicate a use of a longer and diphthongized vowel in the lexical sets \textsc{dress} and comm\textsc{a}, the spellings <sah> ‘sir’ and <whah> ‘where’ indicate not only non-rhoticity, but also a lower \textsc{nurse} vowel and a low monophthongized \textsc{square} vowel and there is an instance of unstressed syllable deletion in \emph{’deed} ‘indeed’. On the grammatical level, the speech of the Kentucky Colonel is marked by differences to the voice of the narrator and that of the reporter as well. The Colonel uses the first- and second-person plural pronouns \emph{we all} and \emph{you all} and there are differences in subject-verb agreement: \emph{is} and \emph{has} mark first- and second-person plural forms\emph{} (e.g. in \emph{you all is fooled},\emph{ we all} [...]\emph{ has good ice skating},\emph{ the} [...] \emph{ice you all has up heah}) and third-person singular \emph{don’t} in \emph{it don’t last so long}.

The anecdote creates a contrast between the north and the south and given the context – the fact that it was published in northern newspapers (in Illinois and Ohio) – makes it not very surprising that the north is presented as superior to the south. The southern elite is characterized as proud and arrogant – exemplified by the Colonel who claims that he is better at ice-skating, an activity that is practiced much more in the north than in the south (“I drive everybody off they ice”), and therefore also at roller-skating. His claim that even the ice in Kentucky is better than in the north is so exaggerated that it creates ridicule of southern arrogance. That southern elitism and pride is unjustified becomes clear as the events in the anecdote unfold and the Colonel’s incapability becomes the subject of laughter. Another aspect of southern pride is highlighted, when the Colonel refers to his home state as “old Kentucky” because this creates the impression that southern pride is based on the past rather than the present, inviting the implicit conclusion that northern pride is based on the present and the future. The grammatical forms deviating from the speech of the northern voices also implicitly create a picture of southern speech being ‘incorrect’ and ‘inferior’ to northern speech. With regard to the social values indexed by non-rhoticity, the article shows that the phonological form is linked to the figure of the southern gentleman, who, in a northern context, is criticized because of his excessive pride and arrogance, which is largely based on the past and lacks any justification in the present.

\begin{ipquote}
\begin{center}
\textstyleStrong{GUTRIPPAH ON SKATES.}\\
“PRIDE GO{\kern0pt}ETH BEFORE A FALL.”\\
\textstyleStrong{How the Gentleman from Kentucky Went Boldly on the Ice and was Ingloriously Sat Upon.}
\end{center}

“I’ll go with you all, sah,” said Colonel Gutrippah. “I used to be a mighty fine skatah when I was a boy sah.”

“I didn’t know that they ever had skating in Kentucky, Colonel,” said the reporter.

“That’s whah you all is fooled, sah,” replied the Colonel. “We all down ouh way has just as good skating as you all up heah, sah, only it don’t last so long, sah. They ice in Kaintucky, when it do{\kern0pt}es come, sah, is hahdah, smoothah, and bettah in every condemned mannah, foh all uses, including skating and mint juleps, sah, than they common soht of ice you all has up heah, sah. Yes, sah, ’deed I will go with you all and show they general assohtment of cussed mechanics that will doubtless be present how they business is done in old Kentucky, sah. Bet you all a hohn of Baeh Grass against a second-hand wooden tooth-pick that I drive everybody off they ice, sah.”

“But, Colonel,” said the reporter, “this is not ice skating. It’s roller skating.”

“Same thing, sah,” replied the Colonel, putting on his fur overcoat. “Any man who can skate on ice can skate on rollers, sah.”

{[Immediately after stepping onto the skate rink, he falls down. He is helped up by two young men who skate with him for a while and then give him a shove that steadies him. However, the Colonel do{\kern0pt}es not know how to stop and eventually collides with a “fat lady weighing something like 250 pounds”. They fall down – the Colonel on his back and the lady on the Colonel’s stomach – which causes the crowd to laugh and a trombone player to ruin his instrument because he falls off his chair laughing.]}

The Colonel rose slowly and painfully to a sitting position, pulled down his vest and straightened his collar. Then deliberately, and with an air of determination, he took off the skates and flung them with great force across the rink. Taking the reporter’s arm he limped painfully out of the building, and all in silence to the door of the hotel. Then he bent down and whispered in his companion’s ear:

“I nevah thought, sah, that a woman could weigh as much as a hoss.” —[Chicago Inter-Ocean].
\end{ipquote}


\begin{figure}[b]
\includegraphics[width=.7\textwidth]{figures/Paulsen-img29.png}
\caption{
Advertisement using the figure of the Kentucky Colonel, published in the \emph{Fayetteville Observer} (Fayetteville, North Carolina) on September 15, 1898\citesource{September151898}, retrieved from Nineteenth-Century U.S. Newspapers
}
\label{fig:key:29}
\end{figure}

\emph{\textup{The figure of the Kentucky Colonel also appears in a different text type in the collection of articles containing} bettah\textup{. It is part of an advertisement which was published four times in the} Fayetteville Observer} (North Carolina) on September 15, 1898\citesource{September151898} (see \figref{fig:key:29}). It advertises a clothing manufacturing company called The Royal Tailors, which was based in Chicago and New York and sold clothes nationwide. In the advertisement, readers are urged to buy clothes to dress well, based on the argument that only rich people can afford to dress poorly, presumably because they have already achieved the status and wealth that others hope to acquire. The main selling point is that the clothes tailored by The Royal Tailors are better than those of other tailors, and to emphasize this point, the voice of “a Kentucky Colonel” is quoted as saying “There aint no bad whiskey, sah, but I may say, sah, that some whisky is bettah than other whiskey, sah”. The distinction between the voice of the company addressing the readers (“Better try The Royal Tailors”) and that of the Colonel is reinforced when the company’s voice draws on the whiskey analogy to advertise the quality of their clothes and, by doing so, imitates the Colonel’s voice using quotation marks again (“while we are not prepared to say there “aint no bad” tailors”). Through this imitation and the explicit separation of voices through quotation marks, the company distances itself from the use of the form \emph{aint} and the use of negative concord as well as from non-rhoticity: It uses \emph{better} whereas the Kentucky Colonel uses the non-rhotic form \emph{bettah}. As in all the articles linking \emph{bettah} to southern speech above, the address term \emph{sir}, whose non-rhotic pronunciation is also indicated by the spelling <sah>, is also linked to the Colonel’s repertoire in the advertisement. \citet[175]{Davies2007} calls the use of address terms like \emph{sir} “a classic form of negative politeness in which social hierarchy is linguistically signaled”. In the context of this advertisement, this has an important function: It directs the readers’ attention to the existence of social hierarchies and shows them a way to signal a higher position in this hierarchy. The claim “There’s a difference!”, which is printed in bold and set apart from the rest of the text in the middle of the anecdote, underlines this focus on social hierarchy as well: Not only is there a hierarchical difference between good and bad whiskey and good and bad tailors, but there is also a hierarchical difference between rich and poor people and wearing the right clothes is supposed to get a person closer to the top. The name of the company, The Royal Tailors, also alludes to a hierarchical society by using the adjective \emph{royal}, which not only establishes a link to English royalty, but also to the southern planter aristocracy with its close ties to England. The use of \emph{aint} and negative concord indicates, however, that the Kentucky Colonel is not constructed as a linguistic (and also not as a social) model. These grammatical deviations from the ‘neutral’ company voice rather create the same hierarchical difference between the superior north and the inferior south as in the anecdote above. Given that The Royal Tailors is a company based in the north and the northeast, this is not surprising. However, using representations of language is a strategy to establish this superiority in a very subtle manner, and the fact that the advertisement was published in a southern newspaper suggests that the creators hoped that it would appeal to a southern audience. It is perhaps the slight ridicule of the Kentucky Colonel that is intended to motivate southern readers to advance in the social hierarchy and that, in order to achieve this aim, they do not have to carry the title of a Colonel nor be rich, but to simply buy clothes from the right tailor.



Another southern figure linked to non-rhoticity is the southern upper-class girl. She is portrayed in an anecdote which was originally published by the \emph{New York Herald} and which was reprinted in \emph{The Aitchison Daily Globe} (Kansas) on May 25, 1891\citesource{May251891}. The narrator describes two girls who ride on a street car in New York City – they are both young and pretty and wear good clothes, which is a signal of a higher social status, but there is also the important difference that one of them is from New York and the other from the south. This difference is the main topic of the anecdote, which creates a representative incident, intended to characterize and differentiate New York and southern upper-class girls in general. The problem that the girls face on board the street car is that all seats are taken by businessmen on their way to work. While the northern girl accepts the situation as normal and tries to continue the conversation, the southern girl decides to change it and by talking loudly to her friend about how she expects the men to do everything they can to deny them the possibility to sit, she provokes two men to give up their seats to prove her wrong. In the end, she points out to her friend that she brought about the favorable change of situation because as a southern girl she knows how to “manage” men, or in other words, how to make them “behave bettah”.


The anecdote addresses two interconnected issues: the characteristics of and the relationship between the male and the female gender and the difference between the north and the south of the United States. The northern men are portrayed as urban businessmen who are focused on work and success and who are not respectful or polite towards women (they do not offer to give up their seats on the street car). This characterization is summarized by the label “brutes” given to the men by the New York girl. This picture of the rude brute stands in opposition to the ideal of the southern gentleman to which the southern girl can be expected to be accustomed, which leads to her surprise when she realizes that none of the men is offering her and her friend a seat. These different male figures are implicitly linked to concepts of tradition and modernity, symbolized by the places that the street car passes in New York City: “Wall street running right up against old Trinity”. Wall Street is a place associated with eastern urban values resulting from the new, modern business life, while the old Trinity church evokes older, traditional and more conservative values. Not only the male figures and their behavior towards women are linked to this opposition, but also the female figures. The northern girl is characterized as practical, strong and independent: She does not rely on the men to offer her a seat, but she can stand on her own feet and hold herself by holding onto the strap. She does not expect help, but she also does not need it because she can help herself. The southern girl, on the other hand, appears soft and weak at first, but it becomes clear that she uses this soft appearance to achieve her goals in an indirect way – in this case to get the men to give up their seats – so that she also appears strong, but in a different way than the northern girl. The author of the anecdote uses the voice of the southern girl to sum up this characterization by using an illustrative metaphor: “The iron hand in the velvet glove; that’s a south’en woman’s fo’te”. She presents her own character and behavior as a model to the New York girl by saying “If you nawthin gyuls understood managing yah men they’d behave bettah”. This shows that the main aim of the anecdote is to create a representative incident based on which two ways of life are compared and evaluated: the southern, traditional one and the northeastern, modern one. The southern girl’s feelings of superiority are constructed as typical of a southern (upper-class) attitude and it is this attitude and the conservative southern way of life that are humorously criticized in the anecdote.

The representation of language is important in conveying the criticism. Only the speech of the southern girl is marked as ‘deviant’ through pronunciation respellings – the speech of the northern girl therefore appears ‘normal’. The linguistic form that stands out is non-rhoticity because apart from a palatal glide insertion (and a lower vowel) in \emph{gyuls} (‘girls’) it is the only form that is represented.\footnote{Note here that the insertion of a palatal glide after a velar consonant has also been used in representations of Black speech (\emph{cyarry} in the article “Dat Deceptious Mule”).} By describing her as speaking “in soft, clear tones and unmistakable southern accents”, non-rhoticity is explicitly linked to southern speech and to her soft and female character. This association with the south is particularly important considering that in the articles containing \emph{deah} AND \emph{fellah}, non-rhoticity is also linked to eastern urban speech, especially New York speech. In this anecdote, however, the New York girl is \emph{not} portrayed as a non-rhotic speaker. This indicates that when differences between the south and the north are highlighted, non-rhoticity is an important marker of southern speech and northern speech is consequently \emph{not} marked as non-rhotic (even in New York City). If, however, the focus is not on a regional difference but on social differences between speakers, non-rhoticity can also be used to mark a northern speaker as someone who wants to appear educated and ‘cultured’ and who wants to signal membership in upper-class circles through the imitation of British English speech, but also as someone who lacks authenticity. Nevertheless, these social differences are also linked to a regional difference, albeit not between north and south but between east and west. This suggests that non-rhoticity is restricted to a particular social group of speakers in the northeast, whose negative characteristics are highlighted by contrasting them with positive characteristics of western (not southern!) speakers. The association between southerners and non-rhoticity, in contrast, rather serves to emphasize the traditional and conservative ways of the south, where men are still gentlemen and where women are not independent per se but need to exert their power in rather indirect ways. \citet[175]{Davies2007} describes indirectness in general as a “classic negative politeness strategy” in the south, and the anecdote reveals how this difference on the pragmatic level between indirect and direct forms of expressing one’s wishes becomes indexically linked to the north-south difference in metadiscursive activity. The case of non-rhoticity consequently illustrates very impressively that one linguistic form can index very different and even opposite social values (southern and northeastern with regard to region, Black and white with regard to ethnicity) and that it is only in context with other linguistic forms and social values established in the text that specific indexical links are created and become interpretable by the reader.

All in all, the anecdote “She Got a Seat” ultimately presents the northern girl as tougher and more progressive because she can stand and does not need to sit like the southern girl, and because she does not have to use indirect politeness strategies to get men to help her. Even though the humor of the anecdote rests on the southern girl getting her way, it also leaves room for doubt whether it was really her ‘management’ that caused the men to give up the seat. One of the men says “No, no; we’re not as bad as that”, which creates the impression that even though northern men might not be like southern gentlemen, they can also not be characterized as uncivilized brutes without morals. They are rather used to modern women who are independent and, if they do need help, express their wishes directly.

\begin{ipquote}
\begin{center}
\textstyleStrong{She Got a Seat.}    
\end{center}
It was about 11 o’clock, on the elevated road—and elsewhere.

At Twenty-third street two pretty young women in good clothes came aboard.

The morning rush was past, but still the car was full of comfortable business men studiously devoted to their newspapers.

Not a seat was vacant, and not one was offered. One of the young women hitched herself to a strap with an air of familiarity with the process; the other looked on and at the men with an expression of intellectual curiosity, not unmixed with scorn.

“\textstyleStrong{You’ll see Wall street running right up against old Trinity},” said she of the strap; as if continuing a previous conversation, “\textstyleStrong{then I want to take you through one of the big office buildings, but we’ll have to wait till papa}”—

“\textstyleStrong{Will we have to stand all the way down theah?}” asked her friend and evident guest irrelevantly, in soft, clear tones and unmistakable southern accents.

“No; not all the way,” replied the New York girl, and then, chaperoning her sex instead of her section, she added sotto voce, “\textstyleStrong{some of these brutes will have to get out before Rector street—they’ll have to give us a seat whether they want to or not}.”

“\textstyleStrong{No, I don’t think they will,}” said the southerner in the same soft, audible tone, and casting a meditative look about her; “\textstyleStrong{I think by their looks they’ll stay aboard and lose money to keep us out of one.}”

A gentleman sitting on one of the cross seats with his back to them now rose with an amused expression, saying:

“\textstyleStrong{No, no; we’re not as bad as that,}” and surrendered his seat, whereupon his vis-a-vis succumbed ruefully to moral suasion and gave up his. When the girls, with smiling thanks, were seated, the southerner winked merrily at her friend and said:

{“\textstyleStrong{If you nawthin gyuls understood managing yah men they’d behave bettah. The iron hand in the velvet glove; that’s a south’en woman’s fo’te.}”—New York Herald

\raggedleft
{[emphasis mine]}\\
}
\end{ipquote}

Non-rhoticity, represented by the search term \emph{bettah}, does not only occur as part of the representation of upper-class southern speech, but also of lower-class southern speech. An example is the article “A Relic of the Past”, written by M. I. Dexter and published in \emph{The Atchison Daily Glob}e (Kansas), on January 9, 1895\citesource{DexterJanuary91895},
and in the \emph{Idaho Statesman} (Boise, Idaho) on January 18, 1895\citesource{DexterJanuary181895}. This article is a report about a sundial in St. Louis and the title “A Relic of the Past” indicates that the general topic is the transformation of St. Louis from an old town to a modern one. The first sentence explicitly describes how in this process the “old city is gradually ridding itself of many of the peculiarities that have heretofore stamped it as more of an old time southern than a modern western town”. The sundial is constructed as a symbol of the past, an object that was once found in almost every southern town (especially in the southwest) but that is now so rare that its peculiarity attracts the interest of many visitors. The person linked to this relic from the past is a janitor who works at the courthouse where the sundial is located. He is interviewed and his description of the dial is rendered in direct speech, indicated through quotation marks (and highlighted in bold here). Phonologically, his speech is prominently marked by non-rhoticity, which is the only phonological form consistently represented. There are three more cases of pronunciation respellings, \emph{they} ‘the’, \emph{kin} ‘can’ and \emph{jest} ‘just’. The first one indicates a longer and diphthongized vowel in \emph{the} and the forms \emph{kin} and \emph{jest} are most likely cases of eye dialect, indicating a weak form with a more central vowel (and not a non-standard pronunciation). On a lexical and pragmatic level, the high frequency of the address term \emph{sir} (represented as \emph{suh} here) is noticeable, as in the other articles analyzed above, which confirms its status as being consistently linked to a southern American linguistic repertoire and linking non-rhoticity to a social system marked by social hierarchy. On a grammatical level, the janitor exhibits a number of differences to the ‘normal’ voice of the reporter. Like the Kentucky Colonel in the advertisement, he uses \emph{ain’t} and \emph{hain’t} as negators as well as negative concord as in \emph{they ain’t no going back} and \emph{that hain’t got no bettah business}. And like Colonel Gutrippah in “Gutrippah on Skates”, he exhibits a different pattern of subject-verb agreement, but in his case, it is the third-person plural form that is marked by an -\emph{s} (\emph{theah’s not many}, \emph{they comes}). However, unmarked forms occur as well (\emph{they all know}). Furthermore, the janitor uses the double demonstratives \emph{this here} and \emph{that there} (\emph{this heah co’thouse}, \emph{that theah old dial}), \emph{what} as a relative pronoun (\emph{theah’s not many but what knows}), an unmarked past tense form in \emph{They use to be two sundials} and existential \emph{they} in the sentence just cited and also in the \emph{they ain’t no going back} cited above. Through the figure of the janitor, these linguistic forms are linked to the south and to an old and outdated way of life. By characterizing him as “quainter even than the old dials”, the indexical link to old-fashioned behavior and attitudes is reinforced. However, the janitor is also described as “delightful” and his description of the old sundial also contains some elements of nostalgia: He makes a reference to the “fine” southern gentleman who is fairly rich (he is wearing a gold watch) and who uses the sundial to set his watch because it is very reliable. Even though the past is depicted in a positive light, it is made clear that this way of life is in the process of being transformed and that it is “entertaining” for ‘modern’ visitors precisely because it is not part of present-day life at the time. The difference between old-fashioned and modern is linked in the article to the regional difference between the south and the west (“an old time southern” vs. “a modern western town”) and through the figure of the janitor non-rhoticity is linked to the former, and rhoticity, by implication, to the latter. The necessity for transformation can therefore be extended to language as well: Non-rhoticity and the other linguistic forms described above are portrayed as old and traditional and even though they are pleasingly peculiar they are not compatible with a modern way of life and therefore need to be changed. Non-rhoticity is thus portrayed as being as much a “relic of the past” as the sundial and the figure of the janitor.

A further aspect that is relevant in this context is the comparison made between the janitor and “our Uncle Samuel”, a figure that became a symbol of the United States \citep[321]{Vile2018}. On the one hand, the reporter finds that the janitor resembles Uncle Sam because of his facial features (“When I first saw his face, I was almost surprised because his trousers were not striped red and white and his vest was not blue and star spangled”). On the other hand, the shape of the body is the complete opposite to that of Uncle Sam (“instead of being thin and angular, he [the janitor] is round and stout and ruddy”). This description can be read as symbolizing the janitor’s belonging to the American nation: While he is an American, he does not correspond to the ideal American, embodied in the Uncle Sam figure. He is a southerner who romanticizes the past (symbolized by his praise of the sundial, which is “running just as good as it did when it was first put theah”) and who complains about the young men who destroy old things (like the dial part of another sundial) to accelerate the transformation to a modern way of life. He is protective of the past and fears, as he is quoted at a later point in the article, that “the county’ll see fit to teah this grand old co’thouse down and put up a great big steel frame modern building”. All in all, non-rhoticity is therefore linked to a conservative, backward-oriented southern speaker who glorifies the past and fears the transformation to modernity.

\begin{ipquote}
\begin{center}
\textstyleStrong{A RELIC OF THE PAST}\\
THE SUNDIAL BEFORE THE OLD COURTHOUSE AT ST. LOUIS\\
\textstyleStrong{Views of the Janitor—A Man Who Is as Quaint as the Dial Itself—Fine Paintings That Cost a Right Smart Heap—The Past Linked to the Present.}\\
{[Special Correspondence.]}    
\end{center}
ST. LOUIS, Jan. 8. —Although this interesting old city is gradually ridding itself of many of the peculiarities that have heretofore stamped it as more of an old time southern than a modern western town, the transformation is still far from complete, and it is this very survival of an occasional feature of other days that renders the place entertaining to the curious visitor.

Standing within a 4 by 5 foot inclosure of iron fence, near the white painted old courthouse, for instance, is an iron post, topped with a broad, flat, circular plate. Rising from the center of the plate is a triangular projection, and there is an iron hinged cover attached, evidently intended to be closed when the weather is bad. Such a contrivance is a rare sight nowadays anywhere, but it would have been a poor sort of town indeed five or six decades ago, especially in the southwest, that was without one, and no one would have been at a loss regarding its use. But there is not a day in a twelvemonth those modern times, so the courthouse janitor told me today, that some stranger do{\kern0pt}es not inquire curiously about the one in the St. Louis courtyard.

“\textstyleStrong{But of co’se,}” said he reflectively, pulling his chin whisker and turning his quid of shaved plug in his check, “\textstyleStrong{theah’s not many but what knows it’s a sundial when they comes to examine it. It’s a fine dial, that. You see, it’s running just as good as it did when it was first put theah. And many’s the fine gentleman, with his kid gloves on, that steps up to that theah old dial, suh, at noon and sets his gold watch by it. They all know, suh, that, notwithstanding a sundial is old fashioned, they ain’t no going back on sun time, suh.}

{“\textstyleStrong{They use to be two sundials to this heah co’thouse. You kin see the post of the othah one out theah on the othah side of the yahd. But the dial paht is gone forevah, suh. You see, they’s a pow’ful lot o’ boys heah in St. Louis that hain’t got no bettah business nights and Sundays than to pound co’thouse dials with rocks and things jest to destroy them.}

\centering {[...]}

This delightful janitor’s name is Griffin, and he is quainter even than the old dials.} {In person, he resembles our Uncle Samuel, save that, instead of being thin and angular, he is round and stout and ruddy. When I first saw his face, I was almost surprised because his trousers were not striped red and white and his vest was not blue and star spangled. {[...]}

\raggedleft
{[emphasis mine]}\\
}
\end{ipquote}

An article which creates a link between non-rhoticity and white southern as well as Black American speech and therefore relates both groups to each other was published in the \emph{Northern Christian Advocate} (Syracuse, New York) on April 3, 1884\citesource{McIntoshApril31884}. It is an account of the missionary and social work of a religious group in the south (in Little Rock, Arkansas), written by a member of that group, Miss E. H. McIntosh, and directed at a northern and eastern audience (their “friends”). The group has installed a “Model Home” called the “Smith Industrial Home”, which is connected to the Philander Smith College, an educational institution, and its chief object is to help Black Americans (“our work is with the colored people”). In the article, the author characterizes the people living in Little Rock through explicit comments and descriptions as well as implicitly by means of direct representations of their speech. She stresses that differences between southerners are mainly based on class and not on race: By saying that “both whites and blacks” are slow “except in the upper circles” and that “whites are little better than the blacks” in having a low morality, she groups both racial groups together as being “of the lower order”. On the other hand, she points out that there are not only white upper-class people, but also “various grades” among Black people, a “colored gentleman” who is “intelligent, polite and interesting” being an example of a Black person with a higher social status. The statement that white lower-class people are not better than Black people functions as a disparagement of white people against the background that Black people are normally constructed as being socially inferior to white people. For the higher classes, white superiority remains unchallenged, as can be seen in the following remark: Even those “who are the most intelligent and most cultured among them [among the colored people] are sadly lacking in many things that they would be so much better for knowing”. So even though there are degrees of ignorance, it remains a major problem and it serves to justify the installment of the “Model Home”. Connected to ignorance is another social characteristic: backwardness. The author uses the voice of an imagined white southern speaker to characterize this particular group of people as backward because of their ignorance (“It is the ignorance of the people what make us so”.). By using the pronouns \emph{they} (“they said”) and \emph{us} (“make us so”), the author emphasizes that the voice is representative of the whole group of lower-class southern people. Using the voice of an insider of the group is an important strategy of characterization because it appears to provide support for the author’s judgements and opinions from an insider’s perspective. Similarly, she uses the representative voice to create the impression that their help is wanted by the people themselves: She lets the voice explicitly ask for it (“What we want is for you northerners to come down heah and teach us bettah”). This shows that the most important division created in the article is that between the south and the north, with most southerners being depicted as slower, less intelligent, less educated, backward and having a lower morality. They are not only presented as needing and wanting help, but also as deserving help because of their friendliness and good character, which is conveyed through their “good-natured” and “easy” laughs. A particular case to reinforce the impression of southern helplessness is presented through the description of the stereotypical “Topsy” figure: a fourteen-year-old girl who sucks her thumb like a much younger child, who is barefoot (she turns on her toe) and not properly dressed, and whose basic needs are not fulfilled (she is hungry and without a home, family, and friends).

The linguistic repertoires of the lower-class southern speakers are illustrated through the voice representing southern white speakers, a voice representing Black speakers in the beginning of the article and the voice of the “Topsy” figure (all the direct quotations are highlighted in bold in the quotation). The co-occurrence of representations of both white and Black southern speech makes this article ideal for exploring which linguistic forms these groups have in common and which forms mark them as different. Non-rhoticity is a phonological form associated with both groups and although it is quite frequent and noticeable through the spelling <ah>, it is also not represented consistently (it is not marked at all if /r/ precedes a consonant in the coda cluster). Apart from non-rhoticity, there is only one form represented in the spelling and linked to all three representations: the alveolar realization of -\emph{ing} (e.g. in \emph{tryin’}, \emph{nothin’}). What differentiates the Black voices from white voices are forms that have been identified as typical of a repertoire of Black speech in articles analyzed above: final consonant cluster reduction (\emph{tole} ‘told’), unstressed syllable deletion (\emph{bout} ‘about’) and TH-stopping (only represented once in \emph{dis}) and the insertion of /j/ and dropping of /h/ in \emph{yeah} ‘here’. White southern speech is not additionally marked on the phonological level. However, on the lexical level white southern speech is characterized by the lexical item used to greet people (\emph{How’dy}), the specific use of the greeting \emph{good evening} (which becomes appropriate as soon as mid-day is past) and the phrase \emph{I reckon} (which is neither part of the author’s voice nor of the other voices which are represented by the author).

Considering grammatical differences, it is more difficult to identify forms that clearly mark a speaker as white or Black. What is noticeable is that the Topsy figure is distinguished on the grammatical level from the other speakers because she almost always uses reduced sentences in which the subject is deleted. The sentences are short and if two clauses are joined, they are not linked by a conjunction (\emph{I’m just so hungry can’t do nothin’ ’t all}). This syntactic simplicity is linked to the description of the simplicity of her life – she lacks the most basic elements needed for survival. The representative Black voice does not exhibit such a reduced syntactic complexity, however, so that this form seems to index a particular type of Black speaker rather than Black ethnicity in general. There is one case of copula deletion (\emph{Our people slow}) in the representation of Black speech, but this is also found in the representation of white southern speech (\emph{we too old}). As in two other articles representing Black speech, the double demonstrative \emph{this here} (\emph{dis yeah work}) is linked to the Black voice here as well, but as it is also used by the white janitor in “A Relic of the Past”, it is doubtful that this form distinguishes Black from white speech. Noticeable elements of southern white speech are \emph{am} as the third-person singular form of \textsc{be} (\emph{the way it am done heah}), the relative pronoun \emph{what} and an invariant third-person singular verb form (\emph{It is the ignorance of the people what make us so}), but all of these forms have been found to mark Black speech in articles above as well. A form linked to the white voice which has so far only been found in one article representing a white southern voice (Colonel Gutrippah) is the marking of the third-person plural (here combined with first-person plural \emph{is} in \emph{We makes lots of money, but we is always poah}). However, Colonel Gutrippah combines this with second-person plural -\emph{s} marking which has also been linked to the Black preacher in “A Negro Revival”, so it is again doubtful that this form indexes the ethnicity of the speaker. A form that has not been used in any of the articles above is the double marking of past tense in \emph{Why didn’t you tole us}, which is linked to the Black voice. There is also a regularized past participle (\emph{knowed}), a form that has been found in the Black preacher’s repertoire in “An Old Time Fo’th” as well.

To conclude, the article “First Impressions on Entering the Field” illustrates that the ethnicity of the southern speakers is primarily indexed on the phonological and the lexical level. On the grammatical level, a reduced, subject-less sentence structure does index a Black figure which is particularly simple-minded and poor, but this makes it a form that does not index Black ethnicity in general but only Black ethnicity co-occurring with other social characteristics. Before drawing any further generalizations based on all the articles analyzed so far, I will analyze articles linking \emph{bettah} and thus non-rhoticity to the third major group of speakers: to mountaineers.

\begin{ipquote}
\begin{center}
\textstyleStrong{First Impressions on Entering the Field.}\\
\textsc{by miss e. h. mcintosh.}    
\end{center}
{Probably most of the readers of the \textsc{Northern} were interested in reading the account of the dedication of the “Model Home” at Little Rock, Ark., which occurred Feb. 25th: and perhaps they would like to know more of this Home which now bears the name “Smith Industrial Home.”

\centering
{[...]}

The habits of the people who belong to this section of the country are slow.} {This may be said of both whites and blacks, except in the upper circles.

\centering
{[...]}

The mind [of the colored people] is not more active than the body.} {When we told the people what we intended to do they said: “Why didn’t you tole us befoah? Why didn’t you come down heah and lecture three, foah months ago, so we’d knowed bout dis yeah work? Our people slow! They can’t take this yeah thing so fast.”

\centering
{[...]}

It is surprising to see how low is the morality of this people.} {I am told that the whites are little better than the blacks in this particular.

\centering
{[...]}

When they meet here they say in greeting \textbf{“How’dy?”} and after twelve o’clock every one says \textbf{“good evening.”}} {If you want anything done as it ought to be done they say: \textstyleStrong{“Oh! they don’t do that way here! I reckon you came from the North, you came from the East, no such thing done in the South—not in Arkansas anyway.”} When you find them doing a thing backward and exclaim: \textstyleStrong{“Why! that is backward!”} they with a good-natured laugh will say: \textstyleStrong{“That’s the way it am done heah,”} you will answer as you feel: “Yes! you do everything backward here!” And with another easy laugh they say: \textstyleStrong{“Yes, mam, we do everything backwards heah, but then you know ouah people doant know any bettah. It is the ignorance of the people what make us so. What we want is for you northerners to come down heah and teach us bettah. We makes lots of money, but we is always poah, for we doant know how to spend it. No use’n us old folks tryin’ to learn; we too old; but you teach ouah girls so they know more than their mothers did.”}

They are, however, not all of the lower order. We find various grades among them. We were shown through the post office of this city the other day by a colored gentleman, superintendent of the mails, as handsome a looking man as you would wish to see, intelligent, polite and interesting [...].

\centering
{[...]}

Then on the other hand we have seen the genuine “Topsy”, who came in to ask for something to eat.} {When we would ask a question she would, with finger in her mouth, turn on her to{\kern0pt}e until you had the back of her head in full view, so I don’t think she had on the original Joseph’s coat but she had parts of it all over her. Her skirt hung to the floor on one side and showed the knees at the other. You judged her to be fourteen years of age. We asked her: “What do you want?”

“\textstyleStrong{Want sothin to eat; I’m just so hungry can’t do nothin’ ’t all.}”

“Where did you come from?”

“\textstyleStrong{Come from no whah.}”

“Where do you stop?”

“\textstyleStrong{Stop no whah.}”

“Where do your friends live?”

“\textstyleStrong{Don’t live no whah. Have no friends. Just so hungry can’t do nothin’ ’t all.}”

This is one of the kind that chews tobacco and ‘dips snuff’.”

\centering
{[...]}

We find that even those who are the most intelligent} {and most cultured among them are sadly lacking in many things that they would be so much better for knowing.

\centering
{[...]}

Friends, these are the people} {whom we have come here to help.

\centering
{[...]}

Then how necessary are these foundation stones! To these our toil shall be given and we hope also the prayers of many of God’s dear children in the North and East.}
\end{ipquote}

For the qualitative analysis of articles constructing non-rhoticity as part of the linguistic repertoire of the figure of the mountaineer, I will give examples of three different text types: a report containing an anecdote, a humorous short dialogue and a long fictional story. The report and the fictional story contain longer stretches of direct representations of speech and were published ten years apart from each other (1885 and 1895). The humorous dialogue was published in the same year as the fictional story (1895), but the representation of speech is very short and provides insights into which forms were focused on to differentiate the mountaineer’s linguistic repertoire from that of other speakers.

The report names the object of interest in the heading of the article: “The Pool Tribe”. It was published on September 20, 1885\citesource{September201885} in the \emph{Galveston Daily News} (Houston, Texas), and it originally appeared in the \emph{New York Sun}. The sub-heading provides the main information as to what the Pool tribe is and how it is evaluated: “A Queer Mountain Race that Live in Pennsylvania”. The reference to the mountains is important – in the first paragraph this is specified further as “a spur of the Appalachian range” and this is where the people described by the author live. They are characterized as “non-descript” and “queer” and compared to prairie dogs, which evokes associations of animal-like, uncivilized behavior. By calling them a “mountain race”, they are also presented as constituting their own race and thus distinguished from the white as well as the Black race. The report includes an anecdote about the reporter meeting a member of the Pool tribe. The man explicitly claims to belong to the group of “Pools” by saying “Wal, I’m one on ’em ”. The anecdotal character of this part of the report invites the reader to regard this man as a typical exemplar of the mountain race and the description of the color of his skin as “a weather beaten, old-copper complexion” underlines his attribution to a different race. Furthermore, the adjectives “weather-beaten” and “bent” indicate a life outdoors and full of hardship. The reporter finds the man “half sitting, half lying on the flat surface of a big boulder, sunning himself”, which suggests that he is not working anymore, and the man offers to help the reporter only if he gets whiskey in return, which, in combination with his repeated request for a drink, implies that he is addicted to alcohol. The reporter then tells the story of how they find a rattlesnake and how the old man catches it, kills it by crushing a stone on its head and then cuts off the rattles to get the money promised by the reporter. The mountain man therefore appears to be tough and unafraid. In the last part of the story he explains to the reporter why he is not afraid of rattlesnakes: His father has made him immune against the poison when he was a baby by cutting a hole in his arm and putting “some o’ de pizon in it”. This anecdote therefore draws several parallels between the man and the rattlesnake. Both the man and the animal were found by the reporter while they were sunning themselves and both have poison in them. When the mountain man says “Wal, I’m one on ’em”, this could also be read as him identifying himself as a rattlesnake and not a Pool. The implicit message that can be constructed based on the anecdote is that mountain men are like rattlesnakes: They are dangerous, but they can be beaten.

This figure of the “Pool” belonging to the “queer” mountain race (the term \textit{mountaineer} is not used in this report, but the figure is explicitly connected to the Appalachian Mountains) is linked to several linguistic forms through representations of his speech in direct quotations (highlighted in bold). Non-rhoticity is quite prominent because it is frequently represented, for example in the address term \emph{mistah}, which the mountain man uses several times. However, there are also cases in which the form is not marked as rhotic (\emph{other} in the quotation below, but there are also several further forms in the remaining part of the story, e.g. \emph{fadder} and \emph{rattler}). Furthermore, voiced TH-stopping and alveolar -\emph{ing} are frequently represented – especially the occurrence of TH-stopping is surprising because in the articles analyzed above it was only linked to Black speakers. Further forms shared by the mountain man and Black speakers in other articles are \emph{little} spelled <leetle> to indicate a high front tense vowel, the \textsc{choice} vowel marked as having a lower onset in \emph{pizon} ‘poison’ and the \textsc{square} vowel being represented as <ee> in \emph{careful}, suggesting a high front monophthong, and as <ah> to indicate a lower and monopthongized vowel in \emph{there} and \emph{where}. This last form was also part of Colonel Gutrippah’s linguistic repertoire. A phonological form not found in other articles is the devoicing of the final consonant in \emph{have} (\emph{haf}). With regard to grammatical differences, the mountain man exhibits second-person singular -\emph{s} marking and he uses the reflexive pronoun \emph{deyselfs} – both forms have been shown to be linked to Black speakers as well. It is also noticeable that \emph{bettah} occurs again in a modal \textsc{better} construction (\emph{Bettah go}), as in seven articles representing Black speech and the first article representing southern upper-class speech. On a lexical level, the use of the phrase \emph{I reckon} links the mountaineer’s repertoire to white and Black southern speech. The phrase\emph{ I’ll have to go you}, which seems to mean ‘I’ll help you’, has not occurred in any article so far and therefore seems to be characteristic of the mountain man in this article.

\begin{ipquote}
\begin{center}
\textstyleStrong{THE POOL TRIBE.}\\
\textstyleStrong{A Queer Mountain Race that Live in Pennsylvania.}\\
{[New York Sun.]}
\end{center}
{\textsc{Towanda}, Pa., September 1.—Two broad, low foothills, a spur of the Appalachian range, comprise almost the whole of Durell township, Bradford county, in the northern tier of Pennsylvania. To the surrounding country this double elevation is known as the Huckleberry mountain, or Pool Hill, the first named from the luxuriant growth of this popular fruit, which covers sides and summit of the mountain. The appellation Pool Hill it takes from the nondescript people who populate its broad sides like prairie dogs, better known as the “Pool tribe.”

\centering
{[...]}    

I camped two nights on the mountain a few weeks ago, and took a stroll up among the rocks, keeping an eye out for rattlesnakes. All at once I heard a call: “Say Mistah”}

It was a bent old man, with white hair, and a weather beaten, old-copper complexion. He was half sitting, half lying on the flat surface of a big boulder, sunning himself. In his hand he held a heavy stick, with two short prongs at the end.

“Hello,” said I; “what do you want?”

“\textstyleStrong{Whah’s yo’ goin’?}”

“Looking for a rattlesnake and a Pool,” I answered.

“\textstyleStrong{Wal, I’m one on ’em, an’ I reckon yo’ won’t have to go mo’n a mile to find th’other. Has yo’ got anythin’ to drink?}”

I had a pocket flask of snake-bite annihilator with me. Pulling it out, I said: “I’ll give you a drink if you will show me a rattlesnake.”

“\textstyleStrong{I’ll haf to go yo’, mistah,}” he said, as he climbed down from the rocks at a pace which left me breathless. Twice on the way he stopped, and, looking up blankly, said: “\textstyleStrong{Godlemity, but I’m dry, mistah.}” Both times I moistened the old man, and again we proceeded. Finally he said:

“\textstyleStrong{Bettah go a leetle kee’ful, mistah. Dey’ll be up dah sunnin’ deyselfs.}” Proceeding cautiously, we climbed up on a rock and looked over on the other side. There, stretched lazily out on a flat stone, lay a large snake.

“\textstyleStrong{Do{\kern0pt}es yo’ want him, mistah?}” said the Pool.

“I’ll give you half a dollar for his rattles,” said I.

{“\textstyleStrong{I’ll haf to go yo’},” was the reply, as he cautiously climbed down from the rock on which we rested and then crept along on all fours until within a few feet of the snake. Then he noiselessly straightened up and reached for the snake with his forked stick, planting it just back of his head. The reptile writhed and twisted and rattled his alarm, but the old man had him pinned fast. After watching him a minute or two, the Pool picked up a stone with his free hand and crushed the snake’s head. {[...]}

\raggedleft
{[emphasis mine]}\\}

\end{ipquote}

The second article I have chosen for analysis is a short humorous dialogue with the heading “Safer, Too”, published as part of the section “Multiple News Items” in the \emph{Boston Daily Advertiser} in Boston, Massachusetts, on July 27, 1895\citesource{July271895}. It is taken from the \emph{Chicago Record}. One of the conversational partners is a “Mountaineer”, while the other is labeled “Visitor”, which underlines that he serves as a ‘neutral’ point of comparison – he is not marked as belonging to a particular group or being from a particular region. The Visitor and the Mountaineer talk about a feud which has apparently been going on for about thirty years. The Visitor asks why the law does not settle it, to which the Mountaineer replies that it should go on until “only one gentleman is left, sah, and then we’ll hang him”. The dialogue marks the Mountaineer as having a belligerent and extremely violent character. A contrast is established between a civilized system, in which laws and institutions enforcing the law ensure peace, and an uncivilized system, in which people take the law into their own hands leading to lynchings and other forms of violence. The Mountaineer’s aim to extinguish all gentlemen can be read as symbolizing his wish to end civilization and refined behavior in favor of uncontrolled violence. These social values are linked to several linguistic forms marking the Mountaineer’s speech as different from that of the Visitor. Non-rhoticity is particularly salient because it is represented several times, especially in the address term \emph{sah} (but not in \emph{thurty} ‘thirty’) and because it is the only deviant form apart from one instance of initial unstressed syllable deletion (\emph{’Bout}) and one representation of a different quality of the \textsc{nurse} vowel (\emph{thurty}). This illustrates that non-rhoticity has been chosen as a primary means of linking the Mountaineer’s otherness to a different linguistic repertoire. The only form that stands out in the visitor’s speech is third-person singular \emph{don’t}, which creates the impression that this form is widespread and not associated with particular places or social characteristics.

\begin{ipquote}
\begin{center}
\textstyleStrong{SAFER, TOO.}\\
Chicago Record.    
\end{center}

Visitor—How long has this feud continued?

Mountaineer—\textstyleStrong{’Bout thurty yeahs, sah.}

Visitor—Why don’t the law step in and settle it?

Mountaineer—Well, \textstyleStrong{sah}, it’s \textstyleStrong{bettah} to let it run on, \textstyleStrong{sah}, till only one gentleman is left, \textstyleStrong{sah}, and then we’ll hang him.
\end{ipquote}

The third example is a long fictional story written by Alfred R. Calhoun and entitled “A Mountain Missionary”. The first chapter was published in the \emph{Yenowine’s News} (Milwaukee, Wisconsin) on March 3, 1895\citesource{CalhounMarch31895}.
The complete story was published in the \emph{Atchison Daily Globe} (Kansas) on March 15, 1895\citesource{CalhounMarch151895}. The main character in Calhoun’s story is a white preacher, Father Peters, who is originally from Ohio but moved with his family into “the Cumberland mountains in southeastern Kentucky” to work there and convert the people living in the mountain region (labeled “mountaineers”) to religion. He is the “mountain missionary” and he is characterized as a man of a very “intense religious spirit”. In the beginning of the story, the mountain region is compared to his old home, northern Ohio, and the two places are presented as complete opposites: Whereas in northern Ohio “[w]ealth, or at least comfort, and intelligence were the rule”, southeastern Kentucky is characterized by violence, poverty and ignorance. This contrast is underlined by contrasting the character of the preacher with a character from the mountain region: Bradley, the blacksmith, who is described as a “man of unusual physical strength”, a “fighter” who had been “shockingly brutal and profane” and had “killed more than one man”. However, as a result of the preacher’s work, Bradley is described as having converted to religion and consequently as having undergone a “remarkable change”. The story is set at the beginning of the Civil War and it describes how the mountaineers split up into secessionists and adherents to the Union, with Father Peters and Bradley belonging to the latter group. As the Union men are in a minority, it is dangerous to be open about their political views, and in the part quoted below Bradley warns Father Peters of the danger posed by the secessionists and urges him to leave. Father Peters, however, refuses to flee, and in the course of the story he is attacked by his opponents (led by a mountaineer called Het Magoone), violence erupts, Bradley gives up his religious sentiments to become a fighter again (for the preacher and the cause of the Union) and ultimately even Father Peters takes up the rifle and shoots enemies before he is killed at the end of story.

The mountaineers are generally depicted in a very negative light and the narrator is condescending in his description of them. The main contrast established in the story is that between the north and the south, with the mountaineers representing the worst part of southern culture. The northern influence, symbolized by Father Peters, is the only positive element in the mountaineer region, but his success (visible in Bradley’s conversion) is ruined when Bradley turns his back on religion again to fight back his opponents. The fact that even Father Peters takes up arms against his religious convictions conveys the strength of the negative southern influence – however, as both Bradley and Father Peters fight for the Northern cause, their fighting is justified, whereas the secessionist mountaineers are condemned because they have caused the fighting by their attack (especially by setting fire to the meeting house of the Union men). While Bradley’s animalistic uncivilized fighting instincts cannot therefore be changed through religion, Father Peters has had enough influence on him that he at least uses them for a good cause. All in all, the culture of the north is portrayed as superior on all levels, and this superiority is constructed to a great extent in relation to religion. The fact that the mountaineer community views Bradley’s conversion to religion very negatively (they find that “religion had took all the pluck and snap out of strong Dick Bradley”) shows that they associate religion (and civilized behavior) with weakness, but the story ultimately presents the religious figure, Father Peters, as the hero: The mountain missionary becomes the martyr symbolizing the enormity of northern strength and efforts to bring some civilization to the southern mountains.

The speech of the mountaineers is represented as different from that of the narrator and that of the preacher and his wife and daughter. In comparison to the linguistic repertoire in the article “The Pool Tribe”, published ten years earlier, non-rhoticity has remained a form marking the mountaineers’ speech (\emph{powaphul}, \emph{no’th}, \emph{fo’}, \emph{ovah}, \emph{heah}). This is not surprising given its salience in the humorous dialogue “Safer, Too”, but it is noticeable that in this story not only do some forms remain rhotic (\emph{yours}, \emph{yer}, \emph{thar}, \emph{Lor’}, \emph{sarmint} ‘sermon’) but that hyper-rhotic forms occur as well (\emph{kinder} and \emph{tomorrer} ‘tomorrow’). This combination of non-rhotic with hyper-rhotic forms is also present in eight articles representing Black speech. Furthermore, there is one case of deletion of pre-vocalic /r/ following a fricative consonant (\emph{f’om}). So far, this deletion has only been represented in the poem “When de Co’n Pone’s Hot” (\emph{th’one}), which makes it another case of an overlap between the linguistic repertoire of Black speakers and mountaineers. Further overlapping forms are final consonant cluster reduction (\emph{mos’}, \emph{Lor’}), a raised \textsc{dress} vowel in \emph{get} (\emph{git}), a lowered \textsc{kit} vowel in \emph{if} (\emph{ef}), a \textsc{choice} vowel nearing \textsc{price} (\emph{p’int} ‘point’ occurs in a part of the story not quoted here) and a lower \textsc{square} vowel (\emph{thar}, \emph{harabouts}) as well as alveolar -\emph{ing} (\emph{brewin}, \emph{needin}). A lower \textsc{square} vowel in \emph{where} is also found in Colonel Gutrippah’s repertoire and thus linked to southern white speech as well. Alveolar -\emph{ing} is also found in “The Pool Tribe”. Two further forms shared with “The Pool Tribe” are representations of a lower \textsc{nurse} vowel (here \emph{sarmint} ‘sermon’ and \emph{larnt} ‘learnt’ in “The Pool Tribe”) and metathesis of /r/ and a vowel (here \emph{purtending} ‘pretending’ and \emph{purty} ‘pretty’ in “The Pool Tribe”). In contrast to the linguistic repertoire represented in “The Pool Tribe” and to representations of Black voices, TH-stopping is not marked here. It is also not represented in the humorous dialogue “Safer, Too”, which suggests that even though it is used by the mountaineer in “The Pool Tribe”, it is not a salient form linked to mountaineer speech. It is a possibility that TH-stopping is used in “The Pool Tribe” to emphasize the “queer” race of the mountaineers (in contrast to white southerners) because since TH-stopping occurs in all representations of Black speech, it is constructed as an index of racial difference. In the story “A Mountain Missionary”, no effort is made to present the mountaineers as racially different, but the emphasis is rather on cultural differences, especially in contrast to the north of the United States. It is also possible, however, that the linguistic repertoire associated with the mountaineer has changed over the ten years between the appearance of “The Pool Tribe” and “Safer, Too” and “The Mountain Missionary”.

\largerpage
Grammatically, the mountaineers in this story use \emph{ain’t} and negative concord (\emph{Thar hain’t no one}) and demonstrative \emph{them} or the double demonstrative \emph{this}/\emph{these here} (\emph{these har hills}). These forms occur in representations of Black as well as of white southern speech. A frequent form that was also linked to Black speech is \emph{a}{}-prefixing. An invariant participle in \emph{some men so give ovah to sin} is reminiscent of invariant past tense forms found in “An Unwelcome Fifteenth” and “Dat Deceptious Mule”. Regarding the pattern of subject-verb agreement, it can be observed that the second-person singular -\emph{s} marking, which is salient in “The Pool Tribe”, is not represented here (\emph{Father Peters, yo’ve preached}). First-person plural marking seems inconsistent because the form \emph{has} is used (\emph{we’uns who’s fought}) alongside unmarked forms (\emph{we want}). Similarly, regarding second-person plural forms, the form \emph{is} occurs alongside unmarked forms, as in \emph{yo’ uns that don’t want to heah Father Peters preach is free to leave}.

A very salient form that is only represented in “A Mountain Missionary” is the use of the first- and second-personal pronouns \emph{we’uns} and \emph{you’uns} (but \emph{we} and \emph{you} are used as well). They are highlighted in the following exchange between Bradley and Het Magoone: “‘Who says so?’ demanded Bradley. / ‘We uns.’ / ‘And who’s yo’uns?’” The pronouns are thus prominently linked to the mountaineers and serve as a contrast to white southern speech, indexed by the personal pronouns \emph{we all} and \emph{you all}. This contrast is not established in the story itself, but in relation to other stories, as for example the anecdote about Colonel Gutrippah. What is also noticeable is the frequent use of \emph{as} as a relative pronoun (\emph{the man ez built hit}, \emph{the man ez kem down heah}), which contrasts with the use of the relative pronoun \emph{what} in representations of Black or white southern speech in other articles (e.g. \emph{de mule wot you bin zoonin’ for} in “Dat Deceptious Mule”).

Finally, the representation of the mountaineer’s speech is also marked by the forms \emph{hit} and \emph{hain’t} and the frequent use of eye dialect, e.g. \emph{sespichis} ‘suspicious’, \emph{ken} ‘can’, \emph{kem} ‘come’, \emph{dainjah} ‘danger’, \emph{hull} ‘whole’, \emph{ez} ‘as’, \emph{religin} ‘religion’, which has so far also been observed in some articles representing Black speech. The use of eye dialect increases the perceived distance of the linguistic repertoire of the represented voice to that of the ‘neutral’ voice of the author or narrator and through the association of eye dialect with illiteracy the speaker is characterized as uneducated, which fits the depiction of the mountaineer as uncivilized and belligerent. An element of humor can be found in the representation of /f/ by <ph> in \emph{powaphul} ‘powerful’, because the graphemes <ph> are normally only found in Greek loanwords, which are associated with a high degree of education. The contrast between this representation and the other forms linked to a lack of education (among others non-rhoticity in the same word) creates irony intended to entertain the educated readers. The impression of the uncivilized and uneducated mountaineer is strengthened further on the lexical level through the frequent use of \emph{doggone}, as in \emph{they’ve got so doggone sespichis} in the quotation below, because \emph{doggone} is a swear word (according to the \emph{OED} \citeyear{doggone}, it is probably “a euphemistic alteration of \emph{God damn}”) and frequent swearing is associated with impolite and uncivilized behavior.

\begin{ipquote}
\begin{center}
\textstyleStrong{A MOUNTAIN MISSIONARY.}\\
By Alfred. R. Calhoun\\
{[Copyright, 1895, by American Press Association.]}\\
CHAPTER 1.\\
\textsc{why he went to Kentucky}   
\end{center}
{Mr. Robert Peters, or Father Peters, as he was called by the mountaineers, was born in Ohio. He was a Cambellite clergyman, and ten years before the war he, with his wife and daughter, moved into the Cumberland mountains in southeastern Kentucky. Only an intense religious spirit could have induced Father Peters to leave his home in the rich lands of the western reserve and to take up his abode among the hills of the Cumberland range. It would be difficult in the United States to find a place and a people in more violent contrast with his old associates. Wealth, or at least comfort, and intelligence were the rule in northern Ohio. There was hardly an exception to poverty and ignorance in his new home.

\centering
{[...]}

Bradley, the blacksmith, who was Mr. Peters’ nearest neighbor, was a man of unusual physical strength, and before the coming of the preacher he had been noted as a fighter.} {Indeed he was credited with having killed more than one man. He had been shockingly brutal and profane and was known far and near as strong Dick Bradley to distinguish him from a cousin of the same name, who was not that strong. If the only good done by the clergyman had been the conversion of the blacksmith, his work in the mountains might well be considered a success. Bradley no longer swore nor drank, nor had he had a fight for years. This remarkable change, added to the fact that he was a man of few words, gave the impression to many that “religion had took all the pluck and snap out of strong Dick Bradley.”

\centering
{[...]}

The blacksmith pulled his chair nearer, and with his big hands to the sides of his mouth to shield his voice he whispered:}

“\textstyleStrong{I’m sorry to tell yo’, Father Peters, that there’s trouble a-brewin harabout, and lots of hit.}”

“Trouble to whom, Brother Bradley?”

“\textstyleStrong{Can’t yo’ guess?}”

“I cannot.”

“\textstyleStrong{Waal, hit’s to yo’ and yours,}” said the blacksmith, with an emphatic shake of the head.

“But surely no one could wish to annoy me. I did not think I had an enemy in the world,” said the alarmed clergyman.

“\textstyleStrong{That’s jest hit,}” responded Bradley. “\textstyleStrong{Thar hain’t no one ez do{\kern0pt}esn’t allow yer the best preacher in the mountains, but the boys say they don’t like yer politics, and so they’ll make trouble.}”

Father Peters protested that he had not meddled in politics, and that since the breaking out of the war a few months before he had guarded his words that he might not give offense.

{“\textstyleStrong{That’s hit. That’s why} \textstyleStrong{they’ve got so doggone sespichis}\textstyleStrong{. Now, Father Peters, yo’ know I’ve allus been a good friend of yer’s, ez I should be a blamed dog not to be, seeing that you took me by the hand and led me up to the light, bless the Lor! But thar’s dainjah all about we uns, mos’ powaphul dainjah, ez me and the wife allowed this night. And we said that ef so be yo’ could go no’th fo’ awhile till the trouble kinder blows ovah that hit’d be bettah fo’ yo’ and fo’ yer friends.}”

\centering
{[...]}\\
{[The secessionist Het Magoone addresses Bradley and Father Peters:]}


“\textstyleStrong{We uns who’s fo’ the south ken wait to git even with yo’ uns who’s fo’ the Union. Thar’s plenty of time to settle them things, but what we want now is to git rid of a man f’om the no’th who comes down har ez a spy and purtending that hit’s God Almighty’s religin. Father Peters, yo’ve preached yo’r last sarmint in these har hills}.”}

“\textstyleStrong{Who says so?}” demanded Bradley.

“\textstyleStrong{We uns.}”

“\textstyleStrong{And who’s yo’uns?}”

“\textstyleStrong{Mo and my friends.}”

\newpage
{“\textstyleStrong{Neither yo’ nor yer friends owns a splinter of this house. From foundation log to clapboard hit’s owned by the man ez built hit—the man ez kem down heah to lead us to do right—but thar’s some men so give ovah to sin, Het Magoone among ’em, that they’re bound to be damned, and I’m mighty glad of hit. Now, yo’ uns that don’t want to heah Feather Peters preach is free to leave. But, by G—, the next man ez tries to break up our worship will find himself needin a hull new top to his head!}”

\raggedleft
{[emphasis mine]}\\
}
\end{ipquote}

To conclude the qualitative analysis of articles containing \emph{bettah}, I have shown that the term and the phonological form it represents are linked to very different groups of speakers, including  speakers contrasting on several social dimensions, like, for example, the white upper-class southern girl, and the Black lower-class man with fifteen children that he can barely feed. This underlines the importance of considering not only one linguistic form in isolation but the way that this form is combined with other forms to create contrasting voices linked to contrasting social characteristics and values. The analysis of the articles has revealed that non-rhoticity is not the only form that is linked to all three main speaker groups, to Black Americans, white southern Americans and mountaineers. Unstressed initial syllable deletion, a lower \textsc{square} vowel, particularly in the function words \emph{here}, \emph{there} and \emph{where}, a lower \textsc{nurse} vowel, different patterns of subject-verb agreement, negation with \emph{ain’t} and negative concord, the double demonstrative \emph{this here} (and related forms), the modal \textsc{better} construction, the forms \emph{hit} and \emph{hain’t}, and the lexical item \emph{I reckon} and the address term \emph{sah} ‘sir’ all occur in the speech of at least one mountaineer, one white southern speaker and one Black American. How are differences between these groups indexed then?



\begin{figure}[b]
\includegraphics[width=.8\textwidth]{figures/Paulsen-img30.png}
\caption{
Humorous dialogue and illustration of a rich and a poor American, published in the \emph{Atchison Daily Globe} (Atchison, Kansas) on September 11, 1888\citesource{September111888}, retrieved from Nineteenth-Century U.S. Newspapers
}
\label{fig:key:30}
\end{figure}

One important element also regards the presence or absence of /r/: Many representations of Black American and of mountaineer speech exhibit several forms that are \emph{not} marked as non-rhotic and, in addition, hyper-rhotic forms. In contrast to that, there is only one article in which several forms of white southern speech are not marked as non-rhotic and that is “First Impressions Entering the Field”, which discusses lower-class white southern Americans and highlights their similarities to Black southern Americans. Hyper-rhotic forms do not occur in any of the representations of white southern speech. How can the co-occurrence of non-rhoticity and hyper-rhoticity in Black and mountaineer speech be explained then, especially if it is taken into account that the analysis of articles containing \emph{deah} AND \emph{fellah} revealed that these two forms constitute end points on a continuum representing the difference between nature (the hyper-rhotic cowboy) and culture (the non-rhotic dude)? First of all, it is important to point out that there are also articles containing \emph{bettah} in which such a continuum is visible, as the article “A Score” shows, which consists of a humorous dialogue accompanied by an illustration of the speakers (\figref{fig:key:30}). It was published in the \emph{Atchison Daily Globe} (Kansas) on September 11, 1888\citesource{September111888}, and represents the voices of two speakers, Mr. Delawney and a man labeled “Hard Up Gent”. The Hard Up Gent is obviously poor and asks Mr. Delawney for a few cents, but the request is declined by Mr. Delawney, who arrogantly accuses the Hard Up Gent of not having brains enough to care for himself by saying “You had bettah ask for bwains instead of money”. To this the Hard Up Gent replies “Well, boss, I asked yer for what I thought yer had the most of”, which not only shows that he is very smart but also critically exposes Mr. Delawney’s unjustified arrogance and makes him appear more ignorant than the Hard Up Gent. The Hard Up Gent’s speech is marked by hyper-rhotic forms (\emph{yer}, \emph{feller}), while Mr. Delawney’s speech exhibits non-rhoticity in \emph{bettah}, \emph{faw} and \emph{yaw}. As non-rhoticity is combined with a labiodental realization of /r/ (\emph{bwains}), Mr. Delawney exhibits two typical forms of the repertoire of the dude or the swell. He therefore represents the end point of culture (pointing out the importance of education and intelligence to lead a successful life), while the Hard-Up Gent represents the end point of nature (having to struggle to fulfill his basic natural needs). The humor of the dialogue is created by showing how the poor man outwits the swell, who thinks so highly of himself, and it therefore serves to criticize upper-class arrogance towards the poor.


I suggest that by keeping rhotic forms and by adding hyper-rhotic forms to the voices of mountaineers and Black Americans, the indexical link to nature (as opposed to culture), which is also established in the dialogue in “A Score” to characterize the Hard Up Gent, is used and strengthened. While in the case of the Hard Up Gent, nature is presented positively by showing how his intelligence and wit make him superior to the over-cultured dude, it is portrayed in a mostly negative sense in the case of Black Americans and mountaineers, however. Nature stands for the absence of education and civilized behavior, which leads humans to rely on their physical strength and animal-like instincts instead of their intellect and morals. For example, the fifteen children of the Black man in “An Unwelcome Fifteenth” imply a lack of sexual restraint having negative consequences for the whole family, the Black man’s use of religion to justify stealing a watermelon to satisfy his craving for the tasty ripe fruit in “Cato’s Soliloquy” suggests a disregard of morals, and the mountaineer helping the reporter in exchange for whiskey in “The Pool Tribe” exposes him as as a person who is very fond of alcohol and whose greatest skill is killing rattlesnakes. In these cases, non-rhoticity does not index culture or refinement in any way, but the main indexical link created in these articles is that between non-rhoticity and lack of culture and civilization. In this line of argumentation, it makes sense that the articles representing white southern speech do not contain hyper-rhotic forms because southerners are not presented as uncivilized or animalistic. On the contrary, in several articles their system of cultural values is implicitly put in relation to that of the north, usually with the aim to criticize it as traditional, conservative, hierarchical and backward in contrast to the more modern and progressive northern one. It is therefore possible that non-rhoticity indexes culture and refinement in these articles as well, especially in those cases where upper-class southerners are portrayed, for example the southern girl in New York. As non-rhoticity also occurs in lower-class southern speech, however, I suggest that the main indexical link is that between non-rhoticity and southernness and this link is also present in the case of Black Americans and mountaineers because these groups are also predominantly associated with the American south in the nineteenth century. While non-rhoticity therefore indexes a variety of social categories and meanings, the addition of hyper-rhotic forms puts an emphasis on the indexical link to natural and uncivilized behavior and cancels potential links to culture and refinement.


Another form that is present in representations of Black American and mountaineer speech is voiced interdental fricative stopping. As this form is the only form next to non-rhoticity that is marked in all representations of Black American speech, it seems to index ethnicity most strongly. I have argued above that this link could also have been used create the impression of racial otherness of the mountaineers in “The Pool Tribe”. However, considering the analysis of \emph{deah} AND \emph{fellah}, voiced interdental fricative stopping is also marked in the speech of the white Philadelphian “waifs”, which suggests that while the form is an important index of ethnicity, it can also index a lower-class status, lack of education and uncivilized behavior more generally. In contrast to voiced interdental fricative stopping, voiced \emph{labiodental} fricative stopping occurs in the speech of Black Americans only, which makes it a strong index of Black American speech. It is present in the majority of articles analyzed here, which makes it all the more striking that it was marked in the first version of Dunbar’s poem “When de Co’n Pone’s Hot”, but not in the versions published after he had become famous. As Dunbar’s poem is an exception in that it creates a positive image of Black American life and identity, this change could indicate that especially voiced labiodental fricative stopping might index very negative social values or perhaps that it became less frequent in actual use and thus less suitable as a marker of Black voices at the end of the nineteenth century. More analyses focusing on this form would be needed to support these suggestions, however.

Further forms which are frequently used in the articles to mark Black American speech are alveolar -\emph{ing} and final consonant cluster reduction. Both forms appear in mountaineer speech as well, which shows that they do not only mark ethnicity, but that they can also mark lower-class status and lack of education. Less frequent forms which are also part of the repertoire of both Black Americans and mountaineers are the deletion of pre-vocalic /r/ in initial consonant clusters, the realization of the \textsc{choice} vowel nearing the \textsc{price} vowel, a raised \textsc{dress} vowel in \emph{get} and a lowered one in \emph{if}, demonstrative \emph{them}, alternative reflexive pronouns and \emph{a}{}-prefixing. In some articles, eye dialect is used to highlight the illiteracy of the speakers. Forms that distinguish the two social groups are fewer, but I have shown in the analysis how they are highlighted: In the case of the mountaineers, it is particularly the personal pronouns \emph{we’uns} and \emph{you’uns} that contrast them with Black as well as with southern speech. Lexical items like the swear word \emph{doggone}, the phrase \emph{I’ll haf to go you} and the address term \emph{mistah} also occur only in representations of mountaineer speech. The relative pronoun \emph{as} differentiates them from white southern and Black voices, which are sometimes marked by using the relative pronoun \emph{what}. A very salient element used in some articles to distinguish Black American speech from mountaineers (and also from white southern Americans) is the use of malapropisms. They serve to highlight the failed attempt by some Black speakers (especially Black preachers) to appear educated and to mark a higher social position. That the mountaineers are not portrayed as using malapropisms shows that they are characterized as not even trying to appear civilized or educated – they are rather depicted as being content with their uncivilized life full of violence and hardship.

With regard to white southern speech, there is one form present in four of the five articles (next to non-rhoticity): the address term \emph{sir} (spelled <sah>). The term is also used by the Black man in “An Unwelcome Fifteenth” and the mountaineer in “Safer, Too”, which shows that it is used to index differences in social hierarchies between the speakers, but its frequent use to mark southern speech shows that it is used to emphasize the southerner’s hierarchical social system. Forms restricted to southern speech are the lexical item \emph{How’dy} and the different use of the greeting \emph{good evening} and the personal pronouns \emph{we all} and \emph{you all}, but they all occur in one article only and are thus not as prominent as other forms here. (However, a detailed analysis of more articles could provide more insights on their prominence.) Two phonological forms unique to southern speech are the spelling of the article \emph{the} as <they> and, in one instance, the spelling of the \textsc{dress} vowel as <ai>, which could be representations of diphthongizations typical of the so-called southern “drawl”.

Considering all articles containing \emph{bettah} analyzed here, a general observation that can be made is that representations of Black speech usually contain a much higher number of linguistic forms marked as different from other voices. Even very short stretches of speech, like Uncle Eben’s “Philosophy” or the caption under the cartoon showing the two boys and their dogs, exhibit roughly the same amount of forms as the much longer quotations of Colonel Gutrippah’s speech or the janitor’s speech in “Relics of the Past”. The highest number of deviating linguistic forms also occurs in two texts representing Black voices: the short story “Dat Deceptious Mule” and “Cato’s soliloquy”. Regarding the sheer frequency of deviating features, representations of mountaineers come second, but as the example of the mountaineer in the humorous dialogue “Safer, Too” illustrates, a restriction to a smaller number of forms is possible, too. Those are non-rhoticity, unstressed initial syllable deletion, a lower \textsc{nurse} vowel and the address term \emph{sah}, so phonological and lexical forms only, which illustrates the emphasis put on the phonological and the lexical level when only a reduced number of forms can be represented (due to the shortness of the dialogue). The voice that is least marked is that of the upper-class southern girl. While she exhibits several non-rhotic forms, the only other forms are yod-insertion after a velar consonant and a lower \textsc{nurse} vowel, which are both indicated in the spelling of only one lexical item (<gyuls>). This establishes a parallel to the representations of the swell and the dude, whose speech is also characterized by a reduced set of forms, which marks them as clearly different and links them to specific social characteristics, but which is at the same time not deviant enough to make them appear uncivilized or uneducated.

All in all, the analysis reveals the complexity in the creation of indexical links between linguistic forms and social characteristics and the high importance of context in the interpretation of these links. To illustrate this important conclusion, I will analyze one last example based on the findings generated in the analyses of the search terms \emph{deah} AND \emph{fellah} as well as \emph{bettah}. It is a report about a new fashion item, shoes of yellow colored leather labeled “Yellow Shoes” in the heading, which was published in the \emph{Evening News} in San Jose, California, on July 10, 1889\citesource{July101889}. The report quotes “a bootblack” as saying “You bettah take dem shoes off; you catch rumahtism suah!” Even without any further information about the bootblack, the representation of his speech indexes his ethnicity. The combination of non-rhoticity, demonstrative \emph{them}, voiced TH-stopping and eye dialect could possibly also index a mountaineer, but it is highly unlikely that a mountaineer would work as a bootblack in a city. This particular combination of linguistic forms makes it clear to the reader that the figure of the dude or the swell and social characteristics associated with this figure cannot be evoked here. In the following part of the report, a “leading dealer in shoes” is quoted as saying that these shoes are particularly fashionable in the east and that they are sold in equal numbers to black shoes there, but that “in San Jose our customers are afraid of them, fearing they may ‘swell’”. This indicates that Americans living in the west are afraid of being negatively evaluated as eastern swells. The dealer is convinced, however, that this fear will “wear off however as many gentlemen are now wearing them, who have nothing, either in their manner or appearance to indicate the ‘swell.’” This shows that there is an awareness of which perceivable signs index a swell, but language is not the focus here. Non-rhoticity in \emph{bettah}, which could be interpreted as a form indexing a swell if one is aware of the figure and the signs indexically linked to it, is however not linked to the swell here at all. Although the article is concerned with both figures, the Black American and the white swell, and therefore offers a \emph{potential} for recognizing the overlap in their linguistic repertoires, this overlap does not play any role at all, because the linguistic as well as the non-linguistic context clearly mark the bootblack as a Black speaker and because it does not invite the reader to create or draw on an indexical link between non-rhoticity and the swell that they might use or recognize in other contexts.

\begin{ipquote}
\begin{center}
\textstyleStrong{YELLOW SHO{\kern0pt}ES.}\\
They are Universally Condemned by the Bootblacks.\\
\textstyleStrong{Said to Attract Rheumatism–The Sale of the Sho{\kern0pt}es Increasing–Extensively Worn in the East}
\end{center}
“\textstyleStrong{You bettah take dem sho{\kern0pt}es off; you catch rumahtism suah!}” said a bootblack to a young man who passed his stand this morning, wearing sho{\kern0pt}es of yellow colored leather.

As the young man passed on, the bootblack stated that before the wearer of the sho{\kern0pt}es had adopted the yellow leather for footwear he had patronized the sho{\kern0pt}e polishing establishment to the amount of \$2.50 to \$3 a month, and that if the custom of wearing canvas and light colored sho{\kern0pt}es continued, and it seemed to be growing in popularity, the trade of the sho{\kern0pt}e polisher would receive a severe blow.

{A leading dealer in sho{\kern0pt}es, speaking of the sunset leather and canvas sho{\kern0pt}es, said “We do not sell a great many of these sho{\kern0pt}es but the demand seems to be increasing. The majority of calls for these sho{\kern0pt}es come from Eastern people and as we do not carry a great variety of styles we cannot always suit them. I am told that in the East the sale of them is equal to that of the ordinary black leather, but \textstyleStrong{in San Jose our customers are afraid of them, fearing they may ‘swell’}. This will soon wear off however as many gentlemen are now wearing them, who have nothing, either in their manner or appearance to indicate the ‘swell.’ As a matter of economy the brown and yellow colored sho{\kern0pt}e will receive ready endorsement, as the amount of money spent with the bootblacks on one pair of black leather sho{\kern0pt}es would pay the original price of the sho{\kern0pt}es several times over.”

\raggedleft
{[emphasis mine]}\\
}
\end{ipquote}


The last phonological form that I will analyze here is yod-dropping based on the collection of articles containing the search term \emph{noospaper}. In the next section I will show how this phonological form connects to the linguistic repertoires and indexical links identified in the analyses above and how it also adds to them in ways that are important for defining an ‘American’ register.


\subsubsection{\emph{noospaper/s}}
\hypertarget{Toc63021239}{}
The first article in the databases that contains the search term \emph{noospaper} was published on November 29, 1816\citesource{November291816}, in the \emph{Western Monitor} in Lexington, Kentucky. It is a letter taken from the \emph{National Pulse}, another Kentucky newspaper, and addressed to “Doctor Hun”, that is to Dr. Anthony Hunn, who was the editor of the \emph{National Pulse} at the time. The writer of the letter, who signs his name as Fearnawt Burly, expresses his pleasure and agreement with a person who, in an article for a Frankfort newspaper, threatened someone (“that feller”) who wanted to “skews” (‘excuse’) Gabriel Slaughter, who was the Governor of Kentucky at the time.\footnote{Gabriel Slaughter, a member of the Democratic-Republican Party, became Governor in October 1816. Shortly after, he appointed John Pope as Secretary of State, which was criticized by Republicans because Pope was considered to be a Federalist.} The threat is quoted and involves physical violence (“I will give it to you under the short ribs”) and this violence is emphatically celebrated by the writer of the letter (“Yes, thats right! gouch hem! bite ’m! kick em, Hurrah for libberty!”). The political stance associated with this threat and celebration of violence is republicanism – the writer of the article for the Frankfort newspaper is said to be “like a troo republican”. The writer of the letter in the \emph{National Pulse} conveys his understanding of republicanism by emphasizing the high value placed on liberty (“Hurrah for libberty!”) and the violent nature of their debates, which do not really allow for contradictions and arguments (“If one said tit to my tat—pang! He should have it behind his yeers”). The positive evaluation of this violent way of arguing and achieving liberty is underlined by contrasting Republicans with “book rats”, a derogatory term for highly educated people, which suggests that the writer of the letter does not evaluate education and a civilized, well-informed argument very positively.


The telling name of the author (Fearnawt Burly) and the exaggerated praise of the Republicans already reveal that this letter has not really been written by a Republican writer, but that it is a piece of fiction full of irony used quite to the contrary to criticize Republican politics and manners. This is not surprising given that the \textit{Western Monitor} was a Federalist newspaper. The linguistic form of the letter is an important part of conveying this criticism. Its most prominent feature is the highly frequent use of eye dialect to create the image of an uneducated and almost illiterate writer (e.g. <munstrusly> ‘monstrously’, <riter> ‘writer’, <Slawter> ‘Slaughter’, <wood> ‘would’, <manached> ‘managed’, <troo> ‘true’). The style is more spoken than written, which is underlined for example by the use of the onomatopoeic word \emph{pang}, and which creates the impression that the writer is not familiar with conventions of written texts at all. The frequent exclamatory sentences suggest that the speaker is emotionally highly involved, and they also serve to characterize the writer as a typical Republican – aggressive and impulsive rather than thoughtful and analytic. Spellings indicating an actual difference in pronunciation are very few. Next to yod-dropping in \emph{noospaper}, there is also a case of initial yod-insertion before a vowel in \emph{yeers} ‘ears’. The spelling of \emph{contradict} (<cunterdict>) indicates a metathesis of /r/ and the following vowel, and \emph{feller} represents a case of hyper-rhoticity. Grammatical forms are first-person singular \emph{is} and the regularized past participle \emph{showed}. Against the background of the frequent use of eye dialect, these forms marked as constituting a differentiable linguistic repertoire are not very salient. The main social characteristics that they are indexically linked to are a Republican political stance, and connected to that, a way of argumentation emphasizing physical strength over civilized and educated behavior. Interestingly, the presence of the hyper-rhotic form \emph{feller} in this article indicates that the link between hyper-rhoticity and the nature end of the nature-culture continuum is already established early in the nineteenth century. The fact that the political debate takes place in Kentucky and in Kentucky newspapers could also link yod-dropping to the south, but there is no indication as to which place the south and southern speech forms should be contrasted to, which makes this not a very salient indexical link.

\begin{ipquote}
\begin{center}
\textit{From the National Pulse.}\\
\end{center}
Doctor Hun!

{I is munstrusly pleased with that riter in the Frankfurt noospaper, who like a troo republican cuts the matter short with that feller who wants to skews Slawter. He says, says he, “if you dont treet that [?] Mungomery with more respect, I will give it to you under the short ribs” Yes, thats right! gouch hem! bite ’m! kick em, Hurrah for libberty! Was I at Frankfert, I’d show them what it was to cunterdict mee! If one said tit to my tat—pang! He should have it behind his yeers. I wood not let them speechify matters in the sembly at all at all! If too of them did cunterdict each other I wood have a ring made round them on the spot, and then, hurrah bois, for libberty! Who hallers enough! Shall be in the rong—woodent that be a show and cheap way of carrying on the Government? That is the way Harry Clay fist his business—and if a man says Harry Clay is not a troo republican, he is a d—d lire! Let ’em come to mee till I put it to him under the short ribbs!

\centering
{[…]}

Docter, Docter! You dont know what we call heer Republicanism!} {If the war had lasted two or three yeers longer, we would have show’d you book rats how the rites of the people shood be manached!—Hurrah for liberty!

\raggedleft
FEARNAWT BURLY.\\}
\end{ipquote}

The second article containing \emph{noospaper} is special because it was reprinted frequently in several newspapers over a time span of eight years (1826–1834) and across a geographical area including Louisiana and Florida in the deep south, North Carolina and Washington D.C. further north and five New England states (Maine, Vermont, New Hampshire, Massachusetts and Rhode Island) in the far northeast of the United States. The original article published in the \emph{Louisiana Advertiser} is not contained in the databases, but it is indicated as the original source in all the other articles. The first article which appears in the databases was published in the \emph{Carolina Observer} (Fayetteville, North Carolina) on June 21, 1826\citesource{June211826}. It is a long dramatic text consisting of stage directions and dialogues between three characters, Mr. Eagernoos, his son John and his neighbor Swallow, which constitute one scene which takes place early in the morning. Mr. Eagernoos sends his son John to the neighbor to borrow a newspaper. He comes back without one and explains to his father how all of the neighbor’s newspapers are already being read by other neighbors. This makes Mr. Eagernoos angry and impatient and he sends his son away to try again. After he exits, Swallow enters the scene and they talk about the problem of obtaining a newspaper to read in the morning, and when John returns several times empty-handed, Eagernoos finally decides to get a subscription to the newspaper to “not be so pestered with the trouble of borrowing from unaccommodating neighbors” anymore. The main message created by the scene is that it is beneficial to have a subscription to a newspaper instead of having to rely on neighbors to borrow them. The text therefore functions as an advertisement for the newspapers, which might be the main reason for the popularity of the article. Newspaper editors and printers were likely to have had an interest in illustrating the advantages of a subscription.

It is in the nature of a dramatic text that the voices of the characters are represented directly. A contrasting linguistic repertoire is created here by marking several forms in Swallow’s speech as different from that of Eagernoos and his son. A very prominent form is yod-dropping because it is the only one that is particularly highlighted. Two strategies are used to achieve this: First, Swallow pronounces his neighbor’s name as \emph{Eagernoos} – the spelling <oo> indicates the yod-less pronunciation of the telling name \emph{Eagernews}. Second, Eagernoos explicitly comments on Swallow’s alternative pronunciation: “you are always inquiring after “Noos” as you call it”. In addition to yod-dropping, Swallow also exhibits a metathesis of /s/ and /k/ in \emph{ask} (\emph{ax}), one instance of definite article reduction (\emph{t’other}), an instance of invariant \emph{be} (\emph{be they reading them now?}) as well as an instance of demonstrative \emph{them}, a relative pronoun \emph{what} and a third-person plural -\emph{s} marking in \emph{them folks what brings the paper}. In general, it is thus Swallow’s speech that is marked as deviant and it is indexically linked to his bad character, which becomes evident when he suggests that Eagernoos should only pay the five dollars asked for at the beginning of the subscription, but not the five dollars at the end because he can count on getting the paper despite not paying the bill. Swallow is thus constructed as a negative example of a subscriber, whose behavior is causing great damage to newspapers. With regard to yod-dropping, it is noticeable, however, that Eagernoos also uses the form \emph{noospaper} once. However, as he explicitly distances himself from saying “noos”, and as he is only shown to drop /j/ this one time, the link between his character and the linguistic form is rather weak. In general, the article indicates that yod-dropping is used in combination with other linguistic forms to underline negative character traits and behavior (Swallow’s immoral attempt to get newspapers without fully paying for them) and to contrast them with the linguistically unmarked speech of a character with positive character traits and behavior (Eagernoos’s change from immorally attempting to read the neighbor’s paper to subscribing for a paper, paying even more than is required and deciding not to lend it to anyone, is precisely the kind of behavior that is beneficial for the newspaper business). As the scene takes place in the south, an indexical link to southernness could also be formed, but, as in the first article, it is not very salient, especially considering that it is mostly Swallow whose speech is marked as deviant. The focus of the article is thus rather on social contrasts than on regional contrasts.

\begin{ipquote}
\begin{center}
\textsc{From the Louisiana Advertiser.}
\end{center}
Oh that my enemy would—Take a Newspaper.

“John! Oh John!—do you hear? run to neighbor Liberal’s and ask him if he will oblige me by the loan of the morning’s paper a few moments, just to look at the ship-news and the advertisements.”

“That’s just what I said yesterday morning, daddy, when I went to borrow the paper, and you know you kept it two hours and he was obliged to send for it.”

“Well, then say something else to him, John, do you hear, John? and give my compliments, John, do you hear?

“Yes, daddy.” (\textit{Exit and returns}.)

“Well, John, have you got the paper!”

{“No, daddy, neighbour Liberal is walking about the room waiting for Mr. Newsmonger to finish reading the Louisiana Advertiser, or Mr. Longwind to drop the Gazette, which he has got almost asleep over.”

\centering
{[…]}

(\textit{Enter Swallow}.)}

“\textstyleStrong{good morning neighbour Eagernoos—any thing noo?}“

“New! fire and faggots, I have sent a dozen times to Liberal there, to request the loan of his paper, only for a moment, and he has the impertinence to refuse me.”

“\textstyleStrong{Refuse you?}”

“Not exactly refused me, but he permits such fellows as Longwind, Neitherside, Scribelerus, and Newsmonger, to pore over them for hours, not only (through a mistaken courtesy,) depriving himself, but his neighbors, from getting early intelligence of that is passing in the world.”

‘\textstyleStrong{My goodness!—be they reading ’em now}?

“Yes” (\textit{sighing})

“\textbf{Well,} \textbf{that’s} \textbf{abominable!} \textbf{Why} \textbf{dont} \textbf{you} \textbf{take} \textbf{a} \textbf{Noospaper} \textbf{yourself?}”

“Why dont you take one? you are always inquiring after “Noos” as you call it.”

“\textstyleStrong{Why I did take one, but the printers dont leave it at my house any more, ’cause I hackeled about the price, and wood’nt pay him}.”

“That’s a good reason for the printer, if it is none for you. Well, John, did you get the paper.”

“No, daddy, just as that Mr. Neitherside was done, in come Mr. Hookit and Mr. Knabit, and I come back.”

“Confound my ill luck!—go back, do you hear? and ask Mr. Liberal if he will be kind enough—do you hear? kind enough to lend me any northern paper he may have, or if he has not one, ask him to lend me yesterday’s paper again, or the day before, or the day before that, or last Saturday’s, or, do you hear? any of the last week’s papers, do year?

“Yes daddy.”

“I am determined on going right away and subscribe for a noospaper: I will not be so pestered with the trouble of borrowing from unaccommodating neighbors.”

“\textstyleStrong{You are right, Mr. Eagernoos, the printers only ax five dollars right down, and then you have a whole year to pay t’other five dollars in, and then you can dispute the bill, and they will send the noospaper three months after that afore it is settled—them folks what brings the paper always throws it into a what had taken it, never thinking the subscriber is done over}.”

“Here comes John—well John, have you got the paper? “No daddy, the neighbors borrowed all the old papers, and Miss Parrot sent to get the morning papers as soon as they were done with.”

{“The devil she did—then I may hang up my fiddle ’till sundown, for when she begins to read ’tis from alpha to omega. Give me my hat, John, do you hear? Never mind breakfast; neighbor Swallow, will you accompany me to the printing office? I will subscribe immediately; five dollars did you say? I will give twenty five before I would suffer such impertinence. If I lend \textit{my paper} I wish I may be—.”

\raggedleft
{[emphasis mine]}\\
}
\end{ipquote}

The articles above show that the search term \emph{noospaper/s} appeared in newspaper articles already very early in the century and that in both cases the term is linked to a southern context, but it is not primarily used to mark region but rather to mark a negative character or political stance. Both these aspects, character and politics, are combined in a set of articles which have been called the “Nasby letters”. The letters are largely responsible for the striking peak in the number of articles containing \emph{noospaper/s} in the 1860s that I have described in \sectref{bkm:Ref6214186}. These letters were written by David Ross Locke, who was born in New York and became a newspaper reporter, editor, printer and owner in Ohio. He did not write the letters using his own name, however, but he constructed a character called Petroleum Vesuvius Nasby, a southern postmaster, who is described by \citet[144--145]{Blair1983} as “a bigot [and] an ignoramus, a hypocrite, a sluggard, an alcoholic, a coward, a bigamist, a thief, a corrupt politician, and a traitor”. His name alludes to his explosive and unrestrained temperament, which is evaluated negatively. He embodies the political views and social behaviors which are in complete opposition to Locke’s own views and norms and the letters are thus satirical pieces of writing intended to expose and criticize southern politics and culture during the Civil War years and afterwards. Locke began writing the letters in 1861 and continued until the 1880s and their popularity and wide circulation in the North made Locke one of the best-known humorists in America \citep[144]{Blair1983}.

The letter I have chosen for analysis here was published in the \emph{St. Louis Globe-Democrat} (Missouri) on September 11, 1876\citesource{September111876}, and it was taken from the \emph{Toledo Blade}, an Ohioan newspaper owned by Locke. The sub-heading “Why the Nigger is a Trubble, a Worriment and an Irritashen—How a Hawty, Shivelrus People Hev Bin Obleeged to Succum to Force” already shows the satirical tone of the letter: Nasby’s display of contempt for Black people does not fit his praise of southerners as haughty and chivalrous. These two topics are elaborated further in the letter. First of all, Nasby claims southerners to be superior to northerners: They are aristocrats and are thus superior to northern men who are just “mer labrin men, or mer men of biznis”, who are not enlightened and not able to understand the situation in the south, which is why they need Nasby to enlighten them. Locke thus ridicules the southerners’ pride in their aristocratic past and their emphasis on their feelings and traditional manners and behaviors, which, in their view, make them superior to hard-working northerners. Having a thoroughly negative character like Nasby express such a view has the effect that the southerners appear deluded and it emphasizes Locke’s position that northern businessman do more for the progress of the nation than lazy southerners, who live off their plantations and other people’s hard work. Similarly, it is clear that it is in fact Nasby who is ignorant and in need of enlightenment and not northern people. This concerns first and foremost his attitude towards Black people: Nasby’s stupidity and lack of knowledge and intelligence disqualifies him and his negative views and rather convinces the readers to distance themselves from his positions. The letter thus rests on the depiction of Nasby as an uneducated and unintelligent brute, whose views are ridiculous and not to be shared by cultured and educated (northern) people, and this depiction is also achieved by means of language. The most notable element of the Nasby letters is the frequent use of eye dialect – examples in the short extract quoted below are \emph{conclooshen}, \emph{enliten}, \emph{nacheral}, \emph{biznis}, \emph{hawty}, \emph{shivelrus}, \emph{succum}, \emph{trubble}, \emph{noboddy}, \emph{bin}, \emph{irritashen}, \emph{absloot}. As pointed out several times already, eye dialect also functions here as an effective means to portray Nasby as so uneducated that he is unable to spell words correctly. It also creates humor and marks the character as inferior to the educated reader who detects the misspellings. Pronunciation respellings are much rarer. In the extract below, they mark alveolar -\emph{ing} (\emph{Reedin}, \emph{bein}, \emph{labrin}), which also extends to nouns (\emph{feelins}), forms of connected speech (weakening in \emph{hev}, \emph{uv}, \emph{kin}, \emph{ez}, \emph{wuz}, and elision in \emph{em}) and the backing of \textsc{strut} (\emph{onrestrained}, \emph{oncontrolled}). While alveolar -\emph{ing} and forms of connected speech are often used to mark the uneducated speaker, \textsc{strut} backing is a phonological form that marks Nasby’s voice as different from that of voices represented in other articles. The representation of yod-dropping in the letter is interesting because by using the spelling <oo> to represent the vowel /uː/ in all words and not just in those in which yod-dropping occurs, Locke draws attention to the vowel, but he does not highlight yod-dropping in particular.\footnote{The question whether <oo> could not also represent /juː/ can be answered by looking at the spelling of \emph{perpetually} as <perpetyooally> occurring later in the article. Here, the <y> seems to be used to represent /j/ before /uː/ (spelled <oo>), which shows that the absence of <y>, for example in <noospaper>, represents a pronunciation without /j/.} This makes the Nasby letter similar to the first letter analyzed here because in both letters eye dialect predominates and <oo> is a case of eye dialect in some words (\emph{troo}) and a marker of yod-dropping in other words (\emph{noospaper}). This connection between eye dialect and yod-dropping reinforces the indexical link between the form and the uneducatedness of the speaker, which is also indicated in the Nasby letter on the grammatical level by the commonly used negation with \emph{ain’t} and negative concord (\emph{ther ain’t no question}), and on the lexical level by representing the pronunciation of \emph{oblige} as /əˈbliːdʒ/.

So overall, even though the Nasby letters are largely responsible for the high frequency of articles containing \emph{noospaper/s} in the 1860s, this does not mean that they contributed to the salience of yod-dropping in discourses on language because yod-dropping is not particularly highlighted here. The prevalence of eye dialect can also be found in other articles containing \emph{noospaper/s} and published in the 1860s: They were written by the humorist writer Charles Farrar Browne under the pen name Artemus Ward and also reached a high popularity and a wide circulation. It is therefore rather eye dialect in combination with a few differential phonological, lexical and grammatical forms that become associated with humorist writings which pointedly ridicule uneducated speakers and their views – in contrast to other articles, the focus is thus more on their differential spelling than on their differential voices.

\begin{ipquote}
\begin{center}
\textstyleStrong{NASBY.}\\
\textstyleStrong{Why the Nigger is a Trubble, a Worriment and an Irritashen—How a Hawty, Shivelrus People Hev Bin Obleeged to Succum to Force.}\\
{[From the Toledo Blade.]}
\end{center}
\textsc{Confedrit X Roads, wich is in the State of Kentucky}, September 4, 1876.—

Reedin Northern noospapers for some weeks past, I hev come to the conclooshen that the people uv the North don’t understand the troo status uv things down here, and I feel it my dooty to enliten em. It is not nacheral that a Northern man kin understand the feelins uv a Southerner. The Northerner never wuz a aristocrat like us—he never wuz a sooperior race. But bein mere labrin men, or mere men of biznis, or sich, they kin hardly be expected to comprehend how some things strike the minds of hawty, shivelrus people, which hev bin obleeged to succum to force.

{That the nigger is a trubble to us ther ain’t no question, and noboddy denies uv it. He is a worriment and an irritashen, and more than that an absloot noosence, and there never kin be peace so long ez he is onrestrained and oncontrolled.

\centering
{[…]}\\
}
\end{ipquote}

An article which also constructs links to uneducatedness in a southern context is a story which is not told by a reporter directly but indirectly using the voice of “E. H. Barclay, a New York traveler at the Lindell”. It is headed “Out-of-Town People” and was published in the \emph{St. Louis Republic} on August 18, 1894\citesource{August181894}. Barclay is quoted at the beginning as saying “I heard a good story once purporting to explain how the town of Rondo, Ark, got its name”. This shows that the story has an anecdotal character because it is said to be based on facts rather than fiction. Barclay then tells the story of the steamboat captain James Crooks, “who sailed on the Red River” and got stuck because he was delayed and the water became too low to continue sailing. He and a passenger then used a steam sawmill and pool tables, which they had on board their boat, to start a business on the river-bank, and around this a town sprang up and because of the gambling resort it was named after the favorite game played there: Rondo. The “natives” of the town are characterized as uncivilized (“the natives were using cuss words, bowie knives and revolvers in settling disputes over the game of rondo”) and they are described as spending much of their time gambling and drinking bad whisky. In the part of the story quoted below, the native inhabitants’ first encounter with newspapers is described. They decided to get a newspaper to be “informed on the doings in the world” and because they were illiterate, they got a schoolteacher to read it to them in exchange for free whisky. The humor of the anecdote is created by contrasting the schoolteacher’s conviction that he is educated and intelligent and thus superior to the other native inhabitants with his actual lack of education, which is exposed when he explains to the natives that the word \emph{immigrant} designates “a little varmint about the size uv a gray squirrel”.

The southern natives are therefore portrayed in a very negative light in the anecdote, the main characteristics highlighted being their ignorance, backwardness and lack of civilized behavior. The figure of the schoolteacher serves to reinforce these qualities because a teacher would be expected to possess a high degree of education and to function as a role model in the community, but he turns out to not be very educated either and his regular consumption of whisky also marks him as being just as uncivilized as the other natives. The heading “Out-of-Town People” also creates a contrast between rural and urban areas of the south, locating the town Rondo in the rural periphery. The voices of the southerners are constructed as different from the voice of the story-teller Barclay, who, as a New Yorker, is not only connected to the north but also to the urban sphere. Next to yod-dropping in \emph{noospaper}, the native inhabitants of the area are also shown to use alveolar -\emph{ing} (\emph{expectin’}, \emph{l’arnin’}), a lower \textsc{square} vowel in \emph{thar} and \emph{hyar}, yod-insertion after /h/ in \emph{here} (\emph{hyar}), a higher \textsc{trap} vowel (\emph{thet} ‘that’, \emph{dern} ‘darn’), a lower \textsc{nurse} vowel (\emph{l’arnin’}), a lower \textsc{kit} vowel in \emph{if} (\emph{ef}) and a lower \textsc{dress} vowel in \emph{well} (\emph{Waal}). Features of connected speech are marked as well (weakenings in \emph{uv}, \emph{wuz} \emph{tuh} and elisions in \emph{’em} and in \emph{more’n}). There is also an instance of eye dialect (\emph{wuddent}). Grammatically, the double demonstrative \emph{this hyar} is a form used by the schoolteacher. All of these forms have also been identified in articles containing \emph{bettah} as forms connected to Black, white southern and mountaineer speech, but it is noticeable that non-rhoticity is not marked here at all. This shows that while non-rhoticity \emph{can} index southernness, it does not have to be present. This article also shows that yod-dropping is also not a very salient form, as it only occurs once and as it is not particularly highlighted. All in all, the representation of speech serves to underline the ignorance and uneducatedness of the (white) southern inhabitants in rural areas and they are depicted as using linguistic forms that evoke these values in other contexts as well, especially in contexts aiming at deriding mountaineers or Black Americans.

\begin{ipquote}
\begin{center}
OUT-OF-TOWN PEOPLE.\\
{[…]}
\end{center}
“Now, the natives at that time, with few exceptions, had never seen or heard of such thing as a newspaper. One of these ignorant natives asked what sort of a dern thing a newspaper was. The passenger explained how \$1 50 would pay for the subscription of a good weekly newspaper published in St. Louis, and thereby the citizens could keep themselves informed on the doings in the world.

“This was a startling innovation to the natives. A collection was taken up in a hat, and the money was sent off for a St. Louis newspaper. In due time the paper came, and then a new difficulty stared them in the face. Who could be found to read it? Upon somebody’s suggestion the school teacher of the neighborhood was selected to do the reading. But the pedagogue had an eye single to business, and seeing that he had a monopoly on the intelligence of the community, he forced a bargain that he was to get his whisky all week free of charge for reading the paper regularly on the day of its arrival at the saloon.

“The first day the paper arrived the country pedagogue wet his whistle at the end of every other sentence, and occasionally when he struck a big word that had been used by a green reporter at a fire, he would stop at a comma, even, and dampen his throat with two glasses of whisky. He finally read an item stating that the corn crop of Texas was magnificent, and that the people of Texas were expecting a large immigration in consequence.

“‘Hold on, thar! What’s thet you read?’ asked one of the natives.

“‘Why, this hyar noospaper says that Texas has a big corn crop,’ replied the school teacher, ‘and thet they’re expectin’ a mighty big immigration on account uv it?’

“‘Waal, what is immigration?’ asked the illiterate native.

“‘Why, you fool, immigration means immigrants coming to the State,’ explained the school teacher.

“‘Yes, but what sort uv a dern thing is an immigrant?’

“‘Now, my friends,’ replied the pedagogue, assuming a look of wisdom, ‘but very few of you have got any book l’arnin’ at all, an’ ef I wuz tuh tell all the Latin name you wuddent know any more’n you do now. But an immigrant is a little varmint about the size uv a gray squirrel. I don’t know much about ’em, but they’re hell on corn.’”
\end{ipquote}

An anecdote containing the search term \emph{noospaper/s} that foregrounds differences between places or regions is headed “Didn’t Know the Place” and it was published in the \emph{Daily Arkansas Gazette} (Little Rock, Arkansas) on July 27, 1883\citesource{July271883}, but the \emph{New York World} is given as its original source. It consists of a dialogue between “a man with Kentucky jeans on”, which implies that he is from Kentucky, and a man whom he meets in the “Broadway corridor of the building”, which implies a New York setting. The Kentucky man asks for the post office, and he is astonished when he finds that his assumptions about what a post office should look like, which people are found there and how the postmaster behaves turn out to be wrong. The Kentucky man’s assumptions are used to convey to the reader the typical make-up of a post office in Kentucky: There are “fellers who sit around on the barrels and tell stories”, there is “a fellow wots got the terbacker”, there is “a minister [who] come in an’ borry a postal card till he gits a whack at the plate” and there is “a postmaster [who] read all the papers an’ postal cards before he sends ’em home”. All these characteristics are negative and make the post office and the people in New York seem superior to those in Kentucky. This in turn is the source for the humor created by the Kentucky man’s last statement: He finds the New York post office to be “the blastgamedest postoffice I ever seen” and decides not to trust them and to “send the letter home” himself, which is ironic because it is apparent to the reader that the New York post office is much more trustworthy than the Kentucky one. The anecdote therefore constructs New York as superior to Kentucky, with postmasters who are responsible men who are hard-working and not sitting around smoking and telling stories.

\begin{ipquote}
\begin{center}
\textstyleStrong{Didn’t Know the Place.}\\
{[New York World.]}
\end{center}

“Where’s the postoffice?” asked a man with Kentucky jeans on and wearing beard from ear to ear around the under part of his jaw that made him look as though he had only put it on for fun.

He was walking up and down the Broadway corridor of the building when he asked the question, and the man he asked told him that he was within the building.

“This is the postoffice?”

“Yes, sir.”

“Where’s the postmaster? I want to mail this letter.”

“Oh, I suppose he’s up stairs in his office.”

“Well, that’s good!” ejaculated the countryman. “Why ain’t he here attending to his business?”

“He is, probably.”

“That’s good, again. I want to get a stamp of him and he’d ought to be here. And you call this the postoffice? Where’s the fellers?”

“What fellows?”

“The fellers who sit around on the barrels and tell stories?”

“We don’t have them here.”

“Where’s the noospapers wot you get out of the boxes an’ read?”

“None here.”

“Where’s the fellow wots got the terbacker?”

“He ain’t around.”

“Don’t the minister come in an’ borry a postal card till he gits a whack at the plate?”

“Not here,”

“An’ don’t the postmaster read all the papers an’ postal cards before he sends ’em home?”

“No.”

“Well, this is the blastgamedest postoffice I ever seen. They can’t git any 8 cents from me. Guess I’ll take the letter home myself,” and he walked away toward Cortlandt street scratching under his hat.
\end{ipquote}

With regard to the voices represented in the anecdote, it therefore not surprising that it is the Kentucky man’s linguistic repertoire which deviates from that of the ‘neutral’ narrator and the New York man. The only form shared with the New York man is the use of \emph{ain’t}, which means that the form is associated mainly with spoken language here. Yod-dropping is only linked to the Kentucky man, which creates an association with southern (or more specifically Kentucky) speech. However, yod-dropping occurs again only once and it is not as salient as another contrast on the phonological level which is established based on the use of hyper-rhoticity: The Kentucky man uses the hyper-rhotic forms \emph{fellers} and \emph{terbacker} and the difference between him and the New York man is highlighted in the exchange “‘Where’s the fellers?’/‘What fellows?’”, which creates a parallel structure directing the readers’ attention to the difference between \emph{feller} and \emph{fellow} and linking it to the difference between uncivilized Kentucky and civilized New York, which fits the values identified for hyper-rhoticity above. Further forms used by the Kentucky man are a raised \textsc{dress} vowel in \emph{get}, relative \emph{what} (\emph{the noospapers wot you get out of the boxes an’ read}, \emph{the fellow wots got}) and third-person singular \emph{don’t} (\emph{Don’t the minister come in}). They mark him as southern (the last two forms have been linked to southern speech in articles above), but also as uncivilized and uneducated (the raised \textsc{dress} vowel has marked the speech of Blacks and mountaineers, and relative \emph{what} has also been linked to Black speech in articles discussed above). On a lexical level, the Kentucky man’s use of the colloquial figurative expression \emph{get a whack at the plate} and of the swear word \emph{blastgamedest} are noticeable, which reinforces the impression that his speech is not very elaborate and that his behavior is rather impolite. Overall, it is striking that apart from yod-dropping, there is no overlap between the repertoire of the southern speakers in “Out-of-Town People” and the Kentucky man in this article. There are also only two forms shared between the Kentucky man and the two figures of the Kentucky colonel (Colonel Gutrippah and the colonel in the advertisement for The Royal Tailors): third-person singular \emph{don’t} and negation with \emph{ain’t}. From this follows that there is no linguistic repertoire that consistently indexes southern speech, but that there are several forms available which can mark southernness, but also other social values. That non-rhoticity is not marked in the speech of the southerners exhibiting yod-dropping shows that it is also not a prerequisite for indexing southernness: It can occur, but it does not have to.

This argument can be supported by the following humorous short dialogue between a “Small Kentuckian” and his “Pap”, published in the \emph{Morning Oregonian} (Portland, Oregon) on January 12, 1894\citesource{January121894}. The young boy asks about the “swearin’ off that the noospapers air talkin’ about”, which his father disparagingly explains to him as being “just some Yankee custom” which southerners are not familiar with.

\begin{ipquote}
Small Kentuckian—Pap, what is this yah swearin’ off that the noospapers air talkin’ about nowadays? Pap—I don’t know. It don’t mattah, anyway. It’s just some Yankee custom.—\textit{Indianapolis Journal}.
\end{ipquote}


As the article was published in the beginning of the new year, it is likely that newspapers contained articles about bad habits that people plan to swear off, that is about New Year’s resolutions. The father’s statement that New Year’s resolutions do not matter in the south sheds a negative light on the region because it makes southerners seem like they are not interested in swearing off bad habits – habits which they are depicted as having in other articles, for example drinking whisky, chewing tobacco and swearing. This negative image is linked to the linguistic forms, which include yod-dropping \emph{and} non-rhoticity here, and thus illustrate that southerners can exhibit both forms as well as only one of them. In addition, alveolar -\emph{ing}, yod-insertion and dropping of /h/ before /j/ in \emph{here} (\emph{yah}) and third-person singular \emph{don’t} in \emph{it don’t mattah} are marked as southern here as well, but they also occur in the repertoires of Black speakers and mountaineers in articles above, illustrating again how many forms are shared by these groups and suggesting that several forms are more generally indexing a lack of eduaction than a specific social group. A form that has not been part of any of the other articles analyzed so far is a higher and fronter vowel in \textsc{start} (\emph{air} ‘are’).


An article illustrating that yod-dropping is also linked to Black speakers is headed “Darkeygraphy”. It was published first in the \emph{Charleston Mercury} on April 13, 1858\citesource{April131858}, then in the \emph{Daily Morning News} (Savannah, Georgia) on April 15, 1858\citesource{April151858}, and finally with slight changes in the \emph{Columbus Tri-Weekly Enquirer} on May 11, 1858\citesource{May111858}. Two Black speakers talk about the suicide of a Californian man at a hotel and the “colored gemman” (Lemuel) accuses “his colored crony, a waiter at a hotel” (Sam) of stealing money from the dead person – an accusation presented as justified because of Sam’s unusually good clothes. Several linguistic forms are present here which, in combination, mark the speaker’s ethnicity, for example voiced TH-stopping and voiceless TH-fronting (in medial position) as well as -stopping (in initial position), voiced labiodental fricative stopping (\emph{neber}, \emph{hab}), final consonant cluster reduction (\emph{lass} ‘last’), alveolar -\emph{ing}, regularized past tense forms (\emph{seed}), absence of \textsc{do}{}-support in wh-questions (\emph{How you suppose I know?}, \emph{What you mean to insenewate?}), malapropisms (\emph{susancide}) and eye dialect (\emph{nite}, \emph{insenewate}). This shows again how an extremely negative stereotyping of Blacks as thieves, combined with the derogatory label “darkey” and the explicit ridicule of their speech as “amusing” is linked to the representation of a large number of differential linguistic forms. Yod-dropping is one of many of these forms and again not very salient, as it occurs only once.

\begin{ipquote}
\textsc{Darkeygraphy}.—The following sample of “darkey” talk is characteristic and amusing:

“So you had a bad susancide at your hous lass nite, Sam,” said a colored gemman, on meeting his colored crony, a waiter at a hotel.

{“Oh, yes, Lemuel, dat we had—it almost scart me into takin’ a drink. He was jis from California, wid heeps of noospapers.

\centering
{[…]}

“Wus dere anything found in de pockets Sam?”}

“How you suppose I know? Do you tink I’d put my hand in to feel? What you mean to insenewate?”

“Oh, nuffin—only I neber seed you hab sich good close on afore, dat’s all.”
\end{ipquote}

So far, I have shown that yod-dropping occurs in articles as part of the representation of southern voices and also of a Black voice. However, unlike non-rhoticity, yod-dropping is also used frequently to represent voices of speakers from other regions. Examples of such voices are Tom Blake, a New York newsboy who used to be a shoeblack in Brooklyn, a Wisconsin deacon and his wife Sarah Jane, Nort Kingsley, an old and grim hunter in Northern California, and Hank Borrows, a giant wagoner in Montana. I will briefly describe the articles that these figures appear in and discuss how the voices are constructed and how salient yod-dropping is as a characteristic of their speech.

Tom Blake is quoted as a witness of the situation of the shoe-blacking industry in New York City in an article which originally appeared in the \emph{Brooklyn Eagle} and was then republished in the \emph{St. Louis Globe-Democrat} (Missouri) on August 31, 1887\citesource{August311887}. He describes how the business has become difficult because of the increasing competition from Italians and Black men, who offer big chairs, newspapers and a good polish which preserves the black color. This has caused him to become a newsboy at the Eagle – he earns more money there than by blacking boots. This contrast to Italians and Black people emphasizes his whiteness and his status as an American. His job marks him as belonging rather to the lower or lower-middle class. His speech is marked by one instance of yod-dropping and, in addition, one instance of hyper-rhoticity (\emph{feller}). So even though he is a New Yorker, he is not marked as non-rhotic at all, which provides further evidence to the argument developed above based on the anecdote “She Got a Seat”, that non-rhoticity is not primarily a maker of northeastern (or more particularly New York) speech, but rather a feature that marks social characteristics, like the imitation of English manners and speech to appear educated and cultured. The newsboy does not possess these characteristics; on the contrary, he is depicted as using alveolar -\emph{ing}, which has been shown to rather mark the opposite. Lexically, the pronunciation of \emph{Italian} as \emph{Eye-talian} also underlines his lack of education and the shortening of \emph{business} to \emph{bis} marks his speech as colloquial. Grammatical forms include the negation with \emph{ain’t} and negative concord (\emph{The bis ain’t no good no more}), second-person singular -\emph{s} and third-person plural -\emph{s} marking and plural \emph{is} in the existential construction \emph{there’s secrets} (compared to third-person plural \emph{are} in the non-existential construction \emph{They’re just as careful}). These forms are all found in representations of southern, Black and mountaineer speech as well, so that, like yod-dropping in \emph{newspaper}, they do not index a particular region here, but rather social characteristics like a lower degree of education and social standing.

\begin{ipquote}
\begin{center}
\textstyleStrong{EVOLUTION OF “SHINES.”}\\
\textstyleStrong{Rise and Progress of the Sho{\kern0pt}e-Blacking Industry.}\\
{[From the Brooklyn Eagle.]}\\
{[…]}
\end{center}

Tom Blake, now an \textit{Eagle} newsboy, who was a sho{\kern0pt}eblack under the old dispensation, delivers himself in the following terms:

{“The bis ain’t any good no more. Five years ago I could rake in a couple of dollars a day easy down by Fulton Ferry. Now, a box ain’t worth 75c a day. The big chairs kills the bis. They’re got up in fine style, an’ you gets a read at a noospaper while the feller blacks yer boots, an’ when he’s through he brushes yer hat an’ coat, an’ al fur 5c. Beside that, there’s secrets in the business. It’s in the blackin’! I don’t care how much you rub at a boot with bad blackin’, the shine’ll die off to black lead in an hour or two. The Eye-talians and neggers ’as got some wrinkle about blackin’ as puts on a tip-top polish which stays there. They’re just as careful about it as the Chinese is about their starch. But there’s more money in selling the \textit{Eagle} on a good route.”

\centering {[…]}\\
}
\end{ipquote}

In the second example, the construction of voices is fairly complex: The article is a letter to the editor, which is supposedly written by a deacon in Wisconsin to the \emph{Milwaukee Daily Journal} on January 20, 1888\citesource{January201888}. There are several clues, however, that the letter is a fictional piece of writing, which has the purpose of expressing views about political topics more indirectly by presenting them as views from people living in Wisconsin. The main clue is the humor created through the contrast between the “subscriber”, the figure of the deacon who is constructed as the writer of the letter, and his wife Sarah Jane. In the course of the letter, it becomes clear that it is in fact Sarah Jane who is smart and educated and who has developed a political view on tariffs based on several calculations that she did and that the deacon presents in the letter. At the very end of the letter, the editor adds a comment confirming that her calculations are correct and thus providing authoritative backing for her analysis. The deacon, in contrast, is portrayed as rather uneducated, which is revealed through several postscripts which the deacon added at the end of the letter. They show that his own writing is in fact quite different from that of his wife and that she has improved the quality of his letter to a large extent. He is depicted as a man who is not intelligent enough to realize that his wife is more educated than he is, although he does acknowledge that “shes a mitey good scholar”. The humor rests on the irony that the deacon describes her as being ashamed of her writing, even though it is more correct than his own writing. The letter therefore implies that that there is a norm for correctness, which provides for example a clear answer to the question whether \emph{rights} should be spelled <rites> or <rights> (again, Sarah Jane knows the answer better than her husband). This norm of correctness is violated sometimes in the letter, thus characterizing the subscriber as not educated enough to know these norms – the use of eye dialect is also prominent because it appears in the heading “Sarah Jane Figgers” and it serves to characterize the deacon as less educated than his wife because it occurs more often in the postscripts than in the letter itself (\emph{bizness}, \emph{rite}, \emph{sez}, \emph{mitey}, \emph{littery}) and the deacon points out explicitly that his wife does not know that he added the postscripts, therefore implying that she did not have a chance to correct them. The only forms indicating a differential pronunciation are \emph{noospaper} (the only instance representing yod-dropping), \emph{kivered}, indicating a fronted \textsc{strut} vowel, and words with final -\emph{ing} spelled <in> to indicate an alveolar pronunciation. Grammatical forms comprise \emph{a}{}-prefixing (\emph{a studyin}), demonstrative \emph{them} (\emph{them letters}), negative concord (\emph{she couldn’t noways}) and subject-doubling (\emph{my wife Sarah Jane she’s a great reader}). Lexically, the adverb \emph{noways} stands out (it is not contained in any of the articles analyzed so far) and the subscriber’s reanalysis of \emph{commas} as a singular form having the plural \emph{commases} is also notable.

Compared to the letters analyzed so far (the letter addressed to Doctor Hun and the Nasby letter), this letter shares with them the characteristic of being fictional, but it differs markedly in two respects. First of all, the subscriber and his wife are shown to be aware of the norms of correctness and to attempt to conform to them. Even though they do not achieve it completely, the number of eye dialect forms and forms indicating differences in phonology, grammar or lexicon is considerably smaller in this letter – they are rather the exception than the rule. This general difference can be linked to a contrast between north and south, as the fictional authors of the first two letters have a southern background, while the subscriber and his wife Sarah Jane live in Wisconsin. However, most of the forms found in the northern letter are also found in the southern letters, including yod-dropping. This indicates that these forms themselves index first and foremost a lack of education and not a place, but that the north, as a place, is indexed through the lower number of such deviating forms. Northern speakers are thus constructed as speaking more correctly than southern speakers. This relates in interesting ways to the difference between males and females – in the northern letter, the wife is constructed as writing more correctly than the subscriber and as I have shown in the analysis of \emph{bettah}, it is the anecdote representing a female southern voice (“She Got a Seat”) which contains the lowest number of deviating forms. The regional difference between north and south which is linked to differences in correctness and thus to the degree of education of its speakers seems thus more salient for male voices than for female voices.

\begin{ipquote}
\begin{center}
\textstyleStrong{SARAH JANE FIGGERS}\\
\textstyleStrong{“SUBSCRIBER” AND HIS WIFE DIG UP SOME INTERESTING FACTS.}\\
\textstyleStrong{The Profit the Wisconsin Farmer Derives from the Tariff Just \$2.27 1-2—The Need of Bounties for Badger Tillers of the Soil—Mr. Granger’s Tobacco Bill.}
\end{center}
{\textsc{To the Editor of The Journal}: I guess they have found out at Madison that I am writing for the noospapers for they have sent me by express (cost me 25 cents) a great big kivered book, full of figures about everything, all the cows and horses and pigs and sheep and hay and cranberries and old soldiers and everything you can think of in Wisconsin. And my wife Sarah Jane she’s a great reader and bully on figgers and she sat up half the night a studyin of it and she said to me when she came to bed (I always go to bed at nine o’clock precisely) she said she believed we could figger out how much the farmers of Wisconsin made on their sheep by that tariff on wool they are makin’ such a fuss about.

\centering
{[…]}

We think it would be best to throw this plagy tariff on wool overboard and the importers also, and give us farmers a bounty on wool as these great statesmen James Blame and John Sherman propose to do for sugar.} {Let them give us 10 cents a pound and wouldn’t you see the wool crop in Wis. Go a kiting. We would soon raise 4 times or 10 times as much wool as we do now and old England could keep her wool at home and the cheating importers would have to shut up shop and the honest farmers would get their rights (should that be spelt rites or rights I say the first S. J. says the second.)

\raggedleft
\textsc{A Subscriber.}

\noindent{P.S.—Please send a copy of this to Senator Sawyer if any man can put this bounty bizness through old Philetus is the man.}}\\
P.S.—I want Sarah Jane to rite them letters to you herself, but she sez she couldn’t noways hav her riten seen by a Editor. Shes awful bashful. She changes the spelling in mine sometimes, and puts in comases and parenteses. Shes a mitey good scholar is S. J.\\
{P.S.—I always like to put in the P. Ses, they look a kind of littery as Uncle Philetus sez. Sarah J. do{\kern0pt}es not know that I have ritten these.

\raggedleft
\textsc{A Subscriber.}

{[We have looked up Sarah Jane’s figures, and beg to say to A Subscriber that S. J. is quite correct. —\textsc{Editor Journal}.]}}
\end{ipquote}

The two examples above show that \emph{noospaper} is constructed as part of the repertoire of not only white and Black southern speakers but also of northeastern (New York City) and northern (Wisconsin) ones. The following article creates a link between \emph{noospaper} and a speaker living in yet another place: in Northern California. It is a short story entitled “Off the Trail”, written by Mariner J. Kent and published in the \emph{Daily Inter Ocean} (Chicago, Illinois) on August 11, 1887\citesource{KentAugust111887}. The story is told from the perspective of the main character, a “reporter”. He rides on a trail through the Sierra Nevada, even though he has been warned by “the old and grim Hunter, Nort Kingsley” that it is very dangerous and that another man who has taken the trail has not come back since. As predicted by the hunter, he encounters several difficulties: At one point his horse sinks into a patch of morass and has to struggle to get out, at another point they get into the middle of a landslide, with large rocks falling down around them and huge amounts of sand threatening to bury them, and finally, the reporter loses the trail and arrives at an “uncanny spot” where he finds the skeleton of the man that the hunter has told him about. In the end, however, he is lucky and finds the trail again and arrives at his destination – his horse, however, falls dead.

The extracts of the article quoted below contain the representations of direct speech in the beginning of the story and they show that the speech of the hunter is constructed as markedly deviant from the voice of the reporter. Yod-dropping, indicated through \emph{noospaper}, appears once and it is thus less salient than other forms (as in the articles above). A more salient form is the lowered \textsc{kit} vowel which not only occurs in \emph{ef} ‘if’ (as in articles representing Black or mountaineer speech), but also in \emph{et} ‘it’ and \emph{sence} ‘since’, suggesting that the raising is not lexically restricted. Furthermore, his speech is characterized by hyper-rhoticity (\emph{ter} ‘to’, \emph{yer} ‘you’, \emph{er} ‘a’) and even though one non-rhotic form occurs as well (\emph{fou’teen}), this seems to be an exception because no other forms are marked as non-rhotic. With respect to rhoticity, his repertoire is thus similar to the repertoire of the western cowboy and it reinforces the link between hyper-rhoticity and the west as a place full of hardships that have to be overcome in order to be successful and full of men who might not be civilized but who are tough and capable of surviving there. In contrast to the southern mountaineer, who combines hyper-rhoticity with non-rhoticity and whose lack of education and civilized behavior is evaluated negatively, the hyper-rhotic western hunter is rather praised for his ability to live in “the unsought and untraveled fastnesses of the great stretching wilderness of mountains”, which is full of dangerous natural phenomena. That the hunter is located at the nature end of the nature-culture continuum is also indicated by forms indicating aspects of connected speech like weakening (\emph{thet} ‘that’, \emph{hes} ‘has’) or elisions (\emph{th’}, \emph{an’}) and by the use of eye dialect (\emph{thot} ‘thought’, \emph{riting} ‘writing’). It is noticeable that the reporter is located in the middle of the continuum as he is neither shown to be non-rhotic nor hyper-rhotic: His use of \emph{fellow} instead of \emph{feller} puts particular emphasis on this absence of hyper-rhoticity and underlines the contrast to the western hunter. Finally, a lower \textsc{nurse} vowel is also indicated in the speech of the westerner (\emph{war’nt}), as in that of Black and white southerners and that of mountaineers. Grammatically, a notable parallel to the southern mountaineer is the use of the relative pronoun \emph{as} (\emph{er chap ez thot thet he could tramp th’ trail}). Furthermore, the hunter’s speech is marked by the third-person singular past tense form \emph{were} (\emph{he war’nt}), a form that has not been found in articles above so far. A form that has also not been represented in any of the articles analyzed above is the use of \emph{mought} as a lexical variant of \emph{might}. So to conclude, although yod-dropping is linked to the western hunter, it is not a very salient form, whereas hyper-rhoticity is more prominent, indexing an uncivilized roughness and toughness, which is evaluated positively in this western context. This positive evaluation is also signaled by the old age of the hunter, which conveys experience and wisdom. A lowered \textsc{kit} vowel is as prominent as hyper-rhoticity and as it is consistently marked it could be an index of western speech or even California more particularly, but more evidence would be needed to support this tentative hypothesis.\footnote{This lowering of \textsc{kit} to \textsc{dress} marked by spellings like <sence> is described by \citet[140]{Wolfram2016} as a part of the Northern California Vowel Shift, which according to them is a “more recent vowel shift”. The fact that it was already indicated at the end of the nineteenth century in the article “Off the Trail” raises the question whether the shift had already started earlier than these authors assume.} Overall, the number of forms marking the westerner’s speech as deviant is also not nearly as high as that of Black or white southern Americans or mountaineers and the use of eye dialect is also much more restricted.

\begin{ipquote}
\begin{center}
\textstyleStrong{OFF THE TRAIL.}\\
\textstyleStrong{A Reporter’s Wild Ride in the Sierra Nevada.}\\
\textsc{by mariner j. kent}\\
Author of “Shot Down a Flume,” “Winning a Scoop,” etc.\\
{[Copyrighted, 1887, by the author]}
\end{center}
“Et’s only fou’teen miles by th’ trail an’ et’s nigh on ter thirty round by the road, which I allow ez safer by er chap ez thot thet he could tramp th’ trail, an’ he hes not turned up sence.”

So urged the old and grim hunter, Nort Kingsley, as we stood in the shadows of Lassen Butte on a balmy August morning in 1877—one of those glorious mornings peculiar to the climate of Northern California. The twin peaks of Lassen towered heavenward in the midst of the mighty crests of the Sierra Nevada, where two mountain spurs seemed to have been hurled together.

“Thank you, Nort,” I said, swinging myself into the saddle, but I’ll take the trail and risk it. Perhaps I may run across the fellow you spoke of.”

{“Maybe yer mought,” rejoined Kingsley, “but ef yer do et will be en er place unpleasant fur riting yer stuff for th’ noospapers. He war’nt er pious chap yer see,” added the old hunter, as he grasped my hand in a farewell clasp.

\centering
{[…]}\\
}
\end{ipquote}

The fourth example of an article linking yod-dropping to voices from several different regions is a long fictional story written by W. Bert Foster and published in the\emph{ Salt Lake Semi-Weekly Tribune} (Salt Lake City, Utah) on November 15, 1898\citesource{FosterNovember151898}. The character whose speech is marked by yod-dropping is Hank Borrows. As indicated by the title “Hank Borrows’s Adventure”, he is the protagonist of the story and the first sentence links him to the northwest of the United States because he is described as “a wagoner in the employ of the Government at a Montana post”. He and his speech are contrasted with that of two other characters, Captain Langdon and Lieutenant Chester, two officers who are stationed in Montana but, in contrast to Hank, not native to the region. While the officers are young and inexperienced (the Lieutenant “was just out from West Point”, a military academy in New York), Hank is old and has accumulated a lot of knowledge and experience over the years. Furthermore, Hank calls the officers “gentlemen”, which indicates their higher social status, but at the same time it is clear that they depend on Hank in this context – not only on his experience, but also on his strength, which is conveyed by the narrator describing him as “a giant in stature—a mighty man of bone and muscle, who could ‘pack’ a mule-load if necessary, and who knew all the trails and mountain passes within 200 miles of the fort”. In the dialogue quoted here, Hank tells the officers that he has discovered signs of another hunter and that it will be dangerous for them as long as the hunter is in the same valley. The officers assume that Hank refers to another human and while Hank lets them believe it at first, he later tells them that he is actually referring to a bear. Furthermore, Hank sees signs of an upcoming blizzard. Even though Hank is depicted as calm and not worried, it is established that the three characters are in a dangerous environment with forces of nature (like snowstorms and wild animals) threatening their lives. The story thus revolves around the contrast between nature and culture again, with the officers representing the side of culture (as men affiliated to an elite academy in the northeast) and Hank representing the side of nature (as a physically strong man from the west who is not afraid of fighting the forces of nature). It is not surprising that Hank’s speech is marked as deviant from that of the officers: Next to the one instance of yod-dropping in \emph{noospapers}, it is noticeable that he exhibits hyper-rhoticity (but not non-rhoticity) like the Northern Californian hunter and also like the Montana cowboy losing a fight to a dude in the anecdote “A Muscular Dude”.\footnote{There is one instance of a representation of a non-rhotic form: \emph{worse} spelled <wuss>. This form belongs to a set of words in which loss of non-prevocalic /r/ occurred very early (see \sectref{bkm:Ref530736302} for a detailed discussion) and it is likely that these non-rhotic forms (e.g. \emph{hoss}, \emph{cuss} and \emph{fust}, where /r/ is deleted before /s/) are linked to a different set of social meanings than non-rhotic forms where /r/ occurs in a different environment (e.g. in final unstressed -\emph{er}, as in \emph{bettah}).} This provides further support for the strong indexical link between hyper-rhoticity and a roughness and toughness needed in natural environments as well as a lack of civilized behavior which can be evaluated negatively (as in “The Muscular Dude”, where the cowboy appears as threatening, wild, and drinking), but also positively as in “Off the Trail” and the story here, where the speakers’ experience and their tough and calm nature are depicted as helpful because they provide support for easterners unused to such difficult environments.

In addition to yod-dropping and hyper-rhoticity, Hank's speech also exhibits forms which have also been found in many articles analyzed above: alveolar \emph{{}-ing} (\emph{huntin’}), a raised \textsc{dress} vowel in \emph{git}, yod-insertion before /h/ followed by a dropping of /h/ in \emph{yere} ‘here’ and a lowered \textsc{square} vowel in \emph{thar} ‘there’. There is one instance of final consonant cluster reduction in \emph{tol’}, but given this low frequency it could also be interpreted as a marker of connected speech, like the weakening of the vowel in \emph{kin} ‘can’. A phonological form that has not been represented so far is yod-coalescence, indicated by spelling \emph{Indians} <Injuns>. As on the phonological level, grammatical forms which distinguish Hank’s speech from that of the officers and that of the narrator have been commonly found in other articles: the double demonstrative with \emph{this here}, negation with \emph{ain’t} and the past tense form\emph{ seen}. Lexically, the use of \emph{worritin’} instead of \emph{worrying} stands out because it has also not been part of any of the other articles; the phrase \emph{I reckon} and the alternative pronunciation of \emph{well} as \emph{wa-al} have been found in other articles above. The latter form is explicitly connected to a “drawl” here (“‘Wa-al,’ drawled Hank again”), a description which probably also applies to \emph{ya-as} ‘yes’.

\begin{ipquote}
\begin{center}
\textstyleStrong{HANK BORROWS’S ADVENTURE}\\
\textstyleStrong{By W. Bert Foster.}
\end{center}
Hank Borrows was a wagoner in the employ of the Government at a Montana post. In winter when teaming was impossible, he hunted in the foothills some distance from the fort. Occasionally some of the younger officers accompanied him, and it was considered something of an honor to go with Hank upon one of his hunting trips. The venture was sure to yield some good returns in the way of game and pelts.

Hank was a giant in stature—a mighty man of bone and muscle, who could “pack” a mule-load if necessary, and who knew all the trails and mountain passes within 200 miles of the fort.

{About Christmas time one winter Hank started out with two of the fort officers—a Capt. Langdon and Lieut. Chester¾bound [sic] for a certain valley, some sixty miles from the fort, which the old guide kne wwell [sic].

\centering
{[...]}

There was enough snow on the ground for good tracking and Langdon and Chester looked forward to some excellent sport. But when Hank returned to camp the first night, after spending the day setting beaver traps along the river, he looked grave.}

“I tell ye how it is,” he said, after supper; “I kinder wish’t I hadn’t brought you gentlemen into this yere valley—that I do!”

“What’s the matter?” queried Langdon.

“Indian signs?” demanded the Lieutenant, eagerly. He was just out from West Point, and had yet to experience his first Ute and Bannock campaign.

“No, no,” returned Hank. “Injuns don’t monkey around in the hills this weather. I’ve seen signs wuss’n that.”

“What was it? Think there’ll be a blizzard?”

“We kin weather a blizzard,” said Hank, calmly, “an’ we may git one. The signs is propeetious, as the weather prophets say in the Denver noospapers. But that ain’t what’s worritin’ me. I seen tracks today that tol’ me thar was somebody in this yere valley that can’t stay here if I’m goin’ ter occupy it, too.”

“Another hunter, Hank?” asked Capt. Langdon. Isn’t there room for two parties?”

“Don’t be hoggish, Hank,” added Chester. “This is a free country.”

“Tain’t free enough for him an’ me,” replied the old wagoner, with a curious smile on his rugged face. “There ain’t room for both of us in this yere valley, an’ I wish I hadn’t brought you gentlemen into it.”

“There must be no shooting scrape, Hank,” said the Captain sternly. “I thought you were a decent, quiet sort of a man—”

“I am—mostly,” said Hank, grinning behind his hand. “But I tell ye there ain’t none of us safe while this chap’s erlive in this yere valley.”

“What d’ye think he’s doing here—hunting?” questioned the Captain.

“Ya-as, I reckon he’s huntin’,” said Hank.

“Who is he? What’s his name?” demanded Lieut. Chester.

{“Wa-al,” drawled Hank again, his eyes twinkling, “I call him ‘Ole Ephr-am.’”

\centering
{[...]}\\
}
\end{ipquote}

So far, the analysis of articles containing \emph{noospaper/s} has focused on those articles which contain direct representations of speech. There are, however, also some articles in the collection of articles containing \emph{noospaper/s} which discuss yod-dropping explicitly and use \emph{noospaper} as an example to illustrate the phenomenon. This search term thus provides an excellent opportunity to compare the indexical links created indirectly in the representations of speech in letters, anecdotes, short humorous dialogues and fictional stories with those created in explicit metadiscursive discussions of the form.

The first article in the databases which discusses yod-dropping explicitly was published originally in the \emph{Washington Star}, a newspaper not contained in the databases but indicated as the source in the article cited below, which appeared in the \emph{Daily Arkansas Gazette} (Little Rock) on September 24, 1879\citesource{September241879}. The article was reprinted on August 29, 1880\citesource{August291880}, almost a year later, in the same newspaper before it appeared in two northeastern newspapers (in the \emph{New Hampshire Sentinel} and in the \emph{Northern Christian Advocate}) on December 30, 1880\citesource{December301880}, and in the \emph{Omaha Daily Herald} (Nebraska) on December 6, 1881\citesource{December61881}. It needs to be noted that the later reprints have shortened the original article to some extent and that they also refer to different sources (like the \emph{Southern Letter}, the \emph{New York Weekly Review} or \emph{Hall’s Journal of Health}). This indicates that the article was very popular and attracted the interest of many newspapers and journals over a time period of at least two years.

The article is a contribution by a reader – his status as a person not working for the paper is indicated by his addition of “if you will allow me the space”. His intention in writing the article is that he wants to draw attention to “a fault in English pronunciation”, which, as indicated by the heading, concerns “The Pronunciation of ‘U’”. He describes this pronunciation as “giving the long ‘u,’ which is in so many [of] our common words, the sound of ‘oo’”. This shows that the “fault” is conceptualized as a differential pronunciation of the vowel and not as an elision of a consonant. This provides further insights into the question of salience: Since /j/ is not represented in the spelling, it is harder to represent its absence graphically and by choosing the vowel grapheme <oo> to mark yod-dropping, the difference can easily be perceived as a vocalic one. This makes yod-dropping different from /h/-dropping, which is unambiguously marked in the spelling by omitting <h>. Nevertheless, since yod-dropping is the only linguistic form discussed in this article, it is of course very salient here. The examples given by the author show that the elision of /j/ occurs in words where /j/ follows an alveolar consonant (\emph{institute}, \emph{duty}, \emph{student}, \emph{Tuesday}, \emph{avenue} and \emph{dupe}), but not in words where it occurs after other consonants (\emph{beauty}, \emph{pew} and \emph{cupid}). His evaluation of the form as a fault and a vulgarism which is not authorized by any dictionary of the English language creates an indexical link between the form and incorrectness, that is a failure to conform to established norms, and thus also to a lack of education. He also links the use of the form to regional difference, however, by claiming that yod-dropping is “exceeding common in the north, rarely heard in the south, or in England, but which seems to be spreading here”. The extent to which the pronunciation is used in the north is quantified by his claim that “ninety-five out of every hundred northerners will say institoot, instead of institute”. Only the most highly educated men in the north (the names he gives as examples are “Wendell Phillips, Charles Sumner, George William Curtis, Emerson, Holmes”) do not use the ‘wrong’ variant, which strengthens the link between education and correct pronunciation and which serves to underline the urgency of the author’s appeal that especially teachers and students need to be made aware of the error in order to be able to correct it. That the author finds yod-dropping to be a purely northern form, which is not found in the south at all (“It is a fault that a southerner also never falls into”), is remarkable given that it has been linked to direct representations of southern speakers in articles above. Yod-dropping is also constructed as an exceptional case by the author himself who states that the southern speaker “has slips enough of another kind” – among them the “vulgarism to call a door a doah”, which refers to non-rhoticity. Non-rhoticity is thus constructed as a southern form and yod-dropping as a northern form, the difference being that non-rhoticity and its evaluation as incorrect and vulgar are already well known and shared by a large number of people in 1879 (“as we all admit”), while yod-dropping is judged to be much less salient, which justifies the writing of the article (“As many of our teachers have never had their attention called to this, I hope they will excuse this notice”). This confirms an observation made based on the analysis of the articles above, namely that yod-dropping is not a very salient form, especially in comparison to non-rhoticity and hyper-rhoticity.

\begin{ipquote}
\begin{center}
\textstyleStrong{The Pronunciation of “U.”}\\
{[Washington Star.]}
\end{center}
As the schools have just opened, and as everybody reads your paper, if you will allow me the space, I wish to call the attention of the teachers and pupils to a fault in English pronunciation, \textstyleStrong{exceeding common in the north, rarely heard in the south, or in England, but which seems to be spreading here}. (We have faults enough in the south without grafting some northern ones upon them.) I refer to the \textbf{vulgarism}—if I may so term it—of giving the long “u,” which is in so many ot [sic] our common words, the sound of “oo.”

For instance, \textstyleStrong{ninety-five out of every hundred northerners} will say institoot, instead of institute, dooty instead of duty—a perfect rhyme to the word beauty. They will call new and news noo and noos—a perfect rhyme to pew and pews—and so on through the dozens and hundreds of similar words. \textstyleStrong{Not a dictionary in the English language authorizes this}. In student and stupid the “u” has the same sound as in cupid, and they should not be pronounced stoodent and stooped, as so many teachers are in the habit of sounding them.

{If it is a \textbf{vulgarism} to call a door a \textbf{doah}—as we all admit—isn’t it as much of a vulgarism to call a newspaper a \textbf{noospaper}? \textstyleStrong{One is northern and the other southern}—that’s the only difference. When the London Punch wishes to burlesque the pronunciation of servants it makes them call the duke the dook, the tutor the tooter, and a tube a toob. \textstyleStrong{You never find the best northern speakers}, such as Wendell Phillips, Charles Sumner, George William Curtis, Emerson, Holmes, and \textstyleStrong{men of that class}, saying noo for new, or Toosday for Tuesday, avenoo for avenue, or calling a dupe a doop. It is a fault that a southerner also never falls into. He has slips enough of another kind, but he do{\kern0pt}esn’t slip on the long “u.” As many of our teachers have never had their attention called to this, I hope they will excuse this notice.

\raggedleft
{[emphasis mine]}\\
}
\end{ipquote}


The second article linking \emph{noospaper} to an explicit discussion of yod-dropping appeared eight years later than “The Pronunciation of ‘U’”, on August 13, 1887\citesource{August131887}, in the \emph{Grand Forks Daily Herald} (Grand Forks, North Dakota). Its heading, “Teaching in the West”, indicates again a relation to teaching and education. It also draws attention to a particular region: the “West”. In the article, the place is specified further as “a school room in lower Arizona”, and the main aim of the article is to report the prescriptive linguistic rules given and enforced by a teacher there to the readers of the newspaper in North Dakota. One of these rules is cited as “They must soften the u in such words as duty and opportunity, and not pronounce them dooty and opportoonity”. As in the article “The Pronunciation of ‘U’”, the difference in pronunciation is conceptualized as a vocalic difference, with the ‘correct’ vowel being ‘softer’ than the ‘incorrect’ one. However, in contrast to that article, yod-dropping is not as salient here because it is only one of several rules and not highlighted in any way. In general, the article shows that yod-dropping was found in western regions because it was apparently deemed necessary to teach the alternative variant in school. The fact that it was written by a western correspondent for a northern newspaper indicates that it is judged to be relevant in the north as well. Moreover, the correspondent explicitly points out in the article that the rules “might find a conspicuous place in many families in the east, where the educated ear is so frequently offended by the mispronunciation of the most common words”, which shows overall that the forms listed in the article are relevant for all American regions. The south-north difference is not emphasized here – it is even possible that the correspondent conceptualizes “the east” as comprising the northeast as well as the southeast. The forms are thus explicitly evaluated as incorrect, and in addition, they are implicitly constructed as non-regional. Taking into consideration other forms judged to be incorrect, like for example alveolar -\emph{ing} (“drop final g’s”), negation with \emph{ain’t}, or elisions which occur in connected speech (“gray deal for great deal”), it becomes clear that they are forms which I have identified in direct representations of several different speakers as well and that they thus serve to signal primarily a lower degree of education or cultural refinement and not the belonging to a specific region.


\begin{ipquote}
\begin{center}
\textstyleStrong{TEACHING IN THE WEST.}
\end{center}
A western correspondent of one of our daily papers gives some \textstyleStrong{rules} that are posted conspicuously in a \textstyleStrong{school room in lower Arizona}. The teacher is an enthusiast in the use of good English, and insists on its use in the school room. The rules that this teacher has made imperative in his school room \textstyleStrong{might find a conspicuous place in many families in the east}, where the educated ear is so frequently offended by the mispronunciation of the most common words. His rules are:

“My scholars must not pronounce dreadful, dretful; or catch, ketch; or \textstyleStrong{newspaper, noospaper}; or society sassiety; or February, Febuary; or Massachusetts, Masschusetts; or eleven, leven; or height, hithe; or drought, drowth; they must not say fur for for, or git red of for get rid of. They must not say anywheres or nowheres, or anyways, or a long ways, or those sort of things, or those kind, for that sort and that kind. They must not say he don’t for he do{\kern0pt}esn’t and they must never use the word ain’t. \textstyleStrong{They must soften the u in such words as duty and opportunity, and not pronounce them dooty and opportoonity}. They must not drop final g’s, or leave out of words their h’s. They must not half pronounce, must not say gray deal for great deal. Every word demands the full, authorized, verbal mention of all its letters.”

{The correspondent says this teacher’s method of teaching is very original, and his success, considering the environment of his pupils, marvellous. {[…]}

\raggedleft
—Christian Union.\\
{[emphasis mine]}\\}
\end{ipquote}

In contrast to the article “Teaching in the West”, the difference between the north and the south found in “The Pronunciation of ‘U’” is also emphasized in the explicit discussion of yod-dropping in the article “The New Woman”, published in the \emph{Daily Picayune} (New Orleans, Louisiana) on September 15, 1895\citesource{September151895}, and which contains a sub-section about “The Speech of Southern Women”, which is quoted below. The main part of the sub-section consists of a quotation of a passage of Theodore Mead’s book “Our Mother Tongue”\footnote{The author’s last name is spelled incorrectly in the article; it is Mead and not Meade.}, in which he generally criticizes “the speech of American women” but at the same time praises the speech of “the women of the southern states” as “pleasant and correct”. As a reason he gives “the pure and more sonorous use of the vowels by southern women than by their northern sisters, and [...] the less harsh employment of the consonants” and the two linguistic forms illustrating this point are the \emph{absence} of yod-dropping, described as “giving its full significance, for instance, to ‘u,’ instead of giving that vowel the sound ‘oo’”, and non-rhoticity because even though he acknowledges that “the ignoring of the ‘r’s” could be seen as an offense, he also finds that the realization of the /r/ is often too harsh and rolling, which makes it sound menacing and unfriendly. Mead thus creates an association between the linguistic forms yod-dropping and rhoticity and the geographical region of the north as well as the social values of masculinity, harshness, unpleasantness and unfriendliness. In contrast, the correctness of yod-retention is highlighted and the incorrectness of non-rhoticity is downplayed by marking it as pleasant and connecting it to “sweet” female voices.

Mead furthermore attempts to explain the difference between the north and the south by suggesting that the southern forms are a result of “the influence that the negroes have had in the speech of the southern people”, thus linking non-rhoticity and yod-retention to Black American speech as well. He suggests that “the negro, by nature, by instinct, discards what is harsh, discordant and unmusical to the ear”, thus creating unusually positive associations between linguistic forms and Black speech. However, this theory is strongly contested by the author of the article, who, after quoting Mead, evaluates it as “very peculiar” and suggests that it is in fact the other way round: “whatever the softness or refinement has come into the negro character and speech has been acquired from contact with the whites of our section”. He emphasizes the connection to refinement by postulating a difference between the “ordinary corn field negro” and “the traditional black mammy or ladies’ maid”, the former’s voice being harsh and unpleasant and the latter’s voice being soft and pleasant as a result of the contact with whites. This link to culture and refinement connects this article to the articles containing \emph{deah} AND \emph{fellah}, in which non-rhoticity is also shown to be linked to fashionable and refined behavior in the (urban) northeast. However, the view that non-rhoticity is a sign of a particularly refined character is criticized and ridiculed in those articles, while it is at the basis of the argument developed by the author of the present article. Considering the results of the analysis of articles containing \emph{bettah}, it is thus remarkable that non-rhoticity is evaluated so positively in this article; however, it is not surprising that the positive evaluation is connected to female speech because it is also the southern girl in “She Got a Seat” who, even though depicted as non-rhotic, is not characterized as uneducated or unrefined.

\begin{ipquote}
\begin{center}
\textstyleStrong{The Speech of Southern Women.}
\end{center}
Mr. Theodore Meade has published a little book on “Our Mother Tongue,” in which he takes occasion to criticise the speech of American women, saying, among other things, that it is really a notable thing, a something that is instantly remarked, when an American woman in speaking has a pleasant voice and uses it with good modulation.

“But,” he continues, “there are women in America who, as a rule, even in conversation, have quite sweet voices and a method of speaking which could be made at once pleasant and correct. I allude to the women of the southern states. This is due, no doubt, to \textstyleStrong{the pure and more sonorous use of the vowels by southern women than by their northern sisters}, and to \textstyleStrong{the less harsh employment of the consonants}. The useful but dangerous ‘r,’ which plays such a sad havoc in the speech of the majority of Americans, is all but ignored by southern women, and many of them are as innocent of ‘r’s’ as cockneys are of ‘h’s’ Probably, however, the musical effect comes to a greater extent from a proper use of the vowels—from giving \textstyleStrong{its full significance, for instance, to ‘u,’ instead of giving that vowel the sound ‘oo.’} A southern woman would never speak of reading a ‘Noo York noospaper;’ she would not eat ‘stoo,’ nor would she go out in the ‘doo,’ She would, however, open the ‘do{\kern0pt}e’ and walk over the ‘flo{\kern0pt}e.’ She knows her ‘u’s’ very well, but has little acquaintance with any ‘r’s’ \textstyleStrong{But even the ignoring of the ‘r’s’ entirely has not in it nearly of the offense of giving them more than their due significance}. A ‘dore’ and a ‘flore,’ where the ‘r’ in each word is long and somewhat rolling, seem something else than what we should be accustomed to, while the pleasant greeting from a friend, ‘Good-morning,’ with a roll in the ‘r,’ has in it something of the sound of menace rather than friendliness.

“I have often wondered why \textstyleStrong{southern women} should use these vowels and consonants \textbf{more pleasantly} than other Americans, and I have reached a conclusion which many southern women—my sisters and other kinsfolk among the rest—will no doubt fail to acquiesce in. They certainly did not achieve these pleasant results through training, because, if anything, they are not as carefully trained at school as girls in the northern states. Nor could it have been entirely by inheritance, for there is not a great difference between the ancestry of the north and south, though in the latter section the people may be a trifle more homogeneously English. The difference, I am persuaded, is due in a very great measure, if not entirely, to the \textstyleStrong{influence that the negro{\kern0pt}es have had in the speech of the southern people}. Children, who start out in life with voices as sweet as the chirping of birds, and tones as pure as the notes of a flute, learn their first words from the nurses in whose charge they are, and southern children are universally reared by negro nurses. […] Now \textstyleStrong{the negro, by nature, by instinct, discards what is harsh, discordant and unmusical to the ear}, and for that reason adopts that which is pleasant and musical to the ear. […]”

{This is certainly a very peculiar theory to advance. The southern woman will hardly appreciate the conclusion that her naturally sweet and harmonious tone of voice has been acquired from negro association. As one who knows whereof she speaks, it is a noteworthy fact that \textstyleStrong{whatever the softness or refinement has come into the negro character and speech has been acquired from contact with the whites of our section. Take the ordinary corn field negro} and contrast the voice with that of the \textstyleStrong{traditional black mammy or ladies’ maid}, and the \textstyleStrong{roughness of the one’s voice and the softness of the other} is at once apparent. No, Mr. Meade, your theory as to the softness of the voices of our southern women is not tenable, as you would find out by short residence.

\raggedleft
{[emphasis mine]}\\
}
\end{ipquote}

Right at the end of the century, on October 20, 1899\citesource{October201899}, an article was published in the \emph{Macon Telegraph} (Georgia) which also argued against the nationwide distribution of yod-dropping. This is already indicated by the heading “By No Means Universal” and the text itself is not only part of the discourse on yod-dropping in America but it also explicitly describes its interaction with the discourse in the English press. The author reacts to a statement made by “Mr. Archer” in the English \emph{Pall Mall Magazine} that yod-dropping is an error which “is distinctly and universally American”. This is denied by the author of “By No Means Universal”, who first argues against the distinctiveness of the error by citing an English newspaper article, published by the \emph{Manchester Guardian}, which points out that yod-dropping can be found in England as well. In this article, Dean Alford’s \emph{Plea for the Queen’s English} is quoted as stating that yod-dropping is “a very offensive vulgarism”, which is “most common in the Midland countries, but found more or less almost anywhere” and which “arises from defective education, or from gross carelessness”. After having established that yod-dropping is not distinctly American, but also an English form, the author of the article in the \emph{Macon Telegraph} continues to argue against the supposed universality of the form by pointing out that it is not universal in any “section” of America except in the northwest. The south is highlighted as a region where the ‘error’ is “rarely if ever made even by the uneducated”. This implies that, according to the author’s judgement, yod-dropping occurs to some extent in other regions, the northeast and the southwest, but that it is a variable form. With regard to the south, the article therefore agrees with the view presented in “The Pronunciation of ‘U’” and “The New Woman”, but with regard to the north, it differs from these articles by making an explicit distinction between the northeast and the northwest. However, the author does not deny that yod-dropping is also found in the northeast; he just claims that it is not universal (not as universal as the 95\% suggested by the author of “The Pronunciation of ‘U’”). The evaluation of the form is the same as in the other articles: It is regarded as incorrect and as signaling a lack of education. Considering that the article takes a perspective which compares England and America, yod-dropping therefore becomes a matter of (linguistic) superiority and inferiority in this article and it is thus clear why the author emphasizes that yod-dropping is not a distinctive and universal form in America: It would make the English in America appear less correct than the English in England. Choosing yod-dropping also allows the author to create a positive evaluation of the south with regard to linguistic correctness, a view which he or she could expect to be liked by the readers of the southern newspaper.

\begin{ipquote}
\begin{center}
\textstyleStrong{By No Means Universal.} 
\end{center}
The \textbf{English press}, following the lead of Mr. Archer who recently discussed “The American Language” in the Pall Mall Magazine, finds much to say about our pronunciation, particularly the substitution of the “oo” for the “u” sound, as in “dooty” and “Doo{\kern0pt}ey,” claiming \textstyleStrong{that this error is distinctly and universally American}. One writer says that an American novelist, “one of the most vigilant of writers,” mimics the speech of an Englishwoman by printing her pronunciation of the word stupid “styoupid” (“stewpid would have been better, being simpler and giving the correct sound of the word when rightly pronounced.) But the Manchester Guardian declares that \textstyleStrong{the error is found in England as well as in America}. It says:

As a matter of fact, this pronunciation is quite English, as Lowell showed in the case of so many habits of speech that we call distinctly American.

“There is a very offensive vulgarism,” wrote Dean Alford in his ‘Plea for the Queen’s English,’ “most common in the Midland countries, but found more or less almost everywhere, giving what should be the sound of the ‘u’ in certain words as if it were ‘oo;’ calling ‘duty’ ‘dooty,’ Tuesday ‘Toosday;’ reading to us that the clouds drop down in the ‘doo; exhorting us’ “Dooly” to do the “dooties” that are “doo’ from us,’ asking to be allowed to see the ‘noospapers” and this is not from incapacity to utter the sound, for though many of these people call new ‘noo,’ no one ever yet called few ‘foo,’ but it arises from defective education, or from gross carelessness.”

{As The Telegraph pointed out when reviewing Mr. Archer’s article, \textstyleStrong{this error of pronunciation is by no means universal in America}. In no section is it universal, \textstyleStrong{unless be it in the Northwest}. As for the \textbf{South}, the improper substitution of the “oo” for the “u” sound is \textstyleStrong{rarely if ever made even by the uneducated}, granted that the speaker be a native.

\raggedleft
{[emphasis mine]}\\
}
\end{ipquote}

However, there is also an article which suggests that the evaluation of yod-dropping is slowly changing in America at the end of the nineteenth century. The article does not contain \emph{noospaper/s} because I found it in a different exploratory search (using the search terms \emph{dooty} AND \emph{pronunciation}), but it offers an important insight into the changing attitude towards the form in question, which is why I add it to the analysis here. It is entitled “Lesson in Pronunciation: Words That Are Frequently Mispronounced Even by Educated People” and it was published in the \emph{Idaho Avalanche} (Silver City, Idaho) on October 18, 1890\citesource{October181890}. It quotes an article by Alfred Ayres, which had been published in the \emph{New York Times}, and which consists of a list of frequently mispronounced words. Yod-dropping is described and evaluated in connection with the word \emph{adduce} and it is first of all noticeable that in contrast to the articles above, yod-dropping is conceptualized as a consonantal difference here (and not as a vocalic one) because the author speaks of the introduction or omission of the “sound of y”. This allows him to construct two extremes: the complete omission on the one hand and the “clear and perfect” realization of /j/ on the other hand. Interestingly, Ayres suggests avoiding both extremes – the omission because it is incorrect and careless and the clear realization because it is “over-nice and consequently smacks of pedantry”. Although this presents the readers with the difficulty of finding out how exactly they should pronounce words like \emph{adduce}, it illustrates a change away from a negative evaluation of yod-dropping. Ayres even points out that “it is not easy to introduce the sound of y” and thus provides the readers with an excuse for a yod-less pronunciation. That yod-retention comes to be associated with pedantry can be interpreted as a sign that yod-retention has become a form used by an educated elite, a small group of people whose insistence on ‘correct’ pronunciation is not generally appreciated, but rather viewed as an ostentatious display of their education and social status.

\begin{ipquote}
\begin{center}
LESSON IN PRONUNCIATION.\\
\textstyleStrong{Words That Are Frequently Mispronounced Even by Educated People.}
\end{center}
{There are in our English at the least three or four thousand words that are frequently mispronounced. Some of these, writes Alfred Ayres in the New York Times, are the following:

\centering
{[…]}  

Adduce. When, in the same syllable, long \textit{u} is preceded by one of the consonants \textit{d}, \textit{t}, \textit{l}, \textit{n}, \textit{s}, or \textit{th}, it is not easy to introduce the sound of y; hence careless speakers omit it, pronouncing duty dooty; tune, toon; lute, loot; nuisance, noosance, etc. And yet to make the \textit{u} in these words as clear and perfect as in mute, cute, etc., is over-nice and consequently smacks of pedantry. The two extremes should be avoided with equal care.}
\end{ipquote}

To conclude the qualitative analysis of selected articles containing \emph{noospaper/s}, I have shown that the main indexical link created is that between yod-dropping and a lack of education. I have analyzed articles containing direct representations of speech, in which the form was linked to voices of different characters, as well as articles containing explicit discussions of the social and regional distribution of the forms. It is striking that while there is some overlap, there are also differences between these types of articles, underlining the importance of taking different text types into consideration. In the fictional letters, anecdotes, reports, stories and short humorous paragraphs, speakers exhibiting yod-dropping are usually portrayed as uneducated, but there are differences with regard to other social values which, interestingly, correlate with region. Southern characters are usually portrayed in a negative way, with the important difference that in earlier articles, yod-dropping is used to differentiate negative from positive southern figures (the Republican whose political views and behaviors are satirized in the letter to “Doctor Hun” in 1816 and the neighbor in the dramatic text published in 1826 in the \emph{Louisiana Advertiser} who does not pay for his newspaper subscription), whereas in the later articles, the negative portrayal of southerners is rather contrasted with a positive image of northerners (the man in Kentucky jeans who does not recognize the post office in New York in 1883, the father of the “Small Kentuckian” ridiculing the “Yankee” custom of New Year’s resolutions in 1894 and the “Out-of-Town People” impersonating southern rural ignorance which is not found to the same extent in the north in 1894). Black speakers are also represented as dropping the yod and they are associated with very negative social values as well, but only in very few articles like for example in “Darkeygraphy” (1858). In contrast to the negative portrayals of southerners and Black people, figures from the north or the west are characterized more positively, despite their lack of education or lower social position: the New York newsboy, a former bootblack, is depicted as someone who knows about business and who is socially upwardly mobile (1887), the illiteracy of the deacon in rural Wisconsin is mainly represented to emphasize the level of education and intelligence of his wife (1888) and the two tough hunters in mountain regions in the southwest and in the northwest are depicted as strong and tough men who help easterners to survive in an environment dominated by natural forces and not cultured civilization (1887 and 1898).

The explicit discussions of yod-dropping confirm the link between the phonological form and a lack of education by calling it a vulgarism and an error and by classifying it thus as incorrect and deviating from a norm. However, in contrast to the representations of direct speech in fictional texts, three articles point out explicitly that yod-dropping is \emph{not} found in the south: “The Pronunciation of ‘U’” (1879), “The New Woman” (1899) and “By No Means Universal” (1899). Furthermore, in “The New Woman” yod-retention by southern women is attributed to the influence of Black speech, a view which is in contrast to the representations of yod-dropping as part of Black speakers’ voices. This shows that explicit and implicit metadiscursive strategies do not necessarily create the same indexical links and can also contradict each other. It is noticeable that despite the emphasis on the south as a yod-retaining region in the explicit discussions of the form, there is at the same time no common agreement on where yod-dropping is actually found in America. In the three articles which consider it a non-southern form, emphasis is placed on “the north” as a region where it does occur; in “By No Means Universal”, it is more specifically the northwest which is highlighted as a region (the only region) where it is universal. “Teaching in the West” (1887), however, places the form in the east and implicitly also in the west (because it needs to be taught there), which creates the impression that it is fairly universal in America. That the universality of yod-dropping is a perception that plays a role in linguistic metadiscourses is indicated by the article “By No Means Universal” because the very need to counter the claim of universality means that this claim is present in the first place.

Taking both types of strategies to create indexical links into account (implicit and explicit ones) therefore reinforces the impression that yod-dropping is not a salient marker of region, but rather of social characteristics like a lack of education or civilization. However, as the article “Lesson in Pronunciation” shows, even this link seems to become weaker because the presence of /j/ is associated with pedantry here, thus marking a speaker negatively as paying too much attention to correctness. In general, the analysis showed that yod-dropping is not particularly highlighted and very often occurs just once, whereas for example non-rhoticity or hyper-rhoticity occur much more frequently when they are represented in articles. This is of course not surprising as /j/ following an alveolar consonant in an onset cluster is of course much less likely to occur in speech than post-vocalic /r/ or a final vowel to which an /r/ can be added. Nevertheless, the authors of the anecdotes could have developed strategies to draw attention to the difference between forms with yod and without yod, but in most cases, they did not employ such strategies. The articles which focus on discussing only yod-dropping (or yod-dropping and non-rhoticity) in more detail evidently draw much attention to the form; they are, however, very small in number (at least in this collection of texts based on the search term \emph{noospaper/s}). Overall, yod-dropping is thus not an index of a specific region or specific social group, but rather of more general social characteristics, and it is also not a very salient one.

\subsection{Indexical links to social personae: a quantitative overview}
\label{bkm:Ref10890365}\hypertarget{Toc63021240}{}
To complete the picture resulting from the detailed qualitative analyses carried out in \sectref{bkm:Ref534386498}, I will now add a quantitative analysis that aims at providing an overview of the more general patterns that can be found when all articles are taken into account and not just selected ones. This analysis will focus on the central social personae and social groups identified in \sectref{bkm:Ref534386498} by asking first in how many articles these groups become linked to particular search terms and thus to the respective phonological forms and secondly, how the frequency of articles linking the social personae to the forms changed over time. Before presenting the result of the analysis, a methodological issue needs to be addressed: It concerns the identification and classification of the social personae in the articles. One problem is that the social persona that the form is linked to is not always explicitly mentioned. Sometimes, just a name is given; sometimes no indication is given at all. In these cases, a decision needed to be made to either classify the social persona as ‘unclear’ or to use available clues to identify the social persona. Dudes and Black Americans could often be identified based on their names – typical names of the dude were for example Cholly, Chappie, Algie as well as Mr. Delawney or Young Mr. Sissy, while typical Black American names were for example Uncle Rastus and Aunt Sheba (the address terms Uncle and Aunt being an indicator of Blackness), but also Sambo and Ebenezer. I largely avoided identifying social personae solely based on their linguistic repertoire because this would have led to a circularity in my argument. Since my question is who the forms are linked to, I could not assign the voice represented or described to a social group or persona based on the forms. This was especially the case with Black Americans because there is some overlap in their linguistic repertoire with southern white speakers and mountaineers, so if in doubt, I classified the social persona as ‘unclear’. However, in most of these unclear cases, the linguistic repertoire points to a Black person. In the case of the dude, there were a few cases where the articles consisted only of a dialogue without any indication of the names of the speaker. If the content \emph{and} the linguistic repertoire pointed to the dude figure, I classified the social personae as dudes. This classification is less likely to be wrong because the linguistic repertoire of the dude is clearly distinct from other repertoires; especially the realization of /r/ with a labiodental approximant is almost always associated with the dude. In most cases, however, the persona linked to the form could be identified clearly, either through explicit labels like “dude”, “darkey” or “mountaineer” or through descriptions like the following which is taken from a fictional story written by Laura B. Marsh and published in the Daily Inter Ocean on April 2, 1893\citesource{MarshApril21893}: “One of them was a white boy—that was Jimmy Price. The other, the hired boy, was perhaps a few years older, and black as ever African could be. For this reason he was called ‘Snow’”.


The second problem was that sometimes several social personae are indexed in one article. This was especially the case in articles which discuss how a particular form is used by fashionable English people and imitated by Americans for this reason. In most cases, I have decided to only consider one social persona – the more important one, which in these cases is the American figure because it is the main motivation for authors to write the article in the first place. Both links were equally prominent in only two articles containing \emph{dawnce}. I therefore counted the respective article twice, once as linking the form to English people and once as linking the form to white Americans. This has the disadvantage of making the overall number of articles containing \emph{dawnce} seem larger, but as there are only two of these cases, this is negligible.

The main social personae or social groups that I have used for the quantitative analyses here are the following ones: First, the white American, which subsumes white Americans of all genders, ages and regions. Within this rather large group, I distinguished one sub-group of white Americans, the American dude. I counted all articles as linking the form to the dude which appeared in the 1880s and 1890s (because this is when the figure appeared in the newspaper discourse) and in which the form is linked to a male figure with the typical characteristics of the dude. This means that I included articles describing male American swells and “Anglomaniacs”, even if they were not labeled as dudes, but I did not include articles featuring female personae like the society girls or salesladies. Further subdivisions did not make sense for the analysis of all search terms because for some search terms, the number of articles linking the forms to white Americans (not dudes) was so small that no sub-groups could be determined. However, the analysis of articles containing the search term \emph{bettah} revealed the sub-groups of white southerners and mountaineers, which is why I will take a closer look at the social personae linked to \emph{bettah} in an additional analysis. The second social persona I used for the quantitative analysis is the English speaker. In the overwhelming majority of cases, these English speakers were male, but I did not restrict the group to males only, so I just labeled it “English”. Thirdly, I distinguished the group of Black people. This group comprised all people whose skin color was described as black or “colored” – in most cases those people were Black Americans, but there were also a few articles in which the form was linked to Black people in other, non-American, contexts, which is why I chose ethnicity as the main criterion for classification. If the article contained links to a social persona that could clearly be identified (it was thus not “unclear”), but that could not be assigned to any of the groups described above, I classified it as “other”. If no characterization of the persona linked to the form was provided at all, I classified it as white American because I assume that this is how the reader of an American newspaper would categorize the persona as well. This means that the characterization of the personae in the “other” group always involved some form of othering which highlighted characteristics that mark the person as un-American – often by emphasizing their status as immigrants and their (former) nationality (e.g. Dutch, German, Irish, Indian).

\figref{fig:key:31} and \figref{fig:key:32} show the number of articles linking the social personae to the search terms (and thus to the phonological forms they represent) and the percentage of articles, respectively. A striking observation that can be made based on these figures is that for almost all search terms one social persona predominates to a great extent. Almost all articles containing \emph{hinglish} link /h/-dropping and -insertion to English speakers – only 7 out of 107 articles link it to white Americans (1) or other social personae (4); in two articles the social persona is unclear. \emph{Noospaper/s} is primarily associated with white Americans – only 6 articles contain links to English speakers and another 6 articles contain links to Black speakers. The number of “other” social personae is highest for this search term (14 articles) – a reason for this is the high number of Dutch and German people portrayed as speakers dropping the yod. \textsc{Twousers} is almost exclusively associated with the dude’s linguistic repertoire (92 out of 103 articles), which is interesting because even though the labiodental realization of /r/ is constructed as an English form imitated by American dudes, it is only in one article that an English speaker is actually portrayed as using the form (the article with the young Oxford swell discussed in \sectref{bkm:Ref534386498}). This indicates that the indexical link to Englishness has become weak and that it is only indirectly present in the indexical link to the dude, the link that has come to predominate. In contrast to \textsc{twousers}, the search term \emph{deah} AND \emph{fellah} is still linked to English speakers in a comparably large proportion of the articles (25 out of 157 articles), which shows that the indexical link to the characteristic of being English is still stronger for non-rhoticity than for the realization of /r/. Nevertheless, the dude figure is linked to non-rhoticity in the overwhelming majority of articles containing \emph{deah} AND \emph{fellah} (116 articles). The search term \emph{bettah} shows that non-rhoticity is also associated with white Americans who are not dudes and, most notably,Black speakers, who are linked to \emph{bettah} in two thirds of the articles (246 out of 374). This shows again how the lexical level interacts with the phonological level in \emph{deah} AND \emph{fellah} and it underlines the importance of using several search terms to confirm whether the results can be attributed mainly to the phonological form or whether the lexical item(s) play a role as well. With regard to the dominance of one social persona, the search term \emph{dawnce} is an exception because even though it is linked to white Americans in almost half of the articles (34 out of 75 articles), the proportion of articles associating the form with other social personae is also fairly large (dudes in 16 articles and English speakers in 12 articles).



\begin{figure}
\includegraphics[width=\textwidth]{figures/Paulsen-img31.pdf}
\caption{
Social personae linked to phonological forms (number of articles per search term)
}
\label{fig:key:31}
\end{figure}
\begin{figure}
\includegraphics[width=\textwidth]{figures/Paulsen-img32.pdf}
\caption{
Social personae linked to phonological forms (percentage of articles per search term)
}
\label{fig:key:32}
\end{figure}

Comparing the results for \emph{dawnce} and \emph{bettah}, it is noticeable that even though the number of articles linking \emph{bettah} to white Americans and dudes is relatively low in comparison to those linking \emph{bettah} to Black speakers (81 vs. 246 speakers), it is higher than the number of articles linking \emph{dawnce} to white speakers and dudes (50). In terms of absolute numbers, the group of white Americans linked to \emph{bettah} is thus prominent enough to deserve closer analysis. Such an analysis makes it possible to assess the relevance of the groups of white southerners and mountaineers in quantitative terms. \figref{fig:key:33} shows the number of articles linking \emph{bettah} to these two groups as well as to dudes and reveals that all three groups are almost equally large, which shows that none of the groups are marginal even though they are not as salient as Black Americans (if salience is based on frequency).


\begin{figure}
\includegraphics[width=.8\textwidth]{figures/Paulsen-img33.pdf}
\caption{
Number of articles linking white Americans to \emph{bettah}
}
\label{fig:key:33}
\end{figure}

In addition to the absolute numbers of articles, it is also important to look at the development of the links between phonological forms and social personae over time. \figref{fig:key:34} focuses on the social personae and shows that from the 1830s on, there is a fairly stable number of articles linking phonological forms to English speakers (mostly between 2.7 and 4.3 articles per million). Links to white Americans also occur throughout the century, with a particularly large proportion in the early second half of the century, due to the high number of articles containing yod-dropping (particularly in the 1860s). Links to Black speakers occur later (in the 1850s), but they increase towards the end of the century and constitute a large proportion of indexical links in the 1880s and 1890s. The social persona that appeared latest is unsurprisingly the dude and it can be seen that this figure quickly assumes a prominent position among all social personae. The largest number of articles containing the search terms appeared in the last two decades and the diagram shows that Black speakers and the dude are the most salient social personae who are linked to the phonological forms investigated in the present study in this important time period.


\begin{figure}
\includegraphics[width=\textwidth]{figures/Paulsen-img34.pdf}
\caption{
Social personae linked to phonological forms in the course of the nineteenth century
}
\label{fig:key:34}
\end{figure}

Concerning indexical links to southerners and mountaineers, a separate analysis shows the temporal development of the appearance of social personae in articles containing the search term \emph{bettah}. \figref{fig:key:35} shows that the links to southerners increase in the last three decades of the nineteenth century, especially towards the end. Mountaineers appear in the 1880s, but only in a low proportion of articles, which then increases greatly in the 1890s, while the articles containing links to the dude decrease slightly. This shows that while in the 1880s the main white figures linked to non-rhoticity (represented by \emph{bettah}) are dudes and, to a much lesser extent, southerners, the proportion of links to southerners and mountaineers becomes much greater in the 1890s and surpasses the number of articles linking it to the dude.




\begin{figure}
\includegraphics[width=.8\textwidth]{figures/Paulsen-img35.pdf}
\caption{
Number of articles (per million) linking white Americans (including dudes) to \emph{bettah} (temporal development)
}
\label{fig:key:35}
\end{figure}

Finally, some important observations can be made by looking at each social persona separately and analyzing how often they were linked to the different search terms in each decade. For English speakers, it is not surprising that they are linked to \emph{hinglish} most frequently in all decades, but the 1860s and, most strikingly, the 1880s show that other search terms are also linked to English speakers (see \figref{fig:key:36}). Especially with regard to non-rhoticity, the figure shows that \emph{deah} AND \emph{fellah} is much more often linked to English speakers than \emph{bettah}. As \emph{deah} AND \emph{fellah} is most prominently linked to the dude, who is portrayed as imitating the English, it is not surprising that the link between this particular search term and English people is stronger than that of \emph{bettah}, which is not as strongly connected to the dude, but also to several other social personae.


With regard to white American speakers, \figref{fig:key:37} shows that until the last two decades, they are indexically linked almost exclusively to \emph{noospaper/s}. In the 1880s, they are as often linked to \emph{noospaper/s} as to \emph{dawnce}, and, to a lesser but still significant extent, to \emph{bettah}. Towards the end of the century, the links to \emph{bettah} increase and \emph{bettah} becomes the form most frequently linked to white Americans because the links to \emph{noospaper/s} decrease slightly and the links to \emph{dawnce} decrease considerably. \figref{fig:key:38} shows the development of the search terms linked to the dude and it can be seen that there are no great changes, except the increase of articles containing \textsc{twousers}, which is responsible for the overall increase of articles creating indexical links to the dude figure in the 1890s. Concerning Black speakers, it is not surprising that they are almost exclusively linked to \emph{bettah}. \figref{fig:key:39} shows additionally that while Black speakers occur already in articles in the middle of the nineteenth century, it is particularly in the 1870s and 1880s that their number increases to a very large extent.

\begin{figure}
\includegraphics[width=\textwidth]{figures/Paulsen-img36.pdf}
\caption{
Phonological forms linked to English speakers (temporal development)
}
\label{fig:key:36}
\end{figure}
\begin{figure}
\includegraphics[width=\textwidth]{figures/Paulsen-img37.pdf}
\caption{
Phonological forms linked to white Americans (temporal development)
}
\label{fig:key:37}
\end{figure}
\begin{figure}
\includegraphics[width=\textwidth]{figures/Paulsen-img38.pdf}
\caption{
Phonological forms linked to dudes (temporal development)
}
\label{fig:key:38}
\end{figure}
\begin{figure}
\includegraphics[width=\textwidth]{figures/Paulsen-img39.pdf}
\caption{
Phonological forms linked to Black speakers (temporal development)
}
\label{fig:key:39}
\end{figure}

To conclude, these quantitative overviews show very well that while English speakers and white American speakers are indexically linked to phonological forms for most of the nineteenth century, the other social personae appear in larger numbers of articles in the last three decades of the century. In fact, Black speakers and dudes are the social personae indexed most often in articles in the 1880s and 1890s and while Black speakers are closely tied to non-rhoticity represented in \emph{bettah}, dudes are portrayed as non-rhotic speakers using \emph{deah} AND \emph{fellah} as well as users of \textsc{twousers} and to a lesser extent of the back vowel in \textsc{bath} (represented by \emph{dawnce}). With regard to non-rhoticity in \emph{bettah}, I have also shown that white southerners and mountaineers become increasingly important in the last two decades as well, but in absolute numbers, they do not appear as frequently as Black speakers. Before discussing how the results of the qualitative and quantitative analyses of the indexical links between social personae and values and phonological forms can be interpreted within the framework of enregisterment, I will present the results of the analysis of indexical values linked to lexical forms in the next section.


\section{Metadiscourses on lexical forms}
\label{bkm:Ref11924664}\hypertarget{Toc63021241}{}
In this section, I will present the analysis of articles containing two variants of a variable: firstly \emph{luggage} and \emph{baggage} and secondly \emph{pants} and \emph{trousers}. The first pair of forms will be analyzed quantitatively in \sectref{bkm:Ref7426387} based on a collection of articles containing both forms appearing within 10 words of each other. The collection is restricted to the database NCNP because it yielded enough articles for a quantitative analysis and since considerable differences to the database AHN are not expected, there are no disadvantages of restricting the analysis to one database. Both pairs of forms will then be analyzed qualitatively in \sectref{bkm:Ref534386276} and \sectref{bkm:Ref9413051} to show which indexical values and social personae become associated with each lexical variant.


\subsection{\emph{luggage} vs. \emph{baggage}: frequency and temporal and regional distribution of articles}
\label{bkm:Ref7426387}\hypertarget{Toc63021242}{}
Using the search term \emph{luggage} n10 \emph{baggage}, I collected all articles containing the two lexical forms within a range of 10 words in the database NCNP. This included lexical items in which the forms function as part of a compound, e.g. \emph{baggage car} or \emph{luggage van}. The search yielded 185 articles, and the first analysis revealed that they could be divided into two groups: Either they contained indexical links between \emph{luggage} and \emph{baggage} and British English or American English respectively or no such links could be found. I therefore used two categories for the following quantitative analyses, “AE/BE yes” for the former group (72 articles) and “AE/BE no” for the latter group (113 articles). In one case, an indexical link to Canadian English was created, and I included the article in the “yes” category because it still distinguishes a North American and a European context. In another case, it was not an English speaker that was indexed but a white American who can be regarded as belonging to the groups of Americans imitating English speakers. Due to the implicit connection to English speech, I also included the article in the category “AE/BE yes”.

\begin{figure}[b]
\includegraphics[height=.4\textheight]{figures/Paulsen-img40.pdf}
\caption{
Number of articles containing \emph{luggage} and \emph{baggage} (within 10 words of each other) and indexical links to American English or British English
}
\label{fig:key:40}
\end{figure}


\figref{fig:key:40} shows the development of the number of articles containing \emph{luggage} and \emph{baggage} over the course of the nineteenth century. The black area shows the proportion of articles containing indexical links to British English and American English and the grey area shows the proportion of articles containing no such links. It can be seen that the overall number of articles increases already towards the mid-century (1830-1850), then decreases slightly until the 1870s, followed by a sharp increase in the 1880s and a smaller decrease in the 1890s. In contrast to the development of all articles containing the two lexical items in close proximity, the number of articles in which \emph{luggage} is indexically linked to British English speech and \emph{baggage} to American English speech increases only in the second half the century, especially in the 1880s. It is only in this decade that the proportion of articles containing indexical links is larger than that without such links.




It is striking that the temporal development of articles containing indexical links between the lexical items and American and British English is similar to that of several sets of articles containing phonological search terms. This can be seen in \figref{fig:key:41}, which shows that \emph{dawnce}, \textsc{twousers,} \emph{deah} AND \emph{fellah} and \emph{bettah} also increase significantly in the 1880s. The decrease in the 1890s also mirrors closely that of articles containing \emph{dawnce} (and to a lesser extent those containing \emph{deah} AND \emph{fellah}). This observation thus highlights the importance of the last two decades of the nineteenth century for the creation and circulation of indexical links.


\begin{figure}
\includegraphics[height=.4\textheight]{figures/Paulsen-img41.pdf}
\caption{
Number of articles containing the search terms in NCNP (per million): comparing phonological search terms to the \emph{luggage} n10 \emph{baggage} (AE/BE yes)
}
\label{fig:key:41}
\end{figure}



\begin{figure}
\includegraphics[height=.4\textheight]{figures/Paulsen-img42.pdf}
\caption{
Regional distribution of articles containing \emph{luggage} n10 \emph{baggage} which link the lexical items to British English and American English respectively
}
\label{fig:key:42}
\end{figure}

With regard to the regional distribution of articles, \figref{fig:key:42} shows the number of articles per state of those 72 articles which indexically link \emph{luggage} and \emph{baggage} to British English and American English respectively. What can be observed is that Wisconsin and Kansas are the states with the highest number of articles (12 articles each), two states in which also many articles containing \textsc{twousers}, \emph{deah} AND \emph{fellah} and \emph{bettah} were published. At the same time, however, several articles were also published in California in the southwest (8 articles) and in Massachusetts (6) and Maine (4) in the northeast, which shows that the circulation of articles differentiating \emph{luggage} and \emph{baggage} as English and American forms is not restricted to a particular region. The absence of articles in Pennsylvania, in the north and in the southeast is as noticeable as the presence of articles in Utah because that differs from the distribution of articles containing other search terms (except for Utah, where also a fairly large number of articles containing \emph{hinglish} was published, which is interesting because in the case of \emph{luggage} and \emph{baggage} it is also the contrast between England and America that is the focus of the creation of indexical links). In general, articles do not appear primarily in the northeast (like those containing \emph{hinglish}), but they also do not primarily appear in the (mid)-west and Pennsylvania (like those containing the non-rhotic search terms and \textsc{twousers}). Due to the absence in the north and in the southwest, their distribution does not seem to be as wide as that of the other search terms. However, as the number of articles is fairly low, it is only possible to identify some tendencies with regard to circulation.




To conclude, the quantitative analysis shows that while articles in which both lexical forms are used within 10 words of each other appear already in the first half and the middle of the century in relatively large numbers, the number of articles containing indexical links between the forms and American and British English only increases notably towards the end of the century. This follows the same pattern as that of those phonological search terms associated with the dude and with Black speakers, and it underlines the importance of the last two decades in the creation of indexical links. The regional distribution shows that articles evaluating \emph{luggage} and \emph{baggage} as British and American forms are not restricted to any particular region, although the absence of articles in the north (except of course Wisconsin) and the southeast is noticeable. The low number of articles needs to be taken into consideration as well when comparing this set of articles to that containing other search terms. The next section will thus focus on a qualitative analysis of selected articles to show how the indexical links to American and British English were created, how the use of the American form (\emph{baggage}) was evaluated and how the context (including the region) possibly influenced this evaluation.


\subsection{\emph{luggage} vs. \emph{baggage}: indexical values and social personae}
\label{bkm:Ref534386276}\hypertarget{Toc63021243}{}
When analyzing indexical links between \emph{luggage} and \emph{baggage} and social values as well as social personae, it should not be forgotten that the majority of articles do not contain such links at all because both terms are used without any apparent difference in either semantic or social meaning. An example of such an article is a travelogue published in the \emph{Daily Rocky Mountain News} on May 3, 1874\citesource{May31874}. The author describes how “baggagemen” transfer “luggage” to “baggage vans”, which shows that the meaning of \emph{luggage} and \emph{baggage} must be the same because the latter contributes the same meaning as the first to compounds (the collection of property that people take with them when they travel is transported by men and by vans).

\begin{ipquote}
\begin{center}
\textstyleStrong{First impressions of Denver.}\\
{[…]}
\end{center}
{Polite and attentive \textbf{baggagemen} attend to the transfer of the passenger's \textbf{luggage} from the cars to elegant \textbf{baggage} \textbf{vans}, and equally courteous conductors act as escort to omnibusses and carriages, as fine as can be found in the streets of New York, Boston or Chicago, drawn by magnificent matched teams.

\raggedleft
{[emphasis mine]}\\
}
\end{ipquote}

A second example of an article without indexical links is a Supreme Court decision published in the \emph{Daily Evening Bulletin} in San Francisco on August 09, 1890\citesource{August091890}. In the decision, \emph{baggage} and \emph{luggage} are used once in a coordinated noun phrase with \emph{and}, which suggests a possible difference in semantic meaning, but they are also used in a coordinated noun phrase with \emph{or}, which suggests that there is no semantic difference, and that they are alternative terms for the same thing. As linguistic precision is particularly important in legal contexts, it is noticeable here that even though both terms are used, they are not distinguished in meaning, which indicates that the Supreme Court acknowledges that both terms are in use, but that there is no semantic difference, at least none that matters legally. There is also no indication of any different social meanings connected to the different lexical forms.\footnote{The author of the article, Associate Justice of the Supreme Court of California John Sharpstein, quotes the definition of \textit{luggage} in the Civil Code: “Luggage may consist of any article intended for the use of a passenger while traveling, or for his personal equipment”. This suggests that the lexical item used in legal texts is \textit{luggage}. Nevertheless, Sharpstein uses both terms, \textit{luggage} and \textit{baggage}, to express the same meaning in his explanation of the Supreme Court decision in the article. It therefore seems as if \textit{baggage} is so commonly used that it appears in legal contexts as well, despite the fact that \textit{luggage} is the term used in the legal text itself.}

\begin{ipquote}
\begin{center}
\textstyleStrong{Supreme Court Decisions.}\\
{[…]}
\end{center}
{The Court finds that at Kansas City, Mo., in January 1888, the plaintiff engaged passage on the defendant's railroad to Coitin, in this State, bought a ticket, paid his fare and checked his trunk, containing, among other things, one lady's gold watch and chain of the value of \$150; [...], which the Court found to be “proper articles of \textbf{luggage} \textbf{and} \textbf{baggage} for the plaintiff to carry as such.” [...] And [the appellant's] contention is that said articles or none of them constituted what in law is defined to be \textbf{luggage} \textbf{or} \textbf{baggage}.

\raggedleft
{[emphasis mine]}\\
}
\end{ipquote}

The first article which creates indexical links to the forms and thus distinguishes them not semantically but with regard to their social meaning was published in the \emph{Weekly Ohio Statesman} (Columbus, Ohio) on September 24, 1845\citesource{September241845}. The correspondent writes a letter to the newspaper from Liverpool and after using the term \emph{baggage}, he adds the information “always called luggage in England” and thus explicitly links the term \emph{baggage} to American English and \emph{luggage} to British English.

\begin{ipquote}
{\centering
\textstyleStrong{Foreign Correspondence of the Statesman.}

\raggedleft
\textsc{Liverpool}, \textit{Clayton Arms Hotel},\\
August 18\textsuperscript{th}, 1845

Travelers frequently speak of the annoyances of the Custom House. Mr. Willis, in one of his recent letters, invokes the retaliation of American Custom House officers upon English travelers, on account of the trouble he received in the passage of his baggage here. He was probably in a bad humor.}

{To make you acquainted with all the particulars of passing \textstyleStrong{baggage (always called luggage in England,)} I will relate exactly what took place with mine. {[…]}

\raggedleft
{[emphasis mine]}\\
}
\end{ipquote}

Even though the article above explicitly creates a difference in social meaning between \emph{baggage} and \emph{luggage}, neither of the forms is evaluated positively or negatively. The author states the difference neutrally as a difference between the two countries. This neutral perspective is adopted in almost half of the articles which link the forms to American and British English respectively (34 out of 72). In the rest of the articles, an evaluation of the terms is expressed – sometimes explicitly, but sometimes also rather implicitly. In the vast majority of cases (35 articles), the American form, \emph{baggage}, is evaluated positively. Only in 3 cases can a decidedly negative evaluation of \emph{baggage} be found, and in one exceptional case, the author states that he is undecided about how to evaluate the form.

As a first step, I will analyze some examples of articles which illustrate how \emph{baggage} becomes evaluated positively as an American form. In the collection of articles obtained for this analysis, the first article conveying such a positive evaluation is headed “The English Language” and it was published in the \emph{New Hampshire Statesman} (Concord, New Hampshire) on September 21, 1866\citesource{September211866}. It is a letter from a correspondent of the \emph{Cincinnati Gazette}, in which he writes about his experiences in England, including differences between English and American speech. The fact that he labels the English expressions “peculiarities” indicates that he regards the American variants positively as a norm from which the English variants deviate. Even though he recommends the readers to study the English forms, this is merely for practical reasons – in his view, knowing about the differences will help the American traveler to communicate in England. The article thus attests to an endonormative orientation visible already in the 1860s. Differences to British usage are observed, but they do not call into question the use of the American variant in any way.

\begin{ipquote}
\begin{center}
\textstyleStrong{The English Language}  
\end{center}
A correspondent of the Cincinnati Gazette, now in England, writes the following:

{Before leaving America I thought it desirable to pay some fresh attention to the French language, in order to travel successfully in France. \textstyleStrong{It never occurred to me that I ought to study the English language, with a special view to facilitating my travels in England}. Nevertheless, I have sometimes been put to inconvenience from my ignorance of \textstyleStrong{some peculiarities in the English mode of speaking}.

\centering
{[...]}

From time to time I have noticed the peculiarities of speech which have struck me ; and if your readers will not accuse me of pedantry, I will give them some extracts from my observations, promising not to deluge them with a general jail delivery of the contents of my notebook.}

{\textstyleStrong{Many of these peculiarities occur in connection with one’s travels.} Thus \textit{railroad} is always here a “railway.” Nobody speaks of \textit{cars}. The train is made up of “carriages.” They do not talk of selling \textit{tickets}, and they have no \textit{ticket office}, so-called. But when they give you a ticket and take your money, they are said to “book” you, and the office is a “booking office.” \textit{\textbf{Baggage}} \textstyleStrong{is “luggage,” and a baggage car is a “luggage van.”} […]

\raggedleft
{[emphasis mine]}\\
}
\end{ipquote}

The second article in the collection with a positive evaluation of \emph{baggage} was published in 1873 and the third one in 1882. The latter is particularly interesting because it combines questions of semantic difference with questions of technical progress and superiority. The title of the article is “English Recklessness” and it was originally published by the \emph{New York Tribune} and then reprinted as a part of a collection of several articles about railroads, first in the \emph{St. Louis Globe-Democrat} (Missouri) on December 20, 1882\citesource{December201882}, and then in the \emph{Cleveland Herald} (Ohio) on December 22, 1882\citesource{December221882}. The author comments on differences between English and American “railroad management”, which manifest themselves in the use of different terms but also in the use of “methods of operating railroads” and “devices for promoting the comfort of passengers”. The author of the article aims to show that while Englishmen regard American innovations as reckless, they are actually reckless themselves because due to a lack of safety measures an accident occurred on a train in England that, according to the author, could not have occurred in America. There is thus a clear ironic undertone, when the author speaks of “the hazards of American recklessness”. In the same vein, it becomes clear that he does not share the British view that “having different names for the same things” is “an unnecessary way of asserting their [American] individuality”, but that, quite to the contrary, Americans have a right to be self-confident due to their superior position regarding technology and operating systems and thus, implicitly, a right to their own names. The use of \emph{baggage} instead of \emph{luggage} is therefore not only linked to American usage, but it also indexes autonomy, individuality and superiority based on technological innovation and progress.

\begin{ipquote}
\begin{center}
\textstyleStrong{English Recklessness.}\\
{[From the New York Tribune]}
\end{center}
{Englishmen have been harassed these many years by American methods of railroad management. The divergence of views dates from the introduction of the system in both countries. The English were annoyed at the outset because the Americans persisted in calling a shunt a switch, a driver an engineer, a stoker a fireman and luggage baggage. This seemed a very unnecessary way of asserting their individuality. Not satisfied with having different names for the same things, the Americans devised characteristic methods of operating railroads, and introduced many devices for promoting the comfort of passengers.

\centering
{[…]}

Englishmen have had for thirty years a characteristic word which they have applied to American railroad methods.} {That word is “reckless.” They have adhered with dogged persistence to many inconveniences of their own system rather than expose themselves to the hazards of American recklessness.

\centering
{[…]}

American railroad officials will read the details of this accident [in England] with a grim smile.} They have been censured so often by English critics of the recklessness of their methods that they will derive consolation from this exhibition of incapacity and carelessness. Passengers in American sleeping-cars are not allowed to smoke in their berths, whether they are drunk or sober, nor can they take their reading-lamp to bed with them; and if an engineer receives warning that something is wrong and that the train must be stopped instantaneously, he is not prevented by the printed regulations of the company from saving the lives of the passengers.
\end{ipquote}

Another important value indexically linked to \emph{baggage} is that of authenticity. This value is established, on the one hand, in relation to Englishmen and, on the other hand, in relation to the group of Americans who hold English linguistic forms in high regards. The latter set of articles connects the discourse on \emph{luggage} and \emph{baggage} to the discourse on phonological forms linked to the American dude and other white American speakers imitating English speech. However, I will discuss an article first that establishes authenticity based on national identity. It is also headed “The English Language” and it was published in the \emph{Galveston Daily News} (Houston, Texas) on October 31, 1886\citesource{October311886}, and in the \emph{Morning Oregonian} (Portland, Oregon) on December 07, 1886\citesource{December071886}. It describes and lists a large number of linguistic differences between England and America, among others lexical items belonging to the “lingo of railways”. The respective paragraph illustrates that \emph{baggage} and \emph{luggage} are one pair among many others, which makes the form not particularly salient in the context of this article. Nevertheless, the attitude expressed towards the American forms is also valid for \emph{baggage}: The author constructs it as a sign of national identity and a marker of “a true and sincere American” and thus creates a link to American nationality as well as to authenticity. Authenticity is, in fact, at the basis of his argument because he finds that if a high value is placed on authenticity, it cannot be detrimental for an American to signal his nationality. On the contrary, hiding one’s nationality would imply that the speaker is untrue and insincere. Another important aspect that becomes visible in this article is the connection to discourses on language beyond newspaper articles: The author explicitly refers to “letters from both sides of the Atlantic”, “books on travel” and “conversation[s] among returned travelers”, which all contribute to the discussion on linguistic differences between England and America. This also shows that discourses on language in England and America are not separate from each other, but that they interact in many ways. Also within newspaper discourse, the article shows the connection between England and America as well as between different regions in America: The author cites an article by a London correspondent of the \emph{Argonaut}, a political journal based in San Francisco, California, thus emphasizing the connection between America and England, and the fact that this article was published in Texas and Oregon shows that discourses on language in America spread to regions far apart in the United States.

\newpage

\begin{ipquote}
\begin{center}
THE ENGLISH LANGUAGE    
\end{center}
{The subject of English pronunciations of different words, names, and places totally out of accord with the spelling, together with the \textstyleStrong{difference that exists in England and America in respect to the expression of the same thing}, has been frequently touched upon in letters from both sides of the Atlantic, referred to incidentally in books of travel, and often made the topic of conversation among returned travelers, whether from Liverpool or New York, says a London correspondent of the Argonaut. But I don’t think the subject has ever been thoroughly gone into. Yet a knowledge of these pronunciations and expressions would be extremely useful to the American traveler in England.

\centering
{[…]}

\textstyleStrong{The lingo of railways differs wonderfully.}} {Railroad is railway; the track is the line, and the rails the metals; the cars are the train; to switch is to shunt; a turnout is a siding; a locomotive is an engine; an engineer a driver, and a fireman a stoker. The conductor is the guard, a car is a carriage, \textstyleStrong{baggage luggage, a baggage car a luggage van}, and a freight train a goods train. A depot is a terminus or a station, and a switch-tender a pointsman or signal man. […]

\centering
{[…]}

Now, I don’t mean to contend, and I hope I shall not be misunderstood as contending, that it is necessary, if, indeed, in all senses desirable, for any American visiting England to in the least sink his \textbf{national identity}.} Far from it. There is nothing that a true and sincere Englishman likes and admires more than \textstyleStrong{a true and sincere American}. {It will be no detriment to a man among true Englishmen that he is an American. But while it is not desirable to hide one’s nationality, and is a sign of bad form to do so, it is equally lacking in good form to parade one’s birthplace and make a blowing-horn of it. […]

\raggedleft
{[emphasis mine]}\\
}
\end{ipquote}


The article thus illustrates very importantly that nationality is connected to authenticity and that this connection forms the basis for evaluating the use of \emph{baggage} positively. The same argument can also be found in other articles, culminating for example in the simple conclusion at the end of a short paragraph published in the \emph{Atchison Daily Globe} on June 16, 1888\citesource{June161888}: “The American language for Americans and the English for the English”.


The value of authenticity not only played a role when the focus was on the contact between English and American people, for example through traveling, but also when an inner-American context was foregrounded because, as with several phonological forms, the use of \emph{luggage} is not only indexically linked to English speakers, but also to a group of Americans imitating English speech. There are two articles which are particularly illustrative examples of how this group is characterized and how this characterization is embedded in the article to express a political view. The first example was published in the \emph{Rocky Mountain News} on August 29, 1883\citesource{August291883}, and its main aim is to criticize a “technical and mischievous decision of Judge Nelson in the Circuit court of the United States, at its sitting in Boston, August 22, upon the admission of Chinese laborers to this country”. The decision interprets a prior act of Congress against such an admission by stating that it applies only to Chinese laborers who come directly from China, which means that Chinese immigrants may enter the country if they arrive from the ports of other countries. The author of the article views this very negatively because he finds that the Chinese pose a threat to the American middle and working classes and he constructs these classes as being in opposition to the wealthy classes, who have an interest in cheap Chinese laborers to promote their own wealth, and whose interests are represented by Judge Nelson in the Boston Court. The following paragraph shows how he constructs this class difference and links it to language:

\begin{ipquote}
{[…] because the \textstyleStrong{middle and laboring classes} of this country, the \textbf{intelligent artisans} and the \textbf{inventive mechanics}, are the \textbf{genuine American citizens}, with no aspirations toward British manners or British strong government, the \textbf{mania} which, in common with \textbf{dudeism}, is at present so prominent among the \textstyleStrong{wealthier classes, whose baggage is suddenly changed to “luggage,”} while the Yankee twang of a few years ago, with its nasal independence of harmony and men, has given place to \textstyleStrong{a badly affected English drawl} and an \textbf{assumed indifference} to the everyday affairs of our everyday world. {[...]}

\raggedleft
{[emphasis mine]}\\
}
\end{ipquote}


The adjectives used to describe the middle and working classes are “intelligent” and “inventive” and by describing them as artisans and mechanics he creates an image that rests on physical skills and force combined with intelligence and inventiveness and it is implicitly conveyed that he regards these traits as crucial in nation-building (because the physical aspects connect to the image of “building” a nation, while intelligence and inventiveness are associated with progress). These Americans are marked as genuine and thus as authentic and their interest in nation-building and progress differentiates them from the wealthier classes, which are depicted as oriented towards England and indifferent to American affairs. This is connected to language because the “badly affected English drawl” indicates their lack of authenticity as Americans. The link to the social personae of the “Anglomaniac” and the dude (discussed in detail in \ref{bkm:Ref7775416}) is alluded to by calling their “aspirations toward British manners or British strong government” a “mania” and by using the term “dudeism”, which implies that by the middle of 1883, the characterological figure of the dude has already been firmly established. The lexical items \emph{baggage} and \emph{luggage} are very salient here because they are the only items linked to the repertoire of the British-oriented wealthier classes. Furthermore, the author underlines the lack of authenticity in their language use by claiming that “baggage is suddenly changed to ‘luggage’”, which shows that \emph{luggage} is not evaluated as a natural, long-established form intrinsically connected to American people, but as something unnatural that is put on quickly by some Americans in an opportunistic attempt to signal social characteristics deemed desirable by the wealthier classes but viewed negatively by the author of the article.


A second example of an article characterizing the group of American speakers indexically linked to \emph{luggage} was published ten years later, on September 23, 1893\citesource{September231893}, in the \emph{Milwaukee Sentinel} (Wisconsin) and on September 29, 1893\citesource{September291893}, in the \emph{Morning Oregonian} (Portland, Oregon). It is a comment on an article criticizing the appointment of James J. Van Alen as minister to Italy, published in the \textit{New York World}. The author claims to disagree with the criticism, but his text is so full of irony that it rather reinforces the criticism to a great extent. The main reason put forward against the appointment is that Van Alen did not earn it based on his hard political work and accomplishments but based on a large sum of money that he contributed to the Democratic campaign fund. The statement that “His friends affirm that he ‘bought the office like a gentleman’” extends the criticism to an entire group of people: the social and political elite in the northeast of the United States. The place linked to this elite is Newport (Rhode Island) because this is where they spend the summer in their “million-dollar cottages”, drive around in their “swell carriages”, ride “dock-tailed horses” and play tennis in “stunning tennis suits”. However, the most important signs indexically linked to these people, in the author’s view, are the monocle and the linguistic repertoire. The monocle (or eyeglass) also plays a prominent role in articles characterizing the dude and, not surprisingly, the language is described as being “as near the English as any man not born in England can hope to acquire”. As in the article criticizing the court decision, \emph{baggage} and \emph{luggage} are very salient because the only other forms mentioned are the variants \emph{brasses} and \emph{checks} and \emph{patron} and \emph{president}. Expletives are mentioned generally, as well as a distinct English accent, but no concrete forms are listed, which shows that the readers are expected to have a general idea of the linguistic repertoire linked to rich Newport people. It is important that this group is distinguished from “mere Americans” who speak “vulgar American” English. The adjectives represent the view of the wealthy classes – their “fine contempt” for other social classes – and serve to underline the ironic stance of the author who aims to characterize the Newport elite as snobbish, condescending and not authentically American despite the fact they are born in America. Their orientation towards England is also depicted as detrimental to America because they follow English political ideas, which are in England’s interest rather than in America’s, and because they “draw revenues from America to spend in Europe” and thus harm the economy. In short, Mr. Van Alen is constructed as a prime example of an elitist northeastern American who is not authentic, whose political position is not earned through hard work but bought with money, and whose political ideas and economic and social behavior are harmful to the nation. The elite’s bad influence on ‘normal’ Americans is represented by the trainmen who adapt linguistically to Newport speech, which reveals the prestige that this linguistic repertoire has acquired, but at the same time, the Newport elite and their attempt to be as English as possible is portrayed so negatively that the readers are urged \emph{not} to follow the example of the trainmen and align with their language use. That they only come near to sounding English but do not fully achieve it (the argument being that only people born in England can speak British English) further underlines the lack of authenticity of Newport speakers and their uselessness as a model for other Americans.

\begin{ipquote}
\begin{center}
MR. VAN ALEN’S APPOINTMENT.    
\end{center}
In making war on President Cleveland for \textstyleStrong{the appointment of Mr. Van Alen as minister to Italy}, The New York World exhibits a narrowness such as we might expect in a provincial newspaper. As long as the offices are to be used for the reward of party workers, there is no reason why the rewards should be limited to the vulgar bosses and those employed by the vulgar bosses. While it is proper that the Poles should be recognized, it is equally proper that the real gentlemen who honor America by being born in this country, should share in the spoils. At \textbf{Newport in summer} we see the heights to which \textbf{the most favored American} may reach. We see it in the \textbf{million-dollar cottages}, in the \textbf{swell carriages} and \textbf{dock-tailed horses} and in the \textbf{stunning tennis suits}, but we see it \textbf{most} in the \textbf{monocle} and the \textbf{expletives} and in \textbf{the language} generally, which is \textstyleStrong{as near the English as any man not born in England can hope to acquire}. Even the trainmen who run into Newport have learned to speak of \textstyleStrong{“the brasses for the luggage”} instead of the \textstyleStrong{vulgar American “checks for the baggage.”}

\textstyleStrong{Mr. Van Alen is the most English} of all the people at Newport. He \textstyleStrong{lives most of the time in dear old London}, he \textstyleStrong{has a fine contempt for mere Americans}, and \textstyleStrong{his ideas of American politics are acquired through The London Times}. Last year he spent a few months at Newport, in response to the suggestion of The London Times that the British in America should organize to advance the cause of tariff reform, he drew on his London bankers for money to fit out the Newport Cleveland club, of which he became the patron, as they say in England, or the president as the vulgar Americans say. Mr. Van Alen did more. He agreed with the Hon. William H. Whitney to give \$50,000 to the Democratic campaign fund if Mr. Cleveland, in event of Democratic success, would appoint him to some office abroad where he could maintain his social life and where his relations would be with Europeans rather than Americans. This sum was much greater than that raised by The World for missionary work in the West.

{We see no reason to complain of Mr. Van Alen’s appointment. He has cultivated \textstyleStrong{a close resemblance to His Royal Highness}, the prince of Wales, \textstyleStrong{he speaks with a distinct English accent}, he \textstyleStrong{wears white spats}, and to the army of Americans who \textstyleStrong{draw revenues from America to spend in Europe} he will be a gladness and a joy. His friends affirm that \textstyleStrong{he “bought the office like a gentleman,”} paying the large price with a draft on London without grumbling. He has a clear title to the office, and since his appointment is eminently satisfactory to the gentlemen and ladies at Newport, who are condescending enough to tolerate America for a few months each year, we do not see why The New York World should complain.

\raggedleft
{[emphasis mine]}\\
}
\end{ipquote}

The articles analyzed so far thus indicate that there was a process of discursive negotiation, in which the association of \emph{luggage} with English usage is a reason for some Americans to adopt the form and for other Americans to reject it. This negotiation is a reflection of the struggle between an exonormative and an endonormative orientation and it is illustrated very well by a short paragraph which pointedly comments on the use of \emph{luggage} and \emph{baggage} as well as some other lexical items (\emph{gown} vs. \emph{dress}, \emph{ticket office} vs. \emph{booking office}, \emph{car} vs. \emph{van}, \emph{engineer} vs. \emph{driver}, \emph{fireman} vs. \emph{stoker}). It was published in the \emph{Atchison Daily Globe} on September 15, 1887\citesource{September151887}, and it shows how the positive indexical values of ‘correct’ and ‘proper’ are re-evaluated as ‘un-American’ and ‘not authentic’: By stating that \emph{gown} “is English, and may be more proper than dress”, the author acknowledges the prestige accorded to the form, which is based on an exonormative view in which English forms are regarded as a model. The continuation of the sentence with “but it is English, and that ought to settle the matter as far as every true American is concerned” implies that this exonormative view is rejected in favor of an American norm. As the value ‘proper’ is associated with the English form, the American form must be evaluated positively in a different way, and this is done by presenting authenticity as a goal and linking it to nationality. This link is signaled by the noun phrase \emph{every true American}, which implies that those who do not follow the American norm are not really Americans. Another important opposition is that between “anglo-maniacs” and “sensible Americans”, which relates the use of the different forms to the opposition between sickness and health. Overall, it is striking that the writer of the article protests against a new development and urges its readers to contain it by warning against the boldness of its proponents. It is thus not the case that the American forms are characterized as ‘new’ and promoted to replace ‘old’ English forms, but, quite to the contrary, American forms are presented as already established (as ‘normal’ forms used by ‘sensible’ Americans) and are now under threat by the new forms (used by “anglo-maniacs”), so that they need to be defended.

\begin{ipquote}
{We protest against the word “gown” as applied to a “dress.” “Gown” is English, and may be \textbf{more proper} than “dress,” \textstyleStrong{but it is English}, and that ought to settle the matter as far as every \textbf{true American} is concerned. In this connection we would suggest that the word \textbf{“luggage”} as applied to “baggage” is beginning to be used, and \textstyleStrong{should be frowned down}. The first thing we know the \textbf{anglo-maniacs} will become so bold as to call a ticket office a booking office, a car a van, an engineer a driver, and a fireman a stoker, and then life to \textbf{sensible Americans} will become simply unbearable.

\raggedleft
{[emphasis mine]}\\
}
\end{ipquote}

The view that an established American usage is to be preferred as a norm over an English norm is also presented in an article headed “Good Form in England” and published in the \emph{Atchison Daily Champion} on February 07, 1891\citesource{February071891}, and in the \emph{Rocky Mountain News} on March 01, 1891\citesource{March011891}. The article was originally published in a New York newspaper, the \emph{New York Ledger}, and it is notable that its argumentation for not adopting ‘English’ forms like \emph{luggage} relies more on practical issues than on emotions connected to values like nationality and authenticity. According to the author, the differences between English and American usage are so numerous (“we could enumerate a thousand other peculiarities”) and so full of detail that it is simply too time-consuming and too difficult (“Americans who ape English usages almost always blunder”) to acquire the English forms. In addition, the acquisition of English forms is also portrayed as unnecessary and not worth the effort (the time “could be put to a better purpose”).

\begin{ipquote}
\begin{center}
\textstyleStrong{GOOD FORM IN ENGLAND}\\
\textstyleStrong{Terms Considered Proper Among the English Gentility}    
\end{center}
{To adequately indicate the divergences between the ways of English society and our own would require a volume, but some striking examples may be given in a few paragraphs.

\centering
{[…]}

\textstyleStrong{Americans who ape English usages almost always blunder} in the use of crests. [What follows is a list of differences, among others \textit{baggage} and \textit{luggage}.]}

{[…] We might enumerate \textstyleStrong{a thousand other peculiarities}, but we have cited enough to show that \textstyleStrong{an American citizen should not easily acquire what in England is called “good form” without an expenditure of time that could be put to a better purpose}.”—N. Y. Ledger.

\raggedleft
{[emphasis mine]}\\
}
\end{ipquote}

Taking into consideration all articles containing the search term \emph{luggage} and \emph{baggage} in the database NCNP, it is noticeable that the number of articles containing \emph{explicit} comments on the lexical items are far more numerous than in the set of articles containing the search terms representing phonological forms. This is not surprising given that pronunciation respellings invite humor and are thus likely to be used in humorous articles which convey evaluations of forms in rather implicit ways. However, there are also some humorous articles focusing on the difference between \emph{luggage} and \emph{baggage}. The article “Terms” is a good example in this regard. It was originally published by the \emph{Detroit Tribune} (Ohio) and reprinted in the \emph{Milwaukee Sentinel} (Wisconsin) on July 01, 1894\citesource{July011894}, and it creates salient indexical links between \emph{baggage} and \emph{luggage} and different American regions. It consists of a very short dialogic exchange between an “Omaha Coxeyite” and a “Boston Coxeyite”, in which the former asks “Any baggage?” and the latter replies “If you mean luggage, no”. The shortness of the dialogue draws a maximum of attention to the lexical items \emph{luggage} and \emph{baggage} and to the different figures using them. Coxeyites were a group of people who were part of a labor movement led by Jacob Coxey and called “Coxey’s Army” in newspapers \citep[327]{Barber2007}. They demanded that the federal government create jobs to counter the massive unemployment caused by an economic crisis, and in order to draw attention to their demands they organized a protest march from Ohio to Washington in 1894. Around 300 supporters left from Ohio, but many others traveled to Washington from different places, so that there were 1,000 people marching down Pennsylvania Avenue on May 1, 1894 \citep[327]{Barber2007}. That Coxeyites were rather poor people is emphasized in the dialogue by pointing out that they did not carry any baggage. Their similarity with regard to social characteristics reinforces the focus on the different regions that become indexically linked to \emph{baggage} and \emph{luggage}. Omaha, the biggest city in Nebraska, is representative of the western parts of the United States, while Boston is representative of the northeastern parts. The humor of the dialogue builds on this contrast between the Bostonians’ social position (being a poor Coxeyite) and his use of \emph{luggage}, a linguistic form associated with the wealthy northeastern elite. The article thus humorously criticizes the prestige of the variant \emph{luggage} in the northeast that extends even to poor Coxeyites. Doing so, the article mainly serves to reinforce the contrast between the northeast and the west that has been found in articles containing \emph{dawnce}, \emph{deah} AND \emph{fellah} and \textsc{twousers} as well.

\begin{ipquote}
\textbf{Terms.}\\   
Detroit Tribune: Omaha Coxeyite—Any baggage?\\
Boston Coxeyite—If you mean luggage, no.
\end{ipquote}

Against the overwhelming majority of positive evaluations of \emph{baggage} there are only three articles in which the form is evaluated negatively. One of these article presents a Supreme Court Decision, which was published by the \emph{Daily Evening Bulletin} (San Francisco, California) on July 20, 1886\citesource{July201886}. It states that “The terms baggage and luggage signify one and the same thing. The former is the form in general use in the United States, while in England the latter prevails”. This statement suggests a neutral perspective, which links the two forms to the two different nations. However, the next statement is: “Our Code has adopted the English expression”. Choosing the British form over the American form signals to the reader that the British form is more appropriate, at least in such an official and institutional context. The American form is in turn implicitly evaluated as inappropriate and consequently as inferior to the British form.

A more explicitly negative evaluation is expressed in the article “‘Luggage,’ not baggage”, published in the \emph{Los Angeles Daily Times} (California) on September 19, 1882\citesource{September191882}. As in the first example, it is acknowledged that \emph{luggage} is the English term and \emph{baggage} the American term, but the latter is judged to be “inadequate, incomprehensive, unsatisfying and limited” and to be used with “undiscriminating and imbecile inappropriateness”. The words used to evaluate \emph{baggage} show that the term is evaluated negatively mainly on linguistic (and not on social) grounds. It is seen as not adequate to designate that which it should designate and thus as limited. However, the adjective \emph{imbecile} also links the linguistic deficiency constructed in the article to a human characteristic, based on which users of the form can be linked to stupidity. The English form is thus linked to being superior because it is “objectively” more appropriate and used by speakers who are smart and educated enough to know that.

\begin{ipquote}
\begin{center}
\textstyleStrong{ “LUGGAGE,” NOT BAGGAGE.}\\
\textstyleStrong{Graphic Account of a Desperate Struggle for it on the San Francisco Wharf.}
\end{center}
{{[Following is an extract from a private letter received in the TIMES office from a personal friend recently arrived at San Francisco from the Hawaiian Islands. It will be read with deep feeling by the average sea traveler:]}

\centering
{[…]}

I found, for the first time, the vast and comprehensive appropriateness of the English word \textstyleStrong{“luggage,”} as compared with our \textstyleStrong{inadequate}, \textstyleStrong{incomprehensive}, unsatisfying and \textstyleStrong{limited} term \textstyleStrong{“baggage,”}} {which we apply with \textstyleStrong{undiscriminating} and \textstyleStrong{imbecile inappropriateness} to things that never could, would or should be “bagged,” “corraled,” or otherwise gathered together, in any other way than by infinite lugging.” {[…]}   

\raggedleft
{[emphasis mine]}\\
}
\end{ipquote}

In a similar vein, lexical differences between England and America are discussed “objectively” in the article “Springs of English”, published in the \emph{Daily Inter Ocean} (Chicago, Illinois) on March 20, 1892\citesource{MetcalfMarch201892}. The author calls for a separation of the judgement of lexical forms and nationality. He rejects any focus on correctness or incorrectness but suggests concentrating on the “pure value of the word”. Linguistic criteria are thus also regarded more highly than social criteria and the “pure value” can be determined by finding out the way which “represents the idea” best. This in turn is determined based on linguistic economy: “the least friction of letters [which] convey the greatest amount of thought”. However, in the case of \emph{baggage} and \emph{luggage}, the proposition is to “sacrifice economy to truth” and to “give to it the comprehensive Latin name ‘impediments’”, so that not only \emph{baggage} but also \emph{luggage} is evaluated negatively as inferior to a third term.

\begin{ipquote}
\begin{center}
\textstyleStrong{SPRINGS OF ENGLISH.}\\
One in Old England, the Other in America.\\
Anglicisms, Americanisms.\\
Curious Differences in Words and Expressions.\\
Differentiation of a Divided People—Habits of Speaking and Eating Contrasted.\\
\textsc{Old Customs Preserved}.
\end{center}

London, March 12.—\textit{Special Correspondence}—

{TRADITION and habit are such linguistic tyrants, that it is not easy to place oneself in a perfectly unprejudiced frame of mind to judge between words of like significance in the same language. From the nature of the case it is impossible to find for the English tongue an unbiased umpire, yet both in England and America there must exist minds of a sufficiently judicial temper \textbf{to separate} themselves for the moment \textbf{from their nationality}, and, presuming either to be equally correct, decide upon the \textbf{pure value of the word}; which the better represents the idea and which will, with the least friction of letters, convey the greatest amount of thought; for in the \textbf{economy of language} is its greatest strength.

\centering
{[…]}

The “baggage car” is a “luggage van,” and, \textstyleStrong{of course, “baggage” is “luggage:”}} {but why—as it is neither universally “bagged” nor “lugged”—not \textbf{sacrifice economy to truth} and give to it the comprehensive Latin name “impediments?”

\raggedleft
{[emphasis mine]}\\
}
\end{ipquote}

To conclude the qualitative analysis of \emph{luggage} and \emph{baggage}, it is striking how positively \emph{baggage} is evaluated as the American form that indexes not only American nationality but also authenticity. It is strongly connected to discourses on non-rhoticity, labiodental /r/ and the back \textsc{bath} vowel because like \emph{luggage} those forms index English speakers or American speakers who are not genuine and sincere and who often fail in their ridiculous attempt to imitate English speech. The oppositions between different classes and regions are found for \emph{luggage} as well: The wealthy upper classes in the northeast of the country use \emph{luggage} because they suffer from ‘anglomania’, while the sensible middle and working classes in the north, midwest and west of the country use \emph{baggage}. Whether the positive evaluation of \emph{baggage} can also be found for \emph{pants}, the American variant of \emph{trousers}, will be the central question guiding the qualitative analysis in the next section.

\subsection{\emph{pants} vs. \emph{trousers}: indexical values and social personae}
\label{bkm:Ref9413051}\hypertarget{Toc63021244}{}
The search for articles containing \emph{pants} within ten words of \emph{trousers} revealed three broad categories of articles, according to which the following qualitative analysis will be structured. They are largely equivalent to those categories established for \emph{baggage} and \emph{luggage:} First, there are articles in which the two forms are used synonymously, which means that no indexical links between the lexical items and different social values are created. Secondly, there are articles in which those links are created and in which \emph{trousers} is evaluated positively as a form that should be preferred to the form \emph{pants}. The third category comprises articles in which the links are also created but in which \emph{pants} is the term that is determined as preferable. The last part of the analysis will focus on one particular text type, namely advertisements, because they provide interesting insights into how the different social meanings linked to the lexical items have an effect on the use of the terms in a context in which language is used to persuade people to buy particular items from particular stores and in which language use has thus direct economic consequences.


The first article in the collection of articles obtained by searching for \emph{pants} and \emph{trousers} occurring within a range of ten words was published in the \emph{Fayetteville Observer} (North Carolina) on May 16, 1861\citesource{May161861}, a month after the outbreak of the Civil War. The article is addressed to volunteer soldiers and provides a set of instructions on proper dress, hygiene, and other aspects deemed important to ensure the “success and efficiency of any army”. By referring to the leg garments as “Pants or Trousers”, the author marks the terms as alternatives, and he does not give any indication as to what motivates the use of the different forms. The fact that \emph{pants} is named first, before “or trousers” is added, and the fact that at the end of the sentence only \emph{pants} is used (“outside of pants”) creates the impression that \emph{pants} is less marked (and possibly more common) than \emph{trousers}. Nevertheless, the absence of any social differences linked to the alternative lexical items makes the article a clear exemplar of the first category.

\begin{ipquote}
\begin{center}
FOR VOLUNTEERS.    
\end{center}
The Duke of Wellington occupied himself a good deal with details of very much the same character as those which we propose to speak of in a few brief papers. He knew how much they had to do with the success and efficiency of any army; and, while in command in Portugal and Spain, found time to discuss in his correspondence the size of “\textit{camp kettles}”

{1. […] \textbf{Pants or Trousers} should be of same color and material with Blouse, cut \textit{full} around hips and knees; and when practicable, leather gaiters should be provided \textit{a la Zouave}, to be worn outside of pants. […]

\raggedleft
{[emphasis mine]}\\
}
\end{ipquote}

The second example of an article belonging to the first category was published in the \emph{Atchison Daily Champion} (Kansas) on May 26, 1888\citesource{May261888}, and therefore at a time when discourses on language in newspaper articles were highly prominent in America. Despite the increasing interest in language use that is apparent in numerous articles of the 1880s analyzed above, in this article several variants are used without any reason being provided for the use of the different forms. The topic of the article is the problem of pants bagging at the knees and it thus requires a repeated reference to the garment in question. The author uses \emph{pantaloons} first (“how to prevent pantaloons from bagging at the knees”), followed by \emph{trousers} (“Your trousers will bag”) and later \emph{pants} (“A great many men pull their pants up on their knees”). \emph{Trousers} is the only variant used a second time (“The smallest part of the trousers is that around the calves”.). The article thus creates the impression that the author, who is identified as a tailor at the end of the article and who is thus professionally involved in the subject he discusses, is aware of the existence of different terms but that he deliberately avoids choosing one over the others. That he uses \emph{trousers} twice could be interpreted as a slight preference for \emph{trousers}, but this preference is not at all salient.

\begin{ipquote}
\begin{center}
\textstyleStrong{Bagging at the Knees.}    
\end{center}
{A great many inquiries are made as to how to prevent \textbf{pantaloons} from bagging at the knees. There is only one answer to these, it can’t be done. Your \textbf{trousers} will bag, and you can’t help it. The bagging can be lessened by frequent pressings and taking good care of them, but as long as men bend their knees in walking their pants will bag. The skin would also, if it didn’t settle back. A great many men pull their \textbf{pants} up on their knees when seated to prevent their bulging. This is very foolish. The smallest part of the \textbf{trousers} is that around the calves of the legs, and, of course, by pulling them up and bending the leg a greater strain is brought to bear on the cloth so it would not stretch so much, but so far no tailor has succeeded in hardly lessening the cause of the complaint. The tailor who do{\kern0pt}es make the discovery will at the same time make a fortune.—Tailor in Globe-Democrat

\raggedleft
{[emphasis mine]}\\
}
\end{ipquote}

That tailors are regarded as experts not only on the clothing they create but also on the linguistic terms that are used to designate such items of clothing is shown in the following article, which, in contrast to the article above, belongs to the second category because the tailor explicitly argues in favor of using the word \emph{trousers}. The topic of the article is the same as the one above, the problem of bagging pants, and it was published only one year later, on February 23, 1889\citesource{February231889}, in \emph{The Wisconsin State Register} (Portage, Wisconsin). The article had originally been published in the \emph{New York Mail and Express}, which is important because it indicates a different regional context in comparison to the article above: The tailor’s opinion in the 1888 article was first expressed in the \emph{Globe-Democrat}, which is a St. Louis newspaper, while in this article a reporter interviews a “fashionable tailor up town”, which indicates that the tailor is from New York. This New York tailor is quoted as saying that “There is but one way to prevent bagging, and that is to pull the trousers up a few inches over the knees whenever you sit down”. Both the adjective \emph{fashionable} used to describe the tailor and the self-confident manner in which he provides his expertise establish the tailor as a person of authority. When asked about his preference for the word \emph{trousers}, he argues that it is “the proper word”, while \emph{pantaloons} is not proper because it “referred originally to the clown in the pantomime” and \emph{pants} is not proper because it is “a very inelegant and inadequate term”. The evaluation is thus mainly based on associations to clownishness on the one hand and elegance on the other hand. The association with clownishness is motivated by drawing on the etymology of the word \emph{pantaloons}, but the elegance of the term \emph{trousers} is not motivated. However, by describing the “bifurcated garment” as “noble”, he connects the elegance of the items to the (suggested) elegance of the word \emph{trousers} and thus creates a reason for using his preferred variant. In general, his evaluation of the terms is rather presented as a matter of fact, which does not require much argumentation. Again, this is possible because he portrays himself as a professional whose opinion can be trusted – a position that is underlined by his comment “You see, I have studied up so as to be posted in every branch of my profession”. His last statement “A tailor may use a goose and not emulate one” plays on the two meanings of \emph{goose} – ‘a tailor’s iron’ and ‘a simpleton’ – and underlines that he wants to be taken seriously as a professional.

\begin{ipquote}
{\centering
WHY TROUSERS BAG.\\
A Fashionable Tailor Tells How to Prevent it Without Trouble.\\
{[...]}\\}

{“Why do I use the word \textbf{trousers}? I think it the \textbf{proper word}. \textbf{Pantaloons} referred originally to the \textbf{clown in the pantomime}, and \textbf{pants} is a \textbf{very inelegant and inadequate} term with which to designate the noble bifurcated garment. As to breeches, it is worse than either pants or pantaloons. You see, I have studied up so as to be posted in every branch of my profession. A tailor may use a goose and not emulate one.” —New York Mail and Express

\raggedleft
{[emphasis mine]}\\
}
\end{ipquote}

The preference of New York tailors for \emph{trousers} also becomes visible in the following statement, published in the \emph{Yenowine’s News} (Milwaukee, Wisconsin) on February 23, 1890\citesource{February231890}, as part of a collection of entertaining short paragraphs in a section labeled “By-the-bye”: “A New York tailor advertises ‘Pants for gents, and trousers for gentlemen.’” The advertisement thus links the difference between \emph{pants} and \emph{trousers} to the difference between \emph{gents} and \emph{gentlemen}. For \emph{pants} and \emph{gents} this is done on a linguistic level: The link between the forms rests on the one hand on their phonological similarity and on the other hand on them being the result of the same morphological process involving shortening and suffixation with a plural -\emph{s}. But social aspects played a role as well: The clipping of \emph{gentlemen} to \emph{gents} was often negatively viewed as slang and it was mentioned together with the clipping of \emph{pantaloons} to \emph{pants} as the following article shows, which was published on January 23, 1856\citesource{January231856}, in the \textit{Charleston Mercury} (South Carolina).\footnote{The article was quite popular: The search in the database NCNP yields five tokens, of which two articles were also published in the south (in the \textit{Daily South Carolinian} in South Carolina, and in the \textit{Fayetteville Observer} in North Carolina), and two articles were published in the east (in the \textit{Daily National Intelligencer} in Washington, D.C., and in the \textit{Bangor Daily Whig \& Courier} in Maine).} In this article, which is entitled “Avoid slang words”, the “author of the behavior book” is cited as stating that “There is no wit [...] in a lady [...] in calling pantaloons ‘pants’ or gentlemen ‘gents’ [...]”. The shortenings are rather labeled “slang words” and explicitly evaluated as “detestable”, which is why they have to be avoided if a lady would like to belong to the “best society”.\footnote{It is notable, however, that the lady is \emph{not} advised to use \emph{trousers} instead of \emph{pantaloons}. This suggests that the latter term is viewed as normal and acceptable in 1856, at least in South Carolina, where the article was published.} This evaluation provides support for interpreting the New York tailor’s advertisement as highlighting the prestige of the word \emph{trousers}. The creation of indexical links works on multiple levels here as the quality of the linguistic terms is linked to the quality of the items of clothing and to the quality of the people wearing the items of clothing. However, the present context of the advertisement also needs to be taken into consideration: Its appearance in a section with short humorous paragraphs and the introduction of the slogan by “a New York tailor advertises” (emphasizing the place New York) implicitly conveys that the New Yorker’s evaluation of \emph{pants} and \emph{trousers} is funny and different from evaluations found in other places, such as Wisconsin, where this article was published. The indexical links created by the advertisement are thus revalorized by ridiculing the New Yorkers’ concern with being gentlemen and using specific terms to signal this status. This shows that the line between the three categories is blurry – an article can contain links to negative social values and present a case of a positive evaluation at the same time simply because of the way it recontextualizes the information in the article and because of the context it is published in.

A similar case of revalorization can be found in an article in which the positive evaluation of \emph{trousers} in New York is also described explicitly. It was published in the \emph{Daily Picayune} (New Orleans, Louisiana) on May 16, 1896\citesource{May161896}. The title, “No Pants”, draws attention to the form in a humorous way because at this point it is not clear that the subject of the article is the lexical form \emph{pants} and not the object it designates. This creates the potential of surprise to readers who might expect that the article is about pants not being worn by some people. The author of the article then describes the social values linked to the forms not only in New York, but also in Boston: As in the articles above, elegance and correctness are contrasted with vulgarity, and a link between the quality of the item of clothing and the term used to designate it is established (\emph{pants} being “hand-me-down articles in a ready-made clothing store”, \emph{trousers} being “the creation of swell tailors”). The homophony of the noun \emph{pants} and the third-person singular form of the verb \emph{pant} is used to link the use of the term \emph{pants} to dogs and thus implicitly to animal-like behavior, in contrast to \emph{trousers}, which is linked to men. In addition, the author emphasizes the importance of social elites and authoritative figures in the determination of the social status of the term. The “elegant and correct Bostonian” is representative of the Boston social elite, while in New York a specific person is named: Theodore “Teddy” Roosevelt, who was the New York City Police Commissioner (the head of the New York City Police Department) at the time. Mentioning Theodore Roosevelt also serves to underline the mechanisms behind the spread of the usage of a lexical item. The political power exercised by him is seen as having direct consequences on language use – not just by people under his direct command (policemen) but also on the general public (“What Teddy Roosevelt says goes in New York”). The fact that this description was published in a New Orleans newspaper has an important effect on the interpretation of the indexical links described: By focusing on the evaluation of \emph{trousers} and \emph{pants} in Boston and New York without relating it to evaluations found in other places, the two cities are implicitly marked as special and contrasted with the rest of the country. Furthermore, the last line in particular reveals an ironic and humorous undertone (“from this on ‘pants’ will be the exclusive property of dogs, who can have them creased or not as the weather permits”). The author here alludes to the importance attached to fashion by upper-class men in Boston and New York and ridicules it by ironically suggesting that in these two cities even dogs wear pants with creases. This ridicule creates intertextual links to articles about the dude, for example to the one discussed above in which Cholly tells his friend that he cannot drown himself because this would take the creases out of his “twousers”. It therefore becomes clear that the author views the situation in Boston and New York rather negatively. As in the article quoting the advertisement by the New York tailor, the links between \emph{pants} and negative social values (inelegant, incorrect, animal-like and thus uncivilized) are revalorized as being specific to New York and Boston, which marks these places negatively as being dominated by a social and political elite that exerts its power even on the language that people use.

\begin{ipquote}
\begin{center}
No Pants.    
\end{center}
{The social status of trousers was settled in Boston a long time ago. The \textstyleStrong{elegant and correct Bostonian } pointed out that \textbf{a dog pants}, but \textbf{a man wears trousers}. Others held that these articles of masculine attire were \textbf{“pants”} when they were the hand-me-down articles in \textbf{a ready-made clothing store}; but \textbf{the creation of a swell tailor} were \textbf{“trousers.”} In \textbf{New York}, Mr. Roosevelt has just made an important decision on this subject. He has ordered that the \textbf{vulgar word “pants,”} referring to bifurcated garments, shall not occur in any report made to the police department. The police are to officially speak of nether clothes as “trousers,” or else be silent on the subject. What Teddy Roosevelt says go{\kern0pt}es in New York, and from this on “pants” will be the exclusive property of dogs, who can have them creased or not as the weather permits.

\raggedleft
{[emphasis mine]}\\
}
\end{ipquote}

A different way of conveying negative values indexed by \emph{pants} is used in the following article, which was published in the \emph{Raleigh Register} (North Carolina) on July 29, 1885\citesource{July291885}. It is a short paragraph containing a brief narrative telling the story of “the maiden” who decides against marrying Fitznoodle because his use of \emph{pants} reveals his uneducatedness and ignorance (““I’ll wed no man so ignorant,” she said; ‘he uses “pants” for “trousers”!’”).  The last comment by the narrator, “And she was right”, shows that the negative evaluation is emphasized and that, in contrast to the two articles above, no revalorization can be observed. It is interesting that in this case no indication as to the place and context of the conversation is given, which implies that the negative evaluation of \emph{pants} is not restricted to a particular region, but rather is a universal phenomenon. This shows that while indexical links are revalorized in some articles, they are not in others, indicating that competing evaluations circulate in newspaper discourses on language.

\begin{ipquote}
\begin{center}
\textstyleStrong{ITEMS ABOUT WOMEN}\\
Grave and Gay—Lively and Severe\\
{[…]}    
\end{center}
“His pants alarm me so,” the maiden said (referring to her poodle, which an unruly cow a chase had led) as she walked with Fitznoodle. “Are they too tight?” the untaught Fitz replied, his indignation rising. (He thought the maiden’s mind he occupied, and her soliloquizing). Sharply she turned, and in her pretty head her eyes glowed like a mouser’s; “I’ll wed no man so ignorant,” she said; “he uses ‘pants’ for ‘trousers’!” And she was right.—\textit{Chicago Tribune}.
\end{ipquote}

It is striking that in none of the above articles evaluating \emph{pants} negatively and \emph{trousers} positively was a reference to British English usage made. In the case of \emph{luggage}, the argument could be found that \emph{luggage} is more proper because it is the English term – but this or a similar point was not made to argue for the use of \emph{trousers}, despite the fact that the association between \emph{trousers} and British English speech can be found in several articles. The article “Pants” (September 30, 1875\citesource{September301875}), which I already analyzed in \sectref{bkm:Ref7775416}, begins with the statement “Pants—or, as they call them in England, trousers—are not a subject which the average citizen cares, except in the presence of his tailor, to discuss”. However, instead of evaluating \emph{trousers} positively as ‘proper English’, the characterization of English people in the article is very negative, so that I assign this article to the third category of articles analyzed in this section: The use of \emph{trousers} is constructed as \emph{not} desirable because it is English and because English people are no role models. I have already described the ridiculous figure of the “young Oxford swell” and the many different kind of “twousers” in his possession. This shows a connection between the lexical and the phonological level because it is not only the lexical item \emph{trousers} but also the realization of /r/ as a labiodental approximant that is linked to this figure. In addition, there are more negative characteristics of English swells mentioned in the article:

\begin{ipquote}
But touching pants, we read with interest, and congratulate the swells thereon, that among the most pronounced authorities in London checks are once more the rage. The check is so large that it takes two men to carry it, or two pairs of trousers to show off the pattern. The swell puts on one pair in the morning, and another in the afternoon. He might with advantage pin upon the first, “to be continued in our next.” Who pays the cheques we do not presume to say. As the boy is father to the man, perhaps he pays the tailor himself. But then, as every swell knows, it is such a deucedly low thing to pay a tailor at all, perhaps he do{\kern0pt}esn’t. May he not hand in his checks, and leave them a legacy to his creditors. But that is a subject beyond our province. We started in on pants; the more we think of their career and their cheques, the more we are inexpressibly reminded of Adam and original sin.
\end{ipquote}


The author thus not only ridicules the fashion of large check patterns, but he also criticizes English swells for often not paying their tailors. This habit is also commented on in a later article, in which it is probably not only seen as applying to English swells but also to American dudes. It was published on October 1, 1892\citesource{October11892}, and contains just a short comment on the difference between \emph{pants} and \emph{trousers}:


\begin{ipquote}
THE PANTS QUESTION.\\
The difference ’twixt pants and trousers is\\
(I think no one has said it)\\
That pants are always sold for cash\\
And trousers bought on credit.\\
—Indianapolis Journal.
\end{ipquote}


It is not made explicit who buys the trousers on credit and the pants with cash, but as the 1875 article indicates, the characteristic of the swells to live on credit and not pay for the fashionable clothes and other items important for their lifestyle is already circulating in newspaper discourses. Using this background knowledge, the readers can infer the respective social personae. Buying something for cash, on the other hand, implies not only having actual financial resources, in contrast to the swells’ possibly just pretended wealth, but also honesty and integrity that is associated with hard work in contrast to the swells’ careless attitude and leisurely lifestyle.


An article in which the author expresses an explicitly positive attitude towards the word \emph{pants} was so popular that it was published in several newspapers across the country. The original was published in the \emph{Milwaukee Sentinel} on October 9, 1889\citesource{October91889}, and four reprints can be found in the NCNP alone, in the \emph{Wisconsin State Register} on November 16, 1889\citesource{November161889}, and on November 23, 1889\citesource{November231889}, in the \emph{Atchison Daily Champion} (Kansas) on November 19, 1889\citesource{November191889}, and in the \emph{Bismarck Daily Tribune} (North Dakota) on January 4, 1890\citesource{January41890}. All of these newspapers are (mid)western or northern ones and the first sentence shows the opposition between them and the New York newspaper \emph{The Sun}. The author reacts to the “regular quarterly attack on the word pants” in the New York newspaper by insisting that the “average American” has adopted \emph{pants} and “regards the word trousers as an English affectation”. The article thus connects to the discourses on \emph{luggage} and \emph{baggage}, in which the same argument can be found. In this article, however, the author extends this argument by linking the linguistic usage question to the American political system and its underlying ideology: By finding that \emph{pants} is “a good democratic term” because it has been adopted by a majority and is not only used by an elitist minority, the author appeals to the pride Americans take in their democracy. Nevertheless, the author also counters the linguistic arguments put forward by the New York newspapers. First of all, using the etymology of the words to argue that \emph{pantaloons} is the best term to designate the object because it originally referred to items of clothing covering entire legs and feet, whereas \emph{trousers} used to designate only those clothing items covering the hip and the thigh. Secondly, the author advances an argument for the shortening of the word \emph{pantaloons} which is based on an iconic relationship between the word and the object: “As the modern leg coverings are pantaloons cut short, why shouldn’t we cut the word short and call it pants?” Overall, this article thus suggests that discourses favoring \emph{trousers} are primarily advanced by New York newspapers and that they are countered by midwestern and northern newspapers by establishing the superiority of the “average American” over the northeastern British-oriented elite, a superiority mainly based on sheer numbers, which is a decisive factor in a democratic society, but also on good linguistic arguments which can be read and interpreted as establishing the position that the ‘average’ midwestern/northern Americans are no less educated than northeastern Americans.

\begin{ipquote}
\begin{center}
\textstyleStrong{“Pants” It Must Be.}    
\end{center}
The \textstyleStrong{regular quarterly attack on the word pants} appears on time in \textstyleStrong{The New York Sun. The war is useless.} The American people have adopted it, and protests, ridicule and arguments are all wasted. Whether we like it or not, pants is here to stay. The \textbf{average American} regards the word \textbf{trousers} as an \textbf{English affectation}, and is no more disposed to adopt it than the word waistcoat for vest or topcoat for overcoat. Since the word pants will stick in the face of all opposition, it is sensible to make the best of it. \textstyleStrong{And there is nothing very bad about it.}. Both The Sun and The Herald declare that pants are not pants, but trousers; but it is also true that trousers are not trousers, but breeches; and that breeches are not anything worn off the stage. Originally trousers were applied to breeches worn by pages—a hip and thigh covering. Pantaloons resemble the leg coverings of today more than trousers or breeches—for pantaloons cover the entire legs and feet. \textstyleStrong{As the modern leg coverings are pantaloons cut short, why shouldn’t we cut the word short and call it pants?} Besides, we have some justification in this in the word pantalet, derived from the word pantaloon. The pantalet, as may be seen in old prints, was a leg-covering for women and children which reached to the sho{\kern0pt}e-top and resembled the modern made leg-covering more than trousers as originally known. The word trousers comes from the French trousse, a bundle—or a bunch about the hips.

{Let us accept \textbf{pants} as \textbf{a good democratic term}, since there is no way to get rid of it.—Milwaukee Sentinel

\raggedleft
{[emphasis mine]}\\
}
\end{ipquote}


The prestige accorded to the word \emph{trousers} is also ridiculed in several short humorous paragraphs. An early example was published on October 17, 1873\citesource{October171873}, in the \emph{Bangor Daily Whig \& Courier} (Maine):


\begin{ipquote}
A Vermont paper says, “Here we are in our new trousers.” It “pants” for more subscribers, probably.
\end{ipquote}


As in some of the articles discussed above, the pun is created based on the homophony of the noun and the conjugated verb \emph{pants}. The implicit criticism conveyed by the statement is that the Vermont newspaper does not use the word \emph{trousers} because they regard it as a better term but solely because they hope to benefit financially by appealing to rich people who prefer \emph{trousers} over \emph{pants}. The use of \emph{trousers} is thus linked to a lack of integrity and authenticity.


Ten years later, on December 11, 1883\citesource{December111883}, a short paragraph consisting of a dialogue between a “youthful Bostonian” and his “Mamma” was published in \emph{The Wisconsin State Register} (Portage, Wisconsin). It ridicules the influence on language use exerted by the social and intellectual elite in Boston: The boy asks his mother whether he may call his trousers \emph{pants} while Mr. Holmes and Mr. Lowell are absent from the city, a question which presents \emph{pants} as the normal and preferred term and \emph{trousers} as the term which is only used because influential men like Oliver Wendell Holmes and James Russell Lowell advocate its use. By employing the figure of a young boy, this contrast between authentic language use and the insincere language use resulting from discourses dominated by an elite is highlighted because it can be expected that children are less susceptible to social pressure – what matters most to them is their mother’s permission. On the other hand, it also illustrates how far-reaching the elite’s influence is because it causes mothers to forbid their children to use the word \emph{pants}. The relationship between the boy and his mother can also be read as mirroring the relationship between ‘average’ Bostonians and the Boston social elite, the first group being infantile and looking to the second group for instructions on how to speak and behave. The article can thus be interpreted as an implicit call to the reader to act as an emancipated adult and choose \emph{pants} over \emph{trousers}.

\begin{ipquote}
Youthful Bostonian—Mamma, aren’t Mr. Holmes and Mr. Lowell both absent from the city? Mamma—I believe they are, dear. Y. B.—Well, can’t I call my trousers “pants,” just while they’re away?—\textit{Life}
\end{ipquote}

As can be expected, the discourses on \emph{pants} and \emph{trousers} also involve articles making reference to the “Anglomaniac” and the dude figure. A good example is the article “As an Englishman”, published in 1887, which I have already discussed in \sectref{bkm:Ref7775416}. The description of the young American trying to appear as English as possible also includes his avoidance of the word \emph{pants} because he considers it a “vulgar blunder”. Furthermore, the article also illustrates the connection between language use and other performable signs by not only describing the “young American Englishman” as using the word \emph{trousers} but also as following the English fashion of wearing trousers turned up. The “Anglicization” process is reflected in his style of wearing his trousers. If he wears them turned up in the rain, there are still traces of Americanness left in him, but if he wears them turned up in the sunshine, he has become a “thorough Englishman”.

\begin{ipquote}
{He would almost die for shame should he make such a vulgar blunder as to say “pants.” The word he uses is “trousers,” “breeches,” or “bags.” He will tell you confidentially, “I pwefew to say bags; it’s awfully English; the best fellows all say it, you know.” In this way do{\kern0pt}es the young citizen proceed to Anglicize himself.

\centering
{[...]}\\
\textsc{With trousers turned up.}

I know a young American Englishman who runs to the window every morning on rising to see if he will have an opportunity of turning his trousers legs up. If the day looks fine he comes from the window with a disappointed air and says “Too bad, by Jove. It isn’t going to wain afteh all.”} {Once he has become a thorough Englishman, however, he will walk through Broadway the sunniest day in the year with his trousers turned up.

\centering
{[...]}\\
}
\end{ipquote}


The fashion of wearing trousers turned up is also illustrated in cartoons, as shown in the two following examples.\footnote{The cartoons discussed in this section have been obtained by searching the NCNP for articles containing the name Cholly and illustrations.} In the cartoon “How Cholly Got Left” (see \figref{fig:key:43}), Cholly is labeled a “Masher”, which, according to the \emph{OED} (\citeyear{masher}), is a slang term first attested in 1872 that could not only be used in the United States for a “a fashionable young man” but also for “a womanizer; a man who makes indecent sexual advances towards women, esp. in public places.” The reader is thus invited to interpret the dude's rather harmless offer to the young lady as an indecent advance and to laugh about his failure when the lady takes the umbrella but leaves him standing in the rain. This contributes to the picture of the dude as being unattractive and unsuccessful in his attempts to impress women. His turned-up pants are clearly visible because the patterned outside is contrasted with the plain inside material. The name Cholly, his behavior, his pants and the eyeglass thus clearly signal his belonging to the group of dudes.


\begin{figure}
\includegraphics[width=.4\textwidth]{figures/Paulsen-img43.png}
\caption{
A cartoon illustrating the dude wearing his trousers turned up in the rain, published in the \emph{Atchison Daily Champion} (Atchison, Kansas) on December 28, 1890\citesource{December281890}, retrieved from Nineteenth-Century U.S. Newspapers
}
\label{fig:key:43}
\end{figure}

The second example depicts Cholly wearing his pants turned up even in the sunshine (see \figref{fig:key:44}). In this case, attention is drawn to the bottom part of his legs by coloring his socks black, while his shoes and pants remain white. His fancy outer appearance is generally ridiculed in the cartoon: The caption contains a dialogue between him and “Miss Soldier Girl” in which they discuss his uselessness as a soldier, which culminates in the girl’s suggestion that he could function as a wigwag signal – a suggestion which alludes to his conspicuous clothing that is likely to be noticed as well as to his incapability of carrying out any task that is more complicated than going back and forth repeatedly and automatically.


\begin{figure}
\includegraphics[width=.6\textwidth]{figures/Paulsen-img44.png}
\caption{
A cartoon illustrating the dude wearing his trousers turned up in the sunshine, published in the \emph{Milwaukee Journal} (Milwaukee, Wisconsin) on September 22, 1898\citesource{September221898}, retrieved from Nineteenth-Century U.S. Newspapers
}
\label{fig:key:44}
\end{figure}

The link between the dude’s use of the lexical item \emph{trousers}, in combination with the labiodental realization of /r/ (indicated by \emph{twousers}), and his huge concern for fashion is also established in the humorous dialogue between Percy Paddleford and Daniel Maginnis (published on April 20, 1895\citesource{April201895}), which I have already discussed in \sectref{bkm:Ref7775416} because it is part of the collection of articles containing the search term \textsc{twousers}. They talk about Hawold Montmowenci who is, according to Percy, a martyr “of the modern type” because he gave up drinking and eating (and smoking) in order to be able to keep up with changes in fashion and who ultimately died of starvation. One of these changes in fashion related to trousers, as the following extract shows:


\begin{ipquote}
“Hewoic cweature!” said the pwopper young man. “But his twials and twibulations were not yet ovah. \textstyleStrong{Just then the fashion changed fwom tight twousers to loose twousers}.”

“Put him in a hole again?” asked the horse doctor?

{“\textstyleStrong{Yaas. But he was made of the twue stuff. He had a gweat and hewoic soul, and he gave up his foah cwackers a day and bought him a pair of loose twousers.}”

\raggedleft
{[emphasis mine]}\\
}
\end{ipquote}


This shows that the dude not only had to take the bottom part of his pants and the right pattern into consideration, but also the fit of the pants. That his concern for fashion poses a threat to his life is of course a great exaggeration, but it can also be found in other articles, as in the following cartoon, published in the \emph{St. Louis Globe-Democrat} on February 5, 1887\citesource{February51887}, and taken from the magazine \emph{Life} (see \figref{fig:key:45}). The cartoonist expresses a warning that the dude “could be ‘carried away’ by the fashion” by depicting how his cane gradually becomes so big that it literally carries him away. The heading “Verbum Sap” is a phrase which is used, according to the \emph{OED} (\citeyear{verbumsap}), “in place of making or concluding a full statement or explanation of something”. The drawings are thus presented as providing a clear conclusion to the matter of the dude's over-emphasis on fashion, which emphasizes its negative consequences. Even though ‘fashion’ comprises mainly items of clothing and accessories here, linguistic forms like \emph{trousers} are connected to the fashion as well, as the analyses above have shown.


\begin{figure}
\includegraphics[width=.5\textwidth]{figures/Paulsen-img45.png}
\caption{
A cartoon depicting the dude being carried away by his own handle, published in the \emph{St. Louis Globe-Democrat} (St. Louis, Missouri) on February 5, 1887\citesource{February51887}, retrieved from Nineteenth-Century U.S. Newspapers
}
\label{fig:key:45}
\end{figure}

The overall conclusion that can be drawn from the analysis of these articles is thus that there are discourses favoring the word \emph{trousers}, which are particularly widespread in northeastern cities, in particular Boston and New York, and highly influenced by the social and political elite. The word is connected to educatedness and indexes elegance as well as the belonging to higher social classes. The prestige of \emph{trousers} is also indirectly visible through its adoption by the dude, who regards it as fashionable and hopes to impress others by using it. On the other hand, there are discourses ridiculing the high prestige attached to \emph{trousers} in the northeast – the dude figure is at the heart of this ridicule, and it is in this connection that the Englishness and thus the un-Americanness of \emph{trousers} is emphasized. But other articles also focus on the lack of authenticity of the use of \emph{trousers} instead of \emph{pants} by comparing people using \emph{trousers} to children who are told which form to use by their parents, for example. \emph{Pants} is presented against \emph{trousers} as a genuine and democratic form because it used by the majority of average and authentic Americans, and even though the shortened form is still considered as ‘slang’ by some people who prefer the long form \emph{pantaloons}, it is also motivated in one of the articles analyzed above as being in an iconic relationship with the object, which has also been shortened.


The qualitative analysis thus reveals the existence of competing discourses which are indicative of a negotiation process on the basis of the social meaning of \emph{trousers} and \emph{pants}. Given this competition, it is particularly interesting to focus on the text type of advertisements because their main aim is to appeal to the readers and convince them to buy the company’s product. The choice of the linguistic item is thus of particularly high importance for the company and it reveals information about how the creators of the advertisements assess the prominence of the different indexical values in discourse and how they use linguistic means to position their company and their product socially.

The first example of an advertisement was published in the \emph{North American} at the very end of the century, on October 5, 1899\citesource{MillersMarketOctober51899} (see \figref{fig:key:46}). The \emph{North American} was published in Philadelphia, Pennsylvania, which makes it likely that the company, Miller’s, is also based in Philadelphia. The advertisement is interesting because it not only prefers the term \emph{trousers}, which appears in big capital letters at the top of the advertisement, but it also explicitly distances itself from the word \emph{pants}, as can be seen in the line below trousers, which contains the line “Not Pants, Mind You” written in small capital letters. The advertisement thus makes use of the value of ‘elegance’ being indexically linked to the word \emph{trousers} – by stating that they sell trousers and not pants they implicitly call attention to the high quality of their product and the social status that potential buyers can convey by wearing the product. This shows that the discourses favoring \emph{trousers} are not restricted to New York and Boston but extend to Philadelphia as well and that the indexical meanings of the lexical items are so well known by the end of the century that they do not require explicit mention in the advertisement.


\begin{figure}[b]
\includegraphics[height=.3\textheight]{figures/Paulsen-img46.png}
\caption{
Advertisement making use of positive indexical values linked to \emph{trousers}, published in the \emph{North American} (Philadelphia, Pennsylvania) on October 5, 1899\citesource{MillersMarketOctober51899}, retrieved from Nineteenth-Century U.S. Newspapers
}
\label{fig:key:46}
\end{figure}

\begin{figure}[b]
\includegraphics[width=.8\textwidth]{figures/Paulsen-img47.png}
\caption{
An advertisement containing an explicit discussion about the use of the terms \emph{pants} and \emph{trousers}, published in the \emph{St. Louis Globe-Democrat} (St. Louis, Missouri) on January 28, 1887\citesource{FWHumphrey&CoJanuary281887}, retrieved from Nineteenth-Century U.S. Newspapers
}
\label{fig:key:47}
\end{figure}


The second and third example are advertisements published in the \textit{St. Louis Globe-Democrat} in Missouri by the company F.W. Humphrey \& Co. In contrast to the example above, they avoid choosing a particular term, but opt for making the choice of a name for the product the subject of explicit discussion in the advertisements. The title of the advertisement published on January 28, 1887\citesource{FWHumphrey&CoJanuary281887}, is thus “What’s in a Name?”, followed by a subtitle in smaller print “That which we call a rose by any other name would smell as sweet” (see \figref{fig:key:47}). This reference to Shakespeare’s play \emph{Romeo and Juliet} not only serves to convey the position of the company in the discourses on language, namely that it is not the name of the product that matters but the product itself, but it also serves to support this position – especially against the members of the intellectual (British-oriented) elite who are expected to prefer the term \emph{trousers} but who are also likely to accept an argument based on Shakespeare. In the text following the title and the subtitle, an anecdote is told about an “uncultured Chicagoan” who was so “narrow-minded in the nomenclature of ‘leg-dressings’” that he would not buy pants in a store where they call them \emph{trousers}. The exclusive preference for the term \emph{pants} is thus linked to a lack of culture and to the city of Chicago and presented negatively in contrast to the city of St. Louis and the company’s view of allowing “the individual to exercise his own sweet will and call them pants, trousers or breeches”. Individuality and freedom are thus presented as the most important values, which are superior to questions of culture, elegance, or social position.



\begin{figure}[b]
\includegraphics[width=.8\textwidth]{figures/Paulsen-img48.png}
\caption{
An advertisement using \emph{trousers} in the title but giving several terms as alternatives in the text, published in the \emph{St. Louis Globe-Democrat} (St. Louis, Missouri) on January 12, 1887\citesource{FWHumphrey&CoJanuary121887}, retrieved from Nineteenth-Century U.S. Newspapers
}
\label{fig:key:48}
\end{figure}

The other advertisement by the same company was published on January 12, 1887\citesource{FWHumphrey&CoJanuary121887}, and even though they chose the term \emph{trousers} here as a title, they emphasize in the text that the choice of terms is “a matter of individual fancy” and not important (see \figref{fig:key:48}). The mention of the possible label of \emph{dude protectors} in addition to the variants \emph{pants}, \emph{pantaloons}, \emph{trousers} and \emph{leg coverings} shows that the creators of the advertisement were aware of the negative indexical meanings linked to \emph{trousers}, but that they chose to position themselves above such evaluations. Nevertheless, their preference for \emph{trousers}, which also appears in big letters in the lower part of the advertisement (“All-Wool Trousers!”) suggests that even a company in the midwest does not opt for \emph{pants} (nor for \emph{pantaloons}) in choosing a title for their advertisement, which implies that the indexical link between elegance, high quality and the form \emph{trousers} must be regarded as quite strong. However, the explicit discussion of their position in the debate on the choice of lexical item shows that they have to defend using \emph{trousers}, which also indicates that they expect several people in the midwest to have a preference for \emph{pants}.



A different strategy of social positioning in discourse is adopted by the company Polack’s Clothing House in the following advertisement (see \figref{fig:key:49}), published in the same newspaper as the two examples above (but five years earlier, on May 16, 1882\citesource{May161882}). They use the title “Pantaloons!”, which suggests that they expect many people in St. Louis to find this term appealing. In addition, both \emph{pants} and \emph{trousers} are used in the text to designate the products sold without any indication of different linguistic or social meanings. This shows that the strategy employed here is similar to that used in Humphrey’s advertisement, namely the decision to avoid choosing just one term, but in contrast to Humphrey’s, Polack’s Clothing House chooses \emph{pantaloons} as a title and avoids any implicit or explicit references to discourses on language.


\begin{figure}
\includegraphics[width=.5\textwidth]{figures/Paulsen-img49.png}
\caption{
An advertisement using \emph{pantaloons}, \emph{pants} and \emph{trousers}, published in the \emph{St. Louis Globe-Democrat} (St. Louis, Missouri) on May 16, 1882\citesource{May161882}, retrieved from Nineteenth-Century U.S. Newspapers
}
\label{fig:key:49}
\end{figure}

A similar avoidance of choosing one term over the other can also be observed in the following advertisement (see \figref{fig:key:50}), published by Kohn, the Clothier and Hatter in the \emph{Morning Oregonian} on May 2, 1893\citesource{KohntheClothierandHatterMay21893}. Even though they use the term \emph{trouser} in the title “Special Trouser Sale!” and in the first line of the text (“We will give you the choice of any pair of trousers in the house [...]”), they then use \emph{pants} when listing the prices (“\$9 Pants, \$8 Pants, \$7 Pants”). In the text at the bottom, \emph{trousers} is used again so that overall, \emph{trousers} is the dominating form, but the fact that \emph{pants} is used as well suggests that the company must have decided that it does not want to repel potential customers who prefer \emph{pants} to \emph{trousers}.


\begin{figure}[t]
\includegraphics[width=.8\textwidth]{figures/Paulsen-img50.png}
\caption{
An advertisement using \emph{trousers} and \emph{pants}, published in the \emph{Morning Oregonian} (Portland, Oregon) on May 2, 1893\citesource{KohntheClothierandHatterMay21893}, retrieved from Nineteenth-Century U.S. Newspapers
}
\label{fig:key:50}
\end{figure}
The last example of an advertisement (see \figref{fig:key:51}) shows that pants can also appear in a title, which is even more notable as the advertisement appeared in the \emph{North American}, so in the same paper that published the advertisement by Miller’s with its exclusive preference for \emph{trousers}. However, \emph{pants} is not the only term in the title, but all three variants are used: “Pants, Pantaloons, or Trousers”. The first sentence of the text can be interpreted in two ways: “Take your choice” can either be understood as taking the choice between the different products sold by the company as well as taking the choice between the different terms in the title. That the company labels their products “Thompson’s Patent Cut Trousers” implies a preference for \emph{trousers}, but this preference is not foregrounded in the advertisement.

\begin{figure}
\includegraphics[width=.6\textwidth]{figures/Paulsen-img51.png}
\caption{
An advertisement using the word \emph{pants} in the title (in addition to \emph{pantaloons} and \emph{trousers}), published in the \emph{North American} (Philadelphia, Pennsylvania) on May 26, 1891\citesource{EOThompsonMay261891}, retrieved from Nineteenth-Century U.S. Newspapers
}
\label{fig:key:51}
\end{figure}

To conclude, the advertisements in particular show that none of the variants (\emph{pantaloons}, \emph{pants}, \emph{trousers}) was evaluated so negatively that it was generally avoided. While the analysis of \emph{luggage} and \emph{baggage} suggested that \emph{baggage} had become the form that was overwhelmingly regarded as positive by the end of the nineteenth century because it was linked to American nationality and authenticity, the analysis of \emph{trousers} and \emph{pants} revealed that while \emph{pants} was associated with similar positive values, the links between \emph{trousers} and positive values like elegance, high quality and a high social position were also still strong. The process of negotiating indexical values linked to the terms was thus still ongoing at the end of the nineteenth century.


