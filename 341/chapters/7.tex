\chapter{Conclusion}
\label{bkm:Ref13747659}\hypertarget{Toc63021256}{}
The present study started out with the question of whether statements about American English, like the one made by the editor in a newspaper article in 1882 saying that the American language “is the English language with the “H’s” in their proper places” should be of interest to linguists studying the emergence of American English as a new variety of English. The theoretical argumentation and the empirical analysis conducted in the present study strongly suggest that this question must be answered in the affirmative. The theory of enregisterment as well as recent research in sociolinguistics and linguistic anthropology provide a detailed account of how language variation on a structural level and the recognition of such variation on a discursive level interact: Registers, defined as cultural models of (linguistic) action, are constructed through metapragmatic and metadiscursive activities that create indexical links between linguistic forms and social values, as in the statement above, which links /h/-retention to the value ‘American’. Registers in turn influence speakers’ selections of linguistic forms in the process of social positioning in interaction, which is essentially an act of identity. These theories thus provide a detailed account of the mechanisms underlying the central tenet of Schneider’s (\citeyear{Schneider2003, Schneider2007}) Dynamic Model, namely that speakers’ attitudes towards and evaluations of linguistic forms are highly relevant because they influence speakers’ linguistic choices in a process of (conscious or unconscious) social identity alignment. The relevance of registers, which I see as equivalent to discursive varieties and as different from structural and perceptual varieties, motivates the application of the theory of enregisterment, which I have visualized in a model in \sectref{bkm:Ref523897668}, in order to investigate the discursive construction of American English in the nineteenth century.


One of the central achievements of the study is the development of a methodological approach to investigating historical enregisterment processes that is based on a discourse-linguistic framework: DIMLAN. The systematic collection of newspapers articles in two large databases by means of specific search terms allowed for quantitative as well as for qualitative analyses of metadiscourses surrounding five phonological and two lexical variants. I analyzed close to 1,200 newspaper articles by focusing on the intratextual layer in order to determine which indexical links were created in these articles and how they were created. Based on these analyses, larger patterns could be discovered and quantified to some extent in order to do justice to the transtextual layer. Particularly the temporal development and the regional distribution of metadiscursive activities in newspaper articles could be investigated by means of quantitative analyses.

The findings of the study provide important insights into the enregisterment of American English, even though the picture is, of course, far from complete since the focus is restricted only to seven linguistic forms. The analysis of the newspaper articles showed that different variants played different roles in the process of enregisterment at different times: /h/-dropping and -insertion as well as yod-dropping were present in newspaper metadiscourses throughout the nineteenth century, but while the former was a very salient form that was essential in delimiting an American English register from a British English register, the latter was much less salient and became more and more non-specific so that it could ultimately index the value ‘American’ and contribute to the delimitation of a non-specific American English register from several specific American English registers. For /h/-dropping and -insertion, the nationality value was consequently most important: Based on nationalist ideologies placing a high value on uniformity, the absence of /h/-dropping and -insertion was constructed as a marker of American English and as a sign of its superiority over a heterogeneous and thus inferior British English. The ““H”s in their proper places” therefore functioned as a basis for American’s linguistic self-confidence. By contrast, yod-dropping was characterized mainly by its non-specificity. Even though it was marked as deviant through the spelling <oo>, it became linked to an unmarked standard American English because the alternative variant, yod-retention, became increasingly associated with southern speech on the one hand and pedantic behavior and attitudes on the other hand. In comparison to these two forms, all the other forms occurred in metadiscourses much later, and they all exhibited a dramatic rise in the last two decades of the nineteenth century.

Non-rhoticity, the back \textsc{bath} vowel and the labiodental realization of pre-vocalic /r/ were all linked to the characterological figure of the dude, which became central in the delimitation of an authentic American English register from an inauthentic American English register. The dude was portrayed as being an American, but not a true and genuine one, because he held English fashions, including linguistic forms, in high regard. His attempts at imitating Englishmen were subject to humor and ridicule in the newspaper articles, which conveyed to the reader that such behaviors, and manners of dressing and speaking were not acceptable for ‘real’ Americans. More peripheral, female figures like the salesladies or the society girls confirm that the authenticity value was important in countering the positive evaluations of non-rhoticity in the urban northeastern centers. Articles which contained both \emph{luggage} and \emph{baggage} in close proximity and which linked these variants to British or American English respectively also rose in frequency in the 1880s. They show that the nationality value remained important because the overwhelmingly positive evaluation of \emph{baggage} as the American variant is based on the strong indexical link between \emph{luggage} and Englishness that made the form an inadequate alternative in America. By contrast, the qualitative analysis of articles containing \emph{pants} and \emph{trousers} demonstrated that in this case the evaluation of \emph{pants} was more contested and rather revolved around the authenticity value. In articles favoring \emph{trousers}, positive associations with educatedness, high social standing and elegance could be identified, whereas articles favoring \emph{pants} constructed the use of the word as a sign of being a true American. That \emph{trousers} was also linked to the dude and occasionally combined not only with a labiodental /r/, but also with non-rhoticity (e.g. in \emph{twousahs}), strengthened associations between \emph{trousers} and a lack of authenticity. This made \emph{trousers} a form that illustrates that enregisterment includes several types of perceivable signs: The word not only linked the level of the lexicon to that of phonology but it also connected linguistic to non-linguistic signs because the newspaper articles also drew attention to the style of trousers worn by the dude. In contrast to the inauthentic dude figure, the authentic American was embodied by figures like the strong and tough American cowboy, hunter or farmer. Through the contrast with the dude, it became implicitly clear that these figures used rhotic forms, front \textsc{bath} vowels and non-labiodental (probably retroflex) realizations of /r/. In some cases, these American figures were represented as using hyper-rhotic forms – I have argued that this created a continuum with rural, uncultivated, but nevertheless true, strong and hard-working Americans using hyper-rhotic forms on the one end, and urban, affected, ignorant, effeminate and lazy Americans using non-rhotic forms on the other end.

Next to the nationality and the authenticity value, the non-specificity value also figured prominently in the final decades of the nineteenth century. Non-rhoticity was constructed as specific not only through its association with the dude but also through indexical links between the form and southern Americans, mountaineers, and, most frequently, Black Americans. Non-rhoticity was not only salient, but the groups it was linked to were portrayed in a negative light in these articles, which made rhoticity, the non-specific variant, appear positive by implication.

The enregisterment of American English thus proceeded by way of delimitation against other registers: British English on the one hand, but also inauthentic as well as more specific regional and social American Englishes on the other hand. These findings have several implications for further research. First, they provide support for the Dynamic Model because the forms that have been enregistered as American in nineteenth-century newspapers are also forms that are now regarded as Standard American English and that have also been shown to have increased in frequency since then. The analyses conducted in this study show in detail how cultural models of (linguistic) action were constructed and how they circulated and thus became potentially relevant to speakers’ social positioning in interaction. The concept of prestige, which was often used in previous research in rather simplistic terms, was elaborated in more detail by identifying the social values based on which forms were constructed as prestigious and how conflicting evaluations of forms are negotiated. This study thus provides a first step towards addressing the need for studies that investigate the connection between the discursive and the structural level. It needs to be complemented by more studies on enregisterment processes that use other search terms, investigate other metadiscursive genres or focus on other linguistic forms. Moreover, it needs to be combined with detailed studies of actual language use in nineteenth century America in order to find out how the emerging registers actually affected speakers’ linguistic choices. Focusing on a specific context (a region and/or social group) will make this task easier.

Secondly, this study supports the claim that a theoretically informed description of American English needs to distinguish carefully between the structural and the discursive level, while at the same time considering their interaction. This means that on the discursive level attention needs to be paid to which linguistic forms are subject of reflexive activities and how these forms are enregistered in the process. I have shown how this study provides a more detailed perspective on Schneider’s account of the emergence of American English, particularly on the shift from an exonormative to an endonormative orientation, by showing how these orientations become visible in metadiscourses relating to specific linguistic forms. A description that includes enregisterment thus comes closer to the “feature-by-feature social-reasons account” demanded by \citet[279]{Trudgill2008b}.

Finally, I have argued that the present study can inform models of language change like the mathematical model by \citegen{Baxter2009} which is based on Croft’s usage-based evolutionary framework. Baxter et al.’s test of Trudgill’s model of new-dialect formation was based on a model of neutral interactor selection to do justice to Trudgill’s claim that social factors do not play a role. That this test failed suggests that in order to account for the emergence of new varieties (at least for the test case of New Zealand English) mechanisms of weighted interactor selection and/or replicator selection need to be included in the model. This means that the value placed on the interactor (the speaker) and/or the replicator (the linguistic form) need to be taken into account. The present study has suggested an approach to determining such values in quantitative terms, but it has also emphasized that this can only be achieved in combination with detailed qualitative analyses.

The editor’s comment on the American language “with the “H”s in their proper places” not only constitutes one quantifiable data point indicating a positive evaluation of /h/-retention and an indexical link to the value ‘American’. It is also part of a larger discourse that extends to other places and other genres, which proves one point: that metadiscursive activities are as complex as language use itself.

