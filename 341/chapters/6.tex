\chapter{Implications and limitations of this study}
\label{bkm:Ref13473653}\hypertarget{Toc63021251}{}\section{Limitations}
\hypertarget{Toc63021252}{}
Every study on enregisterment faces the difficulty that there is a large variety of ways to typify language use, so that a systematic study of such typifications that ultimately lead to registers requires a selection of typifications that are the focus of the study. In the present study, the focus is on typifications which occurred in newspaper discourse in the United States during a time period of 100 years, which means that all observations and conclusions based on the analysis of these articles are necessarily restricted to this particular discourse – even though it is possible to identify connections to discourses in other regions (Great Britain in this case) and discourses in other types of media, for example in books (which can be subdivided into different genres typifying language in different ways, for example non-fictional prescriptivist texts and fictional literary works). Although newspapers were chosen because of their wide regional and social circulation, other typifications of language need to be studied as well to complement the picture.


A second limitation results from the chosen search methodology. The focus on pronunciation respellings is likely to lead to an overrepresentation of articles that contain representations of direct speech and an underrepresentation of articles that contain explicit metalinguistic comments. However, I have argued that representations of direct speech are necessarily linked to speakers using these forms and that repeated associations between speakers, their social characteristics and the linguistic forms they use lead to the creation of metapragmatic stereotypes, often embodied in characterological figures, which offer more concrete models available for social alignment than abstract descriptions of language, so that these rather implicit ways of typifying linguistic forms serve an important function and are thus worth focusing on. Nevertheless, it would be a worthwhile task to search for articles by choosing search terms that are likely to yield more articles containing explicit comments, for example by searching for descriptions like \emph{drop their aitches}. Furthermore, pronunciation respellings already signal deviation and otherness through their alternative spelling, so that the articles are more likely to contain negative evaluations of the phonological forms represented by the search terms. Positive evaluations of these forms are thus likely to be missed or at least underrepresented. However, in several articles, positive evaluations are indirectly visible because they are the reason for people’s negative responses, thus rendering revalorizations of linguistic forms visible to the observer. The complementation of the analysis of phonological forms by an analysis of lexical forms also revealed metadiscursive activity which did not rely on pronunciation respellings and showed on the one hand that the lexical forms fit into the pattern found for phonological forms (including the temporal development), but that in the case of \emph{pants} and \emph{trousers} competing evaluations become more directly observable.

In general, all the results of the study are only based on an investigation of a small set of linguistic forms. The advantage of this approach is that these forms can be analyzed in a very systematic and comprehensive way, but this approach of course has the disadvantage that metadiscourses which are potentially relevant for the enregisterment of American English are not taken into consideration. For example, \citegen{Hodson2017} study on the enregisterment of American English in British fiction revealed that discourse markers like \textit{I guess} or \textit{I calculate} were linked to vulgar American speech and these forms have also occurred in my first exploratory search for articles containing the phrase \emph{American language}, which suggests that an analysis of metadiscourses surrounding these forms in America would yield interesting results to complete the picture. Furthermore, I found early non-rhotic forms like \emph{fust}, \emph{hoss} and \emph{cuss} in the newspaper articles (early because post-vocalic /r/’s were elided first in contexts where they preceded /s/, see \sectref{bkm:Ref530736302}), and they seem to have been enregistered in different ways than forms like \emph{deah} or \emph{bettah}, where post-vocalic /r/ is not in pre-consonantal position. For example, Hank Borrows, the Montana wagoner in the article published in 1898 (discussed in \sectref{bkm:Ref7775416}), represents the western rural American in contrast to the eastern city dude, but he is portrayed as using early non-rhotic forms like \emph{wuss} ‘worse’. Exploring these forms would thus yield insights into how phonological context also matters in metadiscursive activities.

Not only the number of forms is restricted, but also the number of articles found for each form. If compared to the overall number of articles contained in the databases, which is close to 78 million, the number of articles collected for each analysis seems very small. However, as the articles needed to be checked manually as to whether the search term was identified correctly by the recognition software and as the main part of the analysis rested on the qualitative analysis of \emph{all} articles containing the respective search terms, it would not have been possible to investigate a considerably higher number of articles using the methodological framework developed for this study. Despite the small numbers of articles, however, I could identify clear patterns, which are indicative of the fact that the articles analyzed in the study constitute only the metaphorical tip of the iceberg. For example, if it is considered that \emph{bettah} alone yielded 374 articles and that the articles containing \emph{bettah} also contain several other lexical items which are marked as non-rhotic, for example high frequency words like \emph{heah} ‘here’ and \emph{theah} ‘there’, it can be assumed that the number of articles containing similar instances of metadiscursive activity is in fact much higher.

Finally, it is a problem that in the majority of cases, the author of the article is not mentioned and/or the origin of the article is not entirely clear. This makes it difficult to identify the actors in the discursive processes analyzed above. However, it is at the same time also important to note that it was precisely the medium of newspapers that allowed the people creating and publishing the articles to recede into the background in favor of a seemingly impersonal point of view. People were usually not told who had “experienced” the anecdotes or created the cartoons and this had the effect that they were more likely to generalize and not see the content as being the result of a personal subjective view only. However, sometimes a general political stance of the editor and therefore of the newspaper was known and this might not only have had an influence on what was published but also on how people interpreted the content. A closer analysis of who those authors and editors were and how they influenced metadiscursive practices in newspapers would nevertheless enrich the analysis.

To conclude, the present study cannot give a complete picture of how American English was constructed as a discursive variety through enregisterment, but it offers several important insights into enregisterment processes in the nineteenth century that are based on a systematic investigation of a large body of evidence and that have several implications for modeling the emergence of new varieties of English, for a theoretically informed description of the historical development of American English and for modeling of language change in general. I will discuss these theoretical implications in detail in the following sections. It remains a task for further studies to complement the picture drawn here by using further search terms (including search terms that are not pronunciation respellings), by extending the time frame or by focusing on typifications of language occurring in other media.

\section{Implications for modeling the emergence of new varieties of English}
\label{bkm:Ref13994429}\hypertarget{Toc63021253}{}
The theoretical arguments and the results of the analysis provide support for \citegen{Schneider2007} claim that identity constructions are central for the emergence of new varieties of English. The theoretical framework I have developed in Chapter \ref{bkm:Ref523475305}, which relies heavily on \citegen{Agha2007} theory of enregisterment and other theoretical positions developed by linguistic anthropologists (especially \citealt{Silverstein2003, Silverstein2016} and \citealt{Bucholtz2005}) as well as sociolinguists (especially \citealt{Eckert2008, Eckert2014}) and discourse-linguists (especially \citealt{Spitzmuller2011} and \citealt{Spitzmuller2013}), is suitable for shedding light on precisely this interaction between linguistic forms and identity constructions that Schneider postulates and that was subject to heated debate because it contrasts with \citegen{Trudgill2004} deterministic theory of new-dialect formation (see \sectref{bkm:Ref521576818}).\footnote{It also provides a model for the concept of “positive feedback” that plays a role in \citegen{Kretzschmar2014} view on how new varieties emerge.} I argued that a crucial step in this debate is to carefully distinguish between different types of varieties, structural varieties, perceptual varieties and discursive varieties, because only a careful distinction makes it possible to systematically investigate how exactly these levels influence each other. Conceptualizing registers (in Agha’s sense) as discursive varieties has the advantage of emphasizing that they are not independent of language use (which can be thought of as a complex system, following \citealt{Kretzschmar2015b}), but interact with it. In contrast to structural varieties, the definition of discursive varieties puts speakers’ ideas about and evaluations of linguistic forms in the center: In the process of enregisterment, speakers typify linguistic forms and link them to typified social personae and practices. The emerging registers in turn influence the production of linguistic forms: According to Spitzmüller’s model of social positioning (see \figref{fig:key:4}), registers fulfill their function as cultural models of action in interaction because actors position themselves socially (or get positioned by others) not only through alignment with the actor(s) they communicate with directly but also through alignment with these cultural models. These processes of social positioning and alignment constitute identity in interaction and emphasize the relational nature of identity, as pointed out (among others) by \citet{Bucholtz2005}. The theory of enregisterment and the model of social positioning thus provide a detailed picture of the central element of Schneider’s Dynamic Model: the interaction between identity and language use.


In this study, I have identified several registers in nineteenth-century America based on empirically observable metadiscursive activities. I could show for example that since the 1880s non-rhoticity had been discursively constructed as being part of an urban northeastern American register, a southern American register, a mountaineer register and a Black American register, which suggests that for speakers who did not want to position themselves as being part of such groups or as having characteristics associated with these groups, the use of rhotic forms must have seemed more attractive than that of non-rhotic forms. However, the analysis of indexical values associated with linguistic forms revealed the complexity of evaluations, which could contrast with each other. To use non-rhoticity as an example again, the newspaper articles also show that the form must have been evaluated very positively as a sign of cultivation, education, upper-class membership and a close relation to European/British ancestry, especially in northeastern cities, but probably also in the coastal south and perhaps even in (rather urban) regions further west (as indicated by the salesladies’ conversation in Denver, Colorado). Furthermore, I also found glimpses of a positive evaluation of non-rhoticity within the group of Black American speakers because the form was used by the Black poet Paul Dunbar in his celebration of “de co’n pone”, which suggests that for this group of speakers, the use of non-rhoticity was attractive as a signal of African American identity.

A model of linguistic and social action indexically linked to the value ‘American’, which could then serve as a point of reference for speakers to position themselves as Americans, was thus constructed in a complex relation to other models. The fact that these models were linked to stereotypical personae meant that speakers did not exhibit positive or negative alignment with any sort of abstract national identity, but with more concrete models of types of speakers and behaviors. \citegen[176]{Agha2007} concept of negative alignment, that is, a process of self-differentiation and not of convergence, is important in this context because signaling an American identity could also be done by \emph{not} aligning with social personae and practices which were regarded either as inauthentic or too specific to be enregistered as generally American. To put it simply: According to the model of social positioning, Americans living at the end of the nineteenth century who shared the evaluations conveyed in articles ridiculing the dude figure were likely to realize the post-vocalic /r/ and use the word \emph{pants} in order to distance themselves from this social persona and signal their identity as true Americans. The existence of such social personae and characterological figures make the models more concrete than the type of abstract colonial identity that led \citet{Trudgill2008} and \citet{Mufwene2008} to argue against identity being a major force in the emergence of new varieties. This confirms \citegen[82]{Eckert2016} hypothesis that “[a]ccommodation in colonial situations may have more to do with emerging local social types or stances in the colonial situation than with some abstract colonial identification”. Trudgill’s view that identity is only relevant \emph{after} a stable set of forms constituting the new (structural) variety has emerged is also not convincing given the results of this study: While in the case of /h/-dropping and -insertion it could indeed be argued that the retention of /h/ has already been at least the majority variant (if not the only variant) in the United States before it became the focus of discursive activity, in all the other cases the articles suggest that there was considerable variability in the use of the forms \emph{at the same time} that they were subject to implicit and explicit metadiscourses in newspapers. Furthermore, while \citet[88]{Trudgill2004} sees “focusing”, so the stage at which identity supposedly plays a role, as “the process by which the new variety acquires norms and stability”, I have shown in the present study that metadiscourses did not just stabilize a particular set of forms as American, but that they were the site of a negotiation process in which several forms competed for the value ‘American’. It is thus more convincing to regard language use and metadiscourses as interacting forces, revolving around identity: Expressing and ascribing identity through the creation and recognition of indexical values in every instance of communication leads to the creation and recognition of cultural models of actions, which then become points of reference for expressing and ascribing identity in interaction again.

Another point that has been put forward in arguments against identity is that acting based on identity is somehow intentional and that it is unconvincing that the formation of a new variety is a planned and goal-oriented process (see \citealt{Mufwene2008}, as discussed in \sectref{bkm:Ref521576818}). However, \citegen[95]{Schneider2007} view of speakers’ social identity alignments that are at the heart of the Dynamic Model is different:

\begin{quote}
Note that there is no implication made here that these developments have anything to do with language consciousness: accommodation works irrespective of whether the feature selected and strengthened to signal one’s alignments is a salient marker of which a speaker is explicitly aware or an indicator which operates indirectly and subconsciously.
\end{quote}


\citet[606]{Bucholtz2005} support this position in their discussion of the concept of “agency”:


\begin{quote}
From the perspective of an interactional approach to identity, the role of agency becomes problematic only when it is conceptualized as located with\-in an individual rational subject who consciously authors his identity without structural constraints. [...] [A]gency is more productively viewed as the accomplishment of social action.
\end{quote}


While it is clear that the newspaper articles that were of interest for the present study are the result of conscious acts aiming to draw attention to the linguistic forms in question and thus contributing to the enregisterment of these forms, this does not imply that speakers’ alignment with the social personae and practices associated with these forms is in any way conscious. Neither does it imply that the alignment is restricted to only those forms targeted in the articles. If speakers are exposed to speech by actual speakers that they link to such typified social personae, for example in direct interaction, they might (consciously or subconsciously) pick up on other forms and link them to the register as well.


To conclude this discussion, the complexity of understanding and studying identity has already been noted by \citet[361]{Schneider2000}:

\begin{quote}
It is clear that identity as the determining factor of speech performance is a concept more complex and scientifically more difficult to grasp than region and class, encompassing psychological, sociological, and pragmatic components which are fuzzy in themselves; but it is something we need to understand in an increasingly complex and multifarious postmodern world, an appropriate challenge for language variation study in the new millennium.
\end{quote}


The present study has not only shown that this challenge has been successfully addressed by linguistic anthropologists and sociolinguists, but it has also provided a methodological approach to study register formations systematically and thus to provide insights into which cultural models, and, most importantly from a linguistic perspective, which linguistic forms became available for speakers to constitute identity in interaction.


While this study therefore supports the essential role of identity in the emergence of new varieties, it also challenges \citegen[30]{Schneider2007} claim that “to a considerable extent the emergence of PCEs is an identity-driven process of linguistic convergence [...] [which] is followed by renewed divergence only in the end, once a certain level of homogeneity and stability has been reached”. As argued in Chapter \ref{bkm:Ref532290015}, I rather support the view that the existence of different registers is an essential prerequisite for the construction of a unifying national register. I thus agree not only with \citegen{Kretzschmar2014} point that linguistic diversity has always existed, during all stages postulated by the Dynamic Model, but that this diversity has also always been socially indicative. While \citet[296]{Schneider2007} writes that in the last phase of the emergence of a new variety,

\begin{quote}
diversification happened because the various regional, social, and ethnic groups recognized the importance of carving out and signaling their own distinct identities against other groups and also against an overarching nation which, while it is good to be part of, is too big and too distant to be comforting and to offer the proximity and solidarity which humans require.
\end{quote}

I postulate that at least on the discursive level, the construction of regional and social registers was essential for creating models of the negative and deviant “other” against which a neutral national standard register could be constructed. What could mark the starting point of a new phase after the fourth phase of endonormative orientation is the increasingly positive revalorization of social and regional registers. In the newspaper articles analyzed in this study, all specific registers were constructed largely through linking the speech forms to \emph{negative} social personae and values: the uncivilized, uneducated and immoral Black American and mountaineer, the old-fashioned, quaint and also rather uneducated southerner and the affected, ignorant, lazy and unsuccessful city dude. As shown in studies on the enregisterment of regional dialects in the twentieth century (see \sectref{bkm:Ref506884048}), this negative evaluation was replaced by a pride in local practices, including the use of linguistic forms indexically linked to the region. Pittsburgh is a prime example for such a changing revalorization and it is interesting to see that the authenticity value which played such an important role in delimiting an authentic national variety at the end of the nineteenth century then became important in the re-enregisterment of a local variety in the twentieth century.

\section{Implications for a theoretically informed history of American English}
\hypertarget{Toc63021254}{}
It is one of the important contributions of Schneider’s Dynamic Model that it provides a framework for a theoretically informed history of new varieties of English, including American English. Schneider’s own account of the developmental phases of American English constitutes an excellent starting point for such a description. The analysis conducted in the present study adds to it and suggests that some aspects deserve closer attention.


An often-debated issue in research on the history of American English is the role of British influence on linguistic developments in America. This debate revolves especially around the development of non-rhoticity and rhoticity in America (see \sectref{bkm:Ref517077661} and \sectref{bkm:Ref530736302}). Some linguists (e.g. \citealt{Fisher2001}) argue that non-rhoticity became common in the northeast and the coastal south because it was associated with prestigious British speech and that it also declined in frequency because British English lost its prestige (see for example \citegen[296]{Labov2006} claim that rhoticity becoming the prestige form in New York City “reflected the abandonment of the earlier prestige form of Anglophile English”). This view is contested by other linguists, most prominently \citet{Bonfiglio2002}, who argues for a greater role of an inner-American struggle centering on race and ethnicity. Furthermore, I have also pointed out that there were different views with regard to the temporal development of the changes in prestige. \citet{Fisher2001}, for example, argues for a loss of prestige of non-rhoticity after the Civil War and he points out that this shift in prestige was particularly visible in New York, where the colonial elite had lost its influence. By contrast, \citet{Bonfiglio2002} argues for a change in norms in the first half of the twentieth century, while \citet{Labov2006} suggests the years of World War II as the crucial point in time for this change.

The present study contributes to this debate by showing how the prestige of non-rhoticity (and rhoticity respectively) can be studied more systematically within a discourse-linguistic framework. It provides evidence for an increasing amount of attention paid to non-rhoticity in metadiscursive activities in newspaper articles in the second half of the nineteenth century, especially in the last two decades. The occurrence of articles containing evaluations of non-rhoticity was not regionally restricted, but they were especially frequent in the (mid)west and north of the United States, that is, in areas which according to the available evidence have always been rhotic. It can also be assumed that the articles were read by many people: Newspapers in general were widely read, several articles were reprinted in one or more newspapers and many of the articles contained humorous texts (sometimes accompanied by visual elements) and thus provided entertainment for their readers. The metadiscursive activities surrounding post-vocalic /r/ therefore support Fisher’s postulation that non-rhoticity started to lose prestige already in the late nineteenth century. Furthermore, the qualitative analyses of these activities also demonstrate that positive and negative evaluations of /r/ must have co-existed and that British influence did indeed play a role. The writers of the articles explicitly or implicitly convey that English fashions and speech were regarded as desirable in northeastern cities and consequently imitated by Americans – first by upper-class speakers and then by middle-class speakers who wanted to be part of higher circles of society. The negative evaluation of this process of imitation provides evidence of a revalorization of the form: Users of non-rhotic forms were depicted as Americans who lack authenticity by pretending to be English. Metadiscursive activities of this kind are therefore a clear indicator of the shift from an exonormative orientation to an endonormative orientation postulated by the Dynamic Model. Most importantly, they suggest that this shift was noted by a large number of speakers and not just an educated elite who was interested in language. By means of newspaper articles, indexical links between phonological forms and social values could circulate widely even before the advent of radio broadcasting. This study therefore provides support for those accounts that attribute the change in prestige to British influence. It gives a detailed picture of how the link between non-rhoticity and Englishness led to the construction of the social persona of the inauthentic American, embodied by the characterological figure of the dude, which becomes available as a model for negative social alignment – the dude’s character traits are represented so negatively that they do not invite imitation but rather distancing. This in turn makes contrasting ‘genuinely’ American figures appear attractive: the (mid)western farmer, cowboy or hunter. This is in line with \citegen[231]{Bonfiglio2002} finding that constructions of the “western hero as an instantiation of the proper American male” played an important role in the standardization of American English. However, as pointed out by \citet{Minnick2010}, the newspaper articles demonstrate that these constructions were already prevalent in the nineteenth-century, especially in the last two decades. Furthermore, these western American heroic figures were set in opposition to the eastern elite (via the dude figure) and not to Black Americans, so that the claim that the (mid)western parts of the country were constructed as ‘racially pure’ in contrast to ‘ethnically contaminated’ areas in the northeast and the south could not be confirmed for the nineteenth century.\footnote{As newspaper articles representing Black social personae usually located these personae in the south and in northeastern cities (that is, in those areas where most African Americans lived), indexical links between region and race were of course created. However, in the articles analyzed here, these links were not salient because region was not foregrounded in relation to the value of ethnicity but in relation to the value of authenticity.}

However, this study also provides support for the important role of race in the enregisterment of American English. The quantitative analysis of social personae linked to \emph{bettah} showed that in the vast majority of cases these personae were Black Americans. The qualitative analysis of these articles demonstrated that racial othering was achieved in several ways – they all encompassed a negative portrayal of physical traits and social characteristics and, as in the case of the dude, relied to a great extent on humor and ridicule. The enregisterment of Black speech, which included non-rhoticity as well as a large number of other linguistic forms, therefore also created a cultural model that invited negative social alignment. This demonstrates the extent to which metadiscourses were shaped by actors who were white and who had the power to mark Black speech as the deviant “other” and its forms as indexing negative characteristics only. A central argument of \citegen{Bonfiglio2002} study is thus confirmed, although the analysis conducted in the present study suggests that the factor race was already of more importance in the late nineteenth-century than postulated by Bonfiglio. Overall, the discourse-linguistic analysis conducted in this study therefore does justice to the complexity of “prestige”. Prestige depends crucially on who evaluates a linguistic form in which context – a point which is illustrated well by the author of a newspaper article by stating that “[a] hardy backwoodsman may not appear so well in a drawing-room, but he is more attractive on his own ground” (September 8, 1889\citesource{September81889}).

A further contribution of the present study is that it sheds further light on the relation between endonormative orientation and diversification in the history of American English (and thus between phase 4 and 5 of the Dynamic Model). As pointed out above, early endonormative tendencies could already be found in relation to discourses surrounding /h/-dropping and -insertion, which established a sense of linguistic superiority over British English, although the evaluation of this form as incorrect was still shared between Great Britain and America. Diverging evaluations of forms could be observed with striking frequency in the 1880s and 1890s – American forms were “defended” against fashionable English forms, thus providing evidence of the existence of an American norm that was judged superior to a British norm. The term \emph{Hinglishism}, used in an article published in 1881, is an excellent indicator of this development: It not only marked British English as inferior based on /h/-insertion, but also as deviant from an American norm by playing on the negatively connotated term \emph{Americanism}. At the same time, the increasing representations of inner-American linguistic diversity, that is, of specific social and regional forms of speech, also attest to the stability of the American norm at this point. Diversity does not appear threatening anymore when there is a stable agreement on the existence of forms which are “universally” American; in that case, diversity rather contributes to the stabilization of a new American norm (see the argumentation in \sectref{bkm:Ref13994429}). This suggests that the last two decades can be classified as a transition phase in which discourses emphasizing the presence of a positively evaluated American linguistic norm overlapped with discourses showing an interest in inner diversity. The developments observed in newspaper articles in the present study tie in with the “cult of the vernacular” in American literature described by \citet{Jones1999} (see \sectref{bkm:Ref517077661}), which demonstrates that discourses in different genres were likely to have influenced each other.

A final point concerns the characterization of the different phases of the Dynamic Model. \citet[92--93]{Busse2015} for example notes that Richard Grant \citet{White1870} “explicitly disclaim[s] any right of the Americans to set up their own linguistic standard, independent from Britain”, which seems to be a position that does not fit into a phase of endonormative orientation. However, he cites Schneider’s observation of a pro-English movement in the 1870s and 1880s as well as \citegen{Kretzschmar2012} position that exonormative tendencies did not cease to exist until the end of World War II. This illustrates a point that \citet[277]{Schneider2007} makes himself, namely that there has always been an ambivalence with regard to language attitudes in America. Pride in American forms coexisted with a “complaint tradition” upholding British forms as correct. The phases of the Dynamic Model are thus not to be understood in absolute terms, but as describing \emph{tendencies}. Accordingly, \citet[277]{Schneider2007} suggests at one point that “[i]n \emph{quantitative} terms [...] nativization made the balance tip toward the former position” [emphasis mine]. The present study shows how such quantitative claims can be empirically tested: The case of \emph{baggage} and \emph{luggage} shows for example that there were only three articles in the collected sample that valued the English form \emph{luggage} more highly than \emph{baggage}. It remains a task for further research to identify the quantitative evaluative patterns of \emph{pants} and \emph{trousers}, which, according to the qualitative analysis, are not likely to show such a clear preference for \emph{pants}.

 To conclude, the present study suggests a way of approaching the description of American English based on the theoretical framework of the Dynamic Model, but in which the level of structure and the level of discourse are described and investigated in their own right in order to shed light on their interaction. The enregisterment processes identified here show how discursive patterns can be identified – this study thus presents a first step towards a “feature-by-feature social-reasons account” \citep[279]{Trudgill2008b} which shows convincingly how social factors and identity influence the shape of a new variety.

\section{Implications for theories of language change}
\hypertarget{Toc63021255}{}
A final implication of the present study relates to general theories of language change. I would like to draw attention here to \citet{Baxter2009}, who use a mathematical model of language change and computer simulation to test \citegen{Trudgill2004} theory of the emergence of New Zealand English. Their model thus relates directly to the discussion of models of the emergence of new varieties of English and can thus also be related to my study. It is based on Croft’s usage-based evolutionary framework of language change \citep{Croft2000, Croft2006}, which proposes that language change is an evolutionary process that can also be found in other areas, particularly biology. The underlying theory is Hull’s (\citeyear{Hull2001, Hull1988}) General Analysis of Selection (GAS). It consists of the following central elements: first, the \textit{replicator}, which in the case of language is the lingueme, an element of linguistic structure that corresponds to a sociolinguistic variable, secondly a process of \textit{replication}, which is language use in face-to-face interaction, and thirdly the \textit{interactor}, which is an entity that interacts with its environment to cause replication, which in the case of language is the speaker interacting with other speakers.

Within the model, language change is characterized as a differential replication of variants in the process of interaction between speakers. This basic process can be modeled in different ways, depending on which factors are assumed to affect replication. On the one hand, the process can be marked by selection, which means that one variant is selected over other variants. This selection can be influenced by the differential social value assigned to the linguistic variant (\textit{replicator selection}) or to the speaker (\textit{weighted interactor selection}). It can also be neutral, which means that the frequency of interaction between speakers is the only factor influencing the process (\textit{neutral interactor selection}). On the other hand, no selection can be involved, which makes “the successful propagation of a variant [...] a function of the frequency of the variant” and models language change as a purely probabilistic process (\textit{neutral evolution}).

\citet[270]{Baxter2009} argue that \citegen{Trudgill2004} deterministic model of new-dialect formation corresponds to a model of neutral evolution because “his theory is invoking the same usage-based processes as we are, namely that speakers alter their behavior in response to the language they hear around them, and those usage-based processes are probabilistic”. As Trudgill’s model is also based on accommodation between speakers in interaction and thus relies on the frequency of interaction, \citet[271]{Baxter2009} find that “neutral interactor selection” is also a possible mechanism that Trudgill allows for. It is this latter model that they use to test Trudgill’s claims – they basically take the empirical data obtained in the analyses conducted by \citet{Trudgill2004} and \citet{Gordon2004} and apply the model of neutral interactor selection to find out whether the changes in frequency can be explained by this particular type of mechanism (see \citealt{Baxter2009} for the mathematical details). The result of the application is “compelling evidence that neutral interactor selection is unlikely to be solely responsible for the fast convergence of the New Zealand dialect to a homogeneous, stable variety” \citep[284]{Baxter2009}, which means that although they do not deny that the frequency by which language users interact with each other and are thus exposed to linguistic variants does play a role in language change, they strongly argue that “weighted interactor selection and/or replicator selection must be added into the model” \citep[291]{Baxter2009}.

\citegen{Baxter2009} results therefore not only support the important role attributed to social factors in Schneider’s Dynamic Model, but they also suggest a way of making an analysis of enregisterment processes, such as the one conducted in my study, fruitful for providing support for a model of language change that includes mechanisms of weighted interactor selection and/or replicator selection. By means of such an analysis detailed information can be obtained about which social values are attributed to linguistic forms and also to social groups linked to the use of the forms. \citet[269--270]{Baxter2009} point out that factors like the prestige of a variant are not easy to measure – this study, however, suggests a way of approaching such a measurement. By using a well-defined corpus like the databases of newspaper articles, it is possible to quantify metadiscursive activities, as I have shown for example with regard to the numbers of articles associating linguistic forms with different social groups or the numbers of articles evaluating baggage positively, negatively or neutrally. How exactly such a quantification could be carried out and integrated into such a model of language change remains a task for further research, but it would definitely be a valuable addition to the discussion of the role of social factors in language change.
