\hypertarget{Toc63021202}{}\chapter{Introduction}
\hypertarget{Toc63021203}{}
In a letter to the editor which appeared in an Arizona newspaper, the\emph{Tucson Daily Citizen}, on July 25, 1882\citesource{July251882}, a reader asks: “Will you kindly inform me what constitutes the American language?” To this question, the editor replies with only one sentence: “It is the English language with the “H’s” in their proper places”. While the American reader at the end of the nineteenth century thus seemed to be interested in how the “American language” would be defined by the editor, the question is whether the answer given by the editor should be of interest to today’s linguists who aim to study the emergence of American English as a new variety of English.


Two influential models of the process of the emergence of new varieties of English give completely different answers: According to \citegen{Trudgill2004} model of new-dialect formation, the question must be answered in the negative, while according to Schneider’s Dynamic Model, which he first presented in an article in \citeyear{Schneider2003} and elaborated on in much detail in his monograph \emph{Postcolonial English: Varieties around the world} (\citeyear{Schneider2007}), it must be answered in the affirmative. This striking difference between the answers can be explained by the fact that the two models are fundamentally different with respect to an important issue: the role of social factors, especially identity. Trudgill’s model is essentially deterministic, attributing a major role to frequency distributions of variants, while in the Dynamic Model “identity constructions and realignments, and their symbolic linguistic expression, are [...] at the heart of the process of the emergence of PCEs [Postcolonial Englishes]” \citep[28]{Schneider2007}.

This issue has caused considerable debate among linguists – a major contribution to it was \citegen{Kretzschmar2014} article, in which he integrated the Dynamic Model and his own theory of language as a complex system by arguing that they complement each other: While the \textit{internal} development of a variety is characterized by random interactions between speakers in a complex system, causing frequency distributions to change, the evolution of the \textit{perception} of these frequency distributions can be accounted for by the Dynamic Model. The problem is, however, that it is an essential claim of the Dynamic Model that perceptions of and attitudes towards language influence structural developments because it “predicts that via language attitudes a speaker’s social identity alignment will determine his or her language behavior in detail” \citep[95]{Schneider2007}. In other words, the speaker’s recognition of variants and the concept he or she has of a variety influences his or her own linguistic behavior and thus the structural developments. Given the fundamental nature of the disagreement regarding the way that the process of the emergence of new varieties of English can be modeled, it is clear that this debate needs further attention and this study aims to contribute to it in several ways.

The first aim of this study is theoretical: Based on a comparison of Trudgill’s, Schneider’s and Kretzschmar’s models, paying particular attention to how exactly they conceptualize and define the emerging construct, the \textit{variety} or \textit{dialect}, and how they argue for or against the role of social factors, I will present an argument for distinguishing between structural varieties, perceptual varieties and discursive varieties, and for investigating the emergence of these types of varieties in their own right. The focus of this study is on the discursive variety, and I suggest that its development can be studied by using a theoretical framework which originated in linguistic anthropology and redefines and modernizes the notion of \textit{register} as “a cultural model of action [...] which links speech repertoires to stereotypic indexical values”, which “is performable through utterances” and which “is recognized by a sociohistorical population” \citep[81]{Agha2007}. Registers are constructed through processes of \textit{enregisterment}, defined accordingly as “processes and practices whereby performable signs become recognized (and regrouped) as belonging to distinct, differentially valorized registers by a population” \citep[81]{Agha2007}. Studying enregisterment thus directs the focus onto the conceptual or discursive level. It is important, however, that this level is not independent from but instead in a dialectic relationship with the structural level of language use – \citet{Silverstein2003} has described this relationship in much detail by using the notion of \textit{indexical order}. Consequently, the framework is not only useful to investigate the emergence of discursive varieties, but it also provides a theoretical account of the interaction between the discursive level and the structural level. Furthermore, theories of enregisterment also discuss the role of identity in the construction of discursive varieties and I will thus explore how they add to the theoretical basis of Schneider’s Dynamic Model. The concepts of indexicality and enregisterment have already been fruitfully applied in sociolinguistics (e.g. by \citealt{Johnstone2006} and \citealt{Eckert2008}, to name two prominent examples), and they also play a role in the new field of discourse linguistics that was delineated in detail by the German linguists \citet{Spitzmuller2011}. The theoretical framework developed for this study thus integrates approaches from linguistic anthropology, sociolinguistics and discourse linguistics to develop a model of the construction of discursive varieties, which can then inform a general model of the emergence of new varieties.

The second aim is to apply the model in order to contribute to a description of the emergence of American English as a discursive variety in the nineteenth century. The reason for focusing on American English is that it is the case that is discussed in most detail by \citet{Schneider2007}. Even though it is the “best researched postcolonial variety of all”, Schneider’s account is the first one that is “theoretically informed” \citep[250]{Schneider2007}, and using enregisterment as a theoretical framework not only adds to the theoretical basis of such an account, but it also yields insights that shed further light on how American English evolved. The study is restricted to the nineteenth century because it is in this century that Schneider locates (parts of) the phase of nativization and the entire phase of endonormative orientation, claiming that increasingly positive attitudes towards the American variety also fueled its structural differentiation from British English. Other studies on the history of American English also postulate that it was in the nineteenth century that American English became recognized and/or stabilized as a variety of English (e.g. \citealt{Algeo2001a}, \citealt{Bailey2017}, \citealt{Simpson1986}). By applying the model, I aim to show how tracing enregisterment processes leads to insights about the discursive level and how these insights contribute to a theoretically informed description of the history of American English.

As the application of the model requires a methodological approach, the third aim of the study is to develop a method for studying enregisterment processes in a historical context in a systematic and goal-oriented manner. At the heart of enregisterment are reflexive activities – “activities in which communicative signs are used to typify other perceivable signs” \citep[16]{Agha2007} – because it is through these activities that speech differences become linked to social differences and that linguistic forms acquire indexical meanings. For example, a person uttering the statement \emph{Saying jolly sounds so posh} uses linguistic signs (one type of communicative sign) to create an object, the lexical item \emph{jolly}, that is typified through the assignment of the social value ‘posh’. This constitutes an instance of reflexive activity, in this case a very explicit type of reflexive activity, which is metapragmatic because it conveys information about the pragmatic effect of using the word \emph{jolly} (signaling the characteristic of being posh) and also metadiscursive because the object typified is a part of discourse, that is, of “communicative action in the medium of language” \citep[2]{Johnstone2018}. If reflexive activities, which typify linguistic forms, occur repeatedly, these typifications become social regularities and as such lead to registers as large-scale cultural formations. Reflexive activities can take many shapes and forms, but it is essential that they are by definition observable. This makes it possible for the linguist to study them systematically. In this study I propose to use a model for analysis that was developed by the above-mentioned discourse linguists \citet{Spitzmuller2011}: the discourse-linguistic multi-layer analysis (DIMLAN). The model is based on an understanding of discourse that goes beyond the sense given above: It is defined as a linguistic practice through which knowledge is negotiated and constituted (\citealt[53]{Spitzmuller2011}). The central element in this process is the \textit{statement} – it is the smallest unit of discourse. Studying discourses empirically means to study statements by taking into account three levels: the intratextual level (based on the observation that statements typically appear in the context of texts), the transtextual level (because statements and texts are essentially related to each other in discourses) and the level of actors (taking into account that transtextuality is a characteristic that is created through linguistic action by individual people, groups of people or non-personal actors like institutions or political parties). Taking the DIMLAN model as a basis, I approach the empirical study of enregisterment processes in nineteenth-century America by focusing on one metadiscursive genre: newspapers. Newspapers contain texts which contain statements about language (constituting reflexive, metadiscursive activities) and they are electronically accessible and searchable because they have been collected in two large databases, \emph{America’s Historical Newspapers} (AHN) and \emph{Nineteenth-Century U.S. Newspapers} (NCNP), which comprise close to 78 million articles. By searching these databases for specific pronunciation respellings (representing phonological forms) and for lexical items, I obtained collections of statements and texts which are related on a transtextual level (because the same linguistic forms become typified). The intratextual level was examined by analyzing each newspaper article qualitatively in order to identify the indexical links which are created between the linguistic forms and social values and personae. By comparing the articles to each other and using quantitative analyses to identify larger evaluative patterns as well as the regions and the time periods in which the respective texts were produced and received, I analyzed the transtextual level and the level of actors. I argue that these analyses provide insights not only into which linguistic forms were indexically linked to the value of ‘being American’, but also into how, when and where the process of delimiting an American register from other registers proceeded.

The structure of this book reflects these aims. In Chapter \ref{bkm:Ref523475305}, I will give an overview of the models of the emergence of new varieties of English, followed by a discussion of the definition of the notion of \textit{variety} and the role of social factors in these models. I will describe the theoretical framework of enregisterment and discuss how this framework and one of its key concepts, indexicality, have been applied and developed further in sociolinguistic research, in the field of perceptual dialectology and in discourse linguistics. Based on this discussion, I will present a model which captures the relationship between two levels that are relevant in the construction of discursive varieties, namely the structural and the discursive level. The final part of Chapter \ref{bkm:Ref523475305} contains a review of previous research on the history of American English, paying particular attention to how the starting point of the variety is determined and the role that is attributed to the discursive and the structural level in this process.

In Chapter \ref{bkm:Ref13470656}, I will describe the methodological approach of this study by explaining why I focus on metadiscursive activity in newspapers, by describing the process of data collection and by motivating my choice of linguistic forms: The phonological forms are /h/-dropping and -insertion, yod-dropping, a lengthened and backened \textsc{bath} vowel, non-rhoticity and a realization of pre-vocalic /r/ as a labiodental approximant, and the lexical forms are \emph{baggage} (in contrast to \emph{luggage}) and \emph{pants} (in contrast to \emph{trousers}). I will also give reasons as to why metadiscourses on the phonological forms will be identified by searching the databases for specific pronunciation respellings (\emph{hinglish}, \emph{noospaper/s}, \emph{dawnce}, \emph{deah}, \emph{fellah}, \emph{bettah}, and several spelling variants of \textsc{trousers}). Against the background of the distinction between the structural and the discursive level established in Chapter \ref{bkm:Ref523475305}, previous research on these linguistic forms will be reviewed.

In Chapter \ref{bkm:Ref523404731}, I will describe the results of the quantitative and qualitative analyses of the metadiscursive activities surrounding the phonological forms (\sectref{bkm:Ref13472461}) and the lexical forms (\sectref{bkm:Ref11924664}) by focusing on the construction of indexical links between the linguistic forms and social values and social personae. These results will be interpreted in Chapter \ref{bkm:Ref532290015} with regard to the question of what they reveal about the enregisterment, that is, the construction of American English as a discursive variety, in the nineteenth century. I will show that three values are of central importance in this process, ‘nationality’, ‘authenticity’ and ‘non-specificity’, and I will explore each of them in turn. Finally, in Chapter \ref{bkm:Ref13473653}, I will discuss the theoretical implications of this study for modeling the emergence of new varieties of English, for writing a theoretically informed history of the development of American English and for general theories and models of language change. The study will conclude with a summary of the central findings of the study in Chapter \ref{bkm:Ref13747659}, which show that the statement made by the nineteenth-century editor about the characteristics of the "American language" is highly relevant for giving a theoretically informed account of the emergence of American English.

