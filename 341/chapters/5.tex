\chapter[Interpretation]{Interpretation: key values and phases in the enregisterment of American English}
\label{bkm:Ref532290015}\hypertarget{Toc63021245}{}\section{Indexical values and the enregisterment of American English}
\hypertarget{Toc63021246}{}
The overarching goal of the present study is to identify enregisterment processes of American English by systematically investigating and describing the cultural construction of American English as a discursive variety. The analysis carried out in \sectref{bkm:Ref523404731} focused on newspaper articles and a specific set of five phonological and two lexical variables to answer several research questions relating to the creation of indexical links between these linguistic forms and social values and characterological figures. So far, the focus has thus been on the forms and their indexical values, which can be conceptualized as several indexical fields (in the sense of \citealt{Eckert2008}, see \sectref{bkm:Ref506884048}). The next logical step is to interpret these findings with regard to the role played by these indexical values in the construction of a register whose forms have ‘American’ as their main indexical value and which can thus be labeled “American English”. In this chapter, I argue that this construction process revolved around three central values, the nationality value, the authenticity value and the non-specificity value, and that those values affected the process in different phases and in connection with different linguistic forms.


\section[The nationality value]{The nationality value: delimiting American English against British English}
\hypertarget{Toc63021247}{}
As the central aim of the present study is to answer the question of when and how an American register was constructed or continued to be constructed in the nineteenth century, it is obvious that nationality must be a central value in the process because the adjective \emph{American} already indicates a relation to the American nation. This relation is also one of the reasons for focusing on the nineteenth century because at this time the United States of America had been formed as a political union – a nation that was politically independent from other nations. However, the question that remained open and that guided the analysis is if there were linguistic forms which came to index the quality of being ‘American’, and if so, how these indexical links were constructed. The analysis of metadiscursive activity in newspaper articles has revealed that the construction of American linguistic forms revolved centrally around their differentiation from British English forms – and that one crucial value in this process was accordingly the nationality value. The linguistic form that was most important in these metadiscourses was /h/-dropping and -insertion. While the importance of delimitation from British English has of course always played a central part in prior studies on the historical development of American English (see \sectref{bkm:Ref517077661}), the role played by /h/-dropping and -insertion in this process has not been identified so far. A reason for this is that /h/-dropping and -insertion is not a form that was in use in the United States. Research on its distribution and evaluation in metadiscursive activity has focused on England because it was in this geographical area that not only variability was found on the level of structure but where the form also played an important role in delimiting a repertoire of ‘standard’ forms from a repertoire of ‘non-standard’ forms and thus in the discursive construction of a standard British English register. However, the analysis conducted in this study shows that Americans drew precisely on these metadiscourses in England to construct their American register. Two central values in these American metadiscourses were uniformity and superiority – the complete absence of the form in America was interpreted as a marker of American uniformity, and this uniformity in turn was part of the argumentation that established linguistic superiority and connected this to a general cultural superiority.


To shed more light on the construction of American English as uniform based on the absence of /h/-dropping and -insertion, it is important to note that first of all uniformity has been a central value in standardization processes and the construction of national identities throughout Europe. In the case of French, for example, \citet[6]{Lodge2013} emphasizes:

\begin{quote}
the central role which language has played over the past two centuries in the definition of French national identity – the standardized variety of French is more than an efficient vehicle for communication across the vast length and breadth of France; it serves as a powerful symbol fostering among French people a sense of national solidarity (internal cohesion) and a feeling of their uniqueness in comparison with other nations (external distinction).
\end{quote}


Consequently, since the French Revolution, non-standard dialects have been perceived as a threat to uniformity and this even resulted in political action and the persecution of these dialects “with great ruthlessness” \citep[6]{Lodge2013}. This example shows that an ideology emphasizing inner linguistic homogeneity was important for emerging nation states to justify and legitimize their status as a political unit. It is within this ideological framework that /h/-dropping and -insertion became an ideal form for emphasizing the linguistic unity of the United States, especially in contrast to England, which was constructed as linguistically heterogeneous because of the variable use of this form. Secondly, the value of uniformity also tied in with the specific American ideology of being a nation marked by democracy and equality. In England, discourses on /h/-dropping and -insertion served to emphasize social differences between speakers: It became an index of a lower social class and a lack of education and cultivation (see \sectref{bkm:Ref13036568}). Americans used this link to construct the absence of the form as a marker of their social uniformity, in the sense that it marked their lack of emphasis on class differences and the greater level of education of \emph{all} Americans. We can thus observe the identity relations that \citet{Bucholtz2005} have called adequation and distinction (see \sectref{bkm:Ref506884048}): The linguistic and social differences between different groups within the United States were downplayed by emphasizing their similarity (indicated by /h/-retention), while their social and linguistic dissimilarity to Britain was highlighted. Despite this effort on constructing distinctiveness, discourses surrounding /h/-dropping and -insertion also show that Americans still exhibited an exonormative orientation: Their patterns of argumentation rested on the negative indexical values that the form had previously acquired in England. They could not use the form to construct an American English standard against a British English standard because in England the form was also evaluated as non-standard, but this is precisely the reason for why the analysis of enregisterment has to go beyond the distinction between non-standard and standard and look more closely at how standardness is constructed and how this interacts with the specific linguistic forms that are foregrounded. Americans used /h/-dropping and \--insertion to argue that their standard is superior because the ‘correct’ retention of /h/ was not a form used by only a small group of upper-class speakers but by all speakers who were deemed relevant for the imagination of the American nation. This social uniformity, connected to a strong belief in democracy and equality, made the American standard appear better than the English one, despite the fact that /h/-retention was in fact part of both standard registers. This reveals a clear picture of how the shift from an exonormative to an endonormative orientation proceeded in discourses on language: The linguistic norms constructed in England were taken as a basis for establishing American linguistic superiority and thus to create the self-confidence necessary for the subsequent establishment of norms that were in contrast to British norms.


The analysis of articles containing \emph{hinglish} in \sectref{bkm:Ref7775416} has shown in detail how this linguistic self-confidence was created and transmitted. Characterological figures surfacing in anecdotes and humorous paragraphs played an important role because they embodied the struggle over superiority and because they contributed to the construction of a register of British English that stood in contrast to an American one, thus emphasizing not only the linguistic but also the cultural differences between the two nations. The Englishmen appearing in these anecdotes were usually characterized as arrogant and as feeling superior to Americans, but their dropping and insertion of /h/ was used to depict them as being “in fact” vulgar and ignorant. The enregisterment of British English in America thus proceeded by linking /h/-dropping and -insertion to British people. The indexical meanings ‘vulgar’ and ‘incorrect’ had already been present in metadiscourses in England, but they were re-interpreted in America as applying to the majority of British people, thus differentiating between two British registers: on the one hand “standard British English”, which was linked to a small number of upper-class British speakers, and on the other hand a general “British English”, which was linked to the rest of the people living in Britain. This general “British English” was constructed as inferior to “American English”, and the anecdotes describing encounters between arrogant Cockney speakers complaining about American speech and ways of life and competent and relaxed American speakers who put them in their place created representative incidents and invited Americans to identify with the American figures. Interestingly, in two of the anecdotes discussed in \sectref{bkm:Ref7775416} the Americans were judges, so figures who have the power to decide on what is right or wrong, which gave the incorrectness of /h/-dropping \--insertion even more weight. The construction of the vulgar British speaker in contrast to the educated and sensible American speaker has striking parallels to the enregisterment of American English in England in the nineteenth century, as shown by \citegen{Hodson2017} analysis of British fiction from 1800 to 1836. She finds that the stereotype of the “vulgar American” was already present in the earliest novel in her study, Susan Ferrier’s \textit{The Inheritance} (1824), but that this stereotype was not linked to specific linguistic forms yet. However, in an 1853 revision of the same novel, Hodson identifies several changes that Ferrier made with regard to the direct speech of the American figure, Lewinston. She interprets this as evidence that by the middle of the nineteenth century a repertoire of linguistic forms associated with American speech had emerged. As Ferrier did not make any changes to the characterization nor added metalinguistic comments, Hodson concludes that she probably expected her readers to recognize these forms as American, which suggests the existence of an American register. The vulgarity of the American figure remained in place. Like the English figures in American newspaper articles, the American figure in Ferrier’s novel is not only depicted as vulgar and lower class, but also as arrogant and blunt, with “a firm belief in the superiority of American customs” \citep[38]{Hodson2017}. Hodson's findings illustrate the importance of interpreting enregisterment processes with respect to the sociohistorical population that is involved in these processes: In England, American English was enregistered differently than in America. They also show that the mechanisms are similar: By depicting members of the other nation as vulgar and ridiculing their arrogant belief in their superiority, the cultural and linguistic superiority of one’s own nation is constructed. It also suggests that the metadiscourses are linked: It is likely that Americans were aware of how their speech and behavior were evaluated in England and that they reacted to it by applying the same argumentation but twisting it in their favor. This view can be supported by the anecdote of the American traveler in England who is complimented on his good English. The anecdote was quite popular in America and it shows how knowledge of English perceptions of American speech came to circulate in the newly formed American nation.

The quantitative analysis of the frequency of articles containing \emph{hinglish} in \sectref{bkm:Ref6214186} showed that they appeared relatively early and they kept appearing at a relatively low but stable frequency throughout the nineteenth century, which suggests that the nationality value and the delimitation of American English as different from and, more importantly, superior to British English did not lose its importance. The presence of /h/s “in their proper places” continued to be important for the enregisterment of American English; with regard to region, it seems to have been more important in the north and northeast than in other parts of the country. In general, /h/-dropping and \--insertion was a highly salient form and in several cases it was the only one that marked the difference between British and American speakers. Even though upper-class speakers were not associated with the use of the form, the analysis has shown that the negative indexical meanings linked to the form had become so well-known that they could be used to characterize even this speaker group negatively. So while it can be argued that a general British English register was contrasted with a standard British register, the former being inferior to American English and the latter being equal and pointing to shared norms, /h/-dropping and -insertion was also used to shed a negative light on British people in general and to assert American superiority.

With regard to lexical forms, the nationality value also played a role – more so in the case of \emph{baggage} and \emph{luggage} than in the case of \emph{pants} and \emph{trousers}. The analysis in \sectref{bkm:Ref11924664} has shown that if indexical links were created, \emph{baggage} was favored over \emph{luggage} because it was considered the American form. Therefore, \emph{baggage} became enregistered as American English in contrast to \emph{luggage} becoming enregistered as British English. In this case no differentiation between a standard and a general British English was made, which provides clear evidence of an endonormative orientation: The American form was valued highly even though it contrasted with a form that was considered standard in Britain and that was not evaluated as vulgar and incorrect like /h/-dropping and -insertion was. It needs to be noted that in contrast to /h/-dropping and -insertion, these metadiscourses occurred primarily in the last two decades of the nineteenth century.

However, the situation was more complex for \emph{pants} and \emph{trousers}. Although indexical links between \emph{trousers} and British speakers could be identified, \emph{trousers} is also constructed as the ‘proper’ American form in several articles. These are signs of conflict with regard to which form should be favored. Although the nationality value features importantly in some articles arguing for \emph{pants} being the democratic American term since it was used by the majority of speakers in the American nation, it is not at all present in others, where \emph{trousers} is associated with correctness, elegance and cultivation and not with a British nationality. I argue that \emph{pants} and \emph{trousers} present a case where another value comes into play, adding to and interacting with the nationality value: the authenticity value.

\section[The authenticity value]{The authenticity value: delimiting an authentic American English against an inauthentic American English}
\label{bkm:Ref2158524}\hypertarget{Toc63021248}{}
Authenticity was a crucial value in the enregisterment of American English because it helped speakers to make sense of linguistic differences that occurred \emph{within} America, that is, differences between \emph{American} speakers, which could thus not be interpreted as linking speakers to different nationalities (British vs. American). As pointed out in \sectref{bkm:Ref506884048}, in recent research in sociolinguistics and linguistic anthropology, authenticity is not viewed as a quality that is inherent in language itself, but as a claim that speakers make – a claim that sets a process of authentication in motion and which is often accompanied by a process of denaturalization. In order to construct speech forms or other practices as authentic, differing speech forms or practices are constructed as unnatural and not genuine. I argue here that authenticity played a central role in enregistering a set of linguistic forms, among them a back \textsc{bath} vowel, non-rhoticity, a labiodental realization of /r/ and the use of the lexical item \emph{trousers}, as \emph{non}{}-American – not because the users of the forms were not Americans, but because they were constructed as not being \emph{authentic} Americans. In the same process, the alternative variants were thus enregistered as genuine American English forms, used by ‘true’ and ‘real’ American speakers. The analysis shows that characterological figures played a key role in this process. The negotiation of what is authentically American revolved around the opposition between the American dude on the one hand and the American cowboy, hunter and farmer on the other hand. These figures were essential in linking authenticity (or lack thereof) in language with authenticity in other practices, a connection which is also regarded by \citet[109]{Johnstone2014} as being always present in authentication processes. The linguistic forms used by the dude were constructed as a register which was indirectly also linked to the evolving British standard register. The forms used by the dude were evaluated as being ‘fashionable’ in England among educated upper-class speakers, which constitutes evidence of an enregisterment of these forms as “standard British English” in England and a recognition of this process in America. Some articles describe how the indexical values that the forms had acquired in England, i.e. signaling education, cultivation and membership in upper classes, also became relevant in America. These observations tie in with linguistic theories which assume that non-rhoticity was adopted in eastern parts of the United States during the nineteenth century because Americans imitated British speech. The newspaper articles show that this view had already been put forward and transmitted by contemporary observers. They also confirm \citegen[288--289]{Schneider2007} point that even during the endonormative phase exonormative tendencies were still present. However, the dude figure is used to exploit precisely this link to British English speech in order to construct the use of these forms as inauthentic. By parodying the way that men in eastern urban centers of the United States imitated not only fashionable British English forms of speech but also other fashionable practices, like dressing according to English fashion (especially by wearing specific styles of trousers and accessories like the eyeglass and the cane), attention was drawn to the unnatural and affected nature of this imitation – a denaturalizing process that aimed at denying speakers who use this form their ‘true’ American identity.


The dude thus constitutes an embodied metapragmatic stereotype of a speaker whose orientation towards British English norms resulted in a lack of authenticity as an American. The contrasting characterological figures embodied precisely this authenticity as Americans that the dude lacked. Even though the cowboy, the hunter and the farmer were portrayed as uncivilized and uneducated, they also possessed a great amount of practical knowledge based on experience, they had adopted a pragmatic stance towards life, and their strength, toughness and self-reliance made them independent of external (British) influences and fashions. The speech of these ‘genuine’ Americans was marked by several linguistic forms which established a contrast to the speech of the narrators and the linguistic forms present in other newspaper articles, for example negation with \emph{ain’t}, alveolar -\emph{ing}, a lower \textsc{kit} vowel in \emph{if} and hyper-rhoticity. Even though not directly visible in the spelling, the absence of representations of non-rhoticity, back \textsc{bath} vowels, and labiodental realizations of /r/ signaled to the reader that their speech was rhotic, that their \textsc{bath} vowel was front and that their /r/ had an alveolar or retroflex realization. In terms of enregisterment, this means that these figures contributed to the enregisterment of a rural northern and (mid)western American English which was constructed through negative values like lack of civilization and education, but also through positive values like self-reliance, pragmatism, strength and, most importantly, authenticity.

In \sectref{bkm:Ref506884048}, I pointed out that studies on enregisterment have often focused on the enregisterment of regional or social accents and that authenticity was an important value in this process. In Pittsburgh, for example, the revalorization of linguistic forms as positive indexes of place was linked to people expressing their pride in being authentic Pittsburghers. The present study has shown that social and regional aspects connected to authenticity in different ways in nineteenth-century enregisterment processes of American English. The dude was a figure that was linked to northeastern cities, in particular New York City, while the figures of the American cowboy, hunter and farmer were associated with rural areas in the north and the west. The analysis of newspaper articles suggests that metadiscourses which were circulating mostly in the northeast (but also spread to urban centers in the west) and which evaluated non-rhoticity, back \textsc{bath} and labiodental /r/ and \emph{trousers} as signs of cultivation, education, elegance and a belonging to higher social circles were countered by metadiscourses that relied chiefly on authenticity and indirectly also on the nationality value in order to link the alternative variants to a real and genuine American identity.

For phonological forms, the discourses which constructed a positive evaluation of the variants analyzed here can be accessed only indirectly because these variants were criticized and ridiculed in the articles, but for the lexical form \emph{trousers} these competing metadiscourses could be traced in more detail. For example, in the advertisement by the company Humphrey’s, published in the \emph{St. Louis Globe-Democrat} January 28, 1887\citesource{FWHumphrey&CoJanuary281887}, the “uncultured Chicagoan” was described as rejecting the use of \emph{trousers}, which provides evidence of a link between the form \emph{pants} and midwestern cities. However, the adjective \emph{uncultured} also reveals the existence of a link between \emph{pants} and a lack of cultivation. Conceptually, the midwestern cities were midway on the continuum between urban northeastern centers and rural midwestern areas; they were geographically located in the midwest, but they also constituted urban areas. The advertisement solved the dilemma of choosing a lexical variant by leaving the choice up to the individual. The metadiscourses analyzed in this study reveal the cultural models available for motivating such a choice: one emphasizing the value of cultivation and the other emphasizing the value of authenticity. That the latter was endonormative by establishing a model of authentic American linguistic behavior probably contributed to its later success. Articles constructing the American forms as authentic often pointed out the exonormative orientation of the competing model and the social pressure that contributed to its success in the northeast, for example by contrasting the young Bostonian’s wish to use \emph{pants} (because this is the form he would normally use) with the elite’s prescription to use \emph{trousers} (which motivates his mother to forbid the boy to use \emph{pants}). This indicates that \citegen{Bucholtz2005} processes of authorization and illegitimization were at play as well – linguistic authority exerted by a social elite was challenged.

The regional distribution of articles containing \emph{dawnce}, \emph{deah} AND \emph{fellah}, \emph{bettah} and \textsc{twousers} described in \sectref{bkm:Ref6214186} has shown that metadiscourses surrounding these forms circulated most prominently in the (mid)west and the north. It was thus particularly in these regions that the stereotypes of northeastern men, embodied in the dude figure, circulated: men who are unintelligent, lazy, living off inherited wealth and spending their time buying and showing off fashionable clothes and accessories and trying to impress women. However, several articles, especially those aiming at entertaining the readers, were taken from magazines, which were often published in the northeast and had a nationwide readership, which suggests that such discourses were also present the northeast. A further indicator of this presence is the ambivalence that \citet[48]{Bonfiglio2002} identifies in Grandgent’s comments on non-rhoticity in 1899 (see \sectref{bkm:Ref517077661}). Grandgent associated non-rhoticity with cultivation but also with decadence and lack of vitality, so the positively evaluated energy and strength of the rhotic hard-working American farmer or cowboy was also found in expert discourses on language, which were still dominated by the northeastern elite.

The frequency at which articles containing \emph{dawnce}, \emph{deah} AND \emph{fellah}, \emph{bettah} and \textsc{twousers} appeared in newspapers in the course of the nineteenth century clearly reveals that these metadiscourses rose to prominence in the 1880s. Until the 1870s only very few articles containing these features could be found, and none were published before 1840. So claims to authenticity, which were in opposition to constructions of a standard American register based on British ‘fashionable’ forms, were developed particularly towards the end of the nineteenth century, and, as Grandgent’s comments in the last year of the nineteenth century suggest, continued well into the twentieth century. This confirms and adds to \citegen[288]{Schneider2007} observation that the purist movement started after the Civil War and rose to prominence in the 1870s and 1880s. The present study shows, however, that these purist discourses favoring British variants, which were in the process of becoming constructed as ‘standard’, ‘cultivated’ and ‘upper class’ in England, were countered by constructing the alternative variants as being part of an \emph{authentic} American register.

Finally, it is important to note that social personae and characterological figures which played a prominent role in the metadiscourses described in Chapter \ref{bkm:Ref523404731} were mostly male. However, the few cases where female figures were used are particularly interesting. The anecdote about salesladies working in the department store in Denver, Colorado, not only shows that metadiscourses indexically linking non-rhoticity, back \textsc{bath} and labiodental /r/ to upper-class membership, education and cultivation had reached western urban centers, but it also indicates that the effect of these metadiscourses were criticized and ridiculed and that this criticism was not directed at the members of the upper classes themselves, but primarily at the members of the lower middle classes who attempted to imitate upper-class speech to advance socially. The salesladies lacked authenticity and were thus in stark contrast to the proud, pragmatic, intelligent and down-to-earth Montana girl, who was presented as superior to the dim-witted and affected New York dude. The encounter between the Montana girl and the New York dude is also a good example to illustrate the role attributed to masculinity and femininity in these discourses: Particularly in the case of non-rhoticity, the weakening or loss of /r/ was linked to femininity and physical weakness, so that even male figures like the dude were portrayed as effeminate, whereas the rhoticity of the cowboy was associated with masculine characteristics and gave figures like the Montana girl a masculine touch. This impression of masculinity was often increased by adding hyper-rhoticity to the register of the (mid)western and northern authentic American. The enregisterment of hyper-rhoticity in combination with other forms like for example \emph{ain’t} also indicates, however, that the cowboy and his speech were not constructed as models for \emph{all} Americans despite their authentication as genuine Americans. This reveals a third value that came into play and was indispensable for creating a register that came to be recognized by all as ‘American’: non-specificity.

\section[The non-specificity value]{The non-specificity value: delimiting American English against more specific regional and social American Englishes}
\hypertarget{Toc63021249}{}
What I regard as the non-specificity value can be illustrated well by using the following quote by \citet[ix]{Krapp1919}, who explains his understanding of the term \textit{standard speech} at the beginning of the twentieth century:


\begin{quote}
The term standard speech, it will thus be seen, has been used by the author without a very exact definition. Everybody knows that there is no type of speech uniform and accepted in practice by all persons in America. What the author has called standard may perhaps be best defined negatively, as \textbf{the speech which is least likely to attract attention to itself as being peculiar to any class or locality}. As a matter of fact, speech does not often attract notice to itself unless it is markedly peculiar. For the most part, when one is listening to the speech of others, one is intent upon getting the meaning, not upon observing the form. In consequence there is likely to be, even in what we may justly call standard speech, a considerable area of negligible variation, negligible, that is, from the point of view of the practical use of language. To the conscientious and critical listener, many of these variations may seem reprehensible, but only so by the test of some theoretical or ideal standard. [emphasis mine]
\end{quote}


First of all, Krapp distinguishes between a standard “in practice” and a “theoretical or ideal standard” in order to justify the presence of variation in speech that would be evaluated as ‘standard’ by many Americans despite the experts’ demand for uniformity. Secondly, he finds that those linguistic forms are evaluated as ‘standard’ which are not “peculiar to any class or locality”, so in other words, linguistic forms which are non-specific. I argue that Krapp’s distinction between an “ideal” and a “real” standard can be captured within the theoretical framework of enregisterment by separating expert metadiscourses from non-expert metadiscourses. While the uniformity value plays a great role for experts who write dictionaries, grammars and other prescriptive texts aiming at codifying a standard variety comprising a large number of forms, I have shown above that in newspaper discourses the uniformity value mainly played a role in relation to \emph{one} form: /h/-dropping and -insertion. Non-specificity, by contrast, was an important value in newspaper discourses with regard to almost all linguistic forms analyzed in this study. In order to answer the question of which forms became enregistered as ‘American’ for \emph{all} Americans, one must address the question of which forms were constructed as belonging to ‘specific’ American registers and thus as being too specific to become part of a cultural model that all Americans could potentially orientate towards.


Non-rhoticity constitutes the prime example to illustrate the role of the non-specificity value. The absence of post-vocalic /r/ became part of several specific registers: not only of the register of the urban northeastern higher social classes, embodied most prominently by the dude figure, which is why I will refer to it as the “dude register” in the following discussion, but also of a “southern American register”, of “a mountaineer register” and of a “Black American register”. The analysis of the frequency of indexical links between \emph{bettah} and social personae has revealed that the association between non-rhoticity and Black speech dominated in quantitative terms, which indicates that people frequently came into contact with this link. However, as non-rhoticity occurred in combination with a large number of other linguistic forms to index Black speakers, not only phonological, but also grammatical, lexical and pragmatic ones, it was less salient as a shibboleth of a Black American English register than as a shibboleth of a dude register, which consisted of a very restricted set of forms, causing non-rhoticity to stand out more. The southern American English register and the mountaineer register occupied the middle ground with regard to the salience of non-rhoticity: They also comprised a larger number of forms, including grammatical ones, which were completely absent in the dude register, but usually not as many as the Black American register. This has important implications for the non-specificity value: The Black American register is most specific in the sense that it is constructed as comprising the largest number of forms which are specifically (though in many cases not exclusively) linked to it in newspaper articles, whereas the dude register is constructed as least specific because it contains only a small number of forms specific to the register. This in turn had the consequence that the Black American register was constructed as most deviant and least ‘normal’, whereas the dude register, despite its few salient differences, appeared least deviant. In the case of phonological forms, specificity is marked through spelling, whereas in the case of grammatical forms, specificity is marked through the contrast to other speakers and especially to the majority of other newspaper articles. I argue that the register that was constructed as generally American and thus non-specific consisted of forms that were \emph{not} constructed as shibboleths of specific social or regional registers. Non-rhoticity was part of several specific registers, with different degrees of salience. In contrast to that, hyper-rhoticity became enregistered as part of the rural (mid)western register, embodied for example by the cowboy. Against this, rhoticity was  constructed as non-specific – as the form used by authentic Americans who are in no way “peculiar”, to use Krapp’s wording again.

Yod-dropping presents an interesting case because even though the yod-less pronunciation is marked by a specific spelling, thus creating the basis for linking this specific pronunciation to a specific group or locale, the analysis revealed that the form has come to index rather general social characteristics like uneducatedness and lack of cultivation. Even though it was described in articles containing explicit metadiscursive comments as a form occurring in the north and west (and not in the south), it also appeared in representations of direct speech of southerners throughout the nineteenth century, which underlines the non-specificity of the form with regard to region. Nevertheless, in those articles where it was presented as a “fault” of the north and west, one which was not found in the south, this again constructs the south as a specific region that is indexically linked to linguistic forms, in this case to yod-\emph{retention}. In fact, it could be argued that the continuing enregisterment of yod-retention as southern, which is visible in some articles, makes yod-retention too specific to become part of a non-specific American English register. I analyzed one article which provides evidence that yod-retention was increasingly evaluated negatively by linking it to pedantry towards the end of the nineteenth century, which indicates that the indexical links to negative social characteristics weakened and thus the non-specificity of yod-dropping increased. In general, the occurrence of articles containing \emph{noospaper/s} throughout the nineteenth century, without great changes in frequency except for the 1860s, when they appeared frequently in the Nasby letters, shows that yod-dropping had been part of metadiscursive activity in newspaper articles for a long time, but it was not very salient and it was not involved in any claims to nationality or authenticity. However, based on its (increasing) lack of specificity, it became enregistered as part of a non-specific general American English register, while yod-retention was enregistered mainly as (specifically) southern.

The importance of specific registers for the discursive construction of a standard register as the register of an imagined national public is also highlighted by \citet{Frekko2009} who shows convincingly that in the case of Catalonia today, the national public is imagined as fragile precisely because such specific registers do not exist. She finds that an imagined uniformity is as important as an imagined diversity, at different recursive levels:

\begin{quote}
At one taxonomic level, registers are erased in order for one register imagined as standard and homogeneous to count as the named language in contrast with other named national languages. At a lower recursive level, these registers must be imagined to exist in order for the language and its corresponding national public to be able to account for “everyone” in the projected national public. \citep[71]{Frekko2009}
\end{quote}


This can be transferred to the case of nineteenth-century American English and it confirms claims and observations made by \citet{Cooley1992} and \citet{Minnick2010} (see \sectref{bkm:Ref517077661}). In the process of being contrasted with British English, internal differences were erased in metadiscursive activity and particularly the retention of /h/ served as a marker of this uniformity. Yet, linguistic differences were imagined at the same time because the existence of sets of forms to index specific (groups of) speakers was indispensable to the existence of a set of forms that could be imagined as indexing all Americans.


\section{Conclusion: the emerging American English register}
\hypertarget{Toc63021250}{}
To conclude the interpretation of the detailed analyses carried out in Chapter \ref{bkm:Ref523404731}, I argue that the enregisterment of American English was based on three main values: nationality, authenticity and non-specificity. Already fairly early in the nineteenth century, nationality was connected to discourses on linguistic uniformity and superiority, which focused on /h/-dropping and -insertion to enregister a national American English in contrast to British English and establish the superiority of the former against British metadiscourses enregistering American English as vulgar and incorrect in the first half of the nineteenth century. By contrast, authenticity came to play a major role towards the end of the century, particularly in the last two decades, and was used to enregister forms like non-rhoticity, a back \textsc{bath} vowel, a labiodental realization of /r/ and the lexical items \emph{luggage} and \emph{trousers} as a northeastern social register through evaluating them as affected and unnatural. Even though the focus thus shifted from delimiting British from American English to delimiting authentic from non-authentic American English, British English still played an indirect role in the latter process because what made forms used by northeastern Americans non-authentic was the fact that these forms were fashionable in England and thus indexical of a (changing) standard British English. However, these metadiscourses also indirectly provide evidence of a positive evaluation of these forms at least for some speakers, which are based on positive evaluations in England, where they are seen as indexing cultivation, education, elegance and a belonging to higher social circles. In this study, these competing evaluations in metadiscursive activity could be illustrated most clearly through the analysis of \emph{pants} and \emph{trousers}. They show that the negotiation process of which forms became enregistered as American for all Americans did not end in the nineteenth century but continued in the twentieth century. Finally, the construction of further specific American registers was important for the enregisterment of a general, non-specific American English. An important register was the southern American English register because it was not only linked to a region but also to social characteristics, specifically to white southerners. However, with regard to linguistic forms, it was marked by a considerable overlap to the Black American register, a social register based on ethnicity. Social and regional values also overlapped with regard to the mountaineer register. Speech forms, among them non-rhoticity, were linked to uncultivated, uneducated, non-religious, immoral and violent speakers living in the southern mountain regions. The non-authentic dude register was of course also a specific register that served to delimit non-specific and authentic American forms against specific and non-authentic ones, which shows that all three values, nationality, authenticity and non-specificity, interacted in the process of enregistering American English. As in the case of the dude register, the most important period for the construction of the other specific registers in newspaper discourses were the last two decades of the nineteenth century. Only yod-dropping, represented by \emph{noospaper/s}, appeared throughout the nineteenth century, with particularly high frequencies in the 1860s, and these articles provide evidence of an earlier delimitation of a register which, although not very specific, differentiated educated, hard-working and well-mannered Americans from uneducated, uncivilized and lazy Americans, exemplified by David Ross Locke’s Nasby figure.


All in all, the results of the analysis of metadiscursive practices in nineteenth-century American newspapers show that the identity relations described by \citet{Bucholtz2005} play an important role in the construction of a register that had ‘American’ as its main indexical value. Adequation and distinction figured prominently in the emergence of a model of \textit{American} linguistic and social action, not only through the emphasis on national and linguistic uniformity in relation to British heterogeneity, but also through the distinction of a general, non-specific American register from several more specific registers within America. These processes of adequation and distinction were tied to processes of authentication and denaturalization because ‘true’ and ‘genuine’ Americans were constructed as avoiding linguistic forms indexing British speech. Finally, processes of authorization and illegitimization also became visible through the competing discourses on which forms should be used in an American context and on who had the power to exert linguistic authority (either directly through the prescription of forms or indirectly through the use of the forms): the northeastern social elite or the western masses of ‘average’ Americans. As both of these groups were conceptualized as consisting of white and predominantly male speakers, it becomes clear that other groups were excluded in such negotiations of authority: Blacks and mountaineers, both constructed as ethnically other, were completely absent, and white southerners and women in general only rarely played a role, which shows that they were illegitimized and that the variants that they used and that were in contrast with variants used by white, non-southern males, were not even considered to become part of a national \textit{American} register. The implications of these results on theories of the emergence of new varieties of English and other theoretical discussions will be discussed in the following chapter, in which I will also address the limitations of this study.

