\chapter{The emergence of American English: theories, descriptions, and models}
\label{bkm:Ref523475305}\hypertarget{Toc63021204}{}
This chapter lays the theoretical groundwork for the present study. Section \ref{bkm:Ref522870687} presents an overview and a discussion of theories of the emergence of new varieties of English and the way that the emergence of American English is described in these frameworks. This leads to the development of the essential argument of the study in \sectref{bkm:Ref522870698}, namely that the concept of \textit{enregisterment} provides an important perspective on the emergence of new varieties and should be incorporated in the theoretical modeling of the process as well as used to complement methodological approaches to studying it. A central aspect of the argument is the distinction between three types of varieties, namely structural varieties, perceptual varieties and discursive varieties, and in \sectref{bkm:Ref523897668}, I develop a model which illustrates the difference as well as the relationship between structural and discursive varieties based on the framework of enregisterment. Section \ref{bkm:Ref517077661} directs the focus onto the history of American English again by discussing how the development of the variety has been described in works without an underlying theory of the process of emergence and which role is assigned to the structural and the discursive level in these descriptions.


\section{Theories of the emergence of American English as a new variety of English}
\label{bkm:Ref522870687}\hypertarget{Toc63021205}{}
There are several works which have been written on the history of American English during the last 100 years. In general, according to \citet[250]{Schneider2007}, American English is the “best researched postcolonial variety of all”. However, \citet[250]{Schneider2007} also notes a lack of a “theoretically informed history of the language” and he addresses the need for such a history by describing the emergence of American English within the framework of his Dynamic Model. This study continues this line of research, which is why I focus on describing and discussing mainly three theories and models of the emergence of new varieties and the way that they describe this process in the case of American English. In \sectref{bkm:Ref524246106}, I compare and contrast \citegen{Trudgill2004} theory of new-dialect formation, \citegen{Schneider2007} Dynamic Model, which draws heavily on \citegen{Mufwene2001} theory of the “ecology” of language evolution, and Kretzschmar’s view of the emergence of new varieties within his theory of language as a complex system presented in \citet{Kretzschmar2014, Kretzschmar2015, Kretzschmar2015b}. Based on this overview, I will discuss two issues in more detail which are crucial to the debate: the definition of the term \textit{variety} (\sectref{bkm:Ref521000690}) and the role of social factors in the emergence of new varieties (\sectref{bkm:Ref521576818}). In \sectref{bkm:Ref527366308} I finally discuss important consequences of the theoretical debate which are the foundation for my subsequent argument that studying the emergence of new varieties of English needs a careful distinction between structure and discourse in order to be able to investigate the relationship between these two dimensions.


\subsection{Overview of theories and models of the emergence of new varieties of English}
\label{bkm:Ref524246106}\hypertarget{Toc63021206}{}\label{bkm:Ref525715631}
\citegen{Trudgill2004} theory of new-dialect formation and \citegen{Schneider2007} Dynamic Model have in common that they both identify a set of stages or phases that underlie the emergence or formation of new dialects or varieties. Trudgill, who develops his theory mainly based on his study of the history of New Zealand English, argues that “in tabula rasa colonial situations, dialect mixture and new-dialect formation are not haphazard processes” (\citeyear[26]{Trudgill2004}) but a predictable development proceeding from stage 1, which he labels “rudimentary levelling and interdialectal development” to stage 2, labeled “variability and apparent levelling in new-dialect formation” and finally to stage 3 “determinism in new-dialect formation”. In the initial stage, adult speakers of different dialects come into contact in a new place. The communication between these speakers can lead to the leveling of minority and very localized variants because, as Trudgill suggests, speakers need to make themselves understood and the use of linguistic forms unknown to a majority of speakers can inhibit this aim. Additional reasons for this rudimentary leveling can also be that speakers accommodate to particularly salient forms or that they react to normative attitudes which speakers have brought with them from their home country (\citeyear[89-93]{Trudgill2004}). Trudgill regards this stage as the least important one because in his view, “adults are only capable of limited amounts of accommodation” and it is not them but children who are largely responsible for the formation of a new dialect (\citeyear[94]{Trudgill2004}). It is not accommodation but language acquisition which leads to the apparent leveling in stage 2. Children do not notice low frequency forms (below a threshold of roughly 10\%) and do not acquire them, thereby reducing the number of forms which are available to the next generation to a considerable extent. This second generation of children plays a crucial role in stage 3, because from an already reduced number of forms they now select the most common ones. The result is a “final, stable, relatively uniform outcome” in the form of “a stable, crystallized variety” (\citeyear[113]{Trudgill2004}). The final stage is completed by a process called \textit{focusing}, first described and labeled by \citet{LePage1985} and defined by Trudgill as “the process by which the new variety acquires norms and stability” (\citeyear[88]{Trudgill2004}). However, it is important that focusing only occurs when a stable set of forms has emerged through stages 2 and 3, which is why the process of new-dialect formation is essentially deterministic, the new dialect being “a \emph{statistical composite} of the dialect mixture” (\citeyear[123]{Trudgill2004}).


Because the theory is mainly based on the case of New Zealand English, Trudgill does not describe or analyze the emergence of American English using his theoretical framework in any detail. He notes several times that it is harder to study the formation of American English than that of New Zealand English because the mixture processes underlying the formation of the variety took place such a long time ago \citep[2]{Trudgill2004}. In general, he argues that American English went through the same stages as all other colonial dialects, and he notes only one possible difference, namely that comprehensibility played perhaps a more important role in the rudimentary leveling of the first stage because the traditional dialects spoken by the settlers were more different from one another than those in colonies which were settled later. Although his study does not include American English in the analysis, it is therefore still relevant because his theoretical claims apply to American English as well and it is one of the goals of the present study to discuss their validity.

The second model, \citegen{Schneider2007} Dynamic Model, also assumes that the emergence of new varieties of English is characterized by “a uniform underlying process [which] has been effective in all these [contact] situations and explains a wide range of parallel phenomena from one variety to another” \citep[4]{Schneider2007}. However, the phases which he identifies and the mechanisms operating in the process are in many respects different from Trudgill’s. \tabref{tab:2:1} summarizes the key parameters of the different phases and shows that Schneider does not view dialect formation as a deterministic process, but that linguistic effects constitute only one parameter in his model. They result from sociolinguistic conditions, which are a consequence of speakers’ identity constructions, which are in turn caused by the historical and political context.\footnote{\citet{Schneider2007} distinguishes between two speech communities in his model: the Settlers speech community (STL) and the Indigenous speech community (IDG).} Accordingly, \citet[30]{Schneider2007} speaks of a “monodirectional, causal relationship” operating between the parameters. While his model predicts that all post-colonial varieties go through all of these phases, he does not claim that he is able to predict the precise linguistic forms of the new repertoire. His model is not deterministic but explicitly dynamic. Consequently, Schneider does not use the term \textit{formation} to describe the emergence of new varieties but \textit{evolution}, and he aligns himself explicitly with theories of language evolution, particularly with \citegen{Mufwene2001} \textit{feature-pool model}. This model postulates that in a contact situation all linguistic features produced by the speakers are in competition (in a “pool”) and speakers select from this pool. Their choices are influenced by the “ecology” of the contact situation comprising linguistic as well as non-linguistic factors, such as the demographic and political situation and social factors, particularly identity constructions (which identity speakers want to express) and role alignments (which other speakers they want to align with). In Schneider’s Dynamic Model, “identity constructions and realignments, and their symbolic linguistic expression, are also at the heart of the process of the emergence of PCEs [Postcolonial Englishes]” \citep[28]{Schneider2007}. The mechanism that operates in this process is accommodation \citep{Giles1984}, but in contrast to Trudgill, who regards accommodation as an “automatic consequence of interaction” which is “not necessarily driven by social factors such as prestige or identity” \citep[28]{Trudgill2004}, Schneider emphasizes the social nature of the process:

\begin{quote}
Speakers who wish to signal a social bond between themselves will minimize existing linguistic differences as a direct reflection of social proximity: they will tend to pick up forms used by the communication partner to increase the set of shared features and to avoid forms which they realize are not used by their partner and might thus function as a linguistic separator. \citep[27]{Schneider2007}
\end{quote}

This difference is indicative of the fact that the role of social factors is a matter of considerable debate in theories of the emergence of new varieties and it will therefore be discussed in more detail in \sectref{bkm:Ref521576818}.

\begin{sidewaystable}
\scriptsize
\begin{tabularx}{\textwidth}{L{2cm}L{3.5cm}L{3cm}L{4cm}L{4cm}}

\lsptoprule
Phase & History and politics & Identity construction & Sociolinguistics of contact/ use/ attitudes & Linguistic developments/ structural effects\\
\midrule
1: Foundation & STL: colonial expansion: trade, military outposts, missionary activities, emigration/ settlement

IDG: occupation, loss/ sharing of territory, trade & STL: part of original nation

IDG: indigenous & STL: cross-dialectal contact, limited exposure to local languages

IDG: minority bilingualism (acquisition of English) & STL: koinéization; toponymic borrowing; incipient pidginization (in trade colonies)\\
2: Exonormative stabilization & stable colonial status; English established as language of administration, law, (higher) education, … & STL: outpost of original nation, “British-plus-local”

IDG: individually “local-plus British” & STL: acceptance of original norm; expanding contact

IDG: spreading (elite) bilingualism & lexical borrowing (esp. fauna and flora, cultural terms); “-isms”; pidginization/creolization (in trade/plantation colonies)\\
3: Nativization & weakening ties; often political independence but remaining cultural association & STL: permanent resident of British origin

IDG: permanent resident of indigenous origin & widespread and regular contacts, accommodation

IDG: common bilingualism, toward language shift, L1 speakers of local English

STL: sociolinguistic cleavage between innovative speakers (adopting IDG forms) and conservative speakers (upholding external norm; “complaint tradition”) & heavy lexical borrowing;

IDG: phonological innovations (“accent”, possibly due to transfer); structural nativization, spreading from IDG to STL: innovations at lexis – grammar interface (verb complementation, prepositional usage, constructions with certain words/word classes), lexical productivity (compounds, derivation, phrases, semantic shifts); code-mixing (as identity carrier)\\
4: Endonormative stabilization & post-independence, self-dependence (possibly after “Event X”) & (member of) new nation, territory-based, increasingly pan-ethnic & acceptance of local norm (as identity carrier), positive attitude to it; (residual conservatism); literary creativity in new variety & stabilization of new variety, emphasis on homogeneity, codification: dictionary writing, grammatical description\\
5: Differentiation & stable young nation, internal sociopolitical differentiation & group-specific (as part of overarching new national identity) & network construction (increasingly dense group-internal interactions) & dialect birth: group-specific (ethnic, regional social) varieties emerge (as L1 or L2)\\
\lspbottomrule
\end{tabularx}
\caption{
Developmental phases of Schneider’s Dynamic Model (from \citealt[56]{Schneider2007})
}
\label{tab:2:1}
\end{sidewaystable}

With regard to the case of American English, Schneider considers it to be “an almost unique opportunity to observe the entire developmental cycle in hindsight” (\citeyear[251]{Schneider2007}) because it is the oldest and also the best researched variety of all postcolonial varieties. I provide a brief summary of his analysis here by focusing especially on those aspects which illustrate his central thesis that social factors, especially identity constructions, have linguistic effects. In the first phase (roughly from 1587 to 1670), evidence of identity constructions is scarce, but Schneider finds it very likely that the early settlers still perceived themselves as Englishmen \citep[258]{Schneider2007}. Consequently, it is the degree of mixture of speakers coming from different regions and speaking different dialects which has the most effect on linguistic developments. Regions in which the population mixture was highest (as in the case of the Quakers in Pennsylvania) exhibited the highest degree of koinéization, defined by \citet[35]{Schneider2007} as the “emergence of a relatively homogeneous “middle-of-the-road variety”” based on a process in which “speakers […] mutually adjust their pronunciation and lexical usage to facilitate understanding”. In regions with culturally and linguistically more homogeneous settler groups (particularly New England, the South and the Appalachian Mountains), less koinéization occurred. This explains the present-day situation, in which the most distinctive dialects are found in the South and in the East, while the mainstream American variety is located in the Midland, the West and the North \citep[261--262]{Schneider2007}. At the same time, \citet[262]{Schneider2007} argues that social similarities between the settler groups in New England and in tidewater Virginia, namely their middle-to-upper-class background and close ties to the home country, were responsible for the fact that southern and eastern dialects share a number of features, for example lack of rhoticity, yod-dropping and lexical forms like \emph{piazza} ‘veranda’.


In the second phase (ca. 1670-1773), \citet[265]{Schneider2007} distinguishes two English-speaking groups with different identity constructions. One group was of higher social status and lived on the coast and the other group was of lower social status and lived in more inland regions. While the first group still firmly identified with England, the second one adopted an “English-plus” colonial identity, which was influenced by their more frequent contact with other cultural groups and their “frontier experience” \citep[265]{Schneider2007}. Additionally, Schneider argues that non-English speaking groups had a split identity because even though they wanted to adjust in America and leave problems in their home countries behind, they also wanted to retain their cultural and linguistic heritage (\citeyear[266]{Schneider2007}). And lastly, African groups were torn between forces to adjust and the desire to resist these forces and maintain their cultural identity. This combination of identity constructions resulted on the one hand in a stable exonormative orientation (as predicted by the Dynamic Model) among the high-status social group and on the other hand in a bilingualism or multilingualism among groups coming from non-English-speaking countries and in a variable sociolinguistic situation for African Americans who sometimes had extensive contacts with white speakers of English and sometimes primarily intra-ethnic contact without an opportunity to acquire English (\citeyear[266--269]{Schneider2007}). In terms of linguistic effects, Schneider (\citeyear[269--273]{Schneider2007}) finds a high degree of linguistic homogeneity and lexical borrowings as well as innovations (“Americanisms”). With regard to homogeneity, he notes that this is of course not to be seen in absolute terms, i.e. the complete absence of variability. It is rather the case that leveling processes took place as the result of mixing and koinéization, but “cultural and linguistic peculiarities” were retained as well, and English innovations were adopted as a result of the prevailing exonormative orientation. An important example is rhoticity:

\begin{quote}
The most interesting case in point is postvocalic /r/, a sound which was pronounced even in southern British English well into the eighteenth century and disappeared only then, as in modern RP. In other words, the r-lessness of New England and the South must have developed in America, modeling English linguistic fashion – a strong indication of the exonormative linguistic orientation of colonial America. (\citeyear[271]{Schneider2007})
\end{quote}


The term \textit{Americanism}, coined by John Witherspoon in 1781 (see \citealt[272]{Schneider2007}), is another indicator of the exonormative orientation because the lexical items labeled as such were usually evaluated negatively because they did not conform to a British norm. With regard to African American English Schneider finds that there is no evidence of it in the colonial period, although he also notes that “there was room for the development and retention of ethnic speech markers, for the development of linguistic means to signal a non-white, and possibly subliminally counter-European, ethnolinguistic identity” (\citeyear[268]{Schneider2007}).


The crucial phase in the emergence of American English is the third phase, the nativization phase, which \citet[273]{Schneider2007} dates from ca. 1773 to 1828/1848. According to him, the “birth of American English as a concept and as a variety falls into that period” (\citeyear[276]{Schneider2007}) – as an effect of the political independence which was accompanied by changing identity constructions. An American nationalism replaced the orientation towards England and Schneider makes it clear that the “close nexus between political events and linguistic developments (via identity rewritings), and the causal role of the former for the latter, are undisputed” (\citeyear[275]{Schneider2007}). He cites several authors who establish the claim for a language separate from Britain, including Noah Webster’s famous words: “as an independent nation, our honor requires us to have a system of our own, in language as well as in government” (\citealt[20]{Webster1789}, quoted in \citealt[277]{Schneider2007}). There were also voices resisting the call for linguistic independence by continuing to favor British norms (forming a “complaint tradition”) but Schneider argues that “[i]n quantitative terms […] nativization made the balance tip toward the former position” (\citeyear[277]{Schneider2007}). Schneider postulates a clear relationship between the positive evaluation of linguistic difference and actual structural developments:

\begin{quote}
[T]he period of structural nativization was the one during which effects inhibiting divergence disappeared and, in contrast, linguistic differences became actively promoted or at least positively evaluated. Westward expansion, then, brought an increasingly appreciative attitude toward down-to-earth speechways and hence strengthened another powerful factor promoting linguistic nativization. While of course British and American English still have a lot in common and linguistic continuity has also been important, differences between the two major varieties of English kept increasing. (\citeyear[278]{Schneider2007})
\end{quote}


As evidence for these increasing differences he gives examples of lexical borrowings and innovations, stressing the creativity of word-formation processes characteristic of this phase, which found expression especially in many conversions (\citeyear[279]{Schneider2007}). On the level of grammar, he finds an “almost endless” number of innovations at the lexico-grammatical interface, e.g. \emph{different than} (vs. \emph{from, to}) (\citeyear[280]{Schneider2007}). What is also important in this phase is spelling because several differences to British English did not emerge from use (like differences on other levels) but through deliberate language planning and have acquired a symbolic significance in public discourse (e.g. <-or>/<-our>, <-ize>/<-ise>) (\citeyear[281]{Schneider2007}). The phonological level is the only level on which Schneider finds that hardly any evidence is available and if it is, it is difficult to interpret. Schneider consequently does not give any examples on how structural nativization proceeds phonologically (\citeyear[279]{Schneider2007}). He does, however, provide more information on how he conceptualizes the process by arguing that


\begin{quote}
differences between varieties of English, British and American in the present case, not only consist of the ones frequently observed, documented and listed, but they encompass an infinitely larger set of habits and constructions which are hardly ever explicitly noted, most of which are associated with particular lexical items. […] This suggests that structural nativization operates inconspicuously but highly effectively, affecting frequencies and co-occurrence tendencies of individual words and constructions more than anything else. (\citeyear[282]{Schneider2007})
\end{quote}


This shows that he regards actual structural nativization as proceeding below the level of awareness, but it also raises the question of the exact nature of the relationship between consciously expressed attitudes and evaluations and unconsciously proceeding structural changes, a question that remains open in Schneider’s account.


The fourth phase is dated from 1828/1848 to 1898 and, in line with the Dynamic Model, it is characterized by an endonormative orientation based on “a new type of national self-dependence and a national pride based on local, American, achievements” (\citeyear[283]{Schneider2007}). In this phase, the entire continent was settled and controlled by Americans, and this “achievement”, at the cost of Native Americans who were killed and forced away from their lands, was not only a source of national pride but also led to “a second heightened phase of koinéization” (\citeyear[290]{Schneider2007}) resulting in a high degree of uniformity. The uniformity was strengthened by the codification of American English in Webster’s \textit{An American Dictionary of the English Language} (\citeyear{Webster1828}) and Bartlett’s \textit{Dictionary of Americanisms, a Glossary of Words and Phrases usually regarded as peculiar to the United States} (\citeyear{Bartlett1848}). At the same time, regional and social variability obviously continued. \citet[289]{Schneider2007} observes that many literary works which became part of a distinctive American literary canon employed representations of regional and social dialects, but he does not explore the relationship between the literary interest in linguistic variation at a time characterized by the codification of a uniform American variety further. This is something that I will do in the present study, which will show the relevance of this observation to the emergence of American English. Another relevant point that Schneider makes is that the development does not always proceed continually in one direction, but that there can be breaks or even returns to an older phase, as evidenced by “a purist, pro-English movement, which […] gained momentum after the Civil War and in the 1870s and 1880s, after endonormative stabilization” (\citeyear[288]{Schneider2007}). So exonormative orientations, based on different identity constructions, did not cease to exist, even in a phase of endonormative stabilization, and this study will also shed some light on the interplay between these two types of orientations in the nineteenth century and also on the ways that they interact with regional and social variability as well as with national uniformity.

\largerpage
The starting point of the fifth phase, which continues until the present day, is 1898. Schneider regards the Spanish-American War in 1898 as a turning point because it was the first war fought by the whole unified nation against another power, which strengthened feelings of national unity which were then a prerequisite for cultural fragmentation under the umbrella of the nation. “American society is being transformed into a multicultural mosaic, and this process is mirrored by the emergence of distinct varieties of English, each associated with different identities” (\citeyear[294]{Schneider2007}). A very important aspect is that Schneider does not postulate that dialect diversity occurred only in the twentieth century, but he finds that it has always existed. In contrast to earlier centuries, however, he finds that the twentieth century is marked by a “\emph{socially indicative} dialect diversity, an ethnic and regional fragmentation of the population along linguistic lines in perception and production” [emphasis mine] (\citeyear[296]{Schneider2007}). The crucial difference is therefore that diversity comes to index social identities:

\begin{quote}
[T]his diversification happened because the various regional, social, and ethnic groups recognized the importance of carving out and signaling their own distinct identities against other groups and also against an overarching nation which, while it is good to be part of, is too big and too distant to be comforting and to offer the proximity and solidarity which humans require. (\citeyear[296]{Schneider2007})
\end{quote}


This had an impact on linguistic structure in that dialect differences became more pronounced and, in Schneider’s words, more “strictly compartmentalized” (\citeyear[296]{Schneider2007}). As for the nativization period, he stresses that the process operated subconsciously and that it was not the result of an intentional act. As evidence for the developments in the fifth phase he cites several sociolinguistic studies, e.g. Labov’s investigation of Martha’s Vineyard (\citealt{Labov1972} [1963]) and Wolfram \& Schilling-Estes's (\citeyear{Wolfram1996,Wolfram1997}) study of the island Ocracoke, which illustrate how people on the islands used traditional regional variants to symbolize and demarcate their island identity against outsiders and to ensure that they are not absorbed in the mainland group identity. Next to these endangered local varieties he also describes the development of further varieties and their connection to identity constructions and realignments: Southern English, a Northern English marked by the Northern Cities Shift, ethnic speech forms in the European-American groups, Native American English(es), African American English, Chicano English and other Hispanic varieties, Cajun English, Hawaiian Creole and Asian Englishes. To give an example, he cites \citegen{Tillery2003} research which finds that there were two periods of great social change in the South, the first after the Civil War and the following Reconstruction period (marked among other things by immigration of northern Americans) and the second around World War II (marked especially by urbanization), and in both cases these social developments were accompanied by significant linguistic changes. For example, after World War II “the linguistic expression of a new, modern Southern identity was shaped, [which] affected both “Traditional” and “New” Southern features” (\citeyear[299]{Schneider2007}). So non-rhoticity and yod-retention were for example features which were associated with traditional, rural, antebellum culture, whereas rhoticity and yod-dropping have come to symbolize the “New South” (\citeyear[299]{Schneider2007}).


All in all, \citegen{Schneider2007} Dynamic Model stands in stark contrast to \citegen{Trudgill2004} model of new-dialect formation. Trudgill’s stages are very much focused on linguistic developments while Schneider’s developmental phases emphasize the close connection between the historical situation, social factors and linguistic developments. The type of evidence that both linguists draw on reflects this difference. Trudgill relies largely on data from the \emph{Origins of New Zealand English} (ONZE) project and on dialectological research on nineteenth-century British English dialects. While Trudgill therefore keeps a strong focus on linguistic data, Schneider cites an abundance of research not only on linguistic developments in seventeen postcolonial varieties of English but also on social developments in order to support his model. With regard to linguistic effects, his case studies provide many examples, but they do not offer as systematic and detailed an overview as Trudgill’s analysis of New Zealand English. Instead, Schneider emphasizes the common characteristics of sociolinguistic developments, a dimension that is almost completely neglected in Trudgill’s model. These contrary views illustrate the need for more research on the emergence of new varieties of English.

A third theory has been developed by \citet{Kretzschmar2014} as part of his theory of speech as a complex system. In contrast to \citet{Trudgill2004} and \citet{Schneider2007}, he does not postulate the existence of separate phases that lead to the emergence of a new variety because this view is not compatible with his theory of language. A complex system is “a system in which large networks of components with no central control and simple rules of operation give rise to complex collective behavior, sophisticated information processing, and adaptation via learning or evolution” (\citealt[143]{Kretzschmar2014}, citing \citealt[13]{Mitchell2009}). Applied to speech, these components are linguistic forms, variants realizing a variable, and what is “truly stable and systematic about speech” \citep[151]{Kretzschmar2014} is that the token frequency distribution of these forms is nonlinear, leading to a typical A-curve when graphed on a chart \citep[147]{Kretzschmar2014}. This nonlinear distribution occurs at different levels of scale, for example at the level of an individual, a community, a larger region or a nation, and variants which occur at the top at one level of scale can occur in the tail of the curve at another level of scale. Consequently, linguists who identify a variety traditionally do so by identifying the top-ranked variants which occur on a specific level of scale, for example at the level of nation, so that American English, for example, is considered to consist of the variants which occur at the top frequency ranks at the national scale. \citet[151]{Kretzschmar2014}, however, considers it a mistake to focus only on these top-ranked variants in linguistic analysis because the low-frequency variants are in fact highly relevant in language change and in the emergence of new varieties as well. Change occurs because speakers interact and these interactions can result in changes in the frequency distribution of variants. This in turn leads to new variants ranked at the top of the frequency curve, which can then be described as a new variety at some level of scale. \citet[151]{Kretzschmar2014} supports Schneider’s view that the emergence of varieties is \emph{not} a deterministic process:

\begin{quote}
Random interactions between speakers may eventually promote variants at lower frequencies to the top rank, and vice versa, so there is no fixed relationship between input frequencies (say, from settler, indigenous, or adstrate languages) and what will become most common in a new variety.
\end{quote}

He argues that not only language is involved in interactions but human perceptions as well and human perception is not restricted to what is most frequent but also encompasses reactions to the use of variants. In this context, Kretzschmar uses the concept of \textit{positive feedback}, which he regards as equivalent to \citegen{LePage1985} concept of \textit{focusing}:

\begin{quote}
The idea of feedback recognizes that the information content in speech is not just the functional message of some utterance or piece of writing, but also an evaluation of who says what when. \citep[151]{Kretzschmar2014}
\end{quote}


Judging by the reactions of other speakers to the use of linguistic forms, speakers evaluate the success of their use and adapt accordingly. Kretzschmar considers this a better view of Schneider’s and Mufwene’s concept of a \textit{selection process} because, in his view, variants are not selected (and rarely lost). Instead, they simply become more or less frequent through “massive numbers of random interactions between speakers” and “the perceptions of the human agents using language” (\citeyear[152]{Kretzschmar2014}). Unlike Trudgill, who uses \textit{focusing} to explain the stabilization of the forms of the new variety (basically the last step in the process), Kretzschmar assigns a much more central role to focusing because in his view it explains why the process is not deterministic (i.e. the predictable result of input frequencies). It is noticeable, however, that he does not elaborate on how exactly focusing proceeds and which mechanisms are operative in the process of giving and receiving feedback and evaluating the success of linguistic forms. However, he points out that his model and Schneider’s Dynamic Model complement each other precisely because in his view Schneider describes “the evolution of the new society’s perceptions” and not “the internal linguistic history of a new variety” (\citeyear[157]{Kretzschmar2014}). By relegating Schneider’s “new varieties” to the level of perception of usage (to be distinguished from actual usage), he questions the traditional definition of \textit{variety}, of the object whose emergence is controversially modeled, which is why in \sectref{bkm:Ref521000690} I will take a closer look at the understanding of the term in linguistics in general and in the models described in this section in particular.


\citet{Kretzschmar2015} describes the implications of his model for an account of the emergence of American English. For the early phases of settlement, he describes the linguistic situation as a “pool of linguistic features collected from a radically mixed settlement population” (\citeyear[251]{Kretzschmar2015}), a conceptualization that is in line with Mufwene’s feature pool but not with historical accounts on American English such as \citegen{Fischer1989} \emph{Albion Seeds}, which describes the culture and language of the different regions of settlement as having been transplanted from the respective regions in Britain and remaining fairly homogeneous. In this situation, order emerged in the form of the nonlinear distributions characteristic of complex systems. And owing to the scaling property of complex systems, these distributions occurred at different levels of scale, so that Kretzschmar speaks of American English already in this early phase:

\begin{quote}
Right from the beginning somewhat different sets of variants emerged as top-ranked elements in different localities. Also right from the beginning, a particular set of variants emerged at the highest level of scale, American English. (\citeyear[257]{Kretzschmar2015})
\end{quote}


So in contrast to \citet{Schneider2007}, who considers American English to nativize in the late eighteenth and early nineteenth century, Kretzschmar suggests that an American variety, conceptualized as a set of top-ranked elements at the highest level of scale, has already been present from the seventeenth century onwards. He views the nativization that Schneider describes as located on the level of perception (people started to notice and describe differences between American and British English) and argues along the same lines that eighteenth-century comments on uniformity are equally a result of perception and not to be taken as evidence for an actual colonial koiné \citep[258--259]{Kretzschmar2015}. While \citet[270]{Schneider2007} is also critical of strong versions of the koinéization hypothesis (he cites \citealt{Dillard1975} as the strongest one), he nevertheless assumes a “remarkable degree of linguistic homogeneity in the colonies” in phase 2, which is in contrast to the diversification of phase 5. Kretzschmar also disagrees here by stating that diversification also occurred from the seventeenth century onwards, when “noticeable differences, both between American regions and between American and British English” (\citeyear[257]{Kretzschmar2015}) emerged. In order to support this view, he gives examples of linguistic forms which were present then and are still associated with specific regions, e.g. non-rhoticity, lexical items like \emph{chunks} and \emph{tote} and grammatical forms like \emph{hadn’t ought}. The diversification just became \emph{more} noticeable as time progressed (\citeyear[259]{Kretzschmar2015}). Even though Kretzschmar does not postulate the existence of developmental phases in general, he notes a major difference between the eighteenth century and the nineteenth century in the case of American English. While the great population mixture in the early settlements and in the eighteenth century led to the creation of new and independent patterns (\citeyear[259]{Kretzschmar2015}), the nineteenth century was rather marked by an extension of these complex systems from the east to the west as a result of westward migration. Groups of settlers who already lived in western parts of the country, especially Scotch-Irish settlers in the Appalachian mountain regions, could contribute to these complex systems, and Kretzschmar cites Montgomery's (\citeyear{Montgomery1989, Montgomery1991, Montgomery1997}) findings that indeed some Scotch-Irish variants have been retained in these areas, but overall, he finds that Atlas data shows that “major patterns created by historical east-west settlement largely persist” (\citeyear[261]{Kretzschmar2015}). Nevertheless, change always occurs in complex systems, but Kretzschmar regards it as located rather at lower levels of scale (neighborhoods and cities) and not at larger regional levels of scale (\citeyear[261]{Kretzschmar2015}).


This overview of three models of the emergence of new varieties illustrates that linguists are far from achieving a consensus of opinion on the common, underlying operations and mechanisms behind this process. I argue here that some differences between the models are a result of different conceptualizations and definitions of the term \textit{variety} and that it is important to clarify what the term refers to in order to be able to analyze the emergence of new varieties and discuss and ultimately test the models proposed by \citet{Trudgill2004}, \citet{Schneider2007} and \citet{Kretzschmar2014, Kretzschmar2015}. The analysis of the meaning(s) of the term \textit{variety} is therefore going to be the subject of \sectref{bkm:Ref521000690}, before I will pay detailed attention to the role of social factors in \sectref{bkm:Ref521576818} and conclude with a final discussion of the models in \sectref{bkm:Ref527389617}.

\subsection{Definition of the term \textit{variety}}
\label{bkm:Ref521000690}\hypertarget{Toc63021207}{}
A first step in any discussion on how new varieties emerge must be to define what precisely the term \textit{variety} refers to and how a “new” variety can be distinguished from an “old” one. At first, it needs to be noted that the different theories and models foreground different terms. \citet{Trudgill2004} speaks predominantly of \textit{dialects} and particularly of \textit{colonial dialects}, while \citet{Schneider2007} prefers the term \textit{variety} and refers to the varieties that emerge in the different places as \textit{postcolonial varieties} or \textit{Postcolonial Englishes}.\footnote{It is important to note that each author uses \emph{both} terms, \textit{dialect} and \textit{variety}, in their books but foreground different terms by choosing them as the main term for the newly formed or emerging entities.}  It is often indicated that \textit{dialect} and \textit{variety} are used more or less synonymously, with \textit{variety} being a newer and more neutral term. \citet[32]{Meyerhoff2011}, for example, defines variety as a “[r]elatively neutral term used to refer to languages and dialects” which “[a]voids the problem of drawing a distinction between the two, and avoids negative attitudes often attached to the term \emph{dialect}”. \citet[5]{Chambers1998} make a slight distinction between the two terms. They write in their introduction to dialectology that they “shall use ‘variety’ as a neutral term to apply to any particular kind of language which we wish, \emph{for some purpose}, to consider as a single entity” [emphasis mine], while they regard \textit{dialect} as a more particular term which refers to “varieties which are grammatically (and perhaps lexically) as well as phonologically different from other varieties”. It can be inferred from this distinction that \citet{Trudgill2004} foregrounds the term \textit{dialect} in his theory because the main criterion he uses to distinguish a new variety or dialect from an old one is the criterion of structural distinctiveness. He argues that in order to explain the formation of a new dialect, one has to “first decide what the distinctive characteristics of New Zealand English are” \citep[31]{Trudgill2004}. The distinctiveness is established in relation to British dialects, the starting point being a mixture of dialects spoken by the parents of the first New-Zealand-born Anglophones, and the end-point the dialect spoken by the second generation of Anglophone children in New Zealand. Differences which emerged afterwards (since 1890) are not included in the analysis because they are the result of changes occurring only after New Zealand English had been formed \citep[32]{Trudgill2004}. So in order to distinguish and define a “new” variety of New Zealand English against “old” varieties of British English, he looks for evidence of a new set of linguistic features, distinct from other older sets of features, based on two main sources: studies on the history of the English language, especially dialectological research on nineteenth- and twentieth-century English, and the ONZE project, comprising recordings of New Zealanders made by the National Broadcasting operation of New Zealand between 1946 and 1948, which provide insights into the second stage of the formation process because the informants represent the first generation of children born in New Zealand \citep[33]{Trudgill2004}. The linguistic features he analyzes are almost exclusively phonological and located on the segmental level, and the analysis rests heavily on frequencies of use. What distinguishes stage 3, the stage at which he postulates the existence of a distinct New Zealand variety, from stage 2 is that the variety is now characterized by uniformity and stability \citep[113]{Trudgill2004}. Uniformity is achieved because majority variants have survived and minority variants have disappeared, and stability is achieved through focusing. It appears therefore that while the primary criterion in defining the new variety is the structural distinctiveness of a uniform set of features, stability, achieved through focusing, is at least a secondary criterion, which is applicable only after a new set of features has formed. It is noticeable, however, that \citet{Trudgill2004} does not analyze the focusing process in his study on New Zealand English, which again highlights the relative unimportance of the stability criterion.


\citet{Schneider2007} also focuses on the criterion of structural distinctiveness in his model. He speaks of “the birth and growth of structurally distinctive PCEs” \citep[45]{Schneider2007} and describes a PCE as “a new language variety which is recognizably distinct in certain respects from the language form that was transported originally, and which has stabilized linguistically to a considerable extent” \citep[51]{Schneider2007}. In the phase of structural nativization, the degree of difference to the former input varieties increases the most. He states that “this stage results in the heaviest effects on the restructuring of the English language itself; it is at the heart of the birth of a new formally distinct PCE” \citep[44]{Schneider2007}. S-curves, which typically characterize linguistic changes, have a phase of rapid increase of a variant in the middle of the development over time and this is where Schneider locates the phase of structural nativization. In line with \citet{Trudgill2004}, \citet[51]{Schneider2007} regards stability as a second characteristic of a variety, and he argues that it is achieved during a phase which can be seen as corresponding to the later part of the S-curve. However, in stark contrast to \citet{Trudgill2004}, speakers’ perceptions play a role in his conceptualization of a variety as well. He writes that “regional speech differences emerge, stabilize, and become recognizable in the public mind” \citep[9]{Schneider2007}, which shows that in addition to structural difference and stability, public recognition is also a factor to be dealt with in determining what a variety is and how it emerges. In the case study on American English, he finds that the “birth of American English as a concept and as a variety falls into that period [the period between ca. 1773-1828/1848]” \citep[276]{Schneider2007}. In this statement he explicitly distinguishes between the \emph{concept} of a variety and the variety itself, but he does not elaborate on this distinction any further, which is problematic because he assumes a direct relationship between the development of structural differences and people’s perceptions and attitudes:

\begin{quote}
[T]he period of structural nativization was the one during which effects inhibiting divergence disappeared and, in contrast, linguistic differences became actively promoted or at least positively evaluated. Westward expansion, then, brought an increasingly appreciative attitude toward down-to-earth speechways, and hence strengthened another powerful factor promoting linguistic nativization. \citep[278]{Schneider2007}
\end{quote}


So in \citegen{Schneider2007} model, speakers’ perceptions and attitudes, which can be seen as formative of the \emph{concept} of a new variety, did not only play a role in the phase of endonormative stabilization, but they are also supposed to influence structural nativization, that is the emergence of a \emph{structural} variety. \citet[94]{Schneider2007} argues that people usually focus on a few salient distinctive forms when they perceive and evaluate a variety, but that “those properties of a variety which seem specifically distinctive [are] quantitative tendencies of word co-occurrences, recurrent patterns, speech habits, prefabricated phraseology”. Based on this, a conceptual variety can be defined as consisting of a small set of salient linguistic features which are recognized by speakers as distinct and which attract some sort of evaluation. A structural variety can be defined as consisting of a large set of features which make the variety structurally different from another set of features.


Nevertheless, Schneider’s idea of what a variety is, and how the concept of a variety and the structure of a variety are related, remains rather vague. \citet{Kretzschmar2014, Kretzschmar2015, Kretzschmar2015b}, on the other hand, discusses the term \textit{variety} in more detail and he is fairly critical of traditional understandings of the term. As pointed out in \sectref{bkm:Ref525715631}, he relegates \citegen{Schneider2007} Dynamic Model to the level of perception only and argues that this is the case for most descriptions of varieties:

\begin{quote}
What actually makes “new varieties” of English or of any other language, in the sense that we usually mean in linguistics, is that linguists from time to time choose to record their perceptions of the usage of some population of speakers. \citep[157]{Kretzschmar2014}
\end{quote}

These perceptions include only the top-ranked variants of the complex system of speech at a specific level of scale which makes the varieties described by linguists “idealized abstractions” \citep[156]{Kretzschmar2014} which linguists and also lay people are always interested in, but which should not be confused with linguistic reality. According to \citet{Kretzschmar2014}, this reality should best be understood as a complex system:

\begin{quote}
New varieties are not just something to be associated with former colonies; they are emerging all around us every day, as speakers of English form new groups in local neighborhoods, communities of practice, social settings, and new places around the globe in many places besides colonial settings. It is not a process that happens once and is done. The complex system of speech continues to operate, and new order emerges from it all the time. \citep[157]{Kretzschmar2014}
\end{quote}


He does not address \citegen{Schneider2007} distinction between structural variety and conceptual variety, but claims that Schneider views varieties only as “identity-driven discourse constructs” (\citealt[51]{Schneider2007}, cited in \citealt[157]{Kretzschmar2014}), a claim which is clearly not justified given that the nativization phase in the Dynamic Model rests crucially on structural differentiation.\footnote{In fact, \citet[51]{Schneider2007} writes that the homogeneity that is often emphasized in descriptions of new varieties is an “identity-driven discourse construct” and not the variety itself. Still, it makes sense to ask to what extent the conceptual variety in \citegen{Schneider2007} Dynamic Model can be defined as a discourse construct and which role identity plays in the process of its emergence. I will address these questions in more detail in \sectref{bkm:Ref522870698} and \sectref{bkm:Ref523897668}.} In an earlier article, \citet{Kretzschmar2012} make a distinction similar to the one between structure and concept. They argue that \emph{natural} language varieties need to be distinguished from \emph{ideational} language varieties. They discuss the case of Standard American English and argue that it is an \textit{idea} which does not have an empirical basis in language use. Instead, it is “just another manifestation of a particular culture” \citep[156]{Kretzschmar2012}. By calling it an “idea”, they relegate Standard American English to a level that corresponds to Schneider’s conceptual level, so that, in the end, we can derive a threefold distinction between a) the linguistic reality, which is a complex system, b) the perceptual variety, which is based on the linguists’ perceptions of the top-ranked variants on a specific level of scale (e.g. individual, local or national) and c) the conceptual/ideational variety, which is based on people’s idea about language. With regard to the  ideational variety, it is important to note that even though it does not directly reflect language in use, it is “not a myth” but “a very real […] construct” for its speakers \citep[143]{Kretzschmar2012}.


Kretzschmar is not the only one who critically discusses the notion of \textit{variety} in general and in the context of new variety formation in particular. \citet{Leimgruber2013b} also calls for “rethinking the concept of ‘geographical varieties’ of English”. He argues that the starting point for defining a variety, especially in the context of World Englishes, is usually a geographical and political but \emph{not} a linguistic concept:

\begin{quote}
It may seem impractical to completely do away with such a useful concept as the variety, which has for so long been the basic unit of analysis in many fields of linguistics, including World Englishes. It remains, however, that the concept is often under-defined in works setting out to describe such varieties – terms like ‘Singapore English’, ‘Malaysian English’, ‘Welsh English’, etc., are taken for granted because, after all, they contain a geographical component everyone can relate to. The actual linguistic form of the ‘variety’ is then described post hoc, with the analytical unit ‘variety’ conditioning the analysis. \citep[6]{Leimgruber2013b}
\end{quote}

This discussion shows that a clearer definition of what kinds of varieties there are, how they are related and how they can be identified and described is necessary. A recent contribution by \citet{Pickl2016} is helpful in this regard. He draws on the terms \textit{emic} and \textit{etic} to distinguish different types of varieties. Emic dialects are “cognitive concepts of the speakers whose speech is at the same time the object of linguistic investigation” \citep[78]{Pickl2016} and etic dialects are based on objective linguistic analyses of speech productions. Regarding the relationship between emic and etic dialects he states that

\begin{quote}
There is no reason why the fundamental linguistic concept of a dialect variety should differ from the folk linguistic concept. In other words, scholarly or etic ideas about geolinguistic entities can and should have the same principal structure as lay persons’ implicit ideas about dialects in space while being based on transparent – and, as far as possible, objective – criteria that are not derived from the speakers’ ideas, but from scientific reasoning. \citep[78]{Pickl2016}
\end{quote}


The shared “principal structure” is a prototypical one, with the important characteristic of fuzzy category boundaries and linguistic features which have different degrees of typicality. The distinction between emically and etically defined categories is analogous to \citegen{Schneider2007} distinction between a conceptual and a structural variety, but the similarities and differences are elaborated on in more detail. With regard to the emic category, \citet{Pickl2016} refers to perceptual dialectology as a research area, while his own study focuses on the etic category. He defines it more precisely by using a definition by \citet{Berruto2010}:


\begin{quote}
The tendential co-occurrence of variants gives rise to linguistic varieties. Therefore, a linguistic variety is conceivable as a set of co-occurring variants; it is identified simultaneously by both such a co-occurrence of variants, from the linguistic viewpoint, and the co-occurrence of these variants with extralinguistic, social features, from the external, societal viewpoint. (\citealt[229]{Berruto2010}, cited in \citealt[79]{Pickl2016})
\end{quote}

In order to identify sets of co-occurring features, Pickl argues for using statistical methods, more specifically factor analysis, which he regards as superior to other statistical methods like cluster analysis, bipartite spectral graph partitioning and multidimensional scaling (\citeyear[80--83]{Pickl2016}). He supports this view by conducting an analysis on dialect areas in Bavarian Swabia based on data from the dialect atlas \emph{Sprachatlas von Bayerisch-Schwaben} (SBS, \citealt{Konig19962009}). The variability in the region is reduced to 16 factors which account for 62.21\% of the variance in the data. Each factor stands for a recurring pattern of linguistic variants which is particularly strong in a geographical region (and within the region, the locations exhibit different degrees of typicality for the pattern) so that in the end, a combined factor map can be constructed which shows the prototypically structured dialect areas in geographical space. The point is that he finds a way for identifying dialect areas based on a statistical analysis of the data only and without recourse to either lay people’s or linguists’ subjective judgements. Attempts in this direction have been numerous (they belong to the field of dialectometry), but they did not draw on prototype theory or were skeptical of the existence of dialect areas in general (e.g. \citealt{Kretzschmar1996}). Pickl’s suggestion therefore combines the insight that emically and etically defined varieties share a prototypical structure with a suggestion for describing the structure of etic varieties by means of objective criteria, thereby minimizing the influence of ideas on the outcome of the analysis.

\citegen{Pickl2016} approach counters \citegen{Kretzschmar2014} criticism that the varieties perceived and described by linguists are restricted to top-ranked variants and that all those variants which occur at lower frequencies are ignored and considered irrelevant because he aims at identifying abstract varieties in a way that takes the complexity of speech more strongly into account than other methods. For example, \citet[81]{Pickl2016} states that “it is […] impossible for a cluster analysis to come up with anything more subtle than global, exclusively dominant areas; subordinate, non-dominant areas that are determined by smaller numbers of features cannot be identified by cluster analysis”, which is why he proposes factor analysis as a statistical method instead. As his goal is to identify dialect layers overlapping in space, each layer consisting of “congruent distribution areas of co-occurring linguistic forms” \citep[79]{Pickl2016}, he tries to achieve an abstraction that is closer to the linguistic reality than traditional analyses which identified discrete dialect areas based on the presence or absence of distinctive linguistic features.

\citegen{Pickl2016} focus on geographical varieties leaves open the question of how an etic variety can be described which is not only based on regional distributions of data, but also on social ones. It is conceivable, however, that his approach can also identify social varieties if sufficient data are available. \citet[6]{Leimgruber2013b} also points out the need for more data to carry out quantitative studies which complement qualitative analyses conducted by sociolinguists like \citet{Blommaert2010}, who do not consider it important anymore to identify and describe varieties at all but are primarily interested in describing and explaining how speakers draw on linguistic resources in their social interaction with others (see \sectref{bkm:Ref506884048} for details). In general, \citegen{Pickl2016} approach to defining and identifying structural varieties takes the criterion of structural distinctiveness more seriously than other approaches.

To summarize, it seems that linguists distinguish several types of varieties. Based on the discussion above, I propose a distinction between the linguistic reality, conceptualized as a complex system, and three abstract types of varieties: structural varieties, perceptual varieties and discursive varieties, as illustrated in \figref{fig:2:1}.


\begin{figure}
\includegraphics[width=0.8\textwidth]{figures/Paulsen-img01.pdf}
\caption{Types of varieties described by linguists}
\label{fig:2:1}
\end{figure}


While the linguistic “reality”, i.e. the total number of forms used by speakers in particular regional, social and situational contexts, is a complex system as described by \citet{ Kretzschmar2014, Kretzschmar2015, Kretzschmar2015b}, it is possible to identify and describe more abstract patterns of co-occurring forms. Structural varieties are tendential co-occurrences of variants in a particular place (and potentially also in a particular social situation or in relation to social factors). They are determined based on production data, collected and analyzed by the linguist using statistical measures and \citegen{Pickl2016} prototype approach is a very convincing one because it allows for fuzzy category boundaries and does not just identify one dominant pattern but also its overlaps with other non-dominant patterns. Perceptual varieties are tendential co-occurrences of forms in lay people’s perceptions of language variation. They are determined based on perception data, again collected and analyzed by the linguist. Here, perception in a cognitive sense (what is perceivable by the senses) and in a conceptual sense (what is perceivable because people have a cultural concept of the pattern in mind) are both included and it is convincing to assume a prototypical structure of these perceptual categories as well. Discursive varieties take the ideational character of varieties into account, which is for example discussed in \citet{Kretzschmar2012}. They are tendential co-occurrences of forms in speakers’ discursive constructions of patterns of variation and it is the aim of \sectref{bkm:Ref522870698} and \sectref{bkm:Ref523897668} to show how the concept of \textit{enregisterment} is useful in defining and investigating discursive varieties.


It is important to note that the different types of varieties overlap. In determining structural varieties linguists might be influenced by their perceptions already in the process of collecting data, so it is questionable whether a purely fact-driven abstraction of co-occurrence patterns is even possible.\footnote{\citet[6]{Leimgruber2013b} for example criticizes that the collection of corpora for studying World Englishes take a “conceptual linguistic system tied to a particular locale” as a starting point.} At the same time, perception is of course related to structure in that highly frequent variants could be more easily perceivable than low-frequency ones. It is, however, more than doubtful that frequency of use is the only factor influencing perception. If variants are often the focus of \emph{discourses} on language, for example, they are more likely to be perceived by speakers even if they are \emph{used} infrequently. And lastly, perception influences the construction of discursive varieties because speakers are more likely to engage discursively with variants that they perceive. However, it is equally possible that linguistic forms remain part of discursive varieties even though they are neither produced nor cognitively perceived anymore. Against the background of this distinction I compare and discuss the role of social factors, particularly of identity, in the emergence of new varieties of English in the next section.

\subsection{The role of social factors in the emergence of new varieties}
\label{bkm:Ref521576818}\hypertarget{Toc63021208}{}
The roles attributed to social factors in the emergence of new varieties by \citet{Trudgill2004} and \citet{Schneider2007} could not be more different. The following citations illustrate this well. \citet[95]{Schneider2007} writes that


\begin{quote}
The Dynamic Model, supported by accommodation and identity theory, predicts that via language attitudes a speaker’s social identity alignment will determine his or her language behavior in detail. Note that there is no implication made here that these developments have anything to do with consciousness: accommodation works irrespective of whether the feature selected and strengthened to signal one’s alignment is a salient marker of which a speaker is explicitly aware or an indicator which operates indirectly and subconsciously.
\end{quote}


\citet{Trudgill2004}, on the other hand, is skeptical of social factors influencing the formation of the new variety. Especially identity, which assumes a central role in the Dynamic Model, is ruled out as a relevant factor:


\begin{quote}
And it is clear that identity factors cannot lead to the development of new linguistic features. It would be ludicrous to suggest that New Zealand English speakers deliberately developed, say, closer front vowels in order to symbolise some kind of local or national New Zealand identity. This is, of course, not necessarily the same thing as saying that, once new linguistic features have developed, they cannot become emblematic, although it is as well to be sceptical about the extent to which this sort of phenomenon does actually occur also. For example, we can say that the twentieth-century innovation in New Zealand English, whereby the \textsc{kit} vowel became centralised might perhaps now constitute a symbol of New Zealand identity and that the vowel might for that reason in future even become more centralised. But I have to say that I would, personally, find even this unconvincing. Why do New Zealanders need to symbolise their identity as New Zealanders when most of them spend most of their time, as is entirely normal, talking to other New Zealanders? But in any case, we most certainly cannot argue that New Zealanders deliberately centralized this vowel \emph{in order to} develop an identity marker. \citep[157]{Trudgill2004}
\end{quote}

A further argument which he puts forward against the role of social factors relates to the key role played by children in the process of new-dialect formation. In his view, children are not influenced by social factors like prestige or stigma, but “[t]hey simply selected, in most cases, the variants which were most common” \citep[115]{Trudgill2004}. Adults influenced the new variety only marginally through accommodation in the early stages of contact, but even this accommodation “is not necessarily driven by social factors such as prestige or identity, but is most often an automatic consequence of interaction” \citep[28]{Trudgill2004}.

These two extreme positions have attracted the interest of a number of linguists and consequently, a discussion section in \emph{Language in Society} 37 (2008) has been devoted to the issue. \citet{Schneider2008b} and \citet{Trudgill2008b, Trudgill2008c} defend their respective positions and support it with more arguments and evidence. \citet{Mufwene2008} and \citet{Tuten2008} partly agree with Trudgill’s position; \citet{Bauer2008}, \citet{Coupland2008}, \citet{Holmes2008} disagree with Trudgill and argue in favor of \citegen{Schneider2007} view. One of the key issues of the discussion is the role of identity and it is noticeable that it is not understood by all participants in the same way. Trudgill’s (\citeyear{Trudgill2004, Trudgill2008}) argument focuses on “national identity” and all authors concede that if identity is reduced to this national perspective, Trudgill rightly doubts its relevance. However, apart from \citet{Mufwene2008}, all authors emphasize that the question of identity should \emph{not} be reduced to some sort of abstract national identity. \citet[273]{Bauer2008}, for example, argues that “complex kinds of identity are being expressed in the choice of a particular phonetic variant. It will not be as simple as feeling that one is ‘British’ or ‘New Zealand’; it will be much more local and much more specific”. \citet[259]{Tuten2008} also suggests that it is more likely that local or regional identities develop and influence the variety formation. \citet[269]{Coupland2008} goes beyond the local/national distinction by pointing out that identity in general is “often less coherent, less rationalised, more elusive, more negotiated, and more emergent […]. Identities are known to be often multiple and contingent”. This view is echoed by \citet[274]{Holmes2008} who state that “[…] to imply, as Trudgill seems to be doing, that “national identity” can stand for all types of identity deflects our attention from the real sociolinguistic issues”. \citet[265]{Schneider2008b} also points out that identity plays a role in \emph{all} phases of the model, also before the stage of nation-building. Nevertheless, he still emphasizes the importance of national identity (next to all other identity constructions):

\begin{quote}
Of course, there are linguistic forms considered diagnostic of individual postcolonial Englishes, and it is difficult to see how precisely these forms rather than any others should have been selected on a purely deterministic basis, excluding national identity as a factor. The strongest argument for the impact of identity in these processes is the observation that the origin and/or recognition and spread and/or scholarly documentation of these forms typically fall into periods of heightened national or social awareness. \citep[266]{Schneider2008b}
\end{quote}

Related to the question of what kind of identity is supposed to play a role (national, local, individual) is the question of intentionality. In \citegen{Trudgill2004} view, arguing for a role of identity means arguing that people change their linguistic behavior \emph{intentionally} and Mufwene picks up on that point by stating that

\begin{quote}
Trudgill is certainly correct in refuting the position that colonial identity drove the structural divergence of “new dialects” both from their metropolitan kin and from each other. This would be tantamount to claiming that evolution is goal-oriented and the colonists had really planned to be different linguistically. \citep[257]{Mufwene2008}
\end{quote}


While it is convincing that settlers in a colony did not come together and developed the goal to intentionally change their speech in order to mark themselves as different from their country of origin, it is nevertheless unclear why \citet{Trudgill2004, Trudgill2008b} and \citet{Mufwene2008} reduce the question of identity to the abstract national level and interpret effects of identity construction as the result of intentional moves because this is not at all what \citet{Schneider2007} or other sociolinguists claim.


On the contrary, \citet[264]{Schneider2008b} clarifies his position again by emphasizing that he sees identity constructions and linguistic accommodation as closely related. He regards identity as an “individual stance with respect to the social structures of one’s environment” and accommodation as “a process [in which] individuals approach each other’s speech behavior by adopting select forms heard in their environment, thus increasing the set of shared features” and concludes that accommodation is therefore “one of the mechanisms of expressing one’s identity choices” \citep[264]{Schneider2008b}. As a counterargument to Trudgill’s (\citeyear{Trudgill2004, Trudgill2008}) view that accommodation is automatic (and not social) because it is biologically given, he states that it is actually the social nature of the process that is biologically given because human beings are by nature social beings who strive to create group cohesiveness to ensure their survival in a hostile world \citep[264]{Schneider2008b}. \citet[275]{Holmes2008} argue similarly for the importance of identity not only in influencing the direction of accommodation but also in determining the frequency of interactions between people:

\begin{quote}
[N]on-demographic social factors bear directly on frequencies of interaction. Because people bring to each encounter their personal and social identities, as well as knowledge and beliefs about intergroup relations and about the social marking of linguistic variants, interactions with certain social groups will be sought out or avoided. Thus, social factors influence \textit{both} the frequency of interactions \textit{and} the direction of accommodation.
\end{quote}

This is reminiscent of Kretzschmar’s (\citeyear{Kretzschmar2014, Kretzschmar2015}) argument that complex systems evolve and change through massive interactions between speakers, which are naturally constrained by the place that they live in and their social background, which also has an influence on how much they travel and which communication channels they use to get in contact with people who live in other places. While \citet[262]{Kretzschmar2015} is very skeptical of simple correlations between linguistic features and regional and social features, he nevertheless argues that “positive feedback, or focussing, creates the A-curve for every feature” \citep[152]{Kretzschmar2014} and cites from \citegen{LePage1985} \emph{Acts of Identity}:

\begin{quote}
We see speech acts as acts of projection: the speaker is projecting his inner universe, implicitly with the invitation to others to share it. […] The feedback that he receives from those with whom he talks may reinforce him, or may cause him to modify his projections, both in their form and in their content. To the extent that he is reinforced, his behavior in that particular context may become more regular, more focused. (\citealt[181--192]{LePage1985}, cited in \citealt[152]{Kretzschmar2014})
\end{quote}

This shows that \citet{Kretzschmar2014, Kretzschmar2015} argues for a role of social factors and identity, albeit not in an abstract national sense, but in an individual sense and that he regards people’s reactions and social consequences of speech in actual communicative situations as crucial for the creation of A-curves in complex systems of speech.

Even \citet{Mufwene2008}, who strongly argues against a role of identity, regards accommodation as a social process. He states that “[t]he speakers’ mutual accommodations are certainly the social aspect of the mechanisms by which selection from among the competing variants (and language varieties) proceeds” \citep[257]{Mufwene2008}. Why he sees accommodation as a social process influencing the evolution of varieties while at the same time ruling out identity as a factor is less clear; perhaps it is the restriction of identity to an abstract colonial one that makes him skeptical. He does, however, speculate that “[i]f identity has a role to play, it must be in resisting influence from outside one’s community” \citep[258]{Mufwene2008}. As he does not elaborate on this any further, his view on the role of social factors remains very general and it is also not clear how they interact with other linguistic factors like frequency and “simplicity, perceptual salience, semantic transparency, regularity, and more familiarity to particular speakers” \citep[257]{Mufwene2008} and the role of children as “affective filters” \citep[258]{Mufwene2008}.

\citet{Trudgill2008b} defends his position against the criticism by emphasizing again that children play the crucial role in new-dialect formation and he draws on \citegen{Pickering2004} interactive alignment model to support his argument that young children \emph{automatically} accommodate to each other, without any influence of social factors, and that therefore the majority variant survives \citep[279]{Trudgill2008b}. However, \citegen{Tuten2008} discussion of the interactive alignment model emphasizes that its applicability seems rather restricted to young children while “older children and adults could adopt a strategy of non-alignment when appropriate” \citep[261]{Tuten2008}. He argues that when children grow older, they change their social orientation away from their parents towards the peer-group and that they become aware of similarities and differences between them and others. Furthermore, in the process of growing up, people “are heavily socialized to perform in certain ways” \citep[260]{Tuten2008} so that “it may be that accommodation (or automatic interactive alignment) and identity formation (among older children and adolescents) are closely linked” and that “community identity formation and koine formation are simultaneous and mutually dependent processes” \citep[261]{Tuten2008}.

It can be concluded that the overwhelming majority of the arguments are in favor of a position that includes social factors, particularly identity constructions, in the process of new variety formation. This is not to say that identity is unequivocally seen as the main driving force. \citet[265]{Schneider2008b} for example concedes that other factors may be as strong as identity and \citet[267--268]{Coupland2008} warns that “there are dangers in running too freely to causal explanations around identity”. \citegen[279]{Trudgill2008b} strongest point remains that there is no “feature-by-feature social-reasons account” which shows convincingly how social factors and identity influence the shape of a new variety. \citet[266]{Schneider2008b} equally concludes that “[D]esigning a study that will test a straightforward connection between socio-psychological attitudes (including national identity) and the use of specific linguistic forms in these contexts will certainly be a worthwhile task”.

\subsection{Conclusion}
\label{bkm:Ref527366308}\hypertarget{Toc63021209}{}\label{bkm:Ref527389617}
This section has shown that theories of the emergence of new varieties of English still differ to a considerable extent with regard to the underlying mechanisms and phases involved in the process. Even though \citet{Kretzschmar2014} claims that his account is complementary to \citegen{Schneider2007} Dynamic Model, the analysis has shown that this is only true to a limited degree. The only feature-by-feature account is provided by \citet{Trudgill2004} on the emergence of New Zealand English, but as he regards his model to be applicable to all new dialects in tabula rasa situations, his claims can be tested in the American context as well. I have raised two important issues in \sectref{bkm:Ref521000690} and \sectref{bkm:Ref521576818}: The first issue was that the different theories and models conceptualize the emergent new variety in different ways, which has a considerable impact on their claims, and the second issue was that one of the key differences between the theories is the role attributed to social factors in the process. It is not hard to see that these issues overlap. \citet{Trudgill2004}, who regards a new variety as a new linguistic system that is structurally different from other linguistic systems (particularly from those that the settlers brought to the new country), views social factors as irrelevant in the process. They could only become important \emph{after} the formation process in the way that people single out new linguistic features as characteristic of the variety and make them emblematic of it. I interpret this as claiming that social factors could contribute to the emergence of a perceptual and/or discursive variety, but only as a \emph{consequence} of the formation of a new structural variety. \citet{Schneider2007} on the other hand claims that not only a structural variety emerges but a perceptual and/or discursive variety as well, and that the emergence of the latter acts as a driving force in the emergence of the former. The perception, recognition, public documentation and discussion of a variety is therefore in his view \emph{not} just a consequence of already present structural differences, but it is one of the \emph{causes} of structural differentiation. The linguistic features which make the new variety distinct from other varieties are of course much more numerous than those perceived and discussed as distinct and \citet{Schneider2007} claims that they are found on different linguistic levels as well: While the majority of forms of the perceptual and/or discursive variety are located on the phonological and lexical level (with a focus on individual phonological segments and particular words or different spellings of the same word), the majority of structural differences can rather be found on the lexico-grammatical interface. The link between the two levels is supposedly found in the social realm – in the phase of structural nativization, a positive evaluation of forms recognized as distinct leads people to increasingly use distinct forms and patterns because they can express a new (national) identity by aligning themselves with a new model of speech, a discursive variety which subsequently stabilizes in the endonormative phase. However, the description of this process, that is the “micro-level of the relationship between attitudes and the evolution (i.e. selection or avoidance) of individual linguistic forms” \citep[95]{Schneider2007} remains rather vague and is only supported by some examples which do not represent a “feature-by-feature social-reasons account” that \citet{Trudgill2008b} demands. The same is true for Kretzschmar’s (\citeyear{Kretzschmar2014, Kretzschmar2015}) claims that positive feedback in people’s interactions influences the non-linear frequency distributions which form the basis for the idealized abstract structural varieties described by linguists. Concrete evidence or detailed descriptions as to which forms receive feedback, how this feedback is expressed and perceived and how people adapt their linguistic behavior based on the feedback is not provided.

In my view, the question of how new varieties emerge needs to be approached by first of all distinguishing more systematically between the different kinds of varieties because the investigation of each kind of variety requires a different methodology. Structural varieties should be identified based on bottom-up analyses of data, but it needs to be acknowledged as well that a completely data-driven, bottom-up analysis might be difficult to conduct in practice and this type of analysis is especially difficult in historical contexts because of the scarcity of data. Perceptual varieties should be described based on data obtained by methods used in perceptual dialectology, and discursive varieties should be investigated by means of discourse-linguistic methods (see \sectref{bkm:Ref506891065}). The systematic investigation of these different types of varieties should then form the basis for exploring the connection between them. The concept of \textit{enregisterment} is helpful in this regard because it involves a theory of how people construct \textit{registers} as models of linguistic (and social) behavior based on language use in everyday social situations. To conceptualize discursive varieties as registers which are based on structural varieties because they are formed as a result of social interaction (both immediate and mediated) helps to clarify the link between the different types of varieties. It is the aim of the present study to gain insights into the enregisterment of American English by showing how the construction of a discursive variety can be traced systematically in nineteenth-century America. In \sectref{bkm:Ref522870698}, I will therefore describe the concept of \textit{enregisterment} in detail and discuss its relation to several research areas of linguistics.

\section{Enregisterment}
\label{bkm:Ref522870698}\hypertarget{Toc63021210}{}\label{bkm:Ref522888605}
\textit{Enregisterment} is a concept that has been developed by the anthropological linguists Michael \citet{Silverstein2003, Silverstein1979, Silverstein1993} and Asif \citet{Agha2003, Agha2007}. In \sectref{bkm:Ref506883801}, I compare and contrast their definitions of register and enregisterment and sketch the advantages that their models hold for theorizing the emergence of new varieties. In \sectref{bkm:Ref506884048}, I outline and discuss the integration of enregisterment and central concepts to which it is tied (e.g. indexicality and orders of indexicality) in sociolinguistic research in general and show what potential it holds for analyzing social factors in the emergence of new varieties in particular. Taking up \citegen{Kretzschmar2014} observation that Schneider’s Dynamic Model (\citeyear{Schneider2007}) is primarily concerned with the perception of new varieties (and not with actual linguistic usage), I sketch the theoretical assumptions and important findings of the field which is primarily interested in the perception of varieties by non-linguists, namely perceptual dialectology, and discuss its relation to enregisterment in \sectref{bkm:Ref10466352}. Finally, I suggest in \sectref{bkm:Ref506891065} that the newly emerging field of discourse linguistics, which is concerned with the social negotiation of knowledge through linguistic practice \citep[53]{Spitzmuller2011}, can contribute to a study of enregisterment, both from a theoretical and a methodological point of view.


\subsection{The origins of the concept \textit{enregisterment} in linguistic anthropology}
\label{bkm:Ref506883801}\hypertarget{Toc63021211}{}\label{bkm:Ref512260235}
The term \textit{enregisterment} was originally introduced by the anthropologist and linguist Michael Silverstein in the mid-1980s (see \citealt{Silverstein2016} for details on the earlier uses of the term), but elaborated on in most detail in his \citeyear{Silverstein2003} article on indexical order.\footnote{This article is both based on and an elaboration of ideas presented in Silverstein’s prior publications (\citeyear{Silverstein1979, Silverstein1993}).}  He claims that an analysis of any sociolinguistic phenomenon requires an analysis of the indexicality of the linguistic forms used. Indexicality means that the linguistic forms possess the quality to point to aspects of the micro- as well as the macro-context in which the forms are used. For example, in a specific micro-context of wine tasting evaluative phrases like \emph{beautifully complex, very pronounced yellow} and \emph{assertive backbone} index connoisseurship (even though they are not part of specialists’ vocabulary, which comprises words like \emph{bouquet} and phrases like \emph{slightly pasty/acidic texture}). At the same time, this connoisseurship is “macro-sociologically locatable” \citep[226]{Silverstein2003} in that the use of these specific forms also indexes a social distinction between those who know how to describe wine and people who do not. Therefore, the use of the evaluative terms indexes more than connoisseurship: It also indexes social traits of the speaker, such as being well-bred or being at least upwardly mobile and having an interesting character. These two types of indexicality are located by Silverstein on different orders: The connoisseurship indexed by the phrases is on the \emph{n}{}-th order and the social traits on the \emph{n}+1st order of indexicality. This example illustrates that these orders are in a dialectic relationship. The existence of \emph{n}{}-th order indexicality makes it available for what Silverstein terms “ethno-metapragmatic evaluations” \citep[214]{Silverstein2003} which are embedded in a larger cultural schema shaped by social and linguistic ideologies: Being an expert of wine is associated with being part (or trying to be part of) an elite social group and this in turn is connected to expectations about the character and the social and linguistic behavior of that group. Describing a wine as \emph{beautifully complex} therefore comes to index social traits of the speaker such as a high social standing, educatedness and cultivation. This is what Silverstein calls \textit{essentialization}: The social traits are ideologically constructed as essences of persons and thus become “predictable-as-true”. When people believe that the phrase \emph{beautifully complex} is uttered by members of an educated, cultivated and upper-class elite, the utterance of the phrase points to these qualities. As such, it can be used by speakers to signal this macro-social identity. The essentialization (and sometimes even naturalization) of the values indexed by the form makes it possible that \emph{n}+1st order indexicality blends with \emph{n}{}-th order indexicality or even replaces it: Speakers have an idea about “wine talk” and about people engaging in it even though they may never have been part of a wine tasting situation themselves. This in turn opens up the possibility for new \emph{n}+1st order indexical values: The use of the forms can for example be evaluated negatively within a cultural schema that is skeptical or even highly critical of elites and elitist behavior.


The idea of indexical order is a prerequisite for Silverstein’s definition of enregisterment and register. Enregisterment is the process by which “\textit{n}-th- and \textit{n}+1st-order indexicalities are dialectically mediated” through “culturally construing and interpreting contextual formal variation as “different ways of saying ‘the same’ thing”” \citep[216]{Silverstein2003}. Enregisterment is therefore a process of cultural construction. A wine can be described with the phrase \emph{beautifully complex with an assertive backbone} or with the phrase \emph{very good and tasty}; even though both phrases do not denote the exact same thing, they are culturally constructed as expressing the same meaning. \citet[212]{Silverstein2003} calls this phenomenon “metapragmatically imputed denotational equivalence”. Through the cultural schema and metapragmatic evaluations described above, the first phrase which indexes connoisseurship (\emph{n}-th order) is enregistered (\emph{n}+1st order) and becomes part of a lexical register which \citet{Silverstein2003} labels as \textit{oinoglossia}. Registers are defined by Silverstein as “alternate ways of “saying ‘the same’ thing” considered “appropriate to” particular contexts of usage” (\citeyear[212]{Silverstein2003}). This means that even though both phrases are constructed as having the same basic meaning, they are different in that only the phrase \emph{beautifully complex with an assertive backbone} is deemed appropriate in a wine tasting situation – not only because it indexes knowledge about wine but also because it indexes knowledge about appropriate linguistic behavior in social circles where wine tastings are common. Silverstein writes further that “the register’s forms being extractable from the sum total of all possible texts in such a context, a register will consist of particular register shibboleths, at whatever analytic plane of language structure (phonologico-phonetic, morpholexical, morphosyntactic, grammaticosemantic, etc.)” (\citeyear[212]{Silverstein2003}). Applied to the example of the oinoglossia register, this means that this register can be recognized by shibboleths which are found on the lexical level (I assume that this is why Silverstein calls it a lexical register). A phrase like \emph{beautifully complex} indexes the use of the oinoglossia register, but in order to produce a coherent text it must co-occur with other linguistic forms which are also part of the register. To illustrate this, I suggest the following example sentences:
\ea\label{ex:sir} Sir, you must taste this beautifully complex wine!
\ex\label{ex:dudebeautifully} Dude, you must taste this beautifully complex wine!
\ex\label{ex:dudeawesome} Dude, you must taste this awesome wine!
\z 

These sentences show that the address terms and the adjective phrases are in a paradigmatic relationship and that their syntagmatic co-occurrence is determined by the register to which they belong. In (\ref{ex:sir}) and (\ref{ex:dudeawesome}), the address term and the adjective phrase belong to the same register or registers which are compatible with each other, whereas in (\ref{ex:dudebeautifully}) they belong to different registers which are incompatible with each other. The oinoglossia register is created by a higher-order indexicality linking \emph{beautifully complex} to a high social status and the prestige activity of wine tasting - these indexical values are not compatible with the values of American masculinity and non-conformity indexed by the address term \emph{dude} \citep{Kiesling2004}. Sentence (\ref{ex:dudebeautifully}) is unlikely to be produced by speakers, unless they want to create irony and achieve a humorous effect. It is therefore possible that linguistic forms which belong to different registers co-occur syntagmatically, but these combinations are usually marked and open to interpretations involving an even higher-order indexicality ((\emph{n}+1)+1). Silverstein therefore concludes his definition by saying that “[w]hile such shibboleths are strongly salient as indexes that the register is in use, the overall register itself consists of these plus whatever further formal machinery of language permits speakers to make text, such as invariant aspects of the grammar of their language. (A \textit{language} is thus the union of its \textit{registers}.)” (\citeyear[212]{Silverstein2003}). This definition implies that registers are recognized by means of salient linguistic forms (register shibboleths) co-occurring syntagmatically, but that they consist of other forms as well which are invariant or at least not register shibboleths of other incompatible registers.


The aspect of \emph{recognition} is also at the heart of \citegen{Agha2007} concept of enregisterment. He defines it as “processes and practices whereby performable signs become recognized (and regrouped) as belonging to distinct, differentially valorized registers by a population” (\citeyear[81]{Agha2007}). He therefore regards registers as entities that come into existence through recognition by a group of people in a particular cultural and social context and during a particular time period. It is also noteworthy that registers are not restricted to linguistic forms but comprise all kinds of signs which are performable and therefore visible to others. \textit{Semiotic registers} in a wider sense are consequently distinguished from more specific \textit{registers of discourse}. In contrast to Silverstein, whose focus is on the development of a semiotic model of enregisterment (and ultimately a model of language as a union of registers), Agha is very much concerned with modeling the cultural and social processes by which registers are created and changed continually in a socio-historical context. The definitions of semiotic register and register of discourse reflect this \citep[81]{Agha2007}:

\begin{quote}
A \emph{register of discourse}: a cultural model of action
\begin{itemize}
    \item[(a)]which links speech repertoires to stereotypic indexical values
    \item[(b)] is performable through utterances (yields enactable personae/relationships)
    \item[(c)] is recognized by a sociohistorical population
\end{itemize}
A \emph{semiotic register}: a register where language is not the only type of sign-behavior modeled, and utterances not the only modality of action. A register of discourse is a special case. 
\end{quote}

While the indexicality of (linguistic) forms is as central for Agha as for Silverstein in the emergence of registers, it is crucial that Agha defines registers not as repertoires of signs per se, but as cultural models of action. A model of action is only an abstraction of the actual, observable action, but it is necessarily based on that action and serves as a point of reference for all participants in the action and therefore influences the action that it models. The emphasis on action means that only that which is performable and therefore observable can become part of the model. In all these observable instances of action which necessarily happen in a social context, indexical links are created between signs involved in the action and aspects of the context. Agha’s main contribution is to provide a theoretical model for how these links between (linguistic) forms and indexical values, created in every instance of observable action (“the micro-time of interaction”, \citeyear[103]{Agha2007}), become cultural models of actions (“macro-social regularities of culture”, \citeyear[103]{Agha2007}), which then influence actual action again. I outline this process of enregisterment in the following paragraphs.



The key activities in enregisterment are speakers’ \textit{reflexive activities}, “namely activities in which communicative signs are used to typify other perceivable signs” \citep[16]{Agha2007}. This means that several signs are grouped together and assigned a metalinguistic predicate which relates to types of persons, types of interpersonal relationships or types of behaviors. Using the wine example again, the speaker’s use of \emph{beautifully complex with an assertive backbone}, the elegant and expensive jacket that he or she wears and the way he or she greets people with a smile and a handshake are evaluated and typified by participants in the same situation as ‘wine-connoisseur’, ‘superior social standing’ and ‘polite’. In this process, disparate cross-modal signs become icons of categories of personhood, behavior and relationships and the use of these icons comes to index characteristics of its users. The icons are therefore classified by Agha as \textit{indexical icons}, which are \textit{emblematic signs}. It is important to note that while these typifications are perceivable by definition, this does not mean that they are always expressed explicitly and linguistically. On the contrary, they are often implicit and only mediated by overt signs, e.g. by the deferential behavior of others (\citeyear[103]{Agha2007}). Individual face-to-face encounters between people (as in this example) are part of larger “communicative chain processes” and “communicative networks” (\citeyear[69]{Agha2007}) through which typifications are transmitted within a population. \citet[151]{Agha2007} provides a list of common typifications of language use (\tabref{tab:2:2}) which reflects the difference between reflexive activities in direct interpersonal interactions (1.) and in larger more indirect cultural forms of communication (3.). Typifications also occur as the result of interventions of experts (2.).


\begin{table}
\begin{tabularx}{\textwidth}{R{2.5cm}l}
% & Typifications of language use\\

\lsptoprule
 1. & Everyday reflexive behaviors, such as\\
& ~~(a) use of register names\\
& ~~(b) accounts of usage/users\\
& ~~(c) descriptions of ‘appropriate’ use\\
& ~~(d) patterns of ‘next turn’ response behavior\\
& ~~(e) patterns of ratified vs. unratified use\\
 2. & Judgements elicited through\\
& ~~(f) interviews\\
& ~~(g) questionnaires\\
& ~~(h) ‘matched guise’ experiments\\
 3. & Metadiscursive genres such as\\
& ~~(i) traditions of lexicography\\
& ~~(j) grammatology\\
& ~~(k) canonical texts\\
& ~~(l) schooling\\
& ~~(m) popular print genres\\
& ~~(n) electronic media\\
& ~~(o) literary representations\\
& ~~(p) myth\\
& ~~(q) ritual\\
\lspbottomrule
\end{tabularx}
\caption{
Typifications of language use in Agha's framework of enregisterment (from \citealt[151]{Agha2007})
}
\label{tab:2:2}\label{tab:key:2}
\end{table}

It is essential in enregisterment that one instance of metapragmatic activity cannot constitute a register, but that typifications have to be recurrent in the behavior of many speakers. It is only through the transmission and recurrence of evaluations of forms that they become recognized and distinguishable as a register. The frequently recurring typifications become “stereotypes of indexicality” or “metapragmatic stereotypes” (\citeyear[151-153]{Agha2007}) which means that they become social regularities. It is evident that especially the reflexive behaviors in (i)-(q) of \tabref{tab:2:2} contribute to this development because they involve mass communication of some sort where linguistic forms and their indexical values are brought to the attention of a large audience. It is possible that these stereotypes become so widely known and accepted that they become “a routinely background reality for very large groups of people” and therefore “socially routinized metapragmatic constructs (such as beliefs, habits, norms, ideologies)” (\citeyear[29]{Agha2007}). In the transmission process, however, typifications are also negotiated and transformed. Here, Agha draws on Silverstein’s orders of indexicality: If an evaluation of a set of forms becomes so common that it becomes presupposable by many people, it can become subject to reanalysis which could affect both the forms (regrouping) and the indexical values (revalorization). For example, if the indexical link between \emph{beautifully complex with an assertive backbone} (together with other signs) and the image of an ‘expert of wine’ and ‘polite’ behavior and social ‘superiority’ becomes transmitted to a large audience (possibly through advertisements), it may become so strong that it becomes subject to evaluation itself, for example it could be revalorized as an index of snobbery. This does not imply that the former valorization disappears; it is rather the case that there may be competing valorizations which co-exist and depend on the evaluator. What is ‘polite’ and ‘sophisticated’ for a person who is part of higher social circles may be ‘snobbish’ for a person who is not part of these circles. This is why Agha defines registers as dependent on the people recognizing its forms and evaluating them as different from other forms. It also explains why Agha finds questions of boundaries of registers “fruitless and misplaced” (\citeyear[168]{Agha2007}) because any boundaries associated with registers are continually negotiated and reset in the processes of enregisterment. \tabref{tab:2:3} provides an overview of the dimensions of register organization and change that \citet[169]{Agha2007} postulates:


\begin{table}
\begin{tabularx}{\textwidth}{R{1cm}X}
\lsptoprule
A.  & \textbf{Repertoire characteristics} \\
&
  \textit{Repertoire size}: number of forms
\\
&
  \textit{Grammatical range}: number of form-classes in which forms occur
\\
&
  \textit{Semiotic range}: variety of linguistic and non-linguistic signs associated with use
\\
 B. & \textbf{Social range} of enactable (pragmatic) values\\
&
  \textit{Indexical focus}: Stereotypes of speaker-actor, relation to interlocutor, occasion of use, etc.
\\
&
  \textit{Images} (or icons) stereotypically attached to indexical sign-forms: for speaker-focused indexicals, persona types (male/female, upper/lower class, etc.); for interlocutor-focal indexicals, types of relationship (deference, intimacy, etc.)
\\
&
  Positive or negative values associated with the registers
\\
 C. & \textbf{Social domain(s)}: Categories of persons acquainted with the register formation\\
&
  \textit{Domain of recognition}: persons who recognize the register’s forms
\\
&
  \textit{Domain of fluency}: persons fully competent in the register’s use
\\
\lspbottomrule
\end{tabularx}
\caption{Some dimensions of register organization and change (from \citealt[169]{Agha2007}}
\label{tab:2:3}
\end{table}

\largerpage
Registers can be characterized with regard to the three dimensions A--C, but “any such account is merely a sociohistorical snapshot of a phase of enregisterment for particular users” \citep[170]{Agha2007}. Even though Agha places particular emphasis on this processual perspective, he nevertheless also stresses that there are mechanisms which work towards a relative stability and persistence of registers. Institutions of various kinds play an important role here; it is possible, for example, that the forms of a register are codified in dictionaries and grammars and used in educational institutions. This is obviously the case for standard registers whose size is usually not only much larger than that of other registers, but also more resistant to change. Agha uses the example of Received Pronunciation to illustrate how phonological forms come to be enregistered as a national standard of pronunciation in England.\footnote{This article was first published in 2003 and has been re-published in 2007 as Chapter 4 in \emph{Language and social relations}.} Furthermore, processes of \textit{essentialization} and \textit{naturalization}, which also play an important role in Silverstein’s model of enregisterment, establish a natural motivation of the link between the register’s forms and values \citep[74]{Agha2007}. Metapragmatic activity which repeatedly presents the values indexed by the forms as natural qualities of its users causes people to disregard the fact that the indexical link has in fact been socially constructed. This is for example the case when the use of standard forms is so commonly linked to the attribute ‘intelligent’ in metapragmatic activity that language users start considering intelligence as a \emph{natural} characteristic of the people using the forms. Using standard forms even comes to be seen as an essential quality of the group of intelligent people. It is easy to see how normative criteria are based on these essential qualities: In such a scenario, people must use standard forms in order to be seen as intelligent by others. Stabilizing mechanisms are reinforced when they are formulated and backed by authority, for example by institutions of expertise, or by people who have been assigned an expert role. Despite these mechanisms of stabilization, however, registers are always subject to reanalysis and transformation; the difference lies in the speed and the extent to which that happens.


What, then, is the effect of registers? The circulation of metapragmatic stereotypes and images of personhood associated with them provide the ground for role alignments of speakers in interaction. This means that they can either signal their sameness and their co-membership in a social category (symmetric alignment) or their difference (asymmetric alignment) \citep[133]{Agha2007}. This happens in every instance of interaction, but the existence of culturally shared stereotypes also leads to social regularities in role alignment. Alignments are particularly motivated by so-called \textit{characterological figures} linked to registers. Such a figure is “any image of personhood that is performable through a semiotic display or enactment (such as an utterance). Once performed, the figure is potentially detachable from its current animator in subsequent moments of construal and re-circulation” \citep[177]{Agha2007}. This means that links between forms and their indexical values become less abstract, but embodied, and as such more readily inhabited by speakers to signal their social identity.

Comparing \citegen{Silverstein2003} and \citegen{Agha2007} definition of register, it becomes clear that they are very similar in many respects. They see the metapragmatic engagement with perceivable signs (linguistic and others) as the key process in the emergence of registers as it is through this engagement that the signs come to index social values (on the \emph{n}+1st indexical order). They both stress the interrelation between micro-contexts and macro-contexts (Silverstein) and individual face-to-face encounters and large-scale cultural processes (Agha) because registers cannot come to exist on one of these levels only. However, there is also an important difference, which can be explained by the fact that they pursue slightly different aims with their theories. Silverstein’s aim is more (socio-) linguistic in that he wants to show how sociolinguistic analysis needs to be completed by studying not only the \emph{n}th-order of indexicality but also the \emph{n}+1st-order. Agha’s theoretical orientation is more sociological as he aims to describe and explain the role that language has in social life and its impact on social relations. For Silverstein, accordingly, a register comprises all linguistic forms needed to make a text which are judged to be appropriate in context, but while some forms are salient and point to the existence of registers, some forms are not salient or simply invariant. He elaborates on this view in a recent article where he states that

\begin{quote}
Language users evaluate discourse with intuitive metrics of coherence of enregistered features of form co-occurring in text-in-context across segmentable stretches of discourse such as an individual’s contribution to discursive interaction, generally focusing on highly salient ‘register shibboleths’ that reveal a basic register setting around which cluster the untroubled compatibility or indexically marked lack of compatibility of other aspects of usage. \citep[59--60]{Silverstein2016}
\end{quote}


His statement that a language is a union of its registers is basically a theoretical claim about the nature of language. Agha’s definition of registers as models of conduct is not so much language-theoretical but rather social. By emphasizing that models are reference points for social (including linguistic) behavior, he distinguishes them at the same time from actual conduct and language use. Therefore, a register does not comprise all forms needed to produce text in an actual context, but only those forms which are recognized by people as part of the register. It can therefore vary in size – some registers comprise only few forms, while others, especially standard registers, comprise a very large number. It seems as if Silverstein locates registers more on the level of actual language use, while Agha locates them on a discursive level: It is language through which such models are formulated (the terms \textit{discourse} and \textit{discursive} will be discussed extensively in \sectref{bkm:Ref506891065}). Nevertheless, as both stress the interrelation between the level of language use and the discursive level, this difference is more a reflection of their different research aims than a difference between their theories: Silverstein’s registers, located on the level of language use, only exist because of the metapragmatic engagement of language users with some of its forms; Agha’s registers, located on the discursive level, have a bearing on actual language use and in these instances of actual use the register’s forms of course co-occur with other forms. These forms can be congruent with the register or they may “by degrees, cancel the stereotypic values” indexed by the register’s forms, as in the example sentence (\ref{ex:dudebeautifully}) above.


This discussion of Silverstein’s and Agha’s conception of registers and enregisterment shows why enregisterment is a useful theoretical framework for studying the emergence of new varieties: First of all, it provides a way of linking the two levels identified in Schneider’s Dynamic Model, the level of the structure and the level of the concept of a variety, in one theoretical framework, the structural one corresponding to actual language use and the conceptual one corresponding to the discursive level of metapragmatic activity and engagement with linguistic forms (and other signs). At the same time, the theoretical framework of enregisterment also provides a basis for systematically distinguishing between these levels in a study of the emergence of a new variety. This in turn is the basis for providing empirical support for the important role of social factors and people’s perceptions, beliefs and attitudes in the emergence of new varieties if a changing statistical correlation between linguistic forms and social and geographic categories can be shown to correlate with changing metapragmatic activities surrounding these forms. Secondly, Agha and Silverstein elaborate central ideas of the Dynamic Model in much more detail: the interrelation between identity construction and the construction of linguistic difference as well as the emergent \emph{recognition} of forms as differential by a population, which is important because it is part of Schneider’s definition of the term variety itself (see \sectref{bkm:Ref521000690}). The third advantage of the theoretical framework of enregisterment is that it provides a methodological basis for studying the discursive level. By emphasizing that metapragmatic activity is by definition observable it becomes clear that it can be studied empirically. Agha’s and Silverstein’s case studies and examples are mostly qualitative and intended to underline their theoretical arguments; in this study, however, I develop a methodology for extensive case studies which combine both quantitative and qualitative methods to analyze enregisterment processes in nineteenth-century America and contribute to a theoretically informed account of the emergence of American English.

\subsection{Indexing varieties: enregisterment in (historical) sociolinguistics}
\label{bkm:Ref506884048}\hypertarget{Toc63021212}{}
Given its emphasis on theorizing the relation between language and social behavior, it is not surprising that the concept of enregisterment has caught the interest of sociolinguists. Silverstein himself links his order of indexicality explicitly to sociolinguistic research by aligning it with \citegen{Labov1972} order of linguistic variables as indicators, markers, and stereotypes as well as with the difference between dialectal and superposed variability posited by \citet[383--384]{Gumperz1968}. But it was essentially a change of orientation in sociolinguistics, which \citet{Eckert2012} describes using the metaphor of three succeeding waves, that has sparked the interest first of all in indexicality and increasingly also in enregisterment. While the first wave of sociolinguistics was concerned with discovering the systematic relations between linguistic variation and macro-social categories like social class, age, gender as well as race and ethnicity by using mainly quantitative methods, second wave studies rather looked at how language use correlated with social categories which were relevant to a specific group of speakers on a local level by adding qualitative and ethnographic methods to the research design. Third wave studies shifted the focus from correlations to agency: Instead of viewing linguistic variation as a reflection of social categories they concentrate on the way that speakers make use of variation to construct and express social categories and identities. Consequently, the social meaning created through stylistic practice is at the center of interest of third wave studies and the concept of indexical order is very helpful in this respect because it shows how linguistic forms come to index social attributes and how these indexical meanings co-exist but also change over time. \citet{Eckert2008} develops the concept of the \textit{indexical field} which she defines as “a field of potential meanings” of variables, or “constellation of ideologically related meanings, any one of which can  be activated in the situated use of the variable” (\citeyear[454]{Eckert2008}). It is evident here how the emphasis on activation reflects the importance of agency: By activating one or more indexical meanings of a variant, the language user contributes to the maintenance but also to the change of the indexical field, which is in constant flux. Eckert illustrates the concept by using the released /t/ as an example. Drawing on several studies conducted in American contexts, Eckert constructs an indexical field (see \figref{fig:2:2}) consisting of social personae indexed by released /t/, qualities, which are seen as permanent, and stances, which are rather momentary and tied to specific situations but can become constructed as part of people’s identity if they are habitually expressed.


\begin{figure}
\includegraphics[width=0.8\textwidth]{figures/Paulsen-img02.pdf}
\caption{Indexical field of released /t/: social personae (in circles), permanent qualities (in grey boxes) and stances (in italics), my own illustration based on \citet[469]{Eckert2008}}
\label{fig:2:2}
\end{figure}


The indexical field created here emphasizes again the importance of social personae (similar to Agha’s characterological figures) which, according to \citet[470]{Eckert2008}, anchor the process of interpretation because they are less fluid than permanent qualities and situated stances. \citegen[454]{Eckert2008} proposal to study “variation as an indexical system, taking meaning as a point of departure rather than the sound changes or structural issues that have generally governed what variables we study and how we study them” is therefore a programmatic statement for third wave sociolinguistic studies. Using the field metaphor, she makes the point that Silverstein’s indexical order is not to be understood as linear (as it is the case in the first sociolinguistic studies on enregisterment which I will discuss below), but as a continual reconstrual of indexical value. At the same time, she also draws on Agha by claiming that “variables combine to constitute styles” (\citeyear[472]{Eckert2008}) and that styles are “the product of enregisterment” (\citeyear[456]{Eckert2008}). However, she does not employ the term \textit{register} because she regards the common definition of register as “a static collocation of features associated with a specific setting or fixed social category” (\citeyear[456]{Eckert2008}) as too established in sociolinguistics. However, as her own definition of \textit{style} is also very different from the established definitions of the term in traditional variationist studies, her argument against the use of the term \textit{register} is not very convincing – it rather underlines the necessity to develop and redefine established terms and concepts.\footnote{For an extensive overview of the development and definitions of the concept of \textit{style} see e.g. \citet{Coupland2007}.} This is precisely what Agha does for the term \textit{register} by criticizing earlier views (\citeyear[167--170]{Agha2007}). Nevertheless, it becomes clear that Eckert places herself firmly in the field of sociolinguistics and aims at developing her own theoretical framework of style (instead of using a framework from linguistic anthropology) and testing its usefulness for the study of sociolinguistic variation. At the same time, she also stays within the variationist sociolinguistic tradition by focusing on the indexical potential of single variants and by studying actual usage and behavior, not metadiscursive activities. It is interesting that she does identify a “need to examine a far greater range of variables than is commonly done in the field” (\citeyear[472]{Eckert2008}) and a need to address questions of the structure of styles and to model the process of \emph{bricolage}, the process whereby individual (linguistic) resources are “interpreted and combined with other resources to construct a more complex meaningful entity” (\citeyear[456--457]{Eckert2008}), since it is precisely these issues that Agha and Silverstein address in their theory of enregisterment. In a recent article, she elaborates more on the relation between her conception of style and Agha’s register: She describes Agha’s register as “a style that is enduringly associated with some widely recognized character type such as Posh Brit or Surfer Dude” and therefore as “an outcome of stylistic practice” \citep[76]{Eckert2016} in which people do not \emph{use} registers, but refer to them and draw on them in their actual language use when they make smaller or larger interactive moves. By describing register as “a sign at a particularly high level of consensuality and metadiscursivity” \citep[76]{Eckert2016} she locates it on a conceptual level and distinguishes it from actual linguistic and stylistic practice. Although Eckert captures a very important point in Agha’s theory here, namely that registers are \emph{models} of action and not action per se, it is an account of Agha’s theory that is too reductionist. In Agha’s model, a differential metapragmatic treatment of sets of forms is already sufficient to indicate the existence of a register because it points to a differential evaluation of these forms: Users associate one set of forms with different indexical values than another set of forms. In Agha’s view, explicit metadiscourses on speech forms and the values that they index \emph{can} exist and they often do exist, especially in the form of characterological figures embodying these links, but it is not a prerequisite for the existence of registers. Registers may acquire a very large social domain (especially with respect to the domain of recognition), but their domain can theoretically also be much smaller. Instead of viewing registers as specific types of style, Agha distinguishes them in a different way and adds the concept of \textit{enregistered style}. Styles are “patterns of co-occurrence among semiotic devices” \citep[186]{Agha2007} and these devices include linguistic and non-linguistic tokens. Every utterance can therefore be described as a co-occurrence style because in an actual interaction tokens cannot occur in isolation, but they always co-occur and therefore create an observable formal pattern. Only when a formal co-occurrence pattern is differentially evaluated, Agha speaks of enregistered style\textit{s} because they have acquired a cultural intelligibility and significance. It is these enregistered styles that are “reflexively endogenized to a register model” \citep[186]{Agha2007}. \figref{fig:2:3} illustrates this integrative view of styles and registers.



%%please move the includegraphics inside the {figure} environment
%%\includegraphics[width=\textwidth]{figures/Paulsen-img003.emf}


\begin{figure}
\includegraphics[width=.8\textwidth]{figures/Paulsen-img03.eps}
\caption{Style, enregistered style and register based on definitions by \citet{Agha2007}}
\label{fig:2:3}
\end{figure}


Third wave sociolinguistic studies basically study the relation between style and enregistered styles because their aim is to identify the indexical meaning of (combinations of) linguistic variants, thereby paying attention to how they are linked to other signs in the social landscape. This indexical meaning is used by speakers to position themselves socially by making stylistic moves, thereby changing styles and creating the possibility for new indexical meanings.\footnote{In my view, Silverstein’s notion of register can therefore be equated with Agha’s notion of enregistered style.} Agha adds another dimension to these studies by explaining how this process works: Indexical meaning is created through reflexive activities, resulting in a register which is a model of action and therefore not to be equated with a pattern occurring in actual use. Nevertheless, it has an influence on language use as the model is performable in actual stylistic practice. In addition to an empirical study of language use, Agha sees an empirical study of reflexive activity as crucial in identifying how people create and transform social identity and social relations through language.\footnote{In her well-known classification of sociolinguistic studies in three waves, Eckert also states that “every case of variation [discussed in her article] involves enregisterment” (\citeyear[96]{Eckert2012}), but she does not clearly distinguish or relate enregisterment and style in this article. This makes it difficult to follow her argumentation that enregisterment loses its analytic force at some point because nuances of sound, such as fortition or lenition, cannot fruitfully be regarded as components of registers because it is \textit{only} “in continual stylistic practice that nuances of sound take on sufficient meaning to participate in processes of enregisterment” (\citeyear[97]{Eckert2012}).}


Another theoretical framework in which the notion of indexicality plays an important role and which is closely related to Eckert’s theory of style and the indexical field is \citegen{Bucholtz2005} framework for the analysis of identity. They regard identity not as an internal psychological phenomenon, but as a social and cultural phenomenon produced through linguistic and other semiotic practices. Identity, defined as “the social positioning of self and other” (\citeyear[586]{Bucholtz2005}), is linguistically indexed in various ways: through labels, implicatures, stances, styles, or linguistic structures and systems. This incorporation of linguistic structures and systems in their \textit{indexicality principle} is particularly interesting because it is reminiscent of \citegen[18]{Agha2007} idea that “acts of value ascription to language can […] acquire much more generic discursive objects (e.g., entire speech varieties), and become habitual for large groups of evaluators”. \citet[597]{Bucholtz2005} do not elaborate on that point in more detail, nor do they employ the concept of enregisterment, but they do mention the works on language, nationalism and ideology that inform this view, especially that of the linguistic anthropologists \citet{Gal1995}. These two researchers elaborate on the semiotic processes by which differences between languages and dialects are constructed through linguistic ideologies: “[I]deologies interpret linguistic structure, sometimes exaggerating or even creating linguistic differentiation” (\citeyear[993]{Gal1995}). They argue that this construction process is not only shaped by the speakers (“the immediate participants in a sociolinguistic field”, \citeyear[977]{Gal1995}), but also by scholars (“professional observers”, \citeyear[993]{Gal1995}) and criticize that

\begin{quote}
Although it is now a commonplace that social categories—including nations, ethnic groups, races, genders, classes—are in part constructed and reproduced through symbolic devices and everyday practices that create boundaries between them, this analysis is only rarely extended to language. Despite a generation of sociolinguistic work that has persistently provided evidence to the contrary, linguistic differentiation—the formation of languages and dialects—is still often regarded as an a social [sic] process. \citep[969]{Gal1995}
\end{quote}


As a potential reason for this, Gal \& Irvine identify nineteenth-century ideologies which equated one language with one culture – an equation which was then used as a basis to claim nationhood and territory. They argue that precisely because of the idea that language is independent of social activities it could be used for the identification of nations. As the unity of the nation was identified based on the unity of language, linguistic homogeneity became an important characteristic of nations, and ideologically-driven processes of erasure led to the eradication of internal variation to such an extent that people viewed language as free of variation and as a fixed system that should not change. Borders between languages and dialects are therefore socially constructed – a line of argumentation that is continued by Agha’s proposal that these construction processes must be paid attention to and complement (if not replace) traditional variationist analyses:


\begin{quote}
The terms dialect and sociolect describe forms of variation in the denotational system of a language community. […] Dialects may exist and be describable by linguists but groups speaking these dialects may be separated in various ways so that cross-dialect contact among persons does not occur and the existence of dialect differences is not even suspected by most (in principle, by any) speakers of the language. Dialect differences are relevant to social life only insofar as they are experienced through communicative events. How such relevance is construed society internally is an empirical question that will have different answers in different sociohistorical locales. \citep[132]{Agha2007}
\end{quote}


What people view as languages and dialects are registers (in Agha’s sense of the term) which are constructed through an interpretation of linguistic variation through reflexive models – \citet[135]{Agha2007} calls this “the reflexive construal of such ‘-lects’ \emph{as} registers”. Bucholtz \& Hall’s statement that identity can be indexed through linguistic structures and systems can therefore be interpreted by using the framework of enregisterment: Linguistic forms have the potential for being grouped with others and evaluated as a distinct speech variety (a dialect, a sociolect, a language) by speakers in a process which links these forms to aspects of the social identity of speakers using these forms. And it is in fact this potential of forms to become indexes of speech varieties that has been the focus of sociolinguistic studies of enregisterment to date. In the following paragraphs I outline and discuss the most important studies on enregisterment in order to demonstrate that although these studies provide numerous important insights, some theoretical and methodological clarifications and adjustments are necessary to make enregisterment a fruitful theoretical framework for the emergence of new varieties of English.


The first sociolinguistic study using Silverstein’s orders of indexicality and Agha’s concept of enregisterment as a theoretical framework is \citegen{Johnstone2006} study on the enregisterment of \textit{Pittsburghese}. They relate their study to \citegen{Agha2003} case study on the historical enregisterment of Received Pronunciation by emphasizing that enregisterment can also be used to explain how varieties which do not carry overt prestige can nevertheless also become standardized through the development of vernacular norms (\citeyear[80]{Johnstone2006}). Furthermore, they take up Silverstein’s point that \citegen{Labov1972} trichotomy of indicators, markers and stereotypes can be captured by orders of indexicality, the latter being more abstract and providing the advantage of a more nuanced understanding of how relationships between linguistic forms are formed and stabilized (\citeyear[81]{Johnstone2006}). Their study is of great value because it links three fields of research: dialectology, sociolinguistics and linguistic anthropology. They use dialectological evidence to describe linguistic variants which are used in Pittsburgh or in the Pittsburgh metropolitan area, but which are at the same time not limited to this geographical area (the only variant with a small area of occurrence is monophthongal [aː] in the lexical set \textsc{mouth}), and can therefore not be used to identify a set of linguistic forms distinctive of this area. They use sociolinguistic interviews to determine an index score for the use of the monophthongal variant of \textsc{mouth} by five speakers to measure the extent to which they use this local variant. At the same time, the interviews are used to study their perception and their evaluation of this variant. They find that there are two groups of speakers with respect to usage: The two oldest speakers use the local monophthongal variant most of the time, while the middle-aged speakers and the youngest speaker use it hardly at all. In terms of attitudes and perception, they find that one of the older speakers is not aware of using the variant and does not recognize it as local, while the other speaker has learnt to recognize this variant as local and, while knowing about negative evaluations of the variant, regards it as rather neutral (he states that sounding like a working-class Pittsburgher is not “a big deal” (\citeyear[89]{Johnstone2006}). In the second group, the two middle-aged speakers regard the variant as local, incorrect and as signaling working-class membership. One of them, however, also indicates a potential of signaling solidarity by using the variant. The youngest speaker in the second group regards the monophthongal variant as an indicator of local speech and local identity, but in contrast to the other speakers, he does not associate any negative values with it. These findings, which are summarized in \tabref{tab:2:4}, are interpreted by drawing on the anthropological concepts of orders of indexicality and enregisterment.

John K.’s almost invariable use of [aː] and his complete lack of recognition of the form in his own speech is interpreted as the variant being a first-order-index in his speech: He does not use it to express social identity but because he is from the region. According to Johnstone et al., this situation is typical of the time until the 1960s, when the localness of the variant [aː] was observable to outsiders, but not to speakers themselves. Dottie X. confirms this experience but recognizes the variant now as local and working-class. Since she evaluates it neutrally (it is not a problem for her to sound working class), she also does not use the variant to do social work and it is therefore also seen as a first-order index. Arlene C. on the other hand, evaluates the monophthongal variant negatively as incorrect and working-class and uses the diphthongal variant to the extent that even a pattern of hypercorrection can be observed (\citeyear[91]{Johnstone2006}).

\begin{table}
\begin{tabularx}{\textwidth}{p{2cm}>{\raggedright}p{1.7cm}Qp{2.1cm}@{}}

\lsptoprule
~\newline Speaker & \mbox{Use of  [aː]}\newline \mbox{in \textsc{mouth}} &  ~\newline Perception and evaluation of [aː] &   Order of\newline  indexicality\\
\midrule
Dr. John K.,\newline born 1928 & high & {\textbullet} No recognition of the variant in his own speech

{\textbullet} Evaluation of the variant as not local, as working-class and as signaling a lack of education when asked to compare it to the diphthongal variant & First-order\\
\tablevspace
Dottie X.,\newline born 1930 & high & {\textbullet} Recognition of the variant now, but not when she was younger

{\textbullet} Neutral evaluation of the variant as local and working-class & \\
\tablevspace
Arlene C.,\newline \mbox{born ca. 1940} & low & {\textbullet} Recognition of the variant, which is evaluated negatively as incorrect and working-class (and not as a marker of local identity) & Second-order\\
\tablevspace
Barb E.,\newline born 1957 & low & {\textbullet} Recognition of the variant, which is evaluated negatively as incorrect and working-class, but also positively as a marker of social solidarity with fellow Pittsburghers & \\
\tablevspace
Jessica H.,\newline born 1979 & low & {\textbullet} Recognition of the variant, which is evaluated positively as a marker of local Pittsburgh identity & Third-order\\
\lspbottomrule
\end{tabularx}
\caption{
Speakers’ use and perception and evaluation of [aː] in \citegen{Johnstone2006} study on the enregisterment of Pittsburghese and interpretation in terms of orders of indexicality
}
\label{tab:2:4}
\end{table}

In Johnstone et al.’s view, this marks the second-order-indexicality of the variant, as it is avoided by Arlene C. to avoid stigmatization. Barb E.’s use of [aː] is very low, but her evaluation of [aː] is interpreted as an indicator of stylistic variability in her own speech because she regards the use of the variant negatively in some contexts (with its potential to index incorrect and working-class speech), but also positively in other contexts (with its potential to index solidarity among fellow Pittsburghers). This potential to index localness in a positive way is foregrounded in the evaluation of [aː] by Jessica H. As older, negative evaluations of the forms are now in the process of being replaced by a new, positive one, the form is now interpreted as being a third-order index by \citet{Johnstone2006}. A further indication of its third-order status is that it is not stylistically variable in Jessica H.’s own speech and not used in her everyday interactions, but \textit{only} in explicit performances of identity, which she describes in the interview. This performance of local Pittsburgh identity by drawing on a specific set of linguistic forms, including but not limited to [aː], is supported or even made possible by a growing number of metadiscursive activities. \citet{Johnstone2006} analyze twenty newspaper articles about local speech and observe that they have appeared more regularly since the 1950s and 1960s and that they evaluate the supposedly regional forms they describe in a disparaging way (\citeyear[95]{Johnstone2006}). In the 1960s and early 1970s, the attitude shifted mostly due to experts like the University of Pittsburgh dialectologist Robert Parslow, who legitimized these forms and assigned them to a local dialect in several interviews. Other materials analyzed by \citet{Johnstone2006} also constitute evidence of an increasingly favorably evaluation of (supposedly) local forms which are listed in a folk dictionary titled \emph{Sam McCool’s New Pittsburghese: How to Speak Like a Pittsburgher} (\citeyear{McCool1982}, cited by \citealt[96]{Johnstone2006}) and which became commoditized by being written on T-shirts, mugs, postcards, shot glasses and similar products. It is in this context that younger speakers like Jessica H. now recognize forms like [aː] in \textsc{mouth} and associate them with Pittsburgh speech and use them in performances of local identity even though they do not use them in their everyday speech.


The reason for summarizing this study in such detail is that it has been very influential – it is cited in every sociolinguistic work drawing on enregisterment – and has therefore shaped the understanding of theoretical issues, such as the delimitation of the different orders of indexicality and the relationship between these orders and enregisterment, as well as methodological approaches to the study of enregisterment within a broader sociolinguistic framework. There are three theoretical points made by \citet{Johnstone2006} that need closer attention and which I will discuss in detail below: They relate to the role of awareness in delimiting different orders of indexicality, the relationship between orders of indexicality and enregisterment and the relationship between production and recognition of language forms.

Awareness (or consciousness) is an important factor for \citet{Johnstone2006} in delimiting the different orders of indexicality. While a correlation between a linguistic form and an extra-linguistic characteristic is not noticeable to speakers on the level of first-order-indexicality, it is crucial for second-order-indexicality that speakers start to notice the correlation and use it to do social work. When a linguistic form is explicitly discussed in metadiscursive activities and used consciously in performances of identity, it is a third-order index of a particular aspect of identity (here of \emph{local} identity). The difference here seems to be not only one of awareness but also of intentionality: Only when a linguistic form is intentionally and reflexively used to signal a social characteristic is it a third-order index. Their view is in line with Labov’s classification of indicators, markers and stereotypes, which also rests on differing levels of awareness, from not being aware of social correlations (indicators) to not necessarily being aware of them, but definitely reacting to them (markers), to being very aware and explicitly discussing them (stereotypes). It is not in line with \citet{Silverstein2003}, however, who explicitly argues against Labov’s views by stating that

\begin{quote}
Where the Labovian sociolinguistic marker differs from the mere indicator is the inherent interaction of whatever SEC-indexing rates of production of standard with what we might term \emph{register demand} (a species of tasks demands in the normal psychological sense, and having nothing inherently to do with “consciousness,” contra Labov’s speculation). \citep[218]{Silverstein2003}
\end{quote}


This shows that he also regards style shifts (i.e. intra-speaker variation in different situations) as evidence for the respective forms being second-order indexes because speakers react to the “demands” of an existing register by changing those forms which are linked to it. But the process of indicators becoming markers (or first-order indexes becoming second-order indexes) is independent of whether speakers do this consciously or not. This view on the role of awareness in style shifting can also be found in sociolinguistic theories belonging to the third wave. In a recent theoretical article, \citet{Eckert2016} argues against the traditional sociolinguistic emphasis on consciousness and awareness, which is also visible in distinctions like \textit{change from below} and \textit{change from above} as well as \textit{overt prestige} and \textit{covert prestige}. In her view, “consciousness and awareness are not simple matters, and agency does not equal or require awareness” (\citeyear[78]{Eckert2016}). Rather than being external to cognition, “the social is embedded in the unconscious to the same extent, in the same way, and along the same timeline, as the linguistic” (\citeyear[78]{Eckert2016}). To support her argument she cites experimental studies which show that speech perception is influenced by social information about the speakers \citep{DOnofrio2015} and nonlinguistic information like the presence of stuffed toy kangaroos and koalas or toy kiwis invoking associations to Australia and New Zealand respectively \citep{Hay2010}. Using eye-tracking, \citet{DOnofrio2015} shows that persona-based information affects early and automatic speech processing (which is not under conscious control by speakers) because it leads speakers to expect a particular vowel which they associate with the persona. Even though these studies focus on perception, they suggest that speakers’ sensitivity to social information is not necessarily conscious but can be automatic as well. Given these findings, it is doubtful whether consciousness or awareness are helpful constructs in delimiting different orders of indexicality. What is important instead is the metapragmatic engagement with language which is empirically observable evidence for \emph{n}+1st-order indexicality and therefore of enregisterment. If we can observe speakers’ reflexive activities in which they typify perceivable signs, we have evidence of the construction of a register which then influences speakers’ conscious or unconscious perceptions and productions. What makes \emph{n}+1st-order indexicality different from the (\emph{n}+1)+1st-order of indexicality is in Silverstein’s view not primarily a matter of consciousness but of presupposition:


\begin{quote}
Labovian sociolinguistic ‘stereotypes’, of course, are markers that have tilted in the direction of ideological transparency, the stuff of conscious, value-laden, imitational inhabitance – consciously speaking “like” some social type or personified image […]. The values of stereotypes are presupposed in the social-structure-as-indexed according to an ideological model, pure and simple; \textit{n}+1st-order indexicality has become presupposing, in other words, in effect replacing an older n-th-order indexical presupposition. \citep[220]{Silverstein2003}
\end{quote}


When the values indexed on the \emph{n}+1st-order become presupposing, they become open for reanalysis and reinterpretation. The existence of metadiscursive activity involving explicit comments on language use and the use of the features in stylized performances cannot be regarded as evidence for (\emph{n}+1)+1st-order of indexicality per se, but if it occurs repeatedly, it shows that the form-value links are transmitted and become more and more stable and therefore potentially presupposing. Using the example from \sectref{bkm:Ref512260235} again, it is possible that the adjective phrase \emph{beautifully complex with an assertive backbone} is explicitly commented on in a magazine article giving readers suggestions as to how they can describe wine at wine tastings. This contributes to the transmission and stabilization of the oinoglossia register but it is not an indicator of a (\emph{n}+1)+1st-order of indexicality as the values associated with the form are not subject to reinterpretation. However, if an article makes fun of the use of such phrases and suggests that time is better spent drinking wine instead of describing it, the wide recognition of the phrase as belonging to the oinoglossia register is the basis for its negative revalorization. This is an instance of (\emph{n}+1)+1st-order of indexicality. To sum up this discussion, it is not convincing that consciousness or awareness would be helpful concepts in separating the different orders of indexicality. Metapragmatic engagement with language, including (but not restricted to) all types of reflexive activities listed in \tabref{tab:2:2}, is best seen as evidence of second-order indexicality. Metapragmatic activities which build on second-order indexicality to reanalyze and reinterpret these indexical values are evidence of (\emph{n}+1)+1st-order of indexicality.


The second issue is the relation between different orders of indexicality and enregisterment. \citegen{Johnstone2006} study seems to equate the enregisterment process with succeeding orders of indexicality, such that a form is not enregistered when it is a first-order index, somewhat enregistered when it is a second-order index, and fully enregistered when it is a third-order index. This conception is implicitly or explicitly found in most other works on enregisterment. \citet[137--138]{Beal2012} for example describes the process of enregisterment as follows:

\begin{quote}
Enregisterment comes about through the ‘indexing’ of linguistic features as associated with social characteristics of speakers. Applying a language-ideological approach first developed by \citet{Silverstein1976, Silverstein1998}, \citet{Milroy2000, Milroy2004} asserts that there are three orders of indexicality whereby linguistic forms are associated with social categories. The orders relate to ascending levels of awareness within and beyond the speech community:
\begin{itemize}
\item 
\emph{First-order indexicality}: the association of a particular linguistic form and some specific social category. At this stage the association may be noticed by, for example, linguists, but speakers themselves are unaware of it.
\item 
\emph{Second-order indexicality}: speakers may rationalize and justify the link between the linguistic form and a particular social category.
\item
\emph{Third-order indexicality}: forms which have been linked with a certain social category become the subject of overt comment.
\end{itemize}
\end{quote}


This characterization illustrates again the problematic distinction between the different orders of indexicality based on different levels of awareness and the close parallel to Labov’s indicators, markers and stereotypes. It is especially unclear why speakers need a lower level of awareness for rationalizing and justifying the link between the linguistic form and a particular social category than for overtly commenting on it. \citet[167]{Milroy2004} in fact states that “\emph{second-order indexicality} is a metapragmatic concept, describing the noticing, discussion, and rationalization of first-order indexicality”. Her definition therefore includes the \textit{discussion} of linguistic forms, which is by definition overt and explicit, on the level of \emph{second}{}-order indexicality, but also other forms of metapragmatic engagement without recourse to notions of awareness, while third-order indexicality is not defined or discussed in this article at all. In line with \citet{Silverstein2003}, this definition of second-order indexicality rather emphasizes the important role of language ideologies, which are defined by \citet[193]{Silverstein1979} as “sets of beliefs about language articulated by users as a rationalization or justification of perceived language structure or use” (cited by \citealt[166]{Milroy2004}), in the emergence of second-order indexicality. Even though she does not use the concept of enregisterment in her article either, it is telling that her main focus is on first- and second-order indexicality because, as shown in \sectref{bkm:Ref512260235}, this is where Silverstein locates enregisterment. Following his view, forms become already enregistered by becoming second-order indexes. In fact, the data presented and discussed by \citet{Johnstone2006} and \citet{Beal2009, Beal2012} can be interpreted in a different way. In the Pittsburgh case, the analysis shows that the linguistic forms have first been enregistered as non-standard, based on a linguistic ideology associating local forms as deviating from a national prestige variety. A form like [aː] therefore became a second-order index of primarily social characteristics and the resulting register is rather a more general register of \textit{non-standard speech}. The explicit metapragmatic judgments elicited through the interviews and the metadiscursive activities identified in newspaper articles are evidence for this enregisterment process. The results of the interviews (see \tabref{tab:2:4}) also show that the social domain of the register has expanded over time. While Dottie X. did not recognize the variant [aː] and did not associate it with local and incorrect speech when she was young, she recognized the variants and the values indexed by it at the time of the interview. John K. still does not recognize the variant as local. So the social domain of the register has expanded so that it includes Dottie X. but not John K. yet. The expansion of the social domain is a potential factor influencing the transformation of the register and the revalorization of the register’s forms. While localness has played a role before in that it marked the forms as deviating from a national standard, the social repercussions of this deviation were in the foreground and not the indexicality of place. But against the background of ideologies attributing a positive value to regional identity, the index of localness was foregrounded and positive social evaluations were added. They existed parallel to the negative ones or even disappeared for some people, especially young people like Jessica H., who regards it as positive to have a linguistic form which signals her regional identity. The register label \textit{Pittsburghese}, which first appeared in a newspaper article published in 1967 \citep[95]{Johnstone2006}, shows the transformation of the social range of values with its increasing focus on local identity and its positive evaluation. Pittsburghese is therefore a register which builds on a prior register whose form-value links ([aː] indexing local, non-standard, incorrect, uneducated speech) have become a social regularity and as such open for reanalysis in the given sociohistorical context marked by social and geographical mobility ([aː] indexing local identity). This change in social range and social domain is typical for enregisterment and emphasizes the processual character of the concept and its fluidity. In fact, in \citegen{Johnstone2009} article on the commodification and enregisterment of Pittsburghese, she interprets the results in this way by stating that the local forms are already enregistered by becoming second-order indexes. In her book on Pittsburghese published in \citeyear{Johnstone2013}, Johnstone even uses the term \textit{re-enregisterment} to label the transition from second-order to third-order indexicality. Even though, on principle, the prefix \emph{re}- is not necessary here, as the term enregisterment already encompasses fluidity and change (as does the dialectic relationship between \emph{n}th-order and \emph{n}+1st-order of indexicality described by \citealt{Silverstein2003}), it is useful because it adds emphasis to the transition process of one order of indexicality to the next.


The second case is Beal’s analysis of enregisterment in a northern English dialect, which she presents in several articles (e.g. \citeyear{Beal2009, Beal2012, Beal2017}). It has a strong historical focus and shows how linguistic forms come to index localness and are evaluated positively. One example is the creation of “symbolic working heroes” in dialect literature “with the characters of the weaver in Lancashire and Yorkshire, the pitman and keelman in the northeast, embodying the symbolic virtues of the “gradely” or the “canny lad”” (\citeyear[136]{Beal2012}). While these are certainly instances of metadiscursive activity and therefore evidence of the enregisterment of northern dialects, I suggest considering them as third-order indexes not simply because they link form and values explicitly, but because they had already been enregistered as northern and non-standard before. \citet[137]{Beal2012} herself notes parallels between the historical development in Pittsburgh and in northern England: The increasing mobility led to contact between speakers of different dialects and the increasing exposure to different variants increased the potential for higher-order indexicality and enregisterment of forms as belonging to specific regional dialects as well. She describes the rising number of metadiscursive activities in northern England particularly in the second half of the nineteenth century, but in contrast to \citet{Johnstone2006}, she does not elaborate on prior negative evaluations of these local forms as non-standard, incorrect and uneducated. Her argument is that the enregisterment of northern dialects is a reaction against leveling processes, defined by \citet[98]{Trudgill1986} as the “reduction or attrition of \emph{marked} variants”. While Trudgill considers variants as marked which are of relatively low frequency, Beal also considers variants as marked which are socially stigmatized. As an example she cites Watt’s observation that some phonological forms of Tyneside speech are “stereotyped as parochial, unsophisticated, old-fashioned (etc.)” (\citeyear[55]{Watt2002}, cited in \citealt[127]{Beal2012}). Interestingly, this implies the presence of prior negative evaluations of variants which are indicative of second-order indexicality and therefore of enregisterment before the second-half of the nineteenth-century. In fact, in a more recent article \citet{Beal2017} argues that second-order indexical links between linguistic forms and the North of England can already be identified in the sixteenth century and that a \textit{new} order of indexicality started to emerge in the eighteenth century because “forms which have been indexed at the \textit{n}+1-order become associated with another ideological schema” (\citeyear[28]{Beal2017}). To sum up, this discussion shows that the traditional emphasis on the Labovian distinction between indicators, markers, and stereotypes and the accompanying different levels of awareness is not helpful to study enregisterment processes. For this reason, scholars most prominently engaged in promoting the use of enregisterment in sociolinguistics like Johnstone and Beal have changed their definitions and interpretations in the last decade at least to some extent to follow Silverstein and Agha more closely. In my study, I will do so even more by disregarding the concept of awareness completely and by viewing enregisterment as the discursive construction of a cultural model of action which rests on an emerging second-order indexicality, empirically observable through instances of reflexive activity, with an inherent potential for third-order indexicality. Instead of theorizing enregisterment as proceeding on a cline from first- to second- to third-order indexicality (perhaps even implying some sort of completeness at the last order), I emphasize the dialectic relationship between the orders of indexicality, which is the basis for the processual character of enregisterment.

The third issue is the relationship between perception and production. It is related to the question of the role of consciousness and awareness and also to the relationship between orders of indexicality and enregisterment discussed above. Johnstone et al. seem to imply that it is the difference between using forms in “un-self-conscious speech” and using forms in self-conscious, stylized performances of identity which marks the difference between forms which are enregistered and forms which are not (\citeyear[97--99]{Johnstone2006}). They distinguish “the variable, second-order use of regional variants in everyday interaction” from “third-order performances of a person’s knowledge of the sociolinguistic stereotypes that constitute “Pittsburghese””, and they seem to suggest that only the latter case is tied to enregisterment. Their example case to illustrate this is the youngest speaker, Jessica H., who evaluates the variant positively as a marker of Pittsburgh speech, but her frequency of use is as low as for those speakers who evaluate the variant negatively (see \tabref{tab:2:4}). She has a middle-class background and little contact with people using local forms but an “explicit awareness of “Pittsburghese”” (\citeyear[97]{Johnstone2006}), which makes it possible for her to use the forms of the register Pittsburghese to index her identity when it is called for (she describes a situation where a group of college students from different places compare and perform their accents). While this case undoubtedly provides evidence of enregisterment because Jessica H.’s account of her own linguistic behavior is an account of reflexive activity, it is only one potential way of how speakers can position themselves socially through language use. In fact, considering Agha’s definition of enregisterment, it is not the production of forms which is crucial for defining registers but the perception of forms. A register is a register when its forms are recognized by a sociohistorical population – this is independent of whether speakers of the same population actually produce the forms. Nevertheless, Agha suggests analyzing the extent to which speakers are competent in the use of the register as well and therefore distinguishes two social domains: the domain of recognition and the domain of fluency (see \tabref{tab:2:3}). His analysis of the enregisterment of RP illustrates this necessity as the asymmetry between receptive and productive competence is very high. The point is that Agha’s suggestion to view registers as cultural models of action implies that speakers \emph{can} align with them when they use language, but that they do not have to. \citet{Spitzmuller2013} has developed a model of social positioning which extends a sociolinguistic model of stancetaking \citep{DuBois2007} to include what he calls metapragmatic stancetaking as well.


\begin{figure}
\includegraphics[width=.8\textwidth]{figures/Paulsen-img04.pdf}
\caption{A model of social positioning through language (my own illustration based on the German and English versions in \citealt{Spitzmuller2013, Spitzmuller2015, Spitzmuller10062016})}
\label{fig:2:4}\label{fig:key:4}
\end{figure}


This model integrates the two dimensions which are visible in \figref{fig:2:3}: The triangle on the left captures the dimension of actual language use in a communicative situation in which an actor uses and evaluates linguistic forms in a specific way. The triangle on the right, by contrast, captures the dimension of the register as a cultural model of action which links linguistic forms to abstract types of persons (social personae) and practices. So by using a linguistic form, a speaker not only aligns himself or herself with other actors, but also to types of persons and behavior which are indexed by the linguistic forms that he or she uses. The model emphasizes the relation between the concrete linguistic interaction and the abstract register as a social regularity – a relation which is not fixed but highly dynamic. At the same time, it also allows for a separation of the two dimensions to study their interaction in specific cases. The implications for the role of production and perception are that while language use shapes registers, actors can position themselves with respect to registers through evaluation as well as through production.


\citet{Spitzmuller2013} draws on Bucholtz \& Hall’s (\citeyear{Bucholtz2006, Bucholtz2005}) proposition regarding the different ways in which actors construct identity in positioning their selves in relation to other actors and to abstract register models. The first way is a process of \textit{adequation} or \textit{distinction}. Adequation means that an attempt is made to create the impression of similarity to the other actors by downplaying differences and foregrounding similarities. Distinction is complementary to adequation. In this process, differences are foregrounded and similarities are downplayed to create the effect of differentiation to other actors. The second possibility is a process of \textit{authentication} or \textit{denaturalization}. The key concept here is authenticity, but Bucholtz \& Hall emphasize that authenticity as an inherent essence needs to be distinguished from the social process of authentication through which the realness and genuineness of an identity is verified discursively (\citeyear[601]{Bucholtz2005}). Denaturalization is the opposite process because it involves the deconstruction and subversion of claims to authenticity. Assumptions of naturalized and essentialized links between identity and social and other characteristics (e.g. biological characteristics like skin color) are called into question in this process. Actors can therefore establish a relation to other actors by claiming authenticity of their own language use or by denaturalizing it e.g. through parody. The third way is a process of \textit{authorization} or \textit{illegitimation} where identities are affirmed and possibly even imposed or, by contrast, dismissed or ignored. In this process, social institutions play a great role as they give power to ideologies which underlie both processes. All these processes show the relationality of identity, which requires the social positioning of actors in relation to other actors through linguistic and other means. It is for example possible that they evaluate linguistic forms that other speakers use as authentic for the group of people they recognize as indexed by the forms, but that they do not feel the need to align with this group and therefore do not use the features themselves. The decision as to whether to align with a group or not can also be a matter of situational context as in the case of Jessica H. She might feel the need to align with Pittsburgh speakers when establishing her identity in relation to other college students, but obviously not in the interview situation. Coming back to the issue of consciousness and awareness, it needs to be stressed again that \citet{Bucholtz2005} argue in the same way as \citet{Eckert2016} that the construction of identity and social positioning is not by definition a conscious process: “identity may be in part intentional, in part habitual and less than fully conscious, in part an outcome of interactional negotiation, in part a construct of others’ perceptions and representations, and in part an outcome of larger ideological processes and structures” (\citeyear[585]{Bucholtz2005}). To sum up, this discussion shows that it is fruitful to integrate enregisterment with current sociolinguistic theories and models which belong to the third wave rather than with first-wave ideas and concepts like Labov’s indicators, markers and stereotypes.

Since the classic studies discussed above, several sociolinguistic studies have described enregisterment processes in varying contexts.\footnote{The following overview is restricted to studies of enregisterment processes in English-speaking contexts and communities, but it should be noted that there are studies of enregisterment in other contexts as well:  \citet{Cole2010} and  \citet{Goebel2007, Goebel2008, Goebel2012} study enregisterment processes in Indonesia, \citet{Babel2011} and \citet{Romero2012} in South America, \citet{Dong2010} in China, \citet{Eggert2017} in France, \citet{Elmentaler2017} in Germany, \citet{Frekko2009} and \citet{Peter2020} in Spain, \citet{Madsen2013} in Denmark, \citet{Managan2011} in Guadeloupe, \citet{Newell2009} in Côte d’Ivoire, \citet{Park2016} in Korea, \citet{Slotta2012} in Papua New Guinea, \citet{Wilce2008} in Bangladesh.} In studies on enregisterment in English-speaking contexts, most attention has been given to the role of region and regional dialects in this process. Analogous to \citegen{Johnstone2006} study on the enregisterment of Pittsburghese, the enregisterment of several other regional dialects was investigated, with an almost exclusive focus on England and the United States: in England, Sheffieldish and Geordie \citep{Beal2009}, the Black Country dialect in the West Midlands \citep{Clark2013}, the Yorkshire dialect \citep{Cooper2013} and, more specifically, the Barnsley dialect \citep{Cooper2019} and other “distinct sub-‘Yorkshire’ repertoires” \citep[128]{Cooper2020} as well as the Lancashire dialect \citep{RuanoGarcia2020}; in the United States, Copper Country English or Yooperese in Michigan and Wisconsin \citep{Remlinger2009,Remlinger2009b,Remlinger2017}, the Cleveland accent or the northern accent in Ohio \citep{CampbellKibler2012,CampbellKibler2015} as well as Southern speech \citep{Cramer2013}. While social factors necessarily play a role in all the enregisterment processes described in these studies, the labels given to the registers clearly underline the prime importance of positive values associated with the regional culture and language. An interesting case is the “northern accent” in the United States because it demonstrates that a perceived \textit{lack} of social meaning is actually a social meaning in itself and therefore relevant for enregisterment processes. \citet{CampbellKibler2015} explicitly argue against \citegen{Johnstone2006} view that forms like \emph{yinz} and monophthongal [aː] lacked social meaning before they were noticed by speakers by suggesting that “they likely existed within a larger register, with meanings such as normative, unremarkable, or (in the right contexts) American, non-Southern, and native English” \citep[98]{CampbellKibler2015}. In their view, the prevailing language ideology in the United States leads to a positive evaluation of unmarked speech. Consequently,

\begin{quote}
the alternative to a model in which \emph{yinz} indexes Pittsburgh or raised \textsc{trap}/ \textsc{bath} indexes Cleveland is not one in which they index nothing. Rather, it is one in which they both, together with \emph{you}, lower \textsc{trap}/ \textsc{bath}, high front tokens of \textsc{fleece}, the lexical item \emph{cat}, and so on, index normative, unremarkable American English. \citep[98]{CampbellKibler2015}
\end{quote}

This observation is particularly relevant in that it shows enregisterment processes of regional varieties to be embedded in and linked to other enregisterment processes. In a country where a standard language ideology prevails and interacts with the ideology of nationalism, the enregisterment of that standard, unmarked speech proceeds in relation to the enregisterment of non-standard speech, which is marked as regionally or socially restricted and not adequate to be associated with the speech of the nation as a whole. It is essentially this assumption that the present study builds on: Investigating historical enregisterment processes of American English requires not only an analysis of which forms are associated with American speech, but also, and perhaps even more importantly, which forms are excluded from an American register. I will argue that the construction of what is American proceeds against an existing British English register which leads to forms marked as American and not British. At the same time, however, it also proceeds through an internal differentiation process constructing some variants as regionally or socially marked and, consequently, the alternative variants as unmarked forms which are fit to represent the speech of the nation.

The attention paid to region in studies on enregisterment is actually a bit surprising given that that \citegen{Agha2003} seminal study on the enregisterment of \textit{Received Pronunciation} (RP) in England focuses mainly on the establishment, transmission and transformation of \emph{social} values associated with linguistic forms, which led to RP becoming “a status emblem in British society” \citep[231]{Agha2003} and not an emblem of regional belonging. For English, there are only two studies investigating the emergence of a register that is constructed primarily through the creation of indexical links between linguistic variants and \textit{social} characteristics of the speakers: \citet{SchintuMartinez2016} and \citet{But2017} describe the enregisterment of \textit{cant} in eighteenth-century England, highlighting the association of linguistic forms and low speaker status, involvement in criminal activities and negative character traits like maliciousness. A third study, \citegen{Pratt2017} analysis of parodic perfomances of Californian characters, covers a middle ground. The author’s main interest is in the social nature of enregisterment processes as they describe the creation of social stereotypes (in this case the Californian Valley Girl and the Surfer Dude) and how they are linked to a repertoire of phonetic forms. They argue, however, that this creation process is a part of a broader enregisterment process of Californian English, thus showing that social aspects and region are intertwined. The social complexity of the process is also underlined by \citet{Eberhardt2012}, whose study of the enregisterment of Pittsburghese in the African American community demonstrates the importance of \citegen{Agha2007} dimension of social domain: In contrast to the white community in Pittsburgh, local African Americans recognize not only a limited set of linguistic forms as indexical of Pittsburgh speech, but they also recognize different social values: whiteness and, connected to that, negative associations with oppression and racism \citep[367]{Eberhardt2012}.

It should be noted that most of the studies above also include a historical dimension of some kind, doing justice to the processual character of enregisterment. Most notable with respect to this dimension is \citegen{Cooper2013} study because of the methodological framework he developed for studying enregisterment in historical contexts. His use of quantitative and qualitative analyses of the representation of dialect forms in historical texts and their discussion in metapragmatic discourse serves as an inspiration for the methodology of the present study that has a historical focus as well. Furthermore, he describes actual changes in the register’s repertoire of forms over time, using the term \textit{deregisterment} (a term that was first introduced by \citealt{Williams2012}).

In general, most studies on enregisterment that have been conducted to date investigate these processes in England and the United States. In that regard, the present study is not different, as it also aims at tracing enregisterment processes in the United States in the nineteenth-century. However, the focus is not on a particular region in the United States, but on its emergence as a new variety of English, thereby relating theories modeling this process to enregisterment. This has not been done before, even in the few studies that look at enregisterment processes in the context of World Englishes: Moll’s (\citeyear{Moll2017, Moll2014}) analysis of the enregisterment of what she calls “Cyber-Jamaican”, a very specific “digital ethnolinguistic repertoire” \citep[216]{Moll2014}, \citegen{Henry2010} investigation of the enregisterment of Chinglish in China and \citegen{Hodson2017b} analysis of the enregisterment of American English in British novels published between 1800-1836. For the present study, Hodson’s analysis is nevertheless highly relevant. Although it is limited in scope to a rather short time period and only six novels, it nevertheless provides important insights into the enregisterment of American English as a new variety of English in a British context, so \emph{outside} of the United States, in a social domain different from this study, but nevertheless connected to it in several ways. Especially the key value of inferiority/superiority, embodied prominently through the figure of the “vulgar American”, plays a crucial role in enregisterment processes in an American context as well, as the results of the present study will show. Despite the very limited number of studies on enregisterment of new varieties of English, the importance of one of its essential concepts is recognized: In the recently published \emph{Cambridge Handbook of World Englishes}, \citet{Schleef2020} makes a strong case for using indexicality to study identity constructions and the role of these constructions in the development of New Englishes. An insightful application of indexicality is provided by \citet{Leimgruber2012, Leimgruber2013} who uses it to explain the particular mix of linguistic features employed by Singapore English speakers (instead of one of the varieties which are assumed to exist in Singapore, i.e. Singlish, Hokkien or Standard English).

The growing interest in enregisterment is accompanied by even more sociolinguistic studies investigating aspects that are highly relevant for studying enregisterment processes. Three of those deserve particular attention: commodification, authenticity and communication technology. First of all, commodification is concerned with economic motivations for representations of language, thereby highlighting the possibility that language becomes commodified, i.e. “organized and conceptualized in terms of commodity production, distribution, and consumption” (\citealt[207]{Fairclough1992}, cited in \citealt[161]{Johnstone2009}). This is the case when linguistic forms are printed on T-shirts or mugs (for concrete examples see \citealt{Beal2009,Cooper2013,Johnstone2009,Remlinger2009}) or when dialect representations are used to increase the popularity of novels (see Picone’s \citeyear{Picone2014} analysis of literary dialect in nineteenth-century local color novels depicting the American South). This is in part a consequence of enregisterment, as the producers of these goods will only use forms which they expect to be recognized and whose indexical values they judge as attractive, but commodification is also part of the enregisterment process as it increases the domain of recognition, stabilizes the repertoire and can also contribute to changes in evaluation. When linguistic forms are on display as part of economic goods, people who might not have recognized them before because they have a different socio-economic or regional background can learn about them and are therefore enabled to position themselves socially in relation to the forms and the values linked to them. This shows the importance of analyzing economic interests and their effects on enregisterment processes, which will also be a relevant factor in the present study with its focus on newspapers.

The second aspect, authenticity, is central to enregisterment processes as well. The traditional variationist view on authentic speech is that it is the kind of speech which is the least self-conscious and the least influenced by standard norms in the speech community (see \citealt{Eckert2014} for an overview). In this framework, speech in performance, defined as “verbal art” by \citet{Bauman1990}, cannot be authentic because by consciously adapting speech to foreground the poetic function of language a high amount of attention is given to speech. Third wave sociolinguistics have challenged this view of authenticity, however, by arguing that authenticity is not an inherent quality but a claim that speakers make \citep{Eckert2014}. These claims involve attributes that speakers need to possess and based on which speakers can construct themselves or others as authentic or inauthentic. In this framework, speech in performance can be constructed as authentic even though the performing speakers do not use the same speech forms in other natural contexts. \citegen{Beal2009b} analysis of the British Indie Band Arctic Monkeys is a good example of this. The speech of the lead singer during performances of their songs is claimed to be authentic even though he is known to speak differently in contexts which are not part of the performance (interviews, other parts of radio broadcast). Linking her study to Spitzmüller’s model, it becomes clear how the Arctic Monkeys position themselves not just in relation to one register but to several registers in developing their unique singing style. It is crucial that, in this view, authenticity is not equated with non-reflexive language use, but it is the reflexive nature of language use which is essential in constructing authenticity – by engaging in a particular set of cultural practices and using a particular set of linguistic forms a speaker can make claims about being authentic. This view of authenticity therefore highlights the role of performance as “a highly reflexive mode of communication” which “puts the act of speaking on display – objectifies it, lifts it to a degree from its interactional setting and opens it to scrutiny by an audience” \citep[73]{Bauman1990}. As reflexive activities constitute evidence for enregisterment processes, performances are therefore a highly relevant source of data (see \citealt{Johnstone2011} for a detailed analysis of such a performance). Performances also allow for the creation and transmission of embodied stereotypes or what Agha calls \textit{characterological figures}. Such a figure is defined as “any image of personhood that is performable through semiotic display or enactment (such as an utterance). Once performed, the figure is potentially detachable from its current animator in subsequent moments of construal and re-circulation” \citep[177]{Agha2007}. Characterological figures render social personae visible in performance and therefore have an important consequence for the interaction between register models and actual language use – they “motivate patterns of role alignment in interaction” \citep[177]{Agha2007}.\footnote{It is important to point out here that the appearance of characterological figures is of course not restricted to performances – they can also be found in static images like cartoons, which is obviously important for a historical study for which recordings of performances are not available (see e.g. Clark's \citeyear{Clark2020} detailed study of the role of cartoons in enregisterment processes in the West Midlands).} It is convincing that language users are more likely to align their speech with register models linked to characterological figures than to those models consisting of indexical links between speech and only abstract social values like ‘correct’ and ‘proper’. Several studies have already identified such figures: the Lanky (Lancashire), the Tyke (Yorkshire) and the Geordie (northeast England) (see \citealt{Beal2017}), the Yooper in Michigan \citep{Remlinger2009}, the Yinzer in Pittsburgh \citep{Johnstone2017} and the Valley Girl and the Surfer Dude in California \citep{Pratt2017}.\footnote{The Yooper is an excellent example of such a characterological figure. It is described by Remlinger as “stereotypically male: a backwoods, independent do-it-yourselfer who hunts and fishes, rides a snowmobile, drinks beer, spends time at deer camp, and is suspicious of outsiders” (\citeyear[119]{Remlinger2009}). It is noticeable that this description encompasses mostly activities and a general stance towards life; the appearance of this figure seems to be less important. This is supported by a bumper sticker described by Remlinger which reads “Yooper it’s not just a word, it’s a lifestyle” (\citeyear[119]{Remlinger2009}). Remlinger argues for the importance of such a figure in enregisterment by stating that the Yooper and the way of speaking attributed to this figure “reinforces the notion that a distinct and unified dialect exists in the Copper Country, despite the variability of English throughout the area and despite the widespread use of many of these features throughout the Upper Midwest and places as distant as Alaska” (\citeyear[119]{Remlinger2009}).} However, it needs to be noted that enregisterment can also proceed without recourse to a characterological figure embodying imagined speakers of that variety. \citet{CampbellKibler2015} find that when northern Ohioans were asked to describe the northern accent, they offered only diffuse descriptions and did not mention any characterological figures. This can be explained by their argument that “lack of accent [is] a valuable social meaning in and of itself” (\citeyear[115]{CampbellKibler2015}) – even though enregisterment proceeds through the indexical link between linguistic forms and the values of ‘accent-free’, ‘unmarked’ and ‘normal’ speech, these values do not invite an association with a particular figure because such a restriction to a particular speaker would stand in contrast to the generality and unmarkedness of the variants in question. However, characterological figures are still important because they are linked to those varieties which \emph{are} evaluated as accented. \citet[301]{CampbellKibler2012} cites \citet{Preston1997} here, who found that widely circulating figures of the hillbilly and the gangbanger are associated with a “Southern accent” and “Ebonics”, two varieties which she regards as “highly enregistered in the U.S. context” even though a systematic investigation of the enregisterment of these varieties has not been conducted so far. All in all, it needs to be noted that performances and characterological figures are crucial for the development of registers as cultural models of action and their links to language use by inviting speakers’ role alignment. Every one of these reflexive activities is then part of a larger process of enregisterment and because claims to authenticity are processes (referred to as \textit{authentication} by \citealt{Bucholtz2005}) which are not global but usually selective (not all practices and all forms are used to claim authenticity), they possess the potential for change. Speakers not only select those attributes and qualities which are important to them, but they also bring in new ones which have the potential of being interpreted by others as authentic as well and become enregistered in the process. This leads \citet[44]{Eckert2014} to conclude that “[t]o the extent that a linguistic variable is deployed in an authenticity claim, the process of adequation will contribute to its ever-changing indexical field”. Finding an answer to the question which qualities and linguistic forms marked an American as authentic in the nineteenth century will consequently be an important part of the present study.

The last aspect to be discussed in this section is the role of communication technology in enregisterment. \citet{Johnstone2011b} provides an important study, again in the context of Pittsburghese, which shows that it is important to take into account the medium in which instances of reflexive activity appear. Even though it does not necessarily \emph{determine} who gets access and what kind of content is selected, it nevertheless has an influence on it. At the same time, it has to be considered to what extent the media are controlled by individuals, families, companies or institutions. Last but not least, the study indirectly demonstrates that it is not primarily the medium that shapes people’s perception and evaluation of the content, but that other factors play a more important role. Johnstone emphasizes this point in her discussion of the website \textit{Pittsburghese} (http://www.pittsburghese.com) by pointing out that even though the website was intended to be entertaining rather than informative, it turned out that at least some people were willing to take the information on the site to be technical expertise (\citeyear[10]{Johnstone2011b}). Johnstone cites one case of a person who searched for information about Pittsburgh speech and preferred this website over a website provided by linguists (Johnstone herself and Scott F. Kiesling) because he considered the latter to be hard to understand at times and not entertaining enough (\citeyear[6]{Johnstone2011b}). In the context of my study, this has important implications: Even though I analyze only one medium, printed newspaper articles, it is to be expected that other factors, especially the text type (whether the article is for example a news story, a humorous anecdote or an advertisement), have an influence on what people expect from the content, how they interpret it and how their beliefs and views are shaped by it.

Overall, three important conclusions can be drawn from the discussion presented in this section: First of all, sociolinguistics has benefited from the concepts of indexicality and enregisterment, but at the same time it has also contributed greatly to their development. Indexicality proves to be a useful concept in third wave sociolinguistic studies because it helps sociolinguists to theorize and analyze how social meaning is created through stylistic practice in specific instances of language use, and enregisterment can be used to explain how many instances of stylistic practice lead to enregistered styles and registers. At the same time, the development of the concept of the indexical field to identify and describe indexical meanings has advanced the study of indexicality to a great extent, and several studies have underlined the importance of indexicality in the construction and negotiation of social meaning and identity. All the studies which have used the framework of enregisterment illustrate the usefulness of the concept and provide empirical support for and a theoretical elaboration of important factors in enregisterment processes: commodification, performance, characterological figures, authenticity and communication technology. Taken together, the studies show that various research objectives, research methodologies and kinds of data can be integrated when studying enregisterment.

Secondly, it has become clear that especially the integration of enregisterment into sociolinguistic theory can still be advanced further. The  distinction between a register as a model of action located on the discursive level and an enregistered style located on the level of actual language use is helpful for differentiating the concepts of \textit{style} and \textit{register} (which are still often used synonymously) and for theorizing and analyzing the process of social positioning modeled by \citet{Spitzmuller2013}. According to this model, speakers’ stylistic choices in communicative situations are not only influenced by the other actors present in the situation but by registers as well. At the same time, these stylistic choices and therefore instances of language use in particular situations also shape registers. Furthermore, newer sociolinguistic findings suggest that while agency is important in this process, the heavy reliance on levels of awareness, which goes back to Labov and which is an attribute of first wave studies, is not fruitful in analyzing orders of indexicality and stages in the enregisterment process, not least because it is difficult to operationalize. It is, however, not only necessary to integrate enregisterment further into sociolinguistics, but enregisterment also provides an excellent chance for a further integration of sociolinguistics, perceptual dialectology and the field of research labeled discourse linguistics (e.g. by \citealt{Warnke2008b} and \citealt{Spitzmuller2011}). Particularly the model for a discourse-linguistic multi-layer analysis (DIMLAN)\footnote{In the German original, the model is abbreviated as DIMEAN (\emph{diskurslinguistische Mehr-Ebenen-Analyse}). It was first described in \citet{Warnke2008b} and presented in more detail and with minor modifications in \citet{Spitzmuller2011}. An English version was presented by Spitzmüller in Gothenburg in \citeyear{Spitzmuller2014}.} developed by the just mentioned authors and outlined in \sectref{bkm:Ref506891065} is an excellent way of systematizing different ways of accessing and analyzing enregisterment and of providing guidance and orientation. Consequently, I will develop and justify my research questions and methodology with reference to this model (see Chapter \ref{bkm:Ref513018040}).

The third conclusion concerns the relevance of enregisterment for the study of the emergence of new varieties of English. In \sectref{bkm:Ref521576818}, I have outlined the debate on the role of social factors, particularly of identity, in the emergence of new varieties, and the results of the sociolinguistic studies discussed in this section strongly suggest that these factors play an important role not only in everyday interaction but also in large-scale linguistic change and differentiation processes. In her recent theoretical article, \citet[68]{Eckert2016} argues that “social meaning in variation is an integral part of language and that macrosocial patterns of variation are at once the product of, and a constraint on, a complex system of meaning”, leading her to conclude that while the role of identity in language change still requires further research, it is unlikely that it does not play a role at all. Consequently, she points out that it is crucial to investigate the construction of social meaning(s) at the micro-level and hypothesizes that “[a]ccommodation in colonial situations may have more to do with emerging local social types or stances in the colonial situation than with some abstract colonial identification” \citep[82]{Eckert2016}. I would add that rather than foregrounding one or the other it seems fruitful to take a closer look at how exactly local social types and stances might actually be connected to a more abstract colonial identification.

The following section will shed light on the intersection between enregisterment and a branch of linguistics which has already surfaced in this section because it also has a considerable overlap with sociolinguistics: perceptual dialectology.

\subsection{Perceiving varieties: enregisterment and perceptual dialectology}
\label{bkm:Ref10466352}\hypertarget{Toc63021213}{}
One of the leading scholars in the field of perceptual dialectology, Dennis Preston, makes an important point in a recent handbook article, namely that the term \textit{perception} refers to two different things: On the one hand, it refers to ideas that people have about linguistic facts around them and on the other hand, it refers to perceptual abilities, that is the abilities to detect even subtle differences in speech and to identify and differentiate varieties based on these perceptions. Perceptual dialectology is interested in both kinds of perception \citep[199]{Preston2018}. He notes that while earlier studies focused more on the ideas in people’s mind, later studies were increasingly interested in finding out how the acoustic reality is perceived by speakers and how these perceptions relate to production and to people’s ideas. This development was accompanied by a shift from considering more global properties of speech to analyzing the perception of fine linguistic details, by testing for example whether people could perceive slight differences in the degree of diphthongization of a vowel and match these differences to different places. A general finding is that the relationship between ideas, psychological perception and production is complex.


A second important point that Preston makes relates to the relevance of social factors in perceptual dialectology. As suggested by the term \textit{dialectology}, the field is concerned with the identification of dialects which have traditionally been understood mainly in regional terms. In contrast to traditional dialectology, which bases the identification of dialects on speakers’ linguistic productions, the aim of the early studies was to add speakers’ perceptions to the picture and consequently, to identify perceptual dialect areas, either based on respondents’ ratings of degrees of difference of their own dialect to other dialects or based on maps drawn by the respondents themselves (see \citealt{Preston2018} for details). Despite this early focus on regional dialects, social considerations have also played an important role in perceptual dialectology. In the beginning, perceptual dialectologists mainly studied attitudes towards the perceived dialects and evaluative judgements were elicited through a variety of methods, for example by asking respondents to label and provide comments on the maps they have drawn or by asking them to rate perceived dialect regions or samples of actual speech either based on descriptors provided by the linguist (often involving the two important dimensions of pleasantness and correctness) or based on descriptors provided by the speakers themselves (see \citealt{Preston2018} and \citealt{Cramer2016} for details). More recently, elaborate experimental methods have been developed to test the interaction between attitudes and perception, and studies have revealed that speakers’ ideas about and attitudes towards speech can lead to a mismatch between the actual acoustic reality and the perception of this reality. I have already cited some of these studies in \sectref{bkm:Ref506884048}, but I add two older studies here summarized by \citet[197--199]{Preston2018}. The first one was conducted by \citet{Niedzielski1999}, who found that people from Michigan listening to the recorded speech sample of a fellow Michigan speaker did not recognize that this speaker used a higher realization of the vowel /æ/, which is characteristic of the so-called Northern Cities Shift, but identified a lower and more central variant as the form actually produced by the speaker. In her view, the idea that Michigan speakers use \textit{standard} American English and not regional forms influenced the informants’ perception of actual data. Other studies showed that speech perception can also be very accurate and sensitive to fine details. The second study, by \citet{Plichta2005}, shows that American respondents could distinguish different degrees of diphthongization of /ɑɪ/ and also place the slightly varying vowel realizations on a north-south accent continuum. This demonstrates that they not only recognized the monophthongal [ɑː] as a stereotypical southern variant, but that they could also accurately perceive the variants. Nevertheless, the authors could also show an influence of social factors on speakers’ perceptions because when they separated the results by sex of speaker, it became clear that even though the degree of monopthongization was the same (through resynthesis), women’s vowel realizations were rated as more northern and men’s realizations as more southern. As there was no acoustic basis for these differences in rating, the authors concluded that ideas about women’s speech being more correct than men’s speech and northern speech being more standard than southern speech influenced the informants’ perceptions.

Although this overview of the field of perceptual dialectology is not very detailed, it nevertheless illustrates an important point, namely that studies conducted in this field share many theoretical assumptions and aims with studies using enregisterment as a theoretical framework. First and foremost, they assume that nonlinguists’ ideas about and perceptions of language are worth studying and by focusing on lay people’s point of view, both frameworks can be considered part of \textit{folk linguistics}. Secondly, studies on perceptual dialect areas as well as on enregisterment focus on the perception or recognition of linguistic forms, either as part of a perceived regional dialect or register. And thirdly, they share an interest in the evaluation of (sets of) linguistic forms, which is an indispensable part of enregisterment studies because the recognition of difference is assumed to emerge through a differential evaluation of linguistic forms, but which is also central to most studies in perceptual dialectology because of the insight that attitudinal factors can trigger and influence speakers’ perceptions.

One difference between perceptual dialectology and enregisterment is their slightly different focus, the first paying more attention to regional dialects and the second to social models of action, which is also a result of their evolution from different fields of study (dialectology on the one hand and linguistic anthropology on the other hand). Perceptual dialectologists rather emphasize the relevance of their results for explaining language variation and change, while scholars using enregisterment as a theoretical framework stress that their studies are important for explaining how people are connected in social relationships through language. Nevertheless, as shown in \sectref{bkm:Ref506884048}, many studies on enregisterment also focus on the evaluation of specific linguistic forms as indexical of region. Another major difference concerns the methodological approach. For the most part, perceptual dialectologists analyze data which were obtained through their own intervention, while enregisterment studies are more often than not based on data which are not the result of intervention by the researcher. This means that with regard to Agha’s overview of different kinds of typifications of language use (see \tabref{tab:2:2}), perceptual dialectology can provide the best insights into the typifications in category 2, while enregisterment studies contribute most by analyzing typifications in category 1 and 3. This is accompanied by a much heavier reliance on quantitative methods in perceptual dialectology and qualitative methods in studies on enregisterment. However, as shown in \sectref{bkm:Ref506884048}, some enregisterment studies combine quantitative and qualitative methods and \citet[194]{Preston2018} explicitly discusses discourse as a source of evidence for perceptual dialectology, and he illustrates qualitative discourse-analytic methods by extracting presuppositions underlying a short exchange between two friends.

In general, it follows from this discussion that perceptual dialectology and enregisterment are two frameworks which are highly compatible and have a lot of potential to complement each other. In the conclusion of his article, \citet[200]{Preston2018} notes that

\begin{quote}
Respondents delineate areas as distinct or different on the basis of their likes and dislikes with respect to speakers and the stereotypes that respondents hold of them, giving concrete expression to Silverstein’s notion of higher-order \emph{indexicality}, in which the attributes of people (slow, smart, fun-loving, etc.) are assigned to their language variety and, in fact, become intrinsic parts of that variety’s description.
\end{quote}


Indexicality and enregisterment can therefore be used by perceptual dialectologists to explain the processes that lead to the recognition of perceptual dialect areas and the specific evaluations of speech forms connected to these regions, while perceptual dialectology can add a cognitive perspective to enregisterment studies and, together with sociolinguistics, provide the link between abstract register models and actual linguistic production, as modeled by \citet{Spitzmuller10062016, Spitzmuller2013}.


It is therefore not surprising that some studies on enregisterment in an American context have been conducted by researchers with a background in perceptual dialectology (see the articles by \citealt{CampbellKibler2012,CampbellKibler2015,Remlinger2009,Remlinger2009b} discussed in \sectref{bkm:Ref506884048}), but there is definitely more potential for a fruitful integration of these frameworks.

To conclude, perceptual dialectology contributes in important ways to the definition of a variety that relies not on patterns of production, but of perception, including cognitive as well as ideational aspects, and can be integrated with enregisterment, which offers a primarily social and cultural perspective. Before describing a model which brings together different perspectives on the term variety in \sectref{bkm:Ref523897668}, I address the relationship between enregisterment and discourse linguistics in \sectref{bkm:Ref506891065}. As the time frame that this study focuses on is the nineteenth century, the majority of methods used by perceptual dialectologists cannot be used because there are no informants available from whom data could be elicited. Discourse linguistics, however, provides a theoretical framework for integrating a vast amount of methods to study typifications of language in historical texts and other material and is therefore an indispensable resource to give systematicity and coherence to a study of (particularly historical) enregisterment.

\subsection{Constructing varieties: enregisterment and discourse linguistics}
\label{bkm:Ref506891065}\hypertarget{Toc63021214}{}
The concept \textit{discourse} is used in many different fields of research and it is consequently defined, interpreted and analyzed in a variety of ways. \citet{Angermuller2014} provides a comprehensive overview of research on discourse and divides the approaches into two inseparable fields: discourse theory and discourse analysis (visualized in \figref{fig:2:5}).


\begin{figure}
\includegraphics[width=0.8\textwidth]{figures/Paulsen-img05.pdf}
\caption{
Discourse research as discourse analysis and discourse theory (my own illustration based on \citealt[26]{Angermuller2014})
}
\label{fig:2:5}
\end{figure}

The former comprises approaches which are primarily concerned with the connection between language and knowledge, power and subjectivity and which contribute to the development of theories in the social and political sciences as well as cultural studies. The latter focuses on methodology and comprises works which are concerned with empirical approaches to studying discourse as linguistic practice in social contexts. This separation between discourse theory and analysis is reflected in different (yet crucially connected) conceptions of discourse which are described by \citet{Johnstone2018}. The first one is discourse as “actual instances of communicative action in the medium of language” (\citeyear[2]{Johnstone2018}) and it is these instances (or to be more precise, \emph{records} of these instances) which are studied by discourse analysts. The second conception regards discourses as enumerable entities, as “conventional ways of talking that both create and are created by conventional ways of thinking. These linked ways of talking and thinking constitute ideologies (sets of interrelated ideas) and serve to circulate power in society” (\citeyear[2--3]{Johnstone2018}). An example of an approach which combines the two perspectives is \citet{Blommaert2005}, whose central goal is to explain what he calls “language-in-society” (\citeyear[16]{Blommaert2005}), which manifests itself in the shape of discourse which he defines as

\begin{quote}
all forms of meaningful semiotic human activity seen in connection with social, cultural, and historical patterns and developments of use. Discourse is one of the possible names we can give to it, and I follow Michel Foucault in doing so. What is traditionally understood by language is but one manifestation of it; all kinds of semiotic ‘flagging’ performed by means of objects, attributes, or activities can and should also be included for they usually constitute the ‘action’ part of language-in-action. (\citeyear[3]{Blommaert2005})\footnote{This definition of discourse even goes beyond a purely linguistic conception of discourse by including other kinds of semiotic activity – even though this view is not generally shared, the importance of including other ways of meaning-making in the analysis is also pointed out by others. \citet[2]{Johnstone2018} for example notes that “discourse analysts often need to think about the connections between language and other such modes of semiosis, or meaning-making”.}
\end{quote}


This analysis of language-in-society should be “critical” in that it “needs to focus on power effects, and in particular on how inequality is produced in, through, and around discourse” (\citeyear[233]{Blommaert2005}). In contrast to earlier approaches belonging to the school of Critical Discourse Analysis (CDA), he argues for putting more emphasis on the study of “context”, which should not only include “linguistic and textual explicit forms” but also “modes of production and circulation of discourse” (\citeyear[233]{Blommaert2005}). Indexicality plays a crucial role in his approach to discourse analysis because indexical meanings connect linguistic signs and contexts and ultimately link language to larger social and cultural patterns. \citet{Blommaert2005} draws on Silverstein’s orders of indexicality to emphasize that indexical meanings are not randomly created but “systematically reproduced, stratified meanings often called ‘norms’ or ‘rules’ of language, and always typically associated with a particular shape of language” (\citeyear[73]{Blommaert2005}). Although Blommaert does not draw on the concepts of \textit{register} or \textit{enregisterment}, it is not difficult to see how it can be integrated in his approach: The “particular shape of language” associated with indexical meanings is what Agha calls a register. Furthermore, two of the fundamental theoretical principles of discourse analysis developed by Blommaert can be extended by using Agha’s concepts of register and enregistered style.


\begin{quote}
[Principle 3] Our unit of analysis is not an abstract ‘language’ but \emph{the actual and densely contextualised forms in which language occurs in society}. We need to focus on varieties in language, for such variation is at the core of what makes language and meaning social. Whenever the term ‘language’ is used in this book, it will be used in this sociolinguistic sense. One uneasy by-effect of this sociolinguistic use is that we shall often be at pains to find a name for the particular forms of co-occurrence of language. The comfort offered by words such as ‘English’, ‘Zulu’, or ‘Japanese’ is something we shall have to miss. We shall have to address rather complex, equivocal, messy forms of language.

[Principle 4] Language users have \emph{repertoires}, containing different sets of varieties, and these repertoires are the material with which they can engage in communication; they will determine what people can do with language. […] [W]hat people actually produce as discourse will be conditioned by their sociolinguistic background”. (\citeyear[15]{Blommaert2005})
\end{quote}


The notions \textit{repertoire} and \textit{variety} could be described and analyzed in a better way by using the terms \textit{style}, \textit{enregistered style} and \textit{register} as discussed in \sectref{bkm:Ref512260235} and \sectref{bkm:Ref506884048} because they reflect precisely Blommaert’s first principle, namely that “[i]n analysing language-in-society, the focus should be on what language means to its users” (\citeyear[14]{Blommaert2005}). This means that by identifying which repertoires of forms people recognize based on their connection to a specific set of indexical meanings, discourse analysts do not have to give up using words like \textit{English}, but they have to clarify what this word means to language users, i.e. describe the respective register. Registers as recognizable repertoires do not only constitute constraints on discourse, as Blommaert points out in his fourth principle, but they are also constructed through discourse, and this is in my view the essential link between enregisterment and discourse theory and analysis. Because of this theoretical compatibility, discourse analysis can offer a methodological framework for studies on enregisterment, and the results of these studies can in turn be used by discourse analysts in their study of discourses on issues other than language. In what follows, I outline \citegen[185]{Spitzmuller2011} theory and methodology of \textit{discourse linguistics}, which is an excellent way of systematizing the variety of approaches to studying enregisterment discussed in \sectref{bkm:Ref506884048}.


In contrast to \citet{Johnstone2018}, who explicitly considers discourse analysis a research method and not a discipline or subdiscipline of linguistics, \citet[2]{Spitzmuller2011} claim that it is justified to establish a discipline which they call \emph{Diskurslinguistik} (discourse linguistics) because it can contribute to studying the phenomenon of discourse in the same way as other disciplines dealing with discourse (e.g. philosophy, sociology, history, literary studies) and because it can also contribute to gaining insights into language which cannot be generated by other subdisciplines of linguistics (e.g. lexicology, semantics, text linguistics, sociolinguistics). While emphasizing the close relation between discourse linguistics and the Anglo-American tradition of discourse analysis, they distinguish them by claiming that the focus of discourse linguistics is much more on the function of language to constitute society and knowledge (\textit{gesellschafts- und wissenskonstituierende Funktion}). The center of discourse linguistics is therefore the conception of discourse which rests on Foucault, consequently putting more emphasis on the goal of discovering how language shapes the world (as \citealt{Johnstone2018} puts it), whereas the focus of discourse analysis is more on the conception of discourse as instances of communicative action and an inquiry of how language is shaped by the world. Consequently, \citegen{Spitzmuller2011} theoretical framework is very much concerned with a definition of the term \textit{knowledge}, which they describe as “the result of a continuous negotiation, recognition and rejection of insights in discursive practice” (\citeyear[42]{Spitzmuller2011}, my translation). Knowledge is seen as discursively formed and this is only possible through language with the \textit{statement} as the smallest unit (Foucault: \emph{énoncé}, \citeauthor{Spitzmuller2011}: \emph{Äußerung}). Consequently, knowledge is constituted in discourse through statements and it is for this reason that \citet{Spitzmuller2011} suggest an adaptation of Jakobson’s model of factors of verbal communication:


\begin{figure}
\includegraphics[width=.9\textwidth]{figures/Paulsen-img06.eps}
\caption{Factors in the discursive constitution of knowledge (from \citealt[54]{Spitzmuller2011}, my translation)
}
\label{fig:2:6}
\end{figure}

In the center of the model is Foucault’s \textit{statement}, a terminological decision which is meant to reflect that in this model the message is conceived of as instances of communicative action (and not abstract sentences). The speaker or writer is the actor in this process, but it is important to note that actors do not have to be persons, but can also be institutions, specific groups or individuals representing specific social functions (e.g. politicians). The statement links the actor and the hearer or reader. Another important change concerns Jakobson’s \textit{context}. \citet{Spitzmuller2011} speak of \textit{projections} instead, to emphasize that statements to do not refer to any ontological reality, but that they refer to and at the same time evoke projections of this reality. Instead of \textit{contact}, \citet{Spitzmuller2011} use the term \textit{medium} to emphasize that every message is transmitted by means of a medium and the precise nature of the medium has a bearing on how the statement is perceived by the hearers or readers. \citet{Spitzmuller2011} urge discourse linguists to identify precisely which media are used and to include visual elements and interactions between the visual and the linguistic in their analyses. Lastly, perhaps the most important adaptation of Jakobson’s model concerns the \textit{code}. They replace this term by \textit{knowledge} because in order to understand and interpret statements not only a shared code is needed but also shared knowledge. They summarize the factors in their model by stating that


\begin{quote}
the discursive constitution of knowledge is a result of statements which are produced by actors in a medial form, which are perceivable by other actors and which refer to mental content on the basis of shared knowledge which is relevant for the understanding of the statement. (\citealt[57]{Spitzmuller2011}, my translation)
\end{quote}


Based on these six factors they deduct six functions of the discursive constitution of knowledge and, again with explicit reference to Foucault, six \textit{regulatives} of this process, which illustrates that statements do not exist separately from power structures in society but are embedded in them. The factors, functions and regulatives are summarized in what Spitzmüller \& Warnke call a \textit{field model} in \tabref{tab:2:5}.


\begin{table}
\begin{tabularx}{\textwidth}{XXX}
\lsptoprule
\textbf{Factors} & \textbf{Functions} & \textbf{Regulatives}\\
\midrule
speaker/writer & argumentative & hearability\\
hearer/reader & distributive & control of access\\
statement & rhetorical & norms of expression\\
projection & evocative & linguistic conditioning\\
medium & performative & rules of sayability\\
knowledge & metadiscursive & orders of discourse\\
\lspbottomrule
\end{tabularx}
\caption{
Factors, functions and regulatives of the discursive constitution of knowledge (from \citealt[63]{Spitzmuller2011}, my translation)
}
\label{tab:2:5}
\end{table}

This model shows how language, knowledge and power are intrinsically connected. As statements are linked to speakers or writers, they have an argumentative function: These actors argue for or question knowledge and in doing so create knowledge or change it. The distributive function is equally important because in a model which assumes that language has the function to evoke projections (and not to refer to an ontological reality) statements need to be distributed and shared. Both of these factors are regulated by power structures because in order to participate in the discursive constitution of knowledge, speakers have to be able to make themselves heard and hearers have to gain access to the statements made by speakers. In this context, norms of expression regulating the statement and its rhetorical function also come into play because the form of the statement also crucially influences whether speakers are not only able but also willing to hear the statement. It is here that Spitzmüller \& Warnke explicitly mention enregisterment, but they do not explore the connection between enregisterment and norms of expression in any detail (\citeyear[61]{Spitzmuller2011}). However, they stress that norms of expressions are crucial because the forms of language and the values attributed to the forms in a community have a decisive influence on access to and participation in discourse. Following my argumentation above, it is therefore important to take norms of expression into consideration when studying the discursive constitution of knowledge, but it is helpful to use enregisterment and the concept of register in addition to norms of expression to investigate this regulative.\footnote{Register is a wider concept than that of a \textit{norm}. \citet[126]{Agha2007} considers  a \textit{norm} to be an “externally observable pattern of behavior”, a “statistical norm or frequency distribution in some order of behavior”. If this pattern is reflexively recognized as normal by a population, it becomes a “normalized model of behavior” which constitutes a norm for the group of people who recognizes it. The normalized model can then become a normative model, which is “linked to standards whose breach results in sanctions”; it becomes “a norm codified as a standard”. In this line of argumentation, ‘normal’ or ‘standard’ are therefore social values that can be linked to linguistic and other forms of social behavior in enregisterment processes.} Furthermore, it is important to note again that registers do not only regulate discourse, but that they are themselves discursively constructed. The parallels between Agha’s model of enregisterment and Spitzmüller \& Warnke’s model of the discursive constitution of knowledge are evident: Language itself can become the topic of discourse through reflexive activities and a reflexive model of speech is constructed by actors typifying linguistic forms and making these typifications heard by other actors. The typifications have a specific form, are bound to a specific medium and evoke a projection which is bound to shared knowledge. As pointed out in \sectref{bkm:Ref506883801}, the circulation and transmission of typifications is crucial for the formation of a register. One typification does not form a register and this statement is also true for discourse: One statement does not constitute a discourse, but, as \citet[187]{Spitzmuller2011} point out, “[t]he discourse is […] only discourse where intratextual phenomena, actors and transtextual structures interact” [my translation]. The methodological framework that they develop for analyzing discourses is structured based on these three dimensions and presented in \figref{fig:2:6a}.

\begin{figure}
\includegraphics[width=.95\textwidth]{figures/Paulsen-img06a.pdf}
\caption{
\label{bkm:Ref522891320}Layout of the discourse-linguistic multi-layer analysis (DIMLAN) (my own illustration based on \citealt[201]{Spitzmuller2011} and \citealt{Spitzmuller2014})
}
\label{fig:2:6a}
\end{figure}


As registers are discursively constructed, this methodological framework is also useful for studying enregisterment. At the center of discourse analysis as conducted by discourse linguists are statements in concrete temporal-spatial contexts \citep[123]{Spitzmuller2011}, which means that when studying enregisterment the focus has to be on statements which typify other linguistic signs. Note here that Agha defines reflexive activities more widely, as the use of “communicative signs” to “typify other perceivable signs” (\citeyear[16]{Agha2007}), but this integration of linguistic and other signs is also important in discourse analysis where other signs occurring in combination with statements are also seen as part of the process of constituting knowledge. Statements are the smallest units in discourse analysis and they typically occur in texts, a unit which is defined by Spitzmüller \& Warnke as “a multiplicity of statements with syntactic-semantic relations and one/several thematic center/centers in a formal or situational frame” (\citeyear[137]{Spitzmuller2011}). The intratextual layer therefore comprises an analysis of a single text, where the focus can be on words, propositions or the structure of the text. If the focus is on textual structures, the visual structure of the text should also be taken into account, comprising non-linguistic elements like images (and their relation to the rest of the text), typography and the material to which the text is bound. Discourse analysis as understood by Spitzmüller \& Warnke does not stop at the intratextual layer, but regards the transtextual layer as equally important. On this layer, a multiplicity of texts produced by different actors and occurring in different media is analyzed; the relation between these texts is established through communicative/discursive action which is why a third layer is included in the framework, namely the agent layer. They see this layer as mediating between the intra- and the transtextual layer because at this layer statements are “filtered” (\citeyear[173--174]{Spitzmuller2011}). This means on the one hand that discursive actions determine which statements occur as part of the discourse and which position and importance they have in the discourse (this filter is referred to as “discourse rules”). On the other hand, every text produced by an actor is shaped by one or several discourses (this filter is referred to as “discourse shape”). Their intention behind the DIMLAN model is not to provide a set of instructions that needs to be followed consecutively and comprehensively, but they consider it a framework which helps the discourse linguist to decide and argue for a specific method or set of methods. They are convinced that “fixed procedures cannot do justice to the multimodality and linguistic-systematic heterogeneity of discourses” (\citealt[135]{Spitzmuller2011}, my translation). Instead, they argue for the principle of triangulation and a mixture of multiple methods and approaches to studying discourses. In addition to the DIMLAN model summarizing the different \emph{layers} of discourse-linguistic analysis, they also distinguish several objects of study as well as methods and procedures for studying it. These are summarized in \tabref{tab:2:7} and intended to support discourse linguists planning a study. In this section, I will not elaborate on all these objects, methods and procedures in detail, but I will come back to them in Chapter \ref{bkm:Ref513018040} when I develop the aims and the methodology of the present study.


\begin{table}
\begin{tabularx}{.55\textwidth}{XX}
\lsptoprule
\multicolumn{2}{c}{ objects}\\
\midrule
statement & discourse\\
execution of action & product of action\\
event & series\\
\midrule
\multicolumn{2}{c}{ methods}\\
\midrule
thematic & systematic\\
synchronic & diachronic\\
corpus-based & corpus-driven\\
\midrule
\multicolumn{2}{c}{ procedures}\\
\midrule
heuristic & focused\\
individual & collaborative\\
one step & several steps\\
\lspbottomrule
\end{tabularx}
\caption{
Objects, methods and procedures of discourse-linguistic studies (based on \citealt[124--135]{Spitzmuller2011}, my translation).
}
\label{tab:2:7}\label{tab:key:7}
\end{table}


In conclusion, this section shows that discourse linguistics (as developed by \citealt{Spitzmuller2011} by drawing essentially on Foucault’s work but also many other theoretical and methodological approaches to discourse analysis, e.g. by \citealt{Blommaert2005}) is especially beneficial for studies on enregisterment because an understanding of a register as something that is discursively constructed allows researchers to use the methodological framework developed by \citet{Spitzmuller2011} to describe precisely how they approach the analysis of this construction process. This framework has a solid theoretical foundation and while it keeps the necessary openness to a variety of approaches and methods it provides helpful systematicity at the same time. A discourse-linguistic approach to enregisterment is particularly important for studying historical enregisterment processes because methods which rely on the intervention by the researcher to obtain reflexive activities on language for example by interviewing speakers (as used in perceptual dialectology) are obviously not an option in historical contexts.

A further conclusion which can be drawn from \sectref{bkm:Ref522888605} as a whole is that enregisterment can be located at the intersection between several overlapping areas of linguistic research. It is this overlap (visualized in \figref{fig:2:7}) which is responsible for the wide variety of methods and approaches to enregisterment and also for different nuances of conceptualizing enregisterment.


\begin{figure}
\includegraphics[width=.8\textwidth]{figures/Paulsen-img07.eps}
\caption{
Areas of linguistic research and enregisterment
}
\label{fig:2:7}
\end{figure}

In this study, the conception of enregisterment remains very close to \citet{Agha2003, Agha2007}, but important additions and refinements of concepts and methods provided by sociolinguistic studies (e.g. characterological figures, social indexicality, identity construction and social positioning through language, the role of performances and authenticity) will be taken into account in the development of the methodology, the analysis and the interpretation of the data. As the focus of this study is on the nineteenth century, methods used by perceptual dialectologists play no role in this study, but I will heavily rely on the discourse-linguistic framework developed by \citet{Spitzmuller2011}. In combination with the theory of enregisterment and indexicality, this framework is the basis for developing the concept of the \textit{discursive variety}, in contrast to the \textit{structural variety}, and for modeling the relationship between them.



\section{The construction of discursive varieties through enregisterment: a model}
\label{bkm:Ref523897668}\hypertarget{Toc63021215}{}\label{bkm:Ref523899116}
Structural and discursive varieties are both fuzzy and not discrete entities. For purposes of description and investigation, however, they are idealized and abstracted as separate entities. On a structural level this is done by linguists who base their descriptions on frequency distributions of variants they observe in actual linguistic behavior. On a discursive level, the speakers themselves construct a variety through reflexive activities – their metapragmatic and metadiscursive engagement with language. Frequency plays a role here as well. The more frequently typifications of linguistic forms occur, the more they turn into metapragmatic stereotypes and the more stable and visible a register appears.


\begin{figure}
\includegraphics[width=.8\textwidth]{figures/Paulsen-img08.pdf}
\caption{
A model of the construction of discursive varieties through enregisterment (from \citet{Paulsenforthcoming}, reprinted with kind permission from John Benjamins Publishing Company, Amsterdam/Philadelphia)
}
\label{fig:2:8}
\end{figure}


\figref{fig:2:8} illustrates the process of how a register, or discursive variety, is constructed and it therefore synthesizes the discussion in \sectref{bkm:Ref522870698}.\footnote{I have developed this model in collaboration with my colleagues Benjamin Peter and Johanna Gerwin. It is also presented in \citet{Paulsenforthcoming} and in Peter’s study of the discursive construction of Andalusian (\citeyear[116]{Peter2020}). Peter also created a second model that brings particularly the processes of revalorization of variants and thus of re-enregisterment into focus (\citeyear[157]{Peter2020}).} First of all, the model proceeds from the assumption that there is a level of structured variation underlying all abstractions. This level consists of all the forms which are realized in social and communicative contexts and linguists can abstract structural varieties by using statistical methods. If people engage reflexively with linguistic forms, they elevate the forms to a discursive level by making them subject to metapragmatic and metadiscursive activities. In these activities, forms are indexically linked to social and pragmatic values which are linked to larger ideologies, linguistic and non-linguistic, within a cultural and sociohistorical context. In terms of orders of indexicality, the forms therefore move from \emph{n}th-order-indexicality to \emph{n}+1st-order-indexicality in this process. It is usually the case that metapragmatic and metadiscursive activities focus on a few salient forms and mark one set of salient forms as different from another set of forms, thus creating different registers consisting of all forms recognized as co-occurring with the set of salient forms, the so-called \textit{register shibboleths}. As pointed out above, if forms are frequently subject of reflexive activities, this contributes to their salience, but other factors can increase the salience as well, e.g. if the form is foregrounded against other forms or if a form is connected to a conspicuous characterological figure. The key idea behind this model is that there is a potential relationship between the structural and the discursive level not only because forms found on the structural level can become part of a register but also because the register potentially influences speakers’ choice of forms in interaction when they position themselves socially (see Spitzmüller’s (\citeyear{Spitzmuller2013, Spitzmuller2015}) model of social positioning, described in \sectref{bkm:Ref506884048}). Overall, the model captures \citegen[234]{Agha2003} idea that “phonetic varieties have now become objects – or, object discourses – in relation to a metadiscourse linking speech to social classifications”.\footnote{In this case, \citet{Agha2003} refers to the case of Received Pronunciation in England, which is why he restricts the statement to \emph{phonetic} varieties. In principle, this statement can of course be extended to varieties comprising all structural levels.} By linking sets of co-occurring forms to social values, people discursively construct knowledge about the forms, about social characteristics of speakers using the forms as well as about situations and contexts in which the use of the forms is appropriate. It is important to emphasize again that even though the model suggests that there are boundaries, these boundaries are only constructed, either by the linguist, based on objective criteria, or by the speakers themselves, based on the need to build social relations to other speakers. While discrete boundaries seem to be useful or even necessary, they are in fact fuzzy and subject to constant change. As pointed out above, the stability of both structural and discursive registers depends on several factors, e.g. the amount of interactions between people using different sets of linguistic forms, the frequency of typifications of speech, the salience of characterological figures, and the degree of institutionalization and codification.


The model forms the basis for my case studies which focus on the discursive level by investigating when and how American English was enregistered in the United States in the course of the nineteenth century. Before describing my methodology in detail in \sectref{bkm:Ref513018040}, I will end this chapter by coming back to prior research on the historical development of American English in \sectref{bkm:Ref517077661}. The aim of the section is to analyze how historical linguists have determined the beginning(s) of the variety and which importance they attribute to the structural and the discursive level in describing the major developments of the variety.

\section{Structure and discourse in descriptions of the history of American English}
\label{bkm:Ref517077661}\hypertarget{Toc63021216}{}
Historians describing the development of American English face the problem of determining when American English emerged as a new variety of English, that is, as a variety that is distinct from an old variety. This problem is illustrated in a recent article by Richard W. Bailey, which was published as part of a collection of articles on the history of varieties of English edited by \citet{Bergs2017}. On the one hand, he finds that “[h]istorically considered, American English begins to emerge in the 16th century, even before any English speakers reached the shores of the North American continent” (\citeyear[9]{Bailey2017}). On the other hand, he notes that “America was slow to develop a distinctive linguistic identity” (\citeyear[10]{Bailey2017}) and that it was only “by the beginning of the 19th century [that] American English had become a recognized variety of the language” (\citeyear[12]{Bailey2017}). Finally, he argues that “[w]hat is less recognized [by scholars] is the emergence of the meaning of “American English” as a distinct language with certain distinctive properties” (\citeyear[13]{Bailey2017}). These statements show that Bailey’s concept of American English is rather vague: The reason for postulating such an early beginning, which seems paradoxical at first, is probably based on lexical items which had been used by Native Americans before the arrival of English settlers (he cites for example \emph{canoe} and \emph{maize}) and which were borrowed by the English settlers later and constituted a difference between their speech and the language forms used in England. In contrast to that, the American English with a “distinctive linguistic identity” seems to be based on the \emph{recognition} of linguistic differences, which according to Bailey occurred on lower levels of the social scale: While “[h]igh status Americans spoke just like high-status Britons” (\citeyear[12]{Bailey2017}), slaves, servants as well as Scots and Irish immigrants spoke differently. American English as a distinct \textit{language} is defined based on the “certain distinctive properties” accorded to American English by contemporary commentators. One of the most important commentators was Noah Webster, whose views that he proposed and argued for in his dictionaries influenced the perceived properties of American English to a great extent. \citet[16]{Bailey2017} summarizes these properties as follows:


\begin{quote}
Early in the 19th century, the reputation of American English had been settled – at least for some. The language was free of regional variation, at least in comparison to Great Britain. And it was remarkable for its purity which had been achieved through the preservation of the good old ways of Shakespeare and Addison and through the efforts to regularize it by analogy (so \emph{deaf} was like \emph{leaf}), preservation (continuing to employ \emph{air} and \emph{heir} as homophones), or transparency (in a preference for \emph{meeting house} rather than \emph{auditorium}).
\end{quote}


The example of Bailey’s overview article thus illustrates how vaguely American English is defined – structural distinctiveness is not the primary criterion for postulating the existence of a new variety, but only the basis for a discursive construction of the \textit{variety} or the \textit{language} that relies on recognition and valorization. It is not clear, however, how exactly the recognition of different forms of speech in America (used by slaves, servants, Scots and Irish) leads to \emph{one} American variety and it is equally unclear which set of forms is evaluated as constituting a ‘pure’ and ‘homogeneous’ American language.


The issue of defining American English is also addressed by \citet{Algeo2001b} in his preface to the sixth volume of the \emph{Cambridge History of the English Language}, which is concerned with the history of English in North America. He states that

\begin{quote}
[a]ll languages have internal variation ranging in scope from idiolects (the particular ways different persons use the language) to national varieties (standardized forms of the language used in a particular independent political unit). [...] Between the idiolect and the national variety are dialects, regional and social, on various dimensions. (\citeyear[xviii]{Algeo2001b})
\end{quote}

In this vein, he states with regard to the “beginning” of American English that while the “process of differentiation between the English of Britain and that of America began with the first settlement in America” (\citeyear[xvi]{Algeo2001b}), “with the American Revolution, the variations that had developed in the colonies became a new national variety, contrasting with what from this point can be called the British national variety” (\citeyear[xviii]{Algeo2001b}). The existence of independent political units thus seems to be taken as a basis for postulating the existence of two separate national varieties. On the other hand, he problematizes the terms \textit{dialect}, \textit{language variety} and \textit{language} as being abstractions that are usually used metaphorically and thus not to be taken literally: “To talk about language in such metaphors is useful and not to be avoided. But it is wise to remember that such talk is metaphorical, not literal”. That he finds it difficult to define a variety based on the criterion of structural distinctiveness becomes clear when he says that “[b]ecause of the complexities of linguistic systems, it is impossible to speak with confidence about how much alike or how different two speechways are or to compare two dialects with respect to their overall rate or degree of change” (\citeyear[xix]{Algeo2001b}). Consequently, much like \citet{Bailey2017}, \citet{Algeo2001b} also bases the emergence of a distinct American variety mainly on discursive factors and identifies the period from 1776 to 1898 as crucial in that regard because in this “National Period” “the sense of a distinct variety arose, which was standardized especially in dictionaries and spelling books and spread over the continent during the westward expansion” (\citeyear[xx]{Algeo2001b}).

A focus on recognition and thus on discursive factors can also be found in \citeapo{Davis2003} introduction to his collection of texts relating to American English between 1781 and 1921. He attributes a “pivotal position” to American English because “it is the first form of English to be recognized as a distinct new variety” (\citeyear[xi]{Davis2003}). He describes this recognition process further as “the process by which a distinct form of English was picked out of the mass of variation existing on opposite sides of the North Atlantic and on the Atlantic itself, and the various social and political values assigned to that form once its identity had been instituted” (\citeyear[xi]{Davis2003}). In some ways, this description of the process is reminiscent of the definition of enregisterment because it is essentially about a set of forms being recognized as distinct; however, Davis argues here that social values are assigned to the set of forms \emph{after} they have been recognized, whereas Silverstein and Agha emphasize that the differentiation process is essentially based on a differential evaluation of forms, so that the emerging registers are a \emph{consequence} of valorization processes: Forms are recognized \emph{because} they are assigned social values in the process of enregisterment.

Like \citet{Algeo2001b}, Davis dates the starting point of this process of recognition to the eighteenth century, following not only the American independence from Great Britain but also the standardization of English in England. The texts edited and published in this series largely constitute expert discourses on the language spoken in America, starting with selected papers in the first volume, which span the publication of John Witherspoon’s “The Druid” in 1781 (in which he coined the term \textit{Americanism}) to Charles Whibley’s article “The American language”, published in the Blackwood’s Magazine in 1908. The second volume contains glossaries of Americanisms by \citet{Elwyn1859}, \citet{Fallows1883} and \citet{Norton1890}, the third and the fourth volume exhibit the works of the verbal critic Richard Grant White (\citeyear{White1870,White1880}), followed by \citegen{ScheledeVere1872} re-interpretation of \citegen{Bartlett1848} \emph{Dictionary of Americanisms}. The sixth volume is concerned with works by language planners, including \citet{Molee1888} and \citet{Williams1890}. The seventh and eighth volumes focus on articles and books by academics: Brander Matthews (\citeyear{Brander1892b,Brander1901, Brander1909}), George P. Krapp (\citeyear{Krapp1919}) and Gilbert M. Tucker (\citeyear{Tucker1921}). Based on these texts, Davis identifies five “loosely overlapping periods, often corresponding to Americans’ changing political awareness” (\citeyear[xii]{Davis2003}). The first period is marked by an emphasis on linguistic innovation and opposition to British English, which was connected to the new political independence. During the second period, however, many of these presumably innovative forms were identified as older British forms, reflecting a change of focus from innovation to conservatism. Davis links this to “a cultural backlash against the perceived coarseness of frontier-dominated political life in the 1820’s and 1830’s” as well as to “the concerns of the new philology” (\citeyear[xii]{Davis2003}). For the third period, Davis notes an increase in the awareness of American English as a distinct variety, which is reflected in authors trying to define standard American English and establish their authority over other competing positions (a notable example being Richard Grant White). Davis characterizes this period as being marked by great changes, the most important ones being “industrialization, urbanization, intensive immigration, and the expansion of the middle class and its educational institutions”, and these changes were accompanied by great “cultural insecurity” (\citeyear[xii]{Davis2003}). This not only increased the pressure to define a standard but also the difficulty of the task. The fourth period was different from the previous one because the emphasis shifted to the status of American English as a standard and a model in the world, a development which “corresponded to the awakening and growth of the United States as an imperial power from 1890–1914, and its involvement in Latin America and the Pacific” (\citeyear[xiii]{Davis2003}). Davis dates the last period as starting after the first world war because of a “renewed focus” on the distinctiveness of American English, which was especially noticeable in \citegen{Mencken19371919} work on the \textit{American language} \citep[xii]{Davis2003}. Regarding expert discourses on language, his interpretation thus suggests that the crucial period for the emergence of an American standard variety is the third one, which loosely spans from the middle of the nineteenth century to World War I.

Another attempt to define the beginning of a distinct American English variety can be found in \citegen{Simpson1986} \emph{The politics of American English, 1776-1850}. He claims that

\begin{quote}
American English as we recognize it today had been essentially established by 1850. That is, its major deviations from British English had by that time both been proposed in theory (mostly by Noah Webster) and adopted into relatively common (though not uncontested) practice. \citep[11]{Simpson1986}
\end{quote}


This statement shows that on the one hand, Simpson distinguishes between theoretical ideas about language (a discursive dimension) and practice (a structural dimension). On the other hand, he implies a connection between the two – not only temporally, by saying that American English was established by 1850 on both levels, discourse and practice, but also by using the term \textit{adopted}, which suggests that the theoretical ideas have shaped actual practice. The importance he attributes to discourse is highlighted further in his statement that “[t]hanks to the efforts of two generations of linguistic pioneers, Noah Webster foremost among them, and to the spectacular rise in national self-confidence, America had, by about 1850, a version of English that was recognizably its own” \citep[3]{Simpson1986}. Furthermore, he also argues for the date 1850 primarily on the discursive level: First of all, he notes a change in discourse from a focus on social and political differences and tensions to a focus on unity and homogeneity, which is related to democracy having become “the dominant ideology or self-image, so that, in the continuing development of a self-declared pluralistic culture, a struggle of languages has been the harder to perceive where it does exist” \citep[7]{Simpson1986}. Secondly, and this is his central argument, he finds a difference between two traditions of American literature and their representations of language. The first tradition, before 1850, has James Fenimore Cooper as a particularly representative writer, while the second tradition is marked by Transcendentalist authors:


\begin{quote}
In Cooper’s world, language and society are presented as mechanical; the parts remain parts, without combining into any grand whole. Language is always made up of different languages in conflict, and they do not resolve themselves into any democratically representative common language. These are precisely the implications that the Transcendentalists avoid or cover over. \citep[252]{Simpson1986}
\end{quote}

Simpson’s study is thus essentially a study to be located on the level of discourse, with a focus on literary works published between 1776 and 1850. The importance he attributes to the late eighteenth and first half of the nineteenth century in the creation of a distinctly recognized American English is acknowledged by \citet{Jones1999} in his study \emph{Strange talk: The politics of dialect literature in Gilded Age America}; however, as the title suggests, Jones argues that the late-nineteenth century is even more important for the formation of American English: “I agree with Simpson that the real political clash of dialects that we find in Cooper disappears – by about 1850 – with the idealism of the transcendentalists [...]; it reappears, however, in an even more various and extreme way in post-Civil War literature” \citep[215]{Jones1999}. Regarding the cultural background in which such dialect literature was published, he notes that after the Civil War, two changes could be observed with regard to the development of ideas about language: First, “the idea of an American English came into its own” and second, “this recognition of a new national language was accompanied by a new acceptance of dialect” \citep[16--17]{Jones1999}. This acceptance and (scientific) interest in dialect was motivated by ideas that regarded language as crucial “to understand the mind and culture of its speakers”, but, on the other hand, there was also a fear of dialect diversity connected to a fear of “contamination and fragmentation” \citep[17]{Jones1999}. This new literary movement attracted the interest of a wide range of readers, reflecting a “cult of the vernacular” \citep[39]{Jones1999}. Representations of dialect were not only popular in mainstream literature but they were also part of a large number of dialect sketches included in “highbrow literary magazines”, which had a refined readership, and they were used by minority writers (such as Abraham Cahan and Paul Laurence Dunbar) in their works, which were read not only by white but also by Black Americans \citep[7]{Jones1999}. His main conclusion is that “[d]ialect literature rose to prominence in the Gilded Age because it was integral to developing a cultural debate over the state of American English” \citep[210]{Jones1999}. This shows that, like \citet{Simpson1986}, his study focuses on the discursive level by investigating the ideas surrounding American English, which led to the creation and transformation of linguistic ideologies, and by taking a closer look at the role that literature played in this process. Unlike Simpson, however, he regards the late-nineteenth century as the crucial period for the formation of American English.

David Simpson and Gavin Jones are both literary scholars and even though \citet[11]{Simpson1986} makes claims about the influence of ideas about language on actual linguistic practice, their interest is mainly on the discourses on language and not on language use. The importance of the discursive level for investigating the historical development and especially the emergence of an American standard has also been explored by several linguists in more detail. \citet{Cooley1992} provides an important contribution by asking the question: “[U]nder what circumstances are some variant systems recognized as dialects while others are not? In other words, when and how is variation accorded perceptual status and named a language variety itself, a dialect?” \citep[167]{Cooley1992}. She thus distinguishes between the presence of variation and its perception by speakers and uses this distinction to shed new light on the debate about whether early American English is marked by uniformity or diversity (on a structural level) – a debate that also plays an important role in accounts on the emergence of American English that are based on different theories of the emergence of new varieties (see \sectref{bkm:Ref524246106}). \citet{Cooley1992} argues that there was “a co-existence of both diversity and uniformity in early American English” and that “[t]his co-existence may be reconciled by psycholinguistic or sociolinguistic principles rather than denied by arguments for a single state of uniformity or diversity or for sequences of one followed by the other” (\citeyear[168]{Cooley1992}). An important point to consider here is the type of evidence – she finds that for the early American period most sources are descriptions and comments by grammarians, orthoepists and journalists as well as literary dialect, all of which constitute secondary sources which are not independent of the beliefs and perceptions by those who produced them. A change in these beliefs therefore necessarily had an effect on their descriptions, comments and representations of language and such a change was caused by the Revolutionary War and the War of 1812, which led to a “change of social and political allegiance, through which the colonists began to consider themselves Americans rather than Englishmen” \citep[180]{Cooley1992}. In her view, a consequence of this development was a “psycholinguistic, perceptual, change of the standard variety” during the latter part of the eighteenth century and the early part of the nineteenth century, which is similar to Schneider’s postulated change from an exonormative to an endonormative orientation. Her main argument is that the recognition and acknowledgement of language diversity requires the existence of a standard variety, marked by uniformity, against which the diverse sub-varieties can be delimited \citep[180]{Cooley1992}. She regards the appearance of literary dialect in the late 1780s as another indicator of an increasing recognition of an American standard because only the existence of a standard would guarantee the recognition of representations of forms differing from the standard. Overall, her main argument is thus that historical evidence pointing to the uniformity as well as to the diversity of language in the United States needs to be interpreted by taking into account the beliefs of the commentators and writers, which are in turn shaped by social and political changes. It is important to note, however, that  while \citet[183]{Cooley1992} suspects that the growing recognition of a standard also influenced language use (even if unconsciously), she does not explore these connections in more detail.

The question of the relation between literary dialect and the emergence of an American standard is also addressed by \citet{Minnick2010}, who summarizes her main claim as follows:

\begin{quote}
An American standard for English, then, emerges through the contrast to it that the literary representations of vernacular speech provided, since they are replete with information about what is not standard by their very markedness. This contrast contributes to an image of invisibility for ‘standard American’, a prestige variety defined by no identifying characteristics of its own but rather only by what it is not: regionally or socially or racially marked. The ‘otherness’ of the vernacular-speaking characters of the local color tradition, then, is part of an increasingly vivid background against which an image of a ‘standard’ American English that otherwise has no appreciable identity of its own is rendered visible. \citep[181]{Minnick2010}
\end{quote}


Minnick bases her argument on the one hand on the works by Anne Newport Royall, which were published early in the nineteenth century and in which she represented the speech of provincial characters from rural New England and the south, and on the other hand on writers in the Old Southwestern tradition, who created the figure of the rugged uneducated frontiersman or backwoodsman. Interestingly, \citet[178--179]{Minnick2010} finds a strong ambivalence here because this figure and his speech are not only constructed as the “other” against which the narrator’s speech is presented as standard, but the independence and the individualism of the figure is also celebrated in these texts, so that it becomes “a popular if stereotypical symbol for American national identity and particularly for American masculinity”. This leads her to conclude that a positive evaluation of a standard did not automatically coincide with a depreciation of vernacular speech because these “varieties” could “index culturally popular values like masculinity, independence, bravery, and physical strength” \citep[180-181]{Minnick2010} as well. The situation is different for the third group of speakers whose speech is constructed as the “other” in literary works: African Americans. Representations of their speech functioned as implicit evidence of Black inferiority and there were no positive social values connected to it, at least not in works written by white authors. In general, Minnick’s analysis thus locates the emergence of an American standard in the nineteenth century and the emphasis on its construction through contrasting it with “other” speech varieties in literary works clearly locates her work on a discursive level. Minnick’s study also provides an important contribution to the analysis of the discursive construction of American English from a methodological point of view because she advocates the use of quantitative corpus methods for analyzing literary dialect to avoid interpretations and conclusions which are limited because they are only based on an impressionistic analysis of the data.


A very detailed study on the standardization of American English which analyzes discourses on language in America from a sociolinguistic point of view is \citegen{Bonfiglio2002} \emph{Race and the rise of standard American}. His focus is on pronunciation, and thus on the emergence of a standard American accent rather than a standard variety comprising all linguistic levels, and he aims to show “why and how [the] mid(western) accent rose to be perceived as the standard” \citep[1]{Bonfiglio2002}. The linguistic form that is central to his analysis and discussion is the post-vocalic /r/: The two cultural centers of the country, New York and Boston, were marked by the absence of post-vocalic /r/ and yet its presence became a defining feature of the national standard, for which he uses the label \textit{American network standard} because it came to be used by broadcasting media. To uncover reasons for this rather unusual development (in comparison to England and other European countries, where the speech of the cultural centers became the standard), Bonfiglio analyzes on the one hand a multitude of texts to identify the linguistic ideologies of “influential figures in the United States” and how they relate to ideologies of race and ethnicity. On the other hand, he also investigates how the views expressed in these texts were perceived and received by speakers that were heard by many people: popular actors, announcers and politicians. One important result of his study relates to the time period during which the standard became recognized: He regards the early twentieth century as the crucial period and thus explicitly argues against \citegen[225]{Labov2006} speculation that the shift in the prestige of /r/ occurred in the 1940s in connection with changes resulting from World War II. He rules out an earlier starting point for the emerging pronunciation standard because he regards the radio as crucial for the transmission of knowledge of pronunciation: “In 1920, radio had not yet begun its programmed broadcasting. Thus knowledge of pronunciation remained largely local, and impressions of the speech of other regions was not gained directly but spread largely by word of mouth” \citep[47]{Bonfiglio2002}. This affected also the discourse on /r/:

\begin{quote}
[…] both the description and prescription of pronunciation of /r/ remained largely local and tended to generalize based upon the regional custom until the advent of regular radio broadcasts. Non-rhotic /r/ is observed and recommended on the east coast, while rhotic /r/ is preferred in the central states. \citep[54--55]{Bonfiglio2002}
\end{quote}


This view must be challenged based on the studies noting the enormous interest in representations of dialect in the nineteenth century. Even though it is clear that the impression of pronunciation gained through these representations is not as direct as that which can be gained by listening to actual speech, it nevertheless provides the readers with an idea of what people in other regions sounded like. In the following analysis, I will also show that representations of non-rhoticity (and, implicitly, also rhoticity) circulated in nationwide newspaper discourses especially in the last two decades of the nineteenth century, so that discourses on /r/ do not seem to be as local as Bonfiglio suggests.


In addition to the advent of radio broadcasting, Bonfiglio identifies two other reasons for (mid)western speech patterns, particularly rhoticity, to become the standard accent in the early twentieth century. The first are “xenophobic and anti-Semitic movements” (\citeyear[4]{Bonfiglio2002}) at that time. They built on an increasing consciousness of race and ethnicity that Bonfiglio sees emerging in the postbellum period and ultimately leading to a shift in prestige of eastern cities, especially of New York City, from being positively viewed as cultural centers to being negatively viewed as “contaminated” by poor immigrants. This shift in prestige also affected the prestige of the eastern speech patterns, including non-rhoticity, which became seen as racially different and as impure. This emphasis on race, which is underlined by the title of his study, is explicitly directed against the view that the emergence of an American pronunciation standard, or more specifically the shift in prestige of post-vocalic /r/, is a result of differentiation from British English speech – a view that he labels a “myth” \citep[2]{Bonfiglio2002}.

Connected to this is the second reason, namely the construction of the (mid) western region and their speech patterns as an ideal:

\begin{quote}
The (mid)western accent was constructed and desired by forces external to the area itself that projected a preferred ethnicity upon that region and defined it within a power dynamic of difference, i.e. it was precisely not the speech of the ethnically contaminated areas of the northeast metropolis and the south. \citep[8]{Bonfiglio2002}
\end{quote}

His argument is thus that it was not the (mid)west per se that was attractive, but it was the negative image of the south and the northeast that led to its positive valorization – he speaks of “antigravitational forces” (\citeyear[72]{Bonfiglio2002}) here. Similarly, this valorization was not pushed by the (mid)westerners themselves, but was the result of the cultural power exerted by the northeastern population (\citeyear[72]{Bonfiglio2002}). This is for example visible in “the decision of Harvard and other Ivy League Universities to seek the sustenance of their proper ethnicity and culture in rural western regions”, which indicates “the onset of a shift in the prestige discourse of the educated man” \citep[230]{Bonfiglio2002}. An important figure in this valorization process is the “western hero as an instantiation of the proper American male” whose “speech patterns came to function as metonymies of the condition of nostalgia, sentimentality, and tradition” \citep[231]{Bonfiglio2002}. There are interesting parallels to \citegen{Minnick2010} finding that there are positive evaluations of a similar type of masculinity, characterized by physical strength and toughness needed for a life at the frontier, in the dialect representations in the literature of the Old Southwestern tradition. She takes this as evidence that while the speech of these figures was marked as non-standard, it was nevertheless not condemned but indexed positive characteristics as well. This suggests that this positively evaluated masculinity noted by \citet{Bonfiglio2002} has its roots in the early nineteenth century.

It is obvious that Bonfiglio’s analysis must also be located primarily on the discursive level. His basic argument is “that folk linguistic beliefs determined the national standard”, which is why he aims to “illuminate the larger cultural factors that informed the folkish linguistic beliefs in question” \citep[73]{Bonfiglio2002}. However, this argument shows that he also makes claims about changes in language use – the standard is not only an idea, but a model that influences people’s choice of linguistic forms. This is visible, for example, in the statement that “Americans gravitated toward the pronunciation associated with a “purer” region of the country, and they did so in a largely non-conscious manner” \citep[4]{Bonfiglio2002}. This implies that speakers aligned with a model of speech that they evaluated positively, and that this alignment process was not the result of a conscious effort. Similarly, the following statement illustrates his view that a regional diffusion of a linguistic form resulting in a linguistic change in a region is caused by non-linguistic factors: “The migration of the American continuant postvocalic /r/ from the western states eastward, its supplanting of the dropped postvocalic /r/ of the east coast, and its rise to standardization began in the twenties and was precipitated by the axis of radio, anti-immigration, and westward nostalgia” \citep[60]{Bonfiglio2002}. Bonfiglio’s claims and arguments are thus reminiscent of the claims underlying the theoretical framework of enregisterment outlined in \sectref{bkm:Ref512260235} and the model of social positioning developed by \citet{Spitzmuller2015, Spitzmuller10062016} (see \figref{fig:2:4} in \sectref{bkm:Ref506884048}). It thus supports my position that an analysis of the discursive construction of American English is an indispensable part of studying the emergence of American English. His focus on a linguistic form, /r/, and its “culturally constructed value” \citep[7]{Bonfiglio2002} is also interesting from the point of view of enregisterment. However, the strong focus on one form, even if it has a “high cultural visibility” \citep[6]{Bonfiglio2002}, is also problematic as it does not reveal anything about its relation to other forms – if American English is investigated as a process of enregisterment, it is necessary to study how \emph{sets of} perceivable signs, linguistic forms and other non-linguistic signs, become linked to social and cultural values and thus differentiable registers.

Bonfiglio’s major claims, namely that race played an important role in the standardization of American English and that language ideology, that is beliefs about language, are central to the standardization process, are also shared by \citet{Milroy2012}. In their third edition of \emph{Authority in language}, an often-cited work on processes of language standardization and the role played by prescriptivism in these processes, they conclude that

\begin{modquote}
In the US, bitter divisions created by slavery and the Civil War shaped a language ideology focused on racial discrimination rather than on the class distinctions characteristic of an older monarchical society like Britain which continue to shape language attitudes. Also salient in the US was perceived pressure from large numbers of non-English speakers, from both long-es\-tab\-lished communities (such as Spanish speakers in the South-West) and successive waves of immigrants. This gave rise in the nineteenth and twentieth centuries to policies and attitudes which promoted Anglo-conformity. \citep[160]{Milroy2012}
\end{modquote}


Even though they do not cite \citegen{Bonfiglio2002} study, their review of the literature thus confirms his emphasis on race in contrast to the emphasis on class in Great Britain. However, concerning the periodization of the process of standardization in the United States, they regard a different period as particularly important: “Heavy immigration to northern cities between 1880 and 1920 gave rise to conflicts of dominance between immigrant groups and older élites, and to labour conflicts which had the effect of crowding out democratic ideals of equal rights in both north and south” \citep[160]{Milroy2012}. Especially the importance of radio broadcasting is thus not regarded as an important factor in their analysis, which constitutes a major difference to Bonfiglio’s account. However, it needs to be noted that Milroy \& Milroy do not focus on pronunciation alone, which makes radio broadcasting perhaps less important as a factor. On the other hand, they do not analyze the standardization process by focusing on the linguistic forms themselves – it is not clear which forms become standard and which do not, so that the reasoning about the causes for standardization remains on a rather general and abstract level.


With regard to the importance of language ideologies, \citet[162]{Milroy2012} not only agree with \citet{Bonfiglio2002}, but they also link their findings on standardization processes to the understanding of language ideologies proposed by the linguistic anthropologists Woolard, Silverstein and Irvine and Gal. They particularly point out that these beliefs have central social significance by recognizing their essential function of helping language users “to make sense of the socially structured language variability which they observe every day” (\citeyear[162]{Milroy2012}) and also, beyond the individual level, their role in “delimiting and defining salient social groups and indeed whole nations” (\citeyear[163]{Milroy2012}). It is precisely this delimitation process that is at the heart of the theory of enregisterment and this study seeks to identify not only which linguistic forms were constructed as ‘American’ but also which underlying ideologies made this construction possible.

The overview thus far shows that there is much interest in the history of American English which focuses on the discursive level, but the views with regard to the crucial periods for the recognition of American English as a distinct or standard variety vary and depend, at least to a great extent, on the material studied: publications by “experts” on language or other influential figures on the one hand, and literary works, especially dialect literature, on the other hand. This underlines the need for further studies investigating different materials – a need which is met by this study, as it focuses on newspaper articles.

With regard to the structural level, i.e. the structural differentiation between American English and British English as well as between distinct American varieties, it needs to be noted that systematic and detailed studies on the historical development of linguistic forms in America are actually rather scarce. \citet[152]{Montgomery2001} notes for example that “[t]he field has many simplistic statements and assumptions about what must have occurred in new-dialect formation in the American colonies, rather than documentation of input varieties and the extent to which these were maintained”. His article is mainly concerned with summarizing what is known about British and Irish antecedents, but in doing so he also offers theoretical and methodological insights. He criticizes for example that some hypotheses which are not supported by enough linguistic evidence have “achieved a life of their own” as “part of constructed American memory” \citep[111]{Montgomery2001}. An example is the postulation of linguistic ties between Massachusetts, Virginia, or Appalachia with southeastern England, southern or southwestern England, and Ulster respectively \citep[109]{Montgomery2001}. In his view, linguistic studies are required in which several types of evidence are carefully analyzed and interpreted: This evidence should not only comprise secondary sources like popular observations by outsiders, commentary of grammarians and lexicographers and literary attestations but also primary sources like poetic rhymes and original texts and manuscripts because the latter have a higher value in reconstructing structural changes and differences than the former \citep[97]{Montgomery2001}. An example for such a collection of original texts providing insights into the variation present during the earliest phase of settlement is the Early American Corpus containing texts from New England from the 1620s to the 1720s, among others records of Salem witchcraft trials from 1692 \citep{Kyto2004}. In a pilot study, \citet{Kyto2004} finds that there is hardly any evidence of grammatical and phonological forms that can be classified as \textit{dialect input}, that is as forms found in local vernacular British dialects. The few dialect forms that do occur are found in speech-related records. The other forms reflect instead an “early prestige language” that also has its origin in Britain and that was taken by the settlers to America. The fact that these prestige forms constitute the majority of forms in the corpus is explained by \citet[151]{Kyto2004} by the educatedness of most of the authors of the documents included in the corpus. In general, Kytö does not expect “major differences from the language of the mother country” and her analysis and interpretation rather supports the continuity between the (socio-)linguistic variation present in the areas that the settlers originated from and the variation present in the New England settlements.

The difficulty of determining the beginning of structural differentiation based on primary sources can for example be illustrated by the discussion revolving around the beginning of White Southern American English. \citegen{Bailey1997} article addresses the question “When did Southern American English begin” by analyzing several types of primary data, among them phonetic records of southern speakers born in the nineteenth century from the Linguistic Atlas of the Middle and South Atlantic States (LAMSAS) and the Linguistic Atlas of the Gulf States (LAGS), as well as orthographic evidence from the Tennessee Civil War Veterans Questionnaires (TCWVQ), a  collection of documents written by white male Tennessee speakers, most of whom were educated \citep[260]{Bailey1997}. He focuses on twelve phonological features and seven grammatical features, which he regards as distinctively southern (he calls them “long-established stereotypes of SAE”, \citeyear[258]{Bailey1997}) and finds that only six phonological and four grammatical features were clearly part of Southern American English by the middle of the nineteenth century. The other six phonological and two grammatical features appeared or were increasingly used in the nineteenth century, particularly in the last quarter (in the period after 1875). He thus concludes that Southern American English begins in the last quarter of the nineteenth-century and hypothesizes that drastic social changes in the south during that time created a situation which was particularly conducive to the diffusion of linguistic changes \citep[271]{Bailey1997}. The most important changes were the increasing number of villages and towns which became connected through railroad tracks and the increasing geographical mobility.

Bailey’s hypothesis was tested seventeen years later by \citet{Montgomery2014}, who published their results in an article entitled “When did Southern American English really begin?” Between the two studies, new evidence had been found based on which the Corpus of American Civil War letters was created, which provided new primary evidence to shed light on the beginning of Southern American English. Based on an analysis of the grammatical features investigated by \citet{Bailey1997}, the authors find that “the crucial period for the developing distinctiveness of Southern American English must be pushed back at least one generation”. They evaluate Bailey’s argument that the social changes in the last quarter of the nineteenth century correlate with the beginning of Southern American English as not very convincing because based on research on the history of the American South they assume that “other periods probably witnessed substantial innovation and diffusion, too” \citep[334]{Montgomery2014}. In support of their own argument, they stress instead that historians “have argued that by 1830 the South had become a self-conscious region increasingly at odds with the nation at large”, which is why they ask the question of whether “regional consciousness [could] have played a role in the formation of regional standards of speech” \citep[345]{Montgomery2014}.

These two articles discussing the “beginnings” of White Southern American English thus show that especially the scarcity or even absence of primary evidence makes it difficult to reconstruct linguistic changes, especially on the phonological level. Furthermore, differentiation may have proceeded differently on different levels of structure (phonology, grammar, lexicon), which makes the question of when a variety becomes distinct difficult to answer. \citet{Montgomery2014} test \citegen{Bailey1997} hypothesis only on the grammatical level, so that it is not clear whether the earlier beginning that they postulate would also apply to the phonological level. Quantitative statistical measurements, as proposed by \citet{Pickl2016}, are difficult or impossible to apply given the amount of data that is available for such analyses, but this only strengthens \citegen{Montgomery2001} call for finding and analyzing more primary evidence. Lastly, their discussion of extra-linguistic factors influencing structural differentiation shows how a case can be made for both the late nineteenth century and the first half of the nineteenth century, which underlines the need for investigating this relationship in more detail.

This need has been acknowledged and addressed by \citet{Montgomery2015}. He provides further support for locating the beginning of “distinct (type)s of English” (\citeyear[99]{Montgomery2015}) in the South in the period between 1750 and 1850 by not only investigating more primary evidence (letters and a testimony written by semiliterate commoners), but also secondary evidence: 51 primary-level confederate schoolbooks. So despite his emphasis on investigating primary sources, Montgomery suggests here (albeit tentatively) that “the development of distinctive Southern English may have involved ideology leading to \emph{perception} of the South as a distinct region perhaps as much as the reality of one” (\citeyear[99]{Montgomery2015}). His argument is that those forms that were subject to comment in the textbooks must have been salient in some way, and they must have existed since at least the 1840s. He finds that most comments pertained to forms of pronunciation, and he values this evidence not only as providing “glimpses of many features, some of which (such as the drawl) come from a period earlier than previously documented” but also as evidence of southerners’ awareness “of linguistic contact and competition [...] and [of] the new-dialect formation that was the result” \citep[114]{Montgomery2015}. This study is valuable because it draws attention to the discourses surrounding particular linguistic forms in the south, especially their evaluation as correct or incorrect. It can thus be determined which linguistic forms played a role in the discursive construction of southern American English. However, the material can only function as a starting point for further investigations and his article shows that there are still many open questions, for example relating to the prestige (and distribution) of non-rhoticity (see \sectref{bkm:Ref530736302} for a detailed discussion of research on this form) and the relation between this early awareness of southern forms and the “new, modern Southern identity” identified by \citet[299]{Schneider2007} which led to the recognition and increasing use of innovative features which mark the present-day American south as a distinct dialect region.\footnote{\citet{Schneider2007} distinguishes between “traditional” and “new” southern features – the former are associated with the rural pre-Civil War culture and the latter with the modern, urban culture resulting from the social changes following the defeat in the Civil War. For southern (linguistic) identity, the middle of the nineteenth century therefore marks a turning point.}

The difficulties of investigating the emergence of new varieties on a purely structural level also become visible in the case of African American English. In a recent overview and discussion of research on “the origins and history of African American Language”, \citet{Lanehart2017} not only reviews several research positions but also criticizes that research focuses too much on a set of salient linguistic features and that several perspectives on the development of African American Language “use these salient features for various purposes and sometimes in contradictory ways to support their argument” (\citeyear[86]{Lanehart2017}).\footnote{Lanehart prefers the label \textit{African American Language} over \textit{African American English} because \textit{language} is “less limiting” than \textit{English} (\citeyear[86]{Lanehart2017}).} She finally argues that with regard to structural developments, “we simply do not have the artifacts and hard evidence (recordings of nascent AAL) to make a definitive assessment about the origins and history of AAL” (\citeyear[91]{Lanehart2017}). In her discussion, she also deals implicitly with the discursive construction of African American Language by focusing on the contribution of linguistic research to this construction. She describes the Deficit position in the nineteenth century, according to which Blacks are biologically inferior to whites and thus not able to acquire English in the way that whites do – this position continued to be supported throughout the twentieth century. Lanehart argues that not only this position but also the following ones (the two most prevalent ones being the Anglicist position, which claims that African American English is based on British English varieties, and the Creolist position, which purports that the language spoken by African Americans developed from an earlier creole) are influenced by the “ideological and epistemological perspectives of their originators and supporters”, which shows that research is influenced by social, political and cultural circumstances. Research discourse also has an influence on the recognition and evaluation of African American English, but this influence also has its limits, as Lanehart points out: “[W]hen I tell people outside of linguistics about AAL, they seem dumbfounded that anyone would believe that AAL is not historically rooted to Africa since the people who speak it are, hence the African Diaspora” (\citeyear[92]{Lanehart2017}). Her overview thus strengthens the view developed in \sectref{bkm:Ref521000690} that investigations of structure are very often influenced by discourse, and while it might not be possible to completely disentangle structure and discourse in the investigation of language in general and of the historical emergence of new varieties in particular, it is important to critically reflect the ways in which one’s own investigation of structural developments is shaped by the discursive construction of the variety in question.

Lanehart also addresses the issue of defining African American Language – an issue that is also discussed by \citet{Mufwene2001b} in much detail. Mufwene draws the following conclusion:

\begin{quote}
So far, we have done poor jobs either in not reconciling some of our definitions of AAE [African American English] with our analyses, in overemphasizing extreme differences and disregarding similarities with other English vernaculars, or in proposing definitions that ignore the sentiments of native speakers. We might even be better off not even trying to define AAE and just speaking of peculiarities observable among African Americans. There is probably no way of defining AAE – if a language variety can be defined at all – that does not reflect a particular bias, and this problem is true of any language variety in the world. (\citeyear[37]{Mufwene2001b})
\end{quote}


He suggests instead a vague characterization of African American English as an ethnolect – as “English as it is spoken by or among African Americans” (\citeyear[37]{Mufwene2001b}). This puts emphasis on the speakers as a basis for defining the object of investigation – a perspective that is also adopted by \citet{Lanehart2017}. This reflects their research position, namely the importance of viewing language not only as a set of linguistic forms, but as more than that: “[L]anguage is more than the sum of its parts or the handy grammar that we all like to turn to [...]. If language could be learned from reading a grammar book, we could all be multilingual” \citep[91]{Lanehart2017}. In addition to linguistic forms, aspects of perception and recognition are thus crucial, as pointed out by \citet[37]{Mufwene2001b}:


\begin{quote}
I doubt that African Americans utilize just one rigid battery of structural features to identify a person as speaking English in a manner that corresponds to their own. For the purposes of group identity, I think that being able to recognize speech as African American on the family resemblance model, based on a disjunction of kinds of peculiarities, is more realistic than doing so on the basis of whether its speaker has more or fewer specific non-standard features.
\end{quote}

To conclude this overview of the role of structure and discourse in prior research on the emergence of American English and, related to it, other varieties in North America, it can be stated that even though it was not possible to give a complete overview here because of the large amount of literature on their historical development, the question of when American English became a distinct variety has been answered in different ways. As studies on the distribution and change of linguistic features, especially those based on primary evidence like original texts and manuscripts, are rather scarce, the “beginning” of American English has usually been determined primarily on the discursive level, by identifying the point in time that American English came to be \emph{recognized} as distinct from British English. While the Revolutionary War and the following independence from Britain are usually seen as important events because they mark the starting point of the American nation and are thus a prerequisite for the recognition of a \emph{national} variety, some have stressed that this recognition process was completed by 1850, while others found the late nineteenth century or the early twentieth century to be crucial periods for the definition of an American (standard) variety. Despite those differences, there is general agreement that the recognition process not only proceeds through delimitation from British English, but also through the recognition of American sub-varieties against which a uniform, national American variety is constructed. The linguistic differences related to race and ethnicity have been identified as particularly important and also as constituting a striking difference to standardization processes occurring in England. Another difference between England and America that has motivated investigations is that in America it was not the speech of the political, cultural, and economic centers in the northeast that became the standard, but the speech of the rural (mid)west, while in England the London speech patterns came to be recognized as the standard. Despite a focus on discourse, claims have been made about its influence on language use – on the other hand, studies focusing on language use also make claims about the correlation between the changes they observe and social and cultural developments happening at the same time. \citegen{Bonfiglio2002} study is special in this regard because he links discourses surrounding a specific linguistic form, post-vocalic /r/, to what is known about changes in the use of that form by American speakers. As post-vocalic /r/ is also one of the forms investigated in this study, I will summarize the research findings on this particular form in more detail in \sectref{bkm:Ref530736302}. Finally, not only contemporary observers are influenced by ideologies, resulting for example in different assessments of the uniformity and diversity of American English, as shown by \citet{Cooley1992}, but also linguists contributing to the current research debate, particularly (but definitely not restricted to) the investigation of African American English. In this study, I will use enregisterment as a theoretical framework to contribute to the question of how American English was constructed as a discursive variety – but in contrast to prior studies, my analysis focuses on particular linguistic forms and how the social and cultural meanings they acquired led to the recognition of a set of forms as distinctly American. Rather than following the majority of studies investigating discourses on language shaped by language experts, influential figures or authors of literary works, I will focus on newspaper articles because they had a wide and varied readership and because newspapers contained several different text types, for example editorials, news reports and advertisements as well as anecdotes and humorous paragraphs intended to entertain the readers (see \sectref{bkm:Ref524018691} for more details).

\section{Conclusion}
\hypertarget{Toc63021217}{}
The theoretical framework developed in this chapter is intended to provide a basis for gaining deeper insights into the role of social factors and identity in the emergence on new varieties of English. I have demonstrated in \sectref{bkm:Ref522870687} that this point is one of the most contested issues in discussions about models that have been proposed so far: Trudgill’s model of new-dialect formation, Schneider’s Dynamic Model and Kretzschmar’s model of the emergence of varieties in speech as a complex system. I have shown that the arguments they present interact in crucial ways with their definition of the emerging construct, the \textit{variety}, and that an investigation that aims to shed light on the role of social factors needs to distinguish carefully between different types of varieties: structural varieties, perceptual varieties and discursive varieties. This postulation of different types of varieties does not imply that they are to be understood as existing independently of each other – on the contrary, they are crucially connected. However, it is the main argument of this study that in order to explore these connections, they have to be investigated in their own right and this requires a sound theoretical and methodological framework that does justice to the different types of varieties. As this study is primarily interested in the role of social factors, I proposed to focus on the emergence of discursive varieties, and I demonstrated in \sectref{bkm:Ref522870698} that the theory of enregisterment provides a useful framework for this task because it describes how speakers construct registers through engaging reflexively with linguistic forms (and other perceivable signs) and evaluating these linguistic forms in different ways. Linguistic forms thus become indexically linked to different social values and social personae, and the more frequently reflexive activities occur, the more salient these indexical links become and the more speakers are likely to recognize these links and contribute to their persistence or to their change through their own reflexive activities. It is through this process that registers are constructed – discursive varieties which can be understood as cultural models of action consisting of a set of linguistic and non-linguistic forms which are recognized as distinct from other sets of forms by a group of speakers.

As shown in \sectref{bkm:Ref506884048}, the theory of enregisterment and indexicality has already been fruitfully applied in sociolinguistic research, but especially the distinction between the concepts of register and style as well as the role of awareness in the creation and recognition of indexical links and registers deserves closer attention. I have argued here that it makes sense to distinguish style, enregistered style and register: Whereas the first two concepts are located on the level of language use, the third concept operates on a more abstract level because it is essentially a \emph{model} of speech (and action in general) that cannot be identified by investigating speakers’ use of linguistic variants but only by investigating their reflexive activities, in other words their typifications of linguistic forms in metadiscursive practice. In line with Agha’s theory of enregisterment, sociolinguists and linguistic anthropologists have suggested to understand identity as being produced through social practice – practices in which speakers position themselves and others socially. How registers affect this process of social positioning is explained and modeled by \citet{Spitzmuller2013, Spitzmuller2015, Spitzmuller10062016}.

With regard to the perceptual variety and its relation to enregisterment, I demonstrated in \sectref{bkm:Ref10466352} that studies in perceptual dialectology have also drawn on the concept of enregisterment. I have suggested that theories and methods of perceptual dialectology can mainly be used to add a cognitive perspective to the question of how linguistic forms become enregistered. They cannot be used, however, to study enregisterment in a historical context, which is why they do not play a role in the present study. In contrast to that, the new field of discourse linguistics can contribute to the study of enregisterment and thus to the study of the construction of discursive varieties in crucial ways. This can be done first of all by understanding this construction process as the result of discursive action, that is, of linguistic action that constitutes knowledge. Secondly, the methodological framework DIMLAN that takes into account the intratextual layer, the transtextual layer and the agent layer, which functions as a filter between the intra- and the transtextual layers, provides a reference point for developing a methodological approach for the systematic study of discursive activities that bring about socially shared knowledge about language in the form of cultural models of action and thus of registers.

The model presented in \sectref{bkm:Ref523897668} visualizes the enregisterment process and thus the relation between the structural level and the discursive level that have been distinguished in the Dynamic Model. As it is the central aim of the next chapters to apply this model to the investigation of the enregisterment of American English, I have presented an overview of previous research that has described (aspects of) the development of American English and addressed the question of the “beginning” of this variety. I demonstrated that this question has been answered in very different ways and, more often than not, based on data that belongs to the discursive level. The \emph{recognition} of American English as a new variety was foregrounded, whereas the structural differentiation was much less studied. But even in studies that can be located on the discursive level, not much attention has been given to the specific linguistic forms that the recognition of the new variety was based on. Furthermore, it has not been sufficiently investigated which speakers actually recognized linguistic forms as distinct and when and where these processes of recognition could be observed. This underlines the need for a study that aims to investigate the emergence of a discursive \textit{American} variety systematically and empirically. However, I have also outlined important suggestions by \citet{Cooley1992} and \citet{Minnick2010} concerning the manner in which the recognition of a standard American variety proceeded. That these processes can also be identified in the following study of enregisterment in nineteenth-century America will be shown in the remaining chapters of this book.

