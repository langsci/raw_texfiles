\chapter[Tracing enregisterment processes of American English]{Tracing enregisterment processes of American English: aims and methodology}
\label{bkm:Ref513018040}\hypertarget{Toc63021218}{}\label{bkm:Ref13470656}
This chapter describes the methodology used for the present study of enregisterment processes of American English in the nineteenth century. It begins by justifying the focus of the study on one type of material, newspaper articles, and continues by describing the databases and method of data collection and analysis. Section \ref{bkm:Ref523382115} provides an overview of prior research on the phonological variables which are the focus of this study, followed by an overview of what is known about the historical development of the lexical forms investigated in this study in \sectref{bkm:Ref523382196}. Based on this, Section \ref{bkm:Ref523382450} develops the specific research questions of the present study.


\section{Newspapers as a source for enregisterment processes}
\label{bkm:Ref524018691}\hypertarget{Toc63021219}{}
Prior studies on enregisterment have focused on a variety of material (see the overview in \sectref{bkm:Ref506884048}). Many studies have not just analyzed one type of material, but several; by doing so, they not only do justice to Agha’s observation that different genres have a different circulation and therefore contribute in different ways to the creation of speech chain linkages (\citeyear[259]{Agha2003}) but also to the discourse-linguistic call for taking into account a multiplicity of texts, produced by different actors and occurring in different media \citep[187--188]{Spitzmuller2011}. Using only newspaper articles for studying enregisterment therefore seems like a restriction which limits the amount of insight that can be gained. While this is of course true to some extent, there is, however, also an advantage to such a restriction: The procedure adopted here allows for a comprehensive and extensive analysis of one genre by including larger quantities of texts. So far, the only studies which included newspaper articles in their analysis analyzed a small number (20 articles by \citealt{Johnstone2006} and a similar number of articles by \citealt{Remlinger2009}), whereas the present study is based on close to 1,200 articles. The availability of large electronic databases comprising close to 78 million newspaper articles published in the nineteenth century allows for systematic and focused searches and combinations of quantitative and qualitative analyses of the data. For example, the large number of articles makes it possible to analyze the development of speech chain linkages both temporally and regionally and these insights provide important contextual information for analyzing the evaluation of speech forms based on a detailed qualitative analysis of a smaller number of articles.


The large number of newspaper articles, collected in archives and made electronically accessible in databases (see \sectref{bkm:Ref523405958}), has further advantages for studying enregisterment processes. \citet{Agha2003} describes the genre of penny weeklies, popular periodicals which reached a wide readership because of their low price, and argues that they were largely responsible for the expansion of metadiscourses on accent in England due to their large circulation. They did not form an independent speech chain structure but were closely linked to other texts because writers publishing texts in penny weeklies drew on a variety of sources and genres with a lower circulation (prescriptivist works, popular handbooks as well as novels and literary work). I argue that newspaper articles in the United States of the nineteenth century came to fulfill the same important function: They enlarged the social domains of registers by acquainting a large number of people with their linguistic forms and indexical values. Evidence for this claim is provided by historical research on the development of the number of newspapers and their circulation. By 1800, there were already 234 newspapers, most of them weeklies, which together had a circulation of 145,000 copies \citep[149]{Copeland2002}. \citet[453]{Huntzicker1998} estimates that at the beginning of the nineteenth century the readership of each newspaper ranged from at least a few hundred to one thousand people. In the first quarter of the nineteenth century, the number of newspapers increased from about four hundred in 1810 to more than eight hundred in 1825, which “made the United States by far the greatest newspaper country in the world” \citep[88]{Nord2001} at that time. But it was essentially the 1830s in which major changes in the newspaper business started to occur. Not only did the number of newspapers continue to increase to more than 2,300 in 1900 \citep[453]{Huntzicker1998}, but they were increasingly published daily, printed in larger quantities and, in addition to subscriptions, sold as single copies on the streets for a low price (sometimes for just one cent, leading to the term \textit{penny press}). While some newspapers still reached only a few hundred people in 1900, other newspapers (especially those published in big cities) reached more than 100,000 readers \citep[453]{Huntzicker1998}. This development would not have been possible without technological advances. Particularly important were the steam press, which was first used by \emph{The Times} in London in 1814 and which accelerated the speed of printing, so that the output, the number of copies, could be increased greatly, and the steam railways, which made it possible for papers to be distributed across much wider geographical regions \citep[6]{Barker2002b}. But it was not only the case that eastern papers could increase their distribution to new western territories. Settlers moving westwards usually took newspapers with them because, as \citet[232]{Cloud1998} points out, “having a local newspaper represented stability and legitimacy”. The so-called \textit{frontier press} had of course more functions than this rather symbolic one: Next to several political and economic functions, it also provided information and reading material in regions characterized by a high literacy rate but also by a lack of books and libraries \citep[232]{Cloud1998}. In addition, western newspapers often assumed the role of “town boosters” \citep[232]{Cloud1998}: They promoted their communities to outsiders with the aim of getting their articles published in eastern newspapers and to attract the interest of potential new settlers. In the far west of the country, the number of newspapers therefore increased greatly especially in the second half of the nineteenth century. \citet[233]{Cloud1998} notes that there were 11 newspapers in the 11 western states and territories in 1850, and more than 1,000 by 1890. Not just in the West, but in the whole country it was especially in the last two decades of the nineteenth century that the number and circulation of newspapers rose extraordinarily. \citet[228]{Nord2001} regards this period as “the genesis of the modern mass-circulation newspapers in America” because the number of newspapers increased by 78 percent in the 1880s and “the circulation of all dailies jumped 135 percent, from 3.6 to 8.4 million per day”. In addition, he points out that the beginning of the “modern, mass-circulation national magazine” can be found in that period as well \citep[228]{Nord2001}, which is important because magazine articles were often reprinted in newspapers.

\largerpage
While the large number of newspapers, their wide circulation and their presence in even remote regions of the country are certainly indicative of a large social domain, it could be argued that the domain of newspaper readers is nevertheless restricted to particular social groups. Unfortunately, not much is known about the readership of newspapers. \citet[225]{Nord2001} points out that “[n]early all of the research in the history of newspapers and magazines and much of the research in the history of the popular book has centered on the production, not the consumption, of reading materials”. However, one possibility to find out more about newspaper consumption is used in \citegen{Nord2001} study: He analyzes a statistical survey of the family cost-of-living budget, which includes the amount of money spent on books and newspapers.\footnote{Expenditures on books and newspapers are not distinguished in the survey, but the assumption is that if people spent money on books, they would also spend money on newspapers.} The survey was conducted by the Bureau of Labor Statistics headed by Carroll D. Wright from 1890-1891, and \citet{Nord2001} extracted a random sample of one hundred working-class families in the cotton textile industry from the 1891 report to find out to what extent they spent money on newspapers and how this amount was affected by income, region, nationality and type of community.\footnote{The sample focuses on one industry, the cotton textile industry, because “it was the \emph{only} industry well represented across regions and ethnic groups” \citep[229]{Nord2001}. The working-class families come from six states in three regions where this industry was thriving: New England (Massachusetts and New Hampshire), the mid-Atlantic region (New York and Pennsylvania) and the South (Georgia and South Carolina).} His findings show that working-class families did indeed spend money on newspapers and books, but not all to the same degree. One factor influencing the reading expenditures was the degree to which the family’s income depended on children. The more children contributed to the family’s income, the less the family spent on reading materials. Region was an influential factor as well: Southern families spent much less on newspapers and books than northern families, even if the factor of the children’s contribution to the family income was controlled for. Nationality, that is, the difference between native-born workers and immigrants, played a role in the North, where native-born Americans spent more on reading materials, even though they earned less money on average than immigrant working-class families.\footnote{In the South, all families of the sample were native-born Americans so that an influence of nationality could not be observed here.} With regard to the last factor, community, \citet[239]{Nord2001} distinguishes between “traditional interpersonal communities” (\textit{Gemeinschaft}, in Ferdinand Tönnies’s terms) and “modern contractual society” (\textit{Gesellschaft}) and finds that reading as an activity is rather associated with the latter \citep[240]{Nord2001}. This finding leads him to conclude that “[t]hough reading was a common activity for all groups in my sample of working-class families, the more avid readers seem to have been more at home with the institutions of the modern industrial society” \citep[240]{Nord2001} and the less avid readers, especially southerners and a particular immigrant group in the North, French Canadians, were more committed to the institutions of the \emph{Gemeinschaft}, such as family and church. Even though \citegen{Nord2001} study is restricted to working-class families in the late nineteenth century, it provides a very detailed insight into the factors that influenced whether people read newspapers or not and it can be assumed that they also played some role in the rest of the nineteenth-century.

With regard to the early nineteenth century, historians also assume that the readership of newspapers was already quite large and varied. \citet[141]{Copeland2002} investigates the role of the press in the creation of an American public sphere in the period from 1750 to 1820 and finds that

\begin{quote}
[t]he prominence of newspapers and other forms of print reflected both the breadth of popular involvement in public debate and the widespread use of the press to facilitate and promote this process. But it was not just the backcountry farmer, middle-class merchant or elite, educated planters, lawyers and politicians who had access to, and used the public forum afforded by, the press. High literacy rates in America, which exceeded 90 per cent in some regions by 1800, meant that even the poor were more often literate than not, and ensured that access to the public sphere was restricted neither by gender nor by race.
\end{quote}


However, as already indicated by \citegen{Nord2001} study, it is clear that even though access to newspapers was not restricted per se, there were nevertheless huge differences between social groups and between regions. In addition to the factors analyzed by \citet{Nord2001}, the difference between whites and African Americans needs to be considered as well. In an overview of the historical development of literacy in the United States, \citet[49]{Pawley2010} states that “[w]hile for most white Americans the nineteenth century saw an expansion of reading opportunities, for African Americans the picture was more somber”. This was especially true in the South. Even though some slaves were taught how to read or taught themselves, literate slaves were also feared and seen as rebellious and dangerous. Following South Carolina 1740 and Georgia 1755, several slave states passed anti-literacy laws in the first half of the nineteenth century to make literacy learning among African Americans illegal. It is therefore not surprising that the first African American newspaper, \emph{Freedom’s Journal}, was founded in the North, in New York City, in 1827, and that it was aimed at “a small community of free African Americans who lived mostly in northern cities” \citep[26]{Amana1998}. However, after the Civil War, the situation changed. \citet[26]{Amana1998} states that roughly 4 million African Americans were freed and formed new communities that were in need of newspapers. After the end of the Reconstruction period, many African Americans left the South and migrated to the North and the West. The increasing demand for newspapers by an increasingly literate African American readership led to the rise of African American newspapers from 40 papers in the pre-Civil War period to almost 200 papers at the end of the nineteenth century \citep[26]{Amana1998}. This shows that the social domain of newspaper readers increasingly included African Americans as well.


Databases consisting of a large collection of newspapers are therefore a good source for studying enregisterment processes with a potentially large social domain. But for the analysis it is not only important to ask who the readers were but also who was involved in the production and publishing of articles. \citet{DickenGarcia1998} states that modern journalistic practices developed in the nineteenth century. While newspapers in the eighteenth century were often produced by a printer and an assistant only, the nineteenth-century production process involved editors and reporters. The development of the mass press in the 1830s led to an increasing demand for nonpolitical news and changed the way that news were gathered. Before the 1830s, it was common to take articles from other papers, print word-of-mouth reports and notes of congressional sessions. While this practice continued, it then became the task of reporters to identify news stories that could be of interest to readers, to travel to the places where the stories were taking or had already taken place, to gather information about them and to construct an account for the public \citep[585]{DickenGarcia1998}. Eyewitness reporting and correspondent reporting also became important. Essays which expressed the printer’s (political) opinion and which had dominated newspapers before the nineteenth century became separated from news sections and were labeled \textit{editorials}. So in general, reporters wrote most of the news articles, but ultimately the editors and owners of the papers were the ones who decided which articles to print. While most owners, editors and reporters were white and male, there were also approximately 300 female reporters by the 1880s and the growing number of African American papers also suggests that African Americans increasingly became actors in newspaper discourses as well. In addition to news articles, there were other text types which appeared in newspapers and were produced by other actors. Literary texts, such as poems or longer novels, were often printed in newspapers (novels were printed serially), company owners could publish advertisements, and readers themselves had the chance to write letters to the editor, which could potentially be printed as well. During the Civil War era the \textit{column} developed as “an article of moderate length that appears on a regular schedule under the byline of its author” \citep[438]{Riley1998}. The goal of a column was to be interesting and, according to \citet[439]{Riley1998}, the earliest columnists were women because owners hoped to attract a wider female readership. There were also many male columnists; literary authors and humorists started to contribute articles on a regular basis \citep[439]{Riley1998}. Beginning in the 1850s, techniques to print illustrations were taken over from England and improved, so that artists became another group of actors that contributed to newspapers \citep[267]{Everett1998}. Not only were there specialized illustrated newspapers like \emph{Frank Leslie’s Illustrated Newspaper} and\emph{ Harper’s Weekly}, founded in 1855 and 1857 in New York City, but illustrations appeared more and more frequently in other newspapers as well, often as reprints of those appearing in illustrated newspapers and magazines. Illustrations, columns and humorous short texts, which were often published in rubrics called for example “Multiple News Items”, reflect the development that newspapers increasingly “turned into cheap consumer products to be sold for profit” since the 1830s and that this new penny press’ idea of reporting the news was “to tell interesting stories of occurrences” \citep[104]{Nord2001}. It is therefore important to consider that one important aim of the owners was to sell newspapers and to publish articles that promised to be attractive to a large audience.

As shown above, the variety of actors producing and publishing texts went hand in hand with the variety of text types that appeared in newspapers, which makes them a multi-faceted resource for studying enregisterment. Newspaper articles did not only potentially increase the social domain of a register, but they also contributed to the creation and development of the social range of values through the diversity of text types. Newspaper articles were, to use Agha’s words, “linked to earlier genres by a speech chain structure” (\citeyear[259]{Agha2003}). Many oral and written texts which were produced outside of newspapers, like prescriptivist texts, fictional and non-fictional books, political speeches, plays and other performances, conversations on the street and many others, were discussed in newspaper articles, which made these texts not only known to a large number of people, but also provided them with an interpretation and evaluation of these texts. Consequently, studying newspaper articles can potentially also provide insights into how other genres and texts contributed to enregisterment processes. In \sectref{bkm:Ref527971990}, I will give an overview of the databases used for the study and the methods of data collection and analysis.

\section{Databases, data collection and method of analysis}
\label{bkm:Ref523405958}\hypertarget{Toc63021220}{}\label{bkm:Ref527971990}
As pointed out in Chapter \ref{bkm:Ref523475305}, an important goal in any study of enregisterment is the identification of metapragmatic stereotypes, that is “social regularities of metapragmatic typifications [which] can be observed and documented as data” \citep[154]{Agha2007}. Agha observes that


\begin{quote}
[t]he fact that metapragmatic stereotypes are expressible in publicly perceivable signs is not just a matter of convenience to the analyst interested in identifying and studying registers. It is \emph{a necessary condition on the social existence of registers}. (\citeyear[154]{Agha2007})
\end{quote}


The analyst must therefore develop a way to obtain data points which provide evidence for metapragmatic stereotypes and for reasons pointed out in \sectref{bkm:Ref506891065}, I adopt a discourse-linguistic approach to achieve this. With reference to the possibilities summarized in \tabref{tab:key:7}, I outline my approach to data collection and analysis in this section.


First of all, I define the objects of the study as discourses on language in nineteenth-century America which are observable through statements in newspaper articles published in the United States. These articles are products of action and constitute a series of discursive events which then constitute larger discourses on language. As illustrated in \figref{fig:key:9}, the objects of study are therefore located at the intersection of discourses appearing in newspaper articles and discourses on language in general. The initial access to discourses on language is consequently deductive, but to avoid the circularity that often comes with such an approach (studying discourses on language implies that such discourses exist in the first place), I complement this approach by an inductive one by focusing on linguistic variables which I identified through an analysis of statements and not based on prior knowledge about differences between British and American English (see \sectref{bkm:Ref523382115} and \sectref{bkm:Ref523382196} for details). The focus on linguistic forms is very useful in a study on enregisterment because it is essentially the recurrent evaluation of linguistic forms which is at the heart of the construction of a discursive variety and because the forms provide transtextual links between a set of statements and texts.


\begin{figure}
\includegraphics[width=.7\textwidth]{figures/Paulsen-img09.eps}
\caption{
Objects of study
}
\label{fig:key:9}
\end{figure}


Secondly, the concrete methods of data collection follow from this definition of the objects of study. As a basis for the study, I use two large databases which contain close to 78 million newspaper articles published in nineteenth-century America: \emph{America’s Historical Newspapers} (AHN, {\textasciitilde}60 million articles) and \emph{Nineteenth-Century U.S. Newspapers} (NCNP, {\textasciitilde}18 million articles).\footnote{The database \textit{Nineteenth-Century U.S. Newspapers} is a product of Gale, a Cengage Company, and is only available for purchase by institutions. The database \textit{America's Historical Newspapers} was created by Readex, a division of NewsBank, and can also only be accessed through a subscription (usually via libraries or institutions).} The exact number of newspapers, issues and articles in the two databases is listed in \tabref{tab:key:8}.\footnote{The AHN database includes articles from 1690 to 1922, but the present study only includes articles from 1800-1899. There is a small number of newspapers which are included in both databases. It was not possible to determine the exact number of articles which appeared in both databases, but compared to the overall size of the databases, I estimate that the number is so small that it will not skew the results.}


\begin{table}
\begin{tabularx}{\textwidth}{XrYY}
\lsptoprule
& AHN & NCNP  & Both databases\\
\midrule
Newspapers &  835 &  406 &  1,241\\
Issues &  642,573 &  308195 &  950,768\\
Articles &  60,788,035 &  17,943,236 &  78,731,271\\
\lspbottomrule
\end{tabularx}
\caption{Number of newspapers, issues and articles in each database.
}
\label{tab:key:8}
\end{table}

While these collections constitute a corpus of newspaper articles, they cannot be regarded as representative because providing a representative sample of newspaper articles was not the goal of the compilers of the database. The group of experts responsible for choosing the newspapers (five journalism historians for the NCNP and eleven scholars and experts in various fields, mainly historians and librarians, for the AHN) rather aimed at including as many aspects as possible. The company Gale advertises its NCNP database as providing “an as-it-happened window on events, culture and daily life in 19th-century America” and states that it features


\begin{quote}
publications of all kinds, from the political party newspapers at the beginning of the nineteenth century to the mammoth dailies that shaped the nation at the century's end. Every aspect of society and every region of the nation is found in the archive -- rural and urban, large cities and small towns, coast to coast, etc. Includes major newspapers as well as those published by African Americans, Native Americans, women's rights groups, labor groups, the Confederacy, and other groups and interests. \citep{Gale2018}
\end{quote}

Similarly, the flyers advertising the AHN database emphasize the “unparalleled breadth and depth” of the collection (\citeauthor{Readexnd}, Early American Newspapers, series 1) and the focus on the “record of daily life in hundreds of diverse American communities”, so generally including “local and national perspectives” (\citeauthor{Readexndb}, Early American Newspapers). \figref{fig:key:10} and \figref{fig:key:11} show the temporal and the regional distribution of articles.\footnote{The maps showing regional distributions of articles were created using the R packages maps \citep{Becker2016} and ggplot2 \citep{Wickham2009} as well as an R script developed by \citet{DeSante2012}.} Especially the temporal distribution illustrates that the sample is not representative of the development sketched in \sectref{bkm:Ref524018691}. While the NCNP database shows an expected increase of articles over time, the AHN database includes an unexpectedly large number of articles from the first two decades. With regard to the regional distribution, the NCNP database seems more balanced (if also not representative). The AHN database clearly has a stronger focus on the northeast. Nevertheless, it is important to note that if both databases are combined, newspapers from every region of the United States are included.\footnote{For technical reasons, Alaska and Hawaii are not depicted on the maps. Alaska is the territory with the lowest number of articles in the database (2,539 articles, compared to 13,111,797 articles from Massachusetts, the state with the highest number of articles). 62,574 articles were published in Hawaii, which makes the state number 46 on the frequency list (see Appendix). In addition to the 48 states depicted on the map and Alaska and Hawaii, the District of Columbia is listed separately in the newspaper databases. It is much higher in the frequency list (number 12), with 1,940,043 articles published in the two databases. As Hawaii, Alaska and the District of Columbia are not found on the maps, I will list the number of articles underneath the maps which show the results of my searches.}


\begin{figure}
\includegraphics[width=.8\textwidth]{figures/Paulsen-img10.pdf}
\caption{
Number of newspaper articles per decade in the databases AHN and NCNP and in both databases
}
\label{fig:key:10}
\end{figure}


\begin{figure}
\includegraphics[width=\textwidth]{figures/Paulsen-img11.pdf}
\caption{
Number of newspaper articles per state in AHN and NCNP and both databases
}
\label{fig:key:11}
\end{figure}

The advantage of having such a large corpus is that it not only increases the likelihood of finding articles containing statements relating to language, but that it also minimizes the possibility that only a very specific point of view is taken into account. Furthermore, it is not only possible to say when and where statements occurred, but also when and where they did not occur. So in order to obtain statements I collected articles containing the phrase \emph{American language} to identify linguistic forms which were associated with this label. This thematic approach served as a basis for a systematic collection of data, which involved a strong focus on these linguistic forms. I used one or a combination of search terms which represent the linguistic form in question by using a different spelling (for example \emph{hinglish} ‘English’ representing /h/-dropping and -insertion) and searched all articles in the corpus which contained the search terms (see \sectref{bkm:Ref528410938} for more details on pronunciation respellings and their methodological value).\footnote{The articles have been digitized from microfilm using Optical Character Recognition (OCR) technology. According to the information provided by the compilers of the AHN database, newspapers are particularly difficult to digitize because of the wide variety of fonts, formats and other elements and because of the fact that the paper and/or the ink have often deteriorated over time. Consequently, sometimes articles containing words with a similar shape appear in the list of results and have to be excluded manually. More importantly, it is possible that the computer did not find \textit{all} of the articles containing the search term(s) in the database. It should also be noted that I included results in the dataset in which the search term appeared as part of a compound, e.g. in \textit{Hinglishman}.} As the linguistic forms and the search terms guiding the data collection were not pre-defined but the result of an initial data analysis, this approach is not only corpus-based but also corpus-driven. The case studies which involve the analysis of the collected articles containing the search term(s) form the backbone of the study on which the identification and description of discourses on language and therefore the enregisterment of an \textit{American} variety rests. It is therefore obvious that a diachronic approach is used because the question of when enregisterment processes can be observed and how they developed in the nineteenth century is a key question of the present study. The key components of data collection are illustrated in \figref{fig:key:12}.

Thirdly, when it comes to actual procedure, the emphasis is clearly on a focused analysis of articles containing the search term(s). Focusing the discourse analysis on the datasets (each dataset consisting of all articles containing a specific (combination of) search term(s)) makes it possible to systematically analyze discursive structures, including regional distributions and temporal developments. While this approach puts an emphasis on the transtextual layer, the intratextual layer is not neglected. Qualitative analyses of individual articles follow a heuristic procedure in order to find out how the linguistic forms are evaluated, which strategies are employed in this evaluation process and how these evaluations contribute to the emergence of an American register. This heuristic analysis forms the basis for further focused and quantitative procedures because it is then possible to find out how many articles employ the same evaluation and/or the same strategy. Due to the fact that this project is conducted by an individual and not the result of a collaborative research project, the restriction to case studies involving seven linguistic forms was necessary. Nevertheless, I included some additional analyses to assess the reliability of the results, e.g. by comparing the results in the two databases to each other in order to ensure that they are not an artefact of the database used, and by adding datasets based on different search terms representing the same variable to ensure that it is indeed the phonological form and not the lexical item which is subject to evaluation in the article.


%%please move the includegraphics inside the {figure} environment
%%\includegraphics[width=\textwidth]{figures/Paulsen-img013.emf}


\begin{figure}
\includegraphics[width=.8\textwidth]{figures/Paulsen-img12.pdf}
\caption{
Key components of data collection
}
\label{fig:key:12}
\end{figure}

Before stating my precise research questions in \sectref{bkm:Ref523762184}, I will describe the linguistic variables which are at the center of the case studies and provide an overview of prior research on these variables in \sectref{bkm:Ref523762430} and \sectref{bkm:Ref523762437}. In these sections, I also explain which search terms I used for collecting each dataset and justify that choice.


\section{Phonological forms and their written representations}
\label{bkm:Ref523382115}\hypertarget{Toc63021221}{}\label{bkm:Ref523762430}\subsection{Pronunciation respellings and their methodological value}
\label{bkm:Ref528410938}\hypertarget{Toc63021222}{}
Finding instances of phonological forms which are part of enregisterment processes in the nineteenth century is of course difficult because no primary spoken data is available which could provide a record of the exact phonological forms. Historical linguists often rely on spelling to study phonological variation and change (see for example the contributions in \citealt{Hickey2010}), but in nineteenth-century newspaper articles the spelling is already standardized to a high degree. However, there is an exception: spellings which explicitly draw attention to phonological variants and signal a differential pronunciation. \citet{Picone2016} labels them “pronunciation respellings” and emphasizes the importance of distinguishing them from \textit{eye dialect}, a term which also denotes a non-standard spelling, but without a corresponding difference in pronunciation. For example, if \emph{was} is spelled <wuz>, both spellings, <was> and <wuz>, represent the \emph{same} phonological form [wəz] (the weak form of [wɑːz]), which makes <wuz> a case of eye dialect. But if \emph{that} is spelled <dat>, it signals a phonological form ([dæt]) which is \emph{different} to the one indicated by <that> ([ðæt]), which makes <dat> a pronunciation respelling signaling TH-stopping.


These pronunciation respellings are a valuable source for enregisterment processes because they provide a clue as to which phonological form is evaluated and therefore part of the register’s repertoire. This view is shared by \citet[335]{Honeybone2013}, who investigated pronunciation respellings of Liverpool accent features in humorous dialect literature to assess their sociolinguistic salience and who suggest that especially those phonological features which they have identified as being salient are likely to be enregistered for Liverpool English. In general, pronunciation respellings could be subsumed under the category \emph{literary representations} in \citegen{Agha2007} classification of typifications of language use (see \tabref{tab:key:2} in \sectref{bkm:Ref506883801}) even though they do not only appear in literary texts, but also in other text genres, as I will show below. It is typical of this category that even though the phonological forms are explicitly indicated, the evaluations can be very implicit. If they are used in a literary text, for example in the representation of the speech of a character, the form is linked to the social attributes of the character, his or her social relation to other characters and his or her social behavior. These attributes and social relations and their links to language are rarely explicitly described but have to be inferred by the reader. However, explicit evaluations of linguistic forms are also a possibility, for example in editorials containing discussions of language and using such respellings as examples which are supposed to illustrate the pronunciation of the variant. The advantage of using pronunciation respellings is that a search can be conducted which includes all instances of a particular spelling in a corpus. For example, searching the databases for \emph{dawnce} yields all newspaper articles containing this particular form, which makes it possible to identify exhaustively all social values which become indexically linked to it and also to get an overview of when and where they occurred. Due to the fact that pronunciation respellings are by far not as frequent as the same words spelled the regular way, the datasets were also of a manageable size – large enough for quantitative analyses, but small enough for reading every article for qualitative analyses. An analysis of such a set of articles is then comparable to the dataset obtained by a different search term, which makes it possible to find out whether forms are evaluated similarly or differently and therefore belong to the same or a different register.

However, there is also a possible disadvantage to using pronunciation respellings to study the enregisterment of phonological forms. The problem is that respellings rely on already established norms of spelling so that, as \citet[197]{Agha2007} points out, “the reader can only construe defective spelling as an implicit comment on defects of pronunciation”. Similarly \citet[582]{Jaffe2000} write that

\begin{quote}
Our research suggests that it is almost impossible to avoid stigma in the non-standard orthographic representation of others’ low-status speech varieties. \emph{Light} orthography cues voice, but it does so by using stereotyped forms whose meanings are inescapably linked to their use in texts whose aim is to denigrate the speakers being represented. \emph{Heavy} orthographies require too much investment and decoding to allow voice to come through; few readers are able or prepared to sustain the work of attending closely to spelling as a vehicle for voice.
\end{quote}

This implies that the means of representing the phonological forms already bring about a negative social evaluation, signaling a deviation from a norm, defectiveness and incorrectness. In my view, however, even though a negative interpretation of pronunciation respellings is certainly likely, the possibility of a positive interpretation is not ruled out. For example, if the form is innovative or at least has not long been an object of metadiscursive activity, the writer could use the respelling primarily to illustrate the form. Since spelling is a way of rendering abstract descriptions of phonology more vivid and illustrative, and as it is also the best way of representing phonological changes if the genre is written and does not allow for long descriptions (e.g. dramatic texts consisting mostly of dialogue), there is often no alternative for indicating a particular phonological form, even in cases where the evaluation of the form is positive. Moreover, if a new standard register is in the process of being constructed, a deviation, which is normally seen as defective, has at least the potential to be reinterpreted as positive and as an index of a newly emerging norm. Lastly, even if the evaluation of the form is indeed negative, it still constitutes evidence for positive indexical values linked to the alternative phonological form and contributes to the strengthening of the standard register as well as to the clarification of the repertoire and the social range of that register. In any case, it is important to look closely at the context in which the respelling occurs. As the citation by \citet{Jaffe2000} shows, it makes a difference whether there is only one or few respellings or whether there are so many respellings (possibly also in combination with eye dialect) that the text becomes difficult to read and that the voice that is being represented is actually almost silenced in the process. In the analysis, it is therefore important to determine how many other respellings accompany the one that serves as the basis for the collection of the dataset, which ones are particularly salient (through their sheer frequency or due to their foregrounding by other means), which phonological forms these salient respellings represent and whether they are constructed as belonging to the same or a different register.

In the following sections, I will describe the phonological forms and the respective search terms used for the present study and give reasons for the choice of these forms. I will also provide a brief summary of prior research on the historical development of these forms, both on a structural and a discursive level. While it can be argued that the structural development is not relevant in the context of the present study, as it is concerned with the discursive level only, I still see the two levels as linked to each other. Linguistic forms that become enregistered do not come out of nowhere, and a change in the use of these forms (possibly as a consequence of enregisterment processes) might in turn lead to the creation of new indexical values. This is why I will not only discuss what is known about attitudes and evaluations of these forms but also about their use by speakers of English around the world.

\subsection{\emph{hinglish}: /h/-dropping and -insertion}
\label{bkm:Ref13036568}\hypertarget{Toc63021223}{}
The search term \emph{hinglish} is a differential spelling of the word \emph{English}. It draws attention to a process called /h/-\textit{insertion}: the insertion of unetymological /h/ in syllables withouth an onset \citep[65]{Lass2006}. This process is closely linked to the process of /h/-\textit{dropping}, also labeled /h/-\textit{deletion} or /h/-\textit{loss}, which describes the non-existing (or “silent”) pronunciation of the consonant. In an overview on the historical development of /h/, \citet[104--108]{Minkova2014} dates the emergence of /h/ as a distinctive phoneme to the end of the tenth century. Throughout history, it was marked by great instability. First, it became increasingly deleted, then it became used again. The exact dating and the extent of these processes is subject to debate (see \citealt{Milroy1983,Lass2006,Crisma2007,Schluter2009,Schluter2012}), but it is clear that the variability between [h] and Ø was influenced by etymological factors (Romance vs. Germanic origin of the word), linguistic factors (stressed vs. unstressed syllable), regional factors (Northern vs. Midland or Southern varieties) and social factors \citep[105--108]{Minkova2014}. Research on the Present-Day English situation in England shows that /h/-dropping only occurs in unstressed syllables (mainly function words like \emph{he}, \emph{her}, \emph{his}, \emph{her}, \emph{have}, \emph{has} etc.) and is otherwise restricted to very few lexical items (e.g. \emph{heir}, \emph{honest}, \emph{hour}) in Standard Southern British English. Nevertheless, /h/-dropping in stressed syllables has remained present in England until today and is subject to regional and social variation (see \citealt{Ramisch2010}). The existence of /h/-insertion is usually interpreted as a sign of linguistic insecurity of the writers and therefore as evidence for /h/-dropping because the writers seem to have been unsure about where /h/ occurs and therefore inserted it in words which never have been pronounced with an /h/ historically \citep{Milroy1983}. For this reason, I argue that the spelling \emph{hinglish} represents both insertion and deletion, even though it only marks the insertion explicitly.\footnote{This is also supported by \citegen[42]{Milroy1983} assumption that they have “their origin in a single phonological change: loss of underlying /h/ in speech”. Furthermore, the qualitative analysis of newspaper articles shows that they are regularly treated as one phenomenon in metapragmatic reflexive activities.}


The development of /h/-dropping and -insertion in North America has hardly received any attention. Descriptions on the historical development of American English usually only note the absence of /h/-dropping (if it is mentioned at all). \citet[206]{Krapp19252} for example writes that

\begin{quote}
[T]he [town] records do not indicate that at any time or in any region was the loss of \emph{h} [h] in words with this sound in the initial position, or the addition of \emph{h} at the beginning of words with initial vowels, familiar to all in Cockney speech, current in American use.
\end{quote}

In the light of these findings, it is not surprising that there are no studies on the development of attitudes towards and evaluations of the form in an American context. \citegen[206]{Krapp19252} statement which I cited above indicates, however, that the form was well-known in America and that it was associated with the speech of the Cockney, a stereotypical English figure. The mention of this figure suggests that Americans were familiar with discourses on language in England (see also the examples cited by \citealt[131]{Bailey1996}). These discourses have been studied in most detail by \citet{Mugglestone2003} and her analysis shows that /h/-dropping and -insertion occupies a very prominent position in them. She finds that the first comments on the form revealed a rather tolerant attitude towards it and to variation in general. However, this period was followed by

\begin{quote}
a transition, both in terms of comment on /h/ and its attendant connotations [which] seems to come decisively with the second half of the eighteenth century as the nuances of accent were incorporated into the prescriptive consciousness, and the renewed interest in elocution forced a new awareness of the ideals of speech. Fashion, together with a heightened responsiveness to the role of external markers in assigning social status (real or intended), was also to play its part. \citep[99]{Mugglestone2003}
\end{quote}


The evaluation of /h/-dropping since the mid to late eighteenth century has been very negative. \citet[99]{Mugglestone2003} cites a prominent example of such a comment: Thomas Sheridan’s statement that “There is one defect which more generally prevails in the counties than any other, and indeed is gaining ground among the politer part of the world, I mean the omission of the aspirate in many words by some, and in most by others” (\citealt[34]{Sheridan1762}, cited by \citealt[99]{Mugglestone2003}). This citation illustrates the association between /h/\--dropping and defectiveness and incorrectness and it shows the reasoning behind it. Dropping the /h/ is marked as rustic and regional (a feature of speech in the counties as opposed to the fashionable speech in the metropolis, in London) and seen as an index of a lower social standing (people who drop the /h/ were generally not considered to be a member of the “politer part of the world”). According to \citet[115]{Mugglestone2003}, the negative social meanings came to prevail and the absence of /h/ became an important marker of a lack of education (with the underlying argumentation that it signaled the illiteracy of the speaker) and a lack of refinement. As mentioned above, /h/-dropping became linked to the figure of the Cockney, a lower-class Londoner, and it was regularly used by novelists, such as Charles Dickens, to mark figures as socially inferior \citep[122]{Mugglestone2003}. The negative social meanings of /h/-dropping and -insertion consolidated until the 1850s and 1860s and the form was firmly established by then as a “prime symbol” of the social divide between upper and lower classes \citep[113]{Mugglestone2003}. With regard to /h/-insertion, \citet[108]{Mugglestone2003} observes that it was used particularly often to mark “the social (and linguistic) stereotype of the ‘new rich’, and the ‘self-made-man’”, a figure who seeks to become part of the elite social circle, but even though he has acquired wealth, he has not acquired the socially acceptable manners and language, which is made visible by his failed attempt to produce the /h/ word-initially, resulting in frequent /h/\--insertions. All in all, \citet{Mugglestone2003} shows convincingly that the omission and insertion of /h/ became stigmatized to a considerable extent and that the form became a linguistic shibboleth which features prominently in discourses on language. It is therefore not surprising that it is also one of the two linguistic forms which are part of \citegen{Agha2003} analysis of the enregisterment of Received Pronunciation.


This overview shows that while /h/-dropping and -insertion is characterized by variability on the structural level and relatively stable negative evaluations on the discursive level in England, it seems to be of no relevance in an American context. This raises the obvious question of why the form should be part of a study of enregisterment processes of American English at all. The answer is related to the methodological approach used in the present study: It was the initial exploratory survey of newspaper articles containing the search term \emph{American language}{, which} led to two interesting articles pointing to the importance of /h/-dropping and -insertion. The first one is a letter to the editor which appeared in the \emph{Tucson Daily Citizen} on July 25, 1882, and in which a reader asks: “Will you kindly inform me what constitutes the American language?” To this question, the editor replies with only one sentence: “It is the English language with the ‘H’s’ in their proper places”. This answer suggests that the \emph{absence} of /h/-dropping and -insertion in America is used as a defining feature of the newly emerging variety and that it deserves close attention in the present study. In an earlier article, attention to this form is given more indirectly: It is a letter “from an American gentleman at Paris” which was published in five newspapers in 1830, among them in the \emph{Daily National Intelligencer}, a Washington D.C. newspaper. The American describes the “ignorance that prevails in Europe, with regard to our country” and especially with regard to the language spoken in America. He gives anecdotal evidence to support this: “A Yorkshire man, who was my fellow traveller on the top of a coach, upon learning I was an American, complimented me by saying, “Yees talks ez gued Hinglish az hi duz””. This not only indicates an expectation by Englishmen that Americans do not speak “good English”, but also a view held by Americans that they speak English better than Englishmen. That /h/-dropping and -insertion plays a prominent role here is supported by the spellings <Hinglish> (‘English’) and <hi> (‘I’) in the representation of the speech of the Yorkshire man and also by the following imitation of this speech by the American, who replies ironically: “Yees, sur, hand hize cum to Yorkshire, said I, tu perfect my hedication”. The choice of \textit{hinglish} as a search term is based on this article, in which \textit{hinglish} appears a second time as part of the compound Englishman, spelled <Hinglishman>.

The spelling of the word \textit{education} as <hedication> in the letter points to another difference between the speech of the American and the Englishman. The representation of the sounds /juː/ in the second syllable by the letter <i> could indicate the loss of the palatal glide /j/ before the vowel (in addition to the central realization of the vowel). This so-called \textit{yod-dropping} is the second phonological form investigated in the present study, albeit with a different search term, for reasons which I will provide in the next section.

\subsection{\emph{noospaper/s}: yod-dropping}
\hypertarget{Toc63021224}{}
\textit{Yod-dropping} is the term that is often used for a process that involves the elision of the consonant /j/ in onset clusters of a class of words that historically had either /iu/ or /ɛu/.\footnote{This class of words is different from the class of words deriving historically from Middle English /oː/, e.g. \emph{boot} \citep[87]{Lass2006}.}  According to \citet[88]{Lass2006}, this elision process has started  in the eighteenth century in words where /j/ followed /r/ (e.g. in \emph{rue}, \emph{true}) or /l/ (e.g. in \emph{blue}, \emph{glue}). It occurred less commonly after /s/ and the deletion of /t/, /d/, /n/ “has never caught on in the British standards” \citep[88]{Lass2006}. Concerning the present-day variability in the British Isles, \citet[68--69]{Hughes2012} find that yod-dropping after /r/, /l/ and /s/ is almost complete in Received Pronunciation and in many other accents and that in several other regional accents, /j/ is also dropped after other consonants – this process is most advanced in eastern England where yod-dropping occurs before all consonants. Regarding metadiscourses on yod-dropping in England, \citet[88]{Lass2006} notes that it is stigmatized after /t/, /d/ and /n/ and cites \citet{Walker1791} as an early source: He evaluated \emph{noo} ‘new’ and \emph{doo} ‘dew’ as ‘corrupt’ Londonisms. In the nineteenth century, there were different voices: Sweet regarded the pronunciation of \emph{news} without /j/ as ‘vulgar’ (Sweet to Storm, January 21, \citeyear{Sweet1889}), whereas \citet[601]{Ellis1869} listed both pronunciations, with and without /j/, without favoring one of them (both sources are cited in \citealt[473]{MacMahon1998}). The situation in the twentieth century is interesting because in contrast to the retention of /j/ after /t/, /d/ and /n/, which is regarded as a form which is part of Received Pronunciation and which is therefore evaluated as standard and correct, the retention of /j/ after /s/ is “considered amusingly old-fashioned” and affected \citep[69]{Hughes2012}.

In America, yod-dropping has been described as occurring after all coronal consonants, so that it only remained after velars and labials – this development has been labeled “Later Yod Dropping” by \citet[247]{Wells1982}. While he notes that it is a feature of General American today, he also points out that there is variability: “Some easterners and southerners, however, have either /ju/ or the diphthong /ɪu/, and GenAm usage is not entirely uniform” (\citeyear[247]{Wells1982}). An indicator for such variability being present from at least the early nineteenth century is a comment by Grandgent (\citeyear[224--226]{Grandgent1899}, cited by \citealt[196]{Krapp1919}), which indicates, in Krapp’s words, “a confusion in the use of [u] and [ju] in New England speech which was at its height about 1820 and which affected both polite and dialect speech” – a sign of this “confusion” was the insertion of [j] in the onset clusters of words like \emph{bruised} and \emph{smoothed}. This comment suggests that [j] after [r] was already so uncommon that the presence of [j] is interpreted as being as odd as its insertion in a word where historically a [j] had never been present. That this variability continues to at least the early twentieth century is indicated by \citegen[98]{Krapp1919} observation that yod-dropping correlates with region: “It seems to be less general in the South, than it is in New England, whence it has spread to all sections of the country”. This is in line with \citegen[299]{Schneider2007} classification of yod-\emph{retention} as a traditional southern form, which has started to disappear in the new south. \citet{Phillips1994} notes that \citet[174]{Kurath1961} found that yod-retention after [t], [d] and [n] was a form found mostly in the south and the south midland, but she adds that “it is also preferred by some cultivated Northerners”. This indicates that yod-retention was and is not an exclusively southern form and with regard to the change that can be observed regarding yod-dropping she also reports on a study by \citet{Pitts1986} who found that while southerners increasingly drop the yod, there are some northerners, especially newscasters, who produce it \citep[115]{Phillips1994}. A recent study by \citet{Feagin2015} focuses on the development in the south, more specifically in Anniston, Alabama, and finds that while yod-retention was a form used by all social classes in the south, “it has developed into an emerging feature of working-class speech in the younger generation” \citep[361--362]{Feagin2015}). This suggests that there are three developments with regard to the evaluation of yod-dropping: One is the association of yod-dropping with a national norm, which, according to \citet[272]{Schneider2004}, motivates the general increase of yod-dropping in the south. Secondly, there seems to be a retention of an old link between yod-retention and cultivation which seems to have motivated some northerners (particularly newscasters) to use forms with the glide. In contrast to that, the third development is a southern one: Here, yod-retention marks the speech of young working-class speakers, which suggests a possible association between the presence of yod and working-class values that seem attractive to young speakers. However, Feagin finds that, in contrast to other variables (e.g. non-rhoticity), yod-retention or -dropping is not subject to overt comment and therefore part of changes occurring below the level of awareness, which are “interesting as unwitting reflections of community norms” \citep[363]{Feagin2015}. In her view, the overt prestige of forms like rhoticity leads to its adoption by speakers of all social classes, while the covert prestige of forms like yod-retention leads to their maintenance by working-class speakers:

\begin{quote}
These larger economic pressures mean that the local upper class tends to accommodate culturally and linguistically to national pressures, including to the national prestige variety, while the working class does not. In stark contrast, working-class speakers align themselves with the very Southern white rural culture of country music, stock car racing, and hunting, and with speech which could be called the local covert prestige variety. This may explain why the two varieties are diverging as much as they are. \citep[364]{Feagin2015}
\end{quote}


However, \citeapo[72]{Phillips1981} report on her own experience as a student at a southern university in the 1960s shows that yod-dropping is not unnoticed by southern speakers:


\begin{quote}
Before I entered Duke University from my south Georgia high school in 1967, my older brother instructed me that the correct way to pronounce the university’s name was [djuk]. When I arrived on campus, however, I found that all my new acquaintances called it [duk], and that soon became my pronunciation as well. The social significance of the change was not driven home to me, however, until I saw hostile slogans from the nearby University of North Carolina at Chapel Hill reading “Beat Dook”. The antagonism between the Southern state university and the “Northern school set in the South” was crystalized in the pronunciation of that one word.
\end{quote}


Several comments showing that metadiscourses on yod-dropping in America have existed since at least the eighteenth century can be found in \citet{Krapp1919, Krapp19251, Krapp19252}. With regard to prescriptivist comments, he observes that


\begin{quote}
[t]he dictionaries generally authorize only the first of these pronunciations [with [j]] after [d], [t], [θ], [n], [s], and academic authority is very likely to condemn the pronunciation of [uː] as uncultivated, in spite of the fact that it occurs widely in the speech of educated and informed people. It has long been current in America, as is evident from Noah Webster’s defense of [uː] in \textbf{duty}, etc., as the best pronunciation. \citep[95--96]{Krapp1919}
\end{quote}


His statement is thus indicative of an early controversy regarding the evaluation of the presence or absence of yod: While Webster, a very influential figure, advocated yod-less forms, the majority of prescriptivists were opposed to them. With regard to the representation of dialects in literature, \citet[239]{Krapp19251} cites C. Alphonso Smith, a “Southern scholar and historian”, who regards the “vanishing \emph{y}{}-sound” as being “almost a shibboleth of the Southerner to the manner born and helps to differentiate him from the Westerner and the Northerner”. Opposed to this view, \citet[240]{Krapp19251} comments that yod-dropping “is so general as to have little value as a mark of local dialect, especially literary dialect, for dialect writers have rarely made any effort to distinguish by spelling or otherwise between the pronunciation of \textit{tune} with the vowel in \emph{boot} and with the vowel of \emph{mute}”.


Given this controversy, yod-dropping is thus an interesting form for an investigation of discourses on language and the discursive construction of American English – that it played at least some role is suggested by the use of the form \emph{doocid} ‘deuced’ in some articles that contained other search terms. I have chosen a different lexical item, however, because the lexical item \emph{deuced} is linked to a particular characterological figure, the American dude, and I wanted to rule out lexical influence here. I chose \emph{noospaper}, in its singular or its plural form (hence the precise search term \emph{noospaper} OR \emph{noospapers}, represented here as \emph{noospaper/s}) for three reasons: First, the lexical item \emph{newspaper} was not likely to be connected to a specific group of people or to a specific context. Secondly, I expected it to be fairly frequent (especially in newspaper articles) but not too frequent to make a qualitative analysis impossible. The third reason was that it is a fairly long word and therefore more likely to be identified correctly by the OCR software.\footnote{An explorative search for the term \emph{noos} ‘news’ yielded more than 5,000 articles in the NCNP alone, many of which contained \emph{news} or other similar lexical forms. A manual check of all newspaper articles to identify those that actually contained \emph{noos} would have been too time-consuming.}

\subsection{\emph{dawnce}: realization of the \textsc{bath}{}-vowel}
\label{subsection334}
\hypertarget{Toc63021225}{}
What I refer to here as the \textsc{bath}{}-vowel is the vowel which appears in words of the standard lexical set \textsc{bath}, which is defined by \citet[133]{Wells1982} as “comprising those words whose citation form contains the stressed vowel /æ/ in GenAm, but /ɑː/ in RP”. This difference between these two current standard accents is the result of a historical development which \citet[134]{Wells1982} terms the \textsc{trap–bath} split. Lass (\citeyear[103--108]{Lass1999b}, \citeyear[89--90]{Lass2006}) dates the beginning of this split back to the seventeenth century, based on \citet{Cooper1687}, who notes a lengthening of /æ/ in words like \emph{passed}, \emph{cast}, \emph{gasp}, \emph{barge}, and \emph{dart}, while the vowel remained short in words like \emph{pass}, \emph{bar}, and \emph{car}. It seems like consonant clusters (/-rC/ and /-sC/) following the vowel have favored the initial lengthening. In the following years, lengthening expanded to more phonetic contexts and there is evidence for a shift in quality as well. \citet[106]{Lass1999b} cites Flint (1740) who describes a lowering of the lengthened vowel to [aː] before /r/ (in \emph{art}, \emph{dart} and \emph{part}) and variably in \emph{bath}, \emph{castle}, \emph{calf} and \emph{half} (see \citealt{Kokeritz1944}). Another historical source cited by \citet[106]{Lass1999b} is \citet{Nares1784}, who has [aː] in \emph{after}, \emph{ask}, \emph{ass}, \emph{bask}, \emph{mask}, \emph{glass}, \emph{pass}, \emph{plant}, \emph{grant}, \emph{advance}, \emph{alms}, \emph{calm}, and \emph{palm}. \citet{Jones2006} and \citet{MacMahon1998} provide detailed analyses of eighteenth- and nineteenth-century comments on this vowel change and they agree with \citet{Lass1999b, Lass2006} that no clear pattern of the distribution of the lengthened and lowered vowel can be discerned based on the sources. Although the phonological environment clearly had an influence, some contexts seem to have had a stronger effect than others. While [aː] occurred fairly generally before tautosyllabic /r/, it was much more variable before tautosyllabic voiceless fricatives and nasals. The orthoepists often disagree on which words contain the lengthened and lowered vowel, which makes it even more difficult to identify phonological constraints on the distribution of the vowel. \citet[198]{Jones2006} therefore concludes that the “BATH/TRAP split was manifested primarily through lexical distribution”. With regard to the temporal development of the split, \citet[436-437]{MacMahon1998} describes it as “a gradual shift in favor of /aː/” and he finds that “the last decade of the eighteenth century and the first two of the nineteenth must be regarded as the critical period of change”. Next to the phonemic split, which resulted in minimal pairs like \emph{Sam} and \emph{psalm}, the realization of the new phoneme /aː/ has also been subject to variation and change. Innovative retracted variants of /aː/ (e.g. [ɑː]) have been noted by orthoepists since the late eighteenth century, and in many accounts quite detailed descriptions of different vowel realizations are given (\citealt[455--456]{MacMahon1998}; \citealt[193--198]{Jones2006}). \citet[189-190]{Jones2006}, however, warns of any certainty in assessing the exact quality of a vowel based on historical sources:


\begin{quote}
Vowel quality identification in this area of the phonology is notoriously difficult, the low vowel space being relative cluttered with segments distinguished by relatively small shifts in the Hz value of F\textsubscript{2}. [æ], [a], [ɐ], [ɑ] among unrounded variants, with rounded forms in [ɒ] and [œ]. Even for the trained modern observer, it can be difficult without instrumental analysis to be certain about the precise phonetic values of segments like this, especially in connected allegro speech contexts, so that assessing their actual shape from written historical evidence is well-nigh impossible with any degree of certainty.
\end{quote}

Regarding language-external factors influencing the \textsc{bath-trap} split and the different vowel realizations, sources point to region and social class. What can be deduced from the orthoepists’ comments is a restriction of the split to the South of England and a more extreme quality shift of the vowel in London and surrounding areas \citep[106--107]{Lass1999b}. A variety of insights can also be gained based on a study of current variability of /æ/ and /aː/ in Australian English. \citet{Bradley1991} investigates sixty morphemes whose pronunciations are mainly invariant in southeastern British English today, but vary in Australia between the \textsc{trap} vowel /æ/ and the \textsc{palm} vowel /aː/ (e.g. \textit{graph}, \textit{dance}). Based on synchronic data he reconstructs how the change proceeded in southeastern England. He summarizes the interpretation of his findings in \citet{Bradley2008}:

\begin{quote}
[P]laces settled by the early nineteenth century, and primarily by people of lower-socio-economic status, use more \textsc{palm} as in Sydney, Hobart, and Brisbane. Melbourne, settled in the mid-nineteenth century, with a more mixed population, shows a higher proportion of \textsc{trap}. Adelaide, settled later in the nineteenth century primarily by people of middle or higher socio-economic status, uses the highest proportion of \textsc{palm}, and shows a more advanced stage of the shift before nasal + obstruent than elsewhere in Australia, though not as far advanced as New Zealand or modern southeastern British English. This implies that the change in southeastern England was underway during the settlement of Australia, and that \textsc{palm} was a lower-status form in the late eighteenth and early nineteenth century, but had reversed its social value and become a high status form by the late nineteenth century. Furthermore, the change in the nasal + obstruent environment must have followed the prefricative environment by quite some time.
\end{quote}


Furthermore, he suggests that the backing of /aː/ happened fairly late in London and southeastern England because the phonetic realizations of /aː/ in Australia at the time of his study all range between central to front qualities \citep[227]{Bradley1991}.


The \textsc{bath-trap} split and the precise realization of the vowels used in the words belonging to these sets have been commented on and evaluated extensively in works on language in England since the late eighteenth century; \citet[78]{Mugglestone2003} even considers it to have been “one of the most prominent topics for discussion”. \citet[90]{Lass2006} states that while \citet{Cooper1687} and Flint (1740) gave a neutral, descriptive account of the vowels, \citegen[10-11]{Walker1791} comments are the first to contain a clear social evaluation. While he neutrally noted the presence of the “long sound of the middle or Italian \textit{a}” before /r/ (\emph{car}) and before <l> + labial (\emph{calm}, \emph{calf}), he considered it vulgar to use the same vowel in other environments (e.g. in \emph{glass}, \emph{fast}, \emph{after}, \emph{plant}). In the following century, the emerging picture was quite complex. \citet[117]{Bailey1996} describes the commentary as “muddled” because it is often difficult to distinguish between description and evaluation and to determine which vowel quality the commentators refer to. Nevertheless, there are two general and overlapping developments that can be observed. The first is that the \textsc{bath-trap} split continued to be subject of much criticism throughout the nineteenth century. \citet[118-119]{Bailey1996} cites three authors, the anonymous author of \textit{Vulgarities of Speech Corrected} (\citeyear[256--267]{Anonymous1826}, \citet[xxvi--xxvii]{Savage1833} and \citet[25]{Leigh1840}, who consider any vowel but the traditional [æ] in the \textsc{bath}{}-set to be vulgar and improper and associated with Cockney speakers. \citet[81]{Mugglestone2003} finds that Walker has shaped these prevailing negative attitudes considerably, which is for example visible in \citet{Longmuir1864}. However, a second general tendency that can be observed is a change of social evaluation. Jones writes in his \emph{General Pronouncing and Explanatory Dictionary} (\citeyear[ii-iii]{Jones1798}) that “giving to those and similar words [=~the \textsc{bath} set] the flat dead sound of \emph{a} in \emph{lack}, \emph{latch}, \emph{pan} \&c. is encouraging a mincing modern affectation” (cited in \citealt[194]{Jones2006}). This comment indicates a positive evaluation of the innovative lengthened, lowered and retracted variants as ‘normal’, whereas conservative [æ] is linked to a restricted set of affected speakers who aim at sounding elegant and mannered. The attention then started to focus on the precise realization of the vowel in the \textsc{bath}{}-set and on the question of which words actually belonged to the set. As a sort of compromise solution, many commentators developed the concept of an “intermediate” vowel, which was long and had a quality somewhere between [æ] and [ɑ]. This vowel was evaluated positively because it meant that the negative associations linked to the other vowels could be avoided \citep[122--123]{Bailey1996}. A characteristic example of such an evaluation is \citegen[34]{Smith1858} comment, which is cited by \citet[83]{Mugglestone2003}:

\begin{quote}
Avoid a too broad or too slender pronunciation of the vowel \emph{a}, in such words as \emph{command}, \emph{glass}, \emph{pass}, \&c. Some persons vulgarly pronounce the \emph{a} in such words, as of written \emph{ar}, and others mince it as to rhyme with \emph{stand}. […] Equally avoid the extremes of vulgarity and affectation.
\end{quote}


A comment by \citet[428]{Jespersen19091948} shows that the low back realization remained stigmatized at least until the early twentieth century: “A vulgar retracted or rounded [aˑ] is sometimes represented by novelists as \emph{aw}”. Similarly, Ripman (\citeyear[55]{Ripman1906}, cited by \citealt[88]{Mugglestone2003}) observes that “precise speakers” often prefer a “delicate” middle sound and use [aː] instead of [ɑː] due to “an excessive desire to avoid any Cockney leanings in their speech”.


\citet[83]{Mugglestone2003} calls it a paradox that “it was, in the end, not to be the iterated ‘correctness' of this ‘middle' variant, or even Walker's favoured [æ], but instead the theoretically ‘vulgar' [ɑː] which stabilized as one of the dominant markers of RP and the non-regional accent”. In her view, this is a sign that “prevalent attitudes to language will not necessarily affect the ultimate outcome of a linguistic change” and that “the real direction of linguistic change can and does instead regularly run counter to such precepts [found in prescriptive writings]”. How [ɑː] became an index of high social status, education and refinement is not answered in her study. Similarly, \citet{Agha2003} lists [ɑː] as a marker of Received Pronunciation in the twentieth century, but he does not offer any suggestion as to when and how the vowel became enregistered as standard and prestigious.

The historical development of the \textsc{trap}{}-vowel in America has not been studied in much detail. Eighteenth- and nineteenth-century descriptions and modern evidence from dialect atlases suggest that a split of the class into \textsc{trap} and \textsc{bath} through a lengthening, lowering and backing of the vowel to [aː], [ɑː], or even [ɒː] in the \textsc{bath} set occurred only on the American East Coast. \citet{Krapp19252} provides a detailed analysis of early accounts regarding the split and describes the pronunciation as he perceives it in the early twentieth century. He cites Webster’s works as early evidence for a low back vowel in “words of the type of \emph{artist}, \emph{arm}, \emph{clasp}, \emph{after}, \emph{balm}, \emph{grant}, \emph{advance}, etc.” occurring in New England speech \citep[63]{Krapp19252} and concludes that

\begin{quote}
one can infer from Webster’s remarks […] that in New England, as Webster heard the speech of New England, which was probably with a considerable degree of accuracy, the sound of [ɑː] had established itself in several large groups of words, whereas elsewhere in America the sound [æ] or [æː] prevailed in these words. \citep[68]{Krapp19252}
\end{quote}


Based on a variety of other sources exhibiting descriptions and comments on language (mostly dictionaries), \citet[69]{Krapp19252} finds that the \textsc{bath-trap} split and the use of [ɑː] in \textsc{bath} remained regionally restricted, mainly to New England: “the pronunciation of words like \emph{laugh}, \emph{dance}, \emph{branch}, etc. with [ɑː] has established itself nowhere as a popular custom outside of New England”. Regarding the frequency of use he cites \citet{Grandgent1899}, who estimates that [ɑː] was most widely used in New England between 1830 and 1850 and that its use declined towards the end of the century. However, in the first half of the twentieth century, the split is still observed by fieldworkers for several dialect atlas projects on the East Coast of the United States. \citet{Kurath1961} summarize the results for the vowel in \emph{aunt} by stating that it is predominantly a low-front vowel in New England and a low-back to low-central vowel in Tidewater Virginia, whereas outside of these areas, the front vowel [æ] is “nearly universal” \citep[135]{Kurath1961}. The exception are “cultured speakers” in larger cities like New York City, Philadelphia and Charlottesville \citep[135]{Kurath1961}. The results for words like \emph{calf}, \emph{glass} and \emph{dance} (the vowel followed by a voiceless fricative or by /n/ plus a dental consonant) show a similar pattern. In Eastern New England, the low-front vowel [aː] is often found, but usage varies considerably and speakers “freely shift” between this vowel and [æ], which is common elsewhere, except again for “some cultured urban speakers” who use a low-central vowel occasionally \citep[136]{Kurath1961}. In addition, the low-front [aː] is also used by speakers along the coast of South Carolina and Charleston, and \citet[136]{Kurath1961} point out that these speakers are not “cultured” but “common folk”. Low front [aː] is also used in the word \emph{can’t} in large parts of Eastern New England by speakers of all social classes, but it is especially common in important cities such as Boston, Providence, and Newport. The vowel is again a “prestige pronunciation” in some cities outside Eastern New England (Hartford, Manhattan, Rochester, and Philadelphia) and it is used by a few cultured urban speakers in Virginia and occasionally by common speakers in coastal South Carolina and Georgia. That a retracted vowel in \textsc{bath} persisted in Boston until the second half of the twentieth century is reported by \citet[56]{Nagy2008} based on \citegen{Laferriere1977} finding that low-central [aː] occurs in words before /f/ and /θ/ and in some words before /n/ (e.g. \emph{aunt}) and much more variably in words before /s/ and in other words before /n/ (e.g. \emph{dance}). In the southern areas, the use of a low-central or back vowel has declined, and it is only used in a few words like \emph{aunt} or \emph{rather} \citep[97]{Thomas2008}.


\citegen{Krapp19252} account also offers a variety of insights into the evaluation of the \textsc{bath-trap} split and of the different vowels used in \textsc{bath.} He attributes high importance to the fact that Webster recommended a lengthened and retracted vowel in \textsc{bath} in his \emph{American Spelling Book} and in his further dictionaries. Webster explicitly evaluates [æ] as a feature of “mincing, affected pronunciation” in his \emph{Compendious Dictionary} (\citeyear{Webster1806}, cited in \citealt[68]{Krapp19252}), which reflects \citeapo{Jones1798} comment cited above. \citegen[68--69]{Krapp19252} interpretation of Webster’s comments is that he “regarded [ɑː] as the good-home spun pronunciation”, which was to be seen as positive even though it “was at least open to the charge of being a somewhat rustic one”. In the course of the nineteenth century, Webster’s books and New England culture in general became increasingly influential, which, according to \citet[69]{Krapp19252}, led to [ɑː] being evaluated as the refined and elegant variant and losing associations with rustic, common speech. \citet[75]{Krapp19252} attributes an important role to instruction in schools in the establishment and dissemination of positive attitudes towards [ɑː] because the “New England school book and also the New England school and school teacher exerted an enormous influence over American education during the second and third quarters of the nineteenth century”. As in England, \citet[74--75]{Krapp19252} notes the introduction of an “intermediate sound of \emph{a}” in the debate, which he transcribes as [aː]. A well-known commentator who advocates the use of this variant is Richard Grant White, a New Yorker, who writes in his \emph{Words and Their Uses} (\citeyear[62]{White1870}) that “the broad \emph{ah} sound of \emph{a}” indicated “social culture which began at the cradle” whereas the use of “the thick throaty sound of \emph{aw}” or “oftenest, […] the thin, flat sound which it has in “an,” “at,” and “anatomy”” is a sign of a lack of culture and refinement (cited in \citealt[76]{Krapp19252}). Regarding the development of evaluations, \citet[76--77]{Krapp19252} speculates that the “period of highest esteem of [ɑː], or [aː], outside New England, perhaps falls in the two decades from 1870 to 1890”, so forty years after its greatest popularity \emph{within} New England. Despite the similarity to the development of attitudes in England, he decidedly argues against any British influence on American evaluations: “There is no evidence to indicate that any single pronunciation which has become general in America has become so through imitative influence of British pronunciation” \citep[80]{Krapp19252}. Instead, he argues for a parallel development which “was caused by the accident that the sound [ɑː] became established as a popular speech sound in that very section of America which came to be recognized as our first seat of culture, and that in England, by a curious similarity, the situation was exactly the same” \citep[80]{Krapp19252}. However, he also notes that towards the end of the century not only the use of [ɑː] declines, but that the sound is also increasingly evaluated negatively as affected. The comment by Lounsbury (\citeyear[101]{Lounsbury1909}, cited by \citealt[76--77]{Krapp19252}) that the sound [ɑː], “when heard, is already looked upon by many as an affectation” therefore indicates an evaluation which is in complete opposition to Webster’s evaluation of [æ] as affected a century earlier. Next to [æ], [aː] and [ɑː], \citet[36]{Krapp19252} briefly mentions a fourth variant, a back and rounded vowel, which he finds to be “one of the marks of the grotesque pronunciation of the haw-haw type of comic Britisher on the stage” which indicates that there was indeed a variant of the \textsc{bath}{}-vowel associated with British speech. It is at least theoretically possible that the variant targeted in these humorous performances was [ɑː], but that it was exaggerated to an extent that it was in fact [ɔ] or [ɔː].

The view that British influence did not play a role in the evaluation of the vowel is not shared by \citet[141]{Montgomery2001}, who proposes an alternative account. He argues that

\begin{quote}
[æ] in \emph{fast}, \emph{bath}, \emph{aunt} is a colonial lag reflecting earlier emigration, and that alternative pronunciations gained currency in the eighteenth century as New Englanders and Virginians attempted to keep pace with English speech, facilitated by the contact of coastal American cities with England, which was closer than that of the hinterland.
\end{quote}


In this view, the prestige of [ɑː] or [aː] is attributed to the association of these variants with English speech norms, or, as \citet[140]{Montgomery2001} puts it, with “British fashion”.


\citet[136]{Kurath1961} also confirm a perception of [aː] as “refined” and the fact that it was used primarily by “cultured”, urban speakers in those areas where it was not commonly used can also be regarded as evidence of an association of the variant with education and a higher social status. The “intimate commercial and cultural relations with London and with cultured Englishmen throughout the Colonial period and after” are also proposed by \citet[136]{Kurath1961} as a likely reason for the high prestige of the variant. However, they also note that there are speakers in Eastern New England who regard the variant as rustic, which indicates the presence of conflicting evaluations in the first part of the twentieth century. That these conflicting evaluations continue at least in Boston until the 1970s is shown by \citet[105--106]{Laferriere1977}, who identifies “nonlinguistic labels” for what she calls the “backing of [æ]”.\footnote{Unfortunately, \citet{Laferriere1977} does not give any information on how exactly she elicited the labels from the informants.} She finds that it has a “complex social profile”: It is first of all associated with older people, which matches the production data. Secondly, it is linked to Boston identity by middle-aged and older people, who associate themselves with the city. Thirdly, it is stigmatized by speakers “whose ties to the city are more precarious because of higher education or prolonged residence outside the New England area” \citep[105]{Laferriere1977}, even though they tend to use the variant themselves. And lastly, \citet[106]{Laferriere1977} finds that the backed variant is used “as a hypercorrect indication of erudition” and gives examples of older speakers using the backed variant in words in which it normally does not occur (e.g. \emph{Master’s}) to impress others.

This overview shows that the vowel in \textsc{bath} has quite a complex history, not only regarding changes in use but also changes in evaluation. The supposed associations of the innovative variant with New England speech, with culture and refinement \emph{as well as} rusticity, and also with fashionable British speech make it an interesting variant to explore in a study on the enregisterment of American English: Which of these associations can be found in newspaper discourse? Was the situation in nineteenth-century America the same as today, when lengthened and retracted [ɑː] is not considered to be a “general” American variant? The first analyses of articles containing \emph{hinglish} revealed pronunciation respellings representing a lengthened variant of the vowel. In a few cases <h> was used to mark the lengthening (\emph{cahn’t}, \emph{pahdon}) in several cases <r> can be found (\emph{carn’t}, \emph{blarsted}, \emph{’arf} ‘half’). As a lengthened variant has been well-established before <r> (e.g. in \emph{car}, \emph{arm}, \emph{part}), it seems to have been a good indicator of vowel quality. However, there is also the possibility that <r> was used to represent hyper-rhoticity (see \sectref{bkm:Ref530736302} for details on rhoticity and hyper-rhoticity), so <r> is not as reliable an indicator of vowel quality as <h> or <w>, which is the third way of representing the realization of the \textsc{bath} vowel (\emph{rawther}). This last way of representing the vowel additionally indicates a back quality because the combination <aw> usually represents back and rounded [ɔː] (e.g. \emph{awe}, \emph{claw}, \emph{paw}). It is this last spelling variant which I choose to focus on in the present study because it indicates a representation of a more extreme quality within the range of possible variants of the \textsc{bath} vowel and it is therefore more likely to be used to draw attention to a back quality of \textsc{bath} than other variants. As a search term, I decided to use \emph{dawnce} because first of all, the vowel is more variable before /-nC/-clusters than in other environments and secondly because the lexical item \emph{dance} is fairly frequent and has more semantic content than for example \emph{can’t}, \emph{rather}, or \emph{last}. An initial search in the NCNP database revealed that \emph{dawnce} occurs 44 times, while \emph{darnce} occurs three times and \emph{dahnce} only twice, which confirms the hypothesis that the variant <aw> is more often used than other variants for representing the quality of the \textsc{bath} vowel.

The analysis of articles containing \emph{dawnce} revealed articles in which \textit{dawnce} co-occurs with representations of another linguistic form: non-rhoticity. Interestingly, spelling representations of non-rhoticity overlap with representations of the \textsc{bath} vowel. The article “National What-Is-Its” (October 25, 1887\citesource{October251887}) contains for example the forms \emph{dawnce}, \emph{cawn’t}, \emph{pawth} and \emph{chawnce}, in which <aw> represents a back vowel, and the forms \emph{rathaw}{} and \emph{fathaw}, in which <aw> represents non-rhoticity. Before discussing the consequences of this overlap for the choice of search terms, I will summarize prior research on non-rhoticity in the following section.

\subsection{\emph{deah,}  \emph{fellah} {and}\emph{ bettah}: non-rhoticity}
\label{bkm:Ref530736302}\hypertarget{Toc63021226}{}
One of the most studied phonological forms in English (historical) linguistics is non-rhoticity, which is defined in slightly different ways but generally constitutes the absence of /r/ in a post-vocalic and non-prevocalic position (e.g. \emph{car} /kɑː/, \emph{cart} /kɑːt/). Research on the structural development of non-rhoticity, which has often been termed \textit{/r/-loss}, has identified different phases based on the linguistic context in which the loss of /r/ occurred. \citet[91--92]{Lass2006} distinguishes an earlier phase which is characterized by the loss of /r/ without an accompanying lengthening of the preceding vowel (e.g. \emph{arse} /æs/) from a later phase which is characterized by a lengthening of the preceding vowel (e.g. \textit{arm} /ɑːm/). In his view, it is only in the second phase that /r/-loss occurs systematically; before that it is sporadic and lexically restricted. \citet[124]{Minkova2014}, however, regards vowel lengthening only as “one possible outcome rather than an essential stage in the process of /r/-loss”. In her analysis, it is rather the position of /r/ in the syllable which is crucial for distinguishing two developments which not only differ in their temporal occurrence, but also in their underlying mechanism. The first development is the loss of /r/ in the syllable coda preceding one or more tautosyllabic consonants (as in \emph{cart} /kɑːt/), which she terms “pre-consonantal /r/-loss”. The earliest evidence for it stems from the eleventh century: words spelled without an <r> that had contained an <r> before (e.g. OE \textit{gorst} - ME \textit{gost} ‘gorse’).\footnote{\citet{Minkova2014} argues against several others scholars here, who do not regard these early instances as part of a general development of non-rhoticity.} In addition, poetic rhymes of words spelled with and without <r> point to a loss of pre-consonantal /r/. The place of articulation of the following consonant seems to affect the change: /r/ was lost first before coronal consonants ([t], [d], [n], [l], [θ], [ð], [s], [ʃ], and [dʒ]), while it remained before the other non-coronal consonants until the seventeenth century (\citealt[122]{Minkova2014}, citing \citealt{Hill1940}). Nevertheless, \citet[124]{Minkova2014} concludes that “the earlier and later cases of /-rC/ simplification represent a single historical process stretching over more than six centuries” because she assumes that assimilation is the mechanism underlying all cases of pre-consonantal /r/-loss. She also adds that /r/-loss affected “different dialects and different lexical items unevenly”. The second development is the loss of /r/ in the syllable coda without a following consonant (as in \emph{car} /kɑː/), which she terms “post-vocalic /r/-loss”. The reason for separating it from the first development is a different mechanism of change which involves analogy rather than assimilation. The starting point of this type of /r/-loss is a matter of debate, which largely ranks around the interpretation of unetymological <r> in seventeenth-century spellings. \citet[126]{Minkova2014} cites <winder(e)s/wynders> ‘windows’ (1601, 1613), <feller> ‘fellow’ (1639), and <pillars> ‘pillows’ (1673). Insertions of <r> have been (and still are) often interpreted as evidence for loss of post-vocalic /r/. They are either regarded as signs of hyper-correction and linguistic insecurity (using the same argumentation as in the case of /h/-insertions) or, when they appear word-finally preceding a vowel in the following word, as an intrusive /r/ which is typical for non-rhotic, but not for rhotic accents. However, a very detailed study by \citet{Britton2007} shows that it is much more convincing to interpret these spellings as indicating hyper-rhoticity, which he defines as “the appearance, in rhotic accents, of epenthetic, unetymological rhyme-/r/” (\citeyear[525]{Britton2007}).\footnote{It is important to note that \citet[527]{Britton2007} distinguishes hyper-rhotic /r/ from intrusive /r/ because the former is “not attributable to a sandhi development, being probably […] first established in the lexicon, rather than having phonological origins as a linking, hiatus-breaking device”.} His analysis of the \emph{Linguistic Atlas of England} (LAE, \citealt{Orton1978}) reveals that there is modern dialectological evidence for hyper-rhoticity in the \emph{\textup{comm}}\textsc{a} set in rhotic accents of the South and the south-west Midlands (see the maps for \textsc{meadow}, \textsc{yellow}, and \textsc{window}).\footnote{He also mentions the occurrence of hyper-rhotic forms in some items of the \textsc{bath} and \textsc{palm} sets (exemplified in maps of \textsc{last}, \textsc{calf} and \textsc{half}) and of the \textsc{thought} and \textsc{cloth} set (found in maps of \textsc{sawdust}, \textsc{slaughterhouse}, \textsc{straw}, \textsc{walk}, \textsc{daughter}, \textsc{broth}, \textsc{cross}, \textsc{off}, \textsc{fox}). The regional distribution of these forms is more restricted than those in the comm\textsc{a} set, however, and largely confined to west Shropshire \citep[528]{Britton2007}.} This suggests that hyper-rhotic forms also occurred in the history of English, beginning in the seventeenth century with the examples cited above. This is a convincing argument against the loss of post-vocalic /r/ in the seventeenth-century, and it also has repercussions for accounts which suggest that /r/-loss became generally common in the eighteenth century.


With regard to the eighteenth century, \citet[530]{Britton2007} argues that the interpretation of comments by orthoepists (\citealt[37]{Walker1791}, \citealt[264]{Elphinston1786}, Douglas 1779, see \citealt{Jones1991}) on forms like \emph{winder} and \emph{feller} as constituting evidence for intrusive /r/ (and, connected to that, non-rhoticity) is not necessarily the only possible one, but that it is even more reasonable to assume that these forms are instances of hyper-rhoticity. The mechanism which Britton assumes to underlie this unetymological insertion of /r/ is analogy. An analysis of an early and mid-eighteenth century corpus by \citet{Soskuthy2013} reveals that words with an etymological /r/ following a final /-ə/ (e.g. \textit{better}) were much more common than those ending in /-ə/ without being followed by /r/ (e.g. \textit{idea}).\footnote{The same difference holds for /ɔː/ and /ɑː/ where the token frequency of words with /ɔːr/ and  /ɑːr/ is far higher than that of those with /ɔː\#/ and /ɑː\#/ \citep[60]{Soskuthy2013}.} As a consequence, speakers extended the more common /-ər/-pattern in the former class of words to the relatively few words in the latter class. Sóskuthy uses the results mainly to explain the occurrence of intrusive /r/ in non-rhotic accents, but he also agrees with \citet{Britton2007} that they can be used to account for hyper-rhoticity in rhotic accents \citep[77]{Soskuthy2013}. With regard to the early nineteenth century, \citet[529]{Britton2007} cites \citegen{Savage1833} reports of forms like \emph{pillar} ‘pillow’ and \emph{eye-dear} ‘idea’ as features of popular London speech and polite usage. As Savage only provides very few examples of non-rhotic forms in his account, \citet[529]{Britton2007} finds it reasonable to assume that again hyper-rhoticity plays a greater role here than it has commonly been assumed. \citegen{Britton2007} analysis therefore seriously challenges the view that unetymological insertions of <r> in the spellings are always evidence for non-rhoticity. His interpretation suggests that the spread of non-rhoticity in London and the south-east of England is largely a phenomenon of the middle to late nineteenth century.

This view is backed by other research, as for example \citeapo{Jones2006} analysis of orthoepists’ comments on /r/ which leads him to conclude that “post-vocalic [r] loss was very much a minor characteristic of the phonology of late eighteenth-century English” (\citeyear[261]{Jones2006}). Of the few contemporary comments on the development of non-rhoticity in England, Walker’s observation that “[i]n England, and particularly in London, the \emph{r} in \emph{lard}, \emph{bard}, \emph{card}, \emph{regard}, \&c. is pronounced so much in the throat as to be little more than the middle or Italian \emph{a}, lengthened into \emph{laad}, \emph{baad}, \emph{caad}, \emph{regaad}” and that it is “sometimes entirely sunk” in London  (\citeyear[50]{Walker1791}) is probably most well-known. It is often cited as clear evidence for an ongoing change towards non-rhoticity. It should be noted, however, that in all examples but one (\emph{baa} ‘bar’) the /r/ is weakened or lost in pre-consonantal position before /d/. It is therefore more indicative of pre-consonantal /r/-loss than of post-vocalic /r/-loss. In the latter position, /r/ was probably still present which is also supported by \citegen{Savage1833} extensive discussion of /r/-insertions. \citeapo[342]{Jones2006} conclusions therefore need to be restricted to word-final /r/-loss:

\begin{quote}
Savage's data on \textit{r} seem to suggest that in the early part of the nineteenth-century [r]-loss was not yet an active, ongoing change (even among the ‘vulgar' speakers, and certainly not in the \textit{Orthoepy}). On the other hand, syllable-final [r]-adding was ongoing and active and gaining ground everywhere, not least in socially acceptable circles.
\end{quote}

Further evidence for the view that loss of /r/ was by no means general in the middle of the nineteenth-century comes from research on the historical development of (non-)rhoticity in New Zealand English. Many first generation New Zealand-born Europeans whose speech has been documented and analyzed as part of the Origins of New Zealand English project were rhotic to some degrees, which leads \citet[804]{Hay2005} to conclude that the “lack of rhoticity in New Zealand English [...] is not an inherited feature from British English, but rather the result of a change that occurred (or at least went to completion) in New Zealand”.

The picture which emerges is evidently not complete yet, but it seems as if post-vocalic /r/-loss began in the eleventh century when /r/ was followed by a coronal consonant in the same syllable. The loss extended to non-coronal contexts in the seventeenth century. It is likely that early unetymological insertions of <r> in unstressed syllables point to hyper-rhoticity and not to a beginning loss of /r/ in postvocalic and non-preconsonantal positions in the seventeenth century. At what time the loss of /r/ in the latter position became more widespread is not clear – orthoepists’ comments, especially \citet{Ellis1869}, suggest that it was common in London, even among educated speakers, by the middle of the nineteenth century. However, based on a careful interpretation of several types of evidence, \citet[475]{MacMahon1998} concludes that “[t]he evidence for considerable variation between rhoticity and non-rhoticity, with intermediate semi-rhoticity, especially during the later nineteenth century in the educated South of England, is, clearly, very strong”. The continuing presence of rhoticity is also supported by first-generation New Zealand speakers exhibiting some degree of rhoticity. The change to non-rhoticity did not affect all regions. Rhoticity is maintained in Ireland and Scotland to this day (even though there are also studies showing an increasing non-rhoticity at least in some areas and groups of speakers). The LAE maps show that rhoticity has also persisted in some regions in England for a long time, especially in the south-west and the south-west Midlands, but again, non-rhoticity is on the rise here as well.

The discussion on when and how non-rhoticity began to spread in England is also relevant for the reconstruction of the development of non-rhoticity in North America. Two points are important here: first, the assumed rhoticity or non-rhoticity of the settlers based on their origin in England and, second, the two different types of /r/-loss. As shown above, /r/-loss in preconsonantal position was already attested by the time that the first settlers emigrated to America, so it has been argued by \citet{Kurath1971}, \citet{Fisher2001} and \citet{Montgomery2001} (among others) that these early settlers must have brought this type of non-rhoticity to America as well. Evidence for early non-rhoticity is provided by \citet[228--230]{Krapp19252}, who lists 35 words in which <r> has been omitted in the spelling which he found in American town records and in one collection of diaries in the seventeenth and eighteenth century. He also lists 24 instances of unetymological <r>-insertions which he interprets as evidence for non-rhoticity, but in the light of the detailed discussion about these insertions above, it is more likely that these cases are instances of hyper-rhoticity as well.\footnote{This is supported by a comparison of omissions of <r> in the lett\textsc{er} set and insertions of <r> in the comm\textsc{a} set: Omission of <r> in the lett\textsc{er} set occurs only three times (of all 35 examples) and in two of these examples, <r> is omitted before <s> (\emph{Whiticurs} > \emph{Whittacus}, \emph{Rogers} > \emph{Roges}), that is, in a linguistic context (preconsonantal, before /s/) which is typical of early /r/-loss. So while /r/-loss in lett\textsc{er} is hardly attested at all, 16 out of the 25 insertions of <r> (64 \%) occur in the comm\textsc{a} set (e.g. \emph{feler} ‘fellow’, \emph{famerly} ‘family’). To argue that <r> insertions reflect the writer’s non-rhoticity because he is unsure about where to use an <r> in the spelling is not convincing if there are no instances where the <r> is also omitted in the context where it is most frequently inserted.} Furthermore, he cites numerous examples of this early type of /r/-loss in poetic rhymes. This contradicts the assumption by \citet[136]{Newman2014} that non-rhoticity emerged only at the end of the eighteenth century in England and that therefore all early settlers must have been rhotic. Given the evidence described above, this view is only possible if non-rhoticity is defined as necessarily encompassing both types of /r/-loss: pre-consonantal \emph{and} post-vocalic. The question when and why /r/-loss extended to more environments and spread in the United States is more difficult to answer. \citet[171]{Kurath1961} state with regard to the regions where \emph{door} is pronounced without /r/ by the informants interviewed for the Linguistic Atlas of the Eastern United States that

\begin{quote}
{four geographically separated subareas have a mid-central semivowel /ə̯/: Eastern New England as far west as the Connecticut Valley (to the crest of the Green Mountains in Vermont), Metropolitan New York, the Upper South (the southern peninsula of Maryland, Virginia to the Blue Ridge, and north-central North Carolina), and the Lower South (South Carolina and Georgia).}
\end{quote}


With regard to the historical development of the non-rhotic pronunciation of words like \emph{door} in America, they speculate that


\begin{quote}
{[i]n English folk speech of today,} \emph{door}{,} \emph{care}{,} \emph{ear}{ end in unsyllabic /ə̯/ in the eastern counties north of the Thames, in /r/, articulated as a constricted [ɚ], in the south and the west. In all probability both types came to this country with the first colonists and could be heard in all of the colonies. With the acceptance of /doə̯, keə̯, iə̯/ in Standard British English during the eighteenth century, it would seem that this type acquired prestige in the chief American seaports on the Atlantic coast—Boston, New York, Richmond, Charleston—and spread from there to the hinterland. In the inland, on the other hand, the post-vocalic /r/, common from the beginning, came to be generally established, as also in the Quaker-dominated port of Philadelphia and vicinity.} \citep[171]{Kurath1961}
\end{quote}


So on the one hand, they stress that non-rhotic forms have been a result of inheritance, but on the other hand they also stress a parallel development in England and in some English regions. Based on the discussion above, it is plausible to specify that what is inherited is the pre-consonantal /r/-loss, whereas the later parallel change concerns the post-vocalic /r/-loss. However, there is still considerable disagreement about the timing and the causes of the later change. One influential hypothesis is that the increasing prestige and use of non-rhotic forms (in all environments) in England also influenced developments in America. \citet[105]{Bailey1996} also argues along these lines that


\begin{quote}
[i]n North America, only New England, southern New York, and the coastal South (from Baltimore to New Orleans) took part in the general weakening of noninitial \emph{r}, though even in these districts the loss of \emph{r} was not universal. There is nothing surprising in this parallel development for, despite political differences, the cities were part of a single speech community, increasingly so in the first half of the [nineteenth] century, as business and cultural contacts developed between British ports and the sea-coast cities of North America.
\end{quote}


Not only the increasing non-rhoticity in the east, but also the retention of rhoticity in western parts of the United States is explained by drawing on the prestige that non-rhoticity supposedly used to have as a British form:


\begin{quote}
Not everyone rushed to mimic London fashion. Little in the first two centuries of the transatlantic migration of northerners induced a desire to follow London fashions in English. Most North American settlers, had little direct experience of London speech and less interest in imitating it. \citep[107]{Bailey1996}
\end{quote}


Given the finding that by the end of the nineteenth century there was still considerable variability in the use of non-rhotic forms in southeastern England, dating the increasing use of non-rhoticity in the eastern United States to the late eighteenth and early nineteenth century based on developments in England is not convincing – in this line of argumentation it is more likely to have occurred later in the century.


However, there are also scholars who argue against any influence of British speech. For example, Krapp states that even though “the weakening and disappearance of [r] as a final sound and before consonants was particularly noted at about the same time, that is, at the end of the eighteenth century” in England and in America, this is only “a coincidence in the \emph{critical record of} historical development” and not “an actual coincidence in historical development” (\citealt[227]{Krapp19252}, emphasis mine). Similarly, \citet[49--50]{Dillard1992} is very doubtful about British influence on American developments:

\begin{quote}
Most Americans on the East Coast – and even more so those to the west – were busy with concerns of their own rather than with connections to England. \citet{Knights1969} stresses the ‘incredible’ mobility of the Boston population between 1830 and 1860, roughly the period in which the change should have been taking hold. […] Whether such mobility would contribute to a stable, invariant shift of /Vr/ to /V0/ appears to be something less than a matter of socio-linguistic certainty. […] The picture of a stable community, with the top group accepting a British innovation and the lower classes placidly following suit, does not seem to follow.
\end{quote}

\citet{Bonfiglio2002} also looks at the factor of mobility with regard to the case of New York City and argues that with the opening of the Eerie Barge Canal in 1825, immigration to New York from western parts, especially the Great Lakes region, increased significantly and, as a consequence, New Yorkers came into contact with a large number of rhotic speakers. Against this background, he argues that “one must conclude that the dropped /r/ of New York City was not the result of British influence in the nineteenth century, but instead the result of the resistance of indigenous pronunciation to external influences” (\citeyear[51]{Bonfiglio2002}). However, this view requires that non-rhoticity has already been firmly established in New York City by the early nineteenth century – so firmly that the immigrants coming to New York adapted to the speech pattern there.

Linguists not only disagree about when and why non-rhoticity became a feature of the speech in eastern regions of the United States, but also about when and why this process reversed. Very prominently, Labov argued in his study on New York City English in the 1960s that “[t]he introduction of /r/ in the New York City system as a dominant element in the prestige dialect may certainly be traced in the 1930s, but apparently made a great step forward in the years coinciding with World War II” \citep[376]{Labov2006}. In his view, the increasing use of rhotic forms and the shift in norms towards rhoticity being the prestigious form “reflected the abandonment of the earlier prestige form of Anglophile English” \citep[296]{Labov2006}. As already discussed in \sectref{bkm:Ref517077661}, \citet{Bonfiglio2002} disagrees with this view because he not only finds that the crucial shift in prestige occurred already in the first half of the twentieth century, but also because he links it to a sinking prestige of New York City, which becomes viewed as ethnically “contaminated” because of large numbers of African Americans and immigrants living there. \citet{Fisher2001} makes yet a different case: He claims that non-rhoticity lost its prestige after the Civil War because economic and political power shifted away from regions which had traditionally been most closely linked to England (non-rhotic Virginia and Boston) towards New York, Pennsylvania, and the trans-Appalachian Middle West. The last two regions have traditionally been rhotic, but he also finds that more and more rhotic speakers migrated to New York, where the influence of the old non-rhotic Colonial elite, with its ties to England, weakened in the last decades of the nineteenth century, “so that nonrhotic pronunciation lost its prestige” \citep[77]{Fisher2001}. However, Bonfiglio’s study is the only one that is supported by a systematic and detailed analysis of sources, whereas Fisher’s and Labov’s views with regard to causes of the shift of prestige of rhoticity remain rather speculative. Recent studies show that the change towards rhoticity has continued in New York City \citep{Becker2014} and that is has occurred in Boston and New Hampshire as well \citep{Nagy2010}.

In White Southern American English, non-rhoticity is one of the forms that Schneider considers to be traditionally southern, while rhoticity marks the new South \citep[299]{Schneider2007}. \citet[107]{Thomas2008} summarizes the research findings on the southern development, which have identified World War II as a turning point:

\begin{quote}
Before World War II, non-rhoticity was prestigious, appearing most frequently among higher social levels and spreading (except, perhaps, in \textsc{nurse} words). Afterward, rhoticity became prestigious and non-rhoticity became most common among lower social levels. Females have forged ahead of males in this change.
\end{quote}


\citegen{Feagin2015} study on the development of several linguistic variables in Anniston, Alabama, also includes non-rhoticity. She observes that non-rhoticity is almost categorical in the speech of the two older upper class speakers (born 1882 and 1890), but in the speech of the four older rural and urban working class speakers (born 1895, 1899, 1907 and 1911) rhotic forms are already more frequent than non-rhotic forms (63\% rhoticity for the speaker born in 1907 and 84-87\% rhoticity for the rest of the speakers). The two working class speakers born after World War II (1956 and 1957) also have a high degree of rhoticity (86\% and 92\%), but a drastic change can be observed for the upper class: The speakers born in 1953 and 1955 use rhotic forms at a rate of 65\% and 91\%. Even though no major conclusions can be drawn based on this low number of speakers in one location, it only partially confirms the development described by Thomas. While the middle of the twentieth century does seem to constitute a turning point, this only affects the upper class speakers – and even though the rate of rhotic forms in the speech of the upper class speakers is drastically higher than that of the upper class speakers born in the nineteenth century, it does not surpass the rate of the working class speakers born before World War II.


Finally, it has to be noted that the development of non-rhoticity was different for African American English. Even though evidence is scarce, non-rhoticity has shown to be a fairly stable feature. \citet[107]{Thomas2008} states that

\begin{quote}
[i]t should be noted that the dramatic increase in rhoticity applies only to white Southerners; African Americans remain largely non-rhotic, except in the \textsc{nurse} class, and [...] social polarization of the two ethnicities magnified during the civil rights movement may be related to the divergence in rhoticity.
\end{quote}

The discussion about the reasons for the development of the presence or absence of post-vocalic /r/ in the United States shows that the factor of prestige is often drawn on to explain changes. In this regard, the prestige of non-rhoticity in England also matters because it is assumed by some to affect the prestige of the form in America. However, most studies usually take the increase in non-rhoticity in (southeastern) England as a sign of the prestige of the variant – but they do not take studies on metadiscursive activity on /r/ into account which provide a more detailed picture of the social meanings of non-rhoticity or rhoticity respectively. I will thus summarize findings on metadiscourses on /r/ in England before turning to the rather scarce investigations of metadiscursive activities surrounding (non-)rhoticity in America. {}

\citegen[86--94]{Mugglestone2003} overview of attitudes towards the emerging non-rhoticity in England emphasizes the great discrepancy between reality and prescriptive comments. She notes that throughout the nineteenth century the loss of /r/ was commented on negatively. Its use was considered vulgar and attributed to illiterate speakers from lower classes. The stereotype often invoked was that of the Cockney speaker. Even in the second half of the nineteenth century, when postvocalic [r] was probably dropped by large numbers of speakers in London and southern England more generally, its absence was still considered incorrect. An important reason for this attitude was probably the high valuation of education and literacy. The occurrence of <h> in the spelling is an important argument for demanding the pronunciation of [h]; it is only logical that this applies to <r> as well. \citet[88]{Mugglestone2003} cites a comment by the poet Gerard Manley Hopkins (\citeyear{Hopkins22December1880}) on Keats’s rhymes of, for example, \emph{higher} and \emph{Thalia}, which illustrates a way to reconcile the logical incorrectness of non-rhoticity with its obvious widespread presence: He evaluates the rhymes as “most offensive, not indeed to the ear but to the mind”. So even though they \emph{sound} right, they are still to be regarded as incorrect by the educated and literate speaker. \citet[90]{Mugglestone2003} concludes that

\begin{quote}
[t]he use of post-vocalic and final [r] […] was evidently able to achieve the paradoxical status of being present as a prescriptive norm in the ‘best’ English—even while descriptive discussion of non-localized pronunciation was forced to acknowledge its absence in precisely the same positions.
\end{quote}

\citeapo[259--260]{Jones2006} analysis suggests that the negative evaluation of the realization of /r/ as an alveolar or uvular trill played an important role in the metadiscourse on /r/ as well. On the one hand, this realization was described as foreign and provincial and often attributed to Scottish and Irish speech. On the other hand, it was described as rough, harsh, and canine (it was compared to a dog’s snarl). The commentators preferred the ‘softness’ of the alveolar approximant which is regarded as typically English, but also warned about its complete omission. Walker for example comments that “this letter is too forcibly pronounced in Ireland” but at the same time “it is often too feebly sounded in England, and particularly in London, where it is sometimes entirely sunk” (\citeyear[50]{Walker1791}). This shows the link between the negative evaluation of the trilled /r/ and non-rhoticity: They represented opposing poles on a continuum between a highly sonorant realization of /r/ and no constriction at all. This is reflected in the nineteenth-century comment by Smart (\citeyear[33]{Smart1836}, cited by \citealt[339]{Jones2006}) who again states that both extremes have to be avoided. However, \citeapo[603]{Ellis1869} remarks can be interpreted as indicative of a change in attitude, if not in prescriptivists’ writings then in society more generally: He states that in London \emph{farther}, \emph{lord}, \emph{stork}, and \emph{drawers} are “reduced” and pronounced like \emph{father}, \emph{laud}, \emph{stalk}, and \emph{draws} “even in the mouths of educated speakers”. He acknowledges the prevailing prescriptivist tradition by writing “I have usually written (ɹ) final in deference to opinion” but at the same time he emphasizes that the prescribed difference between the word pairs is only upheld by careful speakers “when they are thinking particularly of what they are saying”, which suggests that /r/-less pronunciations had become more and more accepted in everyday life. Those speakers who tried to follow prescriptivist opinion and not omit the /r/ were also criticized, as for example in the anonymous usage guide \emph{Hard Words Made Easy}: “Some of our public speakers, who push the accuracy of utterance beyond a wholesome limit, get the habit of trilling the \emph{r} so much that one would think that they wished to be thought unlettered Scotch or Irish peasants” (\citealt[4]{Anon1855}, cited by \citealt[90]{Mugglestone2003}). In my opinion, it is reasonable to argue that the avoidance of the Scottish or Irish ‘rough’ /r/ was considered to be most important, even if it led to the omission of /r/ in post-vocalic contexts. A text on teaching reading and pronunciation, entitled \emph{The First Part of the Progressive Parsing Lessons} and printed in 1833 (cited by \citealt[248]{Mugglestone2003}) exemplifies this: “\emph{Ar} must be pronounced with the tip of the tongue pressed against the gums of the under teeth, to prevent the \emph{r} having its rough or consonant sound”. It is highly unlikely that any constriction can be achieved if these instructions are followed and it is definitely impossible that an alveolar approximant can be produced if the tip of the tongue is not allowed to come near the alveolar ridge. Mugglestone does not really explain why the loss of post-vocalic /r/, which is condemned throughout the nineteenth century, then becomes part of “a set of regionally neutral ‘standard pronunciation features’” \citep[4]{Mugglestone2003} in the late nineteenth century which are linked with values like correctness and educatedness in the metadiscourse. I suggest that it is plausible that the stigmatization of the trilled /r/, which Ellis calls “decidedly un-English” because it “has a Scotch or Irish twang with it” \citep[603]{Ellis1869}, played an important role here. This can be seen in \citegen[244]{Jespersen1922} account of the change involving /r/:

\begin{quote}
There is one change characteristic of many languages in which it seems as if women have played an important part even if they are not solely responsible for it: I refer to the weakening of the old fully trilled tongue-point \emph{r}. I have elsewhere […] tried to show that this weakening, which results in various sounds and sometimes in a complete omission of the sound in some positions, is in the main a consequence of, or at any rate favoured by, a change in social life: the old trilled point sound is natural and justified when life is chiefly carried on out-of-doors, but indoor life prefers, on the whole, less noisy speech habits, and the more refined this domestic life is, the more all kinds of noises and even speech sounds will be toned down. One of the results is that this original \emph{r} sound, the rubadub in the orchestra of language, is no longer allowed to bombard the ears, but is softened down in various ways, as we see chiefly in the great cities and among the educated classes, while the rustic population in many countries keeps up the old sound with much greater conservatism.
\end{quote}

The omission of /r/ is now regarded as a natural consequence of the weakening of /r/ which is associated with refinement and culture. Consequently, there are no traces of an evaluation of /r/-loss as vulgar. Furthermore, it becomes visible that non-rhoticity is linked to femininity. An earlier example which shows that /r/-loss carries connotations of refined speech and not vulgarity is the \citeyear{Smith1866} publication entitled \emph{Mind your H’s and Take Care of your R’s: Exercises for Acquiring the Use \& Correcting the Abuse of the Letter H with Observations and Additional Exercises on the Letter R} written by Charles William Smith, professor of elocution. He writes that

\begin{quote}
[t]he English language, independently of its copiousness, \&c., is in this respect nearer to perfection than any other modern tongue, and next after the Greek; I mean in sound, for our language is not the same in sound, when well spoken, as it appears upon paper. The English language is most expressive if properly spoken. It abounds in words which seem to paint the thing for which they stand. Of its many elements of strength the aspirate and the \emph{r} are the most important, but unfortunately they are most villainously abused by the vulgar, the educated, and the refined. The two former either leave out the aspirate where it should be given, or prefix it where it should not be given—many committing both faults, while the last drop the \emph{r}, or convert it into \emph{w}. One robs his “’am” to enrich his “hegg,” while the other “\emph{w}enders himself \emph{w}emarkably \emph{w}idiculous” by his senseless affectation”. I know not which is the worse, the Wellerism of the East or the Dundrearyism of the West. \citep[5]{Smith1866}
\end{quote}

Even though /r/-dropping is still evaluated negatively as incorrect, it is interesting that its use is explicitly not attributed to ‘vulgar’ speakers (“Wellerism of the East” invoking the Cockney stereotype) but to ‘refined’ ones. With regard to the latter speakers Smith alludes to the figure Lord Dundreary which was invented by the English playwright Tom Taylor for the very successful play \emph{Our American Cousin} (1858). Lord Dundreary, played by the English actor Edward Askew Southern, is a rather dumb and eccentric English nobleman who apparently lisps and realizes the /r/ as a labiodental approximant [ʋ].\footnote{These forms are indicated by pronunciation respellings in the play: e.g. \emph{yeth} ‘yes’ and \emph{wath} ‘was’ for the lisp and \emph{welations} ‘relations’ for the realization of /r/ as [ʋ]. The latter form will be discussed in more detail in \sectref{bkm:Ref11226173}.} This particular realization of /r/ is actually foregrounded which can be seen in the following elaboration on the letter R in a later chapter:

\begin{quote}
After H, the letter R is the worst used letter in the English Alphabet. By most persons, it is very imperfectly articulated, being given softly at the beginning of words, and altogether omitted when occurring at the end; such words as court, form, lord, being pronounced ca\emph{w}t, fa\emph{w}m, la\emph{w}d. Foppish affectation changes its fine, manly sound into that of \emph{W}, even at the beginning of words. This Dundrearyism is “t\emph{w}uly \emph{w}idiculous,” and more worthy a monkey than a man, yet it has become fashionable, not only among a certain number of brainless “swells,” but has been assumed, together with many other effeminate habits and mannerisms, by brave-hearted gentlemen; whose minced speech and foppish manner completely belie the inner man. \citep[29]{Smith1866}
\end{quote}

Here, weakening and loss of /r/ is attributed to “most persons”, which implies that he does not associate the omission with any particular social class. The focus of his criticism is rather on the realization of /r/ as a labiodental approximant which he evaluates as foppish, affected, ridiculous and unmanly. Interestingly, the association of the changes with softness and femininity found in Jespersen’s comment are also present here. In contrast to Jespersen, however, this association is one of Smith’s main arguments against these changes.

To conclude, metadiscourses on /r/ in England are rather complex and seem to shift over time. While non-rhoticity is condemned as incorrect throughout the century and associated with vulgarity, illiteracy and uneducatedness, there is also some evidence for new perceptions and attitudes in the middle of the nineteenth century which associate non-rhoticity and ‘soft’ realizations of /r/ with refined but effeminate and unmanly speech and behavior, especially when it is combined with a snobbish and affected labiodental /r/. I suggest that the ultimate change of attitude which led to the positive evaluation of non-rhoticity as one of the desirable neutral standard forms (very much visible in \citegen{Jespersen1922} comment) was spurred by the negative attitude towards the realization of /r/ as a trill which was regarded as unpleasant, unrefined and un-English. This is a tentative suggestion, however, which could be investigated more systematically with the methodological framework developed for the present study.

Some associations and evaluations found in metadiscourses on /r/ in England can also be found in metadiscourses on /r/ in the United States. The most comprehensive study on metadiscourses on phonological forms in the United States is provided by \citet{Bonfiglio2002}. His focus is on /r/, and he notes, for example, that Webster, like Walker in England, described the presence of non-rhotic forms but did not recommend their use:

\begin{quote}
Some of the southern people, particularly in Virginia, almost omit the sound of \emph{r} as in \emph{ware}, \emph{there}. […] But there seems to be no good reason for omitting the sound altogether; nor can the omission be defended on the ground, either of good practice or of rules. It seems to be a habit contracted by carelessness. (\citealt[110]{Webster1789}, cited in \citealt[38]{Bonfiglio2002})
\end{quote}


Despite the negative view of non-rhoticity expressed in this influential dictionary, Bonfiglio notes that the form has acquired prestige in the northeast and the south by being associated with cultivation, refinement and higher social circles. \citet[41]{Bonfiglio2002} makes the interesting observation that even \citet[230--231]{Krapp19252} still calls the absence of post-vocalic /r/ a feature of cultivated speech in the early twentieth century. With regard to the beginning of the high prestige of non-rhoticity, I have pointed out above that it is often dated to the late eighteenth century. However, \citet{Montgomery2015} finds that this view must be challenged, at least for the south, based on his analysis of confederate textbooks published in the south in the 1860s. In these textbooks, readers are instructed to “[s]ound the R’s–Poo\emph{r}, not poo-ah; matte\emph{r}, not mattuh; mothe\emph{r}, not mothuh; wa\emph{r}m, not wāäm, \&c” (\citealt[13]{Chaudron1863}, cited by \citealt[108]{Montgomery2015}). At the same time, however, southern children are supposed to practice “to enunciate [r] without harshness” (\citealt[1]{Chaudron1863b}, cited by \citealt[108]{Montgomery2015}). This suggests a similar ambivalence found in prescriptive texts in England: The /r/ should not be omitted, but the ‘harsh’ /r/ (usually a uvular trill) is to be avoided as well.


Another value linked to non-rhoticity in England and America is femininity. In America, however, this link becomes part of an evaluative pattern which links non-rhoticity to weakness – physically, as in the case of women, but also culturally, as in the case of southerners, yet more importantly, in the case of African Americans. Bonfiglio regards this “equation of the weakening of /r/ and cultural feminization and degeneration” as “one of the main thematic motifs in the rise of network standard and the discussion of standard American pronunciation in the broadcast media” (\citeyear[46]{Bonfiglio2002}). In contrast to England, where the negative evaluation of the trilled /r/ did not make this variant seem desirable as an alternative to a non-rhotic pronunciation, the retroflex /r/ came to be associated with very positive values in America. According to Bonfiglio, it was linked to virility, strength, vitality and, very importantly, to a purity that was understood in ethnic terms: While urban eastern areas became ‘contaminated’ by immigration and racial heterogeneity, the western rural parts were constructed as ‘uncontaminated’ areas and thus representing ideal American values. Especially in non-rhotic areas, these conflicting evaluative patterns led to an ambivalence on the part of contemporary observers which continued to exist in the twentieth century. \citet[47--48]{Bonfiglio2002} analyzes, for example, a publication by the Harvard educated Charles H. \citet{Grandgent1920b}, entitled “The dog’s letter” and notes that while Grandgent still regards non-rhoticity more highly than rhoticity, he also associates it with softness, weakness and decay.

The extensive research on the historical development of non-rhoticity summarized above shows that it was and continues to be a highly salient form. At the same time, there are still several open questions, not only with regard to structural but also to discursive developments. First and foremost, the question as to how non-rhoticity became constructed as a prestigious ‘standard’ form in England despite being largely condemned in prescriptivist texts has not been answered convincingly based on an analysis of metadiscursive activities. Similarly, the situation in the United States has shown to be quite complex, which is why the present study aims to complement \citegen{Bonfiglio2002} analysis by focusing on metadiscursive activity in nineteenth-century newspapers.

With regard to the choice of search terms, non-rhoticity is the only phonological form for which I have chosen two different search terms and carried out two separate analyses. The first search term is a combination of two forms, \emph{deah} and \emph{fellah}, and the second search term is \emph{bettah}. There are two reasons for searching for articles in which both \emph{deah} and \emph{fellah} occur: The first is that when I searched for \emph{deah} alone, the recognition software also found articles containing \emph{death}. The number of articles resulting from this search was thus very high, and it would have been too time-consuming to manually exclude articles containing \emph{death}. Adding \emph{fellah} to the search made it possible to obtain a collection of articles in which \emph{deah} was accurately identified because \emph{death} was unlikely to occur with \emph{fellah} in the same article but likely to co-occur with \emph{fellah} since I had found this combination in previous analyses. At first sight, the term \emph{fellah} seems like an unusual choice for investigating non-rhoticity as the alternative spelling <fellow> does not indicate a rhotic pronunciation. However, another common spelling variant was <feller>, and this spelling could indicate a hyper-rhotic form. \citet[279--280]{Schneider2004} draws attention to evidence of a rhotic pronunciation of words like \emph{wash} and \emph{Washington} as well as of words ending in final -\emph{ow} (among others) in early southern America English. He finds that rhoticity in these words is “derived from English sources” (which is in line with \citegen{Britton2007} study of hyper-rhoticity discussed above) and “expanded and redistributed in the American South” \citep[280]{Schneider2004}. Even though Schneider also allows for the possibility that writers did not represent hyper-rhoticity but simply misspelled the words because they were non-rhotic speakers, a representation of hyper-rhoticity is at least a possibility. While <fellah> therefore clearly highlights the \emph{absence} of rhoticity, the alternative <feller> creates an opportunity for writers to draw attention to a (hyper-)rhotic pronunciation. In the following analysis and discussion, I will represent the search term as \emph{deah} AND \emph{fellah} to indicate the co-occurrence of the terms in the same article. However, the first results suggested that not only non-rhoticity was indexically linked to specific social values and personae but also the lexical item \emph{fellow} (often occurring as part of the phrase \emph{my dear fellow}, spelled <me deah fellah>). This is why I conducted a second search using the search term \emph{bettah}. In this case, <bettah> clearly indicates a non-rhotic variant in contrast to the rhotic <better>. Like \emph{deah} and \emph{fellah}, \emph{bettah} also represents a case of non-rhoticity occurring in post-vocalic and non-preconsonantal position, so that all of these search terms allow me to focus on the later phase of the development of non-rhoticity.

\subsection{\textsc{twousers}: non-rhoticity and phonetic realization of /r/}
\label{bkm:Ref11226173}\hypertarget{Toc63021227}{}
In contrast to non-rhoticity, the realization of /r/ as a labiodental approximant [ʋ] has not attracted much interest in linguistics. \citegen{Foulkes2000} study on the development of this realizational variant is the most comprehensive one to date. They describe how the variant is described and often dismissed as a speech defect, an infantilism and an upper-class affectation by linguists and other language observers from the nineteenth century onwards. \citet[354--355]{Jespersen19091948} cites Christmas' (\citeyear{Pegge1844}) edition of Pegge’s \emph{Anecdotes of the English language}, in which the editor adds a footnote stating that some people invariably substitute a \emph{w} for the letter \emph{r} because of their “inability to pronounce the letter” and illustrates this disability with the phrase “Awound the wagged wocks the wagged wascals wun their wure-wall wace” (\citeyear[66]{Pegge1844}). Next to this evaluation of [ʋ] as a speech defect, \citet[355]{Jespersen19091948} also notes that “[t]his \emph{w} is found in some novelists as a constant feature of the speech of noble swells” and he gives the examples of \emph{gwandfather}, \emph{thwee}, \emph{scweeching}, \emph{wight}, \emph{cwied} and \emph{Fwank} from Thackeray’s \emph{Pendennis}, which was published in monthly parts between 1848 and 1850. While research on phonetics and language acquisition confirms the higher complexity of the alveolar approximant [ɹ] in comparison to the labiodental [ʋ], which makes it likely that the variant actually occurred and still occurs in children’s speech \citep[55]{Foulkes2000}, it is not so clear to what extent the stereotypical upper-class usage of [ʋ] was a reality in the nineteenth century. \citeapo[282]{Wells1982b} intuition is that at least in the twentieth century perception and reality diverge: “Although this [variant] is often regarded as an upper-class affectation, I am not convinced that it is nowadays found more frequently among upper-class speakers than among those of other social classes”. In stark contrast to the stereotype of the [ʋ]-pronouncing swell, \citet{Foulkes2000} offer the tentative hypothesis that [ʋ] emerged as a feature of non-standard London accents in the late nineteenth and early twentieth century. They argue that the large number of Jews who migrated to London between 1880 and 1890 were responsible for an increasing frequency of variants similar to [ʋ] in London speech. These variants were probably the result of Yiddish speakers attempting to produce [ɹ] and they might have led to the persistence of the “developmental [ʋ]” used by children in the acquisition process. The assumption underlying this argument is that the adult norms become weaker when variability increases, so that the pressure on children to produce [ɹ] instead of [ʋ] decreases in the light of the higher presence of the latter variant in the community. Therefore, the result is a higher frequency of [ʋ] in non-standard south-eastern accents and a more recent spread of the variant to other urban accents of England. \citet[212]{Altendorf2008} summarize this development by stating that


\begin{quote}
there is plentiful evidence of a dramatic rise in frequency of the labiodental approximant [ʋ] in southern England, and indeed in parts of the North. This feature, formerly regarded as an affectation, a speech defect, or an infantilism, is now heard frequently in the accents of a wide range of English cities, and appears generally to be more favoured by young working-class speakers than by middle-class ones.
\end{quote}

However, research on the development of attitudes towards this form is scarcer. \citet{Foulkes2000} note that with the increase of the variant in London it became associated with the Cockney accent and that the “[ʋ] pronunciation is now commonly used to stereotype a Cockney accent” (\citeyear[37]{Foulkes2000}). This represents an enormous shift: A form that had been associated with affected upper-class swells is now perceived as part of a well-known London working-class accent which is increasingly evaluated positively. \citet[37]{Foulkes2000} somewhat vaguely describe a “present atmosphere where young people are on the whole positively oriented towards the non-standard south-eastern variety” and therefore towards [ʋ]. They support the impression of a positive attitude towards this variant by listing some public figures (musicians, sports stars, actors as well as television and radio presenters) that can be heard using it. At the same time, they emphasize that the variant is still used to achieve a comic effect, for example, in a television advert for Pizza Hut \citep[32]{Foulkes2000}. This advert evokes and plays with the older negative association of the form with defective speech because when Jonathan Ross, a British television presenter, greets the American supermodel Caprice in a Pizza Hut store by pronouncing her name [kæpˈʋiːs] she remarks “You can’t pronounce your \textit{r}’s?”, marking his pronunciation as flawed. However, she immediately adds “I love a man who can’t pronounce his \textit{r}’s”. This phrase creates the humor of the advert because the positive evaluation of the variant as attractive and appealing to a beautiful woman runs counter to the negative values of incorrectness and defectiveness invoked by her first remark. This advert is a particularly interesting example because it reveals the ambiguity in the evaluation of the variant in England at the end of the twentieth century. The older negative values were still well-known, but the use of the variant by a popular public figure and his attractiveness to an American model indicate a change of attitude, even though the positive evaluation of the form is still clearly marked as humorous. Whether intentional or not, there is also an additional American dimension included in the advert: Even though the nationality of the main characters (English vs. American) is not foregrounded, it could play a role in reconciling the two conflicting evaluations. Being American, the model Caprice is an outsider to the British English speech community, which might be used as an explanation for why she is presented as finding the form attractive.

In linguistic research on American English, however, the labiodental approximant [ʋ] has hardly played a role, neither on the structural nor on the discursive level. In my first case studies on the linguistic forms presented above, however, I came across many instances of representations of this variant which are indicated in the spelling by the replacement of the grapheme <r> by the grapheme <w>. The prominence of this variant can be illustrated by the following example: In the humorous newspaper article entitled “A Martyr of the Modern Type, but None the Less a Real Hero”, published on April 20, 1895\citesource{April201895} in the \emph{Idaho Statesman}, a young man called Percy Paddleford gives an example of a modern martyr in the following way:

\begin{ipquote}
“Young Hawold Montmowenci, who was a very deah fweiend of mine—the deah boy!—Hawold Montmowenci was one of the gweatest mawtahs this world has ever pwoduced; ’pon my honah, he was! […] He worked for eight dollahs a week, doncher know. But in spite of the extweme pwices chawged for clothes, he chwished the pwaiseworthy ambition of keeping himself dwessed in the vewy latest fashion. [He continues to describe how his friend spent all his money on new clothes to keep up with the changing fashion.] Yaas; but, doncher know, the noble cweature perwished in the owah of his twiumph. He weally perwished of starvation, but he was the best dwessed wemains that my eyes evah wested upon. Don’t twy to tell me about the old mawtahs! Hawold Montmorwenci was the gweatest mawtah in the whole history of wecorded time”.
\end{ipquote}


It therefore seemed worthwhile to investigate the role of this variant in the enregisterment of American English in more detail. As a search term I chose the lexeme \textsc{trousers}, and I included several spelling variants in my search. Next to those variants indicating the realization of /r/ as [ʋ] by replacing the first <r> by a <w>, I also included those which additionally indicated non-rhoticity by replacing <er> by either <aw> or <ah>. In addition, I included variants using the older spelling <ow> instead of <ou>. As I also wanted to find out how many variants occurred which indicated non-rhoticity without the realization of /r/ as [ʋ], I included spellings which only altered <er> but not the first <r>.\footnote{This makes it possible to assess the prominence of [ʋ] in comparison to non-rhoticity with regard to the one specific lexical item. If many spelling variants occurred which just signal non-rhoticity, the prominence of [ʋ] would be rather low; however, if only few of these spelling variants occur, the prominence of [ʋ] will be rather high.}  I therefore used ten search terms: \emph{twousers}, \emph{twowsers}, \emph{twousaws}, \emph{trousaws}, \emph{twowsaws}, \emph{trowsaws}, \emph{twousahs}, \emph{trousahs}, \emph{twowsahs}, and \emph{trowsahs} which are all represented in the following analysis by \textsc{twousers}. An additional advantage of using \textsc{twousers} as a search term is that it is also of interest on a lexical dimension as I will show in \sectref{bkm:Ref528933812}.


\section{Lexical forms}
\label{bkm:Ref523382196}\hypertarget{Toc63021228}{}\label{bkm:Ref523762437}\subsection{\textit{baggage} vs. \textit{luggage}}
\hypertarget{Toc63021229}{}
According to the \emph{Oxford English Dictionary} (\citeyear{luggage}), \emph{luggage} is a noun derived from the verb \emph{lug}, which is probably of Scandinavian origin. An early meaning of \emph{luggage}, which is now obsolete, is “what has to be lugged about; inconveniently heavy baggage” and also “the baggage of an army”. The current use of the word is indicated as being restricted to Great Britain (where it is “the ordinary word”) and its first sense is given by the \emph{OED} as “The baggage belonging to a traveller or passenger, esp. by a public conveyance”. This shows that the word \emph{baggage} is used to define \emph{luggage}, which indicates that the two terms are basically synonymous. Not surprisingly, \emph{baggage} is also defined by using the word \emph{luggage}: “The collection of property in packages that one takes along with him on a journey; portable property; luggage”. Regarding its use, the \emph{OED} states that \emph{baggage} is “[n]ow rarely used in Great Britain for ordinary ‘luggage’ carried in the hand or taken with one by public conveyance; but the regular term in the U.S.” (\citeyear{baggage}). \emph{Baggage} originates from the Old French word \emph{bagage}, meaning “property packed up for carriage” and it is first attested in 1430, whereas the first occurrence of \emph{luggage} is dated to 1596, so more than a hundred years later. It thus seems that two different forms with different origins but a largely overlapping sense have become used in England at different points in time; and while \emph{luggage} has become the majority form in England, \emph{baggage} has come to dominate in the United States.


That \emph{baggage} and \emph{luggage} are two forms which constitute a linguistic difference between Great Britain and the United States today makes them promising forms to investigate in the present study. However, the primary reason for studying them is again not based on the linguistic situation found today but on newspaper articles which suggest that the forms played a role in nineteenth-century discourses on language. An example for such an article, which I found in the collection of articles containing \emph{hinglish}, is the following anecdote taken from the magazine \emph{Harper’s Round Table} and published in the \emph{Daily Inter Ocean} on March 19, 1896\citesource{March191896}
(and with minor modifications in the \textit{Denver Evening Post} on March 24, 1896\citesource{March241896}):

\begin{ipquote}
\begin{center}
\textstyleStrong{A Boy’s Observation in Europe.}
\end{center}

We had a great time when we landed. All our trunks had to be opened by the custom house inspectors to see if we had any cologne or cigars in ’em. I don’t see why they call them custom house officers though. Their costumes weren’t anything wonderful. It took Pop a half an hour to get his trunks all through because he said the inspector didn't know the language. Pop says he asked him what nation he belonged to and the man said he was Hinglish and Pop told him he'd never heard of any such people, where did they live. In Hingland, the man said. Where's that asked Pop, and the man nearly fainted and then Pop gave him a half crown and the man said he guessed he needn't open any more trunks, because a man as ignorant as he was wouldn't have sense enough to try to smuggle anything in anywhere.

After the trunks were all passed Pop asked a man \textbf{where the baggage car was} and \textbf{that man couldn't speak English either}. He asked Pop what, and Pop says again where's the baggage car, and \textbf{just then an American that had been over before says to the man he means the luggage van}, and the man says oh, wy didn't ee si so. Pop says he thinks that's Welsh, which is a language he never liked anyhow.
[...] [emphasis mine]
\end{ipquote}


This anecdote exaggerates the difficulty of communication between Englishmen and Americans in a humorous way, and it shows that the difference between \emph{luggage} and \emph{baggage} (here as part of the compounds \emph{baggage car} and \emph{luggage van}) are salient forms which are used to illustrate the difference leading to a breakdown in communication.


\subsection{\textit{pants} vs. \textit{trousers}}
\label{bkm:Ref528933812}\hypertarget{Toc63021230}{}
\emph{Pants} is described by the \emph{OED} (\citeyear{pants}) as a shortened form of \emph{pantaloons} – \emph{pantaloons} being a word borrowed from French and Italian and first attested in the late sixteenth century in the form \emph{Pantaloun}, designating a figure of the Italian commedia dell’arte (Italian: \emph{Pantalone}). The second sense of \textit{pantaloon} is given as “Trousers, breeches, or drawers” (\citeyear{pantaloon}) and the different sub-entry senses distinguish different styles, most of which are now historical, rare or obsolete. Sense 2 d. is important because it defines the sense that is related to the time period relevant for the present study:


\begin{quote}
Tight trousers fastened with ribbons or buttons below the calf or (later) with straps passing under the boots, which superseded knee breeches and became fashionable amongst men in the late 18th and early 19th cent. Now hist. exc. where retained as part of a livery or military uniform.
\end{quote}


This sense may have been the basis for the sense 2 f., which indicates a regional differentiation between the United States and England: It is defined as “Trousers (used generally, without indicating a particular style)” and said to be chiefly used in the U.S., with its first attestation from 1834. As it was the case for \emph{luggage} and \emph{baggage}, the use of \emph{trousers} to define \emph{pantaloons} shows that their meaning is basically synonymous. Not surprisingly, \emph{trousers} is also used in the definition of \emph{pants}. Sense 1 a. defines it as “Originally (\emph{colloquial}): pantaloons. Later: trousers of any kind (in early use applied to men’s trousers, but in the 20th cent. extended to include those worn by men and women)”. Its first attestation is from 1835 and it is found to have originated in the United States and to be used in this sense chiefly in North America (as well as in New Zealand, Australia and South Africa). By contrast, sense 3 is chiefly British: “(Men’s or women’s) underpants”, which shows that even though \emph{pants} is used in North America and in Britain, the meaning of the form is different in the different regions.


\emph{Trousers}, on the other hand, is described by the \emph{OED} (\citeyear{trousers}) as an extended form of the noun \emph{trouse}, which was most likely derived from the Irish and Scottish Gaelic word \emph{triubhas} (and not, as it has often been assumed, from Old French \emph{trousse}). The meaning of \emph{trousers} today goes back to the seventeenth century. Sense 2 a. specifies this meaning as follows:

\begin{quote}
An outer garment covering the body from the waist to the ankles, with a separate part for each leg; (originally) a loose-fitting cloth garment of this type worn by men, sometimes over close-fitting breeches or drawers; (now more generally) any of various garments of this type worn by either sex (though traditionally more closely associated with men)
\end{quote}


The first attestation of \emph{trousers} with this meaning is from 1681, but the \emph{OED} notes that its rise in frequency happened more than a hundred years later, from the beginning of the nineteenth century. In this process, the meaning of \emph{trousers} became more clearly distinct from the meaning of \emph{breeches}: Trousers covered the whole leg while breeches only came just below the knee, and trousers were not as tightly fit as breeches.


Using \emph{pants} and \emph{trousers} as lexical forms in the present study is promising for several reasons: First of all, it provides a good case for studying the interaction between different kinds of perceivable signs, because changes in the meaning of the terms were also linked to changes in the items of clothing that they designated. This can be illustrated by the quote illustrating sense 2 d. of \emph{pantaloons}, which is from the \emph{Chambers’s Information for People}, published in 1857: “Pantaloons, which fitted close to the leg, remained in very common use by those persons who had adopted them till about the year 1814, when the wearing of trousers, already introduced into the army, became fashionable”. This indicates that changes in fashion, described here as mainly a change from a tight to a loose fit, were related to a change in the use of terms, at least in Great Britain. Yet, in North America the development seems to have been different, with \emph{pantaloons} remaining the lexical form being applied to the garment covering the legs and extending its meaning to all styles of fashion – the main change was the clipping of the form to \emph{pants}, which was apparently first colloquial but then became the general form. The second reason for studying this form is therefore that it provides interesting insights into which role this differentiation on the lexical level used to play in the discourses on language, that is, in the construction of a discursive variety. The third reason for choosing the form is that the analysis of articles containing \emph{baggage} and \emph{luggage} has already revealed that the difference between \emph{pants} and \emph{trousers} is subject to explicit and implicit metapragmatic activities. For example, several newspaper articles listed important differences between the use of lexical items in England and in America, and they often included \emph{trousers} and \emph{pants}. An example is the article entitled “A Few Verbal Errors” published on February 9, 1882\citesource{February91882} in the \emph{Glendive Times} (Montana). It references a prescriptivist work by Alfred Ayres, \emph{The Verbalist}, and gives examples of “the more common errors in the use of words”, among them “Pants, for pantaloons, or (better still) trousers”. This indicates that, by 1882, \emph{pants} was used in the United States, but that it attracted criticism by prescriptivists, which then circulated in newspaper discourse. A closer analysis of metadiscursive activity surrounding these lexical items thus seems fruitful to investigate if and how \emph{pants}, which is the common form for the item of clothing covering the legs in the United States today, became enregistered as an ‘American’ form in the nineteenth century.

To conclude this discussion of previous research on the linguistic forms chosen for the present study, it becomes clear that their historical development deserves further attention, not only on the structural but also on the discursive level. This study focuses on the discursive level by investigating their role in the enregisterment of American English in the nineteenth century. Before presenting the results of the analysis in Chapter \ref{bkm:Ref523404731}, I will specify the precise research questions guiding the analysis in the following section.

\section{Research questions}
\label{bkm:Ref523382450}\hypertarget{Toc63021231}{}\label{bkm:Ref523762184}
The aim of this study is to investigate enregisterment processes of American English, or, in other words, a register that is indexically linked to the value ‘American’ in the United States in the nineteenth century. Based on the methodological considerations described in this chapter, the analysis will be focused on the following research questions:


\begin{enumerate}
\item
In how many newspaper articles of the two databases (AHN and NCNP) do the selected search terms occur? How are the articles containing the search terms distributed regionally (per state) and temporally (per decade)? Which inferences can be drawn from this about the salience of the linguistic forms in metadiscursive activity, that is, how salient were they, when did they become salient, and in which regions did they become salient?
\item
Which social values and social personae (characterological figures) are indexically linked to the linguistic forms? Which other linguistic forms co-occur with them?
\item
How are the indexical links created? Which strategies are employed and to what extent do they differ depending on the linguistic form? What are the effects of the different strategies and how do they contribute to the salience of the form–meaning links?
\item
How often do the indexical links occur? To what extent do they change over time? To what extent does the creation of the indexical links follow specific patterns? Are these patterns stable or are they changing?
\end{enumerate}

The research questions subsumed under 1) will be answered by means of quantitative analyses because they aim at discovering the frequency of occurrence of articles containing the search terms. The findings obtained in this part provide an important basis for the subsequent analyses because they reveal when, where and to what extent the phonological forms were part of metadiscursive activity in the first place. It is of course possible and even very likely that the linguistic forms have been subject to discussion in other articles which do not use pronunciation respellings (at least not those selected for the analysis here) or which discuss one of the lexical items without using the other variant near to it, but, as I have already pointed out, using specific search terms makes it possible to create a restricted set of texts and to investigate the articles of this set systematically to identify patterns and trace general developments. In the case of the lexical forms, I will investigate more specifically how often the two variants occur within ten words of each other and identify the frequency of occurrence of articles in which the variants are implicitly or explicitly marked as different (be it in terms of semantic meaning or in terms of indexical values associated with the variants). It is these cases which are particularly interesting because, as in the case of pronunciation respellings, attention is drawn to the difference between linguistic forms, which creates the potential for discovering social differences which are linked to these linguistic differences. The general aim of this first part is to assess the salience of the linguistic forms in metadiscursive activity in quantitative terms because this offers an insight into the questions of when the forms have been enregistered, where this enregisterment took place and the extent to which the form became salient register shibboleths.

The questions subsumed under 2) and 3) will be answered by means of qualitative analyses which focus mostly on the intratextual layer. I will analyze specific examples to show the different ways in which indexical links are created in the articles. This includes the crucial question of how the value ‘American’ is constructed (Who is American? What characteristics does ‘being American’ entail?). Furthermore, a central part of the analysis is to identify other discourses and ideologies that are drawn on to establish the indexical links between linguistic forms and social values. In this part, the salience of the form–meaning links is assessed qualitatively by showing how different strategies in the creation of form–meaning links have different effects on the salience of these links. In addition, the analysis also aims at identifying other linguistic forms which are linked to the same social values and characterological figures in order to get a fuller picture of the linguistic repertoire of the registers. Finally, the questions subsumed under 4) focus on the transtextual layer. By adding quantitative elements to the prior qualitative analyses it is possible to identify larger patterns in the creation of links and their role in the enregisterment of American English. This is crucial because, as pointed out in \sectref{bkm:Ref512260235}, enregisterment occurs when the typifications occur repeatedly and form-value links become a social regularity. Moreover, it will be possible to trace developments in the social evaluation of the forms under investigation to discover how stable the form-value links were. The following chapter contains the results of the analysis. It is divided into two parts, the first focusing on the phonological forms and the second on the lexical forms. The analysis will be followed by an interpretation of what the results reveal about historical enregisterment processes of American English in Chapter \ref{bkm:Ref532290015}.

