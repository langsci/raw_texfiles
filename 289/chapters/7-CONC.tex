\chapter{Summary and outlook}

Research into the structure of bilingual speech has traditionally focused on the distribution of elements of language A in the discourse framed by language B. Little attention has been drawn to the inserted elements’ distributional properties reflecting their usage in language A, i.e., in the monolingual mode. This book has shown that distributional properties such as an element's overall occurrence frequency, its co-occurrences with other elements as well as the competition among these co-occurrences decisively influence the structure of bilingual speech. Multimorphemic forms and multiword sequences which are frequent in language A tend to appear in bilingual speech preserving their integrity. A usage-based explanation of this effect refers to the fact that frequently used multimorphemic forms and multiword strings leave strong memory traces in the language user's mind. During language comprehension and production, the mental representations corresponding to frequent multimorphemic and multiword items are activated faster than the representations underlying their component parts. Hence, holistic storage of particular linguistic structures leads to the production thereof en bloc. The analyses in this book present evidence that pieces of language regularly used together in the embedded language, which are also referred to as chunks, appear to repel switches, and conversely, elements with no, or few, frequent companions in the embedded language easily combine with the material of the matrix language in bilingual speech. Furthermore, I have demonstrated that prediction of mixing patterns, or of the choices bilingual speakers make, usually involves other factors. Apart from distributional factors, other factors draw on overlaps and mismatches between the structural patterns of the languages involved. Because under a usage-based view, categorisation, which is responsible for the emergence of abstract linguistic structure, is grounded in the process of similarity identification, not only effects of usage frequency, but also effects of similarity may be usefully interpreted within one framework. Thus, even similarities and differences between equivalent structural patterns of the bilingual speaker's languages are interpreted in terms of the speaker's linguistic experience: bilingual speakers compare, whether consciously or unconsciously, familiar structures and elements of one language with equivalent forms available in the other language. Comparisons resulting in establishing similarity in some aspect of linguistic structure open up possibilities for mixing and transfer and thus lay the ground for linguistic innovations \citet[cf.][]{sebba-09}. However, it is the distributional properties of the inserted material that make the emergence of specific mixing patterns probable.

\begin{sloppypar}
The studies presented in this book investigated language mixing in three frequently reported loci: the adjective-modified noun phrase, the prepositional phrase and plural marking on inserted nouns. Each of these contexts involves the occurrence of embedded-language islands, i.e., multimorphemic or multiword strings of the embedded language in the discourse framed by the matrix language. While co-occurrence frequency, the basis of chunking, was found to facilitate the use of embedded-language islands in each of the examined structural contexts, the frequency of the words constituting the islands also determined the likelihood of their appearance in bilingual sentences, namely, the frequency of the noun in the prepositional phrase and the frequency of the adjective in the adjective-modified noun phrase. 
\end{sloppypar}

In adjective-modified noun phrases involved in code-mixing, the frequency of the adjective was observed to negatively correlate with the tendency to produce a German embedded-language island. Whilst frequent adjectives, which express very general meanings, usually came from the more activated, matrix language, i.e. Russian, low-frequency adjectives were predominantly German. This situation was interpreted in terms of (i) \citeauthor{backus-2001}'s specificity continuum, according to which embedded-language lexical items with specific meanings are frequently involved in code-mixing, and (ii) the strong syntactic projection of inflected German adjectives \citep{auer_projection_2005,auer_syntax_2007}, which are followed exclusively by German nouns in the examined data. Hence, the embedded-language islands structured as adjective-modified noun phrases corresponded either to chunks, or were triggered by German low-frequency adjectives, distinguished by specific meanings. 

In the prepositional phrase, the bias to produce embedded-language islands was observed to positively correlate not only with the probabilistic factor odds, based on the frequency of co-occurrence between the preposition and the noun, but also the frequency of the noun. I interpreted the effect of the noun frequency as evidence for the tendency of high-frequency nouns to trigger their preposition companions, which are part of highly recurrent multiword sequences. Prior discursive context appeared to exert the strongest influence on the choice between German and Russian prepositions with German noun insertions. That is, if a preposition occurred in the immediately preceding discourse, this preposition was likely to appear in the examined prepositional phrase in the same language again. This priming effect was interpreted as evidence for the high accessibility of function words to bilingual speakers after their previous retrieval a moment ago during the same interaction.
As in the previous case studies, plural marking on German noun insertions in Russian sentences exhibited two patterns: the speakers either produced German plurals, i.e., internal embedded-language islands, or combined German stems with Russian inflectional suffixes, fusing number and case. The choice between a mixed constituent and an embedded-language island was found to result from the combination of the following factors: the plural-singular distribution ratio, the phonemic shape of the inserted stem and the morphological case of the slot into which the noun is inserted. According to the minimal adequate regression model fit to the data, the most important predictor of this variation was the plural-singular ratio: plural-dominant nouns, i.e. nouns that are more frequent as plurals than as singulars, tended to retain their German plural marking in mixed sentences, whereas singular-dominant nouns received plural marking from Russian. The stem's phonemic shape was the factor of second importance: nouns with vowels in the stem-final position retained their German plural marking more often than nouns with consonants in the stem-final position. This result was regarded as a morphophonological restriction that the matrix language exerts on inserted noun stems. Finally, the non-equivalence of the German and Russian nominal systems, with varying degrees of syncretism, affected the speakers' choice of variants: German nouns in slots projecting the genitive, the dative, the instrumental and the prepositional case received Russian inflectional suffixes more frequently than German nouns in slots requiring the nominative or the accusative. In effect, bilingual speakers could insert German plurals in slots projecting the core cases more amply than in slots requiring the non-core cases. I interpreted this finding to result from two aspects, namely, (i) a mismatch between case distinctions in the plural forms -- while every non-core case in Russian has a unique plural formative, German plural paradigms virtually neutralise case distinctions -- and (ii) a similarity in form syncretism between German plural paradigms and Russian plural nouns in the core cases, distinguished by nominative-accusative syncretism. Hence, this similarity is of the relational type \citep{gentner-markmann1997}.

When we compare the various predictors identified through the application of statistical modelling, co-occurrence frequency was found to exert an effect on the variation in every case study. It was the most important predictor in the variation in patterns of plural marking on German noun insertions. In this case study as well as in the analysis of switch placement in the prepositional phrase, co-occurrence frequency was modelled in relational terms considering the other items entering the co-occurrence distributions. While in the study of plural-marking, the competition was only between two forms, the plural and the singular (which is equivalent to the base), the competition between the prepositions accompanying a specific noun involved more than ten items. Unlike these studies, the analysis of mixing in the adjective-modified noun phrase dispensed with a similar relational measure because multiple nouns from the data set are used with a virtually unlimited number of adjectives. A comparison of the impacts of co-occurrence frequency on the variation in mixing patterns in the various contexts yields a conclusion that the fewer competitors in a distribution, the more robust the effect of co-occurrence frequency.

As concerns the inter-speaker variation in the utilised data, it was found to be negligible in every case study. This result lent support to my assumption that the speakers represented in the corpus form a rather homogeneous group and their personal preferences in code-mixing did not significantly influence the distribution of the patterns under scrutiny. 

Essentially, a usage-based approach to the analysis of code-mixing, as proposed in this book, proves not only empirically robust but extremely fruitful in providing psychologically plausible explanations for variation in code-mixing patterns, including embedded-language islands. The reported results provide tangible evidence that patterns in code-mixing depend on (i) gradient facts of usage, (ii) similarities between the involved languages’ structural patterns, and (iii) the immediate linguistic and discursive context. Additionally, my findings can be considered as corroborative evidence of the usage-based claim that language users represent multiword and multimorphemic amalgams and heavily rely on them in language production.

Although the results are encouraging, a more accurate and robust comparison of bilingual and monolingual data would be possible when large corpora of spoken language are available. In view of the applied methodology, in future it would be interesting to model recency of linguistic structures in discourse as a gradient rather than a discrete factor. An important question for future studies is to determine a more precise measure of chunk competition. For example, instead of odds it might be worth employing more complex information theoretical measures such as entropy. 

In view of the findings reported in this book, future work in contact linguistics should focus on the followings directions: First, it will be necessary to test the relevance of the explanations proposed for the emergence of embedded language islands to the matrix language surface structure. Most interesting would be an in-depth analysis of the slots which accommodate embedded-language material and the sequences preceding these slots in terms of their interruptibility and cohesion. Particularly exciting would be a study into the relationship between code-mixing, on the one hand, and syntactic constructions and lexical cohesion, on the other hand. It would also be beneficial to apply the approach laid out in the present work to analyses of alternational code-mixing and to examine whether the results are valid for situations of congruent lexicalisation, including the case of dialect-standard mixing. Recent work by \citet{goria-units} provides encouraging evidence for the role of lexically specific constructions and lexical chunks in alternational mixing as well. What seems to be even more promising is to extend the findings of this research to the languages in contact that belong to different typological types because a central issue in the future research agenda will be to investigate the structure of bilingual speech as resulting from (i) perceived and constructed similarity, (ii) patterns of usage and (iii) processing biases \citep[see the papers in][]{hakimov-backus-20}. One direction could explore the relationship between similarities in the morphological systems of the contact languages and the outcomes of language contact. In this regard, the issue of formal and relational similarity could be brought into a sharper focus.
