\documentclass[output=paper,colorlinks,citecolor=brown]{langscibook}
\ChapterDOI{10.5281/zenodo.17158194}
\author{Beáta Gyuris\orcid{0000-0001-8804-7409}\affiliation{HUN-REN Hungarian Research Centre for Linguistics, Budapest and ELTE Eötvös Loránd University, Budapest}}
\title[Marking the type of speaker bias]{Marking the type of speaker bias: Hungarian \textit{nem-e} interrogatives}
\abstract{This paper investigates the use conditions of a noncanonical polar interrogative form type in Hungarian, which contains the (surface) constituent \textit{nem-e} (consisting of a negative and an interrogative particle), and compares them to those of the two canonical negative polar interrogatives. It is shown for the first time that \textit{nem-e} in fact appears in three different construals, which are connected to at least two different dialects. Focusing on \textit{nem-e} interrogatives used by speakers of the Standard Dialect, we point out that they lack non-epistemic speaker expectation bias, cannot be used to encode indirect reproaches, offers or requests, they do not realize initiating moves in discourse, they do not give rise to rhetorical question readings, and they cannot felicitously be responded to by isolated response particles. These properties are accounted for on the basis of the assumption that the focus-background structure of the form involves a focused proposition, which leads to certain restrictions regarding the structure of discourses it can appear in.}

\IfFileExists{../localcommands.tex}{%hack to check whether this is being compiled as part of a collection or standalone
	% add all extra packages you need to load to this file

\usepackage{tabularx,multicol}
\usepackage{url}
\urlstyle{same}

\usepackage{listings}
\lstset{basicstyle=\ttfamily,tabsize=2,breaklines=true}

\usepackage{langsci-basic}
\usepackage{langsci-optional}
\usepackage{langsci-lgr}
\usepackage{langsci-osl}
% \usepackage{./langsci/styles/langsci-lgr}
% \usepackage{./langsci/styles/langsci-osl}
% \usepackage{langsci-gb4e}

\usepackage{tikz}
\usetikzlibrary{patterns,calc}
\pgfdeclarepatternformonly{south east lines}{\pgfqpoint{-0pt}{-0pt}}{\pgfqpoint{3pt}{3pt}}{\pgfqpoint{3pt}{3pt}}{
    \pgfsetlinewidth{0.6pt}
    \pgfpathmoveto{\pgfqpoint{0pt}{3pt}}
    \pgfpathlineto{\pgfqpoint{3pt}{0pt}}
    \pgfpathmoveto{\pgfqpoint{.2pt}{-.2pt}}
    \pgfpathlineto{\pgfqpoint{-.2pt}{.2pt}}
    \pgfpathmoveto{\pgfqpoint{3.2pt}{2.8pt}}
    \pgfpathlineto{\pgfqpoint{2.8pt}{3.2pt}}
    \pgfusepath{stroke}}
    
\usepackage{stmaryrd}
\usepackage{wasysym}
\usepackage{multirow}
\usepackage{caption}
\usepackage{subcaption}
\usepackage{mathrsfs}
\usepackage{qtree}

\usepackage{linguex}


	%pminos do not split footnotes
% \interfootnotelinepenalty=10000 %Footnote in Laporte chapters has to be split SN


%\DeclareIndexNameFormat{default}{%
%\nameparts{#1}%
%\usebibmacro{index:name}%
%{\index[names]}%
%{\namepartfamily}%
%{\namepartgiveni}%
% {}% L1
% {}% L2
%{\namepartprefix}% generates spurious space L3
%{\namepartsuffix}% generates spurious space L4
%}

%  {\DeclareIndexNameFormat{default}{%
%     \usebibmacro{index:name}{\index[names]}{#1}{#3}{#5}{#7}}}

%\DeclareIndexNameFormat{default}{%
%  \usebibmacro{index:name}{\sindex[nom]}{#1}{#3}{#5}{#7}}

%\DeclareIndexNameFormat{default}{%
%  \usebibmacro{index:name}{\sindex[person]}{#1}{#3}{#5}{#7}}
%\DeclareIndexNameFormat{default}{%
%\nameparts{#1} \usebibmacro{index:name}{\sindex[person]]}{\namepartfamily}{‌​\namepartgiven}{\nam‌​epartprefix}{\namepa‌​rtsuffix}}

%\newcommand{\smiley}{:)}

%\renewbibmacro*{index:name}[5]{%
%\usebibmacro{index:entry}{#1}%
%{\iffieldundef{usera}{}{\thefield{usera}\actualoperator}\mkbibindexname{#2}{#3}{#4}{#5}}}

% \newcommand{\noop}[1]{}

%remove for final
%\overfullrule=1mm

\newcommand{\tobi}[2]}}
\renewcommand{\S}[1]{\tobi{#1}{\textsc{*}}}

% this volume references
% puts: [this volume]
% already defined: \citetv
%\newcommand{\citepv}[1]{(\citeauthor{#1} \citeyear*{#1} [this volume])}
\newcommand{\citealtv}[1]{\citeauthor{#1} \citeyear*{#1} [this volume]}

%parentheses around example number
\newcommand{\pref}[1]{(\ref{#1})}

% in-text examples

\newcommand{\lnex}[1]{\textit{#1}} %target lang word
\newcommand{\lnlit}[1]{(lit.: `#1')} %literal reading
\newcommand{\lnlat}[1]{(#1)} % latinization
\newcommand{\lntrans}[1]{`#1'} %translation
\newcommand{\lnexl}[2]%
{\lnex{#1}{} \lnlat{#2}} % ex with latinization
\newcommand{\lnexlat}[3]{\lnex{#1}{} \lnlat{#2}{} \lntrans{#3}} % ex with latinization and tranl.

%ch01
\newcommand{\co}[1]{\mbox{\textbf{#1}}}

%ch09

\newcommand{\cyrbulg}[1]{\begin{otherlanguage*}{bulgarian}#1\end{otherlanguage*}}


%ch10
\newcommand{\nlp}{{\small NLP}}
\newcommand{\mwe}{{\small MWE}}
\newcommand{\rae}{{\small RAE}}
\newcommand{\lvc}{{\small LVC}}
\newcommand{\pos}{{\small P}o{\small S}}
%\newcommand{\todo}[1]{ \textcolor{red}{#1} }

%\renewcommand{\labelenumi}{\theenumi}
%\ainamefmt{{vv}{ll}{, ff}{, jj}} % fullname

\newcommand{\biberror}[1]{{\color{red}#1}}

\newcommand{\osenovaitem}{--~}
	%% hyphenation points for line breaks
%% Normally, automatic hyphenation in LaTeX is very good
%% If a word is mis-hyphenated, add it to this file
%%
%% add information to TeX file before \begin{document} with:
%% %% hyphenation points for line breaks
%% Normally, automatic hyphenation in LaTeX is very good
%% If a word is mis-hyphenated, add it to this file
%%
%% add information to TeX file before \begin{document} with:
%% %% hyphenation points for line breaks
%% Normally, automatic hyphenation in LaTeX is very good
%% If a word is mis-hyphenated, add it to this file
%%
%% add information to TeX file before \begin{document} with:
%% \include{localhyphenation}
\hyphenation{
    Beck-man
    Ngu-yen
    back-chan-nel
    back-chan-nels
    mo-not-o-nous
    ste-reo-typ-i-cal
}

\hyphenation{
    Beck-man
    Ngu-yen
    back-chan-nel
    back-chan-nels
    mo-not-o-nous
    ste-reo-typ-i-cal
}

\hyphenation{
    Beck-man
    Ngu-yen
    back-chan-nel
    back-chan-nels
    mo-not-o-nous
    ste-reo-typ-i-cal
}

	\bibliography{localbibliography}
	\togglepaper[23]
}{}

\begin{document}
	\maketitle

\section{Introduction}\label{sect:intro}
	
	The aim of this paper is to review the use conditions, particularly the 
	``bias profiles'' of different form types of negative polar interrogatives in Hungarian in general, and then focus on a non-canonical form type that contains the (surface) constituent \textit{nem-e}, consisting  of the negative particle \textit{nem} and the interrogative particle \textit{-e}, illustrated in \xref{ex:tortent}. (Until we present our account of the interpretation of interrogatives with \textit{nem-e} in \sectref{sect:account}, they  will be translated into English in terms of  negative polar interrogatives with ``high negation'', as other negative interrogative forms in Hungarian normally are, cf.~\citealt{Gyuris2017}, a.o.. In \sectref{sect:account} we will argue for a more appropriate translation.) 
	
	\ea\label{ex:tortent}
	A and B are wondering why their friends haven't arrived in time for a meeting.\\
	A says:
	\gll Nem-e történt valami az úton?\\
	not-\textsc{q} happened something the way.on\\ 
	\glt `Didn't something happen on the way?'\footnote{\textsc{q} stands for `interrogative particle'. }
	\z
	
	
	We rely on a distinction made in the literature between two dimensions of bias in polar questions (cf. \citealt{sudo2013}). The first one, usually referred to as \textit{evidential bias}, indicates sensitivity to evidence in the context for the positive  or the negative answer ($p$ vs.~$\neg p$) (Cf.~\citealt{ladd81, buring+gunlogson, roelofsen-etal2013}). The second one, which is going to be referred to here as \textit{(speaker) expectation bias} (following \citealt{silk2020}), indicates sensitivity to the speaker's previous expectations regarding the answer. These expectations can stem from the speaker's beliefs, wishes or some set of rules, and are thus referred to in the literature as \textit{epistemic}, \textit{bouletic} or \textit{deontic bias}, respectively.\footnote{Note that \citet{sudo2013} and \citet{gaertner+gyuris17, Gartner2023} refer to all types of (speaker) expectation bias as \textit{epistemic bias} for brevity.} (For relevant discussion, cf.~\citealt{romero+han, Reese2007, reese+asher2009, domaneschietal, silk2020}, a.o.)
	
	Interrogatives containing the constituent \textit{nem-e} complement the inventory of negative polar interrogative forms found in the Standard Dialect, to be introduced below. In traditional descriptive grammars and style guides they have been referred to as a ``substandard'' (cf. \citealt{szasz1905, tompa1961-62, gretsy-kovalovszky}), in modern descriptive grammars as a ``non-standard'' form type of negative interrogatives (cf.~\citealt[2]{kenesei-etal}).\footnote{In spite of the stigmatization of the form by descriptive linguists and language educators, a sociolinguistic survey reported on by \citet{kassai1994} has found that  36,7\%  of 832 participants considered an interrogative with \textit{nem-e} acceptable, and 45,9\% of 812 participants did not correct the form in a text where they were asked to correct what they consider ``mistakes'' . These data indicate how widely the form is used and accepted among speakers.}
	No semantic or pragmatic distinctions between interrogatives with \textit{nem\nobreakdash-e} and the other negative interrogative form types have been mentioned so far in the literature. 

	
	In this first systematic study of the interpretation of interrogatives with  \textit{nem-e}, we argue that they appear in at least two dialects in Hungarian, in different structural environments and with different use conditions.  In the first dialect, they are used \textit{in lieu} of a  standard negative interrogative form.

	In the second dialect, they appear in addition to the canonical negative polar interrogative forms, but are used for a special effect. In the latter dialect, \textit{nem-e}  turns out to be sensitive to the type of the (speaker) expectation bias (epistemic vs.~deontic/bouletic). The central question to be addressed in this paper is how to account for the interpretational features of \textit{nem-e} in the second dialect, with particular attention to its bias profile.  

	
	The paper is structured as follows. First, in \sectref{sect:forms}, we review  previous claims about the bias profiles of the (two) canonical positive and negative polar interrogative form types in Hungarian. \sectref{sect:dialects} presents a set of examples with \textit{nem-e}, and sorts them into two dialects. \sectref{sect:nem-e-dialect-e} focuses on \textit{nem-e} interrogatives in one of these  dialects, investigating the type of negation they encode, the type of the speaker expectation bias they introduce, and further conditions on the use of the form in context. \sectref{sect:account} makes a proposal for an account of the interpretation of \textit{nem-e} interrogatives in the latter dialect, which explains the properties discussed above. The paper ends with the conclusions in \sectref{sect:conclusion}.
	
	
\section{Polar interrogatives in Hungarian: Forms and biases}\label{sect:forms}
	
	This section presents the interrogative forms that can appear in matrix clauses in Hungarian. (For more detailed overviews, cf. \citealt{Gyuris2017, gyuris18}.)
	
	\xref{ex:elutazott-e} represents the form type referred to as \nobreakdash-\textit{e}-\textit{interrogative}, marked by the \nobreakdash-\textit{e} interrogative particle, which cliticizes onto the finite verb. It is pronounced with an end-falling intonation contour. \xref{ex:elutazott?} is a so-called \textit{rise-fall} ($\bigwedge$)-\textit{interrogative}, which is marked by prosodic means, with a global rise-fall tune (L*HL\%, cf. \citealt{ladd96}), peaking on the penultimate syllable.\footnote{For a detailed discussion of Hungarian intonation, including that of polar interrogatives, cf. \citet{varga2002}.}
	
	\ea\label{ex:elutazott-e}
	\gll J\'anos ki-utazott-e Berlinbe?\\
	J\'anos \textsc{vm}-travelled-\textsc{q} Berlin.into\\
	\glt`Did János go to Berlin?'\footnote{\textsc{vm} stands for `verb modifier'. The category of verb modifiers includes verbal prefixes (e.g.~\textit{ki}), bare nominal complements, oblique complements expressing a goal, and non-agentive subjects, cf. \citet[57]{ekiss2002}. In non-negative sentences, these constituents are situated immediately in front of the verb in the absence of a constituent in preverbal focus, but stay behind the verb in case the latter is preceded by the negative particle (\textit{nem}) or a constituent in focus position (to be illustrated below). The verb--\textsc{vm} order will be referred to here as ``inversion''. According to standard Hungarian orthography, a verb is written together as one word with the verbal prefix preceding it. To make the verbal prefix more visible, we will in most cases use a hyphen to connect it with the following verb.}
	\z
	
	\ea\label{ex:elutazott?} 
	\gll J\'anos ki-utazott {Berlinbe} {$\bigwedge$ ?}\\
	J\'anos \textsc{vm}-travelled Berlin.into (\textsc{q})\\
	\glt `Did János go to Berlin?' 
	\z
	
	
	As a comparison between \xref{ex:elutazott?} and \xref{ex:elutazott} illustrates, $\bigwedge$-interrogatives are string-identical to the corresponding  declaratives, which are pronounced with an end-falling tune as a default:
	
	\ea\label{ex:elutazott} 
	\gll J\'anos ki-utazott Berlinbe.\\
	J\'anos \textsc{vm}-travelled Berlin.into\\
	\glt `János went to Berlin.'
	\z
	
	In subordinate clauses, the only interrogative form available is the \nobreakdash-\textit{e} interrogative, as in \xref{ex:tudja-elutazott-e}.\footnote{Note that \xref{ex:tudja-elutazott} can only be analysed as containing an embedded declarative:
		\ea\label{ex:tudja-elutazott}
		\gll Mari tudja, hogy János ki-utazott Berlinbe.\\
		Mari knows that János \textsc{vm}-travelled Berlin.into\\
		\glt `Mari knows that János went to Berlin.'
		\z
	}
	
	\ea\label{ex:tudja-elutazott-e}
	\gll Mari tudja, hogy János ki-utazott-e Berlinbe.\\
	Mari knows that János \textsc{vm}-travelled-\textsc{q} Berlin.into\\
	\glt `Mari knows whether János went to Berlin.'
	\z
	
	
	It is argued in \citet{Gyuris2017} that whereas \nobreakdash-\textit{e}-interrogatives are equally infelicitous in the presence of \textit{compelling contextual evidence} (cf.~ \citealt{buring+gunlogson})  for the positive or the negative answer, $\bigwedge$-interrogatives can be compatible with the presence of compelling contextual evidence  for  the former.\footnote{In case the speaker believes that the contextual evidence is only compatible with the positive answer, the  $\bigwedge$-interrogative form is blocked by a declarative pronounced with multiple rise-fall tunes (cf.~\citealt{gyuris19, varga2010}), the counterpart of English `rising declaratives' (cf.~\citealt{gunlogson2003}).} 
	\xref{ex:sunglasses} shows the use of the two forms in a context with compelling contextual evidence for the positive answer, and \xref{ex:longdistance} illustrates their uses in a neutral context (i.e., one with no compelling evidence for any of the answers).
	

	
	\ea\label{ex:sunglasses} A enters the building in sunglasses and t-shirt. S, who has been sitting in a windowless office during the last couple of hours, wants to know what the weather is like outside. S asks A:
	\ea[\#]{
		\gll Jó idő van-e?\\
		good weather is-\textsc{q}\\
		\glt `Is the weather nice?'
	}
	\ex[]{\gll Jó idő van $\bigwedge$ ?\\
    good weather is (\textsc{q}) \\
    \glt `Is the weather nice?'
	}
	\z
	\z
	
	
	\ea\label{ex:longdistance} A and S talk long-distance on the phone. S wants to know what the weather is like at A's place. S asks A:
	\ea
	\gll Jó idő van-e?\\
	good weather is-\textsc{q}\\
	\glt `Is the weather nice?'	
	\ex  \gll Jó idő van $\bigwedge$ ?\\
    good weather is (\textsc{q}) \\
	\glt `Is the weather nice?'
	\z
	\z
	
	
As noted in \citet{Gyuris2017}, and confirmed experimentally in \citet{gyuris+molnar+mady2020, gyuris+molnar+mady2021}, speakers from different regions judge  the appropriateness of -\textit{e}-{in\-ter\-ro\-ga\-tives} for encoding informal information-seeking questions differently. As opposed to speakers from certain regions in Eastern Hungary and in Transylvania (Romania), speakers from Western Hungary and Budapest tend to accept them only in official, formal contexts (e.g., court or police interrogations), where they are intentionally used to indicate the impartiality of the questioner (cf. \citealt{varga2021}). Nevertheless,  speakers of all dialects give  for \nobreakdash-\textit{e}-interrogatives significantly higher acceptability ratings  in neutral contexts than in  contexts with evidence for the positive answer. 
	
	We turn now to the corresponding negative form types. The negative  counterparts of \xref{ex:elutazott-e} and \xref{ex:elutazott?} are illustrated in \xref{ex:nem-utazott-e} and \xref{ex:nem-utazott?}, respectively. The corresponding negative declarative, which  is string-identical to \xref{ex:nem-utazott?}, is shown in \xref{ex:nem-utazott}.
	
	\ea\label{ex:nem-utazott-e}
	\gll Nem utazott-e	ki	János Berlinbe?\\
	not	travelled-\textsc{q} \textsc{vm} János Berlin.into\\
	\glt `Didn't János go to Berlin?'
	\z
	
	\ea\label{ex:nem-utazott?} 	
	\gll Nem utazott ki János {Berlinbe} {$\bigwedge$?}\\
	not	travelled	\textsc{vm} János 	Berlin.into (\textsc{q})\\ 
	\glt `Didn't János go to Berlin?'
	\z
	
	\ea\label{ex:nem-utazott} 	
	\gll Nem utazott ki János Berlinbe.\\
	not	travelled	\textsc{vm} János 	Berlin.into \\ 
	\glt `János didn't go to Berlin.'
	\z
	
	All of \xxref{ex:nem-utazott-e}{ex:nem-utazott} display inversion between the \textsc{vm} and the verb, due to the fact that the negative particle attracts the verb to NegP, cf. \citet{ekiss2009}. The syntactic structure of \xref{ex:nem-utazott} is shown in \xref{ex:neg-syntax} (Cf.~\citealt{suranyi2009}.):
	
	\ea\label{ex:neg-syntax}   
	[\textsubscript{NegP} Nem [ utazott$_i$ [\textsubscript{IP} ki \textit{t}$_i$ János Berlinbe ]]]
	\z
	
	
	In what follows, standard negative \nobreakdash-\textit{e}-interrogatives of the form illustrated in \xref{ex:nem-utazott-e} will be referred to as \textit{nem V-e} interrogatives, and negative  $\bigwedge$-interrogatives as \textit{nem  $\bigwedge$} interrogatives. 
	
	The bias profiles of \textit{nem V-e} and \textit{nem $\bigwedge$} interrogatives are different, as discussed in \citet{Gyuris2017}. Both of them are compatible with \textit{vala}-indefi\-ni\-tes, which \citet{szabolcsi2002} considers positive polarity items (PPIs). This indicates, following \citet{ladd81}, that both give rise to an ``outside negation'' (ON, non-propositional negation) reading. 
	\textit{Nem $\bigwedge$} interrogatives are also compatible with negative polarity items (NPIs), including phrases with \textit{sem} (that \citealt{ekiss2009} refers to as ``negative polarity item, minimizer''), which indicates, following \citet{ladd81}, that they also give rise to a so-called ``inside negation'' (IN, propositional negation) reading.\footnote{Further diagnostics of ON vs. IN readings include compatibility with \textit{is} `too' vs. \textit{sem} `neither', respectively, to be illustrated in \xxref{ex:neut-exp}{ex:pos-exp}.} 
	The availability of the ON- vs.~IN-readings is illustrated for the two negative interrogative forms in \xref{ex:nem-utazott-vala?}--\xref{ex:nem-utazott-vala-sem?}: 
	
	\ea\label{ex:nem-utazott-vala?}
	\gll Nem utazott-e	ki	János Berlinbe (valamikor / *semmikor)?\\
	not	travelled-\textsc{q} \textsc{vm} János Berlin.into \hspace{0.15cm}at.some.time {} \hspace{0.2cm}never\\
	\glt `Didn't János go to Berlin (at some point/*ever)?'\hspace{1cm}ON, *IN
	\z
	
	\ea\label{ex:nem-utazott-vala-sem?} 	
	\gll  Nem utazott ki János Berlinbe (valamikor / semmikor) $\bigwedge$?\\
	not	travelled	\textsc{vm} János 	Berlin.into \hspace{0.15cm}at.some.time {} never (\textsc{q})\\ 
	\glt `Didn't János (at some point/ever) go to Berlin?'\hspace{1.2cm}ON, IN
	\z
	
	The  examples in \xref{ex:neut-exp}--\xref{ex:pos-exp}, where  \textit{nem V-e} and \textit{nem  $\bigwedge$} interrogatives are presented in contexts with no previous expectation regarding any of the possible answers vs.~with expectation towards the positive answer $p$, respectively, show that both  negative forms are incompatible with contexts where the speaker has no  expectation bias, but that they are both compatible with contexts with bias towards $p$ (independently of the ON/IN distinction).
	
	
	\ea\label{ex:neut-exp}
	\textit{No expectation:}\\
	You told me that you went to a party yesterday. I have no idea who else did (or was supposed to go). I ask:
	\ea[\#]{
		\gll Nem volt-e ott (esetleg) János (is) a buliban?\\
		not was-\textsc{q} there perhaps János too the party.in\\
		\glt \#`Didn't (perhaps) János go to the party (too)?'}
	\ex[\#]{
		\gll Nem  volt ott (esetleg) János (is/sem) a {buliban} {$\bigwedge$?}\\
		not was there perhaps János too/neither the party.in (\textsc{q})\\
		\glt `Didn't (perhaps) János go to the party (too/either)?'}
	\z
	\z
	
	
	\ea\label{ex:pos-exp}
	\textit{Positive expectation:}\\
	You have just told me about Mary's birthday party you went to. I have  no idea who else went (or was supposed to go). I know that John is a good friend of Mary's. I ask:
	\ea Nem volt-e ott (esetleg) János (is) a buliban?
	\ex Nem  volt ott (esetleg) János (is/sem) a buliban $\bigwedge$ ?
	\z
	\z
	
	Having looked at the canonical positive and negative interrogative form types in Hungarian, the next section zooms in on interrogatives with \textit{nem-e}.
	
	
\section{Interrogatives with \textit{nem-e}: Data and dialects}\label{sect:dialects}
	
	In the following examples, the majority of which was taken from the Hungarian National Corpus (HNC)\footnote{\url{http://corpus.nytud.hu/mnsz/index\_eng.html}, cf. \citet{oraveczetal2014}.},  the particle \nobreakdash-\textit{e}  appears cliticized onto the negative particle \textit{nem}.  \xref{ex:akcios}-\xref{ex:figyu}\footnote{Repeated with original spelling.} encode information-seeking questions, in \xref{ex:gyun}, \xref{ex:masodik},  \xref{ex:atvernek}\footnote{\url{https://www.gyakorikerdesek.hu/sport-mozgas__egyeb-kerdesek__2716148-hogyan-nezzem-meg-hogy-nem-e-atvernek} (Last accessed: 15 June 2025)},  and 
	\xref{ex:tudni}\footnote{This example is from the questionnaire reported on in \citet{kassai1994}.} \textit{nem-e} appears in ``embedded root'' environments:
	
	\ea\label{ex:akcios}
	\gll Az üzletközpont útvesztőjéből óriási szatyrokkal betéved néhány civil: ``nem-e itt árulják az akciós rozsdamentes edénykészletet.''\\
	the shopping.centre labyrinth.its.from giant bags.with \textsc{vm}.come.\textsc{3sg} some civilian \hspace{0.17cm}not-\textsc{q} here sell.\textsc{3pl} the sale stainless cookware.\textsc{acc}\\ \jambox*{[HNC]}
	\glt `From the labyrinth of the shopping centre some civilians come in with big shopping bags: ``isn't it here where the stainless steel cookware is sold?'' '
	\z
	
	\ea\label{ex:figyu}
	\gll figyu, vince, nem-e vetted még észre, hogy a mti híreit MINDENKI szószerint hozza le/ismétli, mert valószínűleg ez kikötés? \\
	look.\textsc{subj.2sg} Vince not-\textsc{q} took.\textsc{2sg} still \textsc{vm} that the MTI news.its.\textsc{acc} everybody literally bring.\textsc{3sg} \textsc{vm}/repeat.\textsc{3sg} because probably this requirement\\ \jambox*{[HNC]}
	\glt `Look, Vince, haven't you noticed yet that the news of the MTI (Hungarian News Agency) are brought/repeated by everybody using the same words, because probably that's a requirement?'
	\z
	
	\ea\label{ex:gyun}
	\gll
	nëm-ë gyün el, kérdëzzítëk mëg \hspace{3.1cm}(\citealt{hegedus2001})\\
	not-\textsc{q} come.\textsc{3sg} \textsc{vm} ask.\textsc{subj.2pl} \textsc{vm}\\
	\glt `Isn't he coming? Ask him!'
	\z
	
	\ea\label{ex:masodik}
	\gll Kérdés, hogy nem-e a második emeleti folyosó végén lévő hátsó ajtónál fognak csöngetni?\hspace{4.8cm}(\citealt{nadasdy2004})\\
	question that not-\textsc{q} the second floor.of corridor end.its.on being back door.at will.\textsc{3pl} ring.\textsc{inf}\\
	\glt `It is a question whether they won't ring the bell at the back door at the end of the corridor on the second floor.'
	\z
	
	\ea\label{ex:atvernek}
	\gll Hogyan nézzem meg hogy nem-e átvernek?\\
	how look.\textsc{subj.1sg} \textsc{vm} that not-\textsc{q} \textsc{vm}.deceive.\textsc{3pl}\\
	\glt `How should I find out whether they don't deceive me?' 
	\z
	
	\ea\label{ex:tudni}
	\gll Jó lenne tudni, nem-e lesz vihar. \\
	good be.\textsc{subj.3sg} know.\textsc{inf} not-\textsc{q}  be.\textsc{fut.3sg} storm\\
	\glt `It would be good to know whether there won't be a storm.'
	\z
	
	\noindent \xref{ex:stgallen} realizes a  rhetorical question (cf.~the particle \textit{hiszen} `indeed'):
	
	
	\ea\label{ex:stgallen}
	\gll
	Ráadásul egy ilyen hadüzenetnek megvolnának a történelmi gyökerei is. Hiszen nem-e a magyar kalandozó hadak portyáinak igáját nyögte Szent Gallen büszke kolostora?\\
	in.addition a such declaration.of.war.\textsc{dat} \textsc{vm}.be.\textsc{cond.3pl} the historical roots.its too indeed not-\textsc{q} the Hungarian adventuring troops raids.their.\textsc{dat} yoke.its.\textsc{acc} suffered.\textsc{3sg} Saint Gallen proud cloister.its\\ \jambox*{[HNC]}
	\glt `In addition, such a declaration of war would have its roots in history. Wasn't it the yoke of the raids of Hungarian ``adventuring'' troops that the proud cloister of Saint Gallen suffered from?'
	\z
	
	\noindent The question realized by the next example is to be interpreted as a suggestion for an explanation:
	
	\ea\label{ex:honlap}
	\gll Nem-e az az oka ennek, hogy annyi embernek van megnyilatkozási lehetősége (mindenki szerkeszthet magának honlapot pl.), hogy egyszerűen nem tudjuk átlátni a helyzetet.\\
	not-\textsc{q} that the reason.its this.\textsc{dat} that so.many person.\textsc{dat} is expression opportunity.its \hspace{0.15cm}everybody create.\textsc{poss.3sg} himself.\textsc{dat} homepage.\textsc{acc} e.g. that simply not know.\textsc{1pl} \textsc{vm.}see.\textsc{inf} the situation.\textsc{acc}\\ \jambox*{[HNC]}
	\glt `Isn't the reason for this  that so many people have an opportunity to express themselves (everybody can create a homepage for themselves, for example) that we simply cannot understand the situation?'
	\z
	
	
	\noindent In \xref{ex:tetszik}\footnote{Kondor, Vilmos 2018. A haldokló részvényes. ('The shareholder on his deathbed.') Libri Kiadó, Budapest. (Courtesy of László Simon.)} the interrogative  encodes an  indirect offer:
	
	\ea\label{ex:tetszik}
	\gll – Másvalamit nem-e tetszik kérni? – folytatta a leány.\\
	{} other.something.\textsc{acc} not-\textsc{q} like.\textsc{3sg} ask.\textsc{inf} {} continued.\textsc{3sg} the girl\\
	\glt ` – Don't you wish something else? – the girl continued.'
	\z
	
	\noindent The  interrogative in \xref{ex:fel} is used to encode a threat. (Note the remark by one of the interlocutors in this dialogue about the education of the original speaker whose utterance is reported on here, to be discussed below.)
	
	\ea\label{ex:fel}
	\gll Azt kérdezte egy férfihang, hogy asszonyom, nem-e fél \dots{}  \nobreakdash--{ }Így kérdezte, hogy ``nem-e fél''? Nem lehetett egy akadémikus. Szóval, mit kérdezett? – Nem-e fél a kedves férje, hogy lóba varrjuk? \\
	that.\textsc{acc} asked.\textsc{3sg} a man's.voice that madam not\textsc{-q} afraid {} \hspace{0.3cm}so asked.\textsc{3sg} that \hspace{0.15cm}not\textsc{-q} afraid not be.\textsc{poss.past.3sg} an academic so what.\textsc{acc} asked.\textsc{3sg} {} not\textsc{-q} afraid the kind husband.your that horse.into sew.\textsc{1pl}\\ \jambox*{[HNC]}
	\glt `A man's voice asked, madam, isn't he afraid \dots { } -- Did he ask this way, ``isn't he afraid''? He surely wasn't an academic. So, what did he ask? \nobreakdash--{ }Isn't your dear husband afraid that we sew him into a horse?'
	\z
	
	
	The two examples in \xref{ex:leszerel} and \xref{ex:medve} illustrate the use of the \textit{nem-e} form to make requests. Note the repetition of \textit{-e} in the latter, which will be discussed below:
	
	\ea\label{ex:leszerel}
	\gll Valaki esetleg nem-e tud segíteni a leszerelésben? \\
	somebody possibly not\textsc{-q} can.\textsc{3sg} help.\textsc{inf} the dismantling.in\\ \jambox*{[HNC]}
	\glt `Can't perhaps somebody help in dismantling it?'
	\z
	
	\ea\label{ex:medve}
	\gll Te, medve, nem-e lehetne-e engem arról a listáról kihúzni?\\
	you bear not-\textsc{q} be.\textsc{poss.cond-q} I.\textsc{acc} that.from the list.from \textsc{vm}.delete.\textsc{inf}\\ \jambox*{[HNC]}
	\glt `You, bear, couldn't my name be deleted from that list?'
	\z
	
	The examples provided above might give the impression that interrogatives containing \textit{nem-e} constitute a formal variant of \textit{nem V-e} interrogatives, illustrated in \xref{ex:nem-utazott-e}, a position also taken in \citet{kenesei-etal}. I am going to point out, however, that  \textit{nem-e}  appears in two kinds of syntactic structures, and then argue that these are associated with different use conditions. I will also suggest that the two structures are in fact used in two different dialects. 
	
	The first type of interrogatives with \textit{nem-e}, illustrated in \xref{ex:figyu}--\xref{ex:gyun} and  \xref{ex:leszerel} above, contains verb--\textsc{vm} inversion, as \textit{nem V-e} interrogatives do, cf.~\xref{ex:nem-utazott-e}, but differ from the latter in that \textit{-e} cliticizes onto the negative particle.\footnote{In the case of \xref{ex:leszerel}, the infinitive \textit{segíteni} `help.\textsc{inf}' is the \textsc{vm}.}  I suggest that this configuration is a result of a phonological process, and thus the syntactic structure of  the \textit{nem-e} clause of \xref{ex:gyun} is as shown in \xref{ex:gyun-str}. The latter is either pronounced as in \xref{ex:pron-single}, with \textit{-e} cliticized onto the negative particle, or as in \xref{ex:pron-double}, with \textit{-e} pronounced twice, cf. \xref{ex:medve}.\footnote{I follow \citet{gaertner+gyuris2022}  in taking \textit{-e} to be base-generated in I$^\text{o}$. By contrast, \citet[342]{kenesei1994} considers counterparts of \xref{ex:pron-single} to speak in favor of lowering \textit{-e} from C$^\text{o}$. Discussion of the two approaches -- in particular with respect to  their predictions regarding locality -- is beyond the scope of the current paper.}
	
	
	
	\ea
	\ea\label{ex:gyun-str}
	[ \dots { } [\textsubscript{NegP} nem gyün-e$_i$ [\textsubscript{IP} el $t_i$ ]] \dots {} ]
	\ex\label{ex:pron-single}
	nem-e gyün-\cancel{e} el
	\ex\label{ex:pron-double}
	nem-e gyün-e el
	\z
	\z
	
	
	
	\noindent Interrogatives containing \textit{nem-e} and verb–\textsc{vm} inversion will be referred to as  \textit{nem-e V \textsc{vm}} interrogatives (\textit{nem-e} interrogatives with inversion). \textit{Nem-e V \textsc{vm}} interrogatives   were characteristic of Western Hungarian dialects until the 19th century. However, since in the Northeastern dialect that formed the basis of the standard (literary) dialect of Hungarian \nobreakdash-\textit{e} cliticizes onto the verb, as in \xref{ex:nem-utazott-e}, \textit{nem-e} forms with inversion from other dialects started to be judged as substandard, and got stigmatized (cf. example \xref{ex:fel} above). Informal evidence indicates that speakers who use \textit{nem-e V \textsc{vm}} interrogatives  use them in the same contexts speakers of the Standard Dialect use \textit{nem V-e} interrogatives (with obligatory inversion). This is the reason we refer to the dialect where \textit{nem\nobreakdash-e V \textsc{vm}} interrogatives  are used as the Stigmatized Dialect (Dialect S).\footnote{Stigmatization applies in most cases to any form containing \textit{nem-e}, irrespective of inversion, which makes it very difficult to obtain reliable data about \textit{nem-e}.} In Dialect S, cases of \nobreakdash-\textit{e} doubling, as in \xref{ex:pron-double}, normally mark the speaker's uncertainty, and are often used in indirect requests intended to be very polite.\footnote{Constructions with \textit{-e} doubling are very often used for a stylistic effect, to mock speakers of non-standard dialects.}  In the latter uses \textit{nem-e} can also be analysed as a particle, adjoined to a clausal constituent, like adverbs and other particles are (cf.~\citealt{gaertner+gyuris12}). 
	
	The second type of interrogative with \textit{nem-e}  lacks inversion between the verb and the \textsc{vm}, illustrated in \xref{ex:atvernek}. This latter type will be referred to as the \textit{nem-e  \textsc{vm}\nobreakdash-V} interrogative. It is mostly used by speakers who in other respects speak the Standard Dialect (using  \textit{nem V-e} interrogatives as well), but in a much more restricted range of situations than \textit{nem V-e}  interrogatives, to encode a particular type of  noncanonical question. The dialect where 
	\textit{nem-e \textsc{vm}-V} interrogatives  appear will be referred  to  as  Dialect E (from ``Educated" Dialect).\footnote{Cf.~\citet[224--227]{nadasdy2004}. I thank Ádám Nádasdy for discussion on the use of \textit{nem-e \textsc{vm}-V} interrogatives in the dialect I refer to as Dialect E.}\footnote{
    There are also examples for \textit{nem-e \textsc{vm}-V} interrogatives being used in situations where \textit{nem V-e} interrogatives are used in the Standard Dialect, as in \xref{fn:ex:i}. We will ignore them in what follows.
    \ea\label{fn:ex:i}
    ...\gll még a rúzsát is meg-nézte, hogy ``nem-e elkenődött" \\
    even the lipstick.\textsc{acc} also \textsc{vm}-looked that not-\textsc{q} \textsc{vm}.smeared\\
    \jambox*{[HNC]}
    \glt `...she even looked at her lipstick, “whether it did not smear” ’
    \z
    } 
	
	Note, importantly, that the two types of \textit{nem-e} interrogatives can only be distinguished in case there is a \textsc{vm} in the sentence, and there is no constituent in the immediately preverbal focus position, which automatically forces inversion. (Cf.~\citealt{ekiss2002} for the syntax of the focus position.) \xref{ex:tortent} and \xref{ex:fel}, with no  \textsc{vm}, and  \xref{ex:akcios}, \xref{ex:masodik}, \xref{ex:stgallen}, \xref{ex:honlap} and \xref{ex:fel}, with a preverbal focus constituent, can be analysed as representing both categories out of context. \tabref{ex:summary} presents the inventory of negative interrogative forms in the three dialects distinguished in this work.
	
\begin{table}
	\begin{tabular}{ll}
		\lsptoprule
		Standard Dialect & \textit{nem $\bigwedge$}\\
		& \textit{nem V-e}\\
		\midrule
		Dialect S & \textit{nem $\bigwedge$}\\
		&  \textit{nem-e V \textsc{vm}}\\
		\midrule
		Dialect E & \textit{nem $\bigwedge$} \\
		& \textit{nem V-e} \\
		& \textit{nem-e \textsc{vm}-V}\\
		\lspbottomrule
	\end{tabular}
	\caption{Inventory of negative interrogative forms}
\label{ex:summary}
\end{table}

	The rest of the paper will concentrate on the formal and interpretational features of \textit{nem-e \textsc{vm}-V} interrogatives  in Dialect E. The next section is devoted to a review of its relevant syntactic and semantic properties.
	
	
\section{Properties of \textit{nem-e \textsc{vm}-V} interrogatives in Dialect E}\label{sect:nem-e-dialect-e}
	
\subsection{Type of negation}\label{sect:type_negation}
	\xref{ex:nem-e-elutazott} is a \textit{nem-e \textsc{vm}-V} interrogative from Dialect E:
	
	\ea\label{ex:nem-e-elutazott}
	\gll Nem-e ki-utazott János Berlinbe?\\
	not-\textsc{q} \textsc{vm}-travelled.\textsc{3sg} János Berlin.into\\ 
	\glt `Didn't János go to Berlin?'
	\z
	
	\noindent \xref{ex:nem-e-elutazott-ON} illustrates the compatibility of \xref{ex:nem-e-elutazott} with a PPI (\textit{valamikor} `at some point') and its incompatibility with an NPI (\textit{semmikor} `never'): 
	
	\ea\label{ex:nem-e-elutazott-ON}
	\gll Nem-e ki-utazott János Berlinbe {valamikor \hspace{0.4cm}/} *semmikor?\\
	not-\textsc{q} \textsc{vm}-travelled.\textsc{3sg} János Berlin.into at.some.point  \hspace{0.15cm}never\\ \glt `Didn't János go to Berlin at some point/*ever?'
	\z
	
	\noindent \xref{ex:nem-e-elutazott-ON}
	thus indicates that \textit{nem-e \textsc{vm}-V} interrogatives have an ON, but no IN reading. This might suggest that they can be used felicitously in the same contexts as \textit{nem V-e}  interrogatives. The following subsections will, however, argue against this assumption.   
	
	
\subsection{Expectation bias}\label{sect:exp_bias}
\subsubsection{The data}\label{sect:exp_bias_data}
	
	It has been assumed in the literature (cf.~\citealt{Reese2007, sudo2013}, a.o.) that whenever a polar interrogative form introduces an expectation bias for the positive or the negative answer, the source of this bias can in principle be the speaker's knowledge or beliefs (epistemic bias), her wishes (bouletic bias) or some set of rules (deontic bias). \xref{ex:sugg} illustrates the use of \textit{nem-e \textsc{vm}-V},  \textit{nem V-e} and \textit{nem $\bigwedge$} interrogatives in a context where the expectation bias for the positive answer (`John went to Berlin') is based on the speaker's beliefs:
	
	\ea\label{ex:sugg}  
	\textit{Suggestion scenario}
	\sn A, B and János are colleagues. A and B talk after a meeting.\\
	A: Why wasn't János present at the meeting?\\
	B replies:
	\ea\gll Nem-e  ki-utazott Berlinbe?\\
	not\textsc{-q}	\textsc{vm-}travelled Berlin.into\\	
	\glt `Didn't he go to Berlin?' 
	\ex Nem utazott-e ki Berlinbe?\\ 
	`Didn't he go to Berlin?'
	\ex Nem utazott ki Berlinbe $\bigwedge$ ?\\
	`Didn't he go to Berlin?'
	\z
	\z
	
	\noindent In the \textit{Suggestion scenario}, all  three negative interrogative forms are felicitous. The following example shows, however, that they do not always pattern together:
	
	\ea\label{ex:mother}
	\textit{Reproach scenario}\\
	Mother sees her child kick another child in the sandpit. Mother says to her child:
	\ea[\#]{\gll Nem-e 	sz\'egyelled	magad?\\
		not\textsc{-q}	be.ashamed.\textsc{2sg}	yourself\\
		\glt `Aren't you ashamed?'}\label{ex:mothera}
	\ex[]{Nem sz\'egyelled-e magad? \\
		`Aren't you ashamed?'}\label{ex:motherb}
	\ex[]{Nem sz\'egyelled magad  $\bigwedge$ ? \\
		`Aren't you ashamed?'}\label{ex:motherc}
	\z
	\z
	

	\noindent The  intended interpretation of the negative interrogatives in \xref{ex:mother} is the following. Mother, the authority, thinks that Child should be ashamed of his actions (based on assumed rules of conduct). By asking an information-seeking question, Mother thus indirectly calls Child's attention to the fact that his conduct was inappropriate. This interpretation presupposes that Mother, the speaker, has a deontic bias for  the proposition `Child is ashamed' ($p$). Note that \xref{ex:mothera} would be felicitous in a situation where the purpose of Mother's question were to make a guess about how Child is feeling, since this would be compatible with her having a previous epistemic bias for $p$.\footnote{\xref{ex:mothera} is felicitous in Dialect S in the context provided.}

	\xref{ex:bocsanat} provides another illustration for the scenario above.  As opposed to  \xref{ex:mothera}--\xref{ex:motherc}, \xref{ex:bocsanata}--\xref{ex:bocsanatc} contain  a \textsc{vm}, the bare noun \textit{bocsánatot} `apology.\textsc{acc}', which makes the absence of the \textsc{vm}--verb inversion  visible. The felicity judgments for the three form types pattern with those  pertaining to \xref{ex:mother}.
	
	\ea\label{ex:bocsanat}
	Mother sees her Child kick another child in the sandpit. Mother to Child:
	\ea[\#]{\gll Nem-e  bocs\'anatot k\'ersz?\\
		not\textsc{-q}	apology.\textsc{acc} ask.\textsc{2sg}	\\
		\glt `Don't you apologize?'}\label{ex:bocsanata}
	\ex[]{Nem k\'ersz-e   bocs\'anatot?\\
		`Don't you apologize?'}\label{ex:bocsanatb}
	\ex[]{Nem k\'ersz   bocs\'anatot  $\bigwedge$ ?\\
		`Don't you apologize?'}\label{ex:bocsanatc}
	\z
	\z
	
	The next example shows the  three negative interrogative forms in a context where they are used to encode offers. 
	
	\ea\label{ex:offer} \textit{Offer scenario}\\
	B, a colleague, enters A's office. A wants to offer him some coffee and thus says to him:
	\ea[\#]{\gll Nem-e   meg-innál egy kávét?\\
		nem-\textsc{q} \textsc{vm}-drink.\textsc{cond.}2\textsc{sg} one coffee.\textsc{acc}\\
		\glt `Wouldn't you drink a coffee?'}\label{ex:meginnal-neme}
	\ex[]{Nem  innál-e   meg egy kávét?\\
		`Wouldn't you drink a coffee?'}\label{ex:meginnal-e}
	\ex[]{Nem  innál   meg egy kávét  $\bigwedge$ ?\\
		`Wouldn't you drink a coffee?'}\label{ex:meginnal-rise-fall}
	\z
	\z
	
	\noindent In the \textit{Offer scenario}, the \textit{nem-e \textsc{vm}-V} interrogative form is  infelicitous again, as opposed to the other two. The context justifies the assumption that the questioner has (or acts as if having) a bouletic bias towards the proposition `Addressee would drink a coffee'. (A person making a sincere offer has a preference for the addressee accepting it.)  Note that in case the speaker's aim were to make a guess about what the addressee wishes to drink (or do something) in general, \xref{ex:meginnal-neme} would be just as felicitous as \xref{ex:meginnal-e}--\xref{ex:meginnal-rise-fall}.\footnote{\xref{ex:meginnal-neme} is felicitous in Dialect S in the context provided.}

	
	
	The next example illustrates the use of the negative interrogative forms to make a request:
	
	\ea\label{ex:request}   \textit{Request scenario}\\
	In front of the coffee machine, A addresses her colleague B:
	\ea[\#]{\gll Nem-e  kölcsön-adnál egy százast?\\
		not-\textsc{q} \textsc{vm}-give.\textsc{cond.}2\textsc{sg}  a hundred.\textsc{acc}\\
		\glt `Wouldn't you lend me a hundred forints?'}\label{ex:adnal-neme}
	\ex[]{Nem adnál-e kölcsön egy százast?\\
		`Wouldn't you lend me a hundred forints?'}
	\ex[]{Nem adnál kölcsön egy százast $\bigwedge$ ?\\
		`Wouldn't you lend me a hundred forints?'}
	\z
	\z
	
	\noindent Here again, the \textit{nem-e \textsc{vm}-V} interrogative in \xref{ex:adnal-neme} is infelicitous, as opposed to the other two forms. The context in which the interrogatives are used to make an indirect request indicates bouletic bias of the speaker towards the  answer. If the speaker were making a guess about the intentions of the addressee, \xref{ex:adnal-neme} would be felicitous.\footnote{\xref{ex:adnal-neme} is felicitous in Dialect S in the context provided.} 

	
	Thus, we have shown that \textit{nem-e \textsc{vm}-V} interrogatives in Dialect E (having only an ON reading) differ from  \textit{nem V-e} interrogatives (which also only have an ON reading) and from \textit{nem  $\bigwedge$} interrogatives (which are ambiguous between ON and IN readings): the first form is unavailable to make an indirect reproach, an offer or a request. Since the felicity of the latter three speech acts depends on the speaker having deontic or bouletic (that is, non-epistemic) biases towards the positive answer, one reasonable explanation for 
	the infelicity of (\ref{ex:mother}--\ref{ex:request})  in Dialect E would be that \textit{nem-e \textsc{vm}-V} interrogatives are incompatible with non-epistemic (i.e., deontic or bouletic) biases for the positive answer. In the next section  we look at previous approaches in the literature that were concerned with types of speaker expectation bias, to see whether they can offer a solution for the puzzle.
	
	
\subsubsection{Previous accounts on (speaker) expectation bias}
	

	\citet{Reese2007}, the only theoretically-based proposal on the relation between bias types and IN/ON readings, argues that negative interrogatives with IN readings can have an epistemic, deontic or bouletic expectation bias for the positive answer, while those with ON readings can only be epistemically biased. According to \citet[91]{Reese2007}, the difference follows from the assumption that the biases associated with the two readings are ``distinct kinds of meaning'': the biases of IN readings constitute ``some type of implicature'', whereas those of ON readings are ``entailments, reflecting a speaker commitment which functions as a weak assertion''. Interrogatives with ON readings, which  ``share the distributional properties of questions and assertions'' are accounted for by the author by assigning to them ``a conventionalized complex speech act type \textsc{assertion} \textbullet {} \textsc{question}'', cf.~\citet{asher+lascarides2001, asher+lascarides2003}. 
	
	The claim that deontic and bouletic biases can only arise for IN readings does indeed explain the infelicity of \textit{nem-e \textsc{vm}-V} interrogatives, which can only have ON readings, in contexts \xxref{ex:mother}{ex:request}. However, in these contexts \textit{nem V-e} interrogatives, which also only give rise to ON readings, are all felicitous. This suggests that Reese's general proposal for the  types of  expectation biases available on the basis of the availability of  IN/ON readings cannot be extended to  the relevant Hungarian data.
	
	Contrary to Reese, \citet[284]{sudo2013} assumes that there are negative interrogatives in English with ON readings that ``imply a positive expectation stemming from the norm/rules (deontic) or what the speaker desires (bouletic), rather than what the speaker believes
	to be true'', as in  \textit{Aren't you ashamed of yourselves?}, or \textit{Don't you like it?}, respectively,\footnote{The examples are used by \citet{asher+reese2007}, originally due to \citet{huddleston+pullum}.} although without further theoretical justification.
	
	Having illustrated that \textit{nem-e} interrogatives in Hungarian introduce a distinction between types of expectation biases that has not yet been observed in the literature, we turn to some other properties that determine their use in discourse.
	
	
\subsection{Further discourse properties of \textit{nem-e \textsc{vm}-V} interrogatives}\label{sect:disc-prop}

	
\subsubsection{Unresolved question in the discourse}\label{sect:unresolved}
	
	\xref{ex:sugg}, repeated in \xref{ex:sugg1} below, shows that all the three negative interrogative forms under consideration are felicitous in situations where the aim of the speaker's utterance is to put forward a suggestion for a congruent answer (cf. \citealt{vonstechow1991}) to an unresolved question in the discourse. 

	\ea\label{ex:sugg1}  
	\textit{Suggestion scenario}\\
	A, B and János are colleagues. A and B talk after a meeting.\\
	A: Why wasn't János present at the meeting?\\
	B replies:
	\ea\gll Nem-e  ki-utazott Berlinbe?\\ 
	not\textsc{-q}	\textsc{vm-}travelled Berlin.into\\	
	\glt`Didn't he go to Berlin?' 
	\ex Nem utazott-e ki Berlinbe?\\
	`Didn't he go to Berlin?'
	\ex Nem utazott ki Berlinbe $\bigwedge$ ?\\ 
	`Didn't he go to Berlin?'
	\z
	\z
	
	
	\noindent The unresolved question under consideration  in the dialogue above is the one uttered by A, which activates a set of alternative (full) answers of the type `János wasn't at the meeting because $q$', where $q$ stands for a proposition. $q$ itself thus corresponds to a \textit{term answer} to the unresolved question above.\footnote{Cf.~\citet{krifka2011} for the definitions of \textit{full} vs.~\textit{term} answers.}
	The proposition `He went to Berlin' is offered by B in the dialogue as the  value of $q$.
	
	
	\xref{ex:passed} presents  the three negative interrogatives in a context  with no unresolved question in the context:
	
	\ea\label{ex:passed} \textit{I have to ask something scenario}\\
	A to B:  I have to ask you something.
	\ea[\#]{\gll Nem-e át-ment János a vizsgán?\\
		not-\textsc{q} \textsc{vm}-went\textsc{.3sg} János the exam.on\\
		\glt `Didn't János pass the exam?'}\label{ex:passeda}
	\ex[]{Nem ment-e át János a vizsgán?\\
		`Didn't János pass the exam?'}\label{ex:passedb}
	\ex[]{Nem ment át János a vizsgán $\bigwedge$ ?\\
		`Didn't János pass the exam?'}\label{ex:passedc}
	\z
	\z
	
	\noindent The felicity of \xxref{ex:passedb}{ex:passedc} and the infelicity of  \xref{ex:passeda} confirms the suggestion according to which \textit{nem-e \textsc{vm}-V} interrogatives require the presence of an unresolved question in the context.
	
	
\subsubsection{No coordination}\label{sect:complete}
	
	Here we want to point out that the coordination of two \textit{nem-e \textsc{vm}-V} interrogatives is infelicitous, as opposed to the coordination of two exemplars of the other two negative interrogative form types, as illustrated in \xref{ex:sugg-var} below:
	
	\ea\label{ex:sugg-var} A, B and János are colleagues. A and B talk after a meeting.\\
	A: Why wasn't János present at the meeting?\\
	B replies:
	\ea[\#]{\gll Nem-e  el-felejtette az időpontot és nem-e ki-utazott Berlinbe?\\ 
		not-\textsc{q} \textsc{vm}-forgot the date.\textsc{acc} and not-\textsc{q} \textsc{vm}-travelled Berlin.into\\
		\glt `Didn't he forget the date and didn't he go to Berlin?'}\label{ex:sugga}
	\ex[]{Nem felejtette-e el az időpontot és nem utazott-e ki Berlinbe?\\ 
		`Didn't he forget the date and didn't he go to Berlin?'}\label{ex:suggb}
	\ex[]{Nem felejtette el az időpontot és nem  utazott ki Berlinbe $\bigwedge$ ?\\
		`Didn't he forget the date and didn't he go to Berlin?'}\label{ex:suggc}
	\z
	\z
	
	\noindent Based on the discussion in \sectref{sect:unresolved}, the contrast  between \xref{ex:sugga} and  \xxref{ex:suggb}{ex:suggc} can be interpreted as indicating  that \textit{nem-e \textsc{vm}-V} interrogatives can only be used to suggest a complete congruent answer to the unresolved question (thus making the conjunction of two such forms infelicitous), whereas the other two forms can also be used to suggest a partial answer. 
	
\subsubsection{Possible replies to questions with \textit{nem-e}}\label{sect:possible_answers}
	
	We look next at how to react to questions encoded by \textit{nem-e \textsc{vm}-V} interrogatives. \xref{ex:lehet-nyaral}--\xref{ex:azert} present potential replies to the question in \xref{ex:neme-nyaralni} in the context illustrated:
	
	\ea\label{ex:suntan} A and B see their colleague János from a distance, and note that he has a suntan. 
	\ea{\gll A: Nem-e nyaralni volt?\\
		{} not-\textsc{q} be.on.holiday.\textsc{inf} was\\
		\glt \hspace{0.4cm}`Wasn't he on holiday?'}\label{ex:neme-nyaralni}
	\ex{\gll B: Lehet.\\
		{} maybe\\
		\glt \hspace{0.4cm}`Maybe.'}\label{ex:lehet-nyaral}
	\ex{\gll B: \#Igen (,  nyaralni volt).\\
		{} \hspace{0.2cm}yes {}  be.on.holiday.\textsc{inf} was\\
		\glt \hspace{0.4cm}\#`Yes (, he was on holiday).'}\label{ex:yes-nyaral}
	\ex{\gll B: {}$^\%$De (igen) (,  nyaralni volt).\\
		{} \hspace{0.2cm}but \hspace{0.15cm}yes  {}   be.on.holiday.\textsc{inf} was\\
		\glt \hspace{0.4cm}{}$^\%$`Yes (, he was on holiday).'}\label{ex:but-nyaral}
	\ex{\gll B: Nem, \#(nem volt nyaralni). \\
		{} no \hspace{0.3cm}not was be.on.holiday.\textsc{inf}  \\
		\glt \hspace{0.4cm}`No, \#(he was not on holiday).'}\label{ex:nem-nyaral}
	\ex{\gll B: Nem, a kertben dolgozott. \\
		{} no the garden.in worked\\
		\glt \hspace{0.4cm}`No, he was working in the garden.'}\label{ex:kertben}
	\ex{\gll B: Volt nyaralni, de nem azért barna.\\
		{} was be.on.holiday.\textsc{inf} but not because.of.that brown \\
		\glt \hspace{0.4cm}`He was on holiday, but he does not have a tan because of that.'}\label{ex:nemazert} 
	\ex{\gll B: (Nem,) (volt nyaralni, de) azért barna, mert \hspace{0.4cm}a kertben dolgozott.\\
		{} \hspace{0.15cm}not \hspace{0.15cm}was be.on.holiday.\textsc{inf} but because.of.that brown because \hspace{0.4cm}the garden.in worked\\
		\glt \hspace{0.4cm}`(No,) (he was on holiday, but) he has a tan because he worked in\\ \hspace{0.4cm}the garden.'}\label{ex:azert} 
	\z
	\z
	
	\noindent I suggest that these data indicate that, in spite of appearances, the set of congruent answers to questions realized by \textit{nem-e \textsc{vm}-V} interrogatives does not consist of 
	the denotation of the surface constituent following \textit{nem-e} and its negation. For \xref{ex:neme-nyaralni}, this set would include the propositions `He was on holiday' and `He was not on holiday'. \xref{ex:lehet-nyaral} illustrates the most natural reply to \xref{ex:neme-nyaralni} in the context. 
	Since the response particle \textit{igen} `yes' is infelicitous in answers to canonical forms of negative interrogatives in Hungarian (cf.~\citealt{farkas2009} for an account), it comes as no surprise that, as evidenced by \xref{ex:yes-nyaral},  it is also excluded as answer to a question realized by a \textit{nem-e \textsc{vm}-V} interrogative. It is more unexpected  that  the response particle \textit{de} `but'\footnote{\textit{De} is analogous to German \textit{doch} `but', encoding a reverse polarity reply to a negative polar question.} (with or without the assumed positive answer) is not considered acceptable by all speakers of this dialect in the context under consideration, \xref{ex:but-nyaral}\footnote{I thank Lilla Kamilla Sándor and Viktória Virovec for discussions on the data.} and that the response particle \textit{nem} `not' in \xref{ex:nem-nyaral} does not constitute a felicitous reply in isolation, either, only if followed by the utterance of a declarative disambiguating the answer. 

	\xref{ex:kertben}, where \textit{nem} is followed by the utterance of an alternative answer to the superordinate question (\textit{How did he get a suntan?}), is fine, as is \xref{ex:nemazert}, which gives a polarity-reversing answer to the \textit{nem-e} question, but explicitly denies that it also answers the superordinate question. Finally, \xref{ex:azert} answers the \textit{nem-e} question, but rejects that the latter is identical to the complete answer to the superordinate question, by also explicitly giving an answer to the latter. 
	
	
\subsubsection{No rhetorical question reading}\label{sect:no_add_comp}

	
	The final property of \textit{nem-e \textsc{vm}-V} interrogatives we wish to mention here is that they cannot be used by a speaker in a situation where the denotation of the surface constituent following \textit{nem-e} or its negation is part of the common ground, in other words, where they encode a rhetorical question (cf. \citealt{caponigro+sprouse}). Consider the relevant three forms (containing a covert copula) in a context where they are supposed to realize a rhetorical question:
	
	
	\ea\label{ex:rhet-q} \textit{Rhetorical question scenario}\\
	A and B are talking. They both know that Péter is A's oldest friend.\\
	A: Why is Péter always so helpful?\\
	B replies:
	\ea[\#]{\gll Nem-e	ő 	a legrégebbi bar\'atod?\\
		not-\textsc{q}	he the oldest friend.your\\
		\glt `Isn't he your oldest friend?' \\(Intended: `He is your oldest friend.')}\label{ex:secreta}
	\ex[]{Nem	ő-e	a legrégebbi bar\'atod?\\
		`Isn't he your oldest friend?'\\ (Intended: `He is your oldest friend.')}\label{ex:secretb}
	\ex[]{Nem	ő	a legrégebbi bar\'atod $\bigwedge$ ?\\
		`Isn't he your oldest friend?'\\ (Intended: `He is your oldest friend.')}\label{ex:secretc}
	\z
	\z
	
	\noindent (\ref{ex:secretb}--\ref{ex:secretc}) are available to commit the speaker to the proposition `Péter is your oldest friend', and to indicate, based on the Maxim of Relevance, that this is B's answer to A's question. \xref{ex:secreta}, however, is not available for this purpose in Dialect E (although it would be available in Dialect S). The latter can only be interpreted as a suggestion for an answer to A's question by the speaker, without assuming that the answer is in the common ground, thus, not as a rhetorical question. The next subsection presents the proposal explaining these data.
	
	
\section{\textit{Nem-e \textsc{vm}-V} interrogatives: The account}\label{sect:account}
	
\subsection{Structural assumptions}\label{sect:structural}
	As it was shown in the previous sections, \textit{nem-e \textsc{vm}-V} interrogatives have a more restricted use in Dialect E than the canonical \textit{nem V-e} interrogatives. Thus, the former represent  a special form type having a specific interpretation. One possible approach towards explaining their distribution would be to consider \textit{nem-e}  a discourse/pragmatic particle, which indicates that the rest of the sentence denotes a proposition that the speaker puts forward as a suggested answer to an unresolved question. This proposal, however, fails to account for why clauses containing \textit{nem-e} count as interrogatives, and thus can be embedded under matrix predicates that embed interrogatives, as in \xref{ex:ilike}.\footnote{Note that the embedded (root) interrogative in \xref{ex:ilike} would license the ``reflectivity'' particle \textit{vajon} `I wonder', a diagnostic of interrogative clauses, cf. \citet{kenesei1994} and \citet{kalman2001}.}
	
	\ea\label{ex:ilike}
	\gll Ilike töpreng rajta, hogy nem-e ő bántotta meg {valamivel.}\\
	Ilike contemplates on.it that not\textsc{-q} she offended \textsc{vm} something.with\\ 
	\glt`Ilike contemplates whether it was her who offended him with something.' \jambox*{[HNC]}
	\z
	
	As an alternative, we propose that \textit{nem-e} is the visible subpart of a matrix copular negative interrogative clause, and the rest of the \textit{nem-e} interrogative originates from an embedded  declarative, whose polarity (other things being equal) is positive. The  essentials of the full structure
	of the \textit{nem-e} interrogative in  \xref{ex:e}, including covert parts,
	are shown in \xref{ex:covert}:
	
	\ea\label{ex:e}
	\gll Nem-e ki-utazott Berlinbe?\\
    not-\textsc{q} \textsc{vm}-travelled Berlin.into\\
    \glt `Isn't it that he went to Berlin?'
	\z
	\ea\label{ex:covert}
	\gll [$_{\text{CP}_1}$ \dots {} [\textsubscript{NegP} Nem [\textsubscript{FocP} \sout{az}  \sout{van}-e   [$_{\text{IP}_1}$ \dots { } \hspace{4cm} [$_{\text{CP}_2}$  \sout{hogy} [$_{\text{IP}_2}$ ki-utazott {Berlinbe ]]]] \dots ]}\\
	{} {} {} not {} that be.3\textsc{sg}-\textsc{q}  {} {} {} {} that {} \textsc{vm}-travelled Berlin.into\\
	\glt `Isn't it that he went to Berlin?'
	\z
	
	\noindent Let us consider the properties of the structure in \xref{ex:covert}. First, the  covert expletive \textit{az} is the ``correlate'' of the subordinate declarative clause. It is situated in the preverbal focus position.\footnote{Cf.~\citet{kenesei1994} for a comprehensive account of the syntax of subordinate clauses in Hungarian.} Second,  \textit{az} is followed by the covert copula \textit{van} `be.3\textsc{sg}'. 
	
	Third, the clitic \textit{-e}  ends up attached to the  negative particle \textit{nem} because both the copula and the correlate  are covert. Finally, the complementizer \textit{hogy}, introducing the subordinate clause, also remains  covert.\footnote{Obligatory covertness is the result of a certain degree of grammaticalization having affected \textit{nem-e}. For discussion of the trade-off between compositional and construction-specific properties see \citet{reis1999} and \citet{jacobs2016}.}  It is well known that the preverbal focus position within the Hungarian sentence (which, other things being equal,  hosts the constituent serving as the term answer to the Immediate Question Under Discussion, discussed below, cf. \citealt{gyuris2012}), is associated with an  exhaustive/identificational reading (cf.~\citealt{ekiss2002, szabolcsi94} for general discussion, a.o.). Since the  expletive \textit{az} in the focus position of the main clause functions as a ``placeholder'' for the subordinate declarative clause, 
	we assume that the \textit{nem-e} construction makes the  denotation of the declarative a propositional focus with an exhaustive/identificational reading. These interpretational features are emphasized in the  English translation given in \xref{ex:covert}, which contains a cleft construction, and is thus preferable to the English translations given previously  for \textit{nem-e} interrogatives, in terms of plain negative polar interrogatives. In what follows, we will therefore use the cleft construction in the translations.
	
	

	
	The structural assumptions listed above can account for the lack of inversion between the verb and the verb modifier, which is the default word order in positive declaratives, cf. \xref{ex:elutazott} above. 
	

	
	Additional support for the biclausal analysis is provided by the possibility of \textit{nem-e}  preceding another \textit{nem} `not', illustrated in \xref{ex:nem-e-nem}. Here the second negative particle is a constituent of the ``embedded declarative''.
	
	\ea\label{ex:nem-e-nem} A and B are talking. \\
	A: I thought John went to a conference but his car is in the car park opposite the building.\\ 
    B replies: \gll Nem-e  nem utazott el?\\
	not-\textsc{q} not travelled \textsc{vm}\\
	\glt \phantom{B replies:} `Isn't it that he did not go away (perhaps)?'
	\z
	
	In the next section we present the preliminaries for an account of the interpretation of  \textit{nem-e \textsc{vm}-V} interrogatives listed in \sectref{sect:nem-e-dialect-e} on the basis of the structure postulated above.
	
	
	
\subsection{Interpreting  \textit{nem-e \textsc{vm}-V} interrogatives: Basic assumptions}\label{sect:basic}
\largerpage
	In \sectref{sect:nem-e-dialect-e}  above \textit{nem-e \textsc{vm}-V} interrogatives were  associated with  the interpretational  properties listed in \xref{properties}.  Here, $p$ stands for the denotation of the embedded declarative (cf. CP$_2$ in \xref{ex:covert}) following \textit{nem-e}:

	
	\ea\label{properties} Semantic and discourse properties of \textit{nem-e \textsc{vm}-V} interrogatives in Dialect E
	\ea\label{property1} They give rise to ON readings but not to IN readings. (\sectref{sect:type_negation}.)
	\ex\label{property5} They are infelicitous as indirect reproaches, offers and requests. (\sectref{sect:exp_bias_data}.)
	\ex\label{property2} They are only felicitous in contexts where there is an unresolved question Q in the context such that $p$ counts as a term answer to Q. (\sectref{sect:unresolved}.)
	\ex\label{property3} The conjunction of two \textit{nem-e \textsc{vm}-V}  interrogatives is infelicitous. (\sectref{sect:complete}.) 
	\ex\label{property6} Replies consisting of the isolated response particles \textit{igen} `yes', \textit{de} `but' or \textit{nem} `no' are dispreferred or infelicitous. (\sectref{sect:possible_answers}.)
	\ex\label{property4} They do not give rise to rhetorical question readings. (\sectref{sect:no_add_comp}.) 
	\z
	\z
	
	\noindent The fact that \textit{nem-e} interrogatives are compatible with PPIs, which was used as a diagnostic for ON-readings (property \xref{property1}), follows from analysing the constituent following \textit{nem-e}  as an embedded  positive declarative (which is compatible with PPIs as a default). The incompatibility with NPIs,  which indicates the absence of IN-readings, follows from the general incompatibility of the  interrogative particle \textit{-e}  in the matrix clause with NPIs, as shown in \sectref{sect:forms}.\footnote{Cf.~\citet{gaertner+gyuris2022} for an analysis of the ban on IN readings of \textit{-e}-interrogatives as a syntactic intervention effect.}
	Note that NPIs are still felicitous in \textit{nem-e} interrogatives in case the ``embedded'' declarative contains another negation, which licenses NPIs, as in \xref{ex:nem-e-nem}.
	
	The remaining properties \xxref{property5}{property4} are going to be accounted for by referring to the covert structure shown in  \xref{ex:covert} above,  in which the embedded declarative (CP$_2$) is interpreted as exhaustively focused, a result of  the correlate (placeholding  expletive) \textit{az} being situated in the preverbal focus position of the  matrix clause (CP$_1$). 
	
	For describing the  felicity conditions of \textit{nem-e \textsc{vm}-V} interrogatives in the  discourse we rely on insights from the Question Under Discussion (QUD) framework. (For general discussion, cf.~ \citealt{roberts2012} and \citealt{buring2003}, a.o..)  Here  the (explicit or implicit) question that an utterance (of a  declarative or interrogative) is assumed to  react to is referred to as the \textit{I(mmediate)} QUD, modelled as a set of propositions. For  utterances of declaratives, the  IQUD  is constituted by a (contextually determined) subset of the set of focus alternatives to the declarative. (Cf. \citealt{rooth1992, beaver+clark}, a.o..) As far as the IQUD of utterances of polar interrogatives with (exhaustive) focus is concerned, we follow \citet[26--27]{kamali+krifka}, who take it  to be a contextually determined subset of the set of focus alternatives introduced by the declarative encoding the positive answer. As an illustration, consider \xref{ex:ali},  a polar interrogative with a constituent in the preverbal focus position (followed by \textit{\textsc{vm}-V} inversion). The set of focus alternatives to the positive answer is shown in \xref{ex:ali_alt}. Assuming that the set of contextually relevant individuals contains Anna and Béla, the denotation of \xref{ex:ali_qud} is a  subset of the set of alternatives in \xref{ex:ali_alt}, and thus \xref{ex:ali_qud} denotes  an appropriate IQUD for \xref{ex:ali}.
	
	\ea\label{ex:ali+ali_answer}
	\ea{\gll Anna$_F$ ment el moziba?\\
		Anna went \textsc{vm} movies.into\\
		\glt `Was it Anna who went to the movies?'}\label{ex:ali}
	\ex{\{`It was Anna who went to the movies', `It was Béla who went to the movies', `It was Cili who went to the movies', `It was Anna and Béla who went to the movies', `It was Anna and Cili who went to the movies', `It was Anna, Béla and Cili who went to the movies', \dots { }\}}\label{ex:ali_alt}
	\ex{\gll (Anna és Béla közül) ki ment el a moziba?\\
		Anna and Béla among who went \textsc{vm} the movies.into\\
		\glt `Who (of Anna and Béla) went to the movies?'}\label{ex:ali_qud}
	\z
	\z
	
	\noindent The fact that the focus in the polar interrogative  in \xref{ex:ali} has an exhaustive/identificational reading  entails that the focus alternatives in \xref{ex:ali_alt} exclude each other, and thus that the  positive answer to \xref{ex:ali}  provides a complete answer to the IQUD in \xref{ex:ali_qud}. 
	
	
	Now the assumptions above will be applied to \textit{nem-e \textsc{vm}-V} interrogatives. For simplicity, here we only consider  \textit{nem-e \textsc{vm}-V} interrogatives without a preverbal focus constituent in the ``embedded'' CP$_2$, as in  \xref{ex:covert}.
	We suggest that the focussing of the whole  CP$_2$ indicates that its proposition denotation constitutes a  \textit{term answer} to the IQUD. 
	The fact that CP$_2$ is  exhaustively focused means that the propositions in the set constituting the alternatives to its denotation mutually exclude each other as term answers to the IQUD (and thus that all constitute  maximally informative answers to the latter, cf.~\citealt{dayal1996}). Thus, the IQUD for any \textit{nem-e \textsc{vm}-V} interrogative is to be schematically represented as the set of propositions shown in \xref{ex:QUD-den}, where $p$, $p'$, and $p''$ stand for the denotation of CP$_2$ and its alternatives, $\mathcal{P}$ denotes a property of propositions determined by the specific IQUD (to be discussed below), and exactly one of the propositions in the set can be true at the same time. This means, informally,  that a \textit{nem-e \textsc{vm}-V} interrogative is used for the purpose of asking whether  the denotation of CP$_2$ is equivalent to the only proposition that has the contextually given  property $\mathcal{P}$. 
	
	\ea\label{ex:QUD-den} \{`It is $p$ that has property $\mathcal{P}$', `It is $p'$ that has property $\mathcal{P}$', `It is $p''$ that has property $\mathcal{P}$', \dots\}
	\z
	
	
	Regarding how IQUDs of the form illustrated in \xref{ex:QUD-den}, with mutually exclusive possible answers, are realized in natural language, empirical observations suggest that there are at least two \textit{wh}-interrogative forms available for this purpose. The first one is \textit{why}-interrogatives asking for reasons, which, according to \citet[152]{oshima2007} are presupposed to have ``only one true resolution''  in a particular context.\footnote{It depends on contextual factors ``what counts as a reason'' (\citealt[155]{oshima2007}). Cf.~also \citet{unger1977}.} This means that all possible answers to \textit{why}-questions are also complete. \xref{ex:sugg1}, repeated in \xref{ex:sugg2}, provides an illustration:
	
	\ea\label{ex:sugg2}  
	\textit{Suggestion scenario}\\
	A, B and János are colleagues. A and B talk after a meeting.\\
	\ea A: 
	\gll Miért nem volt János az értekezleten?\\
    why not was János the meeting.on\\
	\glt \hspace{0.4cm}`Why wasn't János at the meeting?'
	\ex B:  \gll Nem-e  ki-utazott Berlinbe?\\
    not-\textsc{q} \textsc{vm}-travelled Berlin.into\\
	\glt \hspace{0.4cm} `Isn't it that he went to Berlin?' 
	\z
	\z
	
	The property of having mutually exclusive possible answers does not apply to the question realized by the  \textit{wh}-interrogative in \xref{ex:saida}, which, however, can denote the IQUD for the two polar negative interrogative form types in \xxref{ex:saidb}{ex:saidc}, but not for the one in \xref{ex:said-ans}.
	
	\ea\label{ex:said}  
	A, B and János are colleagues. A and B talk after a meeting.\\
	\ea\label{ex:saida}
	\gll A: Mit mondott János Mariról?\\
	{} what.\textsc{acc} said János Mari.from\\
	\glt \hspace{0.4cm}`What did János say about Mari?'
	\ex\label{ex:said-ans} B: \#\gll Nem-e  ki-utazott Berlinbe?\\
    not-\textsc{q} \textsc{vm}-travelled Berlin.into\\
	\glt \hspace{0.4cm}`Isn't it that she went to Berlin?' 
	\ex\label{ex:saidb}
	\gll B: Nem azt-e, hogy  ki-utazott Berlinbe?\\
	\hspace{0.5cm} not that.\textsc{acc}-\textsc{q} that \textsc{vm}-travelled Berlin.into\\ 
	\glt\hspace{0.5cm} `Wasn't it that she went to Berlin?'
	\ex\label{ex:saidc} 
	\gll B: Nem azt, hogy  ki-utazott Berlinbe $\bigwedge$?\\
	\hspace{0.5cm} not that.\textsc{acc} that \textsc{vm}-travelled Berlin.into (\textsc{q})\\ 
	\glt \hspace{0.5cm} `Wasn't it that she went to Berlin?'
	\z
	\z
	
	
	
	
	\textit{How}-questions on their method reading, which  have  been suggested to have ``determinate complete answers'' (\citealt[3175]{saeboe2016})\footnote{For a discussion of manner readings of \textit{how}-questions, which are more difficult to associate with unique maximal answers, cf.~\citet{oshima2007, abrusan2011, saeboe2016, schwarz+simonenko}, a.o..} also illustrate a form type denoting appropriate IQUDs for questions realized by \textit{nem-e \textsc{vm}-V} interrogatives, as shown in \xref{ex:rowed} (inspired by (7) in \citealt{saeboe2016}):  
	
	\ea\label{ex:rowed}
	\ea \gll A: Hogy került ez ide?\\
	{} how got this here\\
	\glt \hspace{0.4cm}`How did this guy get here?'
	\ex \gll B: Nem-e ide-evezett a sziget másik oldaláról?\\
	{} nem-\textsc{q} \textsc{vm}-rowed the island other side.from\\
	\glt \hspace{0.4cm}`Isn't it that he rowed here from the other side of the island?'
	\z
	\z
	
	
	Before turning to the formal details of our analysis, we take a brief look at two 
	types of constructions discussed in the literature that resemble \textit{nem-e \textsc{vm}-V} interrogatives in terms of formal features, including focus-marking, and the structure of discourses they appear in. 
	
	
\subsection{Cross-linguistic analogues of \textit{nem-e \textsc{vm}-V} interrogatives}
	
	\citet{sheil16} studies  the  Scottish Gaelic \textit{propositional cleft} (PC), which signals that the clause is not divided into a background and a focus part, because the whole is marked as a ``broad-sized identificational focus''. She argues that declaratives realizing the  PC construction are only felicitous in a context where there is an explicitly given ``super-question of the Immediate QUD'' (\citealt{sheil16}: 35),  the latter being a polar question. The contribution of the PC is then to signal ``a revision to the line of inquiry (an alternative strategy to answering a super-question)'' (\citeyear{sheil16}: 26). 
	\citet[4]{sheil16} also shows, however, that the PC is ungrammatical in interrogatives   and cannot be negated, which makes a detailed comparison with \textit{nem-e} interrogatives difficult. The author argues, in addition, that in spite of superficial similarities, the PC cannot be reduced to the English \textit{it is that} construction (to be discussed below), since it does not share the explanatory or interpretive function of the latter. For example, in the context illustrated in \xref{ex:SG}, where no explanation is asked for, only the PC but not the \textit{it is that} construction is felicitous:
	
	
	\ea\label{ex:SG}
	(Roddy wants to marry a girl, and she insists on him buying her a ring. He gets
	one from a gypsy tinker, and he and the girl agree to marry after the fisheries.)\\
	'S a cheud oidhche bha dannsa aca ann an taighean Gordon,\\
	`And the first night they had a dance in Gordon's houses,'\\
	\gll 's ann a thuit na clachan as an fh\gravis{a}inne.\\
	\textsc{cop} in.3\textsc{msg} \textsc{c.rel} fall.\textsc{past} 
	the.\textsc{pl}  stone.\textsc{pl} out.of the ring\\
	\glt `(\# it's that) the stones fell out of the ring.'\hspace{1.8cm} (\citealt[5, (1.6)]{sheil16})
	\z
	
	\noindent In addition, the PC is also shown by Sheil not to be reducible to  the verum focus construction (the sizes of the focused elements being different), or to sentence focus (which is an information focus, rather than an identificational one). 
		
	The second construction that resembles \textit{nem-e} interrogatives as far as structure and discourse function is concerned is the \textit{it is that} construction in English, illustrated in \xref{ex:uther}:
	
	\ea\label{ex:uther} Not that Uther was ever unkind to me; it was simply that he had no
	particular interest in a girl child. \hspace{3.2cm}(\citealt[11]{delahunty1990})
	\z
	
	\noindent \citet[20]{delahunty1990} considers the \textit{it is that} construction an  ``inferential'': ``the form can be viewed as a pragmatic instruction to its audience to infer a relationship between the construction and its context that goes beyond
	the mere addition of the information conventionally denoted by the clause.''\footnote{More precisely, the relevant connection holds rather between the propositional content of the inferential construction and contextually given material, cf.~\citet[48]{remberger2020}.} In various works, Delahunty suggests that the proposition denoted by the finite clause within the inferential may be interpreted as an implicated premise (explanation, reason, cause) or conclusion (result, consequence, conclusion), or ``taken as a (re)interpretation or reformulation of the target utterance'' (\citealt[301]{delahunty+gatzkiewicz}), the exact choice depending only on the context.
	
	\citet[205]{declerck1992} argues, however, that ``the typical aspects of meaning associated with the inferential \textit{it is that} construction are not of a pragmatic nature but follow from the fact that the
	construction belongs to the class of copular sentences that are `specificational' ''.\footnote{\citet{declerck1992} uses the term `specificational' in the sense of specifying a value for a variable, cf. \citet{higgins1976}, \citet{akmajian1979}, and \citet{Declerck1979}.} This accounts for the fact that the inferential construction is not interpretable in isolation (and unlikely to occur discourse-initially), and  that it gives rise to an ``exhaustiveness implicature'' (\citealt{declerck1992}: 213).\footnote{``When a value, or set of values, is specified for a variable, the hearer has a right to conclude that the listing of values is exhaustive.'' \citep[213]{declerck1992}} 
	Declerck claims that the construction involves two kinds of inferences. 
    \begin{quote}
    ``First, the \textit{that}-clause expresses what the speaker infers to be the correct explanation or interpretation of a situation or speech act. [\dots] Second, in order to understand the sentence the hearer has to infer the variable for which the \textit{that}-clause is presented as value'' (\citealt{declerck1992}:212). 
    \end{quote}
    Importantly, it follows that a negative \textit{it is that}-sentence, as in  \xref{ex:treachery}, does not reject the truth of the \textit{that}-clause but denies that ``this inference is the one that satisfies the variable'' (p.~216):
	
	\ea\label{ex:treachery} It is not that one fears treachery, though of course one does. \\\ \hspace{7cm}(\citealt[216, (22a)]{declerck1992})
	\z
	
		
	\citet{declerck1992} accounts for the distribution of the inferential \textit{it is that} construction by suggesting that 
    \begin{quote}
    ``the only variables that can take a \textit{that}-clause as value are those that lexicalize as nouns that can appear in the copular structure `NP is that'. Nouns like \textit{reason}, \textit{cause}, \textit{explanation}, \textit{interpretation} are of this type: they express a notion whose contents can be specified by a \textit{that}-clause.'' (\citealt{declerck1992}: 220).
    \end{quote}
    He emphasizes that ``[o]ther nouns that have to do with some aspect of a situation (e.g.~time, place) do not share this characteristic'', adding that ``since the value that is specified by an inferential has the form of a \textit{that}-clause, the variable that is inferred is automatically taken from the group `reason / cause / explanation / interpretation' '' (\citealt{declerck1992}: 220). 
	
	This short review  thus shows that  the Scottish Gaelic PC and the inferential construction 
	share with \textit{nem-e \textsc{vm}-V} interrogatives 
	the property of marking  a  proposition as  exhaustively focused,  constituting a term answer to a superordinate question.\footnote{Cf.~\citet{remberger2020} for discussion of the focus-background structure of the latter.} In the next section we turn to issues of how to formally model these interpretational features.
	
	
\subsection{Interpreting  \textit{nem-e \textsc{vm}-V} interrogatives: Formalization}\label{sect:formalization}

	In this section we are going to propose an account of the use conditions of  \textit{nem-e \textsc{vm}-V} interrogatives, which relies on the compositional semantic interpretation of the (partly covert) structure associated with \textit{nem-e \textsc{vm}-V} interrogatives, shown in \xref{ex:covert} above, and repeated here as \xref{ex:covert1}.  
	
	

	\ea\label{ex:covert1}
	\gll [$_{\text{CP}_1}$ \dots {} [\textsubscript{NegP} Nem [\textsubscript{FocP} \sout{az}  \sout{van}-e   [$_{\text{IP}_1}$ \dots { } \hspace{4cm} [$_{\text{CP}_2}$  \sout{hogy} [$_{\text{IP}_2}$ ki-utazott {Berlinbe ]]]] \dots ]}\\
	{} {} {} not {} that be.3\textsc{sg}-\textsc{q}  {} {} {} {} that {} \textsc{vm}-travelled Berlin.into\\
	\glt `Isn't it that he went to Berlin?'
	\z
	
	
	The formal derivation of the meaning of \xref{ex:covert1}, the outlines of which are presented in \xref{ex:denotation_covert}, relies on the following assumptions. First,  the semantic value of polar interrogatives consists of the proposition corresponding to the positive answer and the latter's negation  (cf.~\citealt{hamblin73}). Second,  \xref{ex:covert1} is a construction with exhaustive focus: the proposition denoted by the ``embedded declarative'' CP$_2$ is the only one among its alternatives  that possesses a particular property,  referred to as $\mathcal{P}$. We assume that $\mathcal{P}$ is introduced by the covert copula \textit{van}, and that it is a  property of propositions, which is  contributed by the context, more precisely, by the  IQUD. We also assume that \textit{-e} is responsible for marking the interrogative sentence type, and thus it is interpreted on the periphery (TypeP or ForceP). We assign it a denotation analogous to the polar-question operator in \citet[50]{hamblin73}, along the lines suggested in \citet[14]{uegaki2018}.
	
	\ea\label{ex:denotation_covert}
	\ea\label{ex:CP} 
	$\llbracket$ CP$_2$ $\rrbracket$ = $\lambda w'.$ $went(j,b,w')$
	\ex\label{ex:van}
	$\llbracket$ \textit{van} $\rrbracket$ =  $\lambda r.$ $\lambda w.$ $\mathcal{P}(r)(w)$ 
	\ex\label{ex:FocP}
	$\llbracket$ FocP $\rrbracket$ = $\lambda w.$ $\mathcal{P}(\lambda w'. went(j,b,w'))(w)$ \\ \hspace{1.7cm}$ \wedge$ $ \forall q$ $[q \neq \lambda w'. went(j,b,w')   \rightarrow \neg\mathcal{P}(q)(w)$]    
	\ex\label{ex:den-e}
	$\llbracket$ \textit{-e} $\rrbracket$ = $\lambda p.$ \{$p$, $\neg p$\}
	\ex\label{ex:CP1} 
	$\llbracket$ \xref{ex:covert1} $\rrbracket$ =  \{ $\lambda w.$ $\mathcal{P}(\lambda w'. went(j,b,w'))(w)$ \\
    \hspace{2cm}$ \wedge$ $ \forall q$ $[q \neq \lambda w'. went(j,b,w')   \rightarrow \neg\mathcal{P}(q)(w)$], \\
    \hspace{1.6cm} $ \lambda w.$ $\neg \mathcal{P}(\lambda w'. went(j,b,w'))(w)$ \\
	\hspace{1.9cm} $ \wedge$ $ \forall q$ $[q \neq \lambda w'. went(j,b,w')   \rightarrow \neg\mathcal{P}(q)(w)$]   \}  
	\z
	\z
	
	
	
	\noindent The denotation of CP$_2$ in \xref{ex:CP} is a proposition, identical to the one  denoted by the declarative in \xref{ex:kiutazott-decl}:
	
	\ea\label{ex:kiutazott-decl}
	\gll [\textsubscript{CP} \dots { }[\textsubscript{IP} Ki-utazott Berlinbe ] \dots { }]\\
	{} {}    \textsc{vm}-travelled Berlin.into\\
	\glt `He went to Berlin.'
	\z
	
	
	\noindent \xref{ex:van} shows the denotation of the covert copula \textit{van}, which  takes a proposition as argument and associates with it has a contextually given property $\mathcal{P}$. 
	\xref{ex:FocP} presents the denotation of FocP as a proposition that is true in those possible worlds where `John went to Berlin' has  property $\mathcal{P}$ but no other proposition does.  
	It relies on standard assumptions about the preverbal focus constituent in Hungarian (cf.~\citealt{szabolcsi94}), according to which it is associated with an exhaustive reading.
	\xref{ex:den-e} assigns \textit{-e} the contribution of turning the proposition-denotation of its sister node into a set of propositions consisting of the latter proposition and its negation, discussed above. Additionally, we assume that \textit{nem} contributes non-propositional (outside) negation, which makes a vacuous contribution to the truth conditions. The denotation of the whole structure in \xref{ex:covert1} is shown  in \xref{ex:CP1}. \xref{ex:CP1} makes it clear that the positive answer to a question encoded by a \textit{nem-e \textsc{vm}-V} interrogative is not equivalent to the denotation $p$ of CP$_2$, but to  a proposition that identifies $p$ with the unique proposition that has  property $\mathcal{P}$.
	
	Now we turn to how these proposals can account for  the remaining  properties in \xref{properties}. Property \xref{property2}, according to which \textit{nem-e \textsc{vm}-V} interrogatives are only felicitous in contexts where there is an unresolved
	question Q such that the denotation of  CP$_2$ counts as a term answer to Q, directly follows, given the assumptions of the QUD approach,  from the claim that in the structure under consideration,  CP$_2$ is (exhaustively) focused, and thus gives rise to the IQUD shown in \xref{ex:QUD-den}, repeated in \xref{ex:QUD-den1} below:
	
	\ea\label{ex:QUD-den1} \{`It is $p$ that has property $\mathcal{P}$', `It is $p'$ that has property $\mathcal{P}$', `It is $p''$ that has property $\mathcal{P}$', \dots\}
	\z
	
	The unresolved question Q referred to in \xref{property2} thus corresponds to the appropriate IQUD. 

	Next, we turn to property \xref{property5}: as opposed to the two other negative polar interrogative form types in Hungarian, \textit{nem-e \textsc{vm}-V} interrogatives are infelicitous as indirect reproaches, offers or requests. This was shown in \sectref{sect:exp_bias_data}. For the sake of succinctness, I will sketch an account for the first case, repeated in \xref{ex:mother1}, which is then taken to apply to the other cases \textit{mutatis mutandis}. (The original English  translation given in \xref{ex:mothera} has been replaced by a cleft construction, for reasons discussed in \sectref{sect:structural}.)
	
	
	\ea\label{ex:mother1}
	\textit{Reproach scenario} (repeated with new translation)\\
	Mother sees her Child kick another child in the sandpit. Mother says to Child:
	\ea[\#]{\gll Nem-e 	sz\'egyelled	magad?\\
		not\textsc{-q}	be.ashamed.\textsc{2sg}	yourself\\
		\glt `Isn't it that you are ashamed?'}\label{ex:mothera1}
	\ex[]{Nem sz\'egyelled-e magad? \\
		`Aren't you ashamed?'}\label{ex:motherb1}
	\ex[]{Nem sz\'egyelled magad  $\bigwedge$ ? \\
		`Aren't you ashamed?'}\label{ex:motherc1}
	\z
	\z
	
	A very simple line of explanation can be based on the fact that the reproach in \xref{ex:mother}/\xref{ex:mother1} is an initiating speech act. This is incompatible with uses of \textit{nem-e \textsc{vm}-V}  interrogatives, which, as we have seen above, must be reactive to an IQUD.
	
	A more specific ‐- and stronger -- account emerges if we try to spell out such an IQUD for \xref{ex:mothera}/\xref{ex:mothera1}. Among \textit{why}‐questions, which we limit our
	discussion to here, the one fitting the context of \xref{ex:mother1} most naturally
	would be \xref{ex:why-kicking}.
	
	
	\ea\label{ex:why-kicking} Why are you kicking your friend?
	\z
	
	\noindent And, if we now take inspiration from \citet{delahunty1990}/\citet{declerck1992} and allow ourselves the paraphrase of \xref{ex:mothera}/\xref{ex:mothera1} in \xref{ex:the-reason}, we derive a double infelicity.
	
	\ea\label{ex:the-reason} Isn't the reason for your kicking your friend that you are ashamed?
	\z
	
	\noindent \xref{ex:the-reason} -- and thus \xref{ex:mothera}/\xref{ex:mothera1} -- fails on the basic content‐level, given that being ashamed would be a fairly odd reason for the child to kick his
	friend. And, what's more, if being ashamed is suggested by the mother as a reason for bad behaviour, this cannot be construed as her ``endorsing'' (cf. \citealt{silk2020}) such a feeling to the child, i.e., her epistemic bias cannot be reinterpreted (e.g., via relevance implicature) as deontic bias (``you should be ashamed'').
	
	Note that things are different with the \textit{Suggestion scenario}, \xref{ex:sugg}, and the reinterpretations of \textit{nem-e \textsc{vm}-V} interrogatives in \xref{ex:mother}--\xref{ex:request}  as guesses (\sectref{sect:exp_bias_data}). All of these are limited to epistemic speaker expectation bias, with the first one already containing -- and the others
	in need of accommodating -- an appropriate ``\textit{why}''‐based IQUD.\footnote{An additional possibility that we leave unexplored is that \textit{nem-e \textsc{vm}-V} interrogatives license ``higher‐order'' deontic expectation bias, schematically expressed by \textit{The reason for \dots} \textit{should be that \dots}}\footnote{I thank Hans-Martin Gärtner for discussions on the reasons for the absence of non-epistemic expectation biases for \textit{nem-e \textsc{vm}-V}  interrogatives.}
	
	
	Next, according to property \xref{property3}, the conjunction of two \textit{nem\nobreakdash-e \textsc{vm}-V} interrogatives is infelicitous. This follows from the fact that the answers to two conjoined  \textit{nem-e \textsc{vm}-V} interrogatives, given that they share the same IQUD, cannot be independent of each other. As \xref{ex:denotation_covert} shows, the truth of a positive answer to one \textit{nem-e \textsc{vm}-V} interrogative entails that no other   interrogative having the same IQUD but a different CP$_2$ can be answered positively. 

	
	Now, consider \xref{property4}: \textit{nem-e \textsc{vm}-V} interrogatives, as opposed to other negative interrogatives in Hungarian, cannot be used to signal that the proposition $p$ denoted by   CP$_2$ is  in the common ground according to the speaker. This property  was referred to  as the ``absence of a rhetorical question reading'' above. One of the central claims  of the account proposed here is that  the set of possible answers to questions encoded by \textit{nem-e \textsc{vm}-V} interrogatives is not identical to the denotation of the constituent (CP$_2$) following \textit{nem-e}  and its negation (\{$p$, $\neg p$\}). Rather, due to the focusing of  CP$_2$ in the covert structure the denotation of the polar interrogative under consideration is equivalent to the set of propositions \{`{It is $p$ that has property $\mathcal{P}$}', `{It is not $p$ that has property $\mathcal{P}$}'\}. This means that even if  $p$ is in the common ground, a  \textit{nem-e \textsc{vm}-V} interrogative where  CP$_2$ denotes  $p$ is  not an appropriate means of pointing this out,  since $p$ is not a congruent answer to the latter.
	
	Finally, we turn to property \xref{property6}: the isolated response particles \textit{igen} `yes', \textit{de} `but' or \textit{nem} `no' are dispreferred or even infelicitous as replies to questions realized by \textit{nem-e \textsc{vm}-V} interrogatives. I suggest that the reason, again,  has to do with  the non-identity between the set of agreeing and disagreeing answers to \textit{nem-e \textsc{vm}-V} interrogatives and the set \{$p$, $\neg p$\}, where $p$ equals the denotation of  CP$_2$. This  non-identity makes it difficult (or even impossible) to reconstruct the propositional content of the actual answer on the basis of a response particle alone.
	
	
	
\section{Conclusion}\label{sect:conclusion}

	This paper investigated the use conditions and the bias profile of the  negative interrogative form type in Hungarian that contains the surface constituent \textit{nem-e}, composed of a negative and an interrogative particle. It was pointed out for the first time that different occurences  of \textit{nem}\nobreakdash-\textit{e} may represent different uses and belong to different dialects. We have provided different analyses to three different uses of \textit{nem}\nobreakdash-\textit{e},  attibuting the latter constituent  a biclausal struture on the first one, considering it  a result of movement according to the second one, and analyzing it as  a particle on the third one, respectively.  Concentraing on \textit{nem-e} interrogatives used by speakers of the Standard Dialect, we have pointed out that they only have ON but no IN readings, they are incompatible with (standard) non-epistemic speaker expectation biases, they cannot be interpreted as indirect reproaches, offers or requests, they are only felicitous in contexts with an unresolved superordinate question under discussion, cannot be coordinated, do not have rhetorical question readings, and  cannot felicitously be responded to by isolated response particles.  It was argued that these properties follow from the focus-background structure of the interrogative, which involves an (exhaustively) focused declarative clause subordinated to a covert matrix clause.
	
	
\section*{Acknowledgements}
	I wish to thank the two anonymous reviewers of the paper and the editors of this volume for advice and criticism, Hans-Martin Gärtner, Angelika Kiss, Rebeka Kubitsch, Cecília Sarolta Molnár, Ádám Nádasdy, Eva-Maria Remberger, Lilla Kamilla Sándor, Marianna Varga, and Viktória Virovec for  discussions and suggestions, and the audience of  the SPAGAD workshop \textit{Biased Questions: Experimental Results \& Theoretical Modelling}, as well as audiences at Edinburgh, Newcastle, Vienna and Szeged for valuable comments. Research for the paper was supported by the Hungarian National Research, Development and Innovation Fund (NKFIH), under grant No. K 135038.
	
	
\printbibliography[heading=subbibliography,notkeyword=this]
	
\end{document}
