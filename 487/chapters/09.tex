\documentclass[output=paper,colorlinks,citecolor=brown]{langscibook}
\ChapterDOI{10.5281/zenodo.17158190}

\author{Tue Trinh\orcid{0000-0002-6362-0974}\affiliation{Leibniz-Zentrum Allgemeine Sprachwissenschaft}}
\title{A note on bias and polarity in Vietnamese}
\abstract{Vietnamese has two types of NPIs, simple and complex, and two types of polar questions, yes/no questions and agreement questions. Simple NPIs can occur in both types of polar questions while complex NPIs can occur in yes/no but not in agreement questions. I propose an account for this fact using familiar ingredients of semantic and syntactic analyses. I then discuss some ways in which Vietnamese and English differ with respect to how distinctions in meaning align with distinctions in form.}

%move the following commands to the "local..." files of the master project when integrating this chapter

\IfFileExists{../localcommands.tex}{
   \addbibresource{../localbibliography.bib}
   % add all extra packages you need to load to this file

\usepackage{tabularx,multicol}
\usepackage{url}
\urlstyle{same}

\usepackage{listings}
\lstset{basicstyle=\ttfamily,tabsize=2,breaklines=true}

\usepackage{langsci-basic}
\usepackage{langsci-optional}
\usepackage{langsci-lgr}
\usepackage{langsci-osl}
% \usepackage{./langsci/styles/langsci-lgr}
% \usepackage{./langsci/styles/langsci-osl}
% \usepackage{langsci-gb4e}

\usepackage{tikz}
\usetikzlibrary{patterns,calc}
\pgfdeclarepatternformonly{south east lines}{\pgfqpoint{-0pt}{-0pt}}{\pgfqpoint{3pt}{3pt}}{\pgfqpoint{3pt}{3pt}}{
    \pgfsetlinewidth{0.6pt}
    \pgfpathmoveto{\pgfqpoint{0pt}{3pt}}
    \pgfpathlineto{\pgfqpoint{3pt}{0pt}}
    \pgfpathmoveto{\pgfqpoint{.2pt}{-.2pt}}
    \pgfpathlineto{\pgfqpoint{-.2pt}{.2pt}}
    \pgfpathmoveto{\pgfqpoint{3.2pt}{2.8pt}}
    \pgfpathlineto{\pgfqpoint{2.8pt}{3.2pt}}
    \pgfusepath{stroke}}
    
\usepackage{stmaryrd}
\usepackage{wasysym}
\usepackage{multirow}
\usepackage{caption}
\usepackage{subcaption}
\usepackage{mathrsfs}
\usepackage{qtree}

\usepackage{linguex}


   %pminos do not split footnotes
% \interfootnotelinepenalty=10000 %Footnote in Laporte chapters has to be split SN


%\DeclareIndexNameFormat{default}{%
%\nameparts{#1}%
%\usebibmacro{index:name}%
%{\index[names]}%
%{\namepartfamily}%
%{\namepartgiveni}%
% {}% L1
% {}% L2
%{\namepartprefix}% generates spurious space L3
%{\namepartsuffix}% generates spurious space L4
%}

%  {\DeclareIndexNameFormat{default}{%
%     \usebibmacro{index:name}{\index[names]}{#1}{#3}{#5}{#7}}}

%\DeclareIndexNameFormat{default}{%
%  \usebibmacro{index:name}{\sindex[nom]}{#1}{#3}{#5}{#7}}

%\DeclareIndexNameFormat{default}{%
%  \usebibmacro{index:name}{\sindex[person]}{#1}{#3}{#5}{#7}}
%\DeclareIndexNameFormat{default}{%
%\nameparts{#1} \usebibmacro{index:name}{\sindex[person]]}{\namepartfamily}{‌​\namepartgiven}{\nam‌​epartprefix}{\namepa‌​rtsuffix}}

%\newcommand{\smiley}{:)}

%\renewbibmacro*{index:name}[5]{%
%\usebibmacro{index:entry}{#1}%
%{\iffieldundef{usera}{}{\thefield{usera}\actualoperator}\mkbibindexname{#2}{#3}{#4}{#5}}}

% \newcommand{\noop}[1]{}

%remove for final
%\overfullrule=1mm

\newcommand{\tobi}[2]}}
\renewcommand{\S}[1]{\tobi{#1}{\textsc{*}}}

% this volume references
% puts: [this volume]
% already defined: \citetv
%\newcommand{\citepv}[1]{(\citeauthor{#1} \citeyear*{#1} [this volume])}
\newcommand{\citealtv}[1]{\citeauthor{#1} \citeyear*{#1} [this volume]}

%parentheses around example number
\newcommand{\pref}[1]{(\ref{#1})}

% in-text examples

\newcommand{\lnex}[1]{\textit{#1}} %target lang word
\newcommand{\lnlit}[1]{(lit.: `#1')} %literal reading
\newcommand{\lnlat}[1]{(#1)} % latinization
\newcommand{\lntrans}[1]{`#1'} %translation
\newcommand{\lnexl}[2]%
{\lnex{#1}{} \lnlat{#2}} % ex with latinization
\newcommand{\lnexlat}[3]{\lnex{#1}{} \lnlat{#2}{} \lntrans{#3}} % ex with latinization and tranl.

%ch01
\newcommand{\co}[1]{\mbox{\textbf{#1}}}

%ch09

\newcommand{\cyrbulg}[1]{\begin{otherlanguage*}{bulgarian}#1\end{otherlanguage*}}


%ch10
\newcommand{\nlp}{{\small NLP}}
\newcommand{\mwe}{{\small MWE}}
\newcommand{\rae}{{\small RAE}}
\newcommand{\lvc}{{\small LVC}}
\newcommand{\pos}{{\small P}o{\small S}}
%\newcommand{\todo}[1]{ \textcolor{red}{#1} }

%\renewcommand{\labelenumi}{\theenumi}
%\ainamefmt{{vv}{ll}{, ff}{, jj}} % fullname

\newcommand{\biberror}[1]{{\color{red}#1}}

\newcommand{\osenovaitem}{--~}
   %% hyphenation points for line breaks
%% Normally, automatic hyphenation in LaTeX is very good
%% If a word is mis-hyphenated, add it to this file
%%
%% add information to TeX file before \begin{document} with:
%% %% hyphenation points for line breaks
%% Normally, automatic hyphenation in LaTeX is very good
%% If a word is mis-hyphenated, add it to this file
%%
%% add information to TeX file before \begin{document} with:
%% %% hyphenation points for line breaks
%% Normally, automatic hyphenation in LaTeX is very good
%% If a word is mis-hyphenated, add it to this file
%%
%% add information to TeX file before \begin{document} with:
%% \include{localhyphenation}
\hyphenation{
    Beck-man
    Ngu-yen
    back-chan-nel
    back-chan-nels
    mo-not-o-nous
    ste-reo-typ-i-cal
}

\hyphenation{
    Beck-man
    Ngu-yen
    back-chan-nel
    back-chan-nels
    mo-not-o-nous
    ste-reo-typ-i-cal
}

\hyphenation{
    Beck-man
    Ngu-yen
    back-chan-nel
    back-chan-nels
    mo-not-o-nous
    ste-reo-typ-i-cal
}

   \boolfalse{bookcompile}
   \togglepaper[23]%%chapternumber
}{}

\begin{document}
\maketitle


\section{Observations \label{observation} }

This section describes the differences with respect to distribution and interpretation between two types of NPIs across two types of polar questions in Vietnamese.
%\subsection{Two types of NPI}

\subsection{Two types of polar questions \label{polar} }

I will use the term \textit{polar questions} to describe questions which ask for the truth value of a single proposition. In other words, polar questions are those which are answered felicitously by assertion of a proposition or assertion of its negation. %In English, polar questions take two forms, exemplified in \ref{subjaux}, which involves subject auxiliary inversion, and \ref{decla}, which exhibits declarative word order and often pronounced with rising intonation.
%\ex.
%\a. Is John married?\label{subjaux}
%\b. John is married?\label{decla}
%There are, of course, differences between \ref{subjaux} and \ref{decl}, but one thing they do have in common is that 
Vietnamese has two variants of polar questions. The first, which I will call \textit{yes/no questions}, involves bracketing the predicate of the sentence with the words \textit{có} and \textit{không} \citep{trinh2005aspects, Duffield:2007}. I will gloss \textit{có} and \textit{không} as \textsc{pos} and \textsc{neg}, for reasons which will be clear shortly.%These words are also used as the positive and the negative answer particles, i.e. as the counterparts of English ``yes'' and ``no'', respectively



%\ex.
%\exg.%[A:]
\ea
	\ea
	\gll John đọc Kant\\
	John read Kant\\
	\glt `John reads Kant.'\label{kant}
	\ex
	\gll John có đọc Kant không?\\
	John \textsc{pos} read Kant \textsc{neg}\\
	\glt `Does John read Kant?'\label{cokantkhong}
	\z
\z

I will call \xref{kant} the \textit{prejacent} of \xref{cokantkhong}. More generally, the prejacent of a yes/no question will be (the proposition expressed by) the sentence derived from the question by removing \textsc{pos} and \textsc{neg}. Let us now briefly discuss \textsc{pos} and \textsc{neg} outside the context of a yes/no question. In declaratives, \textsc{pos} and \textsc{neg} are the positive and the negative auxiliary, respectively.\footnote{For arguments that \textsc{pos} and \textsc{neg} are verbal see \citet{trinh2005aspects}. Verbal negation is attested across languages. Finnish is an example \citep{bobaljik1994what}. Note that similarly to English emphatic \textit{do}, \textsc{pos} appears only when the positive sentence bears verum focus. And similarly to English \textit{not}, \textsc{neg} is stressed when the negative sentence bears verum focus.}    %verbs with the former being the affirmative and the latter the negative.

%\begin{multicols}{2}[\columnsep=-4cm]

\ea 
\ea
\gll John có đọc Kant\\
John \textsc{pos} read Kant\\
\glt `John does read Kant.'
\ex
\gll John không đọc Kant\\
John \textsc{neg} read Kant\\
\glt `John does not read Kant.'
\z
\z

These words are also used as the positive and the negative short answer to yes/no questions. The question in \xref{cokantkhong} can be answered with either \textit{có} (\textsc{pos}), which would mean John does read Kant, or \textit{không} (\textsc{neg}), which would mean John does not read Kant. We can analyze these short answers as \xref{coshort} and \xref{khongshort}, which are eliptical sentences with everything but the auxiliary elided \citep{holmberg2016syntax, krifka2013response}.

\ea
\ea \label{coshort}
\sout{Nam} \textsc{pos} \sout{read Kant}
\ex\label{khongshort}
\sout{Nam} \textsc{neg} \sout{read Kant}
\z
\z

%\ex.
%\ag.[A:]
%John có đọc Kant không?\\
%John \textsc{pos} read Kant \textsc{neg}\\
%`Does John read Kant?'

%\vspace{-3pt}

%\begin{multicols}{2}[\columnsep=-1.5cm]
%\ex.[{}]
%\a.[B:]
%Có.\\
%\textsc{pos} (= \sout{Nam} \textsc{pos} \sout{read Kant}) \\
%`He does.' %(= `John does read Kant')
%\b.[C:]
%Không.\\
%\textsc{neg} (= \sout{Nam} \textsc{neg} \sout{read Kant})\\
%`He doesn't.' %(= `John does not read Kant')

%\end{multicols}
%\vspace{-6pt}

The other type of polar questions in Vietnamese is constructed by appending the discourse particle \textit{à} to the end of a declarative sentence \citep{trinh2010asking}.

%\begin{multicols}{2}[\columnsep=-4cm]

\ea
\ea
\gll John đọc Kant à?\\
John reads Kant \textsc{a}\\
\glt `John reads Kant?'\label{agreeread}
\ex
\gll John không đọc Kant à?\\
John \textsc{neg} read Kant \textsc{a}\\
\glt `John doesn't read Kant?'\label{agreenotread}
\z
\z

%\end{multicols}
%\vspace{-6pt}
I will call this type of polar questions \textit{agreement questions}, and the (proposition expressed by) the sentence preceding \textsc{a} the ``prejacent'' of the agreement question. The term ``agreement questions'' is due to the fact that these questions can be described, intuitively, as asking the hearer whether she agrees with the prejacent. Thus, \xref{agreeread} asks whether the hearer agrees that John reads Kant, and \xref{agreenotread}, whether she agrees that John does not read Kant. %The sentence preceding \textsc{a}, i.e. the propositional core of the agreement question, will be called its ``prejacent'', in similar fashion to the case of yes/no questions. The prejacent of \ref{agreeread} is, therefore, the proposition that John reads Kant, and the prejacent of \ref{agreenotread}, the proposition that he does not.

There are two strategies of answering an agreement question. I will call them the \textit{congruent} strategy and the \textit{non-congruent} strategy. The non-congruent strategy consists in answering the agreement question as if it were a yes/no question, which means answering it with either \textsc{pos} or \textsc{neg}. Note that \textsc{pos} expresses a positive and \textsc{neg} expresses a negative sentence independently of whether the prejacent of the agreement question is an positive or a negative sentence. Thus, no matter whether the question is \xref{agreeread} or \xref{agreenotread}, answering it with \textsc{pos} means asserting that John reads Kant, and answering it with \textsc{neg} means asserting that he does not.

%\ex.\label{ynstrategy}
%\ag.[A:]
%John không đọc Kant à?\\
%John \textsc{neg} read Kant \textsc{a}\\
%`John doesn't read Kant?'

%\vspace{-3pt}

%\begin{multicols}{2}[\columnsep=-1.5cm]
%\ex.[{}]
%\a.[B:] 
%Có.\\
%\textsc{pos} (= \sout{Nam} \textsc{pos} \sout{read Kant})\\
%`He does.'
%\b.[C:]
%Không.\\
%\textsc{neg} (= \sout{Nam} \textsc{neg} \sout{read Kant})\\
%`He doesn't.'

%\end{multicols}
%\vspace{-6pt}
 
Recall that an agreement question asks the hearer whether she agrees with the prejacent. The ``congruent'' answering strategy, therefore, should express agreement or disagreement with the prejacent. To express agreement with the prejacent, the response particle \textit{vâng}, which I will gloss as \textsc{arg}, is employed. The closest translation of \textsc{arg} is `that's right', or `that's correct'. Answering \xref{agreeread} with \textsc{arg} means asserting that John reads Kant, and answering \xref{agreenotread} with \textsc{arg} means asserting that he does not, for example. 

%I will gloss \textit{vâng} as \textsc{agree}, and translate it as `that's correct', which expresses its meaning quite accurately.%\newpage

%\begin{multicols}{2}

%\ex. 
%\ag.[A:]
%John đọc Kant à?\\
%John read Kant \textsc{a}\\
%`John reads Kant?'
%\b.[B:] 
%Vâng.\\
%\textsc{agree}\\
%`That's right.' (= `He does')

%\ex. 
%\ag.[A:]
%John không đọc Kant à?\\
%John \textsc{neg} read Kant \textsc{a}\\
%`John doesn't read Kant?'
%\b.[B:] 
%Vâng.\\
%\textsc{agree}\\
%`That's right.' (= `He doesn't')

%\end{multicols}
%\vspace{-6pt}

What if we want to express disagreement with the prejacent? In other words, what is the negative counterpart of \textsc{arg}? It turns out that there is no such word: Vietnamese has a lexical gap. To convey disagreement with the prejacent of an agreement question, we would have to resort to the non-congruent strategy. Suppose the question is \xref{agreeread}, the disagreeing answer would be \textit{không} (\textsc{neg}), which means John does not read Kant. If the question is \xref{agreenotread}, the disagreeing answer would be \textit{có} (\textsc{pos}), which means John does read Kant.   %as exemplified in \ref{ynstrategy}.
%\newpage 
%\begin{multicols}{2}
%\ex. 
%\ag.[A:]
%John đọc Kant à?\\
%John read Kant \textsc{a}\\
%`John reads Kant?'
%\b.[B:] 
%Không.\\
%\textsc{neg}\\
%`He doesn't.'
%\ex. 
%\ag.[A:]
%John không đọc Kant à?\\
%John \textsc{neg} read Kant \textsc{a}\\
%`John doesn't read Kant?'
%\b.[B:] 
%Có.\\
%\textsc{pos}\\
%`He does.' %(= `John doesn't read Kant')
%\end{multicols}
%\vspace{-6pt}
Thus, whereas yes/no questions have a positive and a negative short answer, agreement questions only have a positive short answer.\footnote{Thus, \textit{có} (\textsc{pos}) and \textit{không} (\textsc{neg}) express ``absolute polarity'' while \textit{vâng} expresses ``relative polarity''  in the sense of \citet{roelofsen2015polarity}. For descriptions of similar systems see \citet{holmberg2016syntax, maldonado2023you}.}%\footnote{This fact makes Vietnamese similar to Japanese \citep{}.}

There is, I believe, a possible functional account of this asymmetry. The account will turn on another fact about agreement questions, namely that it is biased towards the positive answer \citep{trinh2010asking}. Suppose the speaker sees John with a copy of \textit{The critique of pure reason} in his hand. In this context, the agreement question in \xref{kanta} is felicitous but the yes/no question in \xref{cokantkhong2} is not.
 
\ea\label{critique1}Context: the speaker sees John with \textit{The critique of pure reason} in his hand.
\ea[]{
\gll John đọc Kant à?\\
John read Kant \textsc{a}\\
}\label{kanta}
\ex[\#]{
\gll John có đọc Kant không?\\
John \textsc{pos} read Kant \textsc{neg}?\\
}\label{cokantkhong2}
\z
\z
%\end{multicols}
%\vspace{-6pt}
In the very same context, the agreement question in \xref{hegela} would be infelicitous, while the yes/no question in \xref{cohegelkhong} would be felicitous.\footnote{Note that if the context is also such that reading Kant entails reading Hegel, \xref{hegela} would be fine. This proves the point I am making. I thank a reviewer for pointing this out.}

\ea
Context: the speaker sees John with \textit{The critique of pure reason} in his hand. \label{critique2}
\ea[\#]{ 
John đọc cả Hegel à?\\
John read also Hegel \textsc{a}}\label{hegela}
\ex[]{
John có đọc cả Hegel không?\\
John \textsc{pos} read also Hegel \textsc{neg}?} \label{cohegelkhong}
\z
\z

%\end{multicols}
%\vspace{-6pt}

We learn two things from \xref{critique1} and \xref{critique2}. First, if there is contextual evidence that $\phi$, a polar question with prejacent $\phi$ is only felicitous when it is formulated as an agreement question. Second, if there is no contextual evidence that $\phi$, a polar question with prejacent $\phi$ is only felicitous when it is formulated as an yes/no question. In other words, yes/no questions require the context to be ``prejacent-neutral'', while agreement questions require it to be ``prejacent-biased''. %We will consider another example to confirm this interpretation of the data. Let context A be one in which the speaker sees John writing with his left hand, and context B be one in which the speaker sees nothing to indicate what Nam's handedness is. The agreement question with prejacent `John is left-handed' is felicitous in context A but not in context B, while the yes/no question with the same prejacent is felicitous on context B but not in context A.
%\begin{multicols}{2}[\columnsep=-4cm]
%\ex. 
%\ag. 
%John {thuận tay trái} à? \\
%John {is left-handed} \textsc{a}? \\
%A / \#B \label{lefta}
%\bg.
%John có {thuận tay trái} không?\\
%John \textsc{pos} {is left-handed} \textsc{neg}?\\
% \#A / B \label{coleftkhong}
%\end{multicols}
%\vspace{-6pt}

I conjecture that the prejacent bias of agreement questions might contribute to the functional pressure on the grammar to have a short answer expressing agreement with the prejacent but no short answer expressing disagreement with the prejacent.\footnote{The idea that answers which agree with the prejacent of the question are preferred by the language system can be found in \citet{roelofsen2015polarity}.} 


\subsection{Two types of NPIs \label{npi} }

Vietnamese is a language that build NPIs from wh-elements. For example, the word \textit{ai}, as an interrogative pronoun, means `who', but as an NPI, means `anyone'. Similarly, \textit{gì} means either `what' or `anything' \citep{Bruening:2006a}. A non-negated declarative sentence would disambiguate such expressions towards the interrogative reading, while a polar question would disambiguate them towards the NPI reading.\footnote{Note that a negated declarative sentence would allow these expressions to be ambiguous between the interrogative and the NPI reading, as exemplified in \xref{jkga}.

\ea \label{jkga}
\gll John không gặp ai\\
John not met who\\
\glt `Who did John not meet?' / `John did not meet anyone'
\z

I use the term \textit{NPI} to describe expressions denoting existential quantifiers whose occurrence is limited to environments that must be characterized semantically as involving negation in some sense. For the purpose of this particular discussion, I will take NPIs to be expressions that can be understood as existential quantifiers in polar questions and under negation, but cannot be so understood in non-negated declarative sentences. Thus, it is possible for something to qualify as an NPI even if its distribution turns out to differ from that of English \textit{anything} with respect to other environments. I believe this terminological practice is common in the literature, and thank a reviewer for pointing out the need to make this clear.}

\ea
\ea
\gll John đang đọc gì\\
John \textsc{prog} read what\\
\glt `What is John reading?'
\ex `Is John reading anything?'
\ea
\gll John có đang đọc gì không\\
John \textsc{pos} \textsc{prog} read anything \textsc{neg}\\
\ex
\gll John đang đọc gì à\\
John \textsc{prog} read anything \textsc{a}\\
\z
\z
\z

This ambiguity extends to \textit{which}-phrases. The Vietnamese word for \textit{which} is \textit{nào}, which combines with singular NPs. As Vietnamese is a classifier language of the East Asian variety, singular number is indicated by a classifier \citep{Chierchia1998, trinh2011nominal}. %\footnote{For an account of this fact see \cite{Chierchia1998, trinh2011nominal}.} 
I will gloss the classifier as \textsc{cl}.

\ea
\ea
\gll John đang đọc quyển sách nào\\
John \textsc{prog} read \textsc{cl} book which\\
\glt `Which book is John reading?'
\ex\label{npiinq}`Is John reading any book?'
\ea\label{ex:09:neutral}
\glt John có đang đọc quyển sách nào không\\
John \textsc{pos} \textsc{prog} read \textsc{cl} book any \textsc{neg} \\
\ex
\gll John đang đọc quyển sách nào à\\
John \textsc{prog} read \textsc{cl} book any \textsc{a}\\
\z
\z
\z

In what follows, we will not be concerned with the interrogative reading of wh-phrases. For this reason, I will gloss \textsc{cl}+NP+\textsc{nao} simply as ``\textsc{any} NP''. %, as shown below. %and consequently, \textsc{bk}+\textsc{cl}+NP+\textsc{nao} as ``\textsc{bk-any} NP'', as exemplified below.
%\begin{multicols}{2}[\columnsep=-2.5cm]
%\exg.
%\ag. 
%John không đọc {quyển sách nào}\\
%John \textsc{neg} read {\textsc{any} book}\\
%`John did not read any book.'
%\bg.
%John không đọc {quyển sách nào}\\
%John \textsc{neg} read {\textsc{any} book}\\
%`John did not read any book.'
%There is morphological means in Vietnamese to force wh-phrases into having an exclusively NPI reading. This is done by prefixing the wh-phrase with the word \textit{bất kỳ}, which I will gloss as \textsc{bk}. Thus, \ref{bkue} is deviant, and \ref{bkde} has only the NPI reading.

NPIs in Vietnamese come in two morphological variants, simple and complex. Those we just discussed are the simple ones. Complex NPIs are built out of simple NPIs by prefixing the latter with the word \textit{bất kỳ} \citep{trinh2020bipartite}, which I will gloss as \textsc{bk}. 

%\ex.
%\ag.\#
%John đang cầm trong tay {bất kỳ} quyển sách nào\\
%John \textsc{prog} hold in hand \textsc{bk} \textsc{cl} book \textsc{nao}\\
%(`Which book is John holding in his hand?')\label{bkue}
%\bg.
%John đang không cầm trong tay {bất kỳ} quyển sách nào\\
%John \textsc{prog} \textsc{neg} hold in hand \textsc{bk} \textsc{cl} book \textsc{nao}\\
%`John is not holding any book in his hand' %/ *`Which book is John not holding in his hand?'
%\label{bkde}

\ea
\gll John có đang đọc {bất kỳ} {quyển sách nào} không\\
John \textsc{pos} \textsc{prog} read \textsc{bk} {\textsc{any} book} \textsc{neg}\\
\glt `Is John reading any book at all?'\label{bkex}
\z

As indicated by the translation in \xref{bkex}, adding \textit{bất kỳ} to the NPI in Vietnamese has a similar interpretive effect as adding \textit{at all} to the NPI in English: it gives rise to the inference that the speaker is biased towards the negative answer, in the sense that she has more reasons to think that the negative answer is correct than to think that the positive answer is. In the case of \xref{bkex}, the inference would be that the speaker strongly suspects that John is not reading any book.\footnote{Note that I am describing the effect of \textit{at all} in canonical, non-negated English yes/no question containing an NPI, as exemplified by the translation of \xref{bkex}. It was pointed out to me that \textit{at all} can occur in a high negation question, e.g. \textit{Isn't John reading any book at all?}, which gives rise to a \textit{positive} speaker's bias (Dan Goodhue p.c.). I have nothing to say about this fact.}  %This effect plays out 
%The lexical meaning of \textsc{bk} is something like `arbitrary' or `random'. In this reading, \textsc{bk} appears in the adjective position, which in Vietnamese is to the right of the noun. One often encounters this contrual of \textsc{bk} in mathematical textbooks. A naturally occuring example is \ref{mathex}.\footnote{The example is a 6th grade assignment taken from this website: https://olm.vn/.} %I will gloss \textit{bất kỳ} as \textsc{bk}.
%\exg.
%Cho n là số {tự nhiên} {bất kỳ}. {Chứng minh} n+3 và 2n+5 là hai số {nguyên tố} {cùng nhau}.\\
%let n be number natural \textsc{bk} prove n+3 and 2n+5 are two number prime together\\
%`Let n be an arbitrary natural number. Prove that n+3 and 2n+5 are co-primes.'\label{mathex}
%In what follows, we will not be concerned with the adjectival reading of \textsc{bk}, and neither will we be concerned with the interrogative reading of wh-phrases. For this reason, I will gloss \textsc{cl}+NP+\textsc{nao} as ``\textsc{any} NP'', as shown below. %and consequently, \textsc{bk}+\textsc{cl}+NP+\textsc{nao} as ``\textsc{bk-any} NP'', as exemplified below.
%\begin{multicols}{2}[\columnsep=-2.5cm]
%\exg.
%\ag. 
%John không đọc {quyển sách nào}\\
%John \textsc{neg} read {\textsc{any} book}\\
%`John did not read any book.'
%\bg.
%John không đọc {bất kỳ} {quyển sách nào}\\
%John \textsc{neg} read \textsc{bk} {\textsc{any} book}\\
%`John did not read any book.'
%\end{multicols}
%\vspace{-6pt}
%{\bf Summary}: Vietnamese (weak) NPIs are built from wh-phrases and come in two morphological variants, namely \textsc{any} NP and \textsc{bk any} NP.
%\subsection{Two types of polar questions}
%\section{NPIs in polar questions}
%This effect plays out in the contrast in \ref{luoi} below.
%\exg.
%\ag. 
%Tôi biết John là một {sinh viên} rất lười / \#chăm. Nó có đang đọc {bất kỳ} {quyển sách nào} không?\\
%I know John is a student very lazy / \phantom{\#}diligent he \textsc{pos} \textsc{prog} read \textsc{bk} {\textsc{any} book} \textsc{neg} \\
%`I know John is a very lazy / \#diligent student. Is he reading any book at all?'\label{luoi}
%\bg.\#
%Tôi biết John là một {sinh viên} rất chăm. Nó có đang đọc {bất kỳ} {quyển sách nào} không?\\
%I know John is a student very lazy he \textsc{pos} \textsc{prog} read \textsc{bk} {\textsc{any} book} \textsc{neg}\\
%`I know John is a very diligent student. Is he reading any book at all?'\label{cham}
%Another way to describe this negative bias brought about by \textsc{bk} is to say that its effect is similar to that of English \textit{at all}, as indicated by the translation in \ref{luoi}. 
%The question \ref{bkex} sounds normal in the context of the claim that John is a very lazy student, but sounds odd in the context of the claim that John is a very diligent student. This contrast is expected if adding \textsc{bk} to the NPI gives rise to a negative bias. Now, u
Simple NPIs, on the other hand, do not induce such negative bias. The question in \xref{ex:09:neutral}, for example, does not give rise to any inference about which answer the speaker strongly suspects to be correct. %It is thus compatible with both the questioner having asserted that John is lazy and her having asserted that John is diligent.

%\exg. 
%Tôi biết John là một {sinh viên} rất lười / chăm. Nó có đang đọc {quyển sách nào} không?\\
%I know John is a student very lazy / diligent he \textsc{pos} \textsc{prog} read {\textsc{any} book} \textsc{neg} \\
%`I know John is a very lazy / diligent student. Is he reading any book?'\label{luoisimple}
%\bg.\#
%Tôi biết John là một {sinh viên} rất chăm. Nó có đang đọc {bất kỳ} {quyển sách nào} không?\\
%I know John is a student very lazy he \textsc{pos} \textsc{prog} read \textsc{bk} {\textsc{any} book} \textsc{neg}\\
%`I know John is a very diligent student. Is he reading any book?'\label{cham}
%Let us now discuss the distribution the two types of NPIs across the two types of polar questions. 
Another difference between simple and complex NPIs pertains to their distribution across the two types of polar questions: whereas simple NPIs are acceptable in both yes/no and agreement questions, as shown by (\ref{cobkanybookkhong}), complex NPIs are acceptable in yes/no questions but give rise to deviance when they occur in agreement questions, as shown by (\ref{complexagree}).

\ea Intended reading: `Is John reading any book at all?'
\ea[]{
\gll John có đang đọc {bất kỳ} {quyển sách nào} không?\\
John \textsc{pos} \textsc{prog} read \textsc{bk} {\textsc{any} book} \textsc{neg}\\
}\label{cobkanybookkhong}
\ex[\#]{
\gll John đang đọc {bất kỳ} {quyển sách nào} à?\\
John \textsc{prog} read \textsc{bk} {\textsc{any} book} \textsc{a}\\
}\label{complexagree}
\z
\z

%\section{Interim summary}

%Vietnamese polar questions come in two variants, yes/no questions and agreement questions. Yes/no question requires the context to be neutral with respect to the prejacent, while agreement questions require the context to be biased toward the prejacent. NPIs in Vietnamese also come in two variants, simple and complex. 

%Simple NPIs are licensed in both yes/no and agreement questions but complex NPIs are licensed only in yes/no questions. In yes/no questions, complex NPIs give rise to negative bias while simple NPIs give rise to no bias. 

%The next section will be devoted to an analysis of these facts.

\section{Analysis \label{analysis} }

We have seen that Vietnamese polar questions come in two variants, yes/no questions and agreement questions. Yes/no questions are prejacent-neutral while agreement questions are prejacent-biased. We have also seen that NPIs in Vietnamese come in two variants, simple and complex. Simple NPIs are acceptable in both yes/no and agreement questions. Complex NPIs are acceptable in yes/no questions but cause deviance in agreement questions. In yes/no questions, complex NPIs give rise to negative bias while simple NPIs do not.

The present section will be devoted to an analysis of these facts.


\subsection{Introducing \textsc{whether} }

For the purpose of this paper I will assume a simplified version of the analysis proposed in \citet{hamblin1973questions, karttunen1977syntax}. % according to which wh-phrases are a variant of existential quatifiers. Consider \textit{who}, for example.  Specifically, whereas quantifiers are of type $\langle \alpha, t\rangle$, wh-phrases are of type $\langle \alpha, \langle st, t\rangle\rangle$. Compare the meaning of {\bf who} and {\bf someone} below.
%\ex.
%\a. 
%$\eval{\text{who}}{} \hspace{0.77cm} = \lambda P \in D_{\langle e,t\rangle}.$ $\lambda p \in D_{st}$. $\exists x \in D_e$. $person(x)$ $\wedge$ $p$ $=$ {$\textasciicircum$}{$P(x)$}\label{whosem}
%\b.
%$\eval{\text{someone}}{} = \lambda P \in D_{\langle e,t\rangle}.$ \censor{{\sout{$\lambda p \in D_{st}$.}}} $\exists x \in D_e$. $person(x)$ $\wedge$ \censor{{\sout{$p$ $=$ {\textasciicircum}}}}$P(x)$
%In this particular case, the wh-phrase of type $\langle \langle e,t\rangle, \langle st, t\rangle\rangle$. It undergoes movement, leaving a trace of type $e$ and introducing a $\lambda$-binder which abstracts over this trace. The quantifier is of type $\langle\langle e,t\rangle,t\rangle$ and also leaves a trace of type $e$ when it moves.
%\begin{multicols}{2}[\columnsep = -0.3cm]
%\ex.
%\a.
%$\eval{\text{who}}{} = \lambda P \in D_{\langle e,st\rangle}.$ $\lambda p \in D_{st}$. $\exists x \in D_e$. $person(x)$ $\wedge$ $p = P(x)$ \vspace{6pt}
%\b.
%\a.
%\Tree [.{\textcolor{black}{$\lambda p$. $\exists x$. $person(x)$ $\wedge$ $p =$ ${\textasciicircum}$$saw(j,x)$} } \node{1}{who} [.{\textcolor{black}{$\lambda y$. $saw(j,y)$}} 1 \qroof{John saw \node{2}{$t_1$}}.{\textcolor{black}{$saw(j,t_1)$}} ] ] 
%\abarnodeconnect[-12pt]{2}{1}
%\anodecurve[b]{2}[b]{1}{2.2cm}\label{johnsawwho}
%\b.
%\Tree [.{\textcolor{black}{$\exists x$. $person(x)$ $\wedge$ $saw(j,x)$} } \node{1}{someone} [.{\textcolor{black}{$\lambda y$. $saw(j,y)$}} 1 \qroof{John saw \node{2}{$t_1$}}.{\textcolor{black}{$saw(j,t_1)$}} ] ] 
%\abarnodeconnect[-12pt]{2}{1}
%\anodecurve[b]{2}[b]{1}{2.2cm}
%\end{multicols}
%\vspace{1cm}
%\ex.[{}]
%\ref{johnsawwho} = $\{$John saw Mary, John saw Sue, John saw Bill, ...$\}$
%\ex.
%\a.
%$\eval{\text{someone}}{} = \lambda P \in D_{\langle e,st\rangle}.$ \textcolor{lightgray}{\sout{$\lambda p \in D_{st}$.}} $\exists x \in D_e$. $person(x)$ $\wedge$ \textcolor{lightgray}{\sout{$p =$}} $P(x)$
%\b. 
%$\eval{\text{who}}{} \hspace{0.77cm} = \lambda P \in D_{\langle e,st\rangle}.$ $\lambda p \in D_{st}$. $\exists x \in D_e$. $person(x)$ $\wedge$ $p = P(x)$
%\ex.
%\Tree [.{\textcolor{gray}{$\lambda p$. $\exists x$. $person(x)$ $\wedge$ $p =$ $came(x)$} } \node{1}{who} [.{\textcolor{gray}{$\lambda y$. $came(y)$}} 1 \qroof{\node{2}{$t_1$} came}.{\textcolor{gray}{$came(t_1)$}} ] ] 
%\abarnodeconnect[-12pt]{2}{1}
%\anodecurve[b]{2}[b]{1}{2cm}
%\vspace{1cm}
%Before discussing polar questions, I will define two functions. 
Let us define two functions. The first is \textsc{yes}, the identity function, and the second is \textsc{no}, the negation function.%\newpage

\ea
\ea
\textsc{yes} $=_{\text{def}}$ $\lambda p \in D_{st}.$ $p$
\ex
\textsc{no} \hspace{0.06cm} $=_{\text{def}}$ $\lambda p \in D_{st}.$ $\neg p$
\z
\z

%These functions are thus of type $\langle st,st\rangle$. %Let $o$, which is mnemonic for ``operator'', stand for $\langle st,st\rangle$. 
We will say that a function $f$ of type $\langle st,st\rangle$ is a ``polarity'', i.e. that $polarity(f)$, if $f$ is either \textsc{yes} or \textsc{no}. For polar questions, I assume the presence of a (overt or) covert \textsc{whether} %, which parallels \textit{who} in its syntax and semantics 
\citep[][]{bennett1977response, Higginbotham:1993, krifka2001structured, guerzonisharvit2014whether}. %The difference is that {\bf who} existentially quantifies over individuals, i.e. things of type $e$, while \textsc{whether} existentially quantifies over polarities, i.e. things of type $h$. 
%Compare the entry of \textsc{whether} in \ref{whethersem} with that of \textit{who} in \ref{whosem}, repeated in \ref{whosem2}.

\ea
%\a.
$\eval{\textsc{whether}}{} $\\ $= \lambda Q \in D_{\langle\langle st,st\rangle,t\rangle}$. $\lambda p \in D_{st}$. $\exists f \in D_{\langle st,st\rangle}$. $polarity(f) \wedge p = Q(f)$\label{whethersem}
\z
%\b.
%$\eval{\text{who}}{} \hspace{1.1cm} = \lambda P \in D_{\langle e,t\rangle}.$ $\lambda p \in D_{st}$. $\exists x \in D_e$. $person(x)$ \hspace{0.01cm} $\wedge$ $p = {\textasciicircum}P(x)$\label{whosem2}

The base position of \textsc{whether} is above TP. %Just like {\bf who}, it undergoes movement. Recall that {\bf who} is of type $\langle \langle e,t\rangle, \langle st, t\rangle\rangle$ and leaves a trace of type $e$ when it moves. Similarly, \textsc{whether} is of type $\langle \langle o,t\rangle, \langle st, t\rangle\rangle$ and 
When it moves, it leaves a trace of type $\langle st,st\rangle$. Predicate abstraction proceeds in the familiar way.

\ea \label{whetherjohnsawmary}
\begin{forest}
 [{$\lambda p \ldotp \exists f\ldotp polarity(f) \wedge p = f(\phi)$} [\textsc{whether}] [{$\lambda g\ldotp g(\phi)$} [ 1 ] [{$t_1(\phi)$} [$t_1$] [$\phi$] ] ] ]
\end{forest}
\z

%\ex.[{}]
%\ref{whetherjohnsawmary} = $\{$\textsc{yes}(John saw Mary), \textsc{no}(John saw Mary)$\}$ %\\
%\phantom{\ref{whetherjohnsawmary}} = $\{$John saw Mary, $\neg$John saw Mary$\}$

\vspace{12pt}

I will assume that at the relevant level of analyis, a yes/no question in Vietnamese whose prejacent is $\phi$ has the logical form $[$\textsc{whether} $\phi]$. Thus, the question in \xref{cokantkhong}, reproduced below in \xref{cokantkhongwhether}, has the logical form in \xref{cokantkhongwhetherlf}, which denotes the set in \xref{cokantkhongwhetherset}.

\ea
\ea \label{cokantkhongwhether}
\gll John có đọc Kant không?\\
John \textsc{pos} read Kant \textsc{neg}\\
%\b.
%$[_\alpha$ \node{1}\textsc{whether} 
\ex \label{cokantkhongwhetherlf}
\begin{forest}
fairly nice empty nodes, for tree={inner sep=0, l=0}
[{$\alpha$} [\textsc{whether}] [ [1] [ [$t_1$] [{$\phi$} [John read Kant , roof ] ] ] ] ]
%\anodecurve[b]{2}[b]{1}{1cm} \vspace{6pt}
\end{forest}
\ex
$\eval{\alpha}{} =$ $\{$\textsc{yes}(John reads Kant), \textsc{no}(John reads Kant)$\}$\label{cokantkhongwhetherset}
\z
\z

%A well-known fact about \textsc{whether} is that it licenses NPIs, and attempts have been made to derive this fact. For this paper, I will assume it as a postulate. 
%\ex.\label{whethernpi}
%Postulate I\\
%\textsc{whether} licenses NPIs in its scope
%Given Postulate I, we predict that NPIs are licensed in yes/no questions. This prediction is borne out, as we have observed.
%\subsection{The evidential marker \textsc{e} }

\subsection{Introducing \textsc{evid} }

What about agreement questions? Recall that agreement questions are prejacent-biased, in the sense that they require the context to contain evidence for the prejacent. I will adopt the analysis proposed in \citet{trinh2014how} and assume the existence of an evidential marker \textsc{evid}, which is attached to TP and has the following interpretation.%\footnote{Thus, \textsc{evid} is the presuppositional counterpart of \citeauthor{fintelgillies2010must}'s \citeyear{fintelgillies2010must} epistemic \textit{must}.}

\ea \label{edef}
$\eval{\textsc{evid $\phi$}}{} =
\left\{\begin{array}{ll}
\eval{\phi}{} & \mbox{if there is contextual evidence that $\phi$}\\
\#   & \mbox{otherwise}
\end{array}\right.$
\z

Thus, $[$\textsc{evid} $\phi]$ presupposes that there is contextual evidence that $\phi$. I propose that at the relevant level of analysis, the agreement question \xref{kantawhether} has the LF in \xref{kantawhetherlf}, and the denotation in \xref{kantawhetherset}.

\ea
\ea \label{kantawhether}
\gll John đọc Kant à?\\
John read Kant \textsc{a}\\
\ex \label{kantawhetherlf}
\begin{forest}
fairly nice empty nodes, for tree={inner sep=0, l=0} [{$\alpha$} [\textsc{whether}] 
		[
		[1] 
			[ [{$t_1$}] 
				[
				[\textsc{evid}]
				[$\phi$ [John read Kant , roof ] ]
					] ] ] ]
\end{forest}
\ex
$\eval{\alpha}{} =$ $\{$\textsc{yes}(\textsc{evid}(John reads Kant)), \textsc{no}(\textsc{evid}(John reads Kant))$\}$\label{kantawhetherset}
\z
\z

Both answers contain $[$\textsc{evid}(John reads Kant)$]$ as a subconstituent. Thus, both answers presuppose that there is contextual evidence that John reads Kant, which means the question presupposes that there is contextual evidence that John reads Kant.\footnote{I assume that if all answers to a question have a presupposition then the question itself inherits that presupposition. This follows from the fact that questions are quantificational structures \citep{Heim:1983, Heim:1992, schlenker2008articulate}.} We thus account for the fact that agreement questions are evidentially biased toward the prejacent.

How do we account for the fact that yes/no questions are prejacent-neutral, i.e. that a yes/no question with prejacent $\phi$ is infelicitous in contexts where there is evidence that $\phi$? I propose that this effect comes about as an anti-presupposition. I will assume the principle of Maximize Presupposition as a primitive of grammar \citep{heim1991artikel}.\footnote{A reviewer raises the question whether MP should be considered a principle of rational communication of a Gricean sort which is external to the language faculty rather than one of grammar. I will not attempt to address this question adequately, as that would take us beyond the scope of this paper. What matters is that I use MP without deriving it, not how it is derived. Nevertheless, I would note that \citet{heim1991artikel} did point out how it would be difficult to derive MP from principles of information exchange. Thus, given contextual, i.e. pragmatic, knowledge, \#\textit{a sun is shining} conveys the exact same amount of information as \textit{the sun is shining}. It is not clear how to explain the contrast between these sentences in terms of their communicative function.}

\ea
Maximize Presupposition (MP)\\
Presuppose as much as possible!
\z

Given MP, a yes/no question will be understood as indicating that there is no contextual evidence for the prejacent, since if there were such evidence, the speaker would have used an agreement question instead \citep{sauerland2008implicated}.





%What about the fact that simple NPIs are licensed in both yes/no and agreement questions while complex NPIs are licensed only in yes/no questions.
%Since \textsc{whether} is present in agreement questions, we also predict that NPIs are licensed in them. This prediction, however, is borne out only in part. We have observed that simple NPIs are licensed in agreement question, but not complex NPIs. We will now come to an explanation as to why that is the case.
%The question whether John reads Kant, in a context where there is evidence that John reads Kant, would have the parse in \ref{biaskant} and the interpretation in \ref{biaskantmeaning}.
%\ex.
%\a.\label{biaskant}
%\Tree [.$\alpha$ \textsc{\node{1}{whether}} [.{$\beta$} 1 [.{$\gamma$} \node{2}{$t_1$} [.{$\delta$} \textsc{e} \qroof{John reads Kant}.{$\epsilon$} ] ] ] ]
%\anodecurve[b]{2}[b]{1}{2cm}\label{whetherjohnsawmary}
%\b.\label{biaskantmeaning}
%$\eval{\alpha}{} =
%\left\{\begin{array}{ll}
%\mbox{$\{\epsilon, \neg\epsilon\}$} & \mbox{if there is contextual evidence that $\epsilon$}\\
%\#   & \mbox{otherwise}
%\end{array}\right.$ 

\subsection{Introducing \textsc{even} }

Let us now address the fact that NPIs, both simple and complex, are acceptable in yes/no questions. A well-known fact about \textsc{whether} is that it licenses NPIs. Various attempts have been made to derive this observation \citep[cf.][]{Ladusaw:1979, krifka1991some, krifka1995semantics, rooij2003negative, guerzonisharvit2007question, guerzonisharvit2014whether, nicolae2015questions, roelofsen2018npis, roelofsenjeong2022focused}. For this paper, I will assume it as a primitive. 

\ea \label{whethernpi}
\textsc{whether} licenses NPIs in its scope
\z

Given \xref{whethernpi}, we predict, correctly, that NPIs of both types are acceptable in yes/no questions. However, we also predict, incorrectly, that NPIs of both types are acceptable in agreement questions as well, given our analyis of agreement questions as containing \textsc{whether}. Our task, therefore, is to specify a distinctive grammatical property of complex NPIs which explains the deviance caused by their occurence in agreement questions. 

Recall a complex NPI consists of a simple NPI plus the element \textsc{bk}. I propose that \textsc{bk}, by itself, has no independent semantics. Rather, it is just the morphological reflex of a c-commanding operator, \textsc{even}.\footnote{For similar ideas see \citet{Heim:1984, Guerzoni:2004, crnic2014against, roelofsenjeong2022focused}.}

\ea \label{complexcond}
\textsc{bk} is the morphological reflex of a c-commanding \textsc{even} in the structure
\z

As its name suggests, \textsc{even} has a meaning akin to that of English \textit{even}. For the purpose of this discussion, we will give \textsc{even} the interpretation in \xref{evendef}.

%ing the fact that NPIs are licensed in DE but not in UE contexts. I will take the ``exhaustification'' approach which has been proposed and defended in many works. This approach has two main ingredients. The first is the assumption that NPIs denote existential quantifiers and introduce subdomain alternatives. Alternatives of sentences containing NPIs are generated by point-wise composition in the standard way.\newpage

%The second ingredient is the assumption of a covert exhaustification operator, \textsc{exh}, which obligatorily attaches to every (declarative) sentence. I will adopt the presuppositional view of \textsc{exh}.

\ea
%\a.
\label{evendef}
$\eval{\textsc{even $\phi$}}{} =
\left\{\begin{array}{ll}
\eval{\phi}{} & \mbox{if $\forall \psi \in \text{ALT}(\phi).$ $\eval{\phi}{} \leq_{\text{likely}} \eval{\psi}{}$}\\
\#   & \mbox{otherwise}
\end{array}\right.$
\z
%\b.
%\label{exhdef}
%$\eval{\textsc{exh $\phi$}}{} = 1$ iff $\forall \psi \in \text{ALT}(\phi).$ $\eval{\psi}{} = 1$ $\rightarrow$ $\eval{\phi}{} \subseteq \eval{\psi}{}$
%$\eval{\textsc{exh $\phi$}}{} \hspace{0.23cm} =
%\left\{\begin{array}{ll}
%\eval{\phi}{} & \mbox{if $\forall \psi \in \text{ALT}(\phi).$ $\eval{\psi}{} = 1 \rightarrow \eval{\phi}{} \subseteq \eval{\psi}{}$}\\
%\#   & \mbox{otherwise}
%\end{array}\right.$

%Thus, [\textsc{exh} $\phi$] asserts that $\phi$ and presupposes that $\phi$ is the strongest proposition among the true alternatives. We predict the following judgements. %Thus, we predict that NPIs are licensed in DE but not in UE environments.

%\ex.
%\a.
%\a.\#
%\textsc{even}(John read any$_{\text{D}}$ book)
%\b.
%\textsc{even}(\textsc{neg}(John read any$_{\text{D}}$ book))\z.
%\b.
%\a.\#
%\textsc{exh}(John read any$_{\text{D}}$ book)
%\b.
%\textsc{exh}(\textsc{neg}(John read any$_{\text{D}}$ book))\z.
%\b.
%\a.\#
%\textsc{exh}(\textsc{even}(John read any$_{\text{D}}$ book))
%\b.
%\textsc{exh}(\textsc{even}(\textsc{neg}(John read any$_{\text{D}}$ book)))\z.
%\b.
%\a.\#
%\textsc{even}(\textsc{exh}(John read any$_{\text{D}}$ book))
%\b.
%\textsc{even}(\textsc{exh}(\textsc{neg}(John read any$_{\text{D}}$ book)))


Thus, $[$\textsc{even} $\phi]$ asserts $\phi$ and presupposes that $\phi$ is the least likely among the alternatives of $\phi$. I will assume, following many works, that NPIs induce ``subdomain'' alternatives. Alternatives of sentences containing NPIs are generated by point-wise composition in the familiar way. In their basic meaning, NPIs are just existential quantifiers \citep{Kadmon:1993, krifka1995semantics, Chierchia:2013}.\footnote{Note that as quantifiers, NPIs must QR to be interpretable. I make the standard assumptions that NPIs have narrowest scope, i.e. that they raise to the smallest clause containing them. This means that \textit{John read any$_{\text{D}}$ book} has the following LF.

\ea
any$_{\text{D}}$ book $\lambda_1$ [John read $t_1$]
\z

I thank a reviewer for pointing out the need to make this clear.}


\ea
\ea $\eval{\text{any$_{\text{D}}$ book}}{} = \lambda P.$ $\exists x$. $x \in \text{D} \cap \eval{\text{book}}{}$ $\wedge$ $P(x)$ $=$ `a book in D'
\ex ALT(any$_{\text{D}}$ book) $=$ $\{$any$_\text{D$'$}$ book $|$ D$'$ $\subseteq$ D$\}$ $=$ $\{$a book in D$'$ $|$ D$'$ $\subseteq$ D$\}$
\ex ALT(John read any$_{\text{D}}$ book) $=$ $\{$John reads any$_\text{D$'$}$ book $|$ D$'$ $\subseteq$ D$\}$
\z
\z

%Here is the property of complex NPIs that would account for the differences between them and simple NPIs.

%\ex.\label{complexcond}
%Condition on complex NPIs\\
%Complex NPIs require a c-commanding \textsc{even} in the structure

%Thus, I will assume that the prefix \textsc{bk} has no semantics of its own but is a morphological reflex of the c-commanding \textsc{even}. 

Simple NPIs do not come with \textsc{bk}. I will take this to mean that they do not come with \textsc{even}. In polar questions with \textsc{even}, \textsc{whether} can be base-generated either above or below \textsc{even}. Consider the first possibility.

\ea \label{deviantq}
\begin{forest}
fairly nice empty nodes, for tree={inner sep=0, l=0} 
[$\alpha$ [\textsc{whether}] [ [1] [ [$t_1$] [ [\textsc{even}] [$\phi$ [John read \textsc{any}$_{\text{D}}$ book , roof ] ] ] ] ] ]
\end{forest}
\sn $\eval{\alpha}{}$ = $\{$\textsc{yes}(\textsc{even}(John read \textsc{any}$_{\text{D}}$ book)), \textsc{no}(\textsc{even}(John read \textsc{any}$_{\text{D}}$ book))$\}$ 
\z

This configuration results in a yes/no question for which both answers, the positive as well as the negative, have the same  unsatisfiable presupposition, induced by the subconstituent in \xref{subcon}.

\ea \# \textsc{even}(John read \textsc{any}$_{\text{D}}$ book) \label{subcon}\\
Presupposition: John reads a book in D $\leq_\text{likely}$ John reads a book in D$'$, for any D$'$ $\subseteq$ D\label{presup}
\z

Given that likelihood respects logical entailments, i.e. that $\phi$ $\leq_\text{likely}$ $\psi$ if $\phi \Rightarrow \psi$, and given the logical truth that for any D and D$'$ such that D$'$ $\subseteq$ D, if John reads a book in D$'$ then John reads a book in D but not vice versa, both answers in \xref{deviantq} presuppose that a weaker sentence is less likely than a stronger sentence, which is necessarily false. I will take this fact to mean that such a parse as \xref{deviantq} will be ruled out as deviant by the grammar.

Having \textsc{even} scoping above the trace of \textsc{whether}, however, results in a polar question with \emph{one} felicitous answer, namely the negative. 

\ea \label{okq}
\begin{forest}
fairly nice empty nodes, for tree={inner sep=0, l=0}  [$\alpha$ [\textsc{whether}] [ [1] [ [\textsc{even}] [ [$t_1$] [$\phi$ [John read any$_{\text{D}}$ book , roof ] ] ] ] ] ]
\end{forest}
\sn $\eval{\alpha}{} =$ $\{$\textsc{even}(\textsc{yes}(John read any$_{\text{D}}$ book)), \textsc{even}(\textsc{no}(John read any$_{\text{D}}$ book))$\}$
\z

The positive answer in this case is equivalent to the positive answer in \xref{deviantq}, and is deviant for the same reason, namely because it has a necessarily false presupposition. The negative answer, however, does not have such a presupposition. Let us consider it.

\ea \label{neganswer}
\textsc{even}(\textsc{no}(John read any$_{\text{D}}$ book))\\
Presupposition: $\neg$John reads a book in D $\leq_\text{likely}$ $\neg$John reads a book in D$'$, for any D$'$ $\subseteq$ D\label{truepresup}
\z

Negation is scale-reversing, so for any D$'$ $\subseteq$ D, if it is not the case that John reads a book in D then it is also not the case that John reads a book in D$'$, but not vice versa. This means the negative answer in \xref{okq} has a trivially true presupposition.

We thus see that if \textsc{even} is present in a polar question, it has to be parsed above the base position of \textsc{whether}, and within this parse, only the negative answer is acceptable. This means that polar questions with \textsc{even} have only the negative answer as the one felicitous answer. And because complex NPIs require a c-commanding \textsc{even}, we predict that for polar questions with complex NPIs, only the negative answer is felicitous. Asking a polar question with a complex NPI, then, amounts to presenting the hearer with the negative answer as the only choice. I propose that this is what brings about the inference that the speaker of such a question is biased towards the negative answer \citep[cf.][]{Guerzoni:2004}. For concreteness, I will take this inference to be a conversational implicature of the question.\footnote{The issue arises, of course, as to how a `yes' answer to such a negatively biased question is still possible. Note that this issue arises in the same way for \citet[][]{Guerzoni:2004}. One plausible response is to say that a `yes' answer, in this case, requires the accomodation of a slightly different question, namely one without the negative bias. Thus, such an answer is also a move to deny the negative presupposition of the biased question.} 

%For concreteness, I will call the negative bias of polar questions containing \textsc{even} a conversational implicature of these questions.

Let us now come (back) to the question why complex NPIs cause deviance in agreement questions, as evidenced by \xref{complexagree}, reproduced below in \xref{complexagree2}.

\ea[\#]{
\gll John đang đọc {bất kỳ} {quyển sách nào} à?\\
John \textsc{prog} read \textsc{bk} {\textsc{any} book} \textsc{a} \\
\glt Intended reading: 'Is John reading any book at all?'}
\label{complexagree2}
\z

Given what we have said, this question will have the parse in \xref{parseagree} and the denotation in \xref{denoteagree}.

\ea \label{badq}
\ea \label{parseagree}
\begin{forest}
fairly nice empty nodes, for tree={inner sep=0, l=0}   
[$\alpha$ [\textsc{whether}] [ [1] [ [\textsc{even}] [ [$t_1$] [ [\textsc{evid}] [$\phi$ [ John is reading any$_{\text{D}}$ book , roof ] ] ] ] ] ] ]
\end{forest}
\ex \label{denoteagree}
%\protectedex{
$\eval{\alpha}{} = \{$\textsc{even}(\textsc{yes}(\textsc{evid}(John is reading any$_{\text{D}}$ book))),
\sn \textsc{even}(\textsc{no}(\textsc{evid}(John is reading any$_{\text{D}}$ book)))$\}$
%}
\z
\z

Let us consider the inferences licensed by this question. Due to the presence of \textsc{evid}, it has the presupposition in \xref{presupq}. And due to the presence of \textsc{even}, it has the implicature in \xref{implicatureq}.

\ea Inferences licensed by \xref{badq}
\ea There is contextual evidence that John is reading a book in D\label{presupq}
\ex The speaker strongly suspects that John is not reading a book in D\label{implicatureq}
\z
\z

I submit that these two inferences are responsible for the question being perceived as deviant. The reason, I claim, is that a rational speaker cannot both strongly suspect $\neg\phi$ while at the same time take some fact in the context to be evidence that $\phi$. Thus, if he really strongly suspects that John is not reading a book, the sight of John reading a book would have to be interpreted by him to be evidence that John is pretending to read a book.\footnote{I admit that this point needs further explication. I hope to pursue this task in future research.}


\section{Interim summary \label{interim} }

Polar questions contain a covert \textsc{whether}, which accounts for the intuition that they ask the hearer to confirm a proposition or to confirm its negation. Polar questions in Vietnamese come in two variants, yes/no questions and agreement questions. Agreement questions contain \textsc{evid}, the evidential marker which introduces the presupposition that its prejacent is supported by contextual evidence. Yes/no questions, in contrast, do not contain \textsc{evid}. Given Maximize Presupposition, yes/no questions anti-presuppose that there is contextual evidence for the prejacent. This accounts for the fact that in prejacent-biased contexts, agreement questions are felicitous while yes/no questions are not, while in prejacent-neutral contexts, the opposite is the case.

NPIs in Vietnamese also come in two variants, simple and complex. Complex NPIs come with a c-commanding \textsc{even} in the structure, which introduces the presupposition that its prejacent is the least likely among the alternatives. Given that NPIs denote existential quantifiers and induce subdomain alternatives, the presence of \textsc{even} brings it about that only the negative answer is felicitous. This accounts for the fact that polar questions containing complex NPIs give rise to the inference that the speaker strongly suspects that the negative answer is correct. Simple NPIs do not come with \textsc{even} and hence do not give rise to such a bias.

An agreement question which contains a complex NPI would be parsed with both \textsc{evid} and \textsc{even}. Such a question would presuppose that there is contextual evidence for the prejacent, and at the same time, would license the inference that the speaker suspects that the prejacent is false. I hypothesize that such an expression represents an odd move in the language game, and hence, would be perceived as odd. This accounts for the fact that complex NPIs in agreement questions gives rise to deviance.

\section{Comparison}

I will conclude this note by discussing some similarities and differences between Vietnamese and English with respect to polar questions and NPIs. I believe that addressing the questions they raise will contribute to the cross-linguistic {re\-search} on the semantics-syntax interface, or more specifically, on how Universal Grammar constrains the way basic building blocks of semantic representation are combined and mapped onto syntactic objects by different languages.

Let us start with the distinction within the class of polar questions in Vietnamese, i.e. the distinction between yes/no and agreement questions. The reader might have noticed that this distinction resembles the distinction in English between ``inversion'' and ``declarative'' questions. Inversion questions are polar questions which exhibit subject auxiliary inversion, such as \xref{subjauxex}, and declarative questions  those which exhibit declarative word order and are often pronounced with rising intonation, such as \xref{declqex}.

\ea
\ea\label{subjauxex}
Does John read Kant?
\ex\label{declqex}
John reads Kant?
\z
\z

It has been pointed out that declarative questions give rise to the inference that there is contextual evidence supporting the prejacent \citep{gunlogson2003true, trinh2014how}. In a context where the speaker has no reason to think that John reads Kant or to think that he does not, \xref{subjauxex} would sound appropriate and \xref{declqex} would sound odd. On the other hand, if the speaker sees John with a copy of \textit{ The critique of pure reason} in his hand, \xref{declqex} would be felicitous.

Can we say that inversion and declarative questions are the English counterparts of Vietnamese yes/no and agreement questions? It turns out that we cannot. The distinctions do align, but not perfectly. Recall, from \xref{critique1} and \xref{critique2}, that Vietnamese yes/no questions and agreement questions are in complementary distribution: yes/no questions are felicitous only in prejacent-neutral contexts and agreement questions are felicitous only in prejacent-biased contexts. The situation with English inversion and declarative questions is different. It turns out that the contexts in which inversion questions are felicitous are a superset of the contexts where declarative questions are felicitous. Specifically, inversion questions are felicitous in prejacent-biased contexts as well. 

\ea Context: the speaker sees John with a copy of \textit{ The critique of pure reason} in his hand.
\ea Does John read Kant?
\ex John reads Kant?
\z
\z

\begin{table}
\begin{tabularx}{.9\textwidth}{lCC@{\qquad}Cc }
\lsptoprule
{} & yes/no   & agreement & inversion & declarative \\
\midrule
%prejacent neutral questions & \ding{51} & \ding{55} & \ding{51} & \ding{55} \\ \hline
prejacent-neutral & \ding{51} & \ding{55} & \ding{51} & \ding{55} \\
prejacent-biased & \ding{55} & \ding{51} & \ding{51} & \ding{51} \\
\midrule
{} & \multicolumn{2}{c}{Vietnamese} & \multicolumn{2}{c}{English} \\
\lspbottomrule
\end{tabularx}
\caption{Prejacent-neutral vs. prejacent-biased}
\end{table}



I turn now to a discussion of the distinction between simple and complex NPIs in Vietnamese. Again, the reader might have noticed that this distinction resembles the distinction between NPIs and so-called ``minimizers'' in English, i.e., expressions such as \textit{lift a finger} or \textit{have a red cent}. In fact, it is \citeauthor{Guerzoni:2004}'s (\citeyear{Guerzoni:2004}) analysis of minimizers that informs the analysis of complex NPIs proposed here. Guerzoni observes that minimizers induce negative bias in polar questions whereas NPIs do not. Thus, \xref{help} can be read as not implying anything about how likely it is that John did something to help, while \xref{finger} clearly implies that it is unlikely that John did something to help.

\ea
\ea Did John do anything to help?\label{help}
\ex Did John lift a finger to help?\label{finger}
\z
\z

Guerzoni accounts for the difference between NPIs and minimizers with respect to negative bias by postulating that minimizers, but not NPIs, come with a c-commanding \textsc{even} in the structure which has to scope above the base position of \textsc{whether}. My account of the same difference between simple and complex NPIs in Vietnamese is just an adoption of her analysis. Can we, then, say that simple NPIs in Vietnamese correspond to NPIs in English while complex NPIs in Vietnamese correspond to minimizers in English?

Again, it turns out that we cannot. Recall that simple NPIs in Vietnamese are acceptable in prejacent-biased polar questions. NPIs in English, however, are not. Suppose I am talking to John on the phone and hear chewing sounds, which I take to be evidence that he is eating while talking. In this context, it seems that I cannot ask him the questions in \xref{chewing}.

\ea\label{chewing} 
Contextual evidence: The hearer is eating.
\ea[\#]{
Are you eating anything?}
\ex[\#]{
You're eating anything?}
\z
\z

Thus, NPIs in English are blocked by prejacent bias. Note that it has been observed that NPIs are deviant in declarative questions \citep{hirst1983interpreting, huddleston1994contrast, gunlogson2002declarative}. This is expected, given that declarative questions are necessarily prejacent-biased. 

Simple NPIs in Vietnamese, however, are not blocked by prejacent bias. Recall that only agreement questions can be prejacent-biased. In the same context, i.e., one where there is evidence that the hearer is eating while talking on the phone, the question in \xref{angi} is completely fine, where \textit{gì} is the word whose interrogative reading is `what' and whose NPI reading is `anything'.

\ea\label{angi}
\gll Anh đang ăn gì à?\\
you \textsc{prog} eat anything \textsc{a}\\
\z

How do complex NPIs in Vietnamese and minimizers in English compare with respect to prejacent-biased questions? It turns out they behave similarly in this case: both are unacceptable. The deviance of \xref{complexagree} evidences this for Vietnamese. For English, we can observe that a question such as \xref{finger} would be utterly inappropriate in contexts where there is evidence that John did do something to help.

Another way in which Vietnamese and English NPIs differ pertains to the so-called ``free choice reading'', or FC reading for short. It has been observed that in English, NPIs embedded under existential modals such as \textit{be allowed to} are, by default, read as wide-scope universal quantifiers \citep{Carlson:1981, Dayal:1998, Menendez-Benito:2010, crnic2019pruning, barlevfox2020free}. The FC reading, however, is impossible for minimizers.

\ea\label{fc}
\ea[]{
John is allowed to do anything to help.}
\sn `$\forall x.$ John is allowed to do $x$ to help'
\ex[\#]{
John is allowed to lift a finger to help.}
\sn Intended reading: John is allowed to do anything to help
\z
\z

In Vietnamese, the situation is, in a sense, the reverse. It is the complex NPIs which can occur, and have the FC reading, under existential modals. Simple NPIs are excluded.

\ea Intended reading: `John is allowed to read any book'
\ea[\#]{
\gll John {được phép} đọc {quyển sách nào}\\
John {is allowed to} read {\textsc{any} book}\\}
\ex[]{
\gll John {được phép} đọc {bất kỳ} {quyển sách nào}\\
John {is allowed to} read \textsc{bk} {\textsc{any} book}\\}
\glt `John is allowed to read any book'
\z
\z

\begin{table}
\begin{tabularx}{.9\textwidth}{ l CCCc }
\lsptoprule
{} & simple NPIs  & complex NPIs & NPIs & minimizers\\ 
\midrule
%prejacent neutral questions & \ding{51} & \ding{55} & \ding{51} & \ding{55} \\ \hline
biased questions & \ding{51} & \ding{55} & \ding{55} & \ding{55}\\
FC reading & \ding{55} & \ding{51} & \ding{51} & \ding{55} \\
\midrule
{} & \multicolumn{2}{c}{Vietnamese} & \multicolumn{2}{c}{English} \\
\lspbottomrule
\end{tabularx}
\caption{Biased questions vs. FC reading}
\end{table}

I hope to account for the facts we just discussed in future research.
 


\section*{Acknowledgements}
This work is supported by the ERC Advanced Grant \textit{Speech Acts in Grammar and Discourse} (SPAGAD), ERC-2007-ADG 787929.

\sloppy
\printbibliography[heading=subbibliography,notkeyword=this]
\end{document}
