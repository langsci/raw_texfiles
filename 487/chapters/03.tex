\documentclass[output=paper,colorlinks,citecolor=brown]{langscibook}
\ChapterDOI{10.5281/zenodo.17158178}
\author{Beste Kamali\affiliation{University of Amsterdam, Institute of Logic, Language, and Computation}\orcid{0000-0002-3292-078X}} 
\title{Evidential bias across clause types}
\abstract{Declarative (polar) questions such as English rising declaratives are associated with a range of effects including positive evidential bias, which are commonly analyzed as arising from the non-interrogative clause type of these questions (\cite{gunlogson, farkas-roelofsen-division}, among others). A detailed comparison between English rising declaratives and Turkish object attachment interrogatives shows that (a) positive evidential bias is not an exclusive feature of declarative questions and (b) in fact a cluster of effects beyond evidential bias is common to rising declaratives and this particular kind of polar interrogative. I argue that both question forms are underlain by the monopolar question meaning in the sense of \citet{krifka15}, which can yield the evidential bias cluster of effects across clause types. The account helps to disentangle types of biased inferences some may still be rooted in clause type, offers a way of approaching variation in bias paradigms, and makes typological predictions regarding the relationship between monopolar question meaning and clause type.}

\IfFileExists{../localcommands.tex}{
   \addbibresource{../localbibliography.bib}
   ınput{../localpackages}
   ınput{../localcommands}
   ınput{../localhyphenation}
   \boolfalse{bookcompile}
   \togglepaper[23]%%chapternumber
}{}

\begin{document}
\maketitle



\section{Introduction}\label{sec:11:intro}

Some polar question forms trigger pragmatic inferences that go beyond the simple truth conditional meaning associated with polar interrogatives, globally call\-ed \textit{bias}. The English rising declarative, a particularly well-researched declarative question, presents a prime case of biased question. In \xref{ex1}, if we first observe the use in context, while the polar interrogative felicitously conveys an unbiased hope to find jumper cables, the declarative question infelicitously appears to presume that such a hope is warranted. Further, certain lexical and grammatical elements such as an adverb of complete ignorance like \textit{by chance} require an unbiased context and are infelicitous in a rising declarative.

\ea\label{ex1}\textit{To passerby in a parking lot}\\
Excuse me, but my battery's dead \ldots
\ea[]{
Do you (by chance) have jumper cables?}\label{ex:11:1a}
\ex[\#]{
You've (by chance) got jumper cables?}\label{ex:11:1b} \jambox*{\citep{gunlogson}}
\z
\z

English rising declaratives are also known for their requirement of public evidence in favor of the content proposition of the question \citep{gunlogson}. This notion has come to be known as \textit{contextual bias} or \textit{evidential bias} \citep{sudo-bias}. In \xref{ex:11:2a}, the speaker infers that it is raining on the basis of the contextual evidence of wet coat, and the rising declarative is completely fitting. Whereas in contexts without such evidence \xref{ex:11:2b} or those with counterevidence \xref{ex:11:2c}, a rising declarative is not felicitous. Hence, the rising declarative is said to bring about inferences of positive evidential bias.\footnote{Throughout, I will be using \quotecite{sudo-bias} terminology that distinguishes these three evidential conditions as positive, zero, and negative bias conditions.}


\ea \label{ex:11:2}
\ea \textsc{[Ev. bias +]} \textit{My friend enters my windowless office in a dripping wet coat.}\\
% Did Ali make dinner yesterday?\\
\sn[]{
It's raining? / Is it raining?}\label{ex:11:2a}
\ex \textsc{[Ev. bias 0]} \textit{Speakers are on the phone, no related utterance has been made.}\\ 
\sn[\#]{
It's raining? / Is it raining?}\label{ex:11:2b}
\ex \textsc{[Ev. bias -]} \textit{My friend enters my windowless office in visibly dry summer clothes.}\\ 
\sn[\#]{
It's raining? / Is it raining?}\label{ex:11:2c}
\z
\z
 
\xref{ex:11:2} also shows that in contrast to the rising declarative, the English polar interrogative is felicitous in all three contexts, hence it is termed \textit{evidentially neutral} or \textit{evidentially unbiased}. The English polar interrogative appears to be unrestricted and unmarked in numerous other respects, including the availability of adverbs of complete ignorance, as seen in \xref{ex:11:1a}.
 
Beyond its bias properties, rising declaratives exhibit a host of other peculiarities, some of which are not straightforwardly relatable to evidential bias or bias in general. \xref{ex:11:3} summarizes some poignant cases including usage restrictions (infelicity when used as exam questions or as polite requests) and more structural features (the inability to support negative polarity items and to take part in polar alternative questions). The data in \xref{ex:11:3} come from \citet{huddleston94} and \citet{gunlogson}.

\ea\label{ex:11:3}
\ea\label{ex:11:3a}
\# The empty set is a member of itself? \jambox*{Exam question} %\citep{gunlogson}
\ex\label{ex:11:3b}
\# You can pass me the salt? \jambox*{Polite request} %\citep{huddleston94}
\ex\label{ex:11:3c}
\# Anybody's home? \jambox*{NPI} %\citep{huddleston94}
\ex\label{ex:11:3d}
\# She ordered coffee, or not? \jambox*{Polar Alt} %(Gunlogson 2001)
\z
\z

Notable analyses of the meaning of English rising declaratives treat the evidential bias inferences as an extension of the generally biased nature of the form, typically only making reference to bias in an unspecified manner. Similarly common across the literature is the centrality of clause type as the root cause of these inferences. In the seminal work of \citet{gunlogson}, declaratives are informative and inherently express commitment by their nature. This accounts for the infelicity in examples such as \xref{ex:11:1b}, \xref{ex:11:3a} and \xref{ex:11:3b}, where a commitment by the addressee cannot be said to be involved. The evidential bias observed in \xref{ex:11:2} is a consequence of informativeness in addition to commitment. Uninformativeness with respect to the addressee is a necessary condition for an utterance to be interpreted as a question act. Declaratives can be interpreted as a question only if the addressee is already understood to be publicly committed to the content proposition. For this reason, inferences based on evidence need to be public. \citet{farkas-roelofsen-division} offer a larger framework of bias and neutrality based on principles of markedness. This account relies heavily on the marked status of question forms such as tags and declarative questions as opposed to the default status of polar interrogatives, argued to result in special discourse effects including bias. For \citet{rudin-rd}, who follows the Table framework of \citet{farkas+bruce10}, the rising tune found both on the polar interrogative and the rising declarative serves to block the addition of a speaker commitment in the form of the content proposition $\phi$ to the discourse commitments and forces the addressee to make one, because Issues must be resolved. What separates the rising declarative and the polar interrogative is that the former, unlike the latter, fails to project a future common ground including $\neg\phi$ and in combination with broader pragmatic principles, this leads to the inference that the speaker takes the addressee to have a bias towards $\phi$, which subsumes but also goes beyond evidential bias.

I introduce in this paper the case of an unmarked polar \textit{interrogative} from Turkish which triggers inferences of evidential bias without any other associations of bias. The data thus shows that neither declarative clause type nor bias in the general sense is a necessary condition for evidential bias to arise. The necessary condition, I argue, is an underlyingly monopolar meaning in the sense of \citet{krifka15}. I further show that, despite its interrogative clause type, this form aligns with rising declaratives in a number of ways including all of the features exemplified in \xref{ex:11:3}, which I dub the Ev+ cluster of effects. Moreover, this cluster is partially observable in evidentially biased forms found in other languages, providing further empirical footing for the approach.

For English rising declaratives, the implication of the analysis is that evidential bias and what I will argue to be related effects are driven by the monopolar denotation of these questions, much in the same spirit as \citet{rudin-rd}. However, I do not propose to supplant the cumulative understanding of rising declaratives with one based on evidential bias. That would be contrary to the empirical evidence, as we will see below. What I aim to do is to illustrate the viability of a crosslinguistically and cross-structurally valid alternative to the analysis of the empirical presentation of rising declaratives that concerns evidential bias and related effects.
 
The paper continues in \sectref{sec:11:2} with the empirical discussion, where Turkish polar questions, and specifically the evidentially biased object attachment questions, are introduced, followed by the introduction of the Ev+ cluster. In \sectref{sec:11:3}, an account is sketched which relies on the monopolar question meaning proposed by \citet{krifka15}, which I argue leads to inferences of positive evidential bias and the Ev+ cluster due to its interactions with focus. \sectref{sec:11:4} documents differences between the Turkish and the English Ev+ forms and suggests that these may indeed be clause type-based. \sectref{sec:11:5} compiles existing findings in unambiguous Ev+ forms in two further languages and illustrates the commonness of the features. \sectref{sec:11:6} reconsiders the connection between declarative clause type and Ev+ and suggests that the connection is through the defaultness of the monopolar meaning. \sectref{sec:11:7} concludes the paper.

\section{The evidentially biased Turkish polar interrogative and the Ev+ cluster}\label{sec:11:2}

Let us delve into the two empirical novelties that necessitate a reconsideration of the connection between evidential bias and clause type and the generalization regarding the family of effects that cluster with evidential bias across clause types. Most of the observations in this section will be based on Turkish polar questions, which I first briefly introduce. 

\subsection{Turkish polar questions at a glance}\label{sec:11:2:1}

Polar questions in Turkish are built by the addition of the vowel-harmonic clitic \textit{-mI} to the corresponding declarative. The clitic is focus-sensitive, therefore it attaches to a focused element \citep{Ladd:1996, kornfilt}.  In \xref{ex:11:4b}, a cleft translation is adopted to facilitate the intended focus interpretation, while Turkish polar questions do not show any signs of being clefts or have any other marked morphosyntax other than carrying the clitic \textit{-mI} (boldfaced throughout). The reader is referred to \citet{kamali-krifka} for a recent and empirically detailed treatment of focus in Turkish polar questions.

\ea\label{ex:11:4}
\ea\label{ex:11:4a}
\gll Ali dün yemek yap-tı.\\
Ali yesterday dinner make-\textsc{past}\\
\glt `Ali made dinner yesterday.'
\ex\label{ex:11:4b}
\gll Ali \textbf{mi} dün yemek yap-tı? \\
Ali MI yesterday dinner make-\textsc{past}\\
\glt `Was it Ali that made dinner yesterday?'
\z
\z
	
When no particular element is in focus, there are two possible attachment sites \citep{kamali-wccfl, kamali-diss}. The clitic may be found on the discourse-new object, or on the verb. I will call these \textit{object attachment} and \textit{verb attachment} questions. Both instances may be found in all-new and out-of-the-blue situations such as \xref{ex:11:5}. Angular brackets signify mutually exclusive potential attachment sites.

\ea \label{ex:11:5} \textit{A, B and Ali are roommates. They meet at the breakfast table for the first time after B's month-long trip.}
\sn A: Good morning! I didn't know you were back.
\sn B: Good morning! You were both asleep when I arrived. Tell me \ldots
\sn \gll Ali dün yemek \textbf{$<$mi$>$} yap-tı{} \textbf{$<$mı$>$}?\\
Ali yesterday dinner \phantom{$<$}MI make-\textsc{past} \phantom{$<$}MI\\
\glt `Did Ali make dinner yesterday?'
\z

Both forms are surface-ambiguous with multiple interpretations available due to limits on attachment site among other things. The clitic is attached to the verb if any overt marking in the verbal morphological complex is focused, such as the verb stem or tense. It is also attached to the verbal complex in cases of verum/polarity focus which require no further marking. Note that \xref{ex:11:5} is a context where none of these focus interpretations is supported without accommodation, which we will notice in various other contexts in due course. I use the term \textit{verb attachment} only to refer to the broad focus/focusless interpretation of polar questions carrying the clitic on the verbal complex.

A similar observation pertains to cases where the clitic is found on the object. Unsurprisingly, this attachment site is used for cases of object focus \xref{ex:11:6a}. Not only that, VP focus is also expressed this way \xref{ex:11:6b}, suggesting that the correspondence from object to VP to broad focus is regulated by focus projection in the sense of \citet{selkirk95}. \xref{ex:11:6} shows this through the example of alternative questions, which in Turkish carry \textit{-mI} on both disjuncts. Note, again, that these construals (the part of the AltQ up to the disjunction) are not felicitous in \xref{ex:11:5} without accommodation. So, again, by the term \textit{object attachment}, I will only be referring to the broad focus construal felicitous in \xref{ex:11:5}.

 \ea\label{ex:11:6}
 \ea\label{ex:11:6a}
 \gll Ali  [yemek]$_{F}$  \textbf{mi} yap-tı, yoksa [tatlı]$_{F}$ \textbf{mı}?\\
 Ali dinner MI  make-\textsc{past} or dessert MI\\
\glt `Did Ali make [dinner]$_F$ or [dessert]$_F$?' 
\ex\label{ex:11:6b}
\gll Ali [$_{VP}$ yemek \textbf{mi} yap-tı]$_{F}$ yoksa [$_{VP}$ ders \textbf{mi} \c{c}alış-tı]$_{F}$?\\
Ali {} dinner MI make-\textsc{past} or {} lesson MI study-\textsc{past}\\
\glt `Did Ali [$_{VP}$ make dinner]$_F$ or [$_{VP}$ study his lesson]$_F$?' 
\z
\z

 A consequence of focus projection is that polar questions lacking a discourse-new object will automatically cause the clitic to be on the verb. Hence, data with unergative verbs or those with a discourse-given object (along with the well-known interactions with differential object marking, cf. \citealt{ozge-indef}) are uninformative for the purposes of this paper and are disregarded.  

The meaning of the verb attachment and the object attachment options differ in interesting ways. In particular, as we will shortly see, object attachment shares important features with the English rising declarative. We will also be referring to verb attachment for comparison and for a fuller understanding of the paradigm. 

\subsection{The evidentially biased polar interrogative}\label{sec:11:2:2}

In \xref{ex:11:5}, both clitic placement options may be possible, but they have different meanings. Object attachment, in particular, implicates the presence of a non-linguistic cue in favor of the content proposition (first intuited to my knowledge by \citealt{gokselkerslake}). Consider possible continuations to the conversation in \xref{ex:11:5}, which are not interchangeable.
\ea\label{ex:11:7}
A: Yes. Why do you ask?
\sn B$^\prime$: \text{[}Continuing from object attachment\text{]} When he cooks, he always leaves the kitchen in this kind of mess, that's why.
\sn B$^{\prime\prime}$: \text{[}Continuing from verb attachment\text{]} I'm starving, that's why.
\z

Indeed, object attachment questions carry positive evidential bias exactly like English rising declaratives.  They are licensed  in the presence of evidence \xref{ex:11:8a}, and not in the absence of evidence \xref{ex:11:8b} or in the presence of counterevidence \xref{ex:11:8c}. For this and other resons we will see throughout, object attachment questions and similar forms are translated with an English rising declarative in \xref{ex:11:8} and elsewhere.\footnote{I save the English polar interrogative to translate verb attachment questions and polar questions where attachment site does not matter for the point being made. As I will explore, neither corresponcence is perfect, but this two-way division is both maximally accurate and, I believe, reader-friendly.} To aide the reader, when a translation of a polar question is itself infelicitous in the given (pragmatic or structural) context, I prefix the judgement in square brackets. I will be referring to positive evidential bias as \textit{Ev+} in short.
\ea\label{ex:11:8}
\ea\label{ex:11:8a} \textsc{[Ev. bias +]} \textit{Speaker observes tell-tale signs of Ali's recent cooking.}
\sn[]{ 
\gll Ali dün yemek \textbf{mi} yap-tı?\\
Ali yesterday dinner MI make-\textsc{past} \\
\glt `Ali made dinner yesterday?'}
\ex\label{ex:11:8b} \textsc{[Ev. bias 0]} \textit{Speakers are on the phone. No related utterance has been made.}
\sn[\#]{
Ali dün yemek \textbf{mi} yaptı?
\glt `[\#]Ali made dinner yesterday?'
}
\ex\label{ex:11:8c} \textsc{$[$Ev. bias -]} \textit{Speaker notices the kitchen looking and smelling exactly as they left it the previous day.}
\sn[\#]{
Ali dün yemek \textbf{mi} yaptı?
\glt `[\#]Ali made dinner yesterday?'
}
\z
\z

From the perspective of what we know about English rising declaratives, we could expect the Ev+ object attachment questions to be declarative questions. This is not correct, however. Both verb attachment questions and object attachment questions are interrogatives. Let us see why.

\largerpage[2]
First,  contexts of unresolved question typically require a true interrogative, barring declarative questions \citep{gunlogson} \xref{ex:11:9b}.  In Turkish, the clitic \textit{-mI} is required in such contexts regardless of position, hence both object and verb attachment align with the English polar interrogative and not with with declarative question \xref{ex:11:9a}.
\ea\label{ex:11:9}
\ea[]{ 
\gll Bir soru cevapsız. Ali  yemek $<$\textbf{mi}$>$ yap-tı{} $<$\textbf{mı}$>$?\\
one question answerless Ali dinner \phantom{$<$}MI make-\textsc{past} \phantom{$<$}MI\\
\glt`One question remains. Did Ali make dinner?'
}\label{ex:11:9a}
\ex[\#]{
One question remains. Ali made dinner?
}\label{ex:11:9b}
\z
\z

Secondly, interrogatives are incompatible with certain biased adverbs such as adverbs of speaker certainty or evidentiality \citep{huddleston94}, whereas declarative questions are fine with them \citep{gunlogson}.  In Turkish, neither object nor verb attachment questions may feature these adverbials \xref{ex:11:10a}, again patterning with interrogatives.
\ea\label{ex:11:10}
\ea[\#]{
\gll Müdür  buna  kuşkusuz/{besbelli}  izin $<$\textbf{mi}$>$ verdi $<$\textbf{mi}$>$?\\
manager   this certainly/evidently  permission \phantom{$<$}MI gave \phantom{$<$}MI\\
\glt `[\#]Has the manager certainly/evidently given permission for this?'
}\label{ex:11:10a}
\ex[]{
The manager has certainly/evidently given permission for this?
}\label{ex:11:10b}
\z
\z

Thirdly, interrogatives, but not declaratives, can be embedded under a rogative predicate. \xref{ex:11:11a} shows a grammatical embedding under \textit{wonder} of both verb and object attachment options. \xref{ex:11:11b} shows that the clitic  is obligatory in this complement clause, indicating, again, that both forms are interrogatives.
\ea\label{ex:11:11}
\ea[]{\gll
Ali  yemek $<$\textbf{mi}$>$ yap-tı{} $<$\textbf{mı}$>$, merak ed-iyor-um.\\
Ali  dinner \phantom{$<$}MI make-\textsc{past} \phantom{$<$}MI wonder do-\textsc{pres-1sg} \\
\glt `I wonder if Ali made dinner.'
}\label{ex:11:11a}
\ex[*]{
Ali yemek yaptı, {} merak ediyorum.
\glt (`I wonder if Ali made dinner.')
}\label{ex:11:11b}
\z
\z

The equally common nominalized embedding form can also be used to show this, but with a few clarifications. \xref{ex:11:12a} shows the baseline example where a declarative is embedded under a non-rogative verb, rejecting a rogative verb. When  the verb attachment  form is nominalized, the clitic has to be replaced by a periphrastic construction expanding the standard nominalization with a V-not-V form \xref{ex:11:12b}. The nominalization of the object attachment question, in contrast, retains the clitic on the embedded object while using the standard nominalization \xref{ex:11:12c} \citep{kamali-diss}. In neither case does nominalization itself suffice to embed the interrogative clauses, indicating that the periphrasis in \xref{ex:11:12b} and the clitic in \xref{ex:11:12c} are interrogative markers. Indeed, neither form accepts a non-rogative matrix predicate and requires a rogative one. The meaning of \xref{ex:11:12b} and \xref{ex:11:12c} are for all relevant purposes identical to the meaning of \xref{ex:11:11a} and \xref{ex:11:11b}.
\ea\label{ex:11:12}
\ea\label{ex:11:12a}
\gll Merve [Ali'nin            yemek {yap-tığ-ın]-ı}             sanıyor / *soruyor.\\
Merve   {\db}Ali-\textsc{gen} dinner {make-\textsc{nomin-3sg-acc}} thinks {} asks\\
\glt `Merve thinks/*asks that Ali made dinner.' 
\ex\label{ex:11:12b}
\gll Merve [Ali'nin yemek yap-ıp {yap-ma-dığ-ın]-ı} soruyor / *sanıyor.\\
Merve Ali-\textsc{gen} dinner make-\textsc{conv} {make-\textsc{neg-nomin-3sg-acc}} asks {} thinks\\
\glt `Merve asks/*thinks if Ali made dinner.' 
\ex\label{ex:11:12c}
\gll Merve [Ali'nin yemek \textbf{mi} {yap-tığ-ın]-ı} soruyor / *sanıyor.\\
Merve Ali-\textsc{gen} dinner MI {make-\textsc{nomin-3sg-acc}} asks {} thinks\\
\glt `Merve asks/*thinks if Ali made dinner.'
\z
\z
%`Merve asks if it's dinner Ali is making.'  

So, we see here a polar question form, the Turkish object attachment question, that is restricted to positive evidential bias contexts while being an interrogative through and through. As it is not a declarative, the Ev+ feature of the Turkish object attachment question cannot be due to clause type. The competing form in the paradigm, the verb attachment question, is also an interrogative, so there are no surprises there.

Now, even if both object attachment and verb attachment questions are interrogatives, there could be a markedness asymmetry between the two forms. This could put the object attachment form in a marked position and may lead to its associations with bias. When we look at the full paradigm with verb attachment questions, we see that this cannot be the case. Departing from the distribution of polar interrogatives in the English paradigm, the competing form in Turkish, verb attachment, is not unmarked. 

Recall that the English polar interrogative is in many ways a default. Directly related to our purposes, it is evidentially neutral \xref{ex:11:2}. The Turkish verb attachment question, in contrast, is not a default. We will see several other reflections of this in due course. For now, observe that it is not evidentially neutral \xref{ex:11:13}. Namely, it is banned from positive evidential contexts \xref{ex:11:13a} (it is Ev-/Ev0, or has ``anti-evidential bias").  As such, it is in complementary distribution with the object attachment form. (As English does not have a form that parallels this distribution, the translations given are approximate.)
 \ea \label{ex:11:13}
 \ea \textsc{[Ev. bias +]} \textit{Speaker observes tell-tale signs of Ali's recent cooking.}
 %Did Ali make dinner yesterday?\\
 \sn[\#]{
 \gll Ali dün yemek  yap-tı{} \textbf{mı}?\\
 Ali yesterday dinner make-\textsc{past} MI\\
\glt (`Did Ali make dinner yesterday?')
}\label{ex:11:13a}
\ex \textsc{[Ev. bias 0]} \textit{Speakers are on the phone. No related utterance has been made.}
\sn[]{
Ali dün yemek  yaptı{} \textbf{mı}?
\glt `Did Ali make dinner yesterday?'}\label{ex:11:13b}
\ex \textsc{$[$Ev. bias -]} \textit{Speaker notices the kitchen looking and smelling exactly as they left it the previous day.}
\sn[]{
Ali dün yemek  yaptı{} \textbf{mı}?
\glt `Did Ali make dinner yesterday?'
}\label{ex:11:13c}
\z
\z

This data shows that Turkish object attachment and verb attachment questions are in complementary distribution with respect to evidential contexts. As they share the pie equally, it is impossible to describe one as the default. For this reason, a markedness-based account, even if it can be formulated bypassing a clause type markedness, cannot easily explain why the object attachment form is biased in a similar way to the English rising declarative, nor why the verb attachment form is anti-biased in this respect.

Let us finally briefly consider Turkish declarative questions. A curious aspect of the Turkish paradigm of question forms is  the very restricted use of declarative questions. If these were categorically missing, one could speculate that the meaning of the object attachment question exceptionally subsumes a normally declarative-related function, or that the paradigm is too unusual to challenge our existing generalizations without further empirical understanding. But declarative questions do exist in Turkish. \xref{ex:11:14} provides a real life example.\footnote{I am not aware of work on the meaning of Turkish declarative questions. My impression is that they are used when the content of the answer can be taken to be almost certain. Sarcastic echoes may also use this form, where the sarcasm appears to be relying exactly on the certainness of the answer. In both cases, most clearly with the airline employee who would not move to the next passenger otherwise, an answer is nevertheless expected.}

\ea \label{ex:11:14} \textit{Airline employee is screening passengers at the check-in queue.}
\sn
\gll Bagaj-ınız yok? Pasaport-unuz yan-ınız-da?\\
baggage-\textsc{poss.2pl} not.exists passport-\textsc{poss.2pl} side-\textsc{poss.2pl-loc}\\
\glt Lit. `You have nothing to check in? You have your passport?'
\z

As expected of a declarative question, these forms may include an adverb of certainty but may not be embedded under \textit{wonder}. 

\ea \label{ex:11:15} \textit{[Same context as \xref{ex:11:14}]}
\ea[]{
\gll Bagaj-ınız yok? Pasaport-unuz herhalde yan-ınız-da?\\
baggage-\textsc{poss.2pl} not.exists passport-\textsc{poss.2pl} presumably side-\textsc{poss.2pl-loc}\\
\glt `You have nothing to check in? You presumably have your passport?'
}\label{ex:11:15a}
\ex[*]{\gll 
Bagaj-ınız yok, merak ed-iyor-um. \\
baggage-\textsc{poss.2pl} not.exists wonder do-\textsc{pres}-\textsc{1sg} \\
\glt (`I wonder if you have nothing to check in.')}\label{ex:11:15b}
\z
\z

What appears to be behind the relative scarcity of declarative questions in Turkish is that they cover a smaller range of uses compared to English declarative questions. Much of the usage space is covered by object attachment questions.

In sum, in Turkish, a proper polar interrogative exhibits the evidential bias profile of English rising declaratives. This form is not more marked than the other major form in the paradigm. Neither is the paradigm too idiosyncratic to be discarded as invalid. Hence, the Turkish facts call for an account of evidential bias that is divorced from clause type or markedness. 

\subsection{The Ev+ cluster}\label{sec:11:2:3}

Now, because the object attachment question is an interrogative and not a declarative question, it could be possible that this construction does not share any other features of the English rising declarative than the evidential bias profile. However, this could not be farther from the truth. In fact, the two questions share a striking number of features. This will lead us to a novel empirical generalization: there is a class of features shared by evidentially positive question forms across clause types, which I will be referring to as the \textit{Ev+ cluster}. 

This exploration of the Ev+ cluster of effects has two dimensions. First, I will be
documenting further parallelisms between English rising declaratives and Turkish object attachment questions, which I will call \textit{unambiguous Ev+ forms} because of their evidential bias profile. For this, I ask the reader to initially pay attention to the parallels between the object attachment distribution given in the example and the provided rising declarative translation (with felicity included in square brackets as before). An alternative translation in the form of a polar interrogative is included to allow for a comparison to the felicitous neutral form in English. I will simultaneously be documenting the complementary distribution between object attachment questions and verb attachment questions in Turkish, which will be visible in the contrasting felicity associated with the two potential attachment sites of the clitic in the examples. I will ask the reader to pay attention to the verb attachment distribution toward the end of the section when we observe the complementary distribution between the two Turkish forms, which will inform us further about the Ev+ cluster.

Now starting with the two unambiguous Ev+ forms, as we have seen in \sectref{sec:11:intro}, rising declaratives are barred from certain uses such as exam questions. This list extends to various sorts of questionnaire questions, and court questions that are not triggered by contextual evidence or conversational content. Turkish object attachment questions, which are interrogatives, exhibit the same distribution.

\protectedex{\ea \textit{Exam question}
\sn
\gll Türkiye İkinci Dünya Savaşı'na $<$\#\textbf{mı}$>$ gir-di $<$\textbf{mi}$>$?\\
Turkey Second World War-\textsc{dat} \phantom{$<$\#}MI enter-\textsc{past} \phantom{$<$}MI \\
\glt `[\#]Turkey fought in WW2?'\\
`Did Turkey fight in WW2?'
\ex \textit{Questionnaire}
\sn 
\gll Son iki hafta i\c{c}inde yumuşatıcı{} $<$\#\textbf{mı}$>$ kullan-dı-nız $<$\textbf{mı}$>$?\\
last two week inside fabric.softener \phantom{$<$\#}MI use\textsc{-past-2pl} \phantom{$<$}MI\\
\glt `[\#]You have used fabric softener in the last two weeks?'\\
`Have you used fabric softener in the last two weeks?'
\ex \textit{Routine traffic checkpoint}
\sn 
\gll Alkol $<$\#\textbf{mü}$>$ al-dı-nız $<$\textbf{mı}$>$?\\
alcohol \phantom{$<$\#}MI take\textsc{-past-2pl} \phantom{$<$}MI\\
\glt `[\#]You have had alcohol?'\\
`Have you had alcohol?'
\z
}

Similarly, neither English rising declaratives nor Turkish object attachment questions may be used in indirect questions expressing polite requests, offers and invitations.

\ea Your roommate says they are coming home and you need bread.
\sn 
\gll Ekmek $<$\#\textbf{mi}$>$ al-ır $<$\textbf{mı}$>$-sın?\\
bread \phantom{$<$}MI get-\textsc{aor} \phantom{$<$}MI-\textsc{2sg}\\
\glt `[\#]You could buy bread?'\\
`Could you buy bread?'\footnote{The clitic occurs to the left of the agreement marker following certain tense-aspect-modality markers.}  
\ex\textit{Hosting a guest.}
\sn
\gll Pasta $<$\#\textbf{mı}$>$ iste-r $<$\textbf{mi}$>$-sin?\\
cake  \phantom{$<$}MI want-\textsc{aor} \phantom{$<$}MI-\textsc{2sg}\\
\glt `[\#]You would like cake?'\\
`Would you like cake?'
\ex\textit{At a ballroom:}
\sn
\gll Benim-le dans $<$\#\textbf{mı}$>$ ed-er $<$\textbf{mi}$>$-sin?\\
me-\textsc{com} dance \phantom{$<$\#}MI do-\textsc{aor} \phantom{$<$}MI-\textsc{2sg}\\
\glt `[\#]You will dance with me?'\\
`Will you dance with me?'\footnote{The nominal in light verb constructions counts as syntactic object.}
\z

Two structural features exhibited by the rising declarative are the impossibility to support negative polarity items and polar alternative questions expressed with \textit{or not}. These, too, hold across clause types, as the Turkish object attachment question is also restricted in this way. In \xref{ex:11:22}, a neg-word's inability to be licensed by object attachment is shown. Note that this is ungrammatical even when the clitic \textit{-mI} attaches to the neg-word, hence presumably has it in its local scope. I use neg-word glossing and assume that neg-words constitute a type of negative polarity item following \citet{Giannakidou:2000}.\footnote{An unlicensed neg-word is considered to be a failure of negative concord, and hence lead to ungrammaticality rather than infelicity. See \citet{kamali-zeijlstra-nc} on Turkish neg-words.}
 
\ea\label{ex:11:22}
\gll Ev-de kimse $<$*\textbf{mi}$>$ var $<$\textbf{mı}$>$ ?\\
house-\textsc{loc} n-body \phantom{sis}MI exists \phantom{ss}MI\\
\glt `[\#]Anybody’s home?’\\
`Is anybody home?’\footnote{Subjects of unaccusatives count as objects.}
\z

In \xref{alt}, the similarly ungrammatical polar alternative question with object attachment is given. The alternative question construction requires the clitic in both alternatives as before, and negation must appear inside the verbal morphological complex as it is a bound morpheme.

\ea\label{alt}
\gll O kahve $<$\#\textbf{mi}$>$ s\"oyle-di $<$\textbf{mi}$>$, yoksa s\"oyle-me-di \textbf{mi}?\\
she coffee \phantom{sis}MI order-\textsc{past} \phantom{ss}MI or order-\textsc{neg-past} MI \\
\glt `[\#]She ordered coffee, or not?’\\
`Did she order coffee, or not?’
\z

So, contexts like exams, questionnaires, court questions, indirect questions, and certain markers of negativity and polarity are systematically rejected by the two unambiguous Ev+ forms we have been looking at -- the English declarative question and the Turkish object attachment question. The familiar wisdom takes it that a polite request or negative polarity licensing do not go with a rising declarative in some way or another because it is not a proper interrogative. With the Turkish object attachment question we have a proper interrogative in our hands that nevertheless has the same set of properties. I refer to this set of properties of unambiguous Ev+ forms that hold across clause types  the \textit{Ev+ cluster}. 

Our characterization of the Ev+ cluster so far relies on contexts from which both the Turkish object attachment question and the English rising declarative are excluded. Are there also contexts which require these forms and exclude others? The answer is yes, but the English polar question paradigm will not help us here. Because of its default and neutral status, no context excludes the English polar interrogative. The Turkish paradigm supplies the contrast we need. 

As we have observed with its evidential bias profile in \xref{ex:11:13}, the Turkish verb attachment question is not a neutral and default form like the English polar interrogative. If the reader now considers the verb attachment distribution in the examples given so far in this section, it will be seen that verb attachment is felicitous in all of the contexts object attachment is ruled out, mirroring the distribution of the English polar interrogative. But unlike the English paradigm, the two forms in Turkish are in complementary distribution: there are also contexts that exclude verb attachment questions and require object attachment questions. Also perfect for English rising declaratives, felicity in these contexts instantiate a positive Ev+ cluster feature. Two such contexts are responses to requests for guesses and echoic responses to all-new utterances, given in \xref{ex:11:24} and \xref{immres}. 

%Now recall that the object attachment question is only one side of the coin in the Turkish paradigm. If the reader now considers the permitted clitic placement in the examples given so far in this section, it will be seen that verb attachment is well-formed in all of the contexts object attachment fails to be licensed. However, as we have observed with its evidential bias profile, verb attachment is also not a default form. Two situations where verb attachment is infelicitous while object attachment is, are given in \xref{ex:11:24} and \xref{immres}. These are responses to requests for guesses and echoic responses to all-new utterances. English rising declaratives are also felicitous in these contexts, as translations indicate.\footnote{ \citet{rudin-rd} shows that echoic uses pose a problem for approaches where rising declaratives encode positive speaker epistemic bias, as they may express contradiction.}

\ea\label{ex:11:24} A:
Guess what happened/why the kitchen is a mess.
\sn B: \gll Ali  yemek  $<$\textbf{mi}$>$ yap-tı{} $<$\#\textbf{mı}$>$ ?\\
Ali  dinner \phantom{ss}MI make-\textsc{past} \phantom{sss}MI\\
\glt \phantom{B:} `Ali made dinner?’\\
\phantom{B:} `Did Ali make dinner?’
\ex \label{immres} 
A: \gll Sonra Ali yemek yap-tı.\\
then Ali dinner make-\textsc{past}\\
\glt \phantom{B:} `And then Ali made dinner.'
\sn B: \gll Ali yemek $<$\textbf{mi}$>$ yaptı{} $<$\#\textbf{mı}$>$?\\
Ali  dinner \phantom{ss}MI made \phantom{sss}MI\\
\glt \phantom{B:} `Ali made dinner?’\\
\phantom{B:} `Did Ali make dinner?’ \\
\phantom{B:} (Is that what you said?/ No way!/ Hm, and then?)
\z

All of this put together, a cluster of features emerges as the Ev+ cluster, which I summarize in \tabref{tab:11:tab1}. Verb attachment questions which do not have an English analogue, cover the complementary set of contexts.

\begin{table}
\caption{The Ev+ cluster based on English rising declarative and Turkish object attachment question}
\label{tab:11:tab1}
\begin{tabularx}{0.8\textwidth}{X rr} 
\lsptoprule
  
{} & English &  Turkish \\  %& Hungarian & Japanese \\ \hline
  \midrule
Unambiguous Ev+ & yes & yes \\ %& yes & yes  \\ 
As exam/questionnaire/court question & no & no \\ %& no & ?  \\ %\hline
As indirect question & no & no \\ %& no & ?   \\ %\hline
With \textit{or not} & no & no \\% & ? & ? \\ %\hline
With NPI/neg-word & no & no \\ %& no & no  \\ \hline
As guess \& echoic & yes & yes \\ %& ? & ?  \\ %\hline
  \lspbottomrule
\end{tabularx} 
\end{table}

The Ev+ cluster includes features shared by the English rising declarative and the Turkish object attachment question as our investigation has shown so far. I take these unexpected parallelisms to be a valuable starting point because they hold across clause types and unrelated languages. More features may be discovered that belong here, but probably not those we readily know not to be shared between the two forms. By this reasoning, the null hypothesis would be that features of the two forms that fall outside of the intersection are independent effects. We have seen some such features in \xref{ex:11:9} through \xref{ex:11:11}. We will revisit these and a few more pieces of similar data to develop a perspective of residual bias effects in \sectref{sec:11:4}. On the other hand, further examination may prove that some features cluster together accidentally. A crosslinguistic perspective, which I will lay out in \sectref{sec:11:5}, may help determine the core of effects that really are due to the same underlying structural phenomenon and tease apart others.

Ideally, the task is to account for the Ev+ cluster, not just positive evidential bias, as the elements of the cluster remain constant across two starkly different constructions. The account must not be based on clause type, because these features are observed in a declarative question on the one hand and an interrogative on the other. Features not shared between the two Ev+ forms, i.e. features that fall outside of the cluster, must stem from other reasons. Finally, the complement set of features, not present in an English form but represented by verb attachment questions in Turkish, must also find an explanation.   

\section{The proposal}\label{sec:11:3}

The Ev+ cluster and its complement can both receive an explanation under the notions of \textit{monopolar} and \textit{bipolar questions} of \citet{krifka15}. Specifically, the monopolar question meaning provides the core meaning which leads to positive evidential bias and the Ev+ cluster, while the Turkish verb attachment question stands out as a true bipolar question manifesting the absence of these effects. The account does not make reference to clause type. Polar interrogatives as well as declarative questions may have this underlying meaning (but see \sectref{sec:11:6} on why declaratives may have to).

\citet{krifka15} argues that there are two possible polar question meanings. The bipolar meaning denotes the speaker's invitation to restrict future commitments in the discourse to either the content proposition $\phi$ or its complement $\neg\phi$, similar in essence to the standard two-membered set of polar question meaning \citep{Hamblin:1973, Karttunen:1977}. These questions may be answered with a conclusive \textit{no} or a negative fragment and take part in polarity alternative questions. I simplify below.

\ea Bipolar question: $\phi$? = Do you commit to $\phi$ or $\neg\phi$?
\sn A: I'm hungry. Did you make dinner (or not)?
\sn B: Yes. / No. / (No,) I didn't.
\z

The second meaning is the monopolar meaning, which denotes an invitation to restrict future commitments to the content question itself without its complement.\footnote{An earlier proposal of this sort is due to \citet{biezma-rawlins12} where a monopolar propositional content is interpreted with a Q operator. Their account is stronger, though, in postulating this meaning to be the default polar question meaning. Hence, the English polar interrogative as well as presumably the rising declarative would receive a monopolar analysis.} These questions may not occur with the polar alternative but rather may occur with an open alternative such as \textit{or what}. Their possible answers are also restricted, making a conclusive \textit{no} answer degraded and a negative fragment even worse. This is due to the absence of the complement proposition $\neg\phi$ from the meaning. 

\ea Monopolar question: $\phi$? = Do you commit to $\phi$?
\sn A: It smells delicious. Did you make dinner (or what)? %/ You made dinner?
\sn B: Yes. / ?No. / \#(No,) I didn't.%/No, a restaurant just opened downstairs.
\z

Krifka already envisions that English rising declaratives are biased towards the \textit{yes} answer because they are monopolar. Differently from Krifka, I would like to connect the monopolar meaning to evidential bias in particular, and along with that to the rest of the Ev+, with no association to clause type.

\subsection{Rising declaratives and object attachment questions are monopolar}\label{sec:11:3:1}

First, let us observe that rising declaratives and object attachment questions both behave in a way expected of monopolar questions. A conclusive \textit{no} answer to a rising declarative or an object attachment question is degraded.  A negative predicate fragment is unacceptable. (For contrast, see \xref{ex:11:cat} and \xref{sprinklers} for examples of what I take to be non-conclusive \textit{no} answers.)
\ea \label{oa-answer}
A: Ali yemek  \textbf{mi} yaptı?\\
\glt \phantom{A:} `Ali made dinner?'
\sn B:
\gll Evet. /?Hayır. /\#(Hayır,) yap-ma-dı.\\
 yes \phantom{/?}no \phantom{/\#(}no make-\textsc{neg-past} \\
\glt \phantom{B:} `Yes. / ?No. / \#(No,) he didn't.'
\z

As we see in \xref{alt} and \xref{polalt-oa}, neither rising declaratives nor object attachment questions can take part in a polar alternative question. The infelicity of the \textit{no} answer and the polar alternative are in line with the missing alternative $\neg\phi$ from the monopolar meaning. 
\ea[\#]{
\gll Ali yemek \textbf{mi} yap-tı, yoksa (Ali yemek) yap-ma-dı{} \textbf{mı}?\\
Ali dinner MI make-\textsc{past} or Ali dinner make-\textsc{neg-past} MI\\
\glt `[\#]Ali made dinner, or (did he) not?'
}\label{polalt-oa}
\z

Recall that the Turkish object attachment question is a proper polar interrogative, so it should in principle be able to take part in a polar alternative question or be answered with a \textit{no} in a conclusive manner, which might not be said that easily of the rising declarative. Still, this polar interrogative diverges from the expected pattern in line with the usage restrictions we have observed earlier.

Where both Turkish object attachment questions and English rising declaratives do occur is contexts where distinct propositional alternatives are evaluated instead of the polar alternative $\neg\phi$, such as $\psi$ \textit{the cat knocked over the shelves}, or \textit{Hasan will order pizza} etc. We can observe the surfacing of such an alternative, for instance, in the form of a voluntary addition to a \textit{no} answer as in \xref{ex:11:cat}. It is noteworthy that the speakers appear to be spontaneously converging on a Question Under Discussion (QUD) \citep{Roberts:1996} like \textit{What happened?}.
\ea \label{ex:11:cat}
A: Ali yemek  \textbf{mi} yaptı?
\glt \phantom{A:}`Ali made dinner?'
\sn B: \gll Hayır, kedi raflar-ı{} devir-di.\\
no cat shelves-\textsc{acc} knock.over-\textsc{past}\\
\glt \phantom{B:} `No, the cat knocked over the shelves.'
\z
 
Another example where distinct propositional alternatives surface under mo-nopolar question forms is propositional alternative questions with distinct propositional alternatives \xref{propalt-oa}. Here, again, $\phi$ and $\psi$ are contrasted. This is in stark contrast to the unavailability of polar alternative questions contrasting $\phi$ and $\neg\phi$ given in \xref{polalt-oa}.
\ea\label{propalt-oa}
\gll Ali yemek \textbf{mi} yap-tı, yoksa Hasan pizza \textbf{mı} s\"oyle-yecek?\\
Ali dinner MI make-\textsc{past} or Hasan pizza MI order-\textsc{fut}\\
\glt `Did Ali make dinner or will Hasan order pizza?' \\
`Ali made dinner or Hasan will order pizza? (Which one is it?)'
\z

\xref{propalt-oa} suggests that a monopolar question with content proposition $\phi$ can project broad focus alternatives similar to an assertion of the proposition $\phi$. Indeed, the same $\phi$ in an assertion \xref{ex:11:32} and a rising declarative \xref{ex:11:33} may be cued by \textit{what happened}, a trigger of broad focus alternatives. \xref{ex:11:33} would be expressed by object attachment in Turkish, skipped here for reasons of space.
\ea\label{ex:11:32} A: What happened?
\sn B: The shelves collapsed.
\sn Other broad focus alternatives not picked up by the speaker: \{Somebody broke the window, officemate fell off the ladder while changing a lightbulb, the loudspeakers malfunctioned \ldots\}
\ex \label{ex:11:33} A: What happened? The shelves collapsed? \\
\sn Other broad focus alternatives not picked up by the speaker: \{Somebody broke the window, officemate fell off the ladder while changing a lightbulb, the loudspeakers malfunctioned \ldots\}
\z

So, both English rising declaratives and Turkish object attachment questions, the difference in their clause types notwithstanding, can be argued to exhibit the monopolar question meaning.\footnote{\citet{rudin-rd} also argues English rising declaratives to be monopolar.} This meaning lacks the polar alternative but allows the generation of broad focus alternatives via focus. In contrast, as we will see below, bipolar questions only lead to the two polar alternatives. One consequence of the monopolar/bipolar division is that the polar alternative $\neg\phi$ and broad focus alternatives \{$\psi$, $\pi$ \ldots\} do not mix.  

\subsection{Verb attachment questions are bipolar}\label{sec:11:3:2}

Equipped with this idea, let us consider bipolar questions. English does not have an unambiguous bipolar question form. The English polar interrogative may be found in an alternative question with distinct propositional alternatives as in \xref{propalt-oa} as well as a polar alternative question as in \xref{polalt-oa}, see \xref{pi}.
\ea \label{pi} 
\ea Did Ali make dinner or will Hasan order pizza?
\ex Did Ali make dinner, or not?
\z
\z

 But Turkish verb attachment questions, complementary to object attachment questions in a range of features reviewed earlier, behave perfectly like unambiguous bipolar questions. In terms of the felicity of a \textit{no} answer and the polar alternative, we see that verb attachment supplies the correct form.

\ea \label{va-answer} 
\ea
A: \gll Ali yemek yaptı{} \textbf{mı}?\\
Ali dinner make-\textsc{past} MI \\
\glt \phantom{A:} `Did Ali make dinner?'
\ex B: \gll Evet. / Hayır. / (Hayır,) yap-ma-dı.\\
yes {} no {} \phantom{(}no make-\textsc{neg-past}\\
\glt \phantom{B:} `Yes. / No. / (No,) he didn't.'
\z
\ex \gll Ali yemek yap-tı{} \textbf{mı}, yoksa (Ali yemek) yap-ma-dı{} \textbf{mı}?\\
Ali dinner make-\textsc{past} MI or (Ali dinner) make-\textsc{neg-past} MI\\
\glt `Did Ali make dinner, or (did he) not?' \label{polalt-va} 
\z
 
Conversely, verb attachment is infelicitous in cases relying on a distinct propositional alternative like \xref{propalt-oa}, illustrating the unavailability of focus alternatives when bipolar polar alternatives are in question \xref{propalt-va}. As with the object attachment question, this form is a proper polar interrogative and should not have a problem being in a proposition-level alternative question, but in fact it does.  

\ea[\#]{
\gll Ali yemek yap-tı{} \textbf{mı}, yoksa Hasan pizza s\"oyle-yecek \textbf{mi}?\\
Ali  dinner make-\textsc{past} MI or Hasan pizza order-\textsc{fut} MI\\
\glt (`Did Ali make dinner, or will Hasan order pizza?')
}\label{propalt-va}
\z

Similarly, \xref{cat-va} shows the closest verb attachment equivalent of the discourse in \xref{ex:11:cat}. Unlike object attachment, with verb attachment there is a clear judgement that B's \textit{no} answer resolves the QUD. An attempted continuation parallel to the corresponding object attachment example is infelicitous \xref{cat-va:a}. One may utter this proposition next in the discourse, but it must be separated from the previous QUD with a full prosodic and topical reset \xref{cat-va:b}. 

\ea\label{cat-va} 
\ea A: Ali yemek  yaptı{ }\textbf{mı}?
\glt `Did Ali make dinner?'
\ex \label{cat-va:a} B: \# Hayır, kedi rafları{ }devirdi.
\glt (`No, the cat knocked over the shelves.’)
\ex \label{cat-va:b} B$^{\prime}$: Hayır. Fark ettin mi? Kedi rafları{ }devirdi.
\glt `No. Have you noticed? The cat knocked over the shelves.'
\z
\z

Based on these tests, verb attachment questions prove to be a good candidate for a true bipolar question. Conversely, object attachment questions and rising declaratives both demonstrate features expected of monopolar questions. Between the two meanings, not only the presence of the polar alternative $\neg\phi$, but also the concomitant availability of distinct focus alternatives varies.

\ea
Bipolar question (\{$\phi$, $\neg\phi$\}) : Verb attachment question (Turkish) \\
Monopolar question ($\phi$) : Object attachment question (Turkish)\\
\phantom{Monopolar questoon ($\phi$) :} Rising declarative (English)
\z

With the proposed distinction, on the one hand the abundantly clear complementary distribution between the two forms in Turkish can be derived in a straightforward way. This is in contrast to English, where the polar interrogative is effectively ambiguous between the two meanings.\footnote{Recall \xref{pi}. Whether this is due to some hidden structural ambiguity or another process is not clear at the moment, but see \sectref{sec:11:6} for some related discussion.} On the other hand, the division collapses the two Ev+ forms together and provides a common underlying reason for the cluster of effects in their monopolar meaning. 

\subsection{How to derive the Ev+ Cluster?}\label{sec:11:3:3}

I will now turn to relating the monopolar question meaning to the Ev+ cluster. This will be a proof of concept rather than a fully fleshed out analysis. Here is the idea in a nutshell: The monopolar meaning, not the bipolar meaning, provides the necessary conditions for the Ev+ cluster. It does not provide sufficient conditions for all of the effects, however, among them positive evidential bias. This particular inference can be said to arise from an accommodation of the QUD in the absence linguistic cues to establish one.


Fist, the necessary conditions. We have seen in \sectref{sec:11:2} that rising declaratives and object attachment questions both work perfectly when the content proposition is a \textit{try-out}: including an inference based on public evidence, guess, or perception of the speaker regarding what the addressee has said. The tried out possibility is one among many.  In contrast, contexts that exclude these two forms and in fact require the bipolar verb attachment form in Turkish are like exams. There is a right answer and a wrong answer which is the logical complement, instead of numerous possibilities.

This difference, I suggest, stems from the difference in the sets of alternatives supported by the two meanings. There are two facets to this, which I will not try to reduce to one in this paper (but presume to be reducible): the first is the possibility of unlimited propositional focus alternatives \{$\phi$, $\psi$, $\pi$ \ldots\} afforded by the focus semantic value of monopolar questions. This fits the function of trying out, because  when venturing a guess or making an inference about the true state of affairs behind my friend's wet coat or the mess in the kitchen, I am picking one possibility among many. Likewise, when I ask a confirmation question or echo the addressee's last statement in surprise or sarcasm, I am picking what I possibly misheard, or pretend to have misheard, as the one utterance by the addressee which could have been any number of similar utterances. Hence, the set against which the content proposition is evaluated is the set of focus alternatives. Let us take the wet coat scenario with the rising declarative as the unambiguously monopolar form. The dialogue in \xref{sprinklers} does not automatically come to an end if an initial guess is wrong. It will be continued with another, perhaps less likely guess \xref{sprinklers:c}.\footnote{A2 indicates A’s second turn in conversation responding to B.} It is in fact odd to stop guessing if the perceived question under discussion has not been resolved while the dialogue is still continuing \xref{sprinklers:a:bar}. 

\ea\label{sprinklers} \textit{A's friend enters their windowless office in a dripping wet coat.}
\ea \label{sprinklers:a} A: It's raining?
\ex B: Nope.
\ex \label{sprinklers:c} A2: Oh dear, someone activated the sprinklers on your floor again?
\ex \label{sprinklers:a:bar}  A2$^\prime$: ??OK!
\z 
\z
  
The second facet is the restricted alternatives of the bipolar question, and the restriction, specifically, to the two polar alternatives \{$\phi$, $\neg\phi$\}. This makes for a very poor choice of form for a try-out, especially in the presence of a monopolar form in the paradigm. One piece of evidence that polar alternatives are not evoked in Ev+ contexts is found in the absence of polar alternative questions in such contexts. \xref{raining} shows that polar alternative questions are infelicitous in Ev+ contexts.\footnote{A reviewer points out the felicity of \textit{It's not sunny?} in this context. This suggests that negative propositions can be part of the set of focus alternatives of a monopolar question without their logical complement (this time the positive \textit{It's sunny?}) being involved. Unfortunately space does not allow me to expand on this interesting lead.}

\ea  \textit{My friend enters my windowless office in a dripping wet coat.}\label{raining}
\sn \#Is it raining, or not? / \#Is it, or is it not, raining? / \# It's raining, or not?
\z
  
  If we consider from this point of view a typical use that rejects our monopolar forms but goes very well with our bipolar form, the restrictions start making sense also in that direction. Take an exam question or a polite request. If met with the answer \textit{no}, the exam is graded 0 and the request is denied. The questioner cannot attempt to request a response to the same question/request with a question from a different angle parallel to making a second guess after a failed one. The question is resolved. Whether \textit{yes} or \textit{no}, one answer excludes the other and becomes immediately conclusive because $\phi$ and $\neg\phi$ are the only alternatives on the table. A clear manifestation of the underlying bipolar meaning in contexts such as court questions and indirect questions is the common use of polar alternatives when necessary pragmatic conditions are met, such as in \xref{42}.  
  
\protectedex{
\ea\label{42}
\ea\textit{Court hearing}
\sn Did you or did you not tell him that if we were going to attack you'd let him know?\footnotemark
\ex\textit{Threat}
\sn Will you or will you not turn that damn radio off?
\z
\z
}
\footnotetext{\url{https://www.washingtonexaminer.com/policy/defense-national-security/milley-under-fire-i-would-never-tip-off-any-enemy} (Last accessed 12 June 2025.)}

What about the \textit{sufficient conditions} to create positive evidential bias? I suggest that contexts of positive evidential bias constitute a particular kind of trying out, namely one where no QUD has been established and hence the speakers must consult the physical environment to construct one. They must do so because expressions with open focus alternatives must be licensed in a QUD that limits the space of salient alternatives \citep{Roberts:1996, biezma-rawlins12}. In this line of thinking, the bias arises as a pragmatic inference based on the knowledge that both interlocutors must converge on a QUD.

Let me start with how linguistic cues delimit the alternatives in some English declarative questions and Turkish object attachment questions. The most straightforward case is an alternative question with distinct propositional alternatives as in \xref{propalt-oa}, repeated as \xref{43}. Even though each alternative is in the form of a monopolar question, the resulting alternative question does not lead to any inferences of bias, nor any evidence needs to be involved. This is because each of the exactly two relevant alternatives are overtly included in the form. The QUD is just that, or something like \textit{which of the following is the case}. 

\ea\label{43}
\gll Ali yemek \textbf{mi} yap-tı, yoksa Hasan pizza \textbf{mı} s\"oyle-yecek?\\
Ali dinner MI make-\textsc{past} or Hasan pizza MI order-\textsc{fut}\\
\glt `Ali made dinner or Hasan will order pizza? (Which one is it?)' 
\z

A guessing challenge is slightly more complicated, but alternatives may nevertheless be construed relatively easily. Consider the rising declarative/object attachment question in \xref{happytoday} in response to the guessing challenge \textit{Guess why I'm so happy today}. Here, the alternatives would be delimited by the knowledge of all possible events that might make the challenger happy, such as winning the lottery or passing an exam. There is no need for particular evidence towards a marriage proposal, because the challenge is simply to pick a likely state of affairs out of many potential ones, which is exactly what the monopolar form delivers. As in \xref{43}, there is no urge to look into the physical environment.
\protectedex{
\ea\label{happytoday} \textit{Your friend said ``Guess why I'm so happy today?"}. You respond.
\sn
\gll Sevgili-n evlenme \textbf{mi} {teklif et-ti}?\\
lover-\textsc{poss.2sg} marriage MI propose-\textsc{past}\\
\glt `Your girlfriend proposed?'
\z
}

\largerpage
Now consider the same question in a context of positive evidence noticed privately by A, as in \xref{engagement}. A wants to know whether B's girlfriend proposed, using the same monopolar form as in \xref{happytoday}, hence indicating that he is considering multiple states of affairs for the true answer. B may be looking at her computer all the while and there may have been no prior exchange about a marriage proposal. Not being aware of A's attention on the ring, the monopolar question is puzzling to B. A must have in mind a QUD with multiple alternatives, to which B has no immediate access. But, assuming A is being cooperative, B must be able to construct a congruent QUD that matches A's. B looks to the physical context for clues and realizes that her ring is a novelty that could lead to the hypothesis that B received a marriage proposal. B may then utter the maximally congruent \xref{engagementB1} or \xref{engagementB2}. Alternatively, B may not try to match A's QUD or fail to do so, in which case she could answer as in \xref{engagementB3}. 
\ea\label{engagement} \textit{A notices his friend B is wearing what looks like an engagement ring.}
\ea A: \gll Sevgili-n evlenme \textbf{mi} {teklif et-ti}?\\
lover-\textsc{2sg} marriage \textsc{mi} propose-\textsc{past}\\
\glt \phantom{A:} `Your girlfriend proposed?'
\ex \label{engagementB1} B: Yes! Cool ring, huh?
\ex \label{engagementB2} B$^{\prime}$: No. You mean my grandma's ring who just passed away? 
\ex \label{engagementB3} B$^{\prime\prime}$: Yes, who told you?/No, why?
\z
\z

Notice that effects of this QUD congruence emerge irrespective of whether the true answer to the question at hand is \textit{yes} as in \xref{engagementB1} or \textit{no} as in \xref{engagementB2}. Maximal question/answer congruence with the monopolar form is achieved only when the QUD is also addressed. Hence, \textit{yes} and \textit{no} alone are perceived to be odd unless the evidence in question is public. 

 QUD is a promising notion to connect the monopolar meaning to the biased inference based on evidence.\footnote{A reviewer suggests that a polar question with narrow focus such as \xref{ex:11:4b} or \xref{tournament} may be subsumed in the monopolar analysis following \citet{krifka14-f}. It becomes an interesting question, then, to what extent they are evidentially biased and whether this would fall out of the proposal given here. I leave this topic to future research, but note that empirically there may be an inference of evidential bias associated with \xref{tournament}.
 
 \ea\label{tournament} Hey, did [LEA]$_F$ win the tournament?
 \ea Felicitous in \textsc{[Ev. bias +]} \textit{I see Lea fans celebrating.}
 \ex ?? in \textsc{[Ev. bias 0]} \textit{Speakers are on the phone, no related utterance has been made.}
\ex Infelicitous in \textsc{[Ev. bias -]} \textit{I see Luke fans celebrating.}
\z
\z
 } Monopolar questions are unambiguously Ev+, that is, they evoke a need to refer to the physical environment to fully make sense, as a result of the pressure to construct a QUD based on the non-linguistic context in the absence of other clues revealing the nature of the relevant alternatives. This way, the requirement of public evidence which \citet{buring-gunlogson} and \citet{gunlogson} address at length would have a pragmatic source, namely based on a matching QUD construal by both parties. 

  
Putting together everything said in this section, I sketched an account that connects English declarative questions with Turkish object attachment interrogatives around the cluster of Ev+ features and across clause types, successfully excluding Turkish verb attachment interrogatives. Both English rising declaratives and Turkish object attachment questions are monopolar. They can support the generation of broad focus propositional alternatives while excluding the polar alternative. The monopolar meaning underlies the Ev+ cluster of effects, because the monopolar form allows the speaker to formulate a question where one of many possibly true states of affairs is picked. Inferences of positive evidential bias arise when interlocutors have to decide on a QUD that restricts that set of states of affairs. Turkish verb attachment questions are bipolar questions and denote the pair of polar alternatives only. Hence this form is employed in cases where only the content proposition and its negation represent potential states of affairs, presenting the complement set of effects to those of the Ev+ cluster. 

One case has been left out, which I will touch upon briefly before closing this section. Regarding negative polarity/concord licensing, our empirical excursion has suggested that it is a bipolar or non-monopolar feature. This could be because bipolar forms have a morphosyntactic representation of the negation in their meaning (related to the polar alternative). Our sketch analysis so far cannot say much more on this topic. 

\section{Residual clause type bias?}\label{sec:11:4}

In this paper, data demanded the generation of an account that can predict evidential bias effects along with the Ev+ cluster both in declarative and interrogative polar questions. Thus the proposed account does not rely on clause type and its consequences in deriving evidential bias. However, widely observed characteristics of English rising declaratives indicating a general bias concerning the addressee's epistemic state are not accounted for. In this account, such effects must come from different sources than the proposed monopolar meaning, so the possibility arises that rising declaratives are biased in multiple ways and the residual non-evidential inferences of bias are in fact due to the clause type of the rising declarative. In this section, I will describe the ways in which our two monopolar forms differ from one another and suggest that these cases may in fact provide the key examples to address this residual bias. Much of the rising declarative data discussed comes from \citet{gunlogson}.

We have seen in examples \xxref{ex:11:9}{ex:11:11} some of the ways in which the Turkish object attachment question and the English rising declarative are different. These are well-known differences between English polar interrogatives and rising declaratives, which we then used to argue that Turkish object attachment questions must be interrogatives. One case was the possibility of embedding under rogative predicates, which may be a direct morphosyntactic effect of the clause type, say, if the clause type is represented in the features of the C head, as commonly assumed. We have also seen that a discourse reference to an unresolved question could only be made with an interrogative. This requires a finer understanding, but one could still hypothesize that the formal qualities of the declarative do not fit the purpose. An intriguing possibility is that the intonation that marks the declarative as a question is only available when it is delivered within the question act. The last piece of data was the unavailability of adverbs like epistemic and evidential adverbs in polar interrogatives. I leave it open why this might be the case, but the interaction between clause type and such adverbials is empirically confirmed (the reader is referred to \citealt{krifka21} for a suggestion).

There are further, less structural, differences between the monopolar object attachment interrogatives and rising declaratives. Let us start with the case of speaker ignorance. We have seen that the English rising declarative cannot occur with an adverbial of complete ignorance like \textit{by chance} \citep{gunlogson} \xref{ex1}. This indicates the presence of a general prior disposition for a certain answer for rising declaratives. Object attachment questions (or verb attachment questions) are not biased in this way \xref{wondering}. 

\ea\label{wondering}
\gll Acaba Ali yemek \textbf{mi} yap-tı?\\
{wondering}	 Ali dinner MI make-\textsc{past}\\
\glt `[\#]Ali made dinner by chance?'
\z

Secondly, while object attachment questions are completely fine (even possibly required) in out-of-the-blue questions, rising declaratives are not. In \xref{cam}, we add some evidentiality to the context of this example to make it convincing, but make it so that the evidence is indirect and there is no conversational common ground. English-speaking informants are not as happy with rising declaratives in such a context  as Turkish-speaking informants are with object attachment questions \xref{cam} \citep[also see ][]{gunlogson}. 

\ea\textit{There was a crashing noise in the next office and I run to help. Behind closed doors, I ask ``Are you okay? \ldots''} \label{cam}
\sn
\gll Raflar \textbf{mı} devril-di?\\
shelves MI collapse-\textsc{past} \\
\glt `[??]The shelves collapsed?'
\z
 
 A similar illustration concerns alternative questions with propositional alternatives. We have seen \xref{accordingly} before without much discussion of the translations adopted. In fact, in line with the previous examples, the Turkish alternative question is just a run-of-the-mill alternative question, while the English declaratives in this alternative question form bring about an additional flavor. The speaker is not asking for information, but asking their interlocutor to finally commit to one of the statements. 

\ea\label{accordingly}
\ea
\gll Ali yemek \textbf{mi} yap-tı, yoksa Hasan pizza \textbf{mı} s\"oyle-yecek?\\
Ali dinner MI make-\textsc{ast} or Hasan pizza MI order-\textsc{fut}\\
\glt `Did Ali make dinner or will Hasan order pizza? (I'll buy the drinks accordingly.)'
\ex
Ali made dinner or Hasan will order pizza? \\
\ldots ??I'll buy the drinks accordingly. \\
\ldots Which one is it? Make up your mind.
\z
\z
 
Finally, object attachment questions make very good (and frequent) newspaper headlines, whereas a declarative question as a newspaper headline is very odd.

\protectedex{
\ea\label{headline} \textit{Newspaper headline}
\sn
\gll Yeni koronavirüs varyantları{} \textbf{mı} gel-iyor?\\
new coronavirus variants MI come-\textsc{pres}\\
\glt `[\#]New coronavirus variants are on the way?'
\z
}

In all these cases, we see that the object attachment question makes no meaning contribution beyond a plain monopolar interrogative while the rising declarative leads to extra inferences. If my account of the Ev+ is on the right track, these unshared effects of bias would be dissociated from the effects of monopolarity. Because the object attachment question in these cases act in line with the English polar interrogative, clause type emerges as a likely culprit for the rising declarative's divergent behavior. Hence, one could still argue for a clause-type-mismatch analysis \textit{\`a la} \citet{gunlogson} or \citet{farkas-roelofsen-division} for these effects. The novelty under such an account would be that English rising declaratives would be exhibiting two types of bias stemming from different sources. 



\section{A crosslinguistic excursus}\label{sec:11:5}

The proposed account attempts to connect a seemingly disparate set of features across clause types with some pre-compositional coherence, but the core of the approach is data-driven. After all, the departure from clause-type-based analyses of evidential bias was enforced by the Turkish polar interrogatives and the outlined account was built on the emerging Ev+ cluster. As the reader will surely have noted, the contrast between a declarative and an interrogative across unrelated languages is informative, but may be partly mired in uncontrolled language-specific factors. What I would like to do now is to evaluate the little available crosslinguistic data on Ev+ forms to confirm the Ev+ cluster and make an initial attempt to interpret crosslinguistic variation in polar question meanings.\footnote{In many languages including many Romance and Slavic languages, declarative questions are neutral. Consequently, they do not count as Ev+ forms.}
 
Two languages with unambiguous Ev+ forms analogous to English rising dec-laratives and Turkish object attachment questions I am aware of are Japanese and Hungarian. Notice that no two languages in the resulting set of languages to be compared are genetically related. The following discussion is based heavily on work on these languages respectively by \citet{sudo-bias} and \citet{gyuris-bias}. 

Japanese has at least three polar question forms, described in detail with respect to their bias profile by \citet{sudo-bias}. Among these, questions with the morpheme \textit{-no} carry positive evidential bias \xref{japaneserain}. The major competing form is morphosyntactically unmarked, which is noteworthy as it shows the absence of a crosslinguistic correspondence between markedness and Ev+ as one might suspect on the basis of English. This second form is also not neutral.   


\ea\label{japaneserain}
\ea \textsc{[Ev. bias +]} \textit{My friend enters my windowless office in a dripping wet coat.}
\sn 
\gll Ima ame futteru no?\\
now rain falls NO\\
\glt `It's raining?'
\ex \textsc{[Ev. bias 0]} \textit{We are looking for a left-handed person. I'm wondering about John, who is not around.}
\sn[\#]{
\gll John-wa hidarikiki-na no?\\
John-\textsc{top} lefty-\textsc{cop} NO\\
\glt `[\#]John is left-handed?'}
\ex \textsc{[Ev. bias -]}  \textit{[Same context as (a)]}
\sn[\#]{
\gll Ima hareteru no?\\
now sun.shines NO\\ \jambox*{\citep{sudo-bias}}
\glt `[\#]It's sunny?'
}
\z
\z

There is no published data on usage restrictions of Japanese \textit{-no} questions to my knowledge, but negative polarity licensing, which is also in the form of negative concord like in Turkish \citep{Watanabe:2004}, is banned as predicted \xref{japanese-nc}.\footnote{Differently from the negative concord asymmetry in the two Turkish polar interrogatives, absence of concord holds across polar question forms in Japanese (Yasutada Sudo, Kazuko Yatsushiro, p.c.). For this reason, it is not beyond doubt that the failure of concord licensing in \xref{japanese-nc} is a manifestation of the Ev+ cluster.  But this possibility is not to be rejected \textit{a priori}, either.} 

\ea[*]{
\gll Daremo kita no?\\
n-body came NO\\ \jambox*{\citep{masakuno}}
\glt (`Did anyone come?')
}\label{japanese-nc}
\z

Polar question forms in Hungarian are described in a good amount of detail by \citet{gyuris-bias}. The unambiguous Ev+ form, termed the  /\textbackslash-declarative (read: rise-fall declarative) following its intonation,  is a declarative question like in English.  In this language, both the Ev+ form and the primary competing form are morphosyntactically unmarked, but the latter has a different intonational phrasing and tests as an interrogative. 

The Hungarian /\textbackslash-declarative exhibits a remarkable subset of Ev+ cluster features such as the infelicity as exam questions and polite requests, and the unavailability of negative polarity licensing \xref{jamboxes} (data from \citealt{gyuris-bias}). Other associated contexts do not contradict any element of the cluster.\footnote{\citet{gyuris-bias} includes infelicity as conversation starters among these, but the category may be too loosely defined to be helpful.}   

\ea\label{jamboxes}
\ea[\#]{
\gll Magyarországnak/\textbackslash{} van/\textbackslash{} tengerpartja/\textbackslash? \\
Hungary-\textsc{dat} be.\textsc{3sg} seashore.its\\ \jambox*{Exam question}
\glt `[\#]Hungary has a seashore?'
} 
\ex[\#]{
\gll Kinyitod/\textbackslash{} az ajtót/\textbackslash? \\
\textsc{pfx}.open.\textsc{2sg} the door.\textsc{acc}\\ \jambox*{Polite request}
\glt `[\#]You are going to open the door?'
}
\ex[*]{
\gll Esik/\textbackslash{} valahol is/\textbackslash{} az es\H{o}/\textbackslash?\\
 falls somewhere too the rain\\ \jambox*{NPI}
\glt `[\#]It's raining anywhere?'
}
\z
\z

In the limited literature with sufficiently detailed descriptions of polar question paradigms, a few studies could not be included because they do not document an unambiguous Ev+ form. Mandarin has an array of polar question forms with associated nuances in bias, but none of the forms in question is strictly Ev+  \citep{ye21, yuan-hara-bias}. The description of \textit{kya:} questions in Hindi-Urdu by \citet{bhatt-dayal} does not include much pragmatic nuance, but it may be noted that the given semantic analysis is monopolar. 


The findings are summarized in \tabref{tab2}.\footnote{As this paper was prepared for publication, \citet{kn-talk} documented that Japanese \textit{-no} questions align with Turkish object attachment questions in terms of the listed Ev+ features.} In this small sample of unrelated languages covering declaratives, interrogatives, and different paradigms of division of labor across competing forms, we see that various Ev+ cluster features remain constant or at least undisturbed. I take this to be a confirmation for the typological reality of unambiguous monopolar question forms across languages and the Ev+ cluster as a good initial characterization of its manifestation. 

\begin{table}
\caption{Emerging Ev+ cluster based on the languages examined}
\label{tab2}
 \begin{tabularx}{\textwidth}{X rrrr}
  \lsptoprule
            & English & Turkish  & Hungarian & Japanese\\
  \midrule
Unambiguous Ev+ & yes & yes & yes & yes  \\ %\hline
As indirect question & no & no & no & ?  \\ %\hline
Exam/questionnaire/court & no & no & no & ?   \\ %\hline
With \textit{or not} & no & no & ? & ? \\ %\hline
NPI/NC & no & no & no & no  \\
As guess \& echoic & yes & yes & ? & ?  \\ %\hline 
  \lspbottomrule
 \end{tabularx}
\end{table}



If this initial review is on the right track, the monopolar meaning is represented and expressed across languages with dedicated forms showing (we expect most) features in the Ev+ cluster. We are not in a position to hypothesize the presence of forms corresponding to this meaning universally, nor the complete empirical profile if they are present. If the meaning is in the repertoire of a given language, its manifestation will depend on language-particular factors such as the morphosyntactic and intonational properties of the structure it is expressed with, or the pragmatics of the division of labor between the questioning forms in the paradigm. Mandarin \textit{ma} questions, which are Ev+/Ev0, are a good candidate to be evaluated from this perspective. At the same time, the convergent features can be used as a starting point to better understand the meaning of these forms as well as their form-meaning correspondence, which I briefly address in the next section.

In contrast to the promising picture on monopolar meaning manifestations, competing forms in this sample of languages fail to paint a coherent crosslinguistic picture corresponding to the proposed implementation of bipolar questions presented here. Why could this be? Foremost, the absence of universality predictions noted for the monopolar meaning applies here as well, so there is no \textit{a priori} reason to assume that unambiguous forms with the bipolar meaning will be observed across languages. It is however possible that some languages show a cluster of effects, yet to be discovered, converging on the bipolar meaning. At the moment, it remains unclear how much of the behavior of the Turkish verb attachment question, which I have argued to instantiate the bipolar meaning, is a pure manifestation of this meaning and how much of it is Turkish-specific. For this reason, it makes more sense to base further typological reasoning on the Ev+/monopolar qualities than apparent bipolar qualities.


\section{Clause type and polar question meaning revisited}\label{sec:11:6}

Now that we have argued against the role of clause type in the Ev+ cluster of effects, the question arises if its apparent correlation with the declarative clause type is entirely coincidental. I suggest the answer is no. The data implies that declarative questions may be crosslinguistically monopolar (cf. English and Hungarian forms). If confirmed, this connection between the declarative clause type and the monopolar meaning may indicate that the monopolar meaning is the default. 

This makes considerable conceptual sense. Of the two distinct meanings we have argued to be part of grammar, the monopolar meaning is the simpler one. It is a question form that just invites a commitment over the content proposition without generating its polar alternative. Hence, the monopolar question meaning in this sense of response solicitor may be the freely available with intonational marking alone, without requiring dedicated morphosyntactic structure.

A consequence of an approach where the monopolar meaning is the default is that it would be expected to be attested more broadly. This prediction is borne out in our small sample. Of the four forms we have analyzed in detail, the Turkish verb attachment interrogative is the only one that is excluded from the monopolar meaning. All others manifest it either unambiguously (the rising declarative and the object attachment question) or possibly ambiguously (the English polar interrogative). Recall, also, that we could find monopolar analogs but no straightforward bipolar analogs in Hungarian and Japanese. 

\ea 
\begin{forest}
[{English PQ forms} [{PI\\Monopolar\\Bipolar}] [{RD\\Monopolar}] ]
\end{forest}
\begin{forest}
[{Turkish PQ forms} [{{VA PI}\\Bipolar}] [{{OA PI}\\Monopolar}] ]
\end{forest}
\z

If the monomopolar meaning is present by default, where does the bipolar meaning come from? For the unambiguous bipolar question in Turkish, the answer cannot be interrogative syntax, as both the monopolar and the bipolar forms are interrogative. A better initial hypothesis would be that object attachment questions do not have a syntactic component driving the bipolar meaning, but verb attachment questions do. \citet{kamali-krifka} assume a Polarity head in the syntax of verb attachment questions in this spirit and \citet{bimono} defends this view based on an asymmetry in negative concord. But if it is the default as we have been entertaining, the monopolar meaning must be barred from arising in verb attachment questions. \citet{bimono} attributes this to a failure of focus projection \`a la \citet{selkirk95} because main prominence is not on its default position.  

The versatility of English polar interrogatives, noted in their evidential neutrality and the lack of distinct forms in polar and propositional alternative questions, implies that they command the union of effects, meaning both meanings are available at the same time. How exactly this ambiguity is encoded is an open question, but under the line of thought pursued here, it may be hypothesized that (part of) the English interrogative syntax drives the bipolar meaning whereas the monopolar meaning ensues by default. 

\section{Conclusion}\label{sec:11:7}
\largerpage
I have argued that not just evidential bias but rather the entire Ev+ cluster of features result from the monopolar question meaning and not from clause type. The sketched approach empirically solidifies the mono- versus bi-polar question dichotomy and their coexistence in grammar, and opens new avenues in approaching the problem of form-meaning correspondence in polar questions. One implication of the approach was the emergence of a residual bias inference that appears to indeed be clause-type-dependent, once evidential bias proper is disentangled from it. Another was the crosslinguistic commonness of an unambiguous monopolar question form sharing a number of the Ev+ cluster features. And finally, I have suggested that the monopolar meaning may correspond to declarative questions more commonly if it is the default. All in all, if correct, this approach points to a a more refined relationship between polar question form and meaning enriched by two universally available polar question meanings and regulated by diverse language-specific factors. 
  



%ıs{Cognition} %add "Cogntion" to subject index for this page



%ıl{Latin} %add "Latin" to language index for this page




\section*{Abbreviations}
\begin{tabularx}{.5\textwidth}{@{}lQ@{}}
\textsc{1/2/3pl} & 1/2/3 plural person \\
\textsc{1/2/3sg} & 1/2/3 singular person \\
\textsc{acc} & accusative \\
\textsc{aor} & aorist \\
\textsc{com} & commitative \\
\textsc{conv} & converbial \\
\textsc{cop} & copula \\
\textsc{dat} & dative \\
\textsc{fut} & future \\
\end{tabularx}%
\begin{tabularx}{.5\textwidth}{@{}lQ@{}}
\textsc{gen} & genitive \\
\textsc{loc} & locative \\
\textsc{neg} & negation \\
\textsc{nomin} & nominalizer \\
\textsc{past} & past \\
\textsc{pfx} & prefix \\
\textsc{poss} & possessive \\
\textsc{pres} & present \\
\textsc{top} & topic \\
\end{tabularx}

\section*{Acknowledgements}
This work is funded by the European Union Marie Sk\l odowska Curie Actions grant EPOQ-101067203 to Beste Kamali. I cordially thank Manfred Krifka for many sessions leading to the ideas in this paper. I also thank Maria Aloni, Beata Gyuris, Craige Roberts, Floris Roelofsen for helpful discussions, Deniz Rudin for comments on a pre-final draft, and two anonymous reviewers for their helpful feedback. 

\sloppy
\printbibliography[heading=subbibliography,notkeyword=this]
\end{document}
