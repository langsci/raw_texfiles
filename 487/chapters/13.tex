\documentclass[output=paper,colorlinks,citecolor=brown]{langscibook}
\ChapterDOI{10.5281/zenodo.17158198}
\author{Daniel Goodhue\orcid{0000-0003-0900-6727}\affiliation{Leibniz-Zentrum Allgemeine Sprachwissenschaft (ZAS)}}
\title[Everything that rises must converge]{Everything that rises must converge: Toward a unified account of inquisitive and assertive rising declaratives}
\abstract{In English, matrix declaratives with a final  rising intonation typical of polar questions are frequently used as a kind of biased question: they can only be used when there is contextual evidence in favor of the proposition denoted by the declarative. However, not all rising declaratives are used to pose a question about their content~-- some are used to assert the content of the declarative, while raising a second issue. In this paper, I offer a unified account of rising declaratives that seeks to explain both of these kinds of uses while positing unitary meanings for clause types and intonations. This goal cannot be achieved if we take the illocutionary force of an utterance to be completely determined by clause type and linguistic intonation, as many recent accounts have done. Instead, I propose that clause type and intonation merely constrain what a speaker could intend to do with them; pragmatic inference must play a key role in enabling an audience to uncover the speaker's illocutionary intention. In other words, there can be no hard and fast conventional discourse effects tied to particular clause type $+$ intonation pairings. I demonstrate that the proposed account enables a derivation of assertive force, and comparisons to other recent accounts are made.}

\IfFileExists{../localcommands.tex}{
   \addbibresource{../localbibliography.bib}
   % add all extra packages you need to load to this file

\usepackage{tabularx,multicol}
\usepackage{url}
\urlstyle{same}

\usepackage{listings}
\lstset{basicstyle=\ttfamily,tabsize=2,breaklines=true}

\usepackage{langsci-basic}
\usepackage{langsci-optional}
\usepackage{langsci-lgr}
\usepackage{langsci-osl}
% \usepackage{./langsci/styles/langsci-lgr}
% \usepackage{./langsci/styles/langsci-osl}
% \usepackage{langsci-gb4e}

\usepackage{tikz}
\usetikzlibrary{patterns,calc}
\pgfdeclarepatternformonly{south east lines}{\pgfqpoint{-0pt}{-0pt}}{\pgfqpoint{3pt}{3pt}}{\pgfqpoint{3pt}{3pt}}{
    \pgfsetlinewidth{0.6pt}
    \pgfpathmoveto{\pgfqpoint{0pt}{3pt}}
    \pgfpathlineto{\pgfqpoint{3pt}{0pt}}
    \pgfpathmoveto{\pgfqpoint{.2pt}{-.2pt}}
    \pgfpathlineto{\pgfqpoint{-.2pt}{.2pt}}
    \pgfpathmoveto{\pgfqpoint{3.2pt}{2.8pt}}
    \pgfpathlineto{\pgfqpoint{2.8pt}{3.2pt}}
    \pgfusepath{stroke}}
    
\usepackage{stmaryrd}
\usepackage{wasysym}
\usepackage{multirow}
\usepackage{caption}
\usepackage{subcaption}
\usepackage{mathrsfs}
\usepackage{qtree}

\usepackage{linguex}


   %pminos do not split footnotes
% \interfootnotelinepenalty=10000 %Footnote in Laporte chapters has to be split SN


%\DeclareIndexNameFormat{default}{%
%\nameparts{#1}%
%\usebibmacro{index:name}%
%{\index[names]}%
%{\namepartfamily}%
%{\namepartgiveni}%
% {}% L1
% {}% L2
%{\namepartprefix}% generates spurious space L3
%{\namepartsuffix}% generates spurious space L4
%}

%  {\DeclareIndexNameFormat{default}{%
%     \usebibmacro{index:name}{\index[names]}{#1}{#3}{#5}{#7}}}

%\DeclareIndexNameFormat{default}{%
%  \usebibmacro{index:name}{\sindex[nom]}{#1}{#3}{#5}{#7}}

%\DeclareIndexNameFormat{default}{%
%  \usebibmacro{index:name}{\sindex[person]}{#1}{#3}{#5}{#7}}
%\DeclareIndexNameFormat{default}{%
%\nameparts{#1} \usebibmacro{index:name}{\sindex[person]]}{\namepartfamily}{‌​\namepartgiven}{\nam‌​epartprefix}{\namepa‌​rtsuffix}}

%\newcommand{\smiley}{:)}

%\renewbibmacro*{index:name}[5]{%
%\usebibmacro{index:entry}{#1}%
%{\iffieldundef{usera}{}{\thefield{usera}\actualoperator}\mkbibindexname{#2}{#3}{#4}{#5}}}

% \newcommand{\noop}[1]{}

%remove for final
%\overfullrule=1mm

\newcommand{\tobi}[2]}}
\renewcommand{\S}[1]{\tobi{#1}{\textsc{*}}}

% this volume references
% puts: [this volume]
% already defined: \citetv
%\newcommand{\citepv}[1]{(\citeauthor{#1} \citeyear*{#1} [this volume])}
\newcommand{\citealtv}[1]{\citeauthor{#1} \citeyear*{#1} [this volume]}

%parentheses around example number
\newcommand{\pref}[1]{(\ref{#1})}

% in-text examples

\newcommand{\lnex}[1]{\textit{#1}} %target lang word
\newcommand{\lnlit}[1]{(lit.: `#1')} %literal reading
\newcommand{\lnlat}[1]{(#1)} % latinization
\newcommand{\lntrans}[1]{`#1'} %translation
\newcommand{\lnexl}[2]%
{\lnex{#1}{} \lnlat{#2}} % ex with latinization
\newcommand{\lnexlat}[3]{\lnex{#1}{} \lnlat{#2}{} \lntrans{#3}} % ex with latinization and tranl.

%ch01
\newcommand{\co}[1]{\mbox{\textbf{#1}}}

%ch09

\newcommand{\cyrbulg}[1]{\begin{otherlanguage*}{bulgarian}#1\end{otherlanguage*}}


%ch10
\newcommand{\nlp}{{\small NLP}}
\newcommand{\mwe}{{\small MWE}}
\newcommand{\rae}{{\small RAE}}
\newcommand{\lvc}{{\small LVC}}
\newcommand{\pos}{{\small P}o{\small S}}
%\newcommand{\todo}[1]{ \textcolor{red}{#1} }

%\renewcommand{\labelenumi}{\theenumi}
%\ainamefmt{{vv}{ll}{, ff}{, jj}} % fullname

\newcommand{\biberror}[1]{{\color{red}#1}}

\newcommand{\osenovaitem}{--~}
   %% hyphenation points for line breaks
%% Normally, automatic hyphenation in LaTeX is very good
%% If a word is mis-hyphenated, add it to this file
%%
%% add information to TeX file before \begin{document} with:
%% %% hyphenation points for line breaks
%% Normally, automatic hyphenation in LaTeX is very good
%% If a word is mis-hyphenated, add it to this file
%%
%% add information to TeX file before \begin{document} with:
%% %% hyphenation points for line breaks
%% Normally, automatic hyphenation in LaTeX is very good
%% If a word is mis-hyphenated, add it to this file
%%
%% add information to TeX file before \begin{document} with:
%% \include{localhyphenation}
\hyphenation{
    Beck-man
    Ngu-yen
    back-chan-nel
    back-chan-nels
    mo-not-o-nous
    ste-reo-typ-i-cal
}

\hyphenation{
    Beck-man
    Ngu-yen
    back-chan-nel
    back-chan-nels
    mo-not-o-nous
    ste-reo-typ-i-cal
}

\hyphenation{
    Beck-man
    Ngu-yen
    back-chan-nel
    back-chan-nels
    mo-not-o-nous
    ste-reo-typ-i-cal
}

   \boolfalse{bookcompile}
   \togglepaper[23]%%chapternumber
}{}

\begin{document}	
\maketitle
	
\section{Introduction}
The main claim of this paper is that one particular rising intonation, the \emph{polar question rise}, has only one specific meaning across its disparate uses, roughly, the speaker does not commit to a relevant proposition.  My account unifies inquisitive and assertive rising declaratives by deriving their distinct global interpretations from the same inputs~-- polar question rises and declarative clauses~-- used in different contexts.\footnote{So the title does not refer to \emph{every} rising intonation, just polar question rises. The point is that polar question rises converge on a single meaning, even though the global interpretation of utterances they appear in may vary.}
	
The following classic examples demonstrate that rising declaratives (RDs) can be used to ask biased questions. The questions are biased in the sense that they require contextual evidence in favor of the proposition denoted by the declarative. Throughout this paper, I use `\rise' (pronounced ``rise") to represent the utterance final rising intonation typical of polar questions in English. The relevant intonation will be discussed further in \sectref{noForm}. 
	
	\exa S is in her windowless office. A has just arrived holding a wet umbrella and raincoat. \label{rainEv}
	\ea S: Hey! It's raining$\nearrow$ \label{rdRainEv}
	\ex S: Hey! Is it raining$\nearrow$ \label{piRainEv} \hfill (based on \citealt[96]{gunlogson03})
	\z
	\z
	
	Intuitively, both the RD in \xref{rdRainEv} and the polar interrogative in \xref{piRainEv} are felicitous means of asking whether it is raining in the context of \xref{rainEv}. Contrast this with \xref{ex:13:rain}:
	
	\exa S is in her windowless office. A has just arrived, and exhibits no evidence whatsoever about the weather outside. \label{ex:13:rain}
	\ea S: \# Hey! It's raining$\nearrow$ \label{rdRain}
	\ex S: Hey! Is it raining$\nearrow$ \label{piRain} \hfill (based on \citealt[95]{gunlogson03})
	\z
	\z
	
	The context of \xref{ex:13:rain} lacks any evidence for rain. Intuitively, the RD in \xref{rdRain} is infelicitous, while the polar interrogative in \xref{piRain} is just fine. In prior work, examples like these establish \xref{bias} as a robust generalization about RDs \citep[see e.g.][]{beun00, gunlogson03, gunlogson08, truckenbrodt06, truckenbrodt09, truckenbrodt12, trinh11, malamud15, farkas17, krifka17, westera17, westera18, jeong18, rudin18, rudin22}.
	
	\exa (Inquisitive) Rising Declaratives are felicitous only if there is contextual evidence in favor of the content of the declarative clause. \label{bias}
	\z
	
	What we learn from such RDs is that declaratives are not reserved for assertions. With a particular rising intonation, declaratives can be used to ask a biased question. 
	
	However, biased questions are not the only use for rising declaratives. RDs can also be used to assert. Consider the RDs in \xref{ladino}, \xref{persimmon}, and \xref{mark}, which are used by S to assert their propositional content.\footnote{\citet{ward85} introduced \xref{ladino} as an example of the rise-fall-rise contour (L*+H L-H\%), which is distinct from the \rise contour used in RDs. Nevertheless, $\nearrow$ is also felicitous in \xref{ladino}, though probably not preferred. On the other hand, \xref{persimmon}, \xref{mark}, and other examples of assertive RDs below are not felicitous with rise-fall-rise.}
	
	\exa A: Do you speak Spanish?\\
	S: I speak Ladino$\nearrow$ \label{ladino} \\
	\phantom{f}\hfill (\citealt[]{jeong18}, \citealt[]{farkas17}, based on \citealt[]{ward85})
	\z
	
	\exa A: What are you eating?\\
	S: This is a persimmon$\nearrow$ \label{persimmon} 
	\z
	
	\exa S isn't sure that he is in the right doctor's office. He says to the receptionist:\\
	S: My name is Mark Liberman\rise \label{mark}\hfill (\citealt[62]{pierrehumbert80}, from Mark Liberman p.c.)
	\z
	
	Besides being used to assert their content, the RDs in \xref{ladino}, \xref{persimmon}, and \xref{mark} also seem to raise a second issue. In \xref{ladino} this is something like \textit{Is Ladino close enough to Spanish for your purposes?}, while in \xref{persimmon} and \xref{mark} this is something like \textit{Is that enough information?} or \textit{Have you heard of persimmons/me before?}. Most of the literature on RDs cited above has either ignored assertive RDs, set them aside, or tried to account for them separately from inquisitive RDs.\footnote{I believe that assertive RDs are likely related to `uptalk' and `high rising terminals'. If so, then they have also been discussed in the sociolinguistics literature \citep[e.g.][a.o.]{mclemore91, fletcher05, ladd08, shokeir08}. I leave a full exploration of this connection to future work.} This  has usually been justified by the claim that there  are two distinct rising contours, one used in inquisitive RDs, the other in assertive RDs. The idea is that each contour makes a distinct meaning contribution resulting in the distinct inquisitive and assertive illocutionary forces observed. In \sectref{noForm}, I will argue that the evidence for this view is weak, and provide further evidence that speaks against it. 
	
	As a result of this empirical evidence, as well as for reasons of theoretical parsimony, I will argue that there is only one relevant rising intonation, \rise, with a single meaning attached to it that can explain its use in both inquisitive and assertive RDs, as well as the fact that it is used in most matrix polar interrogatives (see \citealt{hedberg17} for corpus evidence that over 90\% of American English polar interrogatives rise utterance finally, \emph{pace} \citealt{geluykens88}). Furthermore, I will adopt a semantics for clause types in which declaratives denote propositions while interrogatives denote sets of propositions (answer sets). But if there is a single meaning for \rise, and a single denotation for declaratives, then clearly the combination of these two components alone cannot completely explain the variation in speech act interpretation we see across inquisitive and assertive rising declaratives. In other words, intonation plus clause type does not always determine illocutionary force. This is contrary to the predictions of prior work such as \citet{farkas17}, \citet{jeong18}, and \citet{rudin18}, which claim that an utterance of a specific clause type with a specific intonation results in exactly one discourse update effect (though the details differ substantially across these accounts).
	
	The solution I will propose is to abandon the view that clause type-intonation pairings are specified with conventional discourse effects, and instead allow pragmatic inference to play a greater role in enabling the audience to uncover the illocutionary force intended by the speaker of an utterance. In other words, in order to simultaneously account for inquisitive and assertive rising declaratives, the contributions to illocutionary force made by clause type and intonation need to be weaker than hypothesized in the recent literature on rising declaratives. 
	
	
	
	In \sectref{noForm}, I discuss the intonational facts to motivate a unified account of inquisitive and assertive RDs. Then I lay out the ingredients of the unified account in \sectref{account}. In \sectref{data1}, I apply the account to the basic data, as well as to  incredulous uses of RDs, and I briefly discuss its potential application to rising imperatives. In \sectref{derive}, I show how the account enables a derivation of assertive force. Finally, I briefly compare the account to prior work in \sectref{priorAcc}, and raise issues for future work  in \sectref{openIss}.
	
	
	
	
	
\section{Perspectives on the intonation of RDs} \label{noForm}
	
	
	
	Much work on rising declaratives has claimed that they come with two distinct intonational (phonological) contours, one for inquisitive RDs and another for assertive RDs, with each contour playing a crucial role in determining the illocutionary force of the RD. If this is correct, it means that the two kinds of RDs are orthogonal and can be given independent analyses. In this section, I will cast doubt on this view, and show that the empirical facts are consistent with at least three other views in which intonation does not neatly distinguish the illocutionary force of RDs.  If one of these latter views is correct, it means that the two kinds of RDs are not orthogonal and must be given a unified analysis. I will then develop such an analysis in the remainder of the paper.
	
	
	The earliest empirical evidence motivating the hypothesized intonational distinction between assertive and inquisitive rising declaratives was based on researcher judgments \citep{pierrehumbert80, pierrehumbert90, hirschberg95}. The hypothesis is that assertive RDs have a high rising contour, represented by H* H-H\%, while the standard polar question contour in polar questions and inquisitive RDs is low rising, L* H-H\%.\footnote{A brief primer on the Tones and Break Indices system (ToBI; \citealt{veilleux06}): There are high (H) and low (L) tones. `T*' indicates a pitch accent (a tone on a syllable that is more perceptually salient/stressed), `T\%' an intonational boundary (the end of an intonational unit, usually  the end of a sentence), and `T-' a phrase tone leading up to the boundary tone. The final pitch accent in a sentence is called the \emph{nuclear pitch accent}, and the intonation from that point on is the \emph{nuclear contour}. The nuclear pitch accent is the most salient intonational stress, even though it usually isn't the greatest pitch maximum/minimum acoustically-speaking.} It's clear from the discussions in these references that the relevant perceptual distinction is meant to be in the height of the nuclear pitch accent, with  L* at the bottom of the speaker's range, while H* is in the middle. \citet{truckenbrodt12} builds on ideas in \citet{pierrehumbert90}, \citet{hirschberg95}, and \citet{bartels99}. He analyzes H* as signaling addition of a salient proposition to the common ground, while H- signals the questioning of a salient proposition (L* and L- are treated as meaningless defaults). Thus L* H-H\% questions the proposition uttered, while H* H-H\% asserts the proposition uttered while questioning a related salient proposition.\footnote{One mystery for uses of H* H-H\% in \citeauthor{truckenbrodt12}'s analysis is why the H* always targets the content of the utterance while H- targets some other salient proposition. The account predicts these roles to be reversible, contrary to fact. My analysis in \sectref{account} pursues the idea that \rise can target salient propositions, but derives which propositions are targeted from pragmatics.}\textsuperscript{,}\footnote{Interestingly, \citet[62--63]{pierrehumbert80} does \emph{not} claim that assertive RDs show that an intonational distinction has a strong grammatical link to speech act meaning. On the contrary, despite distinguishing H* H-H\% from L* H-H\%, she takes them to be family members of a \emph{single} yes/no question intonation, and her point is that assertive RDs like \xref{mark} show that this yes/no question intonation does not force the utterance to be interpreted as a yes/no question about the truth of the proposition. She then briefly sketches  an analysis of intonational meaning that is actually in the same spirit as the one I pursue in \sectref{account} through \sectref{derive}.}
	
	\largerpage
	There is some reason to doubt this empirical picture in which L* H-H\% correlates with inquisitive RDs while H* H-H\% correlates with assertive RDs. First, \citet{hedberg17} claim that their corpus data shows that both contours are used in matrix polar interrogatives. Since matrix polar interrogatives only have an inquisitive interpretation, the meaning contribution of H* H-H\% would need to be canceled somehow. Second, \citet{nilsenova06} uses corpus data for stimuli in a comprehension experiment that shows that L* H-H\% and H* H-H\%, as well as L* L-H\%, all significantly increase inquisitive interpretations of declarative clauses. This shouldn't happen if H* H-H\% is the assertive contour for RDs. These results call into question the view that the inquisitive/assertive split in RDs correlates with L* H-H\% and H* H-H\% respectively. 
	
	More recently, a view has emerged in which the inquisitive/assertive split is claimed to correlate with the height of the final H\% boundary tone: Inquisitive RDs are claimed to rise more steeply to a higher final boundary tone than the contour associated with assertive RDs \citep{jeong18, rudin18, westera18}.\footnote{\citet{jeong18} and \citet{rudin18} continue to use the L* H-H\% and H* H-H\% ToBI transcriptions to distinguish these two contours, despite that the key phonetic distinction is in the height of final boundary tones rather than nuclear pitch accents. That's because the phonological distinction between L* H-H\% and H* H-H\%  could in principle result in an observed phonetic distinction in the height of the final H\% boundary tone. Thus the phonological and interpretational claims of \citet{pierrehumbert90}, \citet{hirschberg95} and \citet{truckenbrodt12} could still be viable, even if the claim that the perceptual distinction is in the height of the pitch accent may not be. Thanks to Sunwoo Jeong, Jeffrey Lidz, and Michael Wagner for discussion on this point.} \citet[320ff.]{jeong18} reports on a series of multi-participant comprehension experiments in which the intonation of RDs is manipulated via prosodic resynthesis so that the height of the pitch accent remains constant (neither high nor low, but in the middle of the speaker's range), while the boundary tone is higher in some conditions than others. The results show that the steeper the rise is, the more likely participants are to arrive at an inquisitive interpretation (absent other contextual factors that make inquisitive or assertive interpretations more likely); for the shallowest rise, participants choose between inquisitive and assertive interpretations at around chance levels.\footnote{\citet{jeong18} also shows that relative speaker/addressee knowledgeability strongly impacts interpretation, largely taking precedence over intonation. This factor will play an important role in my analysis as well.} Jeong argues that these results can be explained if there are two intonational contours with meanings that determine the illocutionary force of RDs: a steeper one that leads to inquisitive force, and a shallower one that leads to assertive force. 
	
	\largerpage
	I turn now to showing that there is a competing analysis available (that comes in a few sub-flavors) in which intonation correlates, not with the split between inquisitive and assertive RDs, but instead with a split between incredulous and non-incredulous RDs. To appreciate this, consider the following taxonomy of RDs (cf. the similar taxonomy in \citealt{jeong18}).
	
	\exa
	\begin{forest}
	[{Taxonomy of Rising Declaratives}
	[{Inquisitive RDs}
	[{Incredulous RDs}]
	[{Confirmative RDs}]
	]	
	[{Assertive RDs}]
	] 
    \end{forest}
\label{tax}
	\z
	
	According to \xref{tax}, uses of inquisitive RDs can be sorted into two kinds: incredulous uses and confirmative uses. \xxref{ird}{urd} provide a minimal triple demonstrating the taxonomy in \xref{tax}. In \xref{ird}, S utters the RD incredulously because they are shocked by A's claim that the girl is only nine. 
	
	\exa Incredulous \\ 
	S and A are watching a girl give a very professional performance in a school debate. S thinks that she must be at least 13 years old.\\
	A: I can't believe she's only 9.\\
	S: She's nine$\nearrow$ \label{ird}
	\z
	
	In \xref{crd}, S utters the RD confirmatively; they are not shocked that the girl is nine, they just want to double check that fact.
	
	\exa Confirmative \\ 
	S and A are buying a birthday card for the daughter of A's friend.  While searching for a card for the correct age, S thinks A told him previously that the girl has just turned nine, but he wants to confirm it.\\
	S: She's nine$\nearrow$ \label{crd}
	\z
	
	In \xref{urd}, S asserts that their daughter is nine, but doesn't know whether there is still room for kids in her age group. 
	
	\exa Assertive \\  
	S wants to enroll his daughter in music lessons with A. \\ 
	S: My daughter wants to study tuba. \\
	A: Okay, but there are limited places for each age group, and some age groups have already filled up. How old is she?\\
	S: She's nine$\nearrow$\label{urd}
	\z
	
	I will now lay out three possible analyses of the intonational distinctions to be found among the kinds of RDs in \xref{tax}, and then show how each analysis can explain the experimental results in \citet{jeong18} without positing two phonological intonations that correlate with force in the inquisitive/assertive split.
	
	Starting with the distinction in the height of the final boundary tone postulated by \citet{jeong18}, \citet{rudin18}, and \citet{westera18}, one possible analysis is that it's a paralinguistic distinction, such that all three sub-kinds of RDs in \xref{tax} have the same phonological intonation, best transcribed in ToBI as L* H-H\%, and incredulous RDs have a higher boundary tone due to increased emotional activation, as has been discussed by \citet{gussenhoven04}, \citet{banziger05}, \citet{cresposendra13}, \citet{westera17}, and \citet{goodhue21:lsa}. 
	
	Another possible analysis is that there is a phonological distinction between incredulous and non-incredulous RDs. On this view, incredulous RDs would be phonologically specified for a very high H\% boundary tone, and would be produced with at least two rises, one earlier in the sentence, and another for the nuclear contour (transcribed in ToBI as L* H- L* H-H\%). In \xref{ird}, this would mean a low pitch accent followed by a rise on \emph{she's}, and then a fall back to a low nuclear pitch accent on \emph{nine} that rises to a very high boundary tone. Intuitively, such a double rise is felicitous in \xref{ird}, but infelicitous in both \xref{crd} and \xref{urd}. In the production study of \citet{goodhue16:best}, incredulity contexts elicited double rises to very high boundary tones in the majority of trials.\footnote{The height of the final boundary tone is in principle separable from the phonological contour. So a third analytic option would be that there are two linguistic contours~-- the normal polar question rise and an incredulous double rise~-- and the incredulous double rise generally has higher final H\% boundary tones for paralinguistic reasons, namely because speakers using them are emotionally activated.}\textsuperscript{,}\footnote{The incredulous double rise discussed here is \emph{not} the incredulity contour of \citet{hirschberg92}, which they describe as L*+H L-H\%, identical to the rise-fall-rise contour but with a larger pitch excursion (\citealt{barnes12} argue it is better transcribed as L+H* L-H\%). Crucially, the incredulous double rise and the incredulity contour are perceptually distinct, and likely also have subtly different felicity conditions, which once understood should explain why the incredulity condition of \citet{goodhue16:best} elicited incredulous double rises in over 95\% of trials, but only a single incredulity contour.}
	
	Finally, the height of the final boundary tone may correlate inversely with the speaker's certainty level about the proposition $p$ expressed: the higher the tone, the less certain. The speaker of a confirmative RD tends to think $p$ is likely (the basis for the choice to confirm $p$ as opposed to some other proposition), which results in a lower H\% boundary  tone. The speaker of an incredulous RD takes $p$ to be unlikely (hence the incredulity), resulting in a higher H\% tone (\sectref{incredRDs} will nuance this view of the speaker's possible stances toward $p$ in incredulous RDs). 
	
	Each of these views of possible intonational distinctions among RDs can explain \citeauthor{jeong18}'s (\citeyear[320ff.]{jeong18}) experimental results, which revealed that the steeper the rise is, the more likely participants are to arrive at an inquisitive interpretation, and in the condition with the shallowest rise, participants chose between inquisitive and assertive force at chance levels. Assume the paralinguistic view: Steeper rises imply emotional activation, and emotional activation correlates with incredulity. Furthermore, incredulous RDs are a sub-kind of inquisitive RDs in the taxonomy in \xref{tax}. Thus, when participants hear a steeper rise, they are more likely to infer an inquisitive interpretation. Shallower rises on the other hand are less emotionally activated, and thus non-incredulous, which is consistent with either an inquisitive (confirmative) interpretation or an assertive interpretation. 
	
	Now suppose the linguistic incredulous double rise view is correct: The steeper rises were interpreted as corresponding (albeit imperfectly) to a linguistic contour specified with an incredulity meaning. Since incredulous RDs are a sub-kind of inquisitive RDs in \xref{tax}, this led participants to an inquisitive interpretation. Meanwhile shallower rises were interpreted as corresponding to a normal polar question rise, which is consistent with both confirmative and assertive RD interpretations. 
	
	Finally, on the view that steepness has an inverse correlation with certainty, it is clear why steeper rises would lead to more inquisitive interpretations. What remains to be explained is why the shallowest rises were interpreted as either inquisitive or assertive at chance levels. The answer is that a speaker can be relatively certain, even completely certain, about a proposition $p$ and still ask a question about it, especially if the addressee is known to have more authority over whether $p$ is true or not in the context. The following example from a squib by \citet{sider22} demonstrates this well:
	
	\exa While S is feeding his daughters breakfast, he hears someone stirring upstairs. S's daughters go up, and S hears talking. On a typical day in S's house, this is when S's wife usually rises. S's daughters come back downstairs. \\
	S: Mom's up$\nearrow$ \label{mom}\hfill \citep{sider22}
	\z
	
	\xref{mom} is a confirmative RD. As Sider says, S is relatively certain that the mother is awake. And yet the RD S utters is clearly not an assertion. On my view, it is a question, as evidenced by the fact that the children could provide a \emph{yes} or \emph{no} answer. S is able to ask this question because, even though S has very good evidence that $p$, the children are in a better position to know $p$: they were upstairs with the mother. Thus we understand why, on this third view in which shallower rises are consistent with high certainty levels, participants interpreted the shallowest rise as either inquisitive or assertive at chance levels in Jeong's experiment.
	
	Here is the upshot of this discussion: The experimental results in \citet{jeong18} are compatible with, but do not decide in favor of, the view proposed there in which there are two phonological intonations with meanings that play a direct role in producing inquisitive and assertive illocutionary forces (the illocutionary view). The results are also compatible with three other views in which intonation (whether phonological or paralinguistic) plays an indirect role: two in which intonation distinguishes incredulous RDs from non-incredulous kinds, and a third in which intonational steepness signals uncertainty about the propositional content $p$, with certainty about $p$ being consistent with both inquisitive and assertive force. What these latter three views share in common is that there is an intonation (again, whether phonological or paralinguistic) that is sufficient but not necessary for producing inquisitive force, and another intonation that is consistent with both inquisitive and assertive force. 
	
	Given the complexity of the analytic options, it would be fruitful to investigate the intonations of RDs further via a set of multi-participant production studies in future work. The studies should manipulate the various variables discussed above including contexts to produce the three kinds of RDs in the taxonomy in \xref{tax}, as well as speaker certainty levels. One challenge will be to find a way to ensure that na\"ive participants produce the relevant rising intonation in assertive contexts like \xref{urd}, since a falling declarative would be quite natural there. This could be done by providing the participant with detailed stage directions about their character's thoughts and concerns (e.g. for \xref{urd}, that S is feeling uncertain, not about their daughter's age, but about whether there will be room in her age group's class). 
	
	In the meantime, it is clear that analyses in which a single rising phonological intonation can appear in both inquisitive and assertive RDs are at least as viable as those in which linguistic intonations rigidly correspond to illocutionary force. Moreover, a unified account, besides being more theoretically parsimonious, may also be favored by native speaker judgments. For example, consider again the confirmative inquisitive RD in \xref{crd}. My judgment, and those of others I have consulted, is that this RD is very natural with a shallow rise, no steeper than the rise in the assertive \xref{urd} (as discussed in \S3 of \citealt{goodhue21:lsa}). Likewise for \xref{mom}, \citet{sider22} argues, and I agree, that a natural rise would be shallow (Sider even labels such examples ``slightly rising declaratives").\footnote{Sider goes so far as to argue that slightly rising intonation on a declarative is reserved for a special kind of epistemic tightening speech act that is neither assertion nor question. The challenge for this view is that the same rise is perfectly felicitous in assertive RDs like \xref{urd}, hence my pursuit of a unified account for a more general contour \rise with a more general meaning.} Since these RDs are inquisitive, the illocutionary view incorrectly predicts their rises to be steep. 
	
	\citet[313, 327]{jeong18} already makes an observation that could serve as a rebuttal to this critique of the illocutionary view, writing ``it is likely that the boundary between the two types of intonational configurations that signal assertive vs. inquisitive rising declarative is malleable and heavily dependent on the speaker." The idea is that this variation may swamp the ability to perceive the hypothesized phonological distinction between steep and shallow rises. I agree that such variation is bound to occur, but if Jeong's intonational analysis is correct, it cannot always swamp the distinction, and in fact it must \emph{mostly not} do so. According to Jeong's view, speakers must produce a phonetic distinction between inquisitive and assertive RDs on average in well distinguished contexts~-- if they didn't, then children would never be able to acquire the distinction between the two purported rises. But contrary to the illocutionary view, it seems intuitively clear that well-distinguished contexts that control for factors that might cause paralinguistic phonetic variation (such as emotional activation) make confirmative inquisitive RDs like \xref{crd} and \xref{mom} \emph{less} likely to be phonetically distinguishable from assertive RDs like \xref{urd}, not more.\footnote{Two reviewers expressed some concern about the legitimacy of a researcher using their own production of the relevant examples as evidence, and therefore about the researcher consulting their own judgments (\S3 of \citealt{goodhue21:lsa} is a rough approximation of the argument in the first round manuscript that spurred these reviewer comments). Their comments helped me drastically reshape the argument in this section, and my call for a multi-participant production study above reflects my partial agreement with them. Generally speaking, intonational research will often need to be conducted via multi-participant experimentation because fewer researchers have confident judgments about their tacit intonational knowledge (compared to other linguistic knowledge), and because it can be challenging to tease apart gradient paralinguistic prosody and linguistic intonation. That said, like in other areas of linguistics, researcher judgments (which are really just single-participant auto-experiments) have always played a crucial role in intonational research and we need to preserve that role, in part because trained researcher judgments will often be more reliable than na\"ive speaker judgments, and in part because insisting on multi-participant experiments for all empirical data when many facts can be easily and reliably established via researcher judgments is not an efficient use of resources (see the more detailed metascientific discussions in \citealt{phillips09} and \citealt{jacobson18}). While I think the full intonational facts for RDs should be explored in a multi-participant production study, the fact that some confirmative RDs like \xref{crd} and \xref{mom} are natural with shallow rises of the sort that are also found in assertive RDs like \xref{urd} strikes me as uncontroversial based on researcher judgments.} 
	
	On any of the alternative views I have sketched, such shallow rises on confirmative inquisitive RDs are expected. In the remainder of this paper, I will pursue a unified account of a single linguistic contour \rise with one conventionally associated meaning, and explain why it can appear in both inquisitive and assertive RDs.
	
	
	
	
	
	
\section{The account} \label{account}
	
	I assume that declarative clauses denote propositions, functions from worlds to truth values of type \tp{s,t} as in \xref{dec}. Following \citet{hamblin73}, I treat polar interrogative clauses as denoting sets of their two possible answers as in \xref{pol}, which are characterized by functions from propositions to truth values of type \tp{\tp{s,t},t}. 
	
	\exa
	\ea \label{dec}Declarative: \den{\phi}{} = $p$
	\ex \label{pol}Polar interrogative: \den{?\phi}{} = \{$p, \neg p$\} 
	\z
	\z
	
	I assume a model of context \`a la \citet{farkas10}, which incorporates notions from \citet{hamblin71}, \citet{stalnaker78}, and \citet{roberts12}, and therefore is akin to other approaches to rising declaratives working in this framework \citep[e.g.][]{gunlogson03, gunlogson08, malamud15, farkas17, jeong18, rudin18, rudin22}. 
	
	\exa \label{con}A context $c$ is a tuple $\langle$DC, CG, T, QUD$\rangle$ 
	\ea \emph{DC} is a set of sets of discourse commitments \emph{DC}$_{a}$ for each interlocutor $a$
	\ex \emph{CG} is $\bigcap$\emph{DC}, the common ground, a set of propositions interlocutors are mutually committed to 
	\ex \emph{T}, the table, is a push-down stack of issues (where issues are sets of propositions) 
	\ex \emph{QUD} is a  salient question in \emph{T} \label{qud}
	\z
	\z
	
	Since the questions uttered go onto the table, we might wonder what the role of a separate \emph{QUD} is. Its role is seen most clearly in \xref{urd} (which will be reviewed below), in which S's RD answers a local question at the top of \emph{T}, but in which the relevance of the proposition $q$ targeted by \rise is determined based on its role in a strategy to resolve some larger question. The larger question in \xref{urd} is a prior question, deeper in the push-down stack of \emph{T}. Thus the only requirement on \emph{QUD} is relatively weak, that it be an issue in \emph{T}. 
	
	Since rising declaratives can either be questions or assertions, their conventional or mechanistic effect on the context $c$ needs to be relatively weak. I propose the following, minimal dynamic pragmatics for utterances:
	
	\exa \emph{Utterance function}\label{dgUtt}\\
	\textsc{utterance}$(\psi, c_n) \rightarrow c_{n+1}$ such that
	\ea $T_{n+1} = T_n$ + \den{\psi}{c_n}, \hspace{1cm} if \den{\psi}{c_n} $\in D_{\tp{\tp{s,t},t}}$ \label{uttI}\hfill (\emph{for interrogatives})
	\ex $T_{n+1} = T_n$ + \{\den{\psi}{c_n}\}, \hspace{.58cm} if \den{\psi}{c_n} $\in D_{\tp{s,t}}$  \label{uttD}\hfill (\emph{for declaratives})
	\z
	\z
	
	\xref{dgUtt} has two slightly different effects depending on whether the utterance $\psi$ denotes a proposition (is a declarative) or denotes a set of propositions (is an interrogative). \xref{uttI} says that if $\psi$ is an interrogative, its content is added directly to the table. \xref{uttD} says that if $\psi$ is a declarative, a singleton set of its content is added to the table. These are subcases of a single utterance function, rather than two distinct utterance functions depending on clause type, since they have  the same basic effect of adding utterance content to the table and nothing more (cf. \citealt{farkas10}, \citealt{jeong18}, \citealt{rudin18}, in which different sentence types/intonations are subject to utterance functions differing in whether content is added to the speaker's discourse commitments). The distinction between \xref{uttI} and \xref{uttD} is merely for technical reasons, to make sure that everything added to the table is of the same type, a set of propositions, and therefore an issue.\footnote{An alternative would be to adopt a semantics that gives interrogatives and declaratives the same type \citep[cf.][]{farkas17, rudin18, rudin22}. However I prefer to impose this minor complexity on the utterance function so as to maintain uniformity with the view that the intensions of declarative clauses are functions of type \tp{s,t}, rather than to complicate our semantics of declarative clauses in order to smooth the interface with pragmatics.} While the utterance function in \xref{dgUtt} is inspired by prior work in this domain \citep{farkas10, farkas17, jeong18, rudin18, rudin22}, I'll suggest in \sectref{derive} that my conception of it is somewhat different, closer to a locutionary act than an illocutionary one. 
	
	With the above in place, we are ready to introduce the semantics for \rise. Informally, the idea is that $\nearrow$ conveys that the speaker does not publicly commit to a proposition $q$ that would help to settle the QUD. As a default, $q$ gets its propositional content from the declarative or the prejacent of the polar interrogative uttered because that is the easiest relevant proposition to identify. But $q$ doesn't have to be identified with this overt content $p$. I follow \citet{bartels99} and \citet{truckenbrodt12} in pursuing the idea that intonational meaning can operate on a contextually salient proposition distinct from the propositional content  of the utterance. However, unlike \citepos{truckenbrodt12} account discussed in \sectref{noForm}, my approach does not depend on meanings attached to individual tones in an H* H-H\% contour. It is my view that some contexts make the content $p$ of the clause uttered an unlikely target for the speaker to convey lack of commitment about. In such cases, \rise targets some other proposition, but not just any proposition~-- the one that is most relevant to the QUD, given $p$. Formally, these requirements will be stated as felicity conditions on the use of \rise in \xref{riseden}.\footnote{\rise is defined to compose with propositions, which is necessary in order to have a unified semantics for \rise that can coherently state the lack of commitment conveyed in both RDs and polar questions. As a result, \rise must compose below the $Q$ morpheme in polar interrogatives.}
	
	Before the formal semantics can be introduced, a brief discussion of commitments over time is needed.\footnote{Thanks to an anonymous \emph{Sinn und Bedeutung} reviewer for comments that spurred my thinking on this point.} We expect interlocutors to stand by their commitments from the moment they are made onward, at least until they explicitly change them. This goes for an expression of lack of commitment too~-- we expect the lack of commitment to $q$ that S conveys in using \rise to persist in future developments of the conversation, until something changes it. To make this explicit in the formal semantics, we need to refer to future developments of the context $c$ in a nuanced way. As implied in \xref{dgUtt}, $c_n$ is the context of utterance, and $c_{n+1}$ is the context just after the utterance. Focusing just on the lack of commitment to $q$ conveyed by \rise, suppose $n' > n+1$ and $c_{n'+1}$ is a hypothetical context in which some new evidence has caused S to reconsider their lack of commitment to $q$. Then we can define \rise to require S to lack commitment to $q$ in the context $c_{n+1}$ that immediately follows the utterance context $c_n$, up through the context $c_{n'}$ that is just prior to the one in which new evidence causes S to change their commitments relative to $q$, $c_{n'+1}$.
	
	\exa \label{riseden}
	\den{\nearrow}{c_{n}}$(p)$ is felicitous only if $\exists q \in D_{\tp{s,t}}$  $\exists \Gamma \in D_{\tp{\tp{s,t},t}}$  such that 
	\ea $q \notin  \text{\em DC}_{S_{c_{n+1}}},\dots,\text{\em DC}_{S_{c_{n'}}} ~~\&$ \hfill \label{riseCom}(\emph{Lack of commitment})
	\ex $q \in \Gamma ~~\&~~ p \in \Gamma ~~\&~~ \bigcap \Gamma \in \text{\em QUD}_{c_{n+1}} ~~\&~~ \bigcap(\Gamma-\{q\}) \notin \text{\em QUD}_{c_{n+1}}$ \hfill \label{riseRel}(\emph{Relevance})
	\z
	If felicitous, \den{\nearrow}{c_{n}}$(p)$ = $p$
	\z
	
	\xref{riseCom} says that the speaker $S$ is not committed to some proposition $q$. \xref{riseRel} ensures that $q$ is relevant by requiring $q$ to be part of a strategy $\Gamma$ to address the \emph{QUD}. Without \xref{riseRel}, \rise would be predicted to be felicitous on any assertion, since presumably there is always some non-relevant proposition $q$ that S lacks commitment to.\footnote{Ultimately, this relevance component may be handled via a more general pragmatic condition on relevance (\citealt{grice89:lc}, \citealt{roberts12}, among others), however I have chosen to spell it out here to make its role explicit.} Finally, note that nothing in \xref{riseden} blocks $q$ from being equivalent to the content $p$ of the clause uttered. As I said above, the most likely content for $q$ to take on is $p$:
	
	\exa \emph{Default assumption for the identity of $q$}\\
	$q = p$, unless the context makes this assumption implausible. \label{default}
	\z
	
	I believe the default assumption in \xref{default} holds because $p$ is the easiest propositional content to identify in the context. This sort of content identification is likely related to the resolution of deictic pronouns. However, if the assumption that $q = p$ is highly implausible in the context, then the listener will identify $q$ with some other relevant proposition, as in \xref{urd}, to be discussed further below.\footnote{In contrast to \rise, I will treat the falling intonation typical of assertions of declaratives (H* L-L\%, `\fall') as a meaningless default, discussed in \sectref{derive}.}
	
	Following \citet{stalnaker78}, \citet{lewis79}, \citet{roberts12}, \citet{farkas10}, and others, I assume conversation is a cooperative effort to increase knowledge of the way the world is. This is pursued by asking questions and asserting answers. Both questions and assertions are used to put their content onto a stack of issues to be addressed, the discourse table. When the interlocutors agree to mutually commit to the truth of a proposition in an issue on the table, that proposition is added to a common ground of publicly mutually believed propositions. The more propositions in the common ground, the fewer ways the world might be and the more the interlocutors know about the world. Thus, adding and removing issues from the table is the means by which the purpose of conversation is achieved.
	
	Given this view of the goal of conversation and how it functions, I assume that there is an ever present pressure in conversations:
	
	\exa \emph{Requirement of support for a proposition $p$ in $I$}\\
	When an issue $I$ is added to the table $T$, there is pressure for some interlocutor $a$ to support a proposition $p \in I$ by adding $p$ to their discourse commitments $DC_a$. \label{support}
	\z
	
	Once someone fulfills this support requirement, other interlocutors can agree with the commitment made, thus resolving that issue and adding the proposition to the common ground. \xref{support} will help to explain both why questions usually signal the desire for a response, and why assertions commit the speaker to their propositional content.
	
	
	
	
\section{Application to data} \label{data1}

	
\subsection{The basic data}
	
	One and the same rising declarative form, \emph{She's nine}\rise, can be used to ask a question in \xref{crd}, and to make an assertion in \xref{urd}.
	
% \begin{multicols}{2}
\begin{exe}
\exr{crd}  Confirmative RD\\
S and A are buying a birthday card for the daughter of A's friend. While searching for a card for the correct age, S thinks A told him previously that she just turned nine, but he wants to confirm it.\\
S: She's nine$\nearrow$
\end{exe}
\begin{exe}
\exr{urd}  Assertive RD\\
S wants to enroll his daughter in music lessons with A. \\ 
S: My daughter wants to study tuba. \\
A: Okay, but there are limited places for each age group, and some age groups have already filled up. How old is she?\\
S: She's nine$\nearrow$
\end{exe}
% \end{multicols}
	
	\noindent I define questions and assertions as follows:\footnote{Possible counterexamples to \xref{question} may include rhetorical questions and reflective questions. However it could be argued that rhetorical questions are in fact indirect assertions whose derivation transits through the usual understanding of questions in \xref{question}, while reflective questions are questions in which one talks to oneself, which  fits with \xref{question} when viewed in that light.}
	
	\exa Definition of \emph{question}/\emph{inquisitive}: \\
	If S's intention in uttering $U$ is to raise an issue without settling that issue themselves, and with the expectation that A will settle that issue in reply by committing to a proposition in that issue, then $U$ is a question. \label{question}
	\z
	
	\exa Definition of \emph{assertion}/\emph{assertive}: \\
	If S's intention in uttering $U$ is to settle an issue by committing to a proposition in that issue, then $U$ is an assertion. \label{assertion}
	\z
	
	The solution I'll pursue to explaining why the RD in \xref{crd} is a question while that in \xref{urd} is an assertion is to allow pragmatic inference to play a key role in enabling the audience to uncover the speaker's intended speech act in uttering a rising declarative. Beyond this, there are some additional effects to be explained: \xref{crd} conveys some bias toward the proposition, while \xref{urd} raises a second issue. 
	
	The QUD in \xref{crd} is \emph{How old is the birthday girl?}. S utters a proposition that would answer it, \emph{that she is nine}, which I'll call $q$. According to \xref{dgUtt}, this puts \{$q$\} on the table. However, according to \xref{riseden} and \xref{default}, S's use of \rise conveys that S is not committing to $q$. Given the support requirement in \xref{support}, since S is not committing to $q$ and S is talking only to A, S must intend A to address the issue \{$q$\} that S raised by giving an answer. So given the definition of question in \xref{question}, the account in \sectref{account} explains why the RD in \xref{crd} is regarded as a question. 
	
	The second meaning effect to explain about inquisitive RDs like \xref{crd} is that they convey a bias in favor of the proposition denoted by the declarative. I pursue an explanation somewhat similar to those in both \citet{rudin18} and \citet{westera18}: In choosing to use a RD that denotes only  \emph{that she is nine}, S raises an issue that contains only that one proposition, instead of the two propositions \{\emph{that she is nine}, \emph{that she is not nine}\} that would have been raised via a polar interrogative, \textit{Is she nine?}. Given the support requirement in \xref{support}, S could not have done this if S did not have some reason to think that A was in a position to commit to this particular proposition. Thus we understand why inquisitive rising declaratives necessarily convey a contextual bias as formulated in \xref{bias}: the contextual evidence provides the justification for S to restrict A to the single proposition denoted by the declarative.
	
	Note that the speaker does not themselves need to be biased for the content of the RD. The above logic is met even if S is skeptical of that proposition, so long as S thinks A can and will commit to the proposition, as is the case for incredulous RDs (on which, more below). Note also that private speaker bias for the content of the declarative is not enough to meet the bias condition in \xref{bias}. E.g., if S had private reasons to believe that it is raining in \xref{ex:13:rain}, and S knew that A was arriving from outside, but there was no publicly available contextual evidence of rain, then the RD in \xref{rdRain} would still be infelicitous. It seems that, if S is going to raise an issue that contains only one proposition and ask A to settle it, then there must be contextual evidence available to explain why S would do so. Otherwise, S is required to provide A with more than one alternative.
	
	Now for \xref{urd}: According to \xref{riseden}, \rise conveys that S is not committing to a relevant proposition $q$. However, unlike in \xref{crd}, in \xref{urd} the default assumption that $q$ is equivalent to the propositional content of the RD, \emph{that she is nine}, is made contextually implausible by the fact that S is the clear epistemic authority with respect to that proposition. So the audience can safely assume that S's use of \rise is not meant to convey lack of commitment to the declarative content. Furthermore, since S has not conveyed a lack of commitment to this content, and given S's position of authority, along with the support requirement in \xref{support} and the prior conversational context, the audience can further infer that S intends to commit to the proposition \emph{that she is nine}, settling A's question \textit{How old is your daughter?} (this derivation of assertive commitment is revisited in greater detail in \sectref{derive}). Given the definition of assertion in \xref{assertion}, the account in \sectref{account} explains why the RD in \xref{urd} is regarded as an assertion. 
	
	The second meaning effect to explain about assertive RDs like \xref{crd} is that they raise a second issue. This is caused by \rise, which still conveys that S is not committing to some proposition $q$ that is relevant to the QUD. In \xref{urd}, the goal is to enroll S's daughter in music lessons with A. A has said that whether this can be achieved depends on whether there is room in her age group. S's answer settles what her age group is, but leaves open whether there is still room in that age group. Thus, the proposition $q$ that S expresses lack of commitment about is \emph{that there is still room in the nine-year-old group}, since if this proposition were combined with  the content of the declarative \emph{that she is nine}, they would together form a strategy $\Gamma$ for resolving the QUD, \emph{Can my daughter study tuba with you?}. By working backward from the proposition asserted and the remaining issues that need to be resolved to achieve the goal, the audience can infer the proposition targeted for lack of commitment by \rise in assertive RDs. 
	
	
	
\subsection{Incredulous uses of RDs} \label{incredRDs}
	
	
	On one of the possible analyses I sketched for the intonation of incredulous rising declaratives in \sectref{noForm}, they have their own unique phonological contour, the incredulous double rise, distinct from \rise. If so, then the analysis given for \rise in \sectref{account} won't be required to explain the incredulous RD data. However, on another analysis I sketched, incredulous RDs feature the same linguistic contour \rise found in other RDs, and the difference is only a paralinguistic one affecting the height of the final boundary tone. In case this view turns out to be correct, the following discussion explores how the account of \rise in \sectref{account} can be applied to incredulous RDs.
	
	The speaker's beliefs with respect to the content proposition $p$ of inquisitive RDs can vary. In \xref{rainEv} for example,  S could have no prior bias either way before being confronted with the evidence of A's wet umbrella. And in the confirmative RDs of \xref{crd} and \xref{mom}, S has a prior $p$ bias. Incredulous RDs on the other hand all come with a prior \notp speaker bias, though S's beliefs may or may not change in the face of the contextual evidence for $p$ that is required in order for the inquisitive RD to be felicitous. In \xref{apologize}, S believes that they should not apologize, so there is a prior \notp speaker bias that persists through speaking time.
	
	\exa S believes that they did nothing wrong:\\
	A: You should apologize.\\
	S: I should apologize\rise  \label{apologize}\hfill \citep[based on][292]{pierrehumbert90}
	\z
	
	Unlike \xref{apologize}, the context in \xref{ird} does not establish such a strong form of \notp speaker bias. There the speaker previously thought that the child was older than nine, and thus had a previous \notp bias. But suppose A and S both take A to be in a better position to know the child's age, perhaps because A works at the school and S is merely visiting. In that case S should be inclined to accept A's claim that the child is nine, and so after A's utterance, S should be on the way to believing $p$, even if S may not have completely accepted $p$ yet as there is  still some chance that A could be wrong. 
	
	The prior \notp bias in common across incredulous RDs plays a key role in explaining their incredulity. First, incredulous RDs are a subkind of inquisitive RDs, so according to \xref{riseden} and \xref{default}, the \rise in incredulous RDs conveys S's lack of commitment to the propositional content of the declarative, $p$. Moreover, since they are inquisitive RDs, they require contextual evidence for $p$. It is the contrast between the prior \notp bias and this contextual evidence for $p$ that makes the speakers of incredulous RDs incredulous.
	
	\xref{vaca} exhibits one more kind of incredulous RD that features a prior \notp speaker bias like \xref{ird} and \xref{apologize}, but that differs from them in that the contextual evidence immediately settles the issue in favor of $p$, which creates a challenge for applying the non-commitment analysis of \rise in \xref{riseden} to it.\footnote{Thanks to an anonymous reviewer for \emph{Sinn und Bedeutung} for this example.} 
	
	\exa S believes that her friend A is abroad on vacation and not due back for some time. Then, S bumps into A at the local caf\'e. \\
	S: You're back\rise \label{vaca}
	\z
	
	The truth of $p$ in \xref{vaca} is so obvious to S that she can't help but be publicly committed to it. Alexander Williams noted to me (p.c.) that her use of the pronoun \emph{you} even shows that she has updated her discourse model with A's presence. 
	
	How can S's lack of commitment to $p$ as required by \rise according to \xref{riseden} and \xref{default} be satisfied in \xref{vaca}? One possible response is to say that it can't, similar to the case of assertive RDs like \xref{urd}, and so lack of commitment is interpreted to be about some other relevant proposition $q$. However, this would be counterintuitive, since it would force us to give distinct analyses for different kinds of incredulous RDs: those like \xref{vaca} would be assertions while those like \xref{apologize} would be questions.\footnote{Treating all incredulous RDs as assertions, on the other hand, appears even less likely to work, since examples like \xref{apologize} could never be analyzed as S asserting $p$.} This is  undesirable since \xref{ird}, \xref{apologize}, and \xref{vaca} are intuitively all very similar to one another. All three could be paraphrased as ``S is incredulous that $p$", so we'd like to have a unified analysis of them.
	
	The solution is to extend the analysis I gave for \xref{ird} and \xref{apologize} to \xref{vaca} by taking \xref{vaca} to involve pretense: while S clearly knows that $p$ is true, she is nevertheless shocked at $p$ because just prior to new evidence supporting $p$, S would have happily committed publicly to \notp. Thus S conveys her surprise via the lack of commitment expressed by \rise. Compare this to exclamations of \emph{I can't believe it!} or \emph{I can't believe you're here!} when the proposition is evidently true (thanks to an anonymous reviewer and to Alexander Williams for this comparison). 
	
	
	
	
	
\subsection{A similar illocutionary ambiguity in rising imperatives} \label{riseImps}
	
	
	\citet{rudin18} observes that \rise can appear with imperatives, and that it has a roughly similar effect as with declaratives and interrogatives, conveying a lack of commitment that manifests as suggested actions that the addressee could take but is not required to. For example, consider the imperatives used by the boss in \xref{boss}, which have standard list intonation on the first two (either a slight rise to a plateau or a rise-fall-rise), culminating in a final fall (H* L-L\%, `\fall'), and compare them to the rising imperatives used by the coworker in \xref{coworker}: 
	
	\exa New employee: What should I do now?
	\ea Boss: Take the trash out, wash the sink, (then) take your break\fall \label{boss}
	\ex Coworker: Take the trash out\rise wash the sink\rise take your break\rise \label{coworker}
	\z
	\z
	
	The boss is issuing a set of commands to be carried out in a particular order, while the coworker is merely offering a menu of suggested options. Now consider the boss's use of a rising imperative in \xref{bossR}:
	
	\exa The boss has just told the new employee to replace the trash bags in all of the trash cans, tie up the used ones, and put them by the back door. After doing this, the employee asks:\\
	New employee: What should I do now?\\
	Boss: Put them in the dumpster\rise \label{bossR}
	\z
	
	The boss's rising imperative in \xref{bossR} could be interpreted as a weak suggestion if we assume that the boss is either inappropriately negligent or inappropriately unauthoritative. However, another possible interpretation is that the boss is issuing a normal command, and that \rise raises another issue, roughly \emph{How do you not know this?}, or more rudely  \emph{Are you stupid?} (indeed, the use of \rise in \xref{bossR} has the effect of insulting and belittling the employee). An explanation parallel to that given for the inquisitive/assertive RD split above can be given here, namely the boss has the social authority to issue commands while the coworker does not. This explains why the most natural interpretation of the rising imperatives in \xref{coworker} is a weakening of the commands themselves, while one natural interpretation of the rising imperative in \xref{bossR} is that it retains its usual commanding force, and \rise raises another issue that the audience is left to infer pragmatically. 
	
	I leave a more formal exploration of the interaction of my account of \rise with imperatives to future work. But we can already see the family resemblance to the analysis of declaratives: In both declaratives and imperatives, \rise's meaning usually targets the content of the clause it appears with, but it doesn't \emph{have} to. Whether the audience is cued to search for some other relevant content for \rise to interact with depends on whether the context renders its application to the clausal content improbable.
	
	
	
	
	
	
	
	
	
	
\section{Deriving assertion} \label{derive}
	
	In asserting $p$, whether via a falling or rising declarative, the speaker S intends to commit to $p$, and so from S's perspective, commitment is achieved from the moment the utterance is complete. Likewise, from the perspective of the addressee A, as soon as A has understood S's utterance as an assertion, A will take S to be committed to $p$ starting with whatever context immediately follows the utterance. My account so far leaves open how this assertive commitment comes about. Dynamic semantics/pragmatics models this by treating utterances as functions from contexts to contexts, and assertions in particular as updating discourse commitments in the output context $c_{n+1}$ to reflect the new assertive commitment. The utterance function as I defined it in \xref{dgUtt} is at odds with this view, since it only adds clausal content to the table and has no effect on commitments. This was done intentionally to avoid building commitment into a conventional discourse effect for particular clause types or intonations \citep[\emph{pace}][]{farkas10, lauer13, farkas17, jeong18, rudin18, rudin22}. Instead, my aim was to separate out linguistic meaning from context update, creating space for pragmatics to operate on linguistic meaning and context to produce assertive force. On this view, \xref{dgUtt} can be thought of as a special addition to locutionary force.  When S utters $\phi$, the context is updated with the fact of S having said it. This update includes the syntactic structure and the compositional interpretation computed on it, including the contribution of the prosodic contour, though not yet the full pragmatic consequences of any of these. Going beyond the usual locutionary act, I also take this update to include the addition of the content of $\phi$ to the table $T$, changing that component of our formal model of context. It's from this position in the conversation that the audience can, if need be, draw inferences to recover S's intended illocutionary force. 
	
	To see how this works for assertive force, consider S's assertion of a falling declarative in \xref{fd}: 
	
	\exa A: How old is your daughter?\\
	S: She's nine\fall \label{fd}
	\z
	
	Recall that `\fall' stands for the the falling intonation typical of assertions of declaratives, H* L-L\%. One approach to the analysis of assertive commitment in falling declaratives would be to treat \fall as conveying commitment, an opposite counterpart of \rise. However, I reject this view and instead treat \fall as a meaningless default. There are a few reasons for this. First, \fall does not impose a sufficient condition for the presence of assertive commitment, as it is also the standard contour for constituent questions.\footnote{\citet{bartels99} and \citet{truckenbrodt09} make the argument that constituent questions give rise to an existential presupposition, which \fall expresses commitment to. Whatever the merits of this argument are, there are other reasons below to question that \fall conveys commitment.} Beyond this, \fall also appears in imperative commands/requests, as noted in \sectref{riseImps}, and below I will discuss its presence in falling declarative questions \citep{bartels99, gunlogson08}. Given the breadth of \fall's distribution, it's hard to see how to maintain a unified commitment analysis. \fall appears to be a default elsewhere intonation. Second, \fall does not impose a necessary condition on the presence of assertive commitment, as other, non-falling contours can be used in assertive utterances. For example the rise-fall-rise contour \citep{ward85} and the contradiction contour \citep{liberman74} are two well-studied rising contours that are used in assertions that commit the speaker to the declarative content. Put otherwise, analyzing falling and rising intonation as manipulating the presence or absence of speaker commitment produces an empirically inadequate dichotomy.\footnote{Cf. \citet{rudin18} for an example of an analysis along these lines. \citet{rudin22} revises the analysis so that commitment is built into all utterances (within a restricted domain), and \rise cancels that commitment.} To maintain an intonation-encodes-commitment view while accounting for these other contours, a list of contours that speakers use to make commitments would need to be made. But if commitment is the elsewhere case, it suggests that it shouldn't be baked into lexical meanings for contours. Third, the assertive RDs discussed in this paper are another case of a non-falling contour appearing in assertive utterances. But in this case, the contour, \rise, cannot be added to a list of commitment contours since it frequently appears in inquisitive speech acts. For these reasons, I don't want to lean on \fall to explain assertive commitment. Even if it did play a direct role in producing commitment in all utterances it appears in (which it doesn't), it would still leave assertive commitment unexplained in other utterances that \fall does not appear in. 
	
	Instead I will pursue an approach in which assertive commitment is derived. An anonymous reviewer asked why I don't do this the other way around, arguing that it is more intuitive to have \fall convey commitment and derive lack of commitment from \fall's absence (the reviewer is not the only linguist to have made this argument to me). While the preceding paragraph gives several sufficient reasons not to imbue \fall with commitment, my approach can also be defended on the grounds of the Stalnakerian view of conversation I have assumed. Conversation is about increasing understanding of how the world is, and commitment plays a central role in achieving this goal. So commitment is part of the normal course of conversational events, and is in that sense relatively unmarked. Lack of commitment, on the other hand, presents a detour on the road to achieving the conversational goal. Sometimes detours are necessary, but they are best avoided if possible. Thus the choice to convey lack of commitment is relatively less likely, and therefore more marked, and so more likely to be marked with a meaningful contour than commitment is. 
	
	
	
	So how is assertive commitment to the proposition expressed in \xref{fd} \emph{that she is nine}, call it $p$, derived? In a nutshell, if S had intended to convey lack of commitment to $p$, S should have marked that with \rise; since S didn't do so, S must intend to commit to $p$. In greater detail: According to \xref{dgUtt}, S adds the singleton set of  $p$ to the table $T$. There is nothing in the meaning of the declarative clause itself, or in \fall, or in the utterance function that implies S's commitment to $p$. Given the broadly Stalnakerian view of conversation adopted here, and the support requirement in \xref{support} in particular, I assume that by adding \{$p$\} to $T$, S either intends to commit to $p$ or not (i.e., S intends a lack of commitment to $p$). I also assume a requirement to maximize non-at-issue content (MaxNAI), comparable to Gricean Quantity and Maximize Presupposition, that says that if S intends to convey certain non-at-issue content $\gamma$, and there is a linguistic expression $l$ that conveys $\gamma$, then an utterance $\phi$ that includes $l$ is preferred to an alternative utterance $\psi$ that does not include $l$ but is otherwise identical. MaxNAI can render \rise preferred to \fall in appropriate utterances: There is a linguistic form, \rise, that explicitly conveys lack of commitment. So if S intends to convey lack of commitment to $p$ while uttering declarative $\phi$, S should explicitly do so via \rise. In the case of the falling declarative in \xref{fd}, S has  chosen not to use \rise, and so we can infer that S \emph{doesn't} intend lack of commitment to $p$. But since, in uttering $\phi$, S either intends commitment to $p$ or lack of commitment to $p$, it follows that S intends to commit to $p$. 
	
	Here is the same train of reasoning, in schematic form:
	
	\begin{enumerate}
		

		
		\item S utters ``She's nine$\searrow$"  \label{utt}\hfill \xref{fd}
		\begin{itemize}
			\item $\neg$ S explicitly conveys $p \not\in$ \emph{DC\textsubscript{S}} \hfill (consequence of  \xref{utt})
		\end{itemize}
		
		
		\item $T$ + \{$p$\} \hfill (\xref{utt}, utterance function in \xref{dgUtt})\label{uttT}
		
		
		\item S intends $p \in$ \emph{DC\textsubscript{S}} $\vee$ S intends $p \not\in$ \emph{DC\textsubscript{S}} \hfill (\xref{uttT}, Stalnakerian pragmatics/\xref{support})\label{taut}
		
		\item S$'$ utters ``She's nine$\nearrow$"  \label{ex:13:alt} \hfill (NAI-stronger alternative to \xref{utt})
		\begin{itemize}
			\item S$'$ explicitly conveys $p \not\in$ \emph{DC\textsubscript{S$'$}} \hfill (consequence of \xref{ex:13:alt})
		\end{itemize}
		
		
		\item S intends $p \not\in$ \emph{DC\textsubscript{S}} $\rightarrow$ S explicitly conveys $p \not\in$ \emph{DC\textsubscript{S}} \hfill (\xref{utt}, \xref{ex:13:alt}, MaxNAI)\label{Quant}
		
		\item $\neg$ S intends $p \not\in$ \emph{DC\textsubscript{S}} \hfill (modus tollens on \xref{utt} \& \xref{Quant})\label{notIntend}
		
		\item S intends $p \in$ \emph{DC\textsubscript{S}} \hfill (disjunctive syllogism on \xref{taut} \& \xref{notIntend})
		
	\end{enumerate}
	
	\noindent This is how assertive commitment can be derived for falling declaratives based on the proposed meaning of \rise in \xref{riseden}, combined with a broadly Stalnakerian view of conversation, including the support requirement in \xref{support}, as well as Gricean pragmatics.
	
	Given that assertive commitment is essentially an implicature on this view, it's reasonable to wonder if it can be canceled. The answer may be yes, and the evidence comes from falling declarative questions: 
	
	\exa A and S are colloquium committee organizers. A is in charge of today's colloquium, so both A and S know that A has more information about what is going on with it than S.\\
	A: We have a problem: We need someone to go pick the invited speaker up from the airport, but Kate is on the other side of town.\label{pickle}
	\ea S: And James isn't available\fall \label{pickleF}
	\ex S: And James isn't available\rise \label{pickleR}
	\z
	\z
	
	From A's claim that they have a problem, S is able to infer that the people who would usually be asked to pick the speaker up from the airport are unavailable. But A only mentioned Kate, so S asks about the other usual person James. The felicity of the falling version in \xref{pickleF} shows that falling declaratives are not always assertions, which suggests a cancellation of the derivation sketched above (cf. \citealt[243]{bartels99} and \citealt{gunlogson08} for more discussion of falling declarative questions). 
	
	However, the felicity of the rising version in \xref{pickleR} raises a question for the account I gave above: Why doesn't MaxNAI make the use of \rise necessary for an inquisitive interpretation? Note that there is an intuitive difference between \xref{pickleF} and \xref{pickleR}: In \xref{pickleF}, S seems to be pretty confident about the conclusion they have drawn that James isn't available based on A's claim that they have a problem; but given A's epistemic authority, S wants A to confirm this conclusion, and so S's utterance is still a question looking for a \emph{yes} or \emph{no} response. In \xref{pickleR}, on the other hand, S seems a little less confident that this inference about James is true. So the choice to use \fall instead of \rise in \xref{pickleF} still has an interpretational effect, even if it doesn't lead to full assertive commitment. The reason \xref{fd} leads to full assertion while \xref{pickleF} does not is that A is the clear epistemic authority in \xref{pickle}; A's authority disrupts the derivation of assertion \xref{pickle}.\footnote{Relative authority then seems to play a central role in the force interpretation of utterances, especially if the intonational form of the utterance would otherwise usually be used to convey the opposite force: Assertive rising declaratives are identified in cases where S is the authority on $p$, while falling declarative questions are identified in cases where A is the authority on $p$. Cf. a similar observation in \citet{jeong18}, and see \citet{hansen01} for a related observation about the use of French polar interrogatives with declarative word order from a conversation analytic perspective.} In \citepos{gunlogson08} terms, S is making a dependent commitment in \xref{pickleF}. While these remarks  remain preliminary, the existence of falling declarative questions, distinct from both falling declarative assertions and rising declaratives, further suggests that it is correct to derive assertion pragmatically instead of baking it into the meaning of the declarative+\fall pairing. 
	
	
	
\section{Conclusion} \label{sec:13:conclude}
	
	The advantage of the account I have proposed is its ability to explain how we arrive at distinct illocutionary forces when interpreting one and the same linguistic form. I have posited unitary meanings for linguistic forms (declaratives denote $p$, polar interrogatives denote $\{p, \neg p\}$, \rise conveys lack of commitment to a relevant proposition $q$), as well as a single utterance function that adds utterance content to the table. Speakers can employ the combination of declarative and \rise to assert or to question, thanks primarily to the ability of \rise to apply directly to the declarative content or not. I further observed that pragmatic pressure to grow the common ground in turn leads to pragmatic pressure to raise and resolve issues via interlocutor support for a proposition in the issue. This combined with the meanings of the linguistic forms enables speakers to implicate, and addressees to derive, the distinct discourse effects of inquisitive and assertive RDs such as \emph{She's nine}\rise in \xref{crd} and \xref{urd}, as well as  falling declaratives as in \xref{fd}. Illocutionary force on this view resides purely in the pragmatics, with the audience's ability to recover the speaker's intended force depending only in part on input from the linguistic system. 
	
	
\subsection{Comparison to prior accounts} \label{priorAcc}
	
	\citet{farkas17}, \citet{jeong18}, and \citet{rudin18, rudin22} produce distinct accounts that nevertheless arrive at the same conclusion that clause type $+$ intonation determines illocutionary force as a matter of convention. \citet{farkas17} argue that \rise is a semantic operator that forms polar interrogatives. \citet{rudin18, rudin22} argues that intonation manipulates the utterance function, with \rise calling off commitment to declarative content. The result is that each of these accounts are in their own way too rigid to handle assertive RDs like \xref{urd}, and are forced to set them aside.  \citet{jeong18} meanwhile proposes that there are two phonological contours~-- a steep one corresponding to inquisitive RDs, and a shallow one corresponding to assertive RDs (even if the distinction may often be masked by phonetic variation). I argued against these intonational claims in \sectref{noForm}, in particular arguing that shallow rises appear quite naturally in inquisitive RDs. Furthermore, Jeong's account hypothesizes four distinct sentence types with four overlapping but distinct conventional discourse effects: falling declaratives, polar interrogatives, steep RDs, and shallow RDs. The theoretical advantage of my account is that these overlapping but distinct  discourse update effects emerge from a unitary semantics for clause types and \rise, combined with pragmatics.  
	
	Westera also offers a unified account of inquisitive and assertive RDs \citep{westera13, westera17, westera18}: On this view, \rise is claimed to convey that S is violating a Gricean maxim. A general challenge for this view is that it incorrectly predicts \rise to be felicitous for run-of-the-mill quantity implicatures (e.g. \emph{some} implicates \emph{not all}), since quantity implicatures involve a violation of the maxim of quantity. But this analysis also faces a specific challenge from one of the key phenomena it is meant to explain, assertive rising declaratives: In an example like \xref{urd}, S is respecting all maxims, and so \rise should be infelicitous contrary to fact. First, S's utterance is relevant and informative enough relative to the local question \emph{How old is your daughter?}. Second, an anonymous reviewer for \emph{Sinn und Bedeutung} suggests that Westera would say that S's RD  violates relevance relative to the larger QUD, \emph{Can my daughter study tuba with you?}. While an account of relevance could be stated so that it predicts S's utterance to be irrelevant to the larger QUD, such an account would be undesirable. After all, S's utterance is clearly a relevant step in a strategy to resolve the larger {QUD}, indeed the most helpful step S can take at that juncture, so it would be odd to claim that \rise is felicitous in \xref{urd} because S's utterance is \emph{not} relevant to the larger {QUD}. For a useful comparison, consider a genuine relevance violation example like \xref{ladino} (discussed in \citealt{westera13}).

\begin{exe}
\exr{ladino} 
	A: Do you speak Spanish?\\
	S: I speak Ladino\rise \jambox*{(\citealt[]{jeong18}, \citealt[]{farkas17}, based on \citealt[]{ward85})}
\end{exe}

	In \xref{ladino}, S doubly violates relevance, first for A's local question by not directly answering it, and again for the larger \emph{QUD} because S is unaware of what it is and so is uncertain about the relevance of the present utterance to it. This obviously contrasts with \xref{urd}, in which the relevance of S's utterance to both the local question and the larger {QUD} is quite clear. The account I have proposed has no issue here, since \rise conveys that S lacks commitment to a relevant proposition, \emph{that Ladino is good enough for your purposes (whatever they might be)}. 
	
	
	
	
\subsection{Looking ahead/other issues for future work} \label{openIss}
	
	
	Future work is needed on the view that assertive force is pragmatically derivable. This view of assertion may have consequences for the acquisition of the illocutionary force, since it suggests that children might be able to build the pragmatic category of assertion from more basic components of pragmatics and grammar. 
	
	Another avenue for future work is to more carefully explore and model relative authority between speaker and addressee on both the epistemic and the social dimension, which played crucial roles in explaining assertive RDs and falling declarative questions, as well as rising imperative commands. Perhaps contextual models should include authority parameters along these two dimensions, which could be appealed to when determining whether \rise applies to the clausal content $p$ or to some other relevant proposition $q$ (cf. \citealt{merin97}, who don't discuss rising declaratives, but who do propose that intonational meaning encodes relative social power between speakers and addressees).

\section{Acknowledgements}
This work is supported by the ERC Advanced Grant 787929 \textit{Speech Acts in Grammar and Discourse} (SPAGAD). Thank you to two anonymous reviewers for this volume, Brian Buccola, Valentine Hacquard, Sunwoo Jeong, Jeffrey Lidz, Deniz Rudin, Bernhard Schwarz, Michael Wagner, Alexander Williams, Yu'an Yang, and Manfred Krifka and the SPAGAD project members for helpful discussion. Thank you also to the audiences at Sinn und Bedeutung, the LSA Annual Meeting, the ZAS workshop on \textit{Biased Questions: Experimental Results \& Theoretical Modelling}, the Rutgers Linguistics Colloquium, the University of Chicago Linguistics Colloquium, the University of Maryland Linguistics Colloquium, the Graduate Linguistics Expo at Michigan State University (GLEAMS 2022), the Linguistischer Arbeitskreis K\"oln, and to two seminars. All mistakes are my own. h/t to Flannery O'Connor for the title.

\printbibliography[heading=subbibliography,notkeyword=this]
	
\end{document}
