\documentclass[output=paper,colorlinks,citecolor=brown]{langscibook}
\ChapterDOI{10.5281/zenodo.17158184}
\author{Cory Bill\affiliation{Leibniz-Zentrum Allgemeine Sprachwissenschaft (ZAS)}\orcid{0000-0003-0135-1745} and Todor Koev\affiliation{Ben-Gurion University of the Negev}\orcid{0000-0001-5510-1501}}

\title{Bias in tag questions}
\abstract{We study various kinds of reverse-polarity tag questions in English, arguing that the speaker biases that such questions convey differ across three dimensions: optionality, strength, and polarity. We propose that the bias profile in each case primarily depends on the shape of the tag, while pointing at the central role of polarity focus and answer salience in generating these biases.}

%move the following commands to the "local..." files of the master project when integrating this chapter

\IfFileExists{../localcommands.tex}{%hack to check whether this is being compiled as part of a collection or standalone
   % add all extra packages you need to load to this file

\usepackage{tabularx,multicol}
\usepackage{url}
\urlstyle{same}

\usepackage{listings}
\lstset{basicstyle=\ttfamily,tabsize=2,breaklines=true}

\usepackage{langsci-basic}
\usepackage{langsci-optional}
\usepackage{langsci-lgr}
\usepackage{langsci-osl}
% \usepackage{./langsci/styles/langsci-lgr}
% \usepackage{./langsci/styles/langsci-osl}
% \usepackage{langsci-gb4e}

\usepackage{tikz}
\usetikzlibrary{patterns,calc}
\pgfdeclarepatternformonly{south east lines}{\pgfqpoint{-0pt}{-0pt}}{\pgfqpoint{3pt}{3pt}}{\pgfqpoint{3pt}{3pt}}{
    \pgfsetlinewidth{0.6pt}
    \pgfpathmoveto{\pgfqpoint{0pt}{3pt}}
    \pgfpathlineto{\pgfqpoint{3pt}{0pt}}
    \pgfpathmoveto{\pgfqpoint{.2pt}{-.2pt}}
    \pgfpathlineto{\pgfqpoint{-.2pt}{.2pt}}
    \pgfpathmoveto{\pgfqpoint{3.2pt}{2.8pt}}
    \pgfpathlineto{\pgfqpoint{2.8pt}{3.2pt}}
    \pgfusepath{stroke}}
    
\usepackage{stmaryrd}
\usepackage{wasysym}
\usepackage{multirow}
\usepackage{caption}
\usepackage{subcaption}
\usepackage{mathrsfs}
\usepackage{qtree}

\usepackage{linguex}


   %pminos do not split footnotes
% \interfootnotelinepenalty=10000 %Footnote in Laporte chapters has to be split SN


%\DeclareIndexNameFormat{default}{%
%\nameparts{#1}%
%\usebibmacro{index:name}%
%{\index[names]}%
%{\namepartfamily}%
%{\namepartgiveni}%
% {}% L1
% {}% L2
%{\namepartprefix}% generates spurious space L3
%{\namepartsuffix}% generates spurious space L4
%}

%  {\DeclareIndexNameFormat{default}{%
%     \usebibmacro{index:name}{\index[names]}{#1}{#3}{#5}{#7}}}

%\DeclareIndexNameFormat{default}{%
%  \usebibmacro{index:name}{\sindex[nom]}{#1}{#3}{#5}{#7}}

%\DeclareIndexNameFormat{default}{%
%  \usebibmacro{index:name}{\sindex[person]}{#1}{#3}{#5}{#7}}
%\DeclareIndexNameFormat{default}{%
%\nameparts{#1} \usebibmacro{index:name}{\sindex[person]]}{\namepartfamily}{‌​\namepartgiven}{\nam‌​epartprefix}{\namepa‌​rtsuffix}}

%\newcommand{\smiley}{:)}

%\renewbibmacro*{index:name}[5]{%
%\usebibmacro{index:entry}{#1}%
%{\iffieldundef{usera}{}{\thefield{usera}\actualoperator}\mkbibindexname{#2}{#3}{#4}{#5}}}

% \newcommand{\noop}[1]{}

%remove for final
%\overfullrule=1mm

\newcommand{\tobi}[2]}}
\renewcommand{\S}[1]{\tobi{#1}{\textsc{*}}}

% this volume references
% puts: [this volume]
% already defined: \citetv
%\newcommand{\citepv}[1]{(\citeauthor{#1} \citeyear*{#1} [this volume])}
\newcommand{\citealtv}[1]{\citeauthor{#1} \citeyear*{#1} [this volume]}

%parentheses around example number
\newcommand{\pref}[1]{(\ref{#1})}

% in-text examples

\newcommand{\lnex}[1]{\textit{#1}} %target lang word
\newcommand{\lnlit}[1]{(lit.: `#1')} %literal reading
\newcommand{\lnlat}[1]{(#1)} % latinization
\newcommand{\lntrans}[1]{`#1'} %translation
\newcommand{\lnexl}[2]%
{\lnex{#1}{} \lnlat{#2}} % ex with latinization
\newcommand{\lnexlat}[3]{\lnex{#1}{} \lnlat{#2}{} \lntrans{#3}} % ex with latinization and tranl.

%ch01
\newcommand{\co}[1]{\mbox{\textbf{#1}}}

%ch09

\newcommand{\cyrbulg}[1]{\begin{otherlanguage*}{bulgarian}#1\end{otherlanguage*}}


%ch10
\newcommand{\nlp}{{\small NLP}}
\newcommand{\mwe}{{\small MWE}}
\newcommand{\rae}{{\small RAE}}
\newcommand{\lvc}{{\small LVC}}
\newcommand{\pos}{{\small P}o{\small S}}
%\newcommand{\todo}[1]{ \textcolor{red}{#1} }

%\renewcommand{\labelenumi}{\theenumi}
%\ainamefmt{{vv}{ll}{, ff}{, jj}} % fullname

\newcommand{\biberror}[1]{{\color{red}#1}}

\newcommand{\osenovaitem}{--~}
   %% hyphenation points for line breaks
%% Normally, automatic hyphenation in LaTeX is very good
%% If a word is mis-hyphenated, add it to this file
%%
%% add information to TeX file before \begin{document} with:
%% %% hyphenation points for line breaks
%% Normally, automatic hyphenation in LaTeX is very good
%% If a word is mis-hyphenated, add it to this file
%%
%% add information to TeX file before \begin{document} with:
%% %% hyphenation points for line breaks
%% Normally, automatic hyphenation in LaTeX is very good
%% If a word is mis-hyphenated, add it to this file
%%
%% add information to TeX file before \begin{document} with:
%% \include{localhyphenation}
\hyphenation{
    Beck-man
    Ngu-yen
    back-chan-nel
    back-chan-nels
    mo-not-o-nous
    ste-reo-typ-i-cal
}

\hyphenation{
    Beck-man
    Ngu-yen
    back-chan-nel
    back-chan-nels
    mo-not-o-nous
    ste-reo-typ-i-cal
}

\hyphenation{
    Beck-man
    Ngu-yen
    back-chan-nel
    back-chan-nels
    mo-not-o-nous
    ste-reo-typ-i-cal
}

    \bibliography{localbibliography}
    \togglepaper[23]
}{}

%\RequirePackage{etex}

\begin{document}
\maketitle

\section{Introduction} \label{sect:tagq:intro}
Tag questions are composed of two elements: a \textit{declarative anchor} and a \textit{tag}. While such questions require a response from the addressee, they also seem to convey some prior belief on the part of the speaker as to what the true answer is. Specifically, they seem to convey a belief or bias on the part of the speaker that the proposition expressed by the anchor is true.\footnote{Henceforth, when we talk about a question conveying `bias', we are referring to this speaker-oriented type of belief (i.e., `epistemic bias' or `original bias'). Other kinds of question bias that have been discussed in the literature include `contextual bias', or bias that has to do with evidence available in the context \citep{buring2000, romero2004, sudo2013, northrup2014, domaneschi2017}, as well as `answer bias', or bias about which answer the addressee is going to provide (cf. \citealt{krifka2015, Malamud2015, anderbois2019}). While we have some ideas about how these biases could ultimately be derived from speaker bias, a proper presentation and exploration of these ideas goes beyond the scope of this paper.} For example, \xxref{TQ.1}{TQ.3} suggest that it is indeed raining.

\is{Tag question}
\ea It's raining, isn't it?\label{TQ.1}
\ex It's raining, right?\label{TQ.2}
\ex It's raining, eh?\label{TQ.3}
\z
\il{English}

\noindent
While, as \xxref{TQ.1}{TQ.3} show, there are a variety of forms the tags can take, we will be focusing here on questions with final rising boundary tones and so-called \textit{reverse-polarity tags}, like \xref{TQ.1}. These are questions with a syntactically interrogative tag that is of the opposite polarity to its declarative anchor. We will refer to the variants with negative tags as \textit{negative-tag questions} and the variants with positive tags as \textit{positive-tag questions}.\footnote{For recent discussions regarding how falling boundary tones and/or matching-polarity tags affect the meaning conveyed by tag questions, see \citet{Reese2009}, \citet{Malamud2015}, and \citet{krifka2015}.}

As noted in \citet{Ladd1981a}, reverse-polarity tag questions have been associated with two kinds of intonation patterns: \textit{nuclear} and \textit{post-nuclear}. The nuclear intonation pattern involves a short pause after the anchor and a separate pitch accent on the tag. We will signal this intonation contour by placing a `||' between the anchor and the tag, and by capitalizing the auxiliary verb in the tag, as shown in \xref{Nu}. In contrast, the post-nuclear intonation pattern involves no clear pause after the anchor and no separate pitch accent on the tag, with the pitch contour on the tag merely being a continuation of the contour of the anchor. This pitch accent will be signalled by placing a `$=$' between the anchor and the tag, as shown in \xref{PNu}.\footnote{From a more theoretical point of view \citep{Selkirk2005}, these two prosodic patterns reflect the fact that nuclear tags form their own IntP, while post-nuclear tags are included in the IntP associated with the anchor. }
\is{}
\ea You don't believe in Santa Claus || DO you?\label{Nu} \hfill (nuclear)
\ex You don't believe in Santa Claus $=$ do you?\label{PNu} \hfill (post-nuclear)
\z
\il{English}

\noindent
As presented in \tabref{tq.targeted}, these two axes of variation (tag polarity and intonation pattern) generate four distinct kinds of tag questions, all of which we will be exploring in the current paper.

\begin{table}
 \begin{tabularx}{\textwidth}{lXX}
  \lsptoprule
            & positive-tag  & negative-tag  \\
  \midrule
  post-nuclear  &   \textit{Timmy can't swim $=$ can he?}  & \textit{Timmy can swim $=$ can't he?}   \\
  nuclear  &  \textit{Timmy can't swim || CAN he?}  &   \textit{Timmy can swim || CAN'T he?} \\
  \lspbottomrule
 \end{tabularx}
 \caption{Typology of tag questions.}
\label{tq.targeted}
\end{table}

For each of the tag questions in \tabref{tq.targeted}, we have two goals. The first is to identify the bias profile associated with it. The second is to propose an analysis that derives this profile. We claim that the bias profiles for each of these questions are composed of three elements: polarity (positive vs. negative), strength (weak vs. strong), and optionality (optional vs. obligatory). We will now present each of these elements in turn, including examples of their different settings. In doing so, we will also present the diagnostic tests that we will use throughout this paper to identify the bias profiles of our targeted tag questions. 

We start with bias polarity, which has two possible settings: positive and negative. In this paper, we will not be using any special diagnostics to determine the polarity of the bias associated with our targeted tag questions. This is because the bias polarity is intuitively clear and uncontroversial, and moreover this polarity is revealed by one of our diagnostic tests for bias strength (outlined below). Notice also that, as shown in \xref{Bias.Pos} and \xref{Bias.Neg}, the bias inference has a polarity that matches the polarity of the anchor and is the opposite of the tag's polarity. 

\is{}
\ea You like pasta, don't you?\hfill(positive bias)\\ $\rightsquigarrow$ \textit{The speaker believes that you like pasta.}\label{Bias.Pos}
\ex You don't like pasta, do you?\hfill(negative bias)\\ $\rightsquigarrow$ \textit{The speaker doubts that  you like pasta.}\label{Bias.Neg}
\z
\il{English}

Now for bias strength, which we claim can be \textit{weak} or \textit{strong}. We will use two novel diagnostic tests to determine the strength of the bias; we call these the Follow-up Test and the Weighted Coin Test. In the Follow-up Test, a question is followed up with an epistemic statement which ostensibly reports the strength of the bias conveyed by the original question. Infelicity is expected to be generated when there is a mismatch between the strength of the bias conveyed by the question and the strength of the epistemic follow-up expression. For example, the question in \xref{Bias.Weak} is felicitous when combined with a weak epistemic expression like \textit{suspect} but less felicitous when combined with a strong epistemic expression like \textit{be sure}.\footnote{We chose \textit{suspect} and \textit{be sure} because these expressions are commonly used in English and because they clearly differ in modal strength. More generally, we hypothesize that epistemic expressions of similar strengths give rise to similar judgments. } The pattern presented in \xref{Bias.Strong} is the opposite. This indicates that  post-nuclear negative-tag questions convey a weak bias, whereas  nuclear negative-tag questions convey a strong bias.\footnote{Notice that \xref{Bias.FolUp.Sure1} and \xref{Bias.FolUp.Suspect2} are unacceptable for two different reasons. \xref{Bias.FolUp.Sure1} is bad because it overstates the bias of the tag question, a Quality violation,  while \xref{Bias.FolUp.Suspect2} is bad because it understates that bias, a Quantity violation. The same applies to the Weighted Coin Test, discussed below.} Here and below, in paired examples we will use `\#' to mark a contrast in acceptability, without committing to how strong the claimed contrast is.

\is{}
\ea This is your book $=$ isn't it? That is to say, ...\hfill(weak bias)\label{Bias.Weak}
    \ea I suspected it was.\label{Bias.FolUp.Suspect1} 
    \ex \#I was sure it was.\label{Bias.FolUp.Sure1}
    \z
\ex This is your book || ISN'T it? That is to say, ...\hfill(strong bias)\label{Bias.Strong}
    \ea \#I suspected it was.\label{Bias.FolUp.Suspect2} 
    \ex I was sure it was.\label{Bias.FolUp.Sure2}
    \z
\z
\il{English}

Our second diagnostic for bias strength is the Weighted Coin Test. This test involves a context with a coin that is weighted to land on heads/tails to some degree. The degree of this weighting dictates the strength of the prior belief, which is expected to match the strength of the bias conveyed by the question. When there is a mismatch between the strength of that prior belief (dictated by the weighting of the coin) and the strength of the question bias, we expect the question to be infelicitous. In contrast, when there is a match between these two elements, the question should be felicitous. For example, the  post-nuclear negative-tag question in \xref{WC.Weak} is felicitous only when combined with the weaker prior belief (i.e., a more weakly weighted coin), indicating that this question conveys a weak bias. In contrast, the nuclear negative-tag question in \xref{WC.Strong} is more felicitous when combined with the stronger prior belief (i.e., a more strongly weighted coin), indicating a strong bias.

\is{}
\ea Mary and John are playing with a coin that they both know is designed so that it lands on tails \textit{N\%} of the time. Mary tosses the coin and it lands on the other side of John's legs, such that only John can see the result. John looks confused, causing Mary to say:\label{WC}
    \ea 70\% / \#99\%: It landed on tails $=$ didn't it?\hfill(weak bias)\label{WC.Weak}
    \ex \#70\% / 99\%: It landed on tails || DIDN'T it?\hfill(strong bias)\label{WC.Strong}
    \z
\z
\il{English}

Finally, we claim that there is also variation with regard to the optionality of the bias conveyed by a tag question, which can be optional or obligatory. To determine the optionality features, we will use the By Any Chance Test, proposed in \citet{sadock1971}. This test is based on the observation that the discourse marker \textit{by any chance} only combines with questions that can receive a neutral interpretation, or questions for which any associated bias was only optionally conveyed. This means that an infelicitous combination of \textit{by any chance} and a question can be taken as evidence that the question is obligatorily biased. For example, the felicity of the post-nuclear positive-tag question in \xref{Bias.Opt} indicates that the bias it may convey is optional, while the infelicity of the  post-nuclear negative-tag question in \xref{Bias.Obl} indicates that the bias conveyed is obligatory.

\is{}
\ea By any chance, Julia isn't here $=$ is she? \hfill (optional bias)\label{Bias.Opt}
\ex \#By any chance, Julia is here $=$ isn't she? \hfill (obligatory bias)\label{Bias.Obl}
\z
\il{English}

Our analysis of tag questions is predicated on the assumption that the bias properties of such questions primarily follow from the properties of the tag itself, as previously proposed in \citet{romero2004}. That is, we analyze tag questions as complexes consisting of a declarative and an elliptical question, where the latter conveys a bias with the very same features as would the corresponding independent non-elliptical question. This means that, for example, the post-nuclear negative-tag question in \xref{Bias.Tag} and the high negation question in \xref{Bias.HNQ} share the same bias profile.

\is{}
\ea Julia is here $=$ isn't she? \label{Bias.Tag}
\ex Isn't Julia here? \label{Bias.HNQ}
\z
\il{English}

\noindent 
Against this general parallelism between tags and independent questions, we will show how our analysis can capture the polarity, strength, and optionality settings for each of our chosen tag questions. We will argue that the trigger of bias in tag questions is `polarity focus', i.e., focus marking on some polar element, such as negation or a covert \textsc{verum} operator. Specifically, we will claim that polarity focus triggers a bias that is obligatory and weak, while the semantics of the focused operator determines the polarity feature and may further boost the strength of the bias. 

In the following sections of this paper, we will present and apply the pieces of our analysis gradually, going through the following steps: (i) present a piece of our analysis, (ii) present the profile of the bias conveyed by one or more tag questions, and (iii) explain how the analytical pieces unveiled so far can account for this bias profile. Specifically, in \sectref{Sect2} we present a general analysis of tag questions and show how this analysis can capture the bias profile of post-nuclear positive-tag questions. In \sectref{Sect3}, we provide some short background on polarity focus and show how its effects raise the salience of one of the answers and derive the bias profile of post-nuclear negative-tag questions. \sectref{Sect4} introduces the phenomenon of \textsc{verum} accent and brings together all of the machinery to capture the bias properties of nuclear tag questions. \sectref{Sect5} evaluates previous accounts and \sectref{Sect6} is the conclusion. 

\section{Tag questions}\label{Sect2}

We start by laying out a general analysis of question tagging and then show how our analysis can be fruitfully applied to post-nuclear positive-tag questions.

\subsection{General analysis}

It has been noted by many that tag questions of the form discussed here appear to be hybrid sentences. That is, they combine a declarative and a (VP-elliptical) interrogative clause in a single structure. It is for this reason that previous work dubbed them `queclaratives' \citep{sadock1971}, `part statement and part question' \citep{Rando1980}, `double-barreled speech acts' \citep{Ladd1981a}, `complex speech acts' \citep{Reese2007}, `speech act disjunctions' \citep{krifka2015}, etc. Although the details of these characterizations and their associated analyses differ, the important point is that tag questions encode a regular proposition and a question partition, and these meanings are not further combined into a single semantic object. The core motivation for this claim seems to be that, if the two parts were to be collapsed into a single meaning, the output would either be a regular question (thus losing the bias) or a regular proposition (thus losing the interrogative force). In order to preserve both of these effects, we assume that the link between these two parts of tag questions is mediated by a covert \textsc{tag} operator. What this operator does is take the anchor meaning and the tag meaning as arguments, and create out of them a complex `dot' object of the form $\Diamond p \, \bullet \, Q$, where $Q$ corresponds to the tag question and $\Diamond p$ corresponds to the anchor proposition prefixed by an epistemic possibility operator. An appropriate meaning for \textsc{tag} with these properties is given in \xref{TAG}.\footnote{One plausible line is that the epistemic possibility operator introduced by \textsc{tag} is contributed by the rising boundary tone on the question tag. We leave the issue of the provenance of this operator to future work.}

\is{}
\ea $\llbracket \textsc{tag} \rrbracket  = \lambda Q\lambda p\,.\,\Diamond p \, \bullet \, Q$\label{TAG}
\z
\il{English}
%It is possible to view TAG as the contribution of an INT operator, here specified as `rising intonation'. Along these lines, the same type of expression (i.e. anchor+tag) with 'falling intonation' might contribute a version of TAG without the modal operator in the anchor.
Such metalinguistic dot operators have been previously used as separators between two aspects of meaning that are associated with the same linguistic structure (see \citealt{Pustejovsky1996}; \citealt{potts2005}; \citealt{Asher2011}). What we intend the dot operator to do for us is ship the (modalized version of the) anchor proposition and the tag question partition to the pragmatic component, where these are treated as engendering two separate speech acts, i.e., a modal assertion and a polar question. With this basic analysis of tag questions in place, we will now show how it can account for the bias profile of post-nuclear positive-tag questions.

\subsection{Post-nuclear positive-tag questions}\label{Sec.PNu.Pos}

Post-nuclear positive-tag questions, like \xref{Pos.PNu}, are composed of a negative anchor plus a positive tag, and contain no perceivable pause after the anchor and no separate pitch accent on the tag. As with all the tag questions we will explore, its bias profile is characterized by its optionality, polarity and strength settings.
\is{}
\ea You don't eat fish $=$ do you?\label{Pos.PNu}
\z
\il{English}

Starting with optionality, as mentioned in \sectref{sect:tagq:intro}, we diagnose its setting with the By Any Chance Test. As shown in \xref{Pos.PNu.BAC}, post-nuclear positive-tag questions can be felicitously combined with the discourse marker \textit{by any chance}, indicating the possibility of a neutral reading of this question. In other words, the bias conveyed by  post-nuclear positive-tag questions is optional.
\is{}
\ea By any chance, you don't speak Romanian $=$ do you?\label{Pos.PNu.BAC}
\z
\il{English}
Additional evidence for this neutral reading comes from \citet{Reese2009}, who point out that in a context like \xref{Pos.Pnu.RA} the tag question does not convey any bias. This optionality of bias in post-nuclear positive-tag questions was also noted in \citet{sadock1971} and \citet{Ladd1981a}. 
\is{}
\ea A and B are trying to complete a task at which neither is proficient, but at which Julie is known to be.\label{Pos.Pnu.RA}
\sn A: We need someone who has consulted for us before.  
\sn B: Julie isn't here $=$ is she?
\z
\il{English}

While the bias conveyed by such questions is optional, we would still like to identify the features it has when it is present. As already mentioned, we will employ the same diagnostic tests to identify both the strength and the polarity settings of our targeted biases, starting with the Follow-up Test. As the contrast in \xref{Pos.PNu.TIS} shows, the bias conveyed by  post-nuclear positive-tag questions is negative and weak.
\is{}
\ea Mark isn't a body-builder $=$ is he? That is to say, ...\label{Pos.PNu.TIS}
    \ea I suspected he wasn't.
    \ex \#I was sure he wasn't.
    \z
\z
\il{English}
Our second diagnostic test for strength, the Weighted Coin Test, provides further support that the bias conveyed by post-nuclear positive-tag questions is weak. As shown in \xref{Pos.Pnu.WC.1}, such questions are infelicitous when the speaker's prior belief that the coin would not land on tails is very high. The cause of this infelicity as coming from the strength of the bias is confirmed by the felicity that occurs when the chance of the coin landing on heads is decreased significantly, as shown in \xref{Pos.Pnu.WC.30}.

\is{}
\ea Mary and John are playing with a coin that they both know is designed so that it lands on tails \textit{N\%} of the time. Mary tosses the coin and it lands on the other side of John's legs, such that only John can see the result. John looks confused, causing Mary to say:\label{Pos.PNu.WC}
    \ea 1\%: \#It didn't land on tails $=$ did it?\label{Pos.Pnu.WC.1}
    \ex 30\%: It didn't land on tails $=$ did it?\label{Pos.Pnu.WC.30}
    \z
\z
\il{English}

In sum, the bias profile of  post-nuclear positive-tag questions is optional, negative, and weak. We will now show how this profile can be accounted for with the elements of our analysis introduced so far.

On our analysis,  post-nuclear positive-tag questions are the most basic form of tag question. That is, as presented in \xref{Pos.Pnu.Anyl}, they are composed of a modalized proposition in the anchor and an elliptical positive polar question.

\ea Mary doesn't eat fish $=$ does she?\label{Pos.Pnu.Anyl}
    \ea {[[}$_{\text{CP}}$ Mary$_{i}$ not eat fish] [\textsc{tag} {[}$_{\text{CP}}$ \textsc{q} she$_{i}$ \sout{eat fish}]]]
    \ex $\Diamond \, \lambda w \, . \, \neg eat_w(mary, fish) \, \bullet \, \left\{ \begin{array}{l}
            \lambda w \, . \, eat_w(mary, fish), \\ 
            \lambda w \, . \,\neg eat_w(mary, fish) \\ 
            \end{array} \right\}$ 
    \z
\z
\il{English}

This analysis makes two good predictions about  post-nuclear positive-tag questions. The first good prediction is that it does not say that the anchor proposition is plainly asserted. Instead, we merely predict that the \textit{possibility} of the anchor proposition is asserted. If the anchor proposition was plainly asserted, we would create something like an illocutionary contradiction, where the speaker is both certain about the truth of the anchor proposition (by the norm of assertion) and ignorant about it (by the norm of questioning). Indeed, such a sequence of discourse moves would not be felicitous under normal circumstances (cf. \textit{\#Mary doesn't eat fish. Does she eat fish?}). 

The second good prediction our analysis makes is that  post-nuclear positive-tag questions need not convey any bias. This is because the anchor merely asserts the possibility of the relevant proposition, a very weak statement. Moreover, we analyze the tag as an (elliptical) positive polar question, which is the canonical non-biased polar question form. Since post-nuclear positive-tag questions present a combination of a (negative) possibility and a plain positive polar question, it is unsurprising that such questions need not convey a bias at all. That said, the fact that the negated proposition in the anchor is presented as a possibility may suggest that the speaker is slightly biased in this direction. But because this kind of pragmatic triggering is not directly linked to the semantic properties of the tag, the bias is cancelable. 

In sum, our general analysis of tag questions presented here straightforwardly captures the bias profile (optional, negative, weak) of  post-nuclear positive-tag questions.

\section{Polarity focus}\label{Sect3}

Another crucial piece of our analysis is `polarity focus', or focus applied to a polar element. Therefore, we will start by providing a short background on focus as a general phenomenon, followed by a discussion of its effect when applied to polar elements, especially in post-nuclear negative-tag questions.  

\subsection{Background on focus interpretation}\label{Back.Foc}

A prominent theory of focus, known as `alternative semantics', models focus as a feature $F$ that marks syntactic constituents and generates relevant alternatives (\citealt{Rooth1985, Rooth1992, Rooth1997}; see also \citealt{Jackendoff1972, Hamblin1973, Kratzer1991b, Selkirk1995, Schwarzschild1999, Beaver2008, Buring2019}; a.o.). According to this theory, each linguistic expression has two semantic values: `ordinary' and `focus'. The ordinary semantic value of an expression $\alpha$ is rendered as $\llbracket \alpha \rrbracket^{o}$ and corresponds to its usual denotation. The focus semantic value of $\alpha$ is rendered as $\llbracket \alpha \rrbracket^{f}$ and is always a set, although the nature of its content depends on whether the expression is $F$-marked or not. When $\alpha$ is not $F$-marked, its focus value is the singleton set comprised of the ordinary value of $\alpha$. In contrast, when $\alpha$ is $F$-marked, its focus value is the set comprised of all alternative objects that are of the same semantic type as the ordinary value of $\alpha$. When it comes to complex expressions, the focus semantic value is derived compositionally from the focus values of the immediate constituents, and so focus alternatives project up the tree. Formally, this process is generated via the recursive procedure shown in \xxref{Foc.Lex}{FA}.

\is{}
\ea \label{Foc.Lex} 
    \ea \textit{Non-focused lexical items} \\ $\llbracket \alpha \rrbracket^{f} = \{  \llbracket \alpha \rrbracket^{o} \}$\label{Foc.Lex.NFoc}
    \ex \textit{Focused expressions (lexical or complex)} \\ $ \llbracket  \alpha _{F} \rrbracket^{f} =  \{ \, x \in {D}_{\tau} \, | \, \llbracket \alpha {\rrbracket}^{o} \in {D}_{\tau} \, \} $\label{Foc.Lex.Foc}
    \z
\ex \textit{Pointwise Function Application} \\
		If $\llbracket \alpha \rrbracket^{o} \in {D}_{\sigma \to \tau}$  and  $\llbracket \beta {\rrbracket}^{o} \in {D}_{\sigma}$, then ${ \llbracket  \alpha \,\beta  \rrbracket^{f}} = { \llbracket  \beta \,\alpha \rrbracket^{f}} = \{ \,x(y) \in {D_\tau } \, | \, x \in { \llbracket \alpha \rrbracket^{f}}{\text{ and }}y \in { \llbracket \beta  \rrbracket^{f}}\,\} $.\label{FA}
\z
\il{}

Consider the sentence in \xref{Foc.Mary} as an example. \xref{FA} and \xref{Foc.Lex.NFoc} tell us that the focus semantic value of the predicate \textit{drinks beer} is the singleton set \{drinks beer\}. In turn, \xref{Foc.Lex.Foc} tells us that the focus semantic value of \textit{Mary}$_F$ is the set comprised of all individuals in the domain, e.g. \{Mary, Jane, Susan\}. Combining the two focus values via the compositional rule in \xref{FA}, we obtain the entire range of alternatives, i.e. \{Mary drinks beer, Jane drinks beer, Susan drinks beer\}. This is formalized in \xref{Foc.Mary}.\footnote{Note that \textit{beer} and other non-human objects seem, at least in this example, to be excluded from the focus value of \textit{Mary}. We could capture this by imposing plausibility restrictions on focus alternatives, thus excluding alternatives like \textit{beer drinks beer}.} 

\is{}
\ea MARY drinks beer.\label{Foc.Mary}
    \ea $\text{[}_{\text{TP}} \, \text{Mary}_{F} \, [_{\text{VP}} \, \text{drink beer}]]_{\phi} $ 
    \ex $\llbracket \phi \rrbracket^{o} = \lambda w \, . \, drink_w(mary, beer)$ \\
        $\llbracket \phi \rrbracket^{f} = \left\{ \begin{array}{l}
            \lambda w \, . \, drink_w(mary,beer), \\ 
            \lambda w \, . \,drink_w(jane,beer), \\ 
            \lambda w \, . \,drink_w(susan,beer) \\ 
            \end{array} \right\}  $
    \z
\z
\il{}

The $F$-feature was traditionally thought to lump together two distinct functions of focus, i.e., new information or contrast. However, there is mounting evidence that focus proper is always contrastive and that the new/given information marking is due to an independent discourse strategy \citep{Kratzer2004, Fery2006, Selkirk2008, Katz2011, Beaver2011, Rochemont2013, Buring2019, Kratzer2020, Goodhue2022}. We will adopt this view without discussion and, from here on out, always view focus as signaling a contrast.

Focus marks a phrase whose referent is juxtaposed with the referent of a similar phrase. For example, in \xref{Foc.Contr.Fel}, \textit{Mary} is contrasted with \textit{Jane} and the sentence is felicitous, while in \xref{Foc.Contr.Infel} \textit{beer} finds no appropriate contrasting phrase and so the sentence is odd. 
\is{}
\ea\label{Foc.Contr.Eg}
    \ea Jane drinks beer and MARY drinks beer (too).\label{Foc.Contr.Fel}
    \ex \#Jane drinks beer and Mary drinks BEER (too).\label{Foc.Contr.Infel}
    \z
\z
\il{}
More formally, in order for a contrast to be felicitous, there must be an antecedent that is among the focus alternatives of the focus domain but is different from the ordinary meaning of that domain. This relationship is outlined in \xref{Foc.Contr.Req}, where $C$ is the antecedent and the presuppositional `squiggle' operator $\sim$ marks the focus domain $\phi$.  
\is{}
\ea \textit{Contrasting elements}  (cf. \citealt[90]{Rooth1992}) \\
    $\phi \sim C$ is felicitous only if $C \in  \llbracket \phi \rrbracket ^f$ and $C \ne \llbracket \phi  \rrbracket ^o$.\label{Foc.Contr.Req} 
\z
\il{}
Applied to the second conjunct in \xref{Foc.Contr.Fel}, an appropriate antecedent is presented in \xref{Foc.Contr.Antec}. This antecedent is a member of the focus value of the second conjunct and also differs from its ordinary value, as shown in \xref{Contr.Req.Fel}. The constraint in \xref{Foc.Contr.Req} then correctly predicts that \xref{Foc.Contr.Fel} is felicitous. However, the second conjunct \xref{Foc.Contr.Infel} is expected to be out, as can be seen in \xref{Contr.Req.Infel}. In this latter case, the first condition in \xref{Foc.Contr.Req} is violated. That is, \xref{Foc.Contr.Antec} is not a member of \xref{Contr.Req.Infel}'s focus value.

\is{}
\ea $C = \llbracket \text{Jane drinks beer} \rrbracket ^ o = \lambda w \, . \, drink_w(jane, beer)$\label{Foc.Contr.Antec}
\z
\il{}
\is{}
\ea MARY drinks beer.\label{Contr.Req.Fel}
    \ea ${[}_{\text{TP}} \, \text{Mary}_{F} \, [_{\text{VP}} \, \text{drinks beer}]]_{\phi} \sim C$ 
    \ex $\llbracket \phi \rrbracket^{o} = \lambda w \, . \, drink_w(mary, beer)$ \\
        $\llbracket \phi \rrbracket^{f} = \left\{ \begin{array}{l}
            \lambda w \, . \, drink_w(mary,beer), \\ 
            \lambda w \, . \,drink_w(jane,beer), \\ 
            \lambda w \, . \,drink_w(susan,beer) \\ 
            \end{array} \right\}  $
    \ex $C \in  \llbracket \phi \rrbracket ^f$ \cmark, \, $C \ne \llbracket \phi  \rrbracket ^o$ \cmark 
    \z
\z
\il{}
\is{}
\ea Mary drinks BEER.\label{Contr.Req.Infel}
    \ea ${[}_{\text{TP}} \, \text{Mary} \, [_{\text{VP}} \, \text{drinks beer}_{F}]]_{\phi} \sim C$
    \ex $\llbracket \phi \rrbracket^{o} = \lambda w \, . \, drink_w(mary, beer)$ \\
        $\llbracket \phi \rrbracket^{f} = \left\{ \begin{array}{l}
            \lambda w \, . \, drink_w(mary,beer), \\ 
            \lambda w \, . \,drink_w(mary,wine) \\ 
            \end{array} \right\}  $
    \ex $C \in  \llbracket \phi \rrbracket ^f$ \xmark, \, $C \ne \llbracket \phi  \rrbracket ^o$ \cmark 
    \z
\z
\il{}

Now that we have shown how the phenomenon of (contrastive) focus works generally, we will consider the effects of its application to polar elements. 

\subsection{Polarity focus, answer salience, and question bias}\label{Pol.Foc}
Just like any other phrase, focus can mark an element that conveys the polarity of a sentence, a phenomenon that is often called `polarity focus'.\footnote{The label `polarity focus' is a bit of a misnomer, as it seems to infer that this is some special type of focus. In reality, it is merely run-of-the-mill focus targeted at a polar element. That is, whatever effects polarity focus is claimed to exert should be derived from the semantics of the polar element plus the general theory of focus.}  The individual items that make up the set of polar elements (the potential carriers of polarity focus) is somewhat controversial. Here we take \textit{not} and \textit{really} as two relatively uncontroversial choices for a negative and a positive polar element (cf. \citealt{romero2004}). Other candidate positive elements include \textit{totally}, \textit{so}, and \textit{definitely}.\footnote{Notice that, in addition to their polar use, these elements also have a degree modifier use, as in \textit{really tired}, \textit{totally full}, or \textit{so happy} \citep{romero2004, Beltrama2018}. This is why in order to block the degree modifier use we will avoid sentences with gradable predicates in them altogether. See \citet{Bill2022} for a proposal of how these two uses can be derived from the same basic semantic content.} Crucially, we do not take an accented finite auxiliary to necessarily express polarity focus. In \sectref{Sect4}, we will argue that such forms spell out a covert \textsc{verum} operator whose interpretational effects differ from these of polarity focus. We now discuss the semantics of \textit{not} and \textit{really}, along similar lines to proposals put forward in \citet{Wilder2013}, \citet{Samko2016}, \citet{Goodhue2018a}, and \citet{Gutzmann2020}.

We should note that, as outlined in \citet{Bill2022}, there are good reasons to posit analyses of \textit{really} and certain forms of negation (i.e., `high' or `light' negation), which model them as degree adverbs that are capable of modifying the degree of a speaker's commitment to the prejacent proposition. For simplicity, we will put aside this aspect of their meaning and treat their ordinary semantics as straightforwardly conveying the polarity of the prejacent proposition.  

Starting with negation, we take its ordinary semantics to denote set-theoretic complementation. Its focus semantics has a bit more going on. When $F$-marked, \textit{not} denotes the set consisting of its ordinary value and its positive counterpart. The formal definitions are provided in \xref{Neg.Sem}.   

\is{}
\ea\label{Neg.Sem} 
    \ea $ \llbracket \text{not}_{F} \rrbracket ^o = \llbracket \text{not} \rrbracket ^o =   \lambda p. \overline p $ 
    \ex $ \llbracket \text{not} _{F} \rrbracket ^f = \{ \lambda p.p, \, \lambda p. \overline p \}$
    \z
\z
\il{}

As for \textit{really}, with the simplification noted above, its ordinary meaning can be modeled simply as the identity function on propositions, rendering its plain use redundant and thus infelicitous. Following up on this reasoning, we assume that \textit{really} is inherently $F$-marked, as previously argued in \citet{romero2004}. Its focus value is the same as that of negation and denotes the positive and the negative alternative. This is spelled out in \xref{Rly.Sem}. 

\is{}
\ea\label{Rly.Sem}
    \ea $  \llbracket \text{really} _{F} \rrbracket ^o = \lambda p.p$
    \ex  $ \llbracket \text{really} _{F} \rrbracket ^f = \{ \lambda p.p, \, \lambda p. \overline p \}$
    \z
\z
\il{}

\noindent
In other words, we treat \textit{really} and \textit{not} as polar opposites that give rise to the same set of focus alternatives.\footnote{This does not mean that \textit{really} and \textit{not} occupy the same syntactic slot or that they are in complementary distribution. As it turns out, these two elements can co-occur in the same sentence, cf. \textit{Oliver REALLY isn't from Australia}. In such cases only \textit{really} is obligatorily focus-marked and the utterance contrasts with the positive alternative \textit{Oliver is from Australia}.}

As outlined in \xref{Rly.FocV}, the focus semantic value of a declarative sentence with polarity focus amounts to the ordinary \citet{Hamblin1973}-style denotation of the respective polar question. Such a sentence will typically be used in order to assert the positive prejacent (i.e., \textit{Alex got married}), thus contrasting it with the negative alternative (i.e., \textit{Alex didn't get married}). 

\is{}
\ea Alex REALLY got married.\label{Rly.FocV}  
    \ea  {[}$_{\text{PolP}}$ really$_{F}$ [$_{\text{TP}}$ Alex got married]] 
    \ex $\llbracket [_{\text{TP}}$ Alex got married]$\rrbracket ^ f = \{\, \lambda w \, . \, get.married_w(alex) \, \}$ \\
        $ \llbracket \text{really} _{F} \rrbracket ^f = \{ \lambda p.p, \, \lambda p. \overline p \}$ \\
        \mbox{$\llbracket [_{\text{PolP}} \, \text{really}_{F} \, [_{\text{TP}}$ Alex got married]]$\rrbracket^{f} = \left\{ \begin{array}{l}
            \lambda w \, . \, get.married_w(alex), \\ 
            \lambda w \, . \, \neg get.married_w(alex) \\ 
            \end{array} \right\}  $}
    \z
\z
\il{}

What about cases where \textit{really} occurs in a polar question, as in \xref{RlyQ.Anlys}? Assuming the analysis in \xref{Rly.Sem}, \textit{really} makes no extra contribution to the ordinary semantics of this question. However, it does invoke as focus alternatives the prejacent proposition and its complement. Therefore, a polar question with \textit{really} receives the analysis shown in \xref{RlyQ.Anlys}, where the only possible focus antecedent is the negative polar alternative in \xref{RlyQ.Anlys.C}. Since this alternative entails (in fact, is equivalent with) the negative cell of the question partition, it naturally raises the salience of that cell. We propose that it is for this reason that the negative speaker bias is generated. Also, since this kind of raised salience indicates a mere preference on the part of the speaker, by default this bias is expected to be weak.\footnote{However, application of the Follow-up Test and the Weighted Coin Test suggests that polar questions with \textit{really} convey a strong bias. This can be derived by proposing a more realistic semantics for \textit{really}, according to which this operator raises the level of commitment to the prejacent proposition and thus strengthens the bias (see \citealt{Bill2022}). } As for the bias being obligatory, this follows from the fact that the utterance would be infelicitous unless, as required by \xref{Foc.Contr.Req}, the felicity condition (the contrastive focus interpretation) of the squiggle operator is met. 

\is{}
\ea Does Susan REALLY do weightlifting?\label{RlyQ.Anlys}
    \ea {[}$_{\text{CP}}$ \textsc{q} [$_{\text{PolP}}$ really$_{F}$ [$_{\text{TP}}$ Susan do weightlifting]]$_{\phi}$ $\sim C$]\label{RlyQ.Anlys.LF}
    
    \ex $C = \lambda w \, . \,\neg do_w(susan, weightlifting)$\label{RlyQ.Anlys.C}
    
    \ex $\llbracket \text{PolP} \rrbracket^{o}  = \lambda w \, . \, do_w(susan, weightlifting) = \llbracket \phi \rrbracket^{o}$ \\
        $\llbracket \textsc{q} \rrbracket^{o} = \lambda p \, . \, \{p, \overline{p}\}$ \\
        $\llbracket \text{CP} \rrbracket^{o} = \left\{ \begin{array}{l}
            \lambda w \, . \, do_w(susan,weightlifting), \\ 
            \lambda w \, . \,\neg do_w(susan,weightlifting) \\ 
            \end{array} \right\}$\label{RlyQ.Anyls.Q}
            
    \ex $\llbracket \text{TP} \rrbracket^{f} = \{ \lambda w \, . \, do_w(susan, weightlifting) \}$ \\
        $ \llbracket \text{really} _{F} \rrbracket ^f = \{ \lambda p.p, \, \lambda p. \overline p \}$ \\
        $\llbracket \text{PolP} \rrbracket^{f} = \left\{ \begin{array}{l}
            \lambda w \, . \, do_w(susan,weightlifting), \\ 
            \lambda w \, . \,\neg do_w(susan,weightlifting) \\ 
            \end{array} \right\}  = \llbracket \phi \rrbracket^{f} $
        
    \ex $C \in  \llbracket \phi \rrbracket ^f$ \cmark, \, $C \ne \llbracket \phi  \rrbracket ^o$ \cmark
    \z
\z
\il{}

We propose a very similar analysis for high negation questions like \xref{HNQ.Anlys}, namely the structure in \xref{HNQ.LF}. The main difference is that, in contrast with the question with \textit{really}, focus in high negation questions is manifested by the high structural position of negation rather than by a pitch accent.\footnote{Note that when a sentence signals focus structurally, typically a pitch accent is also placed on the focused element (e.g., in cleft constructions). We argue that this does not happen with high negation questions since it would also produce a verum accent, which -- as we will argue in \sectref{Sect4} -- conveys the presence of a \textsc{verum} operator. Therefore, signaling polarity focus structurally and without a pitch accent allows high negation questions to convey that the underlying structure contains polarity focus but not \textsc{verum}.} Following \citet{rizzi1997}, we call this high structural position FocP.

\is{}
\ea Doesn't Laura live in Italy?\label{HNQ.Anlys}
    \ea {[}$_{\text{CP}}$ \textsc{q} [$_{\text{FocP}}$ not$_{F} \,  $ [$_{\text{TP}}$ Laura live in Italy]]$_{\phi} \sim C$]\label{HNQ.LF}
    \ex $C = \lambda w \, . \, live.in_w(laura,italy)$\label{HNQQ.Anlys.C}
    \ex $\llbracket \text{FocP} \rrbracket^{o} = \lambda w \, . \, \neg                 
        live.in_w(laura,italy) = \llbracket \phi \rrbracket^{o}$ 
    \ex $\llbracket \text{FocP} \rrbracket^{f} = \left\{ \begin{array}{l}
            \lambda w \, . \, live.in_w(laura,italy), \\ 
            \lambda w \, . \,\neg live.in_w(laura,italy) \\ 
            \end{array} \right\} = \llbracket \phi \rrbracket^{f} $
    \ex $C \in  \llbracket \phi \rrbracket ^f$ \cmark, \, $C \ne \llbracket \phi  \rrbracket ^o$ \cmark
    \z
\z
\il{}

\noindent
The derivation of the bias profile is virtually identical to that for questions with \textit{really}, except that in this case the scope of the squiggle operator is the negative focus alternative. That is, given the nature of polarity focus as generating just two polar alternatives, the only possible antecedent that contrasts with the negative focus alternative is the positive alternative in \xref{HNQQ.Anlys.C}. Since this alternative entails (really, matches exactly) the positive cell of the question partition, the salience of that cell is raised and we end up with the intuition of a positive bias. And again, since this salience mechanism indicates a mere preference, the generated bias is weak. Moreover, as with questions with \textit{really}, the bias is obligatory because of the presupposition of the squiggle operator.  

In sum, polarity focus in polar questions raises the salience of one of the answers and leads to the generation of a bias that is weak (by default), obligatory, and of the opposite polarity to the focus domain. We will now show that the same line of explanation applies to tag questions, specifically to post-nuclear negative-tag questions.

\subsection{Post-nuclear negative-tag questions}
Post-nuclear negative-tag questions, like \xref{Neg.PNu}, are composed of a positive anchor and a negative tag. 
\is{}
\ea You like football $=$ don't you?\label{Neg.PNu}
\z
\il{English}

To begin, we will identify its bias profile. Starting with optionality, as shown in \xref{Neg.PNu.BAC}, such questions are infelicitous when combined with the \textit{by any chance} discourse marker. This indicates that a neutral reading of this question is not possible. That is,  the bias conveyed by  post-nuclear negative-tag questions is obligatory.
\is{}
\ea \#By any chance, you speak French $=$ don't you?\label{Neg.PNu.BAC}
\z
\il{English}

Next, we will explore the strength and optionality settings of this bias, starting with the Follow-up Test. As shown by the felicity of the weak but not the strong epistemic follow-up in \xref{Neg.PNu.TIS}, the bias conveyed by post-nuclear negative-tag questions is positive and weak.
\is{}
\ea Mary is a vegan = isn't she? That is to say, ...\label{Neg.PNu.TIS}
\ea[]{I suspected she was.}
\ex[\#]{I was sure she was.}
\z
\z
\il{English}
The Weighted Coin Test in \xref{Neg.PNu.WC} provides further support that the bias is weak. That is, as shown in \xref{Neg.Pnu.WC.99}, such tag questions are degraded when the speaker's prior belief that the coin would land on tails is very high. The cause of this infelicity as coming from the strength of the bias is confirmed by the increase in felicity that occurs when the chance of the coin landing on tails is decreased, as shown in \xref{Neg.Pnu.WC.70}.
\is{}
\ea Mary and John are playing with a coin that they both know is designed so that it lands on tails \textit{N\%} of the time. Mary tosses the coin and it lands on the other side of John's legs, such that only John can see the result. John looks confused, causing Mary to say:\label{Neg.PNu.WC}
    \ea 99\%: ?It landed on tails $=$ didn't it?\label{Neg.Pnu.WC.99}
    \ex 70\%: It landed on tails $=$ didn't it?\label{Neg.Pnu.WC.70}
    \z
\z
\il{English}

In sum, the bias profile of  post-nuclear negative-tag questions is obligatory, positive, and weak. We will now show how this profile can be accounted for using the elements of our analysis introduced so far. 

The structure that we assume for  post-nuclear negative-tag questions is presented in \xref{Neg.PNu.Anyls}. We have our general tag question shape, here consisting of a declarative anchor and an elliptical high negation question. Recall from \sectref{Sec.PNu.Pos} that a tag on its own does not necessarily convey bias, as displayed by the fact that post-nuclear positive-tag questions are only optionally biased. We argue, therefore, that the obligatory nature of the bias in  post-nuclear negative-tag questions is coming from the tag, an (elliptical) high negation question. And as we presented in \sectref{Pol.Foc}, the bias associated with high negation questions is weak, positive and obligatory, exactly the same as the bias associated with post-nuclear negative-tag questions. Therefore, we propose that the bias in such questions is generated in precisely the same manner. That is, it arises because the only possible antecedent that contrasts with the negative focus alternative is the positive alternative in \xref{Neg.PNu.C}. And since this alternative entails the positive cell of the question partition denoted by the tag, the salience of that cell is raised and we get the intuition of a positive bias. And again, this salience mechanism indicates a mere preference, so the generated bias is weak. Moreover, the fact that this focus is derived from the necessary structure of the tag means that the bias is obligatory. This is based on the same explanatory mechanism as with all high negation questions.
\is{}
\ea Phillip rides to work $=$ doesn't he?\label{Neg.PNu.Anyls}
    \ea {[[}$_{\text{CP}}$ Phillip$_i$ ride to work] [\textsc{tag} {[}$_{\text{CP}}$ \textsc{q} [$_{\text{FocP}}$ not$_{F} \, [_{\text{TP}}$ he$_i$ \sout{ride to work}]]$_{\phi} \sim C$]]]\label{Neg.PNu.LF}
    \ex $\Diamond \, \lambda w \, . \, ride.to_w(phillip,work) \, \bullet \, 
            \left\{ \begin{array}{l}
            \lambda w \, . \, ride.to_w(phillip,work), \\ 
            \lambda w \, . \,\neg ride.to_w(phillip,work) \\ 
            \end{array} \right\}$
    \ex $C = \lambda w \, . \, ride.to_w(phillip,work)$\label{Neg.PNu.C}
    \ex $\llbracket \phi \rrbracket^{o} = \lambda w \, . \,\neg ride.to_w(phillip,work)$\label{Neg.PNu.phi}
    \ex $\llbracket \phi \rrbracket^{f} = \left\{ \begin{array}{l}
            \lambda w \, . \, ride.to_w(phillip,work), \\ 
            \lambda w \, . \,\neg ride.to_w(phillip,work) \\ 
            \end{array} \right\}  $
    \ex $C \in  \llbracket \phi \rrbracket ^f$ \cmark, \, $C \ne \llbracket \phi  \rrbracket ^o$ \cmark
    \z
\z
\il{}

Next, we will introduce another important element of our analysis, the \textsc{verum} operator. We will follow by an explanation of how, with this additional element, we can capture the bias conveyed by nuclear tag questions.

\section{Verum}\label{Sect4}
The phenomenon of `verum accent' involves a pitch accent on the finite auxiliary and -- in the case of a declarative sentence -- has the effect of emphasizing the truth of the expressed proposition (\citealt{Hohle1992}). Thus, by uttering \textit{Oliver IS from Australia}, the speaker stresses their belief that it is indeed true that Oliver is from Australia. This section presents the core data on verum accent and our account of it, and then discusses the role of verum accent in deriving the bias profiles of nuclear tag questions.

\subsection{Core data on verum accent}\label{VrmData}
There are certain restrictions on the occurrence of verum accent that any account of it should capture. As \citet{Gutzmann2020} point out, verum accent is felicitous in two kinds of contexts: `contradictory' and `affirmative'. Contradictory contexts are more common and arise when there is some dispute about whether the prejacent is true or false, as  in \xref{Vrm.Contr.Eg}.  
\is{}
\ea\label{Vrm.Contr.Eg}
    \begin{xlist}
    \exi{A:} Oliver is not from Australia.  \hfill (contradictory context)
	\exi{B:} He IS from Australia.   
	\end{xlist}
\z
\il{}

In turn, affirmative contexts come about when the speaker and the addressee agree on the prejacent. We note that this use typically involves `extreme' adjectives, like \textit{amazing}, \textit{awesome}, \textit{}\textit{excellent}, etc. \citep{Cruse1986, Paradis2001, Rett2008, Morzycki2012}. An example of such a context is presented in \xref{Vrm.Aff.Eg}.\footnote{In an affirmative context, verum accent is also possible  with regular predicates, although the result is once again an ``extreme'' interpretation. For example, if \textit{It IS raining} has been uttered as a reaction to \textit{It's raining}, it would suggest a heavy rain and not just a light drizzle (cf. \citealt{Umbach2011} on extreme verbs).}  
\is{}
\ea \textit{After a colloquium talk:}     \hfill (affirmative context)\label{Vrm.Aff.Eg}
    \begin{xlist}
    \exi{A:} Paula is an amazing linguist.           
	\exi{B:} She IS an amazing linguist. 
    \end{xlist}
\z
\il{}

Crucially, a verum accent is not possible in neutral contexts, e.g., when a new issue has been raised by a neutral polar question \citep{Wilder2013, Samko2016, Goodhue2018a, Gutzmann2020}. This is illustrated in \xref{Vrm.Neu.Eg}. 
\is{}
\ea \textit{Out of the blue:}         \hfill (neutral context)\label{Vrm.Neu.Eg}
    \begin{xlist}
    \exi{A:} Is it raining outside? 
	\exi{B:} \#It IS raining.   
    \end{xlist}
\z
\il{}
That is, in order for a verum-marked declarative to be felicitous, the issue must have already been discussed in prior discourse, as in \xxref{Vrm.Contr.Eg}{Vrm.Aff.Eg} above.

Just like in declaratives, when a verum accent features in polar interrogatives, we typically get the intuition of some kind of bias \citep{romero2004}. For example, the question in \xref{Vrm.Q} seems to convey a negative bias.
\is{}
\ea IS Oliver from Australia?\label{Vrm.Q} \\
    $\rightsquigarrow$ \textit{The speaker doubts that Oliver is from Australia.} 
\z
\il{}

Importantly though, the bias associated with verum accent in polar interrogatives is optional, as it can disappear in certain contexts. One such context is \xref{Vrm.Q.Opt1}, where evidence for and against the prejacent has been provided by other parties and the speaker herself does not take a stand. The examples in \xref{Vrm.Q.Opt2} and \xref{Vrm.Q.Opt3} are drawn from the literature and make the same point.    

\is{}
\ea DID Mary join the team? Because some say she did, others say she didn't. \\
	$\not \rightsquigarrow$ \textit{The speaker doubts that Mary joined the team.}\label{Vrm.Q.Opt1} 

\ex \label{Vrm.Q.Opt2}
    \begin{xlist}
	    \exi{A:} Did Karl kick the dog?                 \hfill (\citealt{Gutzmann2020}: 41)
	    \exi{B:} No, Karl didn't kick the dog.
	    \exi{C:} No, he DID kick the dog.
	    \exi{A:} Which is it? DID he kick the dog?  \\
	$\not \rightsquigarrow$ \textit{The speaker doubts that Karl kicked the dog.}
	\end{xlist} 
	
\ex \textit{B wants to know whether Jill will be at a meeting for members of a club. But B lacks an opinion about whether Jill is a member.}\label{Vrm.Q.Opt3} \hfill (\citealt{Goodhue2019}: 473)
    \begin{xlist}
	    \exi{B:} Will Jill be at the meeting?                 
    	\exi{A:} If she's a member, she will.
	    \exi{B:} IS she a member?   \\
	$\not \rightsquigarrow$ \textit{The speaker doubts that Jill is a member. }
	\end{xlist}
\z
\il{}
The By Any Chance Test gives rise to the same result, as shown in \xref{Vrm.Q.Opt4}, providing further evidence that the bias conveyed by a verum accent in polar questions is optional.   
\is{}
\ea DID Mary join the team, by any chance?\label{Vrm.Q.Opt4}
\z
\il{}

Though optional, notice that the bias triggered by verum accent is strong. This is attested by the Follow-up Test, as shown in \xref{Vrm.Q.Str}.
\is{}
\ea IS Oliver from Australia? That is to say, ...\label{Vrm.Q.Str} 
    \begin{xlist}
        \ex ?I suspected he wasn't.
        \ex I was certain he wasn't.
    \end{xlist}
\z
\il{}

We will now present our analysis of verum accent and show how it is able to capture the effects of this accent in declarative and polar interrogative sentences.

\subsection{\textsc{Verum} as a covert operator}\label{Vrm.CovOp}
There are two main approaches to analyzing verum accent. The ``focus approach'' posits that verum accent involves focus on a polarity head and manifests itself as a pitch accent on some element in the left periphery of the sentence \citep{Laka1990, Wilder2013, Samko2016, Goodhue2018a}. This approach analyzes verum accent in essentially the same manner as we have polarity focus in \sectref{Pol.Foc}, with focus being placed on a syntactically realized polarity head. In turn, the `operator approach' contends that a verum accent signals the presence of a covert operator with certain conversational properties \citep{romero2004, Repp2012, Goodhue2019, Gutzmann2020}. For reasons that we explore in detail in \citet{Bill2021}, we favor an explanation that is more in line with the latter approach.\footnote{On the empirical side, the strongest argument comes from \citet{Gutzmann2020}, who argue that \textsc{verum} is overtly lexicalized in various typologically unrelated languages.}

We propose that verum accent manifests the presence of a purely presuppositional \textsc{verum} operator that requires an epistemic conflict regarding the prejacent proposition in the given context. This is stated in \xref{Vrm.Analys}.   
\is{}
\ea $\llbracket\textsc{verum}\rrbracket _c^o (p) = p$,  provided that there is conflicting evidence about $p$ in $c$\label{Vrm.Analys}
\z
\il{}
We assume that conflicting evidence about $p$ involves two mutually exclusive pieces of evidence: a piece of evidence for $p$ and a piece of evidence against $p$. Notice that contrasting evidence alone does not suffice, as such evidence need not produce an epistemic conflict and \textsc{verum} may not be licensed. Thus, if the positive and the negative pieces of evidence are presented as mere possibilities, a verum-marked sentence is degraded, as shown in \xref{Confl.Ev.Weak}.\footnote{Notice that C's utterance in \xref{Confl.Ev.Weak} is not entirely out. The reason, we suggest, is that strong positive evidence can be accommodated from C's (verum-marked) assertion, thus deriving the required conflict with B's utterance.}
\is{}	
\ea\label{Confl.Ev.Weak}
    \begin{xlist}
	\exi{A:} It's possible that Oliver is from Australia. 
	\exi{B:} It's also possible that he is from New Zealand (though).
	\exi{C:} ?No, he IS from Australia.
	\end{xlist}
\z
\il{}
Moreover, note that the strength of the two pieces of evidence does not need to be equal. For example, as shown in \xref{Confl.Ev.Unbal}, it is possible for one side of the evidence to be strong and the other weak, provided the outcome is that they conflict.  
\is{}
\ea\label{Confl.Ev.Unbal}
    \begin{xlist}
        \exi{A:} Oliver is from Australia. 
        \exi{B:} I think he might be from New Zealand, actually.
        \exi{C:} No, he IS from Australia. 
    \end{xlist}
\z
\il{}
In sum, we claim that verum accent indicates the presence of a \textsc{verum} operator which contributes no at-issue content but rather a conflicting evidence presupposition. We will now show how this simple semantics can account for the distribution of \textsc{verum} in declaratives and polar questions. 

Starting with contradiction contexts, recall from example \xref{Vrm.Contr.Eg}, repeated below as \xref{Vrm.Contr.Eg2}, that the prototypical use of verum accent is as a denial, targeting negative utterances. 
\is{}
\ea\label{Vrm.Contr.Eg2} 
    \begin{xlist}
	\exi{A:} Oliver is not from Australia. 
	\exi{B:} No, he IS from Australia.
	\end{xlist}
\z
\il{}
In this case, the conflicting evidence presupposition conveyed by \textsc{verum} is satisfied as follows: the negative evidence comes from the previous utterance, while the positive evidence has two possible sources. One option is that this evidence may be due to a prior positive utterance that A's negative utterance is itself responding to. After all, one would generally not utter a negative sentence if the positive alternative had not been uttered or raised in some way. Even in the absence of such prior utterance, the conflicting evidence presupposition can be accommodated from the fact that the verum-marked sentence is being asserted by B and thus it is strongly supported by the evidence. Either way, the conflicting evidence presupposition is satisfied and \textsc{verum} is licensed.    
	
As for affirmation contexts, we noted earlier that such uses typically involve extreme adjectives (or, more generally, extreme readings of predicates). The example in \xref{Vrm.Aff.Eg2} is a repetition of \xref{Vrm.Aff.Eg} from earlier. 
\is{}
\ea \textit{After a colloquium talk:} \label{Vrm.Aff.Eg2}
    \begin{xlist}
	    \exi{A:} Paula is an amazing linguist.           
	    \exi{B:} She IS an amazing linguist.
    \end{xlist}
\z
\il{}
\citet{Morzycki2012} proposes that extreme adjectives make use of the far end of the scale associated with the respective regular adjective. Following up on this idea, we can say that in \xref{Vrm.Aff.Eg2} the extreme adjective \textit{amazing} is parasitic on the regular adjective \textit{good}, as it refers to extreme degrees of goodness. This derives the required epistemic conflict as follows. Let us assume that $\langle good, amazing \rangle$ forms a Horn-scale, such that a sentence with \textit{amazing} naturally invokes the respective alternative with \textit{good}. In \xref{Vrm.Aff.Eg2}, A's initial utterance of \textit{Paula is an amazing linguist} will invoke the weaker alternative \textit{Paula is a good linguist}. Now, if we allow that this latter alternative be strengthened to \textit{Paula is a good but not an amazing linguist} by some standard scalar mechanism, we get an alternative that contradicts B's verum-marked utterance \textit{She IS an amazing linguist}. In other words, the use of an extreme adjective creates an implicit contraction within the same scale by splitting it into two non-overlapping regions. As a result, the conflicting evidence presupposition is met and \textsc{verum} is licensed once again.    
	
Finally, our semantics for \textsc{verum} straightforwardly derives the observation that verum accent is out in neutral contexts. That is, since such contexts lack conflicting evidence about the prejacent, the presupposition of \textsc{verum} is not satisfied and so a verum-marked sentence is out. 

As for the effects of verum accent in polar questions like \xref{Vrm.Q.Opt}, the bias that is generated is strong, negative, and optional, as already established in \sectref{VrmData}. We will derive this profile by appealing to the semantics we have presented for \textsc{verum} combined with the effects of polarity focus we laid out in \sectref{Pol.Foc}. Specifically, we will claim that the polarity is dictated by polarity focus, whereas the strength and optionality are contributions of \textsc{verum}.
\is{}
\ea DID Mary join the team?\label{Vrm.Q.Opt} \\
	$\rightsquigarrow$ \textit{The speaker doubts that Mary joined the team.}
\z
\il{}
	
In order to derive the optionality of the bias, we propose that verum-marked polar interrogatives may be associated with two homophonous Logical Forms, one with and another without focus marking. While both forms contain \textsc{verum} and thus require conflicting evidence about the prejacent, only the variant in which \textsc{verum} is $F$-marked conveys a bias. That is, we propose that \xref{Vrm.Q.Opt} is ambiguous between \xref{Vrm.LF1} and \xref{Vrm.LF2}.
\is{}
\ea 
	\ea {[}$_{\text{CP}}$ \textsc{q} [$_{\text{PolP}}$ \textsc{verum} [$_{\text{TP}}$ Mary join the team{]]]} \hfill (unbiased)\label{Vrm.LF1}
	\ex {[}$_{\text{CP}}$ \textsc{q} [$_{\text{PolP}}$ \textsc{verum}$_F$ [$_{\text{TP}}$ Mary join the team]]$_{\phi} \sim C$] \hfill (biased)\label{Vrm.LF2}
	\z
\z
\il{}
The ordinary meaning of \xref{Vrm.LF1} is the usual question partition that is comprised of the prejacent proposition and its complement. Since this structure also contains \textsc{verum}, it generates the presupposition of conflicting evidence about the prejacent. This is illustrated in \xref{Vrm.LF1.Der}.
\is{}
\ea \label{Vrm.LF1.Der}
	\ea {[}$_{\text{CP}}$ \textsc{q} [$_{\text{PolP}}$ \textsc{verum} [$_{\text{TP}}$ Mary join the team{]]]}
	\ex $\llbracket \text{CP} \rrbracket^{o}_c = \left\{ \begin{array}{l}
		\lambda w \, . \, join_w(mary,team), \\ 
		\lambda w \, . \,\neg join_w(mary,team) \\ 
	\end{array} \right\}$, \\ 
	provided that there is conflicting evidence about $\lambda w \, . \, join_w(mary,team)$ in $c$
	\z
\z
\il{}
Notably, no part of the evidence needs to originate from the speaker and it can stem from other contextual sources entirely. This accounts for the optionality of the bias associated with verum-marked polar questions.
	
In turn, \xref{Vrm.LF2} gives rise to the same question denotation and conflicting evidence presupposition. However, in this case \textsc{verum} is focus-marked and thus requires an antecedent. Given the contrastive focus interpretation, the only antecedent that meets the squiggle-imposed condition in \xref{Foc.Contr.Req} is the negative question alternative, as shown in \xref{Vrm.LF2.Der}.
\is{}	
\ea \label{Vrm.LF2.Der}
	\ea {[}$_{\text{CP}}$ \textsc{q} [$_{\text{PolP}}$ \textsc{verum}$_F$ [$_{\text{TP}}$ Mary join the team]]$_{\phi} \sim C$] 
	\ex $C = \lambda w \, . \,\neg join_w(mary,team)$ 
	\ex $\llbracket \phi \rrbracket^{o}_{c} = \lambda w \, . \, join_w(mary,team) $, \\
	provided that there is conflicting evidence about $\lambda w \, . \, join_w(mary,team)$ in $c$  
	\ex $\llbracket \phi \rrbracket^{f}_c = \llbracket \text{CP} \rrbracket^{o}_c = \left\{ \begin{array}{l}
		\lambda w \, . \, join_w(mary,team), 	\\ 
		\lambda w \, . \,\neg join_w(mary,team) \\ 
	\end{array} \right\} $, 				\\ 
	provided that there is conflicting evidence about $\lambda w \, . \, join_w(mary,team)$ in $c$
	\ex $C \in  \llbracket \phi \rrbracket ^f$ \cmark, \, $C \ne \llbracket \phi  \rrbracket ^o$ \cmark
	\z
\z
\il{}
The presence of polarity focus in verum-marked polar interrogatives derives the negative speaker bias in the same manner as the other questions with polarity focus. That is, the negative focus antecedent makes salient the negative cell of the question partition, resulting in the generation of negative bias.  
	
Taking stock, we have derived both the optionality and the negative direction of the speaker bias in polar interrogatives with \textsc{verum}. The optionality follows from the assumption that \textsc{verum}, qua polar operator, may (though need not) carry focus marking. The negative direction is due to the fact that when such marking is present, the contrasting antecedent will be resolved to the negative focus alternative. The final element of the bias conveyed by questions with a verum accent is that it is strong. 

We can account for the strength of the bias conveyed by verum-accented questions as follows. Focus marking on a polar element (e.g., negation or \textit{really}) only conveys a preference for one of the question partition cells, and the generated bias is expected to be weak by default. However, \textsc{verum} also introduces the presupposition that the context is conflicted about the prejacent, so the bias gets a boost. That is, in a conflicted context, conventionally the level of certainty required to make a contribution is higher than in a neutral context. For this reason, if biased at all, polar interrogatives with \textsc{verum} are strongly biased.\footnote{One might wonder how our analysis would go accounting for questions containing both a verum accent and a focused polar element, like \xxref{VrmRlyQ}{VrmHNQ}.    
\is{}
\ea DO vegetarians REALLY eat fish?\label{VrmRlyQ}
\ex DO vegetarians NOT eat fish?\label{VrmNQ}
\ex DON'T vegetarians eat fish?\label{VrmHNQ}
\z
\il{}
Basically, such questions would be analyzed as having a structure that contains both \textsc{verum} and polarity focus -- this time not on \textsc{verum}, but on the other polar element, i.e., \textit{really} or \textit{not}. By applying a parallel reasoning to that above, we correctly predict that the resulting speaker biases are strong, obligatory, and of the opposite polarity to that of the focus domain.}

\subsection{Nuclear tag questions}
Nuclear positive-tag/negative-tag questions, like \xref{Neg.Nu} and \xref{Pos.Nu}, are composed of a negative/positive anchor and an opposite polarity tag. Crucially, the prosodic contour of these questions is such that there is a clear break after the anchor, and there is a pitch accent on the auxiliary verb.
\is{}
\ea You haven't watched Star Wars || HAVE you?\label{Pos.Nu}
\ex You have watched Star Wars || HAVEN'T you?\label{Neg.Nu}
\z
\il{English}

Let us identify the bias profiles of these tag questions. Starting with optionality, as shown in \xref{Pos.Nu.BAC} and \xref{Neg.Nu.BAC}, nuclear tag questions of both polarities are infelicitous when combined with \textit{by any chance}. This indicates that a neutral reading of these questions is not possible. That is, the bias conveyed by both positive and negative nuclear tag questions is obligatory.
\is{}
\ea \#By any chance, you don't like dancing || DO you?\label{Pos.Nu.BAC}
\ex \#By any chance, you like dancing || DON'T you?\label{Neg.Nu.BAC}
\z
\il{English}

Now, we will explore the strength and polarity settings of their biases, starting with the Follow-up Test. As shown by the preference for the strong epistemic follow-ups in \xref{Pos.Nu.TIS} and \xref{Neg.Nu.TIS}, the bias conveyed by nuclear tag questions is strong and of the opposite polarity to the tag polarity.
\is{}
\ea Susan doesn't hate exercise || DOES she? That is to say, ...\label{Pos.Nu.TIS}
    \ea \#I suspected she didn't.
    \ex I was sure she didn't.
    \z
\ex Susan hates exercise || DOESN'T she? That is to say, ...\label{Neg.Nu.TIS}
    \ea \#I suspected she did.
    \ex I was sure she did.
    \z
\z
\il{English}
The Weighted Coin Tests in \xref{Pos.Nu.WC} and \xref{Neg.Nu.WC} confirm that the biases are strong. As shown, these tag questions are infelicitous when the speaker's prior belief regarding the prejacent is relatively weak. The cause of this infelicity as coming from the strength of the bias is confirmed by the fact that we get felicity when the chance of the coin landing on tails is increased/decreased to near certainty one way or the other.
\is{}
\ea Mary and John are playing with a coin that they both know is designed so that it lands on tails \textit{N\%} of the time. Mary tosses the coin and it lands on the other side of John's legs, such that only John can see the result. John looks confused, causing Mary to say:\label{Pos.Nu.WC}
    \ea 30\%: \#It didn't land on tails || DID it?\label{Pos.Nu.WC.30}
    \ex 1\%: It didn't land on tails || DID it?\label{Neg.Pnu.WC.1}
    \z
\ex Mary and John are playing with a coin that they both know is designed so that it lands on tails \textit{N\%} of the time. Mary tosses the coin and it lands on the other side of John's legs, such that only John can see the result. John looks confused, causing Mary to say:\label{Neg.Nu.WC}
    \ea 70\%: \#It landed on tails || DIDN'T it?\label{Neg.Nu.WC.70}
    \ex 99\%: It landed on tails || DIDN'T it?\label{Neg.Nu.WC.99}
    \z
\z
\il{English}

In sum, the bias profile of nuclear positive-tag/negative-tag questions is obligatory, negative/positive (respectively), and strong. We now consider how the machinery of our analysis can capture the biases conveyed by nuclear tag questions. 

To do so, we need to bring together all the different pieces of our analysis, including our general analysis of tag questions, polarity focus, and \textsc{verum}. To start with, we claim that the \textsc{verum} operator is necessarily present in nuclear tag questions, due to the prosodic contour associated with the tag. That is, the pitch accent on the auxiliary verb in the tag signals the presence of \textsc{verum} and its semantic effects. As for polarity focus, in the case of nuclear negative-tag questions, it is necessarily generated as a result of the high negation structure. For example, the  tag question in \xref{nNTQ.Anyls} receives the analysis shown below.  
\is{}
\ea Paul goes to church || DOESN'T he?\label{nNTQ.Anyls}
    \ea {[[}$_{\text{CP}}$ Paul$_i$ go to church] [\textsc{tag} {[}$_{\text{CP}}$ \textsc{q} [$_{\text{FocP}}$ not$_{F}$ [$_{\text{PolP}}$ \textsc{verum} [$_{\text{TP}}$ he$_i$ \sout{go to church}]]]$_{\phi}$ $\sim C$]]]\label{nNTQ.LF}
    \ex $\Diamond \, \lambda w \, . \, go.to_w(paul,church) \, \bullet \, 
            \left\{ \begin{array}{l}
            \lambda w \, . \, go.to_w(paul,church), \\ 
            \lambda w \, . \,\neg go.to_w(paul,church) \\ 
            \end{array} \right\}$, \\
            provided the context contains conflicting evidence for and against $\lambda w \, . \, go.to_w(paul,church)$
    \ex  $C = \lambda w \, . \, go.to_w(paul,church)$\label{NegNTQ.C}
    \ex $\llbracket \phi \rrbracket^{o} = \lambda w \, . \,\neg go.to_w(paul,church)$\label{nNTQ.phi}, \\
        provided the context contains conflicting evidence for and against $\lambda w \, . \, go.to_w(paul,church)$
    \ex $\llbracket \phi \rrbracket^{f} = \left\{ \begin{array}{l}
            \lambda w \, . \, go.to_w(paul,church), \\ 
            \lambda w \, . \,\neg go.to_w(paul,church) \\ 
            \end{array} \right\}  $
    \ex $C \in  \llbracket \phi \rrbracket ^f$ \cmark, \, $C \ne \llbracket \phi  \rrbracket ^o$ \cmark
    \z
\z
\il{}
The strong, positive and obligatory bias in \xref{nNTQ.Anyls} arises from the characteristics of the tag, which mirror those of the verum-accented question in \xref{Vrm.LF2.Der}, except that here the polar element focused is negation, instead of \textsc{verum}. Recall that the statement made by the anchor is very weak in that it only posits the possibility of the relevant proposition. On the other hand, the tag is comprised of a focused  high negation and a \textsc{verum} operator. Therefore, as with all the other questions containing polarity focus, an obligatory bias that is of the opposite polarity to that of the focus domain (i.e., a positive bias) is generated. Moreover, the satisfaction of the conflicting evidence presupposition introduced by \textsc{verum} increases the strength of the bias, due to the higher certainty requirements conventionally associated with conflicted contexts.

As for  nuclear positive-tag questions, their bias is generated in precisely the same manner, except that in their case the focused polar element is \textsc{verum}. The details are presented in \xref{nPTQ.Anyls}.
\ea Paul doesn't go to church || DOES he?\label{nPTQ.Anyls}
    \ea {[[}$_{\text{CP}}$ Paul$_i$ not go to church] [\textsc{tag} {[}$_{\text{CP}}$ \textsc{q} [$_{\text{PolP}}$ \textsc{verum}$_{F}$  [$_{\text{TP}}$ he$_i$ \sout{go to church}]]$_{\phi}$ $\sim C$]]]\label{nPTQ.LF}
    \ex $\Diamond \, \lambda w \, . \, \neg go.to_w(paul,church) \, \bullet \, 
            \left\{ \begin{array}{l}
            \lambda w \, . \, go.to_w(paul,church), \\ 
            \lambda w \, . \,\neg go.to_w(paul,church) \\ 
            \end{array} \right\}$, \\
            provided the context contains evidence for and against $\lambda w \, . \, \neg go.to_w(paul,church)$
    \ex  $C = \lambda w \, . \, \neg go.to_w(paul,church)$\label{nPTQ.C}
    \ex $\llbracket \phi \rrbracket^{o} = \lambda w \, . \, go.to_w(paul,church)$, \\
            provided the context contains evidence for and against $\lambda w \, . \, go.to_w(paul,church)$\label{nPTQ.phi}
    \ex $\llbracket \phi \rrbracket^{f} = \left\{ \begin{array}{l}
            \lambda w \, . \, go.to_w(paul,church), \\ 
            \lambda w \, . \,\neg go.to_w(paul,church) \\ 
            \end{array} \right\}  $
    \z
\z

One interesting thing to note is that, in the case of  nuclear positive-tag questions, a neutral or unbiased interpretation that would arise in the absence of polarity focus seems to be ruled out. This is different from polar questions with a verum accent, which may or may not be biased and thus were assumed to only optionally carry polarity focus (see \sectref{Vrm.CovOp}). One way of capturing this difference is by appealing to the presence of the anchor in  nuclear positive-tag questions. That is, the negative anchor presents the negative proposition that would serve as an antecedent for the polarity focus in the tag. Therefore, for discourse coherence purposes, the tag is required to contain polarity focus and a non-biased interpretation is not available. 

In this way, the machinery introduced by our analysis -- namely, a general model of tag questions, polarity focus, and our novel semantics for \textsc{verum} -- can capture the biases conveyed by both nuclear positive-tag and nuclear negative-tag questions.

\section{Alternative accounts}\label{Sect5}
Broadly, we can divide the alternative accounts of tag questions into two groups, based on whether the relevant bias is derived primarily (or solely) from the anchor or from the tag.   

\subsection{Anchor-based approach}
A lot of accounts of bias in tag questions fall under what we call the `anchor-based approach' \citep{Reese2007, Reese2009, krifka2015, Malamud2015, jamieson2018, Woods2021}. While there is some variation in the specific architecture of these accounts, they all subscribe to the general notion that the bias conveyed by tag questions is derived primarily from the declarative anchor. The basic idea here is that the usual discourse effects of producing a declarative utterance are carried over to a tag question in the form of question bias. For example, \citet{krifka2015} assumes a projected discourse development model (called Commitment Space Semantics), where questions are modeled as conversational moves that restrict the possible continuations available to speech participants. In this framework, reverse-polarity tag questions are analyzed as disjunctions of an assertion and a polar question of the opposite polarity. This presents other participants with a choice: they can either `join' the initial speaker in a commitment to the anchor proposition, or can commit themselves to its complement, in which case the initial speaker can either re-commit themselves to the original anchor proposition, or join their interlocutor in accepting its complement. In any case, as far as the typology we are sketching here is concerned, the important thing is that the bias conveyed by tag questions is generated from the anchor component. 

While these accounts perform quite well at capturing some of the basic facts about tag question bias (e.g., their polarity properties), we point out that they suffer from both under- and over-generation problems. Starting with undergeneration, they seem to predict greater uniformity in the biases conveyed by different tag questions than appears to be the case. That is, assuming the bias profiles we have outlined above are correct, it is not clear how these accounts can explain them. For example, it is not clear how they can capture the differences we found in the optionality settings of the biases conveyed by post-nuclear positive-tag vs. negative-tag questions. If the bias were indeed derived from the anchor, then this would predict that all tag questions should be obligatorily biased, as they all include such an anchor component. However, as we showed in \sectref{Sec.PNu.Pos}, post-nuclear positive-tag questions are optionally biased, which presents a challenge for the anchor-based approach.\footnote{This criticism does not necessarily apply to the account proposed in \citet{Reese2009}, which attempts to explain the optionality of the bias in post-nuclear positive-tag questions by positing the presence of a meta-linguistic negation in the anchor which cancels out the usual assertive contribution of the anchor component. Moreover, the analysis presented in \citet{Jamieson2018} is explicitly restricted to nuclear tag questions.}  Somewhat less problematic for this approach are the variations in strength that we observed between post-nuclear and nuclear tag questions. These are less problematic because, while the accounts as they currently stand do not predict this variation, they could easily do so by adopting our (or a similar) analysis of the tag component. That is, they could posit a covert \textsc{verum} operator and thus strengthen the bias in the same manner as we propose. 

The anchor-based approach also seems to over-generate, in the sense that it predicts the possibility of a post-nuclear tag question comprised of a negative anchor plus a negative tag (i.e., a negative matching tag question). That is, on any analysis where a tag questions is composed of an assertion and a polar question, it is unclear why the positive combination of these two elements should be possible, but not the negative combination, as noted in \citet{Cattell1973} and shown in \xxref{Pos.Pnu.PosAnch}{Neg.Pnu.NegAnch}.  

\ea[]{
John drank beer $=$ did he?}\label{Pos.Pnu.PosAnch}
\ex[\#]{
John didn't drink beer $=$ didn't he?}\label{Neg.Pnu.NegAnch}
\z

In contrast, our account is able to capture this asymmetry straightforwardly. That is, the tag component in \xref{Neg.Pnu.NegAnch} is an elliptical high negation question. Since the negation in such questions is focus-marked, it is in need of a contrasting positive antecedent and thus clashes in some sense with the presence of a negative anchor. In contrast, the tag question in \xref{Pos.Pnu.PosAnch} contains an unbiased positive polar question in the tag component, meaning there is no focus marking in the tag and thus no clash with the positive proposition presented in the anchor.  

\subsection{Tag-based approach}
In contrast to the anchor-based approach, what we call the `tag-based approach' attributes the primary source of the bias associated with tag questions to the tag. The account we have presented in this paper is a member of this approach. The only other account that seems to fit in this approach is that presented in \citet{romero2004}. These authors propose that (reverse-polarity) tag questions always contain a covert \textsc{verum} operator within the tag and that their bias properties are derived in the same manner as they are for their matrix question counterparts. Although their \textsc{verum} operator is a conversational/epistemic operator stating that the speaker is certain that the prejacent proposition should be added to the common ground, the question bias is derived in a similar way as on our account.   

This analysis can comfortably capture the polarity settings of the biases conveyed by tag questions. But it has difficulty capturing the ways in which the question bias varies along the two other dimensions we identified, i.e., strength and optionality. Specifically, because \citet{romero2004} propose that the tag component of tag questions always contains a \textsc{verum} operator, they predict that (other than their polarity settings) the bias profiles should be uniform. However, as we noted above, there is considerable variation in both the optionality and strength features of tag questions. That is, the bias conveyed by post-nuclear positive-tag questions is optional, whereas the biases conveyed by all the other forms we investigated are obligatory. Similarly, while the strength of post-nuclear tag questions is weak, the strength of nuclear tag questions is strong. These variations in bias profiles are unexpected if they are all derived from the application of the same \textsc{verum} operator in the tag component of tag questions. 
\section{Summary}\label{Sect6}
In this paper we have focused on a variety of reverse-polarity tag questions in English and have made both empirical and theoretical contributions. Starting with the empirical contributions, we have identified that the speaker biases conveyed by tag questions (and certain other biased questions) vary across three dimensions: optionality, strength, and polarity. Moreover, we have identified the specific bias profiles of our targeted tag questions, which turned out to vary along these three dimensions. 

As for theoretical contributions, we proposed a modular account of how the bias conveyed by the relevant tag questions is generated. We analyzed tag questions as complex expressions consisting of a declarative and an elliptical polar interrogative, where the latter conveys a bias with typically the same features as would the corresponding independent non-elliptical polar interrogative. Specifically, we argued that bias profiles of the investigated tag questions are determined by the presence of polarity focus and the semantics of a covert \textsc{verum} operator. That is, when polarity focus is present, the bias is obligatory, weak and of the opposite polarity to the focus domain. Then, in cases where \textsc{verum} is present (i.e., in nuclear tag questions), the strength of the bias is boosted. The bigger point is that there is nothing mysterious about tag questions: their bias profiles can be derived in a composite way from the elements that make up such questions and whose semantic effects are established independently. Our empirical and theoretical contributions are summarized in \tabref{tab.2.summary}. 

\begin{table}
 \begin{tabular}{l cccr}
  \lsptoprule
  tag question  & optionality & strength  & polarity & analysis\\
  \midrule
  pnPTQ &   optional    &   weak    &   negative    & (anchor) \\
  pnNTQ &   obligatory  &   weak    &   positive    & \textit{not}$_F$ \\
  nPTQ  &   obligatory  &   strong  &   negative    & \textsc{verum}$_F$ (+ anchor)\\
  nNTQ  &   obligatory  &   strong    &   positive    & \textit{not}$_F$ + \textsc{verum}\\
  \lspbottomrule
 \end{tabular}
\caption{Summary of bias profiles and proposed analyses of reverse-polarity tag questions. (Abbreviations: pnPTQ = post-nuclear positive-tag question, pnNTQ = post-nuclear negative-tag question, nPTQ = nuclear positive-tag question, and nNTQ = nuclear negative-tag question)}
\label{tab.2.summary}
\end{table}

Finally, we argued that previous accounts of tag questions do not perform as well in capturing the noted variation in tag question bias.

We close the discussion with one speculative remark. An anonymous reviewer wonders what would justify our claim that languages resort to an operator like \textsc{verum} in order to mark conflicting evidence, especially in view of the fact that polarity focus can play a similar role. Although we cannot provide a definitive answer to this worry, we point out that the semantic effects of polarity focus and \textsc{verum} are not equivalent. That is, polarity focus merely conveys a contrast, indicating that the opposite polar alternative is salient in the context. It thus says nothing about evidence, truth, or similar notions. In turn, \textsc{verum} strengthens this contrast to an epistemic conflict, indicating that there is incompatible evidence regarding the prejacent and that the conversation is in a state of crisis. We hypothesize that it is for this reason that a \textsc{verum} operator is overtly lexicalized in a number of typologically unrelated languages \citep{Gutzmann2020}. Notice also that polarity focus cannot exist in a vacuum as it needs to mark \textit{some} polar operator. Thus, given the kinship between the two mechanisms, it seems plausible that polarity focus and \textsc{verum} feed each other.  

%\section*{Abbreviations}
%\begin{tabularx}{.45\textwidth}{lQ}
%... & \\
%... & \\
%\end{tabularx}
%\begin{tabularx}{.45\textwidth}{lQ}
%... & \\
%... & \\
%\end{tabularx}

\section*{Acknowledgements}
We would like to thank two anonymous reviewers and the participants at the \textit{Biased Questions Workshop} at ZAS for useful feedback. This research was funded by DFG grant KO5704/1-1.

\printbibliography[heading=subbibliography,notkeyword=this]

%The term ``focus'' refers to a certain type of prosodic or morphological marking of a constituent within an utterance, which nominates the targeted expression as input for a set of pragmatic and semantic functions.

%Focus can be interpreted as \textit{semantic} or \textit{pragmatic} \citep{Rooth1985, Rooth1997, Beaver2008, Krifka2012, Buring2019}. Semantic focus has a truth-conditional effect and involves a conventional association between a focus operator (\textit{only}, \textit{even}, \textit{also}, etc.) and a focus-marked constituent in its scope. More important for our purposes is pragmatic focus. In order for pragmatic focus to be felicitous, a certain antecedent needs to be present in the discourse (or else it needs to be accommodated). The requirements regarding the nature of this antecedent depend on the function pragmatic focus plays in a given utterance. 

%\citet{romero2004} claim that the two uses can be distinguished by syntactic position, cf. \textit{Sandra is really clever} (a degree modifier use) vs. \textit{Sandra really is clever} (a polar use). This contrast in position seems real, but it is only visible in the presence of an auxiliary.

%Section 3.1: \footnote{\citet{Wilder2013}, \citet{Samko2016}, \citet{Goodhue2018a}, and \citet{Gutzmann2020} all agree that verum accent is infelicitous in out-of-the-blue contexts. At the same time, \citet{Goodhue2018a} claims that verum-marked sentences may answer questions in neutral contexts, citing \xref{Vrm.Neu.Goodhue} as evidence.
%\is{}
%\ea\label{Vrm.Neu.Goodhue}
    %\begin{xlist}
    %\exi{A:} Did Naomi buy yogurt?
    %\exi{B:} She DID buy yogurt.
    %\end{xlist}
%\z
%\il{}
%In our view, however, B's response clearly signals that the common ground contains evidence against the proposition targeted by \textsc{verum} was unlikely to be true.}

%That is, in \xref{Vrm.Contr.Eg2}, A's utterance would typically be understood as a reaction to a prior suggestion or assertion of \textit{Oliver is from Australia}.

% Section 3.2: The intuition of verum-marked sentences placing emphasis on the truth of the prejacent is then derived straightforwardly. That is, in order to assert $p$ in a discourse context containing conflicting evidence regarding $p$, the conventional certainty requirements are higher than in a neutral context. This results in the sense that the speaker is placing increased emphasis/certainty on the truth of $p$.

% Section 3.2: \footnote{Although we do not propose an explicit analysis of affirmation uses that lack extreme predicates, we think that \citeauthor{Morzycki2012}'s (\citeyear{Morzycki2012}) distinction between \textbf{lexical} and \textbf{contextual} extreme predicates provides a clue of how such cases are to be handled. More specifically, we suggest that such cases involve \underline{contextually} extreme predicates (or respective readings). That is, a context may support a reading of an ordinary predicate whereby its standard falls outside the salient portion of the scale.}

% Section 3.3: Next, we will consider some of the alternative accounts of tag questions and argue that they are unable to capture the relevant data to the same extent as the analysis we have just presented.

\end{document}
