\documentclass[output=paper,colorlinks,citecolor=brown]{langscibook}
\ChapterDOI{10.5281/zenodo.17158176}
\author{Anastasia Giannakidou\orcid{0000-0003-2266-9410}\affiliation{University of Chicago} and Alda Mari\orcid{0000-0002-9479-9110}\affiliation{Institut Jean Nicod, CNRS/ENS/EHESS/PSL}}
\title{Modalization and bias in questions}
\abstract{In this paper, we discuss the presence of modal elements in questions (adverbs, particles, negation) and propose a new theory of bias that relies on the similarities between  polar questions and modal assertions. While the two differ in that the first do not have truth conditions, they are deeply similar in presupposing nonveridical spaces, partitioned into $p$ and its negation. The partition can be indiscriminate, in which case we have nonveridical equilibrium, or bias (positive/negative). We propose an analysis of bias as indicating the presence of a meta-evaluation ranking function O which is also a component of modal structures \citep{giannakidoumari2018a, giannakidoumari2018b, giannakidoumari2021a, giannakidoumari2021b}.  Focus adverbials such as \textsc{REALLY}, modal adverbials, and negation are overt lexicalizations of the ranking, and not indicators of \textsc{verum}, as we show.   Our paper also discusses the phenomenon of \textit{reflection} which arises with the presence of modal particles as a pragmatic dual to bias. We identify a number of properties of reflective questions that set them apart from information and biased questions evidencing enhanced uncertainty which we analyze as an effect of the modal element in a reflective question widening the modal base. }

%move the following commands to the ``local..." files of the master project when integrating this chapter

\IfFileExists{../localcommands.tex}{
   \addbibresource{../localbibliography.bib}
   % add all extra packages you need to load to this file

\usepackage{tabularx,multicol}
\usepackage{url}
\urlstyle{same}

\usepackage{listings}
\lstset{basicstyle=\ttfamily,tabsize=2,breaklines=true}

\usepackage{langsci-basic}
\usepackage{langsci-optional}
\usepackage{langsci-lgr}
\usepackage{langsci-osl}
% \usepackage{./langsci/styles/langsci-lgr}
% \usepackage{./langsci/styles/langsci-osl}
% \usepackage{langsci-gb4e}

\usepackage{tikz}
\usetikzlibrary{patterns,calc}
\pgfdeclarepatternformonly{south east lines}{\pgfqpoint{-0pt}{-0pt}}{\pgfqpoint{3pt}{3pt}}{\pgfqpoint{3pt}{3pt}}{
    \pgfsetlinewidth{0.6pt}
    \pgfpathmoveto{\pgfqpoint{0pt}{3pt}}
    \pgfpathlineto{\pgfqpoint{3pt}{0pt}}
    \pgfpathmoveto{\pgfqpoint{.2pt}{-.2pt}}
    \pgfpathlineto{\pgfqpoint{-.2pt}{.2pt}}
    \pgfpathmoveto{\pgfqpoint{3.2pt}{2.8pt}}
    \pgfpathlineto{\pgfqpoint{2.8pt}{3.2pt}}
    \pgfusepath{stroke}}
    
\usepackage{stmaryrd}
\usepackage{wasysym}
\usepackage{multirow}
\usepackage{caption}
\usepackage{subcaption}
\usepackage{mathrsfs}
\usepackage{qtree}

\usepackage{linguex}


   %pminos do not split footnotes
% \interfootnotelinepenalty=10000 %Footnote in Laporte chapters has to be split SN


%\DeclareIndexNameFormat{default}{%
%\nameparts{#1}%
%\usebibmacro{index:name}%
%{\index[names]}%
%{\namepartfamily}%
%{\namepartgiveni}%
% {}% L1
% {}% L2
%{\namepartprefix}% generates spurious space L3
%{\namepartsuffix}% generates spurious space L4
%}

%  {\DeclareIndexNameFormat{default}{%
%     \usebibmacro{index:name}{\index[names]}{#1}{#3}{#5}{#7}}}

%\DeclareIndexNameFormat{default}{%
%  \usebibmacro{index:name}{\sindex[nom]}{#1}{#3}{#5}{#7}}

%\DeclareIndexNameFormat{default}{%
%  \usebibmacro{index:name}{\sindex[person]}{#1}{#3}{#5}{#7}}
%\DeclareIndexNameFormat{default}{%
%\nameparts{#1} \usebibmacro{index:name}{\sindex[person]]}{\namepartfamily}{‌​\namepartgiven}{\nam‌​epartprefix}{\namepa‌​rtsuffix}}

%\newcommand{\smiley}{:)}

%\renewbibmacro*{index:name}[5]{%
%\usebibmacro{index:entry}{#1}%
%{\iffieldundef{usera}{}{\thefield{usera}\actualoperator}\mkbibindexname{#2}{#3}{#4}{#5}}}

% \newcommand{\noop}[1]{}

%remove for final
%\overfullrule=1mm

\newcommand{\tobi}[2]}}
\renewcommand{\S}[1]{\tobi{#1}{\textsc{*}}}

% this volume references
% puts: [this volume]
% already defined: \citetv
%\newcommand{\citepv}[1]{(\citeauthor{#1} \citeyear*{#1} [this volume])}
\newcommand{\citealtv}[1]{\citeauthor{#1} \citeyear*{#1} [this volume]}

%parentheses around example number
\newcommand{\pref}[1]{(\ref{#1})}

% in-text examples

\newcommand{\lnex}[1]{\textit{#1}} %target lang word
\newcommand{\lnlit}[1]{(lit.: `#1')} %literal reading
\newcommand{\lnlat}[1]{(#1)} % latinization
\newcommand{\lntrans}[1]{`#1'} %translation
\newcommand{\lnexl}[2]%
{\lnex{#1}{} \lnlat{#2}} % ex with latinization
\newcommand{\lnexlat}[3]{\lnex{#1}{} \lnlat{#2}{} \lntrans{#3}} % ex with latinization and tranl.

%ch01
\newcommand{\co}[1]{\mbox{\textbf{#1}}}

%ch09

\newcommand{\cyrbulg}[1]{\begin{otherlanguage*}{bulgarian}#1\end{otherlanguage*}}


%ch10
\newcommand{\nlp}{{\small NLP}}
\newcommand{\mwe}{{\small MWE}}
\newcommand{\rae}{{\small RAE}}
\newcommand{\lvc}{{\small LVC}}
\newcommand{\pos}{{\small P}o{\small S}}
%\newcommand{\todo}[1]{ \textcolor{red}{#1} }

%\renewcommand{\labelenumi}{\theenumi}
%\ainamefmt{{vv}{ll}{, ff}{, jj}} % fullname

\newcommand{\biberror}[1]{{\color{red}#1}}

\newcommand{\osenovaitem}{--~}
   %% hyphenation points for line breaks
%% Normally, automatic hyphenation in LaTeX is very good
%% If a word is mis-hyphenated, add it to this file
%%
%% add information to TeX file before \begin{document} with:
%% %% hyphenation points for line breaks
%% Normally, automatic hyphenation in LaTeX is very good
%% If a word is mis-hyphenated, add it to this file
%%
%% add information to TeX file before \begin{document} with:
%% %% hyphenation points for line breaks
%% Normally, automatic hyphenation in LaTeX is very good
%% If a word is mis-hyphenated, add it to this file
%%
%% add information to TeX file before \begin{document} with:
%% \include{localhyphenation}
\hyphenation{
    Beck-man
    Ngu-yen
    back-chan-nel
    back-chan-nels
    mo-not-o-nous
    ste-reo-typ-i-cal
}

\hyphenation{
    Beck-man
    Ngu-yen
    back-chan-nel
    back-chan-nels
    mo-not-o-nous
    ste-reo-typ-i-cal
}

\hyphenation{
    Beck-man
    Ngu-yen
    back-chan-nel
    back-chan-nels
    mo-not-o-nous
    ste-reo-typ-i-cal
}

   \boolfalse{bookcompile}
   \togglepaper[23]%%chapternumber
}{}

\begin{document}
\maketitle

\section{Questions, bias, and the nonveridical partition} \label{sec:02:intro}

In the literature on speech acts the assumption is made that different syntactic types (declarative, interrogative and imperative) map onto distinct speech acts such as assertions, questions and commands respectively (see for a recent discussion  \citealt{portner2018}).  When it comes to assertions and questions, most formal  analyses  assume a logical language that reflects a clear-cut syntactic distinction between declaratives and interrogatives often containing a designated speech act operator such as \textsc{assert} and $?$ respectively (\citealt{krifka1995}). \citet{ciardelli2013} call those analyses `\textit{syntactically dichotomous}', and propose further that they are also semantically dichotomous in assigning different semantic values to assertions (propositions) and interrogatives, which are taken to denote questions, i.e., sets of propositions. 





While categorically different,   questions and  assertions, however,  come closer in the phenomenon of \textit{bias}. While the denotation of a plain interrogative such as \xref{ex:ernie:veggie} is a request for information, biased questions \xref{ex:ernie:veggie:bias} are not just seeking information but also convey the speaker's expectation that a positive or negative answer is more likely. This is called \textit{bias}:

\ea \label{ex:ernie:veggie} Is Ernie a vegetarian?
\ex \label{ex:ernie:veggie:bias}
    \ea \label{ex:ernie:high:neg} Isn't Ernie a vegetarian?		\jambox*{(high negation: positive bias)}
    \ex \label{ex:ernie:neg:tag} Ernie is a vegetarian, isn't he?  	\jambox*{(negative tag: positive bias)}
    \ex \label{ex:ernie:adv} Is Ernie really a vegetarian? 	\jambox*{(adverb \textit{really}: negative bias)}
\z 
\z



% \ex. \a. Isn't Ernie a vegetarian?		(high negation: positive bias) 
% \b. Ernie is a vegetarian, isn't he?  	(negative tag: positive bias) \\
% \c. Is Ernie really a vegetarian? 	(adverb \textit{really}: negative bias) 

A speaker uttering a plain yes/no interrogative is in a state of ``true'' uncertainty: she does not know if Ernie is a vegetarian and poses the question as a request to find out. The polar question is therefore ``information seeking'' and does not discriminate towards one answer or the other. This state of  neutral uncertainty between $p$ and its negation has been characterized as nonveridical \textit{equilibrium} (\citealt{giannakidou2013, giannakidoumari2016, giannakidoumari2018a, giannakidoumari2018b, giannakidoumari2021a, giannakidoumari2021b})  because the two options are entertained by the speaker as equal possibilities upon asking the question:

\ea \label{nonveridical} Nonveridical equilibrium (= ``True uncertainty'' in \citealt{giannakidou2013})
\sn A partitioned ($p$ and $\neg p$) epistemic or doxastic space $M(i)$ is in non\-ve\-ri\-di\-cal equilibrium if $p$ and $\neg p$ are equal options, i.e., they are not ranked; $i$ is the individual anchor, by default in questions the speaker. 
\z 


Following our earlier work, we take equilibrium to be the default semantic feature of epistemic possibility, questions, and conditionals (for more recent discussion see \citealt{liu2021}), regardless of their discourse function. The egaliatarian state of the nonveridical equilibrium  is neutral because when asking a plain polar question the speaker has no preconception of assuming which answer (yes or no) is true, no priors (i.e. previously held beliefs or assumptions) as to the positive or the negative answer being more likely. The proposition \textit{Ernie is a vegetarian} is not challenged in the context prior to asking the question, and the speaker does not have any preference for a positive or negative answer, no expectations that would make them think that Ernie is or is not a vegetarian. Nonveridical equilibrium is, in other worlds, a state of epistemic neutrality with no preconditions on the context or the speaker's epistemic state regarding the questioned content. 

When asking a biased question, on the other hand, the speaker reveals that they actually do have some prior expectations  that pre-empt  them to discriminate between the two possible answers. They now judge a positive or a negative answer as more likely.  For instance, upon asking \xxref{ex:ernie:high:neg}{ex:ernie:neg:tag} the speaker reveals that they are considering, prior to asking the question, that \textit{Ernie is a vegetarian} is a more likely answer. Having prior expectation is not equivalent to believing that  \textit{Ernie is a vegetarian}; if the speaker had believed that the content is true they wouldn't have asked the question. Likewise in \xref{ex:ernie:adv} with \textit{really}, the speaker is having reasons to think that \textit{Ernie is not a vegetarian} is a more likely answer. Again, this is not a belief that Ernie is a vegetarian; if the speaker believed that already, there would be no need to ask a question. With biased questions uncertainty still exists about what the true answer is, but the speaker comes to question not from a neutral stance but from a discriminating one: she ranks the two possible answers, and has prior assumptions that favor the one or the other.

Bias thus modifies the neutrality of equilibrium in a positive or negative direction by the speaker's priors which are revealed in that the speaker chooses to add certain devices to the question, such as high negation \xref{ex:ernie:high:neg}, a negative tag \xref{ex:ernie:neg:tag}, or the adverb \textit{really} \xref{ex:ernie:adv}. The choice  to add these devices reveals to the audience that the speaker abandons neutrality and questions with preference towards a (positive or negative) answer (\citealt{sadock1971, ladd1981, abels2003, van2003, romero2004, Reese2007, asher2007, krifka2015SALT, malamud2015, roelofsen2015, larrivee2022, giannakidoumari2021a, giannakidoumari2021b} a.o.). In \xxref{ex:ernie:neg:tag}{ex:ernie:adv} the speaker has a positive bias and seems to believe it more likely that Ernie is a vegetarian; in \xref{ex:ernie:adv}, by adding \textit{really}, the speaker intends to show that they doubt that Ernie is a vegetarian, and we talk about \textit{negative bias}.\footnote{Negative bias is also famously observed with negative polarity items (NPIs)  (see \citealt{borkin1971, giannakidou1997, giannakidou2007, van2003, guerzoni2004, guerzoni2007}, a.o.): 

\ea 
\ea Have you spoken to Mary even once?	\jambox*{(NPI: negative bias)}
\ex Did Mary lift a finger to help?		\jambox*{(NPI: negative bias)}
\z
\z


The speaker here has a negative expectation that the addressee has not spoken to Mary or that Mary did not help. In both cases, bias arises because the speaker decided to augment the questions with a focused NPI.} 

Biased questions, then,  are not mere requests for information but rely on the speaker's prior doxastic and epistemic assumptions and expectations.\footnote{There is also contextual bias relating to evidence available in the context, i.e., the common ground (\textit{evidential bias}, \citealt{buring2000}; \citealt{romero2004}; \citealt{sudo2013}; \citealt{northrup2014}; \citealt{domaneschi2017}), or answer bias (\citealt{krifka2015SALT}; \citealt{malamud2015}). We will not discuss common ground bias here; for more details on factors determining belief formation see \citet{giannakidoumari2021a, giannakidoumari2021b}}  In this respect, they lie on a continuum between questions and assertions: they ask whether $p$ but also discriminate towards  $p$, or its negation. Notice that in the tag question \xref{ex:ernie:neg:tag} we actually do have a hybrid declarative and interrogative sentence.   Importantly, the bias can also famously be cancelled with an answer of the unexpected polarity. Bias is a choice that the speaker makes based on their assumptions and contextual knowledge but it is not a common ground presupposition, as we will emphasize, hence it can be objected to. In the case of positive bias, the speaker seems to rank $p$ as more likely,  hence they are more epistemically committed to it;  in the negative bias, $\neg p$ is considered more likely and there is less epistemic commitment to $p$.  Speaker commitment  is a notion  that plays a key role in modality (see \citealt{giannakidoumari2021a}),  and a handy way to refer to the judgment of an epistemic agent towards the veridicality of $p$. In our view, commitment is private (we thus differ for instance from \citealt{krifka2015SALT} and \citealt{geurts2019} who consider commitment a public attitude).


%
\citet{giannakidou2013} connected questions to modal sentences (which she call\-ed \textit{inquisitive assertions}), and \citet{giannakidoumari2018b, giannakidoumari2021a, giannakidoumari2021b} establish a parallel in modality between neutral possibility modals and biased ones (necessity modals):  with possibility modals, the speaker is in a nonveridical information state where $p$ and its negation are equally open possibilities, but with a necessity modal such as \textsc{MUST} there is bias towards the prejacent $p$. Consider:

\ea 
\ea Ernie is a vegetarian. 
\ex Ernie must be a vegetarian.
\ex Ernie may/might be a vegetarian. 
\ex Is Ernie a vegetarian?
\z
\z


Of the sentences above, only the unmodalized assertion  conveys the information that Ernie is a vegetarian, and adds it to the common ground.  Only this one is, then, an assertion of $p$.  The modal sentences, just like the plain question, reveal uncertainty, which we represent as a nonveridical epistemic state, that is to say, a modal base partitioned between Ernie being a vegetarian and not being a vegetarian. The speaker, in choosing to utter modalized sentences, just like with questions, takes a \textit{nonveridical} stance (\citealt{giannakidoumari2021a}, \citealt{giannakidoumari2021b}). The nonveridical stance can be neutral, as with neutral information question and possibility modals, or biased when the speaker chooses to use \textsc{MUST}. This parallelism between possibility modals and unbiased questions on the one hand, and necessity modal \textsc{MUST} and biased questions on the other  will play an important role in our analysis.


While we maintain that the categorical difference between modal assertions and questions lays in the first having and the second lacking truth conditions, we also acknowledge that both convey nonveridical states with two alternatives: $p$ and its negation. Nonveridicality thus ``allows us to see that the distinction between assertion and question is not as basic as we thought", \citet{giannakidou2013} states, and this, she continues, ``seems to support an approach to meaning as semantically non-dichotomous" (\citealt{giannakidou2013} : 117).  In the end, what seems to matter is whether a sentence presents the epistemic agent with one or more possibilities about the world, i.e. whether it reflects  a homogeneous or non-homogenous (partitioned) epistemic space. ``Superficially, this appears to correspond to the contrast between assertion vs. question. However, the more fundamental distinction is between a partitioned or not epistemic space" (\citealt{giannakidou2013}: 126).
  
  
In this paper, we expand on the idea  that the distinction between assertion and question is not categorical semantically, and use the phenomenon of bias as a testing ground to establish a parallel between the structure of questions and the structure of modality.  We argue that questions and possibility modals share an important piece of meaning: they express nonveridical equilibrium by not discriminating between the two alternatives $p$ and $\neg p$ ; they are by default egalitarian. When a bias device is used,  the device manipulates the equilibrium by introducing a ranking function. This is grammaticalized in modality with necessity modals such as \textsc{MUST}, \textsc{SHOULD} and their equivalents (\citealt{giannakidoumari2016, giannakidoumari2018b, giannakidoumari2021a, giannakidoumari2021b}) --  and the expressor of the ranking function  can be an overt adverb such as \textit{She must definitely be a doctor}, or a covert one. We will argue, similarly, that bias inducing devices in questions contribute or manipulate a ranking.  The different effects, positive or negative, are due to the lexical contribution of the bias inducing device as we will show.



The discussion proceeds as follows.  In \sectref{sec:02:2} we outline  the framework of modality we will be using with the bias  inducing ranking function $\mathcal{O}$. In \sectref{sec:02:3} and \ref{sec:02:4} we will propose that biased questions are equivalent to biased modal conjectures in containing a meta-evaluation function $\mathcal{O}$ which ranks the two possibilities, $p$ and its negation. Specifically in \sectref{sec:02:3} we derive negative bias with the adverb \textsc{REALLY}, and, in \sectref{sec:02:4}, we discuss positive bias with negation proposing that negation provides the ranking function. The discourse function of a biased interrogative is still that of a question, but its definedness conditions are those of a biased modal verb. Finally, in \sectref{sec:02:5}, we discuss questions with possibility modals and particles
and identify \textit{reflectiveness} as the inverse of bias, offering an analysis within the modal framework we have established. 


 \section{Nonveridical structure and bias inducing meta-evaluation} \label{sec:02:2}

Modal expressions in human languages --  modal verbs, adverbs, particles --  are common devices to reflect the speaker's judgement towards the truth of a proposition. Almost all analyses of modality assume that non-alethic modal expressions as a class are nonveridical, i.e., they reflect epistemic states that do not entail that the proposition is true (\citealt{kratzer1977, kratzer1986, kratzer1991, giannakidou1997, giannakidou1998, giannakidou1999, giannakidou2013, condoravdi2002, portner2009, Beaver2016, giannakidoumari2016, giannakidoumari2018b, giannakidoumari2021a, giannakidoumari2021b, lassiter2016}; \citealt{von2010} being a notable exception).

In the modal framework of \citet{giannakidoumari2016, giannakidoumari2018b, giannakidoumari2021a, giannakidoumari2021b} (henceforth GM)  the function of epistemic modal expressions is to convey the nonveridical attitude of the speaker: upon hearing or reading a modal sentence, the audience understands that the speaker cannot be fully committed to the truth of the propositional content of the sentence. Modal expressions are therefore characterized as \textit{anti-knowledge} markers in GM: a speaker cannot use a modal if she knows $p$ to be true, e.g. if I see rain falling I cannot say \textit{It might be raining}, or \textit{It must be raining}.   Likewise, when a speaker asks a  question they are in a state of not knowing. Both questions and modals, then,  convey epistemically nonveridical spaces, they have this common logical basis (see also \citealt{sherman2018}). 

  What does it mean for an expression to be veridical or nonveridical? Consider the following declarative sentences:
  
\ea \ea It is raining.
\ex It must be raining. 
\ex It may/might be raining. 
\z
\z

Let us call the tensed unmodalized sentences bare. In asserting a bare sentence the speaker is saying something that they know or believe to be true --  they are veridically committed to it. GM call this the \textit{Veridicality Principle} of co-operative communication, and it follows from abiding by Gricean Quality, which is fundamental to co-operative conversation:


\ea Principle of veridicality for cooperative communication (\citealt{giannakidoumari2021a, giannakidoumari2021b})
\sn A sentence S can be asserted by a speaker A if and only if A is veridically committed to the content $p$ of S (i.e., if and only if A knows or believes $p$ to be true).
\z


Thus, upon hearing an unmodalized sentence the hearer understands that the speaker knows, or has grounds to believe that it is raining. The epistemic attitude of the speaker can therefore be thought of as a veridical state. For a given judge $i$ (by default with epistemic modals, the speaker): 

   \ea 	\textit{Veridical epistemic state}\\ 
  An epistemic state M$(i)$ is veridical with respect to a proposition $p$ iff:    $\forall w [w \in M(i) \rightarrow p(w)]$. 
  \z
   
 A veridical state entails $p$, thus conveying epistemic (or doxastic, or mixed) commitment of $i$ to $p$.\footnote{As we mentioned already, our notion of commitment  is private and subjective, i.e., it represents the set of propositions held by individual linguistic agents (e.g., the speaker, the subject of the attitude in embedding) which \citet{giannakidou1997} and \citet{giannakidoumari2016, giannakidoumari2018b, giannakidoumari2021a, giannakidoumari2021b} call \textit{individual anchors}, recycling a term from \citet{farkas1985}; in our definition above the anchor is called \textit{judge}. Our commitment therefore differs from Krifka's which is ``modelled as a set of propositions, containing the propositions that are publicly shared by the participants" (\citealt{krifka2015SALT}: 328--329). Krifka's commitment corresponds to common ground assumptions, and the goal of speech acts is to ``change a commitment state". Speaker commitment in our view can be entirely private, even solipsistic (\citealt{giannakidoumari2016, giannakidoumari2018a, giannakidoumari2018b, giannakidoumari2021a, giannakidoumari2021b}), as with verbs of belief, imagination, dreaming and pretending.}   But when a speaker chooses to modalize, just like when she choses to pose a question, she is uncertain about $p$. The epistemic state now is partitioned into  $p$ and $\neg p$:

\ea  \textit{Nonveridical epistemic state}\\ 
An epistemic state (a set of worlds) $M(i)$  relative to  an epistemic agent $i$ is \textit{nonveridical}  with respect to a proposition $p$  iff $M(i)$ is partitioned into $p$ and  not $\neg p$ worlds.	
 \z

In a nonveridical state $M(i)$,  $p$ and not $p$ are open possibilities. The partition can be neutral, as in the case of equilibrium, or discriminating as is the case with bias inducing necessity modals (cf. \textit{infra}).   Both modals and questions are evaluated with respect to nonveridical modal spaces $M(i)$, where $i$ is by default the speaker.\footnote{Both questions and modals can be anchored to the addressee or a third party, and we return to the interrogative flip later in the paper. We assume that, by default, $i$ is the speaker.} \citet{giannakidoumari2016} formulate nonveridicality as a definedness condition of modals in the form of  the Nonveridicality Axiom --  which appears to be also a definedness condition of questions:

\ea \textit{Nonveridicality Axiom of modals and questions}
  \ea $\textsc{modal}(M(i))(p)$  can be defined if and only if the modal base $M(i)$ is nonveridical, i.e. only if $M(i)$  contains $p$ and $\neg p$ worlds. 
\ex \label{ques}
  $\textsc{ques}(M(i))(p)$  can be defined if and only if the modal base $M(i)$ is nonveridical, i.e. only if $M(i)$  contains $p$ and $\neg p$ worlds. 
  \z
  \z


The Nonveridicality Axiom is a definedness condition which lexically encodes a speaker presupposition in the Stalnakerian  sense of modals and questions.  We will revisit the presuppositional nature of bias and offer more comments in \sectref{sec:02:3}.

Now consider the difference between possibility and necessity modals. 

\ea \ea Ariadne might  be at the party.
\ex Ariadne must be at the party. 
\z
\z

\largerpage
In both cases, the speaker is uncertain about  Ariadne being at the party, and leaves open the possibility that she might not be. The uncertainty is in nonveridical equilibrium with \textsc{MIGHT}: Ariadne being at the party is considered a mere possibility, and the speaker has no reason to believe Ariadne being in the party is closer to what is the case than Ariadne not being at the party. Possibility modals are thus very much like information seeking questions in this regard: in both cases the speaker is in a  state of true uncertainty and entertains two alternatives equally.\footnote{As we noted earlier, this also holds for the  conditional protasis; see \citet{liu2021} for recent discussion and experimental results.} While questions lack truth conditions and are therefore distinct from modal assertions, it is important to understand that the common semantic core between neutral questions and possibility statements is the nonveridical state: they  both generate alternatives --  and  presuppose equilibrium where the speaker entertains two possible alternatives that are not ranked.

 

On the other hand, when a necessity modal such as \textsc{MUST}/\textsc{SHOULD}  is used, the equilibrium is manipulated towards the prejacent being considered more likely by the speaker. GM coin the term \textit{biased} modals for necessity modals: the speaker is positively biased in favor of the prejacent proposition, though they still are not veridically committed to it. Modal bias reveals an epistemic stance supported by evidence in favor of the proposition, but it does not mean that the speaker knows or believes $p$ to be true, i.e, they are not doxastically or epistemically committed to it. Modals, according to GM and what \citet{von2010} call `the mantra',  remain indicators that the speaker reasons with uncertainty and that they leave both options, $p$ and $\neg p$, open. With biased modals, in addition, the speaker positively discriminates towards $p$, she appears to think of $p$ as a better possibility than $\neg p$. Bias thus reveals that  the two possibilities are ranked. The ranking is done, GM propose, with a metaevaluation function $\mathcal{O}$ which is always present in a modal structure --  and  in a nonveridical structure more generally.

\largerpage
Let us see what the epistemic \textsc{MUST} modal structure is precisely. First, we have the  modal base of epistemic \textsc{MUST} which is nonveridical about the prejacent and contains both $p$ and $\neg p$ worlds.  To derive the truth conditions of \textsc{MUST} the literature assumes (see e.g. \citealt{kratzer1991, portner2009, giannakidoumari2016, giannakidoumari2018b, giannakidoumari2021a, giannakidoumari2021b}) that \textsc{MUST} uses a set of propositions ${\mathcal{S}}$ which describe shared stereotypical/normalcy conditions.  The Kratzer/Portner semantics posits an ordering source Best which ranks worlds according to how close they are to the stereotypical ideal. Our account encodes that the modal base is partitioned into stereotypical and non-stereotypical worlds, but we dissociate stereotypicality from ranking. 
 As an epistemic modal, \textsc{MUST} associates with an epistemic modal base $M(i)$ that contains the worlds compatible with what the speaker knows or believes. $w_0$ is the  world of evaluation, by default the actual world:

\ea $M(i)(t_u)(w_0)$  = $\lambda w' (w'$ is compatible with what is known by the speaker $i$ in $w_0$ at $t_u)$\footnote{Our notation $M(i)$ corresponds to the Kratzerian  notation using set intersection $\cap f_{\text{epistemic}}(w_0,i,t_u)$, where this returns the set of worlds compatible with what it is known in $w_0$ by  $i$.}
\z

Epistemic modality is thus  by default subjective,  and knowledge changes with time. Epistemic modality is therefore parametric to knowledge at $t_u$, as is often  acknowledged  in the literature (see \citealt{portner2009, hacquard2006, hacquard2010, giannakidoumari2016}).  
 
 In the epistemic modal base $M(i)(t_u)(w_0)$, we define Ideal$_{\mathcal{S}}$ as a function over $M(i)(t_u)(w_0)$, still in the spirit of \citet{portner2009}. The output Ideal$_{\mathcal{S}}$  is a subset of $M(i)(t_u)(w_0)$: 

\ea Ideal$_{\mathcal{S}}\,$($M(i)(t_u)(w_0)$) = $\{ w' \in M(i)(t_u)(w_0) : \forall q \in \mathcal{S} \langle w' \in q \rangle \}$
\z

So defined, Ideal$_{\mathcal{S}}$ delivers the worlds in the modal base in which all the propositions in $\mathcal{S}$ are true. $\mathcal{S}$ is a set of propositions that corresponds to common ground norms.\footnote{Since only those worlds are considered in which $all$ the propositions in $\mathcal{S}$ are true, the function Ideal$_{\mathcal{S}}$ determines a cut-off point.} The truth condition for \textsc{MUST} says that $p$ is true in the Ideal$_{\mathcal{S}}$ set of $M(i)$. We assume that, by default, $M(i)$ is projected at the time of utterance in the actual world: 
 
\ea Given a set Ideal$_{\mathcal{S}}$ and the utterance time $t_u$, \\
$\den{\text{\textsc{MUST} (PRES (}p\text{))}}{M,i,\mathcal{S}}$ is defined only if $M(i)$ is nonveridical and  is partitioned into Ideal$_{\mathcal{S}}$ and $\neg$Ideal$_{\mathcal{S}}$ worlds. If defined,\\ 
$\den{\text{\textsc{MUST} (PRES (}p\text{))}}{M,i,\mathcal{S}}$ = 1  iff  $\forall w' \in \, \text{Ideal}_{\mathcal{S}}:   p(w',t_u)$
\z

We will now postulate that Ideal$_{\mathcal{S}}$ and $\neg$Ideal$_{\mathcal{S}}$ worlds are ranked according to an ordering source $\mathcal{O}$, which, in the case of \textsc{MUST}, is introduced by a silent adverb, defined as in \xref{highconf}:\footnote{For more discussion on future shifting and the nonveridicality of \textsc{MUST}, see GM for details. The key observation here is that only in Ideal$_{\mathcal{S}}$ worlds is $p$ true.}


\ea  \label{highconf} For any Ideal$_{\mathcal{S}}$,\\ $\den{\emptyset}{\mathcal{O},\text{M},i,\mathcal{S}}$ = $\lambda q$.  Ideal$_{\mathcal{S}}$ is a better possibility with respect to $\neg$Ideal$_{\mathcal{S}}$ relative to $M(i)$  and $\mathcal{O}$ \& $q$ 
\z

$\emptyset$ is a covert adverb that provides a ``meta-evaluation'' that compares   Ideal$_{\mathcal{S}}$ to its complement in $M(i)$. Ideal$_{\mathcal{S}}$ worlds are privileged by the ranking function $\mathcal{O}$ which ranks the Ideal worlds as better possibilities (in the sense of \citealt{portner2009} and \citealt{kratzer1986}). 



The $\mathcal{O}$ function manifests itself in the form of an adverb or a particle.  


\ea \label{ex:john:sick}
\ea John might be sick perhaps. \label{advonly}
\ex John must probably/definitely be sick. \label{verbonly}
\ex John must be sick. \label{both}
\z
\z


Such structures are characterized as \textit{modal spread} in \citet{giannakidoumari2018b}, and the syntax is called the \textit{modal skeleton}. The above are equivalent semantically in our system, with the example in \xref{both} containing the default null adverb. The modal skeleton always contains the metaevaluation, but with possibility modals $\mathcal{O}$ is empty, hence the equilibrium:



\ea 
\begin{forest}
fairly nice empty nodes, for tree={inner sep=0, l=0}
[ModalP [{$\mathcal{O}$:Adverb/particle}] [ModalP [ [ [Must] [$M(i)$] ] [$\mathcal{S}$] ] [TP] ] ]
\end{forest}
\z


In the GM system, modal spread is the default modal structure,  $\mathcal{O}$  being always part of it. With \textsc{MUST}, the null adverb is akin to \textit{probably} which  induces the intrinsic positive bias of \textsc{MUST} favoring $p$ worlds as better possibilities \xref{ex:john:sick}. Non-biased possibility modals  differ from \textsc{MUST} in that they  convey equilibrium, which is now defined as follows:

 
 

\ea Nonveridical equilibrium (\citealt{giannakidoumari2018b, giannakidoumari2021a, giannakidoumari2021b}) \label{upeq}
\sn A  partitioned ($p$ and $\neg p$) space $M(i)$ is in nonveridical equilibrium if  the ordering $\mathcal{O}$ is empty.
\z


Possibility modals are structures with an empty $\mathcal{O}$. In discussion of the phenomenon of reflection later where we have possibility adverbials in questions, we will see that $\mathcal{O}$ can also induce widening in the modal base in which case we end up with harder to answer or unanswerable questions.

In sum,  the epistemic nonveridical structure involves two ingredients: (i) a partitioned modal base $M(i)$, and (ii) a metaevaluation $\mathcal{O}$ that is either empty (possibility, neutral), or ranks the Ideal$_{\mathcal{S}}$ worlds as better possibilities than non-Ideal$_{\mathcal{S}}$ worlds (bias inducing modals). The preference for higher ranking of Ideal$_{\mathcal{S}}$ is lexically specified: necessity and possibility modals differ lexically in that higher ranking of Ideal$_{\mathcal{S}}$ is only a feature of the former. The availability of $\mathcal{O}$ ranking is not limited to modal expressions but we expect to find it whenever we have a partitioned nonveridical ($p, \neg p$) structure as is the case with questions.
 
We now proceed to show how this analytical framework explains positive and negative bias in questions. We start with \textsc{REALLY}. Our basic idea is that the presence of bias in questions indicates, just like with modal verbs, the presence of a ranking function $\mathcal{O}$. The expressor of $\mathcal{O}$ can be an adverb such as \textit{really}, as we argue next --  but negation can also function as $\mathcal{O}$. The commonality between questions and modals is that they both convey partitioned nonveridical epistemic spaces  which can be ranked and have bias, or can be left unranked in the egalitarian state of equilibrium. Negation can function as $\mathcal{O}$ when it is used as a focus sensitive operator.

    
\section{Deriving negative bias with \textsc{REALLY}-questions} \label{sec:02:3}
    
    
 Questions with the adverb \textit{really}, as mentioned earlier, are known tn the literature to express negative bias. We give below the English example and its Greek and Italian equivalents:

\ea Is Ernie really a communist? 
\ex \gll Ine o Ernie \textit{pragmati/st' alithia} kommunistis? \\
\textsc{cop.3sg} the Ernie truly/really a.communist\\
\glt `Is Ernie really a communist?'
\ex \gll Ernie \`e \textit{davvero} un comunista? \\
Ernie \textsc{cop.3sg} really \textsc{indef} communist \\
\glt `Is Ernie really a communist?'
\z

By using \textsc{REALLY} (\textit{really}, \textit{pragmati}, \textit{st'alithia}, \textit{davvero}) the speaker seems to be genuinely interested in knowing  if Ernie is a communist, like when one utters \textit{I want to know whether Ernie is really a communist}. In addition, as we said at the beginning, in choosing to use \textsc{REALLY} the speaker intends to show that she has negative bias, i.e., she has some doubt that Ernie is a communist (see also a recent discussion of the \textsc{REALLY} effect in \citealt{Bill2020} and more references therein). With a \textsc{REALLY} question, as a reviewer suggests,  the speaker raises the stakes for the addressee to give a truthful answer because her doubt about the asked content is enhanced. The negative bias of \textit{wirklich} (the German counterpart of \textit{really}) is experimentally confirmed in \citet{liu2021} in questions and conditionals --  and it appeared to be one of the most solid experimental findings. \citet{romero2004} propose an influential account of the adverb \textit{really} as \textsc{verum} which we consider after we first lay out our proposal.  \footnote{The \textit{really} effect occurs also in constituent questions, as in \textit{Who did really go to Berlin?}, alternative questions (e.g. \textit{Did he really go to Berlin, or not?}), and in declarative questions  e.g. \textit{He really went to Berlin?}. }



Bias, as we said earlier, is the destruction of the egalitarian state of equilibrium in a positive or negative direction because the speaker has some priors (in the sense of pre-questioning beliefs and expectations) that make them think that \textit{Ernie is not a communist} is a better possibility than \textit{Ernie is a communist}.  It is for this reason that the stakes are higher for the answerer to show that he is. Importantly, the negative bias is not a belief that the prejacent is not true; it is rather, as with \textsc{MUST}, an indication that the speaker is considering a nonveridical partition and compares the likelihoods of $p$ and $\neg p$. Unlike with \textsc{MUST}, the favored proposition now seems to be the negative one. 
  
Bias is anchored to the speaker, who ranks the two possibilities prior to asking a question; it can thus best be understood as a speaker-anchored definedness condition on asking the question. For \citet{stalnaker1978}, and this is important to note, presuppositions are preconditions that need to be satisfied before the common ground can be updated; hence they are requirements on the speaker's knowledge, not on the common ground.\footnote{\citet{von2008} and subsequent literature in the context change potential tradition think of the presuppositional component of the meaning of a sentence as being a requirement on the information state it is used to update. ``Since the information state a sentence is used to update in the ideal case is the common ground, the presuppositional requirements are imposed on the common ground" (\citealt{von2008}: 5). In effect, then, definedness conditions can be understood as common ground presuppositions, or as in the Stalnakerian sense which effectively makes no distinction between felicity conditions of the speaker and common ground.} We ask here how bias comes about with \textsc{REALLY} expressions.  Part of the puzzle is also that \textsc{REALLY} expressions, while positive, they bias towards a negative answer.    
   


Let us observe further that  the negative bias of \textsc{REALLY} holds across nonveridical constructions, including conditionals (\ref{cond}) and imperatives (\ref{imp}). 

\ea \ea If John really studied very hard, he will pass the exam easily. \label{cond} \\ 
Speaker's assumption prior to the question:  John most likely did not study very hard. 
\ex Really, close the door! \label{imp} \\ 
Speaker's assumption prior to the question: the addressee resists closing the door. 
\z
\z

 None of the existing accounts explains this generalized \textsc{REALLY} effect. The prior assumptions could be the speaker's private assumptions but they could also be contextual, as the two types of information intersect. In some cases the prior could be merely a contextual challenge prior to assertion, as in the following example:

\ea A: I know, John told you that he will study very hard for the exam, but I am still worried. \\
B: If John really studied very hard, he will pass the exam easily. And if he says he studies
hard, I believe him. 
\z


In this example, B is entertaining A's uncertainty about whether John studied hard, while also asserting his own favoring position tilting towards the positive. This is consistent with the nature of bias which is speaker-anchored. (We thank a reviewer for bringing up this case). 

 
 In positive veridical assertions, \textsc{REALLY} again is anaphoric to prior assumptions, but now in a positive direction. Observe the following example from \citet{romero2004}:

\protectedex{
\ea
\ea[?]{I am sure I am tired.}
\ex[]{I really am tired.}
\z
\z
}

Here the speaker makes a positive statement against a background assumption that the speaker is not tired.  As \citet{romero2004} observe,  ``other epistemic certainty expressions'' don't have this property, as can be seen. In our view, the contrast here is further evidence that \textsc{REALLY} is not an epistemic attitude, unlike \textit{I am sure}. Crucially, as a focus adverb,  even in positive sentences \textsc{REALLY} depends on a prior assumption, and it  contrasts in polarity with the assertion of the \textsc{REALLY} statement.  \textit{I am not really tired} presupposes a context where the background assumption is that the speaker likely is tired (see \citealt{liu2021} for discussion and references). 

\subsection{The \textsc{REALLY} effect as metaevaluative ranking}

Our analysis goes as follows. \textsc{REALLY} words are focus adverbials, thus anaphoric to an alternative assumption prior to the assertion or question. In the positive assertion, the alternative contextual assumption seems to be that I am likely not tired. In the question, it seems to be that Ernie is likely a communist. \citet{liu2021} call it \textit{contextual} positive bias:


\ea Is Ernie really a communist?
     \ea contextual positive bias: Ernie is likely a communist.
      \ex speaker negative bias: Ernie not being a communist is more likely that Ernie being a communist.
\z
\z      
      
It appears, therefore that \textsc{REALLY} being a focus adverb requires an alternative contextual assumption of the opposite polarity to make sense, i.e., for it to be a reasonable alternative. In other words, the opposite polarity effect follows straightforwardly from  the anaphoricity of focus and the requirement that the focused statement provide new information with respect to the alternative. 

We start with the basic equilibrium partition for questions which is that of possibility modals. The initial state of possibility and information seeking is as follows for epistemic modal \textsc{MIGHT} (\citealt{giannakidoumari2018b, giannakidoumari2021a, giannakidoumari2021b}).%
 
\begin{exe}
\ex $\den{\emptyset \, \textsc{MIGHT} (\textsc{pres} (p\text{))}}{\mathcal{O},\text{M},i}$ is defined only if
\begin{xlist}
\exi{(i)} $M(i)$ is nonveridical and partitioned into $\{p, \neg p\}$ worlds, and if
\exi{(ii)} $\mathcal{O}$ is empty
\sn $\den{\emptyset \, \textsc{MIGHT} (\textsc{pres}(p\text{))}}{\mathcal{O},\text{M},i}$ = 1 iff $\exists w' [w' \in M(i) \wedge p(w')]$
\end{xlist}
\end{exe}

 
 This gives equilibrium, and as we see the meta-evaluative ranking is empty. The tree for questions is as follows.
 
\ea 
\begin{forest}
fairly nice empty nodes, 
for tree={inner sep=0, l=0}
[CP [\textsc{ques} [whether$_1$] ]  [{NonveridicalP} [{$\mathcal{O}$:$\emptyset$/Adverb/Particle/Modals/...}] [{NonveridicalP} [ [ [{...} [$t_1$] ] [$M(i)$] ] [$\mathcal{S}$] ] [TP] ] ] ]
%\draw[semithick,->] (t)..controls +(south west:5) and +(south:5)..(wh);
%\nodeconnect{whether}{mod}    }
\end{forest}
\z

The modal and question structure are distinguished in that the former has quantificational force and truth conditions, but the question does not have truth conditions. 
   
\begin{exe}
\ex $\den{\text{\textsc{ques}} \, \emptyset  \, (\textsc{pres} (p))}{\mathcal{O},\text{M},i}$ is defined only if
\begin{xlist}
\exi{(i)} $M(i)$ is nonveridical and partitioned into  $\{p, \neg p\}$ worlds, and if
\exi{(ii)} $\mathcal{O}$ is empty
\sn $\den{\text{\textsc{ques}} \,  \emptyset  \, (\textsc{pres} (p\text{))}}{\mathcal{O},\text{M},i}$ = $\{p, \neg p\}$
\end{xlist}
\end{exe}

\textsc{REALLY} only has presuppositional content: it introduces the ranking function $\mathcal{O}$ which now says that $\neg p$ is a better possibility than $p$:

\ea Is Ernie really a communist ? 
\ex 
\begin{forest}
fairly nice empty nodes, 
for tree={inner sep=0, l=0}
[CP [\textsc{ques} [whether$_1$] ]  [{NonveridicalP} [{$\mathcal{O}$:REALLY}] [{NonveridicalP} [ [ [{...} [$t_1$] ] [$M(i)$] ] [$\mathcal{S}$] ] [TP] ] ] ]
%\nodeconnect{whether}{mod}  }
\end{forest}
  \z

\begin{exe}
\ex $\den{\textsc{ques}\ \textsc{REALLY}\  (\textsc{pres} (p))}{\mathcal{O},\text{M},i}$ is defined only if
\begin{xlist}
\exi{(i)} the modal base $M(i)$  is nonveridical and partitioned into  $\{p, \neg p\}$ worlds.
\exi{(ii)} $\neg p$ worlds are {better possibilities} than $p$ worlds
\sn $\den{\textsc{ques}\ \textsc{REALLY}\  (\textsc{pres} (p))}{\mathcal{O},\text{M},i}$ = $\{p, \neg p\}$ 
\end{xlist}
\end{exe}

We assume the structure above for both main and embedded questions. The question still denotes  the nonveridical partition $p$ and $\neg p$ --  but the worlds are now ranked by the speaker, and the negative worlds are judged to be better possibilities by the speaker upon asking the question. That the negative worlds are better possibilities means that the answer to the question will more likely fall in the negative space of the partition. In other words, prior to the question, the speaker considers $\neg p$ more likely. This captures the negative bias accurately both in terms of it being a precondition on using \textsc{REALLY} and in terms of the contribution of \textsc{REALLY} being speaker-driven. The ranking is driven by the speaker's priors, as we said including beliefs, but \textsc{REALLY} is not itself an attitude of belief. \textsc{REALLY} does not have truth conditional (assertive)  contribution in questions because questions do not have truth conditions and do not assert. It is unclear to us whether \textsc{REALLY} has assertive content even in assertions: \textit{I am really tired} seems to simply settle the statement (contested in the prior context, we we noted earlier) in the positive, but whether there is additional contribution in the assertion, e.g., a degree reading, is an open question that we will not address here.

\subsection{Why \textsc{verum} doesn't get things right}


Our analysis above differs from the well-known account of \citet{romero2004} in substantial ways. \citet{romero2004}  acknowledges the epistemic root of bias, but makes a number of assumptions that are empirically unjustified thus failing to properly capture the nature of bias. Firstly, it posits that the bias is an `epistemic implicature'. We have shown that bias is stronger than that, it is a definedness condition on the question, a precondition on its use, not a post-condition as an implicature is. The bias, crucially, cannot be cancelled by the same speaker:

\ea Is Ernie really a communist? \#I think so. 
\z


If bias were merely an implicature, the positive continuation  above should be fine. 
 
Secondly, while claiming that bias is an epistemic implicature,  \citet{romero2004} propose a semantics for \textsc{REALLY}-as-\textsc{verum} that is too strong. The claim is that \textsc{REALLY} is the \textsc{verum} operator: \textsc{verum} ``is coming from the lexical item \textit{really}" (\citealt{romero2004}: 641). \citet{romero2004} offer two versions of \textsc{verum} both rendering it an attitude of certainty or subcase thereof. 

In one version, \textsc{verum} is a a veridical operator akin to knowing or believing; \citet{romero2004} call it  ``the run-of-the-mill epistemic operator denotation [...], where $x$ is a free variable whose value is contextually identified with the addressee (or with the individual sum of the addressee and the speaker)": 


\ea  $\den{\textsc{verum}_i}{gx/i}$ =  $\den{\text{really}_i}{gx/i}$ = $\llbracket\text{be sure}\rrbracket(\den{i}{gx/i})$  = \\ $\lambda p _{{\langle}s,t{\rangle}} \lambda w.\forall w' \in \text{Epi}_x (w) [p(w') = 1]$ \jambox*{((40) in \citealt{romero2004})}
\z

The function defined here is, according to \citet{romero2004}, ``the correct denotation for straightforward epistemic expressions like \textit{be sure, be certain}.   But note that, though \textit{really} or \textsc{verum} is often epistemically flavored, it is not interchangeable with pure epistemic expressions like ``\textit{be sure}"" (\citealt{romero2004}: 626). We agree, it is not interchangeable with epistemic or doxastic attitudes, and this is because \textsc{REALLY} is not an epistemic or doxastic attitude. Had \textsc{verum} conveyed the certainty or knowledge of $p$,  it should have been unusable in questions which by definition presuppose uncertainty: if the speaker, or speaker and addressee jointly know or believe that $p$ is true, then why bother asking the question? \textsc{verum} cannot be a veridical operator as above, it cannot entail $p$.\footnote{Even if questions ``flip'' epistemics to the addressee (\citealt{eckardt2019}) as in \textit{Are you sure that Ernie is a communist?}, without  an explicit signal to the addressee, the question remains anchored to the speaker.}

\citet{romero2004} acknowledge the problem and admit that ``the intuition is that \textit{really} or \textsc{verum} is used not to assert that the speaker is entirely certain about the truth of $p$, but to assert that the speaker is certain that $p$ should be added to the Common Ground (CG). That is, rather than a purely epistemic, \textit{really} or \textsc{verum} is a \textit{conversational epistemic operator} [emphasis ours]." (\citealt{romero2004}: 627). They then offer a definition ``FOR-SURE-CG$x$'' in example (43), where ``Epi$_x$($w$) is the set of worlds that conform to $x$'s knowledge in $w$, Conv$_x$($w'$) is the set of worlds where all the conversational goals of $x$ in w are fulfilled CG$_{w''}$ is the Common Ground or set of propositions that the speakers assume in $w''$ to be true (\citealt{stalnaker1978, roberts1996})".

\ea $\den{\textsc{verum}_i}{gx/i}$ =  $\den{\text{really}_i}{gx/i}$   = 
\sn $\lambda p _{{\langle}s,t{\rangle}} \lambda w.\forall w' \in \text{Epi}_x (w) [\forall w'' \in \text{Conv}_x(w') [p \in \text{CG}_{w''}]] = \text{FOR-SURE-CG}_{x}$
\jambox*{((43) in \citealt{romero2004})}
\z


Unfortunately, this understanding of \textsc{verum} is just a variant of their entry (40). \textsc{verum} is now driven by common ground assumptions --  but the \textsc{REALLY} bias (and bias more broadly, as we showed)  is tied to the speaker. The speaker may be making assumptions about Ernie being a communist that could be entirely at odds with the hearer's assumptions, thereby raising the stakes for the latter as we mentioned earlier. More importantly,  \textsc{verum} in this new definition continues to be an attitude  (now that the speaker is certain that $p$ should be added to the Common Ground), and as such the problem of conflicting with the question remains. There is no evidence that \textsc{REALLY} has any assertive content in questions --   and in an actual assertion (\textit{I am really tired}) \textsc{REALLY} contributes only a degree meaning and no ``epistemic implicature".

The most important problem with the \textsc{verum} analysis of \textsc{REALLY}, in our view, is that the \textsc{verum} meaning is akin to a veridical attitude itself, but the bias is, as we argued, at the non-at-issue level  (see also \citealt{liu2012, liu2021}), and it is not an attitude. \textsc{REALLY} is not a propositional attitude: if, in asking \textsc{REALLY} $p$?,  the speaker believed  $p$ or were certain that $p$ should be added to the common ground, then why ask the question? It should be pointless. As an attitude, \textsc{REALLY} asserts content, but this is too strong. In our analysis, \textsc{REALLY} is an adverb that affects the ranking: it ranks the two possibilities given in the nonveridical space, the negative being more likely. \textsc{REALLY} thus turns out to be the dual of \textsc{MUST}. The only additional assumption that we make is that the modal ranking can be realized by expressions that are not strictly speaking modal, such as the adverbs meaning \textsc{REALLY} --   and as we argue next, negation. But this assumption is not extraordinary as the possibility of ranking exists in all nonveridical contexts which can be ranked and have bias, or be left unranked in the egalitarian state of equilibrium.


 
 \section{Focus negation and bias } \label{sec:02:4}



The bias arising with so called high negation (HNQ)  also relies on the speaker's ranking. Since \citet{Ladd1981} the observation has been that  high negation indicates that the questioner believes  that $p$ is true or likely:

\ea 
\ea \gll N'est-il pas la maison? Je pensais qu'il l'était \# Je ne pensais pas qu'il l'était. \\
{not.is} \textsc{neg} \textsc{def.f} house \textsc{1sg} think\textsc{.pst.1sg} that.he {was} {} I not think\textsc{.pst.1sg} \textsc{neg} that.he was \\
\glt `Isn't he home? I thought he was / \#I did not think he was.'
\ex Isn't he home? I thought he was / \#I did not think he was.  \label{ex:02:Xaprim}
\ex \gll Dhen ine spiti? \\
Not is home? \\ \jambox*{(Greek)} 
\glt `Isn't he at home?'
\z
\z


 These questions expect a confirmatory answer, which should be positive (see also \citealt{buring2000, romero2004, goodhue2018}); \textit{a contrario} \citet{krifka2015SALT} states that with negated questions, the speaker checks whether the addressee is ready to express lack of commitment  towards the proposition, which is compatible with expecting a positive answer.  In English,  syntactically high negation can be interpreted in a high or low position. 

\ea Isn't John at home? 
\ea High negation: Isn't John at home? \jambox*{(\textsc{neg} > \textsc{ques})}
\ex Low negation: Is John not at home? \jambox*{(\textsc{ques} > \textsc{neg})}
\z
\z


Across languages the ambiguity does not always arise, and high negation can only be interpreted in a high position. The two readings have different intonational patterns, and only high \textsc{neg} bears focus as the following French example shows:
\ea \label{frNBQfrench} \gll Il n'est pas venu? \\
He {not.is} \textsc{neg} come\\ \jambox*{French}
\glt `Hasn't he arrived?'
\ea Intended interpretation:  It is not true that he has arrived?
\ex Impossible interpretation: Is it true that he has not arrived? 
\z
\z

This data can be replicated for Greek, with focused negation \textit{dhen}:


\ea \label{frNBQgreek}
\gll DHEN irthe? \\
  {not }  came\textsc{.3sg}\\
\glt `Hasn't he arrived?'
\ea Intended interpretation:  It is not true that he has arrived?
\ex Impossible interpretation: Is it true that he has not arrived? 
\z 
\z 


 Current accounts agree on a number of facts that hold uniformly across languages. High negation is focus sensitive (henceforth Focus-\textsc{neg}) and (i) triggers the speaker's expectation that $p$ is true \xref{ex:02:Xabis}; and in the question (ii) it renders the positive answer more likely:


\ea \label{ex:02:Xabis} Isn't he home? I thought he was / \#I did not think he was.
\z





A variety of approaches to negative biased questions have been developed:
(i) \textsc{verum} operator accounts (\citealt{romero2004}; see \citealt{repp2013} for \textsc{falsum}).
(ii) Double speech-act accounts \citep{Reese2007}.
(iii) Commitment (\citealt{krifka2015SALT}) and
(iv) decision based accounts \citet{van2003};
(v) Evidence-based accounts (\citealt{buring2000, sudo2013, roelofsen2015, goodhue2018}), stemming from the work of \citet{Ladd1981}. It would be  impossible  to render justice here to the whole literature, and for the purpose of this paper we only focus here on the interrelations between \textsc{MUST}, Focus-\textsc{neg} and bias; for extended recent discussions see \citet{krifka2017, larrivee2022}.

\citet{larrivee2022} establish a series of correlations between high Focus-\textsc{neg}  in questions and \textsc{MUST}, pretty much in the sprit we outlined for \textsc{REALLY}. They argue that  Focus-\textsc{neg} and \textsc{MUST} share important similarities, namely (i) they are nonveridical, and (ii) they convey the speaker's prior that $p$ is likely. However, they are in complementary distribution in evidential contexts. \textsc{MUST} is felicitous in contexts that are compatible with $p$, whereas negative questions are felicitous in contexts that are \textit{in}compatible with $p$ (called `negative evidence' by \citealt{buring2000}, see also \citealt{sudo2013}). The following examples are from \citet{larrivee2022}. 

\ea John looked pretty happy coming back from school.
\ea -- \gll Il doit avoir réussi son examen de math.\\
he must have succeded his exam of math \\
\glt `He must have passed the big maths test.'
\ex -- \gll \#N'a-t-il pas réussi son examen de math?\\
not.have-\textsc{1sg-}he \textsc{neg} succeeded his exam of math \\
\glt `Didn't he pass the big maths test?'
\label{happypos}
\z
\ex John looked pretty down coming back from school. 
\ea -- \#\gll Il doit avoir réussi son examen de math. \\
he must have succeded his exam of math \\
\glt `He must have passed the big maths test.'
\ex -- \gll N'a-t-il pas réussi son examen de math? \\ 
not.have\textsc{-1sg-}he \textsc{neg} succeeded his exam of math \\
\glt `Didn't he pass the big maths test?' \label{sadneg}
\z
\z


We argue here that the high  Focus-\textsc{neg} introduces a metaevaluation which, like \textsc{MUST}, produces positive bias. The bias is positive because the negation particle functions here as a modal adverb in focus. The negation particle in a question is forced to this meaning for two reasons. First, just like with \textsc{REALLY}, bias inducing negation bears focus and functions as a focus particle. (Low negation, by contrast, is de-accented and functions normally as the expected proposition negating function). Second, being a focus particle means that (i)  there is a set of alternatives in the discourse, and (ii) in questions, the relevant alternative will have to be  $\neg p$. 

If $\neg p$ is the alternative Focus-neg, which needs to express new information, cannot not function as negation. Focus-\textsc{neg} thus has no contribution other than  the ranking presupposition. 

As we noted for \textsc{REALLY}, the focus adverb (here, \textsc{neg}) requires an alternative contextual assumption of the opposite polarity to make sense, i.e., for it to be a reasonable alternative. In other words, the opposite polarity effect follows from  the anaphoricity of focus and the requirement that the focused statement provide new information with respect to the alternative. 


\begin{exe}
\ex $\den{\text{\textsc{ques} Focus-\textsc{neg} (PRES (}p\text{))}}{\mathcal{O},\text{M},i}$ is defined only if
\begin{xlist}
\exi{(i)} the modal base $M(i)$  is nonveridical and partitioned into  $\{p, \neg p\}$ worlds.
\exi{(ii)} $p$ worlds are more preferable than $\neg p$ worlds
\sn $\den{\text{ \textsc{ques} Focus-\textsc{neg}  (PRES (}p\text{))}}{\mathcal{O},\text{M},i}$ = $\{p, \neg p\}$ 
\end{xlist}
\end{exe}

Since a question does not assert $\neg p$ the contribution of Focus-\textsc{neg} as $\mathcal{O}$ arises as a meaning reanalysis. Just like with \textsc{REALLY}, we can assume that \textsc{neg} is in the $\mathcal{O}$ position in the question skeleton --  recall that that was also  the position of \textsc{REALLY}:

\ea 
\begin{forest}
fairly nice empty nodes, 
for tree={inner sep=0, l=0}
[CP [\textsc{ques} [whether$_1$] ]  [{NonveridicalP} [{$\mathcal{O}$:Focus-\textsc{neg}}] [{NonveridicalP} [ [ [{...} [$t_1$] ] [$M(i)$] ] [$\mathcal{S}$ ] ] [TP] ] ] ]
%\nodeconnect{whether}{mod}  
\end{forest}
\z  




Putting the effect of \textsc{REALLY} and negation together, then, we must say that their content is only presuppositional. Thus, our account explains their contribution and the polarity reversal in a very simple way by acknowledging their status as focus operators, and allowing an extended syntactic structure in questions parallel to the  (independently motivated) one for modals with a position for $\mathcal{O}$. No other account that we know of offers an explanation of these facts with such simplicity and no additional \textit{ad hoc} assumptions. 


Before closing, we want to go back to the strength of veridical commitment, and note  that modals and questions mirror each other in terms of strength. In our theory, commitment with modals proceeds according to the following scale: 

\ea \label{VerCom} Scale of veridical commitment (\citealt{giannakidoumari2016, giannakidoumari2018b, giannakidoumari2021a, giannakidoumari2021b})
\sn {\textlangle}$p$, \textsc{MUST} $p$, \textsc{MIGHT} $p$, *{\textrangle};
\sn where $i$ is the speaker, $p$ conveys full commitment of $i$ to $p$; \textsc{MUST} $p$ conveys \textit{partial} commitment of $i$ to $p$, and \textsc{MIGHT} $p$ conveys \textit{trivial} commitment of $i$ to $p$. No modal expression conveys negative commitment. 
\z	 

* in \xref{VerCom} indicates that no epistemic modal can convey negative commitment (see \citealt{ernst2009, homer2015, giannakidoumari2018b} as for how this relates to the neg-raising property of universal epistemic modals).\footnote{Cross-linguistically, attitudes like `doubt' can instead convey negative commitment. As far as we are aware, at least in Indo-European languages, there is no modal verb/auxiliary equivalent.}

With \textsc{REALLY} and Focus-\textsc{neg}, the commitment, which is a presupposition rather than an assertion, proceeds as a mirror image: 

\ea Scale of  commitment in question presuppositions: 
\sn {\textlangle}$p$, FOCUS NEGATION $p$, MIGHT $p$, REALLY $p${\textrangle};
\sn where $i$ is the speaker, $p$ conveys full commitment of $i$ to $p$; HIGH NEGATION $p$ conveys \textit{partial} commitment of $i$ to $p$, MIGHT $p$ conveys \textit{trivial} commitment and REALLY conveys \textit{negative} commitment of $i$ to $p$.
\z

Like \textsc{MUST} and \textsc{MIGHT}, Focus-\textsc{neg} and \textsc{REALLY} questions are nonveridical. Focus-\textsc{neg} and \textsc{MUST} are equivalent in terms of function and commitment in questions, and likewise \textsc{REALLY} contributed ranking but is negatively biased for the reasons we explained earlier. Why modal assertions and questions are mirror images of commitment is a question that we cannot answer fully; but within the GM system the contrast can be simply due to the fact that, in questions, \textsc{REALLY} and negation are forced to function in an alternative way given their opposite polarity anaphoric property due to focus. 


We now complete our exploration of the relation between questions and modality by considering questions where modal particles, including a different kind of negation, are used. The particles, we will show, are equivalent to unbiased possibility modals rather than \textsc{MUST}, and do not rank the two alternative propositions. Rather, they  widen the options and create  vagueness. The effect we observe is an enhancement of equilibrium, not its shrinking (as is the case with bias) --  with the opposite effect of creating more uncertainty.  This operation, which we call `reflection', targets the modal base.



\section{The reflective question: enhancing uncertainty} \label{sec:02:5}
 
In this section, we consider the use of possibility modal particles and verbs in questions, resulting in the question now becoming vague and potentially unanswerable. We call such questions \textit{reflective}. In the literature the term `conjectural' has also been used (see \citealt{littell2010, matthewson2010, eckardt2019, frana2019a} a.o.).  The reflective question is well documented for a number of languages, including Greek, Japanese and Korean (\citealt{kang2018, kang2019}). In discussing it along with bias, we view the two phenomena as pragmatic duals. 

\subsection{Modal particles in questions: The data}

For a long time, it was thought that epistemic modals do not occur in questions \citep{coates1983, drubig2001, jackendoff1972, leech1971, mcdowell1987}. Jackendoff, specifically, claimed that while \textit{may} can either be interpreted deontically or epistemically in a declarative sentence  (\textit{John may leave early tonight}), it can only be interpreted deontically in a question (\textit{May John leave early tonight?}).  Yet \citet{ernst2009} presented examples with modal adverbs in questions (\textit{Is she possibly the murderer?}), and \citet{hacquard2012} offer corpus data with possibility modals in questions. The following examples are from that work:

\ea 
\ea With the owners and the players on opposite sides philosophically and economically, what might they talk about at the next bargaining session?
\ex Might he be blackballed by all institutions of higher learning?
\ex What might the Grizzlies have been like if their leading scorer and rebounder, 6-foot-10 center Brent Smith, had not missed his third straight game because of a sprained ankle?
\z
\z

The authors conclude that epistemic modals of possibility are very natural in questions. 

In Greek, we find possibility modal verbs and particles in questions: the so-called \textit{evaluative subjunctive} (\citealt{giannakidou2016}), and the particles \textit{taxa, mipos, arage} `maybe/possibly', the latter two only used in questions, as we see below with examples from \citet{giannakidou2016} (see earlier discussions in \citealt{giannakidou2009, rouchota1994}):

\ea 
\gll Pjos irthe sto party? \\
Who came\textsc{.3sg} to-the party \\
\glt `Who came to the party?'
\ex 
\gll Pjos $na$ irthe sto party? \\
Who \textsc{subj} came\textsc{.3sg} to-the party \\
\glt `Who might have come to the party?'
\ex
\gll Pjos $bori$ / \/*\textit{prepi} \textit{na} irthe sto party? \label{prepiout} \\
Who  might / must \textsc{subj} came\textsc{.3sg} to-the party \\
\glt `Who might/\#must have come to the party?'
\ex \label{ex:exclude:must}
\gll ($Na$) tou milise (arage/mipos/taxa)?\\
\textsc{subj}V him talked\textsc{-3sg} \textsc{particle} \\
\glt `Might she have talked to him?'
\ex 
\gll Tou milise (\#arage/\#mipos/taxa). \\
 him talked\textsc{-3sg} \textsc{particle} \\
\glt `She talked to him (as-if).' 
\ex
\gll Tou milise?\\
 him talked\textsc{-3sg}\\
\glt `Did she talk to him?'
\z

Here we observe the subjunctive, the possibility modal \textit{bori} and the particles \textit{arage/mipos/taxa} in questions (both polar and wh-questions). As we can see, the particles can spread in a fashion reminiscent of modal spread with modal verbs and adverbs (\textit{He may possibly be here tonight}). The \textsc{MUST} modal is excluded as can be seen in \xref{ex:exclude:must}, indicating that we are not dealing with a bias  related phenomenon.  In the declarative context, only  \textit{taxa} can be used, which means literally `as if, allegedly' and casts doubt on the truth of the proposition is attaches to (see \citealt{giannakidou2022} for more details on the Greek particles; \citealt{ifantidou2001}). The particles correspond to \textit{might}, as indicated in the English translations. 

\citet{giannakidou2016} argues that the particle question differs from the plain one in being open-ended, vague, and primarily self-addressed. It does not require an actual or full answer; for instance, \textit{Pu na evala ta gialia mou?} `Where might I have put my glasses?'  is a question that one poses to herself in a wondering, reflective manner without expectation of knowing the answer.  Bare questions cannot be used this way.  (In fact, first person unmodalized questions are quite uncommon). This open-endedness is found also  in Japanese and Korean discussed recently in \citet{kang2018, kang2019}, and with subjunctive questions in Salish (\quotecite{matthewson2010} `conjectural' questions). English data from \citet{ernst2009, hacquard2012} illustrate the same reflective character for English questions with possibility modals:

\ea Is she possibly/*probably a spy?
\ex Might/*must she be a spy?
\z

These have been described as `weaker' questions. \citet{kang2018, kang2019} in their discussion of a similar use of the particle \textit{nka} in Korean note that by using \textit{nka}, the speaker ``reflects on  her own background assumptions and is not simply requesting information from the addressee". \textit{Nka}-questions, they argue, are \textit{feigned monologues}, i.e., the speaker says something as if it were a monologue without expecting an answer necessarily. Because of the monologic nature of the utterance, it does not necessarily obligate the hearer to respond, and while there may be differences between the particles and across languages, these observations hold also for the Greek and English particle questions above. 

The self-reflective character of particle questions is further evidenced by the fact that directly addressing the hearer is odd. 	
 
\ea[\#]{
\gll Ti \textit{na} efages arage xthes?  \\
what \textsc{subj} ate\textsc{.2sg} araje xthes\\
 \glt `What might you have eaten yesterday?'}
\ex[]{
\gll  Ti efages xthes?  \\
what ate\textsc{.2sg} xthes\\
\glt  `What did you eat yesterday?'}
\ex[\#]{
\gll  Na efages araje xthes?  \\
 \textsc{subj} ate\textsc{.2sg} araje xthes\\
\glt `Might you have eaten yesterday?'}
\ex[]{
 \gll Efages xthes?  \\
ate\textsc{.2sg} xthes\\
 \glt `Did you eat yesterday?'}
 \z
 
 
    
 For the Greek particles, \citet{giannakidou2022} calls this `anti-addressee' effect. Cross-linguistically, the effect  is also observed in Korean (\citealt{kang2019}). This effect has been discussed in the literature by appealing to constraints on the  `interrogative flip', that is to say the phenomenon whereby questions containing an evidential target the addressee mental state (a.o. \citealt{bhadra2017}). Extending the view that evidentials enhance the interrogative flip to epistemic modals, one could argue (see e.g. \citealt{eckardt2019})  that what the modal or particle is questioning is the addressee's knowledge, and that a modal question presupposes lack of knowledge on the part of the addressee. 
 
\citet{mari2021} shows that existential epistemic modal questions \xref{potereques} -- which include Italian future questions \xxref{fut1}{fut2} -- are by default self addressed (see also \citealt{eckardt2019}; \textit{pace} \citealt{IppolitoFarkas2021}).
   
\ea \label{potereques}
   \gll Dove possono essere i miei occhiali ? \\ 
   where might be the my glasses \\
   \glt `Where might my glasses be?' 
   \ex \label{fut1}
   \gll Dove saranno i miei occhiali ? \\ 
   where be\textsc{.fut.3pl} the my glasses \\
   \glt `Where might my glasses be?'
\ex \label{fut2} 
\gll Sar\`a a casa? \\ 
be.\textsc{fut.3pl} at home\\ 
\glt `Might he be home?'
\z


Such questions are akin to questions with \textit{forse} which is the possibility adverb `maybe' in Italian:\footnote{The Italian future has been considered as an epistemic modal (see \citealt{pietrandrea2005, mari2009, giannakidoumari2018a, baranzini2019}; \textit{a contrario}, \citealt{mari2010, frana2019a}; \citealt{eckardt2019} for Italian future as an evidential.) The German modal particle \textit{wohl} is also reported to have similar use of adding a speculative component (the example below is from \citealt{zimmermann2011} with his translation). \citet{zimmermann2011} writes: ``The question above is not about whether or not Hans has invited Mary, but by using \textit{wohl} the speaker indicates her awareness that the addressee may not be fully committed to her answer." (\citealt[2020]{zimmermann2011}):

\ea \gll Hat Hans \textit{wohl} Maria eingeladen? \\
has Hans \textsc{prt} Mary invited\\
\glt 'What do you reckon: Has Hans invited Mary?'
\z
} 

\ea 
\ea
\gll \`E a casa? \\
Is at home \\
\glt `Is he at home?'
\ex 
\gll 
\`E forse a casa? \\
Is maybe at home \\
\glt `Is he maybe at home?'
\z
\z


This way of understanding the impossibility to be addressee-oriented needs to postulate a device to undo the flip and render the question reflective when needed (see discussion in \citealt{frana2019a} and \citealt{mari2021}). Our view that these questions are, to begin with, reflective explains right from the bat the impossibility of being addressee-addressed without needing to stipulate any device to undo a putative flip.  

Characteristically, the Greek negative particle \textit{mi(n)} can also be used in questions with the same reflective flavor (\citealt{chatzopoulou2019}):

\ea
\gll Min eidate ton Jani?\\
\textsc{neg} saw\textsc{.2pl} the John\\
\glt `Did you maybe see John?'
\z


This use of a negative particle, which we can call Modal-\textsc{neg}, remains productive in literary and other registers. \textit{Mi(n)} is the negation of non-indicative contexts in Greek (\citealt{giannakidou1997, Veloudis1980}), \textit{dhen} being the negation in indicative sentences. The \textit{min} negation is the semantic allomorph appearing in subjunctive contexts:

\ea{
\gll \textit{*Min/Dhen} eidate ton Jani.\\
\textsc{neg} saw\textsc{.2pl} the John\\
\glt `You did not see John.'}
\ex{
\gll Na \textit{min/*dhen } deite ton Jani.\\
\textsc{subj} \textsc{neg} see\textsc{.2pl} the John\\
\glt `Do not see John!'}
\z


The particle \textit{mipos}, emerges from a historical path that fused the negation \textit{mi}  with the indicative complementizer \textit{pos}.  Crucially, the reflective negation \textit{min} is not used for bias:


\ea
\gll Dhen/*Min eidate ton Jani?\\
\textsc{neg} saw\textsc{.2pl} the John\\
\glt `Didn't  you see John?'
\ex
\gll  Idate ton Jani, etsi dhen/*min einai? \\
saw\textsc{.2pl} the John, so not is \\
\glt `You saw John, didn't you?'
\z

So, in Greek, there is a bias-producing focus negation \textit{dhen}, and a reflective negation \textit{min}. The fact that negation behaves like a modal particle speaks again to our earlier point: a seemingly non modal element can acquire a modal function in questions because, these lacking truth conditions, it cannot function in the canonical way.\footnote{Salish employs negation in questions too; the following example is from \citet{matthewson2022}:

\ea A: \gll 
\textit{Nee=hl} gwila-n=aa? \\
\textsc{neg=CN} blanket\textsc{=2sg.ii=Q}\\ 
\glt `Do you have a blanket?'
\z

This is a neutral question according to Matthewson, hence this negation seems more akin to a question particle. }

Questions with possibility modals are weakened versions of the information-seeking question --  weaker in that they involve more uncertainty. Notice that  they  can be continued by \textit{Who knows?}, unlike regular information questions:
 
 
\ea A: 
\gll Pjos (arage) na irthe sto party? Pjos kseri! \\
Who \hphantom{(}arage-particle\textsc{.subj} V came\textsc{-3sg} to-the party  who knows\\
\glt `Who might have come to the party?  Who knows!'
\ex
\gll Sar\`a a casa? Chi lo sa!\\ 
be\textsc{.fut.3sg} at home who that knows \\ 
\glt `Might he be home? Who knows !'
 \z

We can thus see that possibility modals  (verbs, adverbs, and particles) create a vague question with more uncertainty,  harder or even impossible to answer. The intent of the speaker, when choosing to use the possibility modal, is not simply to seek information from an addressee, but rather to signal that the question is hard to answer,  that the hearer may not be an authority on giving a true answer, or that there is no addresse at all. Reflections, as we noted earlier, can actually be incompatible with the second person  (what we called anti-addressee effect). And, as \citet{kang2019} observe, they have the flavor of a monologue in being directed towards the speaker herself. 

Let us now proceed to capture the effect of the epistemic modal in the framework we have developed. 

\subsection{Analysis: modal particles widen the modal base}

The first analysis we know that addresses the use of of possibility modals in questions is \citet{giannakidoumari2016}. They argued that, in Greek, possibility modals and the subjunctive are equivalent, and that, for the \textsc{MIGHT} question, the answer set contains modalized propositions:

\ea
$\den{\mbox{Who came to the party?}}{ }$ = \{Bill came to the party, Marina came to the party, Ariadne came to the party, Nicholas came to the party, ...\} \label{17}
\ex
$\den{\mbox{Who might have come to the party?}}{ }$ = \{it is possible that Bill came to the party, it is possible that Marina came to the party, it is possible that  Ariadne came to the party,  it is possible that Nicholas came to the party, ...\} \label{18}
\z

This analysis says that the possibility question is still an information seeking question, but instead of asking about $p$, it asks about POSSIBLY $p$. While a question with \textsc{MIGHT} indeed may ask about the possibility of $p$  --  like, for instance \textit{Is it possible that Ariadne ate?}  --  the reflective effect is more than that.

In our current framework  which uses  metaevaluation, we will argue that the possibility  expression enlarges the spectrum of the possibilities. Possibility expressions \textit{widen} the set of the possibilities considered  attracting attention to $p$, but seeking $p$ in a larger set. 

\begin{exe}
\ex Ernie is possibly a communist? 
\ex 
\begin{forest}
fairly nice empty nodes, 
for tree={inner sep=0, l=0}
[CP [\textsc{ques} [whether$_1$] ]  [{NonveridicalP} [{$\mathcal{O}$:POSSIBLY}] [{NonveridicalP} [ [ [{...} [{$t_1$}] ] [$M(i)$] ] [$\mathcal{S}$ ] ] [TP] ] ] ]
%\nodeconnect{whether}{mod}  
\end{forest}
\ex
$\den{\textsc{ques}\ \textsc{POSSIBLY} (\textsc{pres} (p))}{\mathcal{O},\text{M},i}$ is defined only if
\begin{xlist}
\exi{(i)} $M(i)$ is nonveridical and partitioned into  $\{p, \neg p\}$ worlds.
\exi{(ii)} $\cap \mathcal{O} \supset M(i)$ 
\sn $\den{\textsc{ques}\ \textsc{POSSIBLY} (\textsc{pres} (p))}{\mathcal{O},\text{M},i}$ = $\{p, \neg p\}$
\end{xlist}
\end{exe}


The set of possibilities extends beyond the modal base thus making it harder to think of what would be a ``correct'' answer. This accounts for why reflection  questions seem open-ended, vague,  and potentially lacking answers. The epistemic state of the speaker entertains a broader set of potential answers, not a narrower one as in the case bias were a potential answer is favored by the speaker's prior assumptions.  The category of reflective questions thus illustrates that an interrogative can be manipulated away from the canonical information-seeking mode but not necessarily producing bias. The effect of widening the modal base $\cap \mathcal{O} \supset M(i)$  is again presuppositional.

\section{Conclusions} \label{sec:02:con}


In this paper, we argued that, while polar questions and modal assertions differ in that the first do not have truth conditions, they are deeply similar in  presupposing nonveridical spaces, partitioned into $p$ and its negation. The partition can be indiscriminate, or as we said egalitarian --  in which case we have nonveridical equilibrium. But  the possibility of metaevaluative ($\mathcal{O}$) ranking exists in all nonveridical spaces, hence another commonality between questions and modals  is that while they both start with a core partition between $p$ and $\neg p$, a ranking is always available. When the ranking is contentful, the result is bias.  Focus adverbials such as \textsc{REALLY}, modal adverbials, and negation are overt lexicalizations of the ranking.  The overall analysis offers a novel way of understanding what bias is and how it is derived, capitalizing on a number of independently motivated assumptions about focus and the structure of the nonveridical space. The dichotomy between questions and assertions with modals comes at the discourse level where an interrogative sentence is used as a question (lacking truth conditions), and the declarative as an assertion.

 In terms of what bias actually is, we emphasized that it cannot itself be an attitude (epistemic or otherwise); it is rather a presupposition of metaevaluative ranking.  Our simple idea was that the presence of bias in questions indicates, just like with modal verbs, the presence of a ranking function $\mathcal{O}$. The expressor of $\mathcal{O}$ can be an adverb such as \textit{really} and its crosslinguistic equivalents --  but also negation, which we called Focus-\textsc{neg} and which has no truth conditional contribution in questions when it produces bias. Key to the polarity reversal of bias in questions was the status of the $\mathcal{O}$-contributing expressions (\textsc{REALLY}, negation) as focus operators requiring a contextual alternative of the opposite polarity.
 
 Our analysis  offers, as far as we know, the first unified explanation of the polarity reversal mechanism and makes the welcome prediction that focus will have a polarity reversing effect with other types of focus sensitive expressions.   As a bonus, we have contributed to a deeper understanding of the phenomenon of reflection. Reflection is not just a variant of information seeking. Rather, we identified a number of properties of reflective questions evidencing enhanced uncertainty which we captured as widening.  We showed that reflecting is distinct from bias --  in fact, they are pragmatic duals. Bias relies on ranking, but the role of the modal in a reflective question is to widen the modal base. Overall, the framework of modality we developed in our recent work   --  by dissociating modal force from ranking --  offers a flexible way to understand the deeper relation between asking a question and modalizing, as well as how seemingly un-modal expressions can undertake functions typically associated with modality.
 
\section*{Acknowledgements}

We thank the editors of this volume, Anton Benz, Manfred Krifka and Tue Trinh as well as  the reviewers for their thorough and very helpful comments. This material was presented at a Workshop on \textit{Biased Questions}, and another one on \textit{Multiple Modalities} both at the Humboldt University in Berlin. We want to thank the audiences of these events for very fruitful interactions,  especially Bart Geurts, Mingya Liu, Frank Sode, Carla Umbach and Hedde Zeijlstra for their insights. 
Alda Mari gratefully thanks ANR-17-EURE-0017 FrontCog. 


\printbibliography[heading=subbibliography,notkeyword=this]
\end{document}
