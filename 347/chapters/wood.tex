\documentclass[output=paper]{langscibook}
\ChapterDOI{10.5281/zenodo.8427879}
\author{Jim Wood\affiliation{Yale University}}
\title{Singular \textit{-st} syncretism and featural pied-piping}

\abstract{An often discussed fact about Icelandic dative-nominative constructions
is that nominative objects cannot trigger 1st or 2nd person agreement on the 
finite verb; but when the agreement form is morphologically syncretic with 3rd 
person, the example is judged to improve. What is not often discussed is that the
ameliorative effect of syncretism is stronger when the verb ends in the “middle” 
\textit{-st} morpheme. In this article, I propose that this effect is related to 
another morphological fact about \textit{-st} verbs, namely, that they are always
syncretic across all persons in the singular, but not in the plural. I present a 
syntactic account of this syncretism which captures its morphological properties
and predicts the difference between ameliorative syncreticism when \textit{-st} is 
present and when it is not.}
  
% CUSTOM MACROS
\def\exattr#1{\hfill{} #1}

\tikzset{every tree node/.style={align=center, anchor=north}}

\begin{document}
\SetupAffiliations{mark style=none}
\maketitle

\section{Introduction}\label{woodsecexs}

The Icelandic \sti morpheme is often described as a “middle” or “medio-passive” suffix, though it is acknowledged that \stvs do not comprise a unified class of a certain “voice”. That is, \stvs are a class of verbs bearing a formal resemblance, the \sti morpheme, but from a syntactic perspective, the \stin/non\sti distinction is not analogous to the passive\slash non-passive distinction.  However, there are aspects of the morphosyntax of \stvs which cut across all classes of them, and it is (a subset of) these aspects that are the focus of this paper. More specifically, for all \stvs in all tenses and moods, person distinctions are lost in the singular but not the plural. This syncretism, which will henceforth be referred to as \tit{\sti syncretism}, correlates with a higher acceptability of 1st/2nd person object agreement in dative-nominative (\datnomn) constructions than that found with non\sti syncretism.  

I present an overview of the syntax of \sti put forth in  \citet{wood:refl,Wood2015book} and propose that singular \sti syncretism is derived in the syntax. I then show how the syntactic account of \sti syncretism presented here predicts the kind of improvement seen with 1st and 2nd person singular nominative objects. Crucial to the analysis is the observation that the size of the feature bundle realized as \sti affects the availability of syntactic Agree relations that underlie the syncretism and nominative object agreement. 


\subsection{Syntax and syncretism} 

In a number of reported cases, syntactic constructions can vary in acceptability depending on the availability of syncretic forms. For example, across-the-board (ATB)  movement in Polish is normally only possible when the \textit{wh}-word would have the same morphological case from both conjuncts; but if the different cases happen to be realized with the same morphological form, the result is acceptable (\citealt{citko2005nature,hein2020case}; see also \citealt[fn. 2]{Ximenes:2007vs}). \citet{citko2005nature} proposes that the syntax underlying ATB movement with verbs that assign different cases is fine, but that it fails when the grammar attempts to insert the appropriate case morpheme -- unless the different case forms are morphologically syncretic. 

Many accounts of ameliorative effects of syncretism involve an explanation like this (\citealt{pullum1986phonological,bejar1999multiple,Kratzer:2009jq,Ussery:2009jd,Bjorkman2016}): syncretic forms allow the grammar to realize a syntactic configuration which would otherwise make contradictory demands on the morphology.\footnote{\citet{Savescu:2009al} has a syntactic account of syncretism effects on Romanian clitic order which, like the present one,  involves the intrinsic features of elements in the derivation.}
%where clitic ordering in Romanian is shown to be affected by syncretism. %When direct object and indirect object clitics co-occur, the indirect object clitic usually precedes the direct object clitic.  %\citet{Savescu:2009al} pursues a 
%of these facts which, like the present account, .
%}
Without denying the validity of this kind of explanation (in fact, I will adopt it for certain cases), I will take a different approach to the person syncretism in the singular paradigm of Icelandic \stvsn.

The \sti morpheme, commonly known as the “middle” voice, induces a complete collapse of person distinctions in the singular. An example of this is illustrated in \tabref{woodxes}. Interestingly, along with this syncretism comes an improvement in acceptability of certain \datnom constructions, to be discussed below. I will propose that in this case, both the syncretism itself and the improvement in acceptability are underlain by the syntax, specifically with respect to the size of the feature bundle that is realized as \stin. %If this is correct, then the phenomenon of ameliorative syncretism not only tells us something about the mapping from syntax to morphology, but may in some cases tell us something about the syntax itself.

\begin{table}
 \caption{\label{woodxes}\textit{mylja} `pulverize' -- Present}
    \begin{tabular}{*5{l}} 
    \lsptoprule
         & \multicolumn{2}{c}{Active} & \multicolumn{2}{c}{Middle} \\\cmidrule(lr){2-3}\cmidrule(lr){4-5}
          & \textsc{sg} & \textsc{pl}  & \textsc{sg} & \textsc{pl} \\\midrule
        1 & myl & mylj-um    &         &  mylj-um-st \\
        2 & myl-ur & mylj-ið & myl-st &  mylj-i-st  \\
        3 & myl-ur  & mylj-a &        & mylj-a-st  \\
    \lspbottomrule
    \end{tabular}
\end{table} 



\subsection{Dative-nominative constructions} \label{wooddativ}

Icelandic \datnomn{} constructions exhibit number agreement with 3rd person nominative objects, but cannot agree in person with 1st or 2nd person objects. This holds for verbs which  take dative subjects in the active, as in (\ref{woodboree}), as well as for \datnom constructions which are derived by passivization of a ditransitive, as in (\ref{woodgive}). The significance of the latter is that the properties of \datnom constructions cannot easily be reduced to a special, “quirky” little v selecting for an oblique subject.\largerpage

\ea \label{woodboree}
    \ea[]{
        \gll Henni \tit{höfðu} líkað þeir. \\
            her\dat{} had\gl{3pl} liked they\nom{} \\
        \glt `She had liked them.'}
    \ex[*]{
        \gll Henni \tit{höfðum} líkað við.  \\
            her\dat{} had\gl{1pl} liked we\nom{} \\
        \glt `She had liked us.' \citep[38]{SigurTHsson:1996va}
    }
\z
\ex \label{woodgive}
\ea[]%
%\ex[] {\gll Báðir drengirnir voru gefnir Maríu.  \\
%both boys.the\nom{} were\gl{3pl} given\gl{3pl.m} Mary\dat{} \\
%\glt `Both the boys were given to Mary.'}
%\ex[] {\gll Við vorum gefnir Maríu. \\
%we\nom{} were\gl{1pl} given\gl{3pl.m} Mary\dat{} \\
%\glt `We were given to Mary.'}
{
    \gll Maríu \tit{voru} gefnir báðir drengirnir. \\
        Mary\dat{} were\gl{3pl} given\gl{3pl.m} both boys.the\nom{} \\
    \glt `Mary was given both the boys.'
}
\ex[*]
{
    \gll  Maríu \tit{vorum} gefnir við. \\
    Mary\dat{} were\gl{1pl} given\gl{3pl.m} we\nom{} \\ 
    \glt `We were given to Mary.' \citep[71]{SigurTHsson:1992lj}
}
\z
\z

\begin{sloppypar}
In several approaches to person restrictions on nominative objects, the verb must in some sense agree with both the dative subject and the nominative object  \citep{Boeckx:2000kf,Schutze:2003mh,Koopman:2006zp,SigurTHsson:2008dm,Ussery:2009jd}. Agreement with the dative yields default 3rd person singular agreement (regardless of the actual person/number of the dative), as can be independently verifed by constructions with non-nominative subjects and no nominative object.\end{sloppypar}

\ea 
    \ea[]
    {
        \gll \tit{Hafði} þér ekki leiðst? \\
            had\gl{3sg} you\dat{} not bored \\
        \glt `Were you not bored?' \citep[225]{SigurTHsson:1989dm} %\vspace{-1em}%Sigurðsson (1989:225)}
    }
   
    \ex[]
    {
        \gll \tit{Var} þér boðið í veisluna? \\
            was\gl{3sg} you\dat{} invited to party.the\acc{} \\
        \glt `Were you invited to the party?' \citep[309]{SigurTHsson:1989dm} %\vspace{-1em}%\attr{Sigurðsson (1989:309)}
    }
\z
\z

If the verb agrees with both a dative subject and a non-3rd person object, then there is a feature clash -- the verb must simultaneously be 3rd and 1st/2nd person. However, if the paradigm of a given verb happens to exhibit syncretism for the two forms, the sentence is judged to be improved. The agreement paradigm for \tit{líka} `like' in the past tense has a syncretism between the 1st and 3rd person singular forms, but a distinct form for 2nd person singular (\tabref{tab:wood:2}).\footnote{Since \tit{líka} `like' is an asymmetric dative-nominative verb, where the dative is always the subject, unambiguous 1st/2nd person agreement is generally ungrammatical, so these forms (other than \tsc{1/3sg} \tit{líkaði} and 3\tsc{pl}) \tit{líkuðu} are quite rare; the forms shown are what the agreeing forms would be, based on the general rules of inflection in Icelandic. Einar Freyr Sigurðsson points out to me that these forms are, however, used by many speakers with a more recent, agentive sense of the word \tit{líka}, with a nominative subject, which refers to clicking the “like” button on Facebook.} 

\begin{table}
\caption{\tit{líka} `like'\label{tab:wood:2}}
\begin{tabular}{lll}
 \lsptoprule
 1 & \tit{likaði} & líkuðum   \\
 2 & líkaðir & líkuðuð   \\
 3 & \tit{líkaði} & líkuðu  \\
  \lspbottomrule
 \end{tabular}
\end{table} 

Thus, when a nominative object is 1st person singular, the result is better than when it is 2nd person singular, as shown by the following judgments from \citet{SigurTHsson:1996va}.

\ea \label{woodsing} 
    \judgewidth{??}
    \ea[??]
    {
        \gll Henni líkaði ég. \\
        her\dat{} liked\gl{1/3sg} I\nom{} \\
        \glt \textsc{intended}: ‘She liked me.’
    }
    \ex[\bad]
    {
        \gll Henni líkaðir þú. \\
        her\dat{} liked\gl{2sg} you\gl{sg.nom} \\
        \glt \textsc{intended}: `She liked you.' \citep[33]{SigurTHsson:1996va}
    }
%\attr{Sigurðsson (1996:33)}
    \z
\z

\pagebreak\noindent The claim, then, is that the availability of a form which can express both sets of features allows a way to avoid the feature clash. 

However, it turns out that not all syncretisms are equally ameliorative: if syncretism occurs with the morpheme \stin, in the singular, the ameliorative effect of syncretism is stronger than in other cases of syncretism, and this is not predicted by the analyses outlined above. The data in Table \ref{wood1} from \citet{SigurTHsson:1992lj} show the number of speakers who judged each sentence as “OK” or “?” on the one hand, and “??” or “*” on the other.  

\begin{table}
\caption{Data from \citet[74--76]{SigurTHsson:1992lj}} \label{wood1}
\begin{tabular}{*6{l}}
\lsptoprule
   &           &                    &                & {OK/?} & {??/*} \\\midrule
a. & Henni     & \tit{líkaðir}      & þú.            & 0 & 9 \\ 
   & her\dat{} & liked\gl{2sg}     & you\nom{}      &  &  \\ 
b. & Henni     & \tit{líkaði}       & ég.            & 5 & 4 \\ 
   & her\dat{} & liked\gl{1/3sg}   & I\nom{}        &  &  \\ 
c. & Henni     & \tit{leiddust}     & þið.           & 5 & 4 \\ 
   & her\dat{} & bored\gl{2/3pl}   & you\gl{pl.nom} &   &  \\ 
d. & Henni     & \tit{leiddist}     & ég.            & 8 & 1 \\ 
   & her\dat{} & bored\gl{1/2/3sg} & I\nom{}        &   &\\
\lspbottomrule
\end{tabular}
\end{table}
%\attr{(Data from \citealt[74-76]{SigurTHsson:1992lj})}%Data from Sigurðsson (1992:74-76)}


With \tit{líka} `like', the agreeing form of 2nd person singular is rejected by all speakers, while the syncretic 1st and 3rd person form leads to a split among speakers (see Table \ref{woodweak1}; syncretic forms in italics). The same split is witnessed for the syncretic 2nd and 3rd person plural of the \sti verb \tit{leiðast} `bore'. However, the singular form \tit{leiddist}, which is syncretic across all persons in the singular, is even more improved: only one speaker rejected it outright. 

I will claim that the stronger ameliorative effect of singular \sti syncretism is related to a more general aspect of \sti morphology: the \sti suffix collapses all person distinctions in the singular, and this holds across all inflectional classes, in all tenses and moods, and cannot be due to phonology. 
In the proposed analysis, the presence of \sti prevents the building of the “contradictory” feature bundles which are typically assumed to cause problems in non-syncretic cases. 

\begin{table}
    \caption{Past tense forms of \tit{líka} `like' and \tit{leiðast} `bore'\label{woodweak1}}
    \begin{tabular}{*5{l}}
    \lsptoprule
    & \multicolumn{2}{c}{\tit{líka} `like'} & \multicolumn{2}{c}{\tit{leiðast} `bore'}\\\cmidrule(lr){2-3}\cmidrule(lr){4-5}
    & \tsc{sg} & \tsc{pl}  & \tsc{sg} & \tsc{pl} \\\midrule
    1 & \tit{likaði} & líkuðum  	& \tit{leiddist} & leiddumst \\
    2 & líkaðir      & líkuðuð 		& \tit{leiddist} & \tit{leiddust}  \\
    3 & \tit{líkaði} & líkuðu 		& \tit{leiddist} & \tit{leiddust}  \\
    \lspbottomrule
    \end{tabular}
\end{table}


\subsection{Proposal}

The analysis developed here is basically as follows. Independently of \datnom constructions, the \sti suffix has a Person feature, which I will suggest to be \glf{$-$participant}, but no number feature. This allows it to be merged in an argument position under various conditions. %(see \citealt{wood:middle,wood:refl} for some analysis of \sti argument structure alternations). 
It moves to a clitic position in the inflectional domain lower than the Number (Nm) head (which is lower than Person (Pn)), but higher than verb-phrase-internal arguments. 

The singular syncretism can be understood by adopting \citeauthor{Kratzer:2009jq}'s (2009) proposal that Agree involves $\upphi$-feature union, with the auxiliary assumption that singular number agreement is non-number agreement (see \citealt{Nevins2010:ab}). When Nm establishes an Agree relation with a plural object, Nm takes not only the number features but its other $\upphi$-features as well -- including person. When Pn probes, it has access to these person features only because they have been “pied-piped” past \sti by feature union. They are present on the next inflectional head down, Nm, in line with  \citet{baker2010agreement}. When the object is singular, there is no such pied-piping and Pn can only Agree with \stin. 

The account can then be extended to capture object agreement restrictions in \datnom constructions in a manner very similar to previous analyses (e.g. \citealt{DAlessandro:2003oy,Holmberg:2004gk,Schutze:2003mh,SigurTHsson:2008dm,Ussery:2009jd}). Specifically, feature union builds up “contradictory” $\upphi$-feature bundles, which are highly unacceptable when they correspond to different morphological exponents, but the result improves somewhat when all the features in this bundle are realized by identical exponents. The present account, however, can also explain why \sti can help ameliorate such restrictions more than ordinary syncretism: when there is no featural pied-piping, it allows the syntax to proceed without building up the contradictory feature bundles to begin with. The question for the present account is why such forms are not completely perfect, a question which I will address but not answer. Importantly, the present proposal allows us to understand the three-way distinction between non-syncretic forms, morphologically syncretic forms, and “syntactically” syncretic forms. 


\section{\sti syncretism} 

\subsection{\sti syncretism is meta-paradigmatic and not phonological}

An occasionally noted fact about \stvs is that they are syncretic for person in the singular, but not the plural \parencites[100]{Einarsson:1949xt}[434--440]{Thomson:1987bn}[242]{Anderson:1990sm}[fn. 2]{Taraldsen:1995om}[270]{SigurTHsson:2008dm}. This is odd because usually, when distinctions are collapsed like this, it is in ``marked'' categories like plural, rather than ``unmarked'' categories like singular (cf. \citealt[334]{Ottosson:2008b}).\footnote{See also \citet{aalberse2010:ab} (and the references on page 3 there), where it is argued that neutralization is usually induced in marked categories, the plural being their primary example.}  The \stvs syncretism is thus meta-paradigmatic in \citeauthor{Harley:2008ul}'s (\citeyear{Harley:2008ul}) terms: it occurs with every verb no matter what the morphological shapes of the non\stin{} variant are.\footnote{\citet{Harley:2008ul} cites \citet{Williams:1994zd} as being the first to identify the ``meta-paradigm'' as a phenomenon.}
 In the following tables, this meta-paradigmaticity is illustrated by means of examples across various verb classes, in both strong and weak paradigms. In Table \ref{woodweak}, I show the phenomenon for the present tense paradigm for weak \tit{i}-verbs and weak \tit{a}-verbs.


\begin{table}
\caption{Weak verbs} \label{woodweak}
\begin{subtable}{\linewidth}\centering
\caption{Weak \textit{i-}verb: \textit{gera} `do' -- Present}
\begin{tabular}{*5{l}}
\lsptoprule
  & \tsc{sg} & \tsc{pl}  & \tsc{sg} & \tsc{pl} \\\midrule
1 & ger-i & ger-um  	&  		&  ger-um-st \\
2 & ger-ir & ger-ið 		& ger-i-st 	&  ger-i-st  \\
3 & ger-ir  & ger-a 		& 		& ger-a-st  \\
\lspbottomrule
\end{tabular}
\end{subtable}\medskip\\
\begin{subtable}{\linewidth}\centering
\caption{Weak \textit{a-}verb: \textit{hagga} `budge' -- Present}
\begin{tabular}{*5{l}}
\lsptoprule
  & \tsc{sg} & \tsc{pl}  & \tsc{sg} & \tsc{pl} \\\midrule
1 & hagg-a & högg-um  &   &  högg-um-st \\
2 & hagg-ar & hagg-ið & hagg-a-st &  hagg-i-st  \\
3 & hagg-ar  & hagg-a &       & hagg-a-st \\\lspbottomrule
\end{tabular}
\end{subtable}
\end{table}

In Table \ref{woodwash}, I show a full paradigm in past and present tense, indicative and subjunctive mood, for a particularly irregular strong verb \textit{þvo} `wash'. In both tenses and both moods, the same syncretism occurs. In the present indicative, the 2nd singular \tit{-rð} and the 3rd singular \tit{-r} disappear with \stin, collapsing all person distinctions. In the singular present subjunctive, past subjunctive, and past indicative, the 2nd singular \tit{-r} is lost with \stin. Table \ref{woodbera} shows that when the 2nd singular past tense suffix is itself \stin, as with \tit{bera} `carry', distinctions are still lost and there is no sign of two \sti morphemes.

\begin{table}
\caption{Strong \textit{rð-}verb: \textit{þvo} `wash' -- Full paradigm} \label{woodwash}

\begin{tabular}{*5{l}}
\lsptoprule
  & \tsc{sg} & \tsc{pl}  & \tsc{sg} & \tsc{pl} \\\midrule
\multicolumn{5}{c}{Present} \\\midrule
1 & þvæ & þvo-um  	&  			&  þvo-um-st \\
2 & þvæ-rð & þvo-ið 		& þvæ-st 	&  þvo-i-st  \\
3 & þvæ-r  & þvo 		& 		& þvo-st  \\\midrule
\multicolumn{5}{c}{Past} \\\midrule
1 & þvo-ð-i & þvo-ð-um  	&  			&  þvo-ð-um-st \\
2 & þvo-ð-ir & þvo-ð-uð 		& þvo-ð-i-st 	&  þvo-ð-u-st  \\
3 & þvo-ð-i  & þvo-ð-u 		& 		& þvo-ð-u-st  \\\midrule
\multicolumn{5}{c}{Present subjunctive} \\\midrule
1 & þvo-i & þvo-um  	&  			&  þvo-um-st \\
2 & þvo-ir & þvo-ið 		& þvo-i-st 	&  þvo-i-st  \\
3 & þvo-i  & þvo-i 		& 		& þvo-i-st  \\\midrule
\multicolumn{5}{c}{Past subjunctive} \\\midrule
1 & þvæg-i & þvægj-um  	&  			&  þvægj-um-st \\
2 & þvæg-ir & þvægj-uð 		& þvæ-i-st 	&  þvægj-u-st  \\
3 & þvæg-i  & þvægj-u 		& 		& þvægj-u-st  \\
\lspbottomrule
\end{tabular}
\end{table}



\begin{table}
\caption{Past tense of \tit{bera} `carry' -- Past} \label{woodbera}

\begin{tabular}{*5{l}}
\lsptoprule
  & \tsc{sg} & \tsc{pl}  & \tsc{sg} & \tsc{pl} \\\midrule
1 & bar & bár-um  	&  			&  bár-um-st \\
2 & bar-st & bár-uð 		& bar-st 	&  bár-u-st  \\
3 & bar  & bár-u 		& 		& bár-u-st  \\\lspbottomrule
\end{tabular}
\end{table}

\citet{Anderson:1990sm} observed that this cannot be a (solely) phonological effect. It is true that there are morphophonological effects with the \sti{} suffix. For example, dentals (\textit{s, st, t, tt, d}) are often lost from the stem, as illustrated in Table \ref{woodstem}. In one case, [ð] is lost from the stem in the present tense: \textit{bregð} + \textit{st} $\rightarrow$ \textit{bregst} (Table \ref{woodeth}). Usually, it is retained in the present tense, as exemplified by \tit{býðst} `offer'. This could be (partly) phonotactic, since \tit{býð} and \tit{bregð} have different coda structures.  However, [ð] is usually dropped in supine forms, unless it is preceded by /\textit{á}/ (\tsc{ipa}\,=\,[au]) in the supine stem form \citep[380]{Thomson:1987bn}, so it is also at least partly \tit{morpho}phonological.


\begin{table}
\caption{Dental deletion with \sti{} (data from \citealt[380]{Thomson:1987bn})
\label{woodstem}}  
\begin{tabular}{llllllll}
\lsptoprule
Dental & \sti{} verb & non\sti{} stem & & & & output & \\\midrule
-s- & kjósast & kýs & + & st & $\rightarrow$ & kýst & \textsc{present}\\
-t- & látast & læt & + & st & $\rightarrow$ & læst & \textsc{present}\\
-d- & haldast & held & + & st & $\rightarrow$ & helst & \textsc{present}\\
-st- & brestast & brast & + & st & $\rightarrow$ & brast & \textsc{past}\\
-tt- & hittast & hitt & + & st & $\rightarrow$ & hist & \textsc{supine} \\
\lspbottomrule
\end{tabular}
\end{table}

\begin{table}
\caption{Dental deletion with \sti{} (data from \citealt[380]{Thomson:1987bn})\label{woodeth}} 
\begin{tabular}{llllllll}
\lsptoprule
\sti{} verb & non\sti{} stem & & & & output & \\\midrule
bjóðast & býð & + & st & $\rightarrow$ & býðst & \textsc{present} \\
bregðast & bregð & + & st & $\rightarrow$ & bregst & \textsc{present} \\
sjást & séð & + & st & $\rightarrow$ & sést & \textsc{supine} \\
dást & dáð & + & st & $\rightarrow$ & dáðst & \textsc{supine} \\
\lspbottomrule
\end{tabular}
\end{table}

Given these facts, the question becomes whether these rules are responsible for the meta-syncretism of person in the singular. It turns out that they cannot be  \citep{Anderson:1990sm}. 
One main reason is that [r] is often lost when \sti is added (cf. Table \ref{woodwash}), but the sequence [rst] is allowed, even with \stvsn:

\ea 
\begin{tabular}[t]{@{}lll@{}}
{Attested form} &  {No reason to rule out\ldots{}} & {Actual form} \\
\tit{færst} `move' (supine) &   \stem{þvær}{*þværst} & \tit{þvæst} `wash' \\
\tit{berst}  `carry' (\tsc{sg}, pres, \sti form)  & \stem{sér}{*sérst} & \tit{sést} `see' \\
\end{tabular}
\z
\citet[241]{Anderson:1990sm} points out another near-minimal pair with *\tit{sérst}: the superlative form of `bad', which is \tit{verst} `worst'. This shows that the loss of the inflectional \tit{-r} suffix is not due to the incompatibility of the [r] phone with the \sti suffix.

Another indication that phonology is not to blame for \sti syncretism comes from the form of strong \tit{-ur} verbs, an example of which is given in Table \ref{woodstrong}. If \sti syncretism were due to phonology, the /u/ (\tsc{ipa}\,=\,[ʏ]) would be expected to be retained, predicting, for example, \stem{mylur}{*mylust}, contrary to fact. Instead, the observed form is \stem{mylur}{mylst}, and the same person syncretism in the singular as with all other verbs.\footnote{The loss of certain phones, such as {[ð]} on \stem{berð}{berst}, however, could be derived by phonological deletion. Note that in the case of \stem{bregð}{bregst}, it is a non-inflectional stem [ð] that is deleted, whereas with \stem{berð}{berst}, it is an inflectional suffix \tit{-ð}; since this is the only distinguishing suffix in this subparadigm, it is not possible to tell if this is phonological deletion or not.  Similarly, it may be that [ð] deletion in the 2nd person plural, illustrated for example by \stem{þvo-ið}{þvo-i-st} `wash', is similarly phonological.}

\begin{table}
\caption{Strong verbs} \label{woodstrong} 
\begin{subtable}{\linewidth}\centering
\caption{Strong \textit{-rð-}verb: \textit{sjá} `see' -- Present}
\begin{tabular}{*5{l}}
\lsptoprule
  & \tsc{sg} & \tsc{pl}  & \tsc{sg} & \tsc{pl} \\\midrule
1 & sé    & sjá-um  &       &  sjá-um-st\\ 
2 & sé-rð & sjá-ið  & sé-st &  sjá-i-st \\ 
3 & sé-r  & sjá     &       & sjá-st    \\ 
\lspbottomrule
\end{tabular}
\end{subtable}\medskip\\
\begin{subtable}{\linewidth}\centering
\caption{Strong \textit{-ð-}verb: \textit{bera} `carry' -- Present}
\begin{tabular}{*5{l}}
\lsptoprule
& \tsc{sg} & \tsc{pl}  & \tsc{sg} & \tsc{pl} \\\midrule
1 & ber & ber-um && ber-um-st   \\
2 & ber-ð & ber-ið & ber-st & ber-i-st  \\
3 & ber & ber-a & 		& ber-a-st \\
\lspbottomrule
\end{tabular}
\end{subtable}\medskip\\
\begin{subtable}{\linewidth}\centering
\caption{Strong \textit{-ur-}verb: \textit{mylja} `pulverize' -- Present}
\begin{tabular}{*5{l}}
\lsptoprule
  & \tsc{sg} & \tsc{pl}  & \tsc{sg} & \tsc{pl} \\\midrule
1 & myl & mylj-um  	&  			&  mylj-um-st \\
2 & myl-ur & mylj-ið 		& myl-st 	&  mylj-i-st  \\
3 & myl-ur  & mylj-a 		& 		& mylj-a-st  \\
\lspbottomrule
\end{tabular}
\end{subtable}

\end{table} 


For these reasons, the meta-paradigmatic collapse of person distinctions in the singular with all \stvs cannot be due to phonology. The fact that such a heterogeneous class of suffixes (including \tit{-ur, -r, -rð, -ð, \mbox{-st}}) fails to appear further suggests that it is not due to any simple kind of morphophonology either. I will discuss the particular morphological forms further in \sectref{woodthemo}, after presenting my syntactic account of this syncretism.


\subsection{A syntactic account of singular \sti syncretism}	


My syntactic account of singular \sti syncretism relies on the following assumptions. First, Person and Number are separate probes \citep{SigurTHsson:2008dm,Bejar:2008sw}, and more specifically are separate functional heads in the inflectional domain.\footnote{It is not strictly necessary in the present account that they be separate heads, as I assume, as long as Person and Number probe separately, and Number probes first.}

\ea
    { [ Pn$^0$ [ Nm$^0$ [ T$^0$ [ \dots\ ] ] ] ] }
\z
Second, $\upphi$-Agree is $\upphi$-feature union/unification \citep{Kratzer:2009jq,Harbour:2009mh}.\footnote{The mechanism I adopt is from \citet{Kratzer:2009jq}, but \citet{Harbour:2009mh} has a similar approach. Specifically, he argues that a probe can pick up two sets of features, even if they conflict in feature values, and proposes that there are morphemes in Kiowa which are specifically sensitive to conflicting feature values; see also \citet{oxford2019inverse}. A reviewer points out the present proposal is conceptually similar to \citeauthor{kotek2014wh}'s (\citeyear{kotek2014wh})  notion of parasitic agreement and \citeauthor{vanUrk2015}'s (\citeyear{vanUrk2015}) notion of ``best match'', although the details of these proposals are different enough that they cannot be imported without modification into the present analysis.} The following definitions are taken from \citet{Kratzer:2009jq}.

\ea 
    \ea \textit{Agree}: The $\upphi$-feature set of an unindexed head $\upalpha$ that is in need of $\upphi $-features (the probe) unifies with that of an item $\upbeta$ (the goal) if $\upbeta$ is the closest element in $\upalpha$'s c-command domain that has the needed features. \citep[197]{Kratzer:2009jq}
    \ex \textit{Phi-feature unification}: [Unification] applies to expressions 
    $\upalpha_1, \dots, \upalpha_n$ with associated feature sets $A_1, \dots, A_n$ and 
    assigns to each $\upalpha_1, \dots, \upalpha_n$  the new feature set 
    $\bigcup\,\{A_1, \dots, A_n\}$. \citep[195]{Kratzer:2009jq} %Kratzer (2009:195)
    \z
\z

Third, \sti is an argument clitic which occupies a low clitic position, higher than VoiceP/vP, but lower than Pn/Nm/T, as argued extensively in \citet[ch. 2]{Wood2015book} (see also  \citealt{eythors:1995ab,Kissock:1997gm,SigMin,Svenonius:2005vx,Svenonius:2006zt}).  Thus, \sti can in principle be an intervener for $\upphi$-Agree. %As in \citet{Chomsky:2001mh}, intervention could be circumvented by movement of \sti to the left of the probe; I will assume this does not happen in Icelandic, explaining why it does intervene on the basis of its low enclitic position in the clause.

\ea
{ [ Pn$^0$ [ Nm$^0$ [ T$^0$ [ \dots\ \sti \dots\ [ (DP) Voice$^0$ ] ] ] ] ] }
\z
Fourth, \sti{} has a person feature but no number feature. This is plausibly an independently necessary assumption if \sti merges in an argument position (see \citealt{Wood2015book}). This assumption is supported empirically by the fact that \sti developed diachronically from a 3rd person reflexive which was itself invariant for number, and by the fact that it has no other forms -- it is insensitive to person\slash number.\footnote{In addition, there are precedents in the literature. \citet{DAlessandro:2003oy} argues that Icelandic \sti and Italian impersonal \tit{si} have a person feature which is not 1st or 2nd person, but does not say more about exactly what kind of person feature this is. \citet{Taraldsen:1995om} also claims that Italian \tit{si} is 3rd person and has no number feature.} The specific proposal that the morpheme is \glf{$-$participant} captures the intuition that non-1st/2nd person features are involved that are not quite 3rd person (since there is no specification for [$\pm$\tsc{author}]). 


\begin{sloppypar}
Finally, morphological singular agreement is “non-number" agreement. \citet{Nevins2010:ab} argues for something along these lines, on a number of empirical grounds. The strongest of these is the typological absence of “number-case" constraints analogous to “person-case” constraints.\footnote{He also cites, among other things, agreement phenomena in languages like Georgian, the absence of “inverse” constructions based on number (as opposed to person, where inverse constructions are common), and agreement attraction, which is always for number and not person. 
\begin{exe}
\ex[]{The key to the cabinets are missing.} 
\ex[*]{The story about you are interesting.} 
\end{exe} 
In (i), the plural \tit{cabinets} is able to trigger number agreement on the verb, while in (ii), the embedded \tit{you} is not able to trigger person agreement.} He proposes that while Person features consist of two binary features \glf{$\pm$author, $\pm$participant}, number features are privative and involve either the presence or absence of (for example) \glf{plural}; there is no “singular” feature in the syntax. For the present proposal, what is necessary is that singular DPs do not establish an Agree relation with Nm; \citeauthor{Nevins2010:ab}'s stronger claim entails this. However, in the derivations below I will still represent DPs as though they contain “singular” features, for expositional purposes, since only the absence of singular agreement is important.
\end{sloppypar}


First, I will show how this works for a 1st person singular example without \sti{} (and thus without the syncretism in question).\largerpage[-2]

\ea
{\gll Ég græt. \\
I\nom{} cry\gl{1sg} \\
\glt `I cry.'}
\ex
No\sti -- No Person syncretism in the singular\smallskip\\
{\tabcolsep=.2em\begin{tabular}[t]{@{}ll l l ccl@{}}
a.  & & Pn & Nm  & \lowf{DP}{1sg} & $\rightarrow$ & Nm probes \\ 
b. & & Pn & \lowfb{Nm}{dflt(sg)} &  \lowf{DP}{1sg} & $\rightarrow$ & Pn probes \\ 
c. & & \lowfb{Pn}{1sg} & \lowf{Nm}{dflt(sg)} & \lowfb{DP}{1sg} & $\rightarrow$ & DP moves for EPP \\ 
d. & \lowfb{DP}{1sg} & \lowf{Pn}{1sg} & \lowf{Nm}{dflt(sg)} & \mlowfb{DP}{1sg} \\ 
\end{tabular}}
\z

%\begin{itemize}
In step (b), Nm probes for the nearest plural feature, on the above assumption that singular agreement is “non-number” agreement. It finds no plural feature, and thus takes on the default “singular” feature. %\footnote{More strictly, I would suspect that Nm probes for any `marked' number feature, so that Dual, Paucal, etc., would qualify. In the feature system I am assuming, Nm probes for \glf{$-$singular}. This is not important for present purposes.} 
 In step (c), Pn probes for the nearest Person feature, and finds one on the subject DP. It establishes an Agree relation \citep{Chomsky:2001mh}, and given the assumption that  $\upphi$-Agree is  $\upphi$-feature union, Pn takes the DP's number as well as person. 
Finally, in step (d), the nearest DP, which happens to be the subject, moves to the left of Pn.

Now consider what happens when \sti{} is present and intervenes between Pn and the potential DP goal.
%\end{itemize}

\ea
{\gll Ég meiddi-st. \\
I\nom{} hurt\gl{1/2/3sg\woodst{}} \\
\glt `I got hurt.'}
\ex
\sti -- Person syncretism in the singular \\
{\tabcolsep=.2em
\begin{tabular}[t]{@{}lcl l c ccl@{}} 
 a.  & & Pn & Nm & \lowf{-st}{3} & \lowf{DP}{1sg} & $\rightarrow$ & Nm probes \\ 
b. & & Pn & \lowfb{Nm}{dflt(sg)} & \lowf{-st}{3} & \lowf{DP}{1sg} & $\rightarrow$ & Pn probes \\ 
c. & & \lowfb{Pn}{3} & \lowf{Nm}{dflt(sg)} & \lowfb{-st}{3} & \lowf{DP}{1sg} & $\rightarrow$ & DP moves for EPP \\ 
d. & \lowfb{DP}{1sg} & \lowf{Pn}{3} & \lowf{Nm}{dflt(sg)} & \lowf{-st}{3} & \mlowfb{DP}{1sg} \\ 
\end{tabular}}
\z 

Step (b) is the same as above. 
However, in step (c), \stin{} intervenes between Pn and the DP -- the would-be goal. Since \stin{} has a Pn feature, an Agree relation is established between Pn and \stin{}. 
Finally, the DP moves to the left of Pn to satisfy the EPP. Note that EPP, in this case, is dissociated from agreement. This is a necessary assumption about movement anyway to account for \datnom constructions, where EPP-driven movement of a dative is dissociated from agreement with nominative objects.


Here, I take this dissociation to be even more general, so that Pn can Agree with \stin, but the subject can move to satisfy the EPP (see also \citealt[118]{baker2010agreement}, where non-finite T has an EPP feature which triggers movement even though it is not a probe for agreement).

Now consider how number agreement along with feature union can avoid syncretism.

\ea
{\gll Við gef-um-st upp.  \\
we\nom{} give\tsc{-1pl-st} up \\
\glt `We surrender.'} \citep[3]{Kissock:1997gm} %Kissock (1997:3)}
\z


\ea 
\sti -- No Person syncretism in the plural \\
{\tabcolsep=.2em
\begin{tabular}{@{}l c l l c c c l@{}} 
a. & & Pn & Nm & \lowf{-st}{3} & \lowf{DP}{1pl} & $\rightarrow$ & Nm probes \\
b. & & Pn & \lowfb{Nm}{1pl}   & \lowf{-st}{3} & \lowfb{DP}{1pl} & $\rightarrow$ & Pn probes \\ 
c. & & \lowfb{Pn}{1pl} & \lowfb{Nm}{1pl} & \lowf{-st}{3} & \lowf{DP}{1pl} & $\rightarrow$ & DP moves for EPP \\ 
d. & \lowfb{DP}{1pl} & \lowf{Pn}{1pl} & \lowf{Nm}{1pl} & \lowf{-st}{3} & \mlowfb{DP}{1pl} 
\end{tabular}}
\z

When Nm probes for a plural feature, it finds one on the DP and establishes an Agree relation. Since Agree is feature union, Nm takes on the Person features of the goal as well. 
When Pn probes, it finds the Person features on the Nm head and establishes an Agree relation. It picks up both the Person and Number features of the Nm head.
Thus, establishing an Agree relation with the plural DP allows the Person features to be “pied-piped” across \stin{}, preventing intervention of the latter.

\subsection{The morphology of \sti syncretism} \label{woodthemo}

So far, I have argued that \sti has a 3rd person feature, [$-$\tsc{participant}], so that person agreement past \sti is not possible. It is worth considering how this specific choice of feature leads to the morphological forms observed. It cannot be an ordinary 3rd person feature bundle (e.g. [$-$\tsc{participant},$-$\tsc{author}]) because that would predict the syncretic form to look more like the non\sti 3rd person form than it does. Consider Table \ref{woodfff}. If it were an ordinary 3rd person feature bundle, the expected singular form of \tit{myljast} would be \tit{myl-ur-st}, when in fact it is \tit{myl-st}.

\begin{table}
\caption{Strong \textit{-ur-}verb: \textit{mylja} `pulverize' -- Present} \label{woodfff} 
\begin{tabular}{*5{l}}
\lsptoprule
  & \tsc{sg} & \tsc{pl}  & \tsc{sg} & \tsc{pl} \\\midrule
1 & myl & mylj-um  	&  			&  mylj-um-st \\
2 & myl-ur & mylj-ið 		& myl-st 	&  mylj-i-st  \\
3 & myl-ur  & mylj-a 		& 		& mylj-a-st  \\\lspbottomrule
\end{tabular}
\end{table}

This issue is resolved by assuming that that the \tit{-ur} ending reflects the feature \glf{$-$author}. Where there is a distinction between 2nd and 3rd person, the 2nd person morpheme is \glf{$-$author, $+$participant}. 

\ea
\begin{tabular}[t]{@{}ll@{~}l@{~}l@{}}
a. & \glf{$-$author, $+$participant} & $\leftrightarrow$ & [rð], [ð], \ldots{} \\
b. & \glf{$-$author} & $\leftrightarrow$ & [r], [ur], \ldots{} \\
c. & elsewhere & $\leftrightarrow$ &  ∅, [a], \ldots{} \\
\end{tabular}
\z

Given this much, the intuition that \sti is 3rd person but not fully 3rd person can be captured by saying that it is \glf{$-$participant} (compare Figures \ref{woodtree1} and \ref{woodtree2}). Since there are no forms to realize just this feature, it adopts the “elsewhere” zero agreement allomorph.\footnote{Another possibility, pointed out to me by Neil Myler (p.c.), is that \sti itself is a person agreement morpheme, and that what appears to be plural person agreement is actually just number agreement with allomorphs determined by person. While I find this idea appealing, it is challenged by the fact that \sti appears on infinitive forms and supine forms, neither of which shows agreement inflection of any other kind.} 

\begin{figure}
  \begin{floatrow}
    \captionsetup{margin=.1\linewidth}
    \ffigbox{\begin{forest}
    [Pn [Nm [T] [Nm\\(default)\\\glf{$+$sing}] ] 
        [Pn [{\glf{$-$part}\\$\emptyset$}] ] ] 
 \end{forest}}
            {\caption{Singular agreement with \sti}\label{woodtree1}}
    \ffigbox{\begin{forest}
            [Pn [Nm [T] [Nm\\(default)\\\glf{$+$sing}] ] 
                [Pn [\glf{$-$part}\\∅] [\glf{$-$auth}\\\tbf{-ur}] ] ]  
 \end{forest}}
            {\caption{“True” third-person singular agreement}\label{woodtree2}}
  \end{floatrow}
\end{figure}


\section{Ameliorative effect of \sti syncretism}



Recall from earlier that a range of analyses in the literature cited above argues that both the 1st/2nd person agreement restrictions and the improvement in the context of syncretic forms stem from the verb agreeing in person with both the dative and the nominative.  When verbal inflectional heads successfully Agree with a dative argument, they may continue to probe and, when possible, enter into a Multiple Agree relation with a nominative as well (\citealt{Schutze:2003mh, SigurTHsson:2008dm, Ussery:2009jd, atlamaz2018partial, CoonKeine2020}).\footnote{I assume that dative arguments are special in this regard, and that Multiple Agree does not occur when Pn agrees with a nominative subject, the \sti clitic, etc.}  The Agree relation with the nominative, however, is only licit if the dative subsequently moves to the left of the probe (\citealt{Holmberg:2004gk, Kucerova:2007nn,kucerova2016long, SigurTHsson:2008dm}, see also \citealt{Chomsky:2008od}). In this situation, the probe receives default 3rd person features from the dative (regardless of whether the dative is actually 3rd person), and whatever features the nominative bears. I adopt this analysis as well, but only for a subset of cases, namely person sycretism in the plural and non\sti cases.

Without the analysis of \sti given above, these accounts predict two situations: either there exists a syncretic form, and the example improves, or there exists no syncretic form, and the example is out. These predictions are summarized in \tabref{woodstpred} for the forms in \tabref{woodlike} (syncretic forms italicised in both tables).
 
\begin{table}
\caption{\label{woodlike}Syncretic Forms}
\begin{tabular}{*5{l}}
\lsptoprule
& \multicolumn{2}{c}{{\tit{líka} `like'}} & \multicolumn{2}{c}{{\tit{leiðast} `bore'}} \\ \cmidrule(lr){2-3}\cmidrule(lr){4-5}
& \tsc{sg} & \tsc{pl} & \tsc{sg} & \tsc{pl} \\ \midrule
1 & \tit{likaði} & líkuðum & \tit{leiddist} & leiddumst \\
2 & líkaðir      & líkuðuð & \tit{leiddist} & \tit{leiddust} \\
3 & \tit{líkaði} & líkuðu  & \tit{leiddist} & \tit{leiddust}\\
\lspbottomrule
\end{tabular}
\end{table}

% \ea \label{woodlike} 
% \begin{tabular}[t]{llll}
%  &  & \tbf{\tit{líka} `like'} & \tbf{\tit{leiðast} `bore'}   \\
%  \hline\hline
 
% SG & 1 & \tbf{likaði} & \tbf{leiddist}   \\
%  & 2 & líkaðir &\tbf{leiddist}   \\
%  & 3 & \tbf{líkaði} & \tbf{leiddist}   \\
%   \hline
%  PL & 1 & líkuðum & leiddumst   \\
%  & 2 & líkuðuð & \tbf{leiddust}   \\
%  & 3 & líkuðu & \tbf{leiddust}   \\
%  \end{tabular}
% \z
 
\begin{table}
\caption{\label{woodstpred}Predictions of  Multiple Agree accounts}
\begin{tabular}{llll@{~}l@{~}l}
\lsptoprule
   & Verb & Feature bundle & Syncretic form? & &  \\\midrule
a. & \tit{leiðast} `bore' & \tsc{1/2sg+3} & Yes & $\rightarrow$ & Improved \\
b. & \tit{leiðast} `bore' & \tsc{2pl+3} & Yes & $\rightarrow$ & Improved \\
c. & \tit{leiðast} `bore' & \tsc{1pl+3} & No & $\rightarrow$ & Bad \\\midrule
d. & \tit{líka} `like'    & \tsc{1sg+3} & Yes & $\rightarrow$ & Improved \\
e. & \tit{líka} `like'    & \tsc{2sg+3} & No & $\rightarrow$ & Bad \\
f. & \tit{líka} `like'    & \tsc{1pl+3} & No & $\rightarrow$ & Bad \\
g. & \tit{líka} `like'    & \tsc{2pl+3} & No & $\rightarrow$ & Bad \\
\lspbottomrule
\end{tabular}
\end{table}

However, as shown in \tabref{woodjudge}, there seem to be three classes of acceptability rather than two. Most speakers found the 1st and 2nd plural singular nominative objects with the \sti verb \tit{leiðast} `bore' either OK or “?”. The plural syncretism of \tit{leiðast} in the 2nd/3rd person fared on par with the 1st/3rd person singular syncretism of the non\sti verb \tit{líka} `like', where the judgments split.

\begin{table}
\caption{\label{woodjudge}Acceptability of syncretic and non-syncretic forms (Data from \citealt[74–76]{SigurTHsson:1992lj})}
\begin{tabular}{llllllll} 
\lsptoprule
&\multicolumn{4}{l}{{Improvement due to singular \sti syncreticism}}     & {OK/?} & {??/*} &  \\\midrule
%\\
a. & Henni & \tit{leiddist} & ég. &  & 8 & 1 &  \\ 
 & her\dat{} & bored\gl{1/2/3sg} & I\nom{} &  &  &  &  \\ \tablevspace
 & Henni & \tit{leiddist} & þú. &  & 8 & 1 &  \\ 
 & her\dat{} & bored\gl{1/2/3sg} & you\nom{} &  &  &  &  \\\midrule
& \multicolumn{4}{l}{{Improvement due to syncreticism}}   & {OK/?} & {??/*} &  \\\midrule
b. & Henni & \tit{líkaði} & ég. &  & 5 & 4 &  \\ 
 & her\dat{} & liked\gl{1/3sg} & I\nom{} &  &  &  &  \\\tablevspace 
 & Henni & \tit{leiddust} & þið. &  & 5 & 4 &  \\ 
 & her\dat{} & bored\gl{2/3pl} & you\gl{pl.nom} &  &  &  &  \\\midrule
& \multicolumn{4}{l}{{No syncretism -- no improvement}}   & {OK/?} & {??/*} &  \\ \midrule
c. & Henni & \tit{líkaðir} & þú. &  & 0 & 9 &  \\ 
 & her\dat{} & liked\gl{2sg} & you\nom{} &  &  &  &  \\ \tablevspace
 & Henni & \tit{líkuðum} & við. &  & 0 & 9 &  \\ 
 & her\dat{} & liked\gl{1pl} & we\nom{} &  &  &  &  \\\tablevspace
 & Henni & \tit{líkuðuð} & þið. &  & 0 & 9 &  \\ 
 & her\dat{} & liked\gl{2pl} & you\gl{pl.nom} &  &  &  &  \\\tablevspace
 & Henni & \tit{leiddumst} & við. &  & 2 & 7 &  \\ 
 & her\dat{} & bored\gl{1pl} & we\nom{} &  &  &  & \\\lspbottomrule
\end{tabular}
\end{table}

I now show how the account of \sti syncretism provided above captures these data. Specifically, while my account admittedly predicts the singular \sti cases to be fully grammatical (contrary to fact), it makes the cut in the right direction: it predicts a difference between \tabref{woodjudge}a and \ref{woodjudge}b. There are arguably further constraints on 1st/2nd person nominative objects which account for the fact that the examples in \tabref{woodjudge}a are not perfect (see discussion below). 

First, consider improvement due solely to syncretism \citep[33]{SigurTHsson:1996va}. 

% % \begin{multicols}{2}
\begin{exe}
\exr{woodsing}\judgewidth{??} 
    \ea[??]
    {
        \gll Henni líkaði ég. \\
            her\dat{} liked\gl{1/3sg} I \\
        \glt \textsc{intended}: ‘She liked me.’
    }
    \ex[\bad]
    {
        \gll Henni líkaðir þú. \\
            her\dat{} liked\gl{2sg} you\gl{sg} \\
        \glt \textsc{intended}: ‘She liked you.’ \citep[33]{SigurTHsson:1996va}
    } 
\z
\end{exe}
% % \end{multicols} 

\ea \datnom singular non-agreement (2nd person \textsc{nom})\smallskip\\
\resizebox{\linewidth}{!}{\tabcolsep=.2em
\begin{tabular}[t]{@{}lcllcccl@{}}
a. &  & Pn & Nm & \lowf{\tsc{dat}}{3} & \lowf{\tsc{nom}}{2sg} & $\rightarrow$ & Nm probes \\ 
b. &  & Pn & \lowfb{Nm}{dflt(sg)} & \lowf{\tsc{dat}}{3} &\lowfb{\tsc{nom}}{2sg} & $\rightarrow$ & Pn probes \tsc{dat}/\tsc{nom} \\ 
c. &  & \lowfb{Pn}{2sg,3} & \lowf{Nm}{dflt(sg)}& \lowfb{\tsc{dat}}{3} & \lowfb{\tsc{nom}}{2sg}  & $\rightarrow$ & DP moves for EPP \\ 
d. & \lowfb{\tsc{dat}}{3} & \lowf{Pn}{2sg,3} & \lowf{Nm}{dflt(sg)} & \mlowfb{\tsc{dat}}{3} & \lowf{\tsc{nom}}{2sg}  & 
\end{tabular}}
\z

The Nm head does not pied-pipe any Person features since \tsc{nom} is singular. The Pn head agrees with both \tsc{dat} and \tsc{nom}, and thus has the feature bundle \glf{2sg, 3}. This is ungrammatical since there is no form syncretic for both 2nd and 3rd person singular. However, if \tsc{nom} had been 1st person, there is a syncretic form, so the example improves slightly. Note that even in the syncretic form, the syntax still contains a phi-feature bundle with contradictory values.\footnote{I assume, following \citet{Bjorkman2016}, that when one head has two feature sets of the same type, Vocabulary Insertion must apply twice, once for each feature set, and the result is only grammatical if those two separate competitions result in the same form. For other proposals in the same spirit (but with different details), see \citet{citko2005nature}, \citet{Kratzer:2009jq}, \citet{bhatt2013locating}, \citet{Asarina2013} and  \citet{CoonKeine2020}, among others. \citet{CoonKeine2020} develop an insightful account of ameliorative syncretism in Icelandic \tsc{dat-nom} constructions very much in the spirit of the present paper. Their analysis  does not account for the special effects of singular \sti syncretism, and something different from the present account would have to be said about why \stin{}, despite being third person, intervenes for person agreement.  } 


Now consider what happens when \sti is involved. 


\ea \judgewidth{?}
    \ea[?]
    {
        \gll Henni leiddist ég. \\
            her\dat{} bored\gl{1/2/3sg} I \\
        \glt `She found me boring.' \\
    }
    \ex[?]
    {
        \gll Henni leiddist þú. \\
            her\dat{} bored\gl{1/2/3sg} you\gl{sg} \\
        \glt `She found you boring.' \citep[33]{SigurTHsson:1996va}\\
    }
%\attr{Sigurðsson (1996:33)}
    \ex[]
    {
        \gll Mér leiddist hún. \\
            me\dat{} bored\gl{1/2/3sg} she\gl{sg} \\
        \glt `I found her boring.' \citep[16]{SigOnt} %Sigurðsson (2010a:16)}
    }
\z
\z


\ea \datnom{} singular \sti non-agreement (2nd person \textsc{nom})\smallskip\\
\resizebox{\linewidth}{!}{\tabcolsep=.2em\begin{tabular}[t]{@{}lc l l c c ccl@{}}

a. &  & Pn & Nm & \lowf{-st}{3} &  \lowf{\tsc{dat}}{3} & \lowf{\tsc{nom}}{2sg} & $\rightarrow$ & Nm probes \\ 

b. &  & Pn & \lowfb{Nm}{dflt(sg)} & \lowf{-st}{3} &  \lowf{\tsc{dat}}{3} & \lowf{\tsc{nom}}{2sg} & $\rightarrow$ & Pn probes \sti \\ 

c. &  & \lowfb{Pn}{3} & \lowf{Nm}{dflt(sg)} & \lowfb{-st}{3} &  \lowf{\tsc{dat}}{3} & \lowf{\tsc{nom}}{2sg} & $\rightarrow$ & DP moves for EPP \\ 

d. &  \lowfb{\tsc{dat}}{3} & \lowf{Pn}{3}  & \lowf{Nm}{dflt(sg)} & \lowf{-st}{3} &  \mlowfb{\tsc{dat}}{3} & \lowf{\tsc{nom}}{2sg} & \\ 
\end{tabular}}
\z

This time, when Pn probes, it agrees with \sti rather than \tsc{nom}. Thus, when \sti is present, there is no conflict. The question that arises on my approach is why these examples are marked at all. Unlike above, the syntax here never builds a contradictory feature bundle in the first place. The difference in acceptability judgments thus has to be linked to the different elements present in the syntax.


Finally, consider  \sti with a plural nominative object \citep[33]{SigurTHsson:1996va}.\footnote{Sigurðsson marks both examples as ungrammatical, but recall from \tabref{woodjudge}a that the improvement of (\ref{wood2}b) over (\ref{wood2}a) is comparable to the improvement of (\ref{woodsing}a) over (\ref{woodsing}b).}


\ea \judgewidth{??}\label{wood2} 
   \ea[\bad]
    {
        \gll Henni leiddumst við.\\ 
            her\dat{} bored\gl{1pl} we\\
        \glt \textsc{intended}: ‘She found us boring.’ \\
    }
    \ex[??]
    {
    \gll Henni leiddust þið.\\
         her\dat{} bored\gl{2/3pl} you\gl{pl}\\
        \glt \textsc{intended}: ‘She found you (plural) boring.’ \citep[33]{SigurTHsson:1996va}\\
    }

  %  \ea[\bad]
   % {
    %    \gll Henni leiddumst við. \\
     %       her\dat{} bored\gl{1pl} we \\
    %}
    %\ex[??]
    %{
    %    \gll Henni leiddust þið. \\
    %        her\dat{} bored\gl{2/3pl} you\gl{pl} \\
    %}  
    \z
 %   \attr{\citep[33]{SigurTHsson:1996va}}\footnote{Sigurðsson marks both examples as ungrammatical, but recall from (\ref{woodjudge}a) that the improvement of (\ref{wood2}b) over (\ref{wood2}a) is comparable to the improvement of (\ref{woodsing}a) over (\ref{woodsing}b).}
\z


\ea \datnom{} plural \sti non-agreement (2nd person \textsc{nom})\smallskip\\
{\tabcolsep=.2em\begin{tabular}[t]{@{}lc l c l c c l@{}}

a. &  & Pn &  & Nm & \lowf{-st}{3} & \lowf{\tsc{dat}}{3} & \lowf{\tsc{nom}}{2pl}\\ 
\multicolumn{7}{r}{} & ~ $\rightarrow$ Nm probes\\ 

b. &  & Pn &  & \lowfb{Nm}{2pl} & \lowf{-st}{3} & \lowf{\tsc{dat}}{3} & \lowfb{\tsc{nom}}{2pl}\\
\multicolumn{7}{r}{} & ~ $\rightarrow$ \tsc{dat} moves\\ 

c. &  & Pn & \lowfb{\tsc{dat}}{3} & \lowfb{Nm}{2pl} & \lowf{-st}{3} & \mlowfb{\tsc{dat}}{3} & \lowf{\tsc{nom}}{2pl}\\
\multicolumn{7}{r}{} & ~ $\rightarrow$  Pn probes \tsc{dat}/Nm\\ 

d. &  & \lowfb{Pn}{2pl,3} & \lowfb{\tsc{dat}}{3} & \lowfb{Nm}{2pl} & \lowf{-st}{3} & \mlowf{\tsc{dat}}{3} & \lowf{\tsc{nom}}{2pl}\\
\multicolumn{7}{r}{} & ~ $\rightarrow$  DP moves for EPP\\ 

e. & \lowfb{\tsc{dat}}{3} & \lowf{Pn}{2pl,3} & \mlowfb{\tsc{dat}}{3} & \lowf{Nm}{2pl} & \lowf{-st}{3} & \mlowf{\tsc{dat}}{3} & \lowf{\tsc{nom}}{2pl}\\
\multicolumn{7}{r}{} & ~  \\ 
\end{tabular}}
\z


Here, featural pied-piping allows the contradictory feature bundles to be built. Nm enters into an Agree relation with the nominative, and thus obtains 2nd person plural features. The dative is thus required to move to its left, as discussed above. Pn agrees with the dative and the Nm head, picking up 3rd person and 2nd person plural features.  Thus, plural forms of \tit{leiðast} `bore' pattern like all forms of \tit{líka} `like': they are ungrammatical unless syncretism improves acceptability slightly. The fact that \tit{leiðast} `bore' behaves in the 2nd person plural like the non\sti verb \tit{líka} `like' shows that it is not the \sti morpheme plus syncretism which improves the example per se; \sti only improves it (beyond the non\sti cases) when its presence prevents the syntax from building up the contradictory feature bundle which needs a syncretic form to survive. The syntactic approach to \sti syncretism proposed for here predicts this to be the case. 

As a final remark, note that nothing in the present account  predicts  singular 1st/2nd person objects of \tit{leiðast} `bore' to be less than perfect. Possibly, 1st/2nd person nominatives  are subject to special constraints. Cartographic work often  posits particular positions  for 1st/2nd person (\citealt{Savescu:2009al}). Note that even in infinitive contexts, where agreement should not be an issue, such objects are slightly degraded.


\ea[?] 
{
    \gll Hún vonaðist auðvitað til að leiðast við/þið/þeir ekki mikið. \\
        she hoped of.course for to bore.\tsc{inf} we/you/they\nom{} not much \\
    \glt `She of course hoped not to find us/you/them very boring.' \citep[271]{SigurTHsson:2008dm}
}
%\exattr{\citep[271]{SigurTHsson:2008dm}}
\z

\citet[271]{SigurTHsson:2008dm} suggest that this is due to the difficulty of controlling non-agentive predicates. However, it is suggestive that when agreement is not at issue, 1st/2nd person objects are only slightly degraded. Why they are degraded at all is a question I must set aside for now.\footnote{Einar Freyr Sigurðsson points out to me that in principle this should hold for all dative-nominative verbs, whether \tit{-st} is present or not. However, independent factors may vary, and as far as I know there has not been any thorough study of the matter. See \citet{SigEPP} for a proposal that may bear on the question.} 

% (1. default agreement, 2. plural datives and intervention in expletive constructions 3. speaker variation) 4. why not perfect.

\section{Conclusion}

In this paper I have proposed that  syncretism can shed light on the size of feature bundles involved in  Agree relations and the nature of those relations. At a general level, $\upphi$-features are individually active in Agree relations and in syntactic primitives, but since Agree works as unification, collections of $\upphi$-features are quickly assembled into bundles in the course of the derivation. But they are not always assembled in the same way. The \stin{} clitic has a person feature (proposed to be [$-$\tsc{participant}]) which induces person syncretism in the singular for all \stin{} verbs, but which also "shields" the grammar from building contradictory feature bundles in the presence of 1st and 2nd person nominative objects -- again, though, only in the singular. The special status of plural in this collection of facts stems from what I have called ``featural pied-piping'': the presence of a plural feature leads to the establishment of an Agree relation with the consequence that person features are ``pied-piped'' to the Nm head -- past \stin{}, which now can neither induce syncretism nor shield the grammar from the person features of 1st/2nd person nominative objects. Why plural features are like this remains to be established, but if singular is really the absence of a privative number feature, then perhaps ``singular agreement'' \tit{must be} the absence of number agreement. The broad implication is that the larger a feature bundle is, the harder it will be for the grammar to stop that bundle from being a goal in an Agree relation.  

\section*{Abbreviations}
\begin{multicols}{3}
\begin{tabbing}
\textsc{dflt}\hspace{.5ex} \= default\kill
\textsc{1}   \> 1st person  \\
\textsc{2}   \> 2nd person \\
\textsc{3}   \> 3rd person \\
\textsc{acc} \> accusative    \\
\textsc{dat} \> dative     \\
\textsc{dflt} \> default    \\
\textsc{inf}  \> infinitive\\
\textsc{m}    \> masculine  \\
\textsc{nom} \> nominative   \\
\textsc{pl}   \> plural     \\
\textsc{sg}   \> singular   \\
\textsc{st}   \> \textit{-st} clitic
\end{tabbing}
\end{multicols}

\printbibliography[heading=subbibliography,notkeyword=this]

\end{document}
