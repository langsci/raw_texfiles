\documentclass[output=paper]{langscibook}
\ChapterDOI{10.5281/zenodo.17132453}
\author{Oliver Bunk\orcid{}\affiliation{Humboldt-Universität zu Berlin}}
\title[The anxious heritage speaker?]{The anxious heritage speaker? Language anxiety and insecurity in multilingual contexts}
\abstract{Language anxiety (LA), predominantly studied in second language learners, has recently encompassed the experiences of multilingual heritage speakers. While this expanded research offers important insights, it reveals a thread of Othering in LA studies, portraying heritage speakers as a homogenous group with little power over their linguistic command. Drawing on qualitative interviews, this chapter highlights that heritage speakers’ emotional relationship with language is often shaped more by heightened societal consciousness than mere anxiety. They navigate majority linguistic expectations, not as passive entities dominated by anxiety, but as active agents deftly managing their linguistic choices. This study advocates for a reconceptualized view of multilinguals in LA research, emphasizing their agency and the role of societal ideologies towards multilingualism.}
\IfFileExists{../localcommands.tex}{
  \addbibresource{../localbibliography.bib}
  % add all extra packages you need to load to this file

\usepackage{tabularx,multicol}
\usepackage{url}
\urlstyle{same}

\usepackage{listings}
\lstset{basicstyle=\ttfamily,tabsize=2,breaklines=true}

\usepackage{langsci-basic}
\usepackage{langsci-optional}
\usepackage{langsci-lgr}
\usepackage{langsci-osl}
% \usepackage{./langsci/styles/langsci-lgr}
% \usepackage{./langsci/styles/langsci-osl}
% \usepackage{langsci-gb4e}

\usepackage{tikz}
\usetikzlibrary{patterns,calc}
\pgfdeclarepatternformonly{south east lines}{\pgfqpoint{-0pt}{-0pt}}{\pgfqpoint{3pt}{3pt}}{\pgfqpoint{3pt}{3pt}}{
    \pgfsetlinewidth{0.6pt}
    \pgfpathmoveto{\pgfqpoint{0pt}{3pt}}
    \pgfpathlineto{\pgfqpoint{3pt}{0pt}}
    \pgfpathmoveto{\pgfqpoint{.2pt}{-.2pt}}
    \pgfpathlineto{\pgfqpoint{-.2pt}{.2pt}}
    \pgfpathmoveto{\pgfqpoint{3.2pt}{2.8pt}}
    \pgfpathlineto{\pgfqpoint{2.8pt}{3.2pt}}
    \pgfusepath{stroke}}
    
\usepackage{stmaryrd}
\usepackage{wasysym}
\usepackage{multirow}
\usepackage{caption}
\usepackage{subcaption}
\usepackage{mathrsfs}
\usepackage{qtree}

\usepackage{linguex}


  %pminos do not split footnotes
% \interfootnotelinepenalty=10000 %Footnote in Laporte chapters has to be split SN


%\DeclareIndexNameFormat{default}{%
%\nameparts{#1}%
%\usebibmacro{index:name}%
%{\index[names]}%
%{\namepartfamily}%
%{\namepartgiveni}%
% {}% L1
% {}% L2
%{\namepartprefix}% generates spurious space L3
%{\namepartsuffix}% generates spurious space L4
%}

%  {\DeclareIndexNameFormat{default}{%
%     \usebibmacro{index:name}{\index[names]}{#1}{#3}{#5}{#7}}}

%\DeclareIndexNameFormat{default}{%
%  \usebibmacro{index:name}{\sindex[nom]}{#1}{#3}{#5}{#7}}

%\DeclareIndexNameFormat{default}{%
%  \usebibmacro{index:name}{\sindex[person]}{#1}{#3}{#5}{#7}}
%\DeclareIndexNameFormat{default}{%
%\nameparts{#1} \usebibmacro{index:name}{\sindex[person]]}{\namepartfamily}{‌​\namepartgiven}{\nam‌​epartprefix}{\namepa‌​rtsuffix}}

%\newcommand{\smiley}{:)}

%\renewbibmacro*{index:name}[5]{%
%\usebibmacro{index:entry}{#1}%
%{\iffieldundef{usera}{}{\thefield{usera}\actualoperator}\mkbibindexname{#2}{#3}{#4}{#5}}}

% \newcommand{\noop}[1]{}

%remove for final
%\overfullrule=1mm

\newcommand{\tobi}[2]}}
\renewcommand{\S}[1]{\tobi{#1}{\textsc{*}}}

% this volume references
% puts: [this volume]
% already defined: \citetv
%\newcommand{\citepv}[1]{(\citeauthor{#1} \citeyear*{#1} [this volume])}
\newcommand{\citealtv}[1]{\citeauthor{#1} \citeyear*{#1} [this volume]}

%parentheses around example number
\newcommand{\pref}[1]{(\ref{#1})}

% in-text examples

\newcommand{\lnex}[1]{\textit{#1}} %target lang word
\newcommand{\lnlit}[1]{(lit.: `#1')} %literal reading
\newcommand{\lnlat}[1]{(#1)} % latinization
\newcommand{\lntrans}[1]{`#1'} %translation
\newcommand{\lnexl}[2]%
{\lnex{#1}{} \lnlat{#2}} % ex with latinization
\newcommand{\lnexlat}[3]{\lnex{#1}{} \lnlat{#2}{} \lntrans{#3}} % ex with latinization and tranl.

%ch01
\newcommand{\co}[1]{\mbox{\textbf{#1}}}

%ch09

\newcommand{\cyrbulg}[1]{\begin{otherlanguage*}{bulgarian}#1\end{otherlanguage*}}


%ch10
\newcommand{\nlp}{{\small NLP}}
\newcommand{\mwe}{{\small MWE}}
\newcommand{\rae}{{\small RAE}}
\newcommand{\lvc}{{\small LVC}}
\newcommand{\pos}{{\small P}o{\small S}}
%\newcommand{\todo}[1]{ \textcolor{red}{#1} }

%\renewcommand{\labelenumi}{\theenumi}
%\ainamefmt{{vv}{ll}{, ff}{, jj}} % fullname

\newcommand{\biberror}[1]{{\color{red}#1}}

\newcommand{\osenovaitem}{--~} 
  %% hyphenation points for line breaks
%% Normally, automatic hyphenation in LaTeX is very good
%% If a word is mis-hyphenated, add it to this file
%%
%% add information to TeX file before \begin{document} with:
%% %% hyphenation points for line breaks
%% Normally, automatic hyphenation in LaTeX is very good
%% If a word is mis-hyphenated, add it to this file
%%
%% add information to TeX file before \begin{document} with:
%% %% hyphenation points for line breaks
%% Normally, automatic hyphenation in LaTeX is very good
%% If a word is mis-hyphenated, add it to this file
%%
%% add information to TeX file before \begin{document} with:
%% \include{localhyphenation}
\hyphenation{
    Beck-man
    Ngu-yen
    back-chan-nel
    back-chan-nels
    mo-not-o-nous
    ste-reo-typ-i-cal
}

\hyphenation{
    Beck-man
    Ngu-yen
    back-chan-nel
    back-chan-nels
    mo-not-o-nous
    ste-reo-typ-i-cal
}

\hyphenation{
    Beck-man
    Ngu-yen
    back-chan-nel
    back-chan-nels
    mo-not-o-nous
    ste-reo-typ-i-cal
}
 
  \togglepaper[1]%%chapternumber
}{}

\begin{document}
\maketitle 
%\shorttitlerunninghead{}%%use this for an abridged title in the page headers



\section{Introduction}
\label{8:sec:1}
\largerpage[-1]
Language anxiety (LA) refers to the unease, apprehension, and nervousness experienced by speakers when using or learning a particular language (e.g., \citealt{HorwitzEtAl1986}). LA was initially investigated in the context of foreign and second language acquisition, both in and outside the classroom (e.g., \citealt{Gregersen2020,Horwitz2001,Horwitz2010,HorwitzEtAl1986,MacIntyre2017}). While initial studies primarily revolved around monolingual\footnote{The term “monolingual” commonly describes speakers of one named language, i.e. English, German, Russian etc. However, only very few individuals may actually speak only one language, especially considering the global importance of languages such as English, French, Spanish or Mandarin-Chinese, and linguistic registers (\citealt{Rothman2008,MacSwan2017}). In this chapter, I consider the notion of the “monolingual speaker” as a social construct. However, I will use the term in order to better describe the contrast between speakers that grew up with one vs. multiple language as their family language(s).} learner contexts, recent investigations have expanded to multilingual settings, especially within minority communities (\citealt{ErgütBaş2023,GarciaDeBlakeleyEtAl2017,Jee2022,PradaEtAl2020,Sevinç2014,Sevinç2018,Sevinç2020,Sevinç2022a,SevinçBackus2019,SevinçDewaele2018}). This shift in focus to multilingualism is highly relevant as it includes a growing population that has largely been left untouched by research. Additionally, it adds a new, important perspective, delving into the emotional well-being of heritage speakers and identifying factors that may adversely affect them, e.g., issues of identity (\citealt{SevinçBackus2019}), culture and belonging (\citealt{Jee2022,PradaEtAl2020}), and family attitudes (\citealt{Jee2022,Sevinç2022a}). Thus, such research sheds light on heritage speakers’ experiences and unravels the negative impacts of the contexts they navigate. These insights can provide a basis for improving multilingual experiences. However, there is an underlying risk of Othering within LA studies, echoing Othering practices found in other lines of research, like the use of specific labelling (see \citealt{WieseEtAl2022Multilinguals}, Wiese this volume) and adapting specific perspectives on heritage speakers.

Building on these observations, this chapter investigates the Othering practices within LA research in multilingual contexts. I argue that research on LA generally adopts an inclusive perspective but exhibits patterns of Othering. These patterns often present a unidimensional view, portraying speakers as having limited agency over their language use due to LA. Analyzing interview data that focus on LA in the majority language of heritage speakers, I argue that heritage speakers’ emotional connection to the majority language stems less from anxiety and more from an acute awareness of the expectations set by the prevailing monolingual majority. Consequently, speakers might experience pressure to conform linguistically with this majority; however, they apply specific mechanisms to overcome this pressure. The study, thus, aims to provide an additional perspective in LA research, highlighting speakers’ agency and linguistic ownership and the role of language ideologies in the contexts they navigate. 

Before delving deeper into the topic, it is imperative to clarify that this discussion is not an attempt to spotlight specific authors, projects, or publications for their Othering patterns. Contrary to other subdisciplines in linguistics, recent research on LA stands out due to its use of inclusive language and perspectives. Especially within the recently developing focus on heritage speakers and diverse, multilingual contexts, this line of research has unearthed invaluable insights into the well-being of multilingual speakers, shedding light on the negative emotions they grapple with in their everyday lives. However, this chapter intends to illustrate that even such a progressive and relatively young field is not exempt from the inadvertent Othering processes stemming from certain deep-seated, internalized perspectives on language and speakers. 

The chapter is structured as follows: In Section \ref{sec:08:2}, I discuss the key topics in LA research to outline potential areas of Othering. I introduce Linguistic Insecurity (LI) as a concept closely related to LA and compare both concepts with respect to Othering. Section \ref{8:sec:3} summarizes the main forms of Othering, while Section \ref{8:sec:4} discusses which perspectives might be taken to avoid Othering. Section \ref{8:sec:5} presents an example of how these perspectives can help to enlighten our understanding of LA in heritage speakers, acknowledging their role as active language owners. I present a qualitative analysis of interview data on the linguistic pressure of heritage speakers in Germany, focusing on their use of German. Section \ref{8:sec:6} provides a conclusion.

\section{Language anxiety, insecurity and multilingualism}
\label{8:sec:2}\label{sec:08:2}

Understanding the emotional responses of language learners and speakers, such as Language Anxiety (LA) and Linguistic Insecurity (LI), offers insights into the broader implications of multilingualism. Traditionally, LA has been investigated in the classroom context from the perspective of foreign language learners (\citealt{GardnerMacIntyre1993,HorwitzEtAl1986}).  {The} sources of foreign language anxiety (FLA) are multifaceted, including learner factors like a perceived lack of competence or preparation (\citealt{Horwitz1986}), perfectionism (\citealt{GregersenHorwitz2002}), and competitiveness (\citealt{JinEtAl2015}), and classroom factors like teacher behavior and teaching methods (\citealt{BriesmasterBriesmaster-Paredes2015,HuangEtAl2010}). \citet{HorwitzEtAl1986} further argue that foreign language anxiety interacts with communication apprehension, test anxiety, and fear of social evaluation. FLA can relate to specific domains, such as speaking, listening, reading or writing (\citealt{DewaeleLi2022}).

Studies indicate multilingualism reduces FLA, facilitating language acquisition (\citealt{Dewaele2007,Dewaele2010,ThompsonLee2013}). \citet{Liu2013} reports that frequent use of a foreign language increases self-perceived competence and self-confidence, lowering LA, while  \citet[270]{MacIntyreGardner1989} find that FLA causes “performance deficits”. Thus, studies on LA in the learner context often explore the relationship between anxiety, language achievement and proficiency, i.e. the reasons and effects of LA on language learning (see  \citealt{Luo2013} for a comprehensive review). These complex interactions also play an important role in heritage and majority language anxiety (HLA and MLA, respectively), however, with different effects for the groups investigated. While FLA might lead to negative learning achievements, particularly in the classroom context, HLA and MLA can pose profound socio-emotional challenges for multilingual speakers in their everyday lives. Still, the term “language anxiety” has been adopted for heritage language research.

In his pivotal work, \citet{Tallon2011} introduced the term “heritage language anxiety”, emphasizing the need for detailed research on the anxiety experienced by heritage speakers. \citet[96]{PradaEtAl2020} define HLA as a “specific type of anxiety linked to the negative feelings of physical or emotional discomfort experienced in connection with using the HL”. HLA has been investigated in different languages (Chinese: \citealt{XiaoWong2014}; Korean: \citealt{Jee2016, Jee2022}; Spanish: \citealt{Tallon2009,Tallon2011}; Turkish: \citealt{ErgütBaş2023,Sevinc2014} et seq.). Compared to foreign language learners, heritage speakers tend to experience less anxiety when using their heritage language (\citealt{PradaEtAl2020,Tallon2009}). However, the degree of anxiety seems to be linked to different extra-linguistic factors such as communication partner, communication context, self-perceived proficiency, and socio-biographical aspects.

In a series of studies, \citet{Sevinç2014,Sevinç2018,Sevinç2020,Sevinç2022a,Sevinc2022Mindsets,SevinçBackus2019,SevinçDewaele2018} investigate HLA in three generations of Dutch-Turkish bilinguals in the Netherlands. \citet[162]{SevinçDewaele2018} highlight that the “challenges that immigrant communities face in a language contact situation vary across different geographical, social and political contexts (\citealt{Canagarajah2008}), and across different value systems underpinned by their identity, culture and so forth.” Hence, identity construction, cultural aspects and the societal context play a central role in HLA, which can best be observed when studying different generations. \citet{SevinçDewaele2018} find that the generation that immigrated from Turkey to the Netherlands exhibits very little HLA (though when speaking Turkish around supposedly monolingual Dutch speakers). However, their children, who were born in the Netherlands or arrived there at an early age, experienced more HLA when speaking Turkish in Turkey outside with perceived native Turkish speakers and with Turkish friends in Turkey. This kind of anxiety also applied to the children of the locally born Dutch-Turkish bilinguals. Additionally, this group experienced the highest level of HLA in all social contexts (i.e., communication with family, friends, and “native speakers”). These forms of anxiety manifest physiologically in electrodermal activity \citep{Sevinç2018} and are further influenced by monolingual-oriented family language policy \citep{Sevinc2022Mindsets}.

One of the driving factors in HLA is self-perceived proficiency, with higher self-perceived proficiency being correlated with less HLA (\citealt{Chhuon2011,Sevinç2016,SevinçBackus2019}). This factor might be related to the social and cultural conditions heritage speakers navigate daily. Heritage speakers face expectations concerning their cultural identity and constructed ethnicity, often associated with their languages. Studies brought up the vicious circle in which heritage speakers experience HLA, leading to avoidance and a decrease in proficiency, which, in turn, boosts HLA (see \citealt{Krashen1998}; \citealt{SevinçBackus2019,Wei1996}). Beyond self-perceived proficiency, \citet{SevincDewaele2018} suggest that feelings of social inequality (\citealt{Hudson1996}) and sensations of language pride and panic \citep{Martínez2006} further exacerbate HLA.

While most studies on heritage speakers emphasize the heritage language, a limited number delves into LA within the context of the heritage speakers’ majority language. \citet{SevinçDewaele2018} coined the term “majority language anxiety” (MLA), which they describe as “language anxiety experienced by immigrant or minority community members in the language of the majority of the population in a national context” \citep[176]{SevinçDewaele2018}. The study provided first insights into MLA in three generations of Dutch-Turkish bilinguals. They find that the generation that migrated to the Netherlands and their children experience intense MLA, particularly in conversation with or around supposedly monolingual Dutch speakers. Children of the locally-born generation did not experience MLA. \citet{Jee2022} reports similar findings for English{}-Korean bilingual speakers in Australia, where the immigrant generation felt MLA, while their children did not. However, in contrast to the generations investigated in the Netherlands, speakers from these two generations reported relatively low HLA.

Linking LA to the majority language of multilinguals is a rather new approach, as studies on anxiety in the majority language have mostly been investigated in monolingually raised speakers. First introduced by \citet{Labov2006}, linguistic insecurity (LI) refers to “speakers’ feeling that the variety they use is somehow inferior, ugly or bad’" (\citealt{Meyerhoff2006}: 292). The term, thus, initially referred to felt discrepancies between someone's variety and a valorized variety, often the standard variety. This valorized variety is looked up to, as it represents the baseline of correctness and the norm of an “idealized homogenous language” (\citealt{Lippi-Green2012}: 64). In his study on New York City English, \citet{Labov2006} finds that speakers from the lower middle-class, particularly women, experience LI, related to the desire to social upward movement. LI is thus linked to social factors such as socio-economic class and gender. Similar findings were reported for English in Winnipeg (\citealt{OwensBaker1984}) and Michigan (\citealt{Preston2013}).

LI was long investigated with a focus on monolinguals. However, particularly in the context of French, it has been considered a cross-linguistic topic since the 1990s, encompassing multilingual contexts and language contact settings (\citealt{BretegnierLedegen2002,Calvet1996,FrancardEtAl1993,Robillard1996}). It has since been described in various contexts, such as the workplace (\citealt{Cho2015,Lancereau-ForsterMartinez2018}) or the classroom (\citealt{DaftariTavil2017,GagliardiMaley2010,JantriPhusawisot2021}). LI research involving multilingualism often investigates insecurity in varieties of named languages (Catalan:  \citealt{BaldaquiEscandell2011}; Spanish: \citet{SuárezBüdenbender2010}; English in Singapore: \citealt{FooTan2019}) or minority languages and endangered languages (\citealt{BaldaquiEscandell2011,Dağdeviren-Kırmızıİnan2022,Shulist2022}). Studies in LI also more frequently tend investigate positive strategies in dealing with LI (e.g. \citealt{Lancereau-ForsterMartinez2018,LeeJang2023}) than LA studies.

As this brief review shows, LA and LI, while related, target different speaker groups and adopt distinct perspectives. LA is traditionally associated with language learning and thus learning achievement in foreign language acquisition and was later applied to the heritage language contexts, where research focuses on the socio-emotional well-being of the speakers. LI takes the more Labovian sociolinguistic approach to insecurity, focusing on the hierarchies between linguistic varieties within the same named language and sociolinguistic aspects impacting language use and evaluation. In both domains, speakers reportedly experience discomfort when using certain languages or varieties, and \citet{HorwitzEtAl1986} mention the fear of social evaluation as related to FLA, which supposedly also plays a role in LI. However, speakers manage these feelings in different ways. \citet{Labov2006} identifies hypercorrection as a result of LI, as (monolingual) speakers emulate structures of a more prestigious variety to avoid previously recognized errors. Hypercorrection is thus an active coping strategy that speakers apply to face LI. In the literature on LA, such perspectives that acknowledge speakers' agency seem to be a minority, if not completely absent. Speakers, particularly heritage speakers, are usually portrayed as affected by LA with little power over their languages and negative emotions. In contrast, heritage speakers avoid their heritage language due to anxiety, leading to language loss and social conflict. While this perspective is highly relevant for the investigation of the negative impact of LA on the individual, the sole focus on the heritage language and its avoidance overshadows two factors: First, heritage speakers are owners and in power over all their languages and varieties, including their majority language and heritage language, and second, the significance of the societal context, i.e., societal ideologies and their effects on heritage speakers' associations with languages and varieties.

Current predominating research perspectives on LA in heritage speakers thus run into danger of “Othering” speakers as victims without considering their full linguistic command and broader societal aspects. This kind of Othering also plays out in how speakers and their languages use are labelled. In the next sections, I explore practices of Othering in LA/LI research. I first discuss Othering through labelling as described for other subdisciplines in linguistics (\citealt{WieseEtAl2022Multilinguals}, Wiese, this volume) and through the labels “anxiety” and “insecurity”. Second, I highlight how including specific perspectives can avoid Othering and simultaneously offer important insights, broadening our understanding of LA. I discuss a) the role of different speaker groups and the societal context and b) the importance of repertoires and linguistic agency.

\section{Patterns of Othering in LA/LI research}
\label{8:sec:3}

The literature exhibits several Othering practices identified in \citet{WieseEtAl2022Multilinguals} and Wiese (this volume), particularly concerning labelling a) territorial belonging, b) national membership, and c) linguistic ownership. Drawing from \citet{WieseEtAl2022Multilinguals} and Wiese (this volume), I discuss these patterns using anonymized references. Codes mark the discipline, identified through the terms used in the respective study to describe cases of unease centring around language (LA = Language Anxiety, LI = Linguistic Insecurity), study number (01, 02, 03, …) and publication year. 

\subsection{Territorial belonging, national membership and linguistic ownership}
\label{8:sec:3.1}

The practice of Othering via territorial labels includes terms like “second generation”, “third generation”, and “(im)migrants” (LA02/2023, LA03/2022, LA05/2020, LA06/2022, LA08/2018, LI10/20210, LI12/2016), “immigrant communities” (LA05/2020, LA07/2019) for individuals that are locally born and thus, should be referred to as such, i.e. first- or second-generation locals (\citealt{WieseEtAl2022Multilinguals}). Compared to monolinguals, these speakers are labelled “immigrant peers” (LA02/2023). These speakers are often associated with “home countries” (LA05/2020), meaning where their ancestors originated, while the nations of their birth and residence are termed “host countries” (LA07/2019, LA08/2018). Occasionally, speakers are compared with speakers living in a “foreign country” (LI11/1984).

Othering by attributing national group labels is evident in terms like “Turkish”, “Koreans”, “Spanish”, “Turkish children”, and “Turkish friends”, when referring to speakers born and residing in another country, such as Germany, Australia, or the US (LA02/2023, LI10/2010, LA03/2022, LA06/2022, LA08/2018, LI14/2007). Some studies also include the term “living abroad” (LA02/2023) or “heritage country” (LA17/2020) for locally-born speakers.

Regarding linguistic ownership, many studies categorize heritage speakers as “L2” learners, even if they began acquiring the majority language from birth or early childhood (LA01/2008, LA15/2010, LA17/2013, LA19/2007). Monolinguals are often portrayed as the bearers of a language, setting a benchmark that heritage speakers should “catch up” to, given perceived differences in proficiency or performance (LA02/2023, LA18/2018, LA03/2022). Speakers are also contrasted with “native” speakers (LA18/2018, LA03/2022, LI12/2016). 

\subsection{“Anxiety” and “Insecurity”}

The practice of Othering through labelling also encompasses terms like “anxiety” and “insecurity”. LA research traditionally takes \citegen{Spielberger1983} work on anxiety as a starting point. Spielberger focused on the psychological condition of anxiety, which is then frequently adapted to describe negative feelings and emotions in learners or heritage speakers. Given this context, using the label “anxiety” could inadvertently pathologize it, framing it as a “condition” or “disorder”. Pathologizing LA might divert attention from the external factors that contribute to it, including societal pressures, unrealistic expectations, linguistic hegemony, or discriminatory attitudes towards certain languages. Concurrently, even though LA is often described as situation-specific in contrast to trait and state anxiety (\citealt{HorwitzEtAl1986,MacIntyreGardner1989},) “anxiety” has a more recognized pathological aspect, while “insecurity” often refers to a common temporary human experience, which can lead to anxiety and anxiety disorders. Importantly, LA research seems to single out monolinguals from multilinguals. LA is seldom linked with monolinguals and anxiety towards their only language. Thus, monolinguals’ experiences are constructed as fundamentally distinct from those of multilinguals, potentially pathologizing LA in the latter group but not the former.

Moreover, as FLA is traditionally studied in the classroom contexts (see \sectref{sec:08:2} above), it is often ascribed to external, surmountable reasons. For learners, LA is presented as a natural phenomenon when learning a new language, and thus, it is depicted as resolvable with increasing proficiency (see \citealt{Park2021} for a critical discussion). In contrast, HLA is depicted as an inherent trait of heritage speakers. It is conceptualized as the result of a self-perceived lack of competence, leading to avoidance and social conflict. Given this perspective, research frequently champions heritage language instruction without delving into wider societal factors contributing to HLA, like prevailing ideologies and attitudes.

The labelling practices highlighted above not only Other heritage speakers but also suggest a narrowed viewpoint on them. Including different perspectives in LA research might overcome this potential of Othering and provide new insights into LA in different speaker groups, as discussed in the next section. 

\section{Reconceptualizing perspectives in LA research} \label{8:sec:4}
\subsection{Speaker groups and societal contexts}

Research in LA/LI often generalizes speakers by concentrating on similar demographic groups. They tend to neglect different societal prestige associated with certain languages but focus on speaker groups that often face discrimination against them in specific countries, like heritage speakers of Turkish in Germany or heritage Spanish speakers in the US. However, speakers of languages with high societal prestige may experience LA/LI differently, revealing underlying hegemonic societal patterns. For example, a heritage speaker of German in Namibia or the US or a heritage speaker of English in Germany might face less discrimination and expectations, and therefore, LA/LI might affect them less strongly. This perspective shifts the focus from multilingualism as a potential threat to emotional well-being to larger societal issues related to power relations and the hegemonic status of specific languages, varieties and speakers. Grouping all multilingual speakers’ experiences under a singular narrative, risks obscuring broader societal challenges, especially prevalent ideologies and power dynamics. 

Within this context, focusing on societal circumstances that play a role in LA provides an important factor in understanding LA and LI. LA might arise less from a self-perceived deficiency in proficiency and more from ingrained notions of linguistic competence and expected judgments from one’s environment. The notion of competence is strongly shaped by the majority of society, whose members are themselves impacted by linguistic ideologies. In particular, in the Global North, monolingual ideologies strongly influence the conceptualization of language(s) and their speakers (\citealt{Blackledge2000, Blackledge2002}). These ideologies prioritize monolingual linguistic practices, including the lack of language mixing and code-switching and are often closely related to standard language ideology (\citealt{Lippi-Green2012}), valorizing standard varieties over other varieties. Both, monolingual and multilingual speakers might be affected by these ideologies. However, multilingual speakers to a larger extent by monolingual ideologies. Contextualizing speakers’ feelings of deficient competence with larger societal factors is thus essential to understanding an important aspect of the sources of LA. Simultaneously, it lifts the burden from the heritage speakers and problematizes macro societal responsibilities.

\subsection{Repertoires and agency}

Studies in LA/LI research usually do not consider the present linguistic resources of the speakers but rather imply that improving language competence, often understood as the standard variety of a language, is key to solving issues of anxiety and insecurity. This perspective suggests that the perceived lack or insufficient proficiency of specific varieties as part of the multilingual repertoire contributes to anxiety, overlooking the value of heritage speakers’ linguistic repertoires, encompassing their heritage language skills. While many studies aptly emphasize the importance of heritage language education in addressing LA/LI, acknowledging the linguistic resources that heritage speakers already bring also means accepting their position as native speakers and language ownership. \citet{Schroeder2003} posits that, in heritage language instruction, prioritizing metalinguistic awareness of the speakers' resources outweighs merely teaching the standard variety. Consequently, research implies that the source of LA is the lack of competence in the heritage language, while another factor might be societal acknowledgment of the speakers’ linguistic resources. In addition, studies barely focus on strategies that speakers apply to manage LA/LI but focus on language avoidance, shift, loss, and social conflicts. Refocusing on heritage speakers’ resources and coping mechanisms recognizes linguistic agency without downplaying their daily challenges. This would underline that speakers are owners of their languages, taking the initiative to manage anxiety and insecurity. For instance, in her study, \citet{Versteegh2023} found that German heritage speakers use diverse strategies in digital communication within their heritage language to confront feelings of anxiety and insecurity, and \textcite{Bunk2024} discusses corpus data, indicating that multilingual speakers tend to mark formal language more clearly than monolingual speakers. While this might not apply to all heritage speakers, it is imperative to recognize that acknowledging agency also means capturing diverse coping mechanisms. 

Exploring these coping mechanisms can be achieved by factoring in speakers’ metalinguistic knowledge and their linguistic behavior in the majority language—a perspective largely missing in the current literature (see, however, \citealt{SevinçDewaele2018}). In the following section, I suggest an analysis of LA, focusing on the majority language of heritage speakers in Germany, their experiences of MLA and specific coping mechanisms. In order to avoid the negative implications that come with the terms “anxiety” and “insecurity”, I refer to socio-emotional discomfort that speakers relate to language use with the term “linguistic pressure” (LP). The term encompasses all speakers, mono- and multilinguals, without denying potential group-specific differences.

\section{Awareness and linguistic pressure:  Insights from interviews} \label{8:sec:5}
\subsection{Method and informants}

Data are drawn from two series of semi-structured interviews. The first series was conducted in order to assess linguistic pressure in bilingual speakers in Germany, focusing on German. We conducted four semi-structured individual interviews with four bilingual speakers: two German-Russian bilinguals (31 and 23 years of age) and two German-Turkish bilinguals (54 and 22 years). All speakers were born in Berlin or came to Berlin at a very young age (between 8 months and five years). They grew up in Berlin and reported German and Russian or  German and Turkish as their family languages. Each interview was divided into three parts covering different subtopics. Part I targeted potential factors that might trigger linguistic pressure, i.e., ideologies and attitudes towards multilingualism and standard language. Part II investigated whether speakers experienced linguistic pressure in specific situations, i.e., informal vs. formal situations. Part III asked how speakers adapt their linguistic behavior due to such linguistic pressure. Each interview lasted approximately 30 minutes.

We derived the second series of interviews from the ongoing research project titled “Language Anxiety as a Barrier to Academic Participation”\footnote{PIs: Esther Jahns, University of Oldenburg, Oliver Bunk, Humboldt-University of Berlin}. The project assesses which factors influence academic participation in the university classroom and focuses on the socio-economic and linguistic background of the participants. Semi-structured interviews consisted of several questions centered around these topics. We conducted one focus group interview with four students and 12 individual interviews (all students at the University of Oldenburg). The duration of these interviews ranged between 45 to 60 minutes. For this study, the data provided by a 23-year-old speaker were selected as for them in particular, multilingualism was a central component in academic participation. 

\subsection{Data analysis} 

Data were recorded and transcribed. Qualitative structured content analysis was applied \citep{KuckartzRädiker2022} using \citealt{MAXQDA2022}  (\citealt{VERBISoftware2021}). The questions asked in the semi-structured interviews provided broad categories for data annotation. Based on the interview data, these categories were continuously restructured, broken down into subcategories and merged during the annotation process and before analysis. In the following, I primarily focus on two key topics: 1) the significance of Standard German, formal language, and belonging, and 2) the coping mechanisms employed.

\subsection{Findings}
\subsubsection{Standard German, formal language, and belonging}

Participants distinguished between what they considered “good” and “bad” German, highlighting that formal German is the superior variety. Standard is characterized by an “elevated” (\textit{gehoben}) way of speaking, the usage of “technical terms” (\textit{Fachwörter}) and “grammatical rules and laws” (\textit{grammatikalische Regeln und Gesetze}), and it was associated with formal situations. “Bad German” was identified by a lack of vocabulary (\textit{Wortschatz}), or having an accent (\textit{Akzent}), and the use of “Slang”. Speaking “good” German was often associated with the aspiration for respect, credibility, and recognition of intelligence. This perception of “good” German is not unique to heritage speakers but rather illustrates internalized standard language ideology. As illustrated in \REF{ex:08:1}, Standard German is seen as essential to demonstrate general German command\footnote{Repetitions, restarts, slips of the tongue have been deleted in all quotes to increase readability.}:

\ea\label{ex:08:1}
{mit irgendwelchen Beamtendeutsch, wenn man solche Wörter benutzen kann und anwenden kann, dann würde ich das anwenden, um natürlich auch dem Gegenüber zu zeigen, dass ich auch gut Deutsch sprechen kann.                   (SG\_hT, 21:44)}\\
\z

Notably, Standard German is not just viewed as a refined variety in its own right, but it is also considered essential to be acknowledged by the (monolingual) majority as a competent speaker when this majority constructs the speaker as a “foreigner”. 

\ea\label{ex:08:2}
\ea\label{ex:08:2a}
{  sagen wir mal, man sieht ausländisch aus, […] man   \\
     wird einfach anders wahrgenommen, wenn das Deutsch    \\
     perfekt ist.                                                (CS\_hR, 3:56)}\\
\ex\label{ex:08:2b} { die Leute gehen ja meistens nicht davon aus, dass wenn man ausländisch aussieht, das was aus dem Mund raus kommt richtig ist.           (CS\_hR, 23:25)}\\
\z
\z

\REF{ex:08:2b} indicates that speakers constructed as foreigners are also constructed as having deficient German abilities. This perception lays the ground for high self-expectations concerning speaking German, as illustrated in \REF{ex:08:3}:

\ea\label{ex:08:3}
{das ist auch meine eigene Erwartung an mich selbst, dass ich auch auf diesen, auf diesen Niveau komme und ja, mich genauso verständigen kann, so wie gebürtige Deutsche sage ich mal.              (AC\_hR,16:57)} \\
\z

Even though the informant states that they came to Berlin at a very early age (8 months), they compare their own German to “born Germans” and associate this way of speaking with the norm. This notion might suggest the informant feels their German might not match up to that of monolinguals. If that perception is based on actual differences, this might indicate internalized monolingual ideologies devaluing multilingual practices. However, a second interpretation is that the speaker only \textit{perceives} their language as different from that of monolinguals, while it does not differ. An anonymous reviewer aptly pointed out that this might also reflect general ideologies of language purism or standard language ideology. However, the important point here is the comparison with “native Germans” (\textit{gebürtige Deutsche}), who are typically constructed as monolingual and thus considered benchmark speakers and language owners. Thus, monolingual ideologies and Standard German are two aspects of the same ideal notion of “good” and “correct” German. However, multilingual speakers might be affected more by monolingual ideologies than so-called monolingual speakers, while standard language ideologies impact both groups (see \citealt{Bunk2024} for a more detailed discussion). 

Like other participants, this informant did not explicitly convey feelings of stress, anxiety or insecurity concerning their German, but a heightened awareness of how monolinguals might perceive their way of speaking and how this impacts their perception as a German. This awareness was particularly pronounced in formal communicative situations. Using Standard German in formal situations was not only perceived as proving German proficiency but also prevented the speaker from being constructed as a foreigner and being accepted as a German in-group member. 

\ea\label{ex:08:4}
{Also so wie ich mit meinen ausländischen Freunden rede, rede ich jetzt nicht auf der Arbeit oder in der Universität, […] Deshalb denk ich jetzt nicht, dass man mich irgendwie, also als ungebildet oder seltsam oder ausländisch wahrnimmt, ich denke da nimmt man, man nimmt mich meistens als ganz normale deutsche Frau wahr.                      (CS\_hT, 15:47)} \\
\z

Standard German is thus closely connected with the Germanness of multilingual speakers and is thus essential in preventing societal Othering. In contrast,
monolingual speakers’ Germanness is not judged based on their command of Standard German. However, this seems to be the case for the bilingual speaker in \REF{ex:08:4}. \REF{ex:08:5} encapsulates this perspective. Replying to the question whether the informant thinks someone who is constructed as a foreigner through appearance or name has to prove more that they speak “proper” German than someone who is perceived as a monolingual German speaker, the informant states:

\ea\label{ex:08:5}
{Ja! Ja. Also ich weiß das zu hundert Prozent (CS\_hT, 23:25)}\\
\z

Participants described scenarios where they feel at particular risk of being labelled as ‘foreign’ and when using Standard German is thus particularly relevant. These contexts included governing agencies (“Amt”) and the university. One of the informants highlighted that they felt pressured to demonstrate their German skills to avoid stigmatization and discrimination in classes. Replying to the question whether the informant had the feeling that they had to be “better” compared to monolingual students, they replied:

\ea\label{ex:08:6}
{Ja, auf jeden Fall. Also, ich hab da auch mal ganz oft das Gefühl gerade so auch in Germanistik, […] manchmal fühlt man sich halt auch unwohl, […] und ganz oft ist es dann so: “Ja huch, Sie sprechen ja ganz akzentfrei”.                  (03Fbiac, Pos. 50-52)}\\
\z

These emotions can lead to barriers to participation in the classroom and anxiety. Interestingly, anxiety is not directly related to the majority language:

\ea\label{ex:08:7}
{Angst in dem Sinne, dass du halt abgestempelt wird, weil ich ja halt noch diesen Migrationshintergrund habe und ich mir so denke […] dass man sich so denkt: “Du wirst jetzt abgestempelt.” Und so nach dem Motto: “Ja eben durch diesen Migrationshintergrund kann sie sich sprachlich nicht so ausdrücken, dass es gerade inhaltlich passt.” Und wenn ich […] dann unsicher bin und denke: “Ich kann mich vielleicht fachsprachlich nicht gut genug ausdrücken.” dann halte ich mich auch zurück, damit ich halt nicht einfach abgestempelt werde.   (03Fbiac, Pos. 249-250)}\\
\z

\REF{ex:08:7}  illustrates that anxiety does not stem from the speaker's doubts about their German competencies but from the outside perspective and the evaluation of their German skills based on others reading them as a person with a “migration background”. The example illustrates that the speaker is familiar with the expectations of the supposedly monolingual majority in their classroom. However, these associations might lead to insecurities concerning the speaker’s language skills concerning specialized academic language (‘fachsprachlich’). In order to avoid being stereotyped, the speaker is more hesitant in classroom participation. Still, insecurity does not stem from a self-perceived lack of proficiency but appears to be caused by potentially negative evaluations of classmates.

\subsubsection{Coping mechanisms and agency}

Participants outlined multiple strategies to confront such stigmatization, rooted in their understanding of the significance of Standard German. These strategies encompassed both performative actions and modifications in their linguistic behavior. Performative actions included exploiting specific situations, e.g. conversations on the phone while avoiding others. \REF{ex:08:8} illustrates such a situation:

\ea\label{ex:08:8}
{Am Telefon ist es immer ganz gut, da merkt man zum Beispiel also, hat man mir nicht immer angemerkt, dass ich jetzt nicht Deutsch bin, das hat natürlich auch seine Vorteile, ja.               (SG\_hT, 4:08)}\\
\z

Speaking over the phone lets participants sidestep potential biases based on their appearance. Concurrently, the speaker avoids the pitfalls of self-fulfilling assumptions that suggest a struggle with German if they are read as a foreigner. Similarly, online teaching provides an alternative to avoid Othering and stigmatization at the university, as illustrated in \REF{ex:08:9}.

\ea\label{ex:08:9}
{So in Präsenz kannst du ja nicht sagen: “[…] ich zeig mich jetzt nicht oder, […] dass man nicht sieht, welche Haare ich hab' und da drauf irgendwie keinen Rückschluss auf meinen Hintergrund irgendwie zieht.”  Sondern du kannst ja wirklich sagen: “Hey, wenn ich jetzt meine Kamera ausschalte, würde man ja auch nicht wissen, dass ich 'n Migrationshintergrund hab”.   (03Fbiac, Pos. 354-356)}\\
\z

Informants also demonstrated high resilience and progressive action when discriminated against due to their language. An informant recounted an instance where they were barred from taking a final examination, as their presentation was too erroneous. Their partner and they felt that this was a wrong decision that was not based on their performance but on their “migration background”, which made the teacher, from their perspective, believe that the presentation was, linguistically, not good enough. Both students confronted the teacher and demanded to complete the examination nonetheless. Their evaluation ended up being very positive. However, it left the student with anguish:

\ea\label{ex:08:10}
{Wenn dieser Mensch schon im Vorfeld irgendwie voreingenommen ist und so ähm stereotypisch denkt oder auch so klischeehaft denkt und du dann auch noch wirklich so dich unter ihm positionierst und ihm das Gefühl gibst: “Ok, er hat das erreicht, was er erreichen wollte oder in dir auch bewirken möchte.” Da denke ich immer so: “Nein. Gib diesem Menschen nicht das Gefühl und beweis' ihm das Gegenteil.” Aber im Endeffekt weiß er ja auch nicht, wie es in mir aussieht und so. Und sowas hinterlässt schon Spuren.                               (03Fbiac, Pos. 98)}\\
\z

 The quote highlights the student’s associations between mistakes pointed out by the teacher and the danger of being stereotyped, othered and devaluated due to their “migration background”. An important factor seems to be social hierarchies and power relations: being affected by stereotypes also comes with the risk of giving in to constructed power relations, placing the speaker below the other person, who is not affected by that particular type of stereotyping and Othering. The speaker in \REF{ex:08:10} tackles this aspect by applying another coping strategy that seems to emerge when facing potential language-based stigmatization: proactive confrontation, which might be related to the speaker’s need to prove stereotypes wrong, as indicated in the short inner dialogue.

From a structural perspective, participants mentioned that they often avoided accents and adjusted their speech according to the communicative context, particularly in formal situations, as illustrated in \REF{ex:08:11}.

\ea\label{ex:08:11}
{Wenn ich in einer formellen Situation bin, habe ich das Gefühl, dass ich alles etwas, ich muss halt alles gehobener ausdrücken, die Formulierungen werden teilweise auch länger, und auch viel mehr Nebensätze. […] Entweder kantiger und mit viel Nebensätzen oder halt sehr kurz. Wohingegen, wenn ich jetzt unter meinen Freunden bin […], dann achtet man halt nicht auf jeden Satz, […] und man überlegt einfach ein bisschen weniger und man hat nicht das Gefühl darauf achten zu müssen, dass man sich jetzt besonders artikuliert ausdrückt      (CS\_hT, 17:02)}\\
\z

While this strategy may not be unique to multilinguals, the interview data indicate that multilinguals amplify these differences, primarily due to their heightened awareness of the significance of adhering to Standard German. This aligns with previous studies, indicating multilinguals’ acute cognizance of their language use across diverse communicative contexts (\citealt{BunkPohle2019}) and a tendency to linguistically mark specific communicative situations more intensely. \citet{WieseEtAl2022Multilinguals} e.g. found that bilinguals use more non-canonical patterns in informal situations than monolinguals. Such research has primarily focused on multilinguals’ informal language use, leaving formal registers relatively unexplored (but see \textcite{Bunk2024} for multilinguals’ language use in formal settings). The interview data indicate that multilinguals’ formal language might offer an interesting perspective regarding linguistic pressure, especially given that formal contexts are associated with Standard German and monolingual practices, certain multilingual speakers might feel the need to align with.

\section{Conclusions and outlook} \label{8:sec:6}

This chapter explored Othering practices in research on language anxiety (LA) and linguistic insecurity (LI). I highlighted that these patterns align with previous findings (\citealt{WieseEtAl2022a}, Wiese, this volume) and include specific types of labelling referring to territorial belonging, national membership, and linguistic ownership. Additionally, the literature is dominated by specific perspectives on multilingual speakers and language anxiety, providing a basis for Othering. In particular, the terms “language anxiety” and “linguistic insecurity” might be problematic due to their specific connotations and pathological associations. I suggested the term “linguistic pressure” (LP) as an alternative. I argued that a shift in perspective concerning the speaker groups, the societal context, speakers’ linguistic competence, registers and agency might further diminish the threat of Othering in the field. By presenting interview data on linguistic pressure experienced by German speakers, I demonstrated how a change in perspective can deepen our comprehension of heritage speakers’ coping mechanisms in situations that evoke LP.

The findings indicate that participants do not necessarily experience anxiety in the majority language but are aware of societal expectations they feel pressured to fulfill. Speakers feel they need to prove their German proficiency more than monolinguals, as they are often constructed as foreigners, questioning their place within society. In this process, formal, Standard German is particularly important, as it is not only used to present the speakers as intelligent and educated, features that are traditionally associated with Standard German. It is also viewed as a benchmark used by the monolingual majority to gauge whether a multilingual speaker belongs in society. Recognizing these factors, speakers employ distinct extra-linguistic strategies to counter these challenges, e.g., performative actions. Interestingly, the awareness of the importance of Standard German also seem to play out in the structural plane, e.g. in language use in formal communicative situations. \textcite{Bunk2024}, e.g., found that multilingual speakers use more markers of formal language and less markers of informal language than monolinguals to mark formal communicative situations.

Multilingual speakers, at least those investigated in the interviews, demonstrate full command of their majority language and high meta-linguistic knowledge. They did not report feelings of anxiety or insecurity, even though when asked to describe how they feel in particular situations where they use German. However, they demonstrated a strong awareness of how others perceive their language use. Pressure was mainly caused by the evaluation by the majority society and not by a self-perceived lack of competence. While research primarily focuses on this aspect as one of the driving factors of anxiety, the societal aspect is often overlooked (but see \citealt{Park2021}). In particular, negative societal attitudes towards multilingualism might strongly contribute to LP, as the interview data indicates. In societal contexts with a less negative perspective on multilingualism, LP might not be as strong or completely absent as multilingual practices are perceived as the norm. Given that most LP research involves speakers from the Global North, LA/LI research lacks an important perspective from the Global South, where multilingual practices are often perceived as the norm.

\citet{SevinçAnthonissen2022} offer first insights into such a context in their investigation of language anxiety in Cape Town, South Africa. They show that their informants exhibited only few negative emotions regarding their (multilingual) language use. Their study indicates that LP depends greatly on the societal context. Studies on LP predominantly focus on societies in the Global North, where monoglossic ideologies and a monolingual mindset (\citealt{Clyne2005}) are particularly strong. \citet{Sevinç2020,Sevinc2022Mindsets} highlights that this type of ‘aggressive monolingualism’ can spread from one generation to another and amplify LP in heritage speakers, especially when monoglossic ideologies predominate the perspective on both majority and heritage language. Contexts that perceive multilingualism as the norm might lead to less LP for the individual and different coping mechanisms. Comparing LP in societies with monolingual orientations with societies with multilingual orientations might thus reveal the negative effects that certain ideologies towards multilingualism in most countries in the Global North have on individuals, especially multilingual ones.

LP in the heritage language context is a genuine concern at an individual level. Research underscores its association with negative self-perceptions, shame, language loss and shift, which can lead to intergenerational tension and far-reaching psychological consequences for the individual (\citealt{Purkarthofer2020,Sevinç2020}). However, in addition to the individual level, considering the larger societal context is essential to the question of how LP emerges. This shift in perspective would also imply rethinking the source of LP. Many studies depict LP primarily as an intrinsic condition, mostly triggered by (self-)perceived lack of proficiency and competence that can be overcome by improving these areas (see \citealt{Park2021}: 130). However, anxiety also results as a social and cultural construction if speakers feel they cannot meet certain societal expectations. Overlooking this societal component burdens the speakers, sidelining society’s significant role and the social macro context in which speakers navigate.

\section*{Acknowledgments and funding}

Research for this article was funded by the Deutsche Forschungs-gemeinschaft (DFG, German Research Foundation) for the Research Unit “Emerging Grammars in Language Contact Situations” ({FOR 2537}, Projects P8/313607803 and P9/313607803) and the {CRC 1412} “Register: Language Users' Knowledge of Situational Variation” (Project C07/416591334). I thank the editors of this volume, two anonymous reviewers, and Annika Milena Schimpff for their valuable feedback.

\printbibliography[heading=subbibliography,notkeyword=this]
\end{document}
