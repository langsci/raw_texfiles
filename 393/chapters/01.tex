\documentclass[output=paper,colorlinks,citecolor=brown]{langscibook}
\ChapterDOI{10.5281/zenodo.17132439}
\author{Artemis Alexiadou\orcid{}\affiliation{Leibniz-Zentrum Allgemeine Sprachwissenschaft (ZAS); Humboldt Universität zu Berlin} and Claudio Scarvaglieri\orcid{}\affiliation{Université de Lausanne} and Christoph Schroeder\orcid{}\affiliation{Universität Potsdam} and Heike Wiese\orcid{}\affiliation{Humboldt-Universität zu Berlin}}
\title{Introduction: Multilinguals as Others}
\abstract{}
\IfFileExists{../localcommands.tex}{
  \addbibresource{../localbibliography.bib}
  % add all extra packages you need to load to this file

\usepackage{tabularx,multicol}
\usepackage{url}
\urlstyle{same}

\usepackage{listings}
\lstset{basicstyle=\ttfamily,tabsize=2,breaklines=true}

\usepackage{langsci-basic}
\usepackage{langsci-optional}
\usepackage{langsci-lgr}
\usepackage{langsci-osl}
% \usepackage{./langsci/styles/langsci-lgr}
% \usepackage{./langsci/styles/langsci-osl}
% \usepackage{langsci-gb4e}

\usepackage{tikz}
\usetikzlibrary{patterns,calc}
\pgfdeclarepatternformonly{south east lines}{\pgfqpoint{-0pt}{-0pt}}{\pgfqpoint{3pt}{3pt}}{\pgfqpoint{3pt}{3pt}}{
    \pgfsetlinewidth{0.6pt}
    \pgfpathmoveto{\pgfqpoint{0pt}{3pt}}
    \pgfpathlineto{\pgfqpoint{3pt}{0pt}}
    \pgfpathmoveto{\pgfqpoint{.2pt}{-.2pt}}
    \pgfpathlineto{\pgfqpoint{-.2pt}{.2pt}}
    \pgfpathmoveto{\pgfqpoint{3.2pt}{2.8pt}}
    \pgfpathlineto{\pgfqpoint{2.8pt}{3.2pt}}
    \pgfusepath{stroke}}
    
\usepackage{stmaryrd}
\usepackage{wasysym}
\usepackage{multirow}
\usepackage{caption}
\usepackage{subcaption}
\usepackage{mathrsfs}
\usepackage{qtree}

\usepackage{linguex}


  %pminos do not split footnotes
% \interfootnotelinepenalty=10000 %Footnote in Laporte chapters has to be split SN


%\DeclareIndexNameFormat{default}{%
%\nameparts{#1}%
%\usebibmacro{index:name}%
%{\index[names]}%
%{\namepartfamily}%
%{\namepartgiveni}%
% {}% L1
% {}% L2
%{\namepartprefix}% generates spurious space L3
%{\namepartsuffix}% generates spurious space L4
%}

%  {\DeclareIndexNameFormat{default}{%
%     \usebibmacro{index:name}{\index[names]}{#1}{#3}{#5}{#7}}}

%\DeclareIndexNameFormat{default}{%
%  \usebibmacro{index:name}{\sindex[nom]}{#1}{#3}{#5}{#7}}

%\DeclareIndexNameFormat{default}{%
%  \usebibmacro{index:name}{\sindex[person]}{#1}{#3}{#5}{#7}}
%\DeclareIndexNameFormat{default}{%
%\nameparts{#1} \usebibmacro{index:name}{\sindex[person]]}{\namepartfamily}{‌​\namepartgiven}{\nam‌​epartprefix}{\namepa‌​rtsuffix}}

%\newcommand{\smiley}{:)}

%\renewbibmacro*{index:name}[5]{%
%\usebibmacro{index:entry}{#1}%
%{\iffieldundef{usera}{}{\thefield{usera}\actualoperator}\mkbibindexname{#2}{#3}{#4}{#5}}}

% \newcommand{\noop}[1]{}

%remove for final
%\overfullrule=1mm

\newcommand{\tobi}[2]}}
\renewcommand{\S}[1]{\tobi{#1}{\textsc{*}}}

% this volume references
% puts: [this volume]
% already defined: \citetv
%\newcommand{\citepv}[1]{(\citeauthor{#1} \citeyear*{#1} [this volume])}
\newcommand{\citealtv}[1]{\citeauthor{#1} \citeyear*{#1} [this volume]}

%parentheses around example number
\newcommand{\pref}[1]{(\ref{#1})}

% in-text examples

\newcommand{\lnex}[1]{\textit{#1}} %target lang word
\newcommand{\lnlit}[1]{(lit.: `#1')} %literal reading
\newcommand{\lnlat}[1]{(#1)} % latinization
\newcommand{\lntrans}[1]{`#1'} %translation
\newcommand{\lnexl}[2]%
{\lnex{#1}{} \lnlat{#2}} % ex with latinization
\newcommand{\lnexlat}[3]{\lnex{#1}{} \lnlat{#2}{} \lntrans{#3}} % ex with latinization and tranl.

%ch01
\newcommand{\co}[1]{\mbox{\textbf{#1}}}

%ch09

\newcommand{\cyrbulg}[1]{\begin{otherlanguage*}{bulgarian}#1\end{otherlanguage*}}


%ch10
\newcommand{\nlp}{{\small NLP}}
\newcommand{\mwe}{{\small MWE}}
\newcommand{\rae}{{\small RAE}}
\newcommand{\lvc}{{\small LVC}}
\newcommand{\pos}{{\small P}o{\small S}}
%\newcommand{\todo}[1]{ \textcolor{red}{#1} }

%\renewcommand{\labelenumi}{\theenumi}
%\ainamefmt{{vv}{ll}{, ff}{, jj}} % fullname

\newcommand{\biberror}[1]{{\color{red}#1}}

\newcommand{\osenovaitem}{--~}
  %% hyphenation points for line breaks
%% Normally, automatic hyphenation in LaTeX is very good
%% If a word is mis-hyphenated, add it to this file
%%
%% add information to TeX file before \begin{document} with:
%% %% hyphenation points for line breaks
%% Normally, automatic hyphenation in LaTeX is very good
%% If a word is mis-hyphenated, add it to this file
%%
%% add information to TeX file before \begin{document} with:
%% %% hyphenation points for line breaks
%% Normally, automatic hyphenation in LaTeX is very good
%% If a word is mis-hyphenated, add it to this file
%%
%% add information to TeX file before \begin{document} with:
%% \include{localhyphenation}
\hyphenation{
    Beck-man
    Ngu-yen
    back-chan-nel
    back-chan-nels
    mo-not-o-nous
    ste-reo-typ-i-cal
}

\hyphenation{
    Beck-man
    Ngu-yen
    back-chan-nel
    back-chan-nels
    mo-not-o-nous
    ste-reo-typ-i-cal
}

\hyphenation{
    Beck-man
    Ngu-yen
    back-chan-nel
    back-chan-nels
    mo-not-o-nous
    ste-reo-typ-i-cal
}

  \togglepaper[1]%%chapternumber
}{}

\begin{document}
\maketitle
%\shorttitlerunninghead{}%%use this for an abridged title in the page headers

\section{Multilinguals as Others}

Multilingualism is the normal condition for contemporary as well as historical human societies (e.g., \citealt{Grosjean2010}). However, European nation-state building has led to a strong ``monolingual habitus'' \citep{Gogolin2002} that constructs a community of monolingual speakers as bearers of a nation (\citealt{Ortega2009}, \citealt{Grosjean2010}, \citealt{Cook2016}). This erases or exoticises multilinguistic practices and excludes multilingual speakers. 

The effects of this exclusion are visible in the public discourse on multilingual speakers, where we find a widespread ``Othering'' of multilingual speakers, understood as constructing them as members of a social and linguistic out-group (\citealt{LamontMolnar2002}, \citealt{Dervin2015}). Such Othering is, for example, evident for speakers from heritage language backgrounds, that is, speakers who have grown up with an additional language that is not the societal majority language, typically as a result of migration in an earlier generation (e.g., immigrant parents or grandparents). Even if such speakers have been born and raised locally, their belonging to the national and linguistic in-group is often disputed, and they are socially and linguistically excluded.

Such Othering is not restricted to public discourse but is also found in our own practice as professionals working in linguistics and related fields. In earlier accounts, multilingualism and language contact were often regarded as exceptional. Multilingualism was seen as a cognitive problem, or multilingual speakers were regarded as a data problem. The first pattern is illustrated by a quote from \citet[148]{Jespersen1922}, who states:

\begin{quote}
First of all, the child [...] hardly learns either of the two languages as perfectly as he would have done if he had limited himself to one.~[...] he does not really command the fine points of the language. [...] Secondly, the brain effort required to master two languages instead of one certainly diminishes the child's power of learning other things which might and ought to be learnt.
\end{quote}

This pattern of problematising multilingualism was present all the way to the 1960s \citep{Athanasopoulos2016}, as seen, for instance, in \citegen{Weisgerber1966} claim that early multilingualism leads to mental, cognitive, and moral problems.

The second pattern, seeing multilingual speakers as a potential problem for the data, led to excluding them from linguistic analysis, as evident in the earlier structural linguistic tradition, for instance in \citegen{Saussure1916} focus on an idealised, stable, and implicitly monolingual language system (``forme idéale'') and \citegen{Chomsky1965} assumption of an ideal monolingual (native) speaker-listener.

Since then, the field has moved forward, and today multilingualism is generally accepted as a normal condition of human language in our discipline. Diversity is understood as a central aspect of language, captured, for instance, through approaches to linguistic multi-competence (cf. contributions in Cook \& Li \citealt{Wei2016}), and by such concepts as translanguaging, metrolingualism and others within what \citet{Pennycook2016} called the ``trans-super-poly-metro movement'' (e.g., \citealt{JaspersMadsen2019}).

This has led to a review of research perspectives, categories, and terminology, including: re-evaluations of such concepts as ``native speaker'', in particular in second language acquisition,\footnote{See \citet{Ortega2009, Cook2016, GudmestadEtAl2022, ShadrovaEtAl2021}.} and lately also in other domains such as psycholinguistics and heritage language research;\footnote{Cf. \citet{ChengEtAl2022} for psycholinguistics; \citet{TsehayeEtAl2022, WieseEtAl2021, WieseEtAl2022Multilinguals, RothmannEtAl2023} on heritage language research.} critiques of hegemonic assumptions of monolingualism, standard language, and its links with race/ethnicity;\footnote{For instance, \citet{RosaFlores2017}.} and an engagement with post-colonial studies, which have recently received growing interest in linguistics.\footnote{Cf. \citet{RosaFlores2017, Warnke2019, RashHoran2020, DeumertEtAl2020}.}

This also involved the development of alternatives to hegemonic terminology surrounding multilingualism and multilingual speakers. Examples for this are the concept of ``(trans-)languaging'' for the use of linguistic resources in multilingual practices (e.g., \citealt{García2009}), the introduction of the term ``new speakers'' to replace the notion of second language learners (e.g., \citealt{ORourkePujolar2015}), or an understanding of new linguistic practices among adolescents as ``contemporary urban vernaculars'' \citep{Rampton2010} or ``new urban dialects'' (\citealt{Wiese2013, KerswillWiese2022}), rather than ethnolects. 

So, does that mean that today, we practice what we preach? We believe that there is a still a way to go in order to achieve this. The critical discussion of othering multilingual speakers is still ongoing, and furthermore, it has been mostly restricted to contemporary sociolinguistics and not spread substantially to other areas of linguistics. A closer look at our field shows that the effects of exoticising and problematising multilingualism still overshadow our practice today, and this includes sociolinguistic, contact-linguistic, psycholinguistic, and typological research. 

Addressing this issue, the contributions to this volume take a closer look at Othering practices not only in the public discussion and educational practice, but also in academia, with a focus on linguistics.\footnote{Cf. \citet{Dirim2016} on such related fields as pedagogics and language didactics.} They provide critical reflection of common practices in our own field, and discuss the implications and challenges of this for our research. The contributors address conceptual framing and labelling, methodology, and research biases in a broad spectrum of approaches. They discuss the social context of Othering in linguistics (Scarvaglieri), labelling practices in published work from linguistics and related fields (Wiese), and the construction of multilinguals as Others in a range of linguistic subdisciplines, including heritage language syntactic research (Alexiadou; Bunk), second language acquisition (Gamper, Schroeder, Schlauch \& Steinbock), language teaching (Dirim), descriptive and documentary linguistics (Lüpke), and in outreach activities (Purkarthofer).

\section{Contributions to this volume}

Claudio Scarvaglieri’s paper (Ch.2) starts the volume with a discussion of the social context in which Othering in linguistics occurs, thus serving as a background for the other chapters. As he argues, linguistic research operates within specific societal contexts and is influenced by them. An example are wide-spread folk concepts of ‘Us’ vs. ‘Them’ that are involved in othering multilinguals and can also affect linguistic research. Scarvaglieri presents an analysis of such Othering in public discourse using a corpus of 49 news media texts from Germany, Austria, and Switzerland. He finds patterns of labelling multilinguals as geographic, national, or religious Others that have parallels with those evident in linguistics (see Ch.3). Analysing negative evaluations of multilingual Others, he shows that Others are characterised by deviations from liberal values and blamed for societal problems and challenges.

In Ch.3, Heike Wiese continues with the topic of labelling with a focus on labelling practices in academic discourse. Her analysis of~Othering in published work from linguistics and related fields of sociology and education reveals~recurring topoi that feed into three strands of Othering, namely Othering with respect to territorial belonging,~to national group membership, and~to linguistic ownership. Examples come from publications across different perspectives, subdisciplines, and research domains, underlining how widespread such practices are in our field. Wiese argues that avoiding such Othering is not only important from the point of view of scholarly terminology, but also for linguistic research perspectives: if multilinguals are constructed as Others, this can lead to an implicit bias with negative effects on linguistic research. 

Alexiadou’s contribution (Ch.4) takes up the issue of research perspectives by discussing methodological Othering. Looking at research on heritage speakers’ grammars and lexicons, she targets Othering through monolingual controls. She criticises the notion of the monolingual native speaker as the model speaker, and the deficit framing of heritage speakers in formal and experimental linguistics. She argues that such perspectives have to be overcome, and supports her point through four case studies showing that the same factors influence monolingual as well as heritage speakers’ linguistic behaviour. In her contribution, she analyses heritage speakers and monolingual speakers of Greek, targeting differences with respect to agreement in restrictive relative clauses, agreement mismatches in adjectival modification, gender mismatches, and the use of periphrastic clauses. She shows that these differences are driven by representational economy and analyticity; however, these factors shape not only heritage speakers’ grammars, but also monolingual grammars. Treating heritage speakers’ language as deficient therefore deprives researchers of the ability to investigate and understand important aspects that contribute to linguistic variation.

In Ch.5, Oliver Bunk’s critique targets the study of language anxiety in heritage speakers and second language learners. He argues that while this kind of research has important insights to offer, it also leads to an Othering of multilingual speakers since it presents heritage speakers as~a~homogenous group that have little power over~their linguistic abilities. In contrast, Bunk argues that heritage speakers are not passive subjects acting out of anxiety, but rather agents that actively employ their linguistic choices. He supports his claim with an analysis of qualitative interviews that shows~that heritage~speakers’~emotional relationship with language is often shaped by heightened societal consciousness rather than by mere anxiety. Against this background, he advocates a different view of multilinguals in research on language anxiety, one that takes into consideration their agency and~the role of societal ideologies towards multilingualism.

The contribution of Gamper et al. in Ch.6 adds a focus on language learning and teaching. They target Othering in studies on German as a Second Language (GSL) and analyse how the academic field of GSL conceptualises its ``clientele'' of learners. They review a data sample of 138 papers on child and adolescent learners that were coded according to explicit and implicit features related to what constitutes GSL. Their results reveal GSL to be a vague and unspecific construct, with rarely any explicit criteria for its definition. They argue that this is problematic as it might lead to a construct of GSL speakers that has no base in reality or may be highly biased. Against this background, their analysis reveals patterns of Othering in their data based on a deficit view of GSL speakers in particular and multilingualism in general. The authors stress the necessity to significantly revise and overcome such deficient concepts of GSL and multilingualism and the Othering that goes with it, not only in academia but also in educational policy debates, and propose to limit the concept of ``GSL'' exclusively to newly immigrated learners of German.

\largerpage
In Ch.7, İnci Dirim continues the discussion at the interface of linguistics and didactics, targeting the concept of ``linguistic identity''. This is a concept frequently referred to in studies on language and migration, notably also in the context of German as a second language (see also Ch.6). She critically discusses assumptions on multilingual children and young people where languages other than German are considered to be their ``first languages'': for these speakers, it is often claimed that such ``first languages'' must be valued and taught because they supposedly constitute their ``identity'' or ``linguistic identity''. Dirim challenges this claim about the supposed importance of a ``first language'' for identity both as an empirical fact and as a didactic concept. Her discussion shows that assertions about the importance of a ``first language'' as central to ``identity'' feed into enduring patterns of ethnic demarcations and nationalism, at odds with a globalised society that is shaped by migration. 

In Ch.8, Friederike Lüpke brings in a perspective of the Global South to the discussion of Othering in academia. She critically examines Eurocentric lenses that are still prevalent in cognitive science and linguistics and that support processes of Othering multilinguals. Deconstructing the notion of the monolingual European Self, she analyses its role in describing the non-European Other and in compelling them to perceive themselves through external lenses, thus perpetuating colonial viewpoints. Against this background, the paper advocates a recalibration of ontologies, epistemologies, and methodologies in the description of multilingualism across all settings, in order to normalise dynamic and fluid multilingualisms, in favour of a more comprehensive understanding of multilingualism worldwide that is based on convivial research paradigms.

The final contribution, in Ch.9 by Judith Purkarthofer, rounds off this volume by taking an applied perspective: the chapter addresses the risks of Othering in linguistic outreach activities, i.e., activities designed to transfer scientific linguistic knowledge to members of the public. As Purkarthofer points out, outreach activities are based on specific ideas about their recipients, notably about the recipients’ (lack of) perceived knowledge, their biographies, and their general background. Such activities hence risk Othering their audience by portraying and treating them as lacking certain competences. Reporting on her experiences with an outreach project, she makes use of a model of audience design that distinguishes between different types of audiences, including addressees, auditors, overhearers, and eavesdroppers, and presents its applicability in outreach. She discusses how these groups are or are not addressed in outreach activities, and points out which aspects – including language use, technical access and legal requirements – need to be considered when designing linguistic outreach activities that aim to avoid Othering their audience.

This volume thus unites and showcases studies that shed light on a multitude of aspects related to Othering in linguistics and beyond: the chapters cover different subfields of linguistics and neighbouring disciplines and demonstrate how in each of these fields, multilinguals are treated as deviant and potentially problematic cases that differ from the agreed norm.

\largerpage
Taken together, the contributions in this volume shed light on a deep-rooted us/them distinction at the basis of such Othering practices that transcends different linguistic subfields and societal domains and challenges us to think differently about multilinguals. We believe this to be of great linguistic as well as social importance, given that the way we conceptualise ``us'' vs. the ``Other'' is not only a key challenge to our research practices but can also have far-reaching effects for social cohesion.

\printbibliography[heading=subbibliography,notkeyword=this]
\end{document}
