\documentclass[output=paper]{langscibook}
\ChapterDOI{10.5281/zenodo.17132441}
\author{Claudio Scarvaglieri\orcid{}\affiliation{Université de Lausanne}}
\title{Othering of multilinguals in society}
\abstract{This contribution describes ways of Othering of multilinguals in public discourse. Based on a corpus of 49 texts from Germany, Austria and Switzerland, I first demonstrate that multilinguals are characterized as different from the ‘We’ group and labelled with terms that disregard their multilingual capabilities and instead portray them as geographic, national or religious others. This ‘being-different’ from ‘Us’ is regularly evaluated in negative terms, with multilinguals being blamed for, e.g., antisemitism or homophobia in ‘Our’ society. The data also shows that these labels and the underlying distinctions between ‘Us’ and ‘Them’ persist against criticism and that they are used to explain new social phenomena (like the development of a pandemic). Taken together, this study not only allows us to understand how Othering of multilinguals proceeds in the investigated societies, but also to reconstruct a very narrow concept of the perceived ‘We’ group. As academia in general, and linguistics in particular, is influenced by the societal context in which it operates, such folk concepts of ‘Us’ vs. ‘Them’ have the potential to influence linguistic research – as evidenced by the contributions to this volume. }
\IfFileExists{../localcommands.tex}{
  \addbibresource{../localbibliography.bib}
  % add all extra packages you need to load to this file

\usepackage{tabularx,multicol}
\usepackage{url}
\urlstyle{same}

\usepackage{listings}
\lstset{basicstyle=\ttfamily,tabsize=2,breaklines=true}

\usepackage{langsci-basic}
\usepackage{langsci-optional}
\usepackage{langsci-lgr}
\usepackage{langsci-osl}
% \usepackage{./langsci/styles/langsci-lgr}
% \usepackage{./langsci/styles/langsci-osl}
% \usepackage{langsci-gb4e}

\usepackage{tikz}
\usetikzlibrary{patterns,calc}
\pgfdeclarepatternformonly{south east lines}{\pgfqpoint{-0pt}{-0pt}}{\pgfqpoint{3pt}{3pt}}{\pgfqpoint{3pt}{3pt}}{
    \pgfsetlinewidth{0.6pt}
    \pgfpathmoveto{\pgfqpoint{0pt}{3pt}}
    \pgfpathlineto{\pgfqpoint{3pt}{0pt}}
    \pgfpathmoveto{\pgfqpoint{.2pt}{-.2pt}}
    \pgfpathlineto{\pgfqpoint{-.2pt}{.2pt}}
    \pgfpathmoveto{\pgfqpoint{3.2pt}{2.8pt}}
    \pgfpathlineto{\pgfqpoint{2.8pt}{3.2pt}}
    \pgfusepath{stroke}}
    
\usepackage{stmaryrd}
\usepackage{wasysym}
\usepackage{multirow}
\usepackage{caption}
\usepackage{subcaption}
\usepackage{mathrsfs}
\usepackage{qtree}

\usepackage{linguex}


  %pminos do not split footnotes
% \interfootnotelinepenalty=10000 %Footnote in Laporte chapters has to be split SN


%\DeclareIndexNameFormat{default}{%
%\nameparts{#1}%
%\usebibmacro{index:name}%
%{\index[names]}%
%{\namepartfamily}%
%{\namepartgiveni}%
% {}% L1
% {}% L2
%{\namepartprefix}% generates spurious space L3
%{\namepartsuffix}% generates spurious space L4
%}

%  {\DeclareIndexNameFormat{default}{%
%     \usebibmacro{index:name}{\index[names]}{#1}{#3}{#5}{#7}}}

%\DeclareIndexNameFormat{default}{%
%  \usebibmacro{index:name}{\sindex[nom]}{#1}{#3}{#5}{#7}}

%\DeclareIndexNameFormat{default}{%
%  \usebibmacro{index:name}{\sindex[person]}{#1}{#3}{#5}{#7}}
%\DeclareIndexNameFormat{default}{%
%\nameparts{#1} \usebibmacro{index:name}{\sindex[person]]}{\namepartfamily}{‌​\namepartgiven}{\nam‌​epartprefix}{\namepa‌​rtsuffix}}

%\newcommand{\smiley}{:)}

%\renewbibmacro*{index:name}[5]{%
%\usebibmacro{index:entry}{#1}%
%{\iffieldundef{usera}{}{\thefield{usera}\actualoperator}\mkbibindexname{#2}{#3}{#4}{#5}}}

% \newcommand{\noop}[1]{}

%remove for final
%\overfullrule=1mm

\newcommand{\tobi}[2]}}
\renewcommand{\S}[1]{\tobi{#1}{\textsc{*}}}

% this volume references
% puts: [this volume]
% already defined: \citetv
%\newcommand{\citepv}[1]{(\citeauthor{#1} \citeyear*{#1} [this volume])}
\newcommand{\citealtv}[1]{\citeauthor{#1} \citeyear*{#1} [this volume]}

%parentheses around example number
\newcommand{\pref}[1]{(\ref{#1})}

% in-text examples

\newcommand{\lnex}[1]{\textit{#1}} %target lang word
\newcommand{\lnlit}[1]{(lit.: `#1')} %literal reading
\newcommand{\lnlat}[1]{(#1)} % latinization
\newcommand{\lntrans}[1]{`#1'} %translation
\newcommand{\lnexl}[2]%
{\lnex{#1}{} \lnlat{#2}} % ex with latinization
\newcommand{\lnexlat}[3]{\lnex{#1}{} \lnlat{#2}{} \lntrans{#3}} % ex with latinization and tranl.

%ch01
\newcommand{\co}[1]{\mbox{\textbf{#1}}}

%ch09

\newcommand{\cyrbulg}[1]{\begin{otherlanguage*}{bulgarian}#1\end{otherlanguage*}}


%ch10
\newcommand{\nlp}{{\small NLP}}
\newcommand{\mwe}{{\small MWE}}
\newcommand{\rae}{{\small RAE}}
\newcommand{\lvc}{{\small LVC}}
\newcommand{\pos}{{\small P}o{\small S}}
%\newcommand{\todo}[1]{ \textcolor{red}{#1} }

%\renewcommand{\labelenumi}{\theenumi}
%\ainamefmt{{vv}{ll}{, ff}{, jj}} % fullname

\newcommand{\biberror}[1]{{\color{red}#1}}

\newcommand{\osenovaitem}{--~}
  %% hyphenation points for line breaks
%% Normally, automatic hyphenation in LaTeX is very good
%% If a word is mis-hyphenated, add it to this file
%%
%% add information to TeX file before \begin{document} with:
%% %% hyphenation points for line breaks
%% Normally, automatic hyphenation in LaTeX is very good
%% If a word is mis-hyphenated, add it to this file
%%
%% add information to TeX file before \begin{document} with:
%% %% hyphenation points for line breaks
%% Normally, automatic hyphenation in LaTeX is very good
%% If a word is mis-hyphenated, add it to this file
%%
%% add information to TeX file before \begin{document} with:
%% \include{localhyphenation}
\hyphenation{
    Beck-man
    Ngu-yen
    back-chan-nel
    back-chan-nels
    mo-not-o-nous
    ste-reo-typ-i-cal
}

\hyphenation{
    Beck-man
    Ngu-yen
    back-chan-nel
    back-chan-nels
    mo-not-o-nous
    ste-reo-typ-i-cal
}

\hyphenation{
    Beck-man
    Ngu-yen
    back-chan-nel
    back-chan-nels
    mo-not-o-nous
    ste-reo-typ-i-cal
}

  \togglepaper[1]%%chapternumber
}{}

\begin{document}
\maketitle
%\shorttitlerunninghead{}%%use this for an abridged title in the page headers



\section{Introduction} 

Othering has been described as a language practice that demarcates individuals or groups as distinctly different from the group the current speaker identifies with (\citealt{Dahinden2011}, \citealt{Brons2015}). Othering identifies people as “alien” \citep{Brons2015} from ‘Us’ – the speaker’s group – and as belonging to a different group. Historically, such othering processes have often led to negative evaluations of the Other group (\citealt{ScarvaglieriZech2013}, \citealt{DahindenEtAl2014}; \citealt{ScarvaglieriLuginbühl2023}) and have at times been a basis for drastic and coercive social and political measures directed at that group (\citealt{Drakulic1993}, \citealt{Müller-Funk2016}: 141). In this contribution, I aim to describe ways in which multilinguals are portrayed as Others in society. To this end, I examine data from public discourse in Germany, Switzerland and Austria.\footnote{ \textrm{All three of these countries are multilingual. In this contribution, I will only study the German-language discourse in these countries.} } The data demonstrates 1) that multilinguals in these countries are often portrayed as different from the majority society, 2) the ways in which such Othering proceeds and 3) the types of Othering created. I show that studying processes and practices of Othering not only tells us about how the Other group is perceived, but that it also allows us to better understand how the majority perceives themselves. By describing and discussing Othering in society, this contribution aims to lay the groundwork for the articles in this volume that investigate Othering in science. Since science is connected to social processes in multiple ways, multiple interdependencies between the way multilinguals are perceived in society and science are to be expected. This contribution argues that a clear and objective concept of multilinguals and multilingualism that does not ostracize multilinguals as Others, needs to be the basis of a more open and non-discriminating social discourse about multilinguals. 

In the following, I first present the existing research about Otherness and Othering and discuss the main strands in the discourse analytic literature about the portrayal of immigrants in the mass media. I then briefly describe the analytic procedure and the dataset that was used in this study. The study is exploratory in nature and follows a qualitative approach that allows for a rather detailed, thick description \citep{Geertz1973} of Othering processes but does not claim representativeness or statistical significance. Instead, it aims at uncovering some of the most important patterns in which Othering of multilinguals proceeds in public discourse. In the Findings section, I analyze six examples that illustrate the different patterns of Othering in my data.\footnote{ \textrm{Some of these examples have already been briefly discussed in \citealt{WieseEtAl2022Multilinguals}.}} I first describe practices of Othering and secondly demonstrate that Othering is not restricted to existing social debates, but will also be applied to new emerging problems, like the Covid-19 pandemic or the 2022 New Year’s Eve riots in Germany. Thirdly I will show that Othering frequently leads to negative evaluations of the Other group that allows this group to be blamed for societal problems and thus to portray the ‘own’ group in more positive, desirable terms. In the discussion section, I point out the principles that guide Othering processes and the underlying folk concepts of ‘Us’ vs. ‘They’, i.e., the ideas that guide the creation of group membership in the investigated societies. I close by touching on the connections between scientific and social discourse and by arguing for a different conceptualization of multilinguals in linguistics and language science. 

\section{Research on Othering}

The distinction of same vs. different constitutes a fundamental principle of human perception (\citealt{Freud1913}, \citealt{MaturanaVarela1980}, \citealt{Tauber2015}). This not only holds true from a biological or psychological individualistic perspective but can also been transferred to social processes – people regularly distinguish between those they perceive as belonging to their own groups vs. those perceived as members of another group (\citealt{Gillespie2007}; \citealt{Müller-Funk2016}: 66). While such processes might be seen as “natural” (cf. \citealt[374]{Gadamer2010}) and not per se problematic, the linguistic practice of Othering has been described in more critical terms (\citealt{Dahinden2011}, \citealt{DahindenEtAl2014}; \citealt{WieseEtAl2022Multilinguals}). Linguistic Othering proceeds in two basic steps: First, a distinction is drawn between an In- and an Out-group, a group ‘We’ belong to vs. a group ‘They’ belong to \citep{Reisigl2014}. While the In-group is often not referred to explicitly – implying that who ‘We’ are is self-evident and does not need to be explained – the Other group is often identified via a specific term, e.g., as Jews, Muslims, or people with a ‘migration background’ (see below).\footnote{One of the striking aspects of these labels is that at first sight they appear to denotate a clearly defined social group, while in reality the boundaries of these categories are often very porous and historically undergo a process of gradual extension. In the German Nazi-state, for example, the category of Jews included families and individuals who did not practice their belief for generations or who did not identify as Jewish but had Jewish ancestors. In contemporary discourse in many European countries, people are identified as Muslims if they or their ancestors come from a majority Muslim country, regardless of whether they practice their belief, identify as atheists, agnostics or belong to a religious minority.}  In a second step, negative attributes (like delinquency or abuse of the social welfare system) are regularly ascribed to this Other group. Much linguistic research has described such Othering practices in public discourse in the mass media (see for example \citealt{JungEtAl1997}, \citealt{ReisiglWodak2001}, \citealt{Wengeler2003}, \citealt{ButterweggeHentges2006}; \citealt{ConradAðalsteinsdóttir2017}; \citealt{HolzbergEtAl2018}; \citealt{JorisEtAl2018}; \citealt{AmoresEtAl2020}, \citealt{Figoureux2021}, \citealt{ScarvaglieriLuginbühl2023}). While a complete survey of the literature is not intended here, it has been shown that immigrants – and their offspring, i.e., children born and raised in the respective country of immigration – have been portrayed in more negative than positive terms, i.e., that negative attributes are ascribed to the Other group. Among such attributes, the description of immigrants as criminals is one of the phenomena most investigated in the literature (see e.g., \citealt{Wengeler2003}, \citealt{DeRidder2010}, \citealt{TortEtAl2016}, \citealt{ConradAðalsteinsdóttir2017}, \citealt{Scarvaglieri2018}, \citealt{JorisEtAl2018}, \citealt{HolzbergEtAl2018}, \citealt{Figoureux2021}, \citealt{CzymaravanKlingeren2022}). Other important frames in this context are that of immigrants as intruders (\citealt{VanGorp2006}; \citealt{Scarvaglieri2018}, \citealt{RheindorfWodak2018}) penetrating ‘our’ space despite not belonging ‘here’ or that they produce problems and financial costs for the majority society (\citealt{ScarvaglieriZech2013}, \citealt{TortEtAl2016}, \citealt{ConradAðalsteinsdóttir2017}, \citealt{JorisEtAl2018}, \citealt{Figoureux2021}). These attributions not only present the Other group in a negative light, but also implicitly suggest clear paths of dealing with this Other group – criminals need to be punished, intruders need to be sent away, even more so if they are responsible for social problems and rising social welfare costs. Thus, even if such measures are not explicitly called for in the respective Othering texts, ascribing such characteristics to a social group implies a certain course of action towards them. The same holds true for the water metaphors very frequently used to describe immigrants as a flood or wave of people. These metaphors present Others as a mass of uncountable and uncontrollable objects that threaten to wash over ‘Our’ country and need to be dealt with in ways appropriate to natural disasters (\citealt{Rehbein1993}, \citealt{JungEtAl1997}, \citealt{Wengeler2003}, \citealt{JorisEtAl2018}). Of course, Othering in principle also allows for describing the Other group in more positive or neutral terms. Especially persons fleeing from war zones have been described as victims in the mass media (\citealt{VanGorp2006}, \citealt{ConradAðalsteinsdóttir2017}, \citealt{JorisEtAl2018}, \citealt{HolzbergEtAl2018}) and sometimes they are also being framed as useful or beneficial (most often in economic or demographic terms). These categories are often used in the media to try to balance the reporting on immigrants and portray the Other group in a less negative light, also with the aim of not feeding racist narratives and stereotypes. 

While research has in general been very interested in the description of immigrants in public discourse, even linguists have so far ignored the language aspect of Othering, i.e., the Othering of immigrants based on their multilingual competencies. Researchers have yet to demonstrate language as a source of Othering and to describe the ways in which linguistic Othering in society proceeds (for a first step see \citealt{Scarvaglieri2022}). Therefore, the present contribution pursues the following goals: 

\begin{enumerate}
\item Document cases of Othering of multilinguals in public discourse;
\item Differentiate between different kinds of Othering of multilinguals;
\item Describe the attributes and labels used for Othering multilinguals;
\item Elaborate how, based on the findings of 1)-3), the ‘We’ group is constructed vs. the Other group.
\end{enumerate}

In the following section, I describe the data corpus and the analytic approach that were used to pursue these goals.

\section{Data and analytic approach}

This exploratory study is based on a corpus of 49 texts, including audio- and video files, from Germany, Switzerland, and Austria\footnote{The discourse in the Netherlands and in Dutch-speaking Belgium shows very similar patterns (\citealt{VanMaele2021}, \citealt{Scarvaglieri2022}), but due to space restrictions this contribution will be limited to German-language data.} . All elements of the database are journalistic articles or interviews that contribute to public discourse. The data does not contain forum discussions or contributions to social media, only contributions to traditional news media (online and print versions of newspapers, radio- and tv-programs). 

The data were collected by Heike Wiese, who graciously allowed me to use her data, and by myself. Sampling followed a basic hermeneutic approach – newspaper articles or contributions to radio or TV programs were collected if at first sight they seemed to contribute to the Othering of Multilinguals or – more generally – people portrayed as immigrants. This of course explains the rather small corpus – we did not gather all texts about language, multilingualism, or immigrants, but only such sources that were deemed relevant to the topic of this contribution. 

Following a first reading, listening, or watching of the data, I transcribed the audio and video files. For transcription, I opted to follow conventions established in media linguistics (\citealt{BurgerLuginbühl2014}: 523) that focus on the verbal content of the contributions, less on paralinguistic or nonverbal information (cf. \citealt{Jefferson2004}). I then analyzed the data according to established methods of discourse analysis (\citealt{WodakMeyer2009}, \citealt{WarnkeSpitzmüller2011}), focusing particularly on what was said about multilinguals, how they were portrayed, whether Othering occurred and how it proceeded. My analytic approach thus offers interpretations of the meanings of texts, situates them in the context in which they occur and discusses how their meanings is constructed (cf. \citealt{Richardson2019}). In a fourth analytic step, I distinguished between different types of Othering apparent in the source texts. This distinction was mainly based on which aspects the Othering was grounded in (e.g., geography vs. ethnic or national belonging). 

Sampling and analysis thus followed these four steps: 

\begin{enumerate}
\item  First view (reading, listening, watching) and decision about including the source in the corpus;
\item  Transcription of audio and video sources;
\item  Close linguistic analysis of the sources with regards to Othering of multilinguals;
\item  Differentiation of types of Othering.
\end{enumerate}

As mentioned, the study is exploratory in nature and does not claim representativeness for public discourse as such. Based on the limited dataset, it is also not possible to make statements about the overall frequency of the patterns of Othering apparent in my data. Despite these limitations, this contribution is able to demonstrate that Othering of multilinguals exists in the public discourse of the respective countries and how such Othering proceeds. My study thus also aims to lay the groundwork for future studies that would then be able to use for instance statistical methods and a larger database to examine the representativeness and frequency of the patterns described here. 

\section{Findings}

Despite sampling for Othering of multilinguals, a first look at my data revealed that even in this carefully chosen subset of public discourse, language is often disregarded or treated as a side-issue with limited relevance for the overarching point of the article. This often means that multilingual immigrants or their children or grandchildren are Othered, without mentioning their multilingual capabilities (see extracts 3 and 5 below for examples). One of the effects of this phenomenon is that multilinguals are not framed as persons who – by speaking multiple languages – have more capabilities than the average monolingual member of the majority society. Not mentioning multilingualism when talking about – and Othering – immigrants thus conceals an important asset of the objects of the contribution, which in turn lends itself to portraying them not as important and contributing members of society, but as deficient, problem-inducing Others in need of support (cf. \citealt{ScarvaglieriZech2013}). Contrary to what would be expected, the concealing of their multilingualism thus does not contribute to ‘normalizing’ multilinguals – as one of the attributes that possibly differentiates ‘Them’ from ‘Us’ is not mentioned – but can instead be a part of portraying them in negative terms. The not mentioning of the multilingualism of Others thus contributes to the second step of Othering mentioned above, the negative evaluation of Others or “crude othering” according to \citet[71]{Brons2015}.

One first finding is thus that often the multilingual capabilities of the immigrant population are not mentioned in public discourse. Instead, multilinguals are portrayed as geographic, ethnic, or national Others (cf. \citealt{WieseEtAl2022Multilinguals}), without referring to language. 

A related finding is that language is often used as a trait, to demarcate multilinguals as, for example, ‘non-German-speaking’. In these cases, linguistic competencies are treated like labels, used interchangeably with other labels demarcating the Other group. A first example from a discussion about educational policy in the state parliament of Berlin illustrates this practice (emphasis added).


\ea
{Extract 1: “Unequal opportunities in finishing school”, Tagesspiegel, 22. 3. 2017} \\

\begin{quote}
Melzer [member of parliament; C.,S. ] also wanted to know how students from \textit{immigrant families} compare to those from \textit{German-speaking families}. His question was, however, not answered – instead, Secretary of State Rackles gave him the overall figures – \textit{Germans and migrants} together – as well as the separate figures for \textit{migrants}. To be able to calculate the school completion rates separately for children \textit{with and without a migrant background}, one would first have to grab a calculator. If you don’t do that, all you find out is that 34 percent of \textit{migrants} graduate from high school at the A-level, while the rate for Berlin overall is 47.4\%.\footnote{Extracts are presented in translation. The German originals are provided in the appendix.}
\end{quote}
\z

In this extract, a distinction is made between on the one side “Germans”, “children without a migration background”, and “German-speaking families”, and on the other side “immigrant families”, “migrants” and “children with a migration background”. Language is thus used as a trait to characterize some members of society – those that are “German-speaking” – and to distinguish them from other members who do not have this trait. It also becomes clear that ‘German-speaking’ is just one of the labels assigned to one of the two groups, which by reverse marks the Other group as non-German-speaking. This label is used even though both groups attend the German education system and – to a large extent at least – were born and raised in Germany and thus factually do speak German. This labelling is thus less about the actual ability to speak German, but more about whether the children speak another language besides German. This Othering practice thus leads to the paradoxical situation in which a group of children that speaks German is portrayed as non-German-speaking because they speak German along with another language. Othering here is based on children speaking not only German, but also another language. 

Example 1 also illustrates the many cases in my corpus in which language is presented as a trait acquired by birth, as it is the fact that these children were born into a family that speaks another language besides German that leads to them being Othered as non-German-speaking. It is thus not surprising that this label is used interchangeably with other labels such as “immigrant”, “migrant” or “migration background” that are also permanently attributed to the Other group. 

A similar case is extract two, where language-related labels are again used together with labels related to ethnic or geographic origin (emphasis added): 

\ea\label{ex:02:2}  subexamples
{Extract 2 “Antisemitism among Muslim youth”, Tagesspiegel, 5. 4. 2017}\\
According to current knowledge, it is assumed that it was \textit{Arabic}{}- and \textit{Turkish-speaking} classmates who so maltreated this youth […]. The comparisons show that adolescents of \textit{Turkish and Arab origin} had significantly higher antisemitic attitudes than \textit{German} adolescents […].
\z

The Other group is first labelled as „Arabic- and Turkish-speaking”, then as of “Turkish and Arab origin” and as such negatively compared with members of the ‘We’ group, as “German” adolescents had “significantly” lower “antisemitic attitudes”. Similar to example 1, language is used as a trait to differentiate the ‘We’ group and the Other group, interchangeably with an ethnic and or geographic label (the use of “Turkish” and “Arab” here can be understood as referring to the geographic or the ethnic origin of the group). This differentiation was obviously also the basis of the studies that compared the attitudes among these groups and that the author is referring to. The headline of the article – Antisemitism among Muslim Youth – demonstrates that Turkish- and Arabic-speaking and Turkish and Arabic origin are used as synonyms for “Muslim” – these different ethnic and linguistic groups are here thus grouped together under a religious label that frames them as different from the ‘We’ group in religious terms. Interestingly, the ‘We’ group is not – neither in this article nor in other cases in my corpus – referred to as “Christian”, probably because such a label would be seen as much too broad to be applied to the religiously diverse or neutral members of the ‘We’ group. 

As in extract 1, the fact that both groups do in fact speak German – since the article is about students in secondary schools in Germany – is disregarded. This again shows that language is not seen in relation to communicative ability, but as a trait acquired by birth, to be used alternatively with ethnic, religious or geographic labels. It is important to note that the fact that these problematic attitudes were developed in Germany, i.e., in ‘Our’ society, is not brought up. The responsibility of the majority society for the emergence of such attitudes is not discussed. Instead, these attitudes are portrayed as something that was brought to ‘Us’, was brought ‘Here’, by the Other group. The resulting problem – antisemitic behavior – is thus portrayed as a consequence of the fact that the Other group is ‘Here’, is a part of ‘Our’ society.\footnote{In a very similar example from Switzerland, \textit{homophobic} attitudes and behavior are connected to persons with “migration background”, again without mentioning the fact that these Other persons acquired these attitudes in ‘Our’ society (Neue Zürcher Zeitung, 20. 2. 2020, “Nationality is not the reason why someone beats up gays”).}

Extract 2 thus demonstrates 1) that linguistic labels are used interchangeably with other labels related to ethnic, religious or geographic framings (cf. extract 1); 2) that Othering categories are used to evaluate the Other group in decidedly negative terms and hold them responsible for problems in ‘Our’ country (cf. below extract 3); and 3) that Othering labels are often very broad (“Muslim”), while the ‘We’ group seems to be seen in much more specific terms. Next to this, this example also illustrates a tendency in the German data that certain labels (like migration background) are often applied to youth whose parents or grandparents come from Turkey or from Arabic-speaking countries.\footnote{In fact, people have told me that persons from Poland or other Eastern European countries did not have a migration background as this label referred only to ‘Muslim’ youth or people from the countries mentioned above (oral communication).}  

So far, we have seen examples in which multilinguals are Othered in ways that disregard their multilingual communicative capabilities and instead treat language as a trait, as if it was acquired by birth, in order to label them as members of the Other group. In my dataset, language is usually not the prime reason for Othering (but see my detailed discussion of extract 5, below), but appears to be used as simply one more alternative of referring to the Other group and lending attention to one more aspect in which that group is different. This of course supports the first finding mentioned above, that the multilingual capabilities of multilinguals are usually not referred to as an asset, i.e., as an advantage that the Other group might have over ‘Our’ group or a resource they contribute to ‘Our’ society (for a distinctly positive perspective on immigrant multilingualism see e.g., \citealt{Rehbein2013}). 

One of the alternatives to the “non-German-speaking” label is migration background (\textit{Migrationshintergrund} in German, \textit{migratie achtergrond} in Dutch), a term that is used in all countries that are represented in my dataset. This label was created in 2000 and has since gained popularity but has also received criticism (\citealt{ScarvaglieriZech2013}, \citealt{Will2018}) because, among other things, it essentializes discourse as it cannot be shed (a person will have a migration background their whole life, independent of their own choice) and disregards individual biographies. In Germany, a person is categorized as having a migration background if they do not hold German citizenship by birth or if one or both of their parents did not hold German citizenship when they were born \citep[30]{Destatis2021}. Austria and Switzerland use similar definitions and, more importantly, the term is used in a very similar way in public discourse across the three countries.

Our next example – an article from the German weekly Die Zeit – illustrates, how Othering based on ethnic categories and the “migration background” label not only persists despite explicit criticism but is even transferred to account for new social phenomena. The article investigates the spread of Covid-19 in Germany and tries to explain why different social groups are affected by the disease to varying degrees. The article discusses different criteria – including income, wealth, place of residence –and how they relate to the spread of the virus, and it brings up migration background as a potential factor. Concerning the influence of this aspect, an epidemiologist is quoted as stating: 

\ea
{Extract 3 “The vulnerable”, Die Zeit, 26. 11. 2020}\\
We ask about \textit{mother tongue and knowledge of German}, because that often plays a larger role [than migration background; C.S.]. […] \textit{Migration background} alone does not reveal a lot. 
\z

In this extract, the expert denies the importance of migration background and instead emphasizes language competence. Later in the article, however, the author reintroduces migration background as one of “the reasons behind the reasons”, because, according to the text, migration background increases the likelihood of poverty and suffering from stress, which in turn increases the likelihood of contracting the virus. The migration background label is thus reinstated, despite the criticism expressed earlier in the same article, as part of the “reasons behind the reasons” for the spread of the virus. Language competencies, in contrast, are disregarded in the rest of the article, despite the importance attributed to them by the expert. We thus find another example of neglecting the importance of linguistic abilities in favor of fixed, essentialist labels like migration background. We also find that such labels persist against explicit criticism within the same text and that they are transferred to account for new societal phenomena like the development of a pandemic. The existence of the distinction with vs. without a migration background is thus used to try to explain new emerging problems within a society that – without such distinction – would have to be understood differently (for example in socioeconomic terms, cf. \citealt{ScarvaglieriZech2013}). As a consequence, in Germany a discourse strand developed discussing how much blame for the pandemic was to be put on people with migration background (Frankfurter Allgemeine Zeitung, “A question of social origin”, 28. 4. 2021; Bild, “Covid patients in intensiv-care: 90\% have a migration background“, 11. 10. 2021; taz, “Moral taboo”, 27. 4. 2021\footnote{These contributions can be found at: \url{https://www.faz.net/aktuell/politik/inland/sind-menschen-mit-migrationshintergrund-treiber-der-pandemie-17314887.html} ; \url{https://taz.de/Polarisierung-in-der-Corona-Debatte/!5762645/}; \url{https://www.bild.de/video/clip/bild-tv/corona-intensiv-patienten-90-prozent-haben-migrationshintergrund-77934454.bild.html\#fromWall} (all last seen 22. 1. 2024).} ). 

Extract 3 thus illustrates how the importance of linguistic capabilities is neglected in public discourse, and it also shows how essentialist labels trump language, how they persist against criticism and how they are applied to new societal phenomena.

The productivity of the migration background label was on display also in the aftermath of New Year’s Eve 2022, when youth in large German cities (most prominently in Berlin) attacked representatives of the state, including police, firefighters, and ambulance, using firecrackers and similar make-shift weaponry. Quickly the blame was laid at the feet of youth with migration background. One striking example is a guest commentary at the national German radio program \textit{Deutschlandfunk.} 

\ea
{Extract 4: “Commentary on the New Year’s Eve riots: The view on migrant youth is changing”, Deutschlandfunk, 7. 1. 2023\footnote{The commentary can be found at \url{https://www.deutschlandfunk.de/folgen-aus-der-silvesternacht-100.html} (last seen 22. 1. 2024).}}\\
The images [of the riots, C.S.] ended for everyone the narrative of a disenfranchised, basically well-intentioned migrant youth.
\z

The author, journalist Burkard Ewert, also calls for a debate about “the mostly immigrant perpetrators” and continues by citing the federal minister of the interior who claimed that in “large German cities we have a problem with certain men with migration background”. The gist of the commentary is clear – migrant youth and a “romanticized integration policy” are to be blamed for the events of New Year’s Eve 2022, members of ‘Our’ group are the victims of these events, and ‘We’ as the majority society need to see this as reality and act accordingly. 

This case thus demonstrates that the existing labels (migrant, migration background, Muslim etc.) continue to be transferred to new societal phenomena\footnote{At the time of writing, the migration background label is also used in German discourse to explain the riots in France after the killing of a 17 year old by a French policeman (see e.g., \url{https://www.zeit.de/politik/2023-07/frankreich-unruhen-polizeigewalt-politikpodcast}; last seen 30. 1. 2024)} and that they allow an evaluation of the Other group as well as a call for social or political measures directed at those groups. 

In the following, I discuss another recent case, that once more illustrates how language is used for Othering in combination with other labels and that also shows how the language-based differentiation between ‘Them’ and ‘Us’ can be related to a call for coercive measures. Since this case stems from a 30-minute-long interview and entails many interesting and relevant aspects, I have chosen to discuss it in more detail. 

Extract 5 stems from a radio interview in the regional German public broadcaster SWR 1 (distributed in large parts of Southern Germany, available also on the internet\footnote{The interview was originally broadcast on March 24, 2023, it can be found at: \url{https://www.swr.de/swr1/swr1leute/bildungsforscher-oecd-andreas-schleicher-100.html} (last seen 24. 1. 2024).}) with the global coordinator of the PISA study, the German Andreas Schleicher.\footnote{The PISA study (Programme for International Student Assessment) is a survey organized by the OECD that compares and evaluates the educational systems of a large number of countries around the world.} The interview discusses the performance of German students in a very negative and worried tone. Not unexpectedly – for those familiar with German education discourse –, the interviewer brings up migration as an aspect contributing to the problems of the German education system (transcription according to \citealt{BurgerLuginbühl2014}: 523-524; italics denote emphasis).

\ea
{Extract 5 “Father of the PISA study: Sharp criticism of the German education system”, SWR 1, 24. 3. 2023}\\

Interviewer: Another factor in \textit{our} country - there is a correlation between poor performance in schools and children with an \textit{im}migrant background. Can you explain that?

Expert: Yes, it's not just the immigration background, but the overall social background. Children from socially disadvantaged backgrounds simply don't get the support they need. […]
\z

We see that the interviewer brings up immigrant background (\textit{Zuwanderungshintergrund} in German – another word for migration background, but with the same meaning) as leading to poor performance in schools. This is rejected by the expert, who points out that the social situation of the children is more important – a point widely supported by research (\citealt{OECD2009}, \citealt{LangeGogolin2010}, \citealt{MecherilDirim2010}; \citealt{GreschKirsten2011}; \citealt{ScarvaglieriZech2013}) –, not whether they or their parents migrated. We thus find another example (cf. extract 3) of an expert rejecting the value of the (im-)migration background label.  

The interviewer reacts to this rejection by bringing up language, trying to support her initial point that immigrants create problems for the German education system: 

\begin{quote}
Interviewer: Concerning immigration, it certainly makes a difference whether the children speak their native language at home or \textit{German}. Can you get into contact with the \textit{pa}rents, who sometimes \textit{can}not or don’t \textit{want} to?
\end{quote}

\begin{quote}
Expert: But I believe that parents must be held accountable too. And that is done better in many countries. Overall, it has to be said that there is a lack of relationship work in Germany. [...] It's quite clear that if we don't take language support more seriously, that young people come to elementary school with a good language foundation, then of \textit{course} they will fail later.
\end{quote}

The interviewer first turns the conversation back from social aspects to immigration (“concerning immigration”) and then formulates the proposition that not speaking German at home makes a difference for the children and that there are parents who do not want to speak German at home, even though they could. In his answer, the expert accepts the interviewer’s point that children who speak a language other than German at home with their parents are a problem: “of course [they] will fail later”. This is in stark contrast with linguistic research that has repeatedly shown that children profit from their parents speaking their own language at home and that they are perfectly capable to learn the language of the majority society in schools or day-cares (\citealt{Tracy2008}, \citealt{LangeGogolin2010}, \citealt{Rehbein2013}, \citealt{LohndalEtAl2019}; \citealt{Montanari2019}; \citealt{WieseEtAl2020}). It is unclear whether Andreas Schleicher as the global coordinator of the PISA study is not aware of these findings or unwilling to relay them in this context, where it appears that the interviewer is set on insisting on common-sense knowledge according to which multilingual immigrants create problems for the German education system. 

This topic is expanded in the next turn, with the interviewer quoting a listener who wrote into the program. 

\begin{quote}
Interviewer: NAME from CITY writes and describesWhen I addressed her young, maybe three years old child, a very \textit{well}{}-integrated young Turkish neighbor told me that her daughter did \textit{not} speak German yet. She would first learn only Turkish and then, when she started school, she would be allowed to learn German. Absolutely no \textit{chance}, Turkish parallel world”. And he continues: “I am curious whether you have the \textit{cou}rage to address \textit{this} publicly. I think probably not, rather lead fake debates, because you could be called a racist broadcaster”. So a lot of rage expressed here. Andreas Schleicher, what do you say?
\end{quote}

\begin{quote}
Expert: So the topic of integration is really a crucial one today, and you basically have to get the parents on board early. And I think that's the task of a good education system, to approach these people. I was once in one of the poorest regions of China [...]. That's what we have to expect from a good education system, to seek the conversation early, because the language basics are the basics for everything, right.
\end{quote}

The interviewer refers to a listener who expresses very sharp criticism of a family that is practicing what has scientifically been found to be the most adequate way of multilingual language development: speaking the family language at home and having children acquire the society’s majority language in the educational system (\citealt{Tracy2008}, \citealt{Rehbein2013}, \citealt{Montanari2019}, \citealt{WieseEtAl2020}). The listener connects this with the topos of “parallel societies” and gives the child growing up in this way “no chance”. He also states that the child would only be “allowed” to speak German after entering school, while in reality her parents are probably not actively preventing her from learning German, but are simply speaking their native language with their child. The listener’s strong – and scientifically wrong – statements are classified by the interviewer as expressing “a lot of rage” but not otherwise commented on. In his reaction, the expert again accepts the proposition that immigrants need to avoid speaking their family language at home and should speak German instead. It thus seems clear that the quoted listener, the interviewer and the expert all agree that the Turkish family mentioned in this case it at fault speaking their first language with their child. The expert blames this mistake on the German education system that did not inform the family about the ‘right’ behavior. 

In her next turn, the interviewer continues to bring the topics immigration and multilingualism together: 

\begin{quote}
Interviewer: Just to get back to the topic of immigration […]. So the proportion of fourth graders who \textit{al}ways speak German at home was just under 62\% in 2021, compared with 84\% in 2011. Can conclusions be drawn about this, does something like this drag down a class?
\end{quote}

In this turn, the interviewer brings up immigration for the third time and explicitly connects immigration and speaking a language other than German with problems in the education system (classes being “dragged down”). The proposition, that the problems in the much-maligned German education are due to immigrants speaking a language other than German at home is expressed very clearly and in distinct terms: the classes, the other students and the whole system is going down (cf. \citealt{LakoffJohnson1980} for the metaphorical concepts of ‘up is good’ and ‘down is bad’) because of these students not “always speak[ing] German at home”. 

The expert supports this as well, in unambiguous terms: 

\begin{quote}
Expert: Absolutely. I think that's what we must realize that early language education is the be-all end-all. That's why elementary education is of \textit{crucial} importance, where children have the opportunity to learn the target language German. […] So the parallel worlds usually develop when this school world and the home are basically decoupled, right.
\end{quote}

He agrees emphatically (“absolutely”) and again points out the importance of early language education, without mentioning that this can be achieved multilingually as well, independently of which language is spoken at home. He even takes up the notion of “parallel worlds” or “parallel societies” and relates it to families speaking another language at home, similar to the listener quoted earlier. 

I leave out a short sequence of this interview and bring up the final question about immigration, in which the interviewer seems to propose coercive measures against immigrants speaking their family language at home. 

\begin{quote}
Interviewer: Klaus Klemm, an educational researcher, says that positive discrimination is the way to go, i.e., if there are not enough places in daycares, make sure that children from socially disadvantaged families are given \textit{preference}. Would something like that be a solution, also for \textit{language} support perhaps, to make it \textit{obligatory}, perhaps? 
\end{quote}

The interviewer again shifts from “socially disadvantaged families” to immigrant multilinguals and seems to suggest forcing them to learn German early (“make it obligatory”). The expert again agrees emphatically (“absolutely”) and then strongly argues in favor of positive discrimination. It remains unclear if he understands the question as suggesting coercive measures against immigrant families. 

In this example, that I gave a bit more space to, we thus find a clear connection between immigrants and language. As in other examples, multilingualism is not portrayed as a resource or an asset but reduced to not speaking German. Secondly, the connection between immigration and multilinguals who do not speak German is framed very negatively, mainly as producing problems in the German education system (‘dragging classes down’ or building ‘parallel worlds’). Multiple times during the exchange, speaking a language other than German at home is depicted as highly problematic, with interviewer, expert and even a listener agreeing on this proposition. One would expect that this would leave a strong impression on most listeners of the program, especially since the connection between multilinguals and problems of the education system is made multiple times in very explicit terms and supported by an internationally renowned expert. Thirdly, it appears as if coercive measures are suggested, designed to force immigrant families to speak German (such a suggestion is reminiscent of deeply problematic language policies that historically have led to the extinction of languages in some countries \citep{Brizic2007}). 

We also find that a clear distinction is drawn between ‘Us’ and ‘Them’. This is done for example by the expert pointing out that members of educational institutions need to reach out to immigrant parents and let them know about the importance of speaking German. Immigrants are thus portrayed as special cases that, differently from members of the majority society, need to be contacted and informed separately. Also, the notion of “parallel societies” is brought up multiple times, which suggests that immigrants have willingly chosen to form ‘Their’ own society within ‘Our’ society, to differentiate ‘Themselves’ from ‘Us’. Furthermore, the languages that immigrants speak at home are only mentioned once – by the listener who is upset about his neighbor speaking Turkish with her child – in all other cases, the conversation is concerned with languages Other than German, which literally hints at the Othering that is happening here. This is reinforced in one of the questions – not refuted by the interviewee – that explicitly presents the fact that there are less monolingual German children – fourth graders who “always speak German at home” – than ten years ago as a problem. Multilingualism is thus explicitly portrayed and accepted as a problem for ‘Our’ society.

Finally, the example demonstrates that there is an entire discourse strand that problematizes and Others multilinguals, as various studies or researchers are referred to during the discussion. Those studies are used to support the Othering conducted in this conversation, which makes it clear that this discussion is not an isolated example but based on pre-existing contributions to public and scientific discourse that Othering can rely on. 

A sixth and final example illustrates that linguistic Othering can also be applied to so-called autochthonous minorities. As reported by numerous outlets (extract 6: Der Tagesspiegel , “Saxons vs. Sorbs”, 7. 1. 2018; Neues Deutschland  “Hostility towards Nationals”, 18. 1. 2018\footnote{This article can be read here: \url{https://www.nd-aktuell.de/artikel/1076671.inlaenderfeindlichkeit.html} (last seen 22. 1. 2024)}), the former prime minister of the German state Saxony, Stanislaw Tillich, was publicly labelled non-Saxon: after another politician had been elected as prime-minister, the mayor of a small Saxon town stated that “at least now for the first time we have a Saxon as prime minister” (\textit{Der Tagesspiegel} 7. 1. 2018) and later explained that the other former prime ministers came from other parts of Germany and therefore could not be considered Saxon, while Tillich, who was born and raised in Saxony, was a Sorb – a member of the officially recognized autochthonous Sorbian minority in Saxony – and thus not a Saxon. In this case, Othering is thus not related to immigration but to a person belonging to a multilingual minority. The fact that Tillich was born and raised in Saxony, strongly identifies with the region, and speaks German as one of his native languages thus apparently does not suffice to him being included in the ‘We’ group. Instead, him speaking a language besides German and identifying as a member of a minority, leads to him being Othered. We find that a very narrow concept of who ‘We’ are underlies this Othering process. I discuss this concept in the next section and sum up what I found about Othering of multilinguals in the data. 

\section{Summary and discussion}

Before discussing the findings, I first present a short summary of the patterns of Othering of multilinguals in public discourse that are visible in my data. 

\begin{itemize}
\item  Othering proceeds by portraying people as different and labelling them with terms that express Otherness \citep{WieseEtAl2022Multilinguals}. In the examples discussed above, some of these labels were related to language (non-German speaking, Turkish-speaking), others to ethnic (Turkish, Arab) or geographic origin (migrant) or religion (Muslim). As pointed out, these labels are often very broad and do not describe the Other group in much detail but pick out one aspect to demarcate them as Other.
\item  In my dataset, the multilingual capabilities of multilinguals are often not mentioned. Contrary to what might be expected, this does not work against Othering, but supports the negative evaluation of the Other group (‘crude othering’ according to \citealt{Brons2015}: 71), as an important asset of the group is concealed.
\item  If language is mentioned in these contexts, it is almost always portrayed as a fixed trait, not as communicative ability. The language label (non-German-speaking etc.) is then used interchangeably with other essentialist and broad ways of labelling such as “immigrant”, “migration background” or “Muslim”.
\item  Such essentialist labels (“migration background” etc.) – be they related to language or not –and the corresponding distinctions are upheld and reused even against explicit criticism. They are also applied to new social phenomena (like a pandemic, riots etc.).
\item  Othering is often combined with clearly negative evaluations of the Other group, which makes it possible to blame social problems (homophobia, antisemitism, bad performance of the education system) on the Other group. This also has the effect of portraying the ‘We’ group in more positive terms (cf. \citealt{Gillespie2007}: 580), as they are presented as less responsible for existing societal problems.
\item  The distinction of ‘Us’ vs. ‘Them’ and the clearly negative evaluation of the Other group can lead to the suggestion of coercive measures against this group (cf. \citealt{Drakulic1993}, \citealt{ScarvaglieriZech2013}, \citealt{Brons2015}).
\item  Othering is not restricted to multilinguals who migrated or whose ancestors did so but can also be applied to autochthonous minorities (cf. \citealt{WieseEtAl2022Multilinguals}).
\end{itemize}

These findings thus illustrate how Othering proceeds and the terms in which Others are portrayed and labelled. multilinguals are labelled as linguistic Others, geographic Others, national, ethnic, or religious Others \citep{WieseEtAl2022Multilinguals}. Since the Other group is often if not always portrayed against the backdrop of the ‘We’ group, my findings also allow for conclusions about important traits ascribed to this ‘We’ group. According to my data, a person will be considered a member of the ‘We’ group if they hold the following traits:

\begin{itemize}
\item ‘We’ were born and raised ‘here’, i.e., ‘We’ have no familial or individual experience of cross-border migration. ’Here’ is related to national borders, people who migrated within a country are generally accepted as part of the ‘We’ group; 
\item ’We’ are members of the socially dominant ‘ethnic’ (cf. \citealt{Redder2011}) group;
\item ’We’ are ‘native speakers’ of the socially dominant language and do not speak another language on a native-like level.
\end{itemize}

Anyone who differs in one or multiple of these aspects from these features is perceived and labelled as Other in ways discussed above. The data also shows that the boundaries between these groups are perceived as very sharp – individuals or groups are portrayed as part of either group, they cannot belong to both the ‘We’ and the ‘They’ group.\footnote{There are few exceptions to this rule, namely articles that portray individuals in detail and discuss contradictory feelings of belonging, often of multilingual immigrants or their children. In these contributions however, it is usually pointed out that portraying someone as feeling like they belong to multiple groups is unconventional and difficult and that it goes against the general way of understanding identity and belonging.}  

We thus find that in social public discourse, there exists a commonsense or folk concept about who ‘We’ are vs. who ‘They’ are. \citet[210]{Ehlich1993} understands such everyday or folk concepts (\textit{Alltagskonzepte}) as basic ideas about reality, incl. nature, society or culture, that guide how we act ourselves and how we understand the actions of others in ordinary everyday contexts. Due to their very fundamental nature, many individuals  often are not consciously aware of such folk concepts, and they might perceive them as self-evident presuppositions that need not be criticized Such folk concepts might then appear impossible to discuss or deconstruct. Contrary to methods or theories – which in science are regularly made the objects of intense reflection, discussion and revision – such basic ideas and pre-concepts can therefore become unanalyzed parts of research processes \citep{Ehlich1993}. Since there are multiple connections between science and society – including personal, financial, infrastructural, legal and epistemic connections – science is in general influenced by the social system in which it is conducted (e.g., \citealt{MacIntyre1987}, \citealt{Rosa1998}). This holds especially true for ideas about who we are and what constitutes us as a society – ideas that form the foundation of collective and individual identity \citep{Taylor1989}. As the contributions to this volume show, the unanalyzed, often unconscious notion that Mutlilinguals are Others, is not only present in public discourse but has continued to influence language science for decades and has often formed one of the starting points for scientific undertakings in the linguistic field. I argue that linguistics needs to discuss and deconstruct this distinction to be able to push for a more open and egalitarian societal discourse about linguistic belonging. 

\section*{Acknowledgements}

I thank my co-editors as well as two anonymous reviewers for extensive and very helpful comments on this study. I also thank the participants of the Othering workshop conducted in Berlin in July 2022 for their comments on this study. 

\sloppy\printbibliography[heading=subbibliography,notkeyword=this]
\end{document}
