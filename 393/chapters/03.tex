\documentclass[output=paper]{langscibook}
\ChapterDOI{10.5281/zenodo.17132443}
\author{Heike Wiese\orcid{0000-0002-6310-3045}\affiliation{Humboldt-Universität zu Berlin}}
\title{Othering of multilinguals in linguistics:  Implications from labelling}
\abstract{As a result of European nation-state building, societies in the Global North tend to be dominated by a monolingual and monoethnic habitus that constructs multilingual speakers as members of an out-group. Labelling practices in academic publications reveal that such Othering patterns are not restricted to public or ``lay'' discourse. An analysis of Othering through labelling in published work from linguistics and related fields of sociology and education reveals recurring topoi that feed into three interrelated strands of Othering: Othering with respect to territorial belonging, to national group membership, and to linguistic ownership. These strands mirror the ideological nexus of ``one country, one nation, one language'' in larger society. Examples come from publications across different perspectives, subdisciplines, and research domains, underlining how widespread such practices are in our field. I argue that avoiding such Othering is not only important from the point of view of scholarly terminology, but also for our research perspective: if multilinguals are constructed as Others, this can lead to an implicit bias with negative effects on our research.}
\IfFileExists{../localcommands.tex}{
  \addbibresource{../localbibliography.bib}
  % add all extra packages you need to load to this file

\usepackage{tabularx,multicol}
\usepackage{url}
\urlstyle{same}

\usepackage{listings}
\lstset{basicstyle=\ttfamily,tabsize=2,breaklines=true}

\usepackage{langsci-basic}
\usepackage{langsci-optional}
\usepackage{langsci-lgr}
\usepackage{langsci-osl}
% \usepackage{./langsci/styles/langsci-lgr}
% \usepackage{./langsci/styles/langsci-osl}
% \usepackage{langsci-gb4e}

\usepackage{tikz}
\usetikzlibrary{patterns,calc}
\pgfdeclarepatternformonly{south east lines}{\pgfqpoint{-0pt}{-0pt}}{\pgfqpoint{3pt}{3pt}}{\pgfqpoint{3pt}{3pt}}{
    \pgfsetlinewidth{0.6pt}
    \pgfpathmoveto{\pgfqpoint{0pt}{3pt}}
    \pgfpathlineto{\pgfqpoint{3pt}{0pt}}
    \pgfpathmoveto{\pgfqpoint{.2pt}{-.2pt}}
    \pgfpathlineto{\pgfqpoint{-.2pt}{.2pt}}
    \pgfpathmoveto{\pgfqpoint{3.2pt}{2.8pt}}
    \pgfpathlineto{\pgfqpoint{2.8pt}{3.2pt}}
    \pgfusepath{stroke}}
    
\usepackage{stmaryrd}
\usepackage{wasysym}
\usepackage{multirow}
\usepackage{caption}
\usepackage{subcaption}
\usepackage{mathrsfs}
\usepackage{qtree}

\usepackage{linguex}


  %pminos do not split footnotes
% \interfootnotelinepenalty=10000 %Footnote in Laporte chapters has to be split SN


%\DeclareIndexNameFormat{default}{%
%\nameparts{#1}%
%\usebibmacro{index:name}%
%{\index[names]}%
%{\namepartfamily}%
%{\namepartgiveni}%
% {}% L1
% {}% L2
%{\namepartprefix}% generates spurious space L3
%{\namepartsuffix}% generates spurious space L4
%}

%  {\DeclareIndexNameFormat{default}{%
%     \usebibmacro{index:name}{\index[names]}{#1}{#3}{#5}{#7}}}

%\DeclareIndexNameFormat{default}{%
%  \usebibmacro{index:name}{\sindex[nom]}{#1}{#3}{#5}{#7}}

%\DeclareIndexNameFormat{default}{%
%  \usebibmacro{index:name}{\sindex[person]}{#1}{#3}{#5}{#7}}
%\DeclareIndexNameFormat{default}{%
%\nameparts{#1} \usebibmacro{index:name}{\sindex[person]]}{\namepartfamily}{‌​\namepartgiven}{\nam‌​epartprefix}{\namepa‌​rtsuffix}}

%\newcommand{\smiley}{:)}

%\renewbibmacro*{index:name}[5]{%
%\usebibmacro{index:entry}{#1}%
%{\iffieldundef{usera}{}{\thefield{usera}\actualoperator}\mkbibindexname{#2}{#3}{#4}{#5}}}

% \newcommand{\noop}[1]{}

%remove for final
%\overfullrule=1mm

\newcommand{\tobi}[2]}}
\renewcommand{\S}[1]{\tobi{#1}{\textsc{*}}}

% this volume references
% puts: [this volume]
% already defined: \citetv
%\newcommand{\citepv}[1]{(\citeauthor{#1} \citeyear*{#1} [this volume])}
\newcommand{\citealtv}[1]{\citeauthor{#1} \citeyear*{#1} [this volume]}

%parentheses around example number
\newcommand{\pref}[1]{(\ref{#1})}

% in-text examples

\newcommand{\lnex}[1]{\textit{#1}} %target lang word
\newcommand{\lnlit}[1]{(lit.: `#1')} %literal reading
\newcommand{\lnlat}[1]{(#1)} % latinization
\newcommand{\lntrans}[1]{`#1'} %translation
\newcommand{\lnexl}[2]%
{\lnex{#1}{} \lnlat{#2}} % ex with latinization
\newcommand{\lnexlat}[3]{\lnex{#1}{} \lnlat{#2}{} \lntrans{#3}} % ex with latinization and tranl.

%ch01
\newcommand{\co}[1]{\mbox{\textbf{#1}}}

%ch09

\newcommand{\cyrbulg}[1]{\begin{otherlanguage*}{bulgarian}#1\end{otherlanguage*}}


%ch10
\newcommand{\nlp}{{\small NLP}}
\newcommand{\mwe}{{\small MWE}}
\newcommand{\rae}{{\small RAE}}
\newcommand{\lvc}{{\small LVC}}
\newcommand{\pos}{{\small P}o{\small S}}
%\newcommand{\todo}[1]{ \textcolor{red}{#1} }

%\renewcommand{\labelenumi}{\theenumi}
%\ainamefmt{{vv}{ll}{, ff}{, jj}} % fullname

\newcommand{\biberror}[1]{{\color{red}#1}}

\newcommand{\osenovaitem}{--~} 
  %% hyphenation points for line breaks
%% Normally, automatic hyphenation in LaTeX is very good
%% If a word is mis-hyphenated, add it to this file
%%
%% add information to TeX file before \begin{document} with:
%% %% hyphenation points for line breaks
%% Normally, automatic hyphenation in LaTeX is very good
%% If a word is mis-hyphenated, add it to this file
%%
%% add information to TeX file before \begin{document} with:
%% %% hyphenation points for line breaks
%% Normally, automatic hyphenation in LaTeX is very good
%% If a word is mis-hyphenated, add it to this file
%%
%% add information to TeX file before \begin{document} with:
%% \include{localhyphenation}
\hyphenation{
    Beck-man
    Ngu-yen
    back-chan-nel
    back-chan-nels
    mo-not-o-nous
    ste-reo-typ-i-cal
}

\hyphenation{
    Beck-man
    Ngu-yen
    back-chan-nel
    back-chan-nels
    mo-not-o-nous
    ste-reo-typ-i-cal
}

\hyphenation{
    Beck-man
    Ngu-yen
    back-chan-nel
    back-chan-nels
    mo-not-o-nous
    ste-reo-typ-i-cal
}
 
  \togglepaper[1]%%chapternumber
}{}

\begin{document}
\maketitle 
%\shorttitlerunninghead{}%%use this for an abridged title in the page headers

\section{Background: Language-related dichotomies in society and academia}
\subsection{A monolingual and monoethnic habitus in European nation-states}
\label{bkm:Ref134097123}
European societies have a long tradition of linguistic diversity. Today, we see this, for instance, in urban areas where multilingualism is a normal part of life for many children. The following vignettes characterise the linguistic repertoires of three young people in my neighbourhood, Berlin-Kreuzberg, illustrating the diverse linguistic biographies and repertoires that form a normal part of growing up in such neighbourhoods (original names have been replaced for the purpose of anonymisation):

Ina, age 5, was born in Berlin. Her father speaks Turkish and German with her, her mother German. Her maternal grandfather lives in Togo, and from visiting him, she knows some French and Ewe. In kindergarten, she mostly speaks German and Turkish. Her paternal grandparents immigrated from Anatolia and speak a Turkish dialect that is different from the one she encounters at kindergarten. She has Swabian-speaking cousins on her mother's side, and Bavarian-speaking ones on her father’s side, and knows Kiezdeutsch, an urban contact dialect of German,\footnote{See \citet{ŞimşekWiese2022} for a recent overview of Kiezdeutsch.} from adolescents in her neighbourhood.

Christina, age 12, was also born in Berlin. Her parents immigrated from Switzerland. Her mother speaks Italian and (Germany-)German with her, her father English and (Germany-)German. Her maternal grandparents immigrated to Switzerland from Italy. Her paternal grandparents moved from Greece to Switzerland, but then lived for several years in the US, where her father spent some of this childhood. She learned some Greek from her paternal grandparents, and also understands Swiss German, which her parents use when talking among themselves. In the neighbourhood and in kindergarten, she mostly speaks Germany-German.

Yussuf is 17 and fled to Germany from Afghanistan as a teenager, without his family. In Afghanistan, he spoke mostly Dari and Pashto, in addition to Uzbek with some of his relatives, and he learned some Urdu from popular movies. When fleeing to Germany, he spent some time in Turkey and learned some Turkish there, which he also uses with some of his friends in Berlin. He learned Fārsi in the refugee home where he initially lived when coming to Berlin. He now lives with foster parents and speaks German with them at home.

These brief characterisations already give an impression of the rich linguistic repertoires available to speakers growing up in such neighbourhoods, the diverse resources they can draw on in their daily interactions, through different situational choices, code-switching and bricolage, and such diversity has drawn the interest of sociolinguistics, contact linguistics, and structural linguistics alike. Yet, this multilinguistic reality clashes with a monolingual and monoethnic ideal that is evident in broader society. In public debates, educational policy and practice, multilingualism tends to be regarded as a deviation from a monolingual norm and a difficult challenge for individuals and society (see also \citet{chapters/02}).

This perspective has its roots in 19\textsuperscript{th} century European nation-state building, where the concept of a nation as the bearer of a state played a key role. Crucially, this nation was imagined as a homogeneous entity at linguistic and ethnic levels, with a common, ``national'' language that is owned by one, dominant ethnic group.\footnote{``Ethnicity'' is to be understood as a social category that constructs groups believed to share a common descent – typically geographically associated – and culture. Cf. \citet{Moran2014} on ethnicity and race, \citet{Fought2002} on ethnicity and speech communities; \citet{Wiese2022} on the relevance of (multi-)ethnicity for urban contact dialects.}

Hence, nation-state building relied on a ``contra-factual ideological construction'' (\citealt[182]{BommesMaas2005}) of  ``one country, one nation, one language'' (see also \citealt{HellerMcElhinny2017}). This supported a monolingual and monoethnic societal habitus that is still very much in evidence in European nation-states today (\citealt{Gogolin2002, HüningEtAl2012}), as well as in such countries as the US or Australia that emerged from former European settler colonies.

The construction of nation-states as monolingual and monoethnic creates ``us/them'' dichotomies between an in-group of monolingual native speakers that belong to the nation, and an out-group of others that do not, even though they are in fact part of society. Such dichotomies have been analysed as proxy racism and (neo-)linguicism and have been shown to intersect with social class (\citealt{Wiese2015, Dirim2016}).

Taking Germany as an example, the monoethnic habitus supports a \textit{ius sanguinis} perspective on who counts as a proper German. When a reform of the German citizenship law made it easier to become naturalised, the census introduced the concept of ``Migrationshintergrund'' (``migration background'') for people who hold German citizenship, but were not born to it or have at least one parent for whom this holds. This means that today many German citizens who are born and raised in Germany and have spent their whole lives there are still not acknowledged as proper ``Germans'' (the latter sometimes called ``Biodeutsche'' ``biological Germans'' in the public discussion).\footnote{See \citet{ScarvaglieriZech2013} for a corpus-linguistic analysis of the Othering involved in ``Migrationshintergrund''; \citet{Fuller2021} on ethnonationalist attitudes evident in the term ``Biodeutsche''.} Instead, in public debates, policy, education, and everyday usage alike, they get labelled as Germans with a ``migration background'', which is often used (even in census publications) interchangeably with ``Migrant'', despite the fact that they do not have any migration experience themselves, having never lived outside the country of their birth.

Having a ``migration background'' or being a ``migrant'' is constructed as an outsider status that is a problem for individuals and a challenge for society. The German census motivates singling out people with a ``migration background'' by arguing that these are ``people for whom a need for integration can be identified, at least in principle''.\footnote{``Menschen […], bei denen sich zumindest grundsätzlich ein Integrationsbedarf feststellen lässt'' (Federal Census Office / Statistisches Bundesamt 2019: 4).} In Berlin, the Senate treated the proportion of children and adolescents with a ``migration background'' as a problem for a neighbourhood, using it as a negative indicator in the calculation of its developmental index.

The Othering through ``migration background'' and ``migrant'' ascription is intersectional, with some social groups treated more readily as part of the German in-group than others. The relevance of ethnicity and social class is illustrated by the contrast between two articles that appeared in the same issue of the Berlin newspaper ``Tagesspiegel'', see headlines reproduced in \figref{fig:03:1} below. The one on the left is from an article on a planned minimum quota for employees with a ``migration background'' in Berlin’s public sector as part of affirmative action and diversity management. In the newspaper article, people with ``migration background'' are labelled ``migrants'' and contrasted with ``Germans'', thus implying that someone with a ``migration background'' is not a German, even though the people in question will usually have spent their life in Germany and hold German citizenship. The article on the right reports on British conductor Simon Rattle’s plans to apply for German citizenship. This is phrased as him becoming a ``German'', while remaining a ``Berliner''. So, in contrast to the article on the left, someone who actually is a migrant (Rattle moved to Berlin from the UK) is considered as a ``German'' once he is granted citzenship, and locally embraced as a Berliner.



\begin{figure}
%\subfigure[``Migrant quota in public service. Why a German-cap can do more harm than good.'']{
\includegraphics[width=.45\textwidth]{figures/CollVolOtheringCh3-img001 new.jpg}\label{bkm:Ref134098189}
%}
%\subfigure[ ``Brit Simon Rattle also wants to become a German. He will remain a Berliner.'']{
\includegraphics[width=.45\textwidth]{figures/CollVolOtheringCh3-img002 new.png}\label{bkm:Ref134098184}
%}


\caption{Two articles from Tagesspiegel on January 15\textsuperscript{th}, 2021; on the left: ``Migrant quota in public service. Why a German-cap can do more harm than good.''; on the right: ``Brit Simon Rattle also wants to become a German. He will remain a Berliner.''}
\label{fig:03:1}
\end{figure}

\clearpage
We hence see a stark contrast between, on the one hand, people born and raised in Germany who are Othered as ``migrants'' and excluded from ``Germans'' and, on the other hand, someone who is a migrant but will be accepted as part of the German in-group. To understand this contrast, we need to look at the groups involved: the article on the left is about marginalised social and ethnic groups who would benefit from affirmative action; the article on the right, about a world-famous British artist. This indicates an interaction of ethnic and social Othering with patterns of exclusion from the national in-group.

Linguistic Othering of marginalised groups leads to widespread doubts about their German language competences. At German schools, bilingual students with a ``migration background'' are routinely constructed as in need of German language support (e.g., \citealt{OldaniTruan2022}).\footnote{Cf. Li \citet{Wei2021} on similar evidence from the UK: ``They [students from heritage-Chinese families, H.W.] are British-born and ``have no problem with English''. Yet because of their race, they are categorized as EAL (English as an additional language) learners by the school, whose English is not expected to be ``good enough to do an essay-based subject'' at school or university.'' (Li \citealt{Wei2021}:6).} 

A large programme set up by the German Federal Ministry for Education and Research in 2004 provided funding for research on ``supporting children and adolescents with a migration background''  (``Förderung von Kindern und Jugendlichen mit Migrationshintergrund'', \textit{FörMig}). This programme had its sole focus on presumed problems with the language competences of these young people. In a press release on the start of the funding scheme, the minister was quoted describing as its goal the ``support and integration of migrants'',\footnote{Press release BMBF, February 3\textsuperscript{rd}, 2005: ``Bundesbildungsministerin Edelgard Bulmahn […]: ‚Die Förderung und Integration von Migrantinnen und Migranten muss zu einem zentralen Element der Bildung in allen Bereichen werden.'' (\url{http://web.archive.org/web/20050208063624/www.bmbf.de/press/1376.php}, last access May 24\textsuperscript{th}, 2023)}again using ``migrant'' interchangeably with ``migration background''. This conceptualises multilingual speakers as ``migrants'' who need to be integrated into the very society they were born into, and who need to be supported in the majority language of this society.

This Othering of speakers can lead to a pathologisation of their language use. Bilingual pupils with a “migration background” are overdiagnosed with language disorders in Germany \citep{GagarinaEtAl2020}. In speech-therapeutic practice, they might then be confronted with ethnic Othering up to racism: \citet{TischerGroh2017} interviewed speech therapists about their adolescent patients who used Kiezdeutsch, an urban multiethnolect. As illustrated by the quotes in (\ref{bkm:Ref134095247}), therapists offered a range of characterisations that were Othering at linguistic and ethnic levels \REF[ab]{ex:03:1}, and sometimes openly racist \REF{ex:03:1c}, (German originals; translations by me, H.W.):

\ea\label{ex:03:1}
\label{bkm:Ref134095247}
	\ea “not our German grammar”; “semilingual parents […] quarterlingual children” \label{ex:03:1a}
	\ex “migration children” \label{ex:03:1b}
	\ex “I think it is also somehow a problem of their intellect” \label{ex:03:1c}


% This Othering of speakers can lead to a pathologisation of their language use. Bilingual pupils with a ``migration background'' are overdiagnosed with language disorders in Germany \citep{GagarinaEtAl2020}. In speech-therapeutic practice, they might then be confronted with ethnic Othering up to racism: \citet{TischerGroh2017} interviewed speech therapists about their adolescent patients who used Kiezdeutsch, an urban multiethnolect. As illustrated by the quotes in (\ref{bkm:Ref134095247}), therapists offered a range of characterisations that were Othering at linguistic and ethnic levels (\ref{ex:bkm:Ref134095247a}, \ref{ex:bkm:Ref134095247b}), and sometimes openly racist (\ref{ex:bkm:Ref134095247c}), (German originals; translations by me, H.W.):
%
% \begin{enumerate}
% \item
% \begin{enumerate}\label{bkm:Ref134095247}
% 	\item \label{ex:bkm:Ref134095247a}``not our German grammar''; ``semilingual parents […] quarterlingual children''
% 	\item \label{ex:bkm:Ref134095247b}``migration children''
% 	\item \label{ex:bkm:Ref134095247c}``I think it is also somehow a problem of their intellect''
	
	``emotional superficiality''
	
	``cultural alalia''

	``they also have a different anatomy, they also have completely different faces. Often they have very bulging lips, and then on top of that the puberty'' [laughs]
    \z
\z

Note that these are utterances made in the context of an interview, that is, in a situation where participants tend to monitor themselves more closely and tend to avoid statements they deem to be socially undesirable or unacceptable. That we find such evidence from language professionals could indicate that such opinions are widely shared in society or at least perceived as such. How does this affect academia? Do we find effects of this societal habitus in linguistics?

\subsection{Does this affect linguistics?}

If we are looking for possible influences of a monolingual and monoethnic societal habitus in linguistics, we have to remember that this kind of habitus is primarily characteristic for the Global North, given its origin in European nation-state building. At the same time, research from the Global North has been dominant in academia for a long time. For psychology, \citet{HenrichEtAl2010} pointed out that research traditionally concentrated on what they called WEIRD societies: an abbreviation for ``\textbf{W}estern, \textbf{E}ducated, \textbf{I}ndustrialised, \textbf{R}ich, \textbf{D}emocratic''. In linguistic typology, \citet{Dahl2015} criticised a LOL bias: a focus on languages with the features ``\textbf{L}iterate, \textbf{O}fficial, \textbf{L}ots of users'', which account for less than 1\% of human languages. Both patterns are related to the dominance of the Global North, with its characteristic political, economic, and linguistic make-up.

Accordingly, we might expect to also see effects of the monolingual and monoethnic habitus prevailing in the Global North. This is borne out not only in such earlier structuralist idealisations as de Saussure’s ``forme idéale'' (\citealt{Saussure1916}) or Chomsky’s ``ideal speaker-listener'' \citep{Chomsky1965} that were discussed in the Introduction (\citealt{chapters/01}). We can also see evidence for such a habitus in current conceptualisations of a ``native speaker''.\footnote{For a recent debate see contributions in \citet{GuijarroFuentesEtAl2022}} In research on language acquisition, monolinguals have long been constructed as the primary bearers of a language.\footnote{Cf. criticism in \citet{Ortega2009}; \citet{ORourkePujolar2015}.} In heritage language research, monolinguals are still commonly used as controls (see also \citealt{chapters/05}):\footnote{Cf. \citet{RothmannEtAl2023} for a recent critique of ``monolingual comparative normativity''.} heritage speakers have so far primarily been investigated in the Global North, in societies with a strong monolingual habitus, and monolinguals have been chosen as a control group for ``native-like'' behaviour, ``native competence'' or ``native levels'' of language attainment.

This implies a monolingually biased view on who belongs to the primary speaker group and owns a language, very much in keeping with the societal habitus we discussed. In the present paper I am going to delve deeper into this by looking at a broader range of subdisciplines. In order to uncover possible implicit biases, I will focus on the way multilinguals are labelled. In what follows, I present evidence from linguistics and from related fields of sociology and education, and analyse the different topoi that are evident in labelling practices (\sectref{sec:03:2}). Against this background, I discuss how this might affect our research perspectives (\sectref{sec:03:3}). The final section (\sectref{sec:03:4}) summaries our results and discusses their implications.


\section{Labelling multilinguals as Others}\label{sec:03:2}
\subsection{Evidence from academic writing}

In what follows, I present a qualitative analysis of Othering patterns implicit in labels used for multilinguals in academia. For this analysis, I use as my empirical basis examples of labelling I found when reading literature relevant for my research interests. This is hence not a representative sample, but skewed towards my own reading. The goal was not to identify quantitative patterns, but to analyse the underlying topoi evident in such labels. This said, the examples indicate that Othering through labelling has a wide distribution in linguistics. My examples come from publications across different perspectives, subdisciplines and research domains, including contact linguistics, heritage language research, language acquisition, grammatical analysis, and sociolinguistics, plus some examples from related fields of sociology and education. Geographically, all examples come from the Global North, predominantly from research in Europe, but also from North America and Australia.

I have anonymised all sources. Speaker codes identify the discipline (L – Linguistics; S – Sociology; E – Education), speaker number within a discipline (L01, 02, … S01, …), and publication year. There are two reasons for this anonymisation. First, I use publications as data points here, hence authors are treated as subjects and accordingly anonymised. Second, since these are widespread practices, it would be besides the point to put individual authors on the spot, and might hinder an open discussion. In this context, let me point out that the examples also include a quote from one of my own earlier publications.

As I will show in the following sections, Othering through labelling falls into three main, interrelated strands:
(1) Othering with respect to territorial belonging constructs geographic Others;
(2) Othering with respect to national group membership constructs national Others;
(3) Othering with respect to linguistic ownership constructs linguistic Others. These strands closely follow patterns of Othering in public discourse, as discussed in Section \ref{bkm:Ref134097123}. In what follows, I look at each strand in turn.

\subsection{Territorial belonging: Constructing geographic Others}

Labelling practices in this strand construct children and grandchildren of immigrants as geographic Others, although they are locally born (and non-mobile). This is hence reminiscent of the use of ``migrant'' interchangeably with ``migration background'' mentioned above for the public discussion in Germany.

In some cases, a \textit{migrant status} is perpetuated over several generations, with speakers labelled as (im-)migrants, rather than locals when their ancestors were immigrants. In other cases, a \textit{foreign origin} is perpetuated for later generations, with speakers labelled as originating from other countries, rather than from the country of their birth and upbringing. We can identify two topoi here:

\begin{itemize}
\item Topos 1 \textbf{``Perpetual} \textbf{Migrants}'': Multilinguals are migrants.
\item Topos 2 \textbf{``Not} \textbf{from} \textbf{Here''}: Multilinguals have a foreign origin.
\end{itemize}

The first topos is evident when multilingual speech communities are described as ``immigrant populations'' (L01/2019), and when locally born speakers are labelled as ``migrants'' (S01/2014) and set apart as ``migrant peers'' (L19/2009) of monolingual local speakers. Rather than being acknowledged as first or second generation locals – or indeed just locals – they are labelled as ``second'' or ``third generation (im-)migrants'' (L01/2019, L03/2020, L06/2020, L10/2017, L11/2013, L16/2011, L19/2009, L20/2013, E02/2017), or as coming from ``migrant families of the second or third generation'' (E01/2020).

The second topos manifests itself when foreign ``home countries'' (L01/2019) or ``countries of origin'' (L05/2013, E02/2017) are postulated for locally born speakers, while the country they have lived in all their lives is described as a ``host country'' (L20/2013) or, adopting a perspective of the speaker, they are described as ``living abroad'' (L06/2020). Speakers are characterised, e.g., as of ``Turkish origin'' (L19/2009) or of ``Moroccan'' or ``Turkish descent'' (L07/2014, E02/2017) or as having ``foreign roots'' (S01/2014).

Taken together, this kind of labelling foregrounds an ancestral migration event in the family history and applies it to a generation that has not participated in it, making them new arrivals in the country of their birth (Topos 1), and constructing them as aliens whose home is not the country they live in, but the sending country of those earlier generations in their family history (Topos 2).

\subsection{National group membership: Constructing national Others}
\label{bkm:Ref135038223}
Labelling practices in this strand construct multilingual speakers as members of a foreign out-group or establish dichotomies between them and the local in-group. Again, two topoi are associated with this:

\begin{itemize}
\item Topos 3 \textbf{``Foreign} \textbf{Nationals''}: multilinguals are Turks, Chinese, …
\item Topos 4 \textbf{``Not} \textbf{Our} \textbf{People''}: multilinguals are not German, Dutch, ….
\end{itemize}

Topos 3 is evident when speakers who were born and grew up in, e.g., Germany, the Netherlands, or Australia are labelled ``Chinese'', ``Greek'', Turks'', ``Moroccan'', ``Albanian'', or ``Surinamese'' (L01/2019, L02/2013, L07/2002, L07/2014, L12/2008). In some cases, similarly to the ``Perpetual Migrants'' topos, there is explicit reference to the fact that speakers are already the second generation living in the country, but instead of, e.g., ``second generation Dutch/Australian'', they are labelled as ``second generation Chinese/Greek'' etc. (L01/2019, L13/2019).

Feeding into Topos 4, locally born young people who are multilingual are contrasted to an in-group of, e.g., ``Dutch'' or ``German'' children or adolescents who are constructed as monolingual (L04/2017, L07/2002, L17/2008, E03/2005, S01/2014, S02/2017). This restricts local belonging to monolingual speakers and demarcates multilinguals as outsiders. Especially in the context of the Netherlands, an additional opposition pair sometimes used here is that of ``autochthonous'' vs. ``allochthonous'', which terminologically restricts belonging to one group, marginalising the second group as ``allo-''. While this is in keeping with the state’s census terminology, associating this dichotomy with monolingualism vs. multilingualism in linguistics supports further Othering of multilingual speakers.

These topoi feed into a narrative that sets multilinguals in contrast to the national in-group and casts them as members of another nationality. Especially in the context of European nation-states, this constructs an alien speaker group that is excluded from local belonging.

\subsection{Linguistic ownership: constructing linguistic Others}

Labelling practices within this strand deny multilingual speakers ownership of the languages they speak, even if they grew up with them as a family language or as the dominant language in the society they were born into. My data reveals two topoi:

\begin{itemize}
\item 
Topos 5 \textbf{``Not} \textbf{Native} \textbf{Speakers''}: multilinguals are not native speakers of Spanish, Dutch, …

\item 
Topos 6 \textbf{``Not} \textbf{of} \textbf{Our} \textbf{Language''}: multilinguals are not German-, … speaking

\end{itemize}

Topos 5 is evident through labelling that constructs only monolingual speakers as native speakers and contrasts them with bilinguals. In heritage language research, this is evident when differences between ``native speakers and heritage speakers'' or deviations of ``heritage speakers from native speakers'' are investigated (L01/2019, L14/2018), thus denying native-speakerhood to bilinguals, even though they acquired the language from birth. The other side of the coin is evident when bilinguals are contrasted to ``native speakers'' of the majority language that they speak in addition to their heritage language (S01/2014). Taken together, this kind of labelling denies bilingually grown up speakers native-speakerhood for both of their languages.

In Topos 6, this is taken a step further for the case of the majority language, excluding bilinguals from its speaker base altogether. Such Othering is evident in a paper constructing as ``non-German speaking students'' German university students with a ``migration background'' (L09/2010), referring to speakers who went to school in Germany (and usually were already born there) and graduated from school there – in schools where German would normally be the language of instruction, given the country’s strong monolingual habitus. This might look like an extreme case, and it certainly seems to be less common than Topos 5. However, the paper in question is not a fringe publication, but a contribution to a well-established handbook in which this obviously remained unnoticed throughout the reviewing and editing process, which underscores the normality of such terminology in our field.

\subsection{Three strands: parallels to public discourse}

As mentioned in \ref{bkm:Ref134097123} above, public discourse on language and belonging in the wake of European nation-states is often dominated by a nexus of ``one country, one nation, one language''. My findings indicate that this nexus is not restricted to the general public and ``lay'' discourse: results reveal labelling practices in academic writing that construct multilingual speakers as Others with respect to territorial belonging, national membership, and language ownership, mirroring the trinity of country, nation, and language. \figref{fig:03:2} illustrates this integration of the three strands and the topoi feeding into them.


%%[Warning: Draw object ignored]
\begin{figure}\label{bkm:Ref66953048}
	\begin{tikzpicture}
		\node(belonging)[align=left,text width=3.5cm,rectangle,draw,fill=black!20!white]{\textbf{Territorial belonging} \\ ``Migrants'' \\ ``Not from here''};
		\node(membership)[align=left,text width=3.6cm,rectangle,draw,fill=black!20!white,below=1cm of belonging]{\textbf{National membership} \\ ``Foreign nationals'' \\ ``Not our people''};
		\node(ownership)[align=left,text width=3.5cm,rectangle,draw,fill=black!20!white,below=-2.5mm of belonging, xshift=5cm]{\textbf{Language ownership} \\ ``Not native speakers'' \\ ``Not of our language''};
		\node[rectangle,draw,above=0mm of belonging,xshift=1.03cm]{\textsc{country}};
		\node[rectangle,draw,above=0mm of membership,xshift=1.21cm]{\textsc{nation}};
		\node[rectangle,draw,above=0mm of ownership,xshift=.93cm]{\textsc{language}};
		\draw[line width=1pt,->](belonging)--(membership);
		\draw[line width=1pt,<->](membership.east)--(ownership.south west);
		\draw[line width=1pt,<->](belonging.east)--(ownership.north west);
	\end{tikzpicture}
	\caption{ Othering of multilingual speakers through labelling practices in academia}
	\label{fig:03:2}
\end{figure}

This result suggests that when it comes to the labels we use for multilinguals, the monolingual and monoethnic habitus dominant in larger society exerts its power in academia as well. The linguistic side of this is evident in Othering that denies speakers language ownership, the ethnic side in territorial and national Othering. With respect to the latter, it is presumably not a coincidence that the speakers labelled in this way were invariably speakers of marginalised and racialised groups.\footnote{See \citet{Fuller2012} on bilingual settings constructed as ``immigrant'' vs. ``elite'' in Germany and the US; \citet{RosaFlores2017} on raciolinguistic patterns.} As the contrast discussed for the two \textit{Tagesspiegel} articles in \ref{bkm:Ref134097123} above illustrated, it is these groups that are subject to such Othering in the public debate, rather than those constructed as white middle class.

That labelling practices in academia should so closely mirror negative language ideologies in larger society is surprising. It suggests an unreflective use of terminology that entails biases that presumably none of the publications in question would subscribe to – in fact the very biases that many explicitly criticise. \citet[20]{DaviesLanger2006}, discussing standard language ideologies, remind us that ``\textit{Folk linguistic} views will also be expressed by academic linguists, of course, whenever they are guided by their native instincts rather than their official academic views''. It seems that too often, we are still guided by ``instincts'' on mono- and multilingualism that reflect linguistic and ethnic societal biases, as revealed in the labels we use.

Avoiding such labelling practices is not only important from the point of view of scholarly terminology. The Othering implicit in such labelling can further undermine social equality and societal participation of marginalised groups. A case in point comes from the public discussion in Germany in early 2024. It illustrates, for an extreme example, the kind of narratives that such a conceptualisation as Others can support. Journalists working with the investigative platform ``Correctiv''  revealed a secret meeting of conservative and right\hyp extremist politicians in November 2023 where plans of mass deportations had been discussed of foreign nationals as well as German citizens who were conceptualised as ethnic Others. When these plans became public, they led to a public outcry, with major demonstrations against the extreme right throughout the country lasting for weeks. The interesting point for our discussion is that this deportation plan had been constructed as ``remigration''. Note that this is in fact a term in keeping with the kind of labelling discussed in this paper: by calling second-generation Germans etc. ``\mbox{(im-)}migrants'' we are ultimately blowing in the same horn. If locally born speakers are understood as ``migrants'' from other ``home countries'', then it is only a small step to frame their expulsion into those countries as ``remigration''. When, on the other hand, we conceptualise them as Germans, we recognise such a deportation for what it is.

\largerpage[-2]
Interestingly, some traces of Othering of the people targeted in this deportation plan were evident even in the public outcry against it. An initiative \#ZUSAMMENLAND (``togetherland'') that was joined by the major media, the industrial sector, and all major academic institutions in Germany (incl. the German Science Foundation, the Leibniz Association, the Max Planck Society, the Helmholtz Association, the German Academic Exchange Council and a large number of universities) criticised ``the so-called ``remigration'' of our friends, neighbours, colleagues''. While those ``friends, neighbours, colleagues'' are not framed as non-German, they are also not us: they are not part of the ``we''-group that makes this statement.\footnote{For the full text of the statement (in German) see \url{https://cmk.zeit.de/cms/articles/16974/anzeige/zusammenland/neue-kampagne-zusammenland-vielfalt-macht-uns-stark} (last accessed May 24th, 2024).}

With respect to academia, the Othering evident in the kind of labelling practices discussed in this paper can also affect our research perspective. If multilinguals are constructed as Others, this can lead to a problematic research bias. In the literature, we find some indication for this in different areas. In the following section, I illustrate this with two examples, highlighting problems for approaches to language structure and language use.

\section{Problems for our research perspectives}\label{sec:03:3}
\subsection{Allocating multilinguals’ language use outside native grammars}

A societal monolingual and monoethnic habitus casts multilinguals as linguistic outsiders and leads to doubts about their language competence. This is what we saw, for instance, reflected in such funding schemes as \textit{FörMig} for linguistic and educational research projects in Germany (Section \ref{bkm:Ref134097123}). In the public debate, a frequently voiced assumption is that adolescents with a ``migration background'' do not master German properly, and such voices include well-known linguists.\footnote{See \citet{Wiese2015} for a discussion of linguists’ contributions to the public debate on Kiezdeutsch.}

Along the same lines, if a monolingual and monoethnic habitus leads to constructing multilinguals as being outside the native speaker group, then their linguistic practices might not be seen as part of native grammars.\footnote{See contributions in Guijarro-Fuentes et al. (eds.) (2021/2022) for critical approaches to the concept of ``native speaker''; \citet{WieseEtAl2022a} for evidence from heritage language research.} An example of how this can impact research in linguistics comes from earlier accounts of an urban contact dialect now known as ``Kiezdeutsch'',\footnote{Lit. ``(neighbour-)hood German'' \citep{Wiese2012}.} which emerged in multilingual and multiethnic neighbourhoods in peer-group interactions among adolescents.

In the early 2000s, grammatical characteristics of this variety were typically described as more or less random deviations and omissions. For instance, \citet[4]{Auer2003} states that ``gender gets changed (presumably ad hoc), […] definite and indefinite article forms are often missing, […] the XV…-order of German is transformed into SVO''.\footnote{German original, my translation (H.W.).} The last characteristic refers to an option in Kiezdeutsch to place an adverbial and a subject in front of the finite verb, which violates the verb-second (V2) constraint for standard German main declaratives. \citet{SeltingKern2009} discussed this for a specific prosodic pattern and described it as something normally not found in German including its ``colloquial and/or regionalized varieties''. 

Meanwhile, this word order has been analysed as a systematic verb-third (V3) construction that is a viable option within German grammar,\footnote{Cf. \citet{Wiese2012}; \citet{FreywaldEtAl2015}; \citet{Velde2017}; \citet{Walkden2017}; \citet{AlexiadouLohndal2018}.}it has also been attested in monolinguals, including in spoken language outside Kiezdeutsch,\footnote{See \citet{Schalowski2015}; \citet{WieseMüller2018}; \citet{Bunk2020}.} and it has been shown to have parallels with options found in earlier diachronic stages, suggesting long-standing Germanic roots.\footnote{\citet{Wiese2012}; \citet{Walkden2017}.}

Hence, it is unlikely that such statements would still be made today. The point is, however, that at the time those claims were made without any empirical backing on the range of options outside V2. This might not have been the case if such patterns had first been observed in monolinguals, and it might be linked to a construction of speakers as outsiders. \citet{Auer2003} calls the variety \textit{Türkenslang} ``Turks’ slang'', \citet{KernSelting2006}\footnote{This is an earlier, German version of \citet{SeltingKern2009}. In \citet{SeltingKern2009}, they changed the term to ``Turkish German''.} use the term \textit{Türkendeutsch} ``Turks’ German'', thus conceptualising speakers as ``Turks'', rather than Germans, even though they were born and raised in Germany (and often hold German citizenship), in line with Topoi 3 and 4 discussed in \ref{bkm:Ref135038223} above. This points to a possible implicit bias that can have negative effects on our research: if speakers are not perceived as a legitimate part of the German in-group, their language use will not be allocated within the normal range of German – an Othering of speakers goes hand in hand with an Othering of their language use.

Against this background, \citet{Dittmar2013} states, in a paper on the speakers of this variety, which he refers to as ``multiethnolectal German'' (short: MED),

\begin{quote}
``I advise caution here: we should […] not miss any opportunity to point out that MED speakers […] also have to practice the German standard – it is the only way they can be valuable, indispensable co-designers and creative language changers of our society and remain so in the future!''
\end{quote}

This seems to put Kiezdeutsch speakers outside society: they seem not to be included in the ``us'' in ``\textit{our} society''; speaking Kiezdeutsch puts them at risk of remaining in the out-group. This highlights the link between Othering a way of speaking and Othering a social group associated with it. At the same time, it also shows how this goes hand in hand with doubts about language competence, again parallel to what we found for larger society. In this case, the doubts are on register competences, specifically those in formal, standard-close registers, which are regarded as a precondition for speakers being ``valuable'' members of ``our society''. Apart from the strong standard language ideology evident here, such doubts show that register differentiations are overlooked when it comes to language use that is primarily associated with marginalised multilingual speakers – even though evidence of broader speakers' repertoires in such multilingual settings was available well before that paper was published.\footnote{See, for instance \citet{KallmeyerKeim2003}, \citet{Keim2007}. For register-differentiated use of standard varieties and urban contact dialects by adolescent speakers in other countries, e.g., \citet{Cornips2008} on Dutch, \citet{Ganuza2008} on Swedish.}

And in the same year, \citet{Auer2013} still described the V3 pattern found in Kiezdeutsch as a ``change of XV order'' that ``deeply intervenes in the structure of autochthonous German in its standard and nonstandard forms''. This was despite evidence for the structural integration of this pattern into the German sentence lay-out (e.g., the presence of verbal brackets in V3 and the use of verb-last order in subordinate sentences), together with an overall very low frequency of this pattern, with canonical V2 accounting for the overwhelming majority of main declaratives in speakers’ productions.\footnote{See \citet{Wiese2013} and references therein.}

That such evidence is overlooked and that patterns deviating from standard language are allocated outside authochthonous German, suggests a research bias that casts speakers as allochthonous and that has negative effects on our findings: we miss out on the opportunity to learn something new about native grammars and their options. As research guided by a multilingual, rather than deficit perspective has shown, multilingual settings can support a linguistic dynamics that leads to quantitative advantages and provides us with an interesting spotlight onto linguistic variation and change.\footnote{\citet{Wiese2013}; \citet{KupischPolinsky2022}; \citet{WieseEtAl2022a}.}

\subsection{Erasure of majority language practices in multilinguals}

If multilinguals are seen as linguistic Others in countries with a monolingual societal bias, this can also mean that their use of the majority language is not seen. Such an erasure is evident in two related assumptions often found in research on multilinguals that speak a heritage language:
(1) if a neighbourhood is associated with multilingual and ``migrant-background'' speakers, it is assumed that these speakers encounter the majority language only rarely and are hardly exposed to it at all before formal schooling starts, and
(2) speakers are characterised as using, in private domains, the heritage language almost exclusively.


To illustrate such assumptions, I quote from a recent handbook, \citet{AalberseEtAl2019}. I selected this handbook because it adopts an overall positive approach to multilingualism and multilingual practices, hence finding such claims here illustrates how pervasive they are.

\begin{quote}
``In such areas [Berlin-Kreuzberg, H.W.], Turks encounter relatively few members of the majority culture in their daily lives“ (\citealt{AalberseEtAl2019}: 50)
\end{quote}

\begin{quote}
``In HL [heritage language, H.W.] contexts, there is often an almost exclusive use of the HL in private domains and an equally unrivaled use of the majority language in public ones'' (\citealt{AalberseEtAl2019}: 48f)
\end{quote}

If we unpack this, we find some indications of problematic effects of societal biases in our research. The label ``Turks'' for speakers refers back to their description as ``ethnic Turks'' earlier in the paragraph cited here, which foregrounds ethnicity. This sets speakers outside the national in-group of German speakers similar to Topos 3 above. This perspective is further underlined when they are contrasted with ``members of the majority culture''. According to the latest microcensus in Germany, more than a quarter (27.55\%) of inhabitants have immigrated themselves or have at least one parent who did, and more than a fifth (20.05\%) do not only speak German at home \citep{StatistischesBundesamt2023}. One might hence reasonably argue that the majority culture in Germany is multiethnic and multilingual. However, in this case, only ``ethnic'' Germans or monolingual speakers of German seem to be constructed as its legitimate members.

The claim that very few of these ``members of the majority culture'' are encountered by heritage-Turkish speakers in Berlin-Kreuzberg in their daily lives, is in and of itself not an indication of a research bias. However, to my knowledge there is no empirical backing for this: we are still lacking systematic ethnographic studies on daily linguistic encounters in such neighbourhoods. This suggests that such a claim is considered self-evident, or at least not in need of further evidence.

I have severe doubts about the truth of such claims, though. One reason is that I have been living in that area of Berlin for 30 years now, as a linguist who is naturally always interested in language use, and as a parent who raised two children there. In these roles I have been observing and participating in frequent encounters between people across ``ethnic'' and mono- versus multilingual backgrounds on a daily basis. Speakers of heritage-Turkish, speakers of other heritage languages, and monolingually German speakers mix and interact as neighbours at street festivals, at library events and book nights, in cafés, at shops, in parks and on playgrounds, and in local citizenship initiatives; families meet and parents befriend each other across linguistic and ``ethnic'' boundaries when children attend local kindergartens, schools, and sports clubs. Judging from this, the community seems to be much less segregated than popular stereotypes of ``urban ghettos'' suggest: the fact that there are some shops and hairdressers where it is also possible to be served in Turkish does not mean that people are isolated along language lines, or that German does not play a role in a neighbourhood. It means that the neighbourhood is multilingual, and that Turkish is a relevant heritage language.

Claims about some kind of linguistic isolation of heritage speakers are also doubtful in view of the demographic facts: just over half (55\%) of Kreuzberg residents have a ``migration background'' or have parents who do, according to the census \citep{StatistischesBundesamt2023}. Hence nearly half of Kreuzberg inhabitants presumably do not speak a heritage language, making it highly unlikely that there will be only few encounters with them in daily life.

Along the same lines, whether there is a strict division of labour between an exclusive use of heritage languages at home and of majority languages in public is an empirical question, but we are at present still lacking systematic, large-scale data on actual language choices at home. It is doubtful whether assumptions of such a 1-to-1 correlation of languages along private/public domains will hold up to empirical scrutiny. Apart from code switching and language mixing (which \citealt{AalberseEtAl2019} mention as well), an important point is the use of the majority language in the private domain: there are good reasons to believe that majority languages play a systematic role in private settings for heritage speakers.

Note that heritage language research has so far been mostly concentrated on bilingual speakers who have grown up in countries with a monolingual bias and an accordingly dominant majority language. That the majority language is so dominant means that speakers are bound to encounter it in a broad range of informal situations from early on in their daily life. It will be the language of choice not only with monolingual speakers of the majority language, but often also in encounters with speakers of other heritage languages.

Private settings that hence favour majority language use include interactions on the playground, in the street, and frequently with friends. Given the dominance of the majority language in many everyday situations, we should expect it to enter family communication as well, and there is at least some evidence of this from speakers’ self-reports (see also \citealt{DeHouwer2018}: \S7 for evidence on school language use in family interactions). Here is a transcript from an interview with a heritage-Turkish speaker, 18 years old, growing up in Berlin-Kreuzberg:\footnote{Full video available at \citet{WieseEtAl2014ff}, module ``Ballkontakte'' (\href{http://www.deutsch-ist-vielseitig.de}{{www.deutsch-ist-vielseitig.de}}).}

\begin{quote}
``At home, I speak mostly Turkish – but only with my parents [laughs]. Apart from that, I speak with my siblings relatively – only German.''
\end{quote}

If you grow up in a country with a monolingual habitus, you cannot avoid encountering the majority language from an early age well before school and kindergarten, and with your siblings, you will presumably use it regularly within the family.

This is also what primary school children from heritage language families reported in a study we conducted with students aged 9 to 12 from a state school in Berlin-Kreuzberg. Children named German as the main language they used with friends and siblings, that is, in private settings including the family, and it was only for communication with older generations – parents and grandparents – that heritage languages (mostly Turkish and Kurdish) were mentioned as being relevant as well.\footnote{H. Wiese, P. Seeger, J. Fuller 2015, unpublished interview data.}

Note that the speaker in the above quote starts reporting Turkish at the main language at home, in line with prevailing societal language ideologies that cast speakers as Turkish only. Only later does he correct himself and restricts the use of Turkish to communication with his parents. This suggests that the use of German within family communication might be even stronger than speakers are consciously aware of (and, accordingly, report). Experiences from an outreach project with a Turkish-German bicultural kindergarten and a primary school in Berlin-Kreuzberg support this. In one of the modules, children took a plush toy, a little stork, home with them overnight, and then the next day told the group what range of language use the stork had observed in the child’s family.\footnote{\citet{WieseEtAl2014ff}, module ``Storch Lingi''.} 47 children aged 4 to 8 participated in the project, of whom 31 were from families with a heritage language. Among all children, there was not a single case where German had not been used at home. During the course of the project, a mother of one of the children approached me to tell me she was surprised at what she had discovered about her family’s linguistic practices, saying

\begin{quote}
``I have always been unhappy with M.’s [her son’s, H.W.] Turkish but now I have realised we hardly ever speak it at home; I thought we mostly speak Turkish in the family, but turns out we speak German nearly all the time. No wonder he struggles with his Turkish.''\footnote{Project notes, German original, my translation, H.W.}
\end{quote}

The dominance of German in bilingual speakers’ repertoires is also evident in something like ``Whorfian'' effects we observed in bilingual Turkish-German adolescents in Berlin-Kreuzberg. In a study targeting the information-structural underpinnings of V3, we asked participants to describe a scene to an interlocutor without speaking, using only concrete objects and laminated print-outs of verbs (in their infinitival form). The scene involved a framesetter and a topic, together with an action, and we expected speakers to present both the framesetter and the topic before the action (= the verb), in line with V3 patterns where an adverbial and a subject are placed before the verb.

This expectation was borne out, but there were also some interesting differences between speaker groups. We conducted the study with monolingual speakers of Turkish in Turkey, monolingual speakers of German in Germany, and bilingual Turkish-German speakers in Germany (Berlin-Kreuzberg). Monolinguals were tested in one language, bilinguals were tested twice, once in German and once in Turkish (with German and Turkish verbs, and a German- or Turkish-speaking experimenter, respectively). The most frequent pattern used across all groups had the verb at the end, in line with V3. However, for the two groups in Germany – in contrast to the one in Turkey – this was closely followed by a pattern with the verb in the middle, in line with V2. This suggests that German V2 has an impact on linearisation patterns even in extra-grammatical tasks, and this impact was evident in bilingual speakers irrespectively of whether they acted in the German or in the Turkish condition, pointing to German as a strongly dominant language for them \citep{WieseEtAl2017}. 

The assumption that heritage speakers do not speak the majority language in their family and do not encounter it except in educational institutions hence does not seem to be supported by empirical evidence. That we find this claim to be so pervasive in linguistic research might be an effect of the monolingual societal habitus from which we as researchers are not entirely free.

\section{Conclusion}\label{sec:03:4}

Researchers from the Global North dominate much of academia, including linguistics, and accordingly, their implicit biases can have a substantial effect. In the case of language-related biases, we see a widespread monolingual and monoethnic societal habitus that has its roots in European nation-state building, with its language-ideological nexus of ``one country – one nation – one language''. In this paper, I discussed the Othering of multilingual speakers that such a habitus fosters in public discussion, for the example of Germany, with examples from media, public policy and education, including large research funding schemes. Against this background, I investigated evidence for such Othering in our own professional practices as linguists, starting from the hypothesis that a strong monolingual and monoethnic societal habitus might leave its traces in researchers who grew up in the Global North, leading to implicit biases.

I analysed labelling practices for multilingual speakers in linguistics and related fields of sociology and education and showed that this is in fact the case. Patterns of Othering through labelling that are evident in academic writing are not substantially different from those found in public discourse: I found evidence for labelling practices in academia that exclude multilinguals from geographic, national, and linguistic in-groups and mark them as Others.

Analysis revealed three pairs of topoi in labelling practices in academia:

\begin{itemize}
\item 
``Perpetual Migrants'':   Multilinguals are migrants. \\
``Not from Here'':   Multilinguals have a foreign origin.

\item 
``Foreign Nationals'':   Multilinguals are Turks, Chinese, …\\
``Not Our People'':  Multilinguals are not German, Dutch, ….

\item 
``Not Native Speakers'': Multilinguals are not native speakers of Spanish, Dutch, …\\
``Not of Our Language'': Multilinguals are not German-, … speaking.
\end{itemize}

I showed that each pair of topoi feeds into one of three main, interrelated strands:
(1) Othering with respect to territorial belonging constructs geographic Others;
(2) Othering with respect to national group membership constructs national Others;
(3) Othering with respect to linguistic ownership constructs linguistic Others. These strands hence mirror the ideological nation-state nexus of ``one country, one nation, one language''.


Examples came from publications across different perspectives, subdisciplines, and research domains, underlining how widespread such practices are in our field. I argued that avoiding such labelling practices is not only important from the point of view of scholarly terminology, but also for our research perspective: if multilinguals are constructed as Others, this can lead to an implicit bias that can have negative effects on our research.

I illustrated this problem with two kinds of examples from research on multilingual speakers, coming from the investigation of urban contact dialects and heritage speakers, respectively. I showed that in both areas, we find assumptions that marginalise multilingual speakers and their linguistic practices with respect to the society they grew up in: grammatical patterns found in multilingual speakers were allocated outside the majority language grammar, and relevant areas of their majority language practices were erased. I argued that since these claims are not backed by empirical evidence, they seem to be taken for self-evident, reflecting implicit biases. I showed that the empirical evidence available so far is in fact at odds with such claims, indicating that such biases can be misleading for the linguistic discussion and can prevent us from gaining novel insights into language structure and language use.

That linguistics is not free from bias is not surprising, of course – we are part of society, and as such, we will be influenced by it. However, if we find recurring negative biases across subdisciplines, it is important to reflect on this and aim to overcome the marginalisation of multilinguals implicit here. The analysis of our labelling practices and related problems for our research perspectives has shown that despite an ongoing discussion on multilingual perspectives in linguistics,\footnote{See, for instance, \citet{Canagarajeh2007}, \citet{Wei2016}, \citet{Flores2017}, \citet{KupischRothman2018}, \citet{BayramEtAl2019}, \citet{WieseEtAl2022a}.} we still need to be aware of our own societal prejudices as researchers.

As the examples suggested, the marginalisation of speakers might lead us to more readily assume some kind of linguistic isolation for them both at structural and at usage levels. The association of a country with a specific language negatively affects our perspective on multilinguals, and its association with a specific, ethnically imagined ``nation'' will further add to their marginalisation if they come from a family with migration experience. The ethnolinguistic implications inherited from European nation-state building lead to a construction of such speakers as outsiders and affect our ability as researchers to approach them as legitimate speakers of their languages, including the majority language of the society they grew up in.

\largerpage
Reflecting and overcoming such biases can also make our research more societally relevant. In public discussion and politics, it could, for instance, support an inclusion of multilinguals as members of the majority society and owners of the majority language. In the educational domain, it could lead to a different focus, one that targets negative language ideologies in such key actors as teachers and language therapists, rather than presumed insufficient language competences of marginalised students.\footnote{See also \citet{Flores2019}, \citet{Wei2021} on critiques of assumptions on ``academic language'' in education.}

In linguistics, the results of our analysis contribute to existing calls to adopt a genuinely multilingual perspective, approaching characteristics of multilinguals’ language use as an integral part of language variation, understanding multilinguals as members of the linguistic in-group, and ultimately moving beyond the bilingual\slash monolingual dichotomy in favour of a more inclusive approach to native speakers and native grammars.

\section*{Acknowledgements}

Different aspects of the findings presented here were discussed at a number of talks at universities, workshops and conferences, including the 33\textsuperscript{rd} meeting of Semantics and Linguistic Theory (SALT) at Yale University 2023, the Bilingualism Matters Research Symposium in Edinburgh 2022, Heritage Language Syntax 3 in Paris 2022, the Annual Meeting of the Società di Linguistica Italiana in Brixen 2022, the online lecture series ``Language and Society'' by the universities of Rostock, Greifswald, and Frankfurt/Oder 2022, the workshop ``Do we practise what we preach? The construction of multilinguals as Others in public discourse and academia'' in Berlin 2022, the Sociolinguistics Series of Leiden University 2019, and the Sociolinguistic Lectures Cocktail Series of Cologne University 2019. I thank the participants for constructive discussions and valuable input from different perspectives. Special thanks go also to my group at Humboldt University Berlin: Oliver Bunk, İrem Duman Çakır, Annika Labrenz, Antje Sauermann, and Britta Schulte discussed different aspects of my analyses for this paper and Nicole Wong and Johanna Pott provided much-needed help in checking the sources on societal ideologies.

The analysis of topoi in linguistic labelling practices draws on section 3.1 in Working Paper in Urban Languages and Literatures 302 \citep{WieseEtAl2022Multilinguals}; I am grateful to Ben Rampton for very helpful input on an earlier version of that working paper.

Research for this paper was supported through funding by the Deutsche Forschungsgemeinschaft (DFG, German Research Foundation) for the Research Unit ``Emerging Grammars in Language Contact Situations'' RUEG 2537 (WI 2155/10-1, 10-2, 11-1, 12-1, 13-1) and the Collaborative Research Center ``Registers'' SFB 1412, 416591334.

\sloppy\printbibliography[heading=subbibliography,notkeyword=this]
\end{document}
